\documentclass[aps,pre,showpacs,superscriptaddress,final,floatfix]{revtex4-2}
\usepackage{setspace}
\usepackage{amssymb}
\usepackage{amsmath}
\usepackage{color}
\usepackage{textcomp}
\usepackage[utf8]{inputenc}
\newcommand\ra{\rangle}
\newcommand\la{\langle}
\newcommand\nn{\nonumber}
\usepackage{graphicx}
\graphicspath {{./Figures/}}
%\usepackage{cite}
\usepackage{textcomp}
\usepackage[
%dvipdfmx,
colorlinks=true, 
pdfstartview=FitV, 
linkcolor=magenta, 
citecolor=blue, 
urlcolor=black, 
bookmarks=true,
bookmarksnumbered=true,
pdftitle={},
pdfauthor={}
]{hyperref}


%\usepackage{biblatex}
\usepackage{amsmath}

\begin{document}
\linespread{1.3}\selectfont{}


	
	\title{Quasiperiodicity in the $\alpha-$Fermi-Pasta-Ulam-Tsingou system revisited:  an approach using ideas from wave turbulence }
	
	\author{Santhosh Ganapa}
	\email{sunny.ganapa@kuleuven.be}
	\email{santhosh.eeebitspilani@gmail.com}

	\affiliation{Institute for Theoretical Physics, KU Leuven -- 3001, Belgium}
	\affiliation{International Centre for Theoretical Sciences $-$ Tata Institute of Fundamental Research, Bengaluru -- 560089, India}
%\affiliation{International Centre for Theoretical Sciences, Tata Institute of Fundamental Research, Bengaluru -- 560089, India}



	\begin{abstract} The Fermi-Pasta-Ulam-Tsingou (FPUT) problem addresses fundamental questions in statistical physics, and attempts to understand the origin of recurrences in the system have led to many great advances in nonlinear dynamics and mathematical physics. In this work we revisit the problem and study the cause of quasiperiodic recurrences in the weakly nonlinear $\alpha-$FPUT system. %We use the theory of weakly interacting nonlinear waves in our approach, which is within the purview of wave turbulence.	
	Our aim is to understand if the FPUT recurrences that are observed in the original paper can be captured by the wave turbulence formalism, which is expected only if we are in the weakly nonlinear regime. %If this were so, then the lack of three wave resonances can explain the quasiperiodic behaviour observed in the original paper, and can be removed by a suitable canonical transformation.
	In our work we show that this is not always the case and in particular, the recurrences observed in the original paper are not captured by the wave turbulence formalism. %are strong nonlinear effects. 
	We attribute this disagreement to the presence of small denominators in the canonical transformation used to remove the three wave interactions before arriving at the starting point of wave turbulence. We also show that these small denominators are present even in the weakly nonlinear regime, and they become more significant as the system size is increased. %is done in earlier studies before applying the wave turbulence formalism. 
	We provide numerical evidence to support our claim.	 %even in the weakly nonlinear regime, and 
	%from here 	
	%We show this by deriving an analytic expression for the evolution of the $\alpha-$FPUT system in terms of the dimensionless nonlinearity parameter $\epsilon$ $(<< 1)$ by using perturbation theory,	and then comparing the results with molecular dynamics. We find good agreement between the two for smaller system sizes and nonlinearities, but the agreement gets poorer as the system size increases. We then explain this lack of agreement in terms of the breakdown of perturbation theory, and discuss the onset of strong nonlinearity. %, which can be explained by stricter requirements on the nonlinearity parameter $\epsilon$ (system size dependent) for the applicability of the wave turbulence formalism. %We find that the formalism of wave turbulence is not applicable for a wider range of the nonlinearity parameter than
	We also discuss our results in the context of the problem of equilibration in the $\alpha-$FPUT system, and argue that the wave turbulence formalism needs to be modified for the $\alpha-$FPUT chain in order to %take care of these small denominators.% strong nonlinearity effects. %we need other approaches in order to understand
	explain quasiperiodicity and thermalization in the system for a wider range of nonlinearities, and for larger system sizes.
\end{abstract}

\date{17 Mar 2023}
\maketitle
\textbf{Keywords}: FPUT problem, quasiperiodicity, canonical transformation, wave turbulence, thermalization.

\section{Introduction}
\label{sec:Intro}
The Fermi–Pasta–Ulam–Tsingou (FPUT) problem is a classic study with a long history. It was originally studied in order to understand the cause of thermalization in a macroscopic system. The expectation when the problem was first posed \cite{Fermi1955,dauxois2008fermi} was that nonlinearity should be enough for a system to thermalize. But instead, quasiperiodic behaviour was observed with near perfect recurrences to the initial state. At the time the authors of the paper had remarked: ``Let us say here that the results of our computations show features which were, from the beginning, surprising to us". But this ``surprise" actually lead to a number of studies in the following years aimed at understanding their results, which led to significant developments in statistical physics, nonlinear dynamics and mathematical physics. As far as the question of if and when the FPUT chain thermalizes is concerned, there have been many studies done to investigate this. Some of the methods include the closeness to integrability and soliton solutions \cite{Zabusky1965}, stochasticity threshold \cite{Chirikov1960,Izrailev1966}, Lyapunov exponent studies \cite{Casetti1996,Benettin2018, Liu2021}, breather solutions \cite{Marin1996, Flach2005, Flach2006,Danieli2017} and local equilibration studies \cite{Ganapa2020}. More recently the formalism of wave turbulence \cite{Zakharov1992,Nazarenko2011} was applied to the FPUT chain \cite{Onorato2015,Lvov2018,Pistone2019, Bustamante2019} in order to explain the dependence of equilibration time on the nonlinearity parameter.

In this work, we study quasiperiodicity in the $\alpha-$FPUT system (FPUT system with cubic nonlinearity) in more detail. % by using ideas from wave turbulence. 
 %We do so by deriving an expression for the evolution of the $\alpha-$FPUT system at short times when only the first and the last modes are excited initially. 
Our goal is to understand the cause of quasiperiodic recurrences. Previous studies \cite{Onorato2015} attribute the quasiperiodic behaviour (such as the one observed in the original paper \cite{Fermi1955}) to the lack of three wave resonant interactions. %We check this statement in our study. 
The validity of this statement is important in order for the wave turbulence description to be applicable in this system. The implication of this statement is that the three wave interactions can be removed by a suitable canonical transformation. Removing the three wave interactions is necessary before we start applying the methods of wave turbulence. This process is described in \cite{Onorato2015}, and we will discuss this briefly. In this work, we aim to test if the quasiperiodic behaviour can indeed be attributed to the  lack of three wave resonances. We do this by checking if the canonical transformation used to remove the three wave interactions has information about the quasiperiodic behaviour. We use the canonical transformation to construct the evolution of the $\alpha-$FPUT system at short times. We also use the formalism of wave turbulence to compute the frequencies of the $\alpha-$FPUT chain in terms of perturbative corrections to the harmonic chain, which is necessary in order to construct the evolution even more accurately. We then compare this evolution with that of molecular dynamics simulations. We obtain a good agreement between the two for smaller system sizes and nonlinearities, which suggests that the information about quasiperiodicity is present in the canonical transformation. However, the agreement between them gets poorer as the system size increases. We then explain this lack of agreement in terms of small denominators present in the canonical transformation, which lead to a regime where the lack of three wave resonances is not sufficient to explain the quasiperiodic behaviour. We argue and provide numerical evidence that there is a possibility of divergence in the canonical transformation leading to a breakdown of the perturbation theory, and discuss the onset of strong nonlinearity. We point out that we run into difficulties in this regime even before we start applying the wave turbulence formalism, and argue that we need to understand other physical processes which may be responsible for quasiperiodicity observed in the original paper \cite{Fermi1955}.




 %go on to explain when there is a lack of agreement. %in the wave turbulence regime. We do this by trying to derive the qua


This paper is organized as follows: in Sec.~\ref{sec:Model}, we introduce the $\alpha-$FPUT problem and describe it in terms of variables that are easier to work in the context of wave turbulence. %We then discuss some of the earlier works done on the resolution of the FPUT problem and also on the applications of wave turbulence. 
Our main work is first briefed in Sec.~\ref{sec:Method} and then described in Sec.~\ref{sec:Solution}, where we construct an analytic expression for quasiperiodicity in the $\alpha-$FPUT system by using the canonical transformation used to remove the three wave interactions (and also other non resonant interactions). This canonical transformation is basically a perturbation series in terms of the nonlinearity parameter $\epsilon$, which we define later. We use terms up to first order in our study. In Sec.~\ref{sec:Numerics} we numerically compare our expression with the dynamics of the $\alpha-$FPUT system obtained by evolving the Hamilton's equations of motion. We go on to discuss the anomalies in terms of a potential breakdown of the perturbation series. % for moderate nonlinearities and larger system sizes. 
We finally discuss our results in the context of thermalization in the $\alpha-$FPUT system and conclude our discussion in Sec.~\ref{sec:Conclusion}. The calculation used to derive the frequencies of the $\alpha-$FPUT chain is presented in Appendix.~\ref{app:freq}.

\section{Model and Review of Earlier Work} 
\label{sec:Model}
\subsection{The FPUT problem}
The $\alpha-$FPUT system of $N$ particles, each of mass $m$ is described by the Hamiltonian as a function of the position $q_i$ and momentum $p_i$ of each particle as:
\begin{equation}\label{eq:hamFPU}
H(\boldsymbol{p},\boldsymbol{q})=\sum_{i=0}^{N-1}\left[\frac{p_i^2}{2m}+\frac{\mu (q_{i+1}-q_{i})^2}{2}+\frac{\alpha(q_{i+1}-q_{i})^3}{3}\right]~.
\end{equation}
where we have considered periodic boundary conditions with $q_N \equiv q_0$ and $q_{-1} \equiv q_{N-1}$. If $\alpha=0$, the system reduces to the harmonic chain with $\mu$ being the spring constant. From now on we take $m = \mu = 1$ unless otherwise mentioned. For the harmonic chain, one can make a canonical transformation from $p,q$ variables to the normal mode variables $P,Q$ defined by:
\begin{equation}\label{eq:nm}
Q_k =\frac{1}{N} \sum_{j=0}^{N-1}q_je^{-i2\pi kj/N},~~ P_k =\frac{1}{N} \sum_{j=0}^{N-1}p_je^{-i2\pi kj/N}.  
\end{equation}
The  Hamiltonian then becomes
\begin{align}
H=N\sum_{k=0}^{N-1} E_k,~~{\rm where}~ E_k=\frac{|P_k|^2}{2}+\frac{\omega_k^2 |Q_{k}|^2}{2}~ \label{HNM}
\end{align}
is the energy of each mode and $\omega_k = 2 \sin(k\pi/N), k = 0,1,2,...,N-1$ is the normal mode frequency of the $k^{th}$ mode. The zero mode corresponding to $k=0$ does not participate in the dynamics, and we only deal with initial conditions that have $E_0 = 0$ (zero initial momentum). From Hamilton's equations of motion in these canonical coordinates one can observe that these normal modes decouple. %The evolution of each normal mode can then be found independently of the others.
So, energy exchange does not take place between different normal modes. Hence, most of the phase space is not accessible to a harmonic chain and the system does not thermalize. We would like the normal modes to somehow interact with each other in order to expect thermalization. For that we add a cubic interaction term to the Hamiltonian ($\alpha \neq 0$). %If we now ask whether the system thermalizes, then the answer is not very obvious.
We then get the $\alpha-$FPUT system described by the Hamiltonian Eq.~\eqref{eq:hamFPU}. The origin of this Hamiltonian can be understood as arising from a spring that has restoring force $F$ dependent on the displacement $x$ of the particle connected to it as $F = - \mu x -\alpha x^2$. Thus, $\alpha$ is the origin of nonlinearity in the system. Note that the cubic potential of the $\alpha$-FPUT system implies that the system stays bounded only if the total energy  is sufficiently small and the precise condition is $E< \mu^3/(6 \alpha^2)$, corresponding to all energy contributing to the potential energy of a single particle \cite{Ganapa2020}. In practice this is highly improbable and one can work with energies slightly higher than this bound. This restriction is however not needed in the $\beta$-FPUT system, which has $F = - \mu x -\beta x^3$, where $\beta$ is strength of the nonlinearity. The $\alpha$-FPUT system is studied inspite of this restriction because we can observe higher order effects more easily than the $\beta$ system (one of them is the fact that the first order correction to the harmonic frequency is zero for the $\alpha$-FPUT system (as we will see later in Appendix.~\ref{app:freq}), while it is non-zero for the $\beta$-FPUT system, which has been computed in \cite{Lvov2018}).



For sufficiently small nonlinearity, the energy contribution from nonlinear part of the interaction potential is small and it is a good approximation to assume that the total energy can still be approximated as a sum of energies of the independent harmonic oscillators, i.e, the total energy $E \approx \sum E_k$. In that case, one check of equipartition would be to see if all the $E_k(t)$ converge, at long times, to the same value $e=E/(N-1)$ (perhaps with small fluctuations). This was the approach in \cite{Fermi1955} (where however the fixed boundary condition case was studied). There, energy was initially given to the first normal mode and the time evolution was studied numerically. Contrary to the expectations, the long-time dynamics appeared to be almost periodic, with near perfect returns to the initial condition. The original paper also considered the $\beta$-FPUT system and broken linearity. In all these cases they observed quasiperiodicity instead of thermalization. 

Since the original paper, many different approaches were developed in order to explain the absence of thermalization in the FPUT system, which is sometimes referred to as the FPUT paradox. The FPUT problem has a vast literature and we only refer to some of the review articles \cite{Ford1992, Weissert1997, Berman2005,Gallavotti2008,Benettin2013}. Zabusky and Kruskal have explained this paradox by relating this to the closeness of FPUT system to a completely integrable model \cite{Zabusky1965} in the continuum limit known as the Korteweg–de Vries (KdV) equation, which has soliton solutions. Another approach investigated the role of breather solutions \cite{Marin1996,Flach2005,Flach2006,Christodoulidi2010,Danieli2017} (time-periodic and space-localized solutions) in delaying and even preventing thermal behaviour. %By nature these breather solutions are non-thermal and one might expect them to play a role in delaying or even preventing thermalization. 
In a certain sense the idea is similar to the one relating the presence of solitons in the KdV system to the absence of equilibration in the FPUT system $-$ the difference being that breathers are stable solutions of the discrete system, while the KdV is a continuum approximation. %The quasiperiodic behaviour was linked to presence of soliton solutions in the KdV equation. 
A different approach in resolving the FPUT paradox was developed by Chirikov and Izrailev \cite{Chirikov1960, Izrailev1966} on the basis of the criterion of overlapping of resonances, leading to the stochasticity threshold. %The mechanism of chaotic behavior of dynamical systems was found to be an exponential instability of motion for a wide range of initial conditions. 
%The energy equipartition between different normal modes is explained in terms of interacting nonlinear resonances, which lead to an exponential instability of motion for a wide range of initial conditions and hence thermalization.  According to these studies, the initial conditions used by FPUT in their numerical simulations were chosen below the stochasticity threshold, just in the region corresponding to stable quasi-periodic motion. For the harmonic chain, for small values of $k(k \ll N$), the separation between successive levels scales as $\Delta_\omega \sim 1/N$, while for $N-k \ll N$, $\Delta_\omega \sim 1/N^2$. Hence the stochasticity threshold is larger for low frequency modes. Since the FPUT study had initial conditions with the lowest mode excited and the energy density was small, it is plausible that they were below the threshold. Above this threshold, the FPUT model was shown to behave in accordance with the original expectations of FPUT, revealing strong statistical properties such as energy equipartition among the linear modes. 
This idea has also been studied numerically \cite{Livi1985,Deluca1995,Casetti1996}, where attempts were  made to relate thermalization in the system to the untrapping of the system from its regular region and escape to the chaotic component of its phase space. 
%The unexpected recurrences in the FPUT problem have been linked to the choice of initial conditions used by FPUT, which are set close to breather solutions. 
A recent work done on the $\alpha-$FPUT system \cite{Ganapa2020} studied local equilibration in the system, and investigated the role of initial conditions, choice of observables, choice of the averaging protocol and the role of chaos in thermalization. The FPUT paradox has also been studied by using the methods of wave turbulence \cite{Onorato2015,Lvov2018,Pistone2019, Bustamante2019}, which we will discuss after describing the evolution of $\alpha-$FPUT system in a language that is more appropriate to the formalism of wave turbulence.
%\subsection{Wave turbulence}
\subsection{$\alpha-$FPUT system in the language of wave turbulence}

Wave turbulence \cite{Zakharov1992,Nazarenko2011} refers to the statistical theory of weakly nonlinear dispersive waves. %Wave turbulence can be generally defined as out-of-equilibrium statistical mechanics of random nonlinear waves. %The formalism provides a statistical description of the macroscopic behaviour when the number of particles is large, in which case the microscopic dynamics does not provide much analytical insight.
Wave turbulence arises in a wide variety of contexts. For example, wave turbulence arises in surface waves on water (both gravity and capillary) \cite{Dyachenko1994,Dyachenko1995,Brazhnikov2002,Lukaschuk2009,Cobelli2011,Falcon2022}, nonlinear optical systems \cite{Dyachenko1992,Bortolozzo2009}, sound waves in oceanic waveguides \cite{Gurbatov2005}, shock waves in the solar atmosphere and their coupling to the Earth’s magnetosphere \cite{Ryutova2003}, and magnetic turbulence \cite{David2022} in interstellar gases \cite{Bisnovatyi1995}. The formalism of wave turbulence has also been used to explain anomalous conduction in one-dimensional particle lattices \cite{Devita2022}. Recently, the problem of equilibration in the FPUT system has been approached  by using ideas from wave turbulence \cite{Onorato2015,Lvov2018,Pistone2019, Bustamante2019}. The idea of the approach is to  connect the equilibration issue to the presence of high order resonances between dressed normal modes that appear under repeated canonical transformations. %The resonances between the modes are expected to lead to the irreversible transfer of energy, and hence thermalization.
 Let us now study the $\alpha-$FPUT system in the language of wave turbulence. We mostly repeat the calculations given in \cite{Onorato2015} here. First let us describe the $\alpha-$FPUT system in terms of the normal mode coordinates $a_k$:
\begin{equation}
a_k = \frac{1}{\sqrt{2\omega_k}}(P_k-i\omega_kQ_k).
\end{equation} 
%If $\alpha \neq 0$, there will be coupling among the normal modes. 
Writing in terms of the dimensionless variables,
$$ a_k^\prime = \frac{(\mu/m)^{1/4}}{\sqrt{\sum_k \omega_k \mid a_k(t=0)\mid^2 }}a_k,\ t^\prime = \sqrt{\frac{\mu}{m}}t,\ \omega_k^\prime=\sqrt{\frac{m}{\mu}}\omega_k,$$
the Hamiltonian Eq.~\eqref{eq:hamFPU} can be written in terms of the canonical variables $\{ia_k,a_k^\star\}$ (after removing primes for brevity) as:
\begin{align}
\frac{H}{N}=\sum_{k=1}^{N-1}\omega_k a_k^\star a_k +\epsilon\sum_{k_1,k_2,k_3}V_{k_1,k_2,k_3}[\frac{1}{3}(a_{k_1}a_{k_2}a_{k_3}+a_{k_1}^\star a_{k_2}^\star a_{k_3}^\star)\delta_{k_1,-k_2-k_3} \\ +(a_{k_1}^\star a_{k_2}a_{k_3}+a_{k_1} a_{k_2}^\star a_{k_3}^\star)\delta_{k_1,k_2+k_3}]~.
\end{align}
The matrix $V_{k_1,k_2,k_3}$ weights the transfer of energy between wave numbers $k_1$, $k_2$ and $k_3$ and is defined as:
\begin{equation}\label{eq:transfer}
	V_{k_1,k_2,k_3} = -\frac{1}{2\sqrt{2}}\frac{\sqrt{\omega_{k_1}\omega_{k_2}\omega_{k_3}}}{sign(sin(\frac{\pi k_1}{N})sin(\frac{\pi k_2}{N})sin(\frac{\pi k_3}{N}))}~.
\end{equation}
The dimensionless parameter $\epsilon$ is given by:
\begin{equation}\label{eq:nonlinear}
	\epsilon = \frac{\alpha}{m}(\frac{\mu}{m})^{1/4}\sqrt{\sum_k \omega_k \mid a_k(t=0)\mid^2 } ~.
\end{equation}
This leads to the equations of motion:
\begin{equation}\label{eq:evol}
i\frac{\partial a_{k_1}}{\partial t} = \frac{1}{N}\frac{\partial H}{\partial a_{k_1}^*}=\omega_{k_1} a_{k_1} + \epsilon \sum_{k_2,k_3} V_{k_1,k_2,k_3} (a_{k_2}a_{k_3}\delta_{k_1,k_2+k_3}+2a_{k_2}^\star a_{k_3}\delta_{k_1,k_3-k_2}+a_{k_2}^\star a_{k_3}^\star\delta_{k_1,-k_2-k_3})~.
\end{equation}
The delta function is 1 also if the argument differs mod N. % We denote $a_1\equiv a(k_1,t)$ from now on for brevity.
The variables $k_2$ and $k_3$ run from $-N/2+1$ to $N/2$. 
Eq.~\eqref{eq:evol} describes the evolution of the $\alpha-$FPUT system that is more apt to the formalism of wave turbulence. %The interactions between the different $a_k$'s determine the nonlinearity in the system. 
We are now interested in the weak nonlinearity limit, i.e., $\epsilon \ll 1$.

One can observe from Eq.~\eqref{eq:evol} that the nonlinear behaviour of the $\alpha$-FPUT system is due to the three-wave interactions between $a_{k_1}$, $a_{k_2}$ and $a_{k_3}$, with the strength of coupling determined by $\epsilon$.  First we find out if these three waves interactions are resonant. The way to check for resonances is to do a canonical transformation and then try to remove the three wave interactions. If the canonical transformation does not have a zero denominator then we say that the three wave interactions are not resonant. Otherwise we conclude that there are resonances, and that these interactions cannot be removed by any canonical transformation. This canonical transformation is of the form:

\begin{equation}\label{eq:ct}
a_{k_1}=b_{k_1}+\epsilon\sum_{k_2,k_3}(A_{k_1,k_2,k_3}^{(1)}b_{k_2}b_{k_3}\delta_{k_1,k_2+k_3}+A_{k_1,k_2,k_3}^{(2)}b_{k_2}^\star b_{k_3}\delta_{k_1,k_3-k_2}+A_{k_1,k_2,k_3}^{(3)}b_{k_2}^\star b_{k_3}^\star\delta_{k_1,-k_2-k_3})+O(\epsilon^2)~.
\end{equation}
Substituting this in Eq.~\eqref{eq:evol} and equating the coefficient of $\epsilon$ on both the sides, we get:
$$ A_{k_1,k_2,k_3}^{(1)} = V_{k_1,k_2,k_3}/(\omega_{k_3}+\omega_{k_2}-\omega_{k_1}),$$
$$ A_{k_1,k_2,k_3}^{(2)} =2V_{k_1,k_2,k_3}/(\omega_{k_3}-\omega_{k_2}-\omega_{k_1}),$$
$$ A_{k_1,k_2,k_3}^{(3)} =V_{k_1,k_2,k_3}/(-\omega_{k_3}-\omega_{k_2}-\omega_{k_1}).$$
It is easy to verify using trigonometric identities that for the frequencies $\{\omega_k=2\sin(k\pi/N)\} $ of the harmonic chain %$\alpha-$FPUT system (the harmonic chain that is) 
the denominators in the above transformation $\{\omega_{k_3}+\omega_{k_2}-\omega_{k_1}$, $\omega_{k_3}-\omega_{k_2}-\omega_{k_1}$, $\omega_{k_3}+\omega_{k_2}+\omega_{k_1}\}$ are never zero. Therefore, the canonical transformation Eq.~\eqref{eq:ct} from $a$ to $b$ variables is well defined, the three wave interactions are not resonant and can be removed by the canonical transformation Eq.~\eqref{eq:ct}. This lack of three wave resonances has been linked \cite{Onorato2015} to the quasiperiodic behaviour at short time scales observed in the original paper \cite{Fermi1955}. We will come back to this point later in our work. We also point out that the denominators can get very small, although they are never zero. We will postpone further discussion on this until Sec.~\ref{sec:Numerics}. Now substituting the transformation Eq.~\eqref{eq:ct} in the equations of motion Eq.~\eqref{eq:evol}, we get:
\begin{equation}\label{eq:evol2}
i\frac{\partial b_{k_1}}{\partial t} = \omega_{k_1} b_{k_1} + \epsilon^2 \sum_{k_2,k_3,k_4} T_{k_1,k_2,k_3,k_4} b_{k_2}^\star b_{k_3}b_{k_4}\delta_{k_1+k_2,k_3+k_4} +O(\epsilon^3)~.
\end{equation}
The above equation also has terms including the Kronecker deltas $\delta_{k_1,k_3+k_4+k_2}$, $\delta_{k_1,-k_3-k_4-k_2}$ and $\delta_{k_1,k_4-k_3-k_2}$, which are of order $\epsilon^2$. However, those terms are not resonant and can be removed by higher order terms in the transformation, and are of no use in this analysis. The matrix $T_{k_1,k_2,k_3,k_4}$ depends on $V_{k_1,k_2,k_3}$, and the exact expression described in \cite{Dyachenko1994}, is given in Appendix.~\ref{app:freq}.

Eq.~\eqref{eq:evol2} is the starting point of wave turbulence. While Eq.~\eqref{eq:evol2} describes the $\alpha-$FPUT system for the dispersion relation $\omega_k = 2sin(k\pi/N)$, its structure is the same for an arbitrary dispersion relation $\omega = \omega(k)$. For instance,  $\omega_k = \sqrt{gk}$, $k = 1,2,...$ is the dispersion relation of gravity waves and Eq.~\eqref{eq:evol2} in the continuum limit (sometimes known as the Zakharov equation) has been used to argue in support of the integrability of free-surface hydrodynamics in the one-dimensional case \cite{Dyachenko1994}. It has also been used to study the interaction of gravity waves propagating on the surface of an ideal fluid of infinite depth \cite{Dyachenko1995}. 
%We will use Eq.~\eqref{eq:evol2} as the starting point in our work later in order to study quasiperiodicity in the $\alpha-$FPUT system and give an expression for the analytical solution for the evolution of the norm/al modes $a_k$.

Eq.~\eqref{eq:evol2} has also been used to explain the route to thermalization \cite{Onorato2015} in the $\alpha-$FPUT system, which we briefly mention here. Since there are no three wave resonances in the $\alpha-$FPUT system, quasiperiodic behaviour is observed in the system up to times of order $\epsilon^{-4}$ after which, four wave interactions begin to dominate the dynamics. It can then be shown that four wave resonant interactions though present, are isolated from other quartets and cannot spread the energy across the spectrum. The six wave resonant interactions are interconnected, and are the lowest order interactions that lead to an effective irreversible transfer of energy. Using this analysis, the theory of wave turbulence predicted that the equilibration time scales as $\epsilon^{-8}$ for $N=16, 32, 64$, which was also verified numerically for $N=32$ \cite{Onorato2015}.  In the thermodynamic limit, this is predicted to change to $\epsilon^{-4}$ (and verified numerically \cite{Pistone2019}) because the four wave resonances are interconnected in the thermodynamic limit, which does not happen for finite system sizes.


 


\section{Brief overview of our work}
\label{sec:Method}
The main objective of this study is to understand quasiperiodicity in the $\alpha-$FPUT system by using ideas from wave turbulence. %Strictly speaking, we do not directly use the method of wave turbulence in our work because the scope of this work is limited to understanding quasiperiodicity at short time scales up to order $\epsilon^{-4}$, whereas the formalism of wave turbulence is generally applied to describe the statistical behaviour of the system after a time scale $\epsilon^{-4}$. Instead,
Our work aims to understand the origin of quasiperiodic behaviour in the $\alpha-$FPUT system. Fig.~\ref{FPU} shows the observed quasiperiodic recurrence in the  $\alpha-$FPUT system for $N=32$ and $\epsilon = 0.05$ when the first and the last normal modes are excited initially (with amplitudes $a_1=a_{-1}=1i$). This plot closely resembles the behaviour observed in the original paper \cite{Fermi1955}, where the fixed boundary condition case was studied. 
\begin{figure}[ht]
	\centering
	\includegraphics[width=0.7\textwidth]{FPU.png}
	\caption{Quasiperiodic recurrence in the $\alpha-$FPUT problem: Plot shows the evolution of $E_k$ for different values of $k$ when only $k=1$ and $31$ are excited initially for  $N=32$ and $\epsilon = 0.05$.
	The plot closely resembles the behaviour observed by \cite{Fermi1955}, which however used fixed boundary condition. This is the well-known FPUT paradox. %We aim to derive this behaviour by using perturbation theory, which was used to remove the three wave interactions and we finally arrived at Eq.~\eqref{eq:evol2}.
	}
	\label{FPU}
\end{figure}
%We investigate if we can explain this recurrence by using perturbation theory. %, and we cannot apply the formalism of wave turbulence.
%We instead, investigate the question of when the wave turbulence formalism is applicable to the $\alpha-$FPUT system. in terms of the breakdown of perturbation theory. 
%Our main work is to describe quasiperiodicity in the $\alpha-$FPUT system by using first order perturbation theory. Our goal is to check for the breakdown of perturbation theory when applied to the $\alpha-$FPUT system.
We check if we can extract this behaviour from Eq.~\eqref{eq:evol2}. %We derive an expression for the evolution of the normal modes $a_k$ when the system evolves with $\alpha-$FPUT dynamics.
We rewrite Eq.~\eqref{eq:evol2} here for convenience:
$$i\frac{\partial b_{k_1}}{\partial t} = \omega_{k_1} b_{k_1} + \epsilon^2 \sum_{k_2,k_3,k_4} T_{k_1,k_2,k_3,k_4} b_{k_2}^\star b_{k_3}b_{k_4}\delta_{k_1+k_2,k_3+k_4} +O(\epsilon^3)~. $$
We now summarize our method for constructing the evolution of the $\alpha-$FPUT chain for short times (valid up to timescales of order $\epsilon^{-4}$). We are given $a_k(0)$ for all $k$ and we have to find $a_k(t)$. The basic picture that illustrates our method for finding the evolution is given below:
$$ a_k(0)\xrightarrow[\text{transformation}]{\text{canonical}} b_k(0) \xrightarrow[\text{frequency correction}]{\text{compute}}\xrightarrow[\text{evolution}]{i\frac{\partial b_k}{\partial t} \approx \Omega_k b_k}b_k(t)\xrightarrow[\text{transformation}]{\text{canonical}} a_k(t).$$ 

The idea is that $a_k$ are the normal modes of the harmonic chain and so approximating Eq.~\eqref{eq:evol} by $i\frac{\partial a_{k_1}}{\partial t} \approx \omega_{k_1} a_{k_1}$ is not a good one, for this is the evolution equation of the harmonic chain. However, approximating Eq.~\eqref{eq:evol2} by $i\frac{\partial b_{k_1}}{\partial t} \approx \omega_{k_1} b_{k_1}$ is much better because we are neglecting terms of order $\epsilon$ in the former equation and $\epsilon^2$ in the later. Its solution $b_k(t)= b_k(0)e^{-i\omega_k t}$ describes the approximate evolution of the $\alpha-$FPUT system that is valid up to times of order $\epsilon^{-4}$. While computing the evolution of $b_k$ we also have to renormalize the dispersion relation for consistency in the wave turbulence formalism \cite{Zakharov1999,Nazarenko2011,Onorato2020}. If not included in the renormalized dispersion relation, the nonlinear corrections to the frequency would lead to an unphysical secular growth in the evolution. %In other words, we should also take the frequency corrections to the harmonic chain due to the nonlinearity into account. 
So the evolution of $b_k$ is given by: $b_k(t)= b_k(0)e^{-i\Omega_k t}$ where $\Omega_k$ are the frequencies of the $\alpha-$FPUT chain, which are corrections to those of the harmonic chain. We get  corrections to the harmonic frequencies through the $\epsilon^2$ term in Eq.~\eqref{eq:evol2}, which is explained in more detail in Appendix.~\ref{app:freq}. We thus evolve $b_k$ and transform back to $a_k$, instead of evolving $a_k$ directly. We use the canonical transformation from $a_k$ to $b_k$ (Eq.~\eqref{eq:ct}), which we rewrite here after removing the $\delta$ terms:
\begin{equation}\label{eq:ct2}
	a_{k_1}=b_{k_1}+\epsilon\sum_{k_2}(A_{k_1,k_2,k_1-k_2}^{(1)}b_{k_2}b_{k_1-k_2}+A_{k_1,k_2,k_1+k_2}^{(2)}b_{k_2}^\star b_{k_1+k_2}+A_{k_1,k_2,-k_1-k_2}^{(3)}b_{k_2}^\star b_{-k_1-k_2}^\star)+O(\epsilon^2)~.
\end{equation}
Note that up to first order, we can invert the transformation Eq.~\eqref{eq:ct2} to:
\begin{equation}\label{eq:ct3}
	b_{k_1}=a_{k_1}-\epsilon\sum_{k_2}(A_{k_1,k_2,k_1-k_2}^{(1)}a_{k_2}a_{k_1-k_2}+A_{k_1,k_2,k_1+k_2}^{(2)}a_{k_2}^\star a_{k_1+k_2}+A_{k_1,k_2,-k_1-k_2}^{(3)}a_{k_2}^\star a_{-k_1-k_2}^\star)+O(\epsilon^2)~.
\end{equation}
 
We use the term perturbation theory to describe our method in this work because the construction of quasiperiodicity is being done by using the canonical transformation used to remove the non resonant interactions, which is a perturbation series. As mentioned in the introduction, we use the perturbation series up to first order in our computations (they remove only the three wave interactions), and discuss about higher order terms in Sec.~\ref{sec:Numerics}. We then compare the evolution of the normal mode energies $E_k = \omega_k\mid a_k\mid^2$ obtained by this method with that of molecular dynamics simulations obtained by solving the Hamilton's equations. %predictions of perturbation theory %(we ignore the higher order terms in our analysis, but we defend our omission later) 
%with that of Hamilton's equations of motion for thefor different $\epsilon$ (with $\epsilon\ll 1$).
%If there is a good agreement, then the observed quasiperiodic behaviour can indeed be attributed to the lack of three wave resonances. Otherwise, we relate the breakdown of agreement between perturbation theory and Hamiltonian dynamics to the breakdown of perturbation theory, and argue that Fig.~\ref{FPU} is the consequence of strong nonlinearity effects. We then study the onset of the breakdown in detail, and its relation to the system size. We finally relate our results to the work on problem of thermalization in the $\alpha-$FPUT system \cite{Onorato2015}, which used the formalism of wave turbulence. % We would like to point out that we compute the evolution by using first order perturbation theory and yet, provide conclusive evidence later that we don't have to consider higher order terms in order to expect  a better agreement.





 


\section{Analytical solution of the quasiperiodicity}
\label{sec:Solution}
%Let us now describe our set up.         
Let us construct an expression for quasiperiodicity in the $\alpha-$FPUT system. We initially excite only the first and the last modes.
This is done in order to minimize the number of non zero terms in the sums of Eq.~\eqref{eq:ct2} and Eq.~\eqref{eq:ct3} (the authors of the original FPUT paper \cite{Fermi1955} excited only the first normal mode, and considered fixed boundary condition). So at $t=0$, $a_1(0)$ and $a_{-1}(0)$ are non zero (by $-k$, we mean the $(N-k)^{th}$ mode because of periodic boundary conditions) and the rest of them are zero. %(by $a_{-k}$, we mean $a_{N-k}$).
%Here $a_{-1} = a_{N-1}$ because of the periodic boundary conditions.
Let us now construct an expression for $a_k(t)$ up to first order. For that we have to first evaluate $b_k(0)$ using Eq.~\eqref{eq:ct3}. %Let us do this by first removing the $\delta$ terms in Eq.~\eqref{eq:ct2} and rewriting it as:

%We would like to remind that we are writing $a_1$ for $a_{k_1}$ and similarly for $b_i$. 
For $k_1=1$ one can notice that all the sums in Eq.~\eqref{eq:ct3} are zero since only $a_1$ and $a_{-1}$ are excited originally. For if $k_2 = 1$ then $k_1-k_2 =0$, which is the zero mode that is omitted from the dynamics. We also have $k_1+k_2 =2$ and $-k_1-k_2=-2$, both of which are not excited initially. %and so we have $a_k(0) = 0$ for these wave numbers.
Similarly for $k_2 = -1$ we have $k_1-k_2 =2$ (not excited initially), $k_1+k_2 =0$ and $-k_1-k_2=0$ (zero mode). We can do a similar analysis for $k_1=-1$ and infer that all the sums in Eq.~\eqref{eq:ct3} are zero. We thus have:
$b_1(0) = a_1(0)$, $b_{-1}(0)=a_{-1}(0)$. Similarly, one observes that $b_{k_1} = 0$ for all other values of $k_1$ except $k_1=2$ or $-2$. Note that this argument is valid only because we are computing $b_{k_1}$ up to first order. 

Now consider $k_1 = 2$. For $k_2 = 1$, it is possible to have $k_1-k_2 = 1$ but not $k_1 + k_2 =3$ and $-k_1-k_2 = -3$ (since they are not excited originally). For $k_2 = -1$ it is possible to have $k_1 + k_2 =1$ and $-k_1-k_2 = -1$ but not $k_1-k_2=3$. Thus, there will be three nonzero sums in Eq.~\eqref{eq:ct3} for $k_1=2$. In the same way, one can calculate that for $k_1=-2$, there are three nonzero sums in Eq.~\eqref{eq:ct3} $-$ for $k_2 = 1$, these are $-k_1-k_2=1$, $k_1+k_2=-1$ and for $k_2=-1$ it is $k_1-k_2=-1$. Thus, we get at $t=0$ the following expressions for $b_2$ and $b_{-2}$:
$$ b_2(0) \approx -\epsilon[A_{2,1,1}^{(1)}(a_1(0))^2+A_{2,-1,1}^{(2)}a_{-1}^\star(0) a_1(0)+A_{2,-1,-1}^{(3)}(a_{-1}^\star(0))^2]$$
$$ b_{-2}(0) \approx -\epsilon[A_{-2,-1,-1}^{(1)}(a_{-1}(0))^2+A_{-2,1,-1}^{(2)}a_{1}^\star(0) a_{-1}(0)+A_{-2,1,1}^{(3)}(a_{1}^\star(0))^2]$$
The purpose of computing $b_k(0)$ is two-fold. First, we can find the  frequencies of the $\alpha-$FPUT chain. The procedure to find corrections to the harmonic frequencies is explained in Appendix.~\ref{app:freq}. And second, we use $b_k(0)$ to find $a_k(t)$ through $b_k(t)$. Now we evolve $b_k$ in time as $b_k(t)=b_k(0)e^{-i\Omega_k t}$. Then using the canonical transformation Eq.~\eqref{eq:ct2} from $b_k$ to $a_k$, we get the evolution of $a_k$. Note that while transforming back to $a_k(t)$ we also have to consider the contributions of $b_2(t)$ and $b_{-2}(t)$ that are nonzero in addition to $b_1(t)$ and $b_{-1}(t)$. 
By doing a similar analysis that we have done in the previous paragraphs, we get the following expressions for the evolution of $a_k$ (accurate up to first order):
%\begin{align}\label{eq:evolak}
%a_1(t) = a_1(0)e^{-i\omega_1 t} + \epsilon [2A_{1,-1,2}^{(1)}b_{-1}(0) b_{2}(0) e^{-i(\omega_1+\omega_2)t}+2A_{1,1,-2}^{(3)}b_{1}^\star(0) b_{-2}^\star (0) e^{i(\omega_1+%\omega_2)t}+A_{1,1,2}^{(2)}b_{1}^\star(0) b_{2}(0) e^{i(\omega_1-\omega_2)t}\\+A_{1,-2,-1}^{(2)}b_{-2}^\star(0) b_{-1}(0) e^{i(\omega_2-\omega_1)t}] \\
%a_2(t) = b_2(0)e^{-i\omega_2 t}+\epsilon[A_{2,1,1}^{(1)}b_{1}(0) b_{1}(0)e^{-2i\omega_1t}+A_{2,-1,1}^{(2)}b_{-1}^\star(0) b_{1}(0)+A_{2,-1,-1}^{(3)}b_{-1}^\star(0) b_{-1}%^\star(0)e^{2i\omega_1t}]\\
%a_{-2}(t) = b_{-2}(0)e^{-i\omega_2 t}+\epsilon[A_{-2,-1,-1}^{(1)}b_{-1}(0) b_{-1}(0)e^{-2i\omega_1t}+A_{-2,1,-1}^{(2)}b_{1}^\star(0) b_{-1}(0)+A_{-2,1,1}^{(3)}b_{1}^\star(0) %b_{1}^\star(0)e^{2i\omega_1t}]\\
%a_{-1}(t) = a_{-1}(0)e^{-i\omega_1 t} + \epsilon [2A_{-1,1,-2}^{(1)}b_{1}(0) b_{-2}(0) e^{-i(\omega_1+\omega_2)t}+2A_{-1,-1,2}^{(3)}b_{-1}^\star(0) b_{2}^\star (0) %e^{i(\omega_1+\omega_2)t}+A_{-1,-1,-2}^{(2)}b_{-1}^\star(0) b_{2}(0) e^{i(\omega_1-\omega_2)t}\\+A_{-1,2,1}^{(2)}b_{2}^\star(0) b_{1}(0) e^{i(\omega_2-\omega_1)t}].~ \\
%\end{align}
%Thus, $\alpha-$FPUT system evolves as:
\begin{subequations}\label{eq:evolak}
\begin{equation}
\begin{split}
a_1(t) \approx a_1(0)e^{-i\Omega_1 t} + \epsilon [2A_{1,-1,2}^{(1)}a_{-1}(0) b_{2}(0) e^{-i(\Omega_1+\Omega_2)t}+2A_{1,1,-2}^{(3)}a_{1}^\star(0) b_{-2}^\star (0) e^{i(\Omega_1+\Omega_2)t}+\\A_{1,1,2}^{(2)}a_{1}^\star(0) b_{2}(0) e^{i(\Omega_1-\Omega_2)t}+A_{1,-2,-1}^{(2)}b_{-2}^\star(0) a_{-1}(0) e^{i(\Omega_2-\Omega_1)t}] 
\end{split}
\end{equation}
\begin{equation}
a_2(t) \approx b_2(0)e^{-i\Omega_2 t}+\epsilon[A_{2,1,1}^{(1)}a_{1}(0) a_{1}(0)e^{-2i\Omega_1t}+A_{2,-1,1}^{(2)}a_{-1}^\star(0) a_{1}(0)+A_{2,-1,-1}^{(3)}a_{-1}^\star(0) a_{-1}^\star(0)e^{2i\Omega_1t}]
\end{equation}
\begin{equation}
a_{-2}(t) \approx b_{-2}(0)e^{-i\Omega_2 t}+\epsilon[A_{-2,-1,-1}^{(1)}a_{-1}(0) a_{-1}(0)e^{-2i\Omega_1t}+A_{-2,1,-1}^{(2)}a_{1}^\star(0) a_{-1}(0)+A_{-2,1,1}^{(3)}a_{1}^\star(0) a_{1}^\star(0)e^{2i\Omega_1t}]
\end{equation}
\begin{equation}
\begin{split}
a_{-1}(t) \approx a_{-1}(0)e^{-i\Omega_1 t} + \epsilon [2A_{-1,1,-2}^{(1)}a_{1}(0) b_{-2}(0) e^{-i(\Omega_1+\Omega_2)t}+2A_{-1,-1,2}^{(3)}a_{-1}^\star(0) b_{2}^\star (0) e^{i(\Omega_1+\Omega_2)t}+\\A_{-1,-1,-2}^{(2)}a_{-1}^\star(0) b_{-2}(0) e^{i(\Omega_1-\Omega_2)t}+A_{-1,2,1}^{(2)}b_{2}^\star(0) a_{1}(0) e^{i(\Omega_2-\Omega_1)t}],~
\end{split} 
\end{equation}
\end{subequations}
where $\Omega_k$ are the frequencies of the $\alpha-$FPUT chain. % obtained by second order perturbation theory. %The procedure to derive $\Omega_k$ is derived in the appendix. 
These terms lead to the normal mode energies $E_k = \omega_k\mid a_k\mid^2$ with amplitude of order $\epsilon^2$ for $k=1,2,-1$ and $-2$. %For $k=1$ the amplitude of $E_1$ would be of order $\epsilon^2$.  
We don't mention other $a_k$ here because they are much smaller in magnitude. $E_3$ and $E_{-3}$ would have an amplitude of order $\epsilon^4$. Similarly $E_4$ and $E_{-4}$ would have an amplitude of order $\epsilon^6$, which are also not mentioned here. Note that we have to include terms up to order $\epsilon^2$ in Eqs.~\eqref{eq:ct2}, \eqref{eq:ct3} in order to get accurate expressions for $E_3$ and $E_{-3}$ (and also $E_4$ and $E_{-4}$), which lead to more complicated expressions for $a_k(t)$. %For instance, consider $E_3(t)$. We have to include the term $\epsilon^2 \sum U_{k_1,k_2,k_3,k_4}a_{k_2}^\star a_{k_3}a_{k_4}\delta_{k_1+k_2,k_3+k_4} $ in Eq.~\eqref{eq:ct3}, where $U_{k_1,k_2,k_3,k_4}$ is some function of $k_1,k_2,k_3$ and $k_4$. Since only $k=1$ and $k=-1$ are excited initially, the only non zero term in the sum corresponds to $k_1=3$, $k_2=-1$ and $k_3=k_4=1$. The contribution of this term is thus, of order $\epsilon^2$. Similarly, in Eq.~\eqref{eq:ct2} $b_1$ is of order $1$ and $b_2$ is of order $\epsilon$. Hence, we have for $k_1=3$, $k_2=1$, $k_1-k_2=2$ and $\epsilon A_{3,1,2}^{(1)}b_{1}b_{2}$ will be of order $\epsilon^2$. We thus finally get $E_3=\omega_3\mid a_3\mid^2$ to be of order $\epsilon^4$. 
Finally, note that the modes $k = 5,6,...$ and $k=-5,-6,...$ are not excited if we calculate Eqs.~\eqref{eq:ct2}, \eqref{eq:ct3} only up to first order. 

\section{Numerical results}
\label{sec:Numerics}
We now discuss the numerical results. We initially excite the system to the first and the last normal modes of the harmonic chain (with amplitudes $a_1 = a_{-1} = 1i$ and $a_k = 0$ for other $k$) and then compare Eqs.~\eqref{eq:evolak} with the %evolution of the $\alpha-$FPUT chain. We evolve the 
Hamilton's equations of motion of the $\alpha-$FPUT chain evolved by using a sixth order symplectic integrator \cite{Yoshida1990}. The time-step size is taken to be $0.01$. %The relative energy change for this system at the end of the computation has been verified in \cite{Ganapa2020}.  
Fig.~\ref{1601} shows the evolution of the first two normal modes $E_1 = \omega_1\mid a_1\mid^2$ and $E_2= \omega_2\mid a_2\mid^2$ with system size $N=16$ and $\epsilon = 0.01$. The evolution of Hamilton's equations is plotted in black and the results of perturbation theory Eq.~\eqref{eq:evolak} are shown in magenta. We observe a good agreement between the two.%For the harmonic chain,  $E_1$ and $E_2$ don't change with time but they do for the  $\alpha-$FPUT chain. %We see the quasiperiodic behaviour and also notice that Eqs.~\eqref{eq:evolak} capture the quasiperiodicity very well. 
%, consistent with the Hamiltonian evolution.
\begin{figure}[ht]
	\centering
	\hspace{-35mm}
	\includegraphics[width=0.5\textwidth]{1601_1.png}
	\put (-110,189) {$(a)$}
	%\hspace{1mm}
	\includegraphics[width=0.5\textwidth]{1601_2.png}
	\put (-110,189) {$(b)$}
	\hspace{-40mm}
	\caption{Quasiperiodicity in the $\alpha-$FPUT chain: Plots show the time evolution of $E_1$ (left panel) and $E_2$ (right panel) starting from $a_1,a_{-1} = 1i$ and $a_k = 0$ for other values of $k$. Parameter values for this plot are $N=16$ and $\epsilon= 0.01$. The evolution of Hamilton's equations is plotted in black and the results of perturbation theory Eqs.~\eqref{eq:evolak} are shown in magenta. To a good approximation, $E_1$ and $E_2$ can be seen to oscillate with frequency $2\Omega_1-\Omega_2$.}
	\label{1601}
\end{figure}


\subsection{Dependence on the frequency corrections}
We now demonstrate that it is indeed important to take the frequency corrections into account. We replot Fig.~\ref{1601} for shorter time in Fig.~\ref{frequencycorrection}. This time we also plot Eqs.~\eqref{eq:evolak} with $\Omega_k$ replaced by $\omega_k$, the harmonic frequencies. This is shown in red. The insets show that computing the frequency corrections lead to a noticeably better agreement between Hamiltonian evolution and perturbation theory. This is expected because if we don't take the frequency corrections into account then we get the same frequency of oscillations for all values of $\epsilon$ for which the wave turbulence formalism is applicable. %(which are still small enough so that the wave turbulence formalism is still applicable).
 But we know that as $\epsilon$ is decreased the quasiperiodic frequency should reduce, and so the time period should increase. This is captured by the frequency corrections and so we replace $\omega_k$ with $\Omega_k$. The procedure to calculate the frequency corrections is described in  Appendix.~\ref{app:freq}. % However, if we use $\omega_k$ instead of $\Omega_k$ then we see that the exponentials in Eqs.~\eqref{eq:evolak} have not taken into account the nonlinearity parameter, and therefore the frequency of oscillations do not change for different values of $\epsilon$.
 A more serious problem that would arise by not including the frequency corrections is that these terms that are not included in the frequency correction would then contribute to the dynamics, and these lead to a secular growth as described in \cite{Zakharov1999,Nazarenko2011,Onorato2020}. In the standard perturbation theory such a technique is known as the Poincare-Lindstedt method. 
 
\begin{figure}[ht]
	\centering
	\hspace{0mm}
	\includegraphics[width=0.5\textwidth]{1601_1_1.png}
	\put (-105,250) {$(a)$}
	%\hspace{-10mm}
	\includegraphics[width=0.5\textwidth]{1601_2_1.png}
	\put (-105,250) {$(b)$}
	\hspace{0mm}
	\caption{Dependence on the frequency correction: Same as Fig.~\ref{1601} but for shorter time. Here we illustrate the importance of the frequency corrections. The evolution of Hamilton's equations is plotted in black and the results of perturbation theory Eqs.~\eqref{eq:evolak} are shown in magenta. The result of perturbation theory Eqs.~\eqref{eq:evolak} when $\Omega_k$ is replaced by $\omega_k$ is plotted in red. The insets show the importance of computing the frequency corrections to the harmonic chain.}
	\label{frequencycorrection}
\end{figure}

\subsection{Dependence on the nonlinearity  parameter}
We now move on to study how our method captures the quasiperiodic behaviour for larger values of $\epsilon$. This is shown in Fig.~\ref{16particles} where we plot the evolution of $E_1$ on the left panel and $E_2$ on the right for $N=16$ and $\epsilon = 0.05$. We see that the agreement between Hamiltonian dynamics and perturbation theory is poor now. In fact, there is a breakdown of agreement for $E_1$ even at very small times. This is not expected to happen for $\epsilon = 0.05$ as $0.05$ is still a small perturbation. %may not look surprising at first glance because perturbation theory is expected to work only for small nonlinearities $\epsilon \ll 1$. 
We show in Sec.~\ref{ssec:Math} that the presence of small denominators in Eqs.~\eqref{eq:ct2},~\eqref{eq:ct3} would lead to further restrictions on $\epsilon$.% A_{k_1,k_2,k_3}^{(i)}\ll1$ for $i=1,2,3$. We will now discuss this in more detail.% in order to understand what happens to 
%$\epsilon A_{k_1,k_2,k_3}^{(i)}$ as we change the system size.
\begin{figure}[ht]
	\centering
	\hspace{-35mm}
	\includegraphics[width=0.5\textwidth]{1605_1.png}
	\put (-110,189) {$(a)$}
	%\hspace{1mm}
	\includegraphics[width=0.5\textwidth]{1605_2.png}
	\put (-110,189) {$(b)$}
	\hspace{-40mm}
	\caption{Dependence on the nonlinearity parameter $\epsilon$: %Now we repeat the calculations for higher values of $\epsilon$.
	Plots show the time evolution of $E_1$ (left panel) and $E_2$ (right panel) for $\epsilon = 0.05$ starting from the same initial condition as Fig.~\ref{1601} with $N = 16$. We now observe a breakdown of agreement between Hamiltonian dynamics and perturbation theory for $E_1$ from early times, and also a poorer agreement in the evolution of $E_2$.}
	\label{16particles}
\end{figure}
\subsection{Dependence on the system size}
We now study how our method captures the quasiperiodic behaviour for larger system size. Fig.~\ref{32particles} shows the time evolution of $E_2$ for $\epsilon = 0.01$ (left panel) and $\epsilon = 0.05$ (right panel) for $N=32$ starting from the same initial condition as Fig.~\ref{1601}. We see that $\epsilon = 0.01$ is also not small enough to get a good agreement for $N=32$. From the right panel we see that the quasiperiodic behaviour (Fig.~\ref{FPU}) in the $\alpha-$FPUT chain that resembles the plot in the original paper \cite{Fermi1955} is not captured by perturbation theory. We now explain this lack of agreement by pointing out that the small denominators in Eqs.~\eqref{eq:ct2},~\eqref{eq:ct3} will become more pronounced as the system size is increased.
\begin{figure}[ht]
	\centering
	\hspace{-35mm}
	\includegraphics[width=0.5\textwidth]{3201_2.png}
	\put (-110,189) {$(a)$}
	%\hspace{1mm}
	\includegraphics[width=0.5\textwidth]{3205_2.png}
	\put (-110,189) {$(b)$}
	\hspace{-40mm}
	\caption{Dependence on the system size $N$: Plots show the time evolution of $E_2$ for $\epsilon = 0.01$ (left panel) and $\epsilon = 0.05$ (right panel) starting from the same initial condition as Fig.~\ref{1601} for $N=32$. We now observe a poorer agreement  even for $\epsilon = 0.01$. From the right panel we see that the quasiperiodic behaviour in the $\alpha-$FPUT chain that is observed in Fig.~\ref{FPU} is not at all captured by perturbation theory.}
	\label{32particles}
\end{figure}

\subsection{Evidence of divergence in the canonical transformation}
\label{ssec:Math}
We now relate the observed discrepancies in the agreement between perturbation theory and Hamilton's equations to the presence of small denominators in the canonical transformations Eqs.~\eqref{eq:ct2},~\eqref{eq:ct3}.
For the initial conditions that we have considered, one can observe that as the system size is increased, terms like $\epsilon A_{1,1,2}^{(2)}$ in Eqs.~\eqref{eq:ct2},~\eqref{eq:ct3} and the solution Eqs.~\eqref{eq:evolak} become large even for smaller $\epsilon$ because of the small denominator in $A_{1,1,2}^{(2)}$. This implies that the higher order terms in Eqs.~\eqref{eq:ct2},~\eqref{eq:ct3}, which we have neglected while deriving Eqs.~\eqref{eq:evolak}, become non-negligible when compared to (sometimes even larger than) the lower order terms. This can clearly be noticed by considering higher order terms in Eqs.~\eqref{eq:ct2}. These terms are of the form $\epsilon^2 \sum B_{k_1,k_2,k_3,k_4}b_{k_2}^\star b_{k_3}b_{k_4}\delta_{k_1+k_2,k_3+k_4} $, where $ B_{k_1,k_2,k_3,k_4}$ is given by \cite{Dyachenko1994}:
\begin{equation}\label{eq:B}
\begin{split}
B_{k_1,k_2,k_3,k_4} = A_{k_2,k_3,k_2-k_3}^{(1)}A_{k_4,k_1,k_4-k_1}^{(1)}+ A_{k_2,k_4,k_2-k_4}^{(1)}A_{k_3,k_1,k_3-k_1}^{(1)}- A_{k_1,k_3,k_1-k_3}^{(1)}A_{k_4,k_2,k_4-k_2}^{(1)}- A_{k_2,k_4,k_2-k_4}^{(1)}A_{k_3,k_2,k_3-k_2}^{(1)}\\- A_{k_1+k_2,k_1,k_2}^{(1)}A_{k_3+k_4,k_3,k_4}^{(1)}+ A_{-k_1-k_2,k_1,k_2}^{(3)}A_{-k_3-k_4,k_3,k_4}^{(3)}
\end{split}	
\end{equation}
For $(k_1,k_2,k_3,k_4)=(1,2,1,2)$ we can see that the first term in Eq.~\eqref{eq:B} leads to a term $\sim(\epsilon A_{2,1,1}^{(1)})^2$, which is the square of a small denominator. This indicates that there could be divergences in the canonical transformation, and a possible breakdown of perturbation theory.  %(since $\mid A_{1,1,2}^{(2)}\mid=\mid A_{2,1,1}^{(1)}\mid$). %Thus for higher order terms we get $(\epsilon A_{1,1,2}^{(2)})^n$ and 
Thus, we argue that the divergences in the canonical transformation can be avoided only if Eqs.~\eqref{eq:ct2},~\eqref{eq:ct3} remain bounded. This is possible if $\epsilon A_{1,1,2}^{(2)}\ll1$. Hence the perturbation series Eqs.~\eqref{eq:ct2},~\eqref{eq:ct3} are expected to converge only if $\epsilon A_{1,1,2}^{(2)}\ll1$. We point out that there will be a good agreement between the evolution of the $\alpha-$FPUT chain obtained by perturbation theory and by evolving Hamilton's equations (such as in Fig.~\ref{1601}) when $\epsilon$ is small enough so that $\epsilon A_{1,1,2}^{(2)}\ll 1$. 
Thus for system size $N$ we expect Eqs.~\eqref{eq:ct2},~\eqref{eq:ct3} to remain valid only if:
\begin{equation}\label{eq:breakdown}
	\epsilon\frac{\sqrt{sin^2(\pi/N)sin(2\pi/N)}}{2sin(\pi/N)-sin(2\pi/N)}\ll1~.
\end{equation}
For $N=8$ the limitation becomes $\epsilon\ll0.18$, while for $N=16$ this becomes $\epsilon\ll0.06$ and for $N=32$ it becomes $\epsilon\ll0.02$. As the system size $N$ increases Eqs.~\eqref{eq:ct2},~\eqref{eq:ct3} are expected to converge only if $\epsilon \ll O(N^{-1.5})$. Hence we have to move on to smaller and smaller $\epsilon$ in the thermodynamic limit in order for Eqs.~\eqref{eq:ct2},~\eqref{eq:ct3} to be convergent. Otherwise the higher order terms would be of the order of (or even larger than) the lower order terms, and perturbation theory is expected to break down. 
Note that terms such as $A_{1,-1,2}^{(1)}$ and $A_{1,1,-2}^{(3)}$ do not pose any further restrictions on the nonlinearity parameter because the denominators are not small enough when compared to that of $A_{1,1,2}^{(2)}$. %At the same time, we cannot attribute this to higher order terms in Eqs.~\eqref{eq:ct2},~\eqref{eq:ct3} as we have shown indications that the perturbation series diverges as we have $\epsilon A_{1,1,2}^{(2)}\gtrsim 1$.



Even though we have given indications that the perturbation series Eqs.~\eqref{eq:ct2},~\eqref{eq:ct3} diverges if $\epsilon A_{1,1,2}^{(2)}$ is large enough, we have not proved it mathematically. This is beyond the scope of this study. It may well be argued that all the small denominators cancel in all the higher order terms, and we end up with a convergent series whenever $\epsilon \ll 1$. We now show numerical evidence that this is not the case, and that the small denominators in the higher order do not cancel. We plot in Fig.~\ref{Breakdown} the amplitude of oscillations of $E_2$ and $E_3$ at very short times as a function of $\epsilon$ after evolving the $\alpha-$FPUT system using a symplectic integrator. We have considered the same initial condition as our other plots. 
\begin{figure}[ht]
	\centering
	\hspace{-35mm}
	\includegraphics[width=0.5\textwidth]{E2amp.png}
	\put (-110,189) {$(a)$}
	%\hspace{1mm}
	\includegraphics[width=0.5\textwidth]{E3amp.png}
	\put (-110,189) {$(b)$}
	\hspace{-40mm}
	\caption{Breakdown of perturbation theory for different $N$: Here we plot the amplitude of $E_2$ (left panel) and $E_3$ (right panel) at very short times for different $\epsilon$ after numerically solving the Hamilton's equations of motion starting from the same initial condition as Fig.~\ref{1601}. The slope of the lines in the left panel is $2$, while it is $4$ in the right panel. %In the left panel we observe that for small $\epsilon$ the amplitudes of $E_2$ fit to a line of slope $2$ on a log-log plot. In the right panel we observe that the amplitudes of $E_3$ fit to a line of slope $4$ on a log-log plot.% We see from both the panels that there is a deviation from the straight line at higher $\epsilon$, which is the regime where perturbation theory cannot be used to remove the three wave interactions, and hence we cannot use the wave turbulence formalism. We note that as the system size increases the breakdown happens at a smaller $\epsilon$, consistent with our estimates.
	We can clearly see that the parameters $N=32$ and $\epsilon=0.05$ in Fig.~\ref{FPU} correspond to the strong nonlinearity regime.}
	\label{Breakdown}
\end{figure}
Here are some of the interesting observations:
\begin{itemize}
	\item [--] From the left (right) panel we observe that for small values of $\epsilon$ the amplitudes of $E_2$ ($E_3$) fit to a straight line of slope $2$ ($4$) on a log-log plot, confirming our calculations (predictions) of the $\epsilon^2$ ($\epsilon^4$) dependence of the amplitude of $E_2$ ($E_3$ respectively). %Similarly, from the right panel we notice that for small values of $\epsilon$ the amplitudes of $E_3$ fit to a straight line of slope $4$ on a log-log plot, once again confirming our predictions about the $\epsilon^4$ dependence of the amplitude of $E_3$.
	\item[--] We also find from both the panels that for larger $\epsilon$ the amplitudes deviate from the straight line showing clearly where the higher order terms are not negligible.
	\item[--] %It is interesting to note from both the plots that 
	Deviation from straight lines start to become noticeable when $\epsilon A_{1,1,2}^{(2)}$ is comparable to $1$, providing evidence that small denominators in Eqs.~\eqref{eq:ct2},~\eqref{eq:ct3} do not cancel in higher order terms. Had the terms with small denominators been cancelled, the amplitudes would have scaled as $\epsilon^{2}$ for $E_2$ and $\epsilon^{4}$ for $E_3$ even for those $\epsilon\ll1$ for which $\epsilon A_{1,1,2}^{(2)}$ is large enough. 
	\item[--] And finally note that as the system size is increased the breakdown in agreement is observed at a lower $\epsilon$, consistent with our estimates. 
	
\end{itemize}
This deviation from the straight line shows evidence that for $\epsilon\ll1$ such that $\epsilon A_{1,1,2}^{(2)}$ is large enough, the contribution of the higher order terms in the canonical transformations Eqs.~\eqref{eq:ct2},~\eqref{eq:ct3} is as large as (or even larger than) the lower order terms. This leads to the divergence of the canonical transformations. Thus, we provide evidence of a strong nonlinearity regime where perturbation theory cannot be applied to remove the three wave interactions, and we cannot arrive at Eq.~\eqref{eq:evol2}, which is the starting point of wave turbulence.  We see that we have not escaped the problem of small divisors completely yet. Interestingly, some comments on the mathematical challenges of wave turbulence when applied to the $\alpha-$FPUT system have been made in some of the earlier works done on the subject (for instance \cite{Kramer2002,Biello2002}).

%We also point out that when $\epsilon A_{1,1,2}^{(2)}$ is large enough we cannot attribute this disagreement between perturbation theory and Hamiltonian dynamics in (Fig.~\ref{Breakdown}) to higher order terms (of order $\epsilon^2$) in Eq.~\eqref{eq:evol2} when  $\epsilon A_{1,1,2}^{(2)}$ is comparable to $1$ because that would lead to an incorrect inference that four wave interactions are acting at shorter timescales, since this cannot happen in a system that has quasiperiodicity.

Finally, note from Fig.~\ref{Breakdown} that $\epsilon = 0.05$ is in the strong nonlinearity regime for $N=32$ (shown in Fig.~\ref{FPU}, which was observed in the original paper \cite{Fermi1955}). In the weak nonlinearity regime where perturbation theory is applicable, the modes $k=1$ and $-1$ excited initially can only transfer energy up to four modes, and the fifth mode cannot get excited (as we found in Sec.~\ref{sec:Solution}). However, the fifth mode is found to have energy (as shown in Fig.~\ref{FPU}) and this gives us the initial clue that we may be in the strong nonlinearity regime for $N=32$ and $\epsilon = 0.05$.





%In our method we have initially made a canonical transformation from $a_k$ to $b_k$ and considered corrections upto first order (Eq.~\eqref{eq:ct3}). So, we have made an error of order $\epsilon^2$. While transforming back from $b_k$ to $a_k$ at later times (Eq.~\eqref{eq:ct2}) we have again made an error of order $\epsilon^2$. We now make arguments and show numerical evidence to defend our method. 


%wave turbulence, and hence we have numerical evidence that this quasiperiodicity is not merely due to the absence of three wave resonances. 





\section{Conclusion}
\label{sec:Conclusion}
In this work we studied the cause of quasiperiodicity in the weakly nonlinear $\alpha-$FPUT chain by using ideas from wave turbulence framework. Our aim is to check if the quasiperiodic behaviour observed in the $\alpha-$FPUT system (such as the one in Fig.~\ref{FPU}, which resembles the plot observed in the original paper \cite{Fermi1955}) can be captured in the wave turbulence regime. We used the 
canonical transformation Eq.~\eqref{eq:ct2} used to remove the three wave interactions to construct the solution Eqs.~\eqref{eq:evolak} of the evolution of $a_k$ (given by $a_k = \frac{1}{\sqrt{2\omega_k}}(P_k-i\omega_kQ_k)$) when only $a_1$ and $a_{-1}$ are excited initially.  We also used the wave turbulence formalism to compute the frequencies of the $\alpha-$FPUT chain, which are  perturbative corrections to that of the harmonic chain, in order to construct the evolution even more accurately. We then compared our results with that of the molecular dynamics simulations obtained by solving the Hamilton's equations of motion.
 %The expectation is that if there is a good agreement, then the  observed quasiperiodic behaviour could be attributed to the lack of three wave resonances, and we could understand quasiperiodicity as arising due to the change of variables. %Otherwise, we relate the breakdown of agreement between perturbation theory and Hamiltonian dynamics to the breakdown of perturbation theory, and argue that Fig.~\ref{FPU} is the consequence of strong nonlinearity effects. We then study the onset of the breakdown in detail, and its relation to the system size. 
Our work can be summarized by linking the following observations:
\begin{enumerate}
	\item For the initial conditions that we have considered, we find that the agreement between perturbation theory and Hamiltonian dynamics is good only for $\epsilon A_{1,1,2}^{(2)}\ll1$. This leads to $\epsilon \ll O(N^{-1.5})$, where $N$ is the system size.
	\item There is a small denominator in $A_{1,1,2}^{(2)}$, that is more pronounced as the system size increases. We show that small denominators are also present in the higher order terms in the canonical transformation Eq.~\eqref{eq:ct2}, and perturbation theory is expected to be valid only when  $\epsilon \ll O(N^{-1.5})$. 
	\item Using the molecular dynamics simulations we plotted the amplitude of $E_k = \omega_k\mid a_k\mid^2$ %, the second and third normal mode energies
	as a function of $\epsilon$ for $k=2,3$ at very short times (Fig.~\ref{Breakdown}). We find that there is a deviation from the straight line that is noticeable when $\epsilon A_{1,1,2}^{(2)}$ is comparable to $1$, which demonstrates that the small denominators do not cancel each other in higher orders, which makes a strong point for our claim that the canonical transformation Eq.~\eqref{eq:ct2} could diverge, and we are in the strong nonlinearity regime.
	\item The quasiperiodic behaviour in Fig.~\ref{FPU}, which resembles the plot observed in the original paper \cite{Fermi1955} corresponds to the nonlinearity regime where the small denominators do not cancel each other in higher orders.
	
	
	 %The point of departure from the straight line in  signifies the breakdown of perturbation theory where the formalism of wave turbulence cannot be applied. We infer that the quasiperiodic behaviour observed in the original paper \cite{Fermi1955} is due to strong nonlinearity effects.
	%\item[--] Surprisingly, we find that %even in the strong nonlinearity regime (where wave turbulence fails to capture the quasiperiodicity) % (such as the one observed in \cite{Fermi1955})
	%there is $1/\epsilon^8$ dependence of equilibration time for $N=32$, which is a prediction of the wave turbulence formalism. So, 
	%the equilibration time scales as $1/\epsilon^8$ even in the strong nonlinearity regime for $N = 32$ where the wave turbulence formalism is not applicable
	%\item[--]   We thus argue that we need other approaches to understand quasiperiodicity and thermalization in the $\alpha-$FPUT chain for a wider range of nonlinearities and larger system sizes.
\end{enumerate}

%Surprisingly, we find that %even in the strong nonlinearity regime (where wave turbulence fails to capture the quasiperiodicity) % (such as the one observed in \cite{Fermi1955})
%there is $1/\epsilon^8$ dependence of equilibration time for $N=32$, which is a prediction of the wave turbulence formalism. So, 
%the equilibration time scales as $1/\epsilon^8$ even in the strong nonlinearity regime for $N = 32$ where the wave turbulence formalism is not applicable


Earlier studies \cite{Onorato2015} have predicted that the equilibration time $\tau$ scales as $1/\epsilon^8$ (for finite system sizes), which has also been confirmed numerically. This is found to happen even when $\epsilon A_{1,1,2}^{(2)}$ is comparable to $1$. However, in this regime our work shows that there is evidence for divergence in the canonical transformation used to remove the three wave interactions, and hence there could be a possible breakdown of perturbation theory. 
If the strong nonlinearity is affecting the quasiperiodic behaviour then we also expect it to speed up the rate of thermalization (we should expect faster than the $1/\epsilon^8$ dependence), which is  not happening probably due to finite size effects. We thus point out that the wave turbulence formalism needs to be modified in order to provide a more involved explanation that accounts for quasiperiodicity in the  $\alpha-$FPUT system, while also retaining the ability to predict the $1/\epsilon^{8}$ dependence of the equilibration time. One way to explain the disagreement is to capture the reversible dynamics by using the ideas of \cite{Guasoni2017}, which uses a nonequilibrium spatiotemporal kinetic formulation that accounts for the existence of phase correlations among incoherent waves. This has to be investigated further, and is beyond the scope of this study. 
%We point out that for finite system sizes (such as $N=32$ considered in the original paper \cite{Fermi1955}) this breakdown may not be significant as there are only a finite number of terms in the perturbation series. But this divergence will become a problem when we consider the thermodynamic limit. 

There have been some studies done recently that describe the FPUT system as a perturbation of the Toda system (exponential interaction between the nearest neighbours) \cite{Toda1967,Toda1975,Goldfriend2019,Fu2019}. An interesting question would be to link this description with wave turbulence. This would help us to understand better the difference between weak nonlinearities where wave turbulence predicts the observed quasiperiodic behaviour and we have $\tau \sim \epsilon^{-8}$, and strong nonlinearities where wave turbulence fails to predict the observed quasiperiodic behaviour and we still have  $\tau \sim \epsilon^{-8}$. For the $\beta-$FPUT chain the overlap of resonances leads to $\tau \sim \epsilon^{-1}$ for larger nonlinearities for $N=32$, instead of $\tau \sim \epsilon^{-4}$ for smaller nonlinearities \cite{Lvov2018}. However, the $\alpha-$FPUT chain is bounded only for small nonlinearities, and it is not clear whether to expect the overlap of resonances to take place in the $\alpha-$FPUT chain. But this is an interesting study.

\section*{Acknowledgments} I acknowledge Abhishek Dhar, Amirali Hannani, Amit Apte, Kabir Ramola, Miguel Onorato, Varun Dubey and Wojciech De Roeck for helpful discussions at various stages of this work. I also thank Miguel Onorato for the hospitality at the Summer school on Wave turbulence and beyond in Turin, Italy. I acknowledge the grant $G098919N$ from the Research Foundation – Flanders (FWO), Belgium and the Department of Atomic Energy, Government of India for funding my positions during which this work has been done.
\begin{appendix}
\section{Frequency corrections}
\label{app:freq}
%The above solution for the evolution of $a_k$ is not complete (even upto first order perturbation theory) because the frequency of oscillations are same for all values of the nonlinearity parameter. For example, we expect that for smaller values of the nonlinearity the quasiperiodic frequency should reduce and so the time period should increase. But we see from the solution that the exponential has not taken into account the nonlinearity parameter, and therefore the frequency of oscillations do not change for different values of the nonlinearity parameter (which are still small enough so that the methods of wave turbulence are applicable). Upon inspection, it turns out that the problem with the solution is that we have taken $\{\omega_i\}$, the normal mode frequencies of the harmonic chain in the exponentials. To get a more accurate solution however, me must instead take into account the frequency corrections to the $\{\omega_i\}$ due to the (time independent) perturbation. So instead of $i\frac{\partial b_1}{\partial t} \approx \omega_1 b_1$, we take the evolution equation to be  $i\frac{\partial b_1}{\partial t} \approx \Omega_1 b_1$, where $\{\Omega_i\}$ are the perturbed frequencies. However, for the  $\alpha-$FPUT chain there are no first order corrections to the frequencies and we have to go to the second order. In the formalism of wave turbulence this was done in cite Zakharov's papers  (Nazarenko book) and we reproduce those results here. 
Let us consider the evolution equation Eq.~\eqref{eq:evol2}:
$$i\frac{\partial b_{k_1}}{\partial t} = \omega_{k_1} b_{k_1} + \epsilon^2 \sum_{k_2,k_3,k_4} T_{k_1,k_2,k_3,k_4} b_{k_2}^\star b_{k_3}b_{k_4}\delta_{k_1+k_2,k_3+k_4} +O(\epsilon^3)~.$$
The sum in the above equation can be broken into $2$ parts. The first sum corresponds to $k_1 = k_3$ and $k_2 = k_4$. The second sum corresponds to the remaining terms. The equilibration problem in the $\alpha-$FPUT chain has been discussed in terms of the second sum, which includes the four wave resonances. For the purpose of computing the frequency correction however, we just need the first part. We thus write Eq.~\eqref{eq:evol2} as:
\begin{equation}\label{eq:evol3}
i\frac{\partial b_{k_1}}{\partial t} = \omega_{k_1}b_{k_1}+ 2\epsilon^2 \sum_{k_2} T_{k_1,k_2,k_1,k_2} b_{k_2}^\star b_{k_2}b_{k_1}-\epsilon^2 T_{k_1,k_1,k_1,k_1} b_{k_1}^\star b_{k_1}b_{k_1} + \text{second sum} +O(\epsilon^3)~.
\end{equation}
The factor $2$ accounts for the permutation $k_1 = k_4$ and $k_2 = k_3$. Since $k_1 = k_2 = k_3 = k_4$ has no permutation, the factor of $2$ should not be included for this term and hence we have the subtracted term. Thus, we get the second order correction to the frequencies by writing Eq.~\eqref{eq:evol3} as:
\begin{equation}\label{eq:evol4}
i\frac{\partial b_{k_1}}{\partial t} \approx (\omega_{k_1}+ 2\epsilon^2 \sum_{k_2} T_{k_1,k_2,k_1,k_2} b_{k_2}^\star b_{k_2}-\epsilon^2 T_{k_1,k_1,k_1,k_1} b_{k_1}^\star b_{k_1})b_{k_1}~.
\end{equation}
The corrected frequencies $\Omega$ are given by:
\begin{equation}\label{eq:omegapert}
	\Omega_{k_1} = \omega_{k_1}+\epsilon^2 (2\sum_{k_2} T_{k_1,k_2,k_1,k_2} b_{k_2}^\star b_{k_2}- T_{k_1,k_1,k_1,k_1} b_{k_1}^\star b_{k_1})+O(\epsilon^3)~,
\end{equation}
where the $b_k$ are computed at $t=0$. In this way, $i\frac{\partial b_{k_1}}{\partial t} \approx \Omega_{k_1} b_{k_1}$ is a better approximation to the evolution equation when compared to $i\frac{\partial b_{k_1}}{\partial t} \approx \omega_{k_1} b_{k_1}-$  the former takes into account the frequency corrections to the $\alpha-$FPUT chain due to its nonlinearity. Note that for the $\alpha-FPUT$ chain the first order correction to the harmonic frequencies is zero. In order to compute higher order corrections to the frequencies (the next term turns out to be of order $\epsilon^4$ rather than $\epsilon^3$), we have to similarly split the higher order sums into two. 
Note that the terms responsible for the frequency correction are not included in the subsequent dynamics. If however, we don't normalize the frequencies and instead, use these terms while computing the evolution, then there will be secular terms in equation Eq.~\eqref{eq:evol3} and the formalism of wave turbulence will not be self-consistent \cite{Zakharov1999, Nazarenko2011}.

The general expression for $T_{k_1,k_2,k_3,k_4}$ depends on the transfer matrix $V_{k_1,k_2,k_3}$ (for the $\alpha-$FPUT system $V_{k_1,k_2,k_3}$ is defined in Eq.~\eqref{eq:transfer}) and is given in \cite{Dyachenko1994}. We reproduce the expression here:
\begin{equation}\label{eq:T1234}
\begin{split}
T_{k_1,k_2,k_3,k_4} = -V_{k_1,k_3,k_1-k_3}V_{k_4,k_2,k_4-k_2}[\frac{1}{\omega_{k_3}+\omega_{k_1-k_3}-\omega_{k_1}}+\frac{1}{\omega_{k_2}+\omega_{k_4-k_2}-\omega_{k_4}}]\\-V_{k_2,k_3,k_2-k_3}V_{k_4,k_1,k_4-k_1}[\frac{1}{\omega_{k_3}+\omega_{k_2-k_3}-\omega_{k_2}}+\frac{1}{\omega_{k_1}+\omega_{k_4-k_1}-\omega_{k_4}}]\\-V_{k_1,k_4,k_1-k_4}V_{k_3,k_2,k_3-k_2}[\frac{1}{\omega_{k_4}+\omega_{k_1-k_4}-\omega_{k_1}}+\frac{1}{\omega_{k_2}+\omega_{k_3-k_2}-\omega_{k_3}}]\\-V_{k_2,k_4,k_2-k_4}V_{k_3,k_1,k_3-k_1}[\frac{1}{\omega_{k_4}+\omega_{k_2-k_4}-\omega_{k_2}}+\frac{1}{\omega_{k_1}+\omega_{k_3-k_1}-\omega_{k_3}}]\\-V_{k_1+k_2,k_1,k_2}V_{k_3+k_4,k_3,k_4}[\frac{1}{\omega_{k_1+k_2}-\omega_{k_1}-\omega_{k_2}}+\frac{1}{\omega_{k_3+k_4}-\omega_{k_3}-\omega_{k_4}}]\\-V_{-k_1-k_2,k_1,k_2}V_{-k_3-k_4,k_3,k_4}[\frac{1}{\omega_{k_1+k_2}+\omega_{k_1}+\omega_{k_2}}+\frac{1}{\omega_{k_3+k_4}+\omega_{k_3}+\omega_{k_4}}] ~.
\end{split}
\end{equation}
From this we get the following expression for $T_{k_1,k_2,k_1,k_2}$:
\begin{equation}\label{eq:T1212}
\begin{split}
T_{k_1,k_2,k_1,k_2} = -\frac{2(V_{k_2,k_1,k_2-k_1})^2}{\omega_{k_1}+\omega_{k_2-k_1}-\omega_{k_2}} -\frac{2(V_{k_1,k_2,k_1-k_2})^2}{\omega_{k_2}+\omega_{k_1-k_2}-\omega_{k_1}} -\frac{2(V_{k_1+k_2,k_1,k_2})^2}{\omega_{k_1+k_2}-\omega_{k_1}-\omega_{k_2}}
-\frac{2(V_{-k_1-k_2,k_1,k_2})^2}{\omega_{k_1+k_2}+\omega_{k_1}+\omega_{k_2}} ~.
\end{split}
\end{equation}
Note that the zero mode does not participate in the dynamics and hence, there are no zero denominators in the expression for $T_{k_1,k_1,k_1,k_1}$. Using the expression for $T_{k_1,k_2,k_1,k_2}$ in Eq.~\eqref{eq:omegapert} we thus compute the frequencies of the $\alpha-$FPUT chain.%, which are corrections to the frequencies of the harmonic chain (when the formalism of wave turbulence is valid) due to nonlinearity.
\end{appendix}
%\bibliographystyle{IEEEtran}


\bibliographystyle{unsrt}
\bibliography{references}
\end{document}
