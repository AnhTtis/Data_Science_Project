\section{Implementation proofs}
\label{sec:implementation-proofs}
 In the following section, we provide formal proofs for the correctness of the filters presented in \cref{sec:implementation-appendix} In the setting of finite discrete signals, where we assume the continuous domain signals are periodic.
 For simplicity to the reader, we prove the correctness for 1-dimensional signal, and for upsampling at factor 2. The proofs can be easily generalized to 2D signals and a higher upsampling rate.

\subsection{Definitions}\label{sec:implementation-proofs-defs}
Let $x\left(t\right)$ be a band-limited continuous signal, with CTFT  $X^{F}\left(\omega\right)$ (as defined in \cref{eq:ctft}). $x\left(t\right)$ is periodic with period $NT$, i.e. $x\left(NT+t\right)=x\left(t\right)$. We define discrete sampling as:
\begin{equation}
    x\left[n\right]=x\left(nT\right)\,,    
\end{equation}
and define a finite sampling as taking only one period of the discrete signal, i.e. 
\begin{equation}
    x_{N}\left[n\right]=\left\{ x_{0},...,x_{N-1}\right\} =\left\{ x\left(0\right),x\left(T\right),...,x\left(\left(N-1\right)T\right)\right\} \,.    
\end{equation}

\subsection{Upsample}
As noted in \cref{sec:implementation-appendix}, we assume that $X^{F}(\omega)=0\ \ \forall|\omega|>\frac{\pi}{T}$  and that $X^{F}\left( \frac{\pi}{T}\right) \in \mathbb{R}$ (This a relaxation of Nyquist condition for which $X^{F} \left( \frac{\pi}{T} \right) = 0$). We prove the validity of the following method to upsample $x_{N} \left[ n \right]$  to retrieve:
\begin{equation}\label{eq:x-2n}
    x_{2N}\left[n\right]=x\left(n\frac{T}{2}\right)=\left\{ x\left(0\right),x\left(\frac{T}{2}\right),...,x\left(\left(2N-1\right)\frac{T}{2}\right)\right\} \,.
\end{equation}
The upsample method presented in \Cref{sec:implementation-appendix} is formally shown in \Cref{algo:Up-sample}:

% \begin{algorithm}[h]
% \caption{Upsample}
% \label{algo:Up-sample}
% %\begin{algorithmic}
% % \item {\bfseries Input:}
%     Input: $x_{N}\left[n\right]\in\mathbb{R}^{N}$ \\
%     $x_{z} \gets \left\{ x_{N} \left[ 0 \right],\ 0,\ x_{d} \left[ 1 \right],\ 0\ ,...,\ x_{d} \left[ N-1 \right],\ 0 \right\}$ \\
%     $x_{2N} \gets \mathrm{IDFT}\left\{ \mathrm{DFT} \left\{ x_{z}\right\} H_{2}^{D}\right\} $ \\
%     Output: $x_{2N}\left[ n \right]$ 
% \end{algorithm}

\begin{algorithm}[h]
\caption{Upsample}
\label{algo:Up-sample}
\begin{algorithmic}
\item {\bfseries Input:} $x_{N}\left[n\right]\in\mathbb{R}^{N}$ 
    \item $x_{z} \gets \left\{ x_{N} \left[ 0 \right],\ 0,\ x_{d} \left[ 1 \right],\ 0\ ,...,\ x_{d} \left[ N-1 \right],\ 0 \right\}$ 
    \item $x_{2N} \gets \mathrm{IDFT}\left\{ \mathrm{DFT} \left\{ x_{z}\right\} H_{2}^{D}\right\} $
    \item {\bfseries Output:} $x_{2N}\left[ n \right]$ 
    \end{algorithmic}
\end{algorithm}


% on $x_{N}$ yields $x_{2N}$, noted as ``ideal reconstruction filter''
% with frequency response denoted as $R[k]$.

% \paragraph{Claim 1} \l

\begin{claim} \label{prop:upsample-algo}
Let $x\left(t\right)$ and $x_{N}\left[n\right]$ be a continuous signal and its finite discrete representation as defined in \cref{sec:implementation-proofs-defs}.
 % Let $x\left(t\right)$ be a band-limited continuous signal, with CTFT (defined in \cref{eq:ctft}) $X^{F}(\omega)=0\ \ \forall|\omega|>\frac{\pi}{T}$, and
 % \[
 % x_{N}\left[n\right]=\left\{ x_{0},...,x_{N-1}\right\} =\left\{ x\left(0\right),x\left(T\right),...,x\left(\left(N-1\right)T\right)\right\} \,.    
 % \]
Then the output of \Cref{algo:Up-sample} is 
\[
    x_{2N}\left[n\right]=x\left(n\frac{T}{2}\right)=\left\{ x\left(0\right),x\left(\frac{T}{2}\right),...,x\left(\left(2N-1\right)\frac{T}{2}\right)\right\} \,,
\]
using the described reconstruction filter $H^{D}_{2}$ below:

For even N:
\begin{equation}
    H^{D}_{2} \left[k \right]=\begin{cases}
        1 & 0\leq k<\frac{N}{2},\\
        1 & \frac{3N}{2}+1\leq k\leq2N-1\\
        0.5 & k=\frac{N}{2},k=\frac{3N}{2}\\
        0 & else
    \end{cases}
    \end{equation}
For odd N:
\begin{equation}
    H^{D}_{2} \left[k \right]=\begin{cases}
        1 & 0\leq k<\lfloor\frac{N}{2}\rfloor\\
        1 & \lceil\frac{3N}{2}\rceil\leq k\leq2N-1\\
        0 & else
    \end{cases}
\end{equation}
\end{claim}
% Note that $N$ represents the length of the input  of \ref{algo:Up-sample}, and thus the parameter LPF should be of length $2N$.
\paragraph{Proof (\cref{prop:upsample-algo})}
In order to proof \cref{prop:upsample-algo}, we will show that the DFT of its output equals to $X^{D}_{2N}$, i.e.~the DFT of the signal $x_{2N}$ that is defined in \cref{eq:x-2n}.

\subsubsection{DFT of zero-padded signal}
Recall that in the first step of the \cref{algo:Up-sample} we apply zero padding on the input $x_{N}$.
For a finite discrete signal with length $N$, DFT is defined as \cref{eq:dft}:
\[
X^{D}[k]=\sum_{n=0}^{N-1}x[n]e^{-\frac{j2\pi}{N}nk}\underbrace{=}_{W_{N}\triangleq e^{\frac{j2\pi}{N}}}\sum_{n=0}^{N-1}x[n]W_{N}^{-nk}
\]

\begin{claim} \label{prop:dft-zero-pad}
Let $X_{N}^{D}$ be the DFT of the input $x_{N}$ of \Cref{algo:Up-sample} and $X_{z}^{D}$ be the DFT of the zero-padded signal $x_{z}$ in step 1 of \Cref{algo:Up-sample}. Then:
\begin{equation}
    X_{z}^{D}[k]=\begin{cases}
        X_{N}^{D}[k] & k<N\\
        X_{N}^{D}[k-N] & k\geq N
    \end{cases}\label{eq:1}
\end{equation}
\end{claim}
\paragraph{Proof (\cref{prop:dft-zero-pad})}
\begin{align}
X_{z}^{D} \left[ k \right] &= \sum_{n=0}^{2N-1} x_{z} \left[ n \right] W_{2N}^{-nk} \\
&\underset{(\ast)}{=} \sum_{n'=0}^{N-1}\left(x_{z} \left[ 2n' \right] W_{2N}^{-2n'k} + \underbrace{x_{z}[2n'+1]}_{=0\ \forall n'}W_{2N}^{-(2n'+1)k}\right) \\
 % & \text{ (*) seperate sum to odd and even components}\\
 & =\sum_{n'=0}^{N-1}x_{z} \left[ 2n' \right] W_{2N}^{-2n'k}\\
 & \underset{(\ast \ast)}{=}\sum_{n'=0}^{N-1}x_{N} \left[ n' \right]W_{N}^{-n'k} \\
 % & \text{(**)}W_{2N}^{-2n'k}=e^{\frac{j2\pi}{2N}(-2n'k)}=e^{\frac{j2\pi}{N}(-n'k)}=W_{N}^{-n'k}
\end{align}
In $(\ast)$ we separated the summation to even and odd components, and in $(\ast \ast)$ we used the fact that 
\[
W_{2N}^{-2n'k}=e^{\frac{j2\pi}{2N}(-2n'k)}=e^{\frac{j2\pi}{N}(-n'k)}=W_{N}^{-n'k} \,.
\]
Note that we got a sum of $N$ components, yet $X_{z}^{D}[k]$ is defined for $k=0,1,...,2N-1$.
For $k=0,1,...,N-1$ we got the definition of DFT, meaning 
\begin{equation}
X_{z}^{D} \left[ k \right] = X_{N}^{D} \left[ k \right] \,.    
\end{equation}
For $k=N,...,2N-1$ using the property 
\[
W_{N}^{-n'(N+k)}=\underbrace{W_{N}^{-n'N}}_{=e^{\frac{j2\pi}{N}(-n'N)}=e^{-j2\pi n'}=1}W_{N}^{-n'k}=W_{N}^{-n'k}
\]
we get:
\begin{align}
X_{z}^{D} \left[ k \right] &= \sum_{n'=0}^{N-1}x_{N} \left[ n' \right] W_{N}^{-n'k} \\
&= \sum_{n'=0}^{N-1}x_{N} \left[ n' \right] W_{N}^{-n' \left( N+ \left( k-N \right) \right)} \\
&= \sum_{n'=0}^{N-1}x_{N} \left[ n' \right] W_{N}^{-n' \left( k-N \right) } \\
&= X_{N}^{D}\left[ k-N \right] \,.
\end{align}

\subsubsection{Expressing DFT with CTFT } \label{sec:dft-ctft}
In the previous section we expressed $X_{z}^{D}$ with $X_{N}^{D}$.
In the following section we will express $X_{2N}^{D}$ using $X_{N}^{D}$.
In addition, we assume that $N$ is even, and will show the other case afterwards.
First, we express the DFT of $x_{N}$ with the CTFT of $x$, by using their relations with DTFT:
\begin{align}
X_{N}^{D} \left[ k \right] &= X_{N}^{f} \left( \theta=\frac{2\pi k}{N} \right) \\
&= \frac{1}{T}\sum_{l=-\infty}^{\infty}X^{F}\left( \frac{\theta+2\pi l}{T} \right) \\
&= \frac{1}{T}\sum_{l=-\infty}^{\infty}X^{F} \left( \frac{2\pi k}{NT}+\frac{2\pi l}{T} \right) \\
& \underbrace{=}_{X^{F} \left( \omega \right) = 0\ \ \forall|\omega|>\frac{\pi}{T}}\frac{1}{T} \left( X^{F} \left( \frac{2\pi k}{NT} \right)+X^{F} \left( \frac{2\pi k}{NT}-\frac{2\pi}{T} \right) \right)
\end{align}

\begin{equation} \label{eq:x-n-dft} %\label{eq:2}
\Rightarrow X_{N}^{D}[k]=\frac{1}{T}\begin{cases}
X^{F}(\frac{2\pi k}{NT}) & 0\leq k\leq\frac{N}{2}-1\\
X^{F}(\frac{\pi}{T})+X^{F}(-\frac{\pi}{T}) & k=\frac{N}{2}\\
X^{F}(\frac{2\pi k}{NT}-\frac{2\pi}{T}) & \frac{N}{2}+1\leq k\leq N-1
\end{cases} \,.
\end{equation}
Note that the last two transitions hold considering the limited support
of $X^{F}:$
\[
\underbrace{X^{F}(\frac{2\pi k}{NT})}_{=0\ \ \forall k>\frac{N}{2}}+\underbrace{X^{F}(\frac{2\pi k}{NT}-\frac{2\pi}{T})}_{=0\ \ \forall k<\frac{N}{2}} \,.
\]
Next, we will derive a similar expression for $X_{2N}^{D}[k]$:
\begin{align}
X_{2N}^{D} \left[ k \right] &= X_{2N}^{f} \left( \theta=\frac{2\pi k}{2N} \right) \\
&= \frac{1}{T}\sum_{l=-\infty}^{\infty}X^{F}\left( \frac{\theta+2\pi l}{T/2} \right) \\
&= \frac{1}{T} \left( \underbrace{ X^{F} \left(\frac{2\pi k}{NT} \right)}_{=0\ \ \forall k>\frac{N}{2}}+\underbrace{X^{F}\left( \frac{2\pi k}{NT}-\frac{2\pi}{T/2} \right)}_{=0\ \ \forall k<\frac{3N}{2}} \right)\\
\end{align}
 We get: 
\begin{equation}
\Rightarrow X_{2N}^{D}[k]=\frac{1}{T}\begin{cases}
X^{F}(\frac{2\pi k}{NT}) & 0\leq k\leq\frac{N}{2}\\
0 & \frac{N}{2}+1\leq k\leq\frac{3N}{2}-1\\
X^{F}(\frac{2\pi k}{NT}-\frac{4\pi}{T}) & \frac{3N}{2}\leq k\leq2N-1
\end{cases}\label{eq:3}
\end{equation}

\subsubsection{Expressing $X_{2N}^{D}$ with $X_{N}^{D}$}
Considering the second step of \Cref{algo:Up-sample}, we need to show that applying the filter $H^{D}_{2}$ on $X_{z}^{D}$ yields $X_{2N}^{D}$, meaning
\[
X_{2N}^{D}\left[ k \right] = X_{z}^{D} \left[k \right] H^{D}_{2} \,.
\]
By plugging $X_{N}^{D} \left[ k \right]$ (\cref{eq:x-n-dft}) in $X_{z}^{D} \left[k \right]$ (\cref{eq:1})
we get: 
\begin{align}
X_{z}^{D} \left[k \right] &=\begin{cases}
    X_{N}^{D} \left[ k \right] & k<N\\
    X_{N}^{D}\left[k-N \right] & k\geq N
    \end{cases} \\
 &= \frac{1}{T}\begin{cases}
    X^{F} \left( \frac{2\pi k}{NT} \right) & 0\leq k\leq\frac{N}{2}-1\\
    X^{F} \left(\frac{\pi}{T} \right)+X^{F} \left(-\frac{\pi}{T} \right) & k=\frac{N}{2}\\
    X^{F}\left( \frac{2\pi k}{NT}-\frac{2\pi}{T} \right) & \frac{N}{2}+1\leq k\leq N-1\\
    X^{F} \left(\frac{2\pi \left(k-N \right)}{NT} \right) & N\leq k\leq\frac{3N}{2}-1\\
    X^{F} \left(\frac{\pi}{T}\right)+X^{F} \left(-\frac{\pi}{T}\right) & k=\frac{3N}{2}\\
    X^{F} \left(\frac{2\pi \left(k-N \right)}{NT}-\frac{2\pi}{T} \right) & \frac{3N}{2}+1\leq k\leq2N-1
\end{cases} \,.
\end{align}
Thus, by applying the filter 
\begin{equation}
H^{D}_{2} \left[ k \right]=\begin{cases}
1 & 0\leq k<\frac{N}{2},\\
1 & \frac{3N}{2}+1\leq k\leq2N-1\\
0.5 & k=\frac{N}{2},k=\frac{3N}{2}\\
0 & else
\end{cases}
\end{equation}


we get: 
\begin{equation}
H^{D}_{2} \left[ k \right] X_{z}^{D} \left[ k \right] = \frac{1}{T} \begin{cases}
X^{F} \left( \frac{2\pi k}{NT} \right)                                                              & 0\leq k\leq\frac{N}{2}-1\\
\frac{1}{2}\left(X^{F} \left(\frac{\pi}{T} \right)+X^{F} \left( -\frac{\pi}{T} \right)\right)      & k=\frac{N}{2}\\
0                                                                                                     & \frac{N}{2}+1\leq k\leq N-1\\
0                                                                                                   & N\leq k\leq\frac{3N}{2}-1\\
\frac{1}{2}\left(X^{F} \left( \frac{\pi}{T} \right)+X^{F} \left(-\frac{\pi}{T} \right)\right)                                      & k=\frac{3N}{2}\\
X^{F} \left(\frac{2\pi \left( k-N \right) }{NT}-\frac{2\pi}{T} \right)                              & \frac{3N}{2}+1\leq k\leq2N-1
\end{cases} \,.
\end{equation}

Note that for real $x(t)$, $X^{F}$ is conjugate symmetric, and we assumed  that $X^{F} \left(\frac{\pi}{T} \right) \in \mathbb{R}$. Therefore:
\begin{itemize}
\item for $k=\frac{N}{2},\frac{3N}{2}$: 
\begin{equation}
X^{F} \left(\frac{\pi}{T} \right) + X^{F} \left( -\frac{\pi}{T} \right) = 2X^{F} \left( \frac{\pi}{T} \right)=2X^{F} \left(-\frac{\pi}{T} \right) \,.
\end{equation}
\item for $\frac{3N}{2}+1\leq k\leq2N-1$: 
\begin{equation}
X_{z}^{D} \left[ k \right] = X^{F} \left( \frac{2\pi \left( k-N \right) }{NT}-\frac{2\pi}{T} \right)=X^{F} \left( \frac{2\pi k}{NT}-\frac{2\pi N}{NT}-\frac{2\pi}{T} \right)=X^{F} \left( \frac{2\pi k}{NT}-\frac{4\pi}{T} \right) \,.    
\end{equation}
\end{itemize}
Thus we get: 
\begin{align}
H^{D}_{2} \left[ k \right] X_{z}^{D} \left[ k \right] &= \frac{1}{T}\begin{cases}
X^{F} \left(\frac{2\pi k}{NT}\right)                 & 0\leq k\leq\frac{N}{2}-1\\
X^{F} \left(\frac{\pi}{T}\right)                    & k=\frac{N}{2}\\
0                                                     & \frac{N}{2}+1\leq k\leq N-1\\
0                                                   & N\leq k\leq\frac{3N}{2}-1\\
X^{F} \left( -\frac{\pi}{T} \right)                 & k=\frac{3N}{2}\\
X^{F} \left( \frac{2\pi k}{NT}-\frac{4\pi}{T} \right) & \frac{3N}{2}+1\leq k\leq2N-1
\end{cases} \\
&= \frac{1}{T}\begin{cases}
X^{F} \left( \frac{2\pi k}{NT} \right)              & 0\leq k\leq\frac{N}{2}\\
0                                               & \frac{N}{2}+1\leq k\leq\frac{3N}{2}-1\\
X^{F} \left( \frac{2\pi k}{NT}-\frac{4\pi}{T} \right) & \frac{3N}{2}\leq k\leq2N-1
\end{cases} \\
& \underset{(\ref{eq:3})}{=} X_{2N}^{D} \left[ k \right] \,.  
\end{align}


\subsubsection{Odd $N$}
By repeating the derivations of section \cref{sec:dft-ctft} for odd $N$ we get: 
\begin{equation}
X_{N}^{D} \left[k \right]=\frac{1}{T}\begin{cases}
X^{F}\left(\frac{2\pi k}{NT} \right) & 0\leq k\leq\lfloor\frac{N}{2}\rfloor\\
X^{F}\left(\frac{2\pi k}{NT}-\frac{2\pi}{T} \right) & \lceil\frac{N}{2}\rceil\leq k\leq N-1
\end{cases}\label{eq:2.1} \,.
\end{equation}
 
\begin{equation}
X_{2N}^{D}\left[k\right]=\frac{1}{T}\begin{cases}
X^{F}\left(\frac{2\pi k}{NT}\right) & 0\leq k\leq \lfloor\frac{N}{2} \rfloor\\
0 & \lceil\frac{N}{2}\rceil\leq k\leq\lfloor\frac{3N}{2}\rfloor\\
X^{F}\left(\frac{2\pi k}{NT}-\frac{4\pi}{T}\right) & \lceil\frac{3N}{2}\rceil\leq k\leq2N-1
\end{cases}\label{eq:3.1} \,.
\end{equation}
By plugging $X_{N}^{D}\left[k\right]$ (\ref{eq:2.1}) in $X_{0}^{D}\left[k\right]$ (\ref{eq:1})
we get: 
\begin{align}
X_{z}^{D}\left[k\right] & =\begin{cases}
X_{N}^{D}\left[k\right] & k<N\\
X_{N}^{D}\left[k-N\right] & k\geq N
\end{cases} \\
 &= \frac{1}{T}\begin{cases}
X^{F}\left(\frac{2\pi k}{NT}\right) & 0\leq k\leq\lfloor\frac{N}{2}\rfloor\\
X^{F}\left(\frac{2\pi k}{NT}-\frac{2\pi}{T}\right) & \lceil\frac{N}{2}\rceil\leq k\leq N-1\\
X^{F}\left(\frac{2\pi\left(k-N\right)}{NT}\right) & N\leq k\leq\lfloor\frac{3N}{2}\rfloor\\
X^{F}\left(\frac{2\pi\left(k-N\right)}{NT}-\frac{2\pi}{T}\right) & \lceil\frac{3N}{2}\rceil\leq k\leq2N-1
\end{cases} \,.
\end{align}
Then, by applying the filter 
\begin{equation}
H^{D}_{2}\left[k\right]=\begin{cases}
1 & 0\leq k<\lfloor\frac{N}{2}\rfloor,\\
1 & \lceil\frac{3N}{2}\rceil\leq k\leq2N-1\\
0 & else
\end{cases}
\end{equation}
 we get: 
\begin{align}
H^{D}_{2}\left[k\right]X_{z}^{D}\left[k\right] &= \frac{1}{T}\begin{cases}
X^{F}\left(\frac{2\pi k}{NT}\right) & 0\leq k\leq\lfloor\frac{N}{2}\rfloor\\
% 0 & \lceil\frac{N}{2}\rceil\leq k\leq N-1 \\
% 0 & N\leq k\leq\lfloor\frac{3N}{2}\rfloor \\
0 & \lceil\frac{N}{2}\rceil\leq k\leq \lfloor\frac{3N}{2}\rfloor \\
X^{F}\left(\frac{2\pi\left(k-N\right)}{NT}-\frac{2\pi}{T}\right) & \lceil\frac{3N}{2}\rceil\leq k\leq2N-1
\end{cases} \\
& \underset{(\ref{eq:3.1})}{=} X_{2N}^{D}\left[k\right] \,.
\end{align}

\subsection{LPF}
We use an ``ideal LPF'' before downsampling in BlurPool layers and in alias-free polynomial activations.
As mentioned in \cref{sec:shift-invariance-proof} ``ideal LPF'' in the context of subsampling of infinite discrete signals is a filter that completely eliminates all the frequencies beyond the Nyquist condition.
E.g. in case of subsampling in factor $I$, i.e. 
\[
y \left[ n \right] = x \left[ I n \right] \,,
\]
an ``ideal LPF'' in DTFT domain is implemented by multiplication with the filter:
\[
H^{f}_{1/I} \left(\theta \right) = \begin{cases}
    1  &    \left| \theta \right| < \frac{\pi}{I} \,, \\
    0  &     \left| \theta \right| \geq \frac{\pi}{I}
\end{cases}
\]
When considering the continuous domain, we expect the results to be equal to the discrete representation of the continuous signal after applying ``ideal LPF'', that is multiplication in CTFT domain with the filter 
\[
H^{F}_{1/I} \left(\theta \right) = \begin{cases}
    1  &    \left| \theta \right| < \frac{\pi}{TI}  \\
    0  &     \left| \theta \right| \geq \frac{\pi}{TI}
\end{cases} \,,
\]
where $T$ is the sample rate of $x \left[ n \right]$.
\begin{claim} \label{prop:ideal-lpf}
Let $x\left(t\right)$ and $x_{N}\left[n\right]$ be a continuous signal and its finite discrete representation as defined in \cref{sec:implementation-proofs-defs}.
In addition, let $x_{\mathrm{LPF}} \left( t \right)$ be the continuous signal received by applying $H^{F}_{1/I}$ on $x \left(t \right)$ in the continuous domain, i.e.~:
\begin{equation} \label{eq:x-lpf-ctft}
X_{\mathrm{LPF}}^{F}\left(\omega\right)	=\begin{cases}
X^{F}\left(\omega\right) & \left|\omega\right|<\frac{\pi}{TI}\\
0 & \left|\omega\right|\geq\frac{\pi}{TI}
\end{cases}
\end{equation}
In addition, define $x_{\mathrm{LPF}} \left( t \right)$ discrete representation as: 
\[
x_{\mathrm{LPF}} \left[ n \right] = \left\{ x_{\mathrm{LPF}}\left(0\right),\ x_{\mathrm{LPF}}\left(T\right),...,\ x_{\mathrm{LPF}}\left(\left(N-1\right)T\right)\right\} \,.    
\]
Then applying $\mathrm{LPF}_{1/I}$  on $x_{N}$ gives
 \[
 \mathrm{LPF}_{1/I} \left( x_{N} \right) = x_{\mathrm{LPF}}\left[ n\right]
 \]
where $\mathrm{LPF}_{1/I}$ is defined as multiplication in DFT domain with
\[
H^D_{1/I} \left[ k \right]=\begin{cases}
1 & 0\leq k<\frac{N}{2I}\,,\\
0 & \frac{N}{2I}\leq k \leq N - \frac{N}{2I}\,,\\
1 & N - \frac{N}{2I} < k\leq N-1\,.
\end{cases} \,.
\]
\end{claim}

\paragraph{proof (\cref{prop:ideal-lpf})}
Similarly to \cref{sec:dft-ctft}, using the relations between DFT, DTFT and CTFT we get:

\begin{align}
X_{\mathrm{LPF}}^{D}\left[k\right]	&=  X_{\mathrm{LPF}}^{f}\left(\theta=\frac{2\pi k}{N}\right) \\
&=  \frac{1}{T}\sum_{l=-\infty}^{\infty}X_{\mathrm{LPF}}^{F}\left(\frac{\theta+2\pi l}{T}\right) \\
&= \frac{1}{T}\sum_{l=-\infty}^{\infty}X_{\mathrm{LPF}}^{F}\left(\frac{2\pi k}{NT}+\frac{2\pi l}{T}\right) \\
& \underbrace{=}_{X^{F}(\omega)=0\ \ \forall|\omega|>\frac{\pi}{T}}\frac{1}{T}\left(X_{\mathrm{LPF}}^{F}\left(\frac{2\pi k}{NT}\right)+X_{\mathrm{LPF}}^{F}\left(\frac{2\pi k}{NT}-\frac{2\pi}{T}\right)\right)    \,.
\end{align}
By plugging \cref{eq:x-lpf-ctft} we get:
\begin{equation}
X_{\mathrm{LPF}}^{D}\left[k\right]=\frac{1}{T}\begin{cases}
X^{F}\left(\frac{2\pi k}{2T}\right) & 0\leq k<\frac{N}{2I}\\
0 & \frac{N}{2I}\leq k\leq N-\frac{N}{2I}\\
X_{\mathrm{}}^{F}\left(\frac{2\pi k}{NT}-\frac{2\pi}{T}\right) & N-\frac{N}{2I}<k\leq N-1 \,.
\end{cases}
\end{equation}
Recall that $x_{N}$ satisfies (\cref{eq:x-n-dft}): 
\[ 
X_{\mathrm{N}}^{D}\left[k\right]=\frac{1}{T}\begin{cases}
X^{F}\left(\frac{2\pi k}{2T}\right) & 0\leq k<\frac{N}{2}-1\\
X^{F}\left(\frac{\pi}{T}\right)+X^{F}\left(-\frac{\pi}{T}\right) & k=\frac{N}{2}\\
X_{\mathrm{}}^{F}\left(\frac{2\pi k}{NT}-\frac{2\pi}{T}\right) & \frac{N}{2}<k\leq N-1
\end{cases} \,;
\]
thus we get 
\begin{equation}
X_{\mathrm{LPF}}^{D}\left[k\right]=H_{1/I}^{D}\left[k\right]X^{D}\left[k\right] \,.
\end{equation}
