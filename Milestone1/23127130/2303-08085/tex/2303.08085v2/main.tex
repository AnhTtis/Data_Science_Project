% =========================================
% NeurIPS 2022 Format
% =========================================
\documentclass{article}

% hagay
\usepackage[accsupp]{axessibility}  % Improves PDF readability for those with disabilities.

% if you need to pass options to natbib, use, e.g.:
%     \PassOptionsToPackage{numbers, compress}{natbib}
% before loading neurips_2022


% ready for submission
% \usepackage{neurips_2022}


% to compile a preprint version, e.g., for submission to arXiv, add add the
% [preprint] option:
    \usepackage[preprint]{neurips_2022}


% to compile a camera-ready version, add the [final] option, e.g.:
    % \usepackage[final]{neurips_2022}


% to avoid loading the natbib package, add option nonatbib:
%    \usepackage[nonatbib]{neurips_2022}


\usepackage[utf8]{inputenc} % allow utf-8 input
\usepackage[T1]{fontenc}    % use 8-bit T1 fonts
\usepackage{hyperref}       % hyperlinks
\usepackage{url}            % simple URL typesetting
\usepackage{booktabs}       % professional-quality tables
\usepackage{amsfonts}       % blackboard math symbols
\usepackage{nicefrac}       % compact symbols for 1/2, etc.
\usepackage{microtype}      % microtypography
\usepackage{xcolor}         % colors


% ========================
%% Our macros
% ========================
% graphics
\usepackage[pdftex]{graphicx}

\usepackage{amsmath}
\def\triangleq{\overset{\Delta}{=}}


% Support for easy cross-referencing
\usepackage[capitalize]{cleveref}
\crefname{section}{Sec.}{Secs.}
\Crefname{section}{Section}{Sections}
\Crefname{table}{Table}{Tables}
\crefname{table}{Tab.}{Tabs.}

\newtheorem{claim}{Claim}
\crefname{claim}{Claim}{Claims}

\Crefname{algorithm}{Algorithm}{Algorithms}
\crefname{algorithm}{Algo.}{Algos.}

\usepackage{todonotes}

\newcommand{\hagay}[1]{{[\textcolor{red}{HM: #1}]}}
\newcommand{\daniel}[1]{{[\textcolor{blue}{DS: #1}]}}
\newcommand{\tomer}[1]{{[\textcolor{green}{TM: #1}]}}
% line break in table
\newcommand{\specialcell}[2][c]{\begin{tabular}[#1]{@{}c@{}}#2\end{tabular}}
\newcommand{\specialcellleft}[2][l]{\begin{tabular}[#1]{@{}l@{}}#2\end{tabular}}

\newtheorem{theorem}{Theorem}
\newtheorem{proposition}{Proposition}
\newtheorem{definition}{Definition}
\newtheorem{lemma}{Lemma}


% % algorithms 
\usepackage{algorithm}
\usepackage{algpseudocode}

\bibliographystyle{abbrvnat}
% \bibliographystyle{unsrt}  
% \bibliographystyle{plainnat}
% \bibliographystyle{plain}

% % need for subfigures
\usepackage{caption}
\usepackage{subcaption}



% ========================


% \title{Formatting Instructions For NeurIPS 2022}
\title{Alias-Free Convnets: \\ Fractional Shift Invariance via Polynomial Activations}


% The \author macro works with any number of authors. There are two commands
% used to separate the names and addresses of multiple authors: \And and \AND.
%
% Using \And between authors leaves it to LaTeX to determine where to break the
% lines. Using \AND forces a line break at that point. So, if LaTeX puts 3 of 4
% authors names on the first line, and the last on the second line, try using
% \AND instead of \And before the third author name.


% \author{%
%       Hagay Michaeli \\
%       Department of Electrical and Computer Engineering \\
%       Technion\\
%     Haifa, Israel\\
%     \texttt{hagaymichaeli@gmail.com} \\
%    \And
%   Tomer Michaeli \\
%   Department of Electrical and Computer Engineering \\
%   Technion\\
% Haifa, Israel\\
%  \And
%    Daniel Soudry \\
%   Department of Electrical and Computer Engineering \\
%   Technion\\
% Haifa, Israel\\
% }
\author{
      Hagay Michaeli \ \ \ \ \  
      Tomer Michaeli  \ \ \ \ \  
      Daniel Soudry \\
      Department of Electrical and Computer Engineering \\
      Technion, Haifa, Israel\\
    \texttt{\{hagaymichaeli, daniel.soudry\}@gmail.com, tomer.m@ee.technion.ac.il} 
}

  % David S.~Hippocampus\thanks{Use footnote for providing further information
  %   about author (webpage, alternative address)---\emph{not} for acknowledging
  %   funding agencies.} \\
  % Department of Computer Science\\
  % Cranberry-Lemon University\\
  % Pittsburgh, PA 15213 \\
  % \texttt{hippo@cs.cranberry-lemon.edu} \\
  % % examples of more authors
  % \And
  % Coauthor \\
  % Affiliation \\
  % Address \\
  % \texttt{email} \\
  % \AND
  % Coauthor \\
  % Affiliation \\
  % Address \\
  % \texttt{email} \\
  % \And
  % Coauthor \\
  % Affiliation \\
  % Address \\
  % \texttt{email} \\
  % \And
  % Coauthor \\
  % Affiliation \\
  % Address \\
  % \texttt{email} \\
% }


\begin{document}


\maketitle


\begin{abstract}
Although CNNs are believed to be invariant to translations, recent works have shown this is not the case, due to aliasing effects that stem from downsampling layers. 
The existing architectural solutions to prevent aliasing are partial since they do not solve these effects, that originate in non-linearities.
We propose an extended anti-aliasing method that tackles both downsampling and non-linear layers, thus creating truly alias-free, shift-invariant CNNs\footnote{Our code is available at \href{https://github.com/hmichaeli/alias_free_convnets}{https://github.com/hmichaeli/alias\_free\_convnets/}.}. 
We show that the presented model is invariant to integer as well as fractional (i.e., sub-pixel) translations, thus outperforming other shift-invariant methods in terms of robustness to adversarial translations.
\end{abstract}

%%%%%%%%% BODY TEXT
% \hagay{writing conventions: \\
%         low-pass filter\\
%         aliasing ("effects" unnecessary) \\
%         anti-aliasing \\
%         downsampling \\
%         non-linear, non-linearity \\
%         shift-invariance, shift-invariant \\
%         shift-equivariance, shift-equivariant \\
%             ConvNeXt \\
%            convnet \\
%     }
\section{Introduction}
\label{sec:introduction}
% \begin{itemize}
%     % Diffusion of FL
%     \item {\st{Diffusion of FL}}
%     % Security threats to FL
%     \item {\st{Security threats to FL with particular focus on model poisoning}}
%     % Limitations of existing countermeasures
%     \item {\st{Current countermeasures (e.g., KRUM) and their limitations}}
%     % Proposed method and its advantages
%     \item {\st{Intuitive description of the proposed method and its difference (i.e., advantages) w.r.t. state of the art}}
%     % Main contributions
%     \item {\st{Summary of the main contributions of this work}}
%     % Paper's structure and organization
%     \item {\st{Paper's structure and organization}}
% \end{itemize}

% Diffusion of FL
Recently, {\em federated learning} (FL) has emerged as the leading paradigm for training distributed, large-scale, and privacy-preserving machine learning (ML) systems~\cite{mcmahan2017googleai,mcmahan2017aistats}. 
The core idea of FL is to allow multiple edge clients to collaboratively train a shared, global model without disclosing their local private training data.
%Specifically, an FL system consists of a central server and many edge clients; 
A typical FL round involves the following steps: {\em(i)} the server randomly picks some clients and sends them the current, global model; {\em(ii)} each selected client locally trains its model with its own private data; then, it sends the resulting local model to the server;\footnote{Whenever we refer to global/local model, we mean global/local model {\em parameters}.} {\em(iii)} the server updates the global model by computing an \emph{aggregation function}, usually the average (FedAvg), on the local models received from clients.
% \begin{enumerate}
%     \item[{\em(i)}] the server sends the current, global model to the clients and appoints some of them for training;
%     \item[{\em(ii)}] each selected client locally trains its copy of the global model with its own private data; then, it sends the resulting local model back to the server;\footnote{Whenever we refer to global/local model, we mean global/local model {\em parameters}.}
%     \item[{\em(iii)}] the server updates the global model by computing an \emph{aggregation function} on the local models received from clients (by default, the average, also referred to as FedAvg~\cite{mcmahan2017aistats}).
% \end{enumerate}
This process goes on until the global model converges. %(e.g., after a certain number of rounds or other similar stopping criteria).
%\\
% The advantages of FL over the traditional, centralized learning paradigm are undoubtedly clear in terms of flexibility/scalability (clients can join/disconnect from the FL network dynamically), network communications (only model weights\footnote{We will use \textit{parameters} and \textit{weights} interchangeably.} are exchanged between clients and server), and privacy (each client's private training data is kept local at the client's end and not uploaded to the server).
\\
% Security threats to FL
%However, the growing adoption of FL also raises security concerns~\cite{costa2022covert}, particularly about its confidentiality, integrity, and availability.
Although its advantages over standard ML, FL also raises security concerns~\cite{costa2022covert}. %, particularly about its confidentiality, integrity, and availability~\cite{costa2022covert}.
% OLD, LONG VERSION
% Indeed, some work deals with privacy leakage that may expose the local data of some clients~\cite{melis2019sp}. 
% A large body of work, instead, investigates attacks that usually aim to detriment the predictive accuracy of the learned global model. For instance, \emph{data poisoning} attacks achieve this goal by letting an adversary pollute the training set of some corrupt FL clients with maliciously crafted examples~\cite{jagielski2018sp}.
% Similarly, in \emph{model poisoning} the attacker attempts to tweak the global model weights~\cite{bhagoji2019pmlr} by directly perturbing the local model's weights of some infected FL clients before these are sent to the central server for aggregation, usually via so-called Byzantine attacks. 
% It turns out that Byzantine model poisoning attacks severely impact standard FedAvg; therefore, more robust aggregation functions must be designed to make FL systems secure.
Here, we focus on \emph{untargeted model poisoning} attacks~\cite{bhagoji2019pmlr}, where an adversary attempts to tweak the global model weights %\footnote{We will use the terms \textit{parameters} and \textit{weights} interchangeably.} 
by directly perturbing the local model's parameters of some infected clients before these are sent to the central server for aggregation.
In doing so, the adversary aims to jeopardize the global model \textit{indiscriminately} at inference time.
Such model poisoning attacks severely impact standard FedAvg; therefore, more robust aggregation functions must be designed to secure FL systems.
\\
% In this paper, we focus on designing a novel robust aggregation scheme at the server's end to contrast the effect of Byzantine model poisoning attacks.
%
% Current countermeasures and their limitations
%Several countermeasures have been proposed in the literature to combat model poisoning attacks on FL systems.
% Some methods use simple statistics more robust than plain average to smooth the impact of malicious updates (e.g., Trimmed Mean and FedMedian~\cite{yin2018icml}). 
% Other defenses implement outlier detection techniques to discard malicious updates from the aggregation performed at the server's end. Those are either based on heuristics (e.g., Krum/Multi-Krum~\cite{blanchard2017nips} and Bulyan~\cite{mhamdi2018pmlr}) or data-driven approaches (e.g., K-means clustering~\cite{shen2016acm} or DnC via spectral analysis~\cite{shejwalkar2021ndss}). 
% Finally, some strategies rely on a centralized ``source of trust'' to spot potential malicious updates (e.g., FLTrust~\cite{cao2020fltrust}).
% Several countermeasures have been proposed in the literature to combat model poisoning attacks on FL systems, i.e., to discard possible malicious local updates from the aggregation performed at the server's end. 
% These techniques range from simple statistics more robust than plain average (e.g., Trimmed Mean and FedMedian~\cite{yin2018icml}) to outlier detection heuristics (e.g., Krum/Multi-Krum~\cite{blanchard2017nips} and Bulyan~\cite{mhamdi2018pmlr}) or data-driven approaches (e.g., spectral analysis via K-means clustering~\cite{shen2016acm} or spectral analysis), or methods based on ``source of trust'' (e.g., FLTrust~\cite{cao2020fltrust}).
% OLD, LONG VERSION
%Several countermeasures have been proposed in the literature to combat Byzantine model poisoning attacks on FL systems.
% Descriptive statistics
% For example, Trimmed Mean and FedMedian aggregate local model updates using more robust statistics than standard average~\cite{yin2018icml}.
%
% % Heuristics for outlier detection
% Many existing Byzantine-resilient strategies implement some outlier detection heuristics to discard the model updates sent by potentially malicious clients from the input of the aggregation function.
% One of the most popular heuristics is Krum~\cite{blanchard2017nips}.
% This strategy tries to mitigate the impact of Byzantine attacks by selecting as a global model the local model with the smallest sum of Euclidean distances to {\em all} the other local models.
% Although powerful, Krum requires the server to know (or, at least, estimate) the number of malicious FL clients upfront, which is generally impossible in a realistic attack scenario. %
% Moreover, Krum may become ineffective for complex, high-dimensional model parameter spaces due to the curse of dimensionality.
% Bulyan~\cite{mhamdi2018pmlr} tries to overcome this issue by combining Krum with a variant of Trimmed Mean.
% % Data-driven outlier detection
% Other strategies use data-driven outlier detection techniques -- e.g., via K-means clustering~\cite{shen2016acm} -- to spot potential malicious local model updates. 
% %For instance, Shen et al. propose to cluster local model updates with K-means and thus identify outliers.
%
% % Other techniques
% As far as the server is concerned, any local model received can be from a potential malicious client. 
% FLTrust~\cite{cao2020fltrust} assumes the server acts as a client, i.e., trains a local model on an additional {\em trustworthy} dataset at the server's end and compares it against all the local models from other clients. 
% This way, the server can rely on some ``source of trust'' when discarding potentially malicious clients.
%\\
% Limitations of existing Byzantine-resilient strategies
Unfortunately, existing defense mechanisms either rely on simple heuristics (e.g., Trimmed Mean and FedMedian by~\cite{yin2018icml}) or need strong and unrealistic assumptions to work effectively (e.g., foreknowledge or estimation of the number of malicious clients in the FL system, as for Krum/Multi-Krum~\cite{blanchard2017nips} and Bulyan~\cite{mhamdi2018pmlr}, which, however, cannot exceed a fixed threshold).
Furthermore, outlier detection methods using K-means clustering~\cite{shen2016acm} or spectral analysis like DnC~\cite{shejwalkar2021ndss} do not directly consider the temporal evolution of local model updates received.
Finally, strategies like FLTrust~\cite{cao2020fltrust} require the server to collect its own dataset and act as a proper client, thereby altering the standard FL protocol.
\\
% OLD, LONG VERSION
% Overall, existing Byzantine-resilient strategies are either simple heuristics (e.g., FedMedian) or, if they are more complex, they rely on strong and unrealistic assumptions to work effectively (e.g., knowing the number of malicious clients in the FL system in advance, as for Krum and alike).
% Furthermore, data-driven outlier detection methods do not consider the temporary evolution of local model updates received (e.g., K-means clustering). 
% Finally, strategies like FLTrust requires the server to collect its own dataset and act as a proper client, thereby altering the standard FL protocol.
%
% Description of the proposed method
This work introduces a novel pre-aggregation \textit{filter} robust to untargeted model poisoning attacks. Notably, this filter $(i)$ operates without requiring prior knowledge or constraints on the number of malicious clients and $(ii)$ inherently integrates temporal dependencies. 
The FL server can employ this filter as a preprocessing step before applying \textit{any} aggregation function, be it standard like FedAvg or robust like Krum or Bulyan.
Specifically, we formulate the problem of identifying corrupted updates as a multidimensional (i.e., matrix-valued) time series anomaly detection task. 
The key idea is that legitimate local updates, resulting from well-calibrated iterative procedures like stochastic gradient descent (SGD) with an appropriate learning rate, show \textit{higher predictability} compared to malicious updates. This hypothesis stems from the fact that the sequence of gradients (thus, model parameters) observed during legitimate training exhibit regular patterns, as validated in Section~\ref{subsec:intuition}. %until convergence. 
%This regularity may be more pronounced for smooth convex loss functions, but it can still be captured within an appropriate time window, even for more complex and convoluted loss surfaces. 
%We provide evidence of this claim in Appendix~B, where we show that the average mutual information (i.e., ``predictability''), calculated over pairs of legitimate model updates sent at different FL rounds, is significantly higher than the corresponding computation for a malicious client.
\\
Inspired by the matrix autoregressive (MAR) framework for multidimensional time series forecasting~\cite{chen2021je}, we propose the FLANDERS ({\em \textbf{F}ederated \textbf{L}earning meets \textbf{AN}omaly \textbf{DE}tection for a \textbf{R}obust and \textbf{S}ecure}) filter.
The main advantages of FLANDERS over existing strategies like FLDetector~\cite{zhao2020multivariate} are its resilience to large-scale attacks, where $50\%$ or more FL participants are hostile, and the capability of working under realistic non-iid scenarios.
We attribute such a capability to two key factors: $(i)$ FLANDERS works without knowing a priori the ratio of corrupted clients, and $(ii)$ it embodies temporal dependencies between intra- and inter-client updates, quickly recognizing local model drifts caused by evil players. Below, we summarize our main contributions:

\begin{itemize}
\item[{\em(i)}]
We provide empirical evidence that the sequence of models sent by legitimate clients is more predictable than those of malicious participants performing untargeted model poisoning attacks.
\\
\item[{\em(ii)}] 
We introduce FLANDERS, the first pre-aggregation filter for FL robust to untargeted model poisoning based on multidimensional time series anomaly detection.
\\
\item[{\em(iii)}] 
We integrate FLANDERS into Flower,\footnote{\scriptsize{\url{https://flower.dev/}}} a popular FL simulation framework for reproducibility.
\\
\item[{\em(iv)}] 
We show that FLANDERS improves the robustness of the existing aggregation methods under multiple settings: different datasets, client's data distribution (non-iid), models, and attack scenarios.
\\
\item[{\em(v)}] 
We publicly release all the implementation code of FLANDERS along with our experiments.\footnote{\scriptsize{\url{https://anonymous.4open.science/r/flanders_exp-7EEB}}}
\end{itemize}

% Paper's structure and organization
The remainder of the paper is structured as follows. %some related work and the current state-of-the-art solutions to security issues that FL entails. 
Section~\ref{sec:background} covers background and preliminaries. 
In Section~\ref{sec:related}, we discuss related work.
Section~\ref{sec:problem} and Section~\ref{sec:method} describe the problem formulation and the method proposed. % to tackle it. 
Section~\ref{sec:experiments} gathers experimental results. %, and Section~\ref{sec:limitations} discusses some limitations of this work.
Finally, we conclude in Section~\ref{sec:conclusion}.
 %discusses the limitations of this work and draws future research directions.
%reports conclusions and draws perspectives for future research directions.

%%%%%%% OLD %%%%%%%
%to overcome the resilience of Byzantine failures in distributed Stochastic Gradient Descent computations. 
% The strength of Krum is its time complexity, which is linear in the gradient dimension. 
% However, the robustness of the approach is guaranteed for gradient-based learning applications only when the majority of the clients are not compromised. 
% Besides, the aggregation mechanism of Krum, as well as that of similar methods, is robust from a coarse-grained perspective and does not provide solutions to errors and perturbations that may occur at inference time.
%A related approach to~\cite{blanchard2017nips} is the work of Su et al.~\cite{su2016dc}. Here, the authors propose an iterated approximate agreement to tackle a multi-layer scenario attacked by Byzantine agents. 
%However, the method works efficiently on the sole discrete context and it is inapplicable to continuous state environments.
%\gabri{Maybe, we should just talk about the main limitations of existing countermeasures without digging into their details (or, we can just mention Krum as this is the most popular one). I will move the description of all these methods to the Related Work section.}
\section{PoseRAC Model}
\label{sec4}

\begin{figure*}[t]
\centering
\includegraphics[width=1.0\textwidth]{figure5.pdf}
\caption{Overview of our proposed PoseRAC. For a input video, the repetitive count can be obtained through Pose Estimation, Transformer Encoder, Pose Mapping and Action-trigger, where only the Encoder and the Pose Mapping need to be trained. We use Triplet Margin Loss to train the Encoder while Binary Cross Entropy Loss to train both the Encoder and the Pose Mapping. In addition to achieving the state-of-the-art performance so far, the biggest highlight of our PoseRAC is that it is lightweight enough to be easily trained on a CPU.}
\label{fig5}
\end{figure*}

Given a video $V={\{x_i\}}^{T}_{1}\in \mathbb{R}^{C\times H\times W\times T}$ with $T$ RGB frames, repetitive action counting model aims to predict a certain value $Y$, which is the number of repetitive actions. In this section, we will introduce our PoseRAC in detail.

\subsection{Model Overview}

As shown in Figure \ref{fig5}, PoseRAC consists of four parts. 

\begin{itemize}

\item The first is a state-of-the-art and lightweight Pose Estimation Network~($\S\ref{first}$), which is used to estimate the poses represented by lots of human pose key points from each frame of the original video sequence. 

\item The second is a simple Transformer Encoder~($\S\ref{second}$) to embed the key points of poses into high-level feature space, where the same class have similar distances, while the distances of different classes are far apart.

\item The third is a Pose Mapping Module~($\S\ref{third}$), where the unique mapping relationship between the salient poses and the action classes can be learned. Each pose can be mapped to the action class with the highest probability after the previous encoding.

\item The fourth part is a lightweight Action-trigger Module~($\S\ref{fourth}$). When we get the salient action classification results of all frames of the entire video sequence, we can use this module to calculate the repetition count in a short time.

\end{itemize}

\subsection{Pose Estimation Network}
\label{first}
Our model first converts the video sequence into a sequence of human pose key points, which can be defined as: 
\begin{equation}
\begin{split}
&V={\{x_i\}}^{T}_{1}\in \mathbb{R}^{C\times H\times W\times T}\\
&V\xrightarrow{\mathrm{Pose Estimation}} P={\{p_i\}}^{T}_{1}\in \mathbb{R}^{D\times K\times T}
\end{split}
\label{eq1}
\end{equation}
where each $x_i$ represents a single RGB frame, and each $p_i$ represents the key points of each frame. To express the key points of each frame, we use $D\times K$ sequence, which includes two parts, one ($K$) is the number of key points to fully represent the current pose, the other ($D$) is the dimension of each key point, generally three, which are the two coordinates of the planes and the depth estimation.

Here we use state-of-the-art pose estimation models such as Vitpose\cite{xu2022vitpose} and BlazePose\cite{bazarevsky2020blazepose}. The pose estimation algorithms themselves are not designed by us, but we introduce pose information into the action counting task, which is a novel design not explored by previous work.

Moreover, our pose-level poses estimation processes the primitive information of video, which is similar to the feature extraction network in all video-level algorithms such as I3D\cite{carreira2017quo}, VideoSwinTransformer\cite{liu2022video}, and TSN\cite{wang2016temporal}. But the difference is that the result of video-level incorporates all information, while pose-level only produces core information, which greatly improves the performance. Additionally, using pose information can contribute to the lightweight of model. For instance, for a 1024-frame video, video-level feature extraction with an output dimension of 512 would produce a data volume of $1024\times 512=524288$, while using pose information with 33 key points produces a data volume of only $1024\times 33 \times 3=101376$.

\subsection{Encoding Poses with Transformer}
\label{second}
Here we specify our data representation for the Transformer Encoder, which requires input batch size, sequence length, and embedding dimensions. In our pose-level approach, each frame is a batch, the number of key points in each frame is the sequence length, and the feature dimension of each key point is the embedding dimension.

First we get the pose of each frame ${p_i}\in \mathbb{R}^{D\times K}$ through the Pose Estimation Network, where $i\in {1, 2, \dots, T}$ is the frame index, $K$ is the number of key points, and $D$ is the dimension of each key point. We further define $p_i = {\{k_j\}}^{K}_{1}$ to represent each key point, where $k_j\in \mathbb{R}^D$, and we embed it to obtain richer information. Our embedding projection $\mathrm{\bf{E}}$ is a simple MLP network with ReLU as the activation function. These calculations can be defined as:
\begin{equation}
\begin{split}
\mathrm{\bf{Z}}^0 = [\mathrm{\bf{E}}(k_1), \mathrm{\bf{E}}(k_2), \dots, \mathrm{\bf{E}}(k_K)]^T
\end{split}
\end{equation}
where $\mathrm{\bf{E}}(k_j)\in \mathbb{R}^{D^{\prime}}$ is the embedding feature. Then the next Transformer takes $\mathrm{\bf{Z}}^0$ as input and encodes it with self-attention. Given $\mathrm{\bf{Z}}^0\in \mathbb{R}^{K\times D^{\prime}}$ with $K$ key point features, each of which is $D^{\prime}$-dimensional, $\mathrm{\bf{Z}}^0$ is projected using $\mathrm{\bf{W}}_Q\in \mathbb{R}^{D^{\prime}\times D_q}$, $\mathrm{\bf{W}}_K\in \mathbb{R}^{D^{\prime}\times D_k}$, $\mathrm{\bf{W}}_V\in \mathbb{R}^{D^{\prime}\times D_v}$, where $D_k=D_q$, to extract feature representations query($\mathrm{\bf{Q}}$), key($\mathrm{\bf{K}}$) and value($\mathrm{\bf{V}}$), which can be defined as:
\begin{equation}
\begin{split}
&\mathrm{\bf{Q}}=\mathrm{\bf{Z}}^0\times \mathrm{\bf{W}}_Q\\
&\mathrm{\bf{K}}=\mathrm{\bf{Z}}^0\times \mathrm{\bf{W}}_K\\
&\mathrm{\bf{V}}=\mathrm{\bf{Z}}^0\times \mathrm{\bf{W}}_V
\end{split}
\end{equation}
and the output of self-attention can be computed as:
\begin{equation}
\begin{split}
\mathrm{\bf{Attn}}=\mathrm{Softmax}(\frac{\mathrm{\bf{Q}}\mathrm{\bf{K}}^T}{\sqrt{D_q}})\mathrm{\bf{V}}
\end{split}
\end{equation}
where $\mathrm{\bf{Attn}}\in \mathbb{R}^{K\times D^{\prime}}$. Also, we use common multi-head self-attention (MHSA) to make several self-attention operations calculate in parallel.

Now we introduce the overall architecture of Transformer Encoder, which has $L$ layers with each layer consisting of MHSA and MLP blocks. Also, LayerNorm and Residual Connection are applied before and after every MHSA or MLP block, respectively. Because the number of key points of each frame is  a bit less, so our encoder does not include the downsampling module that other models may have. The overall process can be defined as:
\begin{equation}
\begin{split}
&\mathrm{\bf{\hat{Z}}}^l = \mathrm{MHSA}(\mathrm{LN}(\mathrm{\bf{Z}}^{l-1})) + \mathrm{\bf{Z}}^{l-1}\\
&\mathrm{\bf{Z}}^l = \mathrm{MLP}(\mathrm{LN}(\mathrm{\bf{\hat{Z}}}^l)) + \mathrm{\bf{\hat{Z}}}^l
\end{split}
\end{equation}
where $\mathrm{\bf{Z}}^{l-1}$, $\mathrm{\bf{\hat{Z}}}^l$, $\mathrm{\bf{Z}}^l\in \mathbb{R}^{K\times D^{\prime}}$.


\subsection{Pose Mapping}
\label{third}
Taking the Encoder output $\mathrm{\bf{Z}}^L\in \mathbb{R}^{K\times D^{\prime}}$ as input, Pose Mapping module outputs probability scores $\mathrm{\bf{S}}\in \mathbb{R}^{C}$ of the current frame over all action classes. We perform binary classification after Sigmoid activation for each class, with the two salient poses of each class represented by the same bit data. To realize such a module, we use a very lightweight MLP network, which avoids the complexity. First, the two dimensions $K$ and $D^{\prime}$ of $\mathrm{\bf{Z}}^L$ are flattened into $\mathbb{R}^{KD^{\prime}}$, and then it passes through an MLP module, where the output channels is set to $C$, which can be defined as:
\begin{equation}
\begin{split}
\mathrm{\bf{S}} = \sigma(\mathrm{MLP}(\mathrm{Flatten}(\mathrm{\bf{Z}}^L)))
\end{split}
\end{equation}
where $\sigma$ represents the Sigmoid activation function.

With such Pose Mapping, we can obtain the scores of single frame. It should be noted that we extract the poses of all frames, and use the convenience of matrix operations to obtain scores in parallel, which is actually consistent with the idea of mini batch. So at last, we combine the scores of all frames to get the video score matrix $\mathrm{\bf{\hat{S}}}\in \mathbb{R}^{C\times T}$, where $T$ represents the number of frames in the current video. 


\subsection{Action-trigger Module}
\label{fourth}
We use the lightweight Action-trigger Module to obtain the final output $Y$, the repetitive action count, which has a time complexity of $\mathcal{O}(n)$. First, we get the scores $S_c\in \mathbb{R}^T$ of a given action class from $\mathrm{\bf{\hat{S}}}$. Then, we scan all frames and use the action-trigger mechanism to count when the two salient poses of the action class occur sequentially. We set upper and lower bounds to distinguish the scores of the two salient poses, which cluster non-salient poses in the middle and easily classify the salient poses to the two ends.

\subsection{Losses and Metric Learning}

The modules need to be trained are Embedding, Transformer Encoder and Pose Mapping, and because we perform binary classification for each class, so we use the Binary Cross Entropy Loss, which can be defined as follows:
\begin{gather}
\mathcal{L}_{bce} = -\frac{1}{N}\sum\limits_{i=1}^{N}(\frac{1}{C}\sum\limits_{j=1}^{C}loss(i,j))  \\
 loss(i,j)=y_{ij}\log p_{ij} + (1-y_{ij})\log(1-p_{ij})
\end{gather}
where $N$ represents the batch size (in our method, each frame is a batch), $C$ represents the number of classes, $y$ and $p$ are the labels and our predictions, respectively.

Moreover, we use Metric Learning to improve our Encoder and introduce the Pose Triplet Loss. Given a pose, Encoder produces higher-level features $\mathrm{\bf{Z}}^L$, which should be more representative. As shown in Figure \ref{fig5}, we achieve this with Triplet Margin Loss function, which selects anchors, same class positive samples, and different classes negative samples in a batch. It can be expressed as:
\begin{equation}
\begin{split}
\mathcal{L}_{tri} = \mathrm{max}(\mathrm{CS}(a,p)-\mathrm{CS}(a,n)+\mathrm{margin},0)
\end{split}
\end{equation}
where $a$, $p$, $d$ are anchors, positive and negative samples, and $\mathrm{CS}$ represents the Cosine Similarity to measure the distance between features. We pay more attention to hard samples, where the distances between anchors and negative samples are even smaller than those of positive samples. After Metric Learning, the poses of each action can be distinguishable, which cluster in the high-level space.

At last, our overall training combines these two losses:
\begin{equation}
\begin{split}
\mathcal{L} = \mathcal{L}_{bce} + \alpha\mathcal{L}_{tri}
\end{split}
\end{equation}
where $\alpha$ is the weight factor to control the two losses in the same numeric scale.
\subsection{Implementation Details}

\noindent{\bf Training.} We use the \emph{RepCount-pose} and \emph{UCFRep-pose} dataset we created to train our model. Only the frames with salient poses are inputted into the network instead of the entire video to speed up the fitting.

\noindent{\bf Inference.} During inference, the entire video sequence is inputted into the model. The poses of all frames pass through the Encoder and Pose Mapping, and then enter the Action-trigger Module to output the repetitive count.
\section{Implementation}
\label{sec:impl}

At \company, we have deployed \sysname in our internal clusters to serve daily DL workloads.
The internal clusters consist of heterogeneous GPUs, including NVIDIA T4 GPU and NVIDIA A10 GPU.
Integrated with Kubernetes~\cite{k8s}, \sysname manages thousands of GPUs in each cluster and more than 20,000 GPUs in all.

\parabf{Service manager.}
For online workloads, we use the existing service manager at \company which deploys containers, discovers service, and autoscales horizontal pods.

\parabf{Global manager.}
We modify the Kubernetes scheduler to schedule offline workloads.
The workload profiler takes several dedicated GPUs, whose number is negligible to the total number of GPUs.
When a new offline workload comes, the workload profiler performs a few dry runs of the workload and utilizes the NVIDIA Data Center GPU Manager (DCGM) tools~\cite{dcgm} and NVIDIA Management Library (NVML)~\cite{nvml} libraries to collect GPU metrics.
We collect about 2,000 data for each GPU type to train the speed predictor.
The MLPs of the speed predictor have four layers with hidden size $64\times 64$.
The MLPs are trained with momentum SGD optimizer~\cite{ruder2016overview} in PyTorch v1.8.0~\cite{paszke2019pytorch} until they converge.
\sysname invokes the scheduler periodically to schedule all offline workloads.
When moving workloads, we record checkpoints of offline workloads and restart the workloads after transmitting the models and checkpoints.
As the datasets are usually colossal, we store the datasets in a remote file system and fetch data during the execution.
We implement the scheduler as a third-party plugin to the Kubernetes scheduler.


\parabf{Local executor.}
Each local executor executes online workloads according to the service manager and offline workloads according to the global manager.
DL workloads are executed in Docker containers with our customized components.
We add Best-Effort GPU DevicePlugin in Kubernetes and relevant control paths with Kubelet and \sysprobe for offline workloads.
To control SM percentage, we leverage the environment variable $CUDA\_MPS\_ACTIVE\_THREAD\_PERCENTAGE$ provided by MPS.
The GPU monitor collects resource metrics through DCGM~\cite{dcgm} and NVML~\cite{nvml} for NVIDIA GPU.
The \sysprobe updates the state machine with the collected resource metrics and empirically-set thresholds.
When the state is unhealthy, the \sysprobe will ask the NodeManager in Kubernetes to evict offline workloads.
\bytecuda intercepts nearly 800 CUDA driver APIs for GPU memory allocation and kernel launch.
The GPU memory quota of offline workloads is fixed to $40\%$ as Figure~\ref{fig:motiv_gpu_resource} reports that most online workloads use less than $60\%$ GPU memory.
We adopt the cpuset of Cgroup for CPU isolation.
For memory, \sysname will evict offline workloads if memory usage is higher than a threshold or the kernel swap daemon is busy for a long time.
The parameters to calculate GPU load in Equation~\ref{equ:gpu_load}$\&$\ref{equ:clock_factor} are empirically selected through trial-and-error.

We present in section~\ref{ssec:faces} an application of PnP-HVAE on face images, using a pretrained state-of-the-art hierarchical VAE. 
Next, we study the application of our framework to natural images. To that end, we introduce  in section~\ref{ssec:patchVDVAE}  a patch hierachical VAE architecture, that is able to model natural images of different resolutions. In section~\ref{ssec:app_nat}, we provide deblurring, super-resolution and inpainting experiments to demonstrate the relevance of the proposed method.

Additional results are presented in Appendix~\ref{app:add}. All experiments can be reproduced using the code available at \url{https://github.com/jprost76/PnP-HVAE}.



\subsection{Face Image restoration (FFHQ)}\label{ssec:faces}
We first demonstrate the effectiveness of PnP-HVAE on highly structured data, by performing face image restoration.
Latent variable generative models can accurately model structured images such as face images \cite{karras2019style,vahdat2020nvae,child2021very,kingma2018glow}, and then be used to produce high quality restoration of such data. 
In our experiments, we use the VDVAE model of~\cite{child2021very}, pre-trained on the FFHQ dataset~\cite{karras2019style}, as our hierarchical VAE prior.
VDVAE has $L=66$ latent variable groups in its hierarchy and generates images at resolution $256\times256$.

We compare PnP-HVAE with the intermediate layer optimization algorithm (ILO)~\cite{daras2021intermediate} that is based on a different class of generative models than HVAE. ILO is a GAN inversion method which optimizes the image latent code along with the intermediate layer representation of a StyleGAN to generate an image consistent with a degraded observation.
We use the official implementation of ILO, along with a StyleGAN2 model~\cite{karras2020analyzing, stylegan2pytorch}, that was trained for 550k iterations on images of resolution $256\times256$ from FFHQ.  
As VDVAE and StyleGAN models are not trained on the same train-test split of FFHQ, we chose to evaluate the methods on a subset of 100 images from the CelebA dataset~\cite{liu2018large}. 
For super-resolution, the degradation model corresponds to the application of a gaussian low-pass filter followed by a $\times 4$ sub-sampling, and the addition of a gaussian white noise with $\sigma=3$.
For the deblurring, we considered motion blur and  gaussian kernels, both with a noise level $\sigma=8$. %

We provide quantitative comparisons in table~\ref{table:comp_ILO}, along with a visual comparison of the results in figure~\ref{fig:face_restoration}.
PnP-HVAE has the best  PSNR and SSIM results for all the considered restoration tasks, while ILO provides better results  for the perceptual distance.
By jointly optimizing the image and its latent variable, PnP-HVAE provides  results that are both realistic and consistent with the degraded observation.
On the other hand,  ILO  only optimizes on an extended latent space. This method generates  sharp and realistic images with better LPIPS scores,   
but the results lack  of consistency with respect to the observation, which explains the overall lower PSNR performance. 






\subsection{PatchVDVAE: a HVAE for natural images}\label{ssec:patchVDVAE}
Available generative models in the literature operate on images of  fixed resolutions and
are either restrained to datasets of limited diversity, or even to registered face images~\cite{kingma2018glow,child2021very, vahdat2020nvae, karras2019style}, or requiring additional class information~\cite{brock2018large, dhariwal2021diffusion, song2020score, luhman2022optimizing}.
Fitting an unconditional model on natural images appears to be a more difficult task, as their resolution can change, and their content is highly diverse.
The complexity of the problem can be reduced by learning a prior model on patches of reduced dimension. 
For image restoration problems, the patch model can be reused on images of higher dimensions~\cite{zoran2011learning,prost2021learning,altekruger2022patchnr}. When the model is a full CNN, the prior on the set of the  patches can  be computed efficiently by applying the network on the full image~\cite{prost2021learning}.

We thus introduce  patchVDVAE, a fully convolutional hierarchical VAE.
Contrary to existing HVAE models whose resolution is constrained by the constant tensor at the input of the top-down block, patchVDVAE can generate images of different resolutions by controlling the dimension of the input latent. 
This amounts to defining a prior on patches whose dimension corresponds to the receptive field of the VAE. A similar model is used for image denoising in~\cite{prakash2021interpretable}.

 
For PatchVDVAE architecture, we use the same bottom-up and top-down blocks as VDVAE~\cite{child2021very}, and replace the constant trainable input in the first top-down block by a latent variable, to make the model fully convolutional (details on the  architecture are given in Appendix~\ref{app:details}). 
The training dataset is composed of $128\times 128$ patches extracted from a combination of DIV2K~\cite{agustsson2017ntire} and Flickr2K~\cite{Lim_2017_CVPR_workshops} datasets.
We perform data augmentation by extracting  patches at $3$ resolutions: HR-images and $\times 2$ and $\times 4$ downscaled images. 
The model is trained for $7.10^5$ iterations with a batch size of $64$. Following the recommendation of~\cite{hazami2022efficient}, we use Adamax optimizer with an exponential moving average and gradient smoothing of the variance.
We set the decoder model to be a gaussian with diagonal covariance, as in~\cite{luhman2022optimizing}.
PatchVDVAE is fully convolutional and can generate images of dimension that are multiples of $64$ as illustrated by
figure~\ref{fig:vdvae}.

\newlength{\patchwidth}
\setlength{\patchwidth}{0.135\columnwidth}
\begin{figure}[!ht]
    \centering
    \begin{subfigure}[t]{.34\columnwidth}\hspace{0.1cm}
        \setlength{\tabcolsep}{0.02pt}
\renewcommand{\arraystretch}{0}
        \begin{tabular}{*{2}{p{1.03\patchwidth}}}
            \includegraphics[width=\patchwidth]{figures_arxiv/patchVDVAE/samples/generated/64x64/setup-5-image-0018.png} &
            \includegraphics[width=\patchwidth]{figures_arxiv/patchVDVAE/samples/generated/64x64/setup-5-image-0016.png} \\
            \includegraphics[width=\patchwidth]{figures_arxiv/patchVDVAE/samples/generated/64x64/setup-5-image-0008.png} &
            \includegraphics[width=\patchwidth]{figures_arxiv/patchVDVAE/samples/generated/64x64/setup-5-image-0019.png}   
        \end{tabular}
    \end{subfigure}\hspace{-0.15cm}
    \begin{subfigure}[t]{.64\columnwidth}
\begin{tabular}{cc}\vspace{-0.1cm}
\includegraphics[width=2\patchwidth]{figures_arxiv/patchVDVAE/samples/generated/256x256/setup-2-image-0009.png}&
        \includegraphics[width=2\patchwidth]{figures_arxiv/patchVDVAE/samples/generated/256x256/setup-2-image-0002.png}\end{tabular}

    \end{subfigure}
    \caption{\label{fig:vdvae} Left: $64\times64$ patches samples from our patchVDVAE model trained on patches from natural images.
    Right: PatchVDVAE is fully convolutional and it can generate images of higher resolution (here: $128\times128$).\vspace{-0.2cm}}
\end{figure}

\subsection{Natural images restoration}\label{ssec:app_nat}
We  evaluate PnP-HVAE on natural image restoration.
For each task, we report the average value of the PSNR, the SSIM, and the LPIPS metrics on $20$ images from the test set of the BSD dataset~\cite{MartinFTM01}.\\


\noindent
{\bf Image deblurring.}
In the experiments, we consider $2$ gaussian kernels and $2$ motion blur kernels from~\cite{levin2009understanding}, with $3$ different noise levels 
$\sigma \in \{2.55, 7.65, 12.75\}$.
As a baseline we consider  EPLL~\cite{zoran2011learning}, which learns a prior on image patches with a gaussian mixture model.
We also compare PnP-HVAE  with PnP-MMO and GS-PnP, $2$ competing convergent Plug-and-Play methods based on CNN denoisers.
PnP-MMO~\cite{pesquet2021learning} restricts the denoiser to be contraction in order to guarantee the convergence of the PnP forward-backard algorithm. GS-PnP~\cite{hurault2022gradient} considers a gradient step denoiser and reaches state-of-the-art performances of non converging methods~\cite{zhang2021plug}.
We set the temperature $\tau$  in our method as $0.95$, $0.8$ and $0.6$ for noise levels $2.55$, $7.65$ and $12.75$ respectively, and we let it run for a maximum of $50$ iterations. 
For the three compared methods we use the official implementations and pre-trained models provided by the respective authors. 
Details on the choice of hyperparameters for the concurrent methods are provided in the Appendix~\ref{app:details}
Figure~\ref{fig:deblurring_bsd} illustrates that our method provides correct deblurring results. 

According to table~\ref{tab:deb}, the performance of PnP-HVAE is between those of EPLL and GS-PnP and it outperforms PnP-MMO for large noise levels.\\

\begin{table}
\begin{center}\footnotesize
    \begin{tabular}{>{\centering}m{.3cm}*{5}{c}}
    $\sigma$ &Method & PSNR$\uparrow$ & SSIM$\uparrow$ & LPIPS$\downarrow$  \\ 
    \hline
    \multirow{4}{*}{\vcell{$2.55$}}
    & PnP-HVAE & $27.75$ & $0.79$ & $0.31$\\
    & GS-PNP \cite{hurault2022gradient} & $\mathbf{29.59}$ & $\mathbf{0.84}$ & $\mathbf{0.22}$\\
    & EPLL \cite{zoran2011learning} & $26.49$ & $0.71$ & $0.36$\\ 
    & PnP-MMO \cite{pesquet2021learning} & $\underbar{29.50}$ & $\underbar{0.83}$ & $\underbar{0.20}$ \\ \hline
    \multirow{4}{*}{\vcell{$7.65$}}
    & PnP-HVAE & $\underbar{26.36}$ & $\underbar{0.72}$ & $\underbar{0.40}$\\
    & GS-PNP \cite{hurault2022gradient} & $\mathbf{27.33}$ & $\mathbf{0.77}$ & $\mathbf{0.31}$\\
    & EPLL \cite{zoran2011learning} & $24.04$ & $0.66$ & $0.45$ \\ 
    & PnP-MMO \cite{pesquet2021learning} & $25.34$ & $0.69$ & $0.34$\\
    \hline
    \multirow{4}{*}{\vcell{$12.75$}}
    & PnP-HVAE & $\underbar{25.12}$ & $\mathbf{0.73}$ & $\underbar{0.47}$\\
    & GS-PNP \cite{hurault2022gradient} & $\mathbf{26.32}$ & $\mathbf{0.73}$ & $\mathbf{0.37}$\\
    & EPLL \cite{zoran2011learning} & $23.28$ & $0.61$ & $0.51$ \\ 
    & PnP-MMO \cite{pesquet2021learning} & $22.42$ & $0.53$& $0.54$ \\
    \hline
    &\vspace*{-.3cm}\\
            \multicolumn{2}{c}{Blur and motion kernels}& \multicolumn{3}{c}{
        \includegraphics*[scale=1]{figures_arxiv/kernels/4.png}\;\includegraphics*[scale=1]{figures_arxiv/kernels/7.png}\;\includegraphics*[scale=1]{figures_arxiv/kernels/9.png}\;\includegraphics*[scale=1]{figures_arxiv/kernels/11.png}} 
    \end{tabular}
        \caption{\label{tab:deb}Comparison  of PnP-HVAE  and other restoration methods on deblurring. Results are averaged on $4$ kernels.\vspace{-0.2cm}}% on image deblurring.}
    \end{center}
\end{table}

\begin{figure}
    
    \begin{subfigure}[h]{\linewidth}
        \centering
        \includegraphics*[width=\columnwidth]{figures_arxiv/deb_s255_k7.pdf}\vspace{-0.1cm}
        \caption{Gaussian blur, $\sigma=2.55$}
    \end{subfigure}
    \begin{subfigure}[h]{\linewidth}
        \centering
        \includegraphics*[width=\columnwidth]{figures_arxiv/deb_s765_k11.pdf}\vspace{-0.1cm}
        \caption{Motion blur, $\sigma=7.65$}
    \end{subfigure}\vspace*{-0.1cm}
    \caption{\label{fig:deblurring_bsd} Natural image deblurring\vspace{-0.1cm}}
\end{figure}

\noindent {\bf Effect of the temperature.}
PnP-HVAE gives control on the temperature of the prior over the latent space.
In figure~\ref{fig:temp_effect}, we illustrate that reducing the temperature increases the strength of the regularization prior. In this example the tuning $\tau=0.7$ produces the best performance.\\
\begin{figure}[!ht]
   
    \includegraphics[width=\columnwidth]{figures_arxiv/demo_temp.pdf}\vspace{-0.15cm}
    \caption{ \label{fig:temp_effect} Effect of the temperature in PnP-VAE on a deblurring problem, with $\sigma=7.65$.\vspace{-0.15cm}}
\end{figure}


\noindent
{\bf Image inpainting.}
Next we consider the task of noisy image inpainting. 
We compose a test-set of 10 images from the validation set of BSD~\cite{MartinFTM01} and we create masks
  by occluding diverse objects of small size in the images. 
A gaussian white noise with $\sigma=3$ is added to the images.
As a comparaison, we still consider GS-PnP and EPLL.
For PnP-HVAE, the temperature is set to $\tau=0.6$, and the algorithm is run for a maximum of $200$ iterations, unless the residual $||\x_{k+1}-\x_k||$ is on a plateau.
We provide on Table~\ref{tab:inpainting_bsd} the distortion metrics with the ground truth, as well as a visual
\begin{table}



\begin{center}
    \begin{tabular}{cccc}
        & PSNR$\uparrow$ & SSIM$\uparrow$ &LPIPS$\downarrow$ \\\hline
        PnP-HVAE  & $\mathbf{29.54}$ & $\mathbf{0.93}$ & $\mathbf{0.06}$\\
        GS-PNP & $28.52$ & $\mathbf{0.93}$ & $0.09$\\
        EPLL & $\underline{29.16}$ & $\mathbf{0.93}$ & $\mathbf{0.06}$\\
    \end{tabular}
    \caption{\label{tab:inpainting_bsd}Quantitative evaluation for inpainting on BSD.}
    \end{center}
\end{table}
comparison on figure~\ref{fig:inpainting_bsd}. 
With its hierarchical structure,  PnP-HVAE outperforms the compared methods. \vspace{0.05cm}



\begin{figure}[!h]
    \includegraphics[width=\columnwidth]{figures_arxiv/demo_inp_bsd2.pdf}\vspace{-0.1cm}
    \caption{\label{fig:inpainting_bsd}Natural image inpainting\vspace{-0.3cm}}
\end{figure}











\section{Related work}
% There is extensive recent work on speeding up analytical queries due to the need for consistent execution times in the face of the explosive growth in the volume of available data.
% In this section, we divide existing work into two categories: maintaining data freshness in MVs (\Cref{sec:server_side}) and optimizations for minimizing ad-hoc query latency (\Cref{sec:client_side}).

% \subsection{Maintaining Data Freshness in MVs}
% \label{sec:server_side}
% There exists a variety of data warehousing applications aimed at supporting low-latency analytical queries on fresh data.
% In particular, these applications require efficiency in the propagation of newly ingested data into downstream MVs.
 
\mypara{Efficient MV Refresh}
Incremental view maintenance (IVM) aims to update MVs to reflect newly ingested data, taking advantage of already computed results to perform the update in a manner more efficient than computing from scratch (full refresh)
~\cite{ahmad2012dbtoaster,mcsherry2013differential,armbrust2013generalized,zeng2016iolap, palpanas2002incremental, griffin1995incremental, agiwal2021napa, braun2015analytics}. 
There is an abundance of work in IVM, including incremental updates on duplicate values~\cite{griffin1995incremental}, non-distributive aggregate functions~\cite{palpanas2002incremental}, higher-order views~\cite{ahmad2012dbtoaster}, and sliding windows~\cite{braun2015analytics}. 
More recent works also investigate the scalability aspect of IVM, proposing scale-independent updates~\cite{armbrust2013generalized} and sampled views~\cite{zeng2016iolap}. Since \system is applicable to arbitrary SQL statements, \system is orthogonal to and is fully compatible with existing IVM techniques.

\mypara{MV Refresh Scheduling}
There exist works on scheduling the refresh of a MV set focusing on resolving cyclic dependencies~\cite{folkert2005optimizing}, minimizing weighted average staleness~\cite{golab2009scheduling}, and prioritizing MVs with the highest speedups on predicted future queries~\cite{ahmed2020automated}.
\system's scheduling to speed up the end-to-end refresh of the MV set is not addressed in existing works.

\mypara{DAG Workflow Scheduling}
The execution of workloads consisting of individual jobs with acyclic dependencies is a well-studied topic~\cite{apacheoozie,sparkdag,marchal2018parallel,bathie2020revisiting,baruah2022ilp}; many of these techniques can be applied to MV refresh runs studied in this paper.
Existing workflow scheduling systems such as Apache Oozie~\cite{apacheoozie}, Apache Airflow~\cite{airflow}, and Spark DAG scheduler~\cite{sparkdag} automate the execution of user-defined workflows following a topological order.
There are recent works aimed at finding more optimal execution orders in terms of peak memory usage~\cite{marchal2018parallel, bathie2020revisiting} and execution time on parallel platforms~\cite{baruah2022ilp}.
While \system is designed for use with MV refresh runs/workloads, our technique on joint scheduling and optimization can be reasonably applied to general workloads as a possible future direction.

% \paragraph{Incremental MV indexing}
% Update-optimized indices such as the log-structured merge-trees (LSM)~\cite{o1996log} are used for indexing MVs due to frequent updates induced by data ingestion~\cite{gupta2016mesa,agiwal2021napa}.
% \system is orthogonal to indexing: \system is capable of efficiently performing MV refresh runs regardless of whether the individual MVs are indexed or not.

% \subsection{Ad-hoc Query Latency Reduction}
% \label{sec:client_side}

% The minimization of ad-hoc analytical query response times is a well-studied topic due to latency being negatively correlated with the productivity of a data analyst during a data analysis session~\cite{liu2014effects}.
% Sessions are commonly conducted within visualization systems that contain a variety of optimization techniques to ensure that query response times fall within a certain latency tolerance.

% \mypara{Data prefetching}
% Data is often loaded into memory on a by-need basis in visualization systems to minimize interference with user-issued query computations~\cite{mani2017effective,xin2021enhancing,galakatos2017revisiting, yan2020auto, battle2016dynamic, crotty2016case, jalaparti2018netco}. 
% Query-time data retrieval can be significantly expedited by anticipating the data usage of the user in future queries and pre-loading the data into memory during the downtime between user queries (`think time'). SMART~\cite{mani2017effective} prefetches data for modified versions of current user-issued queries with different filters and dimensions. A-WARE~\cite{crotty2016case} maintains a sample store constantly refined through ingesting data based on speculations of future plots.
% ForeCache~\cite{battle2016dynamic} uses an SVM to predict the user's current analysis phase and accordingly prefetches data tiles partitioned based on different numerical values. NetCo predicts future queries via log analysis, and solves an ILP formulation to prefetch data to maximize the number of SLO-meeting queries~\cite{jalaparti2018netco}.
% In the case of MV refresh workloads, `think time' is nonexistent as individual MVs are refreshed back-to-back, rendering data prefetching techniques non-applicable.

\mypara{Intermediate Data Caching}
Some existing data visualization systems cache user-defined variables to support the typical incremental construction of data visualizations~\cite{zgraggen2016progressive, eichmann2020idebench} during data analysis sessions~\cite{jupyter, rstudio, colab}. 
Recent work proposes a management scheme for these cached variables under a memory constraint that greedily keeps variables with the highest estimated time savings based on predicted future user behavior via neural networks~\cite{xin2021enhancing}.
While useful for data visualization, a greedy approach to memory management fails to achieve satisfactory results compared to \system.

\mypara{Intermediate Result Reuse}

There exist works on storing intermediate results from computations to speedup future computations~\cite{yang2018intermediate, dursun2017revisiting, nagel2013recycling, michiardi2019memory, galakatos2017revisiting}.
Studied topics include the identification of reuse opportunities by finding overlaps in computation graphs of successive jobs~\cite{yang2018intermediate, michiardi2019memory},
selective storage under a space constraint with heuristics such as reuse probability~\cite{dursun2017revisiting}, expected savings~\cite{yang2018intermediate}, and recompute-storage cost difference~\cite{nagel2013recycling},
and rewriting incoming jobs to take advantage of stored intermediates~\cite{galakatos2017revisiting}.
These works share similarity with \system in their selection of items to store under a memory constraint, however, \system's problem setting requires it to uniquely consider the joint (re)ordering of job executions along with the selection of items.

% work that considers both job execution (re)order as well as intermediate result caching with a bounded amount of memory. but notably lack the joint aspect of \system and cannot be used to achieve immediate speedup on an incoming MV refresh run if no intermediates are stored beforehand. 

\mypara{Incremental Query Processing} Incremental processing (IQP) is useful for cases where not all data required for a query is immediately available. Similar to online aggregation~\cite{hellerstein1997online}, initial results of a query are computed on a subset of required data and progressively refined as the rest of the required data arrives in a predictable pattern~\cite{tang2019intermittent,wangtempura}. Tang et al. propose a dynamic programming formulation to pick intermediate states to store in memory given a limited memory budget~\cite{tang2019intermittent}. Tempura rewrites the query plan for more efficient execution based on predicted data arrival patterns~\cite{wangtempura}. While similarities exist between the problem setting of IQP and \system, such as management of bounded memory, \system notably includes additional joint optimization for the order of MV updates.

% \paragraph{Sampling}
% Sampling has seen wide use in visualization systems for reducing the computation time of ad-hoc queries by computing an approximate result over a subset of data as exact results are not always required by the user~\cite{crotty2016case, mani2017effective, zgraggen2014panoramicdata, kraska2021northstar, galakatos2017revisiting, kandula2016quickr}. 
% Commonly studied topics in sampling for ad-hoc queries include complex query sampling~\cite{kandula2016quickr}, rare event aggregation~\cite{kraska2021northstar, galakatos2017revisiting}, and maintaining consistency between related sampled visualizations~\cite{zgraggen2014panoramicdata}.
% Sampling server-side at the MV level compromises the assumptions of downstream applications and is thus not considered in \system.

% \paragraph{Progressive visualization}
% The latency tolerance for time-consuming queries can be circumvented by presenting a partially-computed visualization to the user within the tolerance, which is then incrementally refined until it is fully accurate~\cite{rahman2017ve, zgraggen2016progressive, crotty2015vizdom, kraska2021northstar, kamat2017infiniviz}.
% Example plots which benefit from progressive visualization include bar charts~\cite{kamat2017infiniviz} and heatmaps~\cite{rahman2017ve}.
% Similar to sampling, study on this topic is orthogonal to \system as pushing out partially-updated MVs compromises downstream assumptions.
% We provide some comments on the growth conditions which constituted the majority of our analysis in sections \ref{sec:Hmixing} and \ref{sec:Hsigma}. In the simplest cases of Lemma \ref{lemma:unstableGrowth}, growth was established in an analogous fashion to the old one-step expansion condition (\ref{eq:oldOneStepExpansion}), finding the relevant Jacobians $M_j$ and checking that their expansion factors $K(M_j)$ satisfy
\begin{equation}
    \label{eq:discussionOneStep}
    \sum_j \frac{1}{K(M_j)} <1.
\end{equation}
For the more complicated cases, the inductive method used to establish growth near the accumulation points in Lemma \ref{lemma:unstableGrowth} and the weakened one-step expansion condition (\ref{eq:oneStep}) both address the same fundamental issue: the splitting of unstable curves by singularities into an unbounded number of small components. They circumvent this obstacle in rather different ways, however. While (\ref{eq:oneStep}) generalises (\ref{eq:discussionOneStep}) to ensure an growth of unstable curves `on average' (see \cite{chernov_statistical_2009} for a precise statement), our inductive method is a more direct adaptation of (\ref{eq:discussionOneStep}), using it to generate contradictory geometric conditions which a hypothetical non-growing unstable curve must satisfy. It may be possible to prove Theorem \ref{sec:Hmixing} using (\ref{eq:oneStep}) as the basis for growth. Since we required (\ref{eq:oneStep}) anyway for proving Theorem \ref{thm:HsigmaExp}, this could potentially condense our analysis, but only to a minor extent. A convenience of the method used in section \ref{sec:Hmixing} is that, by way of the `simple intersection' property, it naturally gives geometric information on the images of manifolds, useful for proving the property \textbf{(M)} of Theorem \ref{thm:katok-strelcyn}.

We expect that essentially analogous analysis can be applied to establish mixing properties in a wide class of piecewise linear non-uniformly hyperbolic maps, including those (like the OTM) which sit on the boundary of ergodicity and beyond. While we have relied on the precise partition structure of $H_\sigma$, its fundamental feature (self-similar sequences of elements $A^k$, sharing boundaries with its neighbours $A^{k-1},A^{k+1}$ and accumulating onto some point $p$) is quite typical to return map systems. See, for example, those of various stadium billiards \cite{chernov_chaotic_2006,chernov_improved_2008,chernov_statistical_2009} and LTMs \cite{springham_polynomial_2014}. Indeed, the same method can be used to prove the Bernoulli property for non-monotonic LTMs \cite{myers_hill_mixing_2022}, where monotonicity of the manifold images cannot be assumed and the classical argument \cite{sturman_mathematical_2006} fails. The OTM is the pointwise limit of these maps as the boundary shrinks to null measure. It further has utility in proving growth conditions for maps which are uniformly hyperbolic but possess regions $A_j$ where the hyperbolicity is very weak, signified by $K(M_j) \approx 1$, so that (\ref{eq:discussionOneStep}) fails. Typically this leads to suboptimal bounds on mixing windows, see e.g. \cite{wojtkowski_model_1981,przytycki_ergodicity_1983,myers_hill_family_2022}. The map $H_{(\eta,\eta)}$ for $\eta \approx 1/2$ is another example, possessing weak hyperbolicity over $A_2, A_3$. Letting $\varepsilon = |\eta-1/2|>0$, there is an upper bound $N = N(\varepsilon)$ on escape times from the intersections $A_2\cap \sigma, A_3 \cap \sigma$. The growth lemma then follows by applying the inductive step roughly $N$ times and can be established for arbitrarily small $\varepsilon$, opening the door to establishing optimal mixing windows.

The above gives two examples of piecewise linear perturbations to $H$ where mixing with respect to Lebesgue is preserved and our methods can be applied. Nonlinear perturbations to the shear profiles complicate the analysis in several ways. Firstly as the map's Jacobians takes on a broader range of values, cone invariance becomes an increasingly harder condition to establish. Cones must be widened, giving looser bounds on expansion factors, which may already be weak due to new regions of weaker stretching. This, together with the change from polygonal to curvilinear return time partition elements and nonlinear local manifolds, adds some complexity to showing growth conditions. This does not rule out certain (small) nonlinear perturbations however. There is some leeway in the inequalities which govern cone invariance and growth of local manifolds, the latter of which is not too dissimilar from the piecewise linear setting (see Lemmas \ref{lemma:piecewiseApprox}, \ref{lemma:componentLength}). Certain small perturbations would not alter the \emph{topological} structure of the return time partition, i.e. which elements share boundaries, the key information needed for setting up the induction. Finally while the partition elements would no longer be polygonal, only coarse geometric information is required for verifying each inductive step. Following the above, a potential perturbation could be to replace the linear portions of each shear by a cubic, perturbing the tent profile
\[  f(t) = \begin{cases} 2t & 0 \leq t \leq 1/2, \\ 2(1-t) & 1/2 \leq t \leq 1 ,\end{cases} \]
of the OTM shears to
\[  f_a(t) = \begin{cases} \frac{1}{8} t \left(16 - a + 6at - 8at^{2} \right) & 0 \leq t \leq 1/2, \\ \frac{1}{8}\left(1-t\right)\left( 16 - a + 6a\left(1-t\right) - 8a\left(1-t\right)^{2}\right)  & 1/2 \leq t \leq 1, \end{cases}   \]
for $a>0$. For small enough $a$ the gradient range $f'(t)$ is restricted to small neighbourhoods of $\{ 2, -2\}$ and the escape time partition retains a similar structure. We illustrate this in Figure \ref{fig:perturbations}, showing escapes from the square $S_3$ under the map $G \circ F$, equivalent to escapes from the perturbed $A_3$ under the $G \circ F$, but with a cleaner geometry for comparison. When $a$ is too large the analogy to the OTM breaks down. At $a=16$ the map is twice differentiable everywhere and features a new source of slowed mixing, the Jacobian is the identity at the corner points $x,y \in \{  0, 1/2 \}$ giving locally parabolic behaviour (visible in the escape time partition). 

\begin{figure}
    \centering
    \includegraphics[width=0.24 \linewidth]{0.png}
    \includegraphics[width=0.24 \linewidth]{4.png}
    \includegraphics[width=0.24 \linewidth]{8.png}
    \includegraphics[width=0.24 \linewidth]{16.png}
    \caption{Partition of escape times from $S_3$ under the mapping $F \circ G$ for $a= 0,4,8,16$. }
    \label{fig:perturbations}
\end{figure}
\section{Discussion}
% \paragraph{Conclusions}
In this paper we proposed the Alias-Free Convnet, which for the first time, is guaranteed to eliminate any aliasing effects in the model, to ensure the output is invariant to any input shifts (even sub-pixel ones), and to ensure the internal representations are equivariant to any shifts (even sub-pixel ones). 
We demonstrate this numerically and show this leads to (certified) high performance under adversarial shift-based attacks --- in contrast to existing models which degrade in performance. 
However, this comes at a cost, such as a 1.08\% reduction in standard test accuracy (as methods that increase robustness often reduce accuracy). 
% We discuss additional limitations in \cref{sec:limitaions-append}.  

\subsection{Accuracy}
Although our model has improved robustness in comparison to the baseline model and other shift-invariance methods, it has a lower accuracy on the original test set. 
This result makes sense since robustness often comes at the cost of accuracy. 
Specifically, the property of perfect shift-invariance is architecturally forced in our model, in contrast to other CNNs where it can be violated. This may seem as a reduction of the hypothesis set.
However, in \Cref{table:aal-modifcations-accuracy} we observe that the highest drop in accuracy does not occur at the last modification, where the model becomes shift-invariant, but rather as a result of modifying the Normalization Layer to be alias-free.
Although the proposed alias-free normalization layer was designed to remain similar to the original model LayerNorm, other alias-free normalization methods may exist and lead to a higher accuracy than ours, without hurting robustness. 
Another drop in accuracy occurs as a result of replacing GeLU in polynomial activation, which surprisingly does not happen in the regular setting (models without cyclic convolutions); a polynomial activation in a cyclic setting leads to a reduction of 0.4\% in accuracy (\cref{table:aal-modifcations-accuracy}) while in a non-cyclic setting it leads to a reduction of 0.1\% only (\cref{table:polynomial-vit-results}). 
It implies that additional hyper-parameter tuning might help in the recovery of this accuracy drop.
%In addition, we note again that the modifications in our model effectively remove non-linearities from the baseline model, and that ConvNeXt-AFC is practically a polynomial of the input with a degree which is the depth of the convnet multiplied by two.  
In addition, we note again that the modifications in our model effectively remove non-linearities from the baseline model, and that ConvNeXt-AFC is practically a polynomial of the input with a degree that is an exponent of the convnet's depth.  
Thus, using a wider or deeper convnet might help in closing the accuracy gap.
% To test the hypothesis that accuracy reduction is a result of the Layer Norm modification, we test a model for which this is the only change. \daniel{where are the results?}
% We leave this kind of normalization layer search as future work.
% We leave examining these possible solutions as future work.



% \subsection{Limitations} \label{sec:limitaions-append}

\subsection{Runtime performance}
Although the AFC model has only a small amount of additional parameters in the polynomial activation function, it has a higher computation cost than the baseline (see \cref{table:baenchmark-results}). The main reason for that is that while the activation function in the baseline model is a single pointwise operation, our activation requires upsampling and downsampling which are rather expensive. This issue has been addressed in a previous study \citep{Karras2021Alias-FreeNetworks}, where a similar scheme for partially alias-free activation has been used. They combined all the required operations to a single CUDA kernel which (according to them) led to a speed-up of at least x20 over native PyTorch implementation, and in total to  x10 speed-up in training time.
Our model training time is only 5 times higher than the baseline training time, therefore it is reasonable to assume such efficient implementation may significantly reduce the training time gap.


\begin{table}[h]
\caption{\textbf{Training and evaluation performance.} Train time was measured on Nvidia A6000 x 8 using the maximal possible batch-size per model due to memory constraints. Evaluation time was measured on a single A6000 with batch-size 256.}
  \label{table:baenchmark-results}

  \centering
\resizebox{1.0\linewidth}{!}{
    \begin{tabular}{lll}
    \toprule

Model & Train time [hours] & Eval time [ms per sample] \\ 
\midrule

ConvNeXt-Baseline \cite{Liu2022A2020s}           & 84        &  1.39  \\
ConvNeXt-APS   \cite{Chaman2020TrulyNetworks}    & 93     &   1.56  \\
ConvNeXt-AFC  (ours)                          & 418    &  9.16 \\
\bottomrule
  \end{tabular}
  }
\end{table}

\subsection{Image translations}
\subsubsection{Circular translations}
The guaranteed robustness in the AFC model is limited to circular shifts, similarly to previous work \citep{Chaman2020TrulyNetworks}. 
Applying this kind of translation on a finite image causes edge artifacts and creates an unrealistic image (e.g., see \cref{fig:circular_crop_shifts}).
Although our model has improved robustness even in translations of the frame with respect to the scene (\cref{fig:crop-shift-bilinear}), which may seem as a more practical setting, information-loss makes guaranteed robustness impossible --- for example, consider an image in which a translation cause the classified object to get out of the frame. 
Although the certified robustness may seem not applicable, circular shifts can actually be practically relevant. 
For example, shifting an object over a constant (i.e., uniform) background will seem identical to a circular translation of the entire frame. 
This setting may be relevant in face recognition tasks and medical imaging (see \cref{fig:medical_shifts}). 
In addition, horizontal circular shifts are relevant for panoramic (360$^{\circ}$) cameras, e.g.~in autonomous cars (see \cref{fig:panoram_shifts}).


\begin{figure}[ht]
    \centering
    \begin{subfigure}[b]{0.32\linewidth}
    \centering
    \includegraphics[width=\linewidth]{images/attack_samples_medical_original.pdf}
    \caption{Original image}
    \end{subfigure}
    \begin{subfigure}[b]{0.32\linewidth}
    \centering
    \includegraphics[width=\linewidth]{images/attack_samples_medical_circ_shift.pdf}
    \caption{Circular shifted image}
    \end{subfigure}
\caption{
 \textbf{ Circular translation of a retinal image.} (a) Original Image \citep{Staal2004Ridge-BasedRetina}. (b) Circular translated image. The retinal image has a uniform background, hence circular shift is equivalent to a translation of the object within the image (i.e.~``crop-shift'').
  }
\label{fig:medical_shifts}
\end{figure}

\begin{figure}[ht]
    \centering
    \begin{subfigure}[b]{0.8\linewidth}
    \centering
    \includegraphics[width=\linewidth]{images/attack_samples_panorama_original.pdf}
    \caption{Original image}
    \end{subfigure}
    \begin{subfigure}[b]{0.8\linewidth}
    \centering
    \includegraphics[width=\linewidth]{images/attack_samples_panorama_shift.pdf}
    \caption{Circular shifted image}
    \end{subfigure}
\caption{ \textbf{Circular translation of a panoramic image.} (a) Original Image \citep{Armeni2017JointUnderstanding}. (b) Circular translated image, representing a translation of the camera with respect to the scene, without any edge artifacts.
    }
\label{fig:panoram_shifts}
\end{figure}



\subsubsection{Interpolation kernel}
In \Cref{sec:methods} we proved that AFC is robust even to sub-pixel shifts. %, looking at the input at the continuous domain. 
%From Digital signal perspective, as reflects in our experiments,
%(where we do not have access to continuous signals but rather to digital images)
Our robustness guarantees assume the digital image processed by the network corresponds to point-wise samples of a continuous-space image that had been convolved with a perfect anti-aliasing filter prior to sampling (though, empirically, our method performs well with other types of interpolation kernels, e.g. \cref{fig:crop-shift-bilinear}). %1/2T band-limited function we retrieve using sinc interpolation (i.e.~our upsample implementation, \cref{algo:Up-sample}). 
Although this may seem like a serious limitation, it is in fact the standard setting in any imaging system. %as the point-spread function (PSF) of the camera may differ from the sinc kernel. since upsampling is often performed with smaller interpolation kernels (e.g.~bilinear, bicubic), we argue this is actually the required method in imaging systems:   
Indeed, in any optical imaging system, the image impinging on the detector corresponds to the continuous scene convolved with a low-pass filter. 
This low-pass filter completely zeros out all frequencies above a cutoff frequency that is inversely proportional to the diameter of the aperture 
\citep{Goodman1969IntroductionOptics}. Thus, while in many domains perfect low-pass filtering is challenging to achieve, in Optics, this diffraction limit is in fact impossible to avoid. 
Cameras are designed such that the cutoff frequency of the aperture corresponds to the Nyquist frequency of the CCD array. 
In systems where the aperture can be modified by the user, the CCD array is typically adjusted to the minimal aperture width and an additional anti-aliasing filter element is inserted in front of the CCD \citep{Schoberl2010DimensioningCameras}, so that the Nyquist condition is still met for any chosen aperture diameter within the allowed range. It should be noted, however, that the digital image captured by the sensor typically undergoes a series of nonlinear operations within the image signal processor (ISP) of the camera.
This implies that shift equivariance may be lost already at the camera level, before the image reaches our convnet. Addressing these effects is beyond the scope of our work. 

%Had cameras not been designed this way, images captured by them would exhibit Moiré patterns.

\subsection{Polynomial activation functions}
Polynomial activation functions are relatively cheap to compute, which might motivate their use in the future, regardless of their advantage in the context of aliasing-free convnets. 
Despite this, the high curvature of the polynomial activation seems a barrier to its usage in a wider variety of architectures. 
To achieve good performance on CIFAR, \citet{Gottemukkula2019POLYNOMIALFUNCTIONS} implemented in ResNet \citep{He2015DeepRecognition} a stable version of polynomial activation function, by scaling the pre-activation by the $L_1$ of the layer, which bounds the maximal input element (note that this scaling technique is not aliasing-free and thus was not useful for our purposes). 
We observe this scaling is unnecessary in architectures with sparser usage of activation functions, such as ConvNeXt and ViT; in \Cref{table:polynomial-vit-results}, we show preliminary results suggesting a polynomial activation may be a reasonable substitute for the widely spread GeLU in Transformer-based architectures as well as convnets. 

\subsection{Future work}
This study shows how an aliasing-free convnet can be used to build an image classifier with certified shift robustness. 
Yet, the applications of such convnet are not limited to that purpose; our method can be applied in other domains in which aliasing has been shown to be damaging, such as generative models \citep{Karras2021Alias-FreeNetworks}. 
Furthermore, our method guarantees shift-equivariant internal representation, a stronger property than shift-invariance. 
Future work may examine the importance of this property in other tasks. For example, our method can be naturally expanded to construct a shift-equivariant convnets for segmentation. 

\subsubsection*{Acknowledgments}
 The research of DS was funded by the European Union (ERC, A-B-C-Deep, 101039436). Views and opinions expressed are however those of the author only and do not necessarily reflect those of the European Union or the European Research Council Executive Agency (ERCEA). Neither the European Union nor the granting authority can be held responsible for them. DS also acknowledges the support of Schmidt Career Advancement Chair in AI. TM was supported by grant 2318/22 from the Israel Science Foundation and by the Ollendorff Center of the Viterbi Faculty of Electrical and Computer Engineering at the Technion.


%Bibliography
\bibliography{references, references_manual}  

%%%%%%%%%%%%%%%%%%%%%%%%%%%%%%%%%%%%%%%%%%%%%%%%%%%%%%%%%%%%

\newpage
\appendix

% \section{Limitations and Future Work}

We summarize the limitations we have identified for our method and propose
future research directions.

\textbf{Parallel implementation:} 
With a focus on accuracy and algorithms, our implementation for this work is
serial. Some of the most time-consuming routines in our method can easily
benefit from a parallel implementation, while the same is not obvious for the
SAP solver and the Schur complement computation. Leveraging the power of
parallelization on modern hardware for these computations is an interesting area
for future investigation.

\textbf{Rotational invariance:} 
As with all other linear constitutive models, our linearized model with lagged
rotational component is not rotationally invariant. Thus it is not suitable for
simulation of extreme deformations using large time steps. For those scenarios,
we fall back to traditional nonlinear models with Hessian positive definite
corrections proposed in \cite{bib:teran2005robust}.

\textbf{Self-contact:} 
We do not consider self-contact at the moment due to the lack of support by our
geometry engine. Self-contact can be incorporated into our method by updating the
geometry engine to augment the set of contacts reported.

\textbf{Tunneling at high speeds:} Though our method has a lower computational
cost, it could benefit from continuous collision detection strategies
\cite{bib:li2020ipc} to provide constraints before contact is established. This
would allow to mitigate issues such as objects tunneling past each other at high
speeds. Efficient solution to mitigate this issue is a topic of active research
for the authors.

\textbf{Redundant constraints:} Our geometry engine often introduces a large
number of constraints to resolve contact. Similarly, welding a large number of
deformable mesh vertices to a rigid body (as done in Section
\ref{sec:bubble_gripper}) introduces many constraints. Even though our SAP
solver \cite{bib:castro2022unconstrained} provides existence and uniqueness
guarantees, a large number of constraints hurts performance as can be observed
in the \emph{Soft-bubble} example. We are currently investigating strategies to
significantly reduce the number of constraints without sacrificing accuracy.

\section{Polynomial activation function} \label{sec:poly-append}
\subsection{Coefficients initialization}
The Polynomial activation function is a point-wise polynomial:
\begin{equation}
    \mathrm{Poly}_2(x)=a_0+a_1x+a_2x^2 \,,
\end{equation}
where the coefficients $\{a_0,a_1,a_2\}$ are trainable parameters, which are shared per-channel.
They were initialized by fitting this function to the GeLU, as proposed by \cite{Gottemukkula2019POLYNOMIALFUNCTIONS}, to function as an approximation to the original activation function ConvNeXt works well with. This initialization gives the function presented in \Cref{fig:plot_poly_channels}.
Yet, the activation function may converge to a completely different function by the end of the training. 
Moreover, it may differ significantly between different layers and between different channels in the same layer. 
\Cref{fig:plot_poly_channels} shows the final activation function for five different channels in the first block of a trained model. 



\begin{figure}[ht]
    \centering
    \begin{subfigure}[b]{0.45\linewidth}
    \centering
    \includegraphics[width=\linewidth]{images/plot_gelu_poly_init.pdf}
    \end{subfigure}
    \begin{subfigure}[b]{0.45\linewidth}
    \centering
    \includegraphics[width=\linewidth]{images/plot_poly_channels.pdf}
    \end{subfigure}

\caption{
 \textbf{ The $\mathrm{Poly}_{2}(x)$ activation function.} In the left panel, we see how the coefficients are initialized to fit GeLU in range $\left[ \sqrt{2},\ -\sqrt{2} \right]$ (dashed lines).
  In the right panel we see how, in the trained model, the activation function may change significantly and converge to a different function in each channel.
  }
% \label{performace_steps_var_0.05}
\label{fig:plot_poly_channels}
\end{figure}

\begin{figure}[ht]
\centering
\includegraphics[width=0.45\linewidth]{images/plot_poly_c_search.pdf}
\caption{
  \textbf{Test accuracy on ImageNet200, non-cyclic convolutions, GeLU and polynomial ConvNeXt with different scales.} Scaling with $c = 7$ gave the best results in the scope of our search.}
\label{fig:polynomial-scale-results-imnet200}
\end{figure}



\subsection{Activation scaling}
During our experiments, we found out that scaling the inputs and outputs of the activation function, regardless of the coefficients themselves, may change the final model results.
Thus, we used the activation function:
\begin{equation}
    \mathrm{Poly}_{c} \left( x \right) =  c \mathrm{Poly}_2(c x)
\end{equation}
and were looking for the optimal scale $c$. 
This scaling factor can effectively be seen as a scaling factor of the weights initialization of the pointwise convolution layers before and after the activation function. 
In \cref{fig:polynomial-scale-results-imnet200} we ran a scan over different scale factors and compared the results of the non-cyclic convolution polynomial model, without additional alias-free modifications (ConvNeXt-Tiny), and on the final model (ConvNeXt-AFC). 
The scan was run on ``ImageNet200'', a subset of ImageNet consisting of 200 classes.
We compare the results to the results of non-cyclic convolution GeLU model. 
In the scope of our search, the best result was achieved with scale $c = 7$ for ConvNeXt-Tiny and scale $c=4$ for ConvNeXt-AFC. 
We used scale $c=7$ for the rest of the polynomial models, as it achieved slightly better results on the full dataset.
% . We ran a smaller search to verify the best scale on ImageNet 200 ($c = 7$) also leads to a reasonable result for the full ImageNet \cref{table:polynomial-scale-results-imnet}. 



% \begin{table}
%   \centering
%     \begin{tabular}{lll}
%         \toprule
%         Model &  &Test acc. \\ 
%         \midrule
%         ConvNeXt-baseline (GeLU) & &  81.48 \\
%         \midrule
%         ConvNeXt-polynomial & $c = 1$          & 81.12 \\
%         ConvNeXt-polynomial & $c = 2$          & 81.93 \\
%         ConvNeXt-polynomial & $c = 3$          & 82.57 \\
%         ConvNeXt-polynomial & $c = 4$          & 82.49 \\
%         ConvNeXt-polynomial & $c = 5$          & 82.7 \\
%         ConvNeXt-polynomial & $c = 6$          & 82.44 \\
%         ConvNeXt-polynomial & \textbf{$c = 7$} & \textbf{82.92} \\
%         ConvNeXt-polynomial & $c = 9$          & 81.28 \\
%         ConvNeXt-polynomial & $c = 11$         & 80.53 \\
%         \bottomrule
%     \end{tabular}
% \caption{Test accuracy on ImageNet200, non-cyclic convolutions, GeLU and polynomial ConvNeXt with different scales. Scaling with $c = 7$ gave the best results in the scope of our search.}
%   \label{table:polynomial-scale-results-imnet200}
% \end{table}

% \begin{table}
%   \centering
%     \begin{tabular}{lll}
%         \toprule
%         Model &  &Test acc. \\ 
%         \midrule
%         ConvNeXt-baseline (GeLU) & &  82.11 \\
%         \midrule
%         ConvNeXt-polynomial & $c = 4$          & 81.76 \\
%         ConvNeXt-polynomial & \textbf{$c = 7$} & \textbf{81.98} \\
%         \bottomrule
%   \end{tabular}
% \caption{Test accuracy on ImageNet, polynomial ConvNeXt with different scale.}
%   \label{table:polynomial-scale-results-imnet}
% \end{table}

% \subsection{Aliasing}
% We note that the polynomial coefficients do not change that bandwidth expansion rate, which is only dependent on the polynomial degree.
% This is 



% \begin{table}%
%   \centering
%   \subfloat[][]{\adjustbox{raise=-5pc}}{
%     \begin{tabular}{lll}
%         \toprule
%         Model &  &Test acc. \\ 
%         \midrule
%         ConvNeXt-baseline (GeLU) & &  82.11 \\
%         \midrule
%         ConvNeXt-polynomial & $c = 4$          & 81.76 \\
%         ConvNeXt-polynomial & \textbf{$c = 7$} & \textbf{81.98} \\
%         \bottomrule
%   \end{tabular}

%   }%
%   % \qquad
%   \subfloat[][]{
%         \begin{tabular}{lll}
%             \toprule
%             Model &  &Test acc. \\ 
%             \midrule
%             ConvNeXt-baseline (GeLU) & &  81.48 \\
%             \midrule
%             ConvNeXt-polynomial & $c = 1$          & 81.12 \\
%             ConvNeXt-polynomial & $c = 2$          & 81.93 \\
%             ConvNeXt-polynomial & $c = 3$          & 82.57 \\
%             ConvNeXt-polynomial & $c = 4$          & 82.49 \\
%             ConvNeXt-polynomial & $c = 5$          & 82.7 \\
%             ConvNeXt-polynomial & $c = 6$          & 82.44 \\
%             ConvNeXt-polynomial & \textbf{$c = 7$} & \textbf{82.92} \\
%             ConvNeXt-polynomial & $c = 9$          & 81.28 \\
%             ConvNeXt-polynomial & $c = 11$         & 80.53 \\
%             \bottomrule
%           \end{tabular}
%   }
%   \caption{Here are some tables in a \texttt{table} environment.}%
%   \label{tbl:table}%
% \end{table}
% \input{appendices/attack_description.tex}
% To hide proofs : \newcommand{\maybehide}[1]{}
% To show proofs : \newcommand{\maybehide}[1]{#1}
\newcommand{\maybehide}[1]{#1}

\section{Proofs}
\label{sec:proofs}

\subsection{Weak Open CBV}

\subsubsection{General Lemmas}

\begin{proposition}[{\bf Diamond}]
    \label{prop:diamond}
The relation $\redcbv$ enjoys the diamond property: if $t \redcbv t_i\ (i=1,2)$ and $t_1 \neq t_2$, then there exists $t_3$ such that $t_i \redcbv t_3\ i=1,2$.
\end{proposition}

\propcharnfs*

\maybehide{\begin{proof}
    We are going to show this proposition by splitting the original statement into the two following ones:
    \begin{enumerate}
        \item \label{prop:char-nfs:1} $t \not\dred$ and $\neg\isvalue{t}$ iff $t \in \neutral$.
        \item \label{prop:char-nfs:2} $t \not\dred$ iff $t \in \normal$.
    \end{enumerate}
    The proof now follows by simultaneous induction over both these statements:
    \begin{itemize}
        \item[$\Ra$)] By induction over $t$: 
        \begin{enumerate}
            \item Let $t \not\dred$ and $\neg\isvalue{t}$. We want to show that $t \in \neutral$:
            \begin{itemize}
                \item Case $t = x$ or $t = \lam x.u$. Then $\neg\isvalue{t}$ does not hold. Therefore, the statement holds vacuously.
                \item Case $t = u p$. Since $u p \not\dred$, then, in particular, it must be the case that either $\neg\isabs{u}$ or $\neg\isvalue{p}$ must hold, according to rule (\ruleBeta):
                \begin{itemize}
                    \item Assume $\neg\isabs{u}$ holds. It must be the case that $u \not\dred$, according to rule (\ruleAppL). And it also must be the case that $p \not\dred$, according to rule (\ruleAppR). Therefore, $p \in \normal$, by the \ih (\cref{prop:char-nfs}.\ref{prop:char-nfs:2}). Now, we have to consider $u$, which can be a variable, or not:
                    \begin{itemize}
                        \item Case $u = x$. Then $u p \in x \ \normal \in \neutral$.
                        \item Case $u$ is not a variable. Then $\neg\isvalue{u}$ holds. Therefore, we have $u \in \neutral$, by the \ih (\cref{prop:char-nfs}.\ref{prop:char-nfs:1}). Thus, $u p \in \neutral \ \normal \in \neutral$.
                    \end{itemize}
                    \item Assume $\neg\isvalue{p}$ holds. Then it must be the case that $u \not\dred$, according to rule (\ruleAppL). And that $p \not\dred$, according to rule (\ruleAppR). Therefore, $u \in \normal$, by the \ih (\ref{prop:char-nfs}.\ref{prop:char-nfs:2}), and $p \in \neutral$, by the \ih (\cref{prop:char-nfs}.\ref{prop:char-nfs:1}). Thus, $u p \in \normal \ \neutral \in \neutral$.
                \end{itemize}
            \end{itemize}
            \item Let $t \not\dred$. We want to show that $t \in \normal$:
            \begin{itemize}
                \item Case $t \in \val$. Then, clearly $t \in \normal$.
                \item Case $t \not\in \val$. Then, $\neg\isvalue{t}$ holds. Therefore, $t \in \neutral$, by \cref{prop:char-nfs}.\ref{prop:char-nfs:1}. Thus, in particular, $t \in \normal$.
            \end{itemize}
        \end{enumerate}
        \item[$\La$)] By induction over $t \in \normal$:
        \begin{enumerate}
            \item Let $t \in \neutral$. We want to show that $t \not\dred$ and $\neg\isvalue{t}$:
            \begin{itemize}
                \item Case $t = u p \in x \ \normal$. Then $u = x$ and $p \in \normal$. Since $u = x$, then both rules (\ruleBeta) and (\ruleAppL) cannot be applied. Since $p \in \normal$, then $p \not\dred$, by the \ih (\cref{prop:char-nfs}.\ref{prop:char-nfs:2}). Therefore, rule (\ruleAppR) also cannot be applied. Thus, $u p \not\dred$. And we can conclude, since $\neg\isvalue{u p}$ clearly holds.
                \item Case $t = u p \in \normal \ \neutral$. Then $u \in \normal$ and $p \in \neutral$. Since $u \in \normal$, then $u \not\dred$, by the \ih (\cref{prop:char-nfs}.\ref{prop:char-nfs:2}). Since $p \in \neutral$, then $p \not\dred$ and $\neg\isvalue{p}$ holds, by the \ih (\cref{prop:char-nfs}.\ref{prop:char-nfs:1}). Since $\neg\isvalue{p}$, then rule (\ruleBeta) cannot be applied. Since $u \not\dred$ and $p \not\dred$, then rules (\ruleAppL) and (\ruleAppR) cannot be applied. Therefore, $u p \not\dred$. And we can conclude since $\neg\isvalue{u p}$ clearly holds.
                \item Case $t = u p \in \neutral \ \normal$. Then $u \in \neutral$ and $p \in \neutral$. Since $u \in \neutral$, then $u \not\dred$ and $\neg\isvalue{u}$ holds, by the \ih (\cref{prop:char-nfs}.\ref{prop:char-nfs:1}). Since $p \in \normal$, then $p \not\dred$, by the \ih (\cref{prop:char-nfs}.\ref{prop:char-nfs:2}). Since $\neg\isvalue{u}$, then rule (\ruleBeta) cannot be applied. Since $u \not\dred$ and $p \not\dred$, then rules (\ruleAppL) and (\ruleAppR) cannot be applied. Therefore $u p \not\dred$. And we can conclude since $\neg\isvalue{u p}$ clearly holds.
            \end{itemize}
            \item Let $t \in \normal$. We want to show that $t \not\dred$:
            \begin{itemize}
                \item Case $t \in \val$. Then, clearly $t \not\dred$.
                \item Case $t \not\in \val$. Then, $t \in \neutral$, by definition. Thus, $t \not\dred$ holds, by~\cref{prop:char-nfs}.\ref{prop:char-nfs:1}.
            \end{itemize}
        \end{enumerate}
    \end{itemize}
\end{proof}
}
  
\begin{lemma}[Relevance]
    Let $\Phi \tr \seqi{\Gam}{t}{\tau}{(b,s)}$. Then $\dom{\Gam} \subseteq \fv{t}$.
\end{lemma}

\maybehide{\begin{proof}
    The proof following by induction over $\Phi$. Case $\Phi$ ends with rule (\ruleAx) or (\ruleLamP), then $\Phi$ is clearly relevant. The other cases following easily from the \ih.
\end{proof}}

\subsubsection{Soundness (Auxiliary Lemmas)}

\begin{lemma}
    \label{lem:values-not-neutral}
    Let $\Phi \tr \seqi{\Gam}{t}{\tau}{(b,s)}$. If $t \in \val$, then $\tau \not= \tneutral$.
\end{lemma}

\maybehide{\begin{proof}
    By case analysis on the form of $t \in \val$:
    \begin{itemize}
        \item Case $t = x$. Then we have to consider two additional cases according to the last rule used in $\Phi$:
        \begin{itemize}
            \item Case $\Phi$ ends with rule (\ruleAx), then $\tau$ is of the form $\sig \not= \tneutral$.
            \item Case $\Phi$ ends with rule (\ruleMany), then $\tau$ is of the form $\M \not= \tneutral$.
        \end{itemize}
        \item Case $t = \lam x.t$. Then we have to consider three additional cases according to the last rule used in $\Phi$:
        \begin{itemize}
            \item Case $\Phi$ ends with rule (\ruleLam), then $\tau$ is of the form $\M \ta \del \not= \tneutral$.
            \item Case $\Phi$ ends with rule (\ruleMany), then $\tau$ is of the form $\M \not= \tneutral$.
            \item Case $\Phi$ ends with rule (\ruleLamP), then $\tau = \tabs \not= \tneutral$.
        \end{itemize}
    \end{itemize}
\end{proof}}

\begin{lemma}
    \label{lem:notabs-implies-negabs}
    If $\Phi \tr \seqi{\Gam}{t}{\tau}{(b,s)}$, such that $\Gam$ is tight. If $\tau \in \nott{\tabs}$, then $\neg\isabs{t}$.
\end{lemma}

\maybehide{\begin{proof}
    By induction over $\Phi$:
    \begin{itemize}
        \item Case $\Phi$ ends with rule (\ruleAx), (\ruleApp), (\ruleAppPOne), or (\ruleAppPTwo), then $\neg\isabs{t}$ holds by definition.
        \item Case $\Phi$ ends with rule (\ruleLam), (\ruleMany), or (\ruleLamP),  then $\tau \not\in \nott{\tabs}$. Therefore, these cases do not apply.
    \end{itemize}
\end{proof}}

\begin{lemma}[{\bf Zero Steps and Normal Forms}]
    \label{lem:zero-steps-nfs}
    Let $\Phi \tr \seqi{\Gam}{t}{\tau}{(b,s)}$ be tight. $b = 0$ iff $t \in \normal$.
\end{lemma}

\maybehide{\begin{proof} \mbox{}
    \begin{itemize}
        \item[$\Ra$)] We want to show that, if $b = 0$, then $t \in \normal$. For this, we are going to split the original statement into the two following ones:
        \begin{enumerate}
            \item \label{lem:zero-steps-nfs:1} Let $\Phi \tr \seqi{\Gam}{t}{\tau}{(0,s)}$ be tight and $\neg\isvalue{t}$, then $t \in \neutral$.
            \item \label{lem:zero-steps-nfs:2} Let $\Phi \tr \seqi{\Gam}{t}{\tau}{(0,s)}$ be tight, then $t \in \normal$.
        \end{enumerate}
        The proof now follows by simultaneous induction over both these statements:
        \begin{enumerate}
            \item Let $\Phi \tr \seqi{\Gam}{t}{\tau}{(0,s)}$ be tight and $\neg\isvalue{t}$:
            \begin{itemize}
                \item Case $\Phi$ ends with rule (\ruleAx), (\ruleLam), (\ruleMany), or (\ruleLamP), then $\isvalue{t}$ holds. Therefore, these cases do not apply.
                \item Case $\Phi$ ends with rule (\ruleApp), then $b > 0$. Therefore, this case does not apply.
                \item Case $\Phi$ ends with rule (\ruleAppPOne), then $t$ is of the form $up$ and $\Phi$ is of the following form:
                \[ \begin{prooftree}
                    \hypo{\Phi_u \tr \seqi{\Gam_u}{u}{\nott{\tabs}}{(0,s_u)}}
                    \hypo{\Phi_p \tr \seqi{\Gam_p}{p}{\tightt}{(0,s_p)}}
                    \infer2[(\ruleAppPOne)]{\seqi{\Gam_u + \Gam_p}{up}{\tneutral}{(0,1+s_u+s_p)}}
                \end{prooftree} \]
                where $\tau = \tneutral$, $\Gam = \Gam_u + \Gam_p$ is tight, and $s = 1 + s_u + s_p$. Moreover, $\Gam_u$ and $\Gam_p$ are tight. By the \ih (\cref{lem:zero-steps-nfs}.\ref{lem:zero-steps-nfs:2}) over $\Phi_u$ and $\Phi_p$, we have that $u, p \in \normal$. By~\cref{lem:notabs-implies-negabs}, we have that $\neg\isabs{u}$. Therefore, either $u$ is a variable or $u \in \neutral$ by definition. So, in both cases, we can conclude that $u p \in \neutral$.
                \item Case $\Phi$ ends with rule (\ruleAppPTwo), then $t$ is of the form $up$ and $\Phi$ is of the following form:
                \[ \begin{prooftree}
                    \hypo{\Phi_u \tr \seqi{\Gam_u}{u}{\tightt}{(0,s_u)}}
                    \hypo{\Phi_p \tr \seqi{\Gam_p}{p}{\tneutral}{(0,s_p)}}
                    \infer2[(\ruleAppPTwo)]{\seqi{\Gam_u + \Gam_p}{up}{\tneutral}{(0,1+s_u+s_p)}}
                \end{prooftree} \]
                where $\tau = \tneutral$, $\Gam = \Gam_u + \Gam_p$, and $s = 1 + s_u + s_p$. Moreover, $\Gam_u$ and $\Gam_p$ are tight. By the \ih (\cref{lem:zero-steps-nfs}.\ref{lem:zero-steps-nfs:2}) over $\Phi_u$, we have that $u \in \normal$. By applying~\cref{lem:values-not-neutral} to $\Phi_p$, we have that $\neg\isvalue{p}$. By the \ih (\cref{lem:zero-steps-nfs}.\ref{lem:zero-steps-nfs:1}) over $\Phi_p$, we have that $p \in \neutral$. So, in both cases, we can conclude that $up \in \neutral$.
            \end{itemize}
            \item Let $\Phi \tr \seqi{\Gam}{t}{\tau}{(0,s)}$ be tight:
            \begin{itemize}
                \item Case $\Phi$ ends with rule (\ruleAx), (\ruleLam), or (\ruleLamP). Then, clearly $t \in \val$, so we can conclude immediately.
                \item Case $\Phi$ ends with rule (\ruleMany), then $\tau$ is of the form $\M \not\in \tightt$. Therefore, this case does not apply.
                \item In all the remaining cases $\neg\isvalue{t}$ holds. Then $t \in \neutral$, by \cref{lem:zero-steps-nfs}.\ref{lem:zero-steps-nfs:1}, so $t \in \normal$.
            \end{itemize}
        \end{enumerate}
        \item[$\La)$] We want to show that, if $t \in \normal$, then $b = 0$. The proof follows by induction over $t \in \normal$:
        \begin{enumerate}
            \item Case $t \in \neutral$. Then we have to consider the following additional cases:
            \begin{itemize}
                \item Case $t = xp$, such that $p \in \normal$. Then there are three additional cases to consider:
                \begin{itemize}
                    \item Case $\Phi$ ends with (\ruleApp), then it must be of the following form:
                    \[ \begin{prooftree}
                        \hypo{\seqi{x : \mul{\M \ta \tau}}{x}{\M \ta \tau}{(0,0)}}
                        \hypo{\Phi_p \tr \seqi{\Gam_p}{p}{\M}{(b_p,s_p)}}
                        \infer2[(\ruleApp)]{\seqi{(x : \mul{\M \ta \tau}) + \Gam_p}{xp}{\tau}{(1+b_p,s_p)}}
                    \end{prooftree} \]
                    where $\Gam = (x : \mul{\M \ta \tau}) + \Gam_p$ is tight, $b = 1+b_p$, and $s = s_p$. But, $\mul{\M \ta \tau}$ is not tight, since $\M \ta \tau \not\in \tightt$. Therefore, this case does apply.
                    \item Case $\Phi$ ends with (\ruleAppPOne), then $\Phi$ must be of the following form:
                    \[ \begin{prooftree}
                        \hypo{\seqi{(x : \mul{\tvar})}{x}{\tvar}{(0,0)}}
                        \hypo{\Phi_p \tr \seqi{\Gam_p}{p}{\tightt}{(b_p,b_p)}}
                        \infer2[(\ruleAppPOne)]{\seqi{\Gam_u + \Gam_p}{up}{\tneutral}{(b_p,1+s_u+s_p)}}
                    \end{prooftree} \]
                    where $\tau = \tneutral$, $\Gam = (x : \mul{\tvar}) + \Gam_p$ is tight, $b = b_p$, and $s = 1+ s_u + s_p$. Moreover, $\Gam_p$ is tight. By the \ih over $\Phi_p$, we have that $b_p = 0$. So we can conclude with $b = b_u + b_p = 0$.
                    \item Case $\Phi$ ends with (\ruleAppPTwo). This case is very similar to the case where $\Phi$ ends with rule (\ruleAppPOne).
                \end{itemize}
                \item Case $t = up$, such that $u \in \normal$ and $p \in \neutral$. Then there are three additional cases to consider:
                \begin{itemize}
                    \item Case $\Phi$ ends with (\ruleApp), then it must be of the following form:
                    \[ \begin{prooftree}
                        \hypo{\seqi{\Gam_u}{u}{\M \ta \tau}{(b_u,s_u)}}
                        \hypo{\Phi_p \tr \seqi{\Gam_p}{p}{\M}{(b_p,s_p)}}
                        \infer2[(\ruleApp)]{\seqi{\Gam_u + \Gam_p}{up}{\tau}{(1+b_u+b_p,s_u+s_p)}}
                    \end{prooftree} \]
                    where $\tau = \tau$, $\Gam = \Gam_u + \Gam_p$ is tight, $b = 1 + b_u + b_p$, and $s = s_u + s_p$. By~\cref{lem:tight-spreading}.\ref{lem:tight-spreading:2}, we have that $\M \in \tightt$, which is a contradiction. Therefore, this case does not apply.
                    \item Case $\Phi$ ends with (\ruleAppPOne) or (\ruleAppPTwo). These cases are very similar to the corresponding cases when $t = x p$, such that $p \in \normal$.
                \end{itemize}
                \item Case $t = up$, such that $u \in \neutral$ and $p \in \normal$. Then there are three cases to consider:
                \begin{itemize}
                    \item Case $\Phi$ ends with (\ruleApp), then it must be of the following form:
                    \[ \begin{prooftree}
                        \hypo{\seqi{\Gam_u}{u}{\M \ta \tau}{(b_u,s_u)}}
                        \hypo{\Phi_p \tr \seqi{\Gam_p}{p}{\M}{(b_p,s_p)}}
                        \infer2[(\ruleApp)]{\seqi{\Gam_u + \Gam_p}{up}{\tau}{(1+b_u+b_p,s_u+s_p)}}
                    \end{prooftree} \]
                    where $\tau = \tau$, $\Gam = \Gam_u + \Gam_p$ is tight, $b = 1 + b_u + b_p$, and $s = s_u + s_p$. By~\cref{lem:tight-spreading}.\ref{lem:tight-spreading:2} over $u \in \neutral$, we have that $\M \ta \tau \in \tightt$, which is a contradiction. Therefore, this case does not apply.
                    \item Case $\Phi$ ends with (\ruleAppPOne) or (\ruleAppPTwo). These cases are very similar to corresponding cases when $t = x p$, such that $p \in \normal$, or $t = up$, such that $u \in \normal$ and $p \in \neutral$.
                \end{itemize}
            \end{itemize}
            \item Case $t \in \normal$. Then we can consider the two following additional cases:
            \begin{itemize}
                \item Case $t \in \val$. Then $\Phi$ must end with (\ruleAx), (\ruleLam), (\ruleMany), or (\ruleLamP). With the exception of the case where $\Phi$ ends with rule (\ruleMany), we can conclude $b = 0$ immediately for every other case, by definition. Case $\Phi$ ends with rule (\ruleMany), then $\tau$ is of the form $\M \not\in \tightt$. Therefore, this case does not apply.
                \item Case $t \not\in \val$. Then, $t \in \neutral$, by definition. Therefore, $b = 0$, by \cref{lem:zero-steps-nfs}.\ref{lem:zero-steps-nfs:1}.
            \end{itemize}
        \end{enumerate}
    \end{itemize}
\end{proof}}

\begin{lemma}
    \label{lem:corr-size-counter}
    Let $\Phi \tr \seqi{\Gam}{t}{\tau}{(b,s)}$ be tight. If $b = 0$ then $s = \size{t}$.
\end{lemma}

\maybehide{\begin{proof}
    The proof follows by induction over $\Phi$:
    \begin{itemize}
        \item Case $\Phi$ ends with rule (\ruleAx) or (\ruleLamP). Then $t \in \val$ and $s = 0$. So we can conclude with $\size{t} = 0 = s$.
        \item Case $\Phi$ ends with rule (\ruleLam). Then $\tau$ is of the form $\Gam_u(x) \ta \del \not\in \tightt$, so this case does not apply.
        \item Case $\Phi$ ends with rule (\ruleApp). Then $b > 0$, so this case does not apply.
        \item Case $\Phi$ ends with rule (\ruleMany). Then $\tau$ is of the form $\M \not\in \tightt$, so this case does not apply.
        \item Case $\Phi$ ends with rule (\ruleAppPOne). Then $t = up$ and $\Phi$ must be of the following form:
        \[ \begin{prooftree}
            \hypo{\Phi_u \tr \seqi{\Gam_u}{u}{\nott {\tabs}}{(0,s_u)}}
            \hypo{\Phi_p \tr \seqi{\Gam_p}{p}{\tightt}{(0,s_p)}}
            \infer2[(\ruleAppPOne)]{\seqi{\Gam_u + \Gam_p}{up}{\tneutral}{(0,1+s_u+s_p)}}
        \end{prooftree} \]
        where $\tau = \tneutral$, $\Gam = \Gam_u + \Gam_p$, and $s = 1 + s_u + s_p$. Moreover, $\Gam_u$ and $\Gam_p$ are tight. By the \ih over $\Phi_u$ and $\Phi_p$, we have $s_u = \size{u}$ and $s_p = \size{p}$. So we can conclude with $s = 1 + \size{u} + \size{p} = \size{up}$.
        \item Case $\Phi$ ends with rule (\ruleAppPTwo). This case is very similar to the case where $\Phi$ ends with rule (\ruleAppPOne).
    \end{itemize}
\end{proof}}

\begin{lemma}[{\bf Split for Values}]
    \label{lem:split-values}
    Let $\Phi_v \tr \seqi{\Gam}{v}{\M}{(b,s)}$, such that $\M = \sqcup_{\iI} \M_i$. Then, there exist ($\Phi^i_v \tr \seqi{\Gam_i}{v}{\M_i}{(b_i,s_i)})_{\iI}$, such that $\Gam = +_{\iI} \Gam_i$, $b = +_{\iI} b_i$, and $s = +_{\iI} s_i$.
\end{lemma}

\maybehide{\begin{proof}
    We start by noting that $\Phi_v$ must end with the rule ($\ruleMany$). Therefore, we have $\Gam = +_{\jJ} \Gam_j$, $\M = \mul{\sig_j}_{\jJ}$, $b = +_{\jJ} b_j$, $s = +_{\jJ} s_j$, and $(\Phi^j_v \tr \seqi{\Gam_j}{v}{\sig_j}{(b_j,s_j)})_{\jJ}$, for some $J$. Let $\M_i = \mul{\sig_k}_{\kK_i}$, for each $\iI$, such that $J = +_{\iI} K_i$. Then, by using rule ($\ruleMany$), we can build $\Phi^i_v \tr \seqi{\Gam_i}{v}{\M_i}{(b_i, s_i)}$, for each $\iI$, such that $\Gam_i = +_{\kK_i} \Gam_k$, $b_i = +_{\kK_i} b_k$, and $s_i = +_{\kK_i} s_k$. So we can conclude with $\Gam = +_{\jJ} \Gam_j = +_{\iI} (+_{\kK_i} \Gam_k) = +_{\iI} \Gam_i$, $b = +_{\jJ} b_j = +_{\iI} (+_{\kK_i} b_k) = +_{\iI} b_i$, and $s = +_{\jJ} s_j = +_{\iI} (+_{\kK_i} s_k) = +_{\iI} s_i$.
\end{proof}}

\subsubsection{Completeness (Auxiliary Lemmas)}

\begin{lemma}[{\bf Tight Spreading}]
    \label{lem:tight-spreading}
    Let $\Phi \tr \seqi{\Gam}{t}{\tau}{(b,s)}$, such that $\Gam$ is tight:
    \begin{enumerate}
        \item \label{lem:tight-spreading:1} If $b = 0$ and $\tau$ is not an arrow type or a multi-type, then $\tau \in \tightt$.
        \item \label{lem:tight-spreading:2} If $t \in \neutral$, then $\tau \in \tightt$.
    \end{enumerate}
\end{lemma}

\maybehide{\begin{proof} \mbox{}
    \begin{enumerate}
        \item We want to show that, if $b = 0$ and $\tau$ is not an arrow type or a multiset type, then $\tau \in \tightt$. The proof follows by induction over $\Phi$:
        \begin{itemize}
            \item Case $\Phi$ ends with rule ($\ruleAx$), then it is of the following form:
            \[ \begin{prooftree}
                \infer0[(\ruleAx)]{\seqi{x : \mul{\sig}}{x}{\sig}{(0,0)}}
            \end{prooftree} \]
            such that $\tau = \sig$, $\Gam = x : \mul{\sig}$, and $s = 0$. If $x : \mul{\sig}$ is tight, then $\sig \in \{\tabs, \tvar\}$. Therefore, we can conclude with $\sig \in \{\tabs, \tvar\} \subset \tightt$.
            \item Case $\Phi$ ends with rule (\ruleLam), then $\tau$ is an arrow type. Therefore, this case does not apply.
            \item Case $\Phi$ ends with rule (\ruleApp), then $b > 0$. Therefore, this case does not apply.
            \item Case $\Phi$ ends with rule (\ruleMany), then $\tau$ is a multiset type. Therefore, this case does not apply.
            \item Case $\Phi$ ends with rule (\ruleLamP), then $\tau = \tabs \in \tightt$. 
            \item Case $\Phi$ ends with rules (\ruleAppPOne) or (\ruleAppPTwo), then $\tau = \tneutral \in \tightt$.
        \end{itemize}
        \item We want to show that, if $t \in \neutral$, then $\tau \in \tightt$. By induction over $t \in \neutral$:
        \begin{itemize}
            \item Case $t = xp$, such that $p \in \normal$. Then we have to consider the following three cases depending on the last rule in $\Phi$:
            \begin{itemize}
                \item Case $\Phi$ ends with rule (\ruleApp), then it must be of the following form:
                \[ \begin{prooftree}
                    \hypo{\seqi{x : \mul{\M \ta \tau}}{x}{\M \ta \del}{(0,0)}}
                    \hypo{\Phi_p \tr \seqi{\Gam_p}{p}{\M}{(b_p,s_p)}}
                    \infer2[(\ruleApp)]{\seqi{(x : \mul{\M \ta \tau}) + \Gam_p}{xp}{\del}{(1+b_p,s_p)}}
                \end{prooftree} \]
                where $\Gam = (x : \mul{\M \ta \del}) + \Gam_p$ is tight, $b = 1+b_p$, and $s = s_p$. But, $\mul{\M \ta \del}$ is not tight, since $\M \ta \del \not\in \tightt$. Therefore, this case does apply.
                \item Case $\Phi$ ends with rule (\ruleAppPOne) or (\ruleAppPTwo). Then $\tau = \tneutral \in \tightt$, so we can conclude immediately.
            \end{itemize}
            \item Case $t = up$, such that $u \in \normal$ and $p \in \neutral$. Then we have to consider the following three cases depending on the last rule in $\Phi$:
            \begin{itemize}
                \item Case $\Phi$ ends with rule (\ruleApp), then it must be of the following form:
                \[ \begin{prooftree}
                    \hypo{\Phi_u \tr \seqi{\Gam_u}{u}{\M \ta \tau}{(b_u, s_u)}}
                    \hypo{\Phi_p \tr \seqi{\Gam_p}{p}{\M}{(b_p, s_p)}}
                    \infer2[(\ruleApp)]{\seqi{\Gam_u + \Gam_p}{up}{\tau}{(1+b_u+b_p, s_u+s_p)}}
                \end{prooftree} \]
                where $\Gam = \Gam_u + \Gam_p$ is tight, $b = 1 + b_u + b_p$, and $s = s_u + s_p$. Moreover, $\Gam_p$ is tight. By the \ih over $\Phi_p$, we have that $\M \in \tightt$, which is a contradiction. Therefore, this case does not apply.
                \item Case $\Phi$ ends with rule (\ruleAppPOne) or (\ruleAppPTwo). Then $\tau = \tneutral \in \tightt$, so we can conclude immediately.
            \end{itemize}
            \item Case $t = up$, such that $u \in \neutral$ and $p \in \normal$. Then we have to consider the following three cases depending on the last rule in $\Phi$:
            \begin{itemize}
                \item Case $\Phi$ ends with rule (\ruleApp), then it must be of the following form:
                \[ \begin{prooftree}
                    \hypo{\Phi_u \tr \seqi{\Gam_u}{u}{\M \ta \tau}{(b_u, s_u)}}n
                    \hypo{\Phi_p \tr \seqi{\Gam_p}{p}{\M}{(b_p, s_p)}}
                    \infer2[(\ruleApp)]{\seqi{\Gam_u + \Gam_p}{up}{\tau}{(1+b_u+b_p, s_u+s_p)}}
                \end{prooftree} \]
                where $\Gam = \Gam_u + \Gam_p$ is tight, $b = 1 + b_u + b_p$, and $s = s_u + s_p$. Moreover, $\Gam_p$ is tight. By the \ih over $\Phi_p$, we have that $\M \in \tightt$, which is a contradiction. Therefore, this case does not apply.
                \item Case $\Phi$ ends with rule (\ruleAppPOne) or (\ruleAppPTwo). Then $\tau = \tneutral \in \tightt$, so we can conclude immediately.
            \end{itemize}
        \end{itemize}
    \end{enumerate}
\end{proof}}

\begin{lemma}[{\bf Typability of Normal Forms}]
    \label{lem:typ-nfs}
    If $t \in \normal$, then there exists a tight derivation $\Phi \tr \seqi{\Gam}{t}{\tau}{(b,s)}$, such that $s = \size{t}$.
\end{lemma}

\maybehide{To show this proposition we are going to need to split the original statement into the two following ones:
\begin{enumerate}
    \item \label{prop:typ-nfs:1} If $t \in \neutral$, then there exists a tight derivation $\Phi \tr \seqi{\Gam}{t}{\tneutral}{(b,s)}$, such that $s = \size{t}$.
    \item \label{prop:typ-nfs:2} If $t \in \normal$, then there exists a tight derivation $\Phi \tr \seqi{\Gam}{t}{\tightt}{(b,s)}$, such that $s = \size{t}$.
\end{enumerate}
The proof follows by simultaneous induction over both these statements:
\begin{enumerate}
    \item Let $t \in \neutral$. We want to show that there exists a tight derivation $\Phi \tr \seqi{\Gam}{t}{\tneutral}{(b,s)}$:
    \begin{itemize}
        \item Case $t = up \in x \ \normal$. Then $u = x$ and $p \in \normal$. Therefore, there exists a tight derivation $\Phi_p \tr \seqi{\Gam_p}{p}{\tightt}{(b_p,s_p)}$, by the \ih (\cref{lem:typ-nfs}.\ref{prop:typ-nfs:2}), such that $\size{p} = s_p$. Thus, we can build $\Phi$ as follows:
        \[ \begin{prooftree}
            \infer0[(\ruleAx)]{\seqi{x : \mul{\tvar}}{x}{\tvar}{(0,0)}}
            \hypo{\Phi_p \tr \seqi{\Gam_p}{p}{\tightt}{(b_p,s_p)}}
            \infer2[(\ruleAppPOne)]{\seqi{ x : \mul{\tvar} + \Gam_p}{x p}{\tneutral}{(b_p,1+s_p)}}
        \end{prooftree} \]
        And we can conclude with $\Gam = x : \mul{\tvar} + \Gam_p$, $b = b_p$, and $s = 1+s_p = 1 + \size{x} + \size{p} = \size{xp}$.
        \item Case $t = up \in \normal \ \neutral$. Then $u \in \normal$ and $p \in \neutral$. Therefore, there exists a tight derivation $\Phi_u \tr \seqi{\Gam_u}{u}{\tightt}{(b_u,s_u)}$, such that $\size{u} = s_u$, by the \ih (\cref{lem:typ-nfs}.\ref{prop:typ-nfs:2}), and there exists a tight derivation $\Phi_p \tr \seqi{\Gam_p}{p}{\tneutral}{(b_p,s_p)}$, such that $\size{p} = s_p$ by the \ih (\cref{lem:typ-nfs}.\ref{prop:typ-nfs:1}). Thus, we can build $\Phi$ as follows:
        \[ \begin{prooftree}
            \hypo{\Phi_u \tr \seqi{\Gam_u}{u}{\tightt}{(b_u,s_u)}}
            \hypo{\Phi_p \tr \seqi{\Gam_p}{p}{\tneutral}{(b_p,s_p)}}
            \infer2[(\ruleAppPTwo)]{\seqi{\Gam_u + \Gam_p}{up}{\tneutral}{(b_u+b_p,1+s_u+s_p)}}
        \end{prooftree} \]
        And we can conclude with $\Gam = \Gam_u + \Gam_p$, $b = b_u+b_p$, and $s = 1+s_u+s_p = 1 + \size{u} + \size{p} = \size{up}$.
        \item Case $t = up \in \neutral \ \normal$. Then $u \in \neutral$ and $p \in \normal$. Therefore, there exists a tight derivation $\Phi_u \tr \seqi{\Gam_u}{u}{\tneutral}{(b_u,s_u)}$, such that $\size{u} = s_u$, by the \ih (\cref{lem:typ-nfs}.\ref{prop:typ-nfs:1}), and there exists a tight derivation $\Phi_p \tr \seqi{\Gam_p}{p}{\tightt}{(b_p,s_p)}$, such that $\size{p} = s_p$, by the \ih (\cref{lem:typ-nfs}.\ref{prop:typ-nfs:2}). Thus, we can build $\Phi$ as follows:
        \[ \begin{prooftree}
            \hypo{\Phi_u \tr \seqi{\Gam_u}{u}{\tneutral}{(b_u,s_u)}}
            \hypo{\Phi_p \tr \seqi{\Gam_p}{p}{\tightt}{(b_p,s_p)}}
            \infer2[(\ruleAppPOne)]{\seqi{\Gam_u + \Gam_p}{up}{\tneutral}{(b_u+b_p,1+s_u+s_p)}}
        \end{prooftree} \]
        And we can conclude with $\Gam = \Gam_u + \Gam_p$, $b = b_u+b_p$, and $s = 1 + s_u + s_p = 1 + \size{u} + \size{p} = \size{up}$.
    \end{itemize}
    \item Case $t \in \normal$. We want to show that there exists a tight derivation $\Phi \tr \seqi{\Gam}{t}{\tightt}{(b,s)}$:
    \begin{itemize}
        \item Case $t = x$. Then we can build $\Phi$ as follows:
        \[ \begin{prooftree}
            \infer0[(\ruleAx)]{\seqi{x : \mul{\sig}}{x}{\sig}{(0,0)}}
        \end{prooftree} \]
        by picking $\sig \in \{\tabs, \tvar\}$. And we can conclude with $\Gam = \eset$, $b = 0$, and $s = 0 = \size{x}$.
        \item Case $t = \lam x.u$. Then we can build $\Phi$ as follows:
        \[ \begin{prooftree}
            \infer0[(\ruleLamP)]{\seqi{}{\lam x.u}{\tabs}{(0,0)}}
        \end{prooftree} \]
        And we can conclude with $\Gam = \eset$, $b = 0$, and $s = 0 = \size{\lam x.u}$.
        \item The remaining cases are for when $t \in \neutral$, so they are subsumed by previous cases.
    \end{itemize}
\end{enumerate} 
}

\begin{lemma}[{\bf Merge for Values}]
    \label{lem:merge-values}
    Let $(\Phi^i_v \tr \seqi{\Gam_i}{v}{\M_i}{(b_i,s_i)})_{\iI}$. Then, there exists $\Phi_v \tr \seqi{\Gam}{v}{\M}{(b,s)}$, such that $\Gam = +_{\iI} \Gam_i$, $\M = +_{\iI} \M_i$, $b = +_{\iI} b_i$, and $s = +_{\iI}$.
\end{lemma}

\maybehide{\begin{proof}
    We start by noting that each $\Phi^i_v$ must end with the rule ($\ruleMany$). Therefore, for each $\iI$, we have $\Gam_i = +_{\kK_i} \Gam_k$, $\M_i = \mul{\sig_k}_{\kK_i}$, such that $b_i = +_{\kK_i} b_k$ and $s_i = +_{\kK_i} s_k$, and the following derivations $(\Phi^k_v \tr \seqi{\Gam_k}{v}{\sig_k}{(b_k,s_k)})_{\kK_i}$. Let $J = +_{\iI} K_i$ and $\M = \mul{\sig_j}_{\jJ} = \mul{\sig_k}_{\kK_i, \iI}$. We can use rule ($\ruleMany$) to build $\Phi_v \tr \seqi{\Gam}{v}{\M}{(+_{\jJ} b_j, +_{\jJ} s_j)}$. So we can conclude with $\Gam = +_{\jJ} \Gam_j = +_{\iI} (+_{\kK_i} \Gam_k) = +_{\iI} \Gam_i$, $b = +_{\jJ} b_j = +_{\iI} (+_{\kK_i} b_k) = +_{\iI} b_i$, and $s = +_{\jJ} s_j = +_{\iI} (+_{\kK_i} s_k) = +_{\iI} s_i$.
\end{proof}}

\subsubsection{Soundness and Completeness (Main Results)}

\begin{lemma}[{\bf Substitution and Anti-Substitution}]
    \label{lem:subsantisubs}
    \begin{enumerate} \mbox{}
        \item \label{lem:subs} Let $\Phi_t \tr \seqi{\Gam_t; x : \M}{t}{\tau}{(b_t,s_t)}$ and $\Phi_v \tr \seqi{\Gam_v}{v}{\M}{(b_v,s_v)}$, then there exists $\Phi_{t \subs{x}{v}} \tr \seqi{\Gam_t + \Gam_v}{t \subs{x}{v}}{\tau}{(b_t+b_v,s_t+s_v)}$.
        \item \label{lem:antisubs} Let $\Phi_{t \subs{x}{v}} \tr \seqi{\Gam_{t \subs{x}{v}}}{t \subs{x}{v}}{\tau}{(b,s)}$. Then, there exist $\Phi_t \tr \seqi{\Gam_t; x : \M}{t}{\tau}{(b_t,s_t)}$ and $\Phi_v \tr \seqi{\Gam_v}{v}{\M}{(b_v,s_v)}$, such that $\Gam_{t \subs{x}{v}} = \Gam_t + \Gam_v$, $b = b_t + b_v$, and $s = s_t + s_v$.
    \end{enumerate}
\end{lemma}

\maybehide{\begin{proof} \mbox{}
    \begin{enumerate}
        \item %\begin{proof}
    The proof follows by induction over $\Phi_t$:
    \begin{itemize}
        \item Case $\Phi_t$ ends with rule (\ruleAx). Then $t$ must be a variable and we need to consider two cases:
        \begin{itemize}
            \item Assume $t = y = x$. Then $\Gam_t = \eset$, $\tau = \M$, $t \subs{x}{v} = v$, $b_t = 0$, and $s_t = 0$. So we can take $\Phi_{t \subs{x}{v}} = \Phi_v$ and conclude with $\Gam_t + \Gam_v = \Gam_v$, $b_t + b_v = b_v$, and $s_t + s_v = s_v$.
            \item Assume $t = y \not= x$. Then $\M = \emul$, $\Gam_v = \eset$, $t \subs{x}{v} = t$, $b_v = 0$, and $s_v = 0$. So we can take $\Phi_{t \subs{x}{v}} = \Phi_t$ and conclude with $\Gam_t + \Gam_v = \Gam_t$, $b_t + b_v = b_t$, and $s_t + s_v = s_t$.
        \end{itemize}
        \item Case $\Phi_t$ ends with rule (\ruleLam). Then $t$ must be of the form $\lam y.u$ and $\Phi_t$ must be of the following form (by $\alpha$-conversion):
        \[ \begin{prooftree}
            \hypo{\Phi_u \tr \seqi{\Gam; x : \M}{u}{\tau'}{(b_t,s_t)}}
            \infer1[(\ruleLam)]{\seqi{(\Gam \sm y); x : \M}{\lam y.u}{\Gam(y) \ta \tau'}{(b_t, s_t)}}
        \end{prooftree} \]
        where $\tau = \Gam(y) \ta \tau'$ and $\Gam_t = (\Gam \sm y)$. By the \ih, we have the following derivation $\Phi_{u \subs{x}{v}} \tr \seqi{\Gam + \Gam_v}{u \subs{x}{v}}{\tau}{(b_t + b_v, s_t + s_v)}$. Therefore, we can construct $\Phi_{t \subs{x}{v}}$ as follows:
        \[ \begin{prooftree}
            \hypo{\Phi_{u \subs{x}{v}} \tr \seqi{\Gam + \Gam_v}{u \subs{x}{v}}{\tau'}{(b_t + b_v, s_t + s_v)}}
            \infer1[(\ruleLam)]{\seqi{(\Gam + \Gam_v) \sm y}{(\lam y.u) \subs{x}{v}}{\Gam(y) \ta \tau'}{(b_t + b_v, s_t + s_v)}}
        \end{prooftree} \]
        And we can conclude with $(\Gam + \Gam_v) \sm y = (\Gam \sm y) + \Gam_v = \Gam_t + \Gam_v$, by $\alpha$-conversion.
        \item Case $\Phi_t$ ends with rule ($\ruleApp$). Then $t$ must be of the form $up$ and $\Phi_t$ must be of the following form:
        \[ \begin{prooftree}
            \hypo{\Phi_u \tr \seqi{\Gam; x : \M_1}{u}{\M' \ta \tau}{(b_u, s_u)}}
            \hypo{\Phi_p \tr \seqi{\Del; x : \M_2}{p}{\M'}{(b_p,s_p)}}
            \infer2[(\ruleApp)]{\seqi{(\Gam + \Del); x : \M_1 \sqcup \M_2}{up}{\tau}{(1+b_u+b_p, s_u+s_p)}}
        \end{prooftree} \]
        where $\Gam_t = (\Gam + \Del)$, $\M = \M_1 \sqcup \M_2$, $b_t = 1 + b_u + b_p$, and $s_t = s_u + s_p$. By~\cref{lem:split-values}, we know there exist the following derivations $(\Phi^i_v \tr \seqi{\Gam^i_v}{v}{\M_i}{(b_i,s_i)})_{i \in \{1,2\}}$, such that $\Gam_v = \Gam^1_v + \Gam^2_v$, $b_v = b_1 + b_2$, and $s_v = s_1 + s_2$. By the \ih, we know there exist $\Phi_{u \subs{x}{v}} \tr \seqi{\Gam + \Gam^1_v}{u \subs{x}{v}}{\M' \ta \tau}{(b_u+b_1, s_u+s_1)}$ and $\Phi_{p \subs{x}{v}} \tr \seqi{\Del + \Gam^2_v}{p \subs{x}{v}}{\M'}{(b_p + b_2, s_p + s_2)}$. So we can construct $\Phi_{t \subs{x}{v}}$ as follows:
        \[ \begin{prooftree}
            \hypo{\Phi_{u \subs{x}{v}} \tr \seqi{\Gam + \Gam^1_v}{u \subs{x}{v}}{\M' \ta \tau}{(b_u+b_1, s_u+s_1)}}
            \hypo{\Phi_{p \subs{x}{v}} \tr \seqi{\Del + \Gam^2_v}{p \subs{x}{v}}{\M'}{(b_p+b_2,s_p+s_2)}}
            \infer2[(\ruleApp)]{\seqi{(\Gam + \Del) + (\Gam^1_v + \Gam^2_v)}{(u p) \subs{x}{v}}{\tau}{(1+b_u+b_p+b_1+b_2, s_u + s_p + s_1 + s_2)}}
        \end{prooftree} \]
        And we can conclude with $\Gam_t + \Gam_v = (\Gam + \Del) + (\Gam^1_v + \Gam^2_v)$, $b_t + b_v = 1 + b_u + b_p + b_1 + b_2$, and $s_t + s_v = s_u + s_p + s_1 + s_2$.
        \item Case $\Phi_t$ ends with rule ($\ruleMany$). Then $t$ must be of the form $w$ and $\Phi$ must be of the following form:
        \[ \begin{prooftree}
            \hypo{(\Phi^i_w \tr \seqi{\Gam_i; x : \M_i}{w}{\sig_i}{(b_i,s_i)})_{\iI}}
            \infer1[(\ruleMany)]{\seqi{+_{\iI} \Gam_i; x : \sqcup_{\iI} \M_i}{w}{\mul{\sig_i}_{\iI}}{(+_{\iI} b_i, +_{\iI} s_i)}}
        \end{prooftree} \]
        where $\tau = \mul{\sig_i}_{\iI}$, $\Gam_t = +_{\iI} \Gam_i$, $b_t = +_{\iI} b_i$, and $s_t = +_{\iI} s_i$. By~\cref{lem:split-values}, we have the following derivations $(\Phi^i_v \tr \seqi{\Gam^i_v}{v}{\M_i}{(b^i_v, s^i_v)})_{\iI}$, such that $\Gam_v = +_{\iI} \Gam^i_v$, $b_v = +_{\iI} b^i_v$, and $s_v = +_{\iI} s^i_v$. By the \ih over each $\Phi^i_w$, we have $(\Phi^i_{w \subs{x}{v}} \tr \seqi{\Gam_i + \Gam^i_v}{w \subs{x}{v}}{\sig_i}{(b_i + b^i_v, s_i + s^i_v)})_{\iI}$. Therefore, we can construct $\Phi_{t \subs{x}{v}}$ as follows:
        \[ \begin{prooftree}
            \hypo{(\Phi^i_{w \subs{x}{v}} \tr \seqi{\Gam_i + \Gam^i_v}{w \subs{x}{v}}{\sig_i}{(b_i + b^i_v, s_i + s^i_v)})_{\iI}}
            \infer1[(\ruleMany)]{\seqi{+_{\iI} (\Gam_i + \Gam^i_v)}{w \subs{x}{v}}{\mul{\sig_i}_{\iI}}{(+_{\iI} (b_i + b^i_v), +_{\iI} (s_i + s^i_v))}}
        \end{prooftree} \]
        And we can conclude with $\Gam_t + \Gam_v = +_{\iI} \Gam_i +_{\iI} \Gam^i_v = +_{\iI} (\Gam_i + \Gam^i_v)$, $b_t + b_v = +_{\iI} b_i +_{\iI} b^i_v = +_{\iI} (b_i + b^i_v)$, and $s_t + s_v = +_{\iI} s_i +_{\iI} s^i_v = +_{\iI} (s_i + s^i_v)$.
        \item Case $\Phi_t$ ends with rule (\ruleLamP). Then $t$ must be of the form $\lam y.u$, $\Gam_t = \eset$, $\tau = \tabs$, $\M = \emul$, $\Gam_v = \eset$, $t \subs{x}{v} = \lam y.(u \subs{x}{v}) = (\lam y.u) \subs{x}{v}$, $b_t = b_v = 0$, and $s_t = s_v = 0$. So we can construct $\Phi_{t \subs{x}{v}}$ as follows:
        \[ \begin{prooftree}
            \infer0[(\ruleLamP)]{\seqi{}{(\lam y.u) \subs{x}{v}}{\tabs}{(0,0)}}
        \end{prooftree} \]
        And conclude with $\Gam_t + \Gam_v = \eset$, $b_t + b_v = 0$, and $s_t + s_v = 0$.
        \item Case $\Phi_t$ ends with rule (\ruleAppPOne) or (\ruleAppPTwo), the proof is very similar to when $\Phi_t$ ends with rule (\ruleApp).
    \end{itemize}
%\end{proof}

        \item %\begin{proof}
    The proof follows by induction over $t$:
    \begin{itemize}
        \item Case $t = y$. Then we have to consider two cases:
        \begin{itemize}
            \item Case $t = y \not= x$. Then, $t \subs{x}{v} = y$. Let $\Gam_v = \eset$, $\M = \emul$, $b_v = 0$, and $s_v = 0$. Then, $\Phi_v$ is derivable using rule ($\ruleMany$). We also take $\Phi_t = \Phi_{t \subs{x}{v}}$, so that, in particular $\Gam_t = \Gam_{t \subs{x}{v}}$. Then, we conclude with $\Gam_{t \subs{x}{v}} = \Gam_t + \Gam_v = \Gam_t$, $b = b_t + b_v = b_t$, and $s = s_t + s_v = s_t$.
            \item Case $t = y = x$. Then, $t \subs{x}{v} = v$. Let $\Gam_t = \eset$, $b_t = 0$, and $s_t = 0$. Now, we have to consider two cases depending on the last rule used in $\Phi_{t \subs{x}{v}}$: 
            \begin{itemize}
                \item Case $\Phi_{t \subs{x}{v}}$ ends with rule ($\ruleAx$), then $\tau = \sig$. Let $\Gam_v = \Gam_{t \subs{x}{v}}$, $\M = \mul{\sig}$, $b_v = b$, and $s_v = s$. Then, we can build derivation $\Phi_v$ as follows:
                \[ \begin{prooftree}
                    \hypo{\Phi_{t \subs{x}{v}} \tr \seqi{\Gam_{t \subs{x}{v}}}{v}{\sig}{(b,s)}}
                    \infer1[(\ruleMany)]{\seqi{\Gam_{t \subs{x}{v}}}{v}{\mul{\sig}}{(b,s)}}
                \end{prooftree} \]
                Let $\Gam_t = \eset$, $b_t = 0$, and $s_t = 0$. Then, $\Phi_t \tr \seqi{x : \mul{\sig}}{x}{\sig}{(0,0)}$ is given by rule ($\ruleAx$). So we can conclude with $\Gam_{t \subs{x}{v}} = \Gam_v = \Gam_t + \Gam_v$, $b = b_v = b_t + b_v$, and $s = s_v = s_t + s_v$.
                \item Case $\Phi_{t \subs{x}{v}}$ ends with rule ($\ruleMany$), then $\tau = \mul{\sig_i}_{\iI}$, for some $I$. Let $\Gam_t = \eset$, and $\M = \mul{\sig_i}_{\iI}$. Then, we can build $\Phi_t$ as follows:
                \[ \begin{prooftree}
                    \infer0[(\ruleAx)]{(\seqi{x : \mul{\sig_i}}{x}{\sig_i}{(0,0)})_{\iI}}
                    \infer1[(\ruleMany)]{\seqi{x : \mul{\sig_i}_{\iI}}{x}{\mul{\sig_i}_{\iI}}{(0,0)}}
                \end{prooftree} \] 
                Then, we can take $\Phi_v = \Phi_{t \subs{x}{v}}$, so that $\Gam_v = \Gam_{t \subs{x}{v}}$, $b_v = b$, and $s_v = s$. And we can conclude $\Gam_{t \subs{x}{v}} = \Gam_v = \Gam_t + \Gam_v$, $b = b_v = b_t + b_v$, and $s = s_v = s_t + s_v$.
            \end{itemize}
        \end{itemize}
        \item Case $t = \lam y.u$. Then $t \subs{x}{v} = (\lam y.u) \subs{x}{v} = \lam y.(u \subs{x}{v})$ and we have to consider three cases:
        \begin{itemize}
            \item Case $\Phi_{t \subs{x}{v}}$ ends with rule (\ruleLam), then it must be of the following form:
            \[ \begin{prooftree}
                \hypo{\Phi_{u \subs{x}{v}} \tr \seqi{\Gam_{u \subs{x}{v}}; y : \M'}{u \subs{x}{v}}{\tau'}{(b, s)}}
                \infer1[(\ruleLam)]{\seqi{\Gam_{u \subs{x}{v}}}{\lam y.(u \subs{x}{v})}{\M' \ta \tau'}{(b, s)}}
            \end{prooftree} \]
            where $\tau = \M' \ta \tau'$, and $\Gam_{t \subs{x}{v}} = \Gam_{u \subs{x}{v}}$. By the \ih, we have the following derivations $\Phi_u \tr \seqi{\Gam_u; y: \M'; x : \M}{u}{\del}{(b_u, s_u)}$ and $\Phi_v \tr \seqi{\Gam_v}{v}{\M}{(b_v, s_v)}$, such that $\Gam_{u \subs{x}{v}} = \Gam_u + \Gam_v$, $b = b_u + b_v$, and $s = s_u + s_v$. And we can build $\Phi_{\lam y.u}$ as follows:
            \[ \begin{prooftree}
                \hypo{\Phi_u \tr \seqi{\Gam_u; y : \M'; x : \M}{u}{\tau'}{(b_u, s_u)}}
                \infer1[(\ruleLam)]{\seqi{\Gam_u; x : \M}{\lam y.u}{\M' \ta \tau'}{(b_u, s_u)}}
            \end{prooftree} \]
            So we can pick $\Phi_t = \Phi_{\lam y.u}$, and conclude with $\Gam_{t \subs{x}{v}} = \Gam_{u \subs{x}{v}} = \Gam_u + \Gam_v$, $b = b_u + b_v$, and $s = s_u + s_v$.
            \item Case $\Phi_{t \subs{x}{v}}$ ends with rule (\ruleLamP), then is must be of the following form:
            \[ \begin{prooftree}
                \infer0[(\ruleLamP)]{\seqi{}{\lam y.(u \subs{x}{v})}{\tabs}{(0,0)}}
            \end{prooftree} \]
            where $\tau = \tabs$, $\Gam_{t \subs{x}{v}} = \eset$, $b = 0$, and $s = 0$. Let $\Gam_t = \eset$, $\M = \emul$, $b_t = 0$, and $s_t = 0$. Then, we can build $\Phi_t$ as follows:
            \[ \begin{prooftree}
                \infer0[(\ruleLamP)]{\seqi{}{\lam y.u}{\tabs}{(0,0)}}
            \end{prooftree} \]
            Let $\Gam_v = \eset$, $b_v = 0$, and $s_v = 0$. Then $\Phi_v$ can be constructed by using rule (\ruleMany) with no premises. So we can conclude with $\Gam_{t \subs{x}{v}} = \eset = \Gam_t + \Gam_v$, and $b = 0 = b_t + b_v$, and $s = 0 = s_t + s_v$.
            \item Case $\Phi_{t \subs{x}{v}}$ ends with rule ($\ruleMany$). Then $t \subs{x}{v}$ and $t$ are values, and $\Phi_{t \subs{x}{v}}$ must be of the following form:
            \[ \begin{prooftree}
                \hypo{(\Phi_i \tr \seqi{\Gam_i}{t \subs{x}{v}}{\sig_i}{(b_i,s_i)})_{\iI}}
                \infer1[(\ruleMany)]{\seqi{+_{\iI} \Gam_i}{t \subs{x}{v}}{\mul{\sig_i}_{\iI}}{(+_{\iI} b_i, +_{\iI} s_i)}}
            \end{prooftree} \]
            where $\tau = \mul{\sig_i}_{\iI}$, $\Gam_{t \subs{x}{v}} = +_{\iI} \Gam_i$, $b = +_{\iI} b_i$, and $s = +_{\iI} s_i$. By the \ih over each $\Phi_i$, we have the following derivations $\Phi^i_t \tr \seqi{\Gam^i_t; x : \M_i}{t}{\sig_i}{(b^i_t, s^i_t)}$ and $\Phi^i_v \tr \seqi{\Gam^i_v}{v}{\M_i}{(b^i_v, s^i_v)}$, such that $\Gam_i = \Gam^i_t + \Gam^i_v$, $b_i = b^i_t + b^i_v$, and $s_i = s^i_t + s^i_v$,for each $\iI$. So we can build $\Phi_t$ as follows:
            \[ \begin{prooftree}
                \hypo{(\Phi^i_t \tr \seqi{\Gam^i_t; x : \M_i}{t}{\sig_i}{(b^i_t, s^i_t)})_{\iI}}
                \infer1[(\ruleMany)]{\seqi{+_{\iI} \Gam^i_t; x : \sqcup_{\iI} \M_i}{t}{\mul{\sig_i}_{\iI}}{(+_{\iI} b^i_t, +_{\iI} s^i_t)}}
            \end{prooftree} \]
            such that $\Gam_t = +_{\iI} \Gam^i_t$, $\M = \sqcup_{\iI} \M_i$, $b_t = +_{\iI} b^i_t$, and $s_t = +_{\iI} s^i_t$. By~\cref{lem:merge-values}, we can take the following derivation $\Phi_v \tr \seqi{+_{\iI} \Gam^i_v}{v}{\M}{(+_{\iI} b^i_v, +_{\iI} s^i_v)}$. And we can conclude with $\Gam_{t \subs{x}{v}} = +_{\iI} \Gam_i = +_{\iI} (\Gam^i_t + \Gam^i_v) = +_{\iI} \Gam^i_t +_{\iI} \Gam^i_v = \Gam_t + \Gam_v$, $b = +_{\iI} b_i = +_{\iI} (b^i_t + b^i_v) = +_{\iI} b^i_t +_{\iI} b^i_v = b_t + b_v$, and $s = +_{\iI} s_i = +_{\iI} (s^i_t + s^i_v) = +_{\iI} s^i_t +_{\iI} s^i_v = s_t + s_v$.
        \end{itemize}
        \item Case $t = up$. Then $t \subs{x}{v} = (u \subs{x}{v}) (p \subs{x}{v})$ and we have to consider three cases:
        \begin{itemize}
            \item Case $\Phi_{t \subs{x}{v}}$ ends with ($\ruleApp$), then it must be of the following form:
            \[ \begin{prooftree}
                \hypo{\Phi_{u \subs{x}{v}} \tr \seqi{\Gam_{u \subs{x}{v}}}{u \subs{x}{v}}{\M' \ta \tau}{(b', s')}}
                \hypo{\Phi_{p \subs{x}{v}} \tr \seqi{\Gam_{p \subs{x}{v}}}{p \subs{x}{v}}{\M'}{(b'', s'')}}
                \infer2[(\ruleApp)]{\seqi{\Gam_{u \subs{x}{v}} + \Gam_{p \subs{x}{v}}}{(u \subs{x}{v})(p \subs{x}{v})}{\tau}{(1+b'+b'', s'+s'')}}
            \end{prooftree} \]
            where $\Gam_{t \subs{x}{v}} = \Gam_{u \subs{x}{v}} + \Gam_{p \subs{x}{v}}$, $b = 1+b'+b''$, and $s = s' + s''$. By the \ih over $\Phi_{u \subs{x}{v}}$, we have the following derivations $\Phi_u \tr \seqi{\Gam_u; x : \M_1}{u}{\M' \ta \tau}{(b_u,s_u)}$ and $\Phi^1_v \tr \seqi{\Gam^1_v}{v}{\M_1}{(b^1_v,s^1_v)}$, such that $\Gam_{u \subs{x}{v}} = \Gam_u + \Gam^1_v$, $b' = b_u + b^1_v$, and $s' = s_u + s^1_v$. And by the \ih over $\Phi_{p \subs{x}{v}}$, we have the following derivation $\Phi_{p} \tr \seqi{\Gam_p; x : \M_2}{p}{\M'}{(b_p,s_p)}$ and $\Phi^2_v \tr \seqi{\Gam^2_v}{v}{\M_2}{(b^2_v,s^2_v)}$, such that $\Gam_{p \subs{x}{v}} = \Gam_p + \Gam^2_v$, $b'' = b_p + b^2_v$, and $s'' = s_p + s^2_v$. By~\cref{lem:merge-values}, we can take the following derivation $\Phi_v \tr \seqi{\Gam^1_v + \Gam^2_v}{v}{\M_1 \sqcup \M_2}{(b^1_v+b^2_v, s^1_v + s^2_v)}$, such that $\Gam_v = \Gam^1_v + \Gam^2_v$, $b_v = b^1_v + b^2_v$, and $s_v = s^1_v + s^2_v$. And we can build $\Phi_{up}$ as follows:
            \[ \begin{prooftree}
                \hypo{\Phi_u \tr \seqi{\Gam_u; x : \mul{\sig_i}_{\iI_1}}{u}{\M' \ta \tau}{(b_u,s_u)}}
                \hypo{\Phi_{p} \tr \seqi{\Gam_p; x : \mul{\sig_i}_{\iI_2}}{p}{\M'}{(b_p,s_p)}}
                \infer2[(\ruleApp)]{\seqi{(\Gam_u + \Gam_p); x : \mul{\sig_i}_{\iI}}{up}{\tau}{(1+b_u+b_p, s_u+s_p)}}
            \end{prooftree} \]
            such that $\Gam_t = \Gam_u + \Gam_p$, $b_t = 1+b_u + b_p$, and $s_t = s_u + s_p$. So we can pick $\Phi_t = \Phi_{up}$, and conclude with $\Gam_{t \subs{x}{v}} = \Gam_{u \subs{x}{v}} + \Gam_{p \subs{x}{v}} = \Gam_u + \Gam^1_v + \Gam_p + \Gam^2_v = (\Gam_u + \Gam_p) + (\Gam^1_v + \Gam^2_v) = \Gam_t + \Gam_v$, $b = 1+b'+b'' = 1+b_u + b^1_v + b_p + b^2_v = 1 + (b_u + b_p) + (b^1_v + b^2_v) = b_t + b_v$, and $s = s_u + s^1_v + s_p + s^2_v = (s_u + s_p) + (s^1_v + s^2_v) = s_t + s_v$.
            \item Case $\Phi_{t \subs{x}{v}}$ ends with (\ruleAppPOne) and (\ruleAppPTwo). These cases are very similar to the case where $\Phi_{t \subs{x}{v}}$ ends with rule (\ruleApp).
        \end{itemize}
    \end{itemize}
%\end{proof}
    \end{enumerate}
\end{proof}}

\begin{lemma}[{\bf Split Exact Subject Reduction and Expansion}]
    \label{lem:subjred-subjexp} \mbox{}
    \begin{enumerate} 
        \item \label{lem:subj-red} Let $\Phi_t \tr \seqi{\Gam}{t}{\tau}{(b,s)}$ be tight. If $t \dred t'$, then there exists $\Phi_{t'} \tr \seqi{\Gam}{t'}{\tau}{(b-1,s)}$.
        \item \label{lem:subj-exp} Let $\Phi_{t'} \tr \seqi{\Gam}{t'}{\tau}{(b,s)}$ be tight. If $t \dred t'$, then there exists $\Phi_t \tr \seqi{\Gam}{t}{\tau}{(b+1, s)}$.
    \end{enumerate}
\end{lemma}

\maybehide{\begin{proof} \mbox{}
    \begin{enumerate}
        \item %\begin{proof}
    We will actually prove the following stronger version of the statement, which allows us to reason inductively:

    Let $\Phi_t \tr \seqi{\Gam}{t}{\tau}{(b,s)}$, such that $\Gam$ is tight, and either $\tau$ is tight or $\neg\isvalue{t}$. If $t \dred t'$, then there exists $\Phi_{t'} \tr \seqi{\Gam}{t'}{\tau}{(b-1,s)}$.

    The proof now follows by induction over $\dred$:
    \begin{itemize}
        \item Case $t = (\lam x.u) v \dred u \subs{x}{v} = t'$.Assume that $\Phi_t$ ends with rule (\ruleAppPOne). Then $\lam x.u$ must be assigned type $\nott{\tabs}$, which is not possible by~\cref{lem:notabs-implies-negabs}. Now, assume that $\Phi_t$ ends with rule (\ruleAppPTwo). Then $v$ must be assigned typed $\tneutral$, which is not possible by~\cref{lem:values-not-neutral}. Therefore, $\Phi_t$ must be of the following form:
        \[ \begin{prooftree}
            \hypo{\Phi_u \tr \seqi{\Gam_u; x : \M}{u}{\tau}{(b_u,s_u)}}
            \infer1[(\ruleLam)]{\seqi{\Gam_u}{(\lam x.u)}{\M \ta \tau}{(b_u, s_u)}}
            \hypo{\Phi_{v} \tr \seqi{\Gam_v}{v}{\M}{(b_v,s_v)}}
            \infer2[(\ruleApp)]{\seqi{\Gam_u + \Gam_{v}}{(\lam x.u) v}{\tau}{(1+b_u+b_v, s_u+s_v)}}
        \end{prooftree} \]
        where $\tau \in \tightt$, $\Gam = \Gam_u + \Gam_v$ is tight, $b = 1 + b_u + b_v$, and $s = s_u + s_v$. By~\cref{lem:subsantisubs}.\ref{lem:subs}, we know there exists the following derivation $\Phi_{u \subs{x}{v}} \tr \seqi{\Gam_u + \Gam_v}{u \subs{x}{v}}{\tau}{(b_u+b_v,s_u+s_v)}$. So we can take $\Phi_{t'} = \Phi_{u \subs{x}{v}}$ and conclude with $b - 1 = b_u + b_v$.
        \item Case $t = up \dred u'p = t'$, such that $u \dred u'$. Then $\Phi_t$ must either end with (\ruleApp), (\ruleAppPOne), or (\ruleAppPTwo):
        \begin{itemize}
            \item Case $\Phi_t$ ends with rule (\ruleApp), then it must be of the following form:
            \[ \begin{prooftree}
                \hypo{\Phi_u \tr \seqi{\Gam_u}{u}{\M \ta \tau}{(b_u,s_u)}}
                \hypo{\Phi_p \tr \seqi{\Gam_p}{p}{\M}{(b_p,s_p)}}
                \infer2[(\ruleApp)]{\seqi{\Gam_u + \Gam_p}{up}{\tau}{(1 +b_u+b_p,s_u+s_p)}}
            \end{prooftree} \]
            where $\tau = \tau \in \tightt$, $\Gam = \Gam_u + \Gam_p$ is tight, $b = 1+b_u + b_p$, and $s = s_u + s_p$. Since $u \dred u'$, it is clear that $\neg\isvalue{u}$ holds. Moreover, $\Gam_u$ is necessarily tight. Therefore, by the \ih, there exists $\Phi_{u'} \tr \seqi{\Gam_u}{u'}{\M \ta \tau}{(b_u-1, s_u)}$. Thus, we can build $\Phi_{t'}$ as follows:
            \[ \begin{prooftree}
                \hypo{\Phi_{u'} \tr \seqi{\Gam_u}{u'}{\M \ta \tau}{(b_u-1, s_u)}}
                \hypo{\Phi_p \tr \seqi{\Gam_p}{p}{\M}{(b_p,s_p)}}
                \infer2[(\ruleApp)]{\seqi{\Gam_u + \Gam_p}{u'p}{\tau}{(b_u+b_p,s_u+s_p)}}
            \end{prooftree} \]
            And we can conclude with $b - 1= b_u + b_p$.
            \item Case $\Phi_t$ ends with rule (\ruleAppPOne) or (\ruleAppPTwo), the proof are similar to the one where $\Phi_t$ ends with rule (\ruleApp).
        \end{itemize}
        \item Case $t = up \dred up' = t'$, such that $u \not\dred$ and $p \dred p'$. Then $\Phi_t$ must either end with (\ruleApp), (\ruleAppPOne), or (\ruleAppPTwo):
        \begin{itemize}
            \item Case $\Phi_t$ ends with rule (\ruleApp), then it must be of the following form:
            \[ \begin{prooftree}
                \hypo{\Phi_u \tr \seqi{\Gam_u}{u}{\M \ta \tau}{(b_u,s_u)}}
                \hypo{\Phi_p \tr \seqi{\Gam_p}{p}{\M}{(b_p,s_p)}}
                \infer2[(\ruleApp)]{\seqi{\Gam_u + \Gam_p}{up}{\tau}{(1+b_u+b_p,s_u+s_p)}}
            \end{prooftree} \]
            where $\tau \in \tightt$, $\Gam = \Gam_u + \Gam_p$ is tight, $b = 1 + b_u + b_p$, and $s = s_u + s_p$. Since $p \dred p'$, it is clear that $\neg\isvalue{p}$. Moreover, $\Gam_p$ is necessarily tight. Therefore, by the \ih, we know there exists the following derivation $\Phi_{p'} \tr \seqi{\Gam_p}{p'}{\M}{(b_p-1, s_p)}$. Thus, we can build $\Phi_{t'}$ as follows:
            \[ \begin{prooftree}
                \hypo{\Phi_u \tr \seqi{\Gam_u}{u}{\M \ta \tau}{(b_u, s_u)}}
                \hypo{\Phi_{p'} \tr \seqi{\Gam_p}{p'}{\M}{(b_p-1,s_p)}}
                \infer2[(\ruleApp)]{\seqi{\Gam_u + \Gam_p}{up'}{\tau}{(b_u+b_p,s_u+s_p)}}
            \end{prooftree} \]
            And we can conclude with $b - 1 = b_u + b_p$.
            \item Case $\Phi_t$ ends with rule (\ruleAppPOne) or (\ruleAppPTwo), the proofs are similar to the ones where $\Phi_t$ ends with rule (\ruleApp).
        \end{itemize}
    \end{itemize}
%\end{proof}

        \item %\begin{proof}
    Just like for~\cref{lem:subjred-subjexp}.\ref{lem:subj-red}, we will actually prove the following stronger version of the statement, which allows us to reason inductively:

    Let $\Phi_{t'} \tr \seqi{\Gam}{t'}{\tau}{(b,s)}$, such that $\Gam$ is tight, and either ($\tau \in \tightt$ or $\neg\isvalue{t}$). If $t \dred t'$, then there exists $\Phi_t \tr \seqi{\Gam}{t}{\tau}{(b+1,s)}$.
    
    The proof now follows by induction over $\dred$:
    \begin{itemize}
        \item Case $t = (\lam x.u) v \dred u \subs{x}{v} = t'$. Then $\Phi_{t'} \tr \seqi{\Gam}{u \subs{x}{v}}{\tau}{(b,s)}$ and, by~\cref{lem:subsantisubs}.\ref{lem:antisubs}, there exist the following derivations $\Phi_u \tr \seqi{\Gam_u; x : \M}{u}{\tau}{(b_u, s_u)}$ and $\Phi_v \tr \seqi{\Gam_v}{v}{\M}{(b_v,s_v)}$, such that $\tau \in \tightt$, $\Gam = \Gam_u + \Gam_v$ is tight, $b = b_u + b_v$, and $s = s_u + s_v$. So we can build $\Phi_t$ as follows:
        \[ \begin{prooftree}
            \hypo{\Phi_u \tr \seqi{\Gam_u; x : \M}{u}{\tau}{(b_u, s_u)}}
            \infer1[(\ruleLam)]{\seqi{\Gam_u}{\lam x.u}{\M \ta \tau}{(b_u,s_u)}}
            \hypo{\Phi_v \tr \seqi{\Gam_v}{v}{\M}{(b_v,s_v)}}
            \infer2[(\ruleApp)]{\seqi{\Gam_u + \Gam_v}{(\lam x.u)v}{\tau}{(1+b_u+b_v, s_u+s_v)}}
        \end{prooftree} \]
        And we can conclude with $b + 1 = 1 + b_u + b_v$.
        \item Case $t = up \dred u'p = t'$, such that $u \dred u'$. Then $\Phi_{t'}$ must either end with (\ruleApp), (\ruleAppPOne), or (\ruleAppPTwo):
        \begin{itemize}
            \item Case $\Phi_{t'}$ ends with rule (\ruleApp), then it must be of the following form:
            \[ \begin{prooftree}
                \hypo{\Phi_{u'} \tr \seqi{\Gam_u}{u'}{\M' \ta \tau}{(b_u, s_u)}}
                \hypo{\Phi_p \tr \seqi{\Gam_p}{p}{\M'}{(b_p, s_p)}}
                \infer2[(\ruleApp)]{\seqi{\Gam_u + \Gam_p}{u'p}{\tau}{(1 + b_u + b_p, s_u + s_p)}}
            \end{prooftree} \]
            where $\tau \in \tightt$, $\Gam = \Gam_u + \Gam_p$ it tight, $b = 1 + b_u + b_p$, and $s = s_u + s_p$. Since $u \dred u'$, it is clear that $\neg\isvalue{u}$. Moreover, $\Gam_p$ is tight. Therefore, by the \ih, there exists the following derivation $\Phi_u \tr \seqi{\Gam_u}{u}{\M' \ta \tau}{(b_u + 1, s_u)}$. Thus, we can build $\Phi_{t'}$ as follows:
            \[ \begin{prooftree}
                \hypo{\Phi_u \tr \seqi{\Gam_u}{u}{\M' \ta \tau}{(b_u + 1, s_u)}}
                \hypo{\Phi_p \tr \seqi{\Gam_p}{p}{\M'}{(b_p, s_p)}}
                \infer2[(\ruleApp)]{\seqi{\Gam_u + \Gam_p}{up}{\tau}{(1 + b_u + 1 + b_p, s_u + s_p)}}
            \end{prooftree} \]
            And we can conclude with $b + 1 = (1 + b_u + b_p) + 1 = 1 + b_u + 1 + b_p$.
            \item Case $\Phi_{t'}$ ends with rule (\ruleAppPOne) or (\ruleAppPTwo), the proofs are similar to the one where $\Phi_{t'}$ ends with rule (\ruleApp).
        \end{itemize}
        \item Case $t = up \dred up' = t'$, such that $p \dred p'$. Then $\Phi_{t'}$ must either ends with (\ruleApp), (\ruleAppPOne), or (\ruleAppPTwo):
        \begin{itemize}
            \item Case $\Phi_{t'}$ ends with rule ($\ruleApp$), then it must be of the following form:
            \[ \begin{prooftree}
                \hypo{\Phi_u \tr \seqi{\Gam_u}{u}{\M' \ta \tau}{(b_u, s_u)}}
                \hypo{\Phi_{p'} \tr \seqi{\Gam_p}{p'}{\M'}{(b_p, s_p)}}
                \infer2[(\ruleApp)]{\seqi{\Gam_u + \Gam_p}{u p'}{\tau}{(1 + b_u + b_p, s_u + s_p)}}
            \end{prooftree} \]
            where $\tau \in \tightt$, $\Gam = \Gam_u + \Gam_{p'}$ is tight, $b = 1 + b_u + b_p$, $s_t = s_u + s_p$. Since $p \dred p'$, it is clear that $\neg\isvalue{p}$ holds. Moreover, $\Gam_p$ is tight. Therefore, by the \ih, we have the following derivation $\Phi_p \tr \seqi{\Gam_p}{p}{\M' \ta \tau}{(b_p + 1, s_p)}$. Thus, we can build $\Phi_{t'}$ as follows:
            \[ \begin{prooftree}
                \hypo{\Phi_u \tr \seqi{\Gam}{u}{\M' \ta \tau}{(b_u, s_u)}}
                \hypo{\Phi_p \tr \seqi{\Gam_p}{p}{\M'}{(b_p + 1, s_p)}}
                \infer2[(\ruleApp)]{\seqi{\Gam_u + \Gam_p}{up}{\tau}{(1 + b_u + b_p + 1, s_u + s_p)}}
            \end{prooftree} \]
            And we can conclude with $b + 1 = (1 + b_u + b_p) + 1 = 1 + b_u + b_p + 1$.
            \item Case $\Phi_{t'}$ ends with rule (\ruleAppPOne) or (\ruleAppPTwo), the proofs are similar to the one where $\Phi_{t'}$ ends with rule (\ruleApp).
        \end{itemize}   
    \end{itemize}
%\end{proof}
    \end{enumerate}
\end{proof}}

\begin{theorem}[{\bf Quantitative Soundness and Completeness}]
    \label{thm:soundnesscompleteness}
  \item \label{thm:soundness} If $\Phi \tr \seqi{\Gam}{t}{\tau}{(b,s)}$ is tight, then there exists $u \in \normal$ such that  $t \drred^b u$ with $\size{u} = s$.
  \item \label{thm:completeness} If $t \drred^b u$ with  $u \in \normal$, then there exists a tight type derivation $\Phi_t \tr \seqi{\Gam}{t}{\tau}{(b, \size{u})}$.
\end{theorem}

\maybehide{\begin{proof} \mbox{}
    \begin{enumerate} 
        \item %\begin{proof}
    The proof follows by induction over $b$:
    \begin{itemize}
        \item Case $b = 0$. Then $t \in \normal$, by~\cref{lem:zero-steps-nfs}. And $d = \size{t}$, by~\cref{lem:corr-size-counter}. So we can conclude with $u = t$.
        \item Case $b > 0$. Then $t \not\in \normal$, by~\cref{lem:zero-steps-nfs}. Therefore, there exists $t'$ such that $t \dred t'$, by~\cref{prop:char-nfs}. By\cref{lem:subjred-subjexp}.\ref{lem:subj-red}, there exists $\Phi_{t'} \tr \seqi{\Gam}{t'}{\tau}{(b-1, s)}$. By the \ih, there exists $u \in \normal$, such that $t' \drred^{b-1} u$, such that $d = \size{u}$. So we can conclude with $t \dred t' \drred^{b-1} u$, which means that $t \drred^b u$, as expected.
    \end{itemize}
%\end{proof}
        \item %\begin{proof}
    The proof follows by induction over $b$:
    \begin{itemize}
        \item Case $b = 0$. Then $t = u$, which means that $t \in \normal$. Therefore, we can conclude by~\cref{lem:typ-nfs}.
        \item Case $b > 0$. Then there exists $t'$, such that $t \dred t' \drred^{b-1} u$. By the \ih, there exists a tight derivation $\Phi_{t'} \tr \seqi{\Gam}{t'}{\tau}{(b-1, \size{u})}$. By\cref{lem:subjred-subjexp}.\ref{lem:subj-exp}, there exists a tight derivation $\Phi \tr \seqi{\Gam}{t}{\tau}{(b, \size{u})}$. So, we can conclude.
    \end{itemize}
%\end{proof}
    \end{enumerate}
\end{proof}}
  

  




\subsection{A \texorpdfstring{$\lambda$}{Lambda}-Calculus with Global State}

\subsubsection{General Lemmas}

\propnormalifffinal*

\maybehide{\begin{proof}
    \begin{itemize}
    \item[$\Ra$)] Let $(t, s)$ be \final. We consider two cases:
      \begin{itemize}
            \item Case $(t,s)$ is blocked. We reason by induction on blocked configurations. \begin{itemize}
                \item Case $(t,s) = (\get{l}{x}{u}, s)$, such that $l \not\in \dom{s}$. Then $(t, s) \not\ra$ is straightforward.
                \item Case $(t,s) = (v u, s)$ and $(u,s)$ is blocked.
                  Then by the \ih, we have that $(u,s) \not\ra$. Therefore, $(v u, s) \not\ra$ holds.
            \end{itemize}
          \item Case $t \in \normal$. We reason by induction on
            $\normal$. \begin{itemize}
                \item Case $t=v \in \val$. Then $(v,s) \not\ra$  is straightforward.
                \item Case $t \in \neutral$. Then $t = v u$ and we have to consider two different  cases: \begin{itemize}
                  \item Case  $v= x$ and $u \in \normal$. Then by the \ih, we have $(u,s) \not\ra$. Therefore, $(v u, s) \not\ra$ holds.
                    \item Case $v = (\lam x.p)$ and $u \in \neutral$. Then $u \in \normal$, and by the \ih, we have that $(u,s) \not\ra$. Therefore $(v u,s) \not\ra$ holds.
                \end{itemize}
            \end{itemize}
        \end{itemize}
      \item[$\La$)] Let $t \not \ra$. We reason by
        induction on $t$: \begin{itemize}
            \item Case $t = v$. Then $t \in \normal$. Therefore $(t,s)$ is \final.
            \item Case $t = v u$. Since $(v u, s) \not\ra$, then $(u,s) \not\ra$. By the \ih, we have $(u,s)$ \final. Now, we reason
              by cases: \begin{itemize}
                \item Case $(u, s)$ is blocked. Then, $(v u, s)$ is blocked by definition. 
                \item Case $u \in \normal$. Then we have two cases: \begin{itemize}
                    \item Case $u \in \neutral$. Then $vu \in \normal$. Therefore,  $(t,s)$ is \final.
                    \item Case $u \in \val$ and $v = \lam x.p$. Then $((\lam x.p) u, s) \ra (p \subs{x}{u}, s)$, which yields a contradiction with the hypothesis $t=vu\not\ra$. Thus, this case does not apply.
                \end{itemize}
            \end{itemize}
          \item Case $t = \get{l}{x}{u}$. Since $(\get{l}{x}{u},s) \not\ra$, then $l \not\in \dom{s}$. Therefore, $(\get{l}{x}{u},s)$ is blocked, which implies
$(t,s)$ is \final. 
            \item Case $t = \set{l}{v}{u}$. Then $(\set{l}{v}{u}, s) \ra (u, \upd{l}{v}{s})$, which yields to a contraction with the hypothesis  $t\not\ra$. 
              Therefore, this case does not apply.
        \end{itemize}
    \end{itemize}
\end{proof}
} 

\proptypedunblock*

\maybehide{\begin{proof}
    By induction on $t$: \begin{itemize}
        \item Case $t \in \val $ or $t = \set{l}{v}{t}$. Then the conclusion trivially holds, since clearly $(t,s)$ is not a blocked configuration.
        \item Case $t = \get{l}{x}{t}$. We have two  cases: \begin{itemize}
            \item Case $l \in \dom{s}$. Then $(t,s)$ is clearly unblocked.
            \item Case $l \not\in \dom{s}$. Let $\stype_0 = \conj{(l : \Gam(x))} \splus \stype$. Since $t = \get{l}{x}{u}$, then $\Phi$ must be of the following form:
            \[ \begin{prooftree}
                \hypo{\Phi \tr \seqi{\Gam_u \sm x}{\get{l}{x}{u}}{\comptype{
                    \stype_0}{\ctype}}{(b_u,m_u,d_u)}}
                \hypo{\Phi_s \tr \seqi{\Del}{s}{\stype_0}{(b_s,m_s,d_s)}}
                \infer2[(\ruleConf)]{\seqi{(\Gam_u \sm x) + \Del}{(\get{l}{x}{t}, s)}{\ctype}{(b_u+b_s,1+m_u+m_s,d_u+d_s)}}
              \end{prooftree} \] 
              where $\Gam = \Gam_u \sm x$, $b = b_u+b_s$, $m = 1+m_u+m_s$, and $d = d_u + d_s$. Thus, $l \in \dom{\conj{(l : \Gam_u(x))} \splus \stype}$, and so by
            \cref{lem:states-and-state-types} we have 
          $l \in \dom{s}$, which gives a contradiction with the hypothesis $l \not\in \dom{s}$. Therefore, this case does not apply,
        \end{itemize}
        \item Case $t = v u$. Assume $\Phi_v \tr \seqi{\Gam_v}{v}{\M \ta (\comptype{\stype'}{\ctype})}{(b_v,m_v,d_v)}$ and $\Phi_u \tr \seqi{\Gam_u}{u}{\tcomptype{\stype}{\M}{\stype'}}{(b_u,m_u,d_u)}$. Then $\Phi$ must be of the following form:
        \[ \begin{prooftree}
            \hypo{\Phi_v}
            \hypo{\Phi_u}
            \infer2[(\ruleApp)]{\seqi{\Gam_v + \Gam_u}{v u}{\comptype{\stype}{\ctype}}{(1+b_v+b_u,m_v+m_u,d_v+d_u)}}
            \hypo{\Phi_s \tr \seqi{\Del}{s}{\stype}{(b_s,m_s,d_s)}}
            \infer2[(\ruleConf)]{\seqi{(\Gam_v + \Gam_u) + \Del}{(v u, s)}{\ctype}{(1+b_v + b_u + b_s, m_v + m_u + m_s, d_v + d_u + d_s)}}
        \end{prooftree} \]
        where $\Gam = (\Gam_v + \Gam_u) + \Del$, $b = 1+b_v + b_u + b_s$, $m = m_v + m_u + m_s$, and $d = d_v + d_u + d_s$. Thus, we can build the following derivation for $(u,s)$:
        \[ \begin{prooftree}
            \hypo{\Phi_u \tr \seqi{\Gam_u}{u}{\tcomptype{\stype}{\M}{\stype'}}{(b_u,m_u,d_u)}}
            \hypo{\Phi_s \tr \seqi{\Del}{s}{\stype}{(b_s,m_s,d_s)}}
            \infer2[(\ruleConf)]{\seqi{\Gam + \Del}{(u,s)}{\conftype{\M}{\stype'}}{(b_u+b_s,m_u+m_s,d_u+d_s)}}
        \end{prooftree} \]
        By the \ih, we have that $(u,s)$ is unblocked. Therefore, $(v u, s)$ also unblocked.
    \end{itemize}
\end{proof}} 

\begin{lemma}[Relevance]
    Let $\Phi \tr \seqi{\Gam}{t}{\gtype}{(b,m,d)}$ (resp. $\Phi' \tr \seqi{\Gam}{s}{\stype}{(b',m',d')}$). Then $\dom{\Gam} \subseteq \fv{t}$ (resp. $\dom{\Gam} \subseteq \fv{s}$).
\end{lemma}

\maybehide{\begin{proof}
    The proof following by induction over $\Phi$ (resp. $\Phi'$). Case $\Phi$ (resp. $\Phi'$) ends with rule (\ruleAx), (\ruleAxP), or (\ruleLamP) (resp. rule (\ruleEmp)), then $\Phi$ (resp. $\Phi'$) is clearly relevant. The other cases follow easily from the \ih.
\end{proof}}

\subsubsection{Soundness Lemmas (Auxiliary Lemmas)}

\lemzerocounters*

\maybehide{\begin{proof} \mbox{}
    \begin{enumerate}
        \item \input{proofs/lem-zero-counters}
        \item \input{proofs/lem-zero-size-store}
    \end{enumerate}
\end{proof}} 

\begin{lemma}
    \label{lem:zero-counters-normal}
    Let $\Phi \tr \seqi{\Gam}{t}{\del}{(0,0,d)}$ be tight. If $t \in \normal$, then $\del = \stype \ra \tightt \tim \stype'$ and $\stype =\stype'$.
\end{lemma}

\maybehide{\begin{proof} 
  By induction on $t \in \normal$. We consider two cases:
  \begin{itemize}
    \item Case $t \in \val$. Then such a typing derivation can only end with rule (\ruleAx) followed by rule (\ruleLift) or (\ruleLamP)followed by rule (\ruleLift), in which cases the statement is obvious.
    \item Case $t = vu \in \neutral$. Since the first counter of the derivation is $0$, $\Phi$ can only end with a persistent rule (\ruleAppPOne) or (\ruleAppPTwo). In both cases, we can conclude by applying the \ih to $u \in \normal$ or $u \in \neutral$ and their type derivations, which gives  $\stype = \stype'$.
  \end{itemize}
\end{proof}} 

\lemzeronfs*

\maybehide{\begin{proof}\
  \begin{itemize}
    \item[$\Ra$)] By point (1) of~\cref{lem:zero-counters}.
    \item[$\La$)] By induction on $t$: \begin{itemize}
    \item Case $t \in \val$. There are six cases to consider for $\Phi$:
    \begin{itemize}    
      \item $\Phi$ ends with (\ruleAx). This case does not apply since the resulting type is not a monadic type. %Then $\Phi \tr \seqi{x:\mul{\rdel}}{x}{\rdel}{(0,0,0)}$ and the conclusion holds trivially.
      \item $\Phi$ ends with (\ruleLam). This case does not apply since the resulting type is not a monadic type.
      \item $\Phi$ ends with (\ruleMany). This case does not apply since the resulting type is not a monadic type.
      \item $\Phi$ ends with (\ruleLift). This case does not apply, since $\del = \tcomptype{\stype}{\M}{\stype'}$, but $\M \not\in \tightt$.
      \item $\Phi$ ends with (\ruleAxP). Then $\Phi \tr \seqi{x:\mul{\nott{\tneutral}}}{x}{\tcomptype{\stype}{\nott{\tneutral}}{\stype}}{(0,0,0)}$, with $\stype$ tight, and the conclusion holds trivially.
      \item $\Phi$ ends with (\ruleLamP). Then $\Phi \tr \seqi{}{\lambda x.t}{\tcomptype{\stype}{\vl}{\stype}}{(0,0,0)}$, with $\stype$ tight, and the conclusion holds trivially. 
    \end{itemize}
    \item Case $t = xu$. Then $u \in \normal$, by definition and there are two cases to consider for $\Phi$:
    \begin{itemize}
      \item If $\Phi$ ends with (\ruleApp). Then $\Phi_u \tr \seqi{\Gam_u}{u}{\tcomptype{\stype}{\M}{\stype'}}{(b_u,m_u,d_u)}$, $\Phi_x \tr \seqi{x : \M \ta (\comptype{\stype'}{\ctype})}{x}{\M \ta (\comptype{\stype'}{\ctype})}{(b_x,m_x,d_x)}$, such that $\Gam =  (x:\mul{\M \ta (\comptype{\stype'}{\ctype})}) + \Gam_u$ is tight. Absurd, since $\M \ta (\comptype{\stype'}{\ctype})$ is not tight, therefore this case does not apply.
      \item If $\Phi$ ends with (\ruleAppPOne). Then $\Phi_u \tr \seqi{\Gam_u}{u}{\tcomptype{\stype}{\tightt}{\stype}}{(b_u,m_u,d_u)}$, such that $\Gam = (x: \mul{\tvar})+\Gam_u$ is tight, $b = b_u$, $m =m_u$, $d = d_u+ 1$, and $\stype$ is tight. By the \ih\ on $u$, we have $b_u=m_u=0$, therefore $b = m = 0$.
      \end{itemize}
      \item Case $t = (\lam x.p) u$. Then $u \in \neutral$, by definition and there are two cases to consider for $\Phi$:
      \begin{itemize}
        \item If $\Phi$ ends with (\ruleApp). Then $\Phi_u \tr \seqi{\Gam_u}{u}{\tcomptype{\stype}{\M}{\stype'}}{(b_u,m_u,d_u)}$, $\Phi_{\lam x.p} \tr \seqi{\Gam_{\lam x.p}}{\lam x.p}{\M \ta (\comptype{\stype'}{\ctype})}{(b_p,m_p,d_p)}$, such that $\Gam = \Gam_u + \Gam_{\lambda x.p}$ is tight, $b = 1+b_l+b_u$, $m = m_l+m_u$, $d = d_l+ d_m$. Since $\Gam_u$ is tight and $u\in\neutral$, by~\cref{lem:comp-tight-spreading}, $\M \in \tightt$, which is absurd. Therefore, this case does not apply.
        \item If $\Phi$ ends with (\ruleAppPTwo). Then $\Phi_u \tr \seqi{\Gam_u}{u}{\tcomptype{\stype}{\tneutral}{\stype}}{(b_u,m_u,d_u)}$, such that $\Gam = \Gam_u$ is tight, $b = b_u$, $m=m_u$, $d = d_u+ 1$ and $\stype_f$ is tight. By the \ih\ on $u$, we have $b_u=m_u=0$. Therefore $b = m = 0$.
      \end{itemize}
    \end{itemize}
  \end{itemize}
\end{proof}
}

\begin{lemma}
    \label{lem:states-and-state-types}
    Let $\Phi \tr \seqi{\Del}{s}{\stype}{(b,m,d)}$. If $l \in \dom{\stype}$, then $l \in \dom{s}$.
\end{lemma}
  
\maybehide{\begin{proof}
    We proceed by proving the following stronger version of the statement: 
    
    Let $\Phi_s \tr \seqi{\Del_s}{s}{\stype_s}{(b_s,m_s,d_s)}$. If $l \in \dom{\stype_s}$, then $s \equivstate \upd{l}{v}{q}$, for some value $v$ and store $q$.
    
    The proof follows by induction on $\Phi_s$: 
    \begin{itemize}
        \item Case $\Phi_s$ ends with ($\ruleEmp$). Then the conclusion is vacuously true.
        \item Case $\Phi_s$ ends with ($\ruleUpd$). Then $\Phi_s$ is of the following form: 
        \[ \begin{prooftree}
            \hypo{\Phi_v \tr \seqi{\Gam_v}{v}{\M}{(b_v,m_v,d_v)}}
            \hypo{\Phi_q \tr \seqi{\Del_q}{q}{\stype_q}{(b_q,m_q,d_q)}}
            \infer2[(\ruleUpd)]{\seqi{\Gam_v + \Del_q}{\upd{l'}{v}{q}}{\conj{l' : \M}; \stype_q}{(b_v+b_q,m_v+m_q,d_v+v_q)}}
        \end{prooftree} \]
        where $\Del_s = \Gam_v + \Del_q$, $s = \upd{l'}{v}{q}$, $\stype_s = \conj{l' : \M}; \stype_q$, $b_s = b_v + b_q$, $m_s = m_v + m_q$, and $d_s = d_v + d_q$. Now we consider two  cases: 
        \begin{itemize}
            \item Case $l = l'$. Then we are done.
            \item Case $l \not= l'$. Since we are assuming that $l \in \dom{\stype_s}$, then it must be case that $l \in \dom{\stype_q}$. But, then by the \ih, we have $q \equivstate \upd{l}{w}{q'}$, for some value $w$ and store $q'$. Therefore, $s \equivstate \upd{l'}{v}{\upd{l}{w}{q'}} \equivstate \upd{l}{w}{\upd{l'}{v}{q'}}$.
        \end{itemize}
    \end{itemize}
    The correctness of the original statement now follows easily from the fact that, clearly, if $s \equivstate \upd{l}{v}{q}$, then $l \in \dom{s}$, by Definition~\ref{def:domainS}.
\end{proof}
} 

\begin{lemma}[{\bf Split Lemma}] \mbox{} 
    \label{lem:split-values-stores}
    \begin{enumerate}
        \item {(\bf Values)} \label{lem:com-split-values}  Let $\Phi_v \tr \seqi{\Gam}{v}{\M}{(b,m,d)}$, such that $\M = \sqcup_{\iI} \M_i$. Then, there exist ($\Phi^i_v \tr \seqi{\Gam_i}{v}{\M_i}{(b_i,m_i,d_i)})_{\iI}$, such that $\Gam = +_{\iI} \Gam_i$, $b = +_{\iI} b_i$, $m = +_{\iI} m_i$, and $d = +_{\iI} d_i$.
        \item {\bf (States)} \label{lem:split-state} Let $\Phi_s \tr \seqi{\Gam}{s}{\stype}{(b,m,d)}$, such that $l \in \dom{\stype}$. Then, $s \equivstate \upd{l}{v}{q}$, $\Phi_v \tr \seqi{\Gam_v}{v}{\stype(l)}{(b_v,m_v,d_v)}$ and $\Phi_q \tr \seqi{\Gam_q}{q}{\stype'}{(b_q,m_q,d_q)}$, such that $\Gam = \Gam_v + \Gam_q$, $\stype = \conj{(l : \stype(l))}; \stype'$, $b = b_v+b_q$, $m = m_v+m_q$, and $d = d_v + d_q$.
    \end{enumerate}
\end{lemma}

\maybehide{\begin{proof}
    The proof for values is very similar to the corresponding proof for $\lam_s$, so we are only going to show the split lemma for states.
    The proof follows by induction on the structure of $s$: \begin{itemize}
        \item Case $s = \estate$. Then the statement is vacuously true.
        \item Case $s = \upd{l'}{w}{q'}$. Then $\Phi_s$ is of the form: 
        \[ \begin{prooftree}
            \hypo{\Phi_{w} \tr \seqi{\Gam_{w}}{w}{\M}{(b_w,m_w,d_w)}}
            \hypo{\Phi_{q'} \tr \seqi{\Gam_{q'}}{q'}{\stype_{q'}}{(b_{q'}, m_{q'},d_{q'})}}
            \infer2[(\ruleUpd)]{\seqi{\Gam_{w} + \Gam_{q'}}{\upd{l'}{w}{q'}}{\conj{(l' : \M)}; \stype_{q'}}{(b_w+b_{q'},m_w+m_{q'},d_w+d_{q'})}}
        \end{prooftree} \] where $\Gam = \Gam_{w} + \Gam_{q'}$, $\stype = \conj{(l' : \M)}; \stype_{q'}$, $b = b_w + b_{q'}$, $m = m_w + m_{q'}$, and $d = d_w + d_{q'}$. 
        We  consider two cases: \begin{itemize}
            \item Case $l' = l$. Then we simply take $v = w$ and $q = q'$ and we are done.
            \item Case $l' \not= l$.  Since $l \in \dom{\conj{(l' : \M)}; \stype_{q'}}$ and $l' \not= l$, then $l \in \dom{\stype_{q'}}$. By applying the \ih to $q'$, we have that  $q' \equivstate \upd{l}{w'}{q''}$, $\Phi_{w'} \tr \seqi{\Gam_{w'}}{w'}{\stype_{q'}(l)}{(b_{w'},m_{w'},d_{w'})}$ and $\Phi_{q''} \tr \seqi{\Gam_{q''}}{q''}{\stype_{q''}}{(b_{q''},m_{q''},d_{q''})}$, such that $\Gam_{q'} = \Gam_{w'} + \Gam_{q''}$, $\stype_{q'} = \conj{(l : \stype_{q'}(l))}; \stype_{q''}$, $b_{q'} = b_{w'} + b_{q''}$, $m_{q'} = m_{w'} + m_{q''}$, and $d_{q'} = d_{w'} + d_{q''}$. But $s = \upd{l'}{w}{\upd{l}{w'}{q''}} \equivstate \upd{l}{w'}{\upd{l'}{w}{q''}}$, so we can take $v = w'$, $q = \upd{l'}{w}{q''}$, and consider $\Phi_q$ to be the following derivation:
            \[ \begin{prooftree}
                \hypo{\Phi_{w} \tr \seqi{\Gam_{w}}{w}{\M}{(b_w,m_w,d_w)}}
                \hypo{\Phi_{q''} \tr \seqi{\Gam_{q''}}{q''}{\stype_{q''}}{(b_{q''}, m_{q''}, d_{q''})}}
                \infer2[(\ruleUpd)]{\seqi{\Gam_{w} + \Gam_{q''}}{\upd{l'}{w}{q''}}{\conj{(l' : \M)}; \stype_{q''}}{(b_w+b_{q''}, m_w + m_{q''}, d_w+d_{q''})}}
            \end{prooftree} \] where  $\Gam_q = \Gam_{w} + \Gam_{q''}$ and $\stype_q=\conj{(l' : \M)}; \stype_{q''}$. We can then conclude with the following observations:
            \begin{itemize}
            \item $\Gam_v + \Gam_q = \Gam_{w'} +\Gam_{w} + \Gam_{q''} =
              \Gam_{w} + \Gam_{q'} = \Gam$,
                \item Since $\stype = \conj{(l' : \M)}; \stype_{q'}$ and $l' \not= l$, then $\stype(l) = \stype_{q'}(l)$ and
                \begin{align*}
                    \stype = \conj{(l' : \M)}; \stype_{q'} & = \conj{(l': \M)}; \conj{(l : \stype_{q'}(l))}; \stype_{q''} \\
                    & = \conj{(l : \stype_{q'}(l))}; \stype_{q} \\
                    & = \conj{(l : \stype(l))}; \stype_q
                \end{align*}
              \item $b_v + b_q= b_{w'} + b_{w} + b_{q''}=
                 b_w + b_{q'} = b$, $m_v + m_q= m_{w'} + m_{w} + m_{q''}=
                 m_w + m_{q'} = b$ and
                 $d_v + d_q= d_{w'} + d_{w} + d_{q''}=
                 d_w + d_{q'} = d$.
            \end{itemize} 
        \end{itemize}
    \end{itemize}
\end{proof}
} 

\begin{lemma}
    \label{lem:comp-values-not-neutral}
    Let $\Phi \tr \seqi{\Gam}{t}{\tcomptype{\stype}{\tau}{\stype'}}{(b,m,d)}$. If $t \in \val$, then $\tau \neq \tneutral$.
\end{lemma}

\maybehide{\begin{proof}
    By case analysis on the form of $t \in \val$:
    \begin{itemize}
        \item Case $t = x$. Then we have to consider three cases according to the last rule used in $\Phi$:
        \begin{itemize}
            \item Case $\Phi$ ends with rule (\ruleAx), then $t$ can only be assigned $\sig$. Therefore, this case does not apply.
            \item Case $\Phi$ ends with rule (\ruleMany), then $\tau = \M \neq \tneutral$.
        
            \item Case $\Phi$ ends with rule (\ruleLift). Then $\tau \in \{\tvar, \tabs, \M\}$, which means that $\tau \not= \tneutral$.
        \end{itemize}
        \item Case $t = \lam x.t$. Then we have to consider three cases according to the last rule used in $\Phi$:
        \begin{itemize}
            \item Case $\Phi$ ends with rule (\ruleLam), then $t$ can only be assigned $\sig$. Therefore, this case does not apply.
            \item Case $\Phi$ ends with rule (\ruleMany), then $\tau = \M  \neq \tneutral$.
            \item Case $\Phi$ ends with rule (\ruleLamP), then $\tau = \vl$. Therefore, this case does not apply.
            \item Case $\Phi$ ends with rule (\ruleLift). $\tau \in \{\tabs, \M\}$, which means that $\tau \not= \tneutral$.
        \end{itemize}
    \end{itemize}
\end{proof}} 

\begin{lemma}
    \label{lem:comp-notabs-implies-negabs}
    Let $\Phi \tr \seqi{\Gam}{t}{\tcomptype{\stype}{\tau}{\stype'}}{(b,m,d)}$, such that $\Gam$ is tight. If $\tau \in \nott{\vl}$, then $\neg\isabs{t}$.
\end{lemma}

\maybehide{\begin{proof}
    By induction over $\Phi$:
    \begin{itemize}
        \item Case $\Phi$ ends with rule (\ruleAx), (\ruleApp), (\ruleGet), (\ruleSet), (\ruleAxP) (\ruleAppPOne), or (\ruleAppPTwo), then $\neg\isabs{t}$ holds by definition.
        \item Case $\Phi$ ends with rule (\ruleLam), (\ruleMany), or (\ruleLamP), then $\tau \in \nott{\tabs}$ does not hold. Therefore, these cases do not apply.
    \end{itemize}
\end{proof}} 

\subsubsection{Completeness (Auxiliary Lemmas)}

\begin{lemma}[{\bf Merge for Values}]
    \label{lem:comp-merge-values}
    Let $(\Phi^i_v \tr \seqi{\Gam_i}{v}{\M_i}{(b_i,m_i,d_i)})_{\iI}$. Then, there exists $\Phi_v \tr \seqi{\Gam}{v}{\M}{(b,m,d)}$, such that $\Gam = +_{\iI} \Gam_i$, $\M = +_{\iI} \M_i$, $b = +_{\iI} b_i$, $m = +_{\iI} m_i$, and $d = +_{\iI}$.
\end{lemma}
We omit this proof given its similarity with the proof for system $\syscbv$.

\lemcomtightspreading*

\maybehide{\begin{proof}
  We want to show that, if $t \in \neutral$, then $\tau \in \tightt$, for some $\stype'$. We proceed by induction on the predicate  $t \in \neutral$:
    \begin{itemize}
        \item Case $t = xu$, such that $u \in \normal$. Then we have to consider the following two cases depending on the last rule in $\Phi$:
        \begin{itemize}
            \item Case $\Phi$ ends with rule ($\ruleApp$), then it must be of the following form:
            \[ \begin{prooftree}
                \infer0[(\ruleAx)]{\seqi{x : \mul{\M \ta (\comptype{\stype'}{\ctype})}}{x}{\M \ta (\comptype{\stype'}{\ctype})}{(0,0,0)}}
                \hypo{\Phi_u \tr \seqi{\Gam_u}{u}{\tcomptype{\stype}{\M}{\stype'}}{(b_u,m_u,d_u)}}
                \infer2[(\ruleApp)]{\seqi{(x : \mul{\M \ta (\comptype{\stype'}{\ctype})}) + \Gam_u}{xu}{\comptype{\stype}{\ctype}}{(1+b_u,m_u,d_u)}}
            \end{prooftree} \]
            where $\Gam = (x : \mul{\M \ta (\comptype{\stype'}{\ctype})}) + \Gam_p$ is tight, $b = 1+b_u$, $m = m_u$, and $d = d_u$. But $\M \ta (\comptype{\stype'}{\ctype}) \not\in \tightt$, therefore $\Gam$ is not tight and we have a contraction. Thus, this case does not apply.
            \item Case $\Phi$ ends with rule (\ruleAppPOne), then $\tau = \tneutral \in \tightt$, so we can conclude immediately.
        \end{itemize}
        \item Case $t = (\lambda x.p)u$, such that $u \in \neutral$. Then we have to consider the following two cases depending on the last rule in $\Phi$:
        \begin{itemize}
            \item Case $\Phi$ ends with rule ($\ruleApp$), then it must be of the following form:
            \[ \begin{prooftree}
                \hypo{\seqi{\Gam_p}{\lam x.p}{\M \ta (\comptype{\stype'}{\ctype})}{(b_p,m_p,d_p)}}
                \hypo{\Phi_u \tr \seqi{\Gam_u}{u}{\tcomptype{\stype}{\M}{\stype'}}{(b_u,m_u,d_u)}}
                \infer2[(\ruleApp)]{\seqi{\Gam_p + \Gam_u}{(\lam x.p)u}{\comptype{\stype}{\ctype}}{(1+b_p+b_u,m_p+m_u,d_p+d_u)}}
            \end{prooftree} \]
            where $\Gam = \Gam_u + \Gam_p$ is tight, $b = 1 + b_p + b_u$, $m = m_p + m_u$, and $d = d_p + d_u$. By the \ih on $u$, we have that $\M \in \tightt$, which is a contradiction. Therefore, this case does not apply.
            \item Case $\Phi$ ends with rule (\ruleAppPTwo). Then $\tau = \tneutral \in \tightt$, so we can conclude immediately.
        \end{itemize}
    \end{itemize}
\end{proof}
} 

\typstates*

\maybehide{\begin{proof} \mbox{}
    \begin{enumerate}
        \item \input{proofs/lem-typ-states}
        \item \input{proofs/lem-comp-typ-nfs}
    \end{enumerate}
\end{proof}


} 

\subsubsection{Soundness and Completeness (Main Lemmas)}

\lemcompsubsantisubs*

\maybehide{\begin{proof} \mbox{}
    \begin{enumerate}
        \item %\begin{proof}
    We are going to generalize the original statement by replacing $\del$ with $\gtype$.
    \\ \\
    The proof now follows by induction over the structure of $\Phi_t$:
        \begin{itemize}
            \item Case $\Phi_t$ ends with rule ($\ruleAx$). Then $t$ must be a variable and we must consider two cases:
            \begin{itemize}
                \item Assume $t = y = x$. Then $\Gam_t = \eset$, $\gtype = \M$, $t \subs{x}{v} = v$, $b_t = m_t = d_t = 0$. So we can take $\Phi_{t \subs{x}{v}} = \Phi_v$ and conclude with $\Gam_t + \Gam_v = \Gam_v$, $b_t + b_v = b_v$, $m_t + m_v = m_v$, and $d_t + d_v = d_v$.
                \item Assume $t = y \not= x$. Then $\M = \emul$, $\Gam_v = \eset$, $t \subs{x}{v} = t$, $b_v = 0$, $m_v = 0$, and $d_v = 0$. So we can take $\Phi_{t \subs{x}{v}} = \Phi_t$ and conclude with $\Gam_t + \Gam_v = \Gam_t$, $b_t + b_v = b_t$, $m_t + m_v = m_t$, and $d_t + d_v = d_t$.
            \end{itemize}
            \item Case $\Phi_t$ ends with (\ruleLam). Then $t$ must be of the form $\lam y.u$ and $\Phi_t$ must be of the following form (by $\alpha$-conversion):
            \[ \begin{prooftree}
                \hypo{\Phi_u \tr \seqi{\Gam; x : \M}{u}{\comptype{\stype}{\ctype}}{(b_t,m_t,d_t)}}
                \infer1[(\ruleLam)]{\seqi{(\Gam \sm y); x : \M}{\lam y.u}{\Gam(y) \ta (\comptype{\stype}{\ctype})}{(b_t,m_t,d_t)}}
            \end{prooftree} \]
            where $\Gam_t = (\Gam \sm y)$, and $\gtype = \Gam(y) \ta (\comptype{\stype}{\ctype})$. By the \ih, we have the following derivation $\Phi_{u \subs{x}{v}} \tr \seqi{\Gam + \Gam_v}{u \subs{x}{v}}{\comptype{\stype}{\ctype}}{(b_t+b_v,m_t+m_v,d_t+d_v)}$. Therefore, we can build $\Phi_{t \subs{x}{v}}$ as follows:
            \[ \begin{prooftree}
                \hypo{\Phi_{u \subs{x}{v}} \tr \seqi{\Gam + \Gam_v}{u \subs{x}{v}}{\comptype{\stype}{\ctype}}{(b_t+b_v,m_t+m_v,d_t+d_v)}}
                \infer1[(\ruleLam)]{\seqi{(\Gam + \Gam_v) \sm y}{\lambda y.u \subs{x}{v}}{\Gam(y) \ta (\comptype{\stype}{\ctype})}{(b_t+b_v,m_t+m_v,d_t+d_v)}}
            \end{prooftree} \]
            And we conclude with $(\Gam + \Gam_v) \sm y = (\Gam \sm y) + \Gam_v = \Gam_t + \Gam_v$, by $\alpha$-conversion.
            \item Case $\Phi_t$ ends with ($\ruleApp$). Then $t$ must be of the form $wu$ and $\Phi_t$ must be of following form:
            \[ \begin{prooftree}
                \hypo{\Phi_w \tr \seqi{\Gam; x : \M_1}{w}{\M' \ta (\comptype{\stype'}{\ctype})}{(b_w,m_w,d_w)}}
                \hypo{\Phi_u \tr \seqi{\Del; x : \M_2}{u}{\tcomptype{\stype}{\M'}{\stype'}}{(b_u,m_u,d_u)}}
                \infer2[(\ruleApp)]{\seqi{\Gam + \Del; x : \M_1 \sqcup \M_2}{wu}{\comptype{\stype}{\ctype}}{(1+b_w+b_u,m_w+m_u,d_w+d_u)}}
            \end{prooftree} \]
            such that $\Gam_t = \Gam + \Del$, $\M = \M_1 \sqcup \M_2$, $\gtype = \comptype{\stype}{\ctype}$, $b_t = 1+b_w+b_u$, $m_t = m_w+m_u$, and $d_t = d_w + d_u$. By~\cref{lem:split-values-stores}.\ref{lem:com-split-values}, we know there exist the following derivations $(\Phi^i_v \tr \seqi{\Gam^i_v}{v}{\M_i}{(b_i,m_i,d_i)})_{i \in \{1,2\}}$, such that $\Gam_v = \Gam^1_v + \Gam^2_v$, $b_v = b_1 + b_2$, $m_v = m_1 + m_2$, and $d_v = d_1 + d_2$. By the \ih, we know there exist $\Phi_{w \subs{x}{v}} \tr \seqi{\Gam + \Gam^1_v}{w \subs{x}{v}}{\M' \ta (\comptype{\stype'}{\ctype})}{(b_w+b_1,m_w+m_1,d_w+d_1)}$ and $\Phi_{u \subs{x}{v}} \tr \seqi{\Del + \Gam^2_v}{u \subs{x}{v}}{\tcomptype{\stype}{\M'}{\stype'}}{(b_u+b_2,m_u+m_2,d_u+d_2)}$. %Assume $\Phi_{w \subs{x}{v}} \tr \seqi{\Gam + \Gam^1_v}{w \subs{x}{v}}{\tcomptype{\M'}{\stype'}{\ctype}}{(b_w+b_1,m_w+m_1,d_w+d_1)}$ and $\Phi_{u \subs{x}{v}} \tr \seqi{\Del + \Gam^2_v}{u \subs{x}{v}}{\stype \ta (\comptype{\M'}{\stype'})}{(b_u+b_2,m_u+m_2,d_u+d_2)}$. 
            We can build $\Phi_{t \subs{x}{v}}$ as follows:
            \[ \begin{prooftree}
                \hypo{\Phi_{w \subs{x}{v}}}
                \hypo{\Phi_{u \subs{x}{v}}}
                \infer2[(\ruleApp)]{\seqi{(\Gam + \Del) + (\Gam^1_v + \Gam^2_v)}{(wu) \subs{x}{v}}{\comptype{\stype}{\ctype}}{(1+b_w+b_u+b_1+b_2,m_w+m_u+m_1+m_2,d_w+d_u+d_1+d_2)}}
            \end{prooftree} \]
            And we can conclude with $\Gam_t + \Gam_v = (\Gam + \Del) + (\Gam^1_v + \Gam^2_v)$, $b_t + b_v = 1 + b_w+b_u+b_1+b_2$, $m_t + m_v = m_w+m_u+m_1+m_2$, and $d_t + d_v = d_w+d_u+d_1+d_2$.
            \item Case $\Phi_w$ ends with ($\ruleMany$). Then $t$ must be of the form $w$ and $\Phi_t$ must be of the following form:
            \[ \begin{prooftree}
                \hypo{(\Phi^i_w \tr \seqi{\Gam_i; x : \M_i}{w}{\rdel_i}{(b_i,m_i,d_i)})_{\iI}}
                \infer1[(\ruleMany)]{\seqi{+_{\iI} \Gam_i; x : \sqcup_{\iI} \M_i}{w}{\mul{\rdel_i}_{\iI}}{(+_{\iI}b_i, +_{\iI}m_i, +_{\iI}d_i)}}
            \end{prooftree} \]
            such that $\Gam_t = +_{\iI} \Gam_i$, $\gtype = \mul{\rdel_i}_{\iI}$, $b_t = +_{\iI} b_i$, $m_t = +_{\iI} m_i$, and $d_t = +_{\iI} d_i$. By~\cref{lem:split-values-stores}.\ref{lem:com-split-values}, $(\Phi^i_v \tr \seqi{\Gam^i_v}{v}{\M_i}{(b^i_v,m^i_v,d^i_v)})_{\iI}$, such that $\Gam_v = +_{\iI} \Gam^i_v$, $b_v = +_{\iI} b^i_v$, $m_v = +_{\iI} m^i_v$, and $d_v = +_{\iI} d^i_v$. By the \ih over each $\Phi^i_v$, we have $(\Phi^i_{w \subs{x}{v}} \tr \seqi{\Gam_i + \Gam^i_v}{w \subs{x}{v}}{\rdel_i}{(b_i+b^i_v,m_i+m^i_v,d_i+d^i_v)})_{\iI}$. Therefore, we can build $\Phi_{t \subs{x}{v}}$ as follows:
            \[ \begin{prooftree}
                \hypo{(\Phi^i_{w \subs{x}{v}} \tr \seqi{\Gam_i + \Gam^i_v}{w \subs{x}{v}}{\rdel_i}{(b_i+b^i_v,m_i+m^i_v,d_i+d^i_v)})_{\iI}}
                \infer1[(\ruleMany)]{\seqi{+_{\iI} (\Gam^i_v + \Gam^i_w)}{w \subs{x}{v}}{\mul{\tau_i}_{\iI}}{(+_{\iI}(b_i+b^i_v),+_{\iI}(m_i+m^i_v),+_{\iI}(d_i+d^i_v))}}
            \end{prooftree} \]
            And we can conclude with $\Gam_t + \Gam_v = +_{\iI} \Gam_i +_{\iI} \Gam^i_v = +_{\iI} (\Gam_i + \Gam^i_v)$, $b_t + b_v = +_{\iI} b_i +_{\iI} b^i_v = +_{\iI} (b_i + b^i_v)$, $m_t + m_v = +_{\iI} m_i +_{\iI} m^i_v = +_{\iI} (m_i + m^i_v)$, and $d_t + d_v = +_{\iI} d_i +_{\iI} d^i_v = +_{\iI} (d_i + d^i_v)$.
            \item Case $\Phi_t$ ends with (\ruleLift). Then $t$ is a variable and $\Phi_t$ must be of the following form:
            \[ \begin{prooftree}
                \hypo{\Phi_w \tr \seqi{\Gam; x : \M}{w}{\M'}{(b_t,m_t,d_t)}}
                \infer1[(\ruleLift)]{\seqi{\Gam; x : \M}{w}{\tcomptype{\stype}{\M'}{\stype}}{(b_t,m_t,d_t)}}
            \end{prooftree} \]
            where $\gtype = \tcomptype{\stype}{\M'}{\stype}$. By the \ih, we have $\Phi_{w \subs{x}{v}} \tr \seqi{\Gam + \Gam_v}{w \subs{x}{v}}{\M'}{(b_t+b_v, m_t+m_v, d_t +d_v)}$. Therefore, we can build $\Phi_{t \subs{x}{v}}$ as follows:
            \[ \begin{prooftree}
                \hypo{\Phi_{w \subs{x}{v}} \tr \seqi{\Gam + \Gam_v}{w \subs{x}{v}}{\M'}{(b_t+b_v, m_t+m_v, d_t +d_v)}}
                \infer1[(\ruleLift)]{\seqi{\Gam + \Gam_v}{w \subs{x}{v}}{\tcomptype{\stype}{\M'}{\stype}}{(b_t+b_v, m_t+m_v, d_t +d_v)}}
            \end{prooftree} \]
            And we can conclude.
            \item Case $\Phi_t$ ends with ($\ruleGet$). Then $t$ must be of the form $\get{l}{y}{u}$ and $\Phi_t$ must be of the following form:
            \[ \begin{prooftree}
                \hypo{\Phi_u \tr \seqi{\Gam_u; x : \M}{u}{\comptype{\stype}{\ctype}}{(b_u,m_u,d_u)}}
                \infer1[(\ruleGet)]{\seqi{(\Gam_u \sm y); x : \M}{\get{l}{y}{u}}{\comptype{\conj{(l : \Gam_{u}(y))} \splus \stype}{\ctype}}{(b_u,1+m_u,d_u)}}
            \end{prooftree} \]
          where $\gtype = \comptype{\conj{(l : \Gam_{u}(y))} \splus \stype}{\ctype}$, $\Gam_t = \Gam_u \sm y$, $b_t = b_u$, $m_t = 1+m_u$, and $d_t = d_u$. By the \ih, we have $\Phi_{u \subs{x}{v}} \tr \seqi{\Gam_u + \Gam_v}{u \subs{x}{v}}{\stype \ra \ctype}{(b_u+b_v,m_u+m_v,d_u+d_v)}$. Therefore, we can build $\Phi_{t \subs{x}{v}}$ as follows:
            \[ \begin{prooftree}
                \hypo{\Phi_{u \subs{x}{v}} \tr \seqi{\Gam_u + \Gam_v}{u \subs{x}{v}}{\comptype{\stype}{\ctype}}{(b_u+b_v,m_u+m_v,d_u+d_v)}}
                \infer1[(\ruleGet)]{\seqi{(\Gam_u  + \Gam_v) \sm y}{\get{l}{y}{u} \subs{x}{v}}{\comptype{\conj{(l : \Gam_u(y))} \splus \stype}{\ctype}}{(b_u+b_v,1+m_u+m_v,d_u+d_v)}}
            \end{prooftree} \]
            And we can conclude with $(\Gam_u + \Gam_v) \sm y = (\Gam \sm y) + \Gam_v = \Gam_t + \Gam_v$ by $\alpha$-conversion, $b_t + b_v = b_u+b_v$, $m_t + m_v = 1+m_u+m_v$, and $d_t + d_v = d_u +d_v$.
            \item Case $\Phi_t$ ends with ($\ruleSet$). Then $t$ must be of the form $\set{l}{w}{u}$ and $\Phi_t$ must be of the following form:
            \[ \begin{prooftree}
                \hypo{\Phi_w \tr \seqi{\Gam_w; x : \M_1}{w}{\M'}{(b_w,m_w,d_w)}}
                \hypo{\Phi_u \tr \seqi{\Gam_u; x : \M_2}{u}{\comptype{\conj{(l : \M')}; \stype}{\ctype}}{(b_u,m_u,d_u)}}
                \infer2[(\ruleSet)]{\seqi{\Gam_w + \Gam_u; x : \M_1 \sqcup \M_2}{\set{l}{w}{u}}{\comptype{\stype}{\ctype}}{(b_w+b_u,1+m_w+m_u,d_w+d_u)}}
            \end{prooftree} \]
            where $\gtype = \comptype{\stype}{\ctype}$, $\Gam_t = \Gam_w + \Gam_u$, $\del = \comptype{\stype}{\ctype}$, $b_t = b_w+b_u$, $m_t = 1+m_w + m_u$, and $d_t = d_w + d_u$. By~\cref{lem:split-values-stores}.\ref{lem:com-split-values}, we have $\Phi^1_v \tr \seqi{\Gam^1_v}{v}{\M_1}{(b^1_v,m^1_v,d^1_v)}$ and $\Phi^2_v \tr \seqi{\Gam^2_v}{v}{\M_2}{(b^2_v,m^2_v,d^2_v)}$, such that $\Gam_v = \Gam^1_v + \Gam^2_v$, $b_v = b^1_v + b^2_v$, $m_v = m^1_v + m^2_v$, and $d_v = d^1_v + d^2_v$. By the \ih, we have $\Phi_{w \subs{x}{v}} \tr \seqi{\Gam_w + \Gam^1_v}{w \subs{x}{v}}{\M'}{(b_w+b^1_v,m_w+m^1_v,d_w+d^1_v)}$ and $\Phi_{u \subs{x}{v}} \tr \seqi{\Gam_u + \Gam^2_v}{u \subs{x}{v}}{\comptype{\conj{(l : \M')}; \stype}{\ctype}}{(b_u+b^2_v,m_u+m^2_u,d_u+d^2_v)}$. Assume $\Phi_{w \subs{x}{v}} \tr \seqi{\Gam_w + \Gam^1_v}{w \subs{x}{v}}{\M'}{(b_w+b^1_v,m_w+m^1_v,d_w+d^1_v)}$ and $\Phi_{u \subs{x}{v}} \tr \seqi{\Gam_u + \Gam^2_v}{u \subs{x}{v}}{\comptype{\conj{(l : \M')}; \stype}{\ctype}}{(b_u+b^2_v,m_u+m^2_u,d_u+d^2_v)}$. We can build $\Phi_{t \subs{x}{v}}$ as follows:
            \[ \begin{prooftree}
                \hypo{\Phi_{w \subs{x}{v}}}
                \hypo{\Phi_{u \subs{x}{v}}}
                \infer2[(\ruleSet)]{\seqi{(\Gam_w + \Gam_u) + (\Gam^1_v + \Gam^2_v)}{(wu) \subs{x}{v}}{\comptype{\stype}{\ctype}}{(b_w+b_u+b^1_v+b^2_v,1+m_w+m_u+m^1_v+m^2_v,d_w+d_u+d^1_v+d^2_v)}}
            \end{prooftree} \]
            And we can conclude with $\Gam_t + \Gam_v = (\Gam_w + \Gam_u) + (\Gam^1_v + \Gam^2_v)$, $b_t + b_v = b_w+b_u+b^1_v+b^2_v$, $m_t + m_v = 1+m_w+m_u+m^1_v+m^2_v$, $d_t + d_v = d_w+d_u+d^1_v+d^2_v$.
            \item Case $\Phi_t$ ends with (\ruleAxP). Then $t$ must be a variable and we must consider two cases:
            \begin{itemize}
                \item Assume $t = y = x$. Then $\Gam_t = \eset$, $\gtype = \stype \ta (\comptype{\nott{\tneutral}}{\stype})$, $t \subs{x}{v} = v$, $b_t = m_t = d_t = 0$. Moreover, $\M = \mul{\nott{\tneutral}}$. We have to consider two cases:
                \begin{itemize}
                    \item Case $v = z$. Then $\Phi_v \tr \seqi{z : \mul{\nott{\tneutral}}}{z}{\mul{\nott{\tneutral}}}{(0,0,0)}$. So we can take $\Phi_{t \subs{x}{v}}$ as the following derivation:
                    \[ \begin{prooftree}
                        \infer0[(\ruleAxP)]{\seqi{z : \mul{\nott{\tneutral}}}{z}{\tcomptype{\stype}{\nott{\tneutral}}{\stype}}{(0,0)}}
                    \end{prooftree} \]
                    and conclude with $\Gam_t + \Gam_v = \Gam_v = (z : \mul{\nott{\tneutral}})$, $b_t + b_v = b_v = 0$, $m_t + m_v = m_v = 0$, and $d_t + d_v = d_v$.
                    \item Case $v = \lam z.p$. This case does not apply, by~\cref{lem:comp-notabs-implies-negabs}.
                \end{itemize}
                \item Assume $t = y \neq x$. Then $\M = \emul$, $\Gam_v = \eset$, $t \subs{x}{v} = t$, $b_v = 0$, $m_v = 0$, and $d_v = 0$. So we can take $\Phi_{t \subs{x}{v}} = \Phi_t$ and conclude with $\Gam_t + \Gam_v = \Gam_t$, $b_t + b_v = b_t$, $m_t + m_v = m_t$, and $d_t + d_v = d_t$.
            \end{itemize}
            \item Case $\Phi_t$ ends with (\ruleLamP). Then $t$ is of the form $\lam y.u$, $\Gam_t = \eset$, $\gtype = \tcomptype{\stype}{\vl}{\stype}$, $\M = \emul$, $\Gam_v = \eset$, $t \subs{x}{v} = \lam y.(u \subs{x}{v}) = (\lam y.u) \subs{x}{v}$, $b_t = b_v = 0$, $m_t = m_v = 0$, and $d_t = d_v = 0$. So we can build $\Phi_{t \subs{x}{v}}$ as follows:
            \[ \begin{prooftree}
                \infer0[(\ruleLamP)]{\seqi{}{(\lam y.u) \subs{x}{v}}{\tcomptype{\stype}{\vl}{\stype}}{(0,0,0)}}
            \end{prooftree} \]
            And conclude with $\Gam_t + \Gam_v = \eset$, $b_t = b_v = 0$, $m_t = m_v = 0$, and $d_t = d_v = 0$.
            \item Case $\Phi_t$ ends with (\ruleAppPOne). Then $t$ is of the form $yu$ and we have to consider two cases:
            \begin{itemize}
                \item Case $y = x$. Then $\Phi_t$ must be of the following form:
                \[ \begin{prooftree}
                    \hypo{\seqi{\Gam_u}{u}{\tcomptype{\stype}{\tightt}{\stype}}{(b_u,m_u,d_u)}}
                    \infer1[(\ruleAppPOne)]{\seqi{(x : \mul{\tvar} \sqcup \Gam_u(x)); (\Gam_u \sm x)}{x u}{\tcomptype{\stype}{\tneutral}{\stype}}{(b_u,m_u,1+d_u)}}
                \end{prooftree} \]
                such that $\Gam_t = (\Gam_u \sm x)$, $b = b_u$, $m = m_u$, and $d = 1+d_u$. Then $\M = \mul{\tvar} \sqcup \Gam_u(x)$ and, by~\cref{lem:split-values-stores}.\ref{lem:com-split-values}, we have $\Phi^1_v \tr \seqi{\Gam^1_v}{v}{\mul{\tvar}}{(b^1_v,m^1_v,d^1_v)}$ and $\Phi^2_v \tr \seqi{\Gam^2_v}{v}{\Gam_u(x)}{(b^2_v,m^2_v,d^2_v)}$, such that $\Gam_v = \Gam^1_v + \Gam^2_v$, $b_v = b^1_v + b^2_v$, $m_v = m^1_v + m^2_v$, and $d_v = d^1_v + d^2_v$. By the \ih, we know there exists $\Phi_{u \subs{x}{v}} \tr \seqi{(\Gam_u \sm x) + \Gam^2_u}{u \subs{x}{v}}{\tcomptype{\stype}{\tightt}{\stype}}{(b_u+b^2_v, m_u+m^2_v, d_u+d^2_v)}$. Now, we need to consider two cases:
                \begin{itemize}
                    \item Case $v = z$. Then $\Phi^1_v \tr \seqi{z : \mul{\tvar}}{z}{\mul{\tvar}}{(0,0,0)}$ and $\Phi^2_v \tr \seqi{z : \Gam_u(x)}{z}{\Gam_u(x)}{(0,0)}$. Therefore, we can build $\Phi_{t \subs{x}{v}} = \Phi_{v \subs{x}{v}}$ as follows:
                    \[ \begin{prooftree}
                        \hypo{\Phi_{u \subs{x}{v}} \tr \seqi{(\Gam_u \sm x) + \Gam_u(x)}{u \subs{x}{v}}{\tcomptype{\stype}{\tightt}{\stype}}{(b_u+b^2_v, m_u+m^2_v, d_u+d^2_v)}}
                        \infer1[(\ruleAppPOne)]{\seqi{(z : \mul{\tvar}) + (\Gam_u \sm x + (z : \Gam_u(x)))}{z (u \subs{x}{v})}{\tcomptype{\stype}{\tneutral}{\stype}}{(b_u+b^2_v,m_u+m^2_v,1+d_u+d^2_v)}}
                    \end{prooftree} \]
                    where $(z : \mul{\tvar}) + (\Gam_u \sm x + (z : \Gam_u(x))) = (\Gam_u \sm x) + (z : \mul{\tvar} \cup \Gam_u(x)) = \Gam_u + \Gam_v$, $b_u + b^2_v = b + b^1_v + b^2_v = b + b_v$, $m_u + m^2_v = m + m^1_v + m^2_v = m + m_v$, and $d_u + d^2_v = d + d^1_v + d^2_v = d + d_v$.
                    \item Case $v = \lam z.p$. This case does not apply, since it is not possible to assign $\tvar$ to $\lam z.p$, by~\cref{lem:comp-notabs-implies-negabs}.
                \end{itemize}
                \item Case $y \neq x$. Then, the proof is very similar to when $\Phi_t$ ends with rule ($\ruleApp$).
            \end{itemize}
            \item Case $\Phi_t$ ends with (\ruleAppPTwo), the proof is very similar to when $\Phi_t$ ends with rule (\ruleAppPOne).
        \end{itemize}
%\end{proof}

        \item %\begin{proof}
    We are going to generalize the original statement by replacing $\del$ with $\gtype$.
    \\ \\
    The proof follows by induction over $t$:
    \begin{itemize}
        \item Case $t = y$. Then we have to consider two cases:
        \begin{itemize}
            \item Let $t = y \not= x$. Then $t \subs{x}{v} = y$. Let $\Gam_v = \eset$, $\M = \emul$, $b_v = m_v = d_v = 0$. Then, $\Phi_v$ is derivable using rule ($\ruleMany$) with no premise. We also take $\Phi_t = \Phi_{t \subs{x}{v}}$, so that, in particular $\Gam_t = \Gam_{t \subs{x}{v}}$. Then, we can conclude with $\Gam_{t \subs{x}{v}} = \Gam_t + \Gam_v = \Gam_t$, $b = b_t + b_v = b_t$, $m = m_t + m_v = m_t$, and $d = d_t + d_v = d_t$.
            \item Let $t = y = x$. Then $t \subs{x}{v} = v$. Let $\Gam_t = \eset$, and $b_t = m_t = s_t = 0$. Now we will consider two cases depending on the form of $v$:
            \begin{itemize}
                \item Case $v = z$. Then $t \subs{x}{v} = z$ and we can proceed by case analysis of the last rule in $\Phi_{t\subs{x}{v}}$. In all of them, we can build $\Phi_t$ from $\Phi_{t \subs{x}{v}}$, by simply replacing $x$ with $z$, and $\Phi_v$ as follows:
                \[ \begin{prooftree}
                    \infer0[(\ruleAx)]{\seqi{z : \mul{\sig}}{z}{\sig}{(0,0,0)}}
                    \infer1[(\ruleMany)]{\seqi{z : \mul{\sig}}{z}{\mul{\sig}}{(0,0,0)}}
                \end{prooftree} \]
                And we can conclude since all the counters are zero.
                \item Case $v = \lam z.p$. Then $t \subs{x}{v} = \lam z.p$ and we can proceed by case analysis of the last rule in $\Phi_{t \subs{x}{v}}$. In all of them, we can always build $\Phi_t$ using either (\ruleAx) (case (\ruleApp)), (\ruleAxP) (case (\ruleLamP)),  (\ruleAx) plus (\ruleMany) (case (\ruleMany)), or (\ruleAx) plus (\ruleMany) plus (\ruleLift) (case (\ruleLift)). $\Phi_v$  is either $\Phi_{t \subs{x}{v}}$ (case (\ruleMany)), or it can be built from $\Phi_{t \subs{x}{v}}$ plus rule (\ruleMany) (all other cases).
            \end{itemize}
        \end{itemize}
        \item Case $t = \lam y.u$. Then $t \subs{x}{v} = (\lam y.u)\subs{x}{v} = \lam y.(u \subs{x}{v})$ and we must consider three cases:
        \begin{itemize}
            \item Case $\Phi_{t \subs{x}{v}}$ ends with rule (\ruleLam), then it must be of the following form: 
            \[ \begin{prooftree}
                \hypo{\Phi_{u \subs{x}{v}} \tr \seqi{\Gam_{u \subs{x}{v}}; y : \M'}{u \subs{x}{v}}{\comptype{\stype}{\ctype}}{(b,m,d)}}
                \infer1[(\ruleLam)]{\seqi{\Gam_{u \subs{x}{v}}}{\lam y.(u \subs{x}{v})}{\M' \ta (\comptype{\stype}{\ctype})}{(b,m,d)}}
            \end{prooftree} \]
            where $\gtype = \M' \ta (\comptype{\stype}{\ctype})$ and $\Gam_{t \subs{x}{v}} = \Gam_{u \subs{x}{v}}$. By the \ih, we have $\Phi_u \tr \seqi{\Gam_u; y : \M'; x : \M}{u}{\comptype{\stype}{\ctype}}{(b_u,m_u,d_u)}$ and $\Phi_v \tr \seqi{\Gam_v}{v}{\M}{(b_v,m_v,d_v)}$, such that $\Gam_{u \subs{x}{v}} = \Gam_u + \Gam_v$, $b = b_u + b_v$, $m = m_u + m_v$, and $d = d_u + d_v$. So we can build $\Phi_{\lam y.u}$ as follows:
            \[ \begin{prooftree}
                \hypo{\Phi_u \tr \seqi{\Gam_u; y: \M'; x : \M}{u}{\comptype{\stype}{\ctype}}{(b_u,m_u,d_u)}}
                \infer1[(\ruleLam)]{\seqi{\Gam_u; x : \M}{\lam y.u}{\M' \ta (\comptype{\stype}{\ctype})}{(b_u,m_u,d_u)}}
            \end{prooftree} \]
            And we can pick $\Phi_t = \Phi_{\lam y.u}$, and conclude with $\Gam_{t \subs{x}{v}} = \Gam_{u \subs{x}{v}} = \Gam_u + \Gam_v$, $b = b_u + b_v$, $m = m_u + m_v$, and $d = d_u + d_v$.
            \item Case $\Phi_{t \subs{x}{v}}$ ends with rule (\ruleLamP). Then it must be of the following form:
            \[ \begin{prooftree}
                \infer0[(\ruleLamP)]{\seqi{}{\lam y.(u \subs{x}{y})}{\tcomptype{\stype}{\vl}{\stype}}{(0,0,0)}}
            \end{prooftree} \]
            where $\Gam_{t \subs{x}{v}} = \eset$, $\gtype = \tcomptype{\stype}{\vl}{\stype}$, and $b = m = d = 0$. Let $\Gam_t = \eset$, $\M = \emul$, and $b_t = m_t = d_t = 0$. Then, we can construct $\Phi_t$ as follows:
            \[ \begin{prooftree}
                \infer0[(\ruleLamP)]{\seqi{}{\lam y.u}{\tcomptype{\stype}{\vl}{\stype}}{(0,0,0)}}
            \end{prooftree} \]
            Let $\Gam_v = \eset$, and $b_v = m_v = d_v = 0$. Then $\Phi_v$ can be constructed by using rule ($\ruleMany$) with no premises. So we can conclude with $\Gam_{t \subs{x}{v}} = \eset = \Gam_t + \Gam_v$, and $b = 0 = b_t + b_v$, $m = 0 = m_t + m_v$, and $d = 0 = d_t + d_v$.
            \item Case $\Phi_{t \subs{x}{v}}$ ends with rule ($\ruleMany$). Then $t \subs{x}{v}$ is a value, and $\Phi_{t \subs{x}{v}}$ must be of the following form:
            \[ \begin{prooftree}
                \hypo{(\Phi_i \tr \seqi{\Gam_i}{t \subs{x}{v}}{\rdel_i}{(b_i,m_i,d_i)})_{\iI}}
                \infer1[(\ruleMany)]{\seqi{+_{\iI} \Gam_i}{t \subs{x}{v}}{\mul{\rdel_i}_{\iI}}{(+_{\iI} b_i, +_{\iI} m_i, +_{\iI} d_i)}}
            \end{prooftree} \]
            where $\gtype = \mul{\rdel_i}_{\iI}$, $\Gam_{t \subs{x}{v}} = +_{\iI} \Gam_i$, $b = +_{\iI} b_i$, $m = +_{\iI} m_i$, and $d = +_{\iI} d_i$. By the \ih over each $\Phi_i$, we have the following derivations $\Phi^i_t \tr \seqi{\Gam^i_t; x : \M_i}{t}{\rdel_i}{(b^i_t,m^i_t,d^i_t)}$ and $\Phi^i_v \tr \seqi{\Gam^i_v}{v}{\M_i}{(b^i_v, m^i_v, d^i_v)}$, such that $\Gam_i = \Gam^i_t + \Gam^i_v$, $b = b^i_t + b^i_v$, $m = m^i_t + m^i_v$, and $d = d^i_t + d^i_v$, for each $\iI$. So we can construct $\Phi_t$ as follows:
            \[ \begin{prooftree}
                \hypo{(\Phi^i_t \tr \seqi{\Gam^i_t; x : \M_i}{t}{\rdel_i}{(b^i_t,m^i_t,d^i_t)})_{\iI}}
                \infer1[(\ruleMany)]{\seqi{+_{\iI} \Gam^i_t; x : \sqcup_{\iI} \M_i}{t}{\mul{\rdel_i}_{\iI}}{(+_{\iI} b^i_t, +_{\iI} m^i_t, +_{\iI} d^i_t)}}
            \end{prooftree} \]
            such that $\Gam_t = +_{\iI} \Gam^i_t$, $\M = \sqcup_{\iI} \M_i$, $b_t = +_{\iI} b^i_t$, $m_t = +_{\iI} m^i_t$, and $d_t = +_{\iI} d^i_t$. By~\cref{lem:comp-merge-values}, we can take the following derivation $\Phi_v \tr \seqi{+_{\iI} \Gam^i_v}{v}{\M}{(+_{\iI} b^i_v, +_{\iI} m^i_v, +_{\iI} d^i_v)}$. And we can conclude with $\Gam_{t \subs{x}{v}} = +_{\iI} \Gam_i = +_{\iI} (\Gam^i_t + \Gam^i_v) = +_{\iI} \Gam^i_t +_{\iI} \Gam^i_v = \Gam_t + \Gam_v$, $b = +_{\iI} b_i = +_{\iI} (b^i_t + b^i_v) = +_{\iI} b^i_t +_{\iI} b^i_v = b_t + b_v$, $m = +_{\iI} m_i = +_{\iI} (m^i_t + m^i_v) = +_{\iI} m^i_t +_{\iI} m^i_v = m_t + m_v$, and $d = +_{\iI} d_i = +_{\iI} (d^i_t + d^i_v) = +_{\iI} d^i_t +_{\iI} d^i_v = d_t + d_v$.
        \end{itemize}
        \item Let $t = wu$. Then $t \subs{x}{v} = (wu) \subs{x}{v} = (w \subs{x}{v})(u \subs{x}{v})$, and we have to consider three cases:
        \begin{itemize}
            \item Case $\Phi_{t \subs{x}{v}}$ ends with ($\ruleApp$). Assume $\Phi_{w \subs{x}{v}} \tr \seqi{\Gam_{w \subs{x}{v}}}{w \subs{x}{v}}{\M' \ta (\comptype{\stype'}{\ctype})}{(b',m',d')}$ and $\Phi_{u \subs{x}{v}} \tr \seqi{\Gam_{u \subs{x}{v}}}{u \subs{x}{v}}{\tcomptype{\stype}{\M'}{\stype'}}{(b'',m'',d'')}$. $\Phi_{t \subs{x}{v}}$ must be of the following form:
            \[ \begin{prooftree}
                \hypo{\Phi_{w \subs{x}{v}}}
                \hypo{\Phi_{u \subs{x}{v}}}
                \infer2[(\ruleApp)]{\seqi{\Gam_{w \subs{x}{v}} + \Gam_{u \subs{x}{v}}}{(w \subs{x}{v})(u \subs{x}{v})}{\comptype{\stype}{\ctype}}{(1+b'+b'',m'+m'',d'+d'')}}
            \end{prooftree} \]
            where $\gtype = \comptype{\stype}{\ctype}$, $\Gam_{t \subs{x}{v}} = \Gam_{w \subs{x}{v}} + \Gam_{u \subs{x}{v}}$, $b = 1+b'+b''$, $m = m'+m''$, and $d = d'+d''$. By the \ih over $\Phi_{w \subs{x}{v}}$, we have $\Phi_w \tr \seqi{\Gam_w; x : \M_1}{w}{\M' \ta (\comptype{\stype'}{\ctype})}{(b_w,m_w,d_w)}$ and $\Phi^1_v \tr \seqi{\Gam^1_v}{v}{\M_1}{(b^1_v,m^1_v,d^1_v)}$, such that $\Gam_{w \subs{x}{v}} = \Gam_w + \Gam^1_v$, $b' = b_w + b^1_v$, $m' = m_w + m^1_v$, and $d' = d_w + d^1_v$. And by the \ih over $\Phi_{u \subs{x}{v}}$, we have $\Phi_u \tr \seqi{\Gam_u; x : \M_2}{u}{\tcomptype{\stype}{\M'}{\stype'}}{(b_u, m_u,d_u)}$ and $\Phi^2_v \tr \seqi{\Gam^2_v}{v}{\M_2}{(b^2_v,m^2_v,d^2_v)}$, such that $\Gam_{u \subs{x}{v}} = \Gam_u + \Gam^2_v$, $b'' = b_u + b^2_v$, $m'' = m_u + m^2_v$, and $d'' = d_u + d^2_v$. By~\cref{lem:comp-merge-values}, we can take $\Phi_v \tr \seqi{\Gam^1_v + \Gam^2_v}{v}{\M_1 \sqcup \M_2}{(b^1_v+b^2_v, m^1_v+m^2_v, d^1_v+d^2_v)}$, such that $\Gam_v = \Gam^1_v + \Gam^2_v$, $b_v = b^1_v + b^2_v$, $m_v = m^1_v + m^2_v$, and $d_v = d^1_v + d^2_v$. And we can build $\Phi_{wu}$ as follows:
            \[ \begin{prooftree}
                \hypo{\Phi_w}
                \hypo{\Phi_u}
                \infer2[(\ruleApp)]{\seqi{(\Gam_w + \Gam_u); x : \M_1 \sqcup \M_2}{wu}{\comptype{\stype}{\kappa}}{(1+b_w+b_u,m_w+m_u,d_w+d_u)}}
            \end{prooftree} \]
            such that $\Gam_t = \Gam_w + \Gam_u$, $b_t = 1 + b_w + b_u$, $m_t = b_w + b_u$, and $d_t = d_w + d_u$. So we can pick $\Phi_t = \Phi_{wu}$, and conclude with $\Gam_{t \subs{x}{v}} = \Gam_{w \subs{x}{v}} + \Gam_{u \subs{x}{v}} = (\Gam_w + \Gam^1_v) + (\Gam_u + \Gam^2_v) = (\Gam_w + \Gam_u) + (\Gam^1_v + \Gam^2_v) = \Gam_t + \Gam_v$, $b = 1 + b' + b'' = 1 + b_w + b^1_v + b_u + b^2_v = (1 + b_w + b_u) + (b^1_v + b^2_v) = b_t + b_v$, $m = m' + m'' = m_w + m^1_v + m_u + m^2_v = (m_w + m_u) + (m^1_v + m^2_v) = m_t + m_v$, and $d = d' + d'' = d_w + d^1_v + d_u + d^2_v = (d_w + d_u) + (d^1_v + d^2_v) = d_t + d_v$.
            \item Case $\Phi_{t \subs{x}{v}}$ ends with (\ruleAppPOne) or (\ruleAppPTwo). These cases are very similar to the case where $\Phi_{t \subs{x}{v}}$ ends with (\ruleApp).
        \end{itemize}
        \item Let $t = \get{l}{y}{u}$. Then $t \subs{x}{v} = \get{l}{y}{u \subs{x}{v}}$ and $\Phi_{t \subs{x}{v}}$ must be of the following form:
        \[ \begin{prooftree}
            \hypo{\Phi_{u \subs{x}{v}} \tr \seqi{\Gam_{u \subs{x}{v}}; y : \M'}{u \subs{x}{v}}{\comptype{\stype}{\ctype}}{(b,m',d)}}
            \infer1[(\ruleGet)]{\seqi{\Gam_{u \subs{x}{v}}}{\get{l}{y}{u \subs{x}{v}}}{\comptype{\conj{(l : \M')} \splus \stype}{\ctype}}{(b,1+m',d)}}
        \end{prooftree} \]
        where $\Gam_{t \subs{x}{v}} = \Gam_{u \subs{x}{v}}$ and $m = 1+m'$. By the \ih, we have $\Phi_u \tr \seqi{\Gam_u; y : \M'; x : \M}{u}{\comptype{\stype}{\ctype}}{(b_u,m_u,d_u)}$ and $\Phi_v \tr \seqi{\Gam_v}{v}{\M}{(b_v,m_v,d_v)}$, such that $\Gam_{u \subs{x}{v}} = \Gam_u + \Gam_v$, $b = b_u + b_v$, $m' = m_u + m_v$, and $d = d_u + d_v$. So we can build $\Phi_{\get{l}{y}{u}}$ as follows:
        \[ \begin{prooftree}
            \hypo{\Phi_u \tr \seqi{\Gam_u; y : \M'; x : \M}{u}{\comptype{\stype}{\ctype}}{(b_u,m_u,d_u)}}
            \infer1[(\ruleGet)]{\seqi{\Gam_{u}; x : \M}{\get{l}{y}{u}}{\comptype{\conj{(l : \M')} \splus \stype}{\ctype}}{(b_u,1+m_u,d_u)}}
        \end{prooftree} \]
        And we can pick $\Phi_t = \Phi_{\get{l}{y}{u}}$, and conclude with $\Gam_{t \subs{x}{v}} = \Gam_{u \subs{x}{v}} = \Gam_u + \Gam_v$, $b = b_u + b_v$, $m = 1 + m' = 1 + m_u + m_v = (1 + m_u) + m_v$, and $d = d_u + d_v$.
        \item Let $t = \set{l}{w}{u}$. Then $t \subs{x}{v} = (\set{l}{w}{u}) \subs{x}{v} = \set{l}{w \subs{x}{v}}{u \subs{x}{v}}$. Assume $\Phi_{w \subs{x}{v}} \tr \seqi{\Gam_{w \subs{x}{v}}}{w \subs{x}{v}}{\M}{(b',m',d')}$ and $\Phi_{u \subs{x}{v}} \tr \seqi{\Gam_{u \subs{x}{v}}}{u \subs{x}{v}}{\comptype{\conj{(l : \M)}; \stype}{\ctype}}{(b'',m'',d'')}$. $\Phi_{t \subs{x}{v}}$ must be of the following form:
        \[ \begin{prooftree}
            \hypo{\Phi_{w \subs{x}{v}}}
            \hypo{\Phi_{t \subs{x}{v}}}
            \infer2[(\ruleSet)]{\seqi{\Gam_{w \subs{x}{v}} + \Gam_{u \subs{x}{v}}}{\set{l}{w \subs{x}{v}}{u \subs{x}{v}}}{\comptype{\stype}{\ctype}}{(b'+b'',1+m'+m'',d'+d'')}}
        \end{prooftree} \]
        where $\Gam_{t \subs{x}{v}} = \Gam_{w \subs{x}{v}} + \Gam_{u \subs{x}{v}}$, $b = b'+ b''$, $m = 1 + m'+m''$, and $d = d' + d''$. By the \ih over $\Phi_{w \subs{x}{v}}$, we have $\Phi_w \tr \seqi{\Gam_w; x : \M_1}{w}{\M}{(b_w,m_w,d_w)}$ and $\Phi^1_v \tr \seqi{\Gam^1_v}{v}{\M_1}{(b^1_v,m^1_v,d^1_v)}$, such that $\Gam_{w \subs{x}{v}} = \Gam_w + \Gam^1_v$, $b' = b_w + b^1_v$, $m'= m_w + m^1_v$, and $d' = d_w+d^1_v$. And by the \ih over $\Phi_{u \subs{x}{v}}$, we have $\Phi_u \tr \seqi{\Gam_u; x : \M_2}{u}{\comptype{\conj{(l : \M)}; \stype}{\ctype}}{(b_u,m_u,d_u)}$ and $\Phi^2_v \tr \seqi{\Gam^2_v}{v}{\M_2}{(b^2_v,m^2_v,d^2_v)}$, such that $\Gam_{u \subs{x}{v}} = \Gam_u + \Gam^2_v$, $b'' = b_u + b^2_v$, $m'' = m_u + m^2_v$, and $d'' = d_u + d^2_v$. By~\cref{lem:comp-merge-values}, we can take $\Phi_v \tr \seqi{\Gam^1_v + \Gam^2_v}{v}{\M_1 \sqcup \M_2}{(b^1_v + b^2_v, m^1_v+m^2_v, d^1_v + d^2_v)}$, such that $\Gam_v = \Gam^1_v + \Gam^2_v$, $b_v = b^1_v + b^2_v$, $m_v = m^1_v + m^2_v$, and $d_v = d^1_v + d^2_v$. And we can build $\Phi_{\set{l}{w}{u}}$ as follows:
        \[ \begin{prooftree}
            \hypo{\Phi_w \tr \seqi{\Gam_w; x : \M_1}{w}{\M}{(b_w,m_w,d_w)}}
            \hypo{\Phi_u \tr \seqi{\Gam_u; x : \M_2}{u}{\comptype{\conj{(l : \M)}; \stype}{\ctype}}{(b_u,m_u,d_u)}}
            \infer2[(\ruleSet)]{\seqi{(\Gam_w + \Gam_u); x : \M_1 \sqcup \M_2}{\set{l}{w}{u}}{\comptype{\stype}{\ctype}}{(b_w+b_u, 1+m_w+m_u,d_w+d_u)}}
        \end{prooftree} \]
        such that $\Gam_t = \Gam_w + \Gam_u$, $b_t = b_w + b_u$, $m_t = 1 + m_w + m_u$, and $d_t = d_w + d_u$. So we can pick $\Phi_t = \Gam_{\set{l}{w}{u}}$, and conclude with $\Gam_{t \subs{x}{v}} = \Gam_{w \subs{x}{v}} + \Gam_{u \subs{x}{v}} = (\Gam_w + \Gam^1_u) + (\Gam_u + \Gam^2_v) = (\Gam_w + \Gam_u) + (\Gam^1_v + \Gam^2_v) = \Gam_t + \Gam_v$, $b = b' + b'' = (b_w + b^1_v) + (b_u + b^2_v) = (b_w + b_u) + (b^1_v + b^2_v) = b_t + b_v$, $m = 1+ m' + m'' = 1+ (m_w + m^1_v) + (m_u + m^2_v) = (1 + m_w + m_u) + (m^1_v + m^2_v) = m_t + m_v$, and $d = d' + d'' = (d_w + d^1_v) + (d_u + d^2_v) = (d_w + d_u) + (d^1_v + d^2_v) = d_t + d_v$.
    \end{itemize}
%\end{proof}

    \end{enumerate}
\end{proof}}

\lemexactredexp*

\maybehide{\begin{proof} \mbox{}
    \begin{enumerate} 
        \item %\begin{proof}
  We show a stronger statement of the form:

  Let $(t,s) \red[\gname] (u,q)$. If $\Phi \tr \seqi{\Gam}{(t,s)}{\ctype}{(b,m,d)}$, $\Gam$ is tight,  and ($\ctype$ is tight or $\neg \isvalue{t}$), then $\Phi' \tr \seqi{\Gam}{(u,q)}{\ctype}{(b',m',d)}$, where $\gname =\beta$ implies $b' = b - 1$ and $m' = m$, while  $\gname \in \{\getname, \setname\}$ implies $b'=b$ and $m' = m - 1$.

  We proceed by induction on $(t,s) \ra (u,q)$: 
  \begin{itemize}
    \item Case $(t,s) = ((\lam x.p) v,s) \redbeta (p \subs{x}{v}, s) = (u,q)$. Let $\Phi_{(\lam x.p) v}$ be the sub-derivation for $(\lam x.p) v$ in $\Phi$. Assume that $\Phi_{(\lam x.p)v}$ ends with rule (\ruleAppPTwo). Then $v$ must be assigned type $\comptype{\stype}{\tneutral \tim \stype}$, which is not possible by~\cref{lem:comp-values-not-neutral}. Let $\Phi_0$ be the following derivation:
    \[ \begin{prooftree}
      \hypo{\Phi_p \tr \seqi{\Gam_{\lam x.p}; x : \M}{p}{\comptype{\stype}{\ctype}}{(b_p,m_p,d_p)}}
        \infer1[(\ruleLam)]{\seqi{\Gam_{\lam x.p}}{\lam x.p}{\M \ta (\comptype{\stype}{\ctype})}{(b_p,m_p,d_p)}}
        \hypo{\Phi_v \tr \seqi{\Gam_v}{v}{\M}{(b_v,m_v,d_v)}}
        \infer1[(\ruleLift)]{\seqi{\Gam_v}{v}{\tcomptype{\stype}{\M}{\stype}}{(b_v,m_v,d_v)}}
        \infer2[(\ruleApp)]{\seqi{\Gam_{\lam x.p}+\Gam_v}{(\lam x.p)v}{\comptype{\stype}{\ctype}}{(1+b_v+b_p,m_v+m_p,d_v+d_p)}}
    \end{prooftree} \]
    $\Phi_{(\lam x.p) v}$ must end with rule ($\ruleApp$) and $\Phi$ must be of the following form:
    \[ \begin{prooftree}
        \hypo{\Phi_0}
        \hypo{\Phi_s \tr \seqi{\Del}{s}{\stype}{(b_s,m_s,d_s)}}
        \infer2[(\ruleConf)]{\seqi{\Gam_{\lam x.p}+ \Gam_v + \Del}{((\lam x.p)v, s)}{\ctype}{(1+b_v+b_p+b_s,m_v+m_p+m_s,d_v+d_p+d_s)}}
    \end{prooftree} \]
    where  $\Gam = \Gam_{\lam x.p}+ \Gam_v + \Del$,  $b = 1+ b_v + b_p + b_s$, $m = m_v + m_p + m_s$, and $d = d_v + d_p + d_s$. By~\cref{lem:comp-subs-antisubs}.\ref{lem:comp-subs}, there exists $\Phi_{p \subs{x}{v}} \tr \seqi{\Gam_{\lam x.p} +\Gam_v}{p \subs{x}{v}}{\comptype{\stype}{\ctype}}{(b_v+b_p,m_v+m_p,d_v+d_p)}$, therefore we can build $\Phi_{(p\subs{x}{v},s)}$ as follows:
    \[ \begin{prooftree}
        \hypo{\Phi_{p \subs{x}{v}} \tr \seqi{\Gam_{\lam x.p}+\Gam_v}{p \subs{x}{v}}{\comptype{\stype}{\ctype}}{(b_v+b_p,m_v+m_p,d_v+d_p)}}
        \hypo{\Phi_s \tr \seqi{\Del}{s}{\stype}{(b_s,m_s,d_s)}}
        \infer2[(\ruleConf)]{\seqi{\Gam_{\lam x.p} + \Gam_v + \Del}{(p \subs{x}{v}, s)}{\ctype}{(b_v+b_p+b_s,m_v+m_p+m_s,d_v+d_p+d_s)}}
    \end{prooftree} \]
    We can finally conclude since the first counter is equal to $b-1$, while the second and third remain the same.
  \item Case $(t,s) = (vp,s) \ra (vp',q) = (u,q)$, such that $(p,s) \ra (p',q)$. Then we have three cases for the type derivation $\Phi_p$ of $p$ inside $\Phi$: 
    \begin{itemize}
      \item Case $\Phi_p$ ends with ($\ruleApp$). Let $\Phi_0$ be the following derivation:
      \[ \begin{prooftree}
        \hypo{\Phi_v \tr \seqi{\Gam_v}{v}{\M \ta (\comptype{\stype'}{\ctype})}{(b_v,m_v,d_v)}}
            \hypo{\Phi_p \tr \seqi{\Gam_p}{p}{\tcomptype{\stype}{\M}{\stype'}}{(b_p,m_p,d_p)}}
            \infer2[(\ruleApp)]{\seqi{\Gam_v + \Gam_p}{vp}{\comptype{\stype}{\ctype}}{(1+b_v+b_p,m_v+m_p,d_v+d_p)}}
      \end{prooftree} \]
      $\Phi$ must be of the following form:
        \[ \begin{prooftree}
            \hypo{\Phi_0}
            \hypo{\Phi_s \tr \seqi{\Del}{s}{\stype}{(b_s,m_s,d_s)}}
            \infer2[(\ruleConf)]{\seqi{\Gam_v + \Gam_p + \Del}{(vp, s)}{\kappa}{(1+b_v+b_p+b_s,m_v+m_p+m_s,d_v+d_p+d_s)}}
        \end{prooftree} \]
        where $\Gam = \Gam_v + \Gam_p + \Del$ is tight, $b = 1+b_v+b_p+b_s$, $m = m_v+m_p+m_s$, and $s = d_v+d_p+d_s$. Therefore, we can build the following derivation for $(p,s)$:
        \[ \begin{prooftree}
            \hypo{\Phi_p \tr \seqi{\Gam_p}{p}{\tcomptype{\stype}{\M}{\stype'}}{(b_p,m_p,d_p)}}
            \hypo{\Phi_s \tr \seqi{\Del}{s}{\stype}{(b_s,m_s,d_s)}}
            \infer2[(\ruleConf)]{\seqi{\Gam_p + \Del}{(p,s)}{\conftype{\M}{\stype'}}{(b_p+b_s,m_p+m_s,d_p+d_s)}}
        \end{prooftree} \]
        Since $\Gam$ is tight, then $\Gam_p + \Del$ is tight. Moreover, $(p,s) \ra (p',q)$ implies that $\neg \isvalue{p}$. Then we can apply the \ih, and thus there exists a derivation for $(p',q)$ that must be of the following form:
        \[ \begin{prooftree}
            \hypo{\Phi_{p'} \tr \seqi{\Gam_{p'}}{p'}{\tcomptype{\stype''}{\M}{\stype'}}{(b_{p'},m_{p'},d_{p'})}}
            \hypo{\Phi_q \tr \seqi{\Del_q}{q}{\stype''}{(b_q,m_q,d_q)}}
            \infer2[(\ruleConf)]{\seqi{\Gam_{p'} + \Del_q}{(p',q)}{\conftype{\M}{\stype'}}{(b_{p'}+b_q,m_{p'}+m_q,d_{p'}+d_q)}}
        \end{prooftree} \]
        where $\Gam_{p'} + \Del_q = \Gam_p + \Del$ is tight,  and the counters are related properly. Let $\Phi_0$ be the following derivation:
        \[ \begin{prooftree}
          \hypo{\Phi_v \tr \seqi{\Gam_v}{v}{\M \ta \comptype{\stype'}{\ctype'}}{(b_v,m_v,d_v)}}
            \hypo{\Phi_{p'} \tr \seqi{\Gam_{p'}}{p'}{\tcomptype{\stype''}{\M}{\stype'}}{(b_{p'},m_{p'},d_{p'})}}
            \infer2[(\ruleApp)]{\seqi{\Gam_v+\Gam_{p'}}{vp'}{\comptype{\stype''}{\ctype'}}{(1+b_v+b_{p'},m_v+m_{p'},d_v+d_{p'})}}
        \end{prooftree} \]
        We can build $\Phi_{(u,q)}$ as follows:
        \[ \begin{prooftree}
            \hypo{\Phi_0}
            \hypo{\Phi_q \tr \seqi{\Del_q}{q}{\stype''}{(b_q,m_q,d_q)}}
            \infer2[(\ruleConf)]{\seqi{\Gam_v + \Gam_{p'} + \Del_q}{(vp', q)}{\ctype'}{(1+b_v+b_{p'}+b_q,m_v+m_{p'}+m_q,d_v+d_{p'}+d_q)}}
        \end{prooftree} \]
        where $\Gam_{p'} + \Gam_v + \Del_q = \Gam_v + \Gam_p + \Del = \Gam$, $b' = 1+b_v+b_{p'}+b_q$, $m' = m_v+m_{p'}+m_q$, and $d' = d_v+d_{p'}+d_q$. We can conclude since the counters are related properly according to the \ih.  
        \item Case $\Phi_p$ ends with (\ruleAppPOne) or (\ruleAppPTwo). These two cases are very similar to the previous case.
      \end{itemize}
      \item Case $(t,s) = (\get{l}{x}{p},s) \ra (p \subs{x}{v},s) = (u,q)$, where $s \equivstate \upd{l}{v}{s'}$. Let $\Phi_0$ be the following derivation:
      \[ \begin{prooftree}
        \hypo{\Phi_{p} \tr \seqi{\Gam_{p}}{p}{\comptype{\stype}{\ctype}}{(b_p,m_p,d_p)}}
        \infer1[(\ruleGet)]{\seqi{\Gam_{p} \sm x}{\get{l}{x}{p}}{\comptype{\conj{(l : \Gam_{p}(x))} \splus \stype}{\ctype}}{(b_p,1+m_p,d_p)}}
      \end{prooftree} \]
      $\Phi$ must be of the following form:
        \[ \begin{prooftree}
        \hypo{\Phi_0}
          \hypo{\Phi_{s} \tr \seqi{\Del}{s}{\conj{(l : \Gam_{p}(x))} \splus  \stype}{(b_s,m_s,d_s)}}
          \infer2[(\ruleConf)]{\seqi{(\Gam_{p} \sm x) + \Del}{(\get{l}{x}{p}, s)}{\kap}{(b_p+b_s,1+m_p+m_s,d_p+d_s)}}
        \end{prooftree} \] 
        where $\Gam = (\Gam_{p} \sm x) + \Del$ is tight, $b = b_p + b_s$, $m = 1+ m_p + m_s$, and  $d = d_p + d_s$. Since $\Phi_{s} \tr \seqi{\Del}{s}{\conj{(l : \Gam_{p}(x))} \splus \stype}{(b_s,m_s,d_s)}$, then~\cref{lem:split-values-stores}.\ref{lem:split-state} gives $s \equivstate \upd{l}{v_0}{s'_0}$, but we necessarily have $v_0 = v$ and $s'_0 = s'$. Moreover, the lemma also gives $\Phi_v \tr \seqi{\Del_v}{v}{\Gam_{p}(x) \sqcup \stype(l)}{(b_v,m_v,d_v)}$ and $\Phi_{s'} \tr \seqi{\Del_{s'}}{s'}{\stype'}{(b_{s'},m_{s'},d_{s'})}$, where $\conj{(l : \Gam_{p}(x))} \splus \stype = \conj{(l : \Gam_{p}(x) \sqcup \stype(l))};\stype'$, $\Del = \Del_v + \Del_{s'}$, $b_s=b_v + b_{s'}$, $m_s=m_v + m_{s'}$, and $d_s=d_v + d_{s'}$. Thus, by~\cref{lem:split-values-stores}.\ref{lem:com-split-values} there exist $\Phi^1_v \tr \seqi{\Del^1_v}{v}{\Gam_{p}(x)}{(b^1_v,m^1_v,d^1_v)}$ and $\Phi^2_v \tr \seqi{\Del^2_v}{v}{\stype(l)}{(b^2_v,m^2_v,d^2_v)}$, such that $\Del_v = \Del^1_v + \Del^2_v$, $b_v = b^1_v+b^2_v$, $m_v = m^1_v+m^2_v$, and $d_v = d^1_v+d^2_v$. From $\Phi_{p} \tr \seqi{\Gam_{p}}{p}{\comptype{\stype}{\ctype}}{(b_p,m_p,d_p)}$ and $\Phi^1_v \tr \seqi{\Del^1_v}{v}{\Gam_{p}(x)}{(b^1_v,m^1_v,d^1_v)}$, we obtain $\Phi_{p \subs{x}{v}} \tr \seqi{(\Gam_{p} \sm x) +\Del^1_v}{p\subs{x}{v}}{\comptype{\stype}{\ctype}}{(b_p+b^1_v,m_p+m^1_v,d_p+d^1_v)}$, by~\cref{lem:comp-subs-antisubs}.\ref{lem:comp-subs}. We now construct an alternative type derivation for $s$ of the form:
        \[ \begin{prooftree}
            \hypo{\Phi^2_v \tr  \seqi{\Del^2_v}{v}{\stype(l)}{(b^2_v,m^2_v,d^2_v)}}
            \hypo{\Phi_{s'} \tr \seqi{\Del_{s'}}{s'}{\stype'}{(b_{s'},m_{s'},d_{s'})}}
            \infer2[(\ruleUpd)]{\seqi{\Del^2_v+ \Del_{s'}}{\upd{l}{v}{s'}}{\conj{(l:\stype(l))};\stype'}{(b^2_v+b_{s'},m^2_v+m_{s'},d^2_v+d_{s'})}}
        \end{prooftree} \]
        Let $q = s= \upd{l}{v}{s'}$ and let $\Phi_q$ be this new derivation above. Notice also that $\stype = \conj{(l:\stype(l))}; \stype'$. Then we can construct $\Phi'$ as follows:
        \[ \begin{prooftree}
            \hypo{\Phi_{p\subs{x}{v}}}
            \hypo{\Phi_q}
            \infer2[(\ruleConf)]{\seqi{(\Gam_{p} \sm x) + \Del^1_v + \Del^2_v + \Del_{s'}}{(p \subs{x}{v}, s)}{\kap}{(b,m,d)}}
        \end{prooftree} \]
        Notice that the type environment of the conclusion is $(\Gam_{p} \sm x) + \Del^1_v + \Del^2_v + \Del_{s'} = (\Gam_{p} \sm x) + \Del_v + \Del_{s'} = (\Gam_{p} \sm x) + \Del = \Gam $, and the counters are as expected.
        \item Case $(t,s) = (\set{l}{v}{p},s) \ra (p, \upd{l}{v}{s}) = (u,q)$. Let $\Phi_0$ be the following derivation:
        \[ \begin{prooftree}
          \hypo{\Phi_{v} \tr \seqi{\Gam_v}{v}{\M}{(b_v,m_v,d_v)}}
            \hypo{\Phi_{p} \tr \seqi{\Gam_{p}}{p}{\comptype{\conj{(l : \M)}; \stype}{\ctype}}{(b_p,m_p,d_p)}}
            \infer2[(\ruleSet)]{\seqi{\Gam_v + \Gam_{p}}{\set{l}{v}{p}}{\comptype{\stype}{\ctype}}{(b_v+b_p,1+m_v+m_p,s_v+s_p)}}
        \end{prooftree} \]
        $\Phi$ must be of the following form:
        \[ \begin{prooftree}
          \hypo{\Phi_0}    
            \hypo{\Phi_{s} \tr \seqi{\Gam_{s}}{s}{\stype}{(b_s,m_s,d_s)}}
            \infer2[(\ruleConf)]{\seqi{(\Gam_v + \Gam_{p}) + \Gam_{s}}{(\set{l}{v}{p}, s)}{\kap}{(b_v+b_p+b_s,1+m_v+m_p+m_s,d_v+d_p+d_s)}}
        \end{prooftree} \]
        where  $\Gam = (\Gam_v + \Gam_{p}) + \Gam_{s}$ is tight, $b = b_v+b_p+b_s$, $m=1+m_v+m_p+m_s$ and $d=d_v+d_p+d_s$. Therefore, we can build $\Phi_{\upd{l}{v}{s}}$ as follows:
        \[ \begin{prooftree}
          \hypo{\Phi_{v} \tr \seqi{\Gam_v}{v}{\M}{(b_v,m_v,d_v)}}
          \hypo{\Phi_{s} \tr \seqi{\Gam_{s}}{s}{\stype}{(b_s,m_s,d_s)}}
          \infer2[(\ruleUpd)]{\seqi{\Gam_v + \Gam_{s}}{\upd{l}{v}{s}}{\conj{(l : \M)}; \stype}{(b_v+b_s,m_v+m_s,d_v+d_s)}}
        \end{prooftree} \]
        Assume And we can build $\Phi'$ as follows:
        \[ \begin{prooftree}
            \hypo{\Phi_{p} \tr \seqi{\Gam_{p}}{p}{\comptype{\conj{(l : \M)}; \stype}{\ctype}}{(b_p,m_p,d_p)}}
            \hypo{\Phi_{\upd{l}{v}{s}}}
            \infer2[(\ruleConf)]{\seqi{\Gam_{p} + (\Gam_v + \Gam_{s})}{(p, \upd{l}{v}{s})}{\kap}{(b_v+b_v+b_s,m_v+m_v+m_s,d_v+d_v+d_s)}}
        \end{prooftree} \]
        Notice that the type environment of the conclusion is $\Gam_{p} + (\Gam_v + \Gam_{s}) = \Gam$, and the counters are as expected.
    \end{itemize}
%\end{proof}

        \item %\begin{proof}
    We show a stronger statement of the form:

    Let $(t,s) \red[\gname] (u,q)$. If  $\Phi' \tr \seqi{\Gam}{(u,q)}{\ctype}{(b',m',d)}$, $\Gam$ is tight, and ($\ctype$ is tight or $\neg \isvalue{t}$), then $\Phi \tr \seqi{\Gam}{(t,s)}{\ctype}{(b,m,d)}$, where $\gname =\beta$ implies $b' = b - 1$ and $m' = m$, while $\gname \in \{\getname, \setname\}$ implies $b'=b$ and $m' = m - 1$.

    We proceed by induction on $(t, s) \red (u,q)$: 
    \begin{itemize}
        \item Case $(t,s) = ((\lam x.p) v,s) \redbeta (p \subs{x}{v}, s) = (u,q)$. Then $\Phi'$ must be of the following form:
        \[ \begin{prooftree}
            \hypo{\Phi_{p \subs{x}{v}} \tr \seqi{\Gam_{p \subs{x}{v}}}{p \subs{x}{v}}{\comptype{\stype}{\ctype}}{(b'',m'',d'')}}
            \hypo{\Phi_s \tr \seqi{\Gam_s}{s}{\stype}{(b_s,m_s,d_s)}}
            \infer2[(\ruleConf)]{\seqi{\Gam_{p \subs{x}{v}} + \Gam_s}{(p \subs{x}{v}, s)}{\ctype}{(b''+b_s,m''+m_s,d''+d_s)}}
        \end{prooftree} \]
        such that $\Gam = \Gam_{p \subs{x}{v}} + \Gam_s$, $b' = b''+b_s$, $m' = m''+m_s$, and $d' = d''+d_s$. By~\cref{lem:comp-subs-antisubs}.\ref{lem:comp-antisubs}, there exist $\Phi_p \tr \seqi{\Gam_p; x : \M}{p}{\comptype{\stype}{\ctype}}{(b_p,m_p,d_p)}$ and $\Phi_{v} \tr \seqi{\Gam_v}{v}{\M}{(b_v,m_v,d_v)}$, such that $\Gam_{p \subs{x}{v}} = \Gam_p + \Gam_v$, $b'' = b_p+b_v$, $m'' = m_p+m_v$, and $d'' = d_p + d_v$. We can build $\Phi$ as follows:
        \[ \begin{prooftree}
            \hypo{\Phi_p \tr \seqi{\Gam_p; x : \M}{p}{\comptype{\stype}{\ctype}}{(b_p,m_p,d_p)}}
            \infer1[(\ruleLam)]{\seqi{\Gam_p}{\lam x.p}{\M \ta (\comptype{\stype}{\ctype})}{(b_p,m_p,d_p)}}
            \hypo{\Phi_{v} \tr \seqi{\Gam_v}{v}{\M}{(b_v,m_v,d_v)}}
            \infer1[(\ruleLift)]{\seqi{\Gam_v}{v}{\tcomptype{\stype}{\M}{\stype}}{(b_v,m_v,d_v)}}
            \infer2[(\ruleApp)]{\seqi{\Gam_p + \Gam_v}{(\lam x.p)v}{\comptype{\stype}{\ctype}}{(1+b_p+b_v,m_p+m_v,d_p+d_v)}}
            \hypo{\Phi_s}
            \infer2[(\ruleConf)]{\seqi{(\Gam_p + \Gam_v) + \Gam_s}{((\lam x.t')v, s)}{\ctype}{(1+b_p+b_v+b_s,m_p+m_v+m_s,d_p+d_v+d_s)}}
        \end{prooftree} \]
        such that $b = 1+b_p+b_v+b_s$, $m = m_p+m_v+m_s$, and $d = d_p+d_v+d_s$. And we can conclude with $\Gam = \Gam_{p \subs{x}{v}} + \Gam_s = (\Gam_p + \Gam_v) + \Gam_s$, $b' = b'' + b_s = b_p + b_v + b_s = (1 + b_p + b_v + b_s) - 1 = b - 1$, $m' = m'' + m_s = (m_p + m_v) + m_s = m$, and $d' = d'' + d_s = (d_p + d_v) + d_s = d$.
        \item Case $(t,s) = (vp,s) \ra (vp',q) = (u,q)$, such that $(p,s) \ra (p',q)$. Then we have three cases for the type derivation $\Phi_{p'}$ of $p'$ inside $\Phi'$:
        \begin{itemize}
            \item Case $\Phi_{vp'}$ ends with ($\ruleApp$). Let $\Phi_0$ be the following derivation:
            \[ \begin{prooftree}
                \hypo{\Phi_{v} \tr \seqi{\Gam_v}{v}{\M \ta \comptype{\stype'}{\ctype}}{(b_v,m_v,d_v)}}
                    \hypo{\Phi_{p'} \tr \seqi{\Gam_{p'}}{p'}{\tcomptype{\stype}{\M}{\stype'}}{(b'',m'',d'')}}
                    \infer2[(\ruleApp)]{\seqi{\Gam_v + \Gam_{p'}}{v p'}{\comptype{\stype}{\ctype}}{(1+b_v+b'', m_v+m'', d_v+d'')}}
            \end{prooftree} \]
            $\Phi'$ must be of the following form: 
            \[ \begin{prooftree}
                \hypo{\Phi_0}
                \hypo{\Phi_q \tr \seqi{\Gam_q}{q}{\stype}{(b_q,m_q,d_q)}}
                \infer2[(\ruleConf)]{\seqi{(\Gam_v + \Gam_{p'}) + \Gam_q}{(v p', q)}{\ctype}{(1+b_v+b''+b_q,m_v+m''+m_q,d_v+d''+d_q)}}
            \end{prooftree} \]
            such that $\Gam = (\Gam_v + \Gam_{p'}) + \Gam_q$ tight, $b' = 1+b_v+b''+b_q$, $m' = m_v+m''+m_q$, and $d' = d_v+d''+d_q$. So we can build $\Phi_{(p',q)}$ as follows:
            \[ \begin{prooftree}
                \hypo{\Phi_{p'} \tr \seqi{\Gam_{p'}}{p'}{\tcomptype{\stype}{\M}{\stype'}}{(b'',m'',d'')}}
                \hypo{\Phi_q \tr \seqi{\Gam_q}{q}{\stype}{(b_q,m_q,d_q)}}
                \infer2[(\ruleConf)]{\seqi{\Gam_{p'} + \Gam_q}{(p', q)}{\conftype{\M}{\stype'}}{(b''+b_q,m''+m_q,d''+d_q)}}
            \end{prooftree} \]
            Since $\Gam$ is tight, then $\Gam_{p'} + \Gam_q$ is tight. Moreover, $(p, s) \red (p',q)$ implies $\neg\isvalue{p}$. Then we can apply the \ih, and thus there exists a derivation for $(p,s)$ that must be of the following form:
            \[ \begin{prooftree}
                \hypo{\Phi_p \tr \seqi{\Gam_p}{p}{\tcomptype{\stype''}{\M}{\stype'}}{(b_p,m_p,d_p)}}
                \hypo{\Phi_s \tr \seqi{\Gam_s}{s}{\stype''}{(b_s,m_s,d_s)}}
                \infer2[(\ruleConf)]{\seqi{\Gam_p + \Gam_s}{(p, s)}{\conftype{\M}{\stype'}}{(b_p+b_s,m_p+m_s,d_p+d_s)}}
            \end{prooftree} \]
            where $\Gam_p + \Gam_s = \Gam_{p'} + \Gam_q$ is tight, and either (1) $b''+b_q = b_p+b_s-1$, $m''+m_q=m_p+m_s$, and $d''+d_q = d_p+d_s$, or (2) $b''+b_q = b_p+b_s$, $m''+m_q=m_p+m_s-1$, and $d''+d_q = d_p+d_s$. So, we can build $\Phi$ as follows:
            \[ \begin{prooftree}
                \hypo{\Phi_{v} \tr \seqi{\Gam_v}{v}{\M \ta (\comptype{\stype'}{\ctype})}{(b_v,m_v,d_v)}}
                \hypo{\Phi_p \tr \seqi{\Gam_p}{p}{\tcomptype{\stype''}{\M}{\stype'}}{(b_p,m_p,d_p)}}
                \infer2[(\ruleApp)]{\seqi{\Gam_v + \Gam_p}{vp}{\comptype{\stype''}{\ctype}}{(1+b_v+b_p,m_v+m_p,d_v+d_p)}}
                \hypo{\Phi_s}
                \infer2[(\ruleConf)]{\seqi{(\Gam_v + \Gam_p) + \Gam_s}{(vp, s)}{\kappa}{(1 + b_v + b_p+b_s,m_v+m_p+m_s,d_v+d_p+d_s)}}
            \end{prooftree} \]
            where $\Gam_v + \Gam_p + \Gam_s = \Gam_v + \Gam_{p'} + \Gam_q = \Gam$, $b = 1+b_v+b_p+b_s$, $m = m_v+m_p+m_s$, and $d = d_v+d_p+d_s$. We can conclude since:
            \begin{itemize}
                \item Case (1): $b' = 1 + b_v + b'' + b_q = 1 + b_v + b_p + b_s - 1 = b -1$, and the other counters are easy to check;
                \item Case (2): $m' = m_v + m'' + m_q = m_v + m_p + m_s - 1 = m - 1$, and the other counters are easy to check.
            \end{itemize}
            \item Case $\Phi_{vp'}$ ends with (\ruleAppPOne) or (\ruleAppPTwo). These two cases are very similar to the previous case.
        \end{itemize}
        \item Case $(t,s) = (\get{l}{x}{p},s) \ra (p \subs{x}{v},s) = (u,q)$, such that $s \equivstate \upd{l}{v}{s'}$. Let $\Phi_0$ be the following derivation:
        \[ \begin{prooftree}
            \hypo{\Phi^2_v \tr \seqi{\Gam^2_v}{v}{\M_2}{(b^2_v,m^2_v,d^2_v)}}
            \hypo{\Phi_{s'} \tr \seqi{\Gam_{s'}}{s'}{\stype}{(b_{s'},m_{s'},d_{s'})}}
            \infer2[(\ruleUpd)]{\seqi{\Gam^2_v + \Gam_{s'}}{\upd{l}{v}{s'}}{\conj{(l : \M_2)}; \stype}{(b^2_v+b_{s'},m^2_v+m_{s'},d^2_v+d_{s'})}}
        \end{prooftree} \]
        Then $\Phi'$ must be of the following form:
        \[ \begin{prooftree}
            \hypo{\Phi_{p \subs{x}{v}} \tr \seqi{\Gam_{p \subs{x}{v}}}{p \subs{x}{v}}{\comptype{\conj{(l : \M)}; \stype}{\ctype}}{(b'',m'',d'')}}
            \hypo{\Phi_0}
            \infer2[(\ruleConf)]{\seqi{\Gam_{p \subs{x}{v}} + (\Gam^2_v + \Gam_{s'})}{(p \subs{x}{v}, \upd{l}{v}{s'})}{\ctype}{(b''+b^2_v+b_{s'},m''+m^2_v+m_{s'},d''+d^2_v+d_{s'})}}
        \end{prooftree} \]
        such that $\Gam = \Gam_{p \subs{x}{v}} + (\Gam^2_v + \Gam_{s'})$, $b' = b'' + b^2_v + b_{s'}$, $m' = m'' + b^2_v + b_{s'}$, and $d' = d'' +d^2_v+d_{s'}$. By~\cref{lem:comp-subs-antisubs}.\ref{lem:comp-antisubs}, there exist $\Phi_p \tr \seqi{\Gam_p; x : \M_1}{p}{\comptype{\conj{(l : \M_2)}; \stype}{\ctype}}{(b_p,m_p,d_p)}$ and $\Phi^1_v \tr \seqi{\Gam^1_v}{v}{\M_1}{(b^1_v,m^1_v,d^1_v)}$, such that $\Gam_{p \subs{x}{v}} = \Gam_p + \Gam^1_v$, $b'' = b_p + b^1_v$, $m'' = m_p + m^1_v$, and $d'' = d_p + d^1_v$. Therefore, we can build $\Phi_{\get{l}{x}{p}}$ as follows:
        \[ \begin{prooftree}
            \hypo{\Phi_p \tr \seqi{\Gam_p; x : \M_1}{p}{\comptype{\conj{(l : \M_2)}; \stype}{\ctype}}{(b_p,m_p,d_p)}}
            \infer1[(\ruleGet)]{\seqi{\Gam_p}{\get{l}{x}{p}}{\comptype{\conj{(l : \M_1 \sqcup \M_2)}; \stype}{\ctype}}{(b_p,1+m_p,d_p)}}
        \end{prooftree} \]
        By~\cref{lem:comp-merge-values}, we have $\Phi_v \tr \seqi{\Gam^1_v + \Gam^2_v}{v}{\M_1 \sqcup \M_2}{(b^1_v+b^2_v,m^1_v+m^2_v,d^1_v+d^2_v)}$. Thus, we can build $\Phi_{\upd{l}{v}{s'}}$ as follows:
        \[ \begin{prooftree}
            \hypo{\Phi_v \tr \seqi{\Gam^1_v + \Gam^2_v}{v}{\M_1 \sqcup \M_2}{(b^1_v+b^2_v,m^1_v+m^2_v,d^1_v+d^2_v)}}
            \hypo{\Phi_{s'} \tr \seqi{\Gam_{s'}}{s'}{\stype}{(b_{s'},m_{s'},d_{s'})}}
            \infer2[(\ruleUpd)]{\seqi{(\Gam^1_v + \Gam^2_v) + \Gam_{s'}}{\upd{l}{v}{s'}}{\conj{(l : \M_1 \sqcup \M_2)}; \stype}{(b^1_v+b^2_v+b_{s'},m^1_v+m^2_v+m_{s'},d^1_v+d^2_v+d_{s'})}}
        \end{prooftree} \]
        Finally, we can build $\Phi$ as follows:
        \[ \begin{prooftree}
            \hypo{\Phi_{\get{l}{x}{p}}}
            \hypo{\Phi_{\upd{l}{v}{s'}}}
            \infer2[(\ruleConf)]{\seqi{\Gam_p + (\Gam^1_v + \Gam^2_v) + \Gam_{s'}}{(\get{l}{x}{p}, \upd{l}{v}{s'})}{\ctype}{(b_p+b^1_v+b^2_v+b_{s'},1+m_p+m^1_v+m^2_v+m_{s'},d_p+d^1_v+d^2_v+d_{s'})}}
        \end{prooftree} \]
        such that $b = b_p+b^1_v+b^2_v+b_{s'}$, $m = 1+m_p+m^1_v+m^2_v+m_{s'}$, and $d = d_p+d^1_v+d^2_v+d_{s'}$. And we can conclude with $\Gam = \Gam_{p \subs{x}{v}} + (\Gam^2_v + \Gam_{s'}) = \Gam_p + \Gam^1_v + \Gam^2_v + \Gam_{s'}$, $b' = b'' + b^2_v + b_{s'} = b_p + b^1_v + b^2_v + b_{s'} = b$, and $m' = m'' + m^2_v + m_{s'} = m_p + m^1_v + m^2_v + m_{s'} = (1 + m_p + m^1_v + m^2_v + m_{s'}) - 1 = m - 1$, $d' = d'' + d^2_v + d_{s'} = d_p + d^1_v + d^2_v + d_{s'} = d$.
        \item Case $(t,s) = (\set{l}{v}{p},s) \ra (p, \upd{l}{v}{s}) = (u,q)$. Let $\Phi_0$ be the following derivation:
        \[ \begin{prooftree}
            \hypo{\Phi_{v} \tr \seqi{\Gam_v}{v}{\M}{(b_v,m_v,d_v)}}
            \hypo{\Phi_{s} \tr \seqi{\Gam_s}{s}{\stype}{(b_s,m_s,d_s)}}
            \infer2[(\ruleUpd)]{\seqi{\Gam_v + \Gam_s}{\upd{l}{v}{s}}{\conj{(l : \M)}; \stype}{(b_v+b_s,m_v+m_s,d_v+d_s)}}
        \end{prooftree} \]
        $\Phi'$ must be of the following form:
        \[ \begin{prooftree}
            \hypo{\Phi_p \tr \seqi{\Gam_p}{p}{\comptype{\conj{(l : \M)}; \stype}{\ctype}}{(b_p,m_p,d_p)}}
            \hypo{\Phi_0}
            \infer2[(\ruleConf)]{\seqi{\Gam_p + (\Gam_v + \Gam_s)}{(p, \upd{l}{v}{s})}{\kap}{(b_p+b_v+b_s,m_p+m_v+m_s,d_p+d_v+d_s)}}
        \end{prooftree} \]
        such that $\Gam = \Gam_p + (\Gam_v + \Gam_s)$, $b' = b_p + b_v + b_s$, $m' = m_p + m_v + m_s$, and $d' = d_p + d_v + d_s$. Therefore, we can build $\Phi$ as follows:
        \[ \begin{prooftree}
            \hypo{\Phi_{v} \tr \seqi{\Gam_v}{v}{\M}{(b_v,m_v,d_v,)}}
            \hypo{\Phi_p \tr \seqi{\Gam_p}{p}{\comptype{\conj{(l : \M)}; \stype}{\ctype}}{(b_p,m_p,d_p)}}
            \infer2[(\ruleSet)]{\seqi{\Gam_v + \Gam_p}{\set{l}{v}{p}}{\comptype{\stype}{\ctype}}{(b_v+b_p,1+m_v+m_p,d_v+d_p)}}
            \hypo{\Phi_s}
            \infer2[(\ruleConf)]{\seqi{(\Gam_v + \Gam_p) + \Gam_s}{(\set{l}{v}{p}, s)}{\ctype}{(b_v+b_p+b_s,1+m_v+m_p+m_s,d_v+d_p+d_s)}}
        \end{prooftree} \]
        Notice that the type environment of the conclusion is $(\Gam_v + \Gam_p) + \Gam_s = \Gam$, and the counters are as expected.
    \end{itemize}
%\end{proof}
    \end{enumerate}
\end{proof}}

\compsoundness*

\maybehide{\begin{proof} \mbox{}
    \begin{enumerate}
        \item %\begin{proof}
    The proof follows by induction over $b+m$:
    \begin{itemize}
        \item Case $b+m = 0$. Then $b=m=0$, therefore $t \in \normal$, by point (1) of~\cref{lem:zero-counters}, and  $d = \size{t}$,  by point (2) of~\cref{lem:zero-counters}. Let $u = t$ and $q=s$, then  we can conclude since $\size{(u,q)} = \size{u} =\size{t} = d$.
        \item Case $b+m > 0$. Then $b>0$ or $m>0$, and in either case $t \not\in \normal$,  by~\cref{lem:zero-nfs}. Note that $(t,s)$ is not final because $t$ is unblocked by~\cref{prop:typed-unblock}. Therefore, by~\cref{prop:normal-iff-final} there exists $(t',s')$ such that $(t,s) \gsred (t',s')$. By~\cref{lem-exact-red-exp}.\ref{lem:subj-comp-red}, there exists $\Phi' \tr \seqi{\Gam}{(t',s')}{\ctype}{(b',m',d)}$, such that $b'+m'=b+m-1$. By the \ih, there exists $(u,q)$, such that $u\in \normal$, $(t',s') \gsrred^{(b',m')} (u,q)$ and $d = \size{(u,q)}$. So we can conclude with $(t,s) \gsred (t',s') \gsrred^{(b',m')} (u,q)$, which means that $(t,s) \drred^{(b,m)} (u,q)$, as expected.
    \end{itemize}
%\end{proof}

        \item %\begin{proof}
    By induction over $b + m$: \begin{itemize}
        \item Case $b + m = 0$. Then $b = m = 0$ and $(t,s) = (u,q)$. We can conclude by~\cref{lem:typestatesnfs}.\ref{lem:typ-states} and~\cref{lem:typestatesnfs}.\ref{lem:comp-typ-nfs}.
        \item Case $b + m > 0$. Then there exists $(t',s')$, such that $(t,s) \ra^{(1,0)} (t',s') \rra^{(b-1,m)} (u,q)$ or $(t,s) \ra^{(0,1)} (t',s') \rra^{(b,m-1)} (u,q)$. By the \ih, there exists $\Phi' \tr \seqi{\Gam}{(t',s')}{\kap}{(b',m',\size{(u,q)})}$ tight, such that $b' + m' = b + m - 1$. By~\cref{lem-exact-red-exp}.\ref{lem:comp-subj-exp}, we have $\Phi \tr \seqi{\Gam}{(t,s)}{\kap}{(b'',m'',\size{(u,q)})}$ tight, such that $b'' + m'' = 1+ b' + m'$. Therefore, $b'' + m'' = b + m$, since the fact that $b'' = b$, and $m'' = m$ can be easily checked by a simple case analysis.
    \end{itemize}
%\end{proof}
    \end{enumerate}
\end{proof}}


\section{Implementation} \label{sec:implementation-appendix}

Our theoretical results regarding discrete representation of continuous signals are based on infinite signals, which may seem impractical to real models which work on finite images. However, the results apply in a setting in which we assume that the continuous signals are periodic, and we finitely sample a single period. These assumptions practically limit our discussion to robustness to circular translations, which is the same setting that was considered in previous works \citep{Zhang2019MakingAgain, Chaman2020TrulyNetworks}.
Next, we explain our implementation for the ``ideal LPF'', which was used in BlurPool and phase 3 of \Cref{algo:af-poly}, and for the ``reconstruction filter'', which was used in $\mathrm{Upsample}$ in step 1 of \Cref{algo:af-poly}.
Both of these filters can be implemented using multiplication in the Fourier domain, working with DFT, which is defined for a finite signal with length $N$ as
\begin{equation} \label{eq:dft}
    X^{D}[k] = \mathrm{DFT} \left( x \left[ n \right] \right) \left[ k \right] = \sum_{n=0}^{N-1} x \left[ n \right] e^{-\frac{j 2 \pi}{N}kn } \,.
\end{equation}
Similarly, the inverse of DFT is defined as:
\begin{equation} \label{eq:dft-inv}
    x \left[ n \right]=\mathrm{IDFT} \left( X^D \left[ k \right] \right) \left[ n \right] = \frac{1}{N} \sum_{k=0}^{N-1} X^{D} \left[ k \right] e^{\frac{j 2 \pi}{N}kn } \,.
\end{equation}
For simplicity, all our derivations are for 1-D signals. Our derivations trivially apply to the 2-D case by applying the filters separately on rows and on columns.%, the following operators are presented for 1-D signals. As for being separable in higher dimensions, the proposed filters can be translated to 2-D by 
%\[
%H_{2D} = H H^{T}
%\]
\paragraph{LPF}
\label{par:lpf}
We used a low-pass filter wherever it was necessary to prevent aliasing due to downsampling, namely in BlurPool layers that replace strided convolutions, and alias-free polynomial activations, before subsampling the polynomial results (\cref{algo:af-poly}). We used an ``ideal filter'', i.e.~a filter that eliminates all the frequencies above the cutoff ratio. Practically, this kind of filter can be implemented using multiplication in DFT domain: 
\begin{equation}
    \mathrm{LPF}_{\mathrm{cutoff},c}  \left( x \left[ n \right] \right) \left[ n \right] =   \mathrm{IDFT} \left(\mathrm{DFT} \left(x \left[ n \right] \right) \left[ k \right] H^{D}_{\mathrm{cutoff},c}\left[ k \right] \right) \left[ n \right]\,,
\end{equation}
% where $X^D$ is the DFT of $x \left[ n\right]$ and $H^{2D}$ is a ``rectangle filter'' defined for spatial dimension $N \times N$  and cutoff ratio $c \in \left[ 0,\ 1\right]$ as 
where $H^{D}_{\mathrm{cutoff, c}}$ is a ``rectangle filter'' defined for spatial dimension $N$  and cutoff ratio $c \in \left[ 0,\ 1\right]$ as 

\begin{equation}
% H^{2D} = H H^{T}, 
H^{D}_{\mathrm{cutoff,c}} \left[ k \right]=\begin{cases}
1 & 0\leq k<\frac{N}{2}c\,,\\
0 & \frac{N}{2}c \leq k \leq N - \frac{N}{2}c\,,\\
1 & N - \frac{N}{2}c < k\leq N-1\,.
\end{cases} \,.
\end{equation}

\paragraph{Downsampling}
As mentioned above, all downsampling operations were performed in an alias-free manner, using low-pass filters before subsampling. 
For subsampling at factor $s$, we used LPF with cutoff ratio $c = \frac{1}{s}$.
Then, we used subsampling with a fixed grid:
% \begin{equation}
%     x_{\mathrm{down}} \left[n,\ m \right] = x_{\mathrm{LPF}}[sn,\ sm]
% \end{equation}
\begin{equation}
    x_{\mathrm{down}} \left[n \right] = \mathrm{LPF}_{1/s} \left( x \left[ n \right] \right) \left[ sn \right]
\end{equation}


\paragraph{Upsampling}

In the proof of Algorithm 1, we assume we use ``ideal upsample'', which can be interpreted as a re-sampling in a higher rate of the continuous signal, which was restored using sinc interpolation:
\begin{equation}
    x_{\mathrm{up}_{I}} \left[ m \right] = \sum_{n} x \left[n \right] \mathrm{sinc} \left( \frac{m - nI}{I} \right)
\end{equation}
In practice, and specifically in a finite signal case, upsampling is performed in two steps:
First, we use zero padding and get the intermediate signal 
\begin{equation}
    x_{z}\left[m \right] = \begin{cases}
        x \left[ \frac{m}{I} \right ] &     m = kI\, , \\
       0                               & \text{otherwise.}
    \end{cases}
\end{equation}
Then, the zero-padded signal is convolved with sinc interpolation kernel. This step is equivalent to multiplication in Fourier domain with a rectangle, similarly to the LPF implementation.
% \begin{equation}
%     \mathrm{Upsample}_{I} \left( x \left[ n,\ m \right] \right) =   X_{z_{I}}^{D} \left[ k,\ l \right] H^{2D}_{\mathrm{upsample}_{I}}
% \end{equation}
% \begin{equation}
%     \mathrm{Upsample}_{I} \left( x \left[ n \right] \right) =  IDFT \left\{ DFT_{z_{I}}^{D} \left[ k,\ l \right] H^{2D}_{\mathrm{upsample},I}
% \end{equation}
\begin{equation}
\mathrm{Upsample}_{I}  \left( x \left[ n \right] \right)\left[ n \right] =   \mathrm{IDFT} \left( \mathrm{DFT} \left( x_z \left[ n \right] \right)\left[ k \right] H^{D}_{\mathrm{upsample}, I}\left[ k \right] \right) \left[ n \right]\,,
\end{equation}
Practically we used the following upsample kernel for a signal with spatial dimension $N$.
For even $N$:
\begin{equation}
    H^{D}_{\mathrm{upsample},I} \left[ k \right]=\begin{cases}
1 & 0\leq k<\frac{N}{2},\\
1 & N\left(I - \frac{1}{2}\right)+1\leq k\leq IN-1\\
0.5 & k=\frac{N}{2},k=N\left(I - \frac{1}{2}\right)\\
0 & \text{else}
\end{cases}
\end{equation}
For odd N:
\begin{equation}
H^{D}_{\mathrm{upsample}, I}\left[k\right]=\begin{cases}
1 & 0\leq k<\lfloor\frac{N}{2}\rfloor\\
1 & \lceil N\left(I - \frac{1}{2}\right) \rceil\leq k\leq2N-1\\
0 & \text{else}
\end{cases}	
\end{equation}
% and $H^{2D}_{\mathrm{upsample}_{I}} = HH^T$.
The reason for that is that in practice, we cannot assume the Nyquist condition holds. Specifically, for a finite signal $x\left[ n \right]$ with an even size $N$, we cannot assume that $X^{D} \left[ \frac{N}{2}\right] = X^{D} \left[ \frac{3N}{2}\right] = 0$. Note that for signals with even length, the $\frac{N}{2}$ component in the DFT domain represents the continual frequency of $\frac{\pi}{T}$, and thus, due to aliasing effect, we have
 $X^{D} \left[ \frac{N}{2}\right] = X^{D} \left[ \frac{3N}{2}\right] = X^F \left( \frac{\pi}{T} \right) + X^F \left( \frac{-\pi}{T} \right)$. 
 For a representation of the continuous signal with a higher sampling rate (e.g. the upsampled signal), the overlap in this frequency would not happen, hence we multiply this component by $\frac{1}{2}$ to get $\frac{1}{2} \left( X^F \left( \frac{\pi}{T} \right) + X^F \left( \frac{-\pi}{T} \right) \right) = X^F \left( \frac{\pi}{T} \right)$.
In this equation we use the assumption that $X^F \left( \frac{\pi}{T} \right) \in \mathbb{R}$. 
For a real signal The CTFT is conjugate symmetric, meaning 
$X^F \left( \frac{\pi}{T} \right) = X^F \left( \frac{-\pi}{T} \right)^{*} $. 
Therefore in case $X^F \left( \frac{\pi}{T} \right)$ has an imaginary component, it cannot be retrieved from the sum.
 A more detailed proof of this is given in appendix \cref{sec:implementation-proofs} below.
 % \daniel{A more detailed proof of this is given in appendix section XXX below.}


 
\section{Implementation proofs}
\label{sec:implementation-proofs}
 In the following section, we provide formal proofs for the correctness of the filters presented in \cref{sec:implementation-appendix} In the setting of finite discrete signals, where we assume the continuous domain signals are periodic.
 For simplicity to the reader, we prove the correctness for 1-dimensional signal, and for upsampling at factor 2. The proofs can be easily generalized to 2D signals and a higher upsampling rate.

\subsection{Definitions}\label{sec:implementation-proofs-defs}
Let $x\left(t\right)$ be a band-limited continuous signal, with CTFT  $X^{F}\left(\omega\right)$ (as defined in \cref{eq:ctft}). $x\left(t\right)$ is periodic with period $NT$, i.e. $x\left(NT+t\right)=x\left(t\right)$. We define discrete sampling as:
\begin{equation}
    x\left[n\right]=x\left(nT\right)\,,    
\end{equation}
and define a finite sampling as taking only one period of the discrete signal, i.e. 
\begin{equation}
    x_{N}\left[n\right]=\left\{ x_{0},...,x_{N-1}\right\} =\left\{ x\left(0\right),x\left(T\right),...,x\left(\left(N-1\right)T\right)\right\} \,.    
\end{equation}

\subsection{Upsample}
As noted in \cref{sec:implementation-appendix}, we assume that $X^{F}(\omega)=0\ \ \forall|\omega|>\frac{\pi}{T}$  and that $X^{F}\left( \frac{\pi}{T}\right) \in \mathbb{R}$ (This a relaxation of Nyquist condition for which $X^{F} \left( \frac{\pi}{T} \right) = 0$). We prove the validity of the following method to upsample $x_{N} \left[ n \right]$  to retrieve:
\begin{equation}\label{eq:x-2n}
    x_{2N}\left[n\right]=x\left(n\frac{T}{2}\right)=\left\{ x\left(0\right),x\left(\frac{T}{2}\right),...,x\left(\left(2N-1\right)\frac{T}{2}\right)\right\} \,.
\end{equation}
The upsample method presented in \Cref{sec:implementation-appendix} is formally shown in \Cref{algo:Up-sample}:

% \begin{algorithm}[h]
% \caption{Upsample}
% \label{algo:Up-sample}
% %\begin{algorithmic}
% % \item {\bfseries Input:}
%     Input: $x_{N}\left[n\right]\in\mathbb{R}^{N}$ \\
%     $x_{z} \gets \left\{ x_{N} \left[ 0 \right],\ 0,\ x_{d} \left[ 1 \right],\ 0\ ,...,\ x_{d} \left[ N-1 \right],\ 0 \right\}$ \\
%     $x_{2N} \gets \mathrm{IDFT}\left\{ \mathrm{DFT} \left\{ x_{z}\right\} H_{2}^{D}\right\} $ \\
%     Output: $x_{2N}\left[ n \right]$ 
% \end{algorithm}

\begin{algorithm}[h]
\caption{Upsample}
\label{algo:Up-sample}
\begin{algorithmic}
\item {\bfseries Input:} $x_{N}\left[n\right]\in\mathbb{R}^{N}$ 
    \item $x_{z} \gets \left\{ x_{N} \left[ 0 \right],\ 0,\ x_{d} \left[ 1 \right],\ 0\ ,...,\ x_{d} \left[ N-1 \right],\ 0 \right\}$ 
    \item $x_{2N} \gets \mathrm{IDFT}\left\{ \mathrm{DFT} \left\{ x_{z}\right\} H_{2}^{D}\right\} $
    \item {\bfseries Output:} $x_{2N}\left[ n \right]$ 
    \end{algorithmic}
\end{algorithm}


% on $x_{N}$ yields $x_{2N}$, noted as ``ideal reconstruction filter''
% with frequency response denoted as $R[k]$.

% \paragraph{Claim 1} \l

\begin{claim} \label{prop:upsample-algo}
Let $x\left(t\right)$ and $x_{N}\left[n\right]$ be a continuous signal and its finite discrete representation as defined in \cref{sec:implementation-proofs-defs}.
 % Let $x\left(t\right)$ be a band-limited continuous signal, with CTFT (defined in \cref{eq:ctft}) $X^{F}(\omega)=0\ \ \forall|\omega|>\frac{\pi}{T}$, and
 % \[
 % x_{N}\left[n\right]=\left\{ x_{0},...,x_{N-1}\right\} =\left\{ x\left(0\right),x\left(T\right),...,x\left(\left(N-1\right)T\right)\right\} \,.    
 % \]
Then the output of \Cref{algo:Up-sample} is 
\[
    x_{2N}\left[n\right]=x\left(n\frac{T}{2}\right)=\left\{ x\left(0\right),x\left(\frac{T}{2}\right),...,x\left(\left(2N-1\right)\frac{T}{2}\right)\right\} \,,
\]
using the described reconstruction filter $H^{D}_{2}$ below:

For even N:
\begin{equation}
    H^{D}_{2} \left[k \right]=\begin{cases}
        1 & 0\leq k<\frac{N}{2},\\
        1 & \frac{3N}{2}+1\leq k\leq2N-1\\
        0.5 & k=\frac{N}{2},k=\frac{3N}{2}\\
        0 & else
    \end{cases}
    \end{equation}
For odd N:
\begin{equation}
    H^{D}_{2} \left[k \right]=\begin{cases}
        1 & 0\leq k<\lfloor\frac{N}{2}\rfloor\\
        1 & \lceil\frac{3N}{2}\rceil\leq k\leq2N-1\\
        0 & else
    \end{cases}
\end{equation}
\end{claim}
% Note that $N$ represents the length of the input  of \ref{algo:Up-sample}, and thus the parameter LPF should be of length $2N$.
\paragraph{Proof (\cref{prop:upsample-algo})}
In order to proof \cref{prop:upsample-algo}, we will show that the DFT of its output equals to $X^{D}_{2N}$, i.e.~the DFT of the signal $x_{2N}$ that is defined in \cref{eq:x-2n}.

\subsubsection{DFT of zero-padded signal}
Recall that in the first step of the \cref{algo:Up-sample} we apply zero padding on the input $x_{N}$.
For a finite discrete signal with length $N$, DFT is defined as \cref{eq:dft}:
\[
X^{D}[k]=\sum_{n=0}^{N-1}x[n]e^{-\frac{j2\pi}{N}nk}\underbrace{=}_{W_{N}\triangleq e^{\frac{j2\pi}{N}}}\sum_{n=0}^{N-1}x[n]W_{N}^{-nk}
\]

\begin{claim} \label{prop:dft-zero-pad}
Let $X_{N}^{D}$ be the DFT of the input $x_{N}$ of \Cref{algo:Up-sample} and $X_{z}^{D}$ be the DFT of the zero-padded signal $x_{z}$ in step 1 of \Cref{algo:Up-sample}. Then:
\begin{equation}
    X_{z}^{D}[k]=\begin{cases}
        X_{N}^{D}[k] & k<N\\
        X_{N}^{D}[k-N] & k\geq N
    \end{cases}\label{eq:1}
\end{equation}
\end{claim}
\paragraph{Proof (\cref{prop:dft-zero-pad})}
\begin{align}
X_{z}^{D} \left[ k \right] &= \sum_{n=0}^{2N-1} x_{z} \left[ n \right] W_{2N}^{-nk} \\
&\underset{(\ast)}{=} \sum_{n'=0}^{N-1}\left(x_{z} \left[ 2n' \right] W_{2N}^{-2n'k} + \underbrace{x_{z}[2n'+1]}_{=0\ \forall n'}W_{2N}^{-(2n'+1)k}\right) \\
 % & \text{ (*) seperate sum to odd and even components}\\
 & =\sum_{n'=0}^{N-1}x_{z} \left[ 2n' \right] W_{2N}^{-2n'k}\\
 & \underset{(\ast \ast)}{=}\sum_{n'=0}^{N-1}x_{N} \left[ n' \right]W_{N}^{-n'k} \\
 % & \text{(**)}W_{2N}^{-2n'k}=e^{\frac{j2\pi}{2N}(-2n'k)}=e^{\frac{j2\pi}{N}(-n'k)}=W_{N}^{-n'k}
\end{align}
In $(\ast)$ we separated the summation to even and odd components, and in $(\ast \ast)$ we used the fact that 
\[
W_{2N}^{-2n'k}=e^{\frac{j2\pi}{2N}(-2n'k)}=e^{\frac{j2\pi}{N}(-n'k)}=W_{N}^{-n'k} \,.
\]
Note that we got a sum of $N$ components, yet $X_{z}^{D}[k]$ is defined for $k=0,1,...,2N-1$.
For $k=0,1,...,N-1$ we got the definition of DFT, meaning 
\begin{equation}
X_{z}^{D} \left[ k \right] = X_{N}^{D} \left[ k \right] \,.    
\end{equation}
For $k=N,...,2N-1$ using the property 
\[
W_{N}^{-n'(N+k)}=\underbrace{W_{N}^{-n'N}}_{=e^{\frac{j2\pi}{N}(-n'N)}=e^{-j2\pi n'}=1}W_{N}^{-n'k}=W_{N}^{-n'k}
\]
we get:
\begin{align}
X_{z}^{D} \left[ k \right] &= \sum_{n'=0}^{N-1}x_{N} \left[ n' \right] W_{N}^{-n'k} \\
&= \sum_{n'=0}^{N-1}x_{N} \left[ n' \right] W_{N}^{-n' \left( N+ \left( k-N \right) \right)} \\
&= \sum_{n'=0}^{N-1}x_{N} \left[ n' \right] W_{N}^{-n' \left( k-N \right) } \\
&= X_{N}^{D}\left[ k-N \right] \,.
\end{align}

\subsubsection{Expressing DFT with CTFT } \label{sec:dft-ctft}
In the previous section we expressed $X_{z}^{D}$ with $X_{N}^{D}$.
In the following section we will express $X_{2N}^{D}$ using $X_{N}^{D}$.
In addition, we assume that $N$ is even, and will show the other case afterwards.
First, we express the DFT of $x_{N}$ with the CTFT of $x$, by using their relations with DTFT:
\begin{align}
X_{N}^{D} \left[ k \right] &= X_{N}^{f} \left( \theta=\frac{2\pi k}{N} \right) \\
&= \frac{1}{T}\sum_{l=-\infty}^{\infty}X^{F}\left( \frac{\theta+2\pi l}{T} \right) \\
&= \frac{1}{T}\sum_{l=-\infty}^{\infty}X^{F} \left( \frac{2\pi k}{NT}+\frac{2\pi l}{T} \right) \\
& \underbrace{=}_{X^{F} \left( \omega \right) = 0\ \ \forall|\omega|>\frac{\pi}{T}}\frac{1}{T} \left( X^{F} \left( \frac{2\pi k}{NT} \right)+X^{F} \left( \frac{2\pi k}{NT}-\frac{2\pi}{T} \right) \right)
\end{align}

\begin{equation} \label{eq:x-n-dft} %\label{eq:2}
\Rightarrow X_{N}^{D}[k]=\frac{1}{T}\begin{cases}
X^{F}(\frac{2\pi k}{NT}) & 0\leq k\leq\frac{N}{2}-1\\
X^{F}(\frac{\pi}{T})+X^{F}(-\frac{\pi}{T}) & k=\frac{N}{2}\\
X^{F}(\frac{2\pi k}{NT}-\frac{2\pi}{T}) & \frac{N}{2}+1\leq k\leq N-1
\end{cases} \,.
\end{equation}
Note that the last two transitions hold considering the limited support
of $X^{F}:$
\[
\underbrace{X^{F}(\frac{2\pi k}{NT})}_{=0\ \ \forall k>\frac{N}{2}}+\underbrace{X^{F}(\frac{2\pi k}{NT}-\frac{2\pi}{T})}_{=0\ \ \forall k<\frac{N}{2}} \,.
\]
Next, we will derive a similar expression for $X_{2N}^{D}[k]$:
\begin{align}
X_{2N}^{D} \left[ k \right] &= X_{2N}^{f} \left( \theta=\frac{2\pi k}{2N} \right) \\
&= \frac{1}{T}\sum_{l=-\infty}^{\infty}X^{F}\left( \frac{\theta+2\pi l}{T/2} \right) \\
&= \frac{1}{T} \left( \underbrace{ X^{F} \left(\frac{2\pi k}{NT} \right)}_{=0\ \ \forall k>\frac{N}{2}}+\underbrace{X^{F}\left( \frac{2\pi k}{NT}-\frac{2\pi}{T/2} \right)}_{=0\ \ \forall k<\frac{3N}{2}} \right)\\
\end{align}
 We get: 
\begin{equation}
\Rightarrow X_{2N}^{D}[k]=\frac{1}{T}\begin{cases}
X^{F}(\frac{2\pi k}{NT}) & 0\leq k\leq\frac{N}{2}\\
0 & \frac{N}{2}+1\leq k\leq\frac{3N}{2}-1\\
X^{F}(\frac{2\pi k}{NT}-\frac{4\pi}{T}) & \frac{3N}{2}\leq k\leq2N-1
\end{cases}\label{eq:3}
\end{equation}

\subsubsection{Expressing $X_{2N}^{D}$ with $X_{N}^{D}$}
Considering the second step of \Cref{algo:Up-sample}, we need to show that applying the filter $H^{D}_{2}$ on $X_{z}^{D}$ yields $X_{2N}^{D}$, meaning
\[
X_{2N}^{D}\left[ k \right] = X_{z}^{D} \left[k \right] H^{D}_{2} \,.
\]
By plugging $X_{N}^{D} \left[ k \right]$ (\cref{eq:x-n-dft}) in $X_{z}^{D} \left[k \right]$ (\cref{eq:1})
we get: 
\begin{align}
X_{z}^{D} \left[k \right] &=\begin{cases}
    X_{N}^{D} \left[ k \right] & k<N\\
    X_{N}^{D}\left[k-N \right] & k\geq N
    \end{cases} \\
 &= \frac{1}{T}\begin{cases}
    X^{F} \left( \frac{2\pi k}{NT} \right) & 0\leq k\leq\frac{N}{2}-1\\
    X^{F} \left(\frac{\pi}{T} \right)+X^{F} \left(-\frac{\pi}{T} \right) & k=\frac{N}{2}\\
    X^{F}\left( \frac{2\pi k}{NT}-\frac{2\pi}{T} \right) & \frac{N}{2}+1\leq k\leq N-1\\
    X^{F} \left(\frac{2\pi \left(k-N \right)}{NT} \right) & N\leq k\leq\frac{3N}{2}-1\\
    X^{F} \left(\frac{\pi}{T}\right)+X^{F} \left(-\frac{\pi}{T}\right) & k=\frac{3N}{2}\\
    X^{F} \left(\frac{2\pi \left(k-N \right)}{NT}-\frac{2\pi}{T} \right) & \frac{3N}{2}+1\leq k\leq2N-1
\end{cases} \,.
\end{align}
Thus, by applying the filter 
\begin{equation}
H^{D}_{2} \left[ k \right]=\begin{cases}
1 & 0\leq k<\frac{N}{2},\\
1 & \frac{3N}{2}+1\leq k\leq2N-1\\
0.5 & k=\frac{N}{2},k=\frac{3N}{2}\\
0 & else
\end{cases}
\end{equation}


we get: 
\begin{equation}
H^{D}_{2} \left[ k \right] X_{z}^{D} \left[ k \right] = \frac{1}{T} \begin{cases}
X^{F} \left( \frac{2\pi k}{NT} \right)                                                              & 0\leq k\leq\frac{N}{2}-1\\
\frac{1}{2}\left(X^{F} \left(\frac{\pi}{T} \right)+X^{F} \left( -\frac{\pi}{T} \right)\right)      & k=\frac{N}{2}\\
0                                                                                                     & \frac{N}{2}+1\leq k\leq N-1\\
0                                                                                                   & N\leq k\leq\frac{3N}{2}-1\\
\frac{1}{2}\left(X^{F} \left( \frac{\pi}{T} \right)+X^{F} \left(-\frac{\pi}{T} \right)\right)                                      & k=\frac{3N}{2}\\
X^{F} \left(\frac{2\pi \left( k-N \right) }{NT}-\frac{2\pi}{T} \right)                              & \frac{3N}{2}+1\leq k\leq2N-1
\end{cases} \,.
\end{equation}

Note that for real $x(t)$, $X^{F}$ is conjugate symmetric, and we assumed  that $X^{F} \left(\frac{\pi}{T} \right) \in \mathbb{R}$. Therefore:
\begin{itemize}
\item for $k=\frac{N}{2},\frac{3N}{2}$: 
\begin{equation}
X^{F} \left(\frac{\pi}{T} \right) + X^{F} \left( -\frac{\pi}{T} \right) = 2X^{F} \left( \frac{\pi}{T} \right)=2X^{F} \left(-\frac{\pi}{T} \right) \,.
\end{equation}
\item for $\frac{3N}{2}+1\leq k\leq2N-1$: 
\begin{equation}
X_{z}^{D} \left[ k \right] = X^{F} \left( \frac{2\pi \left( k-N \right) }{NT}-\frac{2\pi}{T} \right)=X^{F} \left( \frac{2\pi k}{NT}-\frac{2\pi N}{NT}-\frac{2\pi}{T} \right)=X^{F} \left( \frac{2\pi k}{NT}-\frac{4\pi}{T} \right) \,.    
\end{equation}
\end{itemize}
Thus we get: 
\begin{align}
H^{D}_{2} \left[ k \right] X_{z}^{D} \left[ k \right] &= \frac{1}{T}\begin{cases}
X^{F} \left(\frac{2\pi k}{NT}\right)                 & 0\leq k\leq\frac{N}{2}-1\\
X^{F} \left(\frac{\pi}{T}\right)                    & k=\frac{N}{2}\\
0                                                     & \frac{N}{2}+1\leq k\leq N-1\\
0                                                   & N\leq k\leq\frac{3N}{2}-1\\
X^{F} \left( -\frac{\pi}{T} \right)                 & k=\frac{3N}{2}\\
X^{F} \left( \frac{2\pi k}{NT}-\frac{4\pi}{T} \right) & \frac{3N}{2}+1\leq k\leq2N-1
\end{cases} \\
&= \frac{1}{T}\begin{cases}
X^{F} \left( \frac{2\pi k}{NT} \right)              & 0\leq k\leq\frac{N}{2}\\
0                                               & \frac{N}{2}+1\leq k\leq\frac{3N}{2}-1\\
X^{F} \left( \frac{2\pi k}{NT}-\frac{4\pi}{T} \right) & \frac{3N}{2}\leq k\leq2N-1
\end{cases} \\
& \underset{(\ref{eq:3})}{=} X_{2N}^{D} \left[ k \right] \,.  
\end{align}


\subsubsection{Odd $N$}
By repeating the derivations of section \cref{sec:dft-ctft} for odd $N$ we get: 
\begin{equation}
X_{N}^{D} \left[k \right]=\frac{1}{T}\begin{cases}
X^{F}\left(\frac{2\pi k}{NT} \right) & 0\leq k\leq\lfloor\frac{N}{2}\rfloor\\
X^{F}\left(\frac{2\pi k}{NT}-\frac{2\pi}{T} \right) & \lceil\frac{N}{2}\rceil\leq k\leq N-1
\end{cases}\label{eq:2.1} \,.
\end{equation}
 
\begin{equation}
X_{2N}^{D}\left[k\right]=\frac{1}{T}\begin{cases}
X^{F}\left(\frac{2\pi k}{NT}\right) & 0\leq k\leq \lfloor\frac{N}{2} \rfloor\\
0 & \lceil\frac{N}{2}\rceil\leq k\leq\lfloor\frac{3N}{2}\rfloor\\
X^{F}\left(\frac{2\pi k}{NT}-\frac{4\pi}{T}\right) & \lceil\frac{3N}{2}\rceil\leq k\leq2N-1
\end{cases}\label{eq:3.1} \,.
\end{equation}
By plugging $X_{N}^{D}\left[k\right]$ (\ref{eq:2.1}) in $X_{0}^{D}\left[k\right]$ (\ref{eq:1})
we get: 
\begin{align}
X_{z}^{D}\left[k\right] & =\begin{cases}
X_{N}^{D}\left[k\right] & k<N\\
X_{N}^{D}\left[k-N\right] & k\geq N
\end{cases} \\
 &= \frac{1}{T}\begin{cases}
X^{F}\left(\frac{2\pi k}{NT}\right) & 0\leq k\leq\lfloor\frac{N}{2}\rfloor\\
X^{F}\left(\frac{2\pi k}{NT}-\frac{2\pi}{T}\right) & \lceil\frac{N}{2}\rceil\leq k\leq N-1\\
X^{F}\left(\frac{2\pi\left(k-N\right)}{NT}\right) & N\leq k\leq\lfloor\frac{3N}{2}\rfloor\\
X^{F}\left(\frac{2\pi\left(k-N\right)}{NT}-\frac{2\pi}{T}\right) & \lceil\frac{3N}{2}\rceil\leq k\leq2N-1
\end{cases} \,.
\end{align}
Then, by applying the filter 
\begin{equation}
H^{D}_{2}\left[k\right]=\begin{cases}
1 & 0\leq k<\lfloor\frac{N}{2}\rfloor,\\
1 & \lceil\frac{3N}{2}\rceil\leq k\leq2N-1\\
0 & else
\end{cases}
\end{equation}
 we get: 
\begin{align}
H^{D}_{2}\left[k\right]X_{z}^{D}\left[k\right] &= \frac{1}{T}\begin{cases}
X^{F}\left(\frac{2\pi k}{NT}\right) & 0\leq k\leq\lfloor\frac{N}{2}\rfloor\\
% 0 & \lceil\frac{N}{2}\rceil\leq k\leq N-1 \\
% 0 & N\leq k\leq\lfloor\frac{3N}{2}\rfloor \\
0 & \lceil\frac{N}{2}\rceil\leq k\leq \lfloor\frac{3N}{2}\rfloor \\
X^{F}\left(\frac{2\pi\left(k-N\right)}{NT}-\frac{2\pi}{T}\right) & \lceil\frac{3N}{2}\rceil\leq k\leq2N-1
\end{cases} \\
& \underset{(\ref{eq:3.1})}{=} X_{2N}^{D}\left[k\right] \,.
\end{align}

\subsection{LPF}
We use an ``ideal LPF'' before downsampling in BlurPool layers and in alias-free polynomial activations.
As mentioned in \cref{sec:shift-invariance-proof} ``ideal LPF'' in the context of subsampling of infinite discrete signals is a filter that completely eliminates all the frequencies beyond the Nyquist condition.
E.g. in case of subsampling in factor $I$, i.e. 
\[
y \left[ n \right] = x \left[ I n \right] \,,
\]
an ``ideal LPF'' in DTFT domain is implemented by multiplication with the filter:
\[
H^{f}_{1/I} \left(\theta \right) = \begin{cases}
    1  &    \left| \theta \right| < \frac{\pi}{I} \,, \\
    0  &     \left| \theta \right| \geq \frac{\pi}{I}
\end{cases}
\]
When considering the continuous domain, we expect the results to be equal to the discrete representation of the continuous signal after applying ``ideal LPF'', that is multiplication in CTFT domain with the filter 
\[
H^{F}_{1/I} \left(\theta \right) = \begin{cases}
    1  &    \left| \theta \right| < \frac{\pi}{TI}  \\
    0  &     \left| \theta \right| \geq \frac{\pi}{TI}
\end{cases} \,,
\]
where $T$ is the sample rate of $x \left[ n \right]$.
\begin{claim} \label{prop:ideal-lpf}
Let $x\left(t\right)$ and $x_{N}\left[n\right]$ be a continuous signal and its finite discrete representation as defined in \cref{sec:implementation-proofs-defs}.
In addition, let $x_{\mathrm{LPF}} \left( t \right)$ be the continuous signal received by applying $H^{F}_{1/I}$ on $x \left(t \right)$ in the continuous domain, i.e.~:
\begin{equation} \label{eq:x-lpf-ctft}
X_{\mathrm{LPF}}^{F}\left(\omega\right)	=\begin{cases}
X^{F}\left(\omega\right) & \left|\omega\right|<\frac{\pi}{TI}\\
0 & \left|\omega\right|\geq\frac{\pi}{TI}
\end{cases}
\end{equation}
In addition, define $x_{\mathrm{LPF}} \left( t \right)$ discrete representation as: 
\[
x_{\mathrm{LPF}} \left[ n \right] = \left\{ x_{\mathrm{LPF}}\left(0\right),\ x_{\mathrm{LPF}}\left(T\right),...,\ x_{\mathrm{LPF}}\left(\left(N-1\right)T\right)\right\} \,.    
\]
Then applying $\mathrm{LPF}_{1/I}$  on $x_{N}$ gives
 \[
 \mathrm{LPF}_{1/I} \left( x_{N} \right) = x_{\mathrm{LPF}}\left[ n\right]
 \]
where $\mathrm{LPF}_{1/I}$ is defined as multiplication in DFT domain with
\[
H^D_{1/I} \left[ k \right]=\begin{cases}
1 & 0\leq k<\frac{N}{2I}\,,\\
0 & \frac{N}{2I}\leq k \leq N - \frac{N}{2I}\,,\\
1 & N - \frac{N}{2I} < k\leq N-1\,.
\end{cases} \,.
\]
\end{claim}

\paragraph{proof (\cref{prop:ideal-lpf})}
Similarly to \cref{sec:dft-ctft}, using the relations between DFT, DTFT and CTFT we get:

\begin{align}
X_{\mathrm{LPF}}^{D}\left[k\right]	&=  X_{\mathrm{LPF}}^{f}\left(\theta=\frac{2\pi k}{N}\right) \\
&=  \frac{1}{T}\sum_{l=-\infty}^{\infty}X_{\mathrm{LPF}}^{F}\left(\frac{\theta+2\pi l}{T}\right) \\
&= \frac{1}{T}\sum_{l=-\infty}^{\infty}X_{\mathrm{LPF}}^{F}\left(\frac{2\pi k}{NT}+\frac{2\pi l}{T}\right) \\
& \underbrace{=}_{X^{F}(\omega)=0\ \ \forall|\omega|>\frac{\pi}{T}}\frac{1}{T}\left(X_{\mathrm{LPF}}^{F}\left(\frac{2\pi k}{NT}\right)+X_{\mathrm{LPF}}^{F}\left(\frac{2\pi k}{NT}-\frac{2\pi}{T}\right)\right)    \,.
\end{align}
By plugging \cref{eq:x-lpf-ctft} we get:
\begin{equation}
X_{\mathrm{LPF}}^{D}\left[k\right]=\frac{1}{T}\begin{cases}
X^{F}\left(\frac{2\pi k}{2T}\right) & 0\leq k<\frac{N}{2I}\\
0 & \frac{N}{2I}\leq k\leq N-\frac{N}{2I}\\
X_{\mathrm{}}^{F}\left(\frac{2\pi k}{NT}-\frac{2\pi}{T}\right) & N-\frac{N}{2I}<k\leq N-1 \,.
\end{cases}
\end{equation}
Recall that $x_{N}$ satisfies (\cref{eq:x-n-dft}): 
\[ 
X_{\mathrm{N}}^{D}\left[k\right]=\frac{1}{T}\begin{cases}
X^{F}\left(\frac{2\pi k}{2T}\right) & 0\leq k<\frac{N}{2}-1\\
X^{F}\left(\frac{\pi}{T}\right)+X^{F}\left(-\frac{\pi}{T}\right) & k=\frac{N}{2}\\
X_{\mathrm{}}^{F}\left(\frac{2\pi k}{NT}-\frac{2\pi}{T}\right) & \frac{N}{2}<k\leq N-1
\end{cases} \,;
\]
thus we get 
\begin{equation}
X_{\mathrm{LPF}}^{D}\left[k\right]=H_{1/I}^{D}\left[k\right]X^{D}\left[k\right] \,.
\end{equation}



\end{document}