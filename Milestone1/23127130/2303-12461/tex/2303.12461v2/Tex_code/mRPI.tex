\subsection{Tracking error characterization: an example}
Since the system dynamics and constraints now are embedded in a linear context, let us provide some insights on how the stability and robustness can be analyzed. More specifically, let us consider the disturbed version of \eqref{eq:drone_dyna} as:
\be 
\ddot \bsig  = h(\psi,\bu) + \bd_\sigma, 
\label{eq:disturbed_drone}
\ee 
where $\bd_\sigma\triangleq[d_{1\sigma},d_{2\sigma},d_{3\sigma}]^\top\in\R^3$ denotes the perturbation acting on the acceleration of the drone and the input $\bu\in\mathcal{U}$ is the constrained input. Then after the transformation \eqref{eq:linearization}, equation \eqref{eq:disturbed_drone} yields:
\be 
\ddot \sigma_i  = v_i + d_{i\sigma}, \,(i\in\{1,2,3\})
\label{eq:disturbed_drone_lin}
\ee 
and, imposingly, the new input $\bv\triangleq[v_1,v_2,v_3]^\top$ is constrained in $\Tilde{\mathcal{V}}$. Then, for \eqref{eq:disturbed_drone_lin} to track the reference signals\footnote{This reference is assumed to be available together with the input reference $\bv_{ref}\triangleq\ddot\bsig_{ref}$ since all the variables can be recovered with the parameterization of the flat output} $\sigma_i^{ref}$ associated with the input reference $\bv_{ref}=\ddot\bsig_{ref}$, let us consider the following controller:
\be 
v_i = \ddot\sigma_{i\,ref} + k_{di}\dot e_i + k_{pi}e_i
\ee 
with $e_i=\sigma_{i,ref}-\sigma_i$. Then the closed loop yields:
\be 
\dot{\blde}_i=\bbm  k_{pi}e_i& 0 \\ 0 &k_{di}\ebm \blde + \bbm0\\1 \ebm  d_{i\sigma}
\label{eq:closed_loop}
\ee 
where $\blde = [e_i, \dot e_i]^\top$. Then to evaluate the controller's performance, as proposed in our previous work \cite{stoican2015rpi,do_ifac_cao}, let us construct the outer approximation of the minimal robust positive invariant (mRPI) set for the system \eqref{eq:closed_loop}, denoted as $\Omega_i$. 