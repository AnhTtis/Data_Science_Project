% ****** Start of file apssamp.tex ******
%
%   This file is part of the APS files in the REVTeX 4.2 distribution.
%   Version 4.2a of REVTeX, December 2014
%
%   Copyright (c) 2014 The American Physical Society.
%
%   See the REVTeX 4 README file for restrictions and more information.
%
% TeX'ing this file requires that you have AMS-LaTeX 2.0 installed
% as well as the rest of the prerequisites for REVTeX 4.2
%
% See the REVTeX 4 README file
% It also requires running BibTeX. The commands are as follows:
%
%  1)  latex apssamp.tex
%  2)  bibtex apssamp
%  3)  latex apssamp.tex
%  4)  latex apssamp.tex
%
\documentclass[%
 reprint,
%superscriptaddress,
%groupedaddress,
%unsortedaddress,
%runinaddress,
%frontmatterverbose, 
%preprint,
%preprintnumbers,
nofootinbib,
%nobibnotes,
%bibnotes,
 amsmath,amssymb,
 aps,
%pra,
%prb,
%rmp,
%prstab,
%prstper,
%floatfix,
]{revtex4-2}

\usepackage{graphicx}% Include figure files
\usepackage{dcolumn}% Align table columns on decimal point
\usepackage{bm}% bold math
%\usepackage{hyperref}% add hypertext capabilities
%\usepackage[mathlines]{lineno}% Enable numbering of text and display math
%\linenumbers\relax % Commence numbering lines
\usepackage{adjustbox}
\usepackage{graphicx}% Include figure files
\usepackage{dcolumn}% Align table columns on decimal point
\usepackage{bm}% bold math
\usepackage{hyperref}% add hypertext capabilities
\usepackage{cleveref}
\usepackage[caption=false]{subfig}
\usepackage[usenames]{color}

\newcommand{\redflag}[1]{{\color{red} #1}}
\newcommand{\blueflag}[1]{{\color{blue} #1}}
\newcommand{\greenflag}[1]{{\color{green} #1}}

%\usepackage[showframe,%Uncomment any one of the following lines to test 
%%scale=0.7, marginratio={1:1, 2:3}, ignoreall,% default settings
%%text={7in,10in},centering,
%%margin=1.5in,
%%total={6.5in,8.75in}, top=1.2in, left=0.9in, includefoot,
%%height=10in,a5paper,hmargin={3cm,0.8in},
%]{geometry}

\begin{document}

\preprint{APS/123-QED}

\title{Relic Neutrino Helicity Evolution in Galactic Magnetic Field and Its Implications}% Force line breaks with \\




\author{Kuo(i:to).Liao}
\email{kl4180@nyu.edu}
 \altaffiliation{%
 Center for Cosmology and Particle Physics, Department of Physics, New York University}%Lines break automatically or can be forced with \\
\author{Glennys R. Farrar}%
 \email{gf25@nyu.edu}
\affiliation{%
 Center for Cosmology and Particle Physics, Department of Physics, New York University  
}%


\date{\today}% It is always \today, today,
             %  but any date may be explicitly specified

\begin{abstract}
We simulate the evolution of the helicity of relic neutrinos as they propagate to Earth through a realistic model of the Galactic magnetic field, improving upon the rough estimates in the pioneering work of Baym and Peng. 
%Following ideas from previous research, we apply Jansson and Farrar(JF-12) \cite{Jansson:2012pc} coherent galactic magnetic field model and updated model with random components to have a comprehensive understanding of relic neutrino helicity evolution in astrophysical magnetic field for the first time.
We find that with magnetic moments consistent with experimental bounds and even several orders of magnitude smaller, the helicity of relic neutrinos rotates with a substantial directional anisotropy.  Averaged over directions this would simply reduce the apparent flux; if the direction of the incident neutrino could be measured, the directional anisotropy in the interaction probability could become a powerful diagnostic. We study the effects of $\nu$ spin rotation on C$\nu$B detection through the inverse tritium decay process. 
%In favor of Python package $Healpix$, relic neutrino's helicity-flip probability and its distribution are presented.%
%e report simulation results of relic neutrino helicity evolution in the astrophysical magnetic field. 
%Following ideas from previous research, we apply Jansson and Farrar(JF-12) \cite{Jansson:2012pc} coherent galactic magnetic field model and updated model with random components to have a comprehensive understanding of relic neutrino helicity evolution in astrophysical magnetic field for the first time.


\end{abstract}
\keywords{Suggested keywords}%Use showkeys class option if keyword
%display desired
\maketitle

%\tableofcontents

\section{Introduction}

Relic neutrinos carry information from the early Universe. They decoupled from the hot plasma about 1 second after the big bang, much earlier than the cosmic microwave background decoupling.
They are decoupled in chirality eigenstates at temperatures $\simeq$ MeV.
When decoupled, they were highly relativistic, and their helicity eigenstates coincided with chiral eigenstates. Previous research \cite{Baym:2020riw}\cite{Baym:2021ksj} studied the effects of astrophysical magnetic field and gravitational inhomogeneities on the helicities of present-day relic neutrinos. The helicity eigenstates of both Dirac and Majorana neutrinos are modified by gravitational effects, with finite mass, and for Dirac neutrinos helicity modifications are also induced by magnetic moments during propagation through astrophysical magnetic fields.

Inverse tritium beta decay \cite{PhysRev.128.1457} (ITBD) is a promising experimental tool \cite{PTOLEMY:2022ldz} to observe relic neutrinos.  Since the ITBD cross-section depends on helicity, in principle ITBD measurements can provide a tool to detect helicity modification. Moreover, with a polarized target, directional anisotropies are in principle accessible \cite{Lisanti:2014pqa} thus detection of a helicity modification which depends on the neutrino arrival direction -- due in our case to the Galactic magnetic field having a non-trivial structure -- may one day be possible to explore observationally.

Our understanding of the Galactic magnetic field developed over the decades. Today the Jasson and Farrar model (JF12) \cite{Jansson:2012pc} fits the largest range of observations and should be a good semi-realistic model to study neutrino helicity evolution. 
Here we focus on Dirac neutrinos with non-zero diagonal magnetic moments. 





\section{Theoretical Background}
\subsection{Relic Neutrino Spin Rotation in Astrophysical Magnetic Field}
Neutrinos are produced in flavor eigenstates, but they arrive at Earth in well-separated mass packets.
The three flavors are superpositions of mass eigenstates via the Pontecorvo-Maki-Nakagawa-Sakata (PMNS) mixing matrix: 
\begin{gather}
 \begin{bmatrix} 
 \nu_{e} \\ \nu_{\mu} \\ \nu_{\tau} \end{bmatrix}
 =
  \begin{bmatrix}
   U_{e1} & U_{e2} & U_{e3} \\
   U_{\mu1} & U_{\mu2} & U_{\mu3} \\
   U_{\tau1} & U_{\tau2} & U_{\tau3} 
   \end{bmatrix}
 \begin{bmatrix} 
 \nu_{1} \\ \nu_{2} \\ \nu_{3}  
 \end{bmatrix}.
\end{gather}
While coupled to ambient plasma, relic neutrinos had a relativistic thermal distribution at temperature $T$. As momenta and temperature redshift identically with the expansion of the Universe, the Fermi-Dirac distribution is preserved even after the decoupling process, so the present momenta $\vec{p}_{0}$ and temperature $T_{0}$, for flavor $\alpha$:
\begin{equation}
    f_{\alpha}(\vec{p_{0}},T_{0})= \frac{1}{e^{|\vec{p_{0}}|/T_{0}}+1},
    \label{fermi}
\end{equation}
where present cosmic neutrino background temperature $T_{0}$ $\simeq$ 1.67 $\times$ $10^{-4}$ eV = $(\frac{4}{11})^{1/3}$ $T_{CMB}$.

Propagating in a magnetic field, the neutrino spin vector rotates in its rest frame \cite{Baym:2020riw}\cite{Baym:2021ksj}:
\begin{equation}     
     \frac{d\vec{S}}{d\tau} =2{\mathbf{\mu}}_{\nu}\index{\n}(\vec{S}\times{\vec{B}}),
\end{equation}
where $\vec{B} $ represents the magnetic field vector in its rest frame, $\tau$ is the neutrino proper time, ${\mathbf{\mu}}_{\nu}\index{\nu}$ is the diagonal magnetic moment of the Dirac neutrino, expressed in Bohr magnetons, $\mu_{B}=1.4$MHz/Gauss.
In the (extended) standard model \cite{Fujikawa:1980yx}:

\begin{equation}
    \mu^{SM}_{\nu} \approx \frac{3G_{F}}{4\sqrt{2}\pi^{2}} m_{\nu}m_{e}\mu_{B} \approx 3\times10^{-21} \frac{m_{\nu}}{0.01eV}\mu_{B},
\end{equation}
this is approximately $10^{-9}$ lower than the present experimental upper limit result from the GEMMA \cite{Beda:2012zz} Reactor and Borexino Phase-II
\cite{Borexino:2017fbd}
$\simeq$ 2.9$\times$$10^{-11}$ $\times$
$\mu_{B}$ \footnote{We also revisited the supersymmetric contribution to the neutrino magnetic moment of massive neutrino. Based on the updated 
LHC limit and the supersymmetry pattern \cite{PhysRevD.28.671}, we found that the SUSY value cannot be larger than the contribution of the standard model result.}.
The neutrino lab frame time $dt$ = $\gamma$ d$\tau$, with the relativistic factor $\gamma$ $\approx$1 for a relic neutrino with a velocity much smaller than the speed of light.  
$\vec{S}$ interacts with the magnetic field and deviates with respect to its initial orientation.
The spin vector $\vec{S}$ will precess around the local $\vec{B}$ as the neutrino propagates through an astrophysical magnetic field in the lab frame:
\begin{equation}     
  \Delta \vec{S} =2{\mathbf{\mu}}_{\nu}\index{\n}\int{\vec{S}\times \vec{B}} \frac{dt}{\gamma}.
\end{equation}

The $\vec{S}$ evolution in terms of path dependence is
\begin{equation}
    \frac{\partial\vec{S}}{\partial \vec{r}}=\frac{1}{|\vec{v}|} \frac{\partial \vec{S}}{\partial {t}}=\frac{2}{\gamma}\frac{{\mathbf{\mu}}_{\nu}\index{\nu}}{|\vec{v}|}{\vec{S}\times \vec{B}}.
\end{equation} 
$\vec{S}_{final}$ can be calculated by integrating over the total propagation path length. 
Assuming that the $\vec{P}$ direction does not change during propagation through the Galaxy, the $\nu$ helicity evolution is well-justified according to  
 \cite{Baym:2020riw}. The probability of helicity flipping can be expressed as a function of the rotation angle $\theta$: $P_{f}=sin^2(\theta/2)$. %For infinitesimal rotation ($\theta \ll 1$), $P_{f}$ $\approx$ $\theta^2$/4.

In the lab frame, with respect to the neutrino's lab frame velocity $\vec{B}_{\perp}^{rest frame}$=$\gamma$$\vec{B}_{\perp}^{lab frame}$ and $\vec{B}_{\parallel}^{lab frame}$=$\vec{B}_{\parallel}^{rest frame}$, so in terms of lab frame magnetic field and time \cite{Baym:2020riw}: 
\begin{equation}
\frac{d\vec{S}_{\perp}}{dt}=(\vec{S}_{\parallel}\times\vec{B}_{\perp}^{lab frame}
+\frac{1}{\gamma} \vec{S}_{\perp}\times \vec{B}_{\parallel}^{lab frame})2{\mu_{\nu}}.
\label{7}
\end{equation}
The cumulative rotation angle $\theta$ at a specific moment compared to neutrino's initial $\vec{S}$ state is $\theta=sin^{-1}({\vec{S}_{\perp}/\vec{S}} )$. When $\theta \ll 1$, $\vec{S}_{\perp}\approx0$, the second term can be neglected in the computation process: Baym-Peng makes this appoximation but we do not.


\subsection{Simulation Method}

In this section, we summarize the essential elements of the simulation procedure.

Following the thermal distribution and its relationship with the neutrino total number density $n$= $\frac{1}{(2\pi)^{3}}$$\int$ f($\vec{p_0}$,$T_0$) $d^3\vec{p_0}$ $\approx$ 56.25$cm^{-3}$ we generate velocity samples for non-relativistic cases based on:
\begin{equation}
    1=\frac{1}{n}\frac{1}{(2\pi)^{3}} \int \frac{1}{e^{m|\vec{v}|/T_{0}}+1}    4\pi m^{3} v^{2}dv.
\end{equation}
The $Astropy.units$ function is imported to track the conversion of units better.


Throughout the simulation of each neutrino trajectory, we assume that the $\nu$ momentum direction is fixed in the laboratory frame. Neutrino's helicity change depends purely on the $\vec{S}$ variation during propagation through the astrophysical magnetic field. The JF12 Coherent Galactic Magnetic Field Model and the coherent model with random components are used in the simulation \cite{Jansson:2012pc}.
The coherent model includes the striated components, poloidal component, toroidal halo, and spiral arm disk components. Along the galactic disk, $>$5 kpc are 8 spiral arms. Between 3-5kpc is the purely azimuthal molecular ring. The magnetic field strength is set as zero at r$<$1kpc and r$>$20kpc.
Field data are presented in Cartesian Coordinates ranging [-20Kpc, 20Kpc], in units of $[\mu G]$, where the galaxy center coordinate is [0,0,0].  

Given the magnetic field model, we apply the Riemann sum to compute the spin vector evolution
\begin{equation}
    \vec{S}_{N} - \vec{S}_{N-1}= 2\frac{\mu_{\nu}}{|\vec{v}|}\times(\vec{S}_{N-1}\times\vec{B}_{N-1}) \delta r,
\end{equation} 
where $\delta {r}$ = $R_{ total}$/$N_{sample}$; $R_{total}$ is the propagation path length of a test neutrino coming from a known direction and N is the number of steps sampled. 

The incoming directions of relic neutrinos are generated as pixelated data. Using the Hierarchical Equal Area isoLatitude Pixelization (HEALPix) library and the Python package $healpy$, we plot values on a skymap with high resolution. The HEALPix resolution parameter (scalar integer) is set to 12288 pixels =$12\times N_{side}$; function $healpy.nside2npix$ from the package returns pixel number at $N_{side}$=32. Every pixel on the map represents a specific direction with latitude and longitude. For our purpose, each neutrino's incoming direction in the simulation is identical to each pixel on the mollweide view projection map. The area of each pixel is $\Omega_{pixel}$=$\pi/3(N_{side})^{2}$ and angular resolution is $(\Omega_{pixel})^{1/2}$ = 1.83 degrees.

We propagate $\nu$ at each pixel outward from Earth to a distance where the local field vanishes and far away from the galactic center, then we reverse the neutrino momentum vector and propagate the neutrino
inward starting with relic helicity eigenstate -1. Along the propagation inward we track the spin rotation of $\nu$.
The Riemann sum process yields the final direction of $\vec{S}$; the angle difference between $\vec{S}_{final}$ and $\vec{S}_{initial}$ is calculated and the
expectation values of rotation for each pixel are computed based on simulation results at different velocity samples.
Attempted magnetic moment values $\mu_{\nu}$ are the GEMMA reactor experimental upper limit $\mu_{\nu}$$<$ 2.9 $\times$ $10^{-11}$ $\mu_{B}$ and $10^{-1}$, $10^{-2}$ and $10^{-3}$ times smaller cases. The $\nu$ mass is assigned to be 0.1eV during the simulation; for this mass, refer to Equation (\ref{fermi}) to have the $\nu$ velocity. 

We note that for non-relativistic neutrinos, $v/c << 1$, the spin rotation is linear in  $ m_\nu$. This is because the spin rotation scales linearly in the time spent traversing the field, hence inversely with the velocity.  But as noted above, the relic neutrino momentum distribution is fixed by cosmology so $v_\nu \sim p_0/m_\nu$.  






\section{Discussion of the results}

We use Healpy Projview and molleweide sky-view visualization to show the all-sky directional anisotropy of relic neutrino spin rotation and helicity-flip. In both FIG.\ref{rotation A} and FIG. \ref{probability flip} Galactic center is $\sim$ 8kpc away from the earth
with coordinates being longitude = 0 and colatitude = 0.

FIG. \ref{rotation A} shows the spin rotation angles (modulo $\pi$) of a 0.1 eV neutrino in the JF12 Galactic magnetic field model as a function of arrival direction, for four different choices of magnetic moment, scaling the GEMMA Reactor experimental bound by 1, 0.1, 0.01 and 0.001.  
For an incoming neutrino at the upper bound value of magnetic moment from the GEMMA Reactor experiment, the total deviation angles between $\vec{S}_{final}$ and $\vec{v}$ ($\vec{S}_{initial}$) range from $\simeq$ 0.792 to $\simeq$ 2.061 radians in the coherent field model and $\simeq$  1.27 to $\simeq$ 1.82 radians in the field model including the random field.  %Thus the helicity-flip probability is of order unity at this scale in most directions over the sky.  
In fact, the spin rotates multiple times for many arrival directions when the moment is within two orders of magnitude of the GEMMA limit. However, at $10^{-3}$ $\times$ $\mu^{GEMMA}_{\nu}$, the rotation angles are in general less than $\pi/2$. 

FIG. \ref{probability flip} shows the helicity flip probability, in the Molleweide view. A large-scale directional anisotropy pattern is visible in the coherent field model for magnetic moments $10^{-1}$ or less than the experimental magnetic moment upper bound, with the high helicity-flip probability directions pointing toward the galactic center.  However, for the more realistic GMF model including the random components, large-scale regions with higher flip probability start to be observable at $10^{-2}$ $\times$ $\mu_{\nu}^{GEMMA}$. For larger magnetic moments the rotation is so large that the neutrino is in a nearly equal superposition of helicity eigenstates over most of the sky. The angular power spectrum is shown in FIG. \ref{power} to quantify how the angular scale of the flip-probability anisotropy depends on magnetic moment.  

Both field model results confirm that regions with longer propagation paths and traversing regions of the stronger fields have a higher helicity-flip probability -- hence the noticeable difference in the Galactic and antigalactic center directions, which could be depicted in FIG.\ref{rotation A}, the center of Molleweide plots represents the direction of Galactic direction


At the same time, the variation of the magnetic field intensity and the direction of the magnetic field vector $\vec{B}$ among different regions along the trajectory affects the helicity-flip probability results. 



\begin{figure*}[!htp]
\hspace*{-5.5em}
\subfloat{%
\includegraphics[height=45mm,width=53.5mm]{1,1angle.png}%
}\hspace*{-0.45em}
\subfloat{%
\includegraphics[height=45mm,width=53.5mm]{2,1angle.png}%
}\hspace*{-0.45em}
\subfloat{%
\includegraphics[height=45mm,width=53.5mm]{3,1angle.png}%
}\hspace*{-0.45em}
\subfloat{%
\includegraphics[height=45mm,width=53.5mm]{4,1angle.png}%
}\hfill
\hspace*{-5.5em}
\subfloat{%
\includegraphics[height=45mm,width=53.5mm]{1,2angle.png}%
}\hspace*{-0.45em}
\subfloat{%
\includegraphics[height=45mm,width=53.5mm]{2,2angle.png}%
}\hspace*{-0.45em}
\subfloat{%
\includegraphics[height=45mm,width=53.5mm]{3,2angle.png}%
}\hspace*{-0.45em}
\subfloat{%
\includegraphics[height=45mm,width=53.5mm]{4,2angle.png}%
}
\caption{Spin vector rotation angle of a 0.1 eV relic neutrino traversing (first row) the JF12 coherent Galactic magnetic field and (second row) the full coherent plus random field.  From left to right the columns are for 
a magnetic moment equal to the GEMMA Reactor experiment bound $\mu_{\nu}^{GEMMA}\approx2.9\times10^{-11} \mu_{B}$ and factor-10 smaller values: 0.1$\mu_{\nu}^{GEMMA}$, 0.01$\mu_{\nu}^{GEMMA}$, and 0.001$\mu_{\nu}^{GEMMA}$.  $Healpy$ $Projview$ polar view has been used for sky-view mapping, with the coordinates being longitude from 0 to 2$\pi$ and colatitude from -$\pi/2$ to $\pi$/2. }
\label{rotation A}
\end{figure*}  








\begin{figure*}[!htp]
\hspace*{-5.0em}
\subfloat{%
\includegraphics[height=37mm,width=53mm]{1,1prob.png}%
}\hspace*{-0.4em}
\subfloat{%
\includegraphics[height=37mm,width=53mm]{2,1prob.png}%
}\hspace*{-0.4em}
\subfloat{%
\includegraphics[height=37mm,width=53mm]{3,1prob.png}%
}\hspace*{-0.4em}
\subfloat{%
\includegraphics[height=37mm,width=53mm]{4,1prob.png}%
}\hfill
\hspace*{-5.0em}
\subfloat{%
\includegraphics[height=37mm,width=53mm]{1,2prob.png}%
}\hspace*{-0.4em}
\subfloat{%
\includegraphics[height=37mm,width=53mm]{2,2prob.png}%
}\hspace*{-0.4em}
\subfloat{%
\includegraphics[height=37mm,width=53mm]{3,2prob.png}%
}\hspace*{-0.4em}
\subfloat{%
\includegraphics[height=37mm,width=53mm]
{4,2prob.png}%
}
\caption{Relic Neutrino Helicity-Flip Probability in Molleweide view, for the same cases shown in FIG. \ref{rotation A}. 
(first row) the JF12 coherent Galactic magnetic field and (second row) the full coherent plus random field.
}
\label{probability flip}
\end{figure*}  



\begin{figure*}[!htp]
\hspace*{-5.87em}
\subfloat[\label{sfig:testa}]{%
\includegraphics[height=35mm,width=59mm]{AngPS1.1.png}%
}\hspace*{-2.0em}
\subfloat[\label{sfig:testa}]{%
\includegraphics[height=35mm,width=59mm]{AngPS1.2.png}%
}\hspace*{-2.0em}
\subfloat[\label{sfig:testa}]{%
\includegraphics[height=35mm,width=59mm]{Angps1.3.png}%
}\hspace*{-2.0em}
\subfloat[\label{sfig:testa}]{%
\includegraphics[height=35mm,width=59mm]{AngPS1.4.png}%
}\hfill
\hspace*{-5.87em}
\subfloat[\label{sfig:testa}]{%
\includegraphics[height=35mm,width=59mm]{angps2.1.png}%
}\hspace*{-2.0em}
\subfloat[\label{sfig:testa}]{%
\includegraphics[height=35mm,width=59mm]{AngPS2.2.png}%
}\hspace*{-2.0em}
\subfloat[\label{sfig:testa}]{%
\includegraphics[height=35mm,width=59mm]{AngPS3.2.png}%
}\hspace*{-2.0em}
\subfloat[\label{sfig:testa}]{%
\includegraphics[height=35mm,width=59mm]{AngPS4.2.png}%
}
\caption{Relic Neutrino Helicity-Flip Probability angular power spectrum for scales of GEMMA Reactor experiment bounds $\mu_{\nu}^{GEMMA}\approx2.9\times10^{-11}\mu_{B}$, 0.1$\mu_{\nu}^{GEMMA}$, 0.01$\mu_{\nu}^{GEMMA}$, 0.001$\mu_{\nu}^{GEMMA}$(a)-(d)  simulation results in JF-12 Coherent Galactic Magnetic Filed Model, (e)-(f) simulation results in JF12 Coherent Galactic Magnetic Filed Plus random components Model; magnetic moment applied are $10^{-1}$ smaller in sequence for each model}
\label{power}
\end{figure*}  

\begin{figure}[h]
\hspace{-2.87em}
    %\centering
    \includegraphics[width=9.5cm,height=6.8cm]{good.png}
    \caption{The sky-averaged helicity flip probability in the JF12 Coherent and Coherent + Random Magnetic Field models, for $m_{\nu}$=0.1 eV
    (green and red data points, respectively, with RMS spread on the sky shown as error bar), overlaid on the prediction using Eq. \ref{eq:BP}. For $m_{\mu}$ $\gtrsim$ $10^{-4}$ times the GEMMA limit, the oscillation in the Baym-Peng model is so rapidly varying, we show only the average value $\frac{1}{2}$.}
    \label{Baym and I}
\end{figure}





\section{Comparison with earlier work}

The Baym-Peng treatment~\cite{Baym:2021ksj} took the
Galactic magnetic field to be entirely random with magnitude $B_{g}$ $\simeq$ 10 $\mu$Gauss, with a coherence length $\Lambda_{g}$ $\simeq$ kpc within a mean crossing distance of the galaxy of $l_{g}$$\simeq$ 16 kpc. In a uniform random field the spin vector of a neutrino undergoes a random walk in the field during propagation, generating a mean-square spin rotation \cite{Baym:2021ksj}
\begin{equation}
\label{eq:BP}
    <\theta^2>\simeq (2\mu_{\nu}B_{g}\frac{\Lambda_{g}}{v})^{2} \frac{l_{g}}{\Lambda_{g}}
\end{equation}
For ($\mu_{\nu}$$B_{g}$$\frac{\Lambda_{g}}{v}$) of order unity and sufficiently large magnetic moment, this predicts cumulative rotation angles large enough for significant $\nu$ helicity-flip.  For $m_\nu = 0.1$ eV,  their predicted spin rotation is generically large and the flip probability is of order one, as long as $\mu_{\nu} \gtrsim 2 \times 10^{-6}$ $\mu^{GEMMA}_{\nu}$, as shown in FIG. \ref{Baym and I}.  By contrast, we find the maximum sky-averaged flip probability is about 20\%, and becomes very small for $\mu_{\nu} \leq 10^{-3}$ $\mu^{GEMMA}_{\nu}$.

Much of the discrepancy is due to the unrealistic field parameters assumed in~\cite{Baym:2021ksj}.  Averaging over the 20 kpc sphere centered on the Galactic center in which the Jansson and Farrar (2012) model is given, the average path length to Earth is 24.4 kpc and the average value of $|B_\perp|$ is $ 0.104 \,\mu$G. (Histograms of $|B_\perp|$ and $l_g$ for all the lines of sight are shown in Fig.~\ref{fig:GMFhists}.) Moreover, the coherence length of the random component of the GMF is typically estimated to be  10 to 100 pc; for our calculations we took it to be 30 pc.  Using our more realistic magnetic field properties in the Baym-Peng expression Eq.~\eqref{eq:BP}, decreases the estimated $<\theta^2>$ by a factor $ \sim 5 \times 10^{-6}$.  

\begin{figure*}[htp!]
\hspace*{-2.2em}
\subfloat[\label{sfig:testa}]{%
\includegraphics[height=55mm,width=75mm]{PropagationHist.png}%
}\hspace*{-0.7em}
\subfloat[\label{sfig:testa}]{%
\includegraphics[height=55mm,width=75mm]{histperp.png}%
}

\caption{Histograms of (a) neutrino propagation length $l_g$  and (b) $|B_{\perp}|$ for all the lines of sight in JF 12 coherent plus random magnetic field model.}
\label{fig:GMFhists}
\end{figure*}  



\section{Effects of helicity flip on the detection of Relic Neutrinos through Inverse Tritium Beta Decay}
A proposed experimental method for detecting relic neutrinos is based on neutrino capture through the inverse tritium beta decay process \cite{PhysRev.128.1457}
$\nu_{j}$ +${}^{3}H$ $\rightarrow$ ${}^{3}He$ $+$ $e^{-}$.

The total cross-section multiplied by the neutrino velocity for capturing $\nu_{e}$ \cite{Peng:2022nvi} is
\begin{equation}
\begin{aligned}
\sigma^{h}(E_{\nu})v= {} &
\frac{G_{F}^2}{2\pi}|V_{ud}|^{2} F(Z,E_{e})\frac{m_{{}^{3} He}}{m_{{}^{3}H}} E_{e}P_{e}\times \\
&(\langle f_{F} \rangle^{2} + (g_{A}/g_{V})^{2} \langle g_{GT} \rangle^{2} ) A^{h},
\end{aligned}
\end{equation}
where $v$ is the neutrino velocity, $V_{ud}$ is the up-down quark element of the CKM matrix, $F(Z,E_{e})$ is the Fermi
Coulomb correction, and
$h$ labels the helicity. We define a polarization-averaged cross-section value
$\overline{\sigma}$ = $\sum_{j=1,2,3}$$\sigma_{j}({\pm})$ $v_{j}$
$\simeq$ 3.834 $\times$$10^{-45}$ cm$^{2}$ for later use.  

The helicity-dependent factor $A^h$ is: 
\begin{equation}
\label{A:factor}
A^{\pm}=1\mp\beta,
\end{equation}
$+$ for right-hand helicity eigenstate, $-$ for left-hand helicity eigenstate;
$\beta$=$v/c$, where $v$ is the $\nu$ velocity. 
For massive neutrinos such that $\beta$ $\ll$ 1, the cross-sections for different helicity states are thus practically speaking indistinguishable and the impact of spin rotation by the GMF is not observable in the total rate. 
A significant impact of neutrino helicity flip in ITBD requires the cross-section to be sensitive to helicity.
%Since $m_{\nu_{1}}$ can be arbitrarily small in the mass normal hierarchy, following the $\nu$ velocity distribution, its contribution to $\hat{s}_{\nu}\cdot\vec{v}_{\nu}$ modulation is $\mathcal{O}$(1).


References \cite{Lisanti:2014pqa}\cite{Tully:2021key} discuss the detection of a cosmic neutrino background arrival direction anisotropy. Such an anisotropy can appear due to density perturbations as for the CMB, or in our case, due to anisotropic impacts of helicity rotation in the GMF.  Equation (4.3) of Ref. \cite{Tully:2021key} gives the differential cross section depending on the direction of ${}^{3}H$ polarization, $\hat{s}_{H}$, the direction of the outgoing electron, the $\nu$ velocity, and $\nu$ spin direction. Neglecting terms giving small contributions:
\begin{equation}
\label{eq:diffXcn}
\frac{d\sigma(\hat{s}_{H},\hat{v}_e)}{d\Omega_{e}}\approx\frac{\overline{\sigma}}{4\pi} [(1-\hat{s}_{\nu}\cdot \vec{v}_{\nu})+ B\, \hat{s}_{H}\cdot(\vec{v}_{\nu}-\hat{s}_{\nu})],
\end{equation}
in units with $c=1$, where $B$ $=$ $\frac{2g_{A}(1+g_{A})}{1+3g^{2}_{A}}$ $\simeq$ 0.99 is an asymmetry parameter from the amplitude calculation of $\nu$ to be captured on the neutron, and term (1-$\hat{s}_{\nu}$$\cdot$$\vec{v}_{\nu})$ is identical to the $A^{\pm}$ in Eq.(\ref{A:factor}), the expression of helicity here is in terms of spin operators. This expression must be summed over all neutrinos contributing to the measurement, i.e., summed over all arrival directions in the sky. %represented in the ITBD fractional difference plots.



For unpolarized ${}^{3}H$, only the $<\hat{s}_{\nu}$$\cdot$$\vec{v}_{\nu}>$ term is of interest because the second term averages to zero. One might think since $v_\nu \sim p_0/m$ for non-relativistic neutrinos, that a lower neutrino mass would lead to a larger $<v>$ and hence larger sensitivity to the spin rotation via $<\hat{s}_{\nu}$$\cdot$$\vec{v}_{\nu}>$.  However, the spin rotation is proportional to the time spent traversing the GMF, and is thus inversely proportional to the mass, so that $<\hat{s}_{\nu}$$\cdot$$\vec{v}_{\nu}>$ is independent of neutrino mass. We show in FIG. \ref{masses} the variation of $<\hat{s}_{\nu}$$\cdot$$\vec{v}_{\nu}>$ as $\mu_{\nu}$ ranges from standard model magnetic moment to experimental upper bound taking $m_{\nu}$= 0.1eV.
%FIG \ref{masses} shows the importance of the neutrino mass on $<\hat{s} \cdot v>$, as a function of the neutrino magnetic moment in units of the GEMMA upper limit. % Lower mass $\nu$ candidates show stronger effects on the differential amplitude results, as $\mu_{\nu}$ approaches the experimental upper bound from the GEMMA experiment, effects are decreased as depicted from logarithm scale plots.

Neutrino spin rotation not only impacts the total event rate but, as can be seen in Eq.~\eqref{eq:diffXcn}, it generates a sensitivity of the event rate to the direction of polarization of the tritium target.
Using the results from our simulations,
we calculate the fractional difference of ITBD event rate for $\hat{s}_{H}$ pointing to a given Galactic coordinate, summing over the neutrino arrival directions, due to the term 
$\hat{s}_{H}\cdot(\vec{v}_{\nu}-\hat{s}_{\nu})$. Repeating this for each direction $ $$\hat{s}_{H}$$ $ on the sky, and dividing by the average at the given $\mu_{\nu}$, yields the ``dipole anisotropy" maps for the ITBD signal fractional difference as a function of its polarization direction, shown in Fig. \ref{anisotropy} taking $m_{\nu} = 0.1$ eV, for the suite of values of $\mu_\nu$ and assuming the flux of cosmic background neutrinos is isotropic.  

\begin{figure}[htp!]
    \hspace{-2.65em}
    \includegraphics[width=9.5cm,height=7.2cm]{s.v.re.png}
    \caption{ Arrival-direction-averaged value of $<$$\hat{s}_{\nu}$$\cdot$$\vec{v}_{\nu}$$>$ versus $\mu_{\nu}$ ranging from the standard model magnetic moment $\mu_{\nu}^{SM}$ $\sim$ $10^{-18}$ $\mu_{B}$ to the experimental upper bound $\mu^{GEMMA}_{\nu}$$\approx$$2.9\times10^{-11}$$\mu_{B}$, in the JF12 coherent GMF. As shown in the text, $<$$\hat{s}_{\nu}$$\cdot$$\vec{v}_{\nu}$$>$ is independent of neutrino mass for non-relativistic neutrinos.}
    \label{masses}
\end{figure}

\begin{figure*}[htp!]
\hspace*{-6.2em}
\subfloat[\label{sfig:testa}]{%
\includegraphics[height=39mm,width=55mm]{DF1.png}%
}\hspace*{-0.7em}
\subfloat[\label{sfig:testa}]{%
\includegraphics[height=39mm,width=55mm]{DF2.png}%
}\hspace*{-0.4em}
\subfloat[\label{sfig:testa}]{%
\includegraphics[height=39mm,width=55mm]{DF3.png}%
}\hspace*{-0.4em}
\subfloat[\label{sfig:testa}]{%
\includegraphics[height=39mm,width=55mm]{DF4.png}%
}

\caption{The
ITBD event rate as a function of the tritium polarization direction, expressed as the fractional difference relative to the mean, in Galactic coordinates and for $m_{\nu} = 0.1$eV and $\nu$ spin rotation in JF12 Coherent Galactic Magnetic Filed Model, for $\mu_{\nu}$= $\mu^{GEMMA}_{\nu}$, $10^{-1}$ $\times$ $\mu^{GEMMA}_{\nu}$, $10^{-2}$ $\times$ $\mu^{GEMMA}_{\nu}$ and $10^{-3}$ $\times$ $\mu^{GEMMA}_{\nu}$ in sequence from left to right}
\label{anisotropy}
\end{figure*}  



\section{ Spin Rotation of solar neutrinos }
In this section, we comment on another possible effect of a neutrino magnetic moment large enough to produce observable effects.  Consider a solar neutrino traveling to Earth, at mass scale $\sim$ $10^{-2}$ eV and energy $\sim$ 100 keV (proton-proton reaction case).  Helioseismological data provide an upper bound on the field strength in the convective zone of the Sun, between 0.71-1 $R_{\odot}$, of $B_{CZ} \lesssim 300$ kG. \cite{Antia:2002ti}.  With these values, even a magnetic moment $\sim$ 2.9$\times$$10^{-11}$$\mu_{B}$ saturating the GEMMA upper bound value yields a negligible spin rotation 
in the convective zone.  In the radiative zone, 0.2$\times$$R_{\odot}$  to  0.7$\times$$R_{\odot}$, an upper bound to  $B_{RZ}$ of  $\approx$ 7 MG is provided by estimation in the axisymmetric toroidal model \cite{Friedland:2002is}.   Here, with a GEMMA or smaller value of the magnetic moment, the magnetic field strength could modify the solar neutrino helicity. Such spin-precession would in general result in a reduced solar neutrino flux observed and induce periodic modulations in solar neutrino signals over the course of a year as the angle of propagation of solar neutrinos reaching a detector on Earth changes with respect to the solar magnetic field.  
The idea of a relic magnetic field with small effects of solar evolution in the radiative zone was proposed in previous research \cite{Friedland:2002is} in an axis-symmetric toroidal field. Following the standard solar model BP2000, eigenmodes of the magnetic field with lifetime $\approx$ 5 Gyr do exist with an upper bound of an order of $\simeq$ MGauss. 

Unlike the magnetic fields, gravitational inhomogeneities affect both Dirac $\nu$ and Majorana $\nu$ helicity. Effects of solar neutrino spin rotation caused by gravitational fields of the sun itself are also estimated by Baym and Peng \cite{Baym:2021ksj}, 
who show that estimated spin rotation is $\simeq$ $10^{-6}$ radians for $\nu$ with $\beta$ = $v_{\nu}$/c $\simeq$ 1, a smaller effect compared to the possible rotation caused by $\mu_{\nu}$ $\times$ $\vec{B}_{\odot}$ discussed above. 

Future detectors and event constructions with better precision will possibly be able to test such predictions. This could result in a better constraint on magnetic moment coupling with the magnetic field in the radiative zone, and at the same time contribute to our understanding of the solar magnetic field structure.




\section{\label{sec:level6}Discussion}

We have explored the effects of a neutrino magnetic moment on the evolution of the helicity of relic neutrinos propagating through the Galaxy to a detector on Earth, by tracking  through a realistic model of the Galactic magnetic field with numerical simulations. 
Our results show that for magnetic moments consistent with experimental bounds and even several orders of magnitude smaller, a relic neutrinos' helicity has a significant rotation, which depends on the arrival direction.
Using these results, we studied the effects of a neutrino magnetic moment on relic neutrino detection through inverse tritium beta decay and calculated the dipole anisotropy in the neutrino detection rate as a function of tritium polarization direction.  We find that  the absolute reduction in the detection rate due to spin rotation is small,
on account of the low velocity of cosmic neutrinos. However, the fractional directional anisotropy is up to a factor-10 larger.  Many difficulties still remain, however, in reaching the required levels of sensitivity or indeed even detecting a relic neutrino signal.

Though neutrino helicity modification due to the astrophysical magnetic field is still not measurable by current experiments, the topic itself could eventually be a significant probe to complete our understanding of the astrophysical magnetic field, cosmic neutrino background studies, constraints for non-zero neutrino magnetic moment value, and whether the neutrino mass is Dirac or Majorana since only Dirac neutrinos experience spin rotation in a magnetic field.



\begin{acknowledgments}
We appreciate helpful discussions with G. Baym, C. Tully, A. Gruzinov, and X. Xu.
The research of GRF is supported by National Science Foundation Grant NSF-PHY-2013199 and by the Simons Foundation. 

\end{acknowledgments}

\bibliography{NeutrinoMS} Produces the bibliography via BibTeX.

\end{document}
%
% ****** End of file apssamp.tex ******
