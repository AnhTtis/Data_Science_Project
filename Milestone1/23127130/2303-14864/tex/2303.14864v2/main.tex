%%%%%%%%%%%%%%%%%%%%%%%%%%%%%%%%%%%%%%%%%%%%%%%%%%%%%%%%%%%%%%%%%%%%%
%%     %%
%% Please do not use \input{...} to include other tex files. %%
%% Submit your LaTeX manuscript as one.tex document. %%
%%     %%
%% All additional figures and files should be attached %%
%% separately and not embedded in the \TeX\ document itself. %%
%%     %%
%%%%%%%%%%%%%%%%%%%%%%%%%%%%%%%%%%%%%%%%%%%%%%%%%%%%%%%%%%%%%%%%%%%%%

%%\documentclass[referee,sn-basic]{sn-jnl}% referee option is meant for double line spacing

%%=======================================================%%
%% to print line numbers in the margin use line no option %%
%%=======================================================%%

%%\documentclass[lineno,sn-basic]{sn-jnl}% Basic Springer Nature Reference Style/Chemistry Reference Style

%%======================================================%%
%% to compile with pdflatex/xelatex use pdflatex option %%
%%======================================================%%

%\documentclass[pdflatex,sn-basic]{sn-jnl}% Basic Springer Nature Reference Style/Chemistry Reference Style

%\documentclass[sn-basic]{sn-jnl}% Basic Springer Nature Reference Style/Chemistry Reference Style
%\documentclass[sn-mathphys]{sn-jnl}% Math and Physical Sciences Reference Style
%%\documentclass[sn-aps]{sn-jnl}% American Physical Society (APS) Reference Style
%%\documentclass[sn-vancouver]{sn-jnl}% Vancouver Reference Style
%%\documentclass[sn-apa]{sn-jnl}% APA Reference Style
%%\documentclass[sn-chicago]{sn-jnl}% Chicago-based Humanities Reference Style
%%\documentclass[sn-standardnature]{sn-jnl}% Standard Nature Portfolio Reference Style
%%\documentclass[default]{sn-jnl}% Default
\documentclass[pdflatex, iicol ]{sn-jnl}% Default with double column layout

\usepackage[mathlines]{lineno}
\usepackage[numbers,sort&compress]{natbib}%
%\usepackage[superscript,biblabel]{cite}
%\bibliographystyle{unsrt}
\bibliographystyle{naturemag}
%\bibliographystyle{plain}

%%%%%=============================================================================%%%%
%%%% Remarks: This template is provided to aid authors with the preparation
%%%% of original research articles intended for submission to journals published 
%%%% by Springer Nature. The guidance has been prepared in partnership with 
%%%% production teams to conform to Springer Nature technical requirements. 
%%%% Editorial and presentation requirements differ among journal portfolios and 
%%%% research disciplines. You may find sections in this template are irrelevant 
%%%% to your work and are empowered to omit any such section if allowed by the 
%%%% journal you intend to submit to. The submission guidelines and policies 
%%%% of the journal take precedence. A detailed User Manual is available in the 
%%%% template package for technical guidance.
%%%%%=============================================================================%%%%

\jyear{2023}%

%% as per the requirement new theorem styles can be included as shown below
%\theoremstyle{thmstyleone}%
%\newtheorem{theorem}{Theorem}% meant for continuous numbers
%%\newtheorem{theorem}{Theorem}[section]% meant for sectionwise numbers
%% optional argument [theorem] produces theorem numbering sequence instead of independent numbers for Proposition
%\newtheorem{proposition}[theorem]{Proposition}% 
%%\newtheorem{proposition}{Proposition}% to get separate numbers for theorem and proposition etc.
%\theoremstyle{thmstyletwo}%
%\newtheorem{example}{Example}%
%newtheorem{remark}{Remark}%
%\theoremstyle{thmstylethree}%
%\newtheorem{definition}{Definition}%

%\raggedbottom
%%\unnumbered% uncomment this for unnumbered level heads
\usepackage{graphics,epsfig,amsfonts,amssymb,amsmath,color}

\newcommand{\Jesus}[1]{\textcolor{red}{\fbox{} {\sl#1}}}

\begin{document}

\title[CMOS Variability]{Bounds to electron spin qubit variability for scalable CMOS architectures}

%%=============================================================%%
%% Prefix	-> \pfx{Dr}
%% GivenName	-> \fnm{Joergen W.}
%% Particle	-> \spfx{van der} -> surname prefix
%% FamilyName	-> \sur{Ploeg}
%% Suffix	-> \sfx{IV}
%% NatureName	-> \tanm{Poet Laureate} -> Title after name
%% Degrees	-> \dgr{MSc, PhD}
%% \author*[1,2]{\pfx{Dr} \fnm{Joergen W.} \spfx{van der} \sur{Ploeg} \sfx{IV} \tanm{Poet Laureate} 
%%  \dgr{MSc, PhD}}\email{iauthor@gmail.com}
%%=============================================================%%

\author*[1]{Jesús D. Cifuentes}\email{j.cifuentes\_pardo@unsw.edu.au}
\author[1,2]{Tuomo Tanttu}%\email{t.tanttu@unsw.edu.au}
\author[1,2]{Will Gilbert}%\email{w.gilbert@unsw.edu.au}
\author[1]{Jonathan Y. Huang}%\email{yue.huang6@unsw.edu.au}
\author[1,2]{Ensar Vahapoglu}%\email{e.vahapoglu@unsw.edu.au}
\author[1]{Ross C. C. Leon}%\email{rossccleon@gmail.com}
\author[1]{Santiago Serrano}%\email{s.serrano@unsw.edu.au}
\author[1]{Dennis Otter}%\email{d.otter@unsw.edu.au} %Check this one 
\author[1]{Daniel Dunmore}%\email{d.dunmore@student.unsw.edu.au}
\author[1]{Philip Y. Mai}%\email{p.mai@unsw.edu.au}
\author[1,3]{Fr\'ed\'eric Schlattner}%\email{frederic.schlattner@gmail.com}
\author[1]{MengKe Feng}%\email{mengke.feng@unsw.edu.au}
\author[4]{Kohei Itoh}%\email{kitoh@appi.keio.ac.jp}
\author[5]{Nikolay Abrosimov}%\email{nikolay.abrosimov@ikz-berlin.de}
\author[6]{Hans-Joachim Pohl}%\email{pohl.vitcon@t-online.de}
\author[7]{Michael Thewalt}%\email{michael\_thewalt@sfu.ca}
\author[1,2]{Arne Laucht}%\email{a.laucht@unsw.edu.au}
\author[1,2]{Chih Hwan Yang}%\email{henry.yang@unsw.edu.au}
\author[1,2]{Christopher C. Escott}%\email{c.escott@unsw.edu.au}
\author[1,2]{Wee Han Lim}%\email{wee.lim@unsw.edu.au}
\author[1,2]{Fay E. Hudson}%\email{f.hudson@unsw.edu.au}
\author[8]{Rajib Rahman}%\email{rajib.rahman@unsw.edu.au}
\author[1,2]{Andrew S. Dzurak}%\email{a.dzurak@unsw.edu.au}
\author*[1,2]{Andre Saraiva}\email{a.saraiva@unsw.edu.au}


\affil[1]{\orgdiv{School of Electrical Engineering and Telecommunications}, \orgname{University of New South Wales}, \orgaddress{\postcode{NSW 2052}, \city{Sydney}, \state{NSW}, \country{Australia}}}

\affil[2]{\orgdiv{Diraq}, \orgaddress{\city{Sydney}, \state{NSW}, \country{Australia}}}


\affil[3]{\orgdiv{Solid State Physics Laboratory, Department of Physics}, \orgname{ETH Zurich}, \orgaddress{\postcode{8093}, \country{Switzerland}}}

\affil[4]{\orgdiv{School of Fundamental Science and Technology}, \orgname{Keio University}, \orgaddress{ \state{Yokohama}, \country{Japan}}}

\affil[5]{\orgdiv{Leibniz-Institut f\"{u}r Kristallz\"{u}chtung}, \orgaddress{\postcode{12489}, \state{Berlin}, \country{Germany}}}


\affil[6]{\orgdiv{VITCON Projectconsult GmbH}, \orgaddress{\postcode{07745}, \state{Jena}, \country{Germany}}}

\affil[7]{\orgdiv{Department of Physics}, \orgname{Simon Fraser University}, \orgaddress{ \postcode{V5A 1S6}, \state{British Columbia}, \country{Canada}}}

\affil[8]{\orgdiv{School of Physics}, \orgname{University of New South Wales}, \orgaddress{\postcode{NSW 2052}, \country{Australia}}}

%\linenumbers



%%==================================%%
%% sample for unstructured abstract %%
%%==================================%%

\abstract{Spins of electrons in CMOS quantum dots combine exquisite quantum properties and scalable fabrication. In the age of quantum technology, however, the metrics that crowned Si/SiO$_2$ as the microelectronics standard need to be reassessed with respect to their impact upon qubit performance. We chart spin qubit variability due to the unavoidable atomic-scale roughness of the Si/SiO$_2$ interface, compiling experiments across 12 devices, and develop theoretical tools to analyse these results. Atomistic tight binding and path integral Monte Carlo methods are adapted to describe fluctuations in devices with millions of atoms by directly analysing their wavefunctions and electron paths instead of their energy spectra. We correlate the effect of roughness with the variability in qubit position, deformation, valley splitting, valley phase, spin-orbit coupling and exchange coupling. These variabilities are found to be bounded, and they lie within the tolerances for scalable architectures for quantum computing as long as robust control methods are incorporated.}

%%================================%%
\keywords{Variability, spin qubits, oxide roughness, scalability}

%%\pacs[JEL Classification]{D8, H51}

%%\pacs[MSC Classification]{35A01, 65L10, 65L12, 65L20, 65L70}

\maketitle



%% -- Figures 1
\begin{figure*}[th]%
\centering
\includegraphics[width=\textwidth]{fig1.pdf}
\caption{ \textbf{$\vert$ Modeling spin qubits.} \textbf{a}, Example scaled architecture of a 49 qubit device. \textbf{b}, The quantum dots are formed below the computer generated rough surface. \textbf{c}, Model of the three-dot devices measured in this paper. The metallic gates are coloured by their order of deposition in different layers. \textbf{d-g}, Comparison between device cross sections in TEM images and computer model. \textbf{d}, TEM image of device T1, showing a cross section of the device located approximately at the position of the violet rectangular region in \textbf{c}. \textbf{e}, TEM of device T2 with a focus on the silicon oxide interface. We highlight in \textbf{d} a square region with the same size. \textbf{f}, Cross-section of model at the green rectangular region in \textbf{c}. \textbf{g}, Atomistic simulation showing the electronic wavefunction of a quantum dot below rough Si-SiO$_2$. \textbf{h}, Average power spectral density (PSD) of the Si/SiO$_2$ interface comparing the interface from transmission electron microscopy (TEM) images of device T1 (blue) and T2 \textbf{d} (cyan), and the computer generated surface in \textbf{a}. \textbf{i}, Average RMS of segments of length $\lambda$ for the random surface generated numerically and for device T1 to T6 (see also Extended Data Fig.~\ref{figA2}). }\label{fig1}
\end{figure*}
%% -- Figures 1


The interface between silicon and its oxide permeates most of the technology that enabled the digital era, and as such it is one of the most studied materials in human history. The practical process engineering advantages of silicon dioxide contrast with the complex chemistry of this material, and the decades of research that has underpinned the development of the CMOS industry. As we approach a new era of quantum technology, this know-how is largely considered a major advantage for technologies such as CMOS spin qubits~\cite{Maurand2016}.The similarity of CMOS spin qubits to MOSFET transistors in materials, design, and fabrication enables the integration of manufacturing techniques exclusive to semiconductor foundries onto the scaling of quantum processors. 


Despite a relatively late start~\cite{Veldhorst2014}, the performance of silicon qubits has led to fidelity levels comparable with more well-established quantum technologies like superconducting or ion-trap qubits~\cite{noiri_fast_2022, mills_two-qubit_2022}. The recent demonstration of repeatable high fidelity two-qubit operations across three nominally identical CMOS devices~\cite{tanttu2023} signals the beginning of an age focused on extensive repeatability of the high performance to achieve scaled architectures. However, despite the improvements in gate uniformity demonstrated by the integration of foundry-level manufacturing techniques~\cite{zwerver_qubits_2022}, new concerns are stirred by the fragility of spins to effects that are ignored in classical transistor technology, such as variations in spin-orbit coupling (impacting one-qubit frequencies), exchange interaction (two-qubit frequencies) and valley splitting (nearest excitation energy).



%has achieved considerable improvements two-qubit devices, pushing up gate fidelities (>99\%)~\cite{tanttu2023}, and operation temperatures (>1K)~\cite{Yang2020} to levels comparable with other semiconductor quantum technologies. \Jesus{ New concerns, however, are stirred by the fragility of spins to insofar ignored effects, such as variations in spin-orbit coupling (one-qubit frequencies), exchange interaction (two-qubit frequencies) and valley splitting (first excited state).}

%\Jesus{Within 9 years since the measurement of the first gate-based spin qubit in silicon~\cite{Veldhorst2014}, CMOS has achieved considerable improvements two-qubit devices, pushing up gate fidelities (>99\%)~\cite{tanttu2023}, and operation temperatures (>1K)~\cite{Yang2020} to levels comparable with other semiconductor quantum technologies~\cite{noiri_fast_2022, mills_two-qubit_2022, camenzind_hole_2022, petit_high-fidelity_2020}. The recent demonstration of repeatable high fidelity two-qubit operations across three identical CMOS devices~\cite{tanttu2023}, starts portraying that the age of "hero devices" has ended, and the aim now is for extensive repeatability of the milestones achieved in scaling architectures.} 

%they including high gate fidelities [Citation], and operation temperatures [Citation], with new developments heading towards scalable global control protocols [Citation], and improved connectivity [Citation].


%Unique levels of miniaturization, integration and yield are inherited from transistors, and they will continue to be benefitted by on-going developments in semiconductor manufacturing. 


 It is only by understanding this variability quantitatively that it is possible to develop a sensible scalable quantum processor architecture. Enabling the breakthrough applications of quantum computation requires millions of qubits to perform error correction~\cite{gidney_how_2021,beverland_assessing_2022}, and it is infeasible to address all of these qubits with individualised pulses catering to their variable parameters. Instead, this variability must be embraced and corrected through a combination of some on-chip electronics and robust quantum control pulses that will be shared among several qubits.

Recent advances in CMOS quantum dot fabrication allowed for sufficient yield to create a number of small-scale quantum processing units and measure their variability. This work combines measurements of 12 qubits across 6 different CMOS quantum dot devices, transmission electron microscopy (TEM) images of cross-sectional cuts of 6 other quantum dot devices and theoretical analysis of quantum properties of electrons in simulated arrays of 49 quantum dots (see Fig.~\ref{fig1}\textbf{a} and \textbf{b}). All devices were fabricated with geometrically identical designs (see structure depicted in Fig.~\ref{fig1}\textbf{c}) and differ only in material stack compositions and spin control methods (Table~\ref{tab:Devices}).


%%% -- Table 1
\begin{table*}[bt]
\centering

\begin{tabular}{|c|c|c|c|c|c|}
\hline
\textbf{Device} &
 \textbf{Dots} &
 \textbf{Configuration} &
 \textbf{Driving} &
 \textbf{Vector Magnet} &
 \textbf{Gate Material } \\ \hline
\textbf{A \includegraphics[height=3mm]{Devices/A11.png}\includegraphics[height=3mm]{Devices/A31.png}\includegraphics[height=3mm]{Devices/A13.png}} & P1, P2 & {\color{olive} (1,1)},{\color{red} (3,1)},{\color{blue} (1,3)} & Antenna & Yes & Pd/Ti + ALD \\ \hline
\textbf{B \includegraphics[height=2mm]{Devices/B.png}\hspace{6mm}} & P2, P3 & (3,1) & Antenna & No & Pd/Ti + ALD\\ \hline
\textbf{C \includegraphics[height=2.5mm]{Devices/C.png}\hspace{5mm}} & P2, P3 & (3,1) & Antenna & No & Pd/Ti + ALD \\ \hline
\textbf{D \includegraphics[height=3mm]{Devices/D.png}\hspace{5mm}} & P1, P2 & (1,3) & Dielectric Resonator & No & Pd/Ti + ALD \\ \hline
\textbf{E \includegraphics[height=2.5mm]{Devices/E.png}\hspace{5mm}} & P1, P2 & (3,1) & Antenna & Yes & Al \\ \hline
\textbf{F \includegraphics[height=2.5mm]{Devices/F.png}\hspace{5mm}} & P2 & 1 electron & Antenna & No & Al \\
\hline

\end{tabular}

\caption{
\label{tab:Devices}\textbf{$\vert$ List of devices used in qubit measurements.} The devices are identical except by differences indicated in here. The colours in device A indicate the three electronic configurations in which qubit data was taken. They can be (1,1) - for 1 electron under gate 1 and 1 electron under gate 2, (3,1) and (1,3). The double quantum dots are formed under the gates P1, P2 or P2, P3 depending on the device. In one of the devices the qubits were driven magnetically with a dielectric resonator instead of an antenna~\cite{Vahapoglu2020,vahapoglu_coherent_2022}. A vector magnet enabled rotations of the magnetic field for the measurements in two of the devices. The last column refers to the gate material. We use a combination of palladium and atomic layer deposition (ALD) alumina for some devices and for others we form the gates with aluminium and isolate them with thermally formed alumina. Differences between these situations are discussed in~\cite{Saraiva2022}. Device F is the only one with a single dot configuration instead of double dot. We use this device only to provide additional data on the valley splitting. }
\end{table*}
%% -- Table 1

Variability may be caused by a few factors, such as strain, fabrication defects, accidental introduction of charged impurities in the oxide and so on~\cite{elsayed_low_2022, Saraiva2022, Cheng2009}. 
Ultimately, most of these sources of variability can be improved with increasingly precise fabrication -- industrial foundries focus most of their efforts in addressing these issues~\cite{zwerver_qubits_2022}. The only source of qubit variability that is inextricable to CMOS technology is the roughness of the interface between the crystalline Si and the amorphous SiO$_2$, which can be seen in the high-resolution TEM image of a device cross-section in Fig.~\ref{fig1}\textbf{d-e}.

Previous theoretical studies have modelled this interface as either single monoatomic steps~\cite{Ferdous2018, Ferdous2018a}, or as a normal distribution with a well-defined roughness amplitude~\cite{Gamble2016}. A more recent work employed this last model to investigate the variability of hole and electron spin qubit frequencies formed at the corners of Si nanowires \cite{martinez_variability_2022}. Their conclusion was that hole spin qubits are more susceptible to the effects of disorder, which has a significant impact on the electrically-driven Rabi frequencies. Here, we focus on magnetically driven electron spin qubits. Electrons have lower spin-orbit coupling, making them less susceptible to electric noise and variability. Moreover, magnetically driven spins have their Rabi frequency determined by the amplitude of the oscillatory magnetic field, which does not lead to variability due to disorder (in reality, Rabi frequencies can still have some variations due to spurious oscillatory electric fields generating a small but measurable spin-orbit drive~\cite{gilbert_-demand_2023}).

%To obtain a complete picture, we have combined data from qubits measured in our facilities with simulations of the impact of disorder on the quantum dot electronic structure.

Our methodology differs from previous approaches in a number of aspects. We validate our theoretical analysis through extensive comparison with observations from experimental results across a variety of qubit devices. The three-dimensional models used for electrostatic simulations were based on TEM and SEM images of our devices (see Fig.~\ref{fig1}\textbf{f},  and methods section). Our model of the Si/SiO2 interface is based on the roughness observed in TEM images (Fig.~\ref{fig1}\textbf{d-e})~\cite{Goodnick1985,Zhao2010,Yoshinobu1995}, allowing us to include more realistic features. By convolving these images with the expected face-centred cubic lattice of monocrystalline silicon we can mathematically discern the interface and analyse the roughness at different scales~\cite{Goodnick1985,Zhao2010}, quantified through its power spectral density (PSD) as a function of the in-plane correlation length scale $\lambda$~\cite{Yoshinobu1995,Persson2014}. Our work incorporates a theoretical study of the realistic scale-dependent fractal structure of the roughness \cite{Yoshinobu1995} and the development of theoretical tools capable of capturing this multiscale physics in a realistic device model (Fig.~\ref{fig1}\textbf{g}). 

Combining multiple TEM images, we obtained in Fig.~\ref{fig1}\textbf{h} a consistent roughness pattern characteristic of a fractal scaling down to the silicon lattice paremeter of the form $\text{PSD}^{\text{1D}}(\lambda) \propto \left( \frac{2\pi}{\lambda} \right)^{-1-2H}$. We estimate a Hurst exponent of $H$=0.28 \cite{Jacobs2017a}. The root-mean-square roughness also scales up with the lateral region $\lambda$ as $\text{RMS} \left(\lambda\right) \propto \left(\frac{2\pi}{\lambda}\right)^{-H}$. As shown in Fig.~\ref{fig1}\textbf{i} this roughness pattern is consistent across all devices measured, and extends up to half a micrometre (details in Extended Data Fig.\ref{figA1} and methods section). We note that these levels of roughness are typical for industry-standard interfaces\cite{Intel2019}. Our computer-generated interface in Figs.~\ref{fig1}\textbf{b,f}~and~\textbf{g} was also designed to mimic these features. 

A direct conclusion from Fig.~\ref{fig1}\textbf{h-i} is that the size of the dots (approximately 10nm for all devices studied) and the separation between dots (approximately 50 nm) will have a large impact on the qubit exposure to surface roughness, and that the interface distortions within a quantum dot are smaller than those between neighbouring dots (approximately one monolayer RMS within a dot compared to 2 monolayers between dots). 

%\textbf{However, the variability distribution each one of the qubit parameters will also depend on other factors, such as, its microscopic origin, its electrical tunability or its susceptibility to disorder. We will analyse these characteristics for qubit parameters and show adequate strategies to cope or compensate the differences for scalable quantum computing.}

%Previous purely theoretical studies have modelled this interface as either single monoatomic steps~\cite{Ferdous2018, Ferdous2018a}, or as a normal distribution with a well-defined roughness amplitude~\cite{martinez_variability_2022, Gamble2016}. Our work incorporates a theoretical study of the realistic scale-dependent fractal structure of the roughness, the development of theoretical tools capable of capturing this multiscale physics in a realistic device model and a validation of this analysis through extensive comparison with data from a number of device measurements.



 %% -- Figures 2
\begin{figure*}[h]%
\centering
\includegraphics[width=1\textwidth]{fig2.pdf}
\caption{\textbf{$\vert$ Quantum dot variability.} \textbf{a}, We define the interface between the silicon lattice and its oxide with a simulated rough surface dividing the atomic sites between two materials. \textbf{b}, Three-dimensional visualization of a quantum dot rough interface simulated atomistically. \textbf{c}, In-plane visualization of the variability in the 7 quantum dots inside the purple rectangle in Fig.~\ref{fig1}\textbf{b}. The 5~nm diameter cyan circle is a static reference to compare the wavefunctions in different simulations. Black asterisks represent the center of each quantum dot $\langle \vec{r} \rangle =\langle \psi \vert \vec{r} \vert \psi \rangle$. \textbf{d}, Variability distribution of dot centres. \textbf{e}, Visualization of the valley oscillations parallel to the $[001]$ lattice orientation. \textbf{f}, Valley splitting distribution of the 49 quantum dots \textit{versus} electric field. We compare our results with experimental data measured in two devices. The electric fields are obtained from COMSOL simulations. \textbf{g}, Correlation between the logarithm of the valley splitting \textit{versus} the centre of the dot in the z axis. \textbf{h}, Distribution of valley phases \textit{versus} the valley splitting. For convenience, we define $\phi_v = 0^\circ$ as the point with the highest density of valley phases. The colour code represents the value of $E_z$ as in \textbf{f,g.} }\label{fig2}
\end{figure*}
%% -- Figures 2



\bmhead{Variability of quantum dot structure and excitation energy} The consistency of this roughness pattern across devices allows to theoretically forecast its impact on qubit performance using computer-generated interfaces, such as shown in Fig.~\ref{fig2}\textbf{a}. To understand how this roughness affects the quantum behaviour of electrons, it is necessary to focus on their wavefunction at the atomic scale (Fig.~\ref{fig2}\textbf{b}). We use an atomistic tight binding model of Si and SiO$_2$ (see Methods) which incorporates relativistic effects and the impact of a magnetic field, yielding eigenstates with realistic spin and valley structure. It can be used to calculate the ground state wavefunction and a few excited states. In addition, we develop techniques to extract this structure and calculate properties of the disordered quantum dot that would not be obtained with a purely spectral analysis, such as the valley phase and the spin g-tensor. The theoretical results discussed refer to 49 simulated dots arranged on a grid of 7$\times$7 on a computer-generated rough Si/SiO$_2$ interface (Fig.~\ref{fig2}\textbf{a}).

We find that the geometry explored here always leads to the successful formation of quantum dots, regardless of the local roughness profile. This is consistent with the yield of measurable quantum dots -- all devices with functional gate electrodes (as determined by their influence on the charge sensing single electron transistor) could form controllable pairs of dots. Roughness mostly alters the quantum dot effective shape and centre position (Fig.~\ref{fig2}\textbf{c}), with location of the electron departing from the potential minimum by less than 5~nm, with a standard deviation of 1.4~nm (Fig.~\ref{fig2}\textbf{d}). The dot position in the geometry studied here is highly tunable by biasing lateral gates (approximately 5~nm/V, see Extended Data Fig.\ref{figA1} and Table \ref{tab:PotentialSweeps}), so that this disturbance can be corrected. 

The excited orbital states are more impacted by the interface roughness. We are particularly interested in the first one, which for a [001] interface corresponds to valley excitation -- the conduction band valleys along $\pm z$ crystal directions are energetically favourable due to the effective mass anisotropy, and the degeneracy between these two valleys is lifted by the sharp interface. The performance of spin qubits is strongly impacted by the interface-induced valley coupling, which creates a superposition between the two valley quantum states. This superposition creates an oscillatory behaviour at the atomic scale, which can be seen in simulations in Fig.~\ref{fig2}\textbf{e}. These oscillations are known to cause variability in valley structure even for interfaces with low levels of disorder. We refer to the relative phase between valleys in this superposition as valley phase and the energy separation between the two states as valley splitting.

To have pure spin systems, valley splittings exceeding the Zeeman energy are desirable. In Fig.~\ref{fig2}\textbf{f} and \textbf{g}, we show how the valley splitting can be controlled tuning the vertical electric field $E_z$, comparing measurements in two devices and the results of the simulations. The surface roughness causes variability in valley splitting of over one order of magnitude for a fixed electric field. The full range of valley splittings spreads from tens of $\mu$eV to a few meV, compatible with observed experimental values~\cite{Yang2013,Gamble2016}.

In general, the field dependence of valley splitting is linear for small ranges of electric field variation. In the absence of back gates (such as those available in SOI devices~\cite{Bourdet2018}), the vertical electric field cannot be tuned independently of the quantum dot chemical potential, which limits the range of tunability of the electric field before a charge transition occurs.

When comparing these valley splittings to the spin splitting, we may ignore the variability in Zeeman energy (which is only of a few parts per thousand). Therefore, if we set a relatively high electric confinement ($\approx$28~meV ${\rm nm}^{-1}$ is sufficient in our simulation) and we tune the magnetic field low enough (<700~mT in our study), all the 49 qubits in the simulation will obey the condition of valley splitting larger than the Zeeman splitting. In a full scale quantum processor, this will translate to a small but finite number of quantum dots that have valley splittings clashing with the spin splitting, which must be dealt with by either changing the number of electrons in the dot~\cite{Leon2020} or discarding that dot from the processor at the firmware level.

Even with a consistently high valley splitting, qubit performance can still be impacted by variations in valley phases between neighbouring dots. The electron density will present Bloch oscillations in the $z$ direction, which are barely visible in Fig.~\ref{fig1}\textbf{f} and were enhanced in Fig.~\ref{fig2}\textbf{e} by taking the difference between the electron densities of both valley states (Methods). Notice that the $z$ oscillations have the same phase across the whole dot instead of conforming to the roughness of the oxide. This implies that the valley phase is well defined even in the presence of surface disorder. This phase has an impact on operations that involve two dots, such as electron tunnelling and exchange coupling, because it determines whether these valley oscillations interfere constructively or destructively~\cite{tariq_impact_2022, Tagliaferri2018}. Fig.~\ref{fig2}\textbf{h} shows the valley phases across the 49 simulated dots, revealing that dots with larger valley splittings (typically above 300~$\mu$eV) tend to have similar valley phases, near zero in our definition.

%% -- Figures 3
\begin{figure*}[h]%
\centering
\includegraphics[width=\textwidth]{fig3.pdf}
\caption{\textbf{$\vert$ Variability of the spin-orbit coupling.} \textbf{a-b}, Comparison between $g$-factor variability in atomistic simulations and measurements in devices A to E. \textbf{a}, Frequency difference $g_1$-$g_2$ \textit{versus} magnetic field angle. \textbf{b}, Top gate Stark shift $dg/dV$ \textit{versus} magnetic field angle. \textbf{c-d}, Distribution of Dresselhaus $\beta$ (\textbf{c}) and Rashba $\alpha$ (\textbf{d}) \textit{versus} vertical electric field $E_z$. \textbf{e-f}, Schematic table showing the sinusoidal dependence of the $g$-factors \textit{versus} in-plane magnetic angle. (\textbf{e}), follows from the anisotropy in the silicon lattice near the interface. (\textbf{f}), In an ideal flat surface, the border of silicon must end in one of the two possible sublattices A (black) or B (gray) and the interface looks different when observed from the $[110]$ and the $1\bar{1}0$ lattice orientations. In a realistic rough surface the border is a mixture of both A and B sub-lattice terminations, which explains the observed g-factor variability \textbf{g}, Distribution of qubit frequencies for two magnetic field orientations: $[110]$ and $[100]$. The bars show an estimate for the maximum gate tunability of the g-factors with the top gate. \textbf{h}, Zoom to the $[100]$ data in \textbf{g}.}\label{fig3} 
\end{figure*}
%% -- Figures 3


\bmhead{Qubit frequency variations} Spin-orbit effects lead to variability in qubit frequencies of the order of $\sim$100~peV, which appears in the form of a variable $g$ factor for the spin subject to an external magnetic field. This variation represents less than 1\% of the qubit frequency. The mean value of the ratio of Zeeman frequency and external magnetic field $f_{\rm Zeeman}/B_0 = g \mu_{\rm B}$ is 27.9~GHz/T, with the differences occurring only at the order of tens of MHz/T. This is a particularity of silicon electrons, whose spin-orbit coupling is among the smallest in all quantum dot technologies. This provides CMOS spin qubits with special protection against disorder and electric fluctuations. However, these very small g-factor variations are still important for qubit operations aimed at higher than 99.9\% fidelity, and they are directly linked to the roughness of the interface at the atomic scale. 

Interface-induced spin-orbit coupling has two flavours in Si/SiO$_2$ interfaces -- Rashba ($\alpha$) and Dresselhaus ($\beta$)~\cite{Ruskov2018}. These two can be experimentally differentiated by measuring the g-factor dependence on the in-plane magnetic field orientation $\varphi$~\cite{Tanttu2019}. The dependence is sinusoidal with the form $g\left(\phi\right)\approx g_0+\alpha+\beta\sin{\left(2\phi\right)}$, where we take $g_0$ to be the theoretical bulk g-factor $g_0=1.9935$ calculated from atomistic simulations including relativistic spin-orbit effects. The difference between the frequencies of any two qubits (Fig~\ref{fig3}.\textbf{a}), as well as the electric field dependence $dg/dV$(Fig~\ref{fig3}.\textbf{b} have the same behaviour. All 12 qubits measured in devices A to E show behaviours consistent with this description. In most cases Dresselhaus dominates both the total spin-orbit effect and its variability -- with the exception of one of the configurations in device A. The Rashba coefficient $\alpha$ is on average one order of magnitude smaller than $\beta$.

We explore theoretically this variability by extracting the g-tensor of electrons in disordered quantum dots from the eigenfunctions calculated by tight binding. The results of simulations, shown as solid lines in Figs.~\ref{fig3}\textbf{a} and \textbf{b}, are then used to extract the dependence of $\alpha$ and $\beta$ on the vertical electric field (Figs.~\ref{fig3}\textbf{c} and \textbf{d}). The Dresselhaus effect emerges from breaking the lattice inversion symmetry near an interface, which explains why it is strongly dependent on the electric field that confines the electron against the oxide. The interface-induced Rashba effect is also dependent on the electric field, but with a weaker dependence. Notice that a couple of simulations escape the overall trend. These are valley-spin degeneracies. While they can be used for fast electrical driving~\cite{Bourdet2018,Corna2018,gilbert_-demand_2023}, they significantly deviate from the target parameters for pure spin qubits. In practice it is possible to tune the valley splitting out of this regime.

We can also observe in Fig.~\ref{fig3}\textbf{c} that the Dresselhaus parameter $\beta$ is bounded between two extreme values for each electric and magnetic field. This can be understood by analysing the perfectly flat [001] interface model, which introduces a distinction between the two sublattices in the diamond structure~\cite{Ferdous2018}. Fig.~\ref{fig3}\textbf{e} shows that in this case the Dresselhaus parameter $\beta$ is maximally positive for one sublattice termination and inverts for the other -- the reason for this inversion can be understood viewing the [001] terminations in Fig.~\ref{fig3}\textbf{f}. In comparison, a rough interface will contain terminations in both sublattices, and the value of $\beta$ will then lie between these two limits (see also Fig.~\ref{figA3}).

The most common strategies explored so far for qubit addressing rely on differences in qubit frequency, which in the absence of magnetic materials occurs naturally due to these g-factor variations. The fact that the spin-orbit effect has a maximum natural spread results in frequency crowding, making it hard to address a given qubit without impacting other qubits with similar frequencies.

Instead, a more scalable pathway relies on a global microwave field acting on all qubits simultaneously~\cite{hansen_pulse_2021,hansen_implementation_2022,seedhouse_quantum_2021}. Ideally, we would either have all qubits in resonance with the global field (forming a dressed qubit) or the ability to electrically tune qubits in and out of resonance. However, the range of electric control of the g-factors is insufficient to tune them into the exact same frequency as can be seen in Fig.~\ref{fig3}\textbf{g} in the case of a magnetic field along [110]. The reduced variability enabled by pointing the field along [001], as in Fig.~\ref{fig3}\textbf{h}, also reduces the Stark shift and does not resolve the problem. Therefore, strategies for qubit control need to be designed to circumvent this variability and tolerate the natural dispersion in qubit frequencies introduced by the oxide interface.

%% -- Figures 4
\begin{figure*}[h]%
\centering
\includegraphics[width=1\textwidth]{fig4.pdf}
\caption{\textbf{$\vert$ Variability of the exchange coupling.} \textbf{a}, Diagram depicting how the dots get closer together when exchange is turned up with the J1 gate. \textbf{b}, Qubit spectra showing two qubit frequencies in device A at the $(3,1)$ configuration. Both frequencies split above $V_{\rm J} = 2.3$~V due to the exchange interactions between the dots. \textbf{c}, Representation of the 3D path of an electron moving inside a double quantum dot generated with path integral Montecarlo (PIMC). The electric potential configuration is plotted in orange and the electron density appears in the xy-plane. \textbf{d}, Exchange \textit{versus} J-gate detuning, comparing PIMC simulations in a rough interface with data from devices A and C. Colours represent the input values in the J-gate (1~V, 1.5~V, 2~V). \textbf{e}, Potential configurations for each value of J. \textbf{f} Double dot wavefunctions simulated with path integral Montecarlo (PIMC) for $V_{\rm J}=1$~V and $V_{\rm J}=2$~V. \textbf{g}, Correlation between exchange coupling and inter-dot distance in PIMC. \textbf{h}, Histogram of the exchange controllability rates $(dJ/dV_{\rm J})$ for each.}\label{fig4}
\end{figure*}
%% -- Figures 5

\bmhead{Exchange interactions under rough interfaces}

Besides the single qubit gate control, roughness also limits the homogeneity of the exchange coupling between neighbouring quantum dots. The differences in valley phase between dots, addressed in Fig.~\ref{fig2}, are the first source of variability in exchange coupling~\cite{tariq_impact_2022}. In the worst case, the valley phases would be random and the resulting exchange coupling would be impacted by the destructive interference of the valley oscillations. The probability of a completely destructive interference is, however, negligibly small. The typical valley interference causes at worst an offset in the exchange coupling of one or two orders of magnitude, which is easily corrected by an offset in the exchange control gate voltage given that the tunability ranges from 6 to 10 decades per Volt. We find in simulations, however, that the disorder in quantum dot position has a stronger impact.

The fact that exchange coupling is a contact interaction means that any effect impacting how the wavefunction tails off from one dot into its neighbouring dot, such as interdot distance and potential barrier height, has an exponential impact~\cite{Li2010,Saraiva2007}. In the devices investigated here, exchange is controlled by the action of the interstitial J gate (see Fig.~\ref{fig4}.a). This gate induces a lateral displacement of the two dots toward each other reducing the interdot distance at a rate of approximately $10$~nm V$^{-1}$ (see Table \ref{tab:PotentialSweeps}). This is contrary to the picture frequently evoked in this scenario which assumes that the J gate controls the electron penetration length into the classically forbidden region between dots without affecting much its position.
%Exchange follows up exponentially and becomes visible in an experiment when the spin-funnels overcome the width of the spin-resonance frequencies ($\sim$1MHz). 
In Fig.~\ref{fig4}\textbf{b} we can see a method of extracting the exchange coupling by measuring the qubit resonant frequency for a randomly initialised pair of spins as a function of the J gate voltage. 

To simulate this system, we use a path integral Monte Carlo approach~\cite{Ceperley1995}. This method is relatively fast and intrinsically includes the effect of interactions in the electron dynamics. For each exchange simulation, we sample realizations of likely paths of two electrons in a three-dimensional double quantum dot potential obtained from a finite elements simulation (see Fig.~\ref{fig4}\textbf{c}). Interface roughness can be readily included by defining a 3.1~eV step potential barrier to simulate the conduction band offset between silicon and the SiO$_2$ layer. The paths of these electrons are allowed to exchange between the dots, and the impact on the path action is used to estimate the exchange coupling~~\cite{Pedersen2010}.

In Fig.~\ref{fig4}\textbf{d} we compare the J-gate tuning of the exchange coupling at random surface realizations with the experimental data obtained from devices A and C. Both of them are made with Pd/Ti gates with ALD oxides, which corresponds to the architecture studied here (Fig~\ref{fig1}\textbf{f}). Exchange control has also been measured in devices with Al gates \cite{tanttu2023}, observing larger control rates and variability due to the the absence of the ALD oxide \cite{Saraiva2022}. We plot the exchange $J$ against $V_J-0.5(V_{\rm P1}+V_{\rm P2})$, which is a rough measure of the voltage bias between the J-gate and the plunger gates. All devices were tuned to the symmetric operation point~\cite{reed_reduced_2016}. We found that the experimental exchange couplings have a lower baseline than our simulations despite a good agreement with the exchange controllability rates $(d\log J/dV_{\rm J}) $(see Fig.~\ref{fig4}\textbf{h}). This is potentially associated to the valley interference in these devices. 

The J-gate tunes the double dot potential (Fig.~\ref{fig4}\textbf{e}), inducing a displacement of both quantum dots to the center by almost $5$~nm each (Fig.~\ref{fig4}\textbf{f}). We can observe in Fig.~\ref{fig4}\textbf{g} that this exchange dependence on displacement is consistent over multiple surface roughness realizations, even when it is perturbed by the variability of the dot centres caused by surface disorder. Because of the strong correlation between the inter-dot distance and the exchange coupling, we can associate the three orders of magnitude of exchange variability to the variations in the interdot distance caused by surface disorder. 

Importantly, the variability in exchange coupling can be compensated with more tunability. In the specific geometry simulated and measured here, the tunability ranges from 6 to 10 decades per volt, large enough to compensate for the interface disorder and consistently hit a target “on” exchange rate across all devices. 
%As our devices have a high controllability between 5-10 decades/V, predicting the exact baseline for exchange can accumulate an important error. While this model has limitations (such as the lack of explicit valley correlations and an inexact knowledge about gate deformation due to fabrication imperfections), the general qualitative conclusions about the role of roughness on exchange coupling variability are independent of these device-specific issues. Our realisation is that exchange control with a barrier gate involves a shift in each quantum dot position across up to 5nm, which exposes the electron to changes in the interface roughness. 

\bmhead{Conclusions}
Besides demonstrating a complete strategy for diagnosing qubit variability for a given choice of qubit design, fabrication process and materials, this study leads to some general conclusions about the general physics of spins under Si/SiO$_2$ interfaces. The main conclusion is that qubit variability in current devices is explained by the roughness of the Si/SiO$_2$ interface, and other effects (such as strain inhomogeneity and geometrical deformation of the gates) can be mitigated down to levels that are, at most, comparable with this intrinsic roughness mechanism.

Secondly, the size of the quantum dots has a major impact on its performance due to the fractal structure of the interface roughness. Smaller dots and pitches between dots restrict electrons to regions of the interface with smaller amplitude of the roughness, reducing the effects of valley interference.

Another conclusion is that electric tuning of qubit frequencies (using spin-orbit effect) and exchange coupling (using barrier gates) both rely to a large extent on moving the quantum dot laterally, dragging it against the rough interface. This may lead to considerations in future designs of quantum dots and the methods for characterising the interface. 

One remaining question is how much improvement can be realistically expected in the interface quality \cite{fang_evolution_1997}, and how it impacts the qubit performance. We address the last question in Extended Data Fig.~\ref{figA4}, showing that the main benefits would be an enhancement of the average valley splitting and a smaller exchange variability. The spin-orbit coupling is not significantly affected due to its intrinsic atomistic dependence (Fig.~\ref{fig3}\textbf{f}). 

Finally, this study realistically sets the ultimate variability of qubit parameters. We may extract, for instance, the voltage offset that would be necessary to bring a qubit parameter to approximately the same value for all qubits in the architecture, which we call the voltage offset deviation $\text{VOD}(x)$ for each parameter $x$ (see Methods section). For example, the typical voltage offset to bring valley splittings to the same range is 0.58~V, while doing the same for g-factors requires 0.23~V if the magnetic field is pointed towards [100] and 9.1~V for a field along [110]. The smallest value is for the exchange coupling variability which can be corrected with only 0.09~V. That clearly reveals that some parameters can be electrically tuned to a target system-wide value while others will require the implementation of strategies to circumvent the variability with a combination of locally generated control signals and robust global control strategies~\cite{hansen_pulse_2021}. These results outline the minimum demands for an architecture that can deal with qubit variations while maintaining high fidelity.

%We may extract, for instance, the ratio between the variability range of each qubit parameter and its electric tunability. This ratio is a rough estimate of the range of voltage offsets that would need to be applied in a multiqubit device to bring that parameter to the same value for all qubits. For example, the typical voltage offset needed to bring all valley splittings to the same range is 0.58~V, while doing the same for g-factors requires 0.23~V if the magnetic field is pointed towards [100] and 9.1~V for a field along [110]. The smallest value is for the exchange coupling variability which can be corrected with only 0.09~V. 


%\textbf{REMINDER: We haven't included references for extended data on the paper for figures \ref{figA3} an \ref{figA4}}

%\textbf{REMINDER: I feel it is a bit weird that we haven't cited jean Michael Niquet's paper on variability~\cite{Martinez2021}.... We should put that somewhere at least.Making a few coments?}


%\textbf{REMINDER: EDSR discussion was lost. Including citations to papers of EDSR with valley splitting~\cite{Bourdet2018,Corna2018}, and Will's paper~\cite{Gilbert2022}... This is an important thing to mention as it explains the weird data point in Fig.~\ref{fig3}\textbf{b-c}. I was even thinking to include a single extended data figure showing how this degeneracy looks like on the paper. }



%\textbf{REMINDER: There is a huge difference between the experimental data in \ref{fig2}\textbf{e}. Are we going to say something about it ? }

%\textbf{REMINDER: We could still put variability tunability numbers if you want. Here is the summary data. *Exchange: 90mV (more tunability than variability)
%\begin{itemize}
% \item Valley splitting: 3.5V ( more variability than tunability)
% \item g[110]: 50V (variability completely dominates tunability)
% \item g[100]: 500mV (variability is a bit higher than tunability.
% \item Exchange: 90mV (more tunability than variability)
%\end{itemize}
% ( For valley splittings and g-factor I have only considered tunability from one gate. 
%We have many. If I add them up numbers will be smaller. }%

%\section{Figures}\label{sec6}





\section{Methods}\label{sec11}

\bmhead{Oxide growth} The SiO$_2$ gate oxide (7.5-8.0~nm) was thermally grown
on the silicon surface in a custom built high quality oxide furnace as part of a standard MOS device fabrication process. 


\bmhead{Spin spectroscopy} To measure the Stark shift and the difference in Zeeman splittings between the spins, we have to be able to measure the Larmor frequency of the two qubits at a given operation point. In these experiments we have double quantum dots which we use as two electron spin qubits. To begin, we initialise both electrons in the same dot forming a singlet state. We then separate the electrons and, depending on the rate of the separation, they will either end up in a $T^-$ state or having an odd parity, for instance an up-down state. To find the Larmor frequency we apply a fixed pulse or adiabatic microwave pulse with an antenna or resonator~\cite{LauchtAdiabatic2014}. This pulse will flip a spin only if the applied frequency corresponds to the Larmor frequency. If the spin is flipped, the parity of the two spins will change too. We then bring the two electrons to a position where they are allowed to tunnel to the same dot only if their total parity is odd. If the parity is even, the electrons stay in their respective dots. We call this the Pauli-spin blockade-based parity readout~\cite{Yang2020,Seedhouse2021PSB}. The resulting charge state is read using a nearby charge sensor. Here we used a single electron transistor. The frequencies where we measured a flip in the parity of the two-spin states will correspond to the Zeeman splitting of one of the two qubits. 


\bmhead{Measurements of valley splitting} A tunnel rate-based spectroscopy method is used as an approximate measure of the valley splitting of quantum dots. The technique is as follows; 1) A repeated square-wave is applied to a dot gate, centred around a dot-to-reservoir charge transition. 2) The frequency of the square-wave is set equal to or faster than the dot-to-reservoir tunnel rate. In this mode, an electron will move in and out of the quantum dot in sync with the square wave, but for only a proportion of the square-wave repetitions due to the tunnel rate. 3) The amplitude of the square wave is swept from zero to 20~mV. The increase in amplitude causes excited state transitions to be accessed from the reservoir, which typically have increased tunnel rates to the quantum dot. 4) The changes in tunnel rate are fitted, and the splitting in amplitude between the ground state transition and excited state transition is multiplied by the dot gate lever-arm to retrieve the excited state energy. Here we assume the first excited state to be the valley excited state. This measurement technique is also explained in detail in~\cite{yang2012}. %sorry will didn't know you where here. Im jsut compiling to fill missing updates and go away


%The specific filters are included in the SI.
\bmhead{Surface characterization from TEM images} We characterize the Si/SiO$_2$ roughness from TEM images of devices T1 to T6 (see Fig.~\ref{figA2}) using a similar procedure to~\cite{Goodnick1985, Zhao2010, Zhao2009}. To filter the interface, we apply image convolutions that enhance the differences between both materials. We then perform a power spectral density (PSD) decomposition of the filtered interface 
 \begin{equation}
 \label{eq:PSD}
\text{PSD}^{\text{1D}}(\lambda) = \mathcal{C}_0 \left( \frac{2\pi}{\lambda} \right)^{-1-2H}
 \end{equation}
 and confirm that the scaling of the Si/SiO$_2$ roughness is characteristic of a fractal self-affine interface~~\cite{Jacobs2017a,Persson2014,Yoshinobu1995}.

We obtained an average Hurst exponent of $H = 0.28 \pm 0.2$ and a roughness amplitude parameter of $\mathcal{C}_0 \approx 1.4$nm$^3$ for our devices. To compute the scaling of the RMS \textit{versus} $\lambda$ we took multiple subsections of the fitted line surface with amplitude $\lambda$ and computed the average RMS for each device. The scaling of the RMS is 
$\text{RMS} \left(\lambda\right)=\frac{C_0}{4\pi H}\left(\frac{2\pi}{\lambda}\right)^{-H}$, and is compatible with the same parameters measured in the PSD. Both, the RMS and PSD profiles can change for TEMs taken at different regions of the same device. This is normal as some line versions of a 2D surface can be smoother than others, and this behaviour is observed even in a surface generated numerically. We found that comparing average 1D parameters from multiple TEM images accounts for a better estimate of the global 2D profile (Extended Data Fig.\ref{figA1}).

\bmhead{Random surface generation} The computer model of the surface in Figure~\ref{fig1}\textbf{b} was generated with a Fourier-filtering algorithm implemented in Matlab~\cite{MonaMahboobKanafi2021}, that takes as input the 1D PSD profile in equation \eqref{eq:PSD} and outputs a random rough surface with similar spectral density. The output surfaces look visually similar to the ones in the TEMs. To calibrate the model we compare the average 1D PSD and 1D RMS scaling profiles, with the edge profile in the TEM images until we find a good match (Extended Data Fig.~\ref{figA2}). 

\bmhead{Modeling of digital twin of the devices} We create a 3D structure of the device in Matlab with the software provided by the DFX library. We begin by using a physical quantum dot electron-beam lithography (EBL) design layout as the primary framework and construct it as a 3D model that closely resembles its appearance in SEM and TEM images. Our software takes into account the levels of oxidation and thermal expansion that occur during the fabrication process to finely imitate the device geometry. We tweak these variables until the 3D model looks similar to transversal and in-parallel views of SEM and TEM images. 

\bmhead{Electrostatic simulations and quantum dot model}
We import the digital twin device into COMSOL Multiphysics and perform electrostatic potential simulations with the integrated Poisson solver. We fit this to a harmonic model with a vertical electric field
\begin{equation} \label{eq2}
 V(x,y,z) = c_x(x-x_c)^2 + c_y(y-y_c)^2 + zE_z, 
\end{equation}
\noindent where $(x_c, y_c)$ is the centre of the parabolic potential, $E_z$ is the electric field in the z-axis (typically $8$ to $40$ meV nm$^{-1}$) and $c_x, c_y$ are the lateral curvatures (approximately 0.3 meV nm$^{-2}$). We simulate potential sweeps from different gates to characterize their impact on the quantum dots (see Fig.~\ref{figA1}). The harmonic model allows to transform these actions to more intuitive parameters such as shifts in the electric confinement, dot movement, ellipticity, etc. A summary of this impact is included in Table \ref{tab:PotentialSweeps}. 

\bmhead{Atomistic simulations with interface roughness} We perform tight binding simulations in NEMO3D in the $sp^3d^5s^*$ 20-band model for Si, which intrinsically includes spin-orbit-interactions~\cite{Klimeck2007}. To include surface disorder, we terminate the silicon lattice with the local section of the rough surface in Figure~\ref{fig1}\textbf{b}. We then label all the lattice sites above (below) the surface as SiO$_2$ (Si), as observed in Figure~\ref{fig2}\textbf{a}. The SiO$_2$ region is modeled with a sp$^3$ tight-binding (TB) model under a virtual crystal approximation (VCA). The SiO$_2$ TB parameters are optimized to reproduce the electrical properties of the oxide, namely the bandgap of 8.9~eV, conduction band offset of 3.15~eV relative to silicon, and conduction effective mass of 0.44m$_0$. The VCA model in TB assumes a well defined crystal structure, in this case zincblende with a lattice constant having the same value of Si, but treats each atom as a fictitious SiO$_2$ atom. This is a standard way to model alloyed (SiGe) or disordered (SiO$_2$) materials under the VCA in the atomistic TB technique. At the interface region whether an atom is marked as a Si atom or SiO$_2$ atom then creates the atomistic disorder profile. The details of the model with parameter values can be found in~\cite{kim_full_2011}.

By loading the surface roughness profile into NEMO3D, we are able to simulate quantum dot wavefunctions with atomic resolution under the correct local symmetries induced by the disordered surface. A limitation of the model is that we cannot simulate atomically disordered Si-O bonds in the oxide. Considering the various geometrical permutations of such bonds in an amorphous solid, it becomes a computationally challenging problem for large scale simulations. However, the VCA model does replicate the bulk electrical properties of SiO$_2$. Despite this limitation, we are able to simulate effectively a Si-SiO$_2$ interface that in general is hard to describe from a tight binding approach.

\bmhead{Valley Phase calculation with atomistic simulation} The atomistic tight binding software outputs the electron densities of the two valley states ($\Vert \Psi_{v_{-}}(x,y,z)) \Vert^2$ and $\Vert \Psi_{v_{+}}(x,y,z) \Vert^2$)~\cite{Saraiva2009}. Their difference may be interpreted in terms of an envelope function $\Psi_{\rm Env}$ multiplied by cosine oscillations with the wavevector of the conduction band minima $k_0 = 0.82 2\pi/a_0$
\begin{equation}
\label{eq:Vosc}
 \begin{aligned}
 \psi_{\text{Osc}}(x,y,z) =& \left\Vert \Psi_{v_{+}}(x,y,z)\right\Vert ^{2}-\left\Vert \Psi_{v_{-}}(x,y,z)\right\Vert ^{2} \\
 \approx & 2\Psi_{\rm Env}(x,y,z)^2\cos\left(-2i\kappa_{0}z+i\phi_v \right).
 \end{aligned}
\end{equation}
A colour plot of $\psi_{\textbf{Osc}}$ is shown in Fig.~\ref{fig2}\textbf{e}. To compute the valley phase we average $\psi_{\text{Osc}}$ over the $x-y$ plane and perform a Fourier transform in the z-axis. There is a distinctive peak with frequency $2\kappa_0$, as it is expected for valley oscillations. The valley phase $\phi_v$ is obtained from the complex phase of the transform at this frequency. 

\bmhead{G matrix computation from atomistic tight-binding} For any magnetic field $\vec{B}$, the spin part of the Hamiltonian of the system is
\begin{equation}
\label{eq:Hzeeman}
H_{\rm Zeeman} = \frac{\mu_B}{2} \vec{\sigma}^T \mathbb{G} \vec{B} = \frac{\mu_B}{2} \vec{\sigma} \cdot g_0 \vec{B}_{eff},
\end{equation}
where $\mathbb{G}$ is the $\mathbb{G}$-matrix and $\vec{B}_{eff} = \frac{1}{g_0} \mathbb{G} \vec{B} $ is the effective magnetic field after including spin-orbit effects. The atomistic tight binding software outputs the Zeeman splitting $E_{\text{Zeeman}}= g_0 \mu_B \Vert \vec{B}_{eff}\Vert$ and also the full wavefunction of the ground state $\Psi_{\downarrow}$ written in a base of atomic positions, orbitals and spins $\vert \textbf{R}, \alpha , s\rangle$. From this we can estimate the mean spin vector $\langle \vec{\sigma_\downarrow} \rangle = \langle \Psi_{\downarrow} \vert \vec{\sigma} \vert \Psi_{\downarrow} \rangle$. This spin aligns anti-parallel to the effective field $g_0 \vec{B}_{eff}$ by definition. In total, we have obtained the magnitude and orientation of the vector $g_0 B_{eff}$. If we perform this computation for three linearly independent magnetic fields $\vec{B}_1,\vec{B}_2,\vec{B}_3$, we will obtain the effective fields $\vec{B}_{\text{eff},1}, \vec{B}_{\text{eff},2}, \vec{B}_{\text{eff},3}$. Then, the linear system $\vec{B}_{\text{eff},i} = \mathbb{G} \vec{B}_{i} $ can be inverted to compute $\mathbb{G} $.

%Explenation of our G-matrices
A typical G-matrix obtained from atomistic simulations is
\begin{equation}
\label{eq:Gmat1}
		\mathbb{G} = 
		 \begin{bmatrix}
			g_0+ \alpha' & \beta' & g_{13}\\
			\beta' & g_0 + \alpha' & g_{23} \\
			0 & 0 & g_{33}
			\end{bmatrix},
\end{equation}
\noindent where the basis is aligned with the lattice orientations $\{[100], [010], [001]\}]$. In here, $g_0 \approx 1.9937$ is the bulk g-factor, as obtained from the asymptotic behavior of the simulations at low electric field. The entries $\alpha' \sim -10^{-3}$ and $\beta' \sim \pm 10^{-2}$ determine the in-plane spin-orbit-coupling and the parameters $g_{13} \sim \pm 10^{-3}, g_{23} \pm \sim 10^{-3} $ and $ g_{33} \sim 2.00192 - \mathcal{O}(10^{-4}) $ describe the out of-plane components. The two remaining entries were calculated to be smaller than $10^{-5}$ at all cases and hence approximated to $0$. To obtain the g-factor at any magnetic field orientation we compute $g(\hat{r}) = \Vert \mathbb{G} \hat{r} \Vert$. If $\hat{r}(\phi)= [\cos\phi \ \sin\phi \ 0]^T$ is an in-plane normal vector

\begin{equation}
\begin{aligned}
\mathbb{G}\hat{r}(\phi)&=\left[\begin{array}{c}
g_{0}\cos\phi+\alpha'\cos\phi+\beta'\sin\phi\\
g_{0}\sin\phi+\alpha'\sin\phi+\beta'\cos\phi\\
0
\end{array}\right] 
\\
&:=
\left[\begin{array}{c}
g_{x}\\
g_{y}\\
0
\end{array}\right].
\end{aligned}
\end{equation}
Then
\begin{equation}
    \begin{aligned}
            \left\Vert \mathbb{G}\hat{r}(\phi)\right\Vert 	&=\sqrt{g_{x}^{2}+g_{y}^{2}} \\
	&\approx\sqrt{g_{0}^{2}+g_{0}\left(\alpha'^{2}+2\beta'^{2}\sin(\phi)\cos(\phi)\right)} \\
	&\approx g_{0}+\frac{\alpha'}{2g_{0}}+\frac{\beta'}{2g_{0}}\sin(2\phi)
    \end{aligned}
\end{equation}

\noindent under the approximation $\alpha',\beta' \ll g_0$ which is valid in this case. After replacing $\alpha := \frac{\alpha'}{2g_{0}}\approx \frac{\alpha'}{4}$ and $\beta := \frac{\beta'}{4}$, we recover the expression that led to the variability distributions in Fig.~\ref{fig3}.

\bmhead{Two electron Hamiltonian for path integral simulations} 
\begin{equation}
\label{eq:H2e}
 H_{2e}(\vec{r}_1(t), \vec{r}_2(t))= 
 \sum_{i=1}^2 H_1(\vec{r}_i(t))
 + \frac{e^2 }{4\pi\epsilon_{Si} \vert \vec{r}_{1}-\vec{r}_{2} \vert },
\end{equation}
\noindent where each single electron hamiltonian is given by 
\begin{equation}
\label{eq:H2e_single}
 \begin{aligned}
 H_1(\vec{r}_i(t))= & \frac{\vec{v}^{\dagger}_{i} M_{\rm Si} \vec{v}_{i}}{2} + V_{\mathrm{DQD}}\left(\vec{r}_{i}\right)\\
 & + V_{{\rm Si-SiO}_2} \sigma(z_i - z_s(\vec{r}_i)).
 \end{aligned}
\end{equation}

\noindent Here $M_{\rm Si}$ is the diagonal matrix $\text{Diag}(0.19, 0.19, 0.98) m_e$ denoting the effective mass of silicon electrons on each lattice orientation and $\vec{v_i}= \frac{d\vec{x}_i}{dt}$ is the velocity of each electron. $\epsilon_{Si}$ is set to $ 11.7 \epsilon_{0}$. $ V_{\mathrm{DQD}}$ is the double quantum dot potential simulated in Comsol as in Fig.~\ref{figA1}.

The oxide interface is defined by a smooth step of $V_{{\rm Si-SiO}_2} = 3.1eV$ with the function $\sigma(z)= \frac{1}{1+e^{-4((z-z_s(\vec{r}))/a_0}}.$ Where $a_0 = 0.543$nm is the silicon lattice parameter. $z_s(\vec{r})$ defines $z$ coordinate of the rough surface at the position of $\vec{r}$ projected in the xy-plane $(r_x,r_y)$. 

\bmhead{Exchange coupling simulations with path integral Monte Carlo (PIMC):} Hundreds of realizations of two-electron paths are sampled with PIMC~\cite{Ceperley1995} with a Metropolis algorithm to minimize the partition function $\mathbb{Z} = e^{-S/\hbar}$, where $S$ is the total action
$S =\sum_{m = 0}^{N_t} \tau H_{2e}(\vec{r}_1(m\tau), \vec{r}_2(m\tau))$. We simulate both, paths that remain in separate quantum dots for the entire simulation and paths that exchange a few times between the quantum dots. The action of electron paths that exchange between the two dots is higher than for non-exchanging paths by an amount $\Delta S$ which allows to compute the exchange energy as $J = \frac{2}{\beta} e^{-\Delta S/\hbar}$~\cite{Pedersen2010}. As the estimates of the operators are computed from average sampled path realizations, the method provides natural error bars determined by the standard deviation of $-\Delta S$. To adapt the method to 3D electrons in MOS quantum dots we had to do modifications to the main algorithm that are detailed in a separate paper. 


\bmhead{Voltage Offset Deviation (VOD)}
\begin{equation}
 \text{VOD}(\sigma) = \operatorname{std}\left(\frac{\sigma-\langle \sigma \rangle}{d \sigma / d V}\right)
\end{equation}
 For any parameter $x$ this metric estimates the deviation of voltage offsets that are needed to bring each variable to the average value $\langle \sigma \rangle$. For the main parameters discussed in this paper we obtained: 
 \begin{enumerate}
 \item Valley Splitting: $ \text{VOD}(\text{VS}) =$ 0.58~V
 \item g-factor [110]: $ \text{VOD}(g_{ [110] }) =$ 9.1~V \\Maximum among in-plane B-field angles
 \item g-factor [110]: $ \text{VOD}(g_{[100]}) =$ 0.23~V \\ Minimum among in-plane B-field angles
 \item Valley Splitting: $ \text{VOD}(J) =$ 0.09~V
 \end{enumerate}
 The values of $\sigma$ are obtained from roughness variability distributions. Only tunings in the range of hundreds of mV can be performed in real experiments. If the VOD is way higher than that, it means that it is not possible to tune $\sigma$ to the same value for all qubits. 
 
 \bmhead{Estimating the tunability of qubit parameters from simulations}
 To compute the tunability of each qubit $d \sigma / d V$ we use simulations for different voltage tunings for the specific gates have a significant impact on each one of the variables. For the exchange ($J$) we focus on the J-gate. In contrast, single qubit parameters like g-factors and valley splittings can be tuned with more than one gate. In here we assume that the action is performed with a top gate and an additional lateral gate so that ${d \sigma / d V = d \sigma / d V_{\rm Top}+ d \sigma /V_{\rm Lat}}$ . At the same time, each of these tunings is divided as \begin{equation}
 \frac{d \sigma }{d V_{\rm Top}} = \frac{d \sigma }{dE_z} \ \frac{dE_z}{dV_{\rm Top}} +
 \frac{d \sigma}{d x} \ \frac{dx}{dV_{\rm Top} }.
 \end{equation} 
 Estimates for the impact of different gates on single quantum dot parameters can be found in Table \ref{tab:PotentialSweeps}. The values for $d \sigma /dE_z$ and $d\sigma /d x$ where simulated from small changes in the simulation parameters as in Figs.~\ref{fig2}\textbf{f} and \ref{fig3}\textbf{c-d}. 
 



%\bmhead{Ethical approval declarations}

% (only required where applicable) Any article reporting experiment/s carried out on (i)~live vertebrate (or higher invertebrates), (ii)~humans or (iii)~human samples must include an unambiguous statement within the methods section that meets the following requirements: 

% \begin{enumerate}[1.]
% \item Approval: a statement which confirms that all experimental protocols were approved by a named institutional and/or licensing committee. Please identify the approving body in the methods section

% \item Accordance: a statement explicitly saying that the methods were carried out in accordance with the relevant guidelines and regulations

% \item Informed consent (for experiments involving humans or human tissue samples): include a statement confirming that informed consent was obtained from all participants and/or their legal guardian/s
% \end{enumerate}

% If your manuscript includes potentially identifying patient/participant information, or if it describes human transplantation research, or if it reports results of a clinical trial then additional information will be required. Please visit (\url{https://www.nature.com/nature-research/editorial-policies}) for Nature Portfolio journals, (\url{https://www.springer.com/gp/authors-editors/journal-author/journal-author-helpdesk/publishing-ethics/14214}) for Springer Nature journals, or (\url{https://www.biomedcentral.com/getpublished/editorial-policies\#ethics+and+consent}) for BMC.

\backmatter

%\bmhead{Supplementary information}


\bmhead{Data Availability}
The datasets generated and/or analysed during this
study are available from the corresponding authors on
reasonable request.



\bmhead{Code Availability}
The analysis codes that support the findings of the
study are available from the corresponding authors on
reasonable request.


\bmhead{Acknowledgments}
We acknowledge support from the Australian Research Council (FL190100167 and CE170100012), the US Army Research Office (W911NF-23-1-0092), and the NSW Node of the Australian National Fabrication Facility. The views and conclusions contained in this document are those of the authors and should not be interpreted
as representing the official policies, either expressed or
implied, of the Army Research Office or the US Govern-
ment. The US Government is authorized to reproduce
and distribute reprints for Government purposes notwith-
standing any copyright notation herein. J.Y.H., S.S., M.F. and J.D.C. acknowledge support from the Sydney Quantum Academy. This project was undertaken with the assistance of resources and services from the National Computational Infrastructure (NCI), which is supported by the Australian Government and includes computations using the computational cluster Katana supported by Research Technology Services at UNSW Sydney. 

\section{Author Contributions}

W.H.L. and F.E.H. fabricated the devices, with
A.S.D.’s supervision, on isotopically enriched 28Si wafers
supplied by K.M.I. (800 ppm), N.V.A., H.-J.P., and
M.L.W.T. (50 ppm). T.T., W.G, J.H , E.V, R.C.C.L, S.S provided measurements of the qubit devices with supervision of C.H.Y.,
A.S., A.L and A.S.D.~ The data was gathered by J.D.C., who used it to estimate the CMOS variability in comparison with one qubit and two qubit simulations.~ J.D.C performed the one qubit simulations with atomistic tight binding in NEMO 3D with the supervision of R.R and A.S.~ For two qubit systems, J.D.C performed the exchange simulations with an in-house Path Integral Monte Carlo code developed by P.Y.M , F.S and J.D.C. with the supervision of A.S.~ D.O coded an initial framework for the fractal analysis of the Si/SiO2 interface that was later improved by J.D.C with the supervision of A.S.~ F.E.H took the TEM images of the devices.~ J.Y.H and D.D coded the software that designs the digital twin of the devices with the supervision of C.C.E. and A.S.~ These digital twins where imported for electrostatic simulations in COMSOL by J.D.C. and D.D. with the supervision of C.C.E.~M.K.F provided theoretical support at all stages. J.D.C and A.S wrote the manuscript, with the input from all authors.

\section*{Declaration of competing interests}

TT, WG, EV, CCE, AL, CHY, WHL, FEH, AS, and ASD declare equity interest in Diraq Pty Ltd. Other authors declare no competing interests.
%Some journals require declarations to be submitted in a %standardised format. Please check the Instructions for %Authors of the journal to which you are submitting to see %if you need to complete this section. If yes, your %manuscript must contain the following sections under the %heading `Declarations':

%\begin{itemize}
%\item Funding
%Conflict of interest/Competing interests (check %journal-specific guidelines for which heading to use)
%\item Ethics approval 
%\item Consent to participate
%\item Consent for publication
%\item Availability of data and materials
%\item Code availability 
%\item Authors' contributions
%\end{itemize}

% \noindent
% If any of the sections are not relevant to your manuscript, please include the heading and write `Not applicable' for that section. 

% %%===================================================%%
% %% For presentation purpose, we have included %%
% %% \bigskip command. please ignore this. %%
% %%===================================================%%
% \bigskip
% \begin{flushleft}%
% Editorial Policies for:

% \bigskip\noindent
% Springer journals and proceedings: \url{https://www.springer.com/gp/editorial-policies}

% \bigskip\noindent
% Nature Portfolio journals: \url{https://www.nature.com/nature-research/editorial-policies}

% \bigskip\noindent
% \textit{Scientific Reports}: \url{https://www.nature.com/srep/journal-policies/editorial-policies}

% \bigskip\noindent
% BMC journals: \url{https://www.biomedcentral.com/getpublished/editorial-policies}
% \end{flushleft}

\bibliography{SurfaceRoughnessPaper}

\begin{appendices}


%!TEX root = main.tex
\onecolumn
\section{Extended Data}


\begin{figure*}[hbt]
\centering
\includegraphics[width=\linewidth]{figA2.pdf}
\caption{ \textbf{$\vert$ Characterization of oxide roughness in 6 devices.} \textbf{a}, 8 TEM images taken from device T2 with range $(\sim 40~nm)$ and interface fit. \textbf{b}, Individual PSD $\mathcal{C}^{1D}(\lambda)$ of each one of the TEM images. The blue line corresponds to the average PSD. \textbf{c},  Comparison of the average PSD of device T2 with the random surface. \textbf{d}, Characterization of the PSD for the five remaining devices. All oxides were grown under the same conditions except for device T4 consisting of a 7.5~nm SiO2 layer grown on isotopically purified $^{29}$Si. The PSD was obtained from TEMs with varying numbers and ranges, with more TEMs yielding a higher degree of precision in the PSD estimate. }
\label{figA2}
\end{figure*}

\newpage

\begin{figure*}[htb]
\centering
\includegraphics[width=\linewidth]{figA1.pdf}
\caption{ \textbf{$\vert$ Potential simulations and gate-impact on quantum dots.} \textbf{a}, Horizontal view of the 3D device model that we input in COMSOL for potential simulations. The green square shows the region where the dots are formed. \textbf{b}, Potential landscape of the device simulated in COMSOL. The set of gate potentials is derived from experiments. A double quantum dot is isolated in the green square region bellow the gates $P1$ and $P2$.\textbf{c}, Zoom into the potential profile at the green region in \textbf{b}. Single dots are formed inside the white rectangles. We fit the potentials inside these regions to the harmonic model in equation \eqref{eq2}. \textbf{d,f}, Evolution of the potential profile over the cyan line in \textbf{c} under the tuning of gates $P1$ (\textbf{d}) and $J1$ (\textbf{f}). \textbf{e,g}, Characterization of the impact of gates $P1$ (\textbf{e}) and $P2$ (\textbf{g}) on each quantum dot. We evaluate this impact over 5 variables: Displacement of the dot position from the mean $(\delta x_c , \delta y_c)$, transversal electric field $E_z$ and curvatures $(c_x, c_y)$. The numbers inside the plots show the slope of the curve. Their units depend on the variable evaluated, for instance , the unit of $dx_c /dV$ is [nm/V].} 
\label{figA1}
\end{figure*}



\newpage

%The units of $dx_i/dV$ are in [nm V$^{-1}$] and the units of $dE_{z}/dV$ are in [eV nm$^{-1}$V$^{-1}$]. }
%\end{table}

%\caption{\label{tab:PotentialSweeps} \textbf{Impact of gate action on each quantum dot} By fitting realistic potentials to the model in equation \eqref{eq2} as shown in Fig.\ref{figA1}, we obtained the the dependence of the  dot parameters on the action of each gate. This provides a simple understanding of h




%\newpage





\newpage
\begin{figure*}[htb]
\centering
\includegraphics[width=\linewidth]{figA3.pdf}
\caption{ \textbf{$\vert$ Investigating spin-orbit correlations.} \textbf{a-c}, Dependence of Dressselhaus term vs: \textbf{a}, Proportion of type A lattice sites on the quantum dot wavefunction. \textbf{b}, Valley splitting. \textbf{c}, Valley phase. \textbf{d-f}, Dependence of Rashba spin orbit coupling vs  \textbf{d}, $\Delta x \Delta y $, where $\Delta x$ ($y$) is the standard deviation of the $x$($y$) site in the ground state wavefunction ($\Delta x^2 = \langle x^2 \rangle - \langle x \rangle^2 )$ . \textbf{e}, Valley splitting. \textbf{f}, Valley phase. These plots do not include near degeneracy points as this data can disturb significantly the scale of the spin-orbit interactions. The leading correlations for each variable are plotted in the first column: Proportion of sub-lattice sites for Dresselhaus and quantum dot area for Rashba. }
\label{figA3}
\end{figure*}




\newpage

\begin{figure*}[htb]
\centering
\includegraphics[width=\linewidth]{figA4.pdf}
\caption{\textbf{$\vert$ Dependence of qubit parameters the surface RMS:} \textbf{a-b}, We generated two additional surfaces with higher (\textbf{a}) and lower (\textbf{b}) RMS that the one used in the paper to understand the potential benefits of improving the surface quality. \textbf{c-d}, 1D RMS and 1D PSD of the original surface and the new surfaces plotted in \textbf{a} and \textbf{b}. The profiles of devices T1 and T2 are also included in both figures for comparison with the dispersion of the measured data. \textbf{e-f}, Spin-orbit coupling dependence on the RMS. Dresselhaus values are slighly more dispersed for smoother interfaces as the results approach to the flat surface limits. \textbf{g-h}, RMS vs valley splitting(\textbf{g}) and two-dot exchange coupling (\textbf{h}). Smoother surfaces lead to higher mean valley splittings \textbf{g} and to  smaller variability in exchange coupling  \textbf{h}.}
\label{figA4}
\end{figure*}




\newpage
%% Table
\begin{table}[t]
\centering
\begin{tabular}{ccccc}
 & \textbf{$dx_1/dV $}     & \textbf{$dE_{z1}/dV$}     & \textbf{$dx_2/dV$}     & \textbf{$dE_{z2}/dV$}   \\
 \textbf{Gates}  & [nm V$^{-1}$] & [eV nm$^{-1}$V$^{-1}$] & [nm V$^{-1}$] & [eV nm$^{-1}$V$^{-1}$]\\
\hline
P1    & { \textbf{-6.74}} & { \textbf{13.42}} & { \textbf{-2.95}} & { -2.11}   \\
P2    & { \textbf{4.95}} & { -0.68}   & { \textbf{5.5}} & { \textbf{14.75}} \\
J1    & { \textbf{6.88}} & { 0.46}   & { \textbf{-3.57}} & { -0.22}   \\
P3   & { 0.04}   & { -0.02}   & { \textbf{0.13}} & { 0.06} \\
J2    & { 0.02}   & { -0.01}   & { \textbf{0.13}} & { 0.06}   
\end{tabular}

\caption{$\vert$ \label{tab:PotentialSweeps} \textbf{Impact of gate action on each quantum dot.} As a result of the harmonic fitting in Fig.~\ref{figA1}~, we obtained the dependence of the dot parameters on the action of each gate. These number are used to estimate tunabilities of qubit parameters from atomistic simulations (see Methods section).}
\end{table}
%\section{Section title of first appendix}\label{secA1}

%An appendix contains supplementary information that is not an essential part of the text itself but which may be helpful in providing a more comprehensive understanding of the research problem or it is information that is too cumbersome to be included in the body of the paper.

%%=============================================%%
%% For submissions to Nature Portfolio Journals %%
%% please use the heading ``Extended Data''. %%
%%=============================================%%

%%=============================================================%%
%% Sample for another appendix section			 %%
%%=============================================================%%

%% \section{Example of another appendix section}\label{secA2}%
%% Appendices may be used for helpful, supporting or essential material that would otherwise 
%% clutter, break up or be distracting to the text. Appendices can consist of sections, figures, 
%% tables and equations etc.

\end{appendices}

%%===========================================================================================%%
%% If you are submitting to one of the Nature Portfolio journals, using the eJP submission %%
%% system, please include the references within the manuscript file itself. You may do this %%
%% by copying the reference list from your.bbl file, paste it into the main manuscript.tex %%
%% file, and delete the associated \verb+\bibliography+ commands.  %%
%%===========================================================================================%%
%~\cite{bib1}
%\bibliography{SurfaceRoughnesPaper.bib}% common bib file





%% if required, the content of.bbl file can be included here once bbl is generated
%\input SurfaceRoughnesPaper.bib

%% Default %%
%%\input sn-sample-bib.tex%

\end{document}
