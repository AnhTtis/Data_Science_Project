%!TEX root = main.tex
\onecolumn
\section{Extended Data}


\begin{figure*}[hbt]
\centering
\includegraphics[width=\linewidth]{figA2.pdf}
\caption{ \textbf{$\vert$ Characterization of oxide roughness in 6 devices.} \textbf{a}, 8 TEM images taken from device T2 with range $(\sim 40~nm)$ and interface fit. \textbf{b}, Individual PSD $\mathcal{C}^{1D}(\lambda)$ of each one of the TEM images. The blue line corresponds to the average PSD. \textbf{c},  Comparison of the average PSD of device T2 with the random surface. \textbf{d}, Characterization of the PSD for the five remaining devices. All oxides were grown under the same conditions except for device T4 consisting of a 7.5~nm SiO2 layer grown on isotopically purified $^{29}$Si. The PSD was obtained from TEMs with varying numbers and ranges, with more TEMs yielding a higher degree of precision in the PSD estimate. }
\label{figA2}
\end{figure*}

\newpage

\begin{figure*}[htb]
\centering
\includegraphics[width=\linewidth]{figA1.pdf}
\caption{ \textbf{$\vert$ Potential simulations and gate-impact on quantum dots.} \textbf{a}, Horizontal view of the 3D device model that we input in COMSOL for potential simulations. The green square shows the region where the dots are formed. \textbf{b}, Potential landscape of the device simulated in COMSOL. The set of gate potentials is derived from experiments. A double quantum dot is isolated in the green square region bellow the gates $P1$ and $P2$.\textbf{c}, Zoom into the potential profile at the green region in \textbf{b}. Single dots are formed inside the white rectangles. We fit the potentials inside these regions to the harmonic model in equation \eqref{eq2}. \textbf{d,f}, Evolution of the potential profile over the cyan line in \textbf{c} under the tuning of gates $P1$ (\textbf{d}) and $J1$ (\textbf{f}). \textbf{e,g}, Characterization of the impact of gates $P1$ (\textbf{e}) and $P2$ (\textbf{g}) on each quantum dot. We evaluate this impact over 5 variables: Displacement of the dot position from the mean $(\delta x_c , \delta y_c)$, transversal electric field $E_z$ and curvatures $(c_x, c_y)$. The numbers inside the plots show the slope of the curve. Their units depend on the variable evaluated, for instance , the unit of $dx_c /dV$ is [nm/V].} 
\label{figA1}
\end{figure*}



\newpage

%The units of $dx_i/dV$ are in [nm V$^{-1}$] and the units of $dE_{z}/dV$ are in [eV nm$^{-1}$V$^{-1}$]. }
%\end{table}

%\caption{\label{tab:PotentialSweeps} \textbf{Impact of gate action on each quantum dot} By fitting realistic potentials to the model in equation \eqref{eq2} as shown in Fig.\ref{figA1}, we obtained the the dependence of the  dot parameters on the action of each gate. This provides a simple understanding of h




%\newpage





\newpage
\begin{figure*}[htb]
\centering
\includegraphics[width=\linewidth]{figA3.pdf}
\caption{ \textbf{$\vert$ Investigating spin-orbit correlations.} \textbf{a-c}, Dependence of Dressselhaus term vs: \textbf{a}, Proportion of type A lattice sites on the quantum dot wavefunction. \textbf{b}, Valley splitting. \textbf{c}, Valley phase. \textbf{d-f}, Dependence of Rashba spin orbit coupling vs  \textbf{d}, $\Delta x \Delta y $, where $\Delta x$ ($y$) is the standard deviation of the $x$($y$) site in the ground state wavefunction ($\Delta x^2 = \langle x^2 \rangle - \langle x \rangle^2 )$ . \textbf{e}, Valley splitting. \textbf{f}, Valley phase. These plots do not include near degeneracy points as this data can disturb significantly the scale of the spin-orbit interactions. The leading correlations for each variable are plotted in the first column: Proportion of sub-lattice sites for Dresselhaus and quantum dot area for Rashba. }
\label{figA3}
\end{figure*}




\newpage

\begin{figure*}[htb]
\centering
\includegraphics[width=\linewidth]{figA4.pdf}
\caption{\textbf{$\vert$ Dependence of qubit parameters the surface RMS:} \textbf{a-b}, We generated two additional surfaces with higher (\textbf{a}) and lower (\textbf{b}) RMS that the one used in the paper to understand the potential benefits of improving the surface quality. \textbf{c-d}, 1D RMS and 1D PSD of the original surface and the new surfaces plotted in \textbf{a} and \textbf{b}. The profiles of devices T1 and T2 are also included in both figures for comparison with the dispersion of the measured data. \textbf{e-f}, Spin-orbit coupling dependence on the RMS. Dresselhaus values are slighly more dispersed for smoother interfaces as the results approach to the flat surface limits. \textbf{g-h}, RMS vs valley splitting(\textbf{g}) and two-dot exchange coupling (\textbf{h}). Smoother surfaces lead to higher mean valley splittings \textbf{g} and to  smaller variability in exchange coupling  \textbf{h}.}
\label{figA4}
\end{figure*}




\newpage
%% Table
\begin{table}[t]
\centering
\begin{tabular}{ccccc}
 & \textbf{$dx_1/dV $}     & \textbf{$dE_{z1}/dV$}     & \textbf{$dx_2/dV$}     & \textbf{$dE_{z2}/dV$}   \\
 \textbf{Gates}  & [nm V$^{-1}$] & [eV nm$^{-1}$V$^{-1}$] & [nm V$^{-1}$] & [eV nm$^{-1}$V$^{-1}$]\\
\hline
P1    & { \textbf{-6.74}} & { \textbf{13.42}} & { \textbf{-2.95}} & { -2.11}   \\
P2    & { \textbf{4.95}} & { -0.68}   & { \textbf{5.5}} & { \textbf{14.75}} \\
J1    & { \textbf{6.88}} & { 0.46}   & { \textbf{-3.57}} & { -0.22}   \\
P3   & { 0.04}   & { -0.02}   & { \textbf{0.13}} & { 0.06} \\
J2    & { 0.02}   & { -0.01}   & { \textbf{0.13}} & { 0.06}   
\end{tabular}

\caption{$\vert$ \label{tab:PotentialSweeps} \textbf{Impact of gate action on each quantum dot.} As a result of the harmonic fitting in Fig.~\ref{figA1}~, we obtained the dependence of the dot parameters on the action of each gate. These number are used to estimate tunabilities of qubit parameters from atomistic simulations (see Methods section).}
\end{table}