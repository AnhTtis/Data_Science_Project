% \vspace{-1em}
\section{Visual Studio Code Investigation}
\label{investigation}
% Besides four aspects in Section~\ref{solution} we have more requirements for picked IDE in real class.

Visual Studio (VS) Code is a lightweight IDE developed and supported by Microsoft, it is free for private or commercial use. The core feature of VS Code is its extension support, users are able to add languages, debuggers, and tools to their own installation to better serve later development. Besides standard extensions released by Microsoft, there are plenty of extensions on the VS Code Extension Marketplace~\cite{marketplace} contributed by third-party organizations and individual developers. We analyze its practicality for introductory-level Python courses from four aspects: accessibility, easy-to-use, functionality, and popularity.



\myparabb{Accessibility. } VS Code provides cross-platform support, it runs on various operating systems, i.e. macOS, Windows, Linux, and on most available hardware with an identical user interface. VS Code has a small download (< 200 MB) and a disk footprint (< 500 MB), it is lightweight, and perfectly fits every student with different devices. 

As a multi-languages supported IDE, almost every major programming language has extension support, it is effortless to switch between different programming languages. Though our class is for Python beginners, students can continually code with VS Code in later programming classes.



\myparabb{Easy-to-use. } VS Code has a compact but simple user interface. Figure~\ref{fig:interface} shows an example. In the center is the editor, where students code their assignments. Under the editor is the panel, it can be displayed in different panels like a debug console or a built-in terminal, the terminal always starts from the root of a certain workspace. On the left side is the sidebar that contains different views, Figure~\ref{fig:interface} shows the view of Explorer to better assist in locating a file. Maintaining a single window of code and debugging effectively increase programming efficiency.


% User Interface
\begin{figure}[t]
    \centering
    \includegraphics[width=0.5\textwidth]{Figures/interface3.png}
    \caption {The basic layout of VS Code user interface~\cite{userinterface} that we introduced to students. The editor is the area to edit files, students can open multiple editors at the same time side by side vertically or horizontally. The panel below the editor is for output or debug information, errors and warnings, or an integrated terminal. The side bar contains different views like the Explorer or Extension Marketplace, to assist students while working on the projects or downloading extensions.}
    \label{fig:interface}
\end{figure}

% user interface: build-in terminal, debugger

% requirement for development

Besides the concise user interface, VS Code employs many features to improve development. One important feature that simplifies programming is the workspace configure setting. A VS Code workspace is the root folder of the current project, configure settings that apply to a specific workspace but not others. As a common scenario, students need to have different settings (e.g. interpreter version, dependent libraries, programming languages) among projects. In our past experience, students may feel confused with the configure settings, especially set/reset environment at the beginning of starting a new project. It will be much easier to work on VS Code workspace, because it allows configuring settings in the context of the current workspace, and always overrides the global user settings. 

Furthermore, VS Code supports remote development. The {\it Remote SSH} extension allows opening a remote folder on any remote machine, virtual machine, or container a running SSH server. Another useful feature favored by many developers is the {\it Command Palette}, Figure~\ref{fig:command} shows an example, where users would have access to all of the functionalities of VS Code, including commands of your installed extensions.  

\begin{figure}[t]
    \centering
    \includegraphics[width=0.5\textwidth]{Figures/commandpalette.png}
    \caption{The red box indicated the Command Palette of VS Code. Students can access all functionality of VS Code and installed extensions through the Command Palette.}
    \label{fig:command}
\end{figure}


\myparabb{Functionality. } Besides the basic features of a source code editor, VS Code offers extensions to increase its functionality. With over 30,000 extensions in Marketplace, users pick favored extensions and customize their installation to improve the programming experience. For example, in our Python course, we recommend students install the Python extension developed by Microsoft, which offers strong support for Python language, it has powerful features such as IntelliSense~\cite{intellisense}, code formatting, debugging, variable explorer, and more. The IntelliSense suggestion pops out while you type, it provides intelligent code completion based on the language semantics and written source code. Figure~\ref{fig:intellisense} shows an example, where IntelliSense suggests using variable {\tt msg} in {\tt print} function. 


\begin{figure}[t]
    \centering
    \includegraphics[width=0.5\textwidth]{Figures/auto.png}
    \caption{An example of code completion supported by VS Code, where the IntelliSense suggests using variable {\tt msg} in {\tt print} function.}
    \label{fig:intellisense}
\end{figure}



\myparabb{Popularity. } VS Code is widely used among real-world developers. In the Stack Overflow 2021 Developer survey~\cite{survey}, VS Code tops the most popular developer environment tool, with 71.06\% of over 80,000 respondents reporting that they use it. Due to the great user amount, bugs/issues are fixed/solved in time. VS Code updates frequently, from June 2020 to May 2021, VS Code development team made 22 releases in total, updating almost every month. 
% https://insights.stackoverflow.com/survey/2021#section-most-popular-technologies-integrated-development-environment

The high popularity tops the reasons why we introduce VS Code to our Python class. Students encounter difficulties in switching IDEs, they are able to use VS Code throughout different programming languages classes, or in their future careers. 
