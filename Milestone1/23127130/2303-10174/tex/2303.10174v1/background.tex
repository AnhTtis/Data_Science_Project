\section{Background and Related Work}

\subsection{Python in Education}

% {\color{orange} Python in Education}

Python has become the most prominent language in college, it tops the list of the most-taught programming languages, especially in introductory computer science courses, it is mainly due to three facts: {\it Python is beginner friendly}, it has a simplified syntax with an emphasis on natural language while most programming languages have complex rules, thus it is perfect for beginners to learn and practice~\cite{khoirom2020comparative, pellet2019beginner, srinath2017python, cutting2021review}. 
% Also learning Python is helpful for students to study in advanced-level courses or do computer science research. 
{\it Python is a powerful tool}, it is applied in many areas like machine learning and big data, especially effective and essential in scientific computing and data analysis~\cite{oliphant2006guide, van2011numpy, scikit-learn, chen2020atmem, chen2020efficient, tan2021toward, tensorflow2015-whitepaper, paszke2017automatic}. Students with diverse backgrounds use Python for different purposes, e.g. analyzing the market in finance, simulating protein structure in biology. Students who pursue a formal education in computer science or computer science-related majors are extremely likely to continue using Python throughout their careers.  
% While most programming languages have complex rules, Python uses a simplified syntax with an emphasis on natural language. Together with a active and supportive community, Python is commonly recommended as the first programming language to learn. 
{\it Python is popular}, according to the PYPL index~\cite{pypl} and Stackoverflow Developer Survey 2021~\cite{survey}, Python is the most popular and the fastest-growing programming language (10.4\%) in the world as of Jun 2022.

% PYPL: http://pypl.github.io/PYPL.html
% Stackoverflow: https://insights.stackoverflow.com/survey/2021#technology

% Python has become the go-to computing language in college, it tops the list of the most-taught programming languages, especially in introductory computer science courses. For students who pursue a formal education in computer science or computer science related majors are extremely likely to be introduced to Python and even more likely to continue using Python throughout their career. 



\subsection{Pedagogical Approaches}
Programming language plays an important role in computing education. Many systematic studies explore the teaching in introductory-level programming classes~\cite{pears2007survey, schulte2006teachers, medeiros2018systematic, salleh2010empirical}. 

Through our observation, a tremendous number of research have been done on how the programming environment can improve students' coding experience in recent years, especially the programming environment for Java and Python, which are the most taught entry-level programming languages in college. 
The selection of the programming environment is typically decided by instructors. Pedagogical tools and Intelligent Tutoring Systems (ITS) ~\cite{pythontutor, van2004teaching, vanlehn2011relative, crow2018intelligent, naser2008developing} have been designed specifically for programming novices. Also, some works report the study of using industry-level programming environments such as Jupyter Notebooks~\cite{al2022jupyter, depratti2019using, barba2019teaching, van2020jupyter} or Eclipse extension~\cite{reis2004taming}, their experience shows how a great IDE improve students' engagement and performance. 




\subsection{IDEs in Education}
Integrated Development Environments (IDEs) refer to software applications that combine all tools needed for a software development project, including an editor, compiler, and debugger. They consolidate different aspects of software development and significantly improve programmers' productivity.


Many types of programming environments exist in the market. Roughly there are two kinds: plain-text/code editor and IDE. Text/code editor does not require complex installation and configuration, some editors are installed by default like NotePad or TextEdit, but they offer limited functionalities and are not directly related to programming. On the other hand, a full-featured IDE combines functionalities in one, integrating all tools developers need to build and test, but IDE requires more disk space/memory or a faster processor, users may suffer more from installation, configuration, even cost (some IDEs' license are expensive). However, there is not a clear boundary exist between them. Users may turn a text/code editor into an IDE by installing plug-ins/extensions. There are always trade-offs between the time spent on installation/configuration and how 
powerful the functionalities are.


% pay more time in installation and configuration


% that offers programmers with extensive development abilities, an IDE that understands your language's syntax well, usually provides features such as build automation, code linting, testing and debugging, those features significantly speed up and simplify your work. The software application development is a complex activity, an IDE combines all tools needed for a software development project, to be brief, an IDE provides an editor, compiler and debugger. 


It is important to introduce IDEs in programming courses, especially in introductory-level courses. 
% Students with little or no programming experience, tend to made more syntax or compilation errors, or mess code formatting
A great IDE helps students from two perspectives. 
First, it helps students write correct code. In our past teaching experience, common errors come up over and over again in students who have no or less programming experience. 
% % For example, a lot of students tend to have inconsistent usage of uppercase and lowercase, they often create variables with uppercase e.g. {\tt Year = 2022}, then try to reference it with lowercase - {\tt if year > 2000}, or import package name with uppercase e.g. {\tt import Math}, and wonder why the code piece doesn't work.
% The debugging for beginners is painful and time-confusing, such errors are pruned with the hint of IDE. With IDE students can debug programs efficiently and write qualified code smoothly. 
Second, it helps students establish good coding habits. Beginners focus more on correctness and tend to produce low-quality code, including messy code format, meaningless variable names, excessive function length, etc. Many IDEs support features like code formatting, variable name suggesting and function length warning,  helping students write clean and decent code that shows professionalism towards industry standards.

% {\color{orange} CITE: Introductory programming: a systematic literature review}