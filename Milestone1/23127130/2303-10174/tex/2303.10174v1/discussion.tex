
\section{Discussions}

\myparabb{Discussions on education-related concerns.}
% \paragraph{Discussions on education-related concerns}
VS Code has some advanced features that raise some debates for their use in introductory-level programming courses, such as code auto-completion and function signature hints. Some argue that supplementary features would develop a blind dependence on IDEs rather than truly understanding the code. In practice, our teaching plan covers all necessary programming knowledge and concepts, and features like auto-completion and function signature hints help students learn faster and more efficiently. Moreover, these features can be customized or completely disabled in VS Code if required.

\myparabb{Discussions on some limitations.}
We can see some limitations in this study. We only evaluate an introductory Python course. 
The reason is that our department offers Python as the introductory programming language. However, VS Code supports multiple programming languages, such as Matlab, Java, and C/C++. It is straightforward to adapt our guidance to courses with other programming languages. We will partner with instructors in our department to report more VS Code experiences.
% In our opinions, features like auto-completion help student learn faster and more efficient, 
% we always teach basic programming concepts in the first place. 


\section{Conclusions and Future Work}
% {\color{red} tutorial -> tutorials}
% {\color{red} computer science -> Computer Science}
This paper describes the experiences of introducing Visual Studio Code in an introductory (CS1) Python programming course at a big engineering university. In this paper, we investigate VS Code as a satisfactory IDE for CS1 programming courses. To better help students, we develop comprehensive VS Code guidance for students with various programming backgrounds. We perform evaluations among students and validate the practicality of VS Code and verify the quality of our VS Code guidance. We periodically update and improve the guidance with the collected feedback. We are now practicing VS Code and our VS Code guidance in more CS1 programming courses with programming languages besides Python.

Our future work is two-fold. First, we will continue exploring useful VS Code extensions available in the marketplace, which can benefit education. Second, we will develop useful VS Code extensions to further support education-related activities. Specifically, we will develop extensions that can better engage students when they are doing programming assignments.


% \begin{figure}[t]
%     \centering
%     \includegraphics[width=0.6\linewidth]{Figures/pie2.pdf}
%     \caption{The distribution of participants' operating system usage.}
%     \label{fig:pie2}
% \end{figure}



% \begin{figure}[t]
%     \centering
%     \includegraphics[width=0.75\linewidth]{Figures/stacked2.pdf}
%     \caption{The installation time spent by participants with different programming backgrounds.}
%     \label{fig:stack2}
% \end{figure}