% \section{Teaching with VS Code}
% We introduced VS Code in a CS1 Python course for non CS students in Fall 2022. The enrollment of the course was 141. The course used a flipped classroom format. Every week, students are expected to start conceptual learning of the topics covered in the week by watching a number of videos and completing the self-check quizzes online. Then, students are expected to attend one 100-minutes long class exercises session, and one 165-minutes long lab exercises session to participate in the active learning activities. Students will also work on weekly homework and project tasks after class (before or after lab) to reinforce the knowledge and skills of the week. Students are expected to create and run Python scripts in VS Code for completing all the programming tasks in lab exercises, homework assignments and projects. To increase engagement, students use both Jupyter Notebook Extension in VS Code and the regular VS Code for completing all the programming tasks in class exercises.

\section{Evaluation and Student Responses}

% \begin{figure}[t]
%     \centering
%     \includegraphics[width=0.9\linewidth]{Figures/stacked.pdf}
%     \caption{The distribution of participants’ responses to questions. The x-axis shows Question 1 to Question 8, and the y-axis indicates the number of participants. }
%     \label{fig:stacked}
% \end{figure}


In this section we evaluate two objectives: 

\begin{itemize}
\item Validate VS Code is a suitable IDE for the introductory-level Python programming courses

\item Verify the value and necessity of VS Code's Guidance

\end{itemize}


\subsection{Course Description}
% The experiment was conducted among students who are enrolled in {\it CSC111 --- Introduction to Computing -- Python}.
We introduce VS Code and VS Code guidance to a CS1 Python programming course in Fall 2022, the course enrollment is 141. This CS1 course is restricted for non-CS students, with an emphasis on basic programming skills and engineering applications. Students learn how to solve problems through writing Python programs, particular elements include: the development of Python programs from specifications; documentation and coding style; use of data types, control structures and data structures; abstractions and verification. 

The course uses a flipped classroom format. Every week, students start conceptual learning of the topics covered in the week by watching videos and completing self-check quizzes online. Then they attend a 100-minute class exercises session and a 165-minute lab exercises session to participate in the active learning activities. Students work on weekly homework assignments and project tasks after class to reinforce the knowledge and skills of the week. Students are required to run the in-class coding exercises via the Jupyter Notebook extension installed in VS Code, and complete the programming tasks (homework assignments, lab exercises, and projects) in VS Code by themselves. 

% \subsection{Experimental Study Setups}
% We conduct an experimental study in two introductory-level programming classes in Summer 2022. The participants are full-time students who are enrolled in: 
% {\it E115 --- Introduction to Computing environment}, an introductory course in Engineering Department, all incoming engineering students (first year, transfer, etc.) are required to complete. The course’s purpose is to better prepare students for using the computing and technology resources of the Engineering Department. 
% And {\it CSC116 --- Introduction to Computing}, also an introductory course in Computer Science Department, with an emphasis on basic computer organization, programming techniques, etc. In total, 34 students participate in our study. 

% Participants are invited to a 60-minute Zoom seminar. During the seminar, participants install and set up a programming environment with VS Code follow guidance, and validate the environment by running a few Python example programs. 
% Data is collected in two parts: participants answer questions during seminar and complete an online survey after the seminar. We aim to collect data regarding participants' programming background and their seminar feedback. 


% Complete survey after seminar, with questions regarding programming background and the seminar experience. 


\subsection{VS Code and VS Code Guidance Evaluation}

% \myparabb{Student/TA Surveys}
We designed surveys with 21 questions, to collect students' feedback on the VS Code and VS Code Guidance at the end of the semester. The survey first collects students' backgrounds, then invites students to rate both VS Code and VS Code guidance based on their programming experience throughout the semester. In total, 42 valid responses were collected.

% \myparabb{Student Background.} 
First, we collect students' background information. As a typical CS1 course, 79\% of students are freshmen and sophomores, 86\% of students did not take any programming course in the past, and 74\% of them have less than one year of coding experience.

% {\color{orange} 65\% of students are willing to learn Python in their future careers.}


% \begin{table*}[ht]
%     \centering
%     \caption{Programming background grading criteria. The left column shows the question descriptions, right five columns show the score students gain for selecting the corresponding answer to questions.}
%     %\vspace{-.5em}
%     % \small
%     %\begin{tabular}{ ||c||c|c|c|c|c|c|c|c|c|c|c|c|c||}
%     \resizebox{\linewidth}{!}{
%     \begin{tabular}{ c|ccccc }
%     \toprule
%     Question Descriptions & 1 & 2 & 3 & 4 & 5\\
%     \midrule 
%     Q1. What is your major? & Not declared & Art, Business, Law & Math/Science & Engineering & Computer Science \\
%     \midrule 
%     Q2. What is your current year of study? & Fresh & Sophomore & Junior & Senior & Graduate\\
%     \midrule 
%     Q3. Do you have interest in programming? & No & just for course & A little & interested & Strong interest\\
%     \midrule 
%     Q4. Have you taken Python-related course in the past? & No & - & - & - & Yes\\
%     \midrule 
%     Q5. Have you tried other programming languages? & No & - & Yes & - & Yes, more than two\\
%     \midrule 
%     Q6. How many years of coding experience do you have? & None & - & less than 1 year & 1-3 year & more than 3 years\\
%     \midrule 
%     Q7. Did you plan to take advanced-level \\ programming courses in the future? & No & Not sure & Yes & - & -\\
%     \midrule 
%     Q8. Do you plan to pursue a major/job in Computer Science? & No & Not sure & - & - & Yes \\
%     \bottomrule
%     % \multicolumn{14}{l}{{\vbox to 2ex{\vfil}} N: Not Optimized; P: Partially Optimized; S: Successfully Optimized.} \\
%     \end{tabular}
%     }
%     %\caption{NO: Not Optimized; PO: Partially Optimized; SO: Successfully Optimized}
%     \label{tab:grading}
%     %\vspace{-1em}
% \end{table*}



% \subsection{Programming Background Analysis}

% \begin{figure}[t]
%     \centering
%     \includegraphics[width=0.65\linewidth]{Figures/pie.pdf}
%     \caption{The distribution of participants’ programming background category.}
%     \label{fig:pie}
% \end{figure}


% % We conduct experiments in two introductory classes in 2022 Summer. 
% This section reports the participants' programming background. Although participants are enrolled in the introductory-level class, it's still necessary to evaluate their programming background. To better classify their programming experiences, we provide questions during the seminar and in the post-seminar survey. 
% We assign scores for each answer to the question, sum up the total score according to participants answers, and classify participants into three categories based on their total scores: beginner, well-grounded and skillful. 


% Table~\ref{tab:grading} explains programming background grading criteria. 
% To check participants' programming background, we spread out eight questions in total throughout the seminar and in post-survey. A score from 1-5 is assigned for answers to each question, based on participants' total score they are classified into three categories of programming background: 

% \begin{itemize}
%     \item Beginner (0-15): one has no or limited programming experience, and requires step-by-step guidance for programming environment setup.
    
%     \item Well-grounded (16-30): one has some programming experience and understands basic concepts in software development, may need helps during programming environment setup.
    
%     \item Skillful (30-40): one has rich programming experience and software development knowledge, and is familiar with programming environment setup.
    
    
% \end{itemize}


% Figure~\ref{fig:stacked} describes the distribution of participants' responses from Question 1 to Question 8. 
% Question 1 and 7 show that 82.4\% of participants major in STEM (Science, Technology, Engineering, and Math), and 67.6\% of participants may take advanced-level programming courses in the future semesters. As described in Section~\ref{investigation}, VS Code has strong functionalities to fulfill different software development requirements, students will be able to use the IDE they are familiar with in advanced-level class/project without extra effort. 
% Question 2 shows that 85.3\% of participants are freshmen and sophomores. It may be too early for college students to think about their future careers in the first two years, but Question 8 shows that there are 30\% participants who feel certain about pursuing a major or job in Computer Science while 47.1\% are not sure about it. 
% VS Code is the most popular IDE in development and is widely-used in industry, in total nearly 80\% participants in introductory-level programming class may continually take advantage of and program with VS Code in their future career.



% % There are 61.7\% participants plan to take advanced-level programming courses in the future.

% As for the coding skill, Question 4 shows that 82.3\% participants have not taken Python-related courses in the past. Question 5 shows that for 73.6\% of participants, Python is the first programming language they learn. Question 6 shows that 85.3\% of participants have none or less than one-year of programming experience. Overall, the majority of introductory-level Python class has limited experience in programming, and only 17.6\% of participants are well-interested in programming. As described in our investigation, VS Code is easy-to-use and beginner-friendly, it helps students write correct Python code with high quality. Involving VS code with guidance in introductory-level Python classes improves the learning experience and cultivates better coding habits for students. In addition, a smooth start to the programming journey builds students' confidence and promotes their interest in programming. 


% Figure~\ref{fig:pie} describes participants' programming background category distribution based on the grading criteria listed on Table~\ref{tab:grading}.
% Almost 60\% of participants are beginners, who require comprehensive guidance for programming environment setup. Furthermore, the programming backgrounds are truly diverse even for students in introductory-level programming classes, tutorials with hierarchical indexing are necessary and improve learning efficiency.


% % \begin{figure}
% %     \centering
% %     \includegraphics[width=0.7\linewidth]{Figures/pie2.pdf}
% %     \caption{Participants' operating system usage distribution.}
% %     \label{fig:pie2}
% % \end{figure}

% We also collect the distribution of participants' operating system usage, 
% % As described in Figure~\ref{fig:pie2},
% 70.6\% participants use MacOS whereas the rest use Windows. The dual-version guidance covers OS choices for all participants. We observe that a variety of hardware and OS versions exist in the same type of OS. 
% Benefit from VS Code good accessibility, massive time will be saved in class since students have identical programming environments and settings neglecting different devices and OS versions. 

% helping instructor saves time xxx TODO!!!


% \myparabb{VS Code Performance.} 

Then students are invited to rate the VS Code and VS Code guidance performance throughout the semester. They rate VS Code from the following four aspects, the degree of satisfaction lies between 1 to 5, in which 1 is strongly dissatisfied and 5 is strongly satisfied.
% VS Code's visual appeal, extension ecosystem, debugging experience, and coding experience. 

\begin{itemize}
    \item Visual appeal: rate the VS Code's user interface, and the editor layout.
    \item Extension ecosystem: rate the way to search/install/uninstall extensions.
    \item debugging experience: rate the VS Code's built-in debugger, whether it helps accelerate students' edit, compile, and debug loop, and if the recommended debugger extensions are helpful. 
    \item editing experience: rate the overall editing experience of VS Code. If the basic editing features (i.e., keyboard shortcuts and Command Palette) are useful and beginner-friendly. Also rate IntelliSense, if code editing features such as code completion, parameter info, and content assist are helpful.
    
\end{itemize}

\begin{table}[H]
    \centering
    \caption{The averaged answers of 42 students' satisfaction over four aspects: visual appeal, extension ecosystem, debugging experience, and editing experience. The degree of satisfaction lies between 1 to 5, in which 1 is strongly dissatisfied and 5 is strongly satisfied.}
    \resizebox{\linewidth}{!}{
    \begin{tabular}{ c|c|c|c|c }
    \toprule
    Aspects & Visual Appeal & Extension & Debugging & Editing\\
    \midrule
    Average Answers & 4.17 & 3.81 & 4.02 & 4.05 \\
    \bottomrule
    % \multicolumn{5}{l}{{\vbox to 2ex{\vfil}} *: the degree of satisfaction lies between 0 to 5.} \\
    \end{tabular}
    }
    \label{tab:VS1}
    %\vspace{-1em}
\end{table}

Based on the responses, 61\% of students do not have IDE or coding platform experience in the past, and 39\% of students have used some other IDEs such as Atom, Spyder, PyCharm/CLion/IntelliJ, etc. , As shown in Table~\ref{tab:VS1}, the average rating for visual appeal, extension ecosystem, debugging experience, and editing experience are 4.17, 3.81, 4.02, and 4.05, respectively.


The students' overall satisfaction with VS Code guidance is 4.2. With the help of detailed tutorials in VS Code guidance, 74\% students consider VS Code quite easy to install and use, and 76\% students consider VS Code as a good IDE to code with. There are 13 students who claim they had issues/trouble with VS Code guidance throughout the semester, and we have collected these issues and fixed/updated them in the VS Code guidance. 



% \myparabb{VS Code Guidance Performance. } Students are invited to rate the VS Code Guidance performance. In the survey, we offer the following questions to evaluate the value of VS Code Guidance:

% \begin{itemize}
%     \item VS Code Installation: with the help of VS Code Guidance
% \end{itemize}

% \begin{itemize}
%     \item 
% \end{itemize}


% \begin{table}[H]
%     \centering
%     \caption{.}
%     \resizebox{\linewidth}{!}{
%     \begin{tabular}{ c|ccccc }
%     \toprule
%     Questions & Visual Appeal & Extension & Debugging & Coding\\
%     \midrule
%     Answers & 4.17 & 2.82 & 2.85 &  \\
%     \bottomrule
%     \multicolumn{5}{l}{{\vbox to 2ex{\vfil}} *: 30 out of 34 participants have not met any installation trouble.} \\
%     \end{tabular}
%     }
%     \label{tab:}
%     %\vspace{-1em}
% \end{table}



% \myparabb{Installation time.}
% {\color{orange} no need}


% \begin{itemize}
%     \item Q9. Do you think VS Code that was introduced in seminar is easy to install? Rate the easiness of the installation (with 1 being hardest and 5 being easiest)
%     \item Q10. Do you think the guidance is helpful and easy to follow?  (with 1 being hard and quite challenging to follow with, 2 being helpful but could have more details, and 3 being very helpful and easy to follow with)
%     \item Q11. Will you recommend this guidance to someone who wants to install VS Code? Rate your recommendation (with 1 being no, 2 being not sure, and 3 being yes)
%     \item Q12. Have you met any installation trouble? 
    
% \end{itemize}

% Table~\ref{tab:average} shows the answers average of Question 9 to Question 12. Overall, the result indicates that VS Code guidance has a positive impact on students' learning phase. However, there are still 4 (out of 34) participants who met installation trouble, we collected their problems and update guidance. As described in Section~\ref{guidanceimp}, keeping guidance updated is essential. 


% \begin{table}[ht]
%     \centering
%     \caption{The averaged answers to Question 9 - Question 12.}
%     \resizebox{\linewidth}{!}{
%     \begin{tabular}{ c|ccccc }
%     \toprule
%     Questions & Q9 & Q10 & Q11 & Q12\\
%     \midrule
%     Answers Averages & 4.17/5 & 2.82/3 & 2.85/3 & 88.2\%* \\
%     \bottomrule
%     \multicolumn{5}{l}{{\vbox to 2ex{\vfil}} *: 30 out of 34 participants have not met any installation trouble.} \\
%     \end{tabular}
%     }
%     %\caption{NO: Not Optimized; PO: Partially Optimized; SO: Successfully Optimized}
%     \label{tab:average}
%     %\vspace{-1em}
% \end{table}

% We trace the time participants spent on installation and setting up the programming environment with VS Code guidance.
% % starting from the beginning of VS Code installation, and ending with running sample Python programs correctly. 
% Figure~\ref{fig:stack2} describes the time usage for participants with different programming backgrounds. The average time spent by beginner, well-grounded, and skillful participants is 26.7, 20.7, and 6.5 minutes, respectively.


% We observe that, with hierarchical guidance, skillful participants accurately locate the information they need by keywords, and finish installation in a short period. Well-grounded participants efficiently skip steps they know and spend relatively longer time than skillful participants. And beginner participants follow guidance step by step, and have the longest time among the three categories of participants.
% The fact that some beginner participants spend much more time than others is because they prefer following video version tutorials.
% There is one well-grounded participant who spent 36 minutes, he spent a longer time because he used a relatively old version of VS Code in Windows 8. To avoid such a problem, we update the tutorials and request installing a consistent version of VS Code in future semesters.

\subsection{Class Averages}

We also compare the class averages of Fall 2022 with Spring 2022, the results are shown in Table~\ref{tab:GPA}. In Fall 2022, the class average total score is 85.10\%, and the class average of all the coding assignments is 89.31\%, whereas 82.86\% and 88.37\% in Spring 2022, respectively. 


\vspace{-0em}
\begin{table}[H]
    \centering
    \caption{The class average of coding assignments and total score in the Spring and Fall 2022 semesters.}
    \resizebox{0.7\linewidth}{!}{
    \begin{tabular}{ c|c|c }
    \toprule
    Class Average & Coding Assignment & Total Score\\
    \midrule
    Spring 2022 & 88.37\% & 82.86\%  \\
    \bottomrule
    Fall 2022 & 89.31\% & 85.10\%  \\
    \bottomrule
    % \multicolumn{5}{l}{{\vbox to 2ex{\vfil}} *: the degree of satisfaction lies between 0 to 5.} \\
    \end{tabular}
    }
    \label{tab:GPA}
    %\vspace{-1em}
\end{table}

\vspace{-1em}
Although the comparison in Table~\ref{tab:GPA} shows that the class averages of Fall 2022 are higher than Spring 2022, we cannot simply accredit the average raise to VS Code or VS Code guidance. In Spring 2022, we used the traditional classroom, and students are required to use Spyder integrated within Anaconda, whereas we use the flipped classroom and VS Code in Fall 2022. But according to the student's experience, we can conclude that both VS Code and VS Code guidance have positive effects and are useful in completing coding assignments.



\subsection{Student Comments}
At the end of the student survey, students are invited to comment on their VS Code and VS Code guidance experience throughout the semester. One student described his/her overall experience as:
\begin{quote}
    I can view Python files and text files alike without even making a new window using VS Code. The interface is visually appealing. The VS Code tutorial is straightforward. Instructions are easy to follow. That's all a tutorial really needs, in my opinion.
\end{quote}

Another student especially liked the user interface of VS Code:
\begin{quote}
     I like how user-friendly VS Code is, the overall interface is easy to understand, allowing us to focus more on learning and applying Python concepts. 

% I like how user-friendly VS Code is. It gives helpful feedback and it's easy to use.
\end{quote}


Some students favor the dark mode in VS Code, one student claimed it improves his/her coding experience:
\begin{quote}
    I like VS Code uses color coordination to make different code types and functions stand out. I also like that by default the background is black it makes it easier to read. 90\% of the time it's very easy to find mistakes within codes. Overall the tutorial is easy to follow. 
\end{quote}
We observe that many students prefer dark mode compared with bright mode, it may be because dark mode or dark theme stands out the syntax highlighting. With a light background, the code texts are mostly darker colors, while more colorful with a dark background. In addition, compared with pedagogical IDEs, changing themes in VS Code is quite easy. Students highly raise the theme extensions in VS Code, one student claimed:


% VS Code supports various theme extensions that are quite easy to install, student really like the wide range of color combinations, one student claims: 

\begin{quote}
    I like the way how VS Code installs extensions, I especially love theme extensions in Marketplace, whenever I am bored with codes I change my theme, and it just becomes a totally new software!
\end{quote}


Some students enjoy the VS Code guidance by simply saying: 
\begin{quote}
    I was pleased with how clear the steps were, and that in most steps there were screenshots to go along with the instructions. 
\end{quote}


We also find students like the hierarchical indexing in the VS Code guidance, one student commented on it as:
\begin{quote}
    It was simple to follow and the links were embedded into the steps. Also, there are pictures and screenshots to help guide you in each step, it helps me to prepare this material before classes.
\end{quote}


From the feedback, we collect many constructive suggestions for both VS Code and VS Code guidance. Some students complain about the complex commands on the terminal: 
\begin{quote}
    The terminal's instructions are complicated and hard to understand.
\end{quote}
It's true that we may involve plenty of terminal instructions at the beginning of class, but the instructor thinks those instructions/commands are required for this class. To fix this problem, we divide the command study into different labs and add detailed explanations on every instruction.  

Another student considered VS Code guidance useful, but suggested we could involve more helper extensions in class:
% \begin{quote}
%     Tutorials are thorough, pretty straight to the point, also easy to understand. But in my opinion, VS Code lacks collaborative elements, especially in the lab. There is an extension that takes about 1 minute to install and log in using university ids that can allow you and your lab partner to collaborate in real time on the same Python file. My lab partner and I did this when we had to do debugging and coding exercises during the lab portion of the course. I feel that collaboration via a `shared' document would be beneficial because, without it, one user mostly does the code by themselves.
% \end{quote}

\begin{quote}
    Tutorials are thorough, pretty straight to the point. But in my opinion, VS Code lacks collaborative elements, especially in the lab. An extension that takes about 1 minute to install allows you and your lab partner to collaborate on the same Python file. My lab partner and I did this when we had to do debugging and coding exercises during the lab portion of the course. I feel that collaboration via a `shared' document would be beneficial because, without it, one user mostly does the code by themselves.
\end{quote}

The live share extension enables students to share screens and collaborate with classmates/TAs/instructors on the same code without the need to sync code or configure the same development tools, settings, or environment~\cite{liveshare}. We are evaluating the feature and will introduce it to students after careful consideration. 

% but we are still evaluating it due to the plagiarized.

% ADD:
% \begin{quote}
%     Loading things such as NumPy was a bit complicated but with a tutorial can be done
% \end{quote}











% \begin{figure}
%     \centering
%     \includegraphics[width=0.8\linewidth]{Figures/stacked2.pdf}
%     \caption{xxx.}
%     \label{fig:stack2}
% \end{figure}



% \begin{table}[ht]
%     \centering
%     \caption{Time spent on VS Code tutorial.}
%     \resizebox{\linewidth}{!}{
%     \begin{tabular}{ c|cccccc }
%     \toprule
%     Minutes & 0-10 & 10-20 & 20-30 & 30-40 & 40-50\\
%     \midrule
%     Participants Number & 3 & 14 & 6 & 9 & 2 \\
%     \bottomrule
%     % \multicolumn{5}{l}{{\vbox to 2ex{\vfil}} *: 30 out of 34 participants do not met any installation trouble.} \\
%     \end{tabular}
%     }
%     %\caption{NO: Not Optimized; PO: Partially Optimized; SO: Successfully Optimized}
%     \label{tab:time}
%     %\vspace{-1em}
% \end{table}






% As a result, we plan to introduce VS Code and tutorials in another introductory level Python programming class with greater capacity of students (more than 200 students are enrolled already).



% \begin{figure}[t]
%     \centering
%     \includegraphics[width=0.6\linewidth]{Figures/pie2.pdf}
%     \caption{The distribution of participants' operating system usage.}
%     \label{fig:pie2}
% \end{figure}



% \begin{figure}[t]
%     \centering
%     \includegraphics[width=0.75\linewidth]{Figures/stacked2.pdf}
%     \caption{The installation time spent by participants with different programming backgrounds.}
%     \label{fig:stack2}
% \end{figure}



% {\color{orange} installation time spent by participants in Summer 2022.

% reorganize here.}

To summarize, students' overall experience with VS Code and VS Code guidance is positive and encouraging, which shows that VS Code and guidance are valuable and promising for the CS1 course. 



% VS Code as an IDE is suitable for the CS1 programming course, together with VS Code Guidance, students 

