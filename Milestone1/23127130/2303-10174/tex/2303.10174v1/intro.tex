\section{Introduction}
% It's challenging for Computer Science educators to choose a proper coding environment for their students in the entry-level programming courses. It needs to support different operating systems. It needs to be easy to install, configure, and use. It's preferred to have interactive features to increase students' engagement in the classroom. It's better to be an environment that they can continue to use in their future high-level courses, and future career. Different educators adopt different ways. However, it's hard to score high on all the features mentioned above. 

% Visual Studio (VS) Code is a lightweight IDE developed and supported by Microsoft, which is free for private or commercial use. It is widely-used in the industry nowadays. In this research study, we want to answer the following research questions:

% (1) Is it appropriate to use VS Code in the entry-level Python courses?

% (2) If so, how to introduce VS Code in the entry-level Python courses?


% Python tops the first teaching programming language at top 50 U.S. universities~\cite{}.
% https://cacm.acm.org/blogs/blog-cacm/176450-python-is-now-the-most-popular-introductory-teaching-language-at-top-us-universities/fulltext
%{\color{blue}Programming environment plays an important role in learning a programming language. It's challenging for Computer Science educators to choose a proper environment for their students in the entry-level programming courses. It needs to be easy enough to install and use. It's better to be an environment that is close to the ones that are used in the real professional world. In the early years, programming environments used in the industry were not easy to use for the beginners. Many computer science education research focus on developing easy-to-learn and easy-to-use programming environment for education purpose, especially for helping entry-level students to reduce the learning curve \cite{} (Python tutor? Need to do some search). However, students need to switch from those environment to the real professional environment later in their advanced courses or at the beginning of their professional career. In recent years, Integrated Development Environments (IDEs) have been developed tremendously. IDEs offer programmers easy-to-use interface and extensive development abilities. They understand language syntax and provide features such as build automation, code linting, testing, and debugging, which accelerate and simplify the coding process. Through the help of IDEs, programmers benefit from efficient programming, testing, and debugging. Nowadays, more and more software engineers use those IDEs in their daily work, and more and more educators start involving IDEs in their courses.}

% {\color{red} Visual Studio (VS) Code is a lightweight IDE developed and sup-
% ported by Microsoft. Because of its easy-to-use, multi-language support, VS Code has become a mainstream IDE in both industry and academia \cite{}. More and more students transit to use VS Code from other IDEs in learning their advanced computer science courses, as it will make them better prepared for their professional career. 

% Based on this trend, we wonder if we can directly introduce VS Code in the entry-level programming course as students' first IDE, in this case, students don't have to transit later. As the introductory computer science educators, we choose to use an IDE that satisfies four features: 1) it is cross-platform supported, students are able to install and use it identically on different OS/hardware; 2) it is easy to use, students can code effectively with a simple and concise interface; 3) it has extensive functionalities, e.g. code compilation, project organization, and multi-languages support, etc.; 4) it is mainstream among professional software engineers, which means the IDE is widely used in industry, bridging the gap between introductory-level class and future career. We think VS Code has all four aforementioned features. However, to our best knowledge, there is not any publication to report using VS Code as the default IDE in the entry-level computer science courses. Moreover, there is no comprehensive guidance on VS Code for educators and students. Since students have various backgrounds, e.g. from diverse majors, with uneven learning speeds, or are familiar with different operating systems, comprehensive guidance is necessary to facilitate the efforts of both students and educators. 

% In this paper, we report our experience of using VS Code as the default IDE in a large Computer Science 1 (CS 1) Python course in Fall 2022. We create the first comprehensive guidance of VS Code for both students and educators by gathering plugins (i.e., VS Code extensions) for Python education and various user experiences. The guidance is highly modularized.  Students who have programming experience jump to the sections they need to learn and skip the sections they already know; students who are beginner programmers follow the guidance step-by-step; students who come across errors target why errors occur and how to solve them. With the help of our guidance, students can quickly acquire the necessary knowledge of the programming environment in the class. We distribute a survey to students at the end of the semester to evaluate the effectiveness of our efforts. From the survey, the students highly value the use of VS code and our guidance. Students with some industry internship experiences acknowledge the usefulness of learning Python with VS Code over other education-oriented IDEs, as VS Code is widely used in the industry.

% Students should be given more opportunities to familiarize themselves with important industry standard tools, and we believe choosing IDE which is popular among industrial help students prepare well.

% }

Integrated Development Environments (IDEs) play an important role in learning a programming language. IDEs offer programmers extensive development abilities. They understand language syntax and provide features such as build automation, code linting, testing, and debugging, which accelerate and simplify the coding process. 
Through the help of IDEs, students benefit from efficient programming, testing, and debugging. Students can further develop better coding habits and flatten the learning curve of a new language.
As a result, more and more instructors start involving IDEs in introductory-level Python courses such as Atom~\cite{atom}, Jupyter Notebook~\cite{perez2015project, al2022jupyter, van2020jupyter}, and many others, which significantly improve student's coding experiences.

However, not all IDEs are suitable for introductory-level Python courses. Choosing a satisfying IDE is difficult for instructors. 
On the one hand, professional IDEs~\cite{vim, eclipse, intelliJ} support an integrated programming environment and many powerful features but with limited support for education. One may need advanced knowledge and plenty of time to use them properly. For students who are new to programming, it is hard to take advantage of IDEs on top of learning a new programming language, because plenty of time is spent on software installation and environment setup. On the other hand, education-focused code editors are easily installed and manipulated, but they are rarely used during professional software development cycles. Students face a big gap when transiting from college to industry~\cite{valstar2020faculty, valstar2020quantitative}, which has some negative impacts on their future careers.

% for students to employ a new programming language. More important, code editors 
\sloppy
Thus, there is urgent to use an appropriate IDE for the introductory-level Python class, which not only provides enough education-related features but also is mainstream among professional software engineers. We aim to use an IDE that satisfies four features: 1) it is cross-platform supported, students are able to install and use it identically on different OS/hardware; 2) it is easy to use, students can code effectively with a simple and concise interface; 3) it has extensive functionalities, e.g. code compilation, project organization, and multi-languages support, etc.; 4) it is mainstream among professional software engineers, which means the IDE is widely used in industry, bridging the gap between introductory-level class and future career.

Thus, we identify Microsoft Visual Studio Code (VS Code) as the desired IDE, which has all four aforementioned features. Moreover, VS Code has been adopted in many advanced courses in our department, such as operating systems, compiler constructions, computer networks, and many others.
However, VS Code does not draw significant attention to CS1 courses. Furthermore, with our study of 20 computer science departments, none of them specify VS Code as the default IDE in the introductory Python courses. Moreover, there is no comprehensive guidance on VS Code for educators and students. 
Since students have various backgrounds, e.g. from diverse majors, with uneven learning speeds, or are familiar with different operating systems, comprehensive guidance is necessary to facilitate the efforts of both students and educators. 

In this paper, we give the first experience report for the use of VS Code in an introductory (CS1) Python course. We create the first comprehensive guidance of VS Code for both students and educators by gathering plugins (i.e., VS Code extensions) for Python education and various user experiences. The guidance is highly modularized. Students who have programming experience jump to the sections they need to learn and skip the sections they already know; students who are beginner programmers follow the guidance step-by-step; students who come across errors target why errors occur and how to solve them. With the help of our guidance, students can quickly acquire the necessary knowledge of the programming environment in the class. We already integrate VS Code and our guidance into the introductory-level Python course in our department. We use a survey to evaluate the effectiveness of our efforts. From the survey, the students highly value the use of VS code and our guidance. Students with some industry internship experiences acknowledge the usefulness of learning Python with VS Code over other education-oriented IDEs, as VS Code is widely used in the industry. 


% \newpage
\myparabb{Contributions.} We make the following four contributions. 

\begin{itemize}
    \item We point out the importance of selecting a proper IDE for introductory-level courses.  
    \item We identify Visual Studio Code as a desired IDE. We investigate it from four aspects and analyze its practicality for introductory-level Python courses.
    \item We propose the first comprehensive guidance\footnote{The VS Code guidance will become public upon this paper's acceptance.} with hierarchical indexing, to guide students with diverse backgrounds.
    \item We report our experiences in using VS Code in an introductory-level programming course and show that VS Code is a satisfactory IDE for the introductory-level Python course. We also verify the value and necessity of the guidance.

\end{itemize}

\myparabb{Organization.} Section 2 reviews the background and related work. Section 3 shows our investigation of VS Code. Section 4 describes the guidance and support we offer to students. Section 5 shows some experimental studies. Section 6 provides the discussion. Section 7 presents conclusions and future work.



% {\color{orange}
% \myparabb{Student preparation for industry: } the day-to-day operations in the industry have a very different set of core skills and tools than what is traditionally presented in a CS curriculum. 

% To raise students' vocational skills --- being fluent in software engineering principles, familiar with the tools, and able to collaborate and communicate well. 

% Students should be given more opportunities to familiarize themselves with important industry standard tools, and we believe choose IDE that is popular among industrial help students prepare well.

% \cite{valstar2020faculty, valstar2020quantitative} 