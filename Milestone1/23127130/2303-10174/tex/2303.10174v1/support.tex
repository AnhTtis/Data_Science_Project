\section{Guidance and Support}

To accelerate the learning process of VS Code for students with various backgrounds, we introduce comprehensive guidance in Python class. The guidance\footnote{Guidance figures are not shown in the paper for anonymous purposes, and the guidance will become public upon this paper's acceptance.} is divided into separate tutorials, the following subsections explain the detail of each tutorial. 


\myparabb{Download and installation of VS Code.}
In this tutorial, students are guided to the download page, and how to choose the correct version according to their OS. We introduce the workspace right after the installation because it is the fundamental unit to manipulate projects. We give examples of how to start a VS Code project, and how to organize files in a project.

% We believe these concepts should be explained in the beginning of learning phase.

% https://code.visualstudio.com/docs/editor/workspaces

% These concepts should be explained. 



\myparabb{Download and installation of Python.}
In our past teaching experience, we suffer three main problems with Python version: 1. not being aware of the difference between Python2 and Python3; 2. installing the wrong version; 3. having both Python2 and Python3 installed on the same device, but don't know how to switch them.

% Or students have both Python 2 and Python 3 installed at the same device, and it is hard to switch Python version due to the change of environment path. 

In the tutorial, we require students to use Python3 and emphasize the difference between Python2 and Python3. Before the installation, students need to check if they have any versions of Python pre-installed. If they don't have or have Python2 installed, they are led to the newest Python3 download link. If they have any version of Python3 installed, then there is no need to install it again. For students with multiple versions of Python installed, we provide a detailed guide on switching Python versions in the next tutorial. 


\myparabb{Set up Python environment.}
Next, students install the Python extension and set up the Python environment in VS Code. In our past teaching experience, students were always confused with the programming environment settings, we pay extra effort into where and how to set/change configure settings. 

Then students are guided to the extension marketplace, to find and install the Python extension developed by Microsoft. The Python extension works on/with any operating system/Python version, the Jupyter extension is included in the Python extension installation bundle.
The Python extension supports code completion and IntelliSense on top of the currently selected Python interpreter, we provide step-by-step guidance with figures on how to select a Python interpreter from {\it Command Palette}. We also provide guidance on how to change programming language mode, for students who want to work with other programming languages in later classes. 


% IntelliSense is a general term for a number of features, including intelligent code completion (in-context method and variable suggestions) across all your files and for built-in and third-party modules~\cite{}

% https://code.visualstudio.com/docs/languages/python


\myparabb{Run Python examples. }
Once set up Python environment, VS Code becomes a real Python IDE. In this tutorial, we guide students to write, run their first simple Python program from scratch, and give explanations on how Python extension improves coding. We require students to generate expected outputs for example programs, such that they have identical settings for the Python environment. 

\myparabb{Install and run Jupyter extension.} We use VS Code's Jupyter extension to run in-class coding exercises to improve student engagement~\cite{perez2015project, al2022jupyter}, through the Jupyter extension students open and run ipynb files on VS Code the same way as Python files. The Jupyter Notebook extension provides basic notebook support for language kernels and allows any Python environment to be used as a Jupyter kernel. In this tutorial, we guide students to install the Jupyter extension, create new Jupyter files, and manage and run the code cells.

\myparabb{Install and use scientific packages.} Python Packages are very efficient to solve complex problems in scientific computing, data visualization, data manipulation, and many other fields. In this tutorial, we guide students to install and import Python libraries, e.g., {\tt Numpy}, {\tt Pandas}, {\tt Matplotlib}, and {\tt SciPy}. Then students are required to learn and use basic APIs, and the example codes are provided. To succeed in this tutorial, students need to plot a certain figure using {\tt Matplotlib} functions.

\subsection{Improvements for Guidance}
\label{guidanceimp}

To better guide students, the guidance comprises optimizations as follows:  

\begin{itemize}
    % \item Though VS Code and Python are well cross-platform supported, a single guidance for all operating systems is not enough. 
    
    \item We provide two versions of guidance for MacOS users and Windows users, respectively. In our experience, a single version of guidance for all OSs is not adequate, minor differences among OSs matter in the introductory-level course. 
    
    \item We provide three ways to access guidance: PDF download, website, and video. The conventional way is to upload guidance to the course learning management system as PDF, but its format (PDF) has limited representations, therefore we add a website version of the guidance. Due to COVID-19, students might not be able to get enough in-person instruction, we also provide video guidance to improve student engagement. 
    
    \item We constantly collect students' feedback, and periodically improve and update the guidance.
    
\end{itemize}



\subsection{Hierarchical Indexing}

We provide detailed descriptions of tutorials, students follow tutorials step-by-step, and set up a programming environment without effort. However, students' preferences regarding tutorials vary with their programming backgrounds. 
Tutorials with a flat structure cannot satisfy all students' requirements. 
Students who prefer video tutorials may not follow the video pace from the beginning. A common scenario is to start with a verbal demonstration and refer to a video when problems occur. It is difficult to locate the corresponding video timestamp from text if tutorials are flat structured. In addition, students with some programming knowledge require more concise messages, e.g. have installed Python and VS Code, but require guidance on Python environment in VS Code, which fail to provide if tutorials are flat structured. For students who are already familiar with the environment setup, it is hard for them to extract key messages from flat-structured tutorials. 


We design hierarchical indexing on tutorials to fulfill diverse requirements. The hierarchical indexing consists of three levels: 1) High-level: we provide an abstract with all key messages at the beginning of each tutorial, such as software names, download links, software versions, etc. 2) Medium-level: we segment each tutorial into several parts and summarize with subtitles. We list subtitles and link the corresponding positions in tutorials. Students can locate the information they need without effort. 3) Low-level: we build mappings between video and verbal demonstration, i.e., verbal demonstrations for each step within a subtitle are linking the corresponding video timestamps. Thus, students switch between verbal and video demonstrations when problems occur. 


%\subsection{Installation Time}
%Prior to introducing VS Code and VS Code Guidance to a large programming course, we invite 34 students who are enrolled in a CS1 programming course in Summer 2022 to try the VS Code Guidance first. Participants are invited to a 60-minute Zoom seminar. Participants install and set up programming environments with VS Code following guidance, and validate the environment by running a few Python example programs. The whole process is completed all by the students themselves with the help of VS Code Guidance. We trace the time that participants spent on the installation and set-up of programming environments, the average time spent by participants is 20.7 minutes. We observe that students respond well to hierarchical guidance. Skillful participants accurately locate the information they need by keywords and finish installation in a short period. Well-grounded participants efficiently skip steps they know and spend relatively longer time than skillful participants. And beginner participants follow guidance step by step and have the longest time among the three categories of participants.
%The fact that some beginner participants spend much more time than others is that they prefer following video version tutorials.

%Of the positive feedback among a small group of students, we are confident about the VS Code Guidance and introducing it to a large group of students in the following Fall 2022, the experience is described in Section 5.

% There is one well-grounded participant who spent 36 minutes, he spent a longer time because he used a relatively old version of VS Code in Windows 8. To avoid such a problem, we update the tutorials and request installing a consistent version of VS Code in future semesters.