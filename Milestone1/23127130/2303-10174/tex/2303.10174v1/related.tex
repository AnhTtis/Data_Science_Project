% \section{Related Literature}

% \subsection{Pedagogical Approaches}
% Programming language plays an important role in computing education. Many systematic studies explore the teaching in introductory-level programming classes~\cite{pears2007survey, schulte2006teachers, medeiros2018systematic, salleh2010empirical}. 

% Through our observation, a tremendous number of research have been done on how the programming environment can improve students' coding experience in recent years, especially the programming environment for Java and Python, which are the most taught programming language in college. 
% The selection of the programming environment is typically decided by instructors. Pedagogical tools and Intelligent Tutoring Systems (ITS) ~\cite{pythontutor, van2004teaching, vanlehn2011relative, crow2018intelligent, naser2008developing} have been designed specifically for programming novices. Also, some works report the study of using industry-level programming environments such as Jupyter Notebooks~\cite{al2022jupyter, depratti2019using, barba2019teaching, van2020jupyter} or Eclipse extension~\cite{reis2004taming}, their experience shows how a great IDE improve students' engagement and performance. 
