In Table~\ref{tab:finetuning_prompts}, we report the prompts we use for finetuning on summaries and finetuning on feedback + refinements. The completion for finetuning on summaries indicates that we can have completions generated from various sources, i.e., either initial summaries from \textit{FeedMe}, refinements generated with our method, or ideal human written summaries. For finetuning feedback + refinements, we first generate the feedback and then the refinement.


\begin{longtable}{p{0.2\textwidth} p{0.4\textwidth} c c }
\toprule
    \textbf{Methods} &  \textbf{Prompt} & \textbf{Completion} \\ 
  \hline
    \makecell{\textsc{Finetuning on} \\ \textsc{Summaries}} & Write an excellent summary of the given text.  & \makecell{ \{\texttt{summary/refinement/human summary}\}}  
    & \\
    & Title:  \{\texttt{title}\} & \\
    & \\
    & Text:  \{\texttt{post}\} & \\
    & \\
    & TL;DR: & \\
   \hline
     \makecell{\textsc{Finetuning on} \\ \textsc{Feedback} \\ \textsc{ + Refinements}} &Write an excellent summary that incorporates the feedback on the given summary and is better than the given summary.  & \makecell{\{\texttt{feedback}\} \\ \\ Improved TL;DR: \{\texttt{refinement}\} \\ \#\#\#} & \\
    & \\
    & Title:  \{\texttt{title}\} & \\
    & \\
    & Text:  \{\texttt{post}\} & \\
    & \\
    & Summary:  \{\texttt{summary}\} & \\
    & \\
    & Feedback on summary: & \\
    & \\
   \hline
\bottomrule
\caption{Prompt templates used for Finetuning on Summaries and Feedback + Refinement.}
\label{tab:finetuning_prompts}
\end{longtable}
