\documentclass[letterpaper, twocolumn, 10pt]{article}
\usepackage[top=2.5cm,left=1.5cm,right=1.5cm,bottom=3cm]{geometry}

\usepackage{authblk}

\usepackage{amsmath,amssymb,amsfonts}
\usepackage{booktabs}
\usepackage{soul}
\usepackage{algorithm}
\usepackage{algorithmicx}
\usepackage[noend]{algpseudocode}
\usepackage{hyperref}
\usepackage{enumitem}
\usepackage{graphicx}
\graphicspath{{./images/}}
\usepackage{float}

\usepackage{comment}

\usepackage{makecell}
\usepackage{multirow}
\usepackage{booktabs}

\setlength{\aboverulesep}{0pt}
\setlength{\belowrulesep}{0pt}

\usepackage{textcomp}

\usepackage[svgnames]{xcolor}

\usepackage[breakable]{tcolorbox}

\usepackage{xspace}

\newcommand{\code}{\texttt}
\newcommand{\cd}{\code}

\newcommand{\ra}[1]{\renewcommand{\arraystretch}{#1}}

\renewcommand\UrlFont{\color{blue}\rmfamily}
\newcommand{\mynote}[2]{\fbox{\bfseries\sffamily\scriptsize{#1}}
	{\small$\blacktriangleright$\textsf{\emph{#2}}$\blacktriangleleft$}}
\newcommand{\GA}[1]{\textcolor{blue}{\mynote{GADL}{#1}}}
\newcommand{\GADL}[1]{\textcolor{blue}{\mynote{GADL}{#1}}}
\newcommand{\GC}[1]{\textcolor{orange}{\mynote{GC}{#1}}}
\newcommand{\LQ}[1]{\textcolor{purple}{\mynote{profLQ}{#1}}}
\newcommand{\CONO}[1]{\textcolor{purple}{\mynote{IlCono}{#1}}}

\newcommand{\mT}{\tau_{\textsc{t}}}
\newcommand{\T}{$\mT$\xspace}
\newcommand{\mU}{\tau_{\textsc{u}}}
\newcommand{\U}{$\mU$\xspace}

\newtcolorbox{mybox}{
	arc=0pt,
	boxrule=0pt,
	colback=Gainsboro,
	width=\columnwidth,   % this option controls the width of the box
	colupper=black
}

\definecolor{babyblueeyes}{rgb}{0.63, 0.79, 0.95}

\newtcolorbox{mybox2}{
	arc=0pt,
	boxrule=0pt,
	breakable,
	colback=babyblueeyes,
	width=\columnwidth,   % this option controls the width of the box
	colupper=black
}

\begin{document}

\title{\Large \bf Adversarial Attacks against Binary Similarity Systems}

\author{{\rm Gianluca Capozzi}}

\author{{\rm Daniele Cono D'Elia}}

\author{{\rm Giuseppe Antonio Di Luna}}

\author{{\rm Leonardo Querzoni} \thanks{\texttt{\{capozzi, delia, diluna, querzoni\}@diag.uniroma1.it}}}

\affil[]{Sapienza University of Rome}

\date{}

% make the title area
\maketitle

\subsection*{Abstract}
In recent years, binary analysis gained traction as a fundamental approach to inspect software and guarantee its security. Due to the exponential increase of devices running software, much research is now moving towards new autonomous solutions based on deep learning models, as they have been showing state-of-the-art performances in solving binary analysis problems. One of the hot topics in this context is binary similarity, which consists in determining if two functions in assembly code are compiled from the same source code. However, it is unclear how deep learning models for binary similarity behave in an adversarial context.

In this paper, we study the resilience of binary similarity models against adversarial examples, showing that they are susceptible to both \textit{targeted} and \textit{untargeted} attacks (w.r.t. similarity goals) performed by {black-box} and {white-box} attackers.
In more detail, we extensively test three current state-of-the-art solutions for binary similarity against two black-box greedy attacks, including a new technique that we call \textit{Spatial Greedy}, and one white-box attack in which we repurpose a gradient-guided strategy used in attacks to image classifiers.

\section{Introduction}

The increasing complexity of source code poses a key challenge to the reliability of large-scale software systems. Software bugs in these systems can lead to safety issues~\cite{bug_safety} for users around the world as well as cause non-negligible financial losses~\cite{bug_loss}. As such, developers have to spend a large amount of time and effort on bug fixing. Consequently, \aprfull (\apr), designed to automatically generate patches to fix software bugs, has attracted wide attention from both academia and industry~\cite{long2016prophet, legoues2012genprog, long2015spr, lou2020can, tufano2018empstudy}. 


To achieve \apr, one popular approach is known as Generate-and-Validate (G\&V)~\cite{qi2015gv, ghanbari2019prapr, lou2020can, le2016hdrepair, legoues2012genprog, wen2018capgen, hua2018sketchfix, martinez2016astor, koyuncu2020fixminder, liu2019tbar, liu2019avatar}, which is typically based on the following pipeline: First, fault localization techniques~\cite{wong2016fl, abreu2007ochiai, zhang2013injecting, papadakis2015metallaxis, li2019deepfl, li2017transforming} are applied to determine the suspicious locations in programs where bugs are likely to exist. Then, the buggy locations are used by the \apr tools to generate a list of patches that replace buggy lines with correct lines. Afterward, each patch is validated against the original test suite to identify any \emph{plausible patches} (i.e., passing all tests in the test suite). Finally, to determine the \emph{correct patches}, developers examine the list of plausible patches to see if any of them can correctly fix the bug. 

Traditional \apr tools can mainly be categorized into heuristic-based~\cite{legoues2012genprog, le2016hdrepair, wen2018capgen}, constraint-based~\cite{mechtaev2016angelix, le2017s3, demacro2014nopol, long2015spr} and \template~\cite{ghanbari2019prapr, hua2018sketchfix, martinez2016astor, liu2019tbar, liu2019avatar}. Among these traditional tools, \template \apr tools~\cite{ghanbari2019prapr, liu2019tbar, benton2020effectiveness} have been able to achieve state-of-the-art results. \Template \apr tools typically leverage pre-defined templates (e.g., adding a nullness check) for bug fixing. However, since these fix templates are typically handcrafted, the number and types of bugs they are able to fix can be limited. 



To address the limitations of traditional \apr, researchers have proposed various \learning \apr tools~\cite{li2020dlfix, chen2018sequencer, jiang2021cure, lutellier2020coconut, zhu2021recoder, ye2022rewardrepair} based on the \nmtfull (\nmt) architecture~\cite{sutskever2014mt} where the input is the buggy code snippets and the goal is to translate the buggy code snippets into a fixed version. To accomplish this, \learning \apr tools require supervised training datasets with pairs of both buggy and fixed code snippets in order to learn how to perform this translation step. These training data are usually obtained by mining historical bug fixes using heuristics/keywords~\cite{dallmeier2007benchmark}, which can be imprecise for identifying bug-fixing commits; even the actual bug-fixing commits can include irrelevant code changes, leading to further pollution in the dataset~\cite{xia2022alpharepair}.
% 
Moreover, it can be hard for such \apr tools to generalize and fix bug types unseen during training. 



To better leverage recent advances in \plmfull{s} (\plm{s}), researchers~\cite{xia2022alpharepair, xia2023repairstudy, kolak2022patch, prenner2021codexws} have directly applied \plm{s} to generate patches without bug-fixing datasets. These \llm-based \apr tools work by either directly generating a complete code function~\cite{prenner2021codexws, xia2023repairstudy} or predict/infill the correct code snippet given its surrounding context~\cite{xia2022alpharepair, xia2023repairstudy}. By directly using \llm{s} that are pre-trained on billions of open-source code snippets, \llm-based \apr tools can achieve state-of-the-art performance on many repair datasets~\cite{xia2022alpharepair}. 


% 
%
%

Traditional \apr tools have long used the insight of the \emph{plastic surgery hypothesis}~\cite{barr2014plastic} where it states that the code ingredients to fix a bug already exist within the same project. Traditional \apr tools have manually designed pattern-~\cite{ghanbari2019prapr, saha2017elixir} or heuristic-based~\cite{jiang2018simfix, legoues2012genprog} approaches to finding and using such relevant code ingredients to generate fixes for bugs. However, the plastic surgery hypothesis has been largely ignored in \llm-based \apr. In fact, \llm provides a unique opportunity to fully automate the plastic surgery hypothesis idea via fine-tuning (learning project-specific information via model updates from the buggy project) and prompting (directly providing relevant code ingredients to the model), and make it directly applicable to different languages (since the \llm{s} are typically multi-lingual).%
Moreover, despite the intensive manual efforts involved, traditional \apr tools still cannot fully leverage project-specific information due to large search space for leveraging/composing existing code ingredients. In contrast, the project-specific information can effectively leveraged by \llm{s} due to their power in code understanding/vectorization, e.g., even partial/imprecise information may still guide \llm{s} in correct patch generation!
 To this end, we ask the question: \emph{How useful is the plastic surgery hypothesis in the era of \plm{s}}?








\mypara{Our Work.} To answer the question, we present \ourtech{\xspace} -- a \llm-based approach that automatically utilizes the plastic surgery hypothesis by systematically combining multiple fine-tuning and prompting strategies for \apr. \ourtech fine-tunes \plm{s} using two novel domain-specific training strategies: \textbf{\epfinetune} -- we fine-tune using the original buggy project by aggressively masking out a high percentage of tokens, which allows \plm to learn project-specific code tokens and programming styles; and \textbf{\rofinetune} -- which only masks out a single continuous code sequence per training sample, allowing the model to get used to the final \csapr task of predicting a single continuous code sequence. Furthermore, we directly leverage the ability for \plm{s} to understand natural language instructions and introduce a novel prompting strategy, \textbf{\idprompting}, which uses information retrieval and static analysis to obtain a list of relevant identifiers for the buggy lines. While such relevant identifiers are critical for fixing some difficult bugs, they may not be seen by the \llm during inference due to limited context window size. Through the use of prompting, we directly tell the model to use these extracted identifiers (relevant code ingredients) to generate the correct code. Finally, to perform repair, we combine all four model variants (including the base model, both fine-tuned models and the base model with prompting) for the final repair.





While our insight of leveraging the plastic surgery hypothesis for \llm-based \apr is generalizable across different types of \plm{s}, to implement \ourtech, we choose a recent \plm{\xspace}, \ctfive~\cite{wang2021codet5}, which is pre-trained on millions of open-source code snippets. \ctfive is an encoder-decoder model trained using \mspfull (\msp) objective where a percentage of tokens are masked out and each continuous masked token sequence is referred to as a masked span. Also, although we only extract relevant identifiers from the current buggy project (since this paper focuses on the plastic surgery hypothesis), our work can be easily extended to obtain other code information (such as relevant statements or functions) from other sources, such as  the massive pre-training corpora~\cite{husain2020codesearchnet} or historical bug-fixing datasets~\cite{jiang2019infer}, which can provide more coding knowledge for \llm{s}. Besides, although we mainly focus on using traditional string comparison algorithms for information retrieval in this paper, these techniques can be easily replaced by other frequency-based retrieval~\cite{robertson2009probabilistic} and neural search (or embedding-based search)~\cite{reimers2019sentence}.
  In summary, this paper makes the following contributions:


%


\begin{itemize}[noitemsep, leftmargin=*, topsep=0pt]
    \item \textbf{Dimension.} This paper is the first to revisit the important plastic surgery hypothesis in the era of \llm{s}. It opens up a new dimension for \llm-based \apr to incorporate previously neglected information from the buggy project itself to boost \apr performance. Furthermore, it demonstrates the promising future of retrieval-based prompting for modern \llm-based \apr.
    \item \textbf{Implementation.} We implement \ourtech based on the recent \ctfive model. We augment the model using two novel fine-tuning strategies: \epfinetune and \rofinetune, along with a novel prompting strategy based on information retrieval and static analysis: \idprompting. We combine the patches generated by all four models together and perform patch ranking to speed up \apr.% 
    \item \textbf{Evaluation Study.} We conduct an extensive evaluation against state-of-the-art \apr tools. On the widely studied \dfj 1.2 and 2.0 datasets~\cite{just2014dfj}, \ourtech is able to achieve the new state-of-the-art results of 89 and 44 correct bug fixes (15 and 8 more than best baseline) respectively.  Furthermore, we perform a broad ablation study to justify our design. \ourtech demonstrates for the first time that the plastic surgery hypothesis can substantially boost \llm-based \apr and advance state-of-the-art \apr, while being fully automated and general. Moreover, even partial/imprecise code ingredients may still effectively guide \llm{s} for \apr!
\end{itemize}


\section{Background on Network Calculus}
\label{sec: background}


\begin{figure*}[tbh]
\centering
\begin{subfigure}[b]{0.3\textwidth}
    \centering
    \includegraphics[width=\linewidth]{images/in-out.png}
    \caption{Arrival and departure data and their relation with delay $d(t)$ and backlog $b(t)$. For a FIFO system, the delay is the horizontal distance between $R(t)$ and $R^*(t)$ but some other multiplexing techniques may shift the data to a later priority, causing a longer delay.}
    \label{fig: data in-out}
\end{subfigure}
\hfill
\begin{subfigure}[b]{0.35\textwidth}
    \centering
    \includegraphics[width=\linewidth]{images/arrival-service.png}
    \caption{Characteristics of an arrival curve and a service curve. From any point of observation, the arriving data never exceeds its arrival curve; the departure data is also never less than the service curve with respect to the data arrival.}
    \label{fig: arrival-service curves}
\end{subfigure}
\hfill
\begin{subfigure}[b]{0.33\textwidth}
    \centering
    \includegraphics[width=\linewidth]{images/bound.png}
    \caption{Delay and backlog bounds of a system. Backlog is the maximum vertical distance between $\alpha(t)$ and $\beta(t)$; FIFO delay is their maximum horizontal distance; but for arbitrary multiplexing, the delay guarantee is when the system clears its buffer, thus it's the intersection of $\alpha(t)$ and $\beta(t)$.}
    \label{fig: system bounds}
\end{subfigure}
\caption{Network calculus framework. We let $R(t)$ and $R^*(t)$ be the arrival and departure data flow of a system; $\alpha(t)$ be the piecewise linear concave arrival curve and $\beta(t)$ be the piecewise linear convex service curve of a system.}
% \hossein{Better to show piece-wise linear concave arrival curve and piece-wise linear convex service curve instead of token-bucket and rate-latency.}}
\end{figure*}

We recall some of the network calculus essentials for a better understanding of the framework used in Saihu. In the following context, we use the following notation: $\mbb{R}^+$ is the set of non-negative real numbers; $[x]_+$ denotes $\max(0, x)$

The data flow is by convention modeled as a left-continuous wide-sense increasing function $R(t): \mbb{R}^+ \mapsto \mbb{R}^+$ with respect to time $t$~\cite{ncbook2001leboudec}. 

A system $\mcal{S}$ receives arrival data described as a cumulative function $R(t)$ and delivers departure data as another cumulative function $R^*(t)$. Figure~\ref{fig: data in-out} illustrates such a system $\mcal{S}$. The benefit of representing a system like this is that we can observe system backlog and delay with such a model. 

\begin{definition}[Backlog and Delay~\cite{ncbook2001leboudec}]
    The backlog of a system at time~$t$ is
    \begin{equation}
        b(t) = R(t) - R^*(t)
    \end{equation}
    
    The virtual delay of a FIFO system at time $t$ is
    \begin{equation}
        d_{FIFO}(t) = \inf \lbp \tau \geq 0 : R(t) \leq R^*(t+\tau) \rbp
    \end{equation}
\end{definition}



The backlog of a system can be viewed as the vertical distance between $R$ and $R^*$. The FIFO (\textit{First-in First-out}) delay is the horizontal distance between $R$ and $R^*$. One may obtain other delay values if the multiplexing technique is not FIFO.

% \begin{figure}
%     \centering
%     \includegraphics[width=0.9\linewidth]{images/in-out.png}
%     \caption{In/out data flow; delay and backlog}
%     \label{fig: data in-out}
% \end{figure}

Since we are interested in the system guarantee instead of a single instance of data flow, we would like to have general bounds to the arrival and departure data flows. Therefore, we define \textit{arrival curve} and \textit{service curve} as the bounds of arrival and departure data flows.

\begin{definition}[Arrival Curve~\cite{ncbook2001leboudec}]
    Given a wide-sense increasing function $\alpha: \mbb{R}^+ \mapsto \mbb{R}^+$, we say that a flow $R(t)$ is $\alpha$-constrained if and only if for all $s \leq t$:
    \begin{equation}
        R(t) - R(s) \leq \alpha(t-s)
    \end{equation}
    We say $R(t)$ has $\alpha$ as an arrival curve.
\end{definition}

\begin{definition}[Service Curve~\cite{ncbook2001leboudec}]
    Given a wide-sense increasing function $\beta: \mbb{R}^+ \mapsto \mbb{R}^+$ and $\beta(0) = 0$. A system $\mcal{S}$ having $R(t)$ and $R^*(t)$ as its arrival and departure flows. We say $\mcal{S}$ offers a service curve $\beta$ if and only if
    \begin{equation}
        R^*(t) \geq (R \otimes \beta)(t) =: \inf_{s \leq t} \lbp R(s) + \beta(t-s) \rbp
    \end{equation}
    where $\otimes$ denotes the min-plus convolution
\end{definition}

Figure~\ref{fig: arrival-service curves} illustrates the arrival and service curves. Any segment of arrival flow $R(t)$ is constrained by arrival curve $\alpha$ and the output curve $R^*(t)$ is always no less than the curve $R\otimes\beta$. As a result, an arrival curve upper bounds the incoming traffic, and a service curve lower bounds the outgoing traffic.

% \begin{figure}
%     \centering
%     \includegraphics[width=\linewidth]{images/arrival-service.png}
%     \caption{Arrival/Service curve}
%     \label{fig: arrival-service curves}
% \end{figure}

We consider 2 special types of curves throughout this paper, \textit{token-bucket} (or sometimes called \textit{leaky-bucket}) curve and \textit{rate-Latency} curve.

\begin{definition}[Token-bucket and Rate-latency~\cite{ncbook2001leboudec}]
    A token-bucket curve $\gamma_{r,b}$ with arrival rate $r$ and burst $b$ is defined as
    \begin{equation}
        \gamma_{r,b}(t) = b + rt
    \end{equation}

    A rate-latency curve $\beta_{R,T}$ with service rate $R$ and latency $T$ is defined as
    \begin{equation}
        \beta_{R,T}(t) = R \lb t - T \rb_+
    \end{equation}
\end{definition}

A token-bucket curve is determined by a burst $b$ and an arrival rate~$r$. Burst represents the maximum possible data volume that can arrive simultaneously, and arrival rate represents the maximum long-term data rate~\cite{bouillard2022tradeoff}.
A rate-latency curve is determined by a latency~$T$ and a service rate~$R$. Latency represents the time a server needs before starting to process the incoming data, and service rate represents the minimum rate to process data after the initial latency.

With the help of arrival and service curves, we can derive delay and backlog bounds for a system $\mcal{S}$ illustrated in Figure~\ref{fig: system bounds}. Suppose a system $\mcal{S}$ has arrival curve $\alpha$ and service curve~$\beta$, its worst-case backlog $b^*$ is the maximum vertical distance between~$\alpha$ and~$\beta$. Similarly, depending on the multiplexing technique applied to the system, its worst-case delay bound $d^*$ is the maximum horizontal distance between $\alpha$ and $\beta$ if $\mcal{S}$ is a FIFO system. If we don't have any information about its multiplexing technique, referred to as arbitrary multiplexing, the best we can say is that when $\alpha$ and $\beta$ intersect each other, where all data has been delivered out of the system. Consequently, the worst-case delay bound for arbitrary multiplexing is the time required for $\mcal{S}$ to clear its buffer.

% \begin{figure}
%     \centering
%     \includegraphics[width=\linewidth]{images/bound.png}
%     \caption{System delay/backlog bounds}
%     \label{fig: system bounds}
% \end{figure}

While a service curve captures the slowest possible output speed of a system, a link's transmission capacity limits the speed as well. Hence, we model this phenomenon using a \textit{greedy shaper} with a sub-additive function $\sigma: \mbb{R}^+ \mapsto \mbb{R}^+$ concatenated with a server. We consider a concatenation as shown in Figure \ref{fig: system}. By convention we assume $\sigma(0) = 0$ and $\beta(t) \leq \sigma(t), \forall t \in \mbb{R}^+$, meaning that the buffer is cleared at the beginning and the service never exceed its physical limitation. With the above definition, such greedy shaper conserves the service provided by the system due to theorem \ref{thm: shaping}.

\begin{figure}[thb]
    \centering
    \includegraphics[width=0.7\linewidth]{images/system.png}
    \caption{Shaping of departure data. A flow that has an arrival curve $\alpha$ feeds into a server with an arrival data flow $R(t)$. The server having service curve $\beta$ takes $R(t)$ and gives a departure data flow $R^*(t)$ to a shaper with shaping function $\sigma$. The shaper takes $R^*(t)$ and shape the data flow as another departure $D(t)$.}
    \label{fig: system}
\end{figure}


\begin{theorem}[Shaping conserves service \cite{ncbook2001leboudec}]
\label{thm: shaping}
Following the system shown in Figure \ref{fig: system}, we have
\begin{equation}
     D = R^* \otimes \sigma \geq \lp R \otimes \beta \rp \otimes \sigma = R \otimes \lp \beta \otimes \sigma \rp = R \otimes \beta
\end{equation}
\end{theorem}

In the following context, we model the shaping function $\sigma$ as a token-bucket curve $\gamma_{C,L}$ with transmission capacity $C$ and the packet size $L$ to capture the link capacity and packetization~\cite{bouillard2022tradeoff}.

\section{Proposed Framework: {\ourmodel}}
\label{model}


In this section, we introduce a novel self-supervised co-training framework {\ourmodel}.
The proposed framework is illustrated in Figure~\ref{fig:intro_model} and works in three phases.
Phase one automatically generates two sets of pseudo labels.
We use a combination of off-the-shelf pre-trained POS and NER taggers, knowledge graph, and GPT-2 scorer for generating the first set of pseudo labels automatically without any hand-crafted rules for matching the slot values.
The other set of pseudo labels is acquired through a zero-shot slot filling model~\cite{liu2020coach}, trained on the out-of-domain dataset.
It is critical to emphasize that both sets of labels are noisy and incomplete which poses serious challenges to training effective models for the task of open-domain slot filling.
Phase two fine-tunes the pre-trained BERT to the slot filling task that effectively transfers the knowledge from the pre-trained language model~(LM) to overcome the issue of label incompleteness to some extent. 
Further, we employ the early stopping technique to minimize the noise in the labels.
The output of this phase is two BERT models that can generate soft labels for self-supervision during co-training in phase three.
Phase three leverages the fine-tuned models and further trains them in an iterative fashion.
Specifically, the proposed peer training approach facilitates high-confidence soft label selection for the other peer to perform training. This phase progressively reduces the noise in the labels and enables effective model fitting. 



\subsection{Phase One: Automatic Label Generation}
To acquire the first set of labels, we perform the following steps.
First of all, off-the-shelf trained POS and NER taggers are used to predict initial estimates of the slot values irrespective of the slot types. Then, the type information of the slot values is queried from the KG and the slot value is tagged for the most appropriate slot in the target domain.
This approach, however, produces low recall. 
To expand the candidate slot values, we generate n-grams of the natural language text and employ a partial matching scheme to query the KG for type information (e.g., \myspecial{Jason} \myspecial{Aldean} = \myspecial{American} \myspecial{singer}) of the n-grams if the entry exists.
This process generates multiple overlapping hypotheses about the slot values.
We replace a span of text that corresponds to a slot value by its type information and a GPT-2 based scorer (see Section~\ref{sec:nlpmodels}) is used to select the best candidate based on the fluency of the text.
Naturally, if a token (or span of tokens) is replaced by its type, the sentence should score higher as compared to the case where an inappropriate substitution is performed. 
We select the best hypothesis if the score is greater than the threshold.
Intuitively, the candidate selection threshold can automatically be searched based on a small validation set from the target domain, making the label generation process fully automatic. 
The other set of noisy labels is acquired by the zero-shot slot filling model~\cite{liu2020coach} that has been trained using an out-of-domain dataset. It is important to highlight that the zero-shot slot filling model does not require any labeled in-domain training example. 
To summarize the automatic label generation phase, both sets of labels are acquired in a fully automatic fashion without any hand-crafting.


In contrast to previous work in weak supervision~\cite{ren2015clustype,he2017autoentity,fries2017swellshark,giannakopoulos2017unsupervised} that obtains a single set of noisy labels and then propose techniques to overcome the challenge of fitting an effective model to the noisy labels, we acquire two sets of complementary labels.
The choice of these two sets of labels is guided by the intuition that they should be complementary and the models trained on these sets of labels should be able to share complementary information with the other to improve the performance in the later phases of the framework.
Essentially, the first set of labels carries information from external knowledge sources, whereas the labels generated through the pre-trained zero-shot slot filling model capture how the slot values are mentioned in other domains.
%
To further elaborate on the motivation and our process for the first set of labels (i.e., labels using KG and other NLP models), the pre-trained LMs have been shown to have a great deal of knowledge~\cite{petroni2019language}, thus should be capable of generating automatic labels with no need of external KG. 
To the best of our knowledge, there exists no work that shows that accurate token-level automatic labeling (e.g., slot filling task) is possible with pre-trained LMs. 
Moreover, such approaches would require heavy prompting in each new target domain, whereas our label generation process is fully automatic and only relies on the readily-available pre-trained NLP models and external KG.

\subsection{Phase Two: LM-assisted Weak Supervision}
Since we do not have access to dataset $\{(\mathbf{X}_n,\mathbf{Y}_n)\}_{n=1}^N$ with true ground-truth labels.
We use pseudo labels generated in phase one, $\{(\mathbf{X}_n,\mathbf{D}_n)\}_{n=1}^N$, to learn 
$f_{m,c}(\cdot; \cdot)$ that outputs the probability of the $m$-th token to take on class $c$. 
We learn $f_{m,c}(\cdot; \cdot)$ by minimizing the following loss over the noisy dataset $\{(\mathbf{X}_n,\mathbf{D}_n)\}_{n=1}^N$: 
$$
\hat\theta = \argmin_{\theta}\frac{1}{N}\sum_{n=1}^{N} \ell(\mathbf{D}_n, f(\mathbf{X}_{n}; \theta)),
\label{eq:stage1}
$$
where $\ell(\mathbf{D}_n, f(\mathbf{X}_{n}; \theta)) = \frac{1}{M} \sum_{m=1}^{M} -\log{f_{m,d_{n, m}}(\mathbf{X}_{n}; \theta)}$. 
We employ the pre-trained multilingual BERT with token-level classification head that uses Adam optimizer \cite{kingma2014adam,Liu2019} with early stopping and multiple random initializations. 


Since slot filling task is similar to the MLM training objective of the BERT, we employ pre-trained BERT as the backbone model.
That is, MLM's goal is to predict the masked tokens using bidirectional contexts. Similarly, slot filling tries to predict the label for a token leveraging both left and right contexts simultaneously, which makes the pre-trained BERT an ideal model of choice that greatly facilitates minimizing incomplete labels.
It is important to highlight that our automatically generated labels are not only incomplete but also potentially wrong.
The training strategies employed in this phase minimize the noise in the label to some extent. 
Specifically, early stopping can provide a strong regularization and would not let the model overfit to the noisy labels, especially wrong labels. 
Moreover, early stopping does not let the model forget the knowledge in the pre-trained model.
Similarly, multiple random initializations enforce robustness. 
Since the model is fine-tuned on the noisy labels, averaging the predictions of multiple models for each token ensures that wrong labels end up with low probabilities and true labels consistently achieve high probabilities.
Using the above-mentioned strategies, we train two slot filling models, which we call the peers. The peer one is trained on the first set of pseudo labels that were generated using POS and NER taggers, KG, and the GPT-2 scorer in phase one. Similarly, peer two is trained using the predictions of the zero-shot slot filling model~\cite{liu2020coach}.
Both models have the same architecture and follow the same training procedures.

\begin{table*}[t!]
\centering
\caption{Dataset statistics.}
\vspace{-7pt}
\label{tab:dataset}
\begin{tabular}{lccccc}
\toprule
\textbf{Dataset}  & \textbf{Dataset Size} & \textbf{Vocab. Size} & \textbf{Avg. Length} & \textbf{\# of Domains} & \textbf{\# of Slots} \\ \hline
\textbf{SGD}      & 188K                  & 33.6K                & 13.8                 & 20                     & 240                  \\
\textbf{MultiWoZ} & 67.4K                 & 10.5K                & 13.3                 & 8                      & 61 \\
\bottomrule
\end{tabular}
\vspace{-7pt}
\end{table*}

\subsection{Phase Three: Self-supervised Co-training}
We introduce an iterative peer training algorithm where both peers generate high-confidence soft labels for training the other peer in the next iteration. 
Theoretically, these peers can be anything, but in this work, 
we explore two of the most promising directions that have shown the promise to minimize the need for manual labeling for the task: zero-shot learning and distant supervision.
This phase uses a self-supervised co-training scheme to exploit the patterns of slot values from other domains through the labels generated by the zero-shot filling model (i.e., peer two)~\cite{liu2020coach} as well as utilize the knowledge in external KGs and pre-trained models via labels provided by the peer one.
Specifically, we initialize the peers trained in phase two and use their pseudo labels to kick-start training in this phase.
Specifically, peer one $f_{m,c}(\cdot; \theta_{\textrm{p1}})$ would generate labels $\{\tilde{\mathbf{Y}}^{(t)}_n = [\tilde{y}_{n,1}^{(t)}, ..., \tilde{y}_{n,m}^{(t)}]\}_{n=1}^{N}$ for peer two $f_{m,c}(\cdot; \theta_{\textrm{p2}})$ at the $t$-th iteration by:
$$
\tilde{y}_{n,m}^{(t)} = \argmax_{c}{f_{m,c}(\mathbf{X}_n; \theta_{\textrm{p1}}^{(t)})}. 
\label{eq:pseudo}
$$

Based on these labels, the peer two can be fine-tuned by: 
$$
\hat\theta_{\textrm{p2}}^{(t+1)} = \argmin_{\theta}\frac{1}{N}\sum_{n=1}^N \ell(\tilde{\mathbf{Y}}_n^{(t)}, f(\mathbf{X}_{n}; \theta)).
\label{eq:self_train1}
$$

Similarly, peer two $f_{m,c}(\cdot; \theta_{\textrm{p2}})$ would generate pseudo labels for peer one $f_{m,c}(\cdot; \theta_{\textrm{p1}})$ that are used to fine-tune peer one. 
We also notice that it is beneficial to stop early during this phase as well, to improve the model fitting and gradually reduce the noise associated with the automatically generated labels.
Since pseudo labels are refined gradually in an iterative way, both peers can benefit from the knowledge contained within the labels of the other while avoiding overfitting.
Furthermore, as an alternative to pseudo labels, we also generate soft labels that are used for confidence re-weighting. 
The high-confidence soft label selection strategy enables better model fitting and efficient learning via better quality of the automatic labels.
Specifically, for the given $m$-th token in the $n$-th training example, the probability for all classes $C$ is $[f_{m,1}(\mathbf{X}_n;\theta),...,f_{m,C}(\mathbf{X}_n;\theta)]$. 
Following ~\cite{xie2016unsupervised}, at $t$-th iteration, peer one generates soft labels, $\{\mathbf{S}_n^{(t)} = [\mathbf{s}_{n,m}^{(t)}]_{m=1}^M \}_{n=1}^N$, as given below:
$$
\mathbf{s}_{n,m}^{(t)} = [s_{n,m,c}^{(t)}]_{c=1}^{C} = \Bigg[  \frac{f_{m,c}^2(\mathbf{X}_n;\theta_{\textrm{peer1}}^{(t)})/p_{c}}{\sum_{c'=1}^C f_{m,c'}^2(\mathbf{X}_n;\theta_{\textrm{peer1}}^{(t)})/p_{c'}}\Bigg]_{c=1}^{C}
\label{eq:soft}
$$ 
where $p_{c} = \sum_{n=1}^N \sum_{m=1}^M f_{m,c}(\mathbf{X}_n;\theta_{\textrm{p1}}^{(t)})$ computes the frequency of the tokens for the $c$-th class. 
Then, peer two $f(\cdot; \theta_{\textrm{p2}}^{(t+1)})$ is fine-tuned by:
$$
\theta_{\textrm{p2}}^{(t+1)} = \argmin_{\theta} \frac{1}{N} \sum_{n=1}^{N} \ell_{\rm KL}(\mathbf{S}_n^{(t)}, f(\mathbf{X}_{n}; \theta)),
$$
where $\ell_{\rm KL}(\cdot,\cdot)$ is the KL-divergence-based loss:
$$
\ell_{\rm KL}(\mathbf{S}_n^{(t)}, f(\mathbf{X}_{n}; \theta))=\frac{1}{M}\sum_{m=1}^M\sum_{c=1}^C - s_{n,m,c}^{(t)} \log f_{m,c}(\mathbf{X}_{n}; \theta).
\label{eq:klloss}
$$

Moreover, we also investigate selecting tokens that have high confidence. 
For instance, we pick high-confidence tokens from the $m$-th input example at the $t$-th iteration by  
$
H^{(t)}_n = \{m : \max_{c} s_{n,m,c}^{(t)} > \epsilon \},
$
where $\epsilon\in [0,1]$ is a threshold that can be searched based on a small validation set. 
Then, peer two $f(\cdot; \theta_{\textrm{p2}}^{(t+1)})$ is fine-tuned by:
$$
\theta_{\textrm{p2}}^{(t+1)} %&= \argmin_{\theta} \frac{1}{N} \sum_{n=1}^{N} \ell_{\rm S-KL}(\bS_n^{(t)}, f(\bX_{n}; \theta)) \\
= \argmin_{\theta} \frac{1}{N|H^{(t)}_n|}\sum_{n=1}^{N} \sum_{m\in H^{(t)}_n}\sum_{c=1}^C - s_{n,m,c}^{(t)} \log f_{m,c}(\mathbf{X}_{n}; \theta).
$$

This phase improves the robustness to effectively fit the model for tokens with high confidence. 
Both peers keep sharing information and their confidence by producing soft labels for their counterparts until they approximate to the true labels while employing early stopping and scheduled learning rates.
It is important to remind that phase three is the most important phase that progressively reduces noise from the labels to a great extent and enables superior performance for the task of open-domain slot filling.
\vspace{-0.1in}
\section{Methodology}
\label{sec:solution}

The overall model architecture of our \model\ is shown in Figure~\ref{fig:arch}.

\begin{figure*}
    \centering
    \includegraphics[width=\linewidth]{material/model_arc_.pdf}
    \vspace{-0.2in}
    \caption{The overall framework of \model. $\mathcal{G}_c$ and $\mathcal{G}_t$ are built to encode the sequences from diversified views (left part). In addition, we generate reasonable interaction-level conformity weights $\omega$ from the rich structure of $\mathcal{G}_t$ (right part). The weights are restrained in normal distribution and empower the cross-view contrastive learning to be adaptive and aware of conformity.}
    \label{fig:arch}
    \vspace{-0.15in}
\end{figure*}

\subsection{Task Formulation}
\noindent \textbf{Notations}. We suppose a recommender with a set of users and items denoted by $\mathcal{U} (u\in \mathcal{U})$ and $\mathcal{V} (v\in \mathcal{V})$, respectively. For each user, his/her engaged subset of items in a temporal order is defined as $\boldsymbol{s}_u=\left(v_1, v_2, \cdots, v_T\right)$. Here, $T$ is the sequence length which varies by users, and indexed by $t$, \ie $1\leq t \leq T$. Following settings in~\cite{bert4rec,iclrec}, we conduct the padding operation over different item sequences ($\boldsymbol{s}_u\ \in \mathcal{S}$) to mitigate the variable length.\\\vspace{-0.12in}

\noindent \textbf{Task}. 
Our objective is to develop a personalized ranking function that takes into account the past item sequences of a user, and predicts the next item ($v_{T+1}$) that the user is most likely to adopt.

% Given the past item sequences, our goal is to learn a personalized ranking function over all candidate items and predict the next item (\ie $v_{T+1}$) that the user is likely to adopt at the future step.

\vspace{-0.1in}
\subsection{Sequential Pattern Encoding}
As of now, Transformer has emerged as the dominant method for encoding sequences, capable of mapping temporally-ordered tokens from different types of sequential data to latent representation space. Examples include textual data~\cite{devlin2018bert} and electronic health data~\cite{poulain2021transformer}. Our sequential pattern encoder is built upon the Transformer, inspired by the effectiveness of this approach in modeling item sequence in~\cite{bert4rec,wu2020sse,yuan2022multi}. This allows us to incorporate temporal context into embeddings, resulting in an effective representation of the user's sequential behavior.

% To date, Transformer has become the most prevalent sequence encoding solution to project temporally-ordered tokens into latent representation space from various sequential data, such as textual data~\cite{devlin2018bert}, electronic health data~\cite{poulain2021transformer}, and user behaviors~\cite{yang2022getnext}. Inspired by the effectiveness of Transformer in modeling item sequence in~\cite{bert4rec,wu2020sse,yuan2022multi}, our sequential pattern encoder is built upon the Transformer, to incorporate temporal context into embeddings.

We start by adding a positional embedding $\mathbf{p}_v$ to the initial item representation $\mathbf{v}_v$ using the operation $\mathbf{h}_v^0 = \mathbf{v}_v \oplus \mathbf{p}_v$, which serves as the input item embedding $\mathbf{h}_v^0$ for the first block of Transformer. We represent each user's item sequence with an embedding matrix $\mathbf{H}_u^0 \in \mathbb{R}^{T \times d}$, where $T$ is the length of the sequence and $d$ is the dimension of the item embedding. The embedding matrix corresponds to the padded item sequence $\boldsymbol{s}_u$ of the user. To capture the correlations between items, we apply a self-attention layer with multi-head ($N$) channels to the user's item embedding matrix:
% Specifically, we first inject the positional embedding $\mathbf{p}_v$ of item $i$ into the initialized item representation $\mathbf{v}_v$ with the operation $\mathbf{h}_v^0 = \mathbf{v}_v \oplus \mathbf{p}_v$ as the input item embedding $\mathbf{h}_v^0$ for the first block of Transformer. We associate each user with an embedding matrix $\mathbf{H}_u^0 \in \mathbb{R}^{T \times d}$ corresponding to the padded item sequence $\boldsymbol{s}_u$. To capture the item-wise correlations, a self-attention layer with multi-head ($N$) channels is applied to user's item embedding matrix:
\begin{align}
    \text{MH}\left({\textbf{H}_u^\ell}\right) &= \left(\text{head}_1 \mathbin\Vert \text{head}_2 \mathbin\Vert  \cdots \mathbin\Vert \text{head}_N\right)\mathbf{W}^D \\
	\text{head}_n &= \text{Attention}\left( \textbf{H}_u^\ell \mathbf{W}^Q_n, \textbf{H}_u^\ell \mathbf{W}^K_n,  \textbf{H}_u^\ell \mathbf{W}^V_n \right),
\end{align}
\noindent $\mathbf{W}^Q_n, \mathbf{W}^K_n, \mathbf{W}^V_n \in \mathbb{R}^{d \times d/N}$ represents the head-specific mapping matrices corresponding to the query, key, value dimension, respectively. $\mathbf{W}^D \in \mathbb{R}^{d \times d}$ is a learnable projection matrix, and $\textbf{H}_u^\ell$ is the embedding matrix of user $u$'s sequence $\boldsymbol{s}_u$ at the $\ell$-th block of Transformer. Here, the self-attention calculation is conducted as: $\text{Attention}\left( \mathbf{Q},\mathbf{K},\mathbf{V}  \right) = \text{softmax}\left( \frac{\mathbf{Q}\cdot \mathbf{K}^\trans}{\sqrt{d/N}} \right)\mathbf{V}$. $\frac{d}{N}$ is the scale factor.

% $\mathbf{W}^Q_n, \mathbf{W}^K_n, \mathbf{W}^V_n \in \mathbb{R}^{d \times d/N}$ represents the head-specific mapping matrices corresponding to the query, key, value dimension, respectively. $\mathbf{W}^D \in \mathbb{R}^{d \times d}$ is a learnable projection matrix. $\textbf{H}_u^\ell$ is the embedding matrix of user $u$'s sequence $\boldsymbol{s}_u$ at the $\ell$-th block of Transformer. Here, the self-attention calculation is conducted as: $\text{Attention}\left( \mathbf{Q},\mathbf{K},\mathbf{V}  \right) = \text{softmax}\left( \frac{\mathbf{Q}\cdot \mathbf{K}^\trans}{\sqrt{d/N}} \right)\mathbf{V}$. $\frac{d}{N}$ is the scale factor.

To inject non-linearity into the embedding generation, a point-wise feed-forward network (FFN) is used for representation transformation within the sequential pattern encoder, which is defined:
% To inject the non-linearity into the embedding generation, we equip our sequential pattern encoder with a point-wise feed-forward network for representation transformation, which is defined as:
\begin{align}
    \label{eq:transformer}
    \text{PFFN}\left(\mathbf{H}_u^{\ell}\right) &= [\text{FFN}\left(\mathbf{h}_1^{\ell}\right)^\trans, \cdots, \text{FFN}\left(\mathbf{h}_T^{\ell}\right)^\trans] \\
	\text{FFN}\left(\mathbf{x}\right) &= \text{GELU}\left(\mathbf{x}\mathbf{W}_1^{\ell} + \mathbf{b}_1^{\ell}\right)\mathbf{W}_2^{\ell}+\mathbf{b}_2^{\ell},
\end{align}
\noindent where $\mathbf{W}_1^{\ell}, \mathbf{W}_2^{\ell}, \mathbf{b}_1^{\ell}, \mathbf{b}_2^{\ell}$ are learnable model parameters as projection and bias terms. $\text{GELU}(\cdot)$ is the activation function.

\subsection{Unifying Sequential and CF Views}
In real-life applications, long-tail sequences with a limited number of items are prevalent in recommendation scenarios~\cite{liu2020long,cl4srec}. These sequences pose challenges to most existing solutions. In particular, short sequences with very few items can hardly provide sufficient contextual signals for neural sequence encoders. This issue affects various types of models, such as self-attention mechanisms~\cite{bert4rec, sasrec}, and graph neural networks~\cite{gcsan, srgnn, mbht, surge}. To tackle the challenge of short sequences with very few items in sequential recommenders, we propose to unify the sequential view of item transitions and the collaborative view of user-item interactions. This design aims to capture the implicit cross-sequence user dependencies, allowing user-wise knowledge transfer in sequential recommender systems. This aspect is largely overlooked in most current solutions.


% In real-life sequential recommenders, long-tail sequence with limited number of items is prevalent in recommendation scenario~\cite{liu2020long,cl4srec}, which poses challenges to most of existing solutions. In particular, short sequences with very few items can hardly provide sufficient contextual signals for neural sequence encoders, such as recurrent neural network~\cite{gru4rec, gru4rec2}, self-attention mechanisms~\cite{bert4rec, sasrec}, and graph neural networks~\cite{gcsan, srgnn, mbht, surge}. To tackle this challenge, we propose to unify the sequential view of item transitions and the collaborative view of user-item interactions. With such design, our model can capture the implicit cross-sequence user dependencies to allow the user-wise knowledge transfer in sequential recommender, which are largely overlooked in most current solutions~\cite{zhang2022enhancing}.

To achieve the goal of unifying the sequential view of item transitions and the collaborative view of user-item interactions, you can start by generating two graphs: \emph{item transition graph} $\mathcal{G}_t$ and \emph{item co-interaction graph} $\mathcal{G}_c$. To be specific, $\mathcal{G}_t$ and $\mathcal{G}_c$ over the item set $\mathcal{V}$ are constructed by following the instructions below:
% Towards this end, we first generate the \emph{item transition graph} $\mathcal{G}_t$ and \emph{item co-interaction graph} $\mathcal{G}_c$ corresponding to the sequential, collaborative views, respectively. To be specific, $\mathcal{G}_t$ and $\mathcal{G}_c$ over the item set $\mathcal{V}$ are constructed by following the instructions below:
\begin{itemize}[leftmargin=*]
\item \textbf{Item Transition Graph $\mathcal{G}_t$}. To capture the transitional relationships among items from the sequential pattern view, adjacent item pairs (\eg $v_{t-1}$, $v_{t}$) in each sequence $\boldsymbol{s}_u$ are connected with an edge in $\mathcal{G}_t$. Given the item sequences of all users $\mathcal{S} = \{ \boldsymbol{s}_1, \boldsymbol{s}_2, \cdots, \boldsymbol{s}_{|U|} \}$, the adjacency matrix $\mathbf{A}_{\mathcal{G}_t} \in \mathbb{R}^{|\mathcal{V}| \times |\mathcal{V}|}$ representing the item correlations in graph $\mathcal{G}_t$ is generated by:
\begin{align}
\label{eq:gt}
    \mathbf{A}^u_{\mathcal{G}_t}(v_p, v_q) = \begin{cases}
                        1, & |p-q|=1 \\
                        0, & \text{otherwise}
    \end{cases} ;\quad
    \mathbf{A}_{\mathcal{G}_t} = \sum_{u=1}^{|U|}\mathbf{A}^u_{\mathcal{G}_t},
\end{align}
\noindent where $\mathbf{A}^u_{\mathcal{G}_t}$ denotes the user-specific item transition connections over sequence $\boldsymbol{s}_u$. Here, $p$ and $q$ denotes the position index in sequence. We sum up $\mathbf{A}^u_{\mathcal{G}_t}$ of all users ($u\in \mathcal{U}$) to obtain $\mathbf{A}_{\mathcal{G}_t}$. The adjacency matrix $\mathbf{A}_{\mathcal{G}t}$ takes into account the transition frequency between items with edge weights in the item transition graph. \\\vspace{-0.12in}

% Hence, the transition frequency between items are considered in $\mathbf{A}_{\mathcal{G}_t}$ to reflect the edge weights in the item transition graph $\mathcal{G}_t$.\\\vspace{-0.12in}

\item \textbf{Item Co-Interaction Graph $\mathcal{G}_c$}. To incorporate collaborative signals to model the cross-user dependencies, we generate another graph $\mathcal{G}_c$ to maintain the item correlations based on their co-interaction patterns. To this end, we firstly construct the interaction matrix $\mathbf{R} \in \mathbb{R}^{|\mathbf{U}| \times |\mathbf{V}|}$ between users and items by setting the entry $\mathbf{R}_{u, v}=1$ if user $u$ has adopted item $v$ and $\mathbf{R}_{u, v}=0$ otherwise. With the operation $\mathbf{A}_{\mathcal{G}_c} = \mathbf{R}^\trans\mathbf{R}$, we obtain the initial correlation strength between items in $\mathbf{A}_{\mathcal{G}_c}$ based on their co-interaction frequency. To filter out less-relevant item-wise connections, we apply \emph{top}-$k(\cdot)$ function to keep highly-relevant connections among items in $\mathbf{A}_{\mathcal{G}_c}$ based on top-$k$ co-interaction frequency of each item. Here, $k$ determines the density of $\mathbf{A}_{\mathcal{G}_c}$.

% In particular, there exists an edge between two items if they are adopted by the same user before. By constructing the interaction matrix $\mathbf{R} \in \mathbb{R}^{|\mathbf{U}| \times |\mathbf{V}|}$ between users and items 

\end{itemize}

After generating the item transition graph $\mathcal{G}_t$ and co-interaction graph $\mathcal{G}_c$, we utilize the graph neural network to project individual item into latent embedding space. Formally, our graph convolution-based message passing is presented as follows:
\begin{equation}
    \label{eq:gcn}
    \mathbf{X}^{(l+1)} = \left(\mathbf{D}_t^{-\frac{1}{2}} \mathbf{A}_{\mathcal{G}_t} \mathbf{D}_t^{-\frac{1}{2}}\right)\mathbf{X}^{(l)};\ 
    \mathbf{Z}^{(l+1)} = \left(\mathbf{D}_c^{-\frac{1}{2}} \mathbf{A}_{\mathcal{G}_c} \mathbf{D}_c^{-\frac{1}{2}}\right)\mathbf{Z}^{(l)}
\end{equation}
\noindent We let $\mathbf{X}^{(l)}$ and $\mathbf{Z}^{(l)}$ respectively denote the embedding matrix of items over the item transition graph ($\mathcal{G}_t$) and the co-interaction graph ($\mathcal{G}_c$) under the $l$-th graph layer. $\mathbf{D}_a$ and $\mathbf{D}_i$ are degree matrices used for graph normalizing. To simplify the model with lightweight GNN architecture, we remove the redundant transformation and activation operations during the message propagation.

\subsection{Adaptive Cross-View Contrastive Learning}
\label{sec:adaptive}
Building on the success of contrastive data augmentation across various domains, including vision learning~\cite{he2020momentum}, text mining~\cite{rethmeier2021primer}, and graph modeling~\cite{zhu2021graph}, our \model\ method harnesses self-supervised signals through contrastive learning across different item semantic views. Nonetheless, the popularity bias is often overlooked, as conformity can entangle real interests and subsequently influence user behaviors~\cite{zheng2021disentangling,chen2021autodebias}. For instance, a user might be influenced by conformity to click on a product or watch a short video, following the actions of others, rather than being genuinely interested in the content. If user interest and conformity are not disentangled when generating augmented signals, contrastive learning methods may focus on incorrect positive pairs, thereby introducing biased information. This can lead to less-interested recommendation.

% With the success of contrastive data augmentation in various domains, vision learning~\cite{he2020momentum}, text mining~\cite{rethmeier2021primer}, and graph modeling~\cite{zhu2021graph}, our \model\ method distills self-supervised signals with the contrastive learning across different item semantic views. However, the popularity bias is ignored when the conformity often entangles the real interests to influence user behaviors~\cite{zheng2021disentangling,chen2021autodebias}. For example, a user may follow others to click a product or view a short-video due to his/her conformity, rather than driven by the real interest. Without disentangling user interest and conformity in generating augmented signals, contrastive learning method may concentrate on the incorrect positive pairs and introduce biased information. 

Intuitively, conformity may vary across users and interactions. For example, user conformity and real interest might be entangled in a complex manner, jointly driving interaction behaviors. This complexity makes it challenging to accurately disentangle conformity from genuine interest, which is essential for providing more helpful augmented SSL signals. To address this challenge, we propose a debiased cross-view contrastive learning approach with adaptive augmentation that incorporates interaction-level conformity. We develop a multi-channel conformity weighting network (CWNet) to calculate the conformity degree of an interaction. By incorporating the estimated conformity degrees into our contrastive learning paradigm, we can adaptively determine the regularization strength. This allows the model to more effectively disentangle user interests from conformity behaviors.

% Intuitively, conformity may vary by users and interactions, \eg user conformity and real interest may be entangled in a complex way and jointly drive the interaction behaviors. To tackle this challenge, we propose a debiased cross-view contrastive learning approach with adaptive augmentation which incorporates the interaction-level conformity. To achieve this goal, we devise a multi-channel conformity weighting network (CWNet) to derive the conformity degree of an interaction. Then, the estimated conformity degrees are incorporated into our contrastive learning paradigm to determine the regularization strength in an adaptive manner.

\subsubsection{\bf Multi-Channel Conformity Weighting Network}
In our CWNet module, we aim to learn the conformity degree of an interaction between user $u$ and item $v$ from three semantic channels.\\\vspace{-0.12in}

\begin{itemize}[leftmargin=*]
\item (1) \textbf{ User-Specific Conformity Influence}. First, we propose to infer the interaction-level (\eg $u-v$) conformity degree by considering the conformity of user $u$ based on his/her past interactions. Given a user with strong conformity, their interactions are more likely to be influenced by popularity bias compared to others who exhibit strong individuality. To obtain the conformity degree of user $u$, we perturb the item transition graph $\mathcal{G}_t$ by removing the edges generated from $u$'s sequence $\boldsymbol{s}u$. This results in the generation of an augmented adjacency matrix $\bar{\mathbf{A}}{\mathcal{G}c}$, where $\bar{\mathbf{A}}^u{\mathcal{G}_t}(v_p, v_q) = 0$ for any two adjacent items $v_p$ and $v_q$ in $\boldsymbol{s}_u$. Subsequently, both the original and augmented item transition graphs are fed into our graph encoder (as per Eq.~\ref{eq:gcn}) to generate two embeddings ($\mathbf{x}_v, \mathbf{x}_v^\prime$) for the target item $v$. The user-specific conformity influence, denoted as $\omega^1_{\left(u,v\right)}$, is estimated using the cosine similarity between the two embeddings ($\mathbf{x}_v, \mathbf{x}v^\prime$), calculated as $\omega^{\alpha}{\left(u,v\right)} = \cos\left( \mathbf{x}_v, \mathbf{x}_v^\prime \right)$. A larger $\omega^u$ score indicates that user $u$'s interactions have little influence over the item graph structures, suggesting that their interaction patterns are more likely to be observed from others, \ie strong user conformity.  \\\vspace{-0.12in} 

\item (2) \textbf{Consistency with Other Users}. We also propose to calculate the conformity from the perspective of considering the sequential behavior consistency between the target user and others. In particular, for a given $u-v$ interaction, we compare the learned transitional patterns of user $u$ with those of other relevant users. To be specific, given the target item $v$, we aggregate the intra-sequence neighboring information using mean-pooling among inner neighbors within the sequence $\boldsymbol{s}_u$. The overall transitional patterns of other correlated users are combined to obtain $\overline{\mathbf{x}}_{O_v}$, which is derived from $v$'s outer neighbors $O_v$ across different user sequences. After that, the transition consistency is measured by $\omega^{\beta}_{\left(u,v\right)} = \cos\left(\overline{\mathbf{x}}_{N_v}, \overline{\mathbf{x}}_{O_v}\right)$. This measure quantifies the degree of consistency between the target user's sequential behavior and that of other users, providing insights into conformity. \\\vspace{-0.12in}

\item (3) \textbf{Subgraph Isomorphic Property}. The isomorphic property of item subgraph is also an important factor in reflecting user conformity with similar interaction patterns. To incorporate this factor into our conformity estimation, we calculate the similarity between item $v$'s embedding $\mathbf{x}_v$ and the representation $\overline{\mathbf{x}}_{O_v}$ aggregated from its outer neighbors, \ie $\omega^{\gamma}_{\left(u,v\right)} = \cos\left(\mathbf{x}_v, \overline{\mathbf{x}}_{O_v}\right)$.
\end{itemize}

\noindent \textbf{Mixing Signals from Different Channels.} We derive the final interaction-level conformity degree by fusing the information from the above three channels. Here, we first adopt mean-pooling over channel-specific results as: $\omega_{\left(u,v\right)} = \frac{1}{3}\sum_{\lambda \in \left\{\alpha,\beta,\gamma \right\}}\omega_{\left(u,v\right)}^{\lambda}$. Following the mapping strategy in~\cite{kgcl,zhu2021graph}, we perform the transformation for $\omega$ values as follows:
\begin{equation}
\label{eq:omega}
    \omega^{(1)} = \text{sigmoid}\left(\omega\right);\ 
    \omega^{(2)} = \frac{\omega^{(1)} - \omega_{min}^{(1)}}{\omega_{max}^{(1)} - \omega_{min}^{(1)}};\ 
    \omega^{(3)} = \frac{\mu_c}{\overline{\omega}^{(2)}} \cdot \omega^{(2)}
\end{equation}
\noindent $\mu_c$ is the hyperparameter that adjusts the mean value $\overline{\omega}$ of $\omega$. We omit the subscript $\left(u,v\right)$ for simplicity and adopt $\omega = \omega^{(3)}$ as the output conformity. Furthermore, to approximate the conformity degrees with normal distribution, we adopt the KL-divergence over the derived conformity results of all interactions:
\begin{equation}
    \label{eq:kl}
    \mathcal{L}_w = \sum_{i=1}^{|\{(u,v)\}|} \phi_i\log\frac{\phi_i}{\omega_i},
\end{equation}
\noindent where $\phi_i$ is generated by random sampling from normal distribution with the hyperparameter $\mu_c$ for the mean and $\sigma$ for the standard deviation. $\omega_i$ is the conformity weighting result.

\subsubsection{\bf Conformity-aware Contrastive Augmentation}
To enhance our \model\ with adaptively debiased augmentation, we integrate the conformity factor into our embedding contrasting paradigm to determine the agreement regularization strength. As discussed before, both sequential and collaborative views are generated through different encoders, \ie Transformer and GNNs. Our \model\ employs contrastive learning (CL) to learn conformity-aware augmented representations from two key dimensions:\\\vspace{-0.12in}

\noindent \textbf{Contrasting from User Dimension}. The first stage of our CL paradigm aims to realize the knowledge transfer across different users. By contrasting user-specific preferences with cross-user global interaction patterns, the learned augmented representations can naturally preserve user-wise implicit dependencies. In this process, the conformity regularizer weakens the impacts of perturbations caused by popularity bias for SSL augmentation. Given the embedding $\mathbf{h}_v$ and $\mathbf{x}_v$ encoded generated by our sequential pattern encoder (Eq.~\ref{eq:transformer}) and transition graph encoder (Eq.~\ref{eq:gcn}), respectively, our debiased contrastive learning paradigm is given as follows:
\begin{equation}
\label{eq:cl1}
    \mathcal{L}_u = -\sum_{u\in \mathcal{U}}\sum_{v \in \boldsymbol{s}_u}  \omega_{\left(u,v\right)} \log \frac{\exp\left(\cos\left(\mathbf{h}_v, \mathbf{x}_v\right)/\tau\right)}{\sum_{v^\prime\in \mathcal{V}} \exp\left(\cos\left(\mathbf{h}_v, \mathbf{x}_{v^\prime}\right)/\tau\right)},
\end{equation}
\noindent In the SSL loss $\mathcal{L}_u$, InfoNCE~\cite{infonce} is adopted for embedding contrasting. By incorporating our learned conformity factor $\omega$, we allow representations $\mathbf{h}_v$ and $\mathbf{x}_v$ to supervise each other adaptively, that is, weighted by the interaction-level conformity. \\\vspace{-0.12in}

% With the control of our learned conformity $\omega$, we let representations $\mathbf{h}_v$ and $\mathbf{x}_v$ supervise each other in an adaptive manner, \ie weighed by the interaction-level $\left(u,v\right)$ conformity context.\\\vspace{-0.12in}

\noindent \textbf{Contrasting from Item Dimension}. The goal of our second stage CL is to extract self-supervision signals by contrasting the global item embedding $\mathbf{x}_v$ with the item semantic representation $\mathbf{z}_v$. Our conformity factor $\omega_{u,v}$ is incorporated into this contrasting process by estimating the uniformity $\psi_{\left(u,v\right)}=1-\omega_{\left(u,v\right)}$. Formally, our item dimension CL loss $\mathcal{L}_v$ is defined as follows:
\begin{equation}
\label{eq:cl2}
    \mathcal{L}_v = -\sum_{u\in \mathcal{U}}\sum_{v \in \boldsymbol{s}_u}  \psi_{\left(u,v\right)} \log \frac{\exp\left(\cos\left(\mathbf{x}_v, \mathbf{z}_v\right)/\tau\right)}{\sum_{v^\prime\in \mathcal{V}} \exp\left(\cos\left(\mathbf{x}_v, \mathbf{z}_{v^\prime}\right)/\tau\right)},
\end{equation}
\noindent In our CL paradigm, instance self-discrimination~\cite{sgl,xia2022hypergraph} is used for generating positive pairs. Representation of different samples are pushed apart as negative pairs to reflect embedding uniformity.

\subsection{Model Training and Prediction}
In the training phase, the last interacted item of each sequence $\boldsymbol{s}_u$ is regarded as the label for model optimization. In the prediction phase, to encourage the cooperation between sequence and collaborative views, we combine view-specific item embeddings into an aggregated representation $\mathbf{p}_v$ with the learnable attentive weights:
\begin{align}
\label{eq:fusion}
    f\left(\mathbf{e}\right) = \frac{\exp\left(\mathbf{a}^\trans \cdot \mathbf{W}_a \mathbf{e}\right)}{\sum_{i=1}^3 \exp\left(\mathbf{a}^\trans \cdot \mathbf{W}_a \mathbf{e}\right)};~~~ \mathbf{p}_v = \sum_{\mathbf{e}\in \left\{\mathbf{h},\mathbf{x},\mathbf{z} \right\}}f\left(\mathbf{e}\right)\mathbf{e}
\end{align}
\noindent where $\mathbf{a}\in\mathbb{R}^{d}$ and $\mathbf{W}_a\in\mathbb{R}^{d\times d}$ are trainable attention parameters. Input embedding $\mathbf{e}$ is selected from the set of view-specific representations, \ie $\mathbf{e} \in \left\{\mathbf{h}_v, \mathbf{x}_v, \mathbf{z}_v \right\}$. $\mathbf{p}_v$ is derived through attentive aggregation with the view-specific importance score $f\left(\mathbf{e}\right)$.

The next item interaction probability $\hat{y}_{u,v}$ is derived as $\hat{y}_{u,v} = \mathbf{p}_{|\boldsymbol{s}_u|}^\trans\mathbf{v}$, where we adopt hidden states of the last item on the sequence as the user embedding. For each user and the ground truth item $v_t$ pair, we utilize the cross-entropy as the loss:
\begin{equation}
    \mathcal{L}_{rec} = \sum_{(u,v_{T+1})\in \mathcal{D}^+} -\log\frac{\exp \hat{y}_{u,v_{T+1}}}{\sum_{v^\prime \in \mathcal{V}}\exp\hat{y}_{u,v^\prime}},
\end{equation}
\noindent where $\mathcal{D}^+$ is the training data set of positive interactions at the $T+1$ timesteps.
To supplement the recommendation loss $\mathcal{L}_{rec}$ with our augmented SSL tasks under a multi-task training framework, we define our joint optimized objective $\mathcal{L}$ as:
\begin{equation}
\label{eq:all}
    \mathcal{L} = \mathcal{L}_{rec} + \lambda_1\left(\mathcal{L}_u+\mathcal{L}_v\right) + \lambda_2\left(\mathcal{L}_w\right),
\end{equation}
\noindent where $\lambda_1$ and $\lambda_2$ are parameters to balance the tasks-specific loss. $\mathcal{L}_w$ is the regularization term with KL-divergence for mixing signals (Eq.~\ref{eq:kl}) in our multi-channel conformity weighting network.\\\vspace{-0.12in}

\noindent \textbf{Time Complexity Analysis}. In our sequential pattern encoder, the computational cost is $O\left(T^2d+Td^2\right)$ where the majority of the cost  is attributed to the item-wise self-attention operations. In our GNN encoder, the graph convolutional message passing and aggregation have a complexity of $O\left(|\mathcal{V}|d^2\right)$. In the cross-view representation aggregation, our \model\ requires a computational cost of $O(d^2)$ for attentional weighting. Owing to the independent nature of our sequential and collaborative relation encoders, the Transformer and GNN encoding can be performed in parallel using the CUDA infrastructure for speeding up computation. In summary, the time complexity of our \model\ is $O\left(\left(|\mathcal{V}|+1\right)d^2\right)$, making it comparable to the state-of-the-art GNN-based sequential recommenders. 

% which is largely dominated by the item-wise self-attention operations~\cite{sasrec}. In our GNN encoder, the graph convolutional message passing and aggregation takes $O\left(|\mathcal{V}|d^2\right)$ complexity. For the cross-view representation aggregation, our \model\ needs $O(d^2)$ cost for attentional weighting. Due to the independent property of our sequential and collaborative relation encoders, Transformer and GNN encoding can be performed in a parallel way using CUDA infrastructure. Overall, the time complexity of our \model\ is $O\left(\left(|\mathcal{V}|+1\right)d^2\right)$, which is comparable to GNN-based state-of-the-art sequential recommenders.

\vspace{-0.1in}
\subsection{Theoretical Analyzes of \model}
\begin{figure}
    \centering
    \includegraphics[width=0.9\linewidth]{material/theo_case.pdf}
    \vspace{-0.2in}
    \caption{Upper part: curve of $0.5f(p)$ and $0.5f(n)$ under $\tau=0.4$. Lower part: distribution area of potential values of $\omega \cdot f(p)$ and $\omega \cdot f(n)$ and random samples within a batch.}
    \label{fig:theo}
    \vspace{-0.2in}
\end{figure}

In this section, we provide an analysis of how the new conformity-aware contrastive learning paradigm benefits the recommendation task. We focus on how to bring theoretical interpretability for the conformity-aware adaptive contrastive learning in Equation~\ref{eq:cl1}-\ref{eq:cl2}. We take Equation~\ref{eq:cl1} for studying because of the symmetry of these two equations. Following the discussion in~\cite{kgcl, sgl, khosla2020supervised}, the gradient of the contrastive objective in Equation~\ref{eq:cl1} can be expressed as:
\begin{equation}
\label{eq:contrib1}
    \nabla \mathcal{L}_u^{(u,v)} = \frac{1}{\tau \|\mathbf{h}_v\|}\left(c(v)+\sum_{v^\prime\in V\setminus\{p\}}c(v^\prime)\right),
\end{equation}
\noindent where $\mathcal{L}_u^{(u,v)}$ is the contrastive loss $\mathcal{L}_u$ for an user-item pair $(u,v)$. $c(v)$ and $c(v^\prime)$ are the gradient contribution from the positive pair $(\mathbf{h}_v,\mathbf{x}_v)$ and the negative pair, respectively. Formally, $c(v)$ and $c(v^\prime)$ are derived using the following formulas:
\begin{equation}
\label{eq:contrib2}
    \begin{aligned}
    c(v) &= \left(\mathbf{\bar{x}}_v - \left(\mathbf{\bar{h}}_v^\trans\mathbf{\bar{x}}_v\right)\mathbf{\bar{h}}_v\right)^\trans\left(P_{vv}-1\right)\\
    c(v^\prime) &= \left(\mathbf{\bar{x}}_{v^\prime} - \left(\mathbf{\bar{h}}_v^\trans\mathbf{\bar{x}}_{v^\prime}\right)\mathbf{\bar{h}}_v\right)^\trans P_{vv^\prime},
    \end{aligned}
\end{equation}
\noindent where $P_{vi} = \exp\left(\mathbf{\bar{h}}_v^\trans\mathbf{\bar{x}}_i / \tau\right) \big/ \sum_{i\in V\setminus\{v\}}\exp\left(\mathbf{\bar{h}}_v^\trans\mathbf{\bar{x}}_i / \tau \right)$. $\mathbf{\bar{x}}, \mathbf{\bar{h}}$ are normalized representations. To this end, we can derive two functions $f(p)$ and $f(n)$ that are proportional to the $L_2$ norm of $c(v)$ and $c(v^\prime)$~\cite{sgl}. Specifically, we have the following derivations:
\begin{equation}
\label{eq:contrib3}
    f_1(p) = \sqrt{1-p^2}\left(\exp\left(\frac{p}{\tau}\right)-1\right);\ f_2(n) = \sqrt{1-n^2}\left(\exp\frac{n}{\tau}\right)
\end{equation}
\noindent where $p = \mathbf{h}_v^\trans\mathbf{x}_v$ is the agreement between the positive pair. $n = \mathbf{h}_v^\trans\mathbf{x}_v^\prime$ denotes the similarity between the negatives. To visualize the impact of $\mathcal{L}_u$ without adaptive weight $\omega$, we plot the curve of $0.5f_1(p)$ and $0.5f_2(n)$ in Figure~\ref{fig:theo}. Note that without $\omega$, the coefficient of $\mathcal{L}_u$ is 0.5 by default. From the curves, it is obvious that the contribution of positive and negative samples at different similarity levels are fixed. This means that the model has difficulty in discriminating among diverse samples. In the context of an interest-driven interaction, it is crucial to dynamically reduce the influence of samples from the conformity modeling view.

% and $n = \mathbf{h}_v^\trans\mathbf{x}_v^\prime$ are the agreement between the positive pair and the similarity between the positive and the negative. To visualize the impact of $\mathcal{L}_u$ without adaptive weight $\omega$, we plot the curve of $0.5f_1(p)$ and $0.5f_2(n)$ in Figure~\ref{fig:theo}. Note that without $\omega$, the coefficient of $\mathcal{L}_u$ is 0.5 by default. From the curves, it is obvious that the contribution of positive and negative samples at different similarity levels are fixed. That is, the model lacks the ability to discriminate among diverse samples. Specifically, for an interest-driven interaction, impact of samples from the conformity modeling view should be dynamically weighted down. 

At this stage, we investigate the advantages of introducing a conformity-aware weight (denoted by $\omega$) in contrastive learning. Specifically, the conformity-aware weight $\omega$ influences the learning process by directly scaling the gradient values. Recall that the distribution of $\omega$ is restrained by normal distribution in Equation~\ref{eq:kl}. The distribution range of $\omega\cdot f_1(p)$ and $\omega\cdot f_2(n)$ creates an area rather than a single curve. We further plot the distribution areas in the lower part of Figure~\ref{fig:theo}. The values are weighted by the interaction-level conformity, falling within the ranges of $(0,f_1(p))$ and $(0,f_2(n))$ following a normal distribution. We also plot the discrete distribution of $\omega\cdot f_1(p)$ and $\omega\cdot f_2(n)$ by sampling two batches of training data. As evident from the distributions, the effect of some samples is enhanced while the influence of others are weakened. This endows the learning process with richer semantics, allowing for a dynamic and adaptive contribution of samples to the contrastive learning gradients with data debiasing. The analyzes also apply to $\mathcal{L}_v$ in Equation \ref{eq:cl2}, since $\gamma = 1-\omega$ has similar properties.

% To this stage, we study what benefits does the introduction of the conformity-aware weight $\omega$ brings. Specifically, $\omega$ weights the contribution to the gradient by directly scaling the gradient values. Recall that the distribution of $\omega$ is restrained by normal distribution in Equation~\ref{eq:kl}. The distribution range of $\omega\cdot f_1(p)$ and $\omega\cdot f_2(n)$ forms an area instead of a curve. We further plot the distribution areas in the lower part of Figure~\ref{fig:theo}. The values are weighted by the interaction-level conformity, range in $(0,f_1(p))$ and $(0,f_2(n))$ following normal distribution. We also plot the discrete distribution of $\omega\cdot f_1(p)$ and $\omega\cdot f_2(n)$ by sampling two batches of training data. Evidently, the effect of some samples is enhanced while some others are weakened. 

% !TEX root = paper.tex

\section{Target systems}\label{chap:targetSystems}

In this section, we illustrate the three models we attacked: Gemini \cite{xu2017neural}, GMN \cite{DBLP:conf/icml/LiGDVK19}, and SAFE \cite{massarelli2021function}. We give a high-level description of their internals and then discuss specific provisions for the Greedy (Section \ref{sec:greedy}) and GCAM (Section \ref{sec:whitebox}) attacks---whereas Spatial Greedy needs no adaptations.

For their choice, we first surveyed recent comparative evaluations (most prominently~\cite{marcelli2022machine}) to identify plausible, performant candidates. We then reviewed the characteristics of the neural networks and the code analysis choices behind them, identifying three distinctly different approaches to problem solving. As we will see, for example, GMN and Gemini rely on CFG properties whereas SAFE does not; GMN accounts for graph similarity using an explicit matching mechanism that is different from the message passing network behind Gemini; SAFE employs a different kind of neural network than the other two systems. 

\subsection{Gemini}\label{sec:gemini}

Gemini \cite{xu2017neural} represents functions in the problem space through their Attributed Control Flow Graph (ACFG). An ACFG is a control flow graph where each basic block consists of a vector of manual features (i.e., node embeddings).

The focal point of this approach consists of a graph neural network (GNN) based on the Structure2vec~\cite{DBLP:conf/icml/DaiDS16} model that converts the ACFG into an embedding vector, obtained by aggregating the embedding vectors of individual ACFG nodes. The similarity score for two functions is given by the cosine similarity of their ACFG embedding vectors. 


\subsubsection{Greedy Attack}\label{sec:black:gemini}
Each ACFG node contributes a vector of 8 manually selected features. Five of these features depend on the characteristics of the instructions in the node, while the others on the graph topology. The model distinguishes instructions from an ISA only for how they contribute to these 5 features.
This enables a gray-box variant of our Greedy attack: we measure the robustness of Gemini using a set of candidates $\texttt{CAND}$ of only five instructions, carefully selected for covering the five features. Later in the paper, we use this variant as the baseline approach for a comparison with Spatial Greedy.



\subsubsection{GCAM Attack}\label{sec:white:gemini}

As described in the previous section, some of the components of a node feature vector $v$ depend on the instructions inside the corresponding basic block. As Gemini maps all possible ISA instructions into 5 features, we can associate each instruction with a deterministic modification of $v$ represented as a vector $u$. We select five categories of instructions and for each category $c_j$ we compute the modification $u_j$ that will be applied to the feature vector $v$. We selected the categories so as to cover the aforementioned features.

When we introduce in the block an instruction belonging to category $c_j$, we add its corresponding $u_j$ modification to the feature vector $v$. Therefore, adding instructions inside the block modifies the feature vector $v$ by adding to it a linear combination vector $\sum_{j}n_j u_j$, where $n_j$ is the number of instructions of category $c_j$ added. Our perturbation $\delta$ acts on the feature vector of the function only in the components corresponding to the added dead branches, by modifying the coefficients of the linear combination above.

Since negative coefficients are meaningless, we avoid them by adding to the optimization problem appropriate constraints. Moreover, we solve the optimization problem without forcing the components of $\delta$ to be integers, as this would create an integer programming problem. Therefore, at the end of the iterative optimization process, we get our problem-space perturbation $\delta_p$ by \textit{rounding} each component of $\delta$. It is immediate to obtain from $\delta_p$ the problem-space modification to our binary function $f_1$. Indeed, in each dead block, we must add as many instructions belonging to a category as the corresponding coefficient in $\delta_p$.

\subsection{GMN}\label{sec:gmn}
Graph Matching Network (GMN)~\cite{DBLP:conf/icml/LiGDVK19} computes the similarity between two graph structures. When functions are represented through their CFGs, GMN offers state-of-the-art performance for the binary similarity problem~\cite{DBLP:conf/icml/LiGDVK19,marcelli2022machine}.

Differently from solutions based on standard GNNs (e.g., Gemini), which compare embeddings built separately for each graph, GMN computes the distance between two graphs as it attempts to match them. In particular, while in a standard GNN the embedding vector for a node captures properties of its neighborhood only, GMN also accounts for the similarity with nodes from the other graph.

\subsubsection{Greedy Attack}\label{sec:black:gmn}
Similarly to the case of Gemini, each node of the graph consists of a vector of manually-engineered features. In particular, each node is a bag of 200 elements, each of which represents a class of assembly instructions, grouped according to their mnemonics. The authors do not specify why they only consider these mnemonics among all the available ones in the \texttt{x86-64} ISA.
Analogously to Gemini, when testing the robustness of this model against the Greedy approach we devise a gray-box variant by considering a set of candidates $\texttt{CAND}$ of 200 instructions, each of which belonging to one and only one of the considered classes.

\subsubsection{GCAM Attack}\label{sec:white:gmn}
Our white-box attack operates analogously to what we presented in Section \ref{sec:white:gemini}.
Similarly to the Gemini case, each dead branch adds a node to the CFG while the feature mapping function transforms each CFG node into a feature vector. The feature vector is a bag of the instructions contained in the node, where assembly instructions are divided into one of 200 categories using the mnemonics.



\subsection{SAFE}\label{sec:safe}
SAFE~\cite{massarelli2021function} is an embedding-based similarity model. It represents functions in the problem space as sequences of assembly instructions.
It first converts assembly instructions into continuous vectors using an instruction embedding model based on the word2vec~\cite{DBLP:conf/nips/MikolovSCCD13} word embedding technique.
Then, it supplies such vectors to a bidirectional self-attentive recurrent neural network (RNN), obtaining an embedding vector for the function. The similarity between two functions is the cosine similarity of their embedding vectors.

\subsubsection{Greedy Attack}\label{sec:black:safe}
The Greedy attack against SAFE follows the black-box approach described in Section \ref{sec:greedy}. Since SAFE does not use manually engineered features, we cannot select a restricted set of instructions that generates all vectors of the feature space for a gray-box variant. We test its resilience against the Greedy approach considering a carefully designed list of candidates $\texttt{CAND}$ composed of random and hand-picked instructions, meaning that the baseline is a black-box attack.


\subsubsection{GCAM Attack}\label{sec:white:safe}
In the feature space, we represent a binary function as a sequence of instruction embeddings belonging to a predefined metric space. The perturbation $\delta$ is a sequence of real-valued vectors initialized with embeddings of real random instructions; each dead block contains four of such vectors. In the optimization process, we modify each embedding $i_j \in \delta$ by a small quantity given by the negative gradient of the loss function $\mathcal{L}$. In other words, every time we optimize the objective function, we alter each $i_j \in \delta$ by moving it in the negative direction identified through the gradient.

Since during optimization we modify instruction embeddings in terms of their single components, we have no guarantee that the obtained vectors are embeddings of real instructions. For this reason, after the optimization process, we compute the problem-space perturbation $\delta_p$ by {\em rounding}, at each iteration, the embeddings in $\delta$ to the closest embeddings in the space of real instruction embeddings.

 \section{Benchmarks and Evaluation}
\label{sec:eval}

We evaluate \krakenSpace to answer the following set of questions:
\begin{itemize}
\item How much improvement does partial evaluation and our implemented compiler optimizations give \kraken? %(\S \ref{sec:eval2})
\item How much faster is our purely functional f-expr language, \krakenSpace, compared to other implementations of fexprs? %(\S \ref{sec:eval1} - \ref{sec:eval2})
\item How does \kraken's performance, with its fexprs, compare to macros? %(\S \ref{sec:eval1}, \S \ref{sec:eval3})
\item How do the different partial evaluation mechanisms/optimizations in \krakenSpace contribute towards reduction in overall runtime?
%\item What does \krakenSpace do internally when we create a data structure and evaluate it for some function? (\S \ref{sec:casestudy})
\end{itemize}

\textbf{Experimental Setup}: 
We ran these experiments in a reproducible Nix environment on a NixOS install \cite{10.1145/1411203.1411255} (Kernel 6.0.0) on a laptop with 8 cores / 16 threads and 64 GB of RAM.
Our code contains the scripts and Nix Flakes needed to reproduce the exact set of dependencies to run our tests.
%The code can be found at \url{https://github.com/limvot/kraken}.

The Kraken benchmarks were run using both the Wasmtime and WAVM WebAssembly engines for most benchmarks.
The Wasmtime WebAssembly engine is one of the most popular, developed by the Bytecode Alliance itself, and uses the CraneLift code generation backend.
The WAVM WebAssembly engine is interesting for its use of LLVM, and it often produces the fastest code on benchmarks but has a higher startup time.
We eliminated the Cfold Wasmtime benchmark due to problems running out of stack space (a known property of the Cfold benchmark).

\textbf{Benchmarks}: 
To showcase the capability of Kraken, we created benchmarks that are commonly implemented in functional languages and have been used as benchmarks in other papers \cite{reinking2021perceus, 10.1145/3547646}.
The benchmarks are
\begin{itemize}
\item Fib - Calculating the nth Fibonacci number
\item RB-Tree - Inserting n items into a red-black tree, then traversing the tree to sum its values
\item Deriv - Computing a symbolic derivative of a large expression
\item Cfold - Constant-folding a large expression
\item NQueens - Placing n number of queens on the board such that no two queens are diagonal, vertical, or horizontal from each other
\end{itemize}
All benchmarks besides Fibonacci use the fexpr version of match for pattern matching in \kraken, which is equivalent to the macro version in NewLisp. We also RB-Tree using NewLisp's~\cite{mueller2018newlisp} version of fexpr match. We modified the sizes of the problems presented to the benchmark to account for the longer running times of some of the less-optimized implementations.
The code for Kraken and NewLisp is very similar, and we should note that it is very unidiomatic NewLisp.
Our goal was not to compare Kraken and NewLisp as implementation languages for Red-Black Trees, but to stress test a single reasonably complex fexpr/macro, namely pattern matching.
% \textbf{Comparison with other languages}: We evaluated \krakenSpace against a language that contains f-exprs, as well as against itself with various optimizations disabled. The only other language we could find which contains a real f-expr mechanism is NewLisp~\cite{mueller2018newlisp} and so we ported \kraken's benchmark implementation to NewLisp.

%The six state-of-the-art languages are Java 17.0.1, Swift 5.4.2, Koka 2.3.2, C++, Haskell 8.10.7, and OCaml 4.12.
%The language choices were taken directly from Perceus reference-counting paper \cite{reinking2021perceus}.
%The Fibonacci benchmark additionally tests Python 3.9.11 and Chez Scheme 9.5.4.
%Koka, Ocaml and Haskell are good comparison points as statically-typed, compiled, functional programming languages, while Chez Scheme is a good comparison point as a mature and industrial strength dynamically-typed Scheme implementation known for its performance. 
%\subsection{Basic Level Comparison}
\subsection{The Effect of Partial Evaluation on Eval Calls}

\begin{table}[h]
\caption{Number of eval calls with no partial evaluation for Fexprs}
	\begin{tabular}{||c | c c c c c ||} 
		\hline
		&Evals & Eval w1 Calls & Eval w0 Calls & Comp Dyn & Comp Dyn\\ 
        & & & & w1 Calls & w0 Calls\\ [0.5ex] 
		\hline\hline
		Cfold 5 & 10897376 & 2784275 & 879066  & 1 & 0 \\ 
		\hline
		  Deriv 2  & 11708558 & 2990090 & 946500 & 1 & 0 \\ 
        \hline
		  NQueens 7 & 13530241 & 3429161 & 1108393 & 1 & 0 \\ 
    \hline
		  Fib 30 & 119107888 & 30450112 & 10770217 & 1 & 0 \\ 
    \hline
		  RB-Tree 10 & 5032297 & 1291489 & 398104 & 1 & 0 \\ 
		\hline
	\end{tabular}
    \label{npe:calls}
 \end{table}

As mentioned before, using fexprs without partial evaluation will prelude optimization and cause a massive amount of repeated work. Table \ref{npe:calls} and Table \ref{pe:calls} show the number of calls to the \krakenSpace runtime's eval function, the number of times the runtime's eval function executed a call to an applicative with wrap\_level=1, the number of times the runtime's eval function executed a call to an operative with wrap\_level=0, the number of compiled dynamic calls to applicatives with wrap\_level=1, and the number of compiled dynamic calls to operatives with wrap\_level=0.
These are shown for \krakenSpace test cases with partial evaluation turned off and turned on. 
\begin{table}[h]
\caption{Number of eval calls in Partially Evaluated Fexprs}
	\begin{tabular}{||c | c c c c c ||} 
		\hline
		&Evals & Eval w1 Calls & Eval w0 Calls & Comp Dyn & Comp Dyn\\ 
        & & & & w1 Calls & w0 Calls\\ [0.5ex] 
		\hline\hline
		Cfold 5 & 0 & 0 & 0  & 0 & 0 \\ 
		\hline
		  Deriv 2  & 0 & 0 & 0 & 2 & 0 \\ 
        \hline
		  NQueens 7 & 0 & 0 & 0 & 0 & 0 \\ 
    \hline
		  Fib 30 & 0 & 0 & 0 & 0 & 0 \\ 
    \hline
		  RB-Tree 10 & 0 & 0 & 0 & 10 & 0 \\ 
		\hline
	\end{tabular}
    \label{pe:calls}
 \end{table}

\begin{table}[h]
\caption{Number of calls to the runtime's eval function for RB-Tree. The table shows the non-partial evaluation numbers -> partial evaluation numbers.}
	\begin{tabular}{||c | c c c c c ||} 
		\hline
		&Evals & Eval w1 Calls & Eval w0 Calls & Comp Dyn & Comp Dyn\\ 
        & & & & w1 Calls & w0 Calls\\ [0.5ex] 
		\hline\hline
		  RB-Tree 7 & 2952848 -> 0 & 757932 -> 0 & 233513 -> 0 & 1 -> 7 & 0 -> 0\\ 
        \hline
		  RB-Tree 8 & 3532131 -> 0 & 906548 -> 0 & 279379 -> 0 & 1 -> 8 & 0 -> 0\\ 
        \hline
		  RB-Tree 9 & 4278001 -> 0 & 1097965 -> 0 & 3383831 -> 0 & 1 -> 9 & 0 -> 0\\ 
		\hline
	\end{tabular}
    \label{pe:rb}
    \vspace{-4mm}
 \end{table}

Without partial evaluation, no compilation can be done because it is impossible to tell if arguments to calls will be evaluated. In all benchmarks, partial evaluation removed all calls to the runtime's eval function, resulting in a completely compiled program. Looking at RB-Tree, there are over a million calls to combiners with wrap level 1 (normal functions), and 398,000 calls to combiners with wrap level 0 (operatives replacing macros). This massive blowup in the number of calls is due to the repeated and exponential re-execution of macro-like-combiners in the definition of other macro-like-combiners, as discussed in the Introduction.

The non-partially-evaluated benchmarks show 1 compiled dynamic call to an applicative (its the first call into eval) and 0 compiled dynamic calls to operatives, because there is no compilation at all. For the partially evaluated benchmarks, there are a few compiled dynamic calls to applicatives due to higher-order function use in the benchmarks, and there are no compiled dynamic calls to operatives, as all operative use has been eliminated.
We also varied the inputs for RB-Tree shown in Table \ref{pe:rb} to give a sense for how the number scale with respect to input size.

The incredible slowdown implied by these tables comes to full fruition in our RB-Tree test in Fig.~\ref{fig:kraken_nqueens_rbtree}.
We kept this run shorter because Kraken's non-partial-evaluating interpreter takes an incredibly long time even for 100 insertions (40 minutes).
The compounding layers of repeated macro-like operative calls in the non-partially-evaluated Kraken version cause a ~70,000x slowdown relative to the partial evaluated, optimized, and compiled version.
For the remaining benchmarks, we remove the naive interpreted \krakenSpace version, as in each case its performance is so bad as to blow out the graph and make it impossible to do any comparison.
In our optimized Kraken, our partial evaluation algorithm is able to fully collapse these levels of inefficiency, evaluate and inline the results, and give the backend more specialized code to optimize, emitting a compiled version that handily beats not only the NewLisp-fexpr implementation but even the NewLisp-macro implementation, as can be seen in Fig.~\ref{fig:kraken_vs_world_fib}.
We kept the benchmark sizes small in this test because the stack limits of NewLisp prevent sizes larger then ~880, while the Tail Call Elimination performed by the \krakenSpace compiler allows us to run much larger benchmarks, including the run of 4,800,000 inserts to the RB-Tree.
This result shows the dramatic effect of partial evaluation and compiler optimizations on runtime for \kraken. Our technique takes the performance of a fully fexpr based language from being completely infeasible to being faster than a macro-based dynamic scripting language currently in use.
% \begin{center}
% \begin{table}[ht]
% \caption{Number of call to the runtime's eval function for Fib. The table shows the non-partial evaluation numbers -> partial evaluation numbers}
% 	\begin{tabular}{||c | c c c c c ||} 
% 		\hline
% 		&Evals & Eval w1 Calls & Eval w0 Calls & Comp Dyn w1 Calls & Comp Dyn w0 Calls\\ [0.5ex] 
% 		\hline\hline
% 		Fib 10 & 8468 -> 0 & 2167 -> 0  & 777 -> 0 & 1 -> 0 & 0 -> 0 \\ 
% 		\hline
% 		  Fib 15  & 87916 -> 0 & 22478 -> 0 & 7961 -> 0 & 1 -> 0 & 0 -> 0 \\ 
%         \hline
% 		  Fib 20 & 969010 -> 0 & 247731 -> 0 & 87633 -> 0 & 1 -> 0 & 0 -> 0 \\ 
%     \hline
% 		  Fib 25 & 10740492 -> 0 & 2745825 -> 0  & 971209 -> 0 & 1 -> 0 & 0 -> 0 \\ 
% 		\hline
% 	\end{tabular}
%     \label{pe:fib}
%  \end{table}
% \end{center}

\begin{figure}[h]
\caption{Constant Fold and Deriv}
\includegraphics[width=0.45\textwidth]{cfold_table.csv_}
\includegraphics[width=0.45\textwidth]{deriv_table.csv_}
\label{fig:kraken_const_deriv}
\vspace{-6mm}
\end{figure}
\subsection{Comparison between Kraken Versions}
Beyond the massive speedup from partial-evaluation, Fig. \ref{fig:kraken_const_deriv} and \ref{fig:kraken_nqueens_rbtree} show the effect of the various compiler optimizations we described by disabling them one by one.
 Our main four optimizations have a strong positive effect on runtime, with the exception of lazy environment instantiation. Lazy environment instantiation helps massively on fib, and some on Deriv, but generally hurts the rest slightly.


\begin{figure}[h]
\caption{N-Queens}
\includegraphics[width=0.45\textwidth]{nqueens_table.csv_}
\includegraphics[width=0.45\textwidth]{slow_rbtree_table.csv_}
\label{fig:kraken_nqueens_rbtree}
\vspace{-4mm}
\end{figure}


\subsection{Comparison against Others}


To give a general idea of our current performance, we also show a Fibonacci benchmark that mostly exercises pure function-call speed and inlining as seen in Fig. ~\ref{fig:kraken_vs_world_fib}.
We include Python and Chez Scheme to give a general idea for where an exemplar slow and an exemplar fast dynamic language would fall.
With the benefit of our partial evaluation, compilation, and leaning upon mature WebAssembly implementations, we beat both, but this should be taken with a grain of salt, as this is a very limited micro-benchmark only meant to give a general sense of the order of magnitude of our performance.



\label{sec:eval1}
\begin{figure}[h]
\caption{Kraken vs. Others. Ordered by fastest to slowest}
\includegraphics[width=0.45\textwidth]{fib_table.csv_}
\includegraphics[width=0.45\textwidth]{rbtree_table.csv_}
\label{fig:kraken_vs_world_fib}
\end{figure}

%\label{sec:eval_nqueens}
%\begin{figure}[h]
%\caption{N-Queens}
%\includegraphics[width=0.45\textwidth]{nqueens_table.csv_}
%\includegraphics[width=0.45\textwidth]{slow_nqueens_table.csv_}
%\label{fig:kraken_nqueens}
%\end{figure}

%\label{sec:eval_nqueens}
%\begin{figure}[h]
%\caption{Kraken, N-Queens, absolute value and log-scale}
%\includegraphics[width=0.45\textwidth]{nqueens_table.csv_}
%\includegraphics[width=0.45\textwidth]{nqueens_table.csv_log}
%\label{fig:kraken_nqueens}
%\end{figure}
%\label{sec:eval_nqueensp}
%\begin{figure}[h]
%\caption{Kraken, N-Queens, absolute value and log-scale}
%\includegraphics[width=0.45\textwidth]{slow_nqueens_table.csv_}
%\includegraphics[width=0.45\textwidth]{slow_nqueens_table.csv_log}
%\label{fig:kraken_nqueensp}
%\end{figure}

%\label{sec:eval_cfold}
%\begin{figure}[h]
%\caption{C-Fold}
%\includegraphics[width=0.45\textwidth]{cfold_table.csv_}
%\includegraphics[width=0.45\textwidth]{slow_cfold_table.csv_}
%\label{fig:kraken_cfold}
%\end{figure}
%\label{sec:eval_cfold}
%\begin{figure}[h]
%\caption{Kraken, C-Fold, absolute value and log-scale}
%\includegraphics[width=0.45\textwidth]{cfold_table.csv_}
%\includegraphics[width=0.45\textwidth]{cfold_table.csv_log}
%\label{fig:kraken_cfold}
%\end{figure}
%\label{sec:eval_cfoldp}
%\begin{figure}[h]
%\caption{Kraken, C-Fold, absolute value and log-scale}
%\includegraphics[width=0.45\textwidth]{slow_cfold_table.csv_}
%\includegraphics[width=0.45\textwidth]{slow_cfold_table.csv_log}
%\label{fig:kraken_cfoldp}
%\end{figure}

%\label{sec:eval_deriv}
%\begin{figure}[h]
%\caption{Deriv}
%\includegraphics[width=0.45\textwidth]{deriv_table.csv_}
%\includegraphics[width=0.45\textwidth]{slow_deriv_table.csv_}
%\label{fig:kraken_deriv}
%\end{figure}
%\label{sec:eval_deriv}
%\begin{figure}[h]
%\caption{Kraken, Deriv, absolute value and log-scale}
%\includegraphics[width=0.45\textwidth]{deriv_table.csv_}
%\includegraphics[width=0.45\textwidth]{deriv_table.csv_log}
%\label{fig:kraken_deriv}
%\end{figure}
%\label{sec:eval_derivp}
%\begin{figure}[h]
%\caption{Kraken, Deriv, absolute value and log-scale}
%\includegraphics[width=0.45\textwidth]{slow_deriv_table.csv_}
%\includegraphics[width=0.45\textwidth]{slow_deriv_table.csv_log}
%\label{fig:kraken_derivp}
%\end{figure}

%\subsection{Comparison against state-of-the-art languages}
%\label{sec:eval3}

%\begin{figure}[h]
%\caption{Kraken vs. S.o.t.A.}
%\includegraphics[width=0.45\textwidth]{cfold_table.csv_}
%\includegraphics[width=0.45\textwidth]{rbtree_table.csv_}
%\label{fig:kraken_vs_world1}
%\end{figure}

%\begin{figure}[h]
%\caption{Kraken vs. S.o.t.A.}
%\includegraphics[width=0.45\textwidth]{deriv_table.csv_}
%\includegraphics[width=0.45\textwidth]{nqueens_table.csv_}
%\label{fig:kraken_vs_world2}
%\end{figure}

% \begin{figure}[h]
% \caption{Kraken vs. S.o.t.A. (Log)}
% \includegraphics[width=0.45\textwidth]{cfold_table.csv_log}
% \includegraphics[width=0.45\textwidth]{rbtree_table.csv_log}
% \label{fig:kraken_vs_world_log_1}
% \end{figure}
% \begin{figure}[h]
% \caption{Kraken vs. S.o.t.A. (Log)}
% \includegraphics[width=0.45\textwidth]{deriv_table.csv_log}
% \includegraphics[width=0.45\textwidth]{nqueens_table.csv_log}
% \label{fig:kraken_vs_world_log_2}
% \end{figure}

%As we noted before with the Fib(30) microbenchmark in Section \ref{sec:eval1}, we remain significantly slower than state-of-the-art compiled languages.
%This is particularly true for memory-intensive benchmarks due to our naive reference-counting and malloc/free implementations.
%However, our results are of a similar order of magnitude to the difference between the state-of-the-art compiled languages and dynamic scripting languages, like Python's results in the Fib(30) microbenchmark.
%We assert that is not a fundamental limitation because the classic f-expr slowness is being eliminated, as shown by Fig. \ref{fig:kraken_vs_newlisp1} and Fig. \ref{fig:kraken_vs_newlisp2}.
%In future work, we plan to expand our compile-time analysis and optimization to implement a modified, dynamic-language version of Perceus reference counting.
%With this change, we belive \krakenSpace can be competitive with these state-of-the-art languages.

%\subsection{Case Study: Red-Black Tree}
%\label{sec:casestudy}

%\begin{figure}[h]
%\caption{Kraken vs. S.o.t.A. - RB-Tree Focus}
%\includegraphics[width=0.4\textwidth]{rbtree_table.csv_}
%\includegraphics[width=0.4\textwidth]{rbtree_table.csv_log}
%\label{fig:kraken_vs_world_rbtree}
%\end{figure}


%To evaluate our partial evaluation algorithm and compiler, we extracted the benchmarks used by the Koka language project from their code repository and added Kraken versions, as well as implementing a naive Fibonacci microbenchmark ourselves to evaluate pure function call speed.\\
%With partial evaluation and the compiler optimizations listed above, we get fairly strong performance on purely numerical computations, such as the naive Fibonacci microbenchmark.
%Unfortunately, the overhead of our unsophisticated reference counting, dynamic type checking, and bounds checking causes poor performance on benchmarks involving data structures relative to mainstream programming language implementations.
%This is not a fundamental limitation, and will be addressed in future work, as recounted in the next section.
%It should be noted, however, that while the performance relative to established language implementations is very poor for the memory-intensive benchmarks (600-900x slower), we still realize a massive speedup compared to an unoptimized and non-partial-evaluated f-expr implementation (100,000x faster)!

\section{Related work}
\noindent \textbf{Video foundation models.}
With sufficient computational power and an abundant source of data, there have been attempts to build a single large-scale foundation model that can be adapted to diverse downstream tasks.
Along with the success of foundations models in the natural language processing domain~\cite{brown2020language,chen2021evaluating,devlin2019bert} and in computer vision~\cite{bertasius2021space,jia2021scaling,radford2021learning}, video data has become another data type of interest, as it has grown in scale due to numerous internet video-sharing platforms.
Accordingly, several methods to train a video foundation model have been proposed.
Due to the innate multi-modality of video data, \textit{i.e.}, a combination of visual $\cdot$ vocal $\cdot$ textual context, most works have centered around the variations of the cross-modal attention mechanism \cite{akbari2021vatt,bertasius2021space,gabeur2020multi,luo2020univl,neimark2021video,tan2021look,wei2020multi,yang2021taco}.
In addition, as most video data lack proper labels or descriptions, contrastive learning methods were studied to learn meaningful feature representations or enhance video-text alignment in a self-supervised manner \cite{akbari2021vatt,kuang2021video,luo2020univl,yang2021taco}.

More specifically, MERLOT \cite{zellers2021merlot} proposed a multi-modal representation learning method for visual commonsense reasoning, which also performed well in twelve video reasoning tasks.
VATT \cite{akbari2021vatt} introduced a multi-modal learning method via contrastive learning. 
The pre-trained model performed well in a variety of vision tasks from image classification to video action recognition and zero-shot video retrieval.
Another representative work, UniVL \cite{luo2020univl} proposed a straightforward pre-training method with auxiliary loss functions. 
After fine-tuning on a specific task, the pre-trained model performed outstandingly in a wide range of tasks of text-to-video retrieval, action segmentation, action step localization, video sentiment analysis, and video captioning.
Other foundation models for multiple video tasks include \cite{li2020hero,sun2019learning,sun2019videobert,zhu2020actbert,fu2021violet,wang2022all}. 

\noindent \textbf{Auxiliary learning.}
In order to enhance the performance of one or a multitude of primary tasks, auxiliary learning methods can be incorporated.
\cite{ruder2017overview} introduced Multi-task learning (MTL) to the deep neural networks by training a single model with multiple task losses to assist learning on the main task.
Such a method is generally adapted to pre-train the foundation models in the self-supervised manner~\cite{li2020hero,sun2019learning,sun2019videobert,zhu2020actbert,fu2021violet,wang2022all}.
However, these various pretext task losses used in the pre-training phase are ignored in the fine-tuning phase, and only the primary task loss is minimized.

Recently, meta-learning methods have been introduced for auxiliary learning.
\cite{liu2019self,navon2020auxiliary,shu2019meta} proposed a meta-learning method in which the model learns auxiliary tasks to generalize well to unseen data. 
In these settings, a separate subset of data is held out as the primary task, while the others are used as auxiliary tasks that aid the primary task's performance.
Similar methods were adopted for computer vision tasks such as semantic segmentation \cite{xu2021leveraging}.
Other domain applications include navigation tasks with reinforcement learning \cite{ye2021auxiliary}, or self-supervised learning methods on graph data \cite{hwang2020self}.

\section*{Acknowledgments}

This work has been carried out while Gianluca Capozzi was enrolled in the Italian National Doctorate on Artificial Intelligence run by Sapienza University of Rome.
This work has been partially supported by project SERICS (PE00000014) under the MUR National Recovery and Resilience Plan funded by the European Union - NextGenerationEU; EU-GUARDIAN project; and Sapienza Ateneo projects (RM1221816C1760BF and AR1221816C754C33). Part of the computational resources have been offered by the AWS Cloud Credit program. 


\bibliographystyle{plain}
% argument is your BibTeX string definitions and bibliography database(s)
\bibliography{bibliography.bib}


% that's all folks
\end{document}


