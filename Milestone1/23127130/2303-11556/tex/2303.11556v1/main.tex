%------------------------------------------------------------------------------
% Beginning of journal.tex
%------------------------------------------------------------------------------
%
% AMS-LaTeX version 2 sample file for journals, based on amsart.cls.
%
%        ***     DO NOT USE THIS FILE AS A STARTER.      ***
%        ***  USE THE JOURNAL-SPECIFIC *.TEMPLATE FILE.  ***
%
% Replace amsart by the documentclass for the target journal, e.g., tran-l.
%
\documentclass{amsart}

\usepackage{hyperref}

%\usepackage{showkeys}
%\usepackage{fullpage}

%     If your article includes graphics, uncomment this command.
\usepackage{graphicx}
\usepackage{accents}

%\usepackage[sort]{natbib}

\usepackage{xcolor}

\usepackage{amsmath, amssymb, amsfonts, amsthm, fullpage}
\usepackage{graphics,graphicx}
\usepackage{accents}
\usepackage{color}

\usepackage{algorithm}
\usepackage{algorithmicx}
\usepackage{algpseudocode}

\usepackage[margin=0.9in]{geometry}

\renewcommand{\div}{\mathrm{div}}
\newcommand{\curl}{\mathrm{curl}}
\newcommand{\Fb}{\mathbf{F}}
\newcommand{\R}{\mathbb{R}}
\newcommand{\hx}{\mathbf{\hat{x}}}
\newcommand{\hy}{\mathbf{\hat{y}}}
\newcommand{\hz}{\mathbf{\hat{z}}}
\newcommand{\hr}{\mathbf{\hat{r}}}
\newcommand{\hq}{\boldsymbol{\hat{\theta}}}
\newcommand{\hf}{\boldsymbol{\hat{\phi}}}
\newcommand{\into}{\int_{\Omega}}
\newcommand{\intg}{\int_{\partial\Omega}}
\newcommand{\Hoo}{\accentset{\scriptstyle o}{H}^1}
\newcommand{\Hd}[1]{H(\text{div},#1)}
\newcommand{\Hdo}[1]{H_0(\text{div},#1)}
\newcommand{\Hdoo}[1]{\accentset{\scriptstyle o}{H}(\text{div},#1)}
\newcommand{\Hoz}{H^1_0}
\newcommand{\Voo}{\accentset{\scriptstyle o}V}
%\newcommand{\Hdo}[1]{\accentset{\scriptstyle o}{H}(\text{div},{#1})}
\newcommand{\norma}[1]{{\left\vert\kern-0.25ex\left\vert\kern-0.25ex\left\vert #1
    \right\vert\kern-0.25ex\right\vert\kern-0.25ex\right\vert}}
\newcommand{\sech}{\text{sech}}

\newcommand{\bd}{\mathbf{d}}
\newcommand{\by}{\mathbf{y}}
\newcommand{\teta}{\tilde{\eta}}
\newcommand{\tu}{\tilde{u}}
\newcommand{\tw}{\tilde{w}}
\newcommand{\tv}{\tilde{v}}
\newcommand{\tf}{\tilde{f}}
\newcommand{\tg}{\tilde{g}}

\newcommand{\Alpha}{\mathrm{A}}
\newcommand{\Beta}{\mathrm{B}}

\newcommand{\be}{\mathbf{e}}
\newcommand{\bk}{\mathbf{k}}
\newcommand{\bx}{\mathbf{x}}
\newcommand{\bn}{\mathbf{n}}
\newcommand{\bu}{\mathbf{u}}
\newcommand{\bw}{\mathbf{w}}
\newcommand{\bz}{\mathbf{z}}
\newcommand{\bv}{\mathbf{v}}
\newcommand{\bF}{\mathbf{f}}
\newcommand{\bH}{\mathbf{H}}
\newcommand{\bU}{\mathbf{U}}
\newcommand{\bR}{\mathbf{R}}
\newcommand{\bS}{\mathbf{S}}
\newcommand{\bL}{\mathbf{L}}
\newcommand{\bI}{\mathbf{I}}
\newcommand{\bphi}{\boldsymbol{\phi}}
\newcommand{\bchi}{\boldsymbol{\chi}}
\newcommand{\bpsi}{\boldsymbol{\psi}}
\newcommand{\bzeta}{\boldsymbol{\zeta}}
\newcommand{\bxi}{\boldsymbol{\xi}}
\newcommand{\dx}{\text{d}\mathbf{x}}
\newcommand{\ds}{\text{d}\mathbf{s}}
\newcommand{\dt}{\Delta t}
\newcommand{\pd}[2]{\partial_{#1}{#2}}
\newcommand{\uh}{u_h}
\newcommand{\vh}{v_h}
\newcommand{\hh}{\eta^h}
\newcommand{\Ac}{\mathcal{A}}
\newcommand{\Bc}{\mathcal{B}}
\newcommand{\Cc}{\mathcal{C}}
\newcommand{\Ec}{\mathcal{E}}
\newcommand{\Nc}{\mathcal{N}}
\newcommand{\Pc}{\mathcal{P}}
\newcommand{\Zc}{\mathcal{Z}}
%\newcommand{\Th}{\mathcal{T}_h}
\newcommand{\RT}[1]{\mathcal{RT}_{#1}}
\newcommand{\BDM}[1]{\mathcal{BDM}_{#1}}
\newcommand{\DL}[1]{\mathcal{DL}_{#1}}
\newcommand{\Th}[1]{\mathcal{T}_{#1}}
\newcommand{\dOmega}{\partial\Omega}
%\newcommand{\Div}{\textrm{div\,}}
\newcommand{\Hdiv}{\text{H(div)}}
\newcommand{\Div}{\nabla\!\cdot\!}
\newcommand{\Curl}{\nabla\!\times\!}
\newcommand{\la}{\langle}
\newcommand{\ra}{\rangle}
\newcommand{\tbn}[1]{{\left\vert\kern-0.25ex\left\vert\kern-0.25ex\left\vert #1 \right\vert\kern-0.25ex\right\vert\kern-0.25ex\right\vert}}

\newcommand{\argmin}{\text{argmin}}
\newcommand{\argmax}{\text{argmax}}

\newcommand{\dd}{\,\mathrm{d}}

\newtheorem{remark}{Remark}[section]
\newtheorem{lemma}{Lemma}[section]
\newtheorem{proposition}{Proposition}[section]
\newtheorem{theorem}{Theorem}[section]
\newtheorem{corollary}{Corollary}[section]
\newtheorem{definition}{Definition}[section]

\AtBeginDocument{%
  \setlength{\belowdisplayskip}{5pt} \setlength{\belowdisplayshortskip}{5pt}
  \setlength{\abovedisplayskip}{5pt} \setlength{\abovedisplayshortskip}{5pt}
}
%\newcommand{\pd}[2]{\frac{\partial{#1}}{\partial{#2}}}
%\newcommand{\pdd}[2]{\frac{\partial^2{#1}}{\partial{#2}^2}}

\newcommand{\RED}[1]{{\color{red}{#1}}}
\newcommand{\BLUE}[1]{{\color{blue}{#1}}}


\begin{document}

\title[Equations for small amplitude shallow water waves over small bathymetric variations]{Small amplitude shallow water wave equations with small bathymetric variations}


%    Information for second author


\author{Samer Israwi}
\address{\textbf{S.~Israwi:} Department of Mathematics, Faculty of Sciences 1, Lebanese University, Beirut, Lebanon}
\email{s\_israwi83@hotmail.com}


\author{Youssef Khalifeh}
\address{\textbf{Y. Khalifeh} Laboratory of Mathematics, Faculty of Sciences 1, Lebanese University, Beirut, Lebanon}
\email{khalifeyoussef78@gmail.com}

\author{Dimitrios Mitsotakis}
%    Address of record for the research reported here
\address{\textbf{D.~Mitsotakis:} Victoria University of Wellington, School of Mathematics and Statistics, PO Box 600, Wellington 6140, New Zealand}
\email{dimitrios.mitsotakis@vuw.ac.nz}

%    Current address


%    \thanks will become a 1st page footnote.
%\thanks{D. Mitsotakis was supported by the Marsden Fund}

%    General info
\subjclass[2000]{35Q35, 76B15, 35G45}

\date{\today}

%\dedicatory{This paper is dedicated to our advisors.}

\keywords{Boussinesq systems, variable bottom topography, energy conservation}


\begin{abstract}

A generalization of the so-called $abcd$-Boussinesq class of systems is derived in the case of variable bottom topography and in two space dimensions.  Some of the new systems preserve appropriate energy functionals and some other can describe water waves in closed basins with well-justified slip-wall boundary conditions. The new systems have appropriate form so that their solutions obey to important laws of physics and mathematics. We show that the new systems are consistent with the Euler equations and we estimate their approximation error. Their derivation is based on the assumption of small bathymetric variations. Having always the applications in mind, we test the validity of some of the new systems against standard benchmarks to conclude that the derivation assumptions are not restrictive. Applications of the new systems include the study of tsunamis, tidal waves as well as waves in ports and lakes.


\end{abstract}

\maketitle



\section{Introduction}

The propagation of water waves is described mathematically by partial differential equations derived by Leonard Euler (1707-1783) in 1757 \cite{Euler1757} for the study of incompressible flow of an ideal fluid. These equations are known to as the incompressible Euler equations of water wave theory or for simplicity the Euler equations. To present the Euler equations we first denote the velocity of the fluid by $(\bu(\bx,z,t),w(\bx,z,t))=(u(\bx,z,t),v(\bx,z,t),w(\bx,z,t))$ separating the vertical component from its horizontal components, the free surface elevation above its undisturbed level by $\eta(\bx,t)$ and the depth of the ocean floor by $D(\bx,t)$. (Note that we consider time-dependent bottom topography to allow the modeling of water waves generated by the motion of the bottom topography.) Here, $\bx=(x,y)\in \mathbb{R}^2$ and $z$ are the horizontal and vertical coordinates, respectively. The horizontal gradient is thus denoted by $\nabla = (\partial_x,\partial_y)$, while $t$ is the time. The Euler equations comprise the equations of mass and momentum conservation laws 
\begin{align}
&\Div\bu+w_z=0\ , \label{eq:mass} \\
&\bu_t+(\bu\cdot\nabla)\bu+w\bu_z+\frac{1}{\rho}\nabla p=0, \quad \text{for $-D(\bx,t)<z<\eta(\bx,t)$}\ , \label{eq:momentu}\\
&w_t+(\bu\cdot\nabla) w+ww_z+\frac{1}{\rho} p_z+g=0\ . \label{eq:momentv}
\end{align}
Since our aim is to model water waves in oceans bounded above by a free surface and bellow by the ocean floor, Euler's equations require the following boundary conditions at the free surface 
\begin{equation}
\eta_t+\bu\cdot\nabla\eta-w=0,\quad \text{on $z=\eta(\bx,t)$}\ ,
\end{equation}
and at the bottom
\begin{equation}
D_t+\bu\cdot\nabla(z+D(\bx,t))+w=0\ , \quad\text{for $z=-D(\bx,t)$}\ .
\end{equation}
Another important ingredient in the theory of water waves is the irrotationality condition
\begin{equation}\label{eq:irrotational}
\Curl (\bu,w)=\begin{pmatrix} w_y-v_z\\ u_z-w_x\\ v_x-u_y \end{pmatrix}=0\ ,
\end{equation}
which is unavoidable when potential flow is assumed.
While the pressure inside the fluid volume is an unknown function, the pressure at the free surface is a given constant \begin{equation}\label{eq:pressbc}
p=p_{\text{atm}}, \quad \text{for $z=\eta(\bx,t)$}\ .
\end{equation}


The great difficulty in the study of the particular equations lies on the fact that the fluid domain $\Omega_t$ comprised the free surface of the water, the impermeable bottom (and perhaps solid wall boundaries) is practically unknown, since it is bounded by the unknown free surface $z=\eta(\bx,t)$. Moreover, the domain is not stationary, which makes the situation even more complicated. For this reason, scientists derived approximations to the Euler equations based on simplification assumptions. The derivation of such approximations started after the discovery of solitary waves by John Scott Russell (1808--1882) and his {\em report on waves} in 1844, \cite{Russell1844}, and continues until today. Examples of such approximations are the so-called Boussinesq systems that describe long waves of small amplitude. The first Boussinesq system was derived by J. Boussinesq himself \cite{Bous1871,Bous1872} and several improvements also called Boussinesq systems have been derived afterwards. It is worth mentioning that mathematically and physically justified water waves systems are still only a few. For example, Euler equations have been proved well-posed (in the Hadamard's sense) only in unbounded domains \cite{wu1997,wu1999,L2005}, while from all Boussinesq systems ever derived only the system of \cite{IKKM2021} has been proved well-posed in bounded domain with slip-wall boundary conditions.


Several Boussinesq systems have been used extensively in the literature for the description of nonlinear and dispersive waves with variable bottom topography. Such systems include the Peregrine system \cite{Pere1967} 
\begin{equation}
\label{eq:Peregrin}
\begin{aligned}
&\eta_t+\nabla\cdot[(D+\eta)\bu]=0\ , \\
&{\bf u}_t +g\nabla\eta+(\bu\cdot \nabla)\bu-\frac{1}{2}D\nabla(\nabla\cdot(D\bu_t))+\frac{1}{6}D^2\nabla(\nabla\cdot\bu_t)=0\ ,
\end{aligned}
\qquad \text{(Classical Peregrine)}
\end{equation}
and the Nwogu (or extended Boussinesq) system \cite{Nwogu93}
\begin{equation}\label{eq:Nwogu}
\begin{aligned}
& \eta_t+\Div((D+\eta)\bu)+\Div[\bar{a}D^2\nabla(\Div(D\bu))+\bar{b}D^3\nabla(\Div\bu)]=0\ ,\\
& \bu_t+g\nabla\eta+(\bu\cdot\nabla)\bu-[\bar{c}D\nabla(\Div(D\bu_t))+\bar{d}D^2\nabla(\Div\bu_t)]=0\ ,
\end{aligned}\qquad \text{(Nwogu)}
\end{equation}
with $\bar{a}=\theta-1/2$, $\bar{b}=1/2[(\theta-1)^2-1/3]$, $\bar{c}=1-\theta$, $\bar{d}=1/2(1-\theta)^2$, and $\theta=1/5$. Although in some cases their Cauchy problems have been proved well-posed, \cite{DuchIsr2018,mits2009,BLIG2022}, they do not preserve any meaningful approximation of the total energy of the Euler equations. While the particular property is not restrictive, it is rather desirable from physical as well as numerical point of view. Specifically, we can ensure physical validity for long periods of time when the solution of a system preserves an energy functional \cite{Feng2010}. Similarly, systems of BBM-BBM-type \cite{mits2009} or the systems with flat bottom topography of \cite{BCL2005} fail at describing water waves in a basin with slip-wall boundary conditions \cite{dms2009,DMS2010,DMS2007}. Some other Boussinesq systems even fail in having traveling waves of finite energy, proving that asymptotic justification does not imply physical relevance, \cite{BDM2007i,BDM2008ii}. For all these reasons in \cite{KMS2020, IKKM2021} we derived and analyzed a new Boussinesq system of BBM-BBM-type which can be used in a straightforward manner in bounded domains with slip-wall boundary conditions. Moreover, its solutions preserve the same energy functional as its non-dispersive counterpart, i.e. the shallow water-waves (Saint-Venant) equations \cite{Whitham2011} making the system attractive from numerical and physical point of view. 

In general, Boussinesq systems extend equations that describe propagation of water waves in one direction such as the KdV equation. KdV equations of various forms have been derived including variants with variable bottom topography \cite{Isrtal13,Israwii10,DurIsrawi12}. 
On the other hand such models cannot describe wave reflections on walls or obstacles. Moreover, high-order Boussinesq equations, including the Serre-Green-Naghdi equations \cite{Serre,GN1976,Lannes13} have only been proved well-posed in unbounded domains \cite{Israwi11}. Note that Serre-Green-Naghdi equations and its optimized counterpart of \cite{CDM2017} preserve an energy functional. In this article, we present the asymptotic derivation of a new class of Boussinesq systems that generalizes the so-called $abcd$-Boussinesq systems of \cite{BCL2005,BCS2002} in the case of variable bottom topography. We also establish their consistency with the Euler equations along the lines of \cite{BCL2005}. These systems generalize also the systems of \cite{Chen03} in two-dimensional domains, and their derivation is based on the same assumption of small bottom variations. Although we do not provide any well-posedness results, these new systems appear to have analogous theoretical properties with the system of \cite{BCL2005} and they can further be used with slip-wall boundary conditions. The well-posedness of the BBM-BBM system was proved in \cite{IKKM2021} while the well-posedness of other Boussinesq systems of the same family such as the Bona-Smith systems is very similar. The new system that corresponds to the Nwogu system can be applied with accurate slip-wall boundary conditions as it is demonstrated in Section \ref{sec:bcs} and is validated against laboratory data in Section \ref{sec:valid}. For the Nwogu system as well as for its regularized counterpart we have justified these set of boundary conditions one-dimensional domains in \cite{mm2023}. We conclude, after also taking into account the results of \cite{IKKM2021,KMS2020}, that the assumption of small bottom variations is not restrictive and the necessary nonlinear and dispersive behaviors of the Boussinesq systems are retained. For the sake of completeness we present a variational derivation of the new energy conservative $abcd$-Boussinesq systems. The new systems have similar energy conservation properties with the $abcd$-Boussinesq systems of \cite{BCL2005,BCS2002} and thus we generalize them in the case of variable bottom topography.

\section{Derivation of the new Boussinesq equations}\label{sec:derivation}

Consider the depth function $D(\bx,t)=D_0+D_b(\bx)+\zeta(\bx,t)$, where $D_0$ is a characteristic (mean) depth and $D_b$ is smooth and varies gently in $\mathbb{R}^2$. If $\lambda_0$ is a typical wavelength, $a_0$ a typical wave height, and $d_0$ a typical order of bottom topography variations, then we consider the (scaled) non-dimensional independent variables 
\begin{equation}\label{eq:indnondim}
\tilde{\bx}=\frac{\bx}{\lambda_0},\quad \tilde{z}=\frac{z}{D_0}, \quad \tilde{t}=\frac{c_0}{\lambda_0}t\ ,
\end{equation}
and also the non-dimensional dependent variables
\begin{equation}\label{eq:depnondim}
\tilde{\bu}=\frac{D_0}{a_0c_0}\bu, \quad \tilde{w}=\frac{\lambda_0}{a_0c_0}w,\quad \tilde{\eta}=\frac{\eta}{a_0},\quad \tilde{D}_b=\frac{D_b}{d_0}\ ,\quad \tilde{p}=\frac{p}{\rho g D_0}\ , \quad \tilde{\zeta}=\frac{\zeta}{a_0}\ ,
\end{equation}
where $c_0=\sqrt{gD_0}$ is the linear speed of propagation. Denoting as usual
$$\varepsilon = \frac{a_0}{D_0},\quad \sigma=\frac{D_0}{\lambda_0}, \quad \beta=\frac{d_0}{D_0}\ ,$$
and assuming that $0<\varepsilon\approx \sigma^2\lesssim \beta\ll 1$ all the aforementioned Boussinesq systems are asymptotically equivalent, in the sense that all of them occur from the Euler equations after appropriate scaling and discarding high-order terms with the same order in $\varepsilon$ and $\sigma^2$. In the new variable the scaled depth will be  $\tilde{D}=1+\beta \tilde{D}_b+\varepsilon\tilde{\zeta}$. Note that $\tilde{D}_{\tilde{t}}=O(\varepsilon)$ and thus we assume a slowly varying bottom. The requirement $\beta\ll 1$ is used to describe small bottom variations compared to the wavelength.

Using the new variables, we write Euler's equations (\ref{eq:mass})--(\ref{eq:momentv}) in non-dimensional form:
\begin{align}
& \tilde{\nabla} \cdot \tilde{\bu}+\tilde{w}_{\tilde{z}}=0\ , \label{eq:eul1}\\
&\varepsilon \tilde{\bu}_{\tilde{t}}+\varepsilon^2 [(\tilde{\bu}\cdot\tilde{\nabla})\tilde{\bu}+\tilde{w}\tilde{\bu}_{\tilde{z}}]+\tilde{\nabla}\tilde{p}=0\ ,  \label{eq:eul2}\\
&\varepsilon\sigma^2 \tilde{w}_{\tilde{t}}+\varepsilon^2\sigma^2 [\tilde{\bu}\cdot \tilde{\nabla} \tilde{w}+\tilde{w}\tilde{w}_{\tilde{z}}]+\tilde{p}_{\tilde{z}}=-1\ ,  \label{eq:eul3}
\end{align}
for $-\tilde{D}<\tilde{z}<\varepsilon\tilde{\eta}$. The irrotationality condition is also transformed to
\begin{align}
&\tilde{\nabla}\times \tilde{\bu}=0\ , \label{eq:eul4}\\
&\bu_z-\sigma^2\tilde{\nabla}\tilde{w}=0\ ,   \label{eq:eul5}
\end{align}
for $-\tilde{D}<\tilde{z}<\varepsilon\tilde{\eta}$. The boundary conditions on the free surface and the bottom are written as
\begin{align}
& \teta_t+\varepsilon(\tilde{\bu}\cdot\tilde{\nabla}\teta)-\tilde{w}=0,\quad \tilde{p}=\frac{p_{\text{atm}}}{\rho g D_0}\quad \text{on}\quad \tilde{z}=\varepsilon\teta\ ,  \label{eq:eul6}\\
& \tilde{\zeta}_{\tilde{t}}+ \tilde{\bu}\cdot \tilde{\nabla}\tilde{D}+\tilde{w}=0\quad \text{on}\quad\tilde{z}=-\tilde{D}\ .  \label{eq:eul7}
\end{align}

First, integrate the mass equation (\ref{eq:eul1}) between $-\tilde{D}$ and $\varepsilon\tilde{\eta}$ to obtain
\begin{equation}
\tilde{w}(\varepsilon\teta)-\tilde{w}(-\tilde{D})=-\int_{-\tilde{D}}^{\varepsilon\teta}\tilde{\nabla}\cdot\tilde{\bu}~d\tilde{z}\ .
\end{equation}
Using the boundary conditions (\ref{eq:eul6}) and (\ref{eq:eul7}) we obtain the equation
\begin{equation}\label{eq:massper}
\teta_{\tilde{t}}+\tilde{\nabla}\cdot[(\tilde{D}+\varepsilon\tilde{\eta})\tilde{\bu}_a]+\tilde{\zeta}_t=0\ ,
\end{equation}
where 
\begin{equation}\label{eq:dpav}
\tilde{\bu}_a(\tilde{\bx},\tilde{t})=\frac{1}{\tilde{D}+\varepsilon\teta}\int_{-\tilde{D}}^{\varepsilon\teta}\tilde{\bu}~d\tilde{z}\ ,
\end{equation}
denotes the depth-averaged horizontal velocity of the fluid. Equation (\ref{eq:massper}) is exact and it is the mass equation of the hydrostatic shallow water wave equations and of Peregrine's original system.

Integrating (\ref{eq:eul1}) from $-\tilde{D}$ to $\tilde{z}$, and using (\ref{eq:eul7}) we have
\begin{equation}\label{eq:eul8}
\tilde{w}=-\tilde{\bu}\cdot \tilde{\nabla}\tilde{D}-\int_{-\tilde{D}}^{\tilde{z}}\tilde{\nabla}\cdot\tilde{\bu}-\tilde{\zeta}_{\tilde{t}}\ .
\end{equation}
After integration of (\ref{eq:eul5}) and using (\ref{eq:eul8}) we have
\begin{equation}\label{eq:eul9}
\tilde{\bu}(\tilde{\bx},\tilde{z},\tilde{t})=\tilde{\bu}_b(\tilde{\bx},\tilde{t})+O(\sigma^2)\ ,
\end{equation}
where $\tilde{\bu}_b(\bx,t)\doteq \tilde{\bu}(\bx,-\tilde{D},\tilde{t})$ denotes the horizontal velocity at the bottom. Substitution of (\ref{eq:eul8}) into the irrotationality condition (\ref{eq:eul5}) and using (\ref{eq:eul9}) yields
\begin{equation}\label{eq:eul10}
\tilde{\bu}_{\tilde{z}}=-\sigma^2(\tilde{z}+1) \tilde{\nabla}(\tilde{\nabla}\cdot \tilde{\bu}_b)-\sigma^2\tilde{\nabla}\tilde{\zeta}_{\tilde{t}}+O(\sigma^4,\varepsilon\sigma^2,\beta\sigma^2)\ .
\end{equation}
Integration of (\ref{eq:eul10}) from $-\tilde{D}$ to $\tilde{z}$ implies
\begin{equation}\label{eq:eul11}
\tilde{\bu}=\tilde{\bu}_b-\sigma^2(\tilde{z}+\frac{\tilde{z}^2}{2})\tilde{\nabla}(\tilde{\nabla}\cdot \tilde{\bu}_b)-\sigma^2(\tilde{z}+1)\nabla\tilde{\zeta}_{\tilde{t}}+O(\sigma^4,\varepsilon\sigma^2,\beta\sigma^2)\ .
\end{equation}
Equation (\ref{eq:eul8}) using (\ref{eq:eul9}) becomes
\begin{equation}\label{eq:eul12}
\tilde{w}=-\tilde{\nabla}\cdot(\tilde{D}\tilde{\bu}_b)-\tilde{z}\tilde{\nabla}\cdot \tilde{\bu}_b-\tilde{\zeta}_{\tilde{t}}+O(\sigma^2)\ ,
\end{equation}
and differentiation with respect to $t$ yields
\begin{equation}\label{eq:eul13}
\tilde{w}_{\tilde{t}}=-\tilde{\nabla}\cdot(\tilde{D}\tilde{\bu}_b)_{\tilde{t}}-\tilde{z}\tilde{\nabla}\cdot {\tilde{\bu}_b}_{\tilde{t}}-\tilde{\zeta}_{\tilde{t}\tilde{t}}+O(\sigma^2)\ .
\end{equation}
Setting $\tilde{P}=\tilde{p}-p_{\text{atm}}/\rho g D_0$ (so as $\tilde{\nabla} \tilde{P}=\tilde{\nabla} \tilde{p}$ and $\tilde{P}(\varepsilon\teta)=0$) and integrating (\ref{eq:eul3}) from $\tilde{z}$ to $\varepsilon\tilde{\eta}$ we obtain
\begin{equation}\label{eq:eul14}
\tilde{P}=\varepsilon\sigma^2(\tilde{z}+\frac{\tilde{z}^2}{2})\tilde{\nabla}\cdot{\tilde{\bu}_b}_{\tilde{t}}+\varepsilon\sigma^2\tilde{z}\tilde{\zeta}_{\tilde{t}\tilde{t}}+\varepsilon\tilde{\eta}-\tilde{z}+O(\varepsilon\sigma^4,\varepsilon^2\sigma^2,\varepsilon\beta\sigma^2)\ .
\end{equation}
Substitution of (\ref{eq:eul11}), (\ref{eq:eul12}) and (\ref{eq:eul14}) into (\ref{eq:eul2}) leads to the approximation of momentum conservation
\begin{equation}\label{eq:eul15}
{\tilde{\bu}_b}_{\tilde{t}}+\tilde{\nabla}\tilde{\eta}+\varepsilon(\tilde{\bu}_b\cdot\tilde{\nabla})\tilde{\bu}_b+\sigma^2\tilde{\nabla}\tilde{\zeta}_{\tilde{t}\tilde{t}}=O(\sigma^4,\varepsilon\sigma^2,\beta\sigma^2)\ .
\end{equation}
Equation (\ref{eq:eul11}) using (\ref{eq:dpav}) becomes
\begin{equation}\label{eq:eul17}
\tilde{\bu}_b=\tilde{\bu}_a-\frac{\sigma^2}{3}\tilde{\nabla}(\tilde{\nabla}\cdot {\tilde{\bu}_a})+\frac{\sigma^2}{2}\tilde{\nabla}\tilde{\zeta}_{\tilde{t}}+O(\sigma^4,\varepsilon\sigma^2,\beta\sigma^2)\ .
\end{equation}
Subsequently, equation (\ref{eq:eul15}) yields the momentum equation
\begin{equation}\label{eq:eul18}
{\tilde{\bu}_a}_{\tilde{t}}+\tilde{\nabla}\tilde{\eta}+\varepsilon({\tilde{\bu}_a}\cdot\tilde{\nabla}){\tilde{\bu}_a}-\frac{\sigma^2}{3}\tilde{\nabla}(\tilde{\nabla}\cdot{\tilde{\bu}_a}_{\tilde{t}})-\frac{\sigma^2}{2}\tilde{\nabla}\tilde{\zeta}_{\tilde{t}\tilde{t}}=O(\sigma^4,\varepsilon\sigma^2,\beta\sigma^2)\ .
\end{equation}
Since $\tilde{\bu}=\tilde{\bu}_a+O(\sigma^2)$, it is implied that $\tilde{\nabla}\times \tilde{\bu}_a=O(\sigma^2)$, which yields that $({\tilde{\bu}_a}\cdot\tilde{\nabla}){\tilde{\bu}_a}=\frac{1}{2}\tilde{\nabla}|\tilde{\bu}_a|^2+O(\sigma^2)$. Therefore, we can further simplify (\ref{eq:eul18}) into 
\begin{equation}\label{eq:eul19}
{\tilde{\bu}_a}_{\tilde{t}}+\tilde{\nabla}\tilde{\eta}+\frac{\varepsilon}{2}\tilde{\nabla}|\tilde{\bu}_a|^2-\frac{\sigma^2}{3}\tilde{\nabla}(\tilde{\nabla}\cdot{\tilde{\bu}_a}_{\tilde{t}})-\frac{\sigma^2}{2}\tilde{\nabla}\tilde{\zeta}_{\tilde{t}\tilde{t}}=O(\sigma^4,\varepsilon\sigma^2,\beta\sigma^2)\ .
\end{equation}
Note that equations (\ref{eq:massper})-(\ref{eq:eul19}) have been also studied in \cite{KMS2020}. Evaluating the horizontal velocity (\ref{eq:eul11}) at depth $\tilde{z}_\theta=-\tilde{D}+\theta(\varepsilon\tilde{\eta}+D)$ where $\sigma^2\tilde{z}_\theta=(\theta-1)\sigma^2+O(\varepsilon\sigma^2,\beta\sigma^2)$, and using (\ref{eq:eul17}) we obtain
\begin{equation}\label{eq:eul20}
\tilde{\bu}_\theta=\tilde{\bu}_a-\frac{\sigma^2}{2}\left(\theta^2-\frac{1}{3}\right)\tilde{\nabla}(\tilde{\nabla}\cdot\tilde{\bu}_a)-\sigma^2\left(\theta-\frac{1}{2}\right)\tilde{\nabla}\tilde{\zeta}_{\tilde{t}}+O(\sigma^4,\varepsilon\sigma^2,\beta\sigma^2)\ ,
\end{equation}
and equivalently
\begin{equation}\label{eq:eul21}
\tilde{\bu}_a=\tilde{\bu}_\theta+\frac{\sigma^2}{2}\left(\theta^2-\frac{1}{3}\right)\tilde{\nabla}(\tilde{\nabla}\cdot\tilde{\bu}_\theta)+\sigma^2\left(\theta-\frac{1}{2}\right)\tilde{\nabla}\tilde{\zeta}_{\tilde{t}}+O(\sigma^4,\varepsilon\sigma^2,\beta\sigma^2)\ .
\end{equation}
Substituting (\ref{eq:eul21}) into (\ref{eq:massper}) and (\ref{eq:eul19}) we obtain the system
\begin{align}
&\teta_{\tilde{t}}+\tilde{\nabla}\cdot[(\tilde{D}+\varepsilon\tilde{\eta})\tilde{\bu}_\theta]-\frac{\sigma^2}{2}\left(\theta^2-\frac{1}{3}\right)\tilde{\nabla}\cdot\tilde{\nabla}(\tilde{\nabla}\cdot\tilde{\bu}_\theta)+\sigma^2\left(\theta-\frac{1}{2}\right)\tilde{\nabla}\cdot\tilde{\nabla}\tilde{\zeta}_{\tilde{t}}+\tilde{\zeta}_t=O(\sigma^4,\varepsilon\sigma^2,\beta\sigma^2)\ , \label{eq:eul22}\\
&{\tilde{\bu}_\theta}_{\tilde{t}}+\tilde{\nabla}\tilde{\eta}+\frac{\varepsilon}{2}\tilde{\nabla}|\tilde{\bu}_\theta|^2-\frac{\sigma^2}{2}(\theta^2-1)\tilde{\nabla}(\tilde{\nabla}\cdot{\tilde{\bu}_\theta}_{\tilde{t}})+\sigma^2(\theta-1)\tilde{\nabla}\tilde{\zeta}_{\tilde{t}\tilde{t}}=O(\sigma^4,\varepsilon\sigma^2,\beta\sigma^2)\ . \label{eq:eul23}
\end{align}
Observe that from (\ref{eq:eul22}) and (\ref{eq:eul23}) we obtain
\begin{equation}\label{eq:eul24}
\tilde{\nabla}\cdot \tilde{\bu}_\theta=-\teta_{\tilde{t}}-\tilde{\zeta}_{\tilde{t}}+O(\varepsilon,\beta,\sigma^2)\ ,
\end{equation}
and
\begin{equation}\label{eq:eul25}
{\tilde{\bu}_{\theta}}_{\tilde{t}}=-\tilde{\nabla}\tilde{\eta}+O(\varepsilon,\sigma^2)\ . 
\end{equation}
Using the classical BBM-trick \cite{BBM1972,Pere1966} with (\ref{eq:eul24}), (\ref{eq:eul25}) and taking arbitrary $\nu,\mu\in \mathbb{R}$ we write 
\begin{equation}\label{eq:rel1}
\tilde{\nabla}\cdot\tilde{\nabla}(\tilde{\nabla}\cdot\tilde{\bu}_\theta)=-\tilde{\nabla}\cdot\tilde{\nabla}(\teta_{\tilde{t}})-\tilde{\nabla}\cdot\tilde{\nabla}({\tilde{\zeta}}_{\tilde{t}})+O(\varepsilon,\beta,\sigma^2)\ ,
\end{equation} 
and 
\begin{equation}\label{eq:rel2}
\tilde{\nabla}(\tilde{\nabla}\cdot{\tilde{\bu}_\theta}_{\tilde{t}})=-\tilde{\nabla}(\tilde{\nabla}\cdot \tilde{\nabla}\tilde{\eta})+O(\varepsilon,\sigma^2)\ .
\end{equation} 
Substituting relations (\ref{eq:rel1})--(\ref{eq:rel2}) into the system (\ref{eq:eul22})--(\ref{eq:eul23}) we obtain the general $abcd$-Boussinesq system
\begin{align}
& \teta_{\tilde{t}}+\tilde{\nabla}\cdot[(\tilde{D}+\varepsilon\tilde{\eta})\tilde{\bu}_\theta]+\sigma^2 \tilde{\nabla}\cdot[a\tilde{\nabla}(\tilde{\nabla}\cdot\tilde{\bu}_\theta)-b\tilde{\nabla}\teta_{\tilde{t}}]+\tilde{a}\sigma^2 \tilde{\nabla}\cdot\tilde{\nabla}\tilde{\zeta}_{\tilde{t}}+\tilde{\zeta}_{\tilde{t}}=O(\sigma^4,\varepsilon\sigma^2,\beta\sigma^2)\ , \label{eq:eul26}\\
& {\tilde{\bu}_\theta}_{\tilde{t}}+\tilde{\nabla}\tilde{\eta}+\frac{\varepsilon}{2}\tilde{\nabla}|\tilde{\bu}_\theta|^2+\sigma^2 \tilde{\nabla}[ c \tilde{\nabla}\cdot \tilde{\nabla}\teta-d\tilde{\nabla}\cdot{\tilde{\bu}_\theta}_{\tilde{t}}]+\sigma^2\tilde{c}\tilde{\nabla}\tilde{\zeta}_{\tilde{t}\tilde{t}}=O(\sigma^4,\varepsilon\sigma^2,\beta\sigma^2)\ .\label{eq:eul27}
\end{align}
where $\tilde{a}=\mu(\theta-1/2)-(1-\mu)(1/2[(\theta-1)^2-1/3])$, $\tilde{c}=(\theta-1)$ and $a,b,c,d$ as in (\ref{eq:abcdcoef2}).

Note that 
\begin{align}
& \tilde{\nabla}(\tilde{\nabla}\cdot\tilde{\bu}_\theta)=\tilde{\nabla}(\tilde{D}^3\tilde{\nabla}\cdot\tilde{\bu}_\theta)+O(\varepsilon,\beta)\ ,\\
&\tilde{\nabla}\teta_{\tilde{t}}=\tilde{D}^2\tilde{\nabla}\teta_{\tilde{t}}+O(\varepsilon,\beta)\ , \\
&\tilde{\nabla}\cdot \tilde{\nabla}\teta=\tilde{\nabla}\cdot (\tilde{D}^2 \tilde{\nabla}\teta)+O(\varepsilon,\beta)\ ,\\
&\tilde{\nabla}\cdot{\tilde{\bu}_\theta}_{\tilde{t}}=\tilde{\nabla}\cdot (\tilde{D}^2{\tilde{\bu}_\theta}_{\tilde{t}})+O(\varepsilon,\beta)\ ,\\
&\tilde{\nabla}\cdot\tilde{\nabla}\tilde{\zeta}_{\tilde{t}}=\tilde{\nabla}\cdot(\tilde{D}^2\tilde{\nabla}\tilde{\zeta}_{\tilde{t}})+O(\varepsilon,\beta)\ ,\\
&\tilde{\nabla}\tilde{\zeta}_{\tilde{t}\tilde{t}}=\tilde{D}\tilde{\nabla}\tilde{\zeta}_{\tilde{t}\tilde{t}}+O(\varepsilon,\beta)\ .
\end{align}
Using the previous approximations we write the general $abcd$-Boussinesq system as
\begin{align}
& \teta_{\tilde{t}}+\tilde{\nabla}\cdot[(\tilde{D}+\varepsilon\tilde{\eta})\tilde{\bu}_\theta]+\sigma^2 \tilde{\nabla}\cdot[a\tilde{\nabla}(\tilde{D}^3\tilde{\nabla}\cdot\tilde{\bu}_\theta)-b\tilde{D}^2\tilde{\nabla}\teta_{\tilde{t}}]+\tilde{a}\sigma^2 \tilde{\nabla}\cdot(\tilde{D}^2\tilde{\nabla}\tilde{\zeta}_{\tilde{t}})+\tilde{\zeta}_{\tilde{t}}=O(\sigma^4,\varepsilon\sigma^2,\beta\sigma^2)\ , \label{eq:eul28}\\
& {\tilde{\bu}_\theta}_{\tilde{t}}+\tilde{\nabla}\tilde{\eta}+\frac{\varepsilon}{2}\tilde{\nabla}|\tilde{\bu}_\theta|^2+\sigma^2 \tilde{\nabla} [ c \tilde{\nabla}\cdot (\tilde{D}^2 \tilde{\nabla}\teta)-d\tilde{\nabla}\cdot (\tilde{D}^2{\tilde{\bu}_\theta}_{\tilde{t}})]+\sigma^2\tilde{c}\tilde{D}\tilde{\nabla}\tilde{\zeta}_{\tilde{t}\tilde{t}}=O(\sigma^4,\varepsilon\sigma^2,\beta\sigma^2)\ .\label{eq:eul29}
\end{align}
The reason for considering such approximations is to ensure that  certain combinations of the parameters $a,b,c,d$ lead to systems of significant interest of \cite{BCS2002} that  preserve meaningful approximations of the total energy even with variable bottom topography. 

Discarding the high-order terms in (\ref{eq:eul28})--(\ref{eq:eul29}) we write the equations (\ref{eq:eul28})--(\ref{eq:eul29}) in dimensional variables as 
\begin{equation}\label{eq:Nwogunabcdd2}
\begin{aligned}
& \eta_t+\Div[(D+\eta)\bu]+\Div\left\{a\nabla(D^3\Div\bu)-bD^2\nabla\eta_t\right\}=-\tilde{a}\Div(D^2\nabla\zeta_t)-\zeta_t\ ,\\
& \bu_t+g\nabla\eta+\tfrac{1}{2}\nabla|\bu|^2 +\nabla\left\{cg\Div(D^2\nabla\eta)-d\Div(D^2\bu_t)\right\}=-\tilde{c}D\nabla\zeta_{tt}\ ,
\end{aligned}
\end{equation}
where
\begin{equation}\label{eq:abcdcoef2}
\begin{aligned}
&a=\frac{1}{2}\left(\theta^2-\frac{1}{3}\right)\mu,\quad b=\frac{1}{2}\left(\theta^2-\frac{1}{3}\right)(1-\mu), c=\frac{1}{2}(1-\theta^2)\nu, \quad d= \frac{1}{2}(1-\theta^2)(1-\nu)\ , \\
&\tilde{a}=\mu\left(\theta-\frac{1}{2}\right)-(1-\mu)\left(\frac{1}{2}\left[(\theta-1)^2-\frac{1}{3}\right]\right), \quad \tilde{c}=(\theta-1)\ .
\end{aligned}
\end{equation}
Note that the system (\ref{eq:Nwogunabcdd2}) has been derived in such a way that the $\Curl\bu$ is preserved for all times $t\geq 0$ for stationary bottom topography, and thus the irrotationality of the water waves is respected exactly. Moreover, we observe that in most cases the presence of the regularization operators $I-b\Div(D^2\nabla\bullet)$ and $I-d\nabla\Div(D^2\bullet)$ requires for well-posedness that $b,d\geq 0$ \cite{DM2008}. As we shall see later, conservation of energy requires additionally $b=d\geq 0$.

Water waves are dispersive by their nature. For modeling purposes we employ the dispersion relation of the linearized equations and with flat bottom topography as a measure of accuracy. Specifically, to study the dispersion characteristics of the system  (\ref{eq:Nwogunabcdd2}) we first consider the expansion of the linear dispersion relation for general periodic solutions of the Euler equations and of the form $e^{i(kx-\omega t)}$, which is
\begin{equation}\label{eq:eulerdisprel}
\frac{c^2_{\text{Euler}}}{gD}=\frac{\tanh(Dk)}{Dk}=1-\frac{1}{3}(Dk)^2+\frac{2}{15}(Dk)^4+O\left((Dk)\right)^6\ ,
\end{equation}
where $k$ is the wavenumber and $\omega$ the frequency of the wave. Recall that the period of a linear wave is defined as $T=2\pi/\omega$, the phase speed is $\omega/k$, the linear  speed of propagation is $\sqrt{gD}$ and the wavelength is $\lambda=2\pi/k$. The system (\ref{eq:Nwogunabcdd2}) with flat bottom $D$ has linear dispersion relation given by the formula
 \begin{equation}\label{eq:abcdisperel}
\begin{aligned}
\frac{c^2_{abcd}}{gD}&=\frac{(1-a(Dk)^2)(1-c(Dk)^2)}{(1+b(Dk)^2)(1+d(Dk)^2)}\\
&=1-(a+b+c+d)(Dk)^2+\left((a + b) (b + c) + (a + b + c) d + d^2\right)(Dk)^4+O\left((Dk)\right)^6\ .
\end{aligned}
\end{equation}
If we choose the coefficients $a,b,c,d$ of (\ref{eq:abcdcoef2}) such that the formula (\ref{eq:abcdisperel}) coincides with (\ref{eq:eulerdisprel}) up to the term of order $(Dk)^4$, then we obtain the optimized extended Boussinesq system for $\nu=0$, $\mu=1$ and $\theta^2=1/5$. This is the system which we call Nwogu system and is practically the only system of the $abcd$ class of systems with such an optimal dispersive behavior. Notable systems of the $abcd$ class that have appeared in various works \cite{BC1998,BCS2002,BS1976,DM2008} are the following:
\begin{itemize}
\item The Peregrine system ($a=b=c=0$, $d=1/3$, i.e, $\nu=\mu=0$, $\theta^2=1/3$)
\item The BBM-BBM system ($a=c=0$, $b=d=1/6$, i.e. $\nu=\mu=0$, $\theta^2=2/3$)
\item The Bona-Smith systems ($a=0$, $b=d=(3\theta^2-1)/6$, $c=(2-3\theta^2)/3$, $2/3<\theta^2\leq1$, i.e. $\nu=0$, $\mu=(4-6\theta^2)/3(1-\theta^2)$)
\item The KdV-KdV system ($a=c=1/6$, $b=d=0$, i.e. $\nu=\mu=1$, $\theta^2=2/3$)
\item The Nwogu system ($a=-1/15$, $b=c=0$, $d=2/5$ i.e. $\nu=0$, $\mu=1$, $\theta^2=1/5$)
\end{itemize}

\section{Consistency with the Euler equations}\label{consistwithEulerr}

In this section we consider the Boussinesq system (\ref{eq:eul28})--(\ref{eq:eul29}) with stationary bottom topography and parameters $a,b,c,d$ as in (\ref{eq:abcdcoef2})
\begin{equation}\label{eq:Nwogunabcd}
\begin{aligned}
& \eta_{t}+\nabla\cdot[(D+\varepsilon\eta)\bu]+\sigma^2 \Div\left\{a\nabla(D^3\Div\bu)-bD^2\nabla\eta_t\right\}=0\ , \\
& {\bu}_{t}+\nabla\eta+\frac{\varepsilon}{2}\nabla|\bu|^2+\sigma^2 \nabla\left\{c\Div(D^2\nabla\eta)-d\Div(D^2\bu_t)\right\}=0\ .
\end{aligned}
\end{equation}
Note that in the notation of this section we have dropped the tilde in nondimensional variables.

Consider also the Euler equations in the nondimensional and scaled Zakharov formulation:
 \begin{equation}\label{eq:S0}
\begin{aligned}
&\eta_t = \cfrac{1}{\varepsilon}\,Z_{\varepsilon}[\varepsilon\eta,\beta b]\psi\ , \\
& \psi_t + \eta + \cfrac{\varepsilon}{2} |\nabla\psi|^{2} - \cfrac{\varepsilon}{\sigma^2}\cfrac{(Z_{\varepsilon}[\varepsilon\eta,\beta b]\psi + \varepsilon\sigma^2\nabla\eta\cdot\nabla\psi)^2}{2(1 + \varepsilon^{2}\sigma^2|\nabla\eta|^2)} = 0 \ ,
\end{aligned}
\end{equation}
where $\psi$ is the velocity potential and $Z_{\varepsilon}[\varepsilon\eta,\beta b]\psi = \sqrt{1+\varepsilon^2|\sigma^2\nabla\eta|^2}\partial_n\Phi_{|z=\varepsilon\eta}$ is the Dirichlet-Neumann operator. Following  \cite{Lannes13} (Section 5.3) one can derive the following four-parameter family of Boussinesq systems of order $O(\varepsilon\sigma^2,\beta\sigma^2)$
\begin{equation}\label{eq:S1}
\begin{aligned}
&\eta_t + \nabla\cdot[(D+\varepsilon\eta)\bu] + \sigma^2\Div [\mathbf{a} D^{2}\nabla(\Div D\bu) + \mathbf{b} D^{3}\nabla(\Div \bu)] = 0\ , \\
& \bu_t + \nabla\eta + \varepsilon(\bu\cdot\nabla)\bu +\sigma^2\nabla[\mathbf{c} D\Div(D\bu_t) + \mathbf{d} D^{2}\Div\bu_t] = 0 \ ,
\end{aligned}
\end{equation}
where $\bu$ is the horizontal velocity at the level $(D + \varepsilon\eta)z+\varepsilon\eta$ and $z=-1+\frac{1}{\sqrt{3}}\sqrt{1+2(\delta -\theta)}$,\\
    with $$ \bold{a} = \frac{\delta + \lambda}{3},\;\;  \bold{b} = -\frac{\theta + \lambda}{3},\;\; \bold{c} = -\frac{\alpha + \delta - 1}{3},\;\; \bold{d} = \frac{\alpha + \theta}{3}\ .$$
    
Here, we will prove consistency results between the asymptotic model (\ref{eq:Nwogunabcd}) and the Euler equations in case of small amplitude topographic variations $\beta = O(\sigma^2)$. However, note that from \cite{Lannes13} the Zakharov system (\ref{eq:S0}) and the Boussinesq system (\ref{eq:S1}) are consistent (see Remark 5.35 in \cite{Lannes13}). In particular, for $\delta = \frac{3}{2} \theta^{2} + \theta -\frac{1}{2} $, $\lambda = -\frac{3}{2} \theta^{2} +2\theta -1$ and $\alpha = -\frac{3}{2}\theta^{2}+2\theta -\frac{3}{2}$, we get Nwogu’s system 
 \begin{equation}\label{eq:Nwogu1}
\begin{aligned}
& \eta_t+\Div((D+\varepsilon\eta)\bu)+\sigma^2\Div[\bar{a}D^2\nabla(\Div(D\bu))+\bar{b}D^3\nabla(\Div\bu)]=0\ ,\\
& \bu_t+\nabla\eta+\varepsilon(\bu\cdot\nabla)\bu-\sigma^2[\bar{c}D\nabla(\Div(D\bu_t))+\bar{d}D^2\nabla(\Div\bu_t)]=0\ ,
\end{aligned}
\end{equation}
where $\bar{a}=\theta-1/2$, $\bar{b}=1/2[(\theta-1)^2-1/3]$, $\bar{c}=1-\theta$ and $\bar{d}=1/2(1-\theta)^2$.  Next we will show in detail the consistency of the new system (\ref{eq:Nwogunabcd}) with Nwogu system (\ref{eq:Nwogu1}). But we shall first introduce the definition of consistency along the lines of \cite{BCL2005}. 
%For the sake of simplicity we assume that $\varepsilon=\sigma^2$, i.e. the Stokes number $S$ is exactly 1.
\begin{definition}\label{defcons}
Let $p, s \in \mathbb{R}$, $\varepsilon_0 > 0$, $T>0$ and let ${(\bu^{\varepsilon},
\eta^{\varepsilon})_{0<\varepsilon<\varepsilon_0}}$ be a family of solutions of a Boussinesq system $(S_1)$ bounded in
$W^{1,\infty}([0,\frac{T}{\varepsilon}];\,H^{p}(\mathbb{R}^2)^{3})$ independently of $\varepsilon$.
This family is called consistent (with regularity $p$ and $s$) with a system $(S_2)$ if
it  satisfies the system $(S_2)$ with a residual of order $\varepsilon^2$ in $L^{\infty}([0,\frac{T}{\varepsilon}];\,H^{s}(\mathbb{R}^2)^{3})$.
\end{definition}

First we show consistency between the classical Nwogu system (\ref{eq:Nwogu}) and the new Nwogu-type system
\begin{equation}\label{eq:Nwogu2}
\begin{aligned}
& \eta_t+\Div((D+\varepsilon\eta)\bu)+\sigma^2a\Div[D^3\nabla(\Div\bu)]=0\ ,\\
& \bu_t+\nabla\eta+\tfrac{\varepsilon}{2}\nabla|\bu|^2 -\sigma^2b\nabla[\Div(D^2\bu_t)]=0\ ,
\end{aligned}
\end{equation}
where $b=\frac{1}{3}-a=\frac{1}{2}(1-\theta^2)>0$, $a<0$. This system is the special case of system (\ref{eq:Nwogunabcd}) for $\nu=0$, $\mu=1$, $0<\theta^2<1/3$.


\begin{lemma}\label{thrm:curl}
 Let $t_0>1$, $s\geq t_0+3$,
$U_0=(\eta_0,\bu_0)\in H^{s}(\mathbb{R}^2)\times H^{s+2}(\mathbb{R}^2)^2$ with $\vert \Curl \bu_0\vert_{H^{s+1}}\leq \sigma^2 C(\vert U_0\vert_{H^s\times H^{s+2}})$. Then,
 the solution $U=(\eta, \bu)\in C([0,\frac{T}{\varepsilon}];\,H^{s}(\mathbb{R}^2)\times H^{s+2}(\mathbb{R}^2)^2)$
of the new Nwogu system of Boussinesq type (\ref{eq:Nwogu2}) with the initial condition $(\eta_0,\bu_0)$ satisfies
$$
\vert \Curl\bu\vert_{L^{\infty}([0,\frac{T}{\varepsilon}],H^{s+1})}\leq\sigma^2\, C(T,\vert U_0\vert_{H^s\times H^{s+2}})\quad \forall\,0<T<T_{max}\ .
$$
\end{lemma}
\begin{proof}
The proof of existence is demonstrated in (\cite[Theorem 1.1]{saut2012cauchy}). Applying the operator $ \hbox{curl}\,( \cdot)$ to the second equation of the system (\ref{eq:Nwogu2}) we see that
$$
\Curl\bu_t= 0\ ,
$$
which implies $\Curl\bu=\Curl \bu_0$, and the result follows.
\end{proof}

\begin{lemma}\label{newformulate}
Let  $\bu$, $\eta$ be solutions either of (\ref{eq:Nwogu1}) or (\ref{eq:Nwogu2}) and $D=1+\beta D_b$ be a bottom parametrization, then
\begin{equation}\label{definewv}
\begin{aligned}
&\Div[D^3\nabla(\Div\bu)]=\Div[D^2\nabla(\Div(D\bu))]+O(\beta)\ ,\\
&\Div(D\bu)=-\eta_t+O(\varepsilon,\sigma^2)\ ,\\
&\bu_t=-\nabla\eta+O(\varepsilon,\sigma^2)\ .
\end{aligned}
\end{equation}
\end{lemma}
\begin{proof}
The first identity follows from the fact that $D(\bx)=1+\beta D_b(\bx)$. The other two follow from the system (\ref{eq:Nwogu2}).
\end{proof}

Here, we present the main result of this section. For simplicity, we take $\beta = \varepsilon = O(\sigma^2)$ for $\sigma^2\ll 1$.
\begin{proposition}\label{propcons}
Let $s$ large enough, $T>0$,  and $(\eta^\varepsilon,\bu^\varepsilon)_{0<\varepsilon<\varepsilon_0}$ be a family of solutions of Nwogu system (\ref{eq:Nwogu1}) bounded in $W^{1,\infty}\big([0,\frac{T}{\varepsilon}]; H^s(\mathbb{R}^2)^3\big)$. If $D(\bx)=1+\beta D_b(\bx)$, then according to the Definition \ref{defcons} this family is consistent with (\ref{eq:Nwogunabcd}).
\end{proposition}
\begin{proof} 
For the sake of simplicity we drop the index $\varepsilon$ from the notation of  $(\eta^\varepsilon,\bu^\varepsilon)$ but we will still assume solutions of (\ref{eq:Nwogu1}). From the first equation of (\ref{eq:Nwogu1}), it suffices to prove that  $$\Div[\bar{a}D^2\nabla(\Div(D\bu)) + \bar{b}D^3\nabla(\Div\bu)] = \Div[a\nabla(D^3\Div\bu) - bD^2\nabla\eta_t]+O(\beta)  $$ 
where $a = A\mu$, $b = A(1-\mu)$, $\mu\in\mathbb{R}$ and $A=\bar{a}+\bar{b}$.

Using Lemma \ref{newformulate} we have 
\begin{align*}
  \Div[\bar{a}D^2\nabla(\Div(D\bu))+\bar{b}D^3\nabla(\Div\bu)] &= \bar{a}\Div[D^3\nabla(\Div\bu)]+\bar{b}\Div[D^3\nabla(\Div\bu)] +O(\beta) \\ 
  &= A(\mu + 1 - \mu)\Div[D^3\nabla(\Div\bu)] +O(\beta) \\
  &= a\Div[D^3\nabla(\Div\bu)] + b\Div[D^3\nabla(\Div\bu)] +O(\beta) \\
   &= a\Div[\nabla(D^3\Div\bu)] + b\Div[D^2\nabla(\Div(D\bu))] +O(\beta) \\
    &= a\Div[\nabla(D^3\Div\bu)] - b\Div[D^2\nabla\eta_t] +O(\beta)\ ,
    \end{align*}
    where $O(\beta) = \beta f$ and $f \in L^{\infty}([0,\frac{T}{\varepsilon}];\,H^{s-5}(\mathbb{R}^2))$.
    Therefore, $$\Div[\bar{a}D^2\nabla(\Div(D\bu))+\bar{b}D^3\nabla(\Div\bu)] = \Div[a\nabla(D^3\Div\bu) - bD^2\nabla\eta_t] \ , $$
  in the sense of consistency.
 Thus, $(\eta^\varepsilon,\bu^\varepsilon)$ is consistent with the first equation in (\ref{eq:Nwogunabcd}) with a residual of order $\beta^2$ in $L^{\infty} ([0,\frac{T}{\varepsilon}];\, H^{s-5}(\mathbb{R}^2))$.
        
  The same situation holds for the second equation of (\ref{eq:Nwogu1}). Here we denote $c = B\nu$, $d = B(1-\nu)$, $\nu\in\mathbb{R}$ and $B=\bar{c}+\bar{d}$. From Lemma \ref{thrm:curl}, and again using Lemma \ref{newformulate} we see that 
  \begin{align*}
      \bar{c}D\nabla(\Div(D\bu_t))+\bar{d}D^2\nabla(\Div\bu_t) &=  \bar{c}\nabla(D\Div(D\bu_t))+\bar{d}\nabla(D^2\Div \bu_t) + O(\beta) \\
      &= \bar{c}\nabla(\Div(D^2\bu_t))+\bar{d}\nabla(\Div ( D^2\bu_t)) + O(\beta) \\
      &= B\nabla(\Div(D^2\bu_t)) + O(\beta) \\
      &= c\nabla(\Div(D^2\bu_t)) + d\nabla(\Div(D^2\bu_t)) + O(\beta) \\
      &= -c\nabla(\Div(D^2\nabla\eta)) + d\nabla(\Div(D^2\bu_t)) + O(\beta) \ ,
  \end{align*} 
where $O(\beta) = \beta g$ and $g \in L^\infty([0,\frac{T}{\varepsilon}];\;H^{s-4}(\mathbb{R}^2)^{2})$. So, 
$$\bar{c}D\nabla(\Div(D\bu_t))+\bar{d}D^2\nabla(\Div\bu_t) = -c\nabla(\Div(D^2\nabla\eta)) + d\nabla(\Div(D^2\bu_t)) \ ,$$
in the sense of consistency. Therefore, $(\eta^\varepsilon,\bu^\varepsilon)$ is consistent with the second equation of (\ref{eq:Nwogunabcd}) with a residual of order $\beta^2$ in $L^{\infty}([0,\frac{T}{\varepsilon}];\,H^{s-4}(\mathbb{R}^2)^{2})$. We conclude that $(\eta^\varepsilon,\bu^\varepsilon)$ is consistent with (\ref{eq:Nwogunabcd}) in the sense of definition \ref{defcons} and the proof is complete.
  \end{proof}
  
The Euler equations are consistent with the System (\ref{eq:Nwogu1}) at order $\varepsilon^2$ in $C([0,\frac{T}{\varepsilon}];\,\dot{H} ^{s+10} \times H^{s+5})$ as it was proved in \cite[Corollary 5.31]{Lannes13}. A direct consequence of this is that the Euler equations are consistent with the new class of systems (\ref{eq:Nwogunabcd}).

\begin{theorem}[Error estimates]
Let $s\geq 0$, $\varepsilon_0>0$, $b\in H^{\infty}(\mathbb{R}^2)$ and let $U_0=(\eta_0^\varepsilon,\psi_0^\varepsilon)_{0<\varepsilon<\varepsilon_0}$ be a bounded family in $H^p(\mathbb{R}^2)\times \dot{H}^{p+1}(\mathbb{R}^2)$($p\geq s$ large enough). Assume moreover that there is $h_{\min}>0$ such that the non-cavitation condition $h=1+\varepsilon\eta_0^\varepsilon+\beta D_b\geq h_{\min}$ is satisfied, and that there exists $T>0$ such that
\begin{itemize}
\item[$-$] There is a unique family of solutions $U=(\eta^\varepsilon_E, \psi^\varepsilon_E)\in C\big([0,\frac{T}{\varepsilon}];\,H^p(\mathbb{R}^2)\times \dot{H}^{p+1}(\mathbb{R}^2)\big)$ of the Euler equations (\ref{eq:S0}) with initial condition $U_0$.
\item[$-$] There is a unique family of solutions $(\eta_B^\varepsilon,\bu_B^\varepsilon)\in C\big([0,\frac{T}{\varepsilon}];\,H^p(\mathbb{R}^2)^3\big)$ of (\ref{eq:Nwogunabcd}) such that $(\eta_B^\varepsilon,\bu_B^\varepsilon)\vert_{(t=0)}=(\eta_0^\varepsilon,\nabla\psi_0^\varepsilon)$.
\end{itemize}
Then, if $(\eta^\varepsilon_E, \bu^\varepsilon_E)_{0<\varepsilon<\varepsilon_0}$ and $(\eta_B^\varepsilon, \bu_B^\varepsilon)_{0<\varepsilon<\varepsilon_0}$ are bounded in $W^{1,\infty}\big([0,\frac{T}{\varepsilon}];\,H^p(\mathbb{R}^2)^3\big)$ w.r.t $\varepsilon$, then for all $0<\varepsilon<\varepsilon_0$ and $t\in \left[0,\frac{T}{\varepsilon}\right]$
\begin{equation}\label{eq:errestim}
|\eta_B^\varepsilon - \eta^\varepsilon_E|_{L^{\infty}([0,t]\times H^s)} + |\bu_B^\varepsilon- \bu^\varepsilon_E|_{L^{\infty}([0,t]\times H^s)} \leq C\;\varepsilon^2t
\end{equation}
where $\bu^\varepsilon_E=\nabla\psi^\varepsilon_E$.
\end{theorem}
\begin{proof}
 The error estimate (\ref{eq:errestim}) follows using the triangle inequality and the results of \cite[Corollary 2]{BCL2005} for both $(\eta^\varepsilon_E, \bu^\varepsilon_E)$ and $(\eta_B^\varepsilon, \bu_B^\varepsilon)$. 

Indeed, performing symmetrization like in \cite{BCL2005}  (using appropriate nonlinear change of variables) to the new Boussinesq systems we obtain a class of systems with symmetric nonlinearity. This is the class of the so-called symmetric systems (see also \cite{BCL2005}), which is denoted by $S_{\theta,\mu,\nu}^{'}$. Let $(\bu,\eta)$ be a solution consistent with $S_{\theta,\mu,\nu}^{'}$ and $(\underline{\theta},\underline{\mu},\underline{\nu})$ fixed satisfying $a=c$. By applying the nonlinear change of variables, we have that $(\bu,\eta)$ is consistent with the class $S_{\underline{\theta},\underline{\mu},\underline{\nu}}^{'}$. Denote by $\Sigma=S_{\underline{\theta},\underline{\mu},\underline{\nu}}^{'}$ the corresponding subclass of symmetric non-linear and dispersive systems.
Moreover, for all $\varepsilon>0$, let $(\bu_{\Sigma}^\varepsilon,\eta_{\Sigma}^\varepsilon)$ be the exact solution of a system of the class $\Sigma$, then
\begin{align*}
|\bu_B^\varepsilon &- \bu^\varepsilon_E|_{L^{\infty}([0,t]\times H^s)} + |\eta_B^\varepsilon - \eta^\varepsilon_E|_{L^{\infty}([0,t]\times H^s)} = \\
&|\bu_B^\varepsilon - (1-\frac{\varepsilon}{2}(1-\theta^{2})\Delta)^{-1}(1-\frac{\varepsilon}{2}(1-\underline{\theta}^{2})\Delta)(\bu_{\Sigma}^\varepsilon(1-\frac{\varepsilon}{2}\eta_{\Sigma}^\varepsilon)) \\
&+ (1-\frac{\varepsilon}{2}(1-\theta^{2})\Delta)^{-1}(1-\frac{\varepsilon}{2}(1-\underline{\theta}^{2})\Delta)(\bu_{\Sigma}^\varepsilon(1-\frac{\varepsilon}{2}\eta_{\Sigma}^\varepsilon)) - \bu^\varepsilon_E|_{L^{\infty}([0,t]\times H^s)} \\
&+ |\eta_B^\varepsilon - \eta^\varepsilon_{\Sigma} + \eta^\varepsilon_{\Sigma} - \eta^\varepsilon_E|_{L^{\infty}([0,t]\times H^s)} \\
&\leq |\bu_B^\varepsilon - (1-\frac{\varepsilon}{2}(1-\theta^{2})\Delta)^{-1}(1-\frac{\varepsilon}{2}(1-\underline{\theta}^{2})\Delta)(\bu_{\Sigma}^\varepsilon(1-\frac{\varepsilon}{2}\eta_{\Sigma}^\varepsilon))|_{L^{\infty}([0,t]\times H^s)} \\
&+ |\eta_B^\varepsilon - \eta^\varepsilon_{\Sigma}|_{L^{\infty}([0,t]\times H^s)} + |\eta^\varepsilon_{\Sigma} - \eta^\varepsilon_E|_{L^{\infty}([0,t]\times H^s)} \\
&+ |(1-\frac{\varepsilon}{2}(1-\theta^{2})\Delta)^{-1}(1-\frac{\varepsilon}{2}(1-\underline{\theta}^{2})\Delta)(\bu_{\Sigma}^\varepsilon(1-\frac{\varepsilon}{2}\eta_{\Sigma}^\varepsilon)) - \bu^\varepsilon_E|_{L^{\infty}([0,t]\times H^s)} \\
& = I + II\ ,
\end{align*}
where 
$$\begin{aligned}
I=&|\bu_B^\varepsilon - (1-\frac{\varepsilon}{2}(1-\theta^{2})\Delta)^{-1}(1-\frac{\varepsilon}{2}(1-\underline{\theta}^{2})\Delta)(\bu_{\Sigma}^\varepsilon(1-\frac{\varepsilon}{2}\eta_{\Sigma}^\varepsilon))|_{L^{\infty}([0,t]\times H^s)} \\ 
&+|\eta_B^\varepsilon - \eta^\varepsilon_{\Sigma}|_{L^{\infty}([0,t]\times H^s)} \ ,
\end{aligned}$$
and
$$\begin{aligned}
II=&|(1-\frac{\varepsilon}{2}(1-\theta^{2})\Delta)^{-1}(1-\frac{\varepsilon}{2}(1-\underline{\theta}^{2})\Delta)(\bu_{\Sigma}^\varepsilon(1-\frac{\varepsilon}{2}\eta_{\Sigma}^\varepsilon)) - \bu^\varepsilon_E|_{L^{\infty}([0,t]\times H^s)} \\
& + |\eta^\varepsilon_{\Sigma} - \eta^\varepsilon_E|_{L^{\infty}([0,t]\times H^s)}\ .
\end{aligned}$$
Using \cite[Corollary 2]{BCL2005} for the solution of the Boussinesq system $(\bu_B^\varepsilon, \eta_B^\varepsilon)$ for all $\varepsilon>0$, we obtain $I \leq C\;\varepsilon^2t$. Now, 
thanks to the Proposition \ref{propcons} and \cite[Corollary 2]{BCL2005} for the solution of the Euler equations $(\bu_E^\varepsilon, \eta_E^\varepsilon)$, for all $\varepsilon>0$, yields $II \leq C\;\varepsilon^2t $, which implies (\ref{eq:errestim}) and completes the proof.
\end{proof}

\begin{remark} Note that different Boussinesq systems have different well-posedness requirements. In the previous theorem we consider $p$ large enough so as to embrace all valid Boussinesq systems. The maximal time of existence also varies depending on the particular system but all systems follow the general results of \cite{DMS2007,saut2012cauchy}. The situation is much different when bounded domains are considered. The results then depend on the choice of the system as well as on the boundary conditions. A discussion on justified slip-wall boundary conditions for Boussinesq systems is presented in Section \ref{sec:bcs}.
\end{remark}


\section{Energy conservation and variational derivation}\label{sec:energy}

In this section we consider $\Omega=\mathbb{R}^2$ while the derivations hold true in an bounded domain as well, under the hypothesis that the solutions along with their derivatives vanish appropriately on the boundary. For example, the boundary conditions
$$\tilde{\bu}\cdot\bn=0,\quad a \tilde{\nabla}(\tilde{D}^3\tilde{\Div}\tilde{\bu})\cdot\bn=0,\quad b \tilde{\nabla}\tilde{\eta}\cdot\bn=0\quad \text{ on }\ \partial\Omega\ ,$$
are adequate to carry the following analysis.

Consider the system (\ref{eq:eul28})--(\ref{eq:eul29}) with stationary bottom $\tilde{D}$. Setting 
$$\tilde{R}=(\tilde{D}+\varepsilon\tilde{\eta})\tilde{\bu}+\sigma^2\left\{a\nabla(\tilde{D}^3\tilde{\Div}\tilde{\bu})-b\tilde{D}^2\tilde{\nabla}{\tilde\eta}_{\tilde{t}}\right\}\ ,$$
and 
$$\tilde{Q}=\tilde{\eta}+\varepsilon\tfrac{1}{2}|\tilde{\bu}|^2 +\sigma^2\left\{c\tilde{\Div}(\tilde{D}^2\tilde{\nabla}\tilde{\eta})-d\tilde{\Div}(\tilde{D}^2{\tilde{\bu}}_{\tilde{t}})\right\}\ ,$$
and using the boundary conditions we have
$$
0 =\int_\Omega {\tilde{\eta}}_{\tilde{t}}\tilde{Q}+{\tilde{\bu}}_{\tilde{t}}\cdot \tilde{R}+\tilde{\Div} \tilde{R}\tilde{Q}+\tilde{R}\cdot \tilde{\nabla} \tilde{Q} = \int_{\Omega} {\tilde{\eta}}_{\tilde{t}}\tilde{Q}+{\tilde{\bu}}_{\tilde{t}}\cdot \tilde{R}\ .
$$
Moreover, if $b=d$ we have
$$
\begin{aligned}
0 &= \int_{\Omega} {\tilde{\eta}}_{\tilde{t}}\tilde{Q}+{\tilde{\bu}}_{\tilde{t}}\cdot \tilde{R}\\
&=\int_{\Omega} {\tilde{\eta}}_{\tilde{t}}\tilde{\eta}+\varepsilon\tfrac{1}{2}{\tilde{\eta}}_{\tilde{t}}|\bu|^2-\sigma^2c\tilde{D}^2\tilde{\nabla}{\tilde{\eta}}_{\tilde{t}}\cdot \tilde{\nabla}\tilde{\eta}+\tilde{D}\tilde{\bu}{\tilde{\bu}}_{\tilde{t}}+\varepsilon\tilde{\eta} \tilde{\bu}\cdot{\tilde{\bu}}_{\tilde{t}}-\sigma^2a\tilde{D}^3 \tilde{\Div}\tilde{\bu}\tilde{\Div} {\tilde{\bu}}_{\tilde{t}}\\
&=\frac{1}{2}\frac{d}{d\tilde{t}}\int_\Omega \tilde{\eta}^2+(\tilde{D}+\varepsilon\tilde{\eta})|\tilde{\bu}|^2-\sigma^2\left[c\tilde{D}^2|\tilde{\nabla}\tilde{\eta}|^2+a\tilde{D}^3(\tilde{\Div}\tilde{\bu})^2 \right]\ .
\end{aligned}
$$
Therefore, for $a\leq 0$, $c\leq 0$ and $b=d\geq 0$ we define the energy functional
\begin{equation}\label{eq:energy}
E(t)=\frac{1}{2}\int_\Omega \tilde{\eta}^2+(\tilde{D}+\varepsilon\tilde{\eta})|\tilde{\bu}|^2-\sigma^2\left[c\tilde{D}^2|\tilde{\nabla}\tilde{\eta}|^2+a\tilde{D}^3(\tilde{\Div}\tilde{\bu})^2 \right]\ ,
\end{equation}
which in dimensional variables can be written as
\begin{equation}\label{eq:energyd}
E(t)=\frac{1}{2}\int_\Omega g\eta^2+(D+\eta)|\bu|^2-\left[cgD^2|\nabla\eta|^2+aD^3(\Div\bu)^2 \right]\ .
\end{equation}
Using the particular energy function, we can derive energy conservative systems of the class (\ref{eq:Nwogunabcdd2}) using the variational method of \cite{CD2012}. We define the kinetic energy as
$$\mathcal{K}=\frac{\rho}{2}\int_{t_1}^{t_2}\int_\Omega (D+\eta)|\bu|^2-aD^3(\Div\bu)^2~d\bx~dt\ ,$$
and the potential energy as
$$\mathcal{P}=\frac{\rho}{2}\int_{t_1}^{t_2}\int_\Omega g[\eta^2-cD^2|\nabla \eta|^2]~d\bx~dt\ .$$
Then we define the action integral
$$\mathcal{I}=\mathcal{K}-\mathcal{P}+\rho\int_{t_1}^{t_2}\int_\Omega [\eta_t+\Div[(D+\eta)\bu]+\Div\left\{a\nabla(D^3\Div\bu)-bD^2\nabla\eta_t\right\}]\phi~d\bx~dt\ ,$$
where the mass conservation is imposed with the help of the Lagrange multiplier $\phi(\bx,t)$. Then the Euler-Lagrange equations for the action integral $\mathcal{I}$ are the following:
\begin{align}
\delta \phi~: & \quad \eta_t+\Div[(D+\eta)\bu]+\Div\left\{a\nabla(D^3\Div\bu)-bD^2\nabla\eta_t\right\}=0\ , \label{eq:var1}\\
\delta \bu~: & \quad \bu-\nabla\phi=0\ , \label{eq:var2}\\
\delta \eta~: & \quad \tfrac{1}{2}|\bu|^2-cg\Div(D^2\nabla\eta)-g\eta-\phi_t+b\nabla\cdot(D^2\nabla\phi_t)-\bu\cdot\nabla\phi=0\ . \label{eq:var3}
\end{align}
Eliminating $\phi$ in (\ref{eq:var3}) using (\ref{eq:var2}) we obtain the momentum conservation equation
$$\bu_t+g\nabla\eta+\tfrac{1}{2}\nabla|\bu|^2 +\nabla\left\{cg\Div(D^2\nabla\eta)-d\Div(D^2\bu_t)\right\}=0\ ,$$
with $b=d$. In this way we re-derived the subclass of systems (\ref{eq:Nwogunabcdd2}) that preserve a reasonable form of energy, and these must have necessarily $b=d$. From (\ref{eq:var2}) we observe that the new systems imply potential flow, and the conservation of the $\Curl\bu$ follows.

\section{Valid slip-wall boundary conditions}\label{sec:bcs}


Practical problems involving water waves are usually posed in bounded polygonal domains $\Omega$. For example, waves in a port, or waves interacting with sides of a basin. Wall boundary conditions are then required to be imposed alongside with the model equations. When the waves can slip on the wall without any friction, then a slip-wall boundary condition of the form $\bu\cdot\bn=0$ needs to be imposed. Here, $\bn$ is the outward unit normal vector on the boundary $\partial\Omega$. While, no other boundary conditions are usually necessary to be imposed, it is interesting to observe that Euler equations obey to additional boundary conditions in an implicit way. For example, the Neumann condition $\nabla\eta\cdot\bn=0$ holds true on $\partial\Omega$ when $\Omega$ is, for example, a polygonal domain, \cite{Khakimzyanov2018a}. To derive the particular boundary condition first observe that $\nabla p\cdot\bn=0$ on $\partial\Omega$. This can be obtained after taking the inner product of equation (\ref{eq:momentv}) with $\bn$ on $\partial\Omega$. Writing the dynamic boundary condition (\ref{eq:pressbc}) as
$p(\bx,\eta(\bx,t),t)=p_{\text{atm}}$,
and taking the horizontal gradient operator we obtain
$\nabla p+p_z\nabla \eta={\bf 0}$.
After multiplication with the normal vector $\bn$ and using the fact that $\nabla p\cdot\bn=0$ on $\partial\Omega$ we obtain the boundary condition $\nabla \eta\cdot \bn=0$ on $\partial\Omega$. If the domain is a general domain in which well-posedness can be established, then by constructing appropriate polygonal approximations $\Omega_n\subset\Omega$ such that $\lim\Omega_n=\Omega$ and the particular Neumann condition can be used in more general situations. This particular Neumann condition is required for the well-posed Boussinesq systems of Bona-Smith and BBM-BBM type in bounded domains with slip-wall boundary conditions, and because of the previous analysis, apparently, it is a physical condition \cite{KMS2020,IKKM2021}.  
Note that for the original Nwogu system (\ref{eq:Nwogu}) in addition to the slip-wall boundary conditions $\bu\cdot\bn=\nabla\eta\cdot\bn=0$ on $\partial\Omega$, it is required that $[\tilde{a}D^2\nabla(\Div (D\bu))+\tilde{b}D^3\nabla(\Div \bu)]\cdot\bn=0$ must be satisfied on the boundary $\partial\Omega$, \cite{WB2002}. This boundary condition is satisfied by the solutions of the Euler equations only when the bottom is flat, while for general bottoms we obtain a simpler condition. 

From the irrotationality condition (\ref{eq:irrotational}) we have $\bu_z=\nabla w$ and thus $\nabla w\cdot \bn=0$ on $\partial \Omega$. Moreover, differentiating the mass equation (\ref{eq:mass}) we obtain $\nabla ( \Div \bu) =-\nabla w_z$, which implies the boundary condition $\nabla (\Div \bu )\cdot \bn =0$ on $\partial \Omega$. Similar conditions hold true for BBM-BBM-type systems, for example, the slip-wall conditions imply that $\nabla(\Div (D^2\bu_t))\cdot\bn=0$ on $\partial\Omega$. If the solution satisfies $\nabla(\Div D^2 \bu)\cdot\bn=0$ on $\partial \Omega$, then this is satisfied for $t\geq 0$. Due to the assumption of mild bottom variations in the derivation of the particular BBM-BBM system, and also for the new Nwogu system, we have that $\nabla(\Div \bu)\cdot \bn\approx 0$, which means that even if the boundary condition is not satisfied exactly, it is a reasonable approximation of the exact boundary condition. The same will be true for Nwogu-type systems of the form (\ref{eq:Nwogunabcdd2}). This shows that some of the new $abcd$-systems (at least the BBM-BBM, Bona-Smith, Nwogu and regularized Nwogu systems) can be used in practical situations in bounded domains with more accurate boundary conditions compared to the other Boussinesq systems. It is worth mentioning that from all known Boussinesq systems only the BBM-BBM system has been proved well-posed in bounded domains with slip-wall boundary conditions  \cite{IKKM2021}, while the well-posedness in unbounded domains follow the work \cite{DMS2007} in a similar manner. The Bona-Smith systems can also been proved well-posed in bounded domain but this will appear in different work.

\section{Experimental validation}\label{sec:valid}

The mild slope assumption used in the derivation of the new  $abcd$-Boussinesq systems with variable bottom topography raises a validity question: What are the minimum  restrictions to the bottom variations to ensure the accuracy of the new systems? This question requires a more detailed study than this one, but in this section we show that the particular assumption is not a significant barrier. On the contrary, the simplicity of the equations and their accuracy is a great attraction for the mathematical modeling of long waves. 

Among the systems we study in this section is the new Nwogu system. Although this system does not preserve any form of energy, it has optimal linear dispersion relation for $\theta^2=1/5$, $\mu=1$ and $\nu=0$ and outperforms other systems in experiments where the bottom variations as well as the steepness of the waves stress the validity of Boussinesq systems. 
For the sake of convenience we rewrite the new Nwogu system in the form
\begin{equation}\label{eq:Nwogunew}
\begin{aligned}
& \eta_t+\Div[(D+\eta)\bu]-\frac{1}{15}\Delta(D^3\Div\bu)=\frac{\sqrt{5}-2}{2\sqrt{5}}\Div(D^2\nabla\zeta_t)-\zeta_t\ ,\\
& \bu_t+g\nabla\eta+\tfrac{1}{2}\nabla|\bu|^2 -\frac{2}{5}\nabla\Div(D^2\bu_t)=\frac{\sqrt{5}-1}{\sqrt{5}}D\nabla\zeta_{tt}\ .
\end{aligned}
\end{equation}
 We also consider the new BBM-BBM system ($\theta^2=2/3$, $\nu=\mu=0$)
 \begin{equation}\label{eq:newbbmbbm}
\begin{aligned}
& \eta_t+\Div[(D+\eta)\bu]-\frac{1}{6}\Div (D^2\nabla\eta_t)=\frac{ 2-\sqrt{6}}{3}\Div(D^2\nabla\zeta_t)-\zeta_t\ ,\\
& \bu_t+g\nabla\eta+\tfrac{1}{2}\nabla|\bu|^2 -\frac{1}{6}\nabla\Div(D^2\bu_t)=\frac{\sqrt{3}-\sqrt{2}}{\sqrt{3}}D\nabla\zeta_{tt}\ ,
\end{aligned}
\end{equation}
which preserves the energy functional (\ref{eq:energyd}) with $a=c=0$. 
We also study the new Peregrine system ($\theta^2=1/3$, $\nu=\mu=0$)
\begin{equation}\label{eq:newperegrine}
\begin{aligned}
& \eta_t+\Div[(D+\eta)\bu]=\frac{\sqrt{3}-2}{2\sqrt{3}}\Div(D^2\nabla\zeta_t)-\zeta_t\ ,\\
& \bu_t+g\nabla\eta+\tfrac{1}{2}\nabla|\bu|^2 -\frac{1}{3}\nabla\Div(D^2\bu_t)=\frac{\sqrt{3}-1}{\sqrt{3}}D\nabla\zeta_{tt}\ .
\end{aligned}
\end{equation}
All the experiments focus on the effects of bottom topography on long one-dimensional waves while more sophisticated experiments will be on focus in future works.

\subsection{Shoaling and reflection of solitary waves}

In the first experiment we study the ability of the new models to describe shoaling waves on gentle slopes \cite{Dodd1998}. In particular we consider the one-dimensional computational domain $\Omega=[-100,20]$ and bottom topography described by the function 
$$D(x)=\left\{ \begin{array}{ll}
0.7, & x\leq 0,\\
0.7-x/50, &x>0 .
\end{array}
\right.
$$
All the variables are dimensional with units in  SI. For the numerical simulation of this experiment we consider wall boundary conditions at the endpoints $x=-100$ and $x=20$, while the numerical method is the one studied in \cite{mm2023} for the Nwogu system, in \cite{ADM2010} for the BBM-BBM and in \cite{AD2012} for the Peregrine system, extended in all cases to the variable bottom topography. For all the experiments we used cubic spline elements and the classical, explicit, fourth-order Runge-Kutta method with four stages. 

We study first the reflection of two right-traveling solitary wave with amplitudes $A_1=0.07~m$ and $A_2=0.12~m$. The two solitary waves have been translated horizontally so as their maximum at $t=0$ is achieved at $x_0=-30$. The two waves first propagate over the sloping bottom and then hit the wall at $x=-20$ where they are reflected back and propagate to the left. We record the free surface elevation at three wave-gauges located at $x=0$, $x=16.25$ and $17.75$.
\begin{figure}[ht!]
  \centering
\includegraphics[width=0.75\columnwidth]{figures/figure3}
  \caption{Reflection of shoaling solitary wave of amplitude $A=0.07~m$ by a vertical wall}
  \label{fig:shoal1}
\end{figure}
 \begin{figure}[ht!]
  \centering  \includegraphics[width=0.75\columnwidth]{figures/figure4}
  \caption{Reflection of shoaling solitary wave of amplitude $A=0.12~m$ by a vertical wall}
  \label{fig:shoal2}
\end{figure}

The results of these experiments for the solitary wave with $A_1=0.07~m$  are presented in Figure \ref{fig:shoal1} while the results of the other solitary wave are presented in Figure \ref{fig:shoal2}. We observe that the solution of the BBM-BBM system agrees better with the experimental data in these particular experiments. This behavior has been observed also in \cite{KMS2020} with a BBM-BBM system that does not preserve any energy functional. Thus, its behaviour should not be attributed to energy conservation but rather to the particular linear structure of the BBM-BBM system.

We close the study of the shoaling waves by studying an experiment of \cite{GSSV1994}.  Specifically, we study the shoaling of a solitary wave with amplitude $A=0.2$ over the bottom topography described by the function
$$D(x)=\left\{ \begin{array}{ll}
1, & x\leq 0,\\
1-x/35, &x>0 .
\end{array}
\right.
$$
We record the free surface elevation at 5 wavegauges that are located at $x=20.96, 22.55, 23.68, 24.68, 25.91$, respectively. Note that in this experiment the variables are all scaled. This experiment apparently is more demanding compared to the previous one since the slope is larger, and for this reason it was required appropriate calibration of the initial data so as the numerical solution matches with the experimental data. Surprisingly, the BBM-BBM system required the least phase calibration. The results are presented in Figure \ref{fig:shoal3}. Once again the BBM-BBM system appears to approximate shoaling waves better compared to the other systems. Similar results have been presented for the classical Nwogu system in \cite{FBCR2015} while the calibration of the initial condition is apparently necessary for all Boussinesq systems. For example, when we tried the classical Peregrine system (\ref{eq:Peregrin}) in comparison with the new Peregrine system we were compelled to use the same calibration to the initial conditions for both systems in order to fit in the phase of the numerical solutions with that of the experimental data. On the other hand, when we tested the same experiment using the Serre-Green-Naghdi system there was no requirement for calibration, \cite{MSM2017}. This implies that even in this extreme for Boussinesq systems situation, the assumption of small bottom variations is not a disadvantage, while Boussinesq systems start loosing accuracy in such extreme situations. This error in the phase can also be observed in the first experiment but it is so small that no calibration is required.
 \begin{figure}[ht!]
  \centering  \includegraphics[width=0.75\columnwidth]{figures/figure5}
  \caption{Recorded wavegauges of  a shoaling solitary wave of amplitude $A=0.2$ on a plain beach of slope $1~:~35$.}
  \label{fig:shoal3}
\end{figure}


\subsection{Periodic waves over a submerged bar} 

Here, we test the same Boussinesq systems against a laboratory experiment designed to test the nonlinear and dispersive properties of Boussinesq systems with variable bathymetry \cite{BB1994}.  The specific experiment has been used to validate other Boussinesq systems including the original Nwogu system, and is one of the standard benchmarks for numerical models \cite{Dingemans1994,KDNS12,WB1999}.

In this experiment, small-amplitude periodic waves are generated by a wavemaker and propagate over a bathymetry defined by the function
$$
D(x,y)=\left\{
\begin{array}{cc}
-0.05x+0.7, & x\in[6,12) \\
0.1, & x\in[12,14) \\
0.1x-1.3, & x\in[14,17] \\
0.4, & \mbox{elsewhere}
\end{array}
 \right.
 $$
which is depicted in Figure \ref{fig:bottom}. 
 \begin{figure}[ht!]
  \centering
  \includegraphics[width=0.9\columnwidth]{figures/figure1}
  \caption{The bathymetry in relation to the sponge layers (light blue regions), wavemaker and wavegauges locations (distances in metres)}
  \label{fig:bottom}
\end{figure}

 \begin{figure}[ht!]
  \centering
  \includegraphics[width=\columnwidth]{figures/figure2}
  \caption{The recorded data at the various wavegauges and the numerical solutions for the new Nwogu, BBM-BBM and Peregrine systems}
  \label{fig:results}
\end{figure}

When the generated waves interact with the upward part of the submerged bar, they shoal and become steep. Then the waves propagate faster over the downward part of the bar covering a wide range of wave-numbers in the spectrum of long waves. The input wave has period $\tau=2.02~s$ and amplitude $A=0.01~m$. The free surface of the water is recorded at  wavegauges located at $x=10.5$, $12.5$, $13.5$, $14.5$, $15.7$, $17.3$, $19.0$ and $21.0$. The waves are generated numerically using the a wavemaker at $x=2.01$ as it is described in \cite{KMS2020}.



The locations of the wavemaker and the wavegauges are depicted by vertical broken lines in Figure \ref{fig:bottom} while the experimentally recorded values of the free-surface elevation at the various gauges are presented in Figure \ref{fig:results}. In the same figure we present the numerical results obtained using the standard Galerkin/Finite element method with cubic splines  combined with the classical Runge-Kutta method of order 4 solving the one-dimensional initial-periodic boundary-value problem for the system (\ref{eq:Nwogunew}) as well as the corresponding BBM-BBM and Peregrine systems (\ref{eq:Nwogunabcdd2}). More details for the particular numerical method are presented in \cite{ADM2010ii} and \cite{mm2023}. We observe that the agreement between the numerical solution and the experimental data is impressive. 

Comparing the results obtained using the BBM-BBM and Peregrine systems and similar numerical methods and wave-generation procedure, \cite{KMS2020}, we observe that the performance of the new Nwogu system is better due to its optimal dispersion properties while overall the errors might be of the same magnitude. Usually the numerical solution of other Boussinesq systems (including Peregrine's system) start diverging from the experimental data after the wavegauge (f) while the phase of the numerical solution  for the new Nwogu system agrees in all wavegauges with the experimental data. Other properties such as the shoaling of solitary waves have been tested for the new BBM-BBM and Peregrine systems in \cite{IKKM2021}.

We conclude that the simplifications followed after the mild bottom variations assumption do not affect the validity of the Boussinesq systems even in a very demanding experiment of nearly breaking waves such as the one presented in this section. 




\section{Conclusions}
The class of $abcd$-Boussinesq systems of water waves theory of \cite{BCS2002} and \cite{BCL2005} is generalized to the new set of $abcd$-Boussinesq systems (\ref{eq:Nwogunabcdd2})--(\ref{eq:abcdcoef2}) in the case of variable bottom topography and in two-dimensional domains. Taking into consideration practical applications the systems  have been appropriately formulated so as to be valid in simulations with slip-wall boundary conditions. We demonstrate also that the new systems can have accurate slip-wall boundary conditions which was not straightforward with their classical counterparts. A subclass of the new systems preserve a meaningful energy functional which makes them appropriate for conservative numerical simulations. The consistency of the new systems with the Euler equations has also been studied and the errors between the Euler equations and the new Boussinesq systems are estimated extending the results of \cite{Lannes13,BCL2005}. Finally, we show that the assumption of smooth bathymetric variations is not restrictive as far as it concerns classical benchmarks. In particular, we were unable to detect any case where the new systems failed compared to the classical systems. 

%\bibliographystyle{plain}
%\bibliography{biblio}
%\bigskip

\begin{thebibliography}{10}

\bibitem{AD2012}
D.~Antonopoulos and V.~Dougalis.
\newblock {Numerical solution of the `classical' Boussinesq system}.
\newblock {\em Mathematics and Computers in Simulation}, 82:984--1007, 2012.

\bibitem{ADM2010ii}
D.~Antonopoulos, V.~Dougalis, and D.~Mitsotakis.
\newblock {Galerkin approximations of periodic solutions of Boussinesq
  systems}.
\newblock {\em Bulletin of the Hellenic Mathematical Society}, 57:13--30, 2010.

\bibitem{ADM2010}
D.~Antonopoulos, V.~Dougalis, and D.~Mitsotakis.
\newblock {Numerical solution of Boussinesq systems of the Bona-Smith family}.
\newblock {\em Applied numerical mathematics}, 60:314--336, 2010.

\bibitem{BB1994}
S.~Beji and J.A. Battjes.
\newblock Numerical simulation of nonlinear wave propagation over a bar.
\newblock {\em Coastal Engineering}, 23:1--16, 1994.

\bibitem{BBM1972}
B.~Benjamin, J.~Bona, and J.~Mahony.
\newblock Model equations for long waves in nonlinear dispersive systems.
\newblock {\em Philosophical Transactions of the Royal Society of London.
  Series A, Mathematical and Physical Sciences}, 272(1220):47--78, 1972.

\bibitem{BC1998}
J.~Bona and M.~Chen.
\newblock {A Boussinesq system for two-way propagation of nonlinear dispersive
  waves}.
\newblock {\em Physica D: Nonlinear Phenomena}, 116:191--224, 1998.

\bibitem{BCS2002}
J.~Bona, M.~Chen, and J.-C. Saut.
\newblock {Boussinesq equations and other systems for small-amplitude long
  waves in nonlinear dispersive media. I: Derivation and linear theory}.
\newblock {\em J.Nonlin. Sci.}, 12:283--318, 2002.

\bibitem{BCL2005}
J.~Bona, T.~Colin, and D.~Lannes.
\newblock Long wave approximations for water waves.
\newblock {\em Arch. Rational Mech. Anal.}, 178:373--410, 2005.

\bibitem{BDM2007i}
J.~Bona, V.~Dougalis, and D.~Mitsotakis.
\newblock {Numerical solution of KdV--KdV systems of Boussinesq equations: I.
  The numerical scheme and generalized solitary waves}.
\newblock {\em Math. Comp. Simul.}, 74:214--228, 2007.

\bibitem{BDM2008ii}
J.~Bona, V.~Dougalis, and D.~Mitsotakis.
\newblock {Numerical solution of Boussinesq systems of KdV--KdV type: II.
  Evolution of radiating solitary waves}.
\newblock {\em Nonlinearity}, 21:2825--2848, 2008.

\bibitem{BS1976}
J.~Bona and R.~Smith.
\newblock A model for the two-way propagation of water waves in a channel.
\newblock {\em Math. Proc. Camb. Phil. Soc.}, 79:167--182, 1976.

\bibitem{Bous1871}
J.~Boussinesq.
\newblock Th\'{e}orie de l'intumescence liquide appel\'{e}e onde solitaire ou
  de translation se propageant dans un canal rectangulaire.
\newblock {\em CR Acad. Sci. Paris}, 72(755--759):1871, 1871.

\bibitem{Bous1872}
J.~Boussinesq.
\newblock Th\'{e}orie des ondes et des remous qui se propagent le long d'un
  canal rectangulaire horizontal, en communiquant au liquide contenu dans ce
  canal des vitesses sensiblement pareilles de la surface au fond.
\newblock {\em J. Math. Pures Appl.}, 17:55--108, 1872.

\bibitem{Chen03}
M.~Chen.
\newblock {Equations for bi-directional waves over an uneven bottom}.
\newblock {\em Math. Comp. Simul.}, 62:3--9, 2003.

\bibitem{CD2012}
D.~Clamond and D.~Dutykh.
\newblock Practical use of variational principles for modeling water waves.
\newblock {\em Physica D}, 241:25--36, 2012.

\bibitem{CDM2017}
D.~Clamond, D.~Dutykh, and D.~Mitsotakis.
\newblock {Conservative modified Serre-Green-Naghdi equations with improved
  dispersion characteristics}.
\newblock {\em Commun. Nonlinear Sci. Numer. Simulat.}, 45:245--257, 2017.

\bibitem{Dingemans1994}
M.W. Dingemans.
\newblock {Comparison of computations with Boussinesq-like models and
  laboratory measurements}.
\newblock Technical Report H1684.12, MAST G8M, Coastal Morphodynamics Research
  Programme, 1994.

\bibitem{Dodd1998}
N.~Dodd.
\newblock {Numerical model of wave run-up, overtopping, and regeneration}.
\newblock {\em J. Waterway Port Coastal and Oc Eng}, 124(2):73--81, 1998.

\bibitem{DM2008}
V.~Dougalis and D.~Mitsotakis.
\newblock {Theory and numerical analysis of Boussinesq systems: A review}.
\newblock In N.~Kampanis, V.~Dougalis, and J.~Ekaterinaris, editors, {\em
  Effective Computational Methods in Wave Propagation}, pages 63--110. CRC
  Press, 2008.

\bibitem{DMS2007}
V.~Dougalis, D.~Mitsotakis, and J.-C. Saut.
\newblock {On some Boussinesq systems in two space dimensions: Theory and
  numerical analysis}.
\newblock {\em ESAIM: Math. Model. Num. Anal.}, 41:825--854, 2007.

\bibitem{dms2009}
V.~Dougalis, D.~Mitsotakis, and J.-C. Saut.
\newblock {On initial-boundary value problems for a Boussinesq system of
  BBM-BBM type in a plane domain}.
\newblock {\em Discrete Contin. Dyn. Syst}, 23:1191--1204, 2009.

\bibitem{DMS2010}
V.~Dougalis, D.~Mitsotakis, and J.-C. Saut.
\newblock {Boussinesq systems of Bona-Smith type on plane domains: theory and
  numerical analysis}.
\newblock {\em J. Sci. Comp.}, 44(2):109--135, 2010.

\bibitem{DuchIsr2018}
V.~Duch\^{e}ne and S.~Israwi.
\newblock {Well-posedness of the Green-Naghdi and Boussinesq-Peregrine
  systems}.
\newblock {\em Ann. Math. Blaise Pascal}, 25:21--74, 2018.

\bibitem{DurIsrawi12}
M.~Duruflé and S.~Israwi.
\newblock A numerical study of variable depth kdv equations and generalizations
  of camassa–holm-like equations.
\newblock {\em Journal of Computational and Applied Mathematics},
  236:4149--4165, 2012.

\bibitem{Euler1757}
L.~Euler.
\newblock Principes g\'{e}n\'{e}raux du moubement des fluides.
\newblock In {\em M\'{e}moires de l'acad\'{e}mie des sciences de Berlin},
  volume~12, pages 274--315, 1757.

\bibitem{Feng2010}
K.~Feng and M.~Qin.
\newblock {\em Symplectic geometric algorithms for Hamiltonian systems}, volume
  449.
\newblock Springer, 2010.

\bibitem{FBCR2015}
A.~Filippini, S.~Bellec, M.~Colin, and M.~Ricchiuto.
\newblock {On the nonlinear behaviour of Boussinesq type models:
  Amplitude-velocity vs amplitude-flux forms}.
\newblock {\em Coastal Engineering}, 99:109--123, 2015.

\bibitem{GN1976}
A.~Green and P.~Naghdi.
\newblock A derivation of equations for wave propagation in water of variable
  depth.
\newblock {\em J. Fluid Mech.}, 78:237--246, 1976.

\bibitem{GSSV1994}
S.~Grilli, R.~Subramanya, I.~Svendsen, and J.~Veeramony.
\newblock Shoaling of solitary waves on plane beaches.
\newblock {\em J. Waterway, Port, Coastal, and Ocean Eng.}, 120:609--628, 1994.

\bibitem{Israwii10}
S.~Israwi.
\newblock variable depth kdv equations and generalizations to more nonlinear
  regimes.
\newblock {\em ESAIM: M2AN}, 44(2):347 -- 370, 2010.

\bibitem{Israwi11}
S.~Israwi.
\newblock {Large time existence for 1D Green-Naghdi equations}.
\newblock {\em Nonlinear Analysis}, 74:81--93, 2011.

\bibitem{IKKM2021}
S.~Israwi, H.~Kalisch, T.~Katsaounis, and D.~Mitsotakis.
\newblock {A regualarized shallow-water waves system with slip-wall boundary
  conditions in a basin: Theory and numerical analysis}.
\newblock {\em Nonlinearity}, 35:750--786, 2022.

\bibitem{Isrtal13}
S.~Israwi and R.~Talhouk.
\newblock Local well-posedness of a nonlinear kdv-type equation.
\newblock {\em Comptes Rendus Mathematique}, 351(23-24):895--899, 2013.

\bibitem{KMS2020}
T.~Katsaounis, D.~Mitsotakis, and G.~Sadaka.
\newblock {Boussinesq-Peregrine water wave models and their numerical
  approximation}.
\newblock {\em J. Comp. Phys.}, 417:109579, 2020.

\bibitem{KDNS12}
M.~Kazolea, A.~Delis, I.~Nikolos, and C.~Synolakis.
\newblock {An unstructured finite volume numerical scheme for extended 2D
  Boussinesq-type equations}.
\newblock {\em Coastal Engineering}, 69:42--66, 2012.

\bibitem{Khakimzyanov2018a}
G.~S. Khakimzyanov and D.~Dutykh.
\newblock {Long wave interaction with a partially immersed body. Part I:
  Mathematical models}.
\newblock {\em Comm. Comp. Phys.}, pages 1--62, 2019.

\bibitem{BLIG2022}
B.~Khorbatly, R.~Lteif, S.~Israwi, and S.~Gerbi.
\newblock Mathematical modeling and numerical analysis for the higher order
  boussinesq system.
\newblock {\em ESAIM: M2AN}, 56:593 -- 615, 2022.

\bibitem{L2005}
D.~Lannes.
\newblock Well-posedness of the water-waves equations.
\newblock {\em Journal of the American Mathematical Society}, 18(3):605--654,
  2005.

\bibitem{Lannes13}
D.~Lannes.
\newblock {\em The water waves problem: mathematical analysis and asymptotics},
  volume 188.
\newblock Americal Mathematical Society, Providence, Rhode Island, 2013.

\bibitem{mm2023}
D.~Mantzavinos and D.~Mitsotakis.
\newblock Extended water wave systems of boussinesq equations on a finite
  interval: Theory and numerical analysis.
\newblock {\em J. Math. Pures Appl.}, 169:109--137, 2023.

\bibitem{mits2009}
D.~Mitsotakis.
\newblock {Boussinesq systems in two space dimensions over a variable bottom
  for the generation and propagation of tsunami waves}.
\newblock {\em Mat. Comp. Simul.}, 80:860--873, 2009.

\bibitem{MSM2017}
D.~Mitsotakis, C.~Synolakis, and M.~McGuinness.
\newblock {A modified Galerkin/finite element method for the numerical solution
  of the Serre-Green-Naghdi system}.
\newblock {\em Int. J. Numer. Meth. Fluids}, 83:755--778, 2017.

\bibitem{Nwogu93}
O.~Nwogu.
\newblock Alternative form of boussinesq equations for nearshore wave
  propagation.
\newblock {\em J. Waterway Port Coastal and Oc Eng}, 119(6):618--638, 1993.

\bibitem{Pere1966}
H.~Peregrine.
\newblock Calculations of the development of an undular bore.
\newblock {\em J. Fluid Mech.}, 25:321--330, 1966.

\bibitem{Pere1967}
H.~Peregrine.
\newblock Long waves on a beach.
\newblock {\em J. Fluid Mech.}, 27:815--827, 1967.

\bibitem{Russell1844}
J.S. Russell.
\newblock Report on waves.
\newblock In {\em Report of the fourteenth meeting of the British Association
  for the Advancement of Science, York}, Plates XLVII--LVII, pages 311--390.
  London: John Murray, 1844.

\bibitem{saut2012cauchy}
J.-C. Saut and L.~Xu.
\newblock {The Cauchy problem on large time for surface waves Boussinesq
  systems}.
\newblock {\em J. Math. Pures Appl.}, 97:635--662, 2012.

\bibitem{Serre}
F.~Serre.
\newblock Contribution \`{a} l'\'{e}tude des \'{e}coulements permanents et
  variables dans les canaux.
\newblock {\em La Houille Blanche}, 8:374--388, 1953.

\bibitem{WB1999}
M.~Walkley and M.~Berzins.
\newblock {A finite element method for the one-dimensional extended Boussinesq
  equations}.
\newblock {\em Int. J. Num. Meth. Fluids}, 29(2):143--157, 1999.

\bibitem{WB2002}
M.~Walkley and M.~Berzins.
\newblock {A finite element method for the two-dimensional extended Boussinesq
  equations}.
\newblock {\em Int. J. Numer. Meth. Fluids}, 39:865--885, 2002.

\bibitem{Whitham2011}
G.B. Whitham.
\newblock {\em Linear and nonlinear waves}, volume~42.
\newblock John Wiley \& Sons, New York, 2011.

\bibitem{wu1997}
S.~Wu.
\newblock Well-posedness in sobolev spaces of the full water wave problem in
  2-d.
\newblock {\em Invent. math.}, 130(1):39--72, 1997.

\bibitem{wu1999}
S.~Wu.
\newblock Well-posedness in sobolev spaces of the full water wave problem in
  3-d.
\newblock {\em J. Amer. Math. Soc.}, 12(2):445--495, 1999.

\end{thebibliography}


\end{document}

%------------------------------------------------------------------------------
% End of journal.tex
%------------------------------------------------------------------------------
