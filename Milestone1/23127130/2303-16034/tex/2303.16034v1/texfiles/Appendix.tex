\section{Error analysis of bare GKP repeaters}
\label{app:error_analysis_bare}

We begin our error analysis by reviewing how Gaussian displacement errors of the form $\exp(\epsilon \hat q_i)$ and $\exp(\epsilon \hat p_i)$, where $\epsilon\in \mathbb R$ is the error magnitude,
propagate across $\CSUM$- and $\CPhase$-gates.
The $\CSUM$-gate, $\exp\left(-\text{i} \hat{q}_1\hat{p}_2\right)$
acts as $CX_{1,2} = \sum_{k=0}^{D-1} \ket{k}\bra{k}_1 \otimes X_2^k$ on GKP qudits,
while the $\CPhase$-gate, $\exp\left(\text{i} \hat{q}_1\hat{q}_2\right)$, 
implements $CZ_{1,2}=  \sum_{k=0}^{D-1} \ket{k}\bra{k}_1 \otimes Z_2^k$~\cite{gkp}.
Hereby, $X = \sum_{k=0}^{D-1} \ket{k+1 \text{ mod }D} \bra{k}$ and $Z=\sum_{k=0}^{D-1} (\text{e}^{2\pi \text{i}/D})^k \ket{k}\bra{k}$ denote the unitary generalizations of the qubit Pauli $X$- and $Z$-gates to the case of $D$-dimensional qudits.
It is well known that single-qudit Pauli errors are propagated across $CX$- and $CZ$-gates via
\begin{align}\label{eq:propagation_discrete}
\begin{split}
    CZ_{1,2} X_1 & = X_1  Z_2\, CZ_{1,2}, \\
    CX_{1,2} X_1 & = X_1  X_2\, CX_{1,2},  \\ \text{and } \
    CX_{1,2} Z_2 & = Z_1^\dagger Z_2\, CX_{1,2},
\end{split}
\end{align}
see, e.g., Refs.~\cite{gkp, MHKB18}.
The error propagation rules of Eq.~\eqref{eq:propagation_discrete} have their bosonic analogs: applying the Baker-Campbell-Hausdorff formula yields
\begin{align}\label{eq:propagation}
\begin{split}
\exp\left(\text{i} \hat{q}_1\hat{q}_2\right)\exp\left(\text{i} \hat{p}_1\right)
&=\exp\left(\text{i} (\hat{p}_1-\hat{q}_2)\right)\exp\left(\text{i} \hat{q}_1\hat{q}_2\right), \\
\exp\left(-\text{i} \hat{q}_1\hat{p}_2\right)\exp\left(\text{i} \hat{p}_1\right)
&=\exp\left(\text{i}( \hat{p}_1+\hat{p}_2)\right)\exp\left(-\text{i} \hat{q}_1\hat{p}_2\right), \\  \text{and } \
\exp\left(-\text{i} \hat{q}_1\hat{p}_2\right)\exp\left(\text{i} \hat{q}_2\right)
&=\exp\left(\text{i}( \hat{q}_2-\hat{q}_1)\right)\exp\left(-\text{i} \hat{q}_1\hat{p}_2\right).
\end{split}
\end{align} 

In the two repeater protocols from Sec.~\ref{sec:2way_postamp} and~\ref{sec:1way_preamp},
every repeater station is responsible for performing a Bell measurement.
This is achieved by a beam splitter, followed by two homodyne measurements.
For both of these homodyne measurements, the results are post-processed (binned) into a measurement outcome of the GKP qudit.
Errors on the GKP qudit lead to errors on the measurement outcomes.
The latter can be described by a Pauli error channel $\mathcal{P}_\text{sq}(X,\sigma^2)$, as in Eq.~\eqref{eq:pauli_shift_probability_gauss}, where the variance $\sigma^2$ comprises all Gaussian noise contributions that have propagated to the measurement device.
As discussed in Sec.~\ref{sec:noise}, we take the following  error sources into account:
\begin{itemize}
    \item Loss that arises when GKP qudits are coupled into an optical fiber. 
    The resulting coupling efficiency is denoted by $\eta_\text{c}$.
    \item Loss that arises during transmission. 
    If the traveling distance is $L_0$, the associated transmittance is given by $\eta = \exp(-L_0/L_\text{att})$, where $L_\text{att}$ is the attenuation distance.
    \item Unavoidable approximation errors of square GKP qudits. These are modeled by a Gaussian channel of variance $\sigma_\text{sq}^2$.
\end{itemize}
Since beam splitters and homodyne measurements only require passive linear optical elements,
we assume they work perfectly.
Similarly, we ignore errors stemming from Gaussian elements, i.e., from $\CSUM$- and $\CPhase$-gates.


For the two-way teleportation protocol from Sec.~\ref{sec:2way_postamp},
transmission and coupling losses lead to a Gaussian error channel with variance $\frac{1}{\eta_c \sqrt{\eta}}-1$, see Sec.~\ref{sec:loss_and_amplification}.
Furthermore, there are three GKP state preparations in the causal light cone of any given measurement. 
All in all, this amounts to a final variance of 
$\sigma_\text{2-way}^2 = 3\sigma_\text{sq}^2+\frac{1}{\eta_\text{c} \sqrt{\eta}}-1$.

For the one-way teleportation protocol from Sec.~\ref{sec:1way_preamp}, the only difference is that the Gaussian error channel arising from losses now has a variance of
$1-\eta_\text{c}\eta$, see Sec.~\ref{sec:loss_and_amplification}.
Therefore, the final variance is given by 
$\sigma_\text{1-way}^2 = 3\sigma_\text{sq}^2 + 1-\eta_\text{c}\eta$.


If $D$ is even, it is possible to directly generate a two-qudit GKP Bell pair
by applying a balanced beam splitter to two grid states~\cite{WalshePRA2020, gkp_syndrome}.
Unlike general Gaussian transformations,
this linear optical transformation does not amplify the noise.
In consequence, the above variances are improved to 
$\sigma_\text{2-way}^2 = 2\sigma_\text{sq}^2+\frac{1}{\eta_\text{c} \sqrt{\eta}}-1$ and
$\sigma_\text{1-way}^2 = 2\sigma_\text{sq}^2 + 1-\eta_\text{c}\eta$.


On the physical level, every Bell measurement is comprised of two homodyne measurements.
Errors on the measurement of one quadrature effectively propagate into $X$-errors on Bob's qudits, while those of the other quadrature lead to $Z$-errors.
By symmetry, the final probability distributions for $X$- and $Z$-errors coincide, and it suffices to compute it in one case.
Ignoring finite size effects\footnote{In principle, the measurements near the ends of the repeater line have smaller error probabilities. Ignoring this slightly underestimates performance, however, the difference is vanishingly small for a large number of repeater stations.} and potential correlations between the error probabilities of different repeater stations, we estimate the final $X$-error distribution $\mathcal{P}_\text{fin}(X) =  \mathcal{P}^{\ast N}_\text{sq}(X, \sigma^2) $ on Bob's qudit as the $N$-fold discrete convolution of $\mathcal{P}_\text{sq}(X, \sigma^2)$, where $N$ denotes the number of repeater stations.
We expect that this estimate captures the general behavior of the performance of GKP qudit repeaters.
In principle, computing this convolution can be sped up by diagonalizing the corresponding error-probability matrix~\cite{MHKB18}.
For our purposes, however, a direct implementation is sufficient.
Then, we compute the outer product $\mathcal{P}_\text{fin}(X, Z) = \mathcal{P}_\text{fin}(X) \otimes \mathcal{P}_\text{fin}(Z)$.
The secret-key rate of the repeater line, finally, is given by $\log_2(D)-H(\mathcal{P}_\text{fin}(X, Z)) = \log_2(D)-2 H(\mathcal{P}_\text{fin}(X))$.

 
\section{Error analysis of GKP repeaters with higher-level codes} 
\label{app:error_analysis_encoded}

In this appendix, we lift our error analysis from App.~\ref{app:error_analysis_bare}
to the logical level.
First, we discuss in App.~\ref{app:teleportation_encoded_analysis} the two repeater protocols from Sec.~\ref{sec:2way_postamp} and~\ref{sec:1way_preamp}.
In App.~\ref{app:half_teleportation_gkp_syndrome_placement}, 
we discuss the optimal placement of the lower-level GKP measurements for the third protocol from Sec.~\ref{sec:1way_preamp_half} and analyze its performance.

\subsection{Logical performance of GKP qudits concatenated with quantum polynomial codes}
\label{app:teleportation_encoded_analysis}
In App.~\ref{app:error_analysis_bare}, 
we showed that the error probability distribution for measurements in repeater stations is given by $\mathcal{P}_\text{sq}(X, \sigma^2)$, where $\sigma_\text{2-way}^2 = 3\sigma_\text{sq}^2+\frac{1}{\eta_\text{c} \sqrt{\eta}}-1$ and
$\sigma_\text{1-way}^2 = 3\sigma_\text{sq}^2 + 1-\eta_\text{c}\eta$ 
for the two-way and one-way teleportation protocol, respectively.
When the protocol is lifted to its logical version,
we still find the same error distribution for each of the measurements of the physical GKP qudits (of which there are $D$).
This is because $\overline{CZ} = (CZ^\dagger)^{\otimes D}$ is semitransversal for the quantum polynomial code with parameters $\llbracket D, 1,\tfrac{D+1}{2} \rrbracket_D$~\cite{Aharonov2008}.

Here, we consider a simple decoder that only corrects errors occurring on a number of qudits not more than half the distance $d=\tfrac{D+1}{2}$.
Thus, the probability that a correctable error pattern occurs at a repeater station is given by 
\begin{align} \label{eq:correctable_error_probability}
    p_\text{cor} = \sum_{k=0}^{\tfrac{d-1}{2}} \binom{D}{k} p_0^{D-k}(1-p_0)^k,
\end{align}
where $p_0 = P_\text{sq}(X^0, \sigma^2)$, as in Eq.~\eqref{eq:pauli_shift_probability_gauss}.
If the decoding attempt fails, we replace the measured state with the maximally mixed state (as a worst-case approximation).
In other words: with probability $1 - p_\text{cor}$, we insert a logical error, uniformly at random from the set $\{1,\ldots, D-1\}$.
Therefore, the error probability distribution on measurement outcomes in any repeater station is given by 
\begin{align}\label{eq:repeater_error_channel}
    P_\text{rep} (X^k) = \begin{cases} p_\text{cor} & \text { if }k=0 \\
    \frac{1}{D-1}(1-p_\text{cor}) & \text { otherwise}.
    \end{cases}
\end{align} 
If the probability of errors is so large that $p_\text{cor}<\frac{1}{D-1}(1-p_\text{cor})$, 
we replace Eq.~\eqref{eq:repeater_error_channel} by the uniform distribution.
Again, ignoring correlations between error distributions on different repeater stations,
we estimate the final error distribution of the encoded repeater line as $\mathcal{P}_\text{fin}(X, Z) = \mathcal{P}^{\ast N} _\text{rep}(X) \otimes \mathcal{P}^{\ast N} _\text{rep}(Z)$.
% and the secret-key rate of the quantum repeater line as $\log_2(D)-2 H(\mathcal{P}^{\ast N}_\text{rep}(X))$.



\subsection{Error analysis of the half-teleportation protocol for various placements of GKP syndrome measurements}
\label{app:half_teleportation_gkp_syndrome_placement}

In this appendix, we discuss how introducing additional ancilla-based measurements of lower-level GKP stabilizers can improve the performance of the one-way half-teleportation protocol with optical pre-amplification from Sec.~\ref{sec:1way_preamp_half}.
Such measurements are pictured in Fig.~\ref{fig:repeaters}~(c) of the main text.
As discussed in Sec.~\ref{sec:loss_and_amplification}, 
every transmission from one repeater station to the next is associated with a Gaussian error channel with variance $\sigma^2_\text{loss} = 1- \eta_\text{c}\eta$, where $\eta = \exp(-L_0/L_\text{att})$.
In every repeater station, all incoming GKP qudits are measured in the $p$-quadrature.
Before this, however, each GKP qudit is coupled via a physical $\CPhase^{\dagger}$-gate to a qudit in the next logical block.
Since the $\CPhase^{\dagger}$-gate spreads $p$-errors into $q$-errors, but $q$-errors are not propagated to the next mode, every error source only has a limited range.
A $p$-error that arises during one transmission, does not directly affect $p$-measurements on the qudit it occurred to, however, it propagates into a $q$-error on the subsequent GKP qudit, which alters the $p$-measurement outcome of that qudit.
Furthermore, 
a $p$-error during GKP state preparation backpropagates through the $\CPhase^{\dagger}$-gate and causes a $q$-error on the readout of the preceding GKP qudit.

In the plain version (without lower-level GKP stabilizer measurements),
errors on physical readouts (in the repeater stations) follow an error distribution $\mathcal{P}_\text{sq}(Z,  2\sigma^2_\text{loss}+3\sigma^2_\text{sq})$, where the variance takes noise from two transmissions and three GKP state preparations into account.
By introducing a lower-level GKP stabilizer measurement in every repeater station, we can correct displacement errors after a single transmission.
In this way, we effectively avoid combining the two transmission loss channels.
Instead, all Gaussian errors in one quadrature are replaced by the discrete Pauli error channel from Eq.~\eqref{eq:pauli_shift_probability_gauss}.
Such discrete qudit Pauli errors will propagate to the measurements in the usual way~\cite{MHKB18}.
Depending on where in the repeater station we place the ancilla-based GKP stabilizer measurement, the final error distribution will vary.
We discuss four options:
\begin{enumerate}[label=(\roman*)]
    \item No additional GKP stabilizer measurements are performed, see Fig.~\ref{fig:error_analysis_no_extra} for the error analysis.
    \item \emph{After} every $CZ$-gate, the (physical) target qudit is subjected to a GKP stabilizer measurement of $S_X= \exp(-\text{i}\sqrt{2\pi D} \hat p)$.
    This is achieved by preparing an ancillary GKP qudit in state $\ket{0}$, applying a $\CSUM$-gate from the ancilla to the repeater qudit, and a $p$-measurement of the ancilla GKP qudit, see Fig.~\ref{fig:error_analysis_always_after} for the error analysis.
    
    \item  \emph{Before} every $CZ$-gate, the control qudit is subjected to a GKP stabilizer measurement of $S_Z= \exp(\text{i}\sqrt{2\pi D} \hat q)$.
    This is achieved by preparing a  GKP ancilla in state $\ket{+}$, applying a $\CSUM$-gate from the repeater qudit to the ancilla, followed by a $q$-measurement of the ancilla, see Fig.~\ref{fig:error_analysis_always_before} for the error analysis.
    \item We \emph{alternate} between options (ii) and (iii), see Fig.~\ref{fig:error_analysis_alternating} for the error analysis.
\end{enumerate}
In option~(i), the error analysis from App.~\ref{app:teleportation_encoded_analysis} with $\sigma^2 =  2\sigma^2_\text{loss}+3\sigma^2_\text{sq}$ applies, see Fig.~\ref{fig:error_analysis_no_extra}.
Both in option~(ii) and~(iii), which we refer to as \emph{symmetric} placements of the GKP stabilizer measurements, it turns out that every $p$-measurement is subject to two discrete Pauli error channels as in Eq.~\eqref{eq:pauli_shift_probability_gauss}, one having variance $2\sigma^2_\text{sq}+\sigma^2_\text{loss}$ and the other one $4\sigma^2_\text{sq}+\sigma^2_\text{loss}$.
Thus, the error analysis from App.~\ref{app:teleportation_encoded_analysis} applies after we insert 
\begin{align} \label{eq:symmetric_placement_error_probs}
p_0^\text{sym} = \sum_{k=0}^{D-1} P_\text{sq}(X^k, 2\sigma^2_\text{sq}+\sigma^2_\text{loss}) P_\text{sq}(X^{D-k}, 2\sigma^2_\text{sq}+\sigma^2_\text{loss})
\end{align}
into Eq.~\eqref{eq:correctable_error_probability}.
Finally, in option~(iv) both GKP stabilizer and logical measurements are subject to Gaussian errors with variance  $3\sigma^2_\text{sq}+\sigma^2_\text{loss}$.
This time, we thus have to insert 
\begin{align} \label{eq:alternating_placement_error_probs}
p_0^\text{alt} = \sum_{k=0}^{D-1} P_\text{sq}(X^k, 3\sigma^2_\text{sq}+\sigma^2_\text{loss}) P_\text{sq}(X^{D-k}, 3\sigma^2_\text{sq}+\sigma^2_\text{loss})
\end{align}
into Eq.~\eqref{eq:correctable_error_probability}.

% In option~(i), the error analysis from App.~\ref{app:teleportation_encoded_analysis} with $\sigma^2 =  2\sigma^2_\text{loss}+3\sigma^2_\text{sq}$ applies, see Fig.~\ref{fig:error_analysis_no_extra}.
% In option~(ii), an ancilla-based measurement of $S_X= \exp(-\text{i}\sqrt{2\pi D} \hat p)$ is performed after each $CZ$-gate.
% In Fig.~\ref{fig:error_analysis_always_after}, we show how displacement errors propagate to the $p$-measurement in any given repeater station.
% By denoting the variances of initial displacement errors by $\sigma^2_{\text{in},q}$ and  $\sigma^2_{\text{in},p}$, we find that the final $p$-variance affecting the $p$-measurement is given by $\sigma^2_{\text{in},p}+2\sigma^2_\text{sq} +1-\eta$.
% Furthermore, the variances at the end of the circuit are given by $\sigma^2_{\text{out},q}=\sigma^2_\text{sq}$ and $\sigma^2_{\text{out},p} =  2\sigma^2_\text{sq}$.
% Since periodic boundary conditions apply, i.e., $\sigma^2_{\text{out},q} = \sigma^2_{\text{in},q} $ and $\sigma^2_{\text{out},p} = \sigma^2_{\text{in},p} $, 
% the continuous displacement errors arriving at the $p$-measurements have a variance of $4 \sigma^2_\text{sq} +1 - \eta$.


% we introduce a Pauli error channel where $Z$-errors are distributed according to $\mathcal{P}(Z,\sigma^2_\text{loss}+4\sigma_\text{sq}^2)$...

% % Show Figures for propagation of variances


\begin{figure}[t]
    \centering	
    \includegraphics[scale=.9]{texfiles/figures/no_extra_gkp_variances.pdf}
    \caption{Propagation of Gaussian errors for the half-teleportation protocol \emph{without} additional GKP stabilizer measurements.
    Because of periodic boundary conditions, we have 
    $\sigma^2_{\text{in},p} = \sigma^2_{\text{out},p} = \sigma^2_\text{sq}$ and 
    $\sigma^2_{\text{in},q} = \sigma^2_{\text{out},q} = 2\sigma^2_\text{sq} + 1-\eta$.
     Therefore, the variance of $q$-errors reaching the $p$-measurements is given by $\sigma^2_\text{sq}+\sigma^2_{\text{in},q} +1-\eta = 3\sigma^2_\text{sq} + 2(1-\eta)$. }
    \label{fig:error_analysis_no_extra}
\end{figure}

\begin{figure}[t]
    \centering	
    \includegraphics[scale=.9]{texfiles/figures/always_after_variances.pdf}
    \caption{Propagation of Gaussian errors for the half-teleportation protocol with additional GKP stabilizer measurements \emph{after} every $CZ$-gate.
    Because of periodic boundary conditions, we have 
    $\sigma^2_{\text{in},p} = \sigma^2_{\text{out},p} = 2\sigma^2_\text{sq}$ and 
    $\sigma^2_{\text{in},q} = \sigma^2_{\text{out},q} = \sigma^2_\text{sq}$.
     Therefore, the variance of $q$-errors reaching the $p$-measurements is given by 
     $\sigma^2_\text{sq}+\sigma^2_{\text{in},q} + 1-\eta = 2\sigma^2_\text{sq} + 1-\eta$.
     In addition to these continuous displacement errors, a discrete Pauli error channel $\mathcal{P}_\text{sq}(Z, \sigma^2_\text{GKP})$ leads to lower-level logical errors on every $X$-measurement, where $\sigma^2_\text{GKP} = 2\sigma^2_\text{sq} + \sigma^2_{\text{in},p} + 1-\eta = 4\sigma^2_\text{sq}+1-\eta$ is the variance of $q$-errors reaching the lower-level GKP stabilizer measurement.
     }
    \label{fig:error_analysis_always_after}
\end{figure}


\begin{figure}[t]
    \centering	
    \includegraphics[scale=.9]{texfiles/figures/always_before_variances.pdf}
    \caption{Propagation of Gaussian errors for the half-teleportation protocol with additional GKP stabilizer measurements \emph{before} every $CZ$-gate.
    Because of periodic boundary conditions, we have 
    $\sigma^2_{\text{in},p} = \sigma^2_{\text{out}, p} = \sigma^2_\text{sq}$ and 
    $\sigma^2_{\text{in},q} = \sigma^2_{\text{out}, q} = 2\sigma^2_\text{sq}$.
     Therefore, the variance of $q$-errors reaching the $p$-measurements is given by 
     $2\sigma^2_\text{sq}+\sigma^2_{\text{in},q} + 1-\eta = 4\sigma^2_\text{sq} + 1-\eta$.
     In addition to these continuous displacement errors, a discrete Pauli error channel $\mathcal{P}_\text{sq}(Z, \sigma^2_\text{GKP})$ leads to lower-level logical errors on every $X$-measurement, where $\sigma^2_\text{GKP} = \sigma^2_\text{sq} + \sigma^2_{\text{in},p} + 1-\eta = 2\sigma^2_\text{sq}+1-\eta$ is the variance of $p$-errors reaching the lower-level GKP stabilizer measurement. Originally, the lower-level GKP stabilizer measurement results in a discrete Pauli error channel $\mathcal{P}_\text{sq}(X, \sigma^2_\text{GKP})$, which is then propagated to a Pauli error channel $\mathcal{P}_\text{sq}(Z, \sigma^2_\text{GKP})$ in the next segment due to the $CZ$-gate.}
    \label{fig:error_analysis_always_before}
\end{figure}


\begin{figure}[t]
    \centering	
    \includegraphics[width=\textwidth]{texfiles/figures/alternating_scheme_variances.pdf}
    \caption{Propagation of Gaussian errors for the half-teleportation protocol with additional GKP stabilizer measurements at \emph{alternating} placements.
    Because of periodic boundary conditions, we have 
    $\sigma^2_{\text{in},p} = \sigma^2_{\text{out}, p} = \sigma^2_\text{sq}$ and 
    $\sigma^2_{\text{in},q} = \sigma^2_{\text{out}, q} = 2\sigma^2_\text{sq}$.
    Therefore, it turns out that the variance of $q$-errors reaching all $p$-measurements is given by 
     $\sigma^2_\text{sq}+\sigma^2_{\text{in},q} + 1-\eta = 3\sigma^2_\text{sq} + 1-\eta$.
   In addition to these continuous displacement errors, a discrete Pauli error channel $\mathcal{P}_\text{sq}(Z, \sigma^2_\text{GKP})$ leads to lower-level logical errors on every $X$-measurement, where $\sigma^2_\text{GKP} = 2\sigma^2_\text{sq} + \sigma^2_{\text{in},p} + 1-\eta 
   = 3\sigma^2_\text{sq}+1-\eta$ is the variance of errors reaching and altering lower-level GKP stabilizer measurements.
    }
    \label{fig:error_analysis_alternating}
\end{figure}


\begin{figure} \centering
	\subfloat[]{\label{fig:skr_VS_L_with_L=2000km_20_squeezing0.999_coupling_noerasure_twoway}\includegraphics[height=0.275\textwidth]{texfiles/figures/half_teleportation_different_schemes_20dB.pdf}}
    \hspace{5mm}
	\subfloat[]{\label{fig:skr_VS_L_with_L=2000km_30_squeezing0.999_coupling_noerasure_twoway_app}\includegraphics[height=0.275\textwidth]{texfiles/figures/half_teleportation_different_schemes_30dB.pdf}}\\
	\caption{Lower bound on the SKR per logical channel use, $\log_2(D)-H(\mathcal{P})$, normalized by the number $N$ of repeater stations for the one-way half-teleportation protocol and various placements of lower-lever GKP stabilizer measurements.
    We plot SKR$/N$ as a function of the repeater spacing $L_0$ for a repeater line of total length $L = N  L_0 = 2000\unitspace\text{km}$, coupling efficiencies $\eta_\text{c}=99.9\%$, and squeezing levels of (a) $s_\text{GKP} = 20\unitspace\text{dB}$ or (b) $s_\text{GKP}=30\unitspace\text{dB}$.
	}
	\label{fig:figappendix}
\end{figure}

In Fig.~\ref{fig:figappendix}, we show how the placement of GKP stabilizer measurements influences the performance of the half-teleportation protocol, using the exact same setting as in Fig.~\ref{fig:figc} of the main text.
Overall, the situation is very similar to that in Fig.~\ref{fig:figc}: for $s_\text{GKP}=20\unitspace\text{dB}$ in Fig.~\ref{fig:figappendix}~(a), only the $\llbracket5,1,3\rrbracket_D$-code (green) offers a nonzero SKR, whereas  for $s_\text{GKP}=20\unitspace\text{dB}$ in Fig.~\ref{fig:figappendix}~(b) also the $\llbracket13,1,7\rrbracket_D$-code (red) and the  $\llbracket17,1,9\rrbracket_D$-code (black) have the potential to distribute secret keys.
We see in Fig.~\ref{fig:figappendix} that  an alternating placement of GKP stabilizer measurements (solid curves) leads to the highest values of SKR$/N$.
For both option~(ii) and~(iii), the symmetric placements (dotted curves) are governed by Eq.~\eqref{eq:symmetric_placement_error_probs}, and therefore lead to the same performance.
We see that not performing any additional GKP stabilizer measurements (dash-dotted curve) 
leads to the lowest performance, which is easily explained by the large variance $2\sigma^2_\text{loss}+3\sigma^2_\text{sq}$.
The other options break the term $2\sigma^2_\text{loss}$ and, therefore, perform better.
For the symmetric placement, the bottleneck is posed by the term $4\sigma^2_\text{sq}$ in Eq.~\eqref{eq:symmetric_placement_error_probs}, which is worse than 
$3\sigma^2_\text{sq}$ in Eq.~\eqref{eq:alternating_placement_error_probs} for the alternating placement.
This explains why the latter performs best.
For a large squeezing value of $s_\text{GKP}=30\unitspace\text{dB}$, the difference between $3\sigma^2_\text{sq}$ and $4\sigma^2_\text{sq}$ is negligible, which causes a nearly perfect overlapping of the dotted and solid curves in Fig.~\ref{fig:figappendix}~(b).

Since the alternating placement of GKP stabilizer measurements has the best performance, we have assumed this option for the one-way half-teleportation protocol throughout the main text of this paper.

\section{Author contributions}

DM initiated the project as a whole and exploring the idea of concatenating GKP qudits with polynomial codes.
FS designed the repeater protocols, derived the analytical model, performed the numerics, and created the figures.
FS and DM designed the study, interpreted the results, and wrote the manuscript. 
PvL supported research and development and helped preparing
the manuscript.

