GKP codes encode a $D$-dimensional qudit within the Fock space $\mathcal{F}$ of a quantum mechanical harmonic oscillator~\cite{gkp}.
We denote the annihilation operator of the oscillator by $\hat{a}$ and its quadrature operators by $\hat{p}=\tfrac{\text{i}}{\sqrt{2}}(\hat{a}^\dagger-\hat{a})$ and $\hat{q}=\tfrac{1}{\sqrt{2}}(\hat{a}^\dagger+\hat{a})$.
For simplicity, we focus in this paper on the square GKP code, which is defined as the $D$-dimensional subspace of $\mathcal{F}$ that is invariant under the action of $S_X = \exp(-\text{i} \sqrt{2\pi D}\hat{p})$ and $S_Z = \exp(\text{i} \sqrt{2\pi D} \hat{q})$.
By repeated non-destructive measurements of the stabilizer operators $S_X$ and $S_Z$, followed by appropriate displacement operations (or, at least, through tracking of the corresponding generalized Pauli frame), one can enforce the state of the oscillator to (effectively) remain 
in the GKP code space.
Since the logical Pauli operators of the square GKP code are given by
${X}=\exp(-\text{i} \sqrt{\tfrac{2\pi}{D}}\hat{p})$ and  ${Z}=\exp(\text{i} \sqrt{\tfrac{2\pi}{D}}\hat{q})$,
it is thereby possible (in the idealizing limit of perfect GKP states) to correct arbitrary displacement errors that are smaller than $\sqrt{{\pi}/{2D}}$ in magnitude.
To implement two-qudit gates between GKP qudits, 
one can utilize common two-mode Gaussian gates.
For example, on the level of GKP qudits, the bosonic  $\CSUM$-gate, $\exp\left(-\text{i}\hat{q}_1\hat{p}_2\right)$, acts as a two-qudit controlled-$X$ gate, $CX = \sum_{k=0}^{D-1} \ket{k}\bra{k}_1 \otimes X^k_2$.
A similarly defined $CZ$-gate is implemented by means of a $\CPhase$-gate, $\exp\left(\text{i}\hat{q}_1\hat{q}_2\right)$.