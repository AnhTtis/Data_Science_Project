Quantum technologies rely on the availability of precisely controllable quantum systems, e.g., qubits, which can be realized with various physical implementations.
In 2000, Gottesman, Kitaev, and Preskill (GKP) proposed a method to encode finite-dimensional quantum systems (qudits) into quantum-mechanical harmonic oscillators~\cite{gkp}.
More recent theoretical developments include further proposals and assessments of GKP state preparation with superconducting devices~\cite{PhysRevX.11.011032, PRXQuantum.2.020101}.
After years of experimental progress, GKP qubits finally have been demonstrated in superconducting microwave cavities~\cite{Campagne-Ibarcq2020, eickbusch_fast_universal_2022, sivak_real_time_2022} and in the harmonic motion of ions~\cite{Fluehmann2019,deNeeve2022}.

In the optical domain, on the other hand, preparing GKP states is notoriously difficult.
The main problem is that reliable and strong nonlinearities are required but not readily available. 
In one approach, Gaussian Boson Sampling~\cite{opticalGKP1, opticalGKP2}, 
one exploits that measurements can induce nonlinear effects.
Here, Gaussian resource states are combined via passive linear optics and partially read out via photon-number resolving measurements. 
In this way, high-quality optical GKP states can be obtained, albeit only probabilistically. 
Gaussian Boson Sampling requires detectors with a sufficiently high level of photon-number resolution as well as increasingly complex linear circuits~\cite{opticalGKP1, opticalGKP2}. 
To shift the experimental burden associated with this, 
alternative approaches have been proposed~\cite{FukuiPRL2022, TakaseGaussBreeding2022}.
If non-Gaussian resource states or non-Gaussian optical elements are available, 
a recursive application of short linear circuits and homodyne measurements is sufficient for the preparation of GKP states~\cite{VasconcelosOL2010, WeigandPRA2018, BudingerCubic2022}.
There also exist alternatives which do not rely on measurements at all~\cite{HastrupNJPQPRA2021, BudingerCubic2022}.
A final option is to combine photon-subtraction- and homodyne-based elements to convert many-mode Gaussian cluster states into non-Gaussian few-mode states, which can be further processed into GKP states~\cite{EatonQuantum2022}.
Such an approach is compatible with measurement-based, continuous-variable quantum computation~\cite{BudingerCubic2022,PhysRevLett.97.110501}.

While the best method for creating optical GKP states has not yet been identified, it is safe to assume that their physical realization will require extremely sophisticated experimental procedures.
Once such technology is available, however, 
it will be comparatively straightforward to extend it to higher-dimensional GKP qudits and to concatenated multi-qubit or -qudit GKP codes.
For example, multiple GKP qubits can be entangled via Gaussian operations~\cite{gkp}.
Furthermore, ordinary beam splitters enable the generation of certain collective GKP ancilla states such as  Bell states with GKP qubits~\cite{WalshePRA2020} or qudits~\cite{gkp_syndrome}, as well as the collective detection of their 
error syndromes~\cite{gkp_syndrome}.
To guide such future experiments, we find it meaningful to investigate the performance of advanced multi-qudit GKP protocols in the realm of quantum communication.

The GKP encoding enables the correction of small displacement errors of the oscillator's quadratures, in particular, those that originate from typical Gaussian error channels such as amplitude damping or photon loss. 
However, large displacement errors cannot be avoided completely, especially for realistic, finitely-squeezed GKP states. 
This can cause misidentification of error syndromes, which leads to discrete logical errors on the affected GKP qudits.

In order to correct such errors, a higher-level quantum error-correcting code (QECC) can be employed to encode a few logical qudits into a larger number of physical GKP qudits~\cite{PRXQuantum.2.020101, gkp_analoginfo, FT_surface_gkp, toric-gkp, PhysRevX.8.021054, biased_gkp, GKP_LDPC}.
Hereby, the error correction capability of the higher-level QECC can benefit from analog information in the single-qudit GKP syndrome measurements~\cite{gkp_analoginfo, FT_surface_gkp, toric-gkp, PhysRevX.8.021054}. 
In order to satisfy the quantum singleton bound $n-k\geq2(d-1)$, every QECC with code parameters $\llbracket n,k,d \rrbracket$ must trade off the number of correctable (arbitrary) single-qudit errors against the number of physical qudits per logical qudit, which are given by $\lfloor(d-1)/2\rfloor$ and $n/k$, respectively~\cite{QMDS, PhysRevA.55.900, Calderbank1998}. 
An optimal trade-off is obtained by those QECCs that meet the quantum singleton bound with equality and are called maximum distance separable (MDS) codes.
While, for qubits, the only~\cite{qubit_QMDS} nontrivial (i.e., $d\geq3$ and $k\geq1$) MDS code encodes one logical qubit into five physical qubits~\cite{PhysRevLett.77.198},
there is a plethora of MDS codes for higher-dimensional qudits.
Such QECCs are explicitly available in the form of {quantum polynomial codes}, which exist for every qudit dimension being a prime power~\cite{PhysRevLett.83.648, Aharonov2008, Ketkar2006, CrossPhd}.

Currently, experimental realizations of long-distance quantum communication protocols 
are limited by the rapid decay of photonic signals that are sent through optical fibers.
This process is formally described by a pure-loss bosonic channel $\mathcal{L}(\eta)$, which arises from mixing the bosonic signal mode with an environmental mode in the vacuum state using a beam splitter with transmittance $\eta$.
In the long-distance limit of $\eta \rightarrow 0$, 
the secret-key capacity of the single-mode pure-loss channel scales linearly with $\eta$~\cite{TGW},
more precisely,
it is given by $-\log_2(1-\eta) \approx  1.44 \,\eta$~\cite{PLOB}.
In consequence, the secret-key rate of point-to-point quantum key distribution (QKD) is exponentially suppressed in the length $L$ of an optical fiber, which typically has a transmittance of  $\eta = \exp(-L/22\unitspace\text{km})$.

To overcome this problem,
quantum repeaters have been proposed~\cite{quantumrepeater_duerer}.
By introducing repeater stations, a long channel is split into multiple shorter ones.
To cope with the loss, 
different strategies have been conceptualized and, subsequently, been classified into three so-called generations of quantum repeaters~\cite{repeater_generations}.
These generations fundamentally differ in their speed of operation 
and in the level of technological maturity required for their realization.

\emph{First-generation quantum repeaters} are based on heralded, probabilistic entanglement distribution~\cite{quantumrepeater_duerer}. 
Once a Bell pair is successfully distributed between two neighboring repeater stations, it is stored in local quantum memories, where it resides until a second Bell pair, which connects the two repeater stations to a third one, is created.
Whenever two parts of different Bell pairs are present in a single repeater station, entanglement swapping can be executed, which results in a single Bell pair 
ranging over a larger distance.
This process is repeated until a long-distance Bell pair is shared between Alice and Bob.
In addition to channel loss, unavoidable operational gate and storage errors pose a challenge for quantum repeaters. 
To cope with such errors, first-generation quantum repeaters employ nested entanglement purification \cite{entanglement_purification}, a probabilistic protocol for the distillation of multiple low-fidelity Bell pairs into a smaller number of states with higher fidelities,
involving two-way classical communication.
In the worst case, entanglement purification has to be performed across the total distance $L$ of the entire quantum repeater chain, which slows down the achievable repetition rate to $c/L$ or less, where $c = 2.14\times 10^{8}\unitspace\tfrac{\text{m}}{\text{s}}$ is the speed of photons in fiber (for both classical and quantum signaling).

To avoid this slow-down, \emph{second-generation quantum repeaters} \cite{repeater_with_encoding} replace entanglement purification by QECCs for the local memories.
With this modification in place, the rate bottleneck is now posed by classical communication between neighboring repeater stations, which are separated by a distance of $L_0$.
Only after a failed entanglement distribution attempt has been heralded, the quantum memories can be freed up for the next attempt.
Therefore, the improved upper bound on the repetition rate is now given by $c/L_0$, which is typically on the order of $ 1 \unitspace\text{kHz}$ for $L_0\sim 100\unitspace\text{km}$  to
$1\unitspace\text{MHz}$ for $L_0 \sim 100\unitspace\text{m}$.
The only possibility to speed up the classical two-way communication
is to reduce $L_0$, i.e., to invest in a larger number of, realistically imperfect, faulty repeater stations whose quantum information must be consistently protected by the QECC.

Finally, \emph{third-generation quantum repeaters} enable ultrafast quantum communication as they dispense with the temporary storage of quantum information and classical two-way communication altogether~\cite{PhysRevLett.104.180503, PhysRevLett.112.250501, PhysRevA.89.032335}.
Instead, these repeaters employ QECCs to correct both channel losses and operational errors.
The repetition rates in this case are only limited by the speed of state preparations, local gate operations, and measurements in the individual repeater stations.
Whereas the preparation of QECC-encoded multi-photon states typically relies on some form of light-matter interaction, all other components of a third-generation quantum repeater can, in principle, be realized in an all-optical fashion~\cite{fabian_repeater_prl, fabian_repeater_pra, logicalBMefficiencies, AzumaNC2015, LeePRA2019}.

In this paper, we theoretically analyze the performance of third-generation quantum repeaters based on optical GKP qudits.
Our investigation also includes cases where the GKP code is concatenated with a higher-level QECC.
Here, we focus on quantum polynomial codes that previously have been considered in combination with multi-mode and Fock-encoded qudits~\cite{Muralidharan_2017, PhysRevA.97.052316, MHKB18,  MHKB19}.
For GKP-encoded states, similar performance studies have only been carried out in the special case of qubits~\cite{PhysRevA.63.022309, roz2020quantum, fukui2020alloptical}.
Our work thus closes the gap between these two approaches to a certain extent as it offers a treatment of the remaining case of GKP qudits.
The consideration of qudits, which can transmit more quantum information  per channel use than qubits, in the context of GKP and third-generation quantum repeaters is particularly attractive due to the existence of hardware-efficient GKP-qudit operations and syndrome extraction routines based on linear-optical elements alone.
In this way, the only fundamental experimental challenge that remains is to provide a supply of suitable multi-mode GKP ancilla states, a problem that can be tackled independently.

This paper is structured as follows.
In Sec.~\ref{sec:2}, we describe the details of our study: 
we begin with introducing the repeater protocols under investigation in Secs.~\ref{sec:protocol} and~\ref{sec:exp_repeaters} 
and proceed with 
our noise model in Sec.~\ref{sec:noise}.
In Sec.~\ref{sec:3}, we present the secret-key rates obtainable with the different GKP qudit repeater protocols and discuss the influence of various experimental parameters.
Finally, in Sec.~\ref{sec:4}, we summarize our results and conclude with a recommendation of the most promising quantum repeater protocol based on GKP qudits as identified in this work.