\subsection{Noise model}
\label{sec:noise}

GKP codes are designed to correct displacement errors.
As we review next, this allows us to model photon loss and imperfect GKP state preparation with incoherent Gaussian displacement channels.
For our error analyses, it will suffice to keep track of their variances. 

\subsubsection{Transmission loss and coupling inefficiencies}
\label{sec:loss_and_amplification}
 
The bosonic pure-loss channel $\mathcal{L}(\eta)$ is commonly used to model fiber loss and coupling inefficiencies in quantum communication protocols~\cite{RevModPhys.77.513, RevModPhys.84.621}.
When $\mathcal{L}(\eta)$ is applied to a GKP state, its quadratures are damped, which shrinks the GKP lattice.
To rescale the lattice, one has to amplify the signal.
Depending on whether this amplification is carried out optically before $\mathcal{L}(\eta)$, optically after $\mathcal{L}(\eta)$, or classically after the measurement of a quadrature operator, the effective error channel on the GKP subspace is altered.

For the one-way protocols in Secs.~\ref{sec:1way_preamp} and~\ref{sec:1way_preamp_half}, we
consider the usage of an optical amplification channel $\mathcal{A}(\eta^{-1})$.
If $\mathcal{A}(\eta^{-1})$ is applied \emph{after} $\mathcal{L}(\eta)$, 
the result is a Gaussian displacement channel with variance $\sigma^2 = (1-\eta)/\eta$~\cite{PhysRevA.63.022309}.
If $\mathcal{A}(\eta^{-1})$ is applied \emph{before} $\mathcal{L}(\eta)$, however, the variance is improved to $\sigma^2 = 1-\eta$, as this avoids amplifying noise that occurs during transmission~\cite{gkp_capacity}.
In our analyses of the one-way protocols, we will therefore consider the latter strategy.
Furthermore, we will assume a total transmittance of
$\eta_\text{tot}=\eta_\text{c}\exp\left(-{L_0}/{L_\text{att}}\right)$,
where $\eta_\text{c}$ denotes the efficiency for coupling into the fiber  ($\eta_\text{c}=0.99$ unless stated otherwise) and $L_\text{att}=22\unitspace\text{km}$ is the attenuation length.

For the two-way teleportation-based protocol in Sec.~\ref{sec:2way_postamp}, 
it is possible and beneficial to replace $\mathcal{A}(\sqrt{\eta}^{-1})$
with a classical amplification of the measured signal. 
Effectively, this turns the loss into a Gaussian error channel with variance
$\sigma^2={1/\sqrt{\eta_\text{tot}}} -1 $~\cite{fukui2020alloptical},
where $\eta_\text{tot}=\eta_\text{c}^2\exp\left(-{L_0}/{L_\text{att}}\right)$ takes into account that,
in a two-way protocol, two signals are coupled into the fiber.

\subsubsection{Approximate GKP state generation}
The second, important noise contribution arises during the preparation of GKP states.
In position basis, the state vector of an ideal square GKP qudit takes the form
\begin{equation}
    \ket{j}=\sum_{k\in\mathbb{Z}}\ket{\hat{q}=\sqrt{\frac{2\pi}{D}}\left(j+D  k\right)}\,,
\end{equation}
where $j\in\{0,\ldots, D-1\}$ labels a computational basis state.
These ideal states are unphysical as they are neither normalizable nor superpositions of finite-width peaks. To describe normalizable, physical instances of GKP states and eventually also predict real-world experimental performances, we instead consider approximate GKP states for
which multiple realizations have been proposed that are essentially\footnote{The state given in Eq.~\eqref{eq:GKP_def1} is not symmetric under exchange of position and momentum. 
However, this state can be squeezed by a factor of $\sqrt{1+\kappa^2\Delta^2}$ to obtain the parameterization given in Eq.~\eqref{eq:GKP_def2}.} equivalent~\cite{gkp, gkp_threshold_prl, approximimate_gkp}.
Normalizability can be restored using an overall slowly decaying Gaussian envelope and the delta peaks can be approximated  with (a still infinite number of) highly squeezed Gaussian peaks.
This results in approximate GKP states of the form
\begin{align}
      \ket{\Tilde{j}} \propto\sum_{k\in \mathbb{Z}}
      \exp\left({- \frac{\pi \kappa^2}{D}(j+Dk)^2}\right)
      \int_{-\infty}^\infty \text{d}q \, 
      \exp\left({-\frac{(q-\sqrt{\frac{2\pi}{D}}(j+Dk))^2}{2\Delta^2}}\right)
      \ket{\hat{q}=q},\label{eq:GKP_def1}
\end{align}
where $\Delta$ and $\kappa$ are squeezing parameters corresponding to the peaks' width in position and momentum representation, respectively.
Alternatively, $\vert{\Tilde{j}}\rangle$ can be interpreted as an ideal GKP state $\ket{j}$ to which coherent Gaussian displacements have been applied, i.e.,
\begin{align}
\ket{\Tilde{j}}
    \propto  \int_{\mathbb{R}^2} {\text{d}u\, \text{d}v} \,
    \exp\left(-\frac{1}{2}\left(\frac{u^2}{\gamma^2}+\frac{v^2}{\delta^2}\right) 
    + \text{i}\left(\frac{-u\hat{p}+v\hat{q}}{\sqrt{2}}\right) \right)
  \ket{j},\label{eq:GKP_def2}
\end{align} 
where the squeezing parameters $\gamma$ and $\delta$ are in one-to-one correspondence to $\Delta$ and $\kappa$, see Thrm.~1 in Ref.~\cite{approximimate_gkp}.
In this work, we only consider the symmetric case of $\gamma=\delta$.
As a further simplification, we assume incoherent Gaussian displacements with variance $\sigma_{\mathrm{sq}}^2$,
which can be understood as a twirling-approximation~\cite[App.~A]{FT_surface_gkp}. 
Numerical simulations confirm that such an approximation does not overestimate the approximate GKP state's fidelity~\cite{gkp_teleportation_twirling}.
Following Refs.~\cite{FT_surface_gkp,gkp_threshold_prl,kosuke_threshold}, we define the \emph{squeezing parameter} (given in dB),
\begin{align} \label{eq:squeezing_parameter}
     s_\text{GKP} = -10\log_{10}\left(\frac{\sigma^2_\text{sq}}{\sigma^2_\text{vac}}\right), 
\end{align} 
where $\sigma^2_\text{vac}=1/2$ denotes the quadrature variance of the vacuum state.

By means of a higher-level QECC, it is possible to concatenate multiple approximate GKP qudits, each of which is modeled by an ideal GKP state followed by Gaussian squeezing errors, into a single logical qudit.
The corresponding unitary encoding circuit may redistribute the error probabilities between the modes, which in principle leads to correlated errors~\cite{MHKB18}.
The resulting error probabilities have a complicated dependence on the selected encoding circuit, 
thus, they cannot be easily captured in full generality in our analytical model.
Therefore, we leave such details for future work.
For the purpose of the present investigation, we are satisfied with a noise model, where unphysical, ideal GKP states are first encoded using a higher-level QECC and, afterward, physicality is restored by 
applying Gaussian squeezing channels i.i.d.~to each qudit, as motivated above. 


\subsubsection{Converting Gaussian noise into Pauli errors}
\label{sec:noise_conversion}


The purpose of the GKP error-correction step shown in Fig.~\ref{fig:repeaters}
is to discretize the continuous displacement errors that build up on the GKP qudits.
%
In general, a single-qudit Pauli error channel is completely described by its joint error probability distribution of $X$- and $Z$-errors~\cite{MHKB18}.
We denote such a distribution by
\begin{align}\label{eq:error_probability_distribution}
	\mathcal{P}(X,Z)=\begin{pmatrix}
	P(X^0,Z^0)&\dots  &P(X^0,Z^{D-1})  \\ 
	\vdots	& \ddots  & \vdots  \\ 
	P(X^{D-1},Z^0)& \dots & P(X^{D-1},Z^{D-1})
	\end{pmatrix} .
\end{align} 
Let us calculate, for a square-lattice GKP qudit, the Pauli error channel that results from a Gaussian noise channel with zero mean and a covariance matrix $\Sigma_\text{sq}=\sigma^2 \mathbb{I}$ (with respect to $q$ and $p$). 
We find that $X$- and $Z$-errors are independent because the same is true for the two Gaussian random variables describing $q$- and $p$-shifts.
In other words, the matrix $\mathcal{P}_\text{sq}(X, Z)=\mathcal{P}_\text{sq}(X)\otimes\mathcal{P}_\text{sq}(Z)$ factors into the  outer product of the error probability vectors that store the marginal distributions of $X$- and $Z$-errors. 
By symmetry of the square lattice, we have $\mathcal{P}_\text{sq}(X)=\mathcal{P}_\text{sq}(Z)$.
The probability to suffer $k\in\{0,\ldots, D-1\}$ shifts can be expressed as
\begin{align} \label{eq:pauli_shift_probability_gauss}
	P_\text{sq}(X^k,\sigma^2)&=\sum_{j\in\mathbb{Z}}\int_{\sqrt{\frac{2\pi}{D}}(jD+k-\frac{1}{2})}^{\sqrt{\frac{2\pi}{D}}(jD+k+\frac{1}{2})}\frac{1}{\sqrt{2\pi \sigma^2}}\exp\left(-\frac{q^2}{2\sigma^2}\right)dq\\
	&=\sum_{j\in\mathbb{Z}}\frac{1}{2}\left(\text{erf}\left(\sqrt{\frac{2\pi}{D}}\frac{jD+k+\frac{1}{2} }{\sigma}\right)-\text{erf}\left(\sqrt{\frac{2\pi}{D}}\frac{jD+k-\frac{1}{2} }{\sigma}\right)\right)\,,\nonumber
\end{align}
where $\text{erf}(x)=\frac{2}{\sqrt{\pi}}\int_{0}^x \exp(-q^2)dq$ is the error function.
For our purposes, it is sufficient to keep only the three terms with $|j|\le 1$.