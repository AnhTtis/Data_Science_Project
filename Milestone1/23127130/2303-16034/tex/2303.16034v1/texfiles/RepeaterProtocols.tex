\subsection{Repeater protocols} \label{sec:protocol}

Since GKP qudits can be encoded into photons, which are the ideal carriers of "flying" quantum information propagating at maximal speed,  they have been envisioned in the context of quantum communication~\cite{gkp_capacity, roz2020quantum, fukui2020alloptical}.
In this paper, we investigate certain quantum communication protocols that only require qudit Clifford operations and generalized Pauli measurements \cite{qudit_clifford}, which can be simply realized with GKP qudits by means of Gaussian optics and homodyne detection, respectively. 
More precisely, we analyze and compare the performance of three third-generation quantum repeater protocols introduced in the following subsections.
For each protocol, the term ``qudit'' may either refer to a bare (physical) GKP qudit or to an ensemble of multiple GKP qudits encoding a single (logical) qudit using a higher-level QECC, in particular, in combination with Knill's error-correction-by-teleportation procedure~\cite{gkp_syndrome, Knill_teleportation}.
Even in the absence of a higher-level QECC, 
our protocols represent instances of error-corrected (third-generation) quantum repeaters, as the availability of GKP syndrome information facilitates the correction of displacement errors to a certain extent.

 \begin{figure} \centering
 	\subfloat[]{\includegraphics[width=0.7\linewidth]{texfiles/figures/two_way_figure.pdf}}\\
 	\vspace{5mm} 
 	
 	\subfloat[]{\includegraphics[width=0.7\linewidth]{texfiles/figures/one_way_preamp_figure.pdf}}\\
 	\vspace{5mm} 
 	
 	\subfloat[ ]{\includegraphics[width=0.8\linewidth]{texfiles/figures/daniel_gkp_scheme.pdf}}
\caption{Unit cells of the quantum repeater protocols considered in this work.
The transmittance $\eta= \exp(-L_0/L_\text{att})$ of the bosonic pure-loss channel $\mathcal{L}(\eta)$ is exponentially suppressed in the distance $L_0$ between adjacent repeater stations (dashed blue boxes).
Here, the qudits can be either individual GKP qudits or logical qudits that are comprised of multiple GKP qudits by means of a higher-level $\llbracket n,1,d\rrbracket_D$ QECC.
\textbf{(a)} In the \textbf{two-way teleportation protocol}, 
every repeater station prepares two qudits in the maximally entangled state
$\ket{\Phi} = \tfrac{1}{\sqrt{D}}\sum_{k=0}^{D-1} \ket{k,k}$.
One of the two qudits is sent forward and the other one backward. %through the quantum repeater chain. 
After propagating a distance of $L_0/2$, at which each physical mode has been subject to a loss channel $\mathcal{L}(\sqrt{\eta})$,
a Bell measurement (BM) is performed.
%
\textbf{(b)} Also in the \textbf{one-way teleportation protocol}, two qudits are prepared in state $\ket{\Phi}$. 
In contrast to (a), only one of the qudits is sent to an adjacent repeater station.
To compensate for loss, a quantum-limited amplification channel 
$\mathcal{A}(\eta^{-1})$ with gain $\eta^{-1}$ is applied to each of the physical GKP modes.
After propagating a distance of $L_0$, a BM combines the forward-moving qudit with the stationary qudit of the subsequent repeater station.
%
\textbf{(c)} The \textbf{one-way half-teleportation protocol} is a GKP-adaptation of a previously-studied discrete-variable protocol~\cite{MHKB19}.
Here, we add measurements to convert displacement errors into Pauli errors.
Overlined ancilla states represent codewords of the higher-level $\llbracket n,1,d \rrbracket_D$ QECC, 
while ancilla states without overscore stand for GKP codewords. 
The $\overline{CX}$-gates correspond to transversal $\CSUM$-gates and the $\overline{CZ}$-gates corresponds to semi-transversal $\CPhase$-gates.
Measurements of $q$ and $p$ denote the measurement of the position and momentum quadrature, respectively. Loss and amplifier channels are again to be understood to act individually and independently on the physical GKP modes.}
 	\label{fig:repeaters}
 \end{figure}

 
\subsubsection{Two-way teleportation protocol with classical post-amplification}
\label{sec:2way_postamp}

The first of the three quantum repeater chains under investigation is portrayed in Fig.~\ref{fig:repeaters}~(a).
For this protocol, every repeater station prepares a pair of qudits in a (logical) Bell state.
One of the qudits is sent in the direction of the next repeater station, while the other one is sent backward.
In the middle between two neighboring repeater stations, the forward- and backward-propagating qudits are joined in a (logical) Bell measurement, which is implementable on the physical level with (transversal) beam splitters and two homodyne detectors per physical Bell measurement~\cite{gkp_syndrome}.
During the transmission from the repeater stations to the central Bell measurement apparatus,
the states of the qudits are altered due to the finite transmittance of the optical fiber channel. For the general case of many physical qudits representing one logical qudit, the optical loss channels act individually and independently (i.i.d.) upon the different modes of the physical multi-mode state that propagates through each fiber segment. 
To facilitate a direct comparison with the other protocols, we denote the channel transmittance by $\sqrt{\eta}= \exp(-L_0/2L_\text{att})$, as the relevant length of the fiber is given by $L_0/2$ here.
Throughout this paper, $L_0$ denotes the distance between two adjacent repeater stations,
and $L_\text{att}=22\unitspace\text{km}$ is the attenuation length of a typical fiber at the telecommunication wavelength of $1550\unitspace\text{nm}$.
In order to compensate for the loss-induced state change (with damped quadrature amplitudes), the classical measurement signal of the Bell measurements needs to be correspondingly amplified by a factor of $\sqrt{\eta}^{-1}$ before decoding the GKP syndrome.
Overall, this protocol produces an imperfect Bell pair ranging from one end of the repeater chain to the other.
Note that classical communication is only needed for post-processing and, 
therefore, it does not slow down the repetition rates of this protocol. 
Further note that for the case of a logical qudit composed of many physical qudits, classical post-amplification is performed individually for each physical Bell measurement to obtain the syndrome of the higher-level QECC~\cite{gkp_syndrome}.

\subsubsection{One-way teleportation protocol with optical pre-amplification} 
\label{sec:1way_preamp}

As a modification of the protocol from Sec.~\ref{sec:2way_postamp}, we also consider a quantum repeater chain where the Bell measurements are executed within the repeater stations, see Fig.~\ref{fig:repeaters}~(b).
Here, only one qudit per Bell pair is transmitted through the fiber channel.
This time, the transmittance is given by $\eta= \exp(-L_0/L_\text{att})$ because the traveling distance of the photons now covers a full repeater segment, i.e., twice the distance as in the previous scenario.
To cope with the fiber losses, an optical pre-amplification channel $\mathcal{A}\left(\eta^{-1}\right)$
is i.i.d.~applied to each (physical) GKP mode before it is sent through the fiber; this step replaces the classical post-amplification of the measurement signal from Sec.~\ref{sec:2way_postamp}.


\subsubsection{One-way half-teleportation protocol with optical pre-amplification}
\label{sec:1way_preamp_half}

The utilization of a Bell measurement (protocols described in Sec.~\ref{sec:2way_postamp} and Sec.~\ref{sec:1way_preamp}) provides GKP syndrome information for both quadratures.
This facilitates the correction of displacement errors on the level of the (physical and logical) GKP qudits.
%
For the final repeater chain under consideration, 
on the other hand,
every repeater station is responsible for preparing and measuring only a single logical GKP qudit, see Fig.~\ref{fig:repeaters}~(c).
This protocol has two core components.
First, a lower-level GKP error correction converts naturally occurring Gaussian displacement errors into Pauli errors on the physical qudits, see Sec.~\ref{sec:noise_conversion}.
Second, a higher-level QECC is utilized to cope with the resulting Pauli errors.
At the start of the repeater chain, Alice prepares two higher-level logical qudits in the state $\ket{\overline +} =   \sum_{k}  \vert{\overline{k}}\rangle/\sqrt{D}$ and entangles them with a logical $\overline{CZ}$-gate.
Since we restrict ourselves to quantum polynomial codes, the $\overline{CZ}$-gate admits a semi-transversal implementation with favorable error-spreading properties~\cite{Aharonov2008}.
Alice stores one of the logical qudits and to the second one, she applies a quantum-limited amplifier with gain $\eta^{-1}$ to each of the physical GKP modes before she sends them jointly through a lossy fiber of transmittance $\eta$ to the first repeater station, where the incoming logical qudit is entangled with a new logical qudit in state $\ket{\overline{+}}$.
A subsequent destructive, (physical) quditwise $p$-measurement
effectively transfers the encoded quantum information onto the next qudit and simultaneously delivers syndrome information involving $X$-stabilizers.
These steps are then repeated at every repeater station.
Besides yielding higher-level $X$-syndromes, the $p$-measurements are also responsible for providing lower-level GKP syndrome information $p \mod \sqrt{{2\pi}/{D}}$.
The physical $CZ$-gates propagate Gaussian $p$-errors on one mode into $q$-errors on the next one.
To prevent these $q$-errors from merging with $q$-errors that occur at the subsequent transmission, we introduce an additional ancilla-based GKP syndrome measurement in every repeater station.
This can be done in multiple ways, as discussed in App.~\ref{app:half_teleportation_gkp_syndrome_placement}.
To complete the protocol, all measurement results are communicated to Bob, who applies a suitable correction operator depending on the measurement outcomes~\cite{MHKB18}. 
Assuming $N$ is even and in the absence of errors, this protocol is equivalent to $N/2$ teleportation subroutines spread over $N+1$ different laboratories. 
For this reason, we refer to this protocol as \emph{half-teleportation}.


\subsection{Some comments on potential realizations of qudit repeaters}\label{sec:exp_repeaters}
To compensate for fiber loss, it is crucial to amplify the signal.
For the two protocols in Secs.~\ref{sec:2way_postamp} and~\ref{sec:1way_preamp}, one may opt between optical pre-amplification and classical post-amplification.
For the {half-teleportation} protocol in Sec.~\ref{sec:1way_preamp_half},
on the other hand, optical pre-amplification is the only option.
This is because the GKP qudits need to be correctly scaled, i.e., they need to be in the GKP code space up to a displacement,
before the $CZ$-gate is applied.
Since classical post-amplification can be carried out conveniently in software, lacking this option may be considered as a disadvantage of the half-teleportation protocol.


While we analyze their performance for GKP qudits, these protocols can be straightforwardly adapted to other qudit encodings, such as multi-mode (MM) qudits, 
which have been experimentally demonstrated in the context of (repeaterless) higher-dimensional quantum key distribution in the form of 
orbital angular momentum~\cite{oam_experiment} and time-bin qudits~\cite{Zhong_2015}.
Two of the three repeater protocols under consideration rely on Bell measurements. For GKP qudits, a deterministic Bell measurement can easily be implemented with static linear optics by employing a balanced beam splitter and continuous-variable homodyne measurements. 
Experimental implementations of Bell measurements for MM-encoded qudits, on the other hand, are disproportionately more involved.
Moreover, deterministic $CX$-gates for MM qudits require strong nonlinearities that are typically mediated through auxiliary matter qudits, which reduces the achievable repetition rates to the order of MHz.
This is in stark contrast to all-optical implementations that can reach GHz repetition rates. 
An attempt to circumvent this shortcoming of MM qudits is based on probabilistic linear optical Bell measurements, enabling an all-optical error correction step at every repeater station~\cite{fabian_repeater_prl,fabian_repeater_pra,logicalBMefficiencies, AzumaNC2015,LeePRA2019}. 
Such probabilistic Bell measurements cannot exceed 50\% for MM qubits in the simplest setting without additional resources such as photonic ancilla states~\cite{Calsamiglia_2001,GriceBM,Ewert34BM}.
For a deterministic Bell measurement, nonlinear optics is required.
Furthermore, probabilistic unambiguous state discrimination measurement of the corresponding two-qudit Bell states, making only use of linear optics and photon counting without ancilla photons, is impossible for MM qudits with $D>2$~\cite{qudit_linear_optics1, qudit_linear_optics2}.
Therefore, overall, the GKP concept and the GKP-based QR protocols presented in this work represent a unique way to combine an increased communication capacity based on photonic qudit encoding with an enhanced loss (and error) robustness based on photonic qudit quantum error correction.