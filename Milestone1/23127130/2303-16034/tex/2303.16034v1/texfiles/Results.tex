The central figure of merit that we employ to compare the performance of different repeater protocols is the secret-key rate (SKR) per channel use.
More precisely, we use $\log_2(D)-H(\mathcal{P})$, which is a lower bound on the two-way capacity~\cite{PLOB}, where $H(\mathcal{P})$ denotes the Shannon entropy of a Pauli error probability distribution $\mathcal{P}$ as in Eq.~\eqref{eq:error_probability_distribution}.
Note that this bound can be achieved by a qudit generalization (using $D+1$ bases, assuming $D$ to be prime) of the six-state protocol~\cite{PhysRevLett.81.3018} in the asymptotic limit, where almost every round the same basis is used~\cite{quditqkdprotocols}.  
Moreover, if $X$- and $Z$-errors are independent, the same rate is obtainable with a generalization of the BB84 protocol~\cite{bb84} (2 bases, arbitrary $D$).\footnote{The secret-key fraction is given by $I(A, B)-I(A, E)=\log_2(D)-H(\vec{q}_{01})-I(A, E)$, where expressions of the mutual information $I(A, E)$ between Alice and Eve are provided in Eqs.~(5) and~(7) of  Ref.~\cite{quditqkdprotocols}.}


\subsection{Repeater performance with GKP error correction only}
\label{sec:result_bare_gkp}

\begin{figure} \centering
	\subfloat[]{\label{fig:complete_L0=500m_Lmax=100000km_099coupling_lo_logx_preamp}\includegraphics[width=0.49\textwidth]{texfiles/figures/complete_L0=500m_Lmax=100000km_099coupling_lo_logx_preamp_highres.pdf}}
	\subfloat[]{\label{fig:complete_L0=500m_Lmax=100000km_099coupling_lo_logx_twowaycc_final}\includegraphics[width=0.49\textwidth]{texfiles/figures/complete_L0=500m_Lmax=100000km_099coupling_lo_logx_twoway_highres.pdf}}
    \vspace{-35mm}
    \begin{minipage}{\textwidth} \small
        \hspace{68mm} $D$
        \hspace{69mm} $D$
    \end{minipage}
    \vspace{26mm}
	\caption{Optimal dimension $D$ of bare GKP qudits utilized in a quantum repeater line with coupling efficiencies $\eta_\text{c}=99\%$ and an intermediate repeater spacing of $L_0=500 \unitspace\text{m}$, where the (a)  one-way or (b) two-way teleportation protocol is used.
    For each choice of total repeater length $L$ and squeezing parameter $s_\text{GKP}$, 
    the qudit dimension is adjusted such that the SKR per channel use, $\log_2(D)-H(\mathcal{P})$, is optimized (inset lines). 
    In the parameter regions of $D=1$, it is not possible to generate secret keys.
}
	\label{fig:figa}
\end{figure}

For near-term applications, it is certainly more convenient to operate a quantum repeater with bare GKP qudits and not with multiple GKP qudits in a QECC.
To guide such initial experiments, we begin our discussion with this important special case.
For the two protocols considered with bare GKP qudits, which are described in Secs.~\ref{sec:2way_postamp} and~\ref{sec:1way_preamp}, 
lower-level error correction is performed via a teleportation step on the logical level of the GKP code, which leads to independent $X$- and $Z$-errors.
As mentioned above, the SKR per channel use is thus given by $\log_2(D)-H(\mathcal{P})$ not only for the generalized six-state protocol ($D$~prime) but also for the generalized BB84 protocol ($D$~arbitrary).
The precise value of $H(\mathcal{P})$  has a complicated dependence on the repeater spacing $L_0$, on the total repeater length $L$, on the squeezing parameter $s_\text{GKP}$ that characterizes approximate GKP states, and on the qudit dimension $D$.
However, we can numerically assess $H(\mathcal{P})$, see App.~\ref{app:error_analysis_bare}.
%
In Fig.~\ref{fig:figa}, we show the optimal choice (color-coded) of qudit dimension $D$ for different values of $L$ and $s_\text{GKP}$, 
where $L_0 = 500\unitspace\text{m}$ is fixed.
Using inset lines, we also display the corresponding (maximal) value of the SKR per channel use.
As expected, the key rate vanishes if the GKP approximation is too bad (small $s_\text{GKP}$) or too much loss accumulates (large $L$).
Since increasing the squeezing poses a core experimental challenge,
the smallest value of $s_\text{GKP}$ at which a nonzero SKR can be achieved is of particular interest.
Below $s_\text{GKP} = 10\unitspace\text{dB}$, neither protocol is suitable for generating secret keys.
For both protocols and for every fixed value of $L$, 
we observe that GKP qubits ($D=2$) represent the leading contender for near-term quantum repeaters based on the GKP code.
To some degree, this result is surprising because in the ideal case, 
the SKR per channel use is given by $\log_2(D)$, and increasing the qudit dimension would be beneficial.
In the presence of noise, however, higher-dimensional GKP qudits have the severe disadvantage of decreased error correction capabilities: a $D$-dimensional GKP qudit can only correct displacement errors that are smaller than $\sqrt{{\pi}/{2D}}$ in magnitude.
Only in the regime of very small errors, i.e., where the qubit GKP protocol has almost reached its maximum performance of $\log_2(D) - H(\mathcal{P}) = \log_2(2)-0 = 1.0$,
it is beneficial to employ qutrits ($D=3$) instead of qubits.
To see such benefits at all, we need at least $s_\text{GKP} \gtrsim 18\unitspace\text{dB}$.
For repeater lines of modest lengths of a few ten kilometers, however, larger squeezing levels of $20\unitspace\text{dB}$-$25\unitspace\text{dB}$ are required to compensate for additional loss.
At some value of $L$, loss errors become so severe that only an unrealistically disproportional improvement of $s_\text{GKP}$ could compensate them.
For the one-way protocol in Fig.~\ref{fig:figa}~(a),
qutrits cease to be the optimal option for repeaters longer than a few hundred kilometers, whereas 
the two-way protocol in Fig.~\ref{fig:figa}~(b) 
can still benefit from qutrits even for repeaters exceeding $L= 10,000\unitspace\text{km}$.
For the latter, however, a squeezing level above $30\unitspace\text{dB}$ is required, which will only be available in the long term (if at all).
The reason for the better performance of the two-way protocol is the lower required amplification factor $\sqrt{\eta}^{-1}$ in the usage of the classical post-amplification, as discussed in Sec.~\ref{sec:loss_and_amplification}. 

Finally note that, in our error analysis, we distinguish the cases of even and odd qudit dimensions.
Only if $D$ is even, we can leverage a beneficial linear-optics protocol for the generation of GKP Bell pairs, see App.~\ref{app:error_analysis_bare}.
For very short repeater chains, we indeed observe that GKP qudits with $D$ even outperform those with $D$ odd.
For larger values of $L$, however, loss errors begin to dominate and parameter regions emerge where the optimal SKR is obtained by odd-dimensional GKP qudits.


\subsection{Repeater performance with both GKP and higher-level error correction}

In comparison to the experimental challenge of creating high-quality GKP qudits in the first place,
concatenating multiple of them into a single logical qudit by means of a higher-level QECC is relatively straightforward.
In the following, we study the performance of third-generation quantum repeaters that make use of $\llbracket D, 1, \tfrac{D+1}{2} \rrbracket_D$ quantum polynomial codes ($D\ge 3$ prime), as reviewed in a related context in App.~A of Ref.~\cite{MHKB18}.
The Pauli weight of the stabilizer generators is immense for quantum polynomial codes, which renders them unsuitable for applications in quantum computing.
For quantum repeaters, on the other hand, this is not an issue, as non-destructive measurements of stabilizer operators are not required.
Instead, destructively measuring all qudits individually is sufficient here.
This facilitates syndrome extraction and decoding in a purely classical manner.
Since the distance of a quantum polynomial code is given by $d=\tfrac{D+1}{2}$, 
any collection of errors that affect no more than $\lfloor\tfrac{d-1}{2}\rfloor =  \lfloor \tfrac{D-1}{4} \rfloor$ qudits can be corrected.
For error patterns that affect more qudits than this, 
we assume (as a worst-case approximation) that a uniformly random logical error occurs. 
This maximizes the Shannon entropy $H(\mathcal{P})$ 
and lower bounds the SKR, $\log_2(D)-H(\mathcal{P})$, that would be achieved if a more sophisticated decoder for correcting specific high-weight errors was used.
Thus, it makes sense for us to limit the discussion to prime qudit dimensions where $D-1$ is a multiple of four. 
We defer our derivation of $H(\mathcal{P})$ for this suboptimal decoder to App.~\ref{app:error_analysis_encoded}.
%
\begin{figure}[t]
	\subfloat[]{\label{fig:skr_GKP_with_L0=100m_squeezing=20dB_coupling=0.99twowaycc}\includegraphics[height=0.3\textwidth]{texfiles/figures/skr_GKP_with_L0=100m_squeezing=20dB_coupling=0.99twowaycc_added_singlegkp_new}}
	\subfloat[]{\label{fig:skr_GKP_with_L0=100m_squeezing=30dB_coupling=0.99twowaycc}\includegraphics[height=0.3\textwidth]{texfiles/figures/skr_GKP_with_L0=100m_squeezing=30dB_coupling=0.99twowaycc_added_singlegkp_new}}\\
	\caption{Lower bound on the SKR per logical channel use, $\log_2(D)-H(\mathcal{P})$,  as a function of the total length~$L$ for a quantum repeater line with coupling efficiencies $\eta_\text{c}=99\%$, an intermediate repeater spacing of $L_0 = 100\unitspace\text{m}$, and squeezing levels of (a) $s_\text{GKP} = 20\unitspace\text{dB}$ or (b) $s_\text{GKP}=30\unitspace\text{dB}$.
    The highlighted area shows the achievable SKR per physical channel use of a bare GKP repeater as in Fig.~\ref{fig:figa} (b).
	}
	\label{fig:figb}
\end{figure}

In Fig.~\ref{fig:figb}, we plot the (lower bound on the) SKR per logical channel use as a function of $L$, where $L_0=100\unitspace\text{m}$ is fixed.
For each of the three repeater protocols introduced in Sec.~\ref{sec:protocol}, we show the SKR for $D=5$ (green), $D=13$ (red), $D=17$ (black), and $D=29$ (blue).
For any fixed value of $D$, we again (as in Fig.~\ref{fig:figa}) observe that the two-way teleportation protocol (dash-dotted curve) from Sec.~\ref{sec:2way_postamp} performs best. 
It is followed by the one-way teleportation protocol (dashed curve) from Sec.~\ref{sec:1way_preamp}.
The least-efficient protocol is the one-way half-teleportation protocol (solid curve) from Sec.~\ref{sec:1way_preamp_half}.
We attribute the poor performance of the latter protocol to the fact that it employs only half as many (compared to the other protocols) logical measurements, which facilitate the correction of errors.


Recall from Sec.~\ref{sec:result_bare_gkp} that for bare GKP repeaters, 
the decreased error-correcting capabilities render higher-dimensional qudits unfeasible for near-term applications.
Since the code distance $d=\frac{D+1}{2}$ grows with $D$, 
one could expect that concatenating bare GKP qudits with quantum polynomial codes would turn the tide.
We see that this is not the case: 
for an optimistic but conceivable value of $s_\text{GKP}=20\unitspace\text{dB}$, we see in Fig.~\ref{fig:figb} (a) that only the smallest code with $D=5$ achieves a nonzero SKR for repeater lengths $L>70\unitspace\text{km}$.
To assess the performance of larger codes, we need to assume exorbitant squeezing levels, e.g., $s_\text{GKP}=30\unitspace\text{dB}$ as in Fig.~\ref{fig:figb} (b).
In this scenario, the $\llbracket5,1,3\rrbracket_5$-code operates near its maximum performance of $\log_2(5) \approx 2.3$ for all considered values of $L$.
Depending on the distance $L$, the largest value of $\log_2(D)-H(\mathcal{P})$ is obtained by a different code:
until $L\approx100\unitspace\text{km}$, the $\llbracket 29,1,15 \rrbracket_{29}$-code achieves a value beyond the optimal performance of $\log_2(17)\approx 4.1$ of the $\llbracket 17,1,9 \rrbracket_{17}$-code.
The latter starts to lose performance after a few thousand kilometers, where it falls behind the $\llbracket 13,1,7\rrbracket_{13}$-code.
%
For comparison, we also show in Fig.~\ref{fig:figb} the performance of the two-way repeater protocol with bare GKP qudits (shaded region), 
where we select the value of $D$ that optimizes the SKR,
as in Fig.~\ref{fig:figa} (b).
For $s_\text{GKP}=20\unitspace\text{dB}$ in Fig.~\ref{fig:figb} (a), bare GKP ququarts ($D=4$) are optimal until  $L \approx 200\unitspace\text{km}$. 
For longer repeaters, too much loss accumulates, and lower-dimensional GKP codes with higher error-correcting capabilities become beneficial:
in a small range of $L$, bare qutrits are the optimal choice, but already for $L\gtrsim300\unitspace\text{km}$, qubits perform best.
As before, this advantage of even dimensions over odd ones is due to improved Bell state availability~\cite{gkp_syndrome}.
For $s_\text{GKP}=30\unitspace\text{dB}$ in Fig.~\ref{fig:figb} (b), 
losses are less of an issue and eight-dimensional GKP qudits are optimal until $L\approx20\unitspace\text{km}$.
For $30\unitspace\text{km} \lesssim L \lesssim 200 \text{km}$, $D=6$ is optimal.
For $1000\unitspace\text{km} \lesssim L \lesssim 50,000 \text{km}$, a bare GKP repeater should operate with $D=4$.

It is important to stress that, from a practical perspective and for the considered parameters, it is not useful to employ higher-level QECCs if the application is quantum key distribution (QKD). 
For example, if $s_\text{GKP}=30\unitspace\text{dB}$ and $L=1000\unitspace\text{km}$, the two-way teleportation protocol with a logical $\llbracket 17,1,9\rrbracket_{17}$-code achieves the largest rate of about four secret bits per logical channel use.
To accomplish this, however, seventeen GKP qudits (entangled in a QECC), i.e., seventeen GKP-encoded and entangled optical modes, need to be transmitted.
With an even lower experimental effort, one could simply transmit in parallel seventeen bare GKP ququarts, i.e., seventeen GKP-encoded but unentangled optical modes, each of which establishes almost two secret bits.
In other words, here the best bare protocol is more efficient than the best higher-level encoded one by a factor of about $8.5$.


\subsubsection{Optimal choice of the repeater spacing}

\begin{figure}[t] \centering
	\subfloat[]{\label{fig:skr_VS_L_with_L=2000km_20_squeezing0.999_coupling_erasure_twoway}\includegraphics[height=0.275\textwidth]{texfiles/figures/skr_VS_L_with_L=2000km_20_squeezing0.999_coupling_noerasure_twoway.pdf}}
    \hspace{5mm}
	\subfloat[]{\label{fig:skr_VS_L_with_L=2000km_30_squeezing0.999_coupling_noerasure_twoway}\includegraphics[height=0.275\textwidth]{texfiles/figures/skr_VS_L_with_L=2000km_30_squeezing0.999_coupling_noerasure_twoway.pdf}}\\
	\caption{Lower bound on the SKR per logical channel use, $\log_2(D)-H(\mathcal{P})$, normalized by the number $N$ of repeater segments ($N-1$ repeater stations) as a function of the repeater spacing $L_0$ for a repeater line of total length $L = N  L_0 = 2000\unitspace\text{km}$, coupling efficiencies $\eta_\text{c}=99.9\%$, and squeezing levels of (a) $s_\text{GKP} = 20\unitspace\text{dB}$ or (b) $s_\text{GKP}=30\unitspace\text{dB}$.
	}
	\label{fig:figc}
\end{figure}

In our discussion of Fig.~\ref{fig:figb}, we have pointed out that no practical benefit is to be expected when switching from bare GKP qudits to a higher-level QECC 
if the repeater spacing is fixed to $L_0 = 100\unitspace\text{m}$.
This raises the question of how the choice of $L_0$ influences this conclusion.
Since implementation cost scales with the total number $N = L/L_0$ of repeater stations, here we focus on SKR$/N$ as a figure of merit.
In a commercial setting, SKR$/N$ is roughly proportional to the return on investment.
In Fig.~\ref{fig:figc}, we plot SKR$/N$ as a function of $L_0$ for a quantum repeater line of fixed length $L= 2000\unitspace\text{km}$.
The colors and line styles have the same meaning as in Fig.~\ref{fig:figb}.
This time, we assume a more optimistic value of $\eta_\text{c} = 99.9\%$, which benefits higher-level QECCs.
Despite this optimistic assumption, we still find that (for QKD) bare GKP qudits outperform those encoded into quantum polynomial codes.
For example, for $s_\text{GKP}=20\unitspace\text{dB}$ in Fig.~\ref{fig:figc} (a), the $\llbracket5,1,3\rrbracket_5$-code performs best among the quantum polynomial codes and reaches the optimal value of SKR/$N$ at a repeater spacing of $L_0 \approx 0.55\unitspace\text{km}$.
For this optimal repeater configuration, the $\llbracket5,1,3\rrbracket_5$-code can generate approximately 1.8 secret bits by transmitting five GKP ququints ($D=5$).
In the same setting, one can generate almost 5.0 secret bits by transmitting five bare GKP qubits (not shown).
The same behavior is observed for $s_\text{GKP} = 30\unitspace \text{dB}$ in Fig.~\ref{fig:figc} (b), 
where the $\llbracket5,1,3\rrbracket_5$-code now 
achieves approximately 2.0 secret bits per logical channel use at the optimal operating point of $L_0 \approx 1.1\unitspace\text{km}$.
In the same setting, transmitting five bare GKP qutrits would generate more than 5.6 secret bits.

From Fig.~\ref{fig:figc}, we can also infer the maximal repeater spacing at which the secret-key rate drops to zero.
For the considered parameters, the best higher-level encoded protocol, i.e., the two-way protocol from Sec.~\ref{sec:2way_postamp} with the $\llbracket5,1,3\rrbracket_5$-code and $s_\text{GKP}=30\unitspace\text{dB}$, 
is operational for all values of $L_0<1.5\unitspace\text{km}$, however, $L_0 \approx 1.1\unitspace\text{km}$ is most effective.
For the one-way protocols from Secs.~\ref{sec:1way_preamp} and~\ref{sec:1way_preamp_half}, the $\llbracket5,1,3\rrbracket_5$-code already fails for $L_0 \approx 0.7\unitspace\text{km}$.
As expected, we find that better repeaters (larger $s_\text{GKP}$, smaller $D$) allow for a larger repeater spacing. 



\subsubsection{Identifying and overcoming noise bottlenecks}

\begin{figure} \centering
    \subfloat[]{\label{fig:squeezing_coupling_phasespace_spacing=500m}
	\begin{minipage}{0.45\textwidth}
	\vspace*{-41.9mm}
	
	\includegraphics[width=\textwidth]{texfiles/figures/squeezing_coupling_phasespace_spacing=500m}
	\end{minipage}
	}
    \hspace{10mm}
	\subfloat[ ]{\label{fig:skr_GKP_with_internal_noise_L=5000km_spacing0.5km_twowaycc}\includegraphics[width=0.45\textwidth]{texfiles/figures/skr_GKP_with_internal_noise_L=5000km_spacing0.5km_twowaycc}}
	
	\caption{(a) Variance $\sigma_\text{in}^2$ of Gaussian noise effectively affecting a physical GKP qudit after it has been coupled into fiber as a function of the squeezing level $s_\text{GKP}$ and coupling efficiency $\eta_\text{c}$. 
    (b) Lower bound on the SKR per logical channel use, $\log_2(D)-H(\mathcal{P})$, as a function of $\sigma_\text{in}^2$ for a quantum repeater line with a total transmission distance of $L = 5000\unitspace\text{km}$ and a repeater spacing of $L_0 = 500\unitspace\text{m}$, where the two-way protocol in combination with a $\llbracket D, 1, \tfrac{D+1}{2}\rrbracket_D$ quantum polynomial code is utilized.}
	\label{fig:figd}
\end{figure}

Before one takes a great effort of building a quantum repeater based on GKP qudits,
it is important to ensure that the experimental building blocks work sufficiently well.
There are multiple components for which improvements might be beneficial or even necessary. 
Thus, it is important to identify and remove the noise bottleneck, which would otherwise diminish the performance.
We distinguish three error mechanisms: input noise, fiber channel losses, and imperfect homodyne measurements. 
Since measurements work comparatively well and we have already discussed the impact of fiber loss,
here we focus on input noise that arises from approximate GKP state preparation and  coupling losses.
As explained in Sec.~\ref{sec:noise},
both processes can be modeled by Gaussian noise.
Errors propagate through the circuit and eventually accumulate on individual measurement results in the repeater stations, which for the two-way post-amplification protocol from Sec.~\ref{sec:2way_postamp} can be described by a Gaussian channel with variance 
\begin{align} \label{eq:variance_input}
    \sigma_\text{in}^2 = 3 \sigma_\text{sq}^2 + \frac{1-\eta_\text{c}}{\eta_\text{c}} \exp\left(\frac{L_0}{2L_\text{att}}\right).
\end{align}
Indeed,
there are three sources from which GKP state preparation errors can propagate to the measurements, which leads to the first term in Eq.~\eqref{eq:variance_input}.
The second term in Eq.~\eqref{eq:variance_input} accounts for coupling losses:
since the variance (incorporating both coupling and fiber channel losses) of a length-$L_0$ link 
in the two-way protocol is given by 
$(\eta_\text{c}\exp(-{L_0}/{2L_\text{att}}))^{-1}-1$,
the noise difference between a link with coupling errors and without is given by
\begin{align}
((\eta_\text{c}\exp(-\tfrac{L_0}{2L_\text{att}}))^{-1}-1 ) 
-
((\exp(-\tfrac{L_0}{2L_\text{att}}))^{-1}-1) 
=
     \frac{1-\eta_\text{c}}{\eta_\text{c}} \exp\left(\frac{L_0}{2L_\text{att}}\right).
\end{align}


In Fig.~\ref{fig:figd} (a), we plot $\sigma_\text{in}^2$ as a function of both $s_\text{GKP}$ and $\eta_\text{c}$.
Recall that $\sigma_\text{sq}^2$ and $s_\text{GKP}$ can be converted into each other via Eq.~\eqref{eq:squeezing_parameter}.
Here, we assume a repeater spacing of $L_0 = 500\unitspace\text{m}$, however, the situation remains almost unchanged if $L_0$ takes any other value between $1\unitspace\text{m}$ and $1\unitspace\text{km}$.
The contour lines in Fig.~\ref{fig:figd} can be used to infer whether one should work on improving $s_\text{GKP}$ or $\eta_\text{c}$: 
since moving along a contour line does not improve performance, 
a series of improvements should instead correspond to a path orthogonal to the contour lines.
For example, for $s_\text{GKP}=6\unitspace\text{dB}$ and $\eta_\text{c} = 0.99$, we have $\sigma_\text{in}^2 \approx 0.4$, which can be reduced to $\sigma_\text{in}^2 \approx 0.2$ if the GKP approximation is improved to  $s_\text{GKP}=9\unitspace\text{dB}$; increasing $\eta_\text{c}$, on the other hand, would not help at all.
Conversely, if coupling losses dominate, e.g.,  $s_\text{GKP}=30\unitspace\text{dB}$ and $\eta_\text{c} = 0.92$, the variance $\sigma_\text{in}^2 \approx 0.1$ can be reduced by a factor of two if coupling efficiencies are improved to $\eta_\text{c} = 0.97$; increasing $s_\text{GKP}$, however, would show no significant effect here.


In Fig.~\ref{fig:figd} (b), we depict the influence of $\sigma_\text{in}^2$ on the SKR obtained with the two-way protocol from Sec.~\ref{sec:2way_postamp} 
for an error-corrected quantum repeater line with $L=5000\unitspace\text{km}$, $L_0 = 500 \unitspace\text{m}$, and a $\llbracket D, 1, \tfrac{D+1}{2}\rrbracket_D$-code.
Here, each physical GKP qudit in every repeater station is affected by a Gaussian channel with variance $\sigma_\text{in}^2$.
Note that also the effect of imperfect homodyne measurements can be inferred from Fig.~\ref{fig:figd} (b) if a corresponding variance term $\sigma_\text{meas}^2$ is added to $\sigma^2_\text{in}$.
As before, we find that a larger value of $D$ both allows for a larger SKR per logical channel use in the low-noise regime and for a smaller noise level to be tolerated before the SKR drops to zero.
We also observe that the parameter range of $\sigma_\text{in}^2$ where the SKR drops from its optimal value to zero is alarmingly small.
This effect is most pronounced for the $\llbracket 5,1,3\rrbracket_5$-code, which has almost optimal performance until $\sigma_\text{in}^2\approx 0.01$ but already for $\sigma_\text{in}^2\approx 0.02$ its SKR is equal to zero.
This showcases that moderate improvements can have a huge impact if they address a noise bottleneck.


\subsubsection{Leveraging lower-level syndrome information to improve higher-level error correction}

So far, we have independently treated the error correction procedures of lower-level GKP qudits and the higher-level QECC.
More precisely, we assumed that, in the first step, displacement errors on the physical GKP qudits are removed.
This may or may not result in a logical GKP qudit error.
Then, in a second step, the higher-level $\llbracket n, 1,d \rrbracket_D$-code deals with potential errors on the GKP qudits:
the correction succeeds if the number of errors with unknown locations is not larger than $\frac{d-1}{2}$.
In this final subsection, we investigate the more general case, where the location of some of the errors are known.
The modified error correction routine succeeds whenever $t_\text{k} +2t_\text{u} <d$, where $t_\text{k}$ and $t_\text{u}$ denote the number of errors with known and unknown locations, respectively.
To obtain some information about error location, one can exploit the continous, ``analog’’ results of the homodyne measurements in the repeater stations~\cite{gkp_analoginfo, toric-gkp}.
If a displacement error of the form $\exp(\text{i}\epsilon\hat{p})$ occurs, the homodyne measurement of $\hat{q}$ reveals the value of $\epsilon$ modulo $\sqrt{2\pi/D}$, which we call analog GKP syndrome.
In particular, every displacement error with  $|\epsilon| < \sqrt{{\pi}/{2D}}$ can be corrected.
The probability of successful error correction is large if $\epsilon$ is small.
When an error of magnitude $\epsilon \approx \sqrt{{\pi}/2D}$ occurs, 
however, the situation is less clear.
Borrowing ideas from Ref.~\cite{fukui2021efficient}, we introduce a discarding parameter $\gamma \in [0,1]$,
and treat any instances of $\epsilon$ which are closer than $\sqrt{\pi/2D}(1-\gamma)$ from the boundary of two bins
as an erasure error with a known location.
In the case $\gamma=1$, we do not discard any qudits, which corresponds to the strategy considered so far.
The other extreme, $\gamma = 0$, corresponds to the absurd approach where all qudits are always discarded.

The advantage of this modification is that, 
for every qudit that is not discarded, 
the probabilities for errors (with unknown locations) are improved from Eq.~\eqref{eq:pauli_shift_probability_gauss} to
\begin{align} 
 P^{(\gamma)}_\text{sq}(X^k,\sigma^2) \propto 
 \sum_{j\in\mathbb{Z}}\int
     _{\sqrt{\frac{2\pi}{D}}(jD+k-\frac{\gamma}{2})}
     ^{\sqrt{\frac{2\pi}{D}}(jD+k+\frac{\gamma}{2})}
     \frac{1}{\sqrt{2\pi \sigma^2}}\exp\left(-\frac{q^2}{2\sigma^2}\right)dq,
\end{align}
where the proportionality constant follows from  $\sum_{k=0}^{D-1}  P^{(\gamma)}_\text{sq}(X^k,\sigma^2) = 1$.
This improvement comes at the expense that we have to introduce an erasure error with probability
\begin{align}
   p_\text{discard} = 1- \sum_{k=0}^{D-1} 
 \sum_{j\in\mathbb{Z}}\int
     _{\sqrt{\frac{2\pi}{D}}(jD+k-\frac{\gamma}{2})}
     ^{\sqrt{\frac{2\pi}{D}}(jD+k+\frac{\gamma}{2})}
     \frac{1}{\sqrt{2\pi \sigma^2}}\exp\left(-\frac{q^2}{2\sigma^2}\right)dq,
\end{align}
however, we can still exploit our knowledge about the location of this error.

Denote the probability that a single GKP qudit is free of errors by $p_0 = P^{(\gamma)}_\text{sq}(X^0, \sigma^2)$.
Then,  the condition $t_\text{k} +2t_\text{u} <d$ and basic combinatorics leads to the probability of a failed error correction attempt
\begin{align} \label{eq:p_fail}
    p_\text{fail}(\gamma) = 1-\sum_{t_\text{k}=0}^{d-1}\binom{n}{t_\text{k}} p_\text{discard} ^{t_\text{k}} (1-p_\text{discard} )^{n-t_\text{k}} 
    \sum_{t_\text{u}=0}^{t_\text{u,max}} \binom{n-t_\text{k}}{t_\text{u}} p_0^{n-t_\text{k}-t_\text{u}} (1-p_0)^{t_\text{u}},
\end{align}
where $t_\text{u,max} = \left\lfloor ({d-t_\text{k}-1})/{2} \right\rfloor$ is the maximal number of correctable errors with unknown locations, assuming that $t_\text{k}$ erasures occurred, and $n$ is the number of physical GKP qudits.

\begin{figure}
	\centering
	\includegraphics[width=0.6\linewidth]{texfiles/figures/erasureplot_n=13}
	\caption{Failure probability $p_\text{fail}$ for decoding the result of a logical measurement for a 
    $\llbracket 13, 1, 7\rrbracket_{13}$-code as a function of the discarding parameter $\gamma$.
    Each physical GKP qudit is subject to Gaussian noise with variance $\sigma^2=0.01$. }
	\label{fig:erasureplotn13}
\end{figure}

In Fig.~\ref{fig:erasureplotn13}, we show how the logical failure rate (red) depends on the discarding parameter $\gamma$ for a $\llbracket13,1,7\rrbracket$-code.
For each physical GKP qudit, we assume that all noise combined (stemming, e.g., from GKP approximation, coupling, or transmission) corresponds to a fairly small but finite variance $\sigma^2=0.01$ of the overall Gaussian noise channel.
For $\gamma = 1$, i.e., without discarding (black), the failure rate has a remarkably low value of $p_\text{fail} \approx 5\times 10^{-11}$, which is due to the low level of noise and the high error-correcting distance of $d=7$.
We observe a local minimum at $\gamma_\text{opt} \approx 0.82$,  where 
the failure rate is improved by more than an order of magnitude to $p_\text{fail}(\gamma_\text{opt}) \approx 3\times 10^{-12}$.
If $\gamma$ is decreased below $\gamma_\text{opt}$, we begin to introduce more erasures than the QECC can deal with, and the failure rate increases.
On the other hand, if $\gamma$ is increased above $\gamma_\text{opt}$, 
then the error rates $P^{(\gamma)}_\text{sq}(X^k,\sigma^2)$ start to deteriorate.
This causes an increasing amount of errors with unknown locations and leads to the rise in $p_\text{fail}$.
A curious effect in Fig.~\ref{fig:erasureplotn13} is that the performance 
in the seldom-discarding regime ($\gamma > 0.96$) is worse than in the never-discarding case ($\gamma = 1$).
We attribute this to the fact, that for $0.96 < \gamma < 1.0$, those cases dominate where only a single erasure error is introduced, i.e., $t_\text{k}=1$ and the number of correctable errors with unknown locations is decreased to $t_\text{u,max} = 2$.
At the same time, the error probabilities $P^{(\gamma)}_\text{sq}(X^k,\sigma^2)$ are only slightly improved because they continuously depend on $\gamma$.
Thus, the performance is worse than for the naive approach with $\gamma=1$, i.e., $t_\text{k}=0$ and $t_\text{u,max}=3$.
