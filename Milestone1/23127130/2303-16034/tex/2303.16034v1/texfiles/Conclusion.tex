In this paper, we have analyzed the performance of third-generation quantum repeaters that operate with higher-dimensional GKP qudits.
We have focused on the GKP square lattice and also considered concatenations with quantum polynomial codes.

The missing component that is currently holding back an experimental realization of such repeaters is efficient sources of high-quality GKP states.
Once such sources are available, however, 
there will be no need for quantum memories or classical two-way communication.
Therefore, the achievable repetition rates will only be limited by fast optical elements for the local processing of GKP qudits.

Our initial motivation for the present investigation was that,
at a first glance,
GKP qudits and quantum polynomial codes seem like a perfect match in the context of quantum repeaters:
GKP states can be encoded into photons, which is crucial for repeaters;
polynomial codes achieve the singleton bound at the expense of high-weight stabilizers, which is a problem for quantum computation but not for quantum repeaters;
polynomial codes require higher-dimensional qudits, which the GKP encoding has to offer.
However, our study revealed that the decreased lower-level error-correcting capabilities of higher-dimensional GKP qudits severely limit their potential benefits.
While this finding might disappoint to a certain extent, it is somewhat good news for experimentalists.
Indeed, the most promising GKP repeater protocol identified in this work is also the one, which is the easiest to implement.

Our recommendation for a first experimental target is a repeater protocol (Sec.~\ref{sec:2way_postamp}) that makes use of two-dimensional GKP qubits.
Admittedly, these qubits will require challenging squeezing levels beyond $10\unitspace\text{dB}$.
However, the identified protocol has the advantage of readily available syndrome measurements based on balanced beam splitters and homodyne measurements alone. 
Furthermore, this protocol is compatible with rescaling the GKP lattice in classical software, 
whereas other protocols would require optical amplifiers to compensate for the loss.

We also found that, in the medium-to-long term,
when squeezing levels above $20\unitspace\text{dB}$ will be available,
the error-correcting capabilities of bare GKP qutrits will suffice to outperform GKP qubits for meaningful repeater lengths.
Only in the very long term, if squeezing levels around $30\unitspace\text{dB}$ can possibly be reached,
we expect some benefit from concatenating multiple GKP qudits using quantum polynomial codes,
however, only for tasks like entanglement distribution where utmost fidelities are important.
For the application of quantum key distribution, on the other hand,
our analysis showed that it is typically more cost-effective to operate bare GKP qudit repeaters instead. With regards to potential experimental realizations, a useful feature
of the case with bare GKP qudits is that the necessary GKP two-qudit
Bell pair for teleportation-based syndrome detection and error correction
can often be directly created by applying a balanced beam splitter upon two
suitable, individual single-mode GKP/grid states~\cite{WalshePRA2020,gkp_syndrome}. In case that the concatenation with the higher-level code is employed,
for potential, future high-fidelity quantum network applications,
the complete syndrome information of the QECC
can still be obtained in one linear-optics step
with no need for any online squeezing operations and also
with no need for separating the physical GKP qudit from
the higher-level code's syndrome measurements and adding extra
GKP ancilla states for the higher-code detections.
This only works, however, provided a suitable logical, higher-level
Bell pair is available~\cite{gkp_syndrome}.


In this paper, we have focused on the cases of single GKP qudits and multiple GKP qudits that are concatenated by means of a higher-level qudit stabilizer code.
This is, however, not the only possibility that can be envisioned.
An interesting open research direction is to study the performance of multi-mode GKP codes that do \emph{not} arise as a concatenation of physical GKP states and a higher-level stabilizer code~\cite{Conrad2022gottesmankitaev, conrad_good_gkp_2023,PRXQuantum.3.010335,lin2023closest,gkp_syndrome}.
For such an analysis, theoretical insights about multi-mode Gaussian channels might become important~\cite{Caruso_2008}.
Moreover, one could analyze how bosonic encodings other than GKP perform in a quantum repeater setting, e.g., cat codes~\cite{cat_codes,li2022memoryless}, spherical codes~\cite{spherical_codes}, etc.~\cite{bosonic_codes,PhysRevX.10.011058}.


\begin{acknowledgments}	
FS and PvL acknowledge financial support from the BMBF in Germany via the projects QR.X, QuKuK, and PhotonQ and
the BMBF/EU for support via the project QuantERA/ShoQC.
DM acknowledges financial support from the BMBF in Germany via the projects RealistiQ, QR.X, and QSolid.

\end{acknowledgments}