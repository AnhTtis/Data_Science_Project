\pdfoutput=1
\documentclass[11pt, notitlepage]{article}
\usepackage{setspace}
\onehalfspace

\usepackage[utf8]{inputenc}
\pretolerance=9999
\hyphenpenalty=9999

\usepackage[a4paper, total={6.5in, 8.5in}]{geometry}
\usepackage{algpseudocode}
\usepackage{setspace}
\usepackage{etoc}
\usepackage{etoolbox}
\AtBeginEnvironment{algorithmic}{\setstretch{1.8}}

\usepackage{amsmath}
\let\openbox\relax
\usepackage{amsthm}
\newtheorem*{theorem}{Theorem}
\usepackage{xcolor}
\usepackage[allbordercolors=white]{hyperref}

\usepackage{graphicx}
\usepackage{authblk}
\title{Fundamental Challenges for On-Chip Diffractive Processing at Gigahertz Speeds}
\author[1,*]{Benjamin Wetherfield}
\author[1]{Timothy D. Wilkinson}
\affil[1]{Electrical Engineering Division, Department of Engineering, University of Cambridge}
\affil[*]{bsw28@cam.ac.uk}
\date{}

\renewcommand{\theequation}{S\arabic{equation}}
\renewcommand{\thesection}{S\arabic{section}}

\begin{document}
\maketitle

\section*{Supplementary Information}
\label{sec:suppl-inform}

\localtableofcontents

\section{Asymptotic Expansions}
\label{sec:asympt-expans}

In order to solve for bounds on $t_{0}$ and $1/t_{0}$, as introduced in the main text, we use asymptotic expansions for complete and incomplete Hankel functions, which are valid for large argument (corresponding to propagation distance). The following asymptotic expansion for the complete Hankel function is standard (see pages 196-197 of Watson's classic treatise \cite{watson}).

\begin{equation}
  \label{eq:Hankel_bound}
  H_{0}^{(1)}(\vert x \vert ) = \sqrt{\frac{2}{i\pi\vert x \vert}}\exp(i\vert x \vert - \pi / 4) + R(\vert x \vert)
\end{equation}
Here, $R$ is bounded for real positive argument:
\begin{equation}
  \label{eq:bound_on_R}
  \lvert R(\lvert x \rvert) \rvert \le \frac{1}{8\lvert x \rvert} \sqrt{\frac{2}{\pi\lvert x \rvert}}
\end{equation}
Second, we prove the following bounds for our defined incomplete Hankel function:
\begin{equation}
  \label{eq:asymptotic_expansion}
  H_{0}^{(1)}(\lvert x \rvert, T) = \frac{2\exp(i\lvert x \rvert T)}{i\pi\lvert x \rvert \sqrt{T^{2} - 1}} + R(\lvert x \rvert, T)
\end{equation}
where the remainder term $R(\lvert x \rvert, T)$ is bounded as follows:
\begin{equation}
  \label{eq:bound_on_remainder}
  \vert R(\lvert x \rvert, T) \lvert \le \frac{4}{\pi\lvert x \rvert^{2}}\frac{T}{(T^{2} - 1)^{3/2}}
\end{equation}
and, more coarsely,
\begin{equation}
  \label{eq:bound_on_incomplete_hankel}
  \left \lvert H_{0}^{(1)}(\lvert x \rvert, T) \right \rvert \le \frac{4}{\pi}\frac{1}{\lvert x \rvert \sqrt{T^{2} - 1}}
\end{equation}
Both of these bounds follow from the following theorem.\footnote{A similar theorem appears in Wong's classic text \protect\cite{wong} (pages 16-17).}

\noindent\hrulefill

\begin{theorem}
 Let $f(t)$ be continuous and infinitely differentiable in the region $[T,\infty]$ such that $f^{(N)}(t)~=~0$ for all $N$ as $t \to \infty$. Then,
\begin{equation}
  \label{eq:thm}
  \int_{T}^{\infty} f(t)\exp(i\lvert x \rvert t) \; dt = \sum_{n=0}^{N-1}\left ( \frac{i}{\lvert x \rvert }\right )^{n + 1}f^{(n)}(T)\exp(i\lvert x \rvert T) + \varepsilon_{N}(\lvert x \rvert, T )
\end{equation}
where the error term $\varepsilon_{N}$ is bounded as follows:
\begin{equation}
  \label{eq:thm_error_bound}
  \lvert \varepsilon_{N}(\lvert x \rvert, T) \rvert \le \left ( \frac{1}{\lvert x \rvert }\right )^{N + 1} \left ( \left \lvert f^{(N)}(T) \right \rvert +  \int_{T}^{\infty}\left \lvert f^{(N+1)}(t) \right \rvert\; dt  \right )
\end{equation}
If, in addition, for each $n$, $f^{(n)}(t) \ge 0$ or $f^{(n)}(t) \le 0$ for all $t \in [T, \infty]$, the error bound becomes:
\begin{equation}
  \label{eq:thm_error_bound_2}
  \lvert \varepsilon_{N}(\lvert x \rvert, T) \rvert \le 2 \left ( \frac{1}{\lvert x \rvert }\right )^{N + 1} \left ( \left \lvert f^{(N)}(T) \right \rvert \right )
\end{equation}
\end{theorem}

\noindent\hrulefill

Hence, setting $f(t)=2(t^{2} - 1)^{1/2}/(i\pi)$, and $T>1$, we have
\begin{equation}
  \label{eq:specific_asymptotic_expansion}
  H_{0}^{(1)}(\lvert x \rvert, T) = \frac{2\exp(i\lvert x \rvert T)}{i\pi}\sum_{n=0}^{N-1}\left ( \frac{i}{\lvert x \rvert }\right )^{n + 1}\frac{d^{n}}{dT^{n}}\left (\frac{1}{\sqrt{T^{2} - 1}}\right ) + R_{N}(\lvert x \rvert )
\end{equation}
with $R_{N}$ satisfying:
\begin{equation}
  \label{eq:specific_asymptotic_error_bound}
  \lvert R_{N}(\lvert x \rvert) \rvert \le \frac{4}{\pi} \left ( \frac{1}{\lvert x \rvert }\right )^{N + 1} \left ( \left \lvert \frac{d ^{N}}{dT^{N}}\frac{1}{\sqrt{T^{2} - 1}} \right \rvert \right )
\end{equation}

Choosing $N = 0$ and $1$, we obtain the specific asymptotic bounds given in equations \eqref{eq:bound_on_incomplete_hankel} and \eqref{eq:bound_on_remainder} respectively. For large enough values of $x$ (or $\rho$ in the context of the main text), larger values of $N$ may yield tighter bounds on $H_{0}^{(1)}(x, T)$.

\begin{proof}[Proof of the theorem]
Integrating by parts $N$ times yields
\begin{equation}
  \label{eq:integration_by_parts}
\int_{T}^{\infty}f(t)\exp(i\vert x \vert t) = \sum_{n=0}^{N-1} \left (\frac{i}{\lvert x \rvert }\right  )^{n+1} \exp(i\lvert x \rvert T) + \varepsilon_{N}(\lvert x \rvert, T)
\end{equation}
where
\begin{equation}
  \label{eq:epsilon}
  \varepsilon_{N}(\lvert x \rvert, T) =  \int_{T}^{\infty} \left (\frac{i}{\lvert x \rvert}\right )^{N}f^{(N)}(t) \exp(i\lvert x \rvert t)  dt
\end{equation}
A further integration by parts gives
\begin{equation}
  \label{eq:further_parts}
  \varepsilon_{N}(\lvert x \rvert, T) =  \left (\frac{i}{\lvert x \rvert}\right )^{N+1}\left [f^{(N)}(T) \exp(i\lvert x \rvert T) +
\int_{T}^{\infty}  f^{(N+1)}(t) \exp(i\lvert x \rvert t) \; dt \right ]
\end{equation}
Hence, by the triangle inequality, we have
\begin{equation}
  \label{eq:triangle}
  \lvert \varepsilon_{N}(\lvert x \rvert, T) \rvert \le  \left (\frac{1}{\lvert x \rvert}\right )^{N+1}\left [\left \lvert f^{(N)}(T) \right \rvert +
\int_{T}^{\infty}\left |  f^{(N+1)}(t) \right | \; dt \right ]
\end{equation}

The final condition in the theorem provides
\begin{align}
  \label{eq:final_condition_outcome}
  \int_{T}^{\infty}\left \lvert f^{(N+1)}(t) \right \rvert \; dt &= \left \lvert\int_{T}^{\infty}  f^{(N+1)}(t)  \; dt\right \rvert \\
  &= \left \lvert f^{(N)}(T) \right \rvert
\end{align}
which completes the final part of the proof.

\end{proof}

\pagebreak
\section{Solving for Bounds}
\label{sec:solving-bounds}

As stated in the main text, the sought criterion on $t_{0}$ and $1/t_{0}$ can be achieved by ensuring that the error incurred from temporal effects is no larger in magnitude than the error accepted from approximating $G_{2D}$ by a complex exponential. Armed with the asymptotic bounds from the previous section (equations (\ref{eq:Hankel_bound}-\ref{eq:bound_on_remainder})), we wish to ensure:
\begin{equation}
  \label{eq:bounding_condition_no_priors}
  \lvert H_{0}^{(1)}(\beta \rho, c_{n}t_{0}/\rho) \rvert \le \lvert R(\beta \rho)  \rvert
\end{equation}
in the case that we \emph{disregard} prior inputs $F_{j}$, per the main text, or,
\begin{equation}
  \label{eq:bounding_condition}
  2\lvert H_{0}^{(1)}(\beta \rho, c_{n}t_{0}/\rho) \rvert \le  \lvert R(\beta \rho)  \rvert
\end{equation}
in the case that we \emph{include} prior inputs $F_{j}$.

Hence, using equations (\ref{eq:asymptotic_expansion}) and (\ref{eq:bound_on_remainder}), we set
\begin{equation}
  \label{eq:set_bound}
  K\cdot\frac{2}{\pi}\left (\frac{1}{X}\frac{1}{\sqrt{T^{2} - 1}} + \frac{2T}{X(T^{2} - 1)^{2}}\right) \le \sqrt{\frac{2}{\pi X}} \frac{1}{8X}
\end{equation}
with $X = \beta \rho$ and $T = c_{n}t_{0} /\rho$, and where $K$ is $1$ or $2$, depending on whether we solve for case \eqref{eq:bounding_condition_no_priors} or \eqref{eq:bounding_condition}.
To find suitable $t_0$, we can seek the slightly relaxed bound
\begin{equation}
  \label{eq:mildly_relaxed_bound}
  \frac{2K}{\pi}\left (\frac{(T^{2} -1)^{3/2}}{X} + \frac{2T}{X}\right)T < \sqrt{\frac{2}{\pi X}} \frac{1}{8X}(T^{2} - 1)^{2}
\end{equation}
and find roots of the resulting quartic in $t_{0}$ obtained by setting left- and right-hand-sides equal. Of the four roots, the one with physical significance is the one close to the positive root of the following quadratic (where the coarser bound \eqref{eq:bound_on_incomplete_hankel} on $\lvert H_{0}^{(1)}(X, T) \rvert $ has been used):
\begin{equation}
  \label{eq:simpler_bound_condition}
  \frac{4K}{\pi}  \frac{T}{X}  - \sqrt{\frac{2}{\pi X}}\frac{1}{8X}(T^{2} - 1) = 0
\end{equation}
yielding
\begin{equation}
  \label{eq:approximate_t_0}
  \min(t_{0}) \approx \frac{16K\sqrt{2}}{\sqrt{\pi}}\frac{n_{e}\rho\sqrt{kn_{e} \rho}}{c}
\end{equation}
where $n_{e}$ is the effective refractive index of the wave in the medium, $k$ is the free-space wavenumber and $c$ is the speed of light in a vacuum, as stated in the main text. For tighter bounds on $t_{0}$, higher order bounded asymptotic approximations of $H_{0}^{(1)}(X,T)$, as given in the previous section, can be used.


\pagebreak
\section{Procedure for Generating Main Plot}
\label{sec:proc-gener-plot}

The following procedure outlines the method for generating the main plot. Perform the following steps for $K=1$ and $K = 2$, where $c$ is the speed of light in a vacuum, $c_{n} = c/n_e$, $n_e =2.84$, $\beta = kn_{e}$, $k = 2\pi / \lambda$ and $\lambda = 1550\times 10^{-9}$:

\begin{algorithmic}
  \algblockdefx{With}{End}[1]{\textbf{With} #1:}{\textbf{End}}
  \State $\textrm{TimeQuartic}(T, X, K) \gets
  \displaystyle \frac{2K}{\pi}\left (\frac{(T^{2} -1)^{3/2}}{X} + \frac{2T}{X}\right)T - \sqrt{\frac{2}{\pi X}} \frac{1}{8X}(T^{2} - 1)^{2}$
  \State $\textrm{FreqQuartic}(F, X, K) \gets \textbf{Simplify }F^{4} \times\textrm{TimeQuartic}(1/F, X, K)$
  \State $\textrm{TimeQuadratic}(T, X, K) \gets
  \displaystyle
  \frac{4K}{\pi}  \frac{T}{X}  - \sqrt{\frac{2}{\pi X}}\frac{1}{8X}(T^{2} - 1) $
  \State $\textrm{FreqQuadratic}(F, X, K) \gets
  \displaystyle
  \textbf{Simplify } F^{2} \times \textrm{TimeQuadratic}(1/F, X, K)$
  \State QuarticRoots[4] $\gets \textbf{Solve }\textrm{FreqQuartic}(\rho f/c_{n}, \beta \rho, K) = 0 \textrm{ in terms of } f$
  \State QuadraticRoots[2] $\gets \textbf{Solve }\textrm{FreqQuadratic}(\rho f/c_{n}, \beta \rho, K) = 0 \textrm{ in terms of } f$
  \With{$\rho \in [1\times 10^{-6}, 1.5\times 10^{-4}]$}
    \State \textrm{PhysicalQuadraticRoot} $\gets \textbf{Select }$ maximum root in QuadraticRoots
    \State \textrm{PhysicalQuarticRoot} $\gets \textbf{Select }$ root in QuarticRoots with same order of magnitude as PhysicalQuadraticRoot
    \State $\textbf{Plot }\textrm{ PhysicalQuadraticRoot against } \rho$
  \End
\end{algorithmic}

\section*{Rights Statement}

This work was supported by the Richard Norman Scholarship grant for the Department of Engineering, University of Cambridge.
For the purpose of open access, the authors have applied a Creative Commons Attribution (CC BY) licence to any Author Accepted Manuscript version arising.

% \bibliographystyle{naturemag}
% \bibliography{bibliography.bib}

\begin{thebibliography}{99}
  \bibitem[S1]{watson}
  Watson, G. N. A Treatise on the Theory of Bessel Functions (Cambridge University Press, Cambridge, 1966), second edn.

  \bibitem[S2]{wong}
  Wong, R. Asymptotic approximations of integrals. (Society for Industrial and Applied Mathematics, 2001).

\end{thebibliography}

\end{document}

%%% Local Variables:
%%% mode: latex
%%% TeX-master: t
%%% End:
