% CVPR 2024 Paper Template; see https://github.com/cvpr-org/author-kit

\documentclass[10pt,twocolumn,letterpaper]{article}


%%%%%%%%% PAPER TYPE  - PLEASE UPDATE FOR FINAL VERSION
% \usepackage{cvpr}              % To produce the CAMERA-READY version
% \usepackage[review]{cvpr}      % To produce the REVIEW version
\usepackage[pagenumbers]{cvpr} % To force page numbers, e.g. for an arXiv version

% \usepackage[subtle]{savetrees}

% Import additional packages in the preamble file, before hyperref
% \usepackage[latin1]{inputenc}
\usepackage[british]{babel}
\usepackage[all]{xy}
\usepackage{amscd}
\usepackage{amssymb}
\usepackage{amsthm}
\usepackage{enumitem}
\usepackage{mathrsfs,bbm}
\usepackage{xcolor,graphicx}
\usepackage{graphics}
\usepackage{soul}
\usepackage{comment}
\usepackage[all]{xy}
\usepackage{amscd}
\usepackage{amssymb,amsmath,latexsym}
\usepackage{amsthm}
\usepackage{enumitem}
\usepackage{mathrsfs,bbm}
\usepackage{dsfont}
\usepackage{tikz-cd}
\usepackage[T1]{fontenc}
\usepackage[utf8]{inputenc}  
 %
%%%%%%%%%%%%%%%%%%%%%%%%%%%%%%%%%%
%pagestyle
%%%%%%%%%%%%%%%%%%%%%%%%%%%%%%%%%%
%\pagestyle{plain}
\textwidth=430pt
\headsep=.7cm
\evensidemargin=15pt
\oddsidemargin=15pt
\leftmargin=0cm
\rightmargin=0cm
%%
%%%%%%%%%%%%%%%%%%%%%%%
\newcommand*\fixitem {\item[]%
  \refstepcounter{enumi}\hskip-\leftmargin\labelenumi\hskip\labelsep}
\newtheorem*{mainthm}{Main Theorem}
\newtheorem*{mainthm1}{Theorem}
\newtheorem*{maincor}{Corollary}
\usepackage[colorlinks=true]{hyperref}
\DeclareMathOperator{\Forall}{\forall}
\DeclareMathOperator{\Exists}{\exists}
\DeclareMathOperator{\ord}{ord}
\newcommand{\phiD}{\varphi_D}
\newcommand{\phiDI}{\varphi_{\mathbf{D}_I}}
\newcommand{\phiDIj}{\varphi_{\mathbf{D}_I (j)}}
\newcommand{\phiH}{\varphi_H}
\newcommand{\phiTimes}{\phiD \otimes \phiH}
\newcommand{\phiTimesDI}{\varphi_{\mathbf{D}_I} \otimes \phiH}
\newcommand{\R}{\mathscr{A}}
\newcommand{\X}{\mathscr{X}}
\newcommand{\Xf}{\mathscr{X}_{(k_0 ,i)}[r_0]}
\newcommand{\Xfr}{\mathscr{X}_{(k_0,i)}[r]}
\newcommand{\hotimes}{\widehat{\otimes}}
\newcommand{\C}{\mathbb{C}_p}
\newcommand{\V}{\mathscr{V}}
\newcommand{\B}{\mathscr{B}}
\newcommand{\dualD}{\mathfrak{D}}
\newcommand{\Dg}{\mathbf{D}}
\newcommand{\DD}{\mathcal{D}^0}
\newcommand{\DDg}{\mathcal{D}}
\newcommand{\DV}{\mathcal{D}}
\newcommand{\W}{\mathscr{W}_N}
\newcommand{\Ao}{\mathbf{A}^\circ}
\newcommand{\AoK}{\mathbf{A}^\circ_{\K}}
\newcommand{\AK}{\mathbf{A}_{/\K}}
\newcommand{\OOO}{\mathscr{A}^\circ}
\newcommand{\K}{\mathcal{K}} 
\newcommand{\OK}{\mathcal{O}_{\K}}
\newcommand{\varprojlog}[1]{\underleftarrow{\log\!^{#1}}}
\newcommand{\T}{\mathscr{T}}
\newcommand{\TT}{\mathbf{T}}
\newcommand{\VV}{\mathbf{V}}
\newcommand{\HH}{\mathcal{H}}
\newcommand{\hh}{\mathcal{H}^+}
\newcommand{\HG}[2]{\mathcal{H}_{#1}(#2)}
\newcommand{\hhl}{\mathcal{H}^{+,[l]}}
\newcommand{\hhj}{\mathcal{H}^{+,[j]}}
\newcommand{\hhjj}{\mathcal{H}^{+,[l,l']}}
\newcommand{\GS}{G_{\mathbb{Q},S}}
\newcommand{\Rf}{R_{(k_0 ,i)}[r_0]}
\newcommand{\Rfr}{R_{(k_0 ,i)}[r]}
\newcommand{\parT}{\langle T\rangle}
\newcommand{\Zf}{Z_{(k_0 ,i)}[r_0]}
\newcommand{\Zfr}{\mathscr{Z}_{(k_0 ,i)}[r]}
\newcommand{\ZFf}{\mathscr{Z}_{(k_0 ,i)}[r_0]}
\newcommand{\ZFfr}{\mathscr{Z}_{(k_0 ,i)}[r]}
\newcommand{\ZF}{\mathscr{Z}}

% It is strongly recommended to use hyperref, especially for the review version.
% hyperref with option pagebackref eases the reviewers' job.
% Please disable hyperref *only* if you encounter grave issues, 
% e.g. with the file validation for the camera-ready version.
%
% If you comment hyperref and then uncomment it, you should delete *.aux before re-running LaTeX.
% (Or just hit 'q' on the first LaTeX run, let it finish, and you should be clear).
\definecolor{cvprblue}{rgb}{0.21,0.49,0.74}

\usepackage{xurl}

%\PassOptionsToPackage{hyphens}{url}%\usepackage{hyperref}

\usepackage[pagebackref,breaklinks,colorlinks,citecolor=cvprblue]{hyperref}

%%%%%%%%% PAPER ID  - PLEASE UPDATE
\def\paperID{10974} % *** Enter the Paper ID here
\def\confName{CVPR}
\def\confYear{2024}


\title{Multi-modal learning for geospatial vegetation forecasting} %GreenEarthNet: 

\author{Vitus Benson\textsuperscript{1,2,3,*} \and Claire Robin\textsuperscript{1,2} \and Christian Requena-Mesa\textsuperscript{1,2} \and Lazaro Alonso\textsuperscript{1} \and Nuno Carvalhais\textsuperscript{1,2} \and José Cortés\textsuperscript{1} \and Zhihan Gao\textsuperscript{4} \and Nora Linscheid\textsuperscript{1} \and Mélanie Weynants\textsuperscript{1} \and Markus Reichstein\textsuperscript{1,2} \\
\textsuperscript{1} Max-Planck-Institute for Biogeochemistry  \quad \textsuperscript{2} ELLIS Unit Jena  \quad \textsuperscript{3} ETH Zürich  \\ \textsuperscript{4} Hong Kong University of Science and Technology  \quad \textsuperscript{*} {\tt vbenson@bgc-jena.mpg.de}
}


%%%%%%%%%%% TODO %%%%%%%%%%%%%%%%%%%

\iffalse

Dataset:
    - Rename as EarthNet2021 v2 (?)
    - Rephrase as a complete remake of EarthNet2021 (instead of expansion)
    - Or: GreenEarthNet Europe.
    - Say for compatibility locations + setup are based on EarthNet2021. Else many things different.
    - 

    
Figures:
    - Fig 1
    - Fig 2
    --> Look at accepted CVPR papers and copy their style !
Methods:
    - Need to rename ConvLSTM-meteo somehow. This should enhance visibility as a methodological contribution
    - Change order in Table also, renamed ConvLSTM-meteo should be last
Title:
    - More catchy. Show we are in scope of CVPR
Abstract:
    - Needs another iteration, also more in line with Rebuttal.
Discussion:
    - Talk about domain gap (like in Rebuttal)
Related Work
    - Need to be extended
    - Drop Fig 7 & Sec 4.5 instead
Other stuff
    - Other comments from ICCV reviews
    - Especially Re: Clarity we are not doing Gapfilling.

DEADLINE: Nov 10th 2023

\fi


%%%%%%%%%%%%%%%%%%%%%%%%%%%%%%%%%%%%





\begin{document}
\maketitle


%%%%%%%%% ABSTRACT
\begin{abstract}
    The innovative application of precise geospatial vegetation forecasting holds immense potential across diverse sectors, including agriculture, forestry, humanitarian aid, and carbon accounting. To leverage the vast availability of satellite imagery for this task, various works have applied deep neural networks for predicting multispectral images in photorealistic quality. However, the important area of vegetation dynamics has not been thoroughly explored. Our study breaks new ground by introducing GreenEarthNet, the first dataset specifically designed for high-resolution vegetation forecasting, and Contextformer, a novel deep learning approach for predicting vegetation greenness from Sentinel 2 satellite images with fine resolution across Europe. Our multi-modal transformer model Contextformer leverages spatial context through a vision backbone and predicts the temporal dynamics on local context patches incorporating meteorological time series in a parameter-efficient manner. The GreenEarthNet dataset features a learned cloud mask and an appropriate evaluation scheme for vegetation modeling. It also maintains compatibility with the existing satellite imagery forecasting dataset EarthNet2021, enabling cross-dataset model comparisons. Our extensive qualitative and quantitative analyses reveal that our methods outperform a broad range of baseline techniques. This includes surpassing previous state-of-the-art models on EarthNet2021, as well as adapted models from time series forecasting and video prediction. To the best of our knowledge, this work presents the first models for continental-scale vegetation modeling at fine resolution able to capture anomalies beyond the seasonal cycle, thereby paving the way for predicting vegetation health and behaviour in response to climate variability and extremes. We provide open source code and pre-trained weights to reproduce our experimental results under \url{https://github.com/vitusbenson/greenearthnet} \cite{benson_2024_10793870}.


    %Precise geospatial vegetation forecasting is valuable for a variety of applications in agriculture, forestry, humanitarian aid or carbon accounting. To leverage the vast availability of satellite imagery for this task, various works have begun training deep neural networks on predicting multispectral images in photorealistic quality, yet derived vegetation dynamics have not received as much attention. In this work, we introduce the first dataset tailored for high resolution vegetation forecasting, namely GreenEarthNet, and a deep learning-based solution, namely Contextformer, for predicting vegetation greenness from the Sentinel 2 satellite at fine resolution across Europe. Our multi-modal transformer model Contextformer leverages spatial context through a vision backbone and predicts the temporal dynamics on local context patches given driving meteorological variables in a parameter-efficient manner. The GreenEarthNet dataset introduces a learned cloud mask and an appropriate evaluation scheme for vegetation modeling, while maintaining compatibility with the satellite imagery forecasting dataset EarthNet2021, allowing the comparison of models across datasets. Qualitative and quantitative experiments demonstrate superior performance of our approach over a wide variety of baseline methods, including previous state-of-the-art approaches on EarthNet2021 and also adapted models from time series forecasting and video prediction. To the best of our knowledge, this work presents the first models for continental-scale vegetation modeling at fine resolution able to capture anomalies beyond the seasonal cycle, thereby paving the way for predictive assessments of vegetation status.
\end{abstract}


\begin{figure}[t]
    \centering
    \includegraphics[width=\columnwidth]{figures/fig1-2w3.pdf}  %fig1-v2-draft.pdf}
    \caption{Future vegetation status $\hat{V}$ is predicted with deep learning models $f$ from past satellite imagery $X$, past and future weather $C$ and elevation $E$. The dataset GreenEarthNet spans across Europe with minicubes split into train (red markers), temporal OOD test (ood-t, yellow) and spatio-temporal OOD test (ood-st, blue) subsets.}
    \label{fig:task}
\end{figure}

\section{Introduction}

Optical satellite images have been proven useful for monitoring vegetation status. This is essential for a variety of applications in agricultural planning, forestry advisory, humanitarian assistance or carbon monitoring. In all these cases, prognostic information is relevant: Farmers want to know how their farmland may react to a given weather scenario \cite{wolanin.etal_2019}. Humanitarian organisations need to understand the localized impact of droughts on pastoral communities for mitigation of famine with anticipatory action \cite{meshesha.etal_2020}. Afforestation efforts need to consider how their forests react to future climate \cite{sturm.etal_2022}. However, providing such prognostic information through fine resolution vegetation forecasts is challenging as it requires a model that considers ecological memory effects \cite{kraft.etal_2019}, spatial interactions and the influence of weather variations. Deep neural networks have proven successful at modeling relationships in space, time or across modalities. Hence, their application to vegetation forecasting given a sufficiently large dataset seems natural.

So far, deep learning in the domain of vegetation forecasting can be roughly grouped into two categories: low-resolution global vegetation forecasting and high-resolution local satellite imagery forecasting. The former \cite{ji.peters_2004,kraft.etal_2019, barrett.etal_2020, lees.etal_2022, zeng.etal_2022, shamseddin.gall_2023, martinuzzi.etal_2023} builds upon long-term measurements of vegetation status from the coarse resolution AVHRR and MODIS satellites. These methods overlook the heterogeneity within each pixel, e.g. a grassland will react very different than a forest, two neighbouring fields can have almost opposite dynamics depending on the type of crops (\textit{e.g.} winter wheat vs. maize), and vegetation on a north-facing slope close to a river will generally react different to drought stress than on a rocky south-facing slope. The latter \cite{requena-mesa.etal_2021, kladny2024enhanced, diaconu.etal_2022, robin.etal_2022, smith.etal_2023, tseng.etal_2023}, in contrast, aims at modeling the field-scale heterogeneity as observed from the Sentinel 2 and Landsat satellites through self-supervised learning. However, these approaches have so far focused on perceptual image quality instead of vegetation dynamics, which renders their suitability for vegetation forecasting uncertain. 

%So far, deep learning in the domain of vegetation forecasting can be roughly grouped into two categories: low-resolution global vegetation forecasting and high-resolution local satellite imagery forecasting. The former \cite{ji.peters_2004,kraft.etal_2019, barrett.etal_2020, lees.etal_2022, zeng.etal_2022, shamseddin.gall_2023, martinuzzi.etal_2023} builds upon long-term measurements of vegetation status from the AVHRR and MODIS satellites, at a resolution that can differ between ecosystems, but suppresses the heterogeneity within. The latter \cite{requena-mesa.etal_2021, kladny2024enhanced, diaconu.etal_2022, robin.etal_2022, smith.etal_2023, tseng.etal_2023} particularly aims at modeling the field-scale heterogeneity as observed from the Sentinel 2 and Landsat satellites through self-supervised learning, yet so far with a focus on perceptual image quality instead of vegetation dynamics. However, the former overlooks the heterogeneity within each pixel, e.g. a grassland will react very different than a forest, two neighbouring fields can have almost opposite dynamics, think winter wheat vs. maize, and vegetation on a north-facing slope close to a river will generally react different to drought stress than on a rocky south-facing slope.

The largest dataset for high resolution vegetation forecasting \cite{xiong.etal_2022} is called EarthNet2021 \cite{requena-mesa.etal_2021}. In theory, a model trained on EarthNet2021 can forecast satellite images of high perceptual quality. It is, however, harder to assess if this skill also translates to predicting derived vegetation dynamics. Here, the suitability of EarthNet2021 is limited by a faulty cloud mask, insufficient baselines and a poorly interpretable evaluation protocol. For instance, natural vegetation follows a strong seasonal cycle, making a vegetation climatology a necessary baseline to compare with, which is lacking on EarthNet2021. 

In this paper, we approach continental-scale modeling of vegetation dynamics. To achieve this, we predict remotely sensed vegetation greenness at 20m conditioned on coarse-scale weather. For this, we introduce the \emph{GreenEarthNet} dataset. It includes Sentinel 2 bands red, green, blue and near-infrared and a high quality deep learning-based cloud mask, which allows to distinguish between anomalies due to data corruption and those due to meteorological and anthropogenic influence. The training locations and spatial and temporal dimensions of GreenEarthNet are kept consistent with the EarthNet2021 dataset, which enables the re-use of the leading models ConvLSTM \cite{diaconu.etal_2022}, SGED-ConvLSTM \cite{kladny2024enhanced} and Earthformer \cite{gao.etal_2022} as baselines for vegetation forecasting. In other words, GreenEarthNet is a complete remake of the EarthNet2021 dataset, removing all its weaknesses and enabling self-supervised learning for geospatial vegetation forecasting. To advance state-of-the-art on the new dataset, we present a light-weight transformer model: the \emph{Contextformer}. It utilizes a Pyramid Vision Transformer \cite{wang.etal_2021a, wang.etal_2022c} as vision backbone and local context patches to make use of spatial interactions. It then models temporal dynamics conditioned on coarse-scale meteorology with a temporal transformer encoder. Finally, it puts a strong prior on persistence through a delta-prediction scheme starting from an initial, gap filled observation. Fig.~\ref{fig:task} presents a sketch of our GreenEarthNet approach: Future vegetation state ($\hat{V}$) is predicted from past satellite image spectra ($X$), past and future weather data ($C$), and elevation information ($E$) via a deep neural network: the Contextformer ($f$).

Our major \textbf{contributions} can be summarized as follows. \textbf{(1)} We present the GreenEarthNet dataset, the first large-scale dataset suitable for  prediction of within-year vegetation dynamics, including a learned cloud mask and a new evaluation scheme. \textbf{(2)} We introduce the Contextformer, a novel multi-modal transformer model suitable for vegetation forecasting, leveraging spatial context through its vision backbone, and forecasting the temporal evolution of small context patches with a temporal transformer. \textbf{(3)} We compare the Contextformer against a previously unseen variety of state-of-the-art models from related tasks and find it outperforms all of them both across metrics.

%Find our source code at \url{https://github.com/earthnet2021/earthnet-models-pytorch}.


\begin{figure*}[t]
    \centering
    \includegraphics[width=\textwidth]{figures/contextformer.pdf}
    \caption{The architecture of our proposed Contextformer.}
    \label{fig:contextformer}
\end{figure*}
\section{Related Work}
%HERE: Aim for 1/2 to 3/4 of a page for this!

\textbf{Vegetation forecasting}
%HERE: Write about previous approaches to Vegetation modeling: 1. Coarse Resolution, 2. Fine Resolution // EarthNet. Write that the Fine Resolution is necessary for field-scale analysis, but this has not yet been thoroughly done on a continental scale, and this work will fill the gap.
There is a growing interest in vegetation growth forecasting driven by the democratization of machine learning techniques, the availability of remote sensing data, and the urgency to address climate change \cite{ferchichi.etal_2022, kang2016forecasting, cui2020forecasting, lees2022deep}. Numerous studies in vegetation modeling use coarse resolution data from satellites like AVHRR or MODIS \cite{ji.peters_2004, kraft.etal_2019, lees.etal_2022, zeng.etal_2022, martinuzzi.etal_2023}. 
Since 2015, Sentinel-2 has provided high-resolution satellite imagery (up to $10$m), enabling more localized modeling. The introduction of EarthNet2021 \cite{requena-mesa.etal_2021} marked the first dataset for self-supervised Earth surface forecasting, which contains predicting satellite imagery and derived vegetation state with a focus on perceptual quality. Subsequently, the ConvLSTM model \cite{shi.etal_2015} has been widely used for satellite imagery prediction \cite{diaconu.etal_2022, kladny2024enhanced, ahmad.etal_2023, robin.etal_2022, ma.etal_2022a}, hence we are including it as a baseline.
%for continental-scale vegetation forecasting, facilitating deep field-scale analysis and aiding in the imputation of cloudy time steps \cite{meraner.etal_2020, yang.etal_2022}. It has also contributed to popularizing the interest and challenge of the task within both the remote sensing and machine learning communities.

\textbf{Spatio-temporal learning}
%HERE: Write about local time series modeling and LSTMs and under what circumstances that can be effective. Then write about ConvLSTM and where this was successful. And then write about Video Prediction, what appraoches have succeeded there. Finish saying since a priori for all of those models we could find some grounding that they may work for Vegetation forecasting, we try all of them as baselines.
Learning spatio-temporal dynamics (as in the case of vegetation forecasting) is a challenge across many disciplines. Often, temporal dynamics dominate, so local time series models can be effective. For instance in traffic, weather or electricity forecasting, time series models such as LSTM \cite{hochreiter.schmidhuber_1997}, Prophet \cite{taylor.letham_2018}, Autoformer \cite{wu.etal_2021a} or NBeats \cite{zeng.etal_2023} yield useful performance. Still, often spatial interactions are important or at least offer additional predictive capacity. For instance in video prediction, ConvNets \cite{babaeizadeh.etal_2021, gao.etal_2022a}, ConvLSTM \cite{shi.etal_2015}, ConvLSTM successors \cite{wang.etal_2023, wu.etal_2021}, PredRNN \cite{wang.etal_2017, wang.etal_2023}, SimVP \cite{tan.etal_2023} and transformers \cite{gupta.etal_2022a, nash.etal_2022a, gao.etal_2022} have been found skillful. Often, the necessity of modeling the spatial component translates to Earth science: spatio-temporal deep learning is being applied for precipitation nowcasting \cite{ravuri.etal_2021a, shi.etal_2017}, weather forecasting \cite{bi.etal_2022, lam.etal_2022, pathak.etal_2022}, climate projection \cite{nguyen.etal_2023}, and wildfire modeling \cite{kondylatos.etal_2022}. Hence, when evaluating our Contextformer model, we need to do so against strong baselines from video prediction \cite{shi.etal_2015, gao.etal_2022, wang.etal_2023, tan.etal_2023}, as a priori one might expect them to outperform also on vegetation forecasting. However, vegetation forecasting does present some unique challenges, it builds upon multi-modal data fusion and requires capturing across-scale relationships (in time and space), which may prove challenging for existing video prediction models and thus interesting to the computer vision community.

%To capture the spatio-temporal patterns in vegetation forecasting, the ConvLSTM model \cite{shi.etal_2015} has been widely used for satellite imagery prediction \cite{diaconu.etal_2022, kladny2024enhanced, ahmad.etal_2023, robin.etal_2022, ma.etal_2022a}, applied in regional crop yield modeling \cite{schwalbert.etal_2018, engen.etal_2021}, and regional vegetation forecasting \cite{ferchichi.etal_2022, yu.etal_2022}. Recently, transformer-based models also have been proposed \cite{gao.etal_2022, tseng.etal_2023}. We have chosen \cite{shi.etal_2015, gao.etal_2022, wang.etal_2017, tan.etal_2023} as promising baseline models for the vegetation forecasting task. 
%However, vegetation forecasting presents several unique challenges. The evolution of vegetation occurs in the temporal dimension, influenced by spatial context (e.g., trees, which do not move, are influenced by surrounding water availability). The task involves different time-lagged effects and heterogeneous data with varying spatial-temporal resolutions. These complexities make it a challenging task for models trained on video prediction datasets within the machine learning community.

\textbf{Multi-modal transformers for data fusion}
%HERE: Write about traditional data fusion approaches and then how transformers can be leveraged in that regard. Finish with Vision transformers and their role in geospatial foundation models.
Levering remote sensing data often means multi-modal data fusion. Recently, machine learning methods have shown significant advancements in fusing different satellite sensors compared to traditional approaches \cite{steinhausen2018combining, whyte2018new, audebert2018beyond, dalla2015challenges, li2022deep}. This includes recent work on combining Sentinel 2 and SAR data to impute cloudy Sentinel 2 images \cite{wang.etal_2019, meraner.etal_2020, yang.etal_2022}. Gapfilling vegetation time series could also be done with the models presented in this study, as they leverage meteorology to inform the imputation \cite{stucker.etal_2023}. However, as gapfilling is just done in retrospective, one should rather resort to complementary satellite data like SAR.

%Audebert et al. \cite{audebert2018beyond} utilize a multi-kernel convolutional layer to aggregate information at multiple scales for remote sensing segmentation.

Transformers \cite{vaswani.etal_2017} offer a compelling approach to handle multi-modal data \cite{jaegle.etal_2022}. Their efficacy in remote sensing has been shown multiple times \cite{ma2022crossmodal, gao.etal_2022, cong.etal_2022, tseng.etal_2023}. In particular, geospatial foundation models \cite{mendieta.etal_2023, reed.etal_2023, cong.etal_2022, yao.etal_2023, smith.etal_2023, tseng.etal_2023} make use of this approach, often through masked token modeling \cite{he.etal_2022} with Vision Transformers \cite{dosovitskiy.etal_2020}. Our Contextformer follows this line of research, yet in contrast to geospatial foundation models, it is more targeted for vegetation forecasting and only utilizes a pre-trained vision transformer as a vision backbone.

%Their attention mechanisms facilitate focused data processing and the capture of long-range dependencies.
%Ma et al. \cite{ma2022crossmodal} developed a multimodal fusion method based on Transformer architecture. SatMAE \cite{cong.etal_2022} incorporates positional encoding for temporal and spectral dimensions to handle temporal and multi-spectral input data. Earthformer \cite{gao.etal_2022} employs Cuboid attention, a space-time attention block, for spatio-temporal forecasting tasks. Finally, Presto \cite{tseng.etal_2023} is a lightweight Transformer-based model designed for remote sensing pixel-timeseries data, capable of handling inputs from different sensors and data products. However, these models suffer from large architecture, making fine-tuning on smaller datasets difficult \cite{gao.etal_2022, cong.etal_2022}. Although \cite{tseng.etal_2023} offers a lighter alternative, it overlooks the spatial dimension and is constrained by a coarse monthly temporal resolution—two crucial factors for vegetation forecasting tasks.


%\paragraph{Spatio-temporal learning}
%The ConvLSTM \cite{shi.etal_2015} was first introduced for precipitation nowcasting. Subsequently, spatio-temporal forecasting of the Earth system has gained traction, with strong results not only on precipitation nowcasting \cite{ravuri.etal_2021a, shi.etal_2017}, but also on weather forecasting \cite{bi.etal_2022, lam.etal_2022, pathak.etal_2022}, climate projection \cite{nguyen.etal_2023} and wildfire modeling \cite{kondylatos.etal_2022}. Beyond the Earth system, video prediction is spatio-temporal learning. Recent video prediction models use LSTM \cite{}ConvNets \cite{babaeizadeh.etal_2021, gao.etal_2022a}, ConvLSTM successors \cite{wang.etal_2023, wu.etal_2021}. A sub-area of video prediction uses action conditioning: predicting future frames by giving a future action in video games \cite{oh.etal_2015} or robot experiments \cite{babaeizadeh.etal_2018, finn.etal_2016}. Recently, transformers gained attention \cite{gupta.etal_2022a, nash.etal_2022a}. 


%\paragraph{Vegetation forecasting task}
%Vegetation modeling from remote sensing has a long tradition at coarse resolution, e.g. from the AVHRR or MODIS satellites \cite{ji.peters_2004,kraft.etal_2019, lees.etal_2022, zeng.etal_2022, martinuzzi2023learning}. Since 2015, the Sentinel 2 satellites deliver imagery at high resolution (up to $10$m). Several studies have used this data for regional crop yield modeling \cite{schwalbert.etal_2018, engen.etal_2021} and regional vegetation forecasting \cite{ferchichi.etal_2022, yu.etal_2022}. With EarthNet2021 \cite{requena-mesa.etal_2021}, the first dataset for continental-scale satellite imagery forecasting was introduced. Subsequent works leveraged the ConvLSTM model \cite{shi.etal_2015} for satellite imagery prediction \cite{diaconu.etal_2022, kladny2024enhanced, ahmad.etal_2023} and for vegetation prediction in Africa \cite{robin.etal_2022, ma.etal_2022a}. Another line of work focuses on imputing cloudy time steps \cite{meraner.etal_2020, yang.etal_2022}, yet often with a focus on historical gapfilling instead of near-realtime information. 


\section{Methods}

\subsection{Task}
We predict the future NDVI, a remote sensing proxy of vegetation state ($V^t \in \mathbb{R}^{H\times W}, t \in [T+1, T+K]$) conditioned on past satellite imagery ($X^t \in \mathbb{R}^{H\times W}, t \in [1, T]$), past and future weather ($C^t \in \mathbb{R}, t \in [1, T+K]$) and static elevation maps ($E \in \mathbb{R}^{H\times W}$). Hence, denoting a model $f(.;\theta)$ with parameters $\theta$, we obtain vegetation forecasts as:
\begin{align}
    \hat{V}^{T+1:T+K} = f(X^{1:T}, C^{1:T+K}, E; \theta)
\end{align}
In this paper most models are deep neural networks, trained with stochastic gradient descent to maximize a Gaussian Likelihood. More specifically, the optimal parameters $\theta^{*}$ are obtained by minimizing the mean squared error over valid pixels $V_{*}^t = V^t \odot M_Q^t \odot M_L$, where $M_Q \in \{0,1\}^{H\times W}$ masks pixels that are cloudy, cloud shadow or snow, $M_L \in \{0,1\}^{H\times W}$ masks pixels that are not cropland, forest, grassland or shrubland and $\odot$ denotes elementwise multiplication. Hence the training objective (leaving out dimensions for simplicity) is
\begin{align}
    \theta^{*} = \underset{\theta}{arg min} \frac{\sum (V - \hat{V})^2 \odot M_Q \odot M_L}{\sum M_Q \odot M_L}
\end{align}
In this work $H=W=128\text{px}$, $T=10$ and $K=20$.

%  \begin{figure*}[t]
%     \centering
%     \includegraphics[width=\textwidth]{figures/fig2draft-0.2}
%     \caption{Simplified view of evaluated models. Baselines (a,b), weather-guided deep learning (c,d,e,f).}
%     \label{fig:models}
% \end{figure*}


\subsection{Our proposed Contextformer model}\label{sec:model}
To tackle the vegetation forecasting task, we develop the Contextformer, a multi-modal transformer model operating on local spatial context patches (hence the name) and trained with self-supervised learning for predicting vegetation state across Europe. Next to historical satellite imagery, it leverages an elevation map and meteorological data.

\textbf{Overview} Our proposed Contextformer follows an \emph{encode-process-decode} \cite{battaglia.etal_2018a} configuration. Encoders and decoders operate in the spatial domain without temporal fusion, while the processor translates latent features temporally in local context patches. We use two encoders (meteo and vision), a temporal transformer processor and a decoder that predicts the delta from the last cloud free NDVI observation (see Fig.~\ref{fig:contextformer}).

\textbf{Encoders} The meteo encoder (for weather) and the delta decoders are parameterized as multi-layer perceptrons (MLPs) (Fig.~\ref{fig:contextformer} red boxes). For the vision encoder (Fig.~\ref{fig:contextformer} yellow boxes), we follow the MMST-ViT model for crop yield prediction \cite{lin.etal_2023} and use a Pyramid Vision Transformer (PVT) v2 B0 \cite{wang.etal_2021a, wang.etal_2022c}, which is particularly suitable for dense prediction tasks. It divides the images of each time step into patches of $4\times4$ px and then creates an embedding for each of the patches with a global receptive field. In other words, information is exchanged spatially at each time step, but not across time steps. We merge multi-scale features from the different stages of an ImageNet pre-trained PVT v2 B0, upscale them to our patches, concatenate and project. The resulting features for each image stack (satellite \& elevation) contain multi-scale and spatial context information.

\textbf{Masked Token Modeling} During training, we drop out patches in a masked token modeling approach \cite{he.etal_2022} and replace them with a learned masking token. We randomly ($p=0.5$) switch between inference mode, where we drop all patches for time steps $10$ to $30$, and random dropout mode, where we mask $70\%$ of the patches for time steps $3$ to $30$. At test time, we only use inference mode: the model never sees future satellite imagery. In addition, we use the cloud mask to drop every patch with at least one cloudy pixel. After applying the masking, a sinusoidal temporal positional encoding \cite{tseng.etal_2023} and the weather embeddings from the meteo encoder are added to each patch.

\textbf{Processor} The temporal transformer (Fig.~\ref{fig:contextformer} green box) processes patches in parallel, i.e. it exchanges information across the $30$ time steps, but spatially only within each $4\times4$ px patch.The idea here is that for ecosystem processes, spatial context is crucial but does not change dynamically. Therefore, separating spatial and temporal processing enhances efficiency. We maintain a small local context ($4\times4$ px) within the temporal encoder due to Sentinel 2's sub-pixel inaccuracies causing slight pixel shifts over time. This approach significantly reduces the model's memory cost during training (by $16\times$), enabling larger batch sizes. Our temporal transformer is implemented following Presto's transformer encoder \cite{tseng.etal_2023}, which is based on the standard vision transformer \cite{dosovitskiy.etal_2020}.

\textbf{Output} Our Contextformer leverages the persistence within the vegetation dynamics by predicting only a deviation from an initial state. More specifically, we compute the last cloud free NDVI observation from the historical period ($10$ time steps) using the cloud mask and use it as initial prediction $\hat{V}^0$. Then, the delta decoder predicts a deviation $\delta^i$ for each of the future period token embeddings (Fig.~\ref{fig:contextformer} right side). The final NDVI prediction is computed as $\hat{V}^i = \hat{V}^0 + \delta^i$.
%A similar delta framework has previously been leveraged for training the SGED-ConvLSTM \cite{kladny2024enhanced}, that model, however, predicts only one-step ahead in an iterative fashion, hence the deviation to the previous time step was predicted. In our multi-step prediction setting, this would correspond to a cumulative sum on the outputs, which is not desirable for training gradients.
A similar delta framework was used in training the SGED-ConvLSTM \cite{kladny2024enhanced}. However, that model predicts only one step ahead in an iterative fashion, predicting the deviation to the previous time step. In our multi-step prediction setting, this would result in a cumulative sum on the outputs, which is undesirable for training gradients.

\subsection{GreenEarthNet Dataset}
%HERE: Need to change this all to the GreenEarthNet framing! In particular, write less about en21, to make clear it is a complete remake.
%EarthNet2021 \cite{requena-mesa.etal_2021} is a dataset for Earth surface forecasting, that is weather-conditioned satellite image prediction.

We present GreenEarthNet, a tailored dataset for high-resolution geospatial vegetation forecasting. It contains spatio-temporal minicubes \cite{loaiza.etal_2023}, that are a collection of $30$ 5-daily satellite images ($10$ historical, $20$ future), $150$ daily meteorological observations and an elevation map. Spatial dimensions are $128\times 128$px ($2.56\times 2.56$km). To enable cross-dataset model comparisons, we re-use the training locations and predictor dimensions from the EarthNet2021 \cite{requena-mesa.etal_2021} dataset for Earth surface forecasting. %Hence, in the following, we focus on the difference to EarthNet2021, even though our dataset is generated completely independently and does not re-use actual data from EarthNet2021.


\begin{table}[]
    \centering
    \begin{tabularx}{\columnwidth}{Xcccc}
    \toprule
    & Works /w & & & \\
    Algorithm & GreenEarthNet & Prec & Rec & F1 \\
    \midrule
    Sen2Cor & Yes	& 0.83 & 0.60 & 0.70 \\
    FMask & No & 0.85 & 0.85 & 0.85\\
    KappaMask & No & 0.74 & 0.88 & 0.81\\
    \arrayrulecolor{black!30}\midrule
    UNet RGBNir & Yes & \underline{0.91} & 0.90 & \underline{0.90}\\
    \emph{UNet+Sen2CorSnow} & Yes & 0.83 & \textbf{0.93} & 0.88\\
    \arrayrulecolor{black!30}\midrule
    UNet 13Bands & No & \textbf{0.94} & \underline{0.92} & \textbf{0.93}\\
    \arrayrulecolor{black}\bottomrule
    \end{tabularx}
    \caption{Precision, recall and F1-score of different Sentinel~2 cloud masking algorithms.}
    \label{tab:cloudmask}
\end{table}





\textbf{Satellite and Meteo Layers} GreenEarthNet  includes Sentinel 2 \cite{louis.etal_2016} satellite bands blue, green, red, and near-infrared at $20$m (consistent with EarthNet2021) and E-OBS \cite{cornes.etal_2018} interpolated meteorological station data, which represents high quality meteorology over Europe \cite{bandhauer2022evaluation}. More specifically, the meteorological drivers wind speed, relative humidity, and shortwave downwelling radiation, alongside the rainfall, sea-level pressure, and temperature (daily mean, min \& max) are included. To enable reproducible research and minicube generation anywhere on Earth, we open source a Python package called \emph{EarthNet Minicuber}\footnote{\url{https://github.com/earthnet2021/earthnet-minicuber}}, which generates multi-modal minicubes in a cloud native manner: only downloading the data chunks actually needed, instead of a full Sentinel 2 tile.


\begin{figure*}[t]
    \centering
    \includegraphics[width=\textwidth]{figures/Figure3.png}
    \caption{Qualitative Results of Contextformer for one OOD-t minicube located near Oradea, Romania. The top-left shows timeseries for all pixels (mean and std. dev.) and for a single pixel (green square on top right). The right side shows image timeseries of cloud-masked target and predicted NDVI alongside their difference.}
    \label{fig:qualitative}
\end{figure*}

% The cloudmask in EarthNet2021 is faulty.
\textbf{Cloud mask} Vegetation proxies derived from optical satellite imagery are only meaningful if observations with clouds, shadows and snow are excluded such that anomalies due to clouds can be distinguished from vegetation anomalies. We train a UNet with Mobilenetv2 encoder \cite{sandler.etal_2018} on the CloudSEN12 dataset \cite{aybar.etal_2022} to detect clouds and cloud shadows from RGB and Nir bands. Tab.~\ref{tab:cloudmask} compares precision, recall and F1 scores for detecting faulty pixels. Our approach outperforms Sen2Cor \cite{louis.etal_2016} (used in EarthNet2021), FMask \cite{qiu.etal_2019} and KappaMask \cite{domnich.etal_2021} baselines by a large margin. If using Sen2Cor in addition, to allow for snow masking, precision drops, but recall increases: i.e. the cloud mask gets more conservative. Using all 13 Sentinel 2 L2A bands is better than just using 4 bands, however such a model is not directly applicable to GreenEarthNet.

\textbf{Test sets} Due to meso-scale circulation patterns, weather has high spatial correlation lengths. For GreenEarthNet, we design test sets ensuring independence not only in the high-resolution satellite data but also in the coarse-scale meteorology between training and test minicubes.
More specifically, we introduce the subsets
\begin{itemize}[noitemsep,topsep=0pt]
    \item \emph{Train}, 23816 minicubes in years 2017-2020
    \item \emph{Val} 245 minicubes close to training locations, year 2020
    \item \emph{OOD-t test} same locations as Val, years 2021-2022
    %\item \emph{IID} containes 245 minicubes from the EarthNet2021 IID testset, stratified by Sentinel 2 tile, year 2020
    \item \emph{OOD-s test}, 800 minicubes stratified over $1\degree \times 1\degree$ lat-lon grid cells outside training regions, years 2017-2019
    \item \emph{OOD-st test} same as OOD-s, but years 2021-2022
\end{itemize}
\emph{OOD-t} is the main test set used throughout this study. It tests the models' ability to extrapolate in time: i.e. we allow it to learn from past information about a location and want to know how it would perform in the future. \emph{val} follows the same reasoning and hence allows for early stopping of models according to their temporal extrapolation skill. \emph{OOD-s} and \emph{OOD-st} test spatial extrapolation, as well as spatio-temporal extrapolation. The test sets are compatible with EarthNet2021: the \emph{Val}/\emph{OOD-t} locations are inside the EarthNet2021 IID tests set and the \emph{OOD-s}/\emph{OOD-st} locations are far away from EarthNet2021 training data. For all test sets, we create minicubes over four periods during the European growing season \cite{rotzer.chmielewski_2001} each year: Predicting March-May (MAM), May-July (MJJ), July-September (JAS) and September-November (SON).

\textbf{Additional Layers} We add the ESA Worldcover Landcover map \cite{zanagadaniele.etal_2021} for selecting only vegetated pixels during evaluation, the Geomorpho90m Geomorphons map \cite{amatulli.etal_2020} for further evaluation and the ALOS \cite{tadono.etal_2016}, Copernicus \cite{esa_2021} and NASA \cite{crippen.etal_2016} DEMs, to provide uncertainty in the elevation maps. Finally, we provide georeferencing for each minicube, enabling their extension with further data.
%In contrast to EarthNet2021, we only provide one vector instead of a 3D tensor of meteorology, dropping the meso-scale surrounding of each minicube. This reduced the memory footprint of each minicube by $>5x$ and makes the task easier. Finally, we provide proper georeferencing, which was missing in EarthNet2021.



\subsection{Evaluation}
We resort to traditional metrics in environmental modeling:
\begin{itemize}[noitemsep,topsep=0pt]
    \item $R^2$, the squared pearson correlation coefficient
    \item $\text{RMSE}$, the root mean squared error
    \item $\text{NSE} = 1 - \dfrac{MSE(V, \hat{V})}{Var[V]}$, the nash-sutcliffe efficiency \cite{nash.sutcliffe_1970}, a measure of relative variability
    \item $|\text{bias}| = | \overline{V} - \overline{\hat{V}} |$, the absolute bias
\end{itemize}
In addition, we propose to measure if a model is better than the NDVI climatology, by computing the \emph{Outperformance score}: the percentage of minicubes, for which the model is better in at least 3 out of the 4 metrics. Here, better means their score difference (ordering s.t. higher=better) exceeds $0.01$ for RMSE and $|\text{bias}|$ and $0.05$ for NSE and $R^2$. We also report the RMSE over only the first 25 days (5 time steps) of the target period. 

We compute all metrics per pixel over clear-sky timesteps. We then consider only pixels with vegetated landcover (cropland, grassland, forest, shrubland), no seasonal flooding (minimum NDVI $>0$), enough observations ($\geq10$ during target period, $\geq3$ during context period) and considerable variation (NDVI std. dev $>0.1$). All these pixelwise scores are grouped by minicube and landcover, and then aggregated to account for class imbalance. Finally, the macro-average of the scores per landcover class is computed. In this way, the scores represent a conservative estimate of the expected performance of dynamic vegetation modeling during a new year or at a new location. 



\begin{table*}[t]
    \centering
    \begin{tabularx}{\textwidth}{>{\footnotesize\scshape}cXccccccc}
    \toprule
    & Model & $R^2$ $\uparrow$ & RMSE $\downarrow$ & NSE $\uparrow$ &   $|\text{bias}|$ $\downarrow$ & $\underset{\text{Climatology}}{\text{Outperform}}$ $\uparrow$& $\underset{25 \text{ days}}{\text{RMSE}}$ $\downarrow$& \#Params\\
    \midrule
    \parbox[t]{1mm}{\multirow{3}{*}{\rotatebox[origin=c]{90}{non-ML}}}& Persistence & 0.00 &      0.23 &      -1.28 &      0.17 &      21.8\% &      0.09 & 0 \\
    & Previous year &  0.56 &      0.20 &      -0.40 &      0.14 &      19.3\% &      0.18  & 0 \\
    & Climatology & 0.58 &      0.18 &      -0.34 &      0.13 &       n.a. &      0.16 & 0 \\
    \arrayrulecolor{black!30}\midrule
    \parbox[t]{1mm}{\multirow{3}{*}{\rotatebox[origin=c]{90}{local TS}}} & Kalman filter &      0.41 &      0.19 &      -0.57 &      0.13 &      27.0\% &      0.16  & $\mathcal{O}$(10)\\
    & LightGBM &      0.51 &      0.17 &      -0.22 &      0.12 &      42.2\% &      0.11  & n.a.\\
    & Prophet & 0.57 &      0.16 &      -0.05 &      0.11 &      60.6\% &      0.13  & $\mathcal{O}$(10) \\
    \arrayrulecolor{black!30}\midrule
    \parbox[t]{1mm}{\multirow{3}{*}{\rotatebox[origin=c]{90}{EN21}}} & ConvLSTM \cite{diaconu.etal_2022} & 0.51 &      0.18 &      -0.37 &      0.12 &      43.9\% &      0.12  & 0.2M \\
    & SG-ConvLSTM \cite{kladny2024enhanced} & 0.53 &      0.19 &      -0.33 &      0.14 &      45.8\% &      0.11  & 0.7M \\
    & Earthformer \cite{gao.etal_2022} &  0.49 &      0.17 &      -0.27 &      0.12 &      47.2\% &      0.11  &  60.6M\\
    \arrayrulecolor{black!30}\midrule
    \parbox[t]{1mm}{\multirow{5}{*}{\rotatebox[origin=c]{90}{This Study}}} & ConvLSTM \cite{robin.etal_2022} & 0.58 \footnotesize{$\pm$0.01} & 0.16 \footnotesize{$\pm$0.00} &  -0.13 \footnotesize{$\pm$0.02} & 0.11 \footnotesize{$\pm$0.00} & 53.1\% \footnotesize{$\pm$1.2\%} & 0.11 \footnotesize{$\pm$0.00} & 1.0M \\
    & Earthformer \cite{gao.etal_2022} & 0.52 &      0.16 &      -0.13 &      0.10 &      56.5\% & 0.09 & 60.6M\\
    & PredRNN \cite{wang.etal_2023} & \textbf{0.62} \footnotesize{$\pm$0.00} & 0.15 \footnotesize{$\pm$0.00} &  0.03 \footnotesize{$\pm$0.00} & 0.10 \footnotesize{$\pm$0.00} & 64.7\% \footnotesize{$\pm$1.2\%} & 0.10 \footnotesize{$\pm$0.00} & 1.4M \\
    & SimVP \cite{tan.etal_2023} & 0.60 \footnotesize{$\pm$0.00} & 0.15 \footnotesize{$\pm$0.00} &  0.03 \footnotesize{$\pm$0.01} & \textbf{0.09} \footnotesize{$\pm$0.00} & 64.1\% \footnotesize{$\pm$1.0\%} & 0.10 \footnotesize{$\pm$0.00} & 6.6M \\
    & Contextformer (Ours) & \textbf{0.62} \footnotesize{$\pm$0.00} & \textbf{0.14} \footnotesize{$\pm$0.00} &  \textbf{0.09} \footnotesize{$\pm$0.01} & \textbf{0.09} \footnotesize{$\pm$0.00} & \textbf{66.8\%} \footnotesize{$\pm$0.3\%} & \textbf{0.08} \footnotesize{$\pm$0.00} & 6.1M \\
    %\quad no weather & 0.48& 0.49 & -8.14 &  0.46 & 0.0\% & 34.1M & $\sim$0.05s\\
    \arrayrulecolor{black}\bottomrule
    \end{tabularx}
    \caption{Quantitative Results. Mean ($\pm$std. dev.) are computed from three different random seeds.}
    \label{tab:quantitative}
\end{table*}



\subsection{Baselines}\label{sec:baselines}
We evaluate Contextformer against diverse baselines representative of various model classes, including non-ML methods, time series forecasting models, the top 3 performers on the EarthNet2021 benchmark \cite{requena-mesa.etal_2021}, classical, and two state-of-the-art video prediction models. This choice aims to account for uncertainty regarding the optimal models for vegetation forecasting, considering factors such as the relevance of spatial context. While existing spatio-temporal earth surface forecasting models are expected to serve as strong baselines due to task similarity, recent advancements in video prediction, leveraging the perspective of satellite image time series as a video, may also offer a competitive advantage.


%HERE: Introduce baselines and why they were chosen. But keep technical details rather brief.

%We compare against several baselines:

%This study focuses modeling around meteo-guided deep learning. We study in-depth four models which are representative for their respective model class. Two models perform next-frame prediction and leverage internal memory (ConvLSTM and PredRNN) to perform iterative roll-out. Two models perform next-cuboid prediction (SimVP and Earthformer), thereby modeling the full target period temporal dynamics at once. PredRNN, SimVP and Earthformer follow an \emph{encode-process-decode} \cite{battaglia.etal_2018a} configuration. Encoders and decoders operate in the spatial domain without any temporal fusion, while the processor translates latent features spatio-temporally. For encoding and decoding we leverage ConvNets, which are the standard in the domain of satellite remote sensing. The models are sketched in fig.~\ref{fig:models} and described below.

\textbf{Non-ML baselines} We evaluate three non-ML baselines related to ecological memory: persistence \cite{requena-mesa.etal_2021} (last cloud free NDVI pixel), previous year \cite{robin.etal_2022} (linearly interpolated) and climatology (mean NDVI seasonal cycle). 

\textbf{Local time series models} We compare against three common time series models: Kalman filter, LightGBM \cite{ke.etal_2017} and Prophet \cite{taylor.letham_2018} from the Python library darts \cite{herzen.etal_2022}. These are trained on timeseries from a single pixel and applied to forecast this pixel, given future weather as covariates. They are expensive to run: a single minicube takes $\sim3$h on an 8-CPU machine, $\mathcal{O}(10^4)$ slower than deep learning. We also evaluate a global timeseries model: the LSTM (implemented as ConvLSTM with 1x1 kernel). The time series models should be strong if spatial interactions are less predictive for vegetation evolution.


\textbf{EarthNet2021 models} 
We also evaluate the Top-3 models from the EarthNet2021 challenge leaderboard\footnote{\url{https://web.archive.org/web/20230228215255/https://www.earthnet.tech/en21/ch-leaderboard/}} using their trained weights: a regular ConvLSTM \cite{diaconu.etal_2022}, an encode-process-decode ConvLSTM called SGED-ConvLSTM \cite{kladny2024enhanced} and the Earthformer \cite{gao.etal_2022}, a transformer model using cuboid-attention. 

Additionally, we train and fine-tune both the ConvLSTM and Earthformer on GreenEarthNet. For the ConvLSTM, we follow the original Shi et al. \cite{shi.etal_2015} encoding-forecasting setup, which is different from ConvLSTM flavors studied on EarthNet2021 \cite{diaconu.etal_2022, kladny2024enhanced} but has demonstrated improved performance on a similar problem in Africa \cite{robin.etal_2022}. We condition the Earthformer \cite{gao.etal_2022} through early fusion during historical steps and latent fusion during future steps.

%\textbf{LSTM} A much faster timeseries model is the LSTM when trained globally. We implement a pixelwise LSTM as a ConvLSTM with 1x1 kernel size. It can not make use of spatial context, but does use temporal memory.


\textbf{Video prediction models}
We adapt two state-of-the-art video prediction models (PredRNN and SimVP) and two basic UNet-based approaches. The next-frame UNet \cite{rasp.etal_2020} predicts autoregressively one step ahead. The next-cuboid UNet \cite{requena-mesa.etal_2021} directly predicts all time steps, taking the historical time steps stacked in the channel dimension. PredRNN \cite{wang.etal_2017, wang.etal_2023} is an autoregressive model with improved information flow. We generalize the action-conditioned PredRNN \cite{wang.etal_2023} by using feature-wise linear modulation \cite{perez.etal_2018} for weather conditioning on the inputs. SimVP \cite{tan.etal_2023} performs direct multi-step prediction through an encode-process-decode ConvNet, we adapt it with weather conditioning by feature-wise linear modulation \cite{perez.etal_2018} on the latent embeddings at each stage of the processor.


%Video prediction akin to satellite imagery forecasting and vegetation forecasting. We evaluate two basic UNets and two state-of-the-art models. The next-frame UNet predicts autoregressively without memory the vegetation of the next time step. This is a common baseline for weather prediction \cite{rasp.etal_2020}. The next-cuboid UNet operates in chunks, stacking all context time steps along the channel dimension and outputting all target time steps simultaneously, akin to SimVP but with early spatio-temporal fusion. 

%We propose a network, inspired by the action-conditioned PredRNN \cite{wang.etal_2023}, that employs ConvNet encoder and decoder and conditioning in each memory cell, generalizing action-conditioning through feature-wise linear modulation \cite{perez.etal_2018} for weather conditioning.

%The PredRNN video prediction model \cite{wang.etal_2017, wang.etal_2023} is a ConvLSTM with two memory states: one for longer-term dynamics that passes information at the same depth level over time, and one for more complex short-term dynamics, with an information flow that zigzags through levels over time (called ST-LSTM). We propose a similar network to the action-conditioned PredRNN \cite{wang.etal_2023}: using ConvNet encoder and decoder and conditioning in each memory cell. We generalize their action-conditioning by using feature-wise linear modulation \cite{perez.etal_2018} for weather conditioning on the inputs. 

%The SimVP video prediction model \cite{tan.etal_2023} is an encode-process-decode model with a ConvNet processor called Gated Spatiotemporal Attention Translator. It achieves temporal modeling by stacking the features of all time steps along the channel dimension. Each block then processes first with depth-wise convolution in the spatial domain, then with channel-wise convolutions in the temporal domain and finally gates with an attention layer. We achieve weather conditioning by feature-wise linear modulation \cite{perez.etal_2018} on the latent embeddings at each stage of the processor.
%Video prediction, akin to satellite imagery and vegetation forecasting, is assessed using two basic UNets and two advanced models. The next-frame UNet predicts the next vegetation frame autoregressively, a standard baseline for tasks like weather prediction \cite{rasp.etal_2020}. The next-cuboid UNet operates in chunks, stacking all context time steps along the channel dimension and outputting all target time steps simultaneously, akin to SimVP but with early spatio-temporal fusion. Our proposed network, inspired by the action-conditioned PredRNN \cite{wang.etal_2023}, employs ConvNet encoder and decoder with conditioning in each memory cell, generalizing action-conditioning through feature-wise linear modulation \cite{perez.etal_2018} for weather conditioning. The PredRNN video prediction model \cite{wang.etal_2017, wang.etal_2023} is a ConvLSTM with two memory states: one for longer-term dynamics passing information at the same depth level over time and another for more complex short-term dynamics with zigzagging information flow through levels over time (ST-LSTM). The SimVP video prediction model \cite{tan.etal_2023} is an encode-process-decode model with a ConvNet processor called Gated Spatiotemporal Attention Translator. It achieves temporal modeling by stacking features of all time steps along the channel dimension, processing with depth-wise convolution in the spatial domain, channel-wise convolutions in the temporal domain, and gating with an attention layer. Weather conditioning is achieved through feature-wise linear modulation \cite{perez.etal_2018} on latent embeddings at each stage of the processor.



%\textbf{ConvLSTM-meteo} We follow the original ConvLSTM work \cite{shi.etal_2015} and use an encoding-forecasting setup (fig.~\ref{fig:models}c). It consists of two networks, each containing two ConvLSTM cells, without parameter sharing: One for the context period which works with past satellite imagery and past weather, and one for the target period, only using future weather as input. This is in contrast to the ConvLSTM flavors previously studied on EarthNet2021 \cite{diaconu.etal_2022, kladny2024enhanced}, but has been shown to work better on a similar problem in Africa \cite{robin.etal_2022}. 

%\textbf{PredRNN-meteo} The PredRNN video prediction model \cite{wang.etal_2017, wang.etal_2023} is a ConvLSTM with two memory states: one for longer-term dynamics that passes information at the same depth level over time, and one for more complex short-term dynamics, with an information flow that zigzags through levels over time (called ST-LSTM). We propose a similar network to the action-conditioned PredRNN \cite{wang.etal_2023}: using ConvNet encoder and decoder and conditioning in each memory cell. We generalize their action-conditioning by using feature-wise linear modulation \cite{perez.etal_2018} for weather conditioning on the inputs (fig.~\ref{fig:models}e).

%\textbf{SimVP-meteo} The SimVP video prediction model \cite{tan.etal_2023} is an encode-process-decode model with a ConvNet processor called Gated Spatiotemporal Attention Translator. It achieves temporal modeling by stacking the features of all time steps along the channel dimension. Each block then processes first with depth-wise convolution in the spatial domain, then with channel-wise convolutions in the temporal domain and finally gates with an attention layer. We achieve weather conditioning by feature-wise linear modulation \cite{perez.etal_2018} on the latent embeddings at each stage of the processor (fig.~\ref{fig:models}d).

%\textbf{Earthformer-meteo} The Earthformer Earth system forecasting model \cite{gao.etal_2022} uses cuboid attention to process spatio-temporal chunks of information. It uses different cuboids of each input tensor as tokens for self- and cross-attention. Multiple cuboid attention modules are composed in a UNet like architecture. To tame the memory usage of the attention mechanisms, ConvNet encoder and decoder are used. Weather conditioning is achieved with early fusion during context steps and latent fusion during target steps (fig.~\ref{fig:models}f).



\subsection{Implementation details} We build all of our ConvNets with a PatchMerge-style architecture similar to the one in Earthformer \cite{gao.etal_2022}. For SimVP and PredRNN, such encoders and decoders are more powerful, but also slightly more parameter-intensive, than the variants used in the original papers. We use GroupNorm \cite{wu.he_2018} and LeakyReLU activation \cite{xu.etal_2015b} for the ConvNets, and  and ConvLSTMs. For the Contextformer, we use LayerNorm \cite{ba2016layer} and GELU activation \cite{hendrycks2023gaussian}. For ConvNets, skip connections preserve high-fidelity content between encoders and decoders. Our framework is implemented in PyTorch, and models are trained on Nvidia A40 and A100 GPUs. We use the AdamW \cite{loshchilov.hutter_2022} optimizer and tune the learning rate and a few hyperparameters per model. More implementation details can be found in the supplementary materials.


\section{Experiments}




\subsection{Baseline comparison}
We conduct experiments for predicting vegetation state across Europe in 2021 and 2022 at $20m$ resolution and compare the Contextformer against a wide range of baselines. The quantitative results are shown in table~\ref{tab:quantitative}. For Contextformer, ConvLSTM, PredRNN and SimVP, we report the mean ($\pm$std. dev.) from three different random seeds. Earthformer has an order of magnitude more parameters, making training more expensive, which is why we only report one random seed. We find the Contextformer outperforms (or performs on par with) every baseline on all metrics. It achieves $R^2 = 0.62$ and $0.14$ RMSE on the full 100 days lead time, which is further improved to $0.08$ RMSE during the first 25 days lead time. The closest competitors are PredRNN and SimVP, with PredRNN having on par $R^2 = 0.62$ and SimVP on par $|\text{bias}| = 0.09$.

The Contextformer and the other video prediction baselines trained in this study are the first models to outperform the Climatology baseline: the ConvLSTM reaches $53.1\%$ outperformance score, while the Contextformer achieves $66.8\%$ (with consistent ranking across thresholds, see sec.~\ref{sec:robustoutpf}). For the top-3 models (PredRNN, SimVP and Contextformer) and all metrics, differences to the climatology are highly significant when tested for all pixels (with Wilcoxon signed-rank test, $\alpha = 0.001$), but also for each land cover or for smaller subsets of $100$ minicubes. ConvLSTM and Earthformer have overall lower skill. They mostly excels at RMSE and $|\text{bias}|$, where they can perform similar to other methods, yet have way lower performance for NSE and $R^2$.


\begin{figure}
    \centering
    \includegraphics[width=\columnwidth]{figures/fig4}
    \caption{Model performance comparing meteo-guided models (blue) with the ablation not using weather (black bar is std. dev. from three random seeds).}
    \label{fig:horizon}
\end{figure}


The models trained on EarthNet2021 (ConvLSTM \cite{diaconu.etal_2022}, SGED-ConvLSTM \cite{kladny2024enhanced} and Earthformer \cite{gao.etal_2022}) perform poorly. None of the approaches consistently beats the Climatology, particularly the $R^2$ is much lower (from $0.49$ for Earthformer up to $0.53$ for SGED-ConvLSTM). Likely, this is a result of the focus on perceptual quality that was reflected in the EarthNet2021 metrics, as well as the overall lower data quality due to a faulty cloud mask. 

Finally, other local time series baselines and non-ML baselines also underperform the Contextformer. The strongest pixelwise model is Prophet \cite{taylor.letham_2018}, with an outperformance score of $60.6\%$, followed by the climatology. Note, all of these baselines have access to a lot more information than the deep learning-based models (6 years vs. 50 days). Hence, this model comparison gives an indication, that spatial context is useful for vegetation forecasting, but leveraging them is challenging, as temporal dynamics are more dominant. In addition, the local time series models are all very slow, compared to the deep learning solutions presented in this work, which perform predictions within seconds (see sec.~\ref{sec:speed}).



\begin{table}
\centering
\begin{tabularx}{\columnwidth}{Xccc}
\toprule
& Original & Shuffled & \\
Model & $R^2$ $\uparrow$ & $R^2$ $\uparrow$ & Diff $\uparrow$ \\
\midrule
Climatology & 0.58 & - & - \\
\arrayrulecolor{black!30}\midrule
1x1 LSTM & 0.53 & 0.53 & 0.00 \\
Next-frame UNet & 0.51 & 0.48 & -0.03 \\
Next-cuboid UNet & 0.56 & 0.43 & -0.13 \\
\arrayrulecolor{black!30}\midrule
ConvLSTM & 0.58 & 0.46 & -0.12 \\
PredRNN & 0.62 & 0.45 & -0.17 \\
SimVP & 0.60 &  0.49 & -0.11 \\
Contextformer & 0.62 & 0.55 & -0.07 \\
\arrayrulecolor{black}\bottomrule
\end{tabularx}
\caption{Model skill when spatial interactions are broken through shuffling.}
\label{tab:spatiotemp}
\end{table}



%However, previous work on satellite imagery forecasting is applicable, since the NDVI, our vegetation proxy, can be derived from the red and near-infrared channels. Hence, we evaluate the Top-3 models from the EarthNet2021 challenge leaderboard\footnote{\url{https://web.archive.org/web/20230228215255/https://www.earthnet.tech/en21/ch-leaderboard/}} using their trained weights: a regular ConvLSTM \cite{diaconu.etal_2022}, an encode-process-decode ConvLSTM called SGED-ConvLSTM \cite{kladny2024enhanced} and the Earthformer \cite{gao.etal_2022}.

%We compare these against three Non-ML baselines: persistence, previous year and climatology. Note, the climatology uses a lot more information than our models (6 years vs. 50 days). Additionally, we compare with Kalman filter, LightGBM \cite{ke.etal_2017} and Prophet \cite{taylor.letham_2018}, local timeseries forecasting models, which also work with the full timeseries instead of just 50 context days.

%We introduce four new model variants: ConvLSTM-meteo, PredRNN-meteo, SimVP-meteo and Earthformer-meteo (see sec.~\ref{sec:models}). These are weather-guided extensions of four state-of-the-art approaches to video prediction, each belonging to a different model class. For ConvLSTM-meteo, PredRNN-meteo and SimVP-meteo, we report the mean ($\pm$std. dev.) from three different random seeds. Earthformer-meteo has an order of magnitude more parameters, making training more expensive, which is why we only report one random seed.

%The quantitative results are shown in table~\ref{tab:quantitative}. Both the climatology and Prophet are strong baselines, which outperform all of the top-3 models from the EarthNet2021 challenge. SimVP, PredRNN and ConvLSTM outperform all baselines on all metrics except for the 25-day RMSE, where a persistence baseline is slightly stronger. For all three models and metrics, differences to the climatology are highly significant when tested for all pixels (with Wilcoxon signed-rank test, $\alpha = 0.001$), but also for each land cover or for smaller subsets of $100$ minicubes. Earthformer-meteo, has overall lower skill. It mostly excels at RMSE and $|\text{bias}|$, where it can perform similar to other methods, yet has way lower performance for NSE and $R^2$. Here, NSE may be weak and RMSE good because we aggregate over the full dataset, hence indicating spatial patterns of model skill. %When not using the weather, SimVP and PredRNN exhibit large drops in performance, especially for $R^2$ and $NSE$.

Qualitative results of the Contextformer model for one minicube are reported in fig.~\ref{fig:qualitative}. The model clearly learns the complex dynamics of vegetation, with a strong seasonal evolution of the crop fields. It interpolates faithfully those pixels, which are masked in the target, and contains strong temporal consistency. However, as the lead time increases, predictions become less explicit, with a tendency towards oversmoothing.

\subsection{Weather guidance}
Our meteo-guided models benefit from the weather conditioning. Fig.~\ref{fig:horizon} compares ConvLSTM, PredRNN, SimVP and Contextformer (blue) against a variant without weather conditioning (orange). For all metrics, using weather outperforms not using it. The ConvLSTM has the largest performance gain due to meteo-guidance, yet it is also the weakest model. This could possibly be due to the ConvLSTMs smaller receptive field and hence lower capacity at leveraging spatial context, which may to some degree compensate predictive capacity from weather. %This could possibly be since it does not explicitly model memory effects, but rather learns to disentangle the temporal evolution in one piece. The ConvLSTM without weather has only slightly lower skill than the SimVP-meteo.  

For PredRNN and SimVP, we conduct an extended ablation study on weather guidance (see supplementary material). Weather conditioning methods (concatenation, FiLM \cite{perez.etal_2018}, and cross-attention \cite{rombach.etal_2022}) have a minor impact on performance when applied appropriately: cross-attention is most useful with latent fusion, FiLM outperforms concatenation, and is suitable for early fusion.

\subsection{The role of spatial interactions}
Unlike video prediction, satellite images show minimal spatial movement. Field and forest boundaries remain mostly fixed, with the largest variations occurring within these edges over time. Hence, it is unclear whether spatio-temporal models, accounting for interactions, are suitable for modeling vegetation dynamics. However, at $20m$ resolution, lateral processes may occur, not captured by predictors. For example, grasslands near a river or on a north-facing slope may react differently to meteorological drought. Additionally, weather affects trees at the forest edge differently from those in the center.

We compare model performance with spatially shuffled input, i.e. explicitly breaking spatial interactions \cite{requena-mesa.etal_2019}. We shuffle across batch and space, to also destroy image statistics. We evaluate Contextformer, ConvLSTM, PredRNN, and SimVP, skipping Earthformer due to high training cost. In addition we also study three baselines: a pixelwise (1x1) LSTM, the next-frame UNet and the next-cuboid UNet (see sec.~\ref{sec:baselines}). The pixelwise LSTM is a global timeseries model unable to capture spatial interactions. The next-frame UNet models spatial interactions, but does not consider temporal memory. All other models can leverage spatio-temporal dependencies, though the ConvLSTM only has a small local receptive field ($\sim100$m around each pixel). The results are reported in tab.~\ref{tab:spatiotemp}. As can be expected, the pixelwise LSTM can be trained with spatial shuffled pixels without performance loss. All other models, though, exhibit a drop in performance under pixel shuffling. For Contextformer, ConvLSTM, PredRNN and SimVP, $R^2$ drops by at least $0.07$ and $RMSE$ increases by at least $0.04$. %For PredRNN, SimVP and Next-cuboid UNet it can be very large, as they have large receptive fields. For the next-frame UNet it is smaller, as it itself is not a very skillful model. The Conv\-LSTM also exhibits only a small performance drop, which may be due to its local receptive field. In turn, this may indicate that spatial interactions relevant for vegetation response to weather are of rather local nature and long-range interactions seldom important.



\begin{table}
\centering
\begin{tabularx}{\columnwidth}{Xcccc}
\toprule
Ablation & $R^2$ $\uparrow$ & RMSE $\downarrow$ & $\underset{\text{Climatology}}{\text{Outperf}}$ $\uparrow$ \\
\midrule
MLP vision encoder & 0.58 & 0.15 & 58.3\%\\
\arrayrulecolor{black!30}\midrule
PVT encoder (frozen) & 0.57 & 0.17 & 46.1\% \\
PVT encoder & 0.62 & 0.15 & 62.3\% \\
\arrayrulecolor{black!30}\midrule
\quad /w cloud mask token & 0.61 & 0.16 & 61.8\% \\
\quad /w learned $\hat{V}^0$ & 0.62 & 0.16 & 60.6\% \\
\quad /w last pixel $\hat{V}^0$ & 0.62 & 0.15 & 65.1\% \\
%\quad /w $\hat{V}_0$ \& mask clouds & 0.62 & 0.14 & 66.9\% \\
%\quad /w $\hat{V}_0$ \& $\bar{\mu}$ & 0.62 & 0.15 & 65.7\% \\
\arrayrulecolor{black!30}\midrule
Contextformer-6M & 0.62 & 0.14 & 66.8\% \\
Contextformer-16M & 0.61 & 0.14 & 67.3\% \\
\arrayrulecolor{black}\bottomrule
\end{tabularx}
\caption{Model ablations. The Contextformer uses a PVT encoder, a cloud mask token and the last cloud free pixel as $\hat{V}_0$.}
\label{tab:ablation}
\end{table}


\subsection{Ablation Study of Contextformer components}
We conduct experiments to show how each key component in our Contextformer affects predictive skill. Tab.~\ref{tab:ablation} lists the results of our ablation studies. First, we find that continued training of a pre-trained PVT vision encoder (outperformance score $62.3$\%) outperforms both a MLP vision encoder and a frozen pre-trained PVT. Second, adding the delta-prediction scheme with an initial vegetation state estimate $\hat{V}^0$ constructed by the last historical cloud free NDVI pixel further improves the outperformance to $65.1\%$ -- the version directly predicting NDVI is \emph{PVT encoder}. Instead using a learned MLP decoder to estimate $\hat{V}^0$ is inferior. Third, using the cloud mask to drop out faulty tokens from the PVT encoder decreases model skill, if used alone, but if used on top of the delta-prediction scheme in the final model Contextformer-6M, it gives another boost to $66.8\%$ outperformance. Finally, scaling the model size of the Contextformer to 16M parameters is not helpful when trained on GreenEarthNet, indicating the need for an even larger dataset for further performance gains.



\begin{figure}
    \centering
    \includegraphics[width=\columnwidth]{figures/R2_over_predtime.pdf}
    \caption{Contextformer model skill for different seasons and landcover on the OOD-t test set.}
    \label{fig:seasons}
\end{figure}



\begin{table}
\centering
\begin{tabularx}{\columnwidth}{Xcccc}
\toprule
& \multicolumn{2}{c}{OOD-s} & \multicolumn{2}{c}{OOD-st} \\
Model & $R^2$ $\uparrow$ & RMSE $\downarrow$ & $R^2$ $\uparrow$ & RMSE $\downarrow$ \\
\midrule
Climatology & 0.50 & 0.15 & 0.56 & 0.19 \\
%ConvLSTM & 0.47 & 0.17 & 0.52 & 0.16 \\
%Earthformer & 0.47 &      0.15 &      0.47 &      0.16 \\
%PredRNN & 0.54 & 0.15 & 0.58 & 0.15 \\
%SimVP & 0.50 & 0.15 & 0.54 & 0.15 \\
Contextformer & 0.54 & 0.15 & 0.58 & 0.14 \\
\bottomrule
\end{tabularx}
\caption{Model skill at spatial (OOD-s) and spatio-temporal (OOD-st) extrapolation.}
\label{tab:extrapol}
\end{table}

\subsection{Contextformer Strengths and Limitations}

The OOD-t test set includes minicubes from four 3-month periods over two years. Fig.~\ref{fig:seasons} shows Contextformer's model skill. Yearly variations are significant. Growing season prediction was better in 2022 until September, then it switched, and 2021 performed better. First half of the season is usually better predicted than the second half, likely due to anthropogenic influences (harvest, mowing, cutting, and forest fires). These events are challenging to predict from weather covariates and may be interpreted as random noise.

We assess the performance at spatio-(temporal) extrapolation of the Contextformer on the OOD-s and OOD-st test sets and report in tab.~\ref{tab:extrapol}. The Contextformer can extrapolate in space and time. However, the margin to the climatology does shrink. Here, more training data might help: spatial extrapolation is theoretically not necessary for modeling vegetation dynamics (only temporal extrapolation is). Practically speaking, however, it does help to increase inference speed and enable potential applicability over large areas.

For these practical situations another aspect needs to be studied in future work: at inference time weather comes from uncertain weather forecasts. Here, we first wanted to learn the impact of weather on vegetation and thus took the historical meteo data which has the least error. We expect the weather forecast uncertainty (represented by realizations / scenarios) to mostly propagate, but not present a covariate shift larger than the inter-annual variability, which our models are robust to (OOD-t evaluation).


\section{Conclusion}
We proposed Contextformer, a multi-modal transformer model designed for fine-resolution vegetation greenness forecasting. It leverages spatial context through a Pyramid Vision Transformer backbone while maintaining parameter efficiency. The temporal component is a transformer that independently models the dynamics of local context patches over time, incorporating meteorological data. We additionally introduce the novel GreenEarthNet dataset tailored for self-supervised vegetation forecasting and compare Contextformer against an extensive set of baselines.

Contextformer outperforms the previous state-of-the-art, especially on nash-sutcliffe efficiency and surpasses even strong freshly trained video prediction baselines. To our knowledge, we are the first to consider a climatology baseline and outperforming it with models. Given the pronounced seasonality of vegetation dynamics, this suggests real-world applicability for our models, particularly the Contextformer, in crucial scenarios like humanitarian anticipatory action or carbon monitoring.

%We proposed the Contextformer, a multi-modal transformer model for forecasting vegetation greenness at fine resolution. In leverages spatial context through a Pyramid Vision Transformer backbone, but still remains a parameter-efficient model as its temporal component is a transformer that independently models the dynamics of local context patches through time, incorporating meteorological data. We introduce the novel GreenEarthNet dataset, which is tailored for self-supervised vegetation forecasting, and compare the Contextformer against an extensive set of baselines. The Contextformer outperforms the previous state-of-the-art by a large margin and also beats the much stronger freshly trained video prediction baselines. To the best of our knowledge, we present the first study considering a climatology baseline and outperforming it with models, which, given the strong seasonality of vegetation dynamics, indicates real-world usefulness of our models, especially the Contextformer, in impactful usecases such as humanitarian anticipatory action or carbon monitoring.  

%\iffalse
{%\small
\paragraph{Code.} We provide code and pre-trained weights to reproduce our experimental results under \url{https://github.com/vitusbenson/greenearthnet} \cite{benson_2024_10793870}.\\
\textbf{Author contributions.} VB experiments, figures, writing. CR experiments, writing. CRM supervision, figures, writing. ZG experiments. LA figures. NC, JC, NL, MW writing. MR funding, supervision, writing. All authors contributed to discussing the results. \\
\textbf{Acknowledgments.} We are thankful for invaluable help, comments and discussions to Reda ElGhawi, Christian Reimers, Annu Panwar and Xingjian Shi. MW thanks the European Space Agency for funding the project DeepExtremes (AI4Science ITT). CRM and LA are thankful to the European Union’s DeepCube Horizon 2020 (research and innovation programme grant agreement No 101004188). NL and JC acknowledge funding from the European Union’s Horizon 2020 research and innovation programme under grant agreement No 101003469. 
}
%\fi
\clearpage
%\clearpage
%\iffalse
{\small
\bibliographystyle{ieeenat_fullname}
\bibliography{earthnet}
}


\clearpage
%\fi
%\iffalse
\appendix
\section{Model details}
\subsection{Cloud masking}
\paragraph{Baselines (Table 1)}
The baselines reported in table 1 are taken from CloudSEN12 \cite{aybar.etal_2022}. Sen2Cor \cite{louis.etal_2016} is the processing software from ESA used to produce the Scene Classification Layer (SCL) mask, which was also introduced in EarthNet2021 \cite{requena-mesa.etal_2021}. FMask \cite{qiu.etal_2019} is a processing software originally designed for NASA Landsat imagery, but now repurposed to also work with Sentinel 2 imagery. It requires L1C top-of-atmosphere reflectance from all bands to be produced (EarthNet2021 only containes L2A bottom-of-atmosphere reflectance from four bands). KappaMask \cite{domnich.etal_2021} is a cloud mask based on deep learning, in table 1 we reported scores from the L2A version, which uses all 13 L2A bands as input.

\paragraph{UNet Mobilenetv2 (Table 1)}
Our UNet with Mobilenetv2 encoder \cite{sandler.etal_2018} was trained in two variants, one with RGB and near-infrared bands of L2A imagery (i.e. works with EarthNet2021) and one with all 13 bands of L2A imagery. We adopted the exact same implementation that was benchmarked in the CloudSEN12 paper \cite{aybar.etal_2022}, with the only difference being that in the paper, L1C imagery was used (which is often not useful in practical use-cases). In detail, this means we trained the UNet with Mobilenetv2 encoder using the Segmentation Models PyTorch Python library\footnote{\url{https://segmentation-models-pytorch.readthedocs.io/en/latest/}}. We used a batch size of 32, random horizontal and vertical flipping, random 90 degree rotations, random mirroring, unweighted cross entropy loss, early stopping with a patience of 10 epochs, AdamW optimizer, learning rate of $1e^{-3}$, and a learning rate schedule reducing the learning rate by a factor of 10 if validation loss did not decrease for 4 epochs.


\subsection{Vegetation modeling}

\paragraph{Contextformer (Table 2,3,4,5 Figure 3,4,5)}
Our Contextformer is a combination of a spatial vision encoder: Pyramid Vision Transformer (PVT) v2 B0 \cite{wang.etal_2021a, wang.etal_2022c} with pre-trained ImageNet1k weights from the PyTorch Image Models library\footnote{\url{https://github.com/rwightman/pytorch-image-models}} and a temporal transformer encoder. We use a patch size of $4\times 4$ px. We use an embedding size of $256$ and the temporal transformer has three self-attention layers with $8$ heads, each followed by an MLP with $1024$ hidden channels. We use LayerNorm \cite{ba2016layer} for normalization and GELU \cite{hendrycks2023gaussian} as non-linear activation function. The model is trained with masked token modeling, randomly ($p=0.5$) flipping between inference mask (future token masked) and random dropout mask ($70\%$ of image patches masked, except for the first 3 time steps). We train for 100 epochs with a batch size of 32, a learning rate of $4e^{-5}$ and with AdamW optimizer on $2$ NVIDIA A100 GPUs. We train three models from the random seeds 42, 97 and 27.

\paragraph{Local timeseries models (Table 2)}
We train the local timeseries models (table 2) at each pixel. For a given pixel we extract the full timeseries of NDVI and weather variables at 5-daily resolution. All variables are linearly gapfilled and weather is aggregated with min, mean, max, and std to 5-daily. The whole timeseries before each target period is used to train a timeseries model, for the target period the model only receives weather. The Kalman Filter runs with default parameters from darts \cite{herzen.etal_2022}. The LightGBM model gets lagged variables from the last 10 time steps and predicts a full 20 time step chunk at once. For Prophet we again use default parameters.

\paragraph{EarthNet models (Table 2)}
For running the leading models from EarthNet2021 we utilize the code from the respective github repositories: ConvLSTM \cite{diaconu.etal_2022}\footnote{\url{https://github.com/dcodrut/weather2land}}, SGED-ConvLSTM \cite{kladny2024enhanced}\footnote{\url{https://github.com/rudolfwilliam/satellite_image_forecasting}} and Earthformer \cite{gao.etal_2022} \footnote{\url{https://github.com/amazon-science/earth-forecasting-transformer/tree/main/scripts/cuboid_transformer/earthnet_w_meso}}. We derive the NDVI from the predicted satellite bands red and near-infrared:
\begin{align}
    NDVI = \frac{NIR - Red}{NIR + Red + 1e^{-8}}
\end{align}

\paragraph{ConvLSTM (Table 2,3, Figure 4,5)}
Our ConvLSTM contains four ConvLSTM-cells \cite{shi.etal_2015} in total, two for processing context frames and two for processing target frames. Each has convolution kernels with bias, hidden dimension of 64 and kernel size of 3. We train for 100 epochs with a batch size of 32, a learning rate of $4e^{-5}$ and with AdamW optimizer. We train three models from the random seeds 42, 97 and 27.

\paragraph{PredRNN (Table 2,3, Figure 4)}
Our PredRNN contains two ST-ConvLSTM-cells \cite{wang.etal_2017} Each has convolution kernels with bias, hidden dimension of 64 and kernel size of 3 and residual connections. We use a PatchMerge encoder decoder with GroupNorm (16 groups), convolutions with kernel size of 3 and hidden dimension of 64, LeakyReLU activation and downsampling rate of 4x. We train for 100 epochs with a batch size of 32, a learning rate of $3e^{-4}$ and with AdamW optimizer. We use a spatio-temporal memory decoupling loss term with weight 0.1 and reverse exponential scheduling of true vs. predicted images (as in the PredRNN journal version \cite{wang.etal_2023}). We train three models from the random seeds 42, 97 and 27.

\paragraph{SimVP (Table 2,3, Figure 4)}
Our SimVP has a PatchMerge encoder decoder with GroupNorm (16 groups), convolutions with kernel size of 3 and hidden dimension of 64, LeakyReLU activation and downsampling rate of 4x. The encoder processes all 10 context time steps at once (stacked along the channel dimension). The decoder processes 1 target time step at a time. The gated spatio-temporal attention processor \cite{tan.etal_2023} translates between both in the latent space, we use two layers and 64 hidden channels. We train for 100 epochs with a batch size of 64, a learning rate of $6e^{-4}$ and with AdamW optimizer. We train three models from the random seeds 42, 97 and 27.

\paragraph{Earthformer (Table 2)}
Our Earthformer is a transformer combined with an initial PatchMerge encoder (and a final decoder) to reduce the dimensionality. The encoder and decoder use LeakyReLU activation, hidden size of 64 and 256 and downsample 2x. In between, the transformer processor has a UNet-type architecture, with cross-attention to merge context frame information with target frame embeddings. GeLU activation and LayerNorm, axial self-attention, 0.1 dropout and 4 attention heads are used. Weather information is regridded to match the spatial resolution of satellite imagery and used as input during context and target period. We train for 100 epochs with a batch size of 32, a maximum learning rate of $1e^{-4}$, linear learning rate warm up, cosine learning rate shedule and with AdamW optimizer.

\paragraph{1x1 LSTM (Table 4)}
Our 1x1 LSTM is implemented as a ConvLSTM with kernel size of 1. We train for 100 epochs with a batch size of 32, a learning rate of $4e^{-5}$ and with AdamW optimizer.

\paragraph{Next-frame UNet (Table 4)}
Our next-frame UNet has a depth of 5, latent weather conditioning with FiLM, a hidden size 128, kernel size 3, LeakyReLU activation, GroupNorm (16 groups), PatchMerge downsampling and nearest upsampling. We train for 100 epochs with a batch size of 64, a learning rate of $6e^{-4}$ and with AdamW optimizer.

\paragraph{Next-cuboid UNet (Table 4)}
Our next-cuboid UNet has a depth of 5, latent weather conditioning with FiLM, a hidden size 256, kernel size 3, LeakyReLU activation, GroupNorm (16 groups), PatchMerge downsampling and nearest upsampling. We train for 100 epochs with a batch size of 64, a learning rate of $6e^{-4}$ and with AdamW optimizer.


\section{Weather ablations}
\subsection{Methods}
Most of our baseline approaches have been originally proposed to handle only past covariates. Here, we condition forecasts on future weather. A-priori it is not known how to best achieve this weather conditioning. For PredRNN and SimVP, we compare three approaches, each fused at three different locations. The approaches operate pixelwise, taking features $x_{in} \in \mathbb{R}^d$ and conditioning input $c_{i} \in \mathbb{R}^{n}$ for weather variable $i$. The conditioning layers $g(\cdot, \cdot; \phi)$ with parameters $\phi$ then operate as
\begin{align}
    x_{out} = g(x_{in}, c; \phi) \in \mathbb{R}^d
\end{align}
We parameterize $g$ with neural networks. %Fig.~\ref{fig:weathercond} provides an overview of the three approaches:

\paragraph{CAT} First concatenates $x_{in}$ and a flattened $c$ along the channel dimension, and then performs a linear projection to obtain $x_{out}$ of same dimensionality as $x_{in}$. In practice we implement this with a 1x1 Conv layer.

\paragraph{FiLM} Feature-wise linear modulation \cite{perez.etal_2018} generalizes the concatenation layer before. It produces $x_{out}$ with linear modulation:
\begin{align}
  x_{out} = x_{in} + \sigma(\gamma(c; \phi_{\gamma})\odot N(f(x_{in}; \phi_{f})) + \beta(c; \phi_{\beta}))
\end{align}
Here, $f$ is a linear layer, $\gamma$ and $\beta$ are MLPs, $N$ is a normalization layer and $\sigma$ is a pointwise non-linear activation function.

\paragraph{xAttn} Cross-attention is an operation commonly found in the Transformers architecture. In recent works on image generation with diffusion models it is used to condition the generative process on a text embedding \cite{rombach.etal_2022}. Inspired from this, we propose a pixelwise conditioning layer based on multi-head cross-attention. The input $x_{in}$ is treated as a single token query $Q$. Each weather variable $c_{i}$ is treated as individual tokens, from which we derive keys $K$ and values $V$. The result is then just regular multi-head attention $MHA$ in a residual block:
\begin{align}
    x_{out} &= x_{in} \\
    &+ f(N(MHA(Q(x_{in}; \phi_{Q}), K(c; \phi_{K}), V(c; \phi_{V}))); \phi_{f})
\end{align}
Here, $f$ is either a linear projection or a MLP and $N$ is a normalization layer.

Each of the three approaches we apply at three locations throughout the network:

\paragraph{Early fusion} Just fusing all data modalities before passing it to a model. This Early CAT has been previously used for weather conditioning in satellite imagery forecasting %\cite{diaconu.etal_2022, kladny2024enhanced}.

\paragraph{Latent fusion} In the encode-process-decode framework, encoders are meant to capture spatial, and not temporal, relationships. Hence, latent fusion conditions the encoded spatial inputs twice: right after leaving the encoder and before entering the decoder.

\paragraph{All (fusion everywhere)} In addition, we compare against conditioning at every stage of the encoders, processors and decoders. All CAT has been applied to condition stochastic video predictions on random latent codes \cite{lee.etal_2018}.

\subsection{Results}
Fig.~\ref{fig:horizon2} summarizes the findings by looking at the RMSE over the prediction horizon. For the first 50 days, most models are better than the climatology, afterwards, most are worse. If using early fusion, FiLM is the best conditioning method. For latent fusion and fusion everywhere (all), xAttn is a consistent choice, but FiLM may sometimes be better (and sometimes a lot worse). CAT in general should be avoided, which is consistent with the theoretical observation, that CAT is a special case of FiLM.

For SimVP, the best weather guiding method is latent fusion with FiLM. For PredRNN, the best method is early fusion with FiLM. This is likely due to the difference in treatment of the temporal axis. For SimVP, early fusion would merge all time steps, hence, latent fusion is a better choice. For PredRNN on the other hand, early fusion handles only a single timestep. 

\begin{figure}
    \centering
    \includegraphics[width=\columnwidth]{figures/predhorizon.pdf}
    \caption{Model performance (RMSE) when using different ways of weather conditioning over varying prediction horizons.}
    \label{fig:horizon2}
\end{figure}




%\iffalse
\section{Contextformer Strengths and Limitations continued}
We show spatial extrapolation skills for more models in table~\ref{tab:extrapol2}.
\begin{table}
\centering
\begin{tabularx}{\columnwidth}{Xcccc}
\toprule
& \multicolumn{2}{c}{OOD-s} & \multicolumn{2}{c}{OOD-st} \\
Model & $R^2$ $\uparrow$ & RMSE $\downarrow$ & $R^2$ $\uparrow$ & RMSE $\downarrow$ \\
\midrule
Climatology & 0.50 & 0.15 & 0.56 & 0.19 \\
ConvLSTM & 0.47 & 0.17 & 0.52 & 0.16 \\
Earthformer & 0.47 &      0.15 &      0.47 &      0.16 \\
PredRNN & 0.54 & 0.15 & 0.58 & 0.15 \\
SimVP & 0.50 & 0.15 & 0.54 & 0.15 \\
Contextformer & 0.54 & 0.15 & 0.58 & 0.14 \\
\bottomrule
\end{tabularx}
\caption{Same as table~\ref{tab:extrapol}, but extended. Model skill at spatial (OOD-s) and spatio-temporal (OOD-st) extrapolation.}
\label{tab:extrapol2}
\end{table}

Reassured by spatial extrapolation capabilities, we present a map of $R^2$ for the Contextformer in fig.~\ref{fig:mapR2}a. Cropland regions on the Iberian peninsula and in northern France, as well as forests in the Balkans are regions with great applicability of the model. For the former two, this may be explained by many training samples in those regions, for the last, it cannot. Grasslands and forests in Poland and highly heterogenous regions (mountains, near cities, near coasts) are more challenging for the model.

Geomorphons capture local terrain features, derived from first and second spatial derivatives of elevation. Fig.~\ref{fig:mapR2}b shows densities of RMSE of the ConvLSTM for different geomorphons from  the Geomorpho90m map \cite{amatulli.etal_2020}. Generally, the model performs well across all classes. Summits and Depressions, two rather extreme types, seem to be slightly easier to predict. Homogeneous terrain (red: flat, shoulder, footslope) has a larger tail towards high error. This may be as those regions are typically where there is a lot of anthropogenic activity, possibly leading to dynamics less covered by the predictors (harvest, clear-cut, etc.). 

\begin{figure}
    \centering
    \begin{subfigure}[c]{0.49\columnwidth}
    \includegraphics[width=\textwidth]{figures/fig6a.pdf}
    \end{subfigure}
    \begin{subfigure}[c]{0.49\columnwidth}
    \includegraphics[width=\textwidth]{figures/fig6b.pdf}
    \end{subfigure}
    
    \caption{Panel a) shows a map of $R^2$ on OOD-t and OOD-st test sets and panel b) shows probability densities of RMSE per geomorphon. Both for Contextformer.}
    \label{fig:mapR2}
\end{figure}

%\fi



\section{Performance per landcover type}
Fig.~\ref{fig:map2} shows the model performance per landcover type.

\begin{figure}
    \centering
    \includegraphics[width=\columnwidth]{figures/map_R2_predrnn.pdf}
    \caption{Model performance per landcover. Maps represent $R^2$ on OOD-t and OOD-st test sets of PredRNN.}
    \label{fig:map2}
\end{figure}

\section{Robustness of Outperformance Score}\label{sec:robustoutpf}
The choice of thresholds in the outperformance score (the percentage of samples where a model outperforms the climatology baseline by at least the threshold on at least 3 out of 4 metrics) is a heuristic. To assess its robustness, we re-evaluated five of our models over a wide range of possible threshold values. Fig.~\ref{fig:outpf} shows a consistency of the ranking, in particular our Contextformer outperforms all other models in all settings.

\begin{figure}[t]
    \centering
    \includegraphics[width=\columnwidth]{figures/fig_outperf.pdf}
    \caption{Outperformance score robustness (shown: marginals).}
    \label{fig:outpf}
\end{figure}

\section{Inference speed}\label{sec:speed}
Computing inference speed is highly platform and batch size dependent. To make it somewhat fair, we compare models by running 1024 samples on an A40 GPU (48GB), with the largest batch size (bs) fitting in memory, we perform 10 repetitions and report the mean and std. dev.: Contextformer $29.3$s{\footnotesize{$\pm 0.4$}} (bs 72), SimVP $6.7$s{\footnotesize{$\pm 0.8$}} (bs 96), PredRNN $16.2$s{\footnotesize{$\pm 0.2$}} (bs 512), ConvLSTM $37.1$s{\footnotesize{$\pm 1.8$}} (bs 256). For comparison predicting a single sample with one of the local time series models takes $>$1h on a single CPU.

\section{Downstream task: carbon monitoring}

Carbon monitoring is of great importance for climate change mitigation, especially in relation to nature-based solutions. The gross primary productivity (GPP) represents the amount of carbon that is taken up by plants through photosynthesis and subsequently stored. It is not directly observable. At a few hundred research stations around the world with eddy covariance measurement technology, it can be indirectly measured. For carbon monitoring, it would be beneficial to measure this quantity everywhere on the globe. It has been shown \cite{pabon-moreno.etal_2022} that Sentinel 2 NDVI is correlated to GPP measured with eddy covariance. We build on this correlation to show how our models could potentially be leveraged to give near real-time estimates of GPP and to study weather scenarios.

Fig.~\ref{fig:gpp} compares modeled with observed GPP at the Fluxnet site Grillenburg (identifier DE-Gri) in eastern Germany distributed by ICOS \cite{degri_icos}. First, we fit a linear model between observed NDVI and GPP for the years 2017-2019. Here, interpolated grassland NDVI pixels (fig.~\ref{fig:gpp}b, inside red boundaries) are used. Next, we perform an out-of-sample analysis and find an $R^2 = 0.53$ for 2020-01 to 2021-04 (fig.~\ref{fig:gpp}a, blue line). Finally, we forecast GPP with our PredRNN model from May to July 2021(fig.~\ref{fig:gpp}a, orange line). The resulting forecast has decent quality at short prediction horizons, but low skill after 75 days (fig.~\ref{fig:gpp}c). These results show a way to leverage models from this paper for near real-time carbon monitoring. However, for application at scale, it is likely beneficial to use a more powerful GPP model (e.g. random forest \cite{pabon-moreno.etal_2022} or light-use efficiency \cite{bao.etal_2022}), fitted across many Fluxnet sites.

\begin{figure}
    \centering
    \includegraphics[width=0.95\columnwidth]{figures/figgpp}
    \caption{Panel a) shows timeseries of observed (green) and modeled GPP (blue from NDVI observations, orange from NDVI prediction). Panel b) shows a satellite image of the Grillenburg Fluxnet site with grassland boundaries in red. Panel c) shows the RMSE over prediction horizons.}
    \label{fig:gpp}
\end{figure}





%{\small
%\bibliographystyle{ieeenat_fullname}
%\bibliography{earthnet}
%}



% \begin{abstract}
The current study investigated possible human-robot kinaesthetic interaction using a variational recurrent neural network model, called PV-RNN, which is based on the free energy principle.
Our prior robotic studies using PV-RNN showed that the nature of interactions between top-down expectation and bottom-up inference is strongly affected by a parameter, called the meta-prior, which regulates the complexity term in free energy.
% The current study examines how the behaviours of robots alter by changing the meta-prior $w$ in human-robot kinaesthetic interaction.
The current study examines how changing the meta-prior $w$ in the interaction phase affects the counter force generated when an experimenter attempts to induce movement pattern transitions familiar to the robot through its prior training.
The study also compares the counter force generated when trained transitions are induced by a human experimenter and when untrained transitions are induced.
Our experimental results indicated that (1) the human experimenter needs more/less force to induce trained transitions when $w$ is set with larger/smaller values, (2) the human experimenter needs more force to act on the robot when he attempts to induce untrained as opposed to trained movement pattern transitions.
Our analysis of time development of essential variables and values in PV-RNN during bodily interaction clarified the mechanism by which gaps in actional intentions between the human experimenter and the robot can be manifested as reaction forces between them.


%% Hiroki writing 2022-11-4
%Current study investigates the dynamics of the latent states during human-robot kinaesthetic interaction using PV-RNN.
%We have achieved to observe and analyse the internal state of an RNN model based on the free energy principle, during real-time human-robot interaction.
%Essential characteristics observed in the previous study of this variational recurrent neural network model, PV-RNN, is that by changing a meta prior $w$, the balance between the top-down intention and the bottom-up perceptual reality changes.
%In the current study, we examined how changing the weighting parameter $w$ between accuracy and complexity in free energy principle affects the humanoid robot's behaviour through human-robot interaction. We have conducted some human-robot kinaesthetic interaction experiments with various $w$ and quantitatively analysed the latent variable and the force applied to the humanoid robot. We have observed that the force required to change the robot's intention has increased, both when the top-down intention was strengthened by changing the $w$ and when corresponding switch of its primitive was against the experience of the RNN during its training. The study confirms through quantitative analysis that by increasing or decreasing the $w$ in PV-RNN, humanoid robot leads or follows the human counterpart during the human-robot kinaesthetic interaction.

\begin{comment}
Comment from Jun #2
・最後にQualitativeな結果(インパクト)が欲しい
・Current study investigates the problem on~と書き出すのが一般的
・最初の一文と最後の一文を対応させる
・最後の一文はもう少しAbstractかつ包括的に
\end{comment}

\begin{comment}
Comment from Jun #1
We investigated how the kinaesthetic human-robot interaction can affect the internal state of a model based on the free energy principle. 
=> how the internal state is affected is not the most important point in this study. This part should be rewritten.

The key function of this variational recurrent neural network model, PV-RNN, is that by changing a meta prior $w$, it takes a balance between the "complexity” term and the ”accuracy” term which corresponds to a top-down intention and a bottom-up perceptual reality in the free energy principle, respectively. 
=> This is not key function of PV-RNN. It is an essential characteristics observed in the previous study. The grammar after $w$ is something strange. Rewrite these.

This research has conducted a human-robot interaction experiment with a robotic agent in a kinaesthetic sense.
=> The sentence is not good. "in a kinaesthetic sense" is grammatically wrong.
MODIFIED => "In the current study human-robot interaction experiments using the kinaesthetic sense were conducted."

We investigated that when human forces the agent to switch primitives from one to another, larger force was required both when the human intention is conflictive against the top-down the intention of the agent and when the agent has a stronger top-down intention by modifying the $w$.
=> You should write the essential results of the experiments rather than what we investigated and also how these results could contribute to the studies on human-robot interaction.
\end{comment}

\end{abstract}    
% \section{Introduction}
\label{sec:intro}
\begin{figure}[t]
\begin{center}
    \includegraphics[width=1\linewidth]{figures/teaser.pdf}
\end{center}
\vspace{-0.1in}
\caption{\textbf{{\em Foggy} vs {\em Clear} NeRF.} Our \ournerf gets rid of reconstruction errors manifested as foggy ``floaters" in the density volume without additional input or significant computational overhead. 
%
Below are density profiles along a given ray before and after our geometry correction procedure, where we discard density peaks corresponding to floaters.
}
\label{fig:teaser}
\vspace{-0.2in}
\end{figure}



%The emergence of 
Neural Radiance Fields (NeRFs)~\cite{mildenhall2020nerf}  %and its variants 
have made revolutionary contributions in %photo-realistic 
novel view synthesis~\cite{barron2021mip,barron2022mip}, 
autonomous driving~\cite{rematas2022urban,tancik2022block}, digital human~\cite{hong2022headnerf,zhao2022humannerf}, and 3D content generation~\cite{eg3d,poole2022dreamfusion,lin2022magic3d}.
%by leveraging a multi-layer perceptron (MLP) to implicitly model the mapping from input 5D coordinates (i.e., 3D coordinates $\mathbf{x} = (x,y,z)$ and 2D viewing directions $\mathbf{d}=(\theta,\phi)$) to volume density $\sigma$ and view-dependent emitted radiance color $\mathbf{c} = (r,g,b)$. 
%
%They then use traditional volume rendering mechanisms on the obtained continuous 5D function (i.e., MLP) to generate novel views. 
To date, unfortunately, most NeRF-based methods encounter challenges when tackling large-scale cluttered scenes (e.g., Fig.~\ref{fig:teaser}):
\begin{enumerate}[leftmargin=0.16in, topsep=2pt,itemsep=-1ex,partopsep=1ex,parsep=1ex]
\item Input observations used for NeRF are often too sparse  compared to forward-facing or synthetic looking-inward scenes;
%\item Recovering fine-grained objects within a large volume is challenging for NeRF; %in capturing details accurately.
\item View-dependent visual effects give rise to ambiguity, resulting in a ``foggy" density field as shown in Fig.~\ref{fig:teaser}. 
%
Such artifacts are particularly pronounced in indoor scenes strewn with view-dependent appearances, such as specular highlights, glossy surface reflections from man-made objects. 
\end{enumerate}

Despite attempts to enhance NeRF's rendering quality given suboptimal input, such as using 3D conical frustums~\cite{barron2021mip,barron2022mip}, physically-grounded augmentations~\cite{chen2022aug}, and misalignment correction~\cite{jiang2022alignerf},  these challenges have yet to be fully resolved.
%
Depth supervision~\cite{deng2022depth, wei2021nerfingmvs} or proxy geometry~\cite{xu2021scalable,wu2022scalable} images can help alleviate the challenges in handling large-scale with sparse input, at the expense of %but they come at the cost of requiring 
expensive pre-processing or additional input.
%
Another line of work~\cite{wang2021neus, oechsle2021unisurf, wang2022neuris} achieves better reconstruction of surface geometry by using signed distances instead of volume density as scene representation. However, they sacrifice the ability to synthesize photo-realistic novel views.

%We observe that NeRF has been suffering from foggy ``floater" artifacts in large-scale cluttered scenes.
%
%Such artifacts are particularly pronounced in indoor scenes strewn with view-dependent appearances from man-made objects. 
%
To address the above issues, we propose an extension to NeRF, dubbed as {\bf \ournerf}, which enforces effective {\em appearance} and {\em geometry} constraints conducive to accurate colors and 3D densities estimation. We believe \ournerf can contribute beyond novel view synthesis, such as NeRF object detection~\cite{hu2022nerf}, NeRF object segmentation~\cite{zhi2021place, liu2022unsupervised, fan2022nerf,ren2022neural}, and NeRF registration~\cite{goli2022nerf2nerf}, where the rooms for improvement are substantial if more accurate color and density estimation are available.

Correspondingly, there are two steps in \ournerf. First, for appearance correction, the view-independent and view-dependent color components are predicted from the underlying 3D scene, which is combined to produce the final color estimation (Fig.~\ref{fig:toaster}).
%
The view-independent component (diffuse color and shading) captures the overall scene color, while the view-dependent component (highlights or reflections) captures color variations due to changes in viewing angle.
%
\ournerf then discards these view-dependent appearances in the training views to prevent them from interfering with the density estimation.
%
Second, a simple and effective geometry correction procedure will be performed to further eliminate the foggy ``floaters" or density errors. This geometry correction procedure is based on an assumption in line with traditional ray tracing in computer graphics.
\begin{comment}
% xh: basically copying method
On the other hand, ClearNeRF performs a geometric correction procedure performed on each traced ray during inference to refine the density estimation and better tackle the floater artifacts. 
%
The geometry correction procedure assumes that there should only be one salient peak along each traced ray during NeRF inference. 
Only the salient peak closest to the ray origin (the camera center) corresponds to  true geometry while the others will be manifested as foggy floaters hovering in the density volume. 
%
This assumption is in line with traditional ray tracing in computer graphics where in the absence of noise, only one intersection per ray should be returned to indicate the closest ray-object intersection.
%
\end{comment}
%%%%%%%%%%%
%As shown in Fig.~\ref{fig:teaser}, when reconstructing an indoor scene with sparse input and highly view-dependent objects, NeRF produces severe floating artifacts due to its attempt to explain view-dependent appearances.
%
Experiments verify that our proposed \ournerf can effectively get rid of floater artifacts without additional input.% or significant computational overhead. 


In summary, our contributions include the following:
\begin{itemize}[leftmargin=0.16in, topsep=2pt,itemsep=-1ex,partopsep=1ex,parsep=1ex]
    \item We propose a concise method for decomposing view-independent and view-dependent appearance during NeRF training and eliminate the interference of view-dependent appearance.
    \item We propose a geometric correction procedure performed on each traced ray during inference to refine the density estimation and better tackle the floater artifacts.
    \item Extensive experiments and ablations verify the effectiveness of our core designs and results in improvements over the vanilla NeRF and other state-of-the-art alternatives.
    %without additional computational resources or other inputs.
\end{itemize}




% \input{sec/2_formatting}
% \input{sec/3_finalcopy}
% {
%    \small
%    \bibliographystyle{ieeenat_fullname}
%    \bibliography{main}
%}

% WARNING: do not forget to delete the supplementary pages from your submission 
% \clearpage
%\setcounter{page}{1}
\maketitlesupplementary
The supplementary material for our work  \textit{SC-VAE: Sparse Coding-based Variational Autoencoder with Learned ISTA} is structured as follows:
%Sec. \ref{section1_s} provides the detailed information of the encoder and decode architecture of the SC-VAE model. 
%Sec. \ref{section2_s} shows the visualization of the dictionary atoms.
%Sec. \ref{section3_s} shows the training loss on the ImageNet dataset with different number of downsampling (upsampling) blocks ($d$) in the encoder (decoder) of the SC-VAE model.
%Sec. \ref{section4_s} shows the visualization results of an unofficial implementation of VIT-VQGAN \cite{yu2021vector}. 
%Sec. \ref{section5_s} shows additional manipulation and interpolation results on FFHQ dataset. 
%Sec. \ref{section6_s} shows additional image patches clustering results on FFHQ and ImageNet datasets. 
%Sec. \ref{section7_s} shows additional unsupervised image segmentation results.
Section \ref{section1_s} details the encoder and decoder architecture of the SC-VAE model. In Section \ref{section2_s}, the dictionary atoms are visualized. In Section \ref{section3_s}, we provide the training losses on the ImageNet dataset when varying the number of downsampling (upsampling) blocks ($d$) in the encoder (decoder) of the SC-VAE model. In Section \ref{section4_s}, the visualized reconstruction results of an unofficial implementation of VIT-VQGAN \cite{yu2021vector} are provided. 
We provide  additional manipulation and interpolation results on the FFHQ dataset in Section \ref{section5_s}, while  additional clustering results of image patches on both FFHQ and ImageNet  are provided in Section \ref{section6_s}. Supplementary unsupervised image segmentation results are given in Section \ref{section7_s}.

%Additional results on image patches clustering and unsupervised image segmentation on FFHQ and ImageNet datasets are then presented in Sec. 2 and Sec. 3, respectively.
\setcounter{section}{0}

\section{The Encoder and Decoder Architecture of SC-VAE} \label{section1_s}
The SC-VAE model's encoder and decoder architecture mirrors that of VQGAN \cite{esser2021taming}. Details about the architecture are provided in Table \ref{figure:encoder_decoder}.
%The encoder and decoder architecture in the SC-VAE model are the same as the architecture used in VQGAN \cite{esser2021taming}, which is described in Table \ref{figure:encoder_decoder}. 
$H$, $W$ and $C$ denote the height, width
and the number of channels of an input image, respectively.
$C'$ and $C''$ represent the number of channels of the feature maps that are produced as outputs by the intermediate layers of the encoder and decode network.
In our experiment, $C'$ and $C''$ were set to $128$ and $512$, respectively. $n$ denotes the number of dimensions of each latent representation, which was set to $256$.
The variable $d$ represents the number of blocks used for downsampling and upsampling. Therefore, we can calculate the height ($h$) and width ($w$) of the encoder's output feature maps by dividing the height ($H$) and width ($W$) of input images by $2$ raised to the power of $d$.

\begin{table}[thbp!]
\centering
\caption{High-level architecture of the encoder and decoder of the SC-VAE model. $H$, $W$, and $C$ refer to the height, width, and the number of channels of an input image. 
$C'$ and $C''$ represent the number of channels of the feature maps from intermediate layers in the encoder and decoder networks. $n$ denotes the number of dimensions of each latent representation, while $d$ represents the number of downsampling (upsampling) blocks. Note that $h=\frac{H}{2^{d}}$, $w=\frac{W}{2^d}$.} 
\resizebox{1\linewidth}{!}{%
\begin{tabular}{c|c}
  \toprule
   &  $x\in \mathbb{R}^{H\times W\times C} $\\
   &  2D Convolution $\rightarrow \mathbb{R}^{H\times W\times C'}$\\
   &  $d \times$\{Residual Block, Downsample Block\} $\rightarrow \mathbb{R}^{h\times w\times C''}$\\
   &  Residual Block $\rightarrow \mathbb{R}^{h\times w\times C''}$\\
  Encoder &  Non-Local Block $\rightarrow \mathbb{R}^{h\times w\times C''}$\\
   &  Residual Block $\rightarrow \mathbb{R}^{h\times w\times C''}$\\
   &  Group Normalization \cite{wu2018group} $\rightarrow \mathbb{R}^{h\times w\times C''}$ \\
   &  Swish Activation Function \cite{ramachandran2017searching} $\rightarrow \mathbb{R}^{h\times w\times C''}$\\
   &  2D Convolution $\rightarrow E(x) \in \mathbb{R}^{h\times w\times n}$\\
  \midrule
   & $\tilde{E}(x)\in \mathbb{R}^{h\times w\times n} $  \\
   &  2D Convolution $\rightarrow \mathbb{R}^{h\times w\times C''}$  \\
   &  Residual Block $\rightarrow \mathbb{R}^{h\times w\times C''}$\\
    & Non-Local Block $\rightarrow \mathbb{R}^{h\times w\times C''}$\\
  Decoder & Residual Block $\rightarrow \mathbb{R}^{h\times w\times C''}$\\
    & $d\times$\{Residual Block, Upsample Block\} $\rightarrow \mathbb{R}^{H\times W\times C'}$\\
   & Group Normalization \cite{wu2018group} $\rightarrow \mathbb{R}^{H\times W\times C'}$\\
    & Swish  Activation Function \cite{ramachandran2017searching}
    $\rightarrow \mathbb{R}^{H\times W\times C'}$\\
    & 2D Convolution $\rightarrow G(\tilde{E}(x)) \in \mathbb{R}^{H\times W\times C}$\\
  \bottomrule
\end{tabular}}
\label{figure:encoder_decoder}
\end{table}

\section{Visualization of Dictionary Atoms}
\label{section2_s}
Figure \ref{figure:dictionary_visualization} demonstrates the $512$ columns (atoms) of the pre-determined Discrete Cosine Transform (DCT) dictionary. Each atom is of dimension $256$, which corresponds to the size of $16 \times 16$ images when shaped.
%We reshape all atoms into an image with a $16\times 16$ resolution.

\begin{figure}[tbp]
\centering
\includegraphics[width=8cm]{./Figures/visualization_of_dictionary.png}
\caption{$512$ atoms of the Discrete Cosine Transform (DCT) dictionary. All atoms were reshaped into a $16 \times 16$ image.}
\label{figure:dictionary_visualization}
\end{figure}

\section{Training Losses}  \label{section3_s}
%Training losses of inherent noises around the 140th epoch under different auxiliary dataset sizes (K)
Figures \ref{figure:TLImagenet32x32}, \ref{figure:TLImagenet16x16}, \ref{figure:TLImagenet4x4} and \ref{figure:TLImagenet1x1} show  the training losses over $120,000$ training steps on the
ImageNet dataset.
The number of downsampling (upsampling) blocks ($d$) in the encoder (decoder) of the SC-VAE model are $3, 4, 6$ and $8$, respectively.
%with the number of downsampling (upsampling) blocks ($d=3,4,6$ and $8$, respectively) in the encoder (decoder) of the SC-VAE model. 
%As is shown in these figures, the LISTA networks of the SC-VAE models converge to a fixed point no matter which downsampling (upsampling) block $d$ is used. However, SC-VAE suffer from image reconstruction when increasing $d$.
As depicted in these figures, the LISTA networks within the SC-VAE models consistently converge to a stable point regardless of the chosen downsampling (upsampling) block $d$. However, increasing $d$ leads to worse image reconstructions ($\mathcal{L}_{rec}$) in SC-VAE.

\begin{figure}[tbp]
\centering
\includegraphics[width=7.5cm]{./Figures/Imagenet32x32.png}
\caption{The training losses over $120,000$ training steps on the ImageNet dataset. The number of  downsampling (upsampling) blocks ($d$) in the encoder (decoder) of the SC-VAE model was set to $3$ and the height ($h$) and width ($w$) of latent representations were $32$. (a) Total loss $\mathcal{L}_{SC-VAE}$. (b) Image reconstruction loss $\mathcal{L}_{rec}$. (c)The mean of latent representations reconstruction loss $\frac{1}{hw}\mathcal{L}_{latent}$.}
\label{figure:TLImagenet32x32}
\end{figure}

\begin{figure}[tbp]
\centering
\includegraphics[width=7.5cm]{./Figures/Imagenet16x16.png}
\caption{The training losses over $120,000$ training steps on the ImageNet dataset. The number of  downsampling (upsampling) blocks ($d$) in the encoder (decoder) of the SC-VAE model was set to $4$ and the height ($h$) and width ($w$) of latent representations were $16$. (a) Total loss $\mathcal{L}_{SC-VAE}$. (b) Image reconstruction loss $\mathcal{L}_{rec}$. (c) The mean of latent representations reconstruction loss $\frac{1}{hw}\mathcal{L}_{latent}$.}
\label{figure:TLImagenet16x16}
\end{figure}

\begin{figure}[tbp]
\centering
\includegraphics[width=7.5cm]{./Figures/Imagenet4x4.png}
\caption{The training losses over $120,000$ training steps on the ImageNet dataset. The number of  downsampling (upsampling) blocks ($d$) in the encoder (decoder) of the SC-VAE model was set to $6$ and the height ($h$) and width ($w$) of latent representations were $4$. (a) Total loss $\mathcal{L}_{SC-VAE}$. (b) Image reconstruction loss $\mathcal{L}_{rec}$. (c) The mean of latent representations reconstruction loss $\frac{1}{hw}\mathcal{L}_{latent}$.}
\label{figure:TLImagenet4x4}
\end{figure}

\begin{figure}[tbp]
\centering
\includegraphics[width=7.5cm]{./Figures/Imagenet1x1.png}
\caption{The training losses over $120,000$ training steps on the ImageNet dataset. The number of  downsampling (upsampling) blocks ($d$) in the encoder (decoder) of the SC-VAE model was set to $8$ and the height ($h$) and width ($w$) of latent representations were $1$. (a) Total loss $\mathcal{L}_{SC-VAE}$. (b) Image reconstruction loss $\mathcal{L}_{rec}$. (c) The mean of latent representations reconstruction loss $\frac{1}{hw}\mathcal{L}_{latent}$.}
\label{figure:TLImagenet1x1}
\end{figure}

%\noindent
%\noindent\textbf{Learnbale ISTA.} The architecture of our Learnable ISTA network is shown in Table 2.
%\noindent\textbf{Attention Network for $\alpha$ Estimation.}  Our neural network architecture follows the backbone of PixelCNN++ [52], which is a U-Net [48] based on a Wide ResNet [72]. We replaced weight normalization [49] with group normalization [66] to make the implementation simpler. Our 32 × 32 models use four feature map resolutions (32 × 32 to 4 × 4), and our 256 × 256 models use six. All models have two convolutional residual blocks per resolution level and self-attention blocks at the 16 × 16 resolution between the convolutional blocks [6].






% \begin{table*}[!htbp]
% \centering
% \caption{High-level architecture of the Learnable ISTA of our SC-VAE. Note that $k$ is the number of the unfolded ISTA block.} 
% \begin{tabular}{c}
%   \toprule
%   Learnable ISTA \\
%   \midrule
%   $E(x)\in \mathbb{R}^{h\times w \times n} $ \\
%   Filter Matrix $\rightarrow \mathbb{R}^{h\times w\times K}$ \\
%   $k\times$\{Shrinkage Function, Mutual Inhibition Matrix, Addition Operator\} $\rightarrow \mathbb{R}^{h\times w\times K}$\\
%   Shrinkage function$\rightarrow Z\in \mathbb{R}^{h\times w\times K}$\\
%   \bottomrule
% \end{tabular}
% \end{table*}

\section{Image Reconstruction}  \label{section4_s}
Reconstruction results from unofficial implementation\footnote{https://github.com/thuanz123/enhancing-transformers} of VIT-VQGAN \cite{yu2021vector} are presented in Figure \ref{figure:ViT-VQGAN_Visualization}.
%Figures \ref{figure:ViT-VQGAN_Visualization} shows visualizations from unofficial implementation\footnote{https://github.com/thuanz123/enhancing-transformers} of VIT-VQGAN \cite{yu2021vector}. 
VIT-VQGAN \cite{yu2021vector} achieved visually appealing results. However, similar to VQ-GAN \cite{esser2021taming} and RQ-VAE \cite{lee2022autoregressive}, it faced challenges in accurately reconstructing intricate details and complex patterns.
%as VQ-GAN\cite{esser2021taming} and RQ-VAE\cite{lee2022autoregressive}. 
Additionally, its generalization performance was inferior to that of our model.

\begin{figure}[tbp]
\centering
\includegraphics[width=7.0cm]{./Figures/ViT-VQGAN_Visualization.png}
\caption{Image reconstructions from an unofficial implementation of VIT-VQGAN \cite{yu2021vector} and the SC-VAE models trained
on ImageNet dataset. Original images in the top two rows are
from the validation set of ImageNet dataset. Two external images are shown in the last two rows to demonstrate the generalizability of different methods. The numbers denote the shape of
latent codes and the learned codebook (dictionary) size, respectively.
SC-VAE achieved improved image reconstruction compared to VIT-VQGAN \cite{yu2021vector}. Zoom in to see the details in the red square area.}
\label{figure:ViT-VQGAN_Visualization}
\end{figure}

\section{Image Generation}  \label{section5_s}
Additional interpolation and manipulation results can be found in Figures \ref{figure:image_interpolation_supple} and \ref{figure:image_manipulation_supple}, respectively.

\begin{figure}[tbp]
\centering
\includegraphics[width=7.0cm]{./Figures/image_interpolation_supple.png}
\caption{Interpolation between the sparse code vectors of two samples from the SC-VAE$^{\dag}$ model trained on FFHQ.}
\label{figure:image_interpolation_supple}
\end{figure}

\begin{figure*}[tbp]
\centering
\includegraphics[width=14.5cm]{./Figures/image_manipulation_supple4.png}
\caption{Manipulating sparse code vectors on FFHQ. 
Each block contains five seed images used to infer the latent sparse code vector in the SC-VAE$^{\dag}$ model.
The disentangled attributes associated with the $i$-th component of a sparse code vector $z$ and a traversal range are shown on the top of each block.}
\label{figure:image_manipulation_supple}
\end{figure*}



% \begin{figure}[tbp]
% \centering
% \includegraphics[width=8cm]{./Figures/IG_Age.png}
% \caption{IG-Age.}
% \label{figure:IG_Age}
% \end{figure}

% \begin{figure}[tbp]
% \centering
% \includegraphics[width=8cm]{./Figures/IG_sunglasses.png}
% \caption{IG-sunglasses.}
% \label{figure:IG_sunglasses}
% \end{figure}

% \begin{figure}[tbp]
% \centering
% \includegraphics[width=8cm]{./Figures/IG_Azimuth.png}
% \caption{IG-azimuth.}
% \label{figure:IG_azimuth}
% \end{figure}

% \begin{figure}[tbp]
% \centering
% \includegraphics[width=8cm]{./Figures/IG_Fringe.png}
% \caption{IG-Fringe.}
% \label{figure:IG_Fringe}
% \end{figure}


% \begin{figure}[tbp]
% \centering
% \includegraphics[width=8cm]{./Figures/IG_skin color.png}
% \caption{IG-skin color.}
% \label{figure:IG_skin color}
% \end{figure}


% \begin{figure}[tbp]
% \centering
% \includegraphics[width=8cm]{./Figures/image_interpolation_supple.png}
% \caption{Interpolation in the latent space between two samples from a model trained on FFHQ.}
% \label{figure:interpolation}
% \end{figure}

\section{Image Patches Clustering}  \label{section6_s}
%Figures \ref{figure:s1} and \ref{figure:s2} exhibit more image patches clustering outcomes for the FFHQ and ImageNet datasets, respectively. 
Figures \ref{figure:s1} and \ref{figure:s2} showcase additional qualitative results of image patches clustering on FFHQ and ImageNet datasets, respectively.
These results were obtained utilizing the pre-trained SC-VAE$^\curlyvee$ model specific to each dataset with a downsampling block $d=4$.
\begin{figure*}[h!]
\centering
\includegraphics[width=16cm]{./Figures/patches_cluster_ffhq_supple_50.png}
\caption{50 randomly selected image patch clusters from the validation set of the FFHQ dataset generated by clustering the learned sparse code vectors of the pre-trained SC-VAE$^\curlyvee$ model
using the K-means algorithm. Each row represents one cluster. Image patches with similar patterns were grouped together.}
\label{figure:s1}
\end{figure*}

\begin{figure*}[h!]
\centering
\includegraphics[width=16cm]{./Figures/imagenet_cluster_patches_V3.png}
\caption{50 randomly selected image patch clusters from the validation set of the ImageNet dataset generated by clustering the learned sparse code vectors of the pre-trained SC-VAE$^\curlyvee$ model
using the K-means algorithm. Each row represents one cluster. Image patches with similar patterns were grouped together.}
\label{figure:s2}
\end{figure*}

% \begin{figure*}[h!]
% \centering
% \includegraphics[width=16cm]{./Figures/segmentation_ffhq_supple3.png}
% \caption{FFHQ.}
% \label{figure:5}
% \end{figure*}

\section{Unsupervised Image Segmentation} \label{section7_s}
\subsection{Qualitative Analysis on FFHQ and ImageNet}
%Figures \ref{figure:s3} and \ref{figure:s4} contain additional qualitative unsupervised image segmentation results on FFHQ and ImageNet datasets, respectively. 
%We utilized two SCVAE models that were pre-trained on the training set of the FFHQ and ImageNet dataset, respectively. These models had a downsampling block of $d = 3$ and a sparsity penalty of $\lambda = 2$. 
%We employed two SC-VAE$^\curlywedge$ models that had been pre-trained on the training sets of the FFHQ and ImageNet datasets, respectively. These models had a downsampling block $d=3$.
Additional qualitative unsupervised image segmentation results on the FFHQ and ImageNet datasets can be found in Figures \ref{figure:s3} and \ref{figure:s4}, respectively. We utilized two SC-VAE$^\curlywedge$ models pre-trained on the training sets of FFHQ and ImageNet, each employing a downsampling block $d=3$.
\subsection{Quantitative comparisons to prior work}
%Figure \ref{figure:Flower_CUB} shows more qualitative results on  Flowers \cite{nilsback2008automated} and Caltech-UCSD Birds-200-2011 (CUB) \cite{WahCUB_200_2011}. Flowers \cite{nilsback2008automated} consists of $8,189$ images of $102$ classes of flowers, with segmentation masks obtained by an automated algorithm developed specifically for segmenting flowers in color photographs \cite{nilsback2007delving}. CUB \cite{WahCUB_200_2011} consists of $11,788$ images of $200$ classes of birds and segmentation masks. Flowers and CUB contain $1,020$ and $1,000$ test images, respectively.
%Figure \ref{figure:Flower_CUB} shows more qualitative results on  Flowers \cite{nilsback2008automated} and Caltech-UCSD Birds-200-2011 (CUB) \cite{WahCUB_200_2011} datasets.
Figure \ref{figure:Flower_CUB} displays additional qualitative results from the Flowers \cite{nilsback2008automated} and Caltech-UCSD Birds-200-2011 (CUB) \cite{WahCUB_200_2011} datasets.\\
\subsubsection{Evaluation Metrics}
\textbf{Intersection of Union (IoU).} %The IoU score measures the overlap of two regions A and B by calculating the ratio of intersection over union, according to
The IoU score quantifies the overlap between two regions. This is achieved by evaluating the ratio of their intersection to their union.
\begin{align}
    \textup{IoU}(A, B) = \frac{|A\cap B|}{|A\cup B|}. \nonumber
\end{align}
%where we use the inferred mask and ground-truth mask as $A$ and $B$ respectively for evaluation.\\
$A$ denotes the ground-truth mask, while $B$ denotes the inferred mask.\\
%as $B$ for assessment purposes.\\
\textbf{DICE score.} Similarly, the DICE score is defined as:
\begin{align}
    \textup{Dice}(A, B) = \frac{2|A\cap B|}{|A|+ |B|}.\nonumber
\end{align}
\noindent
Higher is better for both scores.\\
\subsubsection{Dataset Details}
\textbf{Flowers.} The Flowers \cite{nilsback2008automated} dataset consists of $8,189$ images across $102$ different flower classes. Additionally, it includes segmentation masks generated by an automated algorithm designed explicitly for color photograph flower segmentation \cite{nilsback2007delving}. 
%The images in this dataset have large scale, pose and light variations.\\
The dataset contains images that exhibit substantial variations in scale, pose, and lighting.
Flowers \cite{nilsback2008automated} contains $1,020$ test images.\\
\textbf{CUB.} The CUB \cite{WahCUB_200_2011} dataset contains $11,788$ images covering $200$ bird classes, along with their segmentation masks. 
%Each image is further annotated with $15$ part locations and $1$ bounding box. We use theprovided bounding box to extract a center square from the image, and scale it to $128\times 128$ pixels.
Every image comes with annotations for $15$ part locations, $312$ binary attributes, and $1$ bounding box. We utilized the given bounding box to crop a central square from the image. The CUB dataset includes $1,000$ test images.\\
\textbf{ISIC-2016.} The ISIC-2016 \cite{gutman2016skin} dataset is a public challenge dataset dedicated to Skin Lesion Analysis for Melanoma Detection. Derived from the extensive International Skin Imaging Collaboration (ISIC) archive, it represents a significant collection of meticulously curated dermoscopic images of skin lesions. Within this challenge, a subset of $900$ images is designated as training data, while $379$ images serve as testing data, aiming to provide representative samples for analysis.
%The ISIC-2016 \cite{gutman2016skin} dataset is a public challenge dataset of Skin Lesion Analysis Towards Melanoma Detection released with ISBI 2016. This dataset is based on the International Skin Imaging Collaboration (ISIC) Archive, which is the largest publicly available collection of quality controlled dermoscopic images of skin lesions. The challenge employs a subset of representative images with $900$ images as training data and $379$ images as testing data.

%For all experiments, we resized the input images into a resolution of $256\times 256$ and  generated a $32\times 32$ binary mask for each image utilizing the pre-trained SC-VAE$^\curlywedge$ on ImageNet dataset, a spectral clustering algorithm and boundary connectivity information. The inferred binary mask and ground truth mask were resized to $128\times 128$ to calculate the IoU and DICE scores.
For our experiments, we resized the input images into a resolution of $256\times 256$.
Subsequently, we generated a binary mask of size $32\times 32$ per image by employing the pre-trained SC-VAE$^\curlywedge$ on the ImageNet dataset, along with a spectral clustering algorithm and boundary connectivity information \cite{zhu2014saliency}. To compute the IoU and DICE scores, both the inferred binary mask and the ground truth mask were resized to $128\times 128$.
%\subsubsection{Baseline Methods}
\label{section3}
\begin{figure*}[h!]
\centering
\includegraphics[width=16cm]{./Figures/segmen_ffhq_supple3.png}
%\caption{Additional unsupervised image segmentation results. Images are from the validation set of the FFHQ dataset.}
\caption{Additional unsupervised image segmentation results. These results were generated by grouping sparse code vectors into $5$ categories per image, utilizing the pre-trained SC-VAE$^{\curlywedge}$ model and the K-means algorithm. Images are from the validation set of the FFHQ dataset.}
\label{figure:s3}
\end{figure*}

\begin{figure*}[h!]
\centering
\includegraphics[width=16cm]{./Figures/segmentation_imagenet_supple.png}
%\caption{Additional unsupervised image segmentation results by applying K-means algorithm to cluster sparse code vectors per image into $5$ categories using the SC-VAE$^{\curlywedge}$ model. Images are from the validation set of the ImageNet dataset.}
\caption{Additional unsupervised image segmentation results. These results were generated by grouping sparse code vectors into $5$ categories per image, utilizing the pre-trained SC-VAE$^{\curlywedge}$ model and the K-means algorithm. Images are from the validation set of the ImageNet dataset.}
\label{figure:s4}
\end{figure*}

\begin{figure*}[tbp]
\centering
\includegraphics[width=18cm]{./Figures/flower_cub_isic2016_supple2.png}
\caption{Additional unsupervised image segmentation results on Flowers \cite{nilsback2008automated} (\textit{Left Panel}), CUB \cite{WahCUB_200_2011} (\textit{Middle Panel}) and ISIC-2016 \cite{gutman2016skin} (\textit{Right Panel}). (a) input image. (b) ground truth mask. (c) and (e) segmentation results by clustering sparse code vectors per image into $2$ or $3$ classes using a spectral clustering algorithm. (d) and (f) boundary connectivity information \cite{zhu2014saliency}
was used to decide the foreground and background.}
\label{figure:Flower_CUB}
\end{figure*}

\clearpage
\clearpage
{
   \small
   \bibliographystyle{ieee_fullname}
   \bibliography{egpaper_arxiv_V2}
}

\end{document}
