\documentclass{article}
\usepackage{authblk}
\usepackage[utf8]{inputenc}
\usepackage{cite}
\usepackage{graphicx}
\usepackage{subcaption}
\usepackage{amsmath}

% Keywords command
\providecommand{\keywords}[1]
{
  \small	
  \textbf{\textit{Keywords:}} #1
}


\title{Chemical reactions between carbon atoms confined in single-walled carbon nanotube – a molecular dynamics study }
\author[1]{Somayeh Eskandari*}
% \ead{email1@domain.ca}
\author[1]{János Koltai}
\author[2]{István László}
\author[3]{Mehran Vaezi}
\author[1]{Jenő Kürti*}

\affil[1]{Department of Biological Physics, Eötvös University, Pázmány Péter sétány 1/A, 1117 Budapest, Hungary}
\affil[2]{Department of Theoretical Physics, Budapest University of Technology and Economics, Budafoki út 8, 1111 Budapest, Hungary}
\affil[3]{Institute for Nanoscience and Nanotechnology (INST), Sharif University of Technology, Tehran, Iran}

%*) Corresponding Authors: eskandari.somaye@gmail.com and jeno.kurti@ttk.elte.hu 


\begin{document}
%\date{30 January 2023}


%\today at \currenttime

\maketitle

\begin{abstract}
Carbon nanotubes can serve as one-dimensional nanoreactors for in-tube synthesis of various nanostructures. Experimental observations have shown that chains, inner tubes or nanoribbons can grow by thermal decomposition of organic/organometallic molecules encapsulated in carbon nanotubes. The result of the process depends on the temperature, the diameter of the nanotube, and the type and amount of material introduced inside the tube. Nanoribbons are particularly promising materials for nanoelectronics. Motivated by recent experimental results observing the formation of carbon nanoribbons inside carbon nanotubes, molecular dynamics calculations were performed with the open source LAMMPS code to investigate the reactions between carbon atoms confined within a single walled carbon nanotube. 

Our results show that the interatomic potentials behave differently in quasi-one-dimensional simulations of nanotube-confined space than in three-dimensional simulations. In particular, the Tersoff potential performs better than the widely used ReaxFF potential, in describing the formation of carbon nanoribbons inside nanotubes. The Tersoff potential is a valuable tool for modelling reactions within the nanotubes. We also found a temperature window where the nanoribbons were formed with the fewest defects, i.e. with the largest flatness and the most hexagons, 
which is in agreement with the experimental temperature range.
\end{abstract} \hspace{10pt}

\keywords{molecular dynamics, LAMMPS, interatomic potential, carbon nanotube, carbon nanoribbon, nanoreactor}


\section{Introduction}
In addition to their attractive electrical, optical and mechanical properties, carbon nanotubes are also exceptional materials because, when filled with various atoms or molecules, they can serve as one-dimensional nanocontainers or even nanoreactors for chemical reactions.

One-dimensional confinement in the nanotubes offers the possibility of new types of processes that do not occur in three dimensions, or occur in very different ways, and can thus provide new approaches to different chemical syntheses.

Many good reviews have been published on the encapsulation of various organic and inorganic materials in carbon nanotubes and their chemical reactions within nanotubes \cite{ref01,ref02,ref03,ref04,ref05}. Here we mention just a few of the rich array of relevant experimental studies. 

For the first time, lead was successfully introduced into multi-walled carbon nanotubes (MWCNTs) \cite{Inorg_1}. This has since been replicated with many other metals and other inorganic components \cite{Inorg_2, Inorg_3, Inorg_4, Inorg_5, Inorg_6, Inorg_7, Inorg_8}. The ability of single-walled carbon nanotubes (SWCNTs) to trap organic molecules in their interiors was first demonstrated experimentally on nanopeapods, where $C_{60}$ fullerenes were found to form a “string of pearls” inside SWCNTs of suitable diameter of about 1.4 nm \cite{Peapod_1}. It was shown later that by annealing peapods the fullerene molecules within the host nanotube coalesce and transform into an inner nanotube \cite{Peapod_2, Peapod_3}. The family of peapods that can be obtained by loading nanotubes with different fullerenes, endohedral fullerenes and molecules derived from fullerenes is very rich \cite{Peapod_4,Peapod_5,Peapod_6,Peapod_7,Peapod_8,Peapod_9,Peapod_10,Peapod_11,Peapod_12,Peapod_13,Peapod_14,Peapod_15,Peapod_16}. 
The filling of nanotubes with fullerenes even opens up the possibility of implementing quantum computing with SWCNTs filled with endohedral fullerenes \cite{benjamin2006towards}
Many other organic  \cite{Org_1,Org_2,Org_3,Org_4,Org_5,Org_6} 
and organometallic \cite{Mc_1,Mc_2,Mc_3,Mc_4,Mc_5,Mc_6} molecules can also be encapsulated in carbon nanotubes. 
Numerous experimental observations have shown that linear carbon chains \cite{Cchain_1,Cchain_2,Cchain_3,Cchain_4,Cchain_5,Cchain_6,Cchain_7}, internal carbon tubes \cite{DW_1,DW_2,DW_3,DW_4} or carbon nanoribbons \cite{GNR_1,GNR_2,GNR_3,GNR_4,GNR_5,GNR_7,GNR_8,GNR_9,GNR_10,GNR_11,GNR_12} can be grown by heating organic or organometallic molecules entrapped in SWCNTs.


Carbon nanoribbons are particularly promising materials for nanoelectronic applications because, unlike infinite graphene, they have a non-zero band gap due to their finite width \cite{GNR_13}.
In a recent paper \cite{GNR_12} the synthesis of narrow graphene nanoribbons (GNRs) by high-temperature (800-1200 K) vacuum annealing of ferrocene molecules inside single-walled carbon nanotubes (SWCNTs) was described. 
The formation of two specific armchair graphene nanoribbons, 6-AGNR and 7-AGNR, in (18,0) nanotube was experimentally demonstrated. (6 or 7 is the number of carbon atoms along the width of the ribbon.)
In another recent study, it was demonstrated by tip-enhanced Raman scattering and HRTEM that the encapsulation of 1,2,4-trichlorobenzene molecules in SWCNTs and subsequent annealing in the temperature range of 800-1100 K resulted in the formation of 6-AGNR graphene nanoribbons with lengths of tens of nanometers \cite{cadena2022molecular}.
It should be mentioned that the same kind of procedure but using boron nitride nanotubes instead of carbon nanotubes resulted also in the formation of graphene nanoribbons inside the tubes \cite{cadena2023encapsulation}. These experimental studies \cite{GNR_12,cadena2022molecular} were the main motivation for our molecular dynamics calculations.

In contrast to much experimental work, there is little theoretical work in the literature on the chemical reactions inside carbon nanotubes, and molecular dynamics simulations in particular are rare.

Kim and Tománek have theoretically investigated the microscopic mechanism of fullerene fusion in peapods by calculating total energy change along the optimum reaction path in phase space \cite{Tomanek2004_PRB}.
Their results highlighted the important role of confinement.

Nishio et al studied the formation of $Si$ nanowires inside $(13,0)$ and $(14,0)$ SWCNTs by molecular dynamics simulations \cite{nishio2008formation} where the $Si$ and $C$ atoms were modeled by Tersoff potential \cite{tersoff1988empirical,tersoff1989modeling}. 

In a previous work, we investigated the stability of bamboo defects on the inner tube of double-walled carbon nanotubes grown from peapods by molecular dynamics simulation using the molecular dynamics package DL\_POLY with Tersoff potential \cite{zolyomi2010characteristics}.

Diels-Alder reactions in confined space inside $(8,8)$ and $(9,9)$ SWCNT were investigated by molecular dynamics simulations using CHARMM27 force field in \cite{smith2014confined}.   

Molecular dynamics simulations were carried out by Calvaresi and Zerbetto in order to get insights into adsorption, packing, and fluxes of molecules inside carbon nanotubes \cite{Calvaresi2014_JMaterChemA}.

Marforio et al investigated the aromatic bromination of N-phenylacetamide inside carbon nanotubes by ONIOM calculations \cite{Marforio2017_JPhysChem}.

Previously, we performed molecular dynamics simulations with a self-developed DFT-adjusted tight-binding method for a system where a $(14,0)$ SWCNT was loaded with carbon pentagons (modelling ferrocene molecules without iron and hydrogen) \cite{laszlo2017molecular}. The calculations were carried out at two temperatures, 2000K and 3000K, and the concentration of the carbon atoms varied between 23 – 59 $atoms/nm$. For 33 $atoms/nm$ and 2000K defected ribbon structure was formed whereas in other cases, depending on the actual parameters, chains, inner tubes or even closed cage like molecules were obtained.

The aim of our present work was to perform systematic molecular dynamics simulations for the simplest possible situation, namely to investigate what happens when purely carbon atoms are confined in a single-walled carbon nanotube at a suitable concentration and temperature. This can help us better understand the actual experimental observations, where the reactions take place between larger organic molecules inside the nanotube.
The results of \cite{laszlo2017molecular}, together with the new experimental observations by Kuzmany et al mentioned above \cite{kuzmany2021well}, formed the starting point for the present work. However, instead of the time-consuming diagonalization of a Hamiltonian we used interatomic potentials between atoms to calculate the forces acting on the atoms, which is a much faster procedure. 
Calculations were carried out using LAMMPS, which is a versatile code and is one of the most widely used freely available molecular dynamics simulation packages \cite{LAMMPS1995,LAMMPS2022}. LAMMPS has a number of different inter-atomic potentials, the performance of which depends on the purpose for which they are used and the conditions under which they are used.

For simulating carbon-related processes, the ReaxFF (Reactive Force Field) potential is nowadays most commonly used, in which the large number of parameters are fitted to a large training set obtained from DFT calculations \cite{van2001reaxff,chenoweth2008development}. The ReaxFF potential has been successfully applied to a number of systems and reactions, e.g. oxidation and pyrolysis of hydrocarbon fuels \cite{wang2010chemistry}, mechanical properties of graphene \cite{li2018reaxff}, graphitization of amorphous carbon \cite{de2016graphitization}, carbonization of polymers \cite{kowalik2019atomistic,rajabpour2021low}, impact deformation of peapods \cite{de2020carbon}, structural and mechanical properties of graphene, carbon nanotubes and fullerenes \cite{fthenakis2022evaluating}.
Before starting the simulations inside the nanotube, we repeated the procedure described in \cite{de2016graphitization} to see how the method works for 3-dimensional graphitization. Using the ReaxFF potential, we reproduced their published result. The only difference was that in a simple cubic lattice we used a smaller supercell containing only 14 x 14 x 14 carbon atoms instead of 32 x 32 x 32 carbon atoms, otherwise the procedure was exactly the same. 
The method was then adapted to the case where carbon atoms are confined in a nanotube. Unfortunately, the results were disappointing: instead of hexagonal structures, we obtained many triangles with carbon chains in between (see the detailes later in the Results and Discussion part). The exact structure depended on the concentration of carbon atoms and the temperature. That is, we have seen from our own experience that there is a significant difference between three-dimensional and one-dimensional simulations, even when using the same interatomic potential. To say the least, there is no room for parallel graphene ribbons inside a nanotube. To the best of our knowledge, the ReaxFF potential has not yet been used to simulate processes inside nanotubes. 
We therefore set out to complement ReaxFF with four additional potentials to systematically investigate how these potentials perform when carbon atoms are confined inside a carbon nanotube. The four additional potentials are: Tersoff \cite{tersoff1988empirical,tersoff1989modeling} and its improved versions, AIREBO (adaptive intermolecular reactive empirical bond order) \cite{AIREBO}, REBO-II (2nd generation reactive empirical bond order) \cite{REBO-II} and LCBOP-I (long range carbon bond order potential) \cite{LCBOP-I}. Each potential is suitable for molecular dynamics studies of similar systems.


\section{Methodology}

All molecular dynamics simulations were performed using the open source software LAMMPS. The conditions were chosen as closely as possible to the experimental conditions, with the significant simplification that instead of organic or organometallic molecules, only single carbon atoms were placed inside the nanotubes, randomly arranged as the starting geometry. The reason for considering only carbon atoms, without hydrogen atoms, was that our main goal was to systematically compare five different interatomic potentials - Tersoff, Rebo, Airebo, Lcbop and ReaxFF - and of these, in case of Tersoff potential one cannot include hydrogen atoms.

The initial geometry was the same in all cases: using PACKMOL software, 99 randomly arranged separate carbon atoms were confined inside a $3 \mathrm{~nm}$ long piece of $(18,0)$ SWCNT. The diameter of the tube was $1.4$ $\mathrm{nm}$. Due to the periodic boundary condition, the geometry was repeated along the nanotube. The density of free carbon atoms of 33 atoms $/ \mathrm{nm}$ was in line with the already mentioned experimental observations \cite{GNR_12}. The nanotube wall was kept rigid after, the carbon atoms in the wall did not move at all during the simulation, the wall merely served to lock the carbon atoms inside. Figure 1 shows the initial geometry from the front (left) and from the side (right). To get a better overview of the position of the free carbon atoms, only part of the tube is shown on the right side of Figure 1.

\begin{figure}[!ht]
    \centering
    \includegraphics[width=12cm]{images/m1.jpg}
    \caption{Initial geometry; 99 carbon atoms randomly arranged inside $(18,0)$ SWCNT with length of $3 \mathrm{~nm}$.}
    \label{fig1}
\end{figure}

After performing many convergence and stability calculations in order to find the optimal numerical parameters, we applied the following procedure in all cases. The time step throughout the modelling was $0.1 \mathrm{fs}$. First, we used a conjugate gradient method for a few steps to minimize the forces on the atoms. In this way we avoided the nonphysical high forces between the randomly arranged carbon atoms. This was followed by equilibration at $300 \mathrm{~K}$ for $10 \mathrm{ps}$ by applying an NPT ensemble, where the environmental temperature was controlled by Nosé-Hoover thermostat. After equilibration, while the SWCNT was kept rigid, the system was heated to raise the temperature from $300 \mathrm{~K}$ to annealing temperatures of $T_{\text {anneal }}=1200 \mathrm{~K}, 3000 \mathrm{~K}$ and $4000 \mathrm{~K}$ in a few ps at a rate of $410 \mathrm{~K} / \mathrm{ps}$. Finally the system was annealed for $100 \mathrm{ps}$ at $\mathrm{T}_{\text {anneal }}$. The heating and annealing processes were carried out using an NVT ensemble, which also adjusts the temperature using a Nosé-Hoover thermostat. 

To quantitatively track the change in structure, we used a self-written Python code that calculates different geometric features from the initial random structure every $100 \mathrm{fs}$ during the simulation. These geometric features are the average number of bonds per carbon atom (a bond is defined when the distance between two carbon atoms is less than $1.7$ \AA ), the distribution of $C-C$ bond lengths, the number of different polygons from triangles to octagons, and the average distance from the best inplane fit of the atom positions (planarity).

\section{Results and Discussion}
\subsection{ReaxFF potential}

The most commonly used potential for simulating carbon-related processes today is ReaxFF \cite{van2001reaxff,chenoweth2008development}. In ReaxFF the large number of parameters are fitted to a large training set obtained from DFT calculations. We will see, however, that the ReaxFF potential performs poorly in describing the formation of hexagonal nanoribbons within carbon nanotubes. 
The final structure using ReaxFF potential after $100 \mathrm{ps}$ annealing at three different temperatures of 1200 $\mathrm{K}, 3000 \mathrm{~K}$ and $4000 \mathrm{~K}$ are shown in Figures 2a, 2b and 2c, respectively.

\begin{figure}[!ht]
     \centering
     \begin{subfigure}{\textwidth}
         \centering
         \includegraphics[width=10cm]{images/m2.jpg}
         \caption{Final result at $1200 \mathrm{~K}$.}
          \label{fig2a}
     \end{subfigure}

     \begin{subfigure}{\textwidth}
         \centering
         \includegraphics[width=10cm]{images/m3.jpg}
         \caption{Final result at $3000 \mathrm{~K}$.}
            \label{fig2b}
     \end{subfigure}

     \begin{subfigure}{\textwidth}
         \centering
         \includegraphics[width=10cm]{images/m4.jpg}
         \caption{Final result at $4000 \mathrm{~K}$.}
            \label{fig2c}
     \end{subfigure}
    %\caption{ The final structure using ReaxFF potential after 100 ps annealing at temperatures of $1200 \mathrm{~K}$, $3000 \mathrm{~K}$, and $4000 \mathrm{~K}$.}
     
\caption{ The final structure using ReaxFF potential after 100 ps annealing at three temperatures of $1200 \mathrm{~K}$ (a), $3000\mathrm{~K}$ (b), and $4000 \mathrm{~K}$ (c). At $1200 \mathrm{~K}$ Chains are formed, which are located close to the inner wall of the nanotube. There appeared non-physical triangles, as well. At $3000 \mathrm{~K}$, shorter and longer chains connect several polygons, including non-physical ones such as triangles and rhombuses. At $4000 \mathrm{~K}$, except of triangles, there is no polygon and mostly long chain structures are formed.}
\label{fig2}
\end{figure}


Figure \ref{fig2a} clearly shows that $1200 \mathrm{~K}$ is not an appropriate annealing temperature for the ReaxFF potential, as only chains and non-physical triangles are formed after 100ps. It is noticeable that the chains preferentially "stick" to the inner wall of the nanotube. This is likely a consequence of the long-range interaction incorporated in ReaxFF. When the annealing temperature is $3000 \mathrm{~K}$, various polygons such as pentagons, hexagons and heptagons are formed, but the non-physical triangles still appear and even rhombuses can be observed, see Figure \ref{fig2b}. At $T_{\text {anneal}}=4000 \mathrm{~K}$, almost no polygons are formed, but chain-like structures with cross-links between the chains, some non-physical triangles and some carbon pairs appear, see Figure \ref{fig2c}.

The change of various geometrical features during the simulation can be followed in Figure \ref{fig3}. The lower left panel of Figure \ref{fig3} shows how the average number of bonds per carbon atom increases and converges during the simulation at the three different temperatures. Remember: a bond is defined when the distance between two carbon atoms is less than $1.7$ \AA. Convergence is achieved after about 60 ps at all three temperatures. The converged value is about (or slightly less than) 2 at $1200 \mathrm{~K}$, and it is about $2.5$ at $3000 \mathrm{~K}$, and it becomes again about 2 at $4000 \mathrm{~K}$. Just for comparison: this value is 2 for infinite linear chain; 3 for infinite graphene and for fullerenes and nanotubes; $2.5$ for coronene, 3-2/N for N-AGNR etc. The converged values in the bottom left of Figure \ref{fig3} correspond nicely to the structures shown in Figure \ref{fig2a}, Figure \ref{fig2b} and Figure \ref{fig2c} . Furthermore, at higher temperatures the fluctuations are larger.

In order to quantify further the qualitative differences seen in Figure \ref{fig2a}, Figure \ref{fig2b} and Figure \ref{fig2c}, additional quantities were calculated. These values in Figure \ref{fig3} refer to the unit cell with a length of $3 \mathrm{~nm}$.

\begin{figure}[!ht]
    \centering
    \includegraphics[width=12cm]{images/output_N=-1_Reax.png}
    \caption{Increase of the average number of bonds per carbon atom during the MD simulation (lower left). The occurrence of $C-C$ bond lengths at the end of the MD simulation (top left). The number of hexagons during the MD simulation (top right). The average distance (in Å) from best planar fit during the MD simulation (lower right). $T_{\text {anneal }}$ is $1200 \mathrm{~K}$ (blue), $3000 \mathrm{~K}$ (red) and $4000 \mathrm{~K}$ (black). All for ReaxFF potential.}
    \label{fig3}
\end{figure}

The histograms at the top left panel of Figure \ref{fig3} illustrate the distribution of the $C-C$ distances (bond lengths) at the end of the molecular dynamics simulation for the three different temperatures. The distribution has two maxima at all three temperatures: one at $1.2$ \AA, corresponding to chains, and one at around $1.45$ \AA, corresponding to polygons in the structure. At $1200 \mathrm{~K}$ the contribution of chains and at $3000 \mathrm{~K}$ that of polygons is more significant, in agreement with the structures shown in Figure \ref{fig2a} and Figure \ref{fig2b}. At $4000 \mathrm{~K}$ again more chains and less polygons are formed, in agreement with the structure shown in Figure \ref{fig2c}. Furthermore, there is a general tendency for the distribution to widen with increasing temperature. The number of polygons and especially hexagons is low at any temperature. This is shown in top right panel of Figure \ref{fig3} for the three temperatures. At $1200 \mathrm{~K}$, no hexagons are formed at all, while at $3000 \mathrm{~K}$ there are a few hexagons along with a few pentagons, heptagons and octagons and even with some non-physical triangles and rhombuses. In particular, the number of triangles is high at $3000 \mathrm{~K}$, reaching 10 (not shown in Figure \ref{fig3}). At $4000 \mathrm{~K}$, again, no hexagons are formed, similar to $1200 \mathrm{~K}$. The number of other polygons is also almost zero, except for the triangles, which are almost the same number as at $3000 \mathrm{~K}$.

We are interested in the possibility of creating planar structures, such as carbon nanoribbons. However, carbon atoms are far from being/staying in the same plane at any temperature for ReaxFF potential. The non-planarity at $1200 \mathrm{~K}$ is very high, fluctuating around $2.1$ \AA, which is practically the same as the nonplanarity of the random initial geometry. At $3000 \mathrm{~K}$ and $4000 \mathrm{~K}$ it is slightly lower, but still high, fluctuating around $1.7$ \AA, as shown in the lower right panel of Figure \ref{fig3}.


In summary, the ReaxFF potential performs poorly in describing the formation of hexagonal nanoribbons within SWCNTs. At all three temperatures, carbon atoms are far from forming a planar structure, the tendency to form chains is high, and too many non-physical structures, triangles, are formed.

\subsection{Tersoff potential}
Unlike the ReaxFF potential, the Tersoff potential is more suitable for describing the formation of nanoribbons, as we will see. 
The final structure using Tersoff potential after $100 \mathrm{ps}$ annealing at three different temperatures of $1200\mathrm{K}$, $3000 \mathrm{~K}$ and $4000 \mathrm{~K}$ are shown in Figures \ref{fig4a}, \ref{fig4b} and \ref{fig4c}, respectively.

\begin{figure}[!ht]
%\centering
 \begin{subfigure}{\textwidth}
         \centering
         \includegraphics[width=9cm]{images/m6.jpg}
    \caption{Final Result at $1200 \mathrm{~K}$}
    \label{fig4a}
\end{subfigure}

 \begin{subfigure}{\textwidth}
\centering
         \includegraphics[width=9cm]{images/m7.jpg}
    \caption{Final Result at $3000 \mathrm{~K}$}
    \label{fig4b}
\end{subfigure}

 \begin{subfigure}{\textwidth}
         \centering
         \includegraphics[width=9cm]{images/m8.jpg}
    \caption{Final Result at $4000 \mathrm{~K}$}
    \label{fig4c}
    \end{subfigure}
  
    \caption{The final structure using Tersoff potential after 100 ps annealing at three temperatures of $1200 \mathrm{~K}$ (a), $3000\mathrm{~K}$ (b), and $4000 \mathrm{~K}$ (c).At $1200 \mathrm{~K}$, 22 hexagons are formed from free carbon atoms inside the tube, and the structure is close to be planar.At $3000 \mathrm{~K}$, there are fewer hexagons and the structure bends.At $4000 \mathrm{~K}$, the non-planarity is increased, the structure bends more and a cage-like geometry appears at the right part.}
    \label{fig4}
\end{figure}

At $1200 \mathrm{~K}$, a ribbon-like structure with 22 hexagons $(\operatorname{per} 3 \mathrm{~nm}$ ) is formed and the geometry is more or less flat. In contrast, annealing at $3000 \mathrm{~K}$ and $4000 \mathrm{~K}$ results in a reduction in the number of hexagons and the geometry is not flat.

The change of various geometrical features during the simulation can be followed in Figure \ref{fig5}. The lower left panel of Figure \ref{fig5} shows how the average number of bonds per carbon atom increases and converges during the simulation at the three different temperatures. The convergence is faster than with ReaxFF potential. The converged value is slightly above $2.5$ at $1200 \mathrm{~K}$, whereas it is slightly below $2.5$ at higher temperatures. Furthermore, at higher temperatures the fluctuations are larger.
 

\begin{figure}[!ht]
    \centering
    \includegraphics[width=12cm]{images/output_N=-1_Tersoff.png}
    \caption{ Increase of the average number of bonds per carbon atom during the MD simulation (lower left). The occurrence of $C-C$ bond lengths at the end of the MD simulation (top left). The number of hexagons during the MD simulation (top right). The average distance (in Å) from best planar fit during the MD simulation (lower right). $T_{\text {anneal }}$ is $1200 \mathrm{~K}$ (blue), $3000 \mathrm{~K}$ (red) and $4000 \mathrm{~K}$ (black). All using Tersoff potential.}
    \label{fig5}
\end{figure}

The histograms at the top left panel of Figure \ref{fig5} illustrate the distribution of the $C-C$ distances (bond lengths) at the end of the molecular dynamics simulation for the three different temperatures. At higher temperatures, the distributions are slightly wider. It is worth noting that, looking at the $1200 \mathrm{~K}$ histogram at larger distances (not shown here), after the first maximum of about $1.45$ \AA, the next maxima are at $\sqrt{3} * 1.45$ {\AA} and $2 * 1.45$ \AA, which correspond nicely to the distances of the second and third neighbours in a perfect hexagonal grid.

It is instructive to observe how polygons are formed in the MD simulation. At $1200 \mathrm{~K}$, almost only hexagons are formed, sometimes with two or three heptagons and octagons. At $3000 \mathrm{~K}$ and $4000 \mathrm{~K}$, there are fewer hexagons and more larger polygons. The time dependence of the number of hexagons is shown in top right panel of Figure \ref{fig5} for the three temperatures. At $1200 \mathrm{~K}$, the number of hexagons converges to 22, but at $3000 \mathrm{~K}$ and $4000 \mathrm{~K}$ the number of hexagons fluctuates between 10 and 15.

As shown in Figures \ref{fig4a}, \ref{fig4b} and \ref{fig4c}, the most striking difference between the results of the three temperatures is not only in the number of hexagons but also in the flatness as well. This can be quantified by calculating the average distance of the atoms from the plane fitted to the points using the method of least squares. As shown in the lower right panel of Figure \ref{fig5}, at $1200 \mathrm{~K}$ the non-planarity converges to a small value, with an average deviation from the best in-plane fit of about $0.6$ \AA. At $3000 \mathrm{~K}$ and $4000 \mathrm{~K}$, however, the non-planarity is much larger, fluctuating around $1.25$ {\AA} and $1.5$ \AA, respectively.

In summary, by applying the Tersoff potential at $1200 \mathrm{~K}$, a near-planar hexagonal structure is formed and stabilized a few times in $10 \mathrm{ps}$, although the structure is not defect-free. At annealing temperatures of $3000 \mathrm{~K}$ and $4000 \mathrm{~K}$, the structure is non-planar and consists of far fewer hexagons.

In addition to the Tersoff potential and the ReaxFF potential, three other potentials were also investigated: AIREBO, REBO-II and LCBOP-I.

\subsection{AIREBO potential}

AIREBO is a long-range extension of the Tersoff potential \cite {AIREBO}.
The final structure using AIREBO potential after $100 \mathrm{ps}$ annealing at three different temperatures of $1200 \mathrm{~K}, 3000 \mathrm{~K}$ and $4000 \mathrm{~K}$ are shown in Figures 6a, 6b and 6c , respectively.

\begin{figure}[!ht]
    %\centering
 \begin{subfigure}{\textwidth}
     \centering
    \includegraphics[width=9cm]{images/m10.jpg}
    \caption{Final Result at $1200 \mathrm{~K}$}
    \label{fig6a}
\end{subfigure}

\begin{subfigure}{\textwidth}
    \centering
    \includegraphics[width=9cm]{images/m11.jpg}
    \caption{Final Result at $3000 \mathrm{~K}$.}
    \label{fig6b}
\end{subfigure}

\begin{subfigure}{\textwidth}
    \centering
    \includegraphics[width=9cm]{images/m12.jpg}
    \caption{Final Result at $4000 \mathrm{~K}$}
    \label{fig6c}[h]
    \end{subfigure}
    
    \caption{The final structure using AIREBO potential after $100 \mathrm{ps}$ annealing at three temperatures of $1200 \mathrm{~K}$ (a), $3000 \mathrm{~K}$ (b) and $4000 \mathrm{~K}$ (c). At $1200 \mathrm{~K}$ much of the geometry contains chain-like structures, and a non-physical triangle is also formed at this temperature. At $3000 \mathrm{~K}$, the number of polygons is increased, and a tube-like structure is formed.At $4000 \mathrm{~K}$, a more compact structure appeared, the atoms are entangled with each other.}
\end{figure}

Figure \ref{fig6a} clearly shows that a non-planar chain-like structure is formed at $1200 \mathrm{~K}$ and a triangle is also visible. It is striking that the chains are preferably located close to the inner wall of the nanotube. As in ReaxFF, this is due to the presence of long-range interaction in AIREBO. Annealing at $3000 \mathrm{~K}$ results in a larger number of polygons, forming a tube-like structure at this temperature, see Figures \ref{fig6b}. When the temperature is increased to $4000 \mathrm{~K}$, a more compact structure with many defects is formed, see Figures \ref{fig6c}. At higher temperatures, short-term interactions seem to become more important than long-range ones.

The change of various geometrical features during the simulation can be followed in Figure 7. The lower left panel of Figure 7, shows how the average number of bonds per carbon atom increases and converges during the simulation at the three different temperatures. The converged value is about $2.2$ at $1200 \mathrm{~K}$, whereas it is about $2.5$ at higher temperatures. Furthermore, at higher temperatures the fluctuations are larger.


\begin{figure}[!ht]
    \centering
    \includegraphics[width=12cm]{images/output_N=-1_Airebo.png}
    \caption{Increase of the average number of bonds per carbon atom during the MD simulation (lower left). The occurrence of $C-C$ bond lengths at the end of the MD simulation (top left). The number of hexagons during the MD simulation (top right). The average distance (in Å) from best planar fit during the MD simulation (lower right). $T_{\text {anneal }}$ is $1200 \mathrm{~K}$ (blue), $3000 \mathrm{~K}$ (red) and $4000 \mathrm{~K}$ (black). All using AIREBO potential.}
    \label{fig7}
\end{figure}

The histograms at the top left panel of Figures \ref{fig7} illustrate the distribution of the $C-C$ distances (bond lengths) at the end of the molecular dynamics simulation for the three different temperatures. The distribution widens as the temperature increases, and the position of the center slides up slightly from $1.35$ {\AA} to $1.45$ \AA.
This is consistent with the fact that the number of chain-like pieces with shorter bond lengths decreases with increasing temperature.

The number of hexagons is low at any temperature. This is shown in top right panel of Figures \ref{fig7} for the three temperatures. At $1200 \mathrm{~K}$, no hexagons are formed at all, while at $3000 \mathrm{~K}$ and $4000 \mathrm{~K}$ there are a few hexagons along with a few pentagons, heptagons and octagons. The structure becomes fullerene like at $4000 \mathrm{~K}$, but with many defects.

Carbon atoms are far from being in the same plane at any temperature. The non-planarity converges to $1.9$ {\AA} at $1200 \mathrm{~K}$, decreases to $1.5$ {\AA} at $3000 \mathrm{~K}$ but is still high, and fluctuates between $1.5$ {\AA} and $1.75$ {\AA} at $4000 \mathrm{~K}$, as shown in the lower right panel of Figure \ref{fig7}. In summary, the use of the AIREBO interatomic potential leads to a non-planar structure full of defects chain-like at lower temperatures and cage-like at higher temperatures and thus performs worse than the Tersoff potential in describing the formation of graphene nanoribbons inside SWCNTs.


\subsection{REBO-II potential}
REBO-II is also an extension of the Tersoff potential, but unlike AIREBO, it does not involve long-range interaction \cite{REBO-II}.
The final structure using REBO-II potential after $100 \mathrm{ps}$ annealing at three different temperatures of $1200 \mathrm{~K}, 3000 \mathrm{~K}$ and $4000 \mathrm{~K}$ are shown in Figures 8a, 8b, and 8c, respectively.

\begin{figure}[!ht]
    \centering
   \begin{subfigure}{\textwidth}
     \centering
    \includegraphics[width=9cm]{images/m14.jpg}
    \caption{The final result at $1200 \mathrm{~K}$.}
    \label{fig8a}
\end{subfigure}

 \begin{subfigure}{\textwidth}
     \centering
    \includegraphics[width=9cm]{images/m15.jpg}
    \caption{The final result at $3000 \mathrm{~K}$.}
    \label{fig8b}
\end{subfigure}

 \begin{subfigure}{\textwidth}
     \centering
    \includegraphics[width=9cm]{images/m16.jpg}
    \caption{The final result at $4000 \mathrm{~K}$.}
    \label{fig8c}
\end{subfigure}
\caption{The final structure using REBO-II potential after $100 \mathrm{ps}$ annealing at three temperatures of $1200 \mathrm{~K}$ (a), $3000 \mathrm{~K}$ (b), and $4000 \mathrm{~K}$ (c). At $1200 \mathrm{~K}$, only one hexagon is formed from the free carbon atoms in the tube and several chain structures and large polygons appear. At $3000 \mathrm{~K}$ , the structure is mainly chain-like with several polygons. At $4000 \mathrm{~K}$ a cage-like structure appears at the end of the simulations.}
\end{figure}

\begin{figure}[!ht]
    \centering
    \includegraphics[width=12cm]{images/output_N=-1_Rebo.png}
    \caption{Increase of the average number of bonds per carbon atom during the MD simulation (lower left). The occurrence of $C-C$ bond lengths at the end of the MD simulation (top left). The number of hexagons during the MD simulation (top right). The average distance (in Å) from best planar fit during the MD simulation (lower right). $T_{\text {anneal }}$ is $1200 \mathrm{~K}$ (blue), $3000 \mathrm{~K}$ (red) and $4000 \mathrm{~K}$ (black). All using REBO-II potential.}
    \label{fig9}
\end{figure}


As shown in Figure \ref{fig8a}, at $1200 \mathrm{~K}$, chain-like structures are formed and some larger polygons appear. The chains do not prefer to be close to the inner wall of the nanotubes. This is because, unlike AIREBO, there is no long-range interaction in REBO-II. The structure at $3000 \mathrm{~K}$ is basically similar to that of $1200 \mathrm{~K}$. In contrast, at $4000 \mathrm{~K}$ a cage-like structure is formed.

The change of various geometrical features during the simulation can be followed in Figure \ref{fig9}. The lower left panel of Figure \ref{fig9} shows how the average number of bonds per carbon atom increases and converges during the simulation to about $2.25$ at all three temperatures.


The histograms at the top left panel of Figure 9 illustrate the distribution of the $C-C$ distances at the end of the molecular dynamics simulation for the three different temperatures. The distribution widens as the temperature increases, and the position of the center slides up slightly from $1.4$ {\AA} to $1.5$ \AA.

The number of hexagons is low at any temperature. This is shown in top right panel of Figure 9 for the three temperatures. At $1200 \mathrm{~K}$, only one hexagon is formed, while at $3000 \mathrm{~K}$ and $4000 \mathrm{~K}$ there are a few hexagons along with a few pentagons, heptagons and octagons.

Carbon atoms are far from being in the same plane at any temperature. The non-planarity fluctuates around $1.5$ {\AA} at all three temperatures, as shown in the lower right panel of Figure 9.
Similar to Airebo and Tersoff potentials, at 4000 K, we observe a curved structure. 

In summary, the REBO-II interatomic potential leads to a non-planar structure with a small number of hexagons at all three temperatures, and thus performs worse than the Tersoff potential in describing the formation of graphene nanoribbons inside SWCNTs. 

\subsection{LCBOP potential}

LCBOP-I is a simultaneous bond-order and long-range extension of REBO \cite{LCBOP-I}.
The final structure using LCBOP potential after 100 ps annealing at three different temperatures of 1200 K, 3000 K and 4000 K are shown in Figures \ref{fig10a}, \ref{fig10b} and \ref{fig10c}, respectively.

\begin{figure}[!ht]
    \centering
    
 \begin{subfigure}{\textwidth}
     \centering
    \includegraphics[width=9cm]{images/m18.jpg}
    \caption{The final result at $1200 \mathrm{~K}$.}
    \label{fig10a}
\end{subfigure}

  \begin{subfigure}{\textwidth}
     \centering
    \includegraphics[width=9cm]{images/m19.jpg}
    \caption{The final result at $3000 \mathrm{~K}$.}
    \label{fig10b}
    \end{subfigure}
    
 \begin{subfigure}{\textwidth}
     \centering
    \includegraphics[width=9cm]{images/m20.jpg}
    \caption{The final result at $4000 \mathrm{~K}$.}
    \label{fig10c}
    \end{subfigure}
    
    \caption{The final structure using LCBOP potential after $100 \mathrm{ps}$ annealing at three temperatures of $1200 \mathrm{~K}$ (a), $3000 \mathrm{~K}$ (b), and $4000 \mathrm{~K}$ (c). At $1200 \mathrm{~K}$, no polygons appear, only chain structures of different lengths are formed. At $3000 \mathrm{~K}$, a few polygons are formed beside chains. At $4000 \mathrm{~K}$, only chains appeared.}
\end{figure}

As shown in Figure \ref{fig10a}, at $1200 \mathrm{~K}$, a non-planar chain-like structure without any polygon is appeared. The chains are preferably positioned close to the inner wall of the nanotube. This is a consequence of the long-range interaction present in LCBOP. At $3000 \mathrm{~K}$ some polygons are formed while at $4000 \mathrm{~K}$ again only chains can be observed.

The change of various geometrical features during the simulation can be followed in Figure \ref{fig11}. The lower left panel of Figure \ref{fig11} shows how the average number of bonds per carbon atom increases during the simulation. At $3000 \mathrm{~K}$ and $4000 \mathrm{~K}$ convergence is fast, while at $1200 \mathrm{~K}$ convergence is very slow. The converged value is around 2 at all three temperatures.


\begin{figure}[!ht]
    \centering
    \includegraphics[width=12cm]{images/output_N=-1_Lcbop.png}
    \caption{Increase of the average number of bonds per carbon atom during the MD simulation (lower left). The occurrence of $C-C$ bond lengths at the end of the MD simulation (top left). The number of hexagons during the MD simulation (top right). The average distance (in Å) from best planar fit during the MD simulation (lower right). $T_{\text {anneal }}$ is $1200 \mathrm{~K}$ (blue), $3000 \mathrm{~K}$ (red) and $4000 \mathrm{~K}$ (black). All using LCBOP potential.}
    \label{fig11}
\end{figure}

The histograms at the top left panel of Figure \ref{fig11} illustrate the distribution of the $C-C$ distances at the end of the molecular dynamics simulation for the three different temperatures. The distribution widens as the temperature increases, and the position of the center slides up slightly from $1.3$ {\AA} to $1.4$ \AA.

The number of polygons and especially hexagons is low at any temperature. This is shown in top right panel of Figure \ref{fig11} for the three temperatures. At $1200 \mathrm{~K}$, no hexagon is formed, while at $3000 \mathrm{~K}$ and 4000 $\mathrm{K}$ there are a few hexagons along with a few pentagons, heptagons and octagons.

Carbon atoms are far from being in the same plane at any temperature. The non-planarity fluctuates around $1.75$ {\AA} at $1200 \mathrm{~K}$ and around $1.5$ {\AA} at $3000 \mathrm{~K}$ and $4000 \mathrm{~K}$, as shown in the lower right panel of Figure \ref{fig11}.

In summary, using the LCBOP potential, a non-planar chain-like structure is formed at $1200 \mathrm{~K}$, while some polygons are formed at an annealing temperature of $3000 \mathrm{~K}$. However, the number of hexagons is small and the atoms are far from being in a plane, and thus LCBOP performs worse than the Tersoff potential in describing the formation of graphene nanoribbons inside SWCNTs.
It is concluded that of the five interatomic potentials investigated, the Tersoff potential is the best for describing the formation of carbon nanoribbon-like structures within SWCNTs. When the Tersoff potential is used, the structure will be most planar and the number of hexagons formed will be the largest.

\subsection{Temperature window for Tersoff potential}


In agreement with the experiments, our calculations clearly show the important role of the annealing temperature. According to the experiments, the optimal annealing temperature for the formation of carbon nanoribbons from molecules loaded into the SWCNT is around $1100 \mathrm{~K}$. We have investigated in detail the temperature dependence for the best performing Tersoff potential. We carried out the calculations between $500 \mathrm{~K}$ and $1700 \mathrm{~K}$ in steps of $100 \mathrm{~K}$. To reach a more reliable conclusion, we repeated the simulations with four different random seed numbers, giving the carbon atoms four different random initial velocity distribution. The simulation process itself was identical to the one described earlier. Since the convergence in the case of Tersoff potential is already reached after $20 \mathrm{ps}$ (see Figure \ref{fig4a}), the annealing time in the new simulations was set to $40 \mathrm{ps}$ instead of $100 \mathrm{ps}$. In each case, we calculated the evolution of both the number of hexagons and the planarity (as was defined earlier) during the simulation. Figure \ref{fig12} shows the averaged values 
of mentioned parameters for four simulations with different random initial velocity distributions.


\begin{figure}[ht!]
    \centering
    \includegraphics[width=12cm]{images/m22.jpg}
    \caption{Number of hexagons (circles, blue line, left axis) and non-planarity (triangles, orange line, right axis), averaged over four random initial velocity distribution. The error bars indicate the standard deviations.}
    \label{fig12}
\end{figure}

As can be seen in the Figure \ref{fig12}, there is a temperature window between about $1000 \mathrm{~K}$ and $1200 \mathrm{~K}$ where the nanoribbons are formed with the least defects. In this interval, both the number of hexagons and the planarity of the system are at a maximum. Surprisingly, this is close to the experimentally observed optimum temperature range \cite{GNR_12,cadena2022molecular}, although this is not actually expected since the simulation was done with carbon atoms only, whereas in the real experiment, annealing occurs for nanotubes filled with organic/organometallic molecules.

\section{Conclusion}

Molecular dynamics simulations were carried out on carbon atoms confined in a carbon nanotube with a diameter of $1.4 \mathrm{~nm}$. By comparing five different interatomic potentials (ReaxFF, Tersoff, AIREBO, REBO-II, LCBOP), we investigated when a nanoribbon-like structure can form from a random initial geometry. From this comparison, we conclude that the Tersoff potential provides by far the best performance in describing the formation of carbon nanoribbons inside SWCNTs. When using the Tersoff potential, the structure will be closest to being planar and the number of hexagons generated will be the largest.
This is probably because, as our experience has shown, the Tersoff potential is more prone to hold bond angles around 120° than other potentials, and is therefore more likely to form hexagons.
In general, the Tersoff potential is a valuable tool for modelling reactions in nanotubes.

Our experience shows that the other four potentials do not lead to a planar structure. The reasons for this may be complex: the function describing the potential itself, in particular the consideration of long-range interaction, but also whether the simulation is performed in one, two or three dimensions.  The poor performance of ReaxFF may be particularly surprising, since in three dimensions the ReaxFF potential describes graphite formation very well.
However, our results point out that there is a significant difference between three-dimensional simulations and quasi-one-dimensional simulations in confined space, even when using the same interatomic potential. To say the least, there is no room for parallel graphene ribbons inside the nanotube.
Parametrizing the 2D and 1D versions of ReaxFF can be an interesting task to allow it to perform well on systems such as molecules inside SWCNTs.

For the Tersoff potential, we found a temperature window of around $1000 \mathrm{~K}-1200 \mathrm{~K}$, where the nanoribbons formed with the fewest defects, i.e. the highest planarity and the most hexagons. 
This is close to the temperature range that experiments show is most suitable for the formation of carbon nanoribbons inside carbon nanotubes \cite{GNR_12,cadena2022molecular}.

It should be mentioned that in real experiments, carbon nanoribbons inside SWCNTs are mostly formed by hydrocarbon molecules, which are themselves planar, whereas here we only investigated the interaction of carbon atoms. However, our results can serve as a good starting point for simulations with molecules used in the experiments.


\section*{Acknowledgements}
This research was supported by the Ministry of Culture and Innovation and the National Research, Development and Innovation Office within the Quantum Information National Laboratory of Hungary (Grant No. 2022-2.1.1-NL-2022-00004) and by NKFIH Grant No. K-134437. 
S.E. greatly acknowledges the support from Stipendium Hungaricum No. 23921.
We appreciate the useful discussions with Viktor Zólyomi.

\section*{Data Availability}
The data that support the findings of this study are available
from the corresponding authors upon request.






\bibliographystyle{unsrt}
\bibliography{bib01}


\end{document}
