\vspace{-0.2cm}
\section{Future work} \label{future}
Despite the significant progress made in CDFSL, it still presents unique challenges that require further investigation. Therefore, we outline several promising future research directions, focusing on problem setups, applications, and theoretical developments within the field.

\vspace{-0.2cm}
\subsection{Problem Setups}
\textit{Active Learning based CDFSL.}
In Section~\ref{challenge}, we discussed the challenge of CDFSL, primarily due to domain gap and task shifts, which is especially problematic when the domains are vastly different and target domain data is scarce. Addressing this challenge requires expanding and efficiently utilizing shared information between these domains. Active learning (AL)~\cite{10167762}, which identifies the most informative samples for labeling, has gained attention in both domain adaptation~\cite{ac_da,ac_da1} and few-shot learning~\cite{ac_fsl,ac_fsl1}. For example, \cite{ac_da} improves target domain recognition by prioritizing samples with high uncertainty and diversity. Similarly, \cite{ac_fsl} integrates FSL and AL into FASL, an iterative platform for efficient text classification. Given the ability of AI to select the most informative samples, it is well-suited to enhance both cross-domain and cross-task learning in CDFSL, making it a promising area for future research.

\textit{Source-free CDFSL.} 
Pre-trained models, such as CLIP, pose unique challenges in the CDFSL context due to the inaccessibility of their source data and pre-training strategies \cite{xu2024enhancing, zhuo2024prompt, xu2024step}. Existing CDFSL methods, which focus on improving model transferability by leveraging source data and optimizing pre-training strategies, are not well-suited to these conditions. Moreover, with growing concerns about data privacy, accessing source data in real-world scenarios may infringe on intellectual property rights. Therefore, it is imperative to develop a CDFSL approach that operates independently of source data and pre-training details. This necessitates the exploration of source-free cross-domain few-shot learning (SF-CDFSL) setting tailored to address these constraints effectively. In SF-CDFSL, the algorithms improve FSL performance on the target domain using only a pre-trained source model, without access to the source data. While this approach is well-established in domain adaptation~\cite{sfda1,sfda2,sfda3}, it remains in its early stages for CDFSL. Notable efforts include ~\cite{parameter_fix_6}, which adjusts batch normalization to align source and target distributions,~\cite{xu2024enhancing}, which uses information maximization for implicit alignment, and~\cite{zhuo2024prompt}, which uses limited target domain data to enhance diversity and applies VPT to adapt large-scale models. Given the unique challenges of CDFSL, further research in source-free settings is essential for safeguarding data privacy and usage of pre-trained models.

\textit{Prompt Tuning based CDFSL.} 
Fine-tuning is a common method for adapting pre-trained models in CDFSL, but it becomes costly as model sizes grow. To address this, prompt learning has emerged in natural language processing (NLP) \cite{ptnlp}. Prompt tuning (PT)~\cite{dapt1,zhuo2024prompt} uses minimal prompts to guide the model in understanding the target task, rather than fine-tuning all model parameters. This approach significantly reduces computational cost and storage requirements while maintaining competitive performance on downstream tasks. PT has proven effective in FSL tasks in NLP~\cite{ptnlpfsl1,ptnlpfsl2} and has shown promise in solving domain adaptation problems~\cite{dapt1} in computer vision. Recent works~\cite{zhuo2024prompt, xu2024step} have attempted to apply PT to address the CDFSL problem, achieving competitive performance. However, its effectiveness heavily depends on the prompt design, often requiring careful crafting or optimization to achieve optimal results. In summary, exploring the potential of PT in CDFSL represents a promising avenue for future research.

\textit{Incremental CDFSL.}
Current CDFSL methodologies focus on addressing FSL tasks in the target domain but often suffer from catastrophic forgetting, where performance on the source domain declines. A robust model should retain knowledge from both domains and tasks. Thus, addressing catastrophic forgetting in CDFSL is a critical challenge. Recent advances in incremental and continuous learning have been applied in FSL to tackle task-incremental issues~\cite{cdfsil_1,cdfsil_2}. For example, \cite{cdfsil_1} stabilizes network topology to minimize forgetting of previous classes, while \cite{cdfsil_2} focuses on updating classifiers incrementally, preventing the erasure of knowledge in the feature extractor. Building on these techniques, exploring domain-incremental learning in CDFSL is crucial. The goal is to extend the model to accommodate new domains and tasks while maintaining performance on prior ones.


\vspace{-0.2cm}
\subsection{Applications}

\textit{Rare Cancer Detection.}  
Cancer is a serious disease that demands early detection, and detecting rare cancers is particularly challenging due to the limited availability of data. While FSL has been employed for rare cancer detection~\cite{rcd1}, gathering sufficient auxiliary data from the same distribution as the target data can be challenging. This makes CDFSL an ideal solution, as it allows for the use of auxiliary data from different domains.


\textit{Intelligent Fault Diagnosis.}  
Intelligent fault diagnosis~\cite{app-diagnosis} involves detecting machine faults early using diagnostic methods. However, creating an ideal training dataset for diagnostic models is challenging. To address this challenge,~\cite{app-dia2} incorporated data from other domains and applied FSL algorithms. This makes intelligent fault diagnosis a promising application for CDFSL, allowing models to leverage auxiliary data and improve diagnostic accuracy in the face of limited training samples.

\textit{Solving Algorithmic Bias.}  
AI algorithms depend on training data to address many real-world problems, but inherent biases in the data can be learned and amplified by these algorithms. For example, when certain groups are underrepresented in a dataset, the algorithm may make poor predictions for those groups, which can lead to algorithmic bias~\cite{fairness}. This presents a critical ethical challenge in artificial intelligence. Ideally, AI systems should mitigate bias rather than exacerbate it. CDFSL offers a potential solution by focusing on reducing bias in datasets and generalizing across domains and tasks, effectively addressing domain and task shifts. Furthermore, CDFSL can mitigate performance loss caused by limited data from underrepresented groups, contributing to fairer outcomes.

\vspace{-0.3cm}
\subsection{Theories}
\vspace{-0.1cm}
\textit{Invariant Risk Minimization (IRM).}  
Machine learning systems often capture all correlations in the training data, including spurious correlations arising from data biases. To ensure generalization to new environments, eliminating such spurious correlations is essential. Invariant Risk Minimization (IRM)~\cite{irm} is a learning paradigm that addresses this by estimating nonlinear, invariant, causal predictors across multiple training environments, reducing a system’s dependence on data biases. In CDFSL, spurious correlations learned in the source domain should be discarded when adapting to tasks in the target domain. Thus, despite being in its early stages, IRM shows significant promise for CDFSL due to its emphasis on domain and task migration.

\textit{Multiple Source Domain Organization.}  
While some CDFSL approaches leverage multiple source domains to enhance FSL performance on the target domain, theoretical research on effectively organizing these source domains remains limited. Questions regarding how to select and optimize source domains for maximum FSL performance remain underexplored. Advancing theoretical work in this area could significantly enhance multi-source domain applications in CDFSL. A promising reference is provided by~\cite{mul_theory}, which offers a theoretical foundation for organizing multi-source domains. This could pave the way for more effective and rational strategies in multi-source domain CDFSL.

\textit{Domain Generalization.}  
The ultimate goal of CDFSL is to generalize not only to specific domains but to all domains. Theoretical research on domain generalization~\cite{generalization_theory} is crucial to achieving this. Leveraging this research can enable CDFSL to evolve into a few-shot domain generalization problem, allowing models to generalize across a wide range of domains, thereby enhancing their robustness and adaptability.