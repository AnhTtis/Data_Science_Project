\vspace{-0.3cm}
\section{Conclusion} \label{conclusion}
Cross-domain few-shot learning (CDFSL) significantly enhances the capabilities of few-shot learning (FSL) by utilizing diverse domain samples to improve performance on target domains, thereby addressing the critical challenge of limited labeled data. This survey provides a thorough review of CDFSL, starting from foundational definitions and advancing to its formalization, while distinguishing it from related fields such as domain adaptation and multi-task learning. A primary challenge within CDFSL is the unreliable two-stage empirical risk minimization, which complicates the learning of effective shared features and hinders adaptability to new tasks. We categorize existing strategies into four main approaches: $\mathcal{D}$-Extension, $\mathcal{H}$-Constraint, and $\Delta$-Adaptation, along with hybrid methods, each presenting unique strengths and weaknesses that influence their effectiveness in different scenarios. Looking forward, research should focus on specific areas such as integrating active learning techniques to enhance shared knowledge acquisition, exploring source-free CDFSL approaches to safeguard data privacy, and utilizing prompt learning to streamline model adaptations. By addressing these challenges and exploring these avenues, CDFSL can significantly impact practical applications across various domains, paving the way for robust and adaptable machine learning solutions.