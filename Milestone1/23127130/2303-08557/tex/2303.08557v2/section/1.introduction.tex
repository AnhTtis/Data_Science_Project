\section{Introduction}
% During the past decade, under the joint driving force of big image data and the availability of powerful computing hardware, Machine Learning techniques, in particular Deep Learning~\cite{dl}, have brought revolutionary progress for various computer vision tasks including fundamental ones like image classification~\cite{ic1,ic2}, segmentation~\cite{sg1,app-segmentation}, and synthesis~\cite{is1,is2}, and object detection~\cite{od1,app-detection}. For instance, deep learning has achieved (91.10$\%$ top-1 and 99.02$\%$ top-5) accuracy on the ImageNet image classification challenge, exceeding the cognitive abilities of human beings at 95$\%$ top-5. These capabilities are impressive and unprecedented, especially considering the intrinsic advantages of automation, such as processing data at a much larger scale and efficiency than humans. While it appears that the issue has been resolved, it is important to note that this is merely an experimental outcome within a closed dataset. These huge achievements have been credited to supervised deep learning demanding adequate data and labeling, which, however, remains a substantial disparity from the practical implementation. Firstly, data labeling is an expensive and time-consuming process in many fields, including industrial inspection, endangered species identification, and underwater scene analysis. To address this issue, researchers have explored the use of semi-supervised learning algorithms~\cite{semisurvey}. However, these algorithms often require strict assumptions, such as the smoothness assumption, cluster assumption, manifold assumption, \etc., and have high requirements for training data, such as the need for unlabeled data to be from the same category as labeled data and be evenly distributed. These limitations make them challenging to apply in practice. Furthermore, in certain fields, such as medical imaging, military applications, and remote sensing, data privacy concerns can make it difficult to collect large samples, resulting in only a few available samples.

\textcolor{black}{Deep learning~\cite{dl} has been highly successful in computer vision~\cite{ic1,ic2,sg1,app-segmentation,is1,is2,od1,app-detection}, largely due to the availability of large-scale labeled datasets. However, in many practical scenarios, obtaining such large amounts of labeled data is difficult or costly.} To address this challenge, Few-shot learning (FSL) aims to enable models to learn new tasks with only a limited number of labeled samples. Consequently, this problem has garnered significant attention in both academia and industry due to its broad real-world applications. While humans can easily distinguish between objects after seeing only a few examples, machines struggle to achieve similar efficiency. In domains such as natural scene images, large datasets are readily available, but FSL is crucial in scenarios where collecting large amounts of data is difficult. Since the problem was first introduced in 2006~\cite{fsl-1}, numerous methods have been proposed to tackle the challenges of FSL~\cite{fslsurvey,fslsurvey22,fslsurvey20,fsl18,fslsurvey1}.
%Solving problems with limited supervised information using few-shot learning (FSL) is feasible, as suggested by biological evidence~\cite{human-like}. Humans have an excellent ability to recognize new objects with only a few examples. For instance, children can easily distinguish between a `cat' and a `dog' after seeing only a few pictures, a skill that machines have yet to match at a human-like level. Additionally, in domains like natural scene images, it is relatively easy to acquire large amounts of data. Inspired by human rapid learning and transfer learning, researchers aim to enable deep learning models to learn new categories with only a few samples after being trained on large datasets. Therefore, the goal of FSL is to leverage prior knowledge to learn new tasks with only a few labeled samples, which has attracted significant attention due to its crucial industrial and academic applications. Since this problem was first introduced in 2006~\cite{fsl-1}, numerous research methods have been proposed~\cite{fslsurvey,fslsurvey22,fslsurvey20,fsl18,fslsurvey1}.

% With the development of FSL, limited training data, domain variations, and task modifications make FSL more challenging, leading to the emergence of variants such as semi-supervised FSL~\cite{semifsl}, unsupervised FSL~\cite{ufsl1,ufsl2}, zero-shot learning (ZSL)~\cite{zsl1}, cross-domain FSL (CDFSL)~\cite{feature-wise,bscd-fsl}, and more. These variants are regarded as distinctive cases of FSL tasks in terms of both samples and domain learning. This paper focuses on CDFSL variants. \textcolor{black}{The FSL problem restricts prior knowledge and target tasks to come from the same domain, which means that a large amount of data from the target domain are still required. This is a very strict assumption because in some fields, such as medicine, military, remote sensing, rare species detection, etc., there are not enough data to provide prior knowledge. Taking medical X-ray disease detection as an example, FSL can in principle use X-ray images from different tasks, as prior knowledge, which are, however, still scarce due to data privacy and collection costs. Therefore, researchers try to relax this assumption and obtain prior knowledge from domains with rich data, such as natural scene images, to solve the small-sample X-ray detection problem. However, the large domain difference between natural scenes and medical X-rays leads to the performance degradation of FSL. CDFSL addresses the performance degradation in FSL due to domain gaps between auxiliary data that provide prior knowledge and the data in FSL tasks.} Figure~\ref{int} illustrates the difference between FSL and CDFSL. It has practical applications in many fields with limited supervision information, such as rare cancer detection, video event detection~\cite{multi-task}, object tracking~\cite{osl}, and gesture recognition~\cite{dad}. For instance, in the rare cancer detection, obtaining high-quality supervised cancer samples is typically a challenging and expensive process, and there are legal concerns related to patient privacy. In this case, CDFSL can be used to detect rare cancers by utilizing the prior knowledge acquired from a large amount of natural scene images. However, it combines the challenges of both transfer learning and FSL, namely the existence of domain gaps and class shift between the auxiliary and target data, and the scarcity of sample sizes in the target domain, making it a more challenging task. Therefore, after researchers evaluated the cross-domain problem in FSL approaches in 2019~\cite{clc,rffl},~\cite{feature-wise} introduced the concept of CDFSL for the first time and proposed corresponding solutions in 2020. Since then, CDFSL has gained widespread attention as a branch of FSL, and numerous related works have been published in top publications~\cite{bscd-fsl,st,dynamic,hybrid_1,feature_reweight_6}. Figure~\ref{imaging} presents the milestones of CDFSL technologies from 2019 to the present, including representative CDFSL methods and related benchmarks.
With the development of FSL, challenges such as limited training data, domain variations, and task modifications have led to the emergence of various FSL variants, including semi-supervised FSL~\cite{semifsl}, unsupervised FSL~\cite{ufsl1,ufsl2}, zero-shot learning (ZSL)\cite{zsl1}, and cross-domain FSL (CDFSL)\cite{feature-wise,bscd-fsl}, among others. These variants represent distinctive cases of FSL in terms of sample availability and domain learning. This paper focuses specifically on CDFSL variants. \textcolor{black}{The traditional FSL problem assumes that both prior knowledge and target tasks come from the same domain, which is often restrictive in real-world applications. CDFSL addresses this issue by overcoming the domain gap between auxiliary data (which provides prior knowledge) and the target data in FSL tasks. For example, in medical X-ray disease detection, while FSL could theoretically leverage prior knowledge from X-ray images, such data remains scarce due to privacy concerns and high collection costs. Researchers have thus sought to acquire prior knowledge from domains rich in data, such as natural scene images, to solve small-sample X-ray detection problems. However, the significant domain gap between these domains often leads to performance degradation in FSL. However, CDFSL faces challenges from both transfer learning and FSL, including domain gaps, class shifts, and the scarcity of labeled samples in the target domain, making it a more complex task. Since its formal introduction in 2020~\cite{feature-wise}, CDFSL has garnered widespread attention, with numerous methods published in top venues~\cite{bscd-fsl,st,dynamic,hybrid_1,feature_reweight_6}.} Figure~\ref{imaging} presents the milestones of CDFSL technologies from 2019 to the present, showcasing representative CDFSL methods and related benchmarks.
\begin{figure}[t]
	\centering
  \vspace{-0.3cm}
 	\includegraphics[width=0.9\linewidth]{CDFSLProblem-10.pdf}
  \vspace{-0.3cm}
	\caption{\textcolor{black}{The difference of few-shot learning and cross-domain few-shot learning.}}
 \vspace{-0.4cm}
	\label{int}
\end{figure}

% So far, several existing surveys have made detailed summaries and prospects for FSL~\cite{fsl18,fslsurvey,fslsurvey1,fslsurvey20,fslsurvey22}.~\cite{fsl18} divides FSL into experience learning and concept learning, discussing how to use data from other domains to augment small sample data or rectify existing knowledge. More recently,~\cite{fslsurvey} investigates the minimization of empirical risk and define FSL in terms of experience, task, and performance, while also introducing CDFSL as one of the branches of FSL. Both~\cite{fslsurvey1} and~\cite{fslsurvey20} introduce CDFSL as a variation of FSL.~\cite{fslsurvey1} discusses meta-learning, non-meta-learning, and hybrid meta-learning approaches to FSL, and briefly outlines the pioneering work~\cite{feature-wise} in CDFSL, while~\cite{fslsurvey20} discusses benchmarks and additional works in CDFSL. Furthermore, a taxonomy is provided from the perspective of prior knowledge in~\cite{fslsurvey22}. This paper represents the task shift in FSL as a cross near-domain problem and indicates that existing work cannot address the cross distance-domain problem. All the aforementioned works envision cross-domain issues in FSL as a potential direction. However, there is currently a lack of systematic literature that summarizes and discusses the various related works for CDFSL. \textcolor{black}{Currently, there are two elementary surveys on CDFSL~\cite{wang2023survey,deng2023survey}. \cite{wang2023survey}, classifies existing methods into three categories: benchmark, single source, and multiple source, based on the number of source domains. Meanwhile, \cite{deng2023survey} organizes CDFSL algorithms into two paradigms: data augmentation and feature alignment. In contrast, to stimulate future research and enable newcomers to better understand this challenging problem, this paper, for the first time, provides a classification grounded in theoretical analysis and presents a comprehensive review of the CDFSL problem, offering deeper insights into its underlying principles.} Firstly, this paper collects and analyzes a large body of literature on the topic. The analysis of the reference index reveals that prior to the formal proposal of CDFSL, some works had already focused on the cross-domain issues in the field of FSL~\cite{clc, rffl}. Immediately afterward, its introduction as a branch topic of FSL, CDFSL has gained significant attention and been widely explored. In addition, we define CDFSL using the machine learning definition~\cite{ml,erm1} and transfer learning theory~\cite{tltheory}. Secondly, the analysis of many related papers shows that the unique issue of CDFSL is the unreliable two-stage empirical risk minimization problem. The details are discussed in Section~\ref{background}. Hence, all the CDFSL work requires organization through a scientific taxonomy to address its specific challenges. Next, regarding the question of how to transfer knowledge in CDFSL, \textcolor{black}{this paper provides a comprehensive overview of existing approaches and systematically categorizes them into four distinct categories: $\mathcal{D}$-Extension, $\mathcal{H}$-Constraint, $\Delta$-Adaptation, and hybrid approaches.} To facilitate the understanding of CDFSL and provide a comprehensive evaluation of existing methods, the paper also compiles and introduces a comprehensive collection of relevant datasets and benchmarks. The information related to these datasets and benchmarks is presented in detail, providing valuable insights for researchers and practitioners alike. The paper then goes on to analyze and compare the performance of the different approaches, providing a comprehensive understanding of the state-of-the-art in CDFSL, as discussed in Section~\ref{methods} and Section~\ref{performance}. Finally, we explore future research directions for CDFSL by considering three perspectives, including problem set-ups, applications, and theories, which provide a comprehensive understanding of the field and its potential for future growth. Contributions of this survey can be summarized as follows:
%Several surveys have provided comprehensive overviews of FSL and its variants, with works like \cite{fsl18} categorizing FSL into experiential and conceptual learning, and \cite{fslsurvey} focusing on empirical risk minimization, introducing CDFSL as a branch of FSL. Other surveys, such as \cite{fslsurvey1,fslsurvey20}, discuss meta-learning approaches and benchmarks, while \cite{fslsurvey22} emphasizes the challenges in addressing cross-domain issues within FSL. Two recent surveys on CDFSL~\cite{wang2023survey,deng2023survey} classify methods based on data sources and algorithmic paradigms, focusing on benchmark, single-source, and multi-source categories. In contrast, this paper offers the first classification grounded in theoretical analysis, specifically TSERM (two-stage empirical risk minimization), which we identify as a key issue in CDFSL tasks. Detailed discussion of TSERM and its implications can be found in Section~\ref{background}. Building on TSERM, we organize CDFSL research into four categories: $\mathcal{D}$-Extension, $\mathcal{H}$-Constraint, $\Delta$-Adaptation, and hybrid approaches. We also compile datasets and benchmarks to evaluate these methods, providing a deeper understanding of the core principles and future research directions for CDFSL. Finally, we explore future research directions by considering problem setups, applications, and theories, offering a comprehensive understanding of the field and its growth potential. The contributions of this survey can be summarized as follows:
So far, several surveys have provided comprehensive overviews and future directions for FSL~\cite{fsl18,fslsurvey,fslsurvey1,fslsurvey20,fslsurvey22}. For example,\cite{fsl18} categorizes FSL into experiential and conceptual learning, while\cite{fslsurvey} focuses on empirical risk minimization and defines FSL by experience, task, and performance, introducing CDFSL as a branch of FSL. Both~\cite{fslsurvey1} and~\cite{fslsurvey20} highlight CDFSL as a variant of FSL, discussing meta-learning approaches and benchmarks. Lastly,~\cite{fslsurvey22} offers a taxonomy based on prior knowledge and emphasizes that current methods have yet to fully tackle cross-domain challenges. Collectively, these works point to cross-domain learning as a promising area for future FSL research. \textcolor{black}{Currently, there are two elementary surveys on CDFSL~\cite{wang2023survey,deng2023survey}. \cite{wang2023survey} classifies methods into benchmark, single source, and multiple source categories, while~\cite{deng2023survey} categorizes algorithms into data augmentation and feature alignment paradigms.} In contrast, to stimulate future research and help newcomers better understand this challenging problem, this paper offers the first classification grounded in theoretical analysis and provides a comprehensive review, offering deeper insights into the core principles of CDFSL. Firstly, this paper compiles and analyzes a broad range of literature on the topic. An analysis of the reference index reveals that even before the formal introduction of CDFSL, some works had already tried to solve cross-domain issues within the FSL framework~\cite{clc, rffl}. Following its formal introduction as a branch of FSL, CDFSL has garnered significant attention. Additionally, we define CDFSL using both machine learning theory~\cite{ml,erm1} and transfer learning principles~\cite{tltheory}. Secondly, our analysis highlights that the unique challenge in CDFSL lies in the unreliable nature of two-stage empirical risk minimization. The details are discussed in Section~\ref{background}. To address these challenges, \textcolor{black}{the paper organizes CDFSL research into four categories: $\mathcal{D}$-Extension, $\mathcal{H}$-Constraint, $\Delta$-Adaptation, and hybrid approaches.} We also compile relevant datasets and benchmarks to evaluate the methods, and analyze their performance, as discussed in Sections~\ref{methods} and~\ref{performance}. Finally, we explore future research directions for CDFSL by considering three perspectives, including problem set-ups, applications, and theories, which provide a comprehensive understanding of the field and its potential for future growth. Contributions of this survey can be summarized as follows:

\begin{itemize}
    \item We analyzed existing CDFSL papers and provided a comprehensive survey. We also defined CDFSL formally, connecting it to classic ML~\cite{ml,erm1} and transfer learning theory~\cite{tltheory}. This helps guide future research in the field. % It is general enough to include existing CDFSL works while specific sufficient to clarify the goal of CDFSL. This definition helps set future research targets in the CDFSL area.
    \item We listed relevant learning problems for CDFSL with examples, clarifying their relation and differences. This helps position CDFSL among various learning problems. We also analyzed unique issues and challenges of CDFSL, helping to explore a scientific taxonomy for CDFSL work.
    \item \textcolor{black}{We conducted an extensive literature review, organizing it into a unified taxonomy based on $\mathcal{D}$-Extension, $\mathcal{H}$-Constraint, $\Delta$-Adaptation, and hybrid approaches.} We introduced applicable scenarios for each taxonomy to help discuss its pros and cons. We also presented datasets and benchmarks for CDFSL, summarizing insights from performance results to improve understanding of CDFSL methods.
    \item We proposed promising future directions for CDFSL in problem set-ups, applications, and theories, based on current weaknesses and potential improvements.
\end{itemize}

\begin{figure}%[b]
	\centering
  %\vspace{-0.3cm}
	\includegraphics[width=\linewidth]{response/crop_pipeline.pdf}
 \vspace{-0.5cm}
	\caption{\textcolor{black}{Chronological milestones of CDFSL from 2019 to the present, including representative CDFSL approaches and related benchmarks. Key events include the release of Meta-Dataset~\cite{meta-dataset} and BSCD-FSL~\cite{bscd-fsl} in 2020, the introduction of pioneering works such as~\cite{feature-wise}, and subsequent contributions like~\cite{feature_reweight_1,lscdfsl}. Later works~\cite{st,dynamic,hybrid_1,hybrid_4,hybrid_2} explored new setups, while~\cite{boosting,ata,data_target_1,feature_reweight_5,parameter_weight_2,confess,feature_reweight_9} focused on improving performance. Please see Section~\ref{methods} for details.}}
 \vspace{-0.4cm}
	\label{imaging}
\end{figure}

The remainder of this survey is organized as follows: Section \ref{background} provides an overview of CDFSL, including its definition, challenges, and taxonomy. Section \ref{methods} covers approaches to CDFSL in detail, while Section \ref{performance} presents performance results and evaluates methods. Section \ref{future} explores future directions in set-ups, applications, and theories. Finally, Section \ref{conclusion} concludes the survey.