\section{Conclusion} \label{conclusion}
Cross-domain few-shot learning (CDFSL) significantly enhances the capabilities of few-shot learning (FSL) by utilizing diverse domain samples to improve performance on target domains, thereby addressing the critical challenge of limited labeled data. This survey provides a thorough review of CDFSL, starting from foundational definitions and advancing to its formalization, while distinguishing it from related fields such as domain adaptation and multi-task learning. A primary challenge within CDFSL is the unreliable two-stage empirical risk minimization, which complicates the learning of effective shared features and hinders adaptability to new tasks. We categorize existing strategies into four main approaches: $\mathcal{D}$-Extension, $\mathcal{H}$-Constraint, and $\Delta$-Adaptation, along with hybrid methods, each presenting unique strengths and weaknesses that influence their effectiveness in different scenarios. Looking forward, research should focus on specific areas such as integrating active learning techniques to enhance shared knowledge acquisition, exploring source-free CDFSL approaches to safeguard data privacy, and utilizing prompt learning to streamline model adaptations. By addressing these challenges and exploring these avenues, CDFSL can significantly impact practical applications across various domains, paving the way for robust and adaptable machine learning solutions.
%Cross-domain few-shot learning (CDFSL) is a branch of few-shot learning (FSL) that enhances FSL performance on a target domain using samples from different domains, removing the limitation that the source and target domains must be the same. This approach reduces the need for large amounts of supervised data, which is often a challenge in industrial applications. In this survey, we provide a comprehensive review of CDFSL, starting from the definition of supervised learning and leading to the formalization of CDFSL. We highlight its similarities and differences with related topics such as semi-supervised domain adaptation, unsupervised domain adaptation, domain generalization, and multi-task learning. The primary challenge in CDFSL is the unreliable two-stage empirical risk minimization, along with the difficulty of learning high-quality shared features. We categorize existing solutions into four main strategies: $\mathcal{D}$-Extension, $\mathcal{H}$-Constraint, and $\Delta$-Adaptation, along with hybrid approaches, and assess the strengths and weaknesses of each. Additionally, we introduce key datasets and benchmarks used in CDFSL and review the performance of various techniques. Finally, we discuss future research directions, including exploring new problem setups, applications, and theoretical advancements in CDFSL.