\section{Conclusion} \label{conclusion}
Cross-domain few-shot learning (CDFSL) is a branch of few-shot learning (FSL) that allows models to improve FSL performance on the target domain using samples from other domains, thereby eliminating the constraint of the source and target domains being the same in FSL. It reduces the burden of gathering vast quantities of supervised data for various industrial applications. In this survey, we present a thorough and systematic review of CDFSL, beginning with the definition of supervised learning, naive FSL problem, and leading to the definition of CDFSL. We explore the similarities and distinctions between CDFSL and related topics, such as semi-supervised domain adaptation, unsupervised domain adaptation, domain generalization, few-shot learning, and multi-task learning. Furthermore, we shed light on the main challenge of CDFSL, which is the unreliable two-stage empirical risk minimization, and the difficulties of acquiring excellent shared features. We categorize different approaches to address these challenges as instance-guided, parameter-based, feature post-processing, and hybrid approaches, and examine the advantages and limitations of each one. We also introduce datasets and benchmarks used in CDFSL, and the performance of different techniques. Lastly, we discuss the future directions of CDFSL, including the exploration of problem setups, applications, and theories.
%Cross-domain few-shot learning (CDFSL) aims at processing the FSL tasks on the target domain with the help of the source domain. As a branch of FSL, CDFSL breaks the barriers between different domains in FSL problems, which allows the model to train FSL tasks on the target domain by using samples from other domains. And it also helps to relieve the burden of collecting large-scale supervised data in diverse industrial applications. In this survey, we reviewed CDFSL comprehensively and systematically. Firstly, we define supervised learning, naive FSL problem, and leads to the definition of CDFSL. And we also discuss the similarities and differences of CDFSL with closely related issues such as semi-supervised domain adaptation, unsupervised domain adaptation, domain generalization, few-shot learning, and multi-task learning. Then we illustrate the core issue (two-stage empirical risk minimization) of CDFSL. The difficulty of obtaining excellent shared features in the first stage resulted in the unreliable empirical risk minimizer in the second stage. Based on the issue mentioned above, we summarize the unique challenges of CDFSL. Understanding the special issue and challenges helps categorize different works into instance-guided, feature post-processing, and parameter-based approaches according to how they solve the core issue and challenges. Instance-guided approaches introduce the related examples, feature post-processing methods transfer features to the shared one, and parameter-based techniques search for shared features by tuning additional parameters. In addition, the cons and pros of the methods of different categories were discussed and summarized. Finally, we discuss the future work of CDFSL and the directions of exploration in terms of problem setups, algorithmic bias, applications, and theories.