\section{Future work} \label{future}
Despite significant progress in CDFSL, it continues to present unique challenges that require attention. As such, we outline several promising research directions for the future, which we discuss in terms of problem setups, applications, and theories, respectively.

\subsection{Problem Setups}
\textit{Active Learning-based CDFSL.}
In Section~\ref{challenge}, we discussed the challenge of limited shared knowledge between source and target in CDFSL, caused by domain gaps and task shifts, which is especially pertinent when the source and target domains are vastly different and the target domain data is scarce. To address this challenge, it is vital to find ways to expand and fully utilize the shared information between the source and target. Active learning (AL), which selects the most informative samples for labeling, has gained increasing traction in domain adaptation~\cite{ac_da,ac_da1} and few-shot learning~\cite{ac_fsl,ac_fsl1}. For example,\cite{ac_da} enhances the weights of samples with significant uncertainty in classification and diversity to boost the recognition performance of the target domain. Moreover,\cite{ac_fsl} combines FSL and AL into FASL, a speedy and iterative platform for training text classification models. As AL selects the most informative data, it is well-suited for the CDFSL problem, as it can facilitate cross-domain and cross-task learning. Therefore, incorporating AL to solve the CDFSL problem is a promising avenue for further research.

\textit{Transductive CDFSL.} 
Transductive inference refers to the prediction of individual test samples by observing specific training samples. In cases where training samples are limited and test samples are abundant, the category discriminant model generated through inductive reasoning often yields suboptimal performance. Transductive reasoning, on the other hand, exploits information from unlabeled test samples to identify clusters and enhance classification accuracy. Numerous studies have successfully applied transductive inference to tackle FSL problems, resulting in promising outcomes~\cite{tfsl1,tfsl2,tfsl3}. As a subfield of FSL, utilizing transductive inference to improve CDFSL performance is an encouraging avenue to explore.

\textit{Incremental CDFSL.}
Current CDFSL methodologies are designed to tackle FSL tasks on the target domain but often suffer from catastrophic forgetting, leading to a decline in performance on the source domain. However, a good model should retain previous knowledge from both domains and tasks. However, an effective model must preserve prior knowledge from both domains and tasks. Thus, addressing catastrophic forgetting in CDFSL is of significant concern. Recent advancements in incremental learning and continuous learning have been adopted in FSL to combat task incremental issues~\cite{cdfsil_1,cdfsil_2,cdfsil_3}. For instance,\cite{cdfsil_1} stabilizes the network's topology to minimize the forgetting of previous classes. In contrast,\cite{cdfsil_2} solely updates the classifiers in each incremental session to avoid erasing the feature extractor's knowledge. Encouraged by these techniques, future research in domain-incremental is also essential. Thus, the objective of this setup is to train the model to expand to new domains and tasks while maintaining performance on previous domains and tasks.

\textit{Interpretability-guided CDFSL.}
Current techniques for CDFSL rely on black-box feature generation, which hinders understanding of which features are optimal for generalization and what factors influence the model's performance. Recent research by~\cite{parameter_weight_3} introduces attention to identify the importance of each sample area. However, this approach still needs refinement for cross-domain and cross-task settings. More recently,\cite{cr_1,cr_2} have introduced causal reasoning to explain the causal relationships between factors in FSL, rendering the model more interpretable and capable of acquiring shared knowledge. For example,\cite{cr_1} proposes a Structural Causal Model (SCM) to mine the causal relationships between pre-trained knowledge, sample features, and labels in FSL. Therefore, research focused on interpretability-guided feature representation is a promising direction to enhance the performance of CDFSL models.

\textit{Multi-modal/Multi-view CDFSL.} 
We can enhance the performance of CDFSL by incorporating additional modal information from different modalities, as it has been proved in zero-shot learning~\cite{zsl} that information from diverse modalities can aid in processing unseen tasks. In particular, multi-modal CDFSL can furnish additional insights from varying viewpoints, further enhancing the performance of CDFSL. Therefore, exploring multi-modal CDFSL is a promising research direction to pursue.

\textit{Imbalanced CDFSL.} 
The current CDFSL tasks assume an equitable number of labeled samples in various categories, which may not accurately reflect real-world scenarios. Nonetheless, existing research in FSL has tackled data imbalance problems using techniques such as data augmentation and class imbalance loss. For instance,\cite{im_fsl} proposes a data augmentation method to rebalance the original imbalanced data, while\cite{im_fsl2} suggests a class imbalance loss to tackle the imbalance problem in FSL. Hence, such technologies can be adapted to address the imbalance issue in CDFSL.

\subsection{Applications}
As CDFSL can tackle both domain and task shift problems and few-shot learning problems simultaneously, it has found applications in various computer vision (CV) fields where data is limited. This section will highlight some promising CDFSL applications, including detecting rare forms of cancer~\cite{cancer}, object tracking~\cite{osl}, intelligent fault diagnosis~\cite{app-diagnosis}, and addressing AI algorithm bias, \etc.

\iffalse
\textbf{Computer Vision.}
At present, many practical applications in computer vision involve CDFSL, such as image segmentation~\cite{app-segmentation}, object detection~\cite{app-detection}, image generation~\cite{app-generation}, etc. Furthermore, it also has attracted a lot of attention in many emerging applications, such as re-identification, semantic segmentation~\cite{app_cv1}, etc. Although there have many applications of CDFSL in computer vision, its exploration is still in the early stage. Hence, there are still many exploration directions of CDFSL in computer vision applications.

\textbf{Natural Language Processing (NLP).} CDFSL currently has related applications in natural language processing, for instance intent classification~\cite{app_nlp1}, sentiment classification~\cite{app_nlp2}, etc. Nevertheless, there are still many fields in NLP that strongly demand CDFSL, such as machine translation for minority languages (ancient Chinese poems, Mongolian, Romanian, etc.). Generally, the application exploration of CDFSL on NLP needs to be more.

\textbf{Speech Signal Processing.} CDFSL also has related applications in speech signal processing~\cite{app_speech1,app_speech2}. However, its exploration in speech signal processing is still minimal. Combined with multimodal CDFSL, it has a wide range of application prospects in speech understanding and synthesis.

\textbf{Other Fields.} Furthermore, several CDFSL works in hyperspectral image processing~\cite{Hyperspectral1,parameter_fix_8}. Besides,~\cite{app-micro} applies CDFSL on micro-expression recognition. And the cross-domain few-shot facial expression recognition is solved in~\cite{app-rec2}. Currently, CDFSL still has application prospects in various fields, such as intelligent diagnosis~\cite{app-diagnosis,app-dia2}, agricultural~\cite{app_ag1}, marine~\cite{app_ma1}, etc.
\fi

\textit{Rare Cancer Detection.}
Cancer is a severe disease that requires early detection. The detection of rare cancers is particularly critical due to the scarcity of data. Several studies have employed few-shot learning to address rare cancer detection~\cite{rcd1,rdc2}. However, acquiring a large amount of auxiliary data from the same distribution as the target data is often challenging, necessitating the use of CDFSL in rare cancer detection. CDFSL permits the utilization of auxiliary data from other domains, significantly relaxing the constraints on the source data in FSL, and enhancing low detection rates due to the paucity of medical samples. Therefore, CDFSL is a promising approach to overcome the challenge of rare cancer detection.

\textit{Object Tracking.}
Object tracking~\cite{ot} is a crucial computer vision task that entails predicting the location of selected objects in subsequent frames based on their initial locations in the first frame. This task closely resembles the FSL task setting, which involves classification using minimal data. Consequently, some researchers~\cite{ot1} have applied FSL to object tracking. However, domain gaps frequently exist between auxiliary data and target data due to variations in devices and data acquisition methods. Existing FSL techniques have not effectively tackled these domain gaps. Therefore, CDFSL has emerged as a promising direction for addressing object tracking challenges.

\textit{Intelligent Fault Diagnosis.}
Intelligent fault diagnosis~\cite{app-diagnosis} is the process of detecting machine faults at an early stage using various diagnostic methods. However, establishing an ideal dataset for training intelligent diagnostic models is a challenging task. To address this issue,~\cite{app-dia2} introduced data from other domains and utilized few-shot algorithms. As a result, intelligent fault diagnosis represents a promising application direction for CDFSL.

\textit{Solving Algorithmic Bias.}
AI algorithms currently rely on training data to solve many real-life problems. However, inherent biases in the data can be compiled and amplified by the algorithms. For instance, when there is less information about a particular group in a dataset, the algorithm trained on this data set may make poor predictions for that group, leading to algorithmic bias~\cite{fairness}. This is a critical ethical issue in artificial intelligence. A good AI algorithm should reduce bias in a dataset rather than amplifying it. CDFSL is a potential exploration direction for addressing algorithm bias, as it focuses on reducing bias in datasets and generalizing to the new domains and tasks by addressing domain shift and task shift. Furthermore, CDFSL can help minimize the performance loss caused by having few samples of a specific group in datasets.

\subsection{Theories}
\textit{Invariant Risk Minimization (IRM).} 
Machine learning systems can often pick up all correlations present in the training data, including those that are spurious due to existing data biases. To ensure generalization to new environments, it is crucial to discard such spurious correlations that do not hold in the future. Invariant Risk Minimization (IRM) is a learning paradigm proposed by~\cite{irm} that estimates nonlinear, invariant, causal predictors from multiple training environments to mitigate the over-reliance of machine learning systems on data biases. Although still in its early stages of exploration, IRM is crucial for CDFSL due to the migration of domains and tasks between the source and target domains. In CDFSL, spurious correlations learned in the source domain must be discarded when adapting to the target domain tasks, making the development of IRM important for CDFSL. By exploring IRM for CDFSL, we can significantly enhance the performance on the target domain in CDFSL.

\textit{Multiple Source Domain Organization.} 
Although some current works in CDFSL aim to utilize multiple source domains to improve FSL performance on the target domain, there is still limited theoretical research on how to effectively organize these source domains, including how to select and utilize them to maximize FSL performance. Developing relevant theoretical research in this area can greatly advance the application of multi-source domains in CDFSL. An excellent reference direction for this is provided by ~\cite{mul_theory}, which offers theoretical support for organizing multi-source domains. This could lead to more rational and superior works on multi-source domain CDFSL.

\textit{Domain Generalization.} 
The further goal of CDFSL should be not only generalize to a specific domain but to all domains. Theoretical research on domain generalization~\cite{generalization_theory} is essential to support this goal. Utilizing this research, CDFSL can be transformed into a few-shot domain generalization learning problem, ultimately enabling models to generalize across various domains.