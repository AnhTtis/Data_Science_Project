\section{Datasets and benchmarks}\label{evaluation}
This section exhibits the evaluation tools for CDFSL models, the datasets used in CDFSL task are described in Section~\ref{data}, Section~\ref{benchmark} collects the existing benchmarks utilized for CDFSL problem. And we summarize in Section~\ref{sum}.

\subsection{Datasets} \label{data}
Annotated datasets are used to provide a fair comparison among diferent algorithms and architectures. Furthermore, the growth in complexity, size, annotation number, and transfer difficult of datasets increases the challenge level, encouraging in constant development of new and improved techniques. The most used datasets for the CDFSL task as follows: 
\begin{itemize}
    \item \textit{mini}ImageNet\cite{miniimagenet}: \textit{mini}ImageNet dataset is composed of 60000 images selected from the dataset ImageNet, with a total of 100 categories. Each category has 600 images, and the size of each image is $84 \times 84$.
    
    \item \textit{tiered}ImageNet\cite{tieredimagenet}: This dataset is selected from the ImageNet dataset, including 34 categories, each category contains 10-30 sub-categories (classes). There are 608 classes and 779165 images in this dataset, each class has multiple samples of varying numbers.
    
    \item Plantae\cite{plantae}: Plantae dataset is one of dataset iNat2017. There are 2101 categories and 196613 images in this dataset.
    \item Places\cite{places}: Places dataset contains more 10 million images comparising 400+ unique scene categories. This dataset features 5000 to 30000 training images in each class, consistent with real-world frequencies of occurrence. The image size in this dataset is $200 \times 200$.
    \item Stanford Dogs\cite{dogs}: The Stanford Dogs dataset contains over 20,000 images of 120 breeds of dogs from around the world. This dataset has been built using images and annotation from ImageNet for the task of fine-grained image categorization. It was originally collected for fine-grain image categorization, a challenging problem as certain dog breeds have near identical features or differ in colour and age..
    \item Stanford Cars\cite{cars}: The Cars dataset is a fine-grained classification dataset about cars. It contains 16,185 images of 196 classes of cars. The data is split into 8,144 training images and 8,041 testing images.
    \item CUB\cite{cub}: Images in CUB dataset overlap with images in ImageNet. It is a fine-grained classification dataset about birds that contain 11788 images in 200 categories. The size of images in this dataset is $84 \times 84$.
    \item CropDiseases\cite{cropdiseases}: CropDiseases dataset consists of about 87000 RGB images of healthy and diseased crop leaves which is categorized into 38 different classes. The total dataset is divided into 80/20 ratio of training and validation set. The image size in this dataset is $256 \times 256$.
    \item EuroSAT\cite{eurosat}: EuroSAT is a dataset for land use and land cover classification. The dataset is based on Sentinel-2 satellite images covering 13 spectral bands and consisting out of 10 classes with in total 27,000 labeled and geo-referenced images. Each class includes 2000-3000 images, the size of these images is $64 \times 64$.
    \item ISIC 2018\cite{isic1}\cite{isic2}: ISIC 2018 dataset includes 10015 dermoscopic lesion images from 7 categories for training, 193 images for evaluate, and 1512 images for testing. The size of each image is $600 \times 450$.
    \item ChestX\cite{chestx}: ChestX-ray14 is currently the largest lung X-ray database provided by the NIH Research Institute, which contains 14 lung diseases (atelectasis, consolidation, infiltration, pneumothorax, edema, emphysema, fibrosis, More than 100,000 X-ray front views of fluid, pneumonia, pleural thickening, cardiac hypertrophy, nodules, masses, and hernias), categories 1-14 correspond to 14 lung diseases, and category 15 indicates no disease was found. The size of images in this dataset is $1024 \times 1024$.
    \item Omniglot\cite{omniglot}: The Omniglot dataset is composed of 1,623 handwritten characters from 50 different languages, each with 20 different handwritings, which constitutes an dataset that contains extremely large number of sample categories (1623), but a very small sample number (20) of samples in each category. The size of each image in this dataset is $28 \times 28$.
    \item CIFAR-FS\cite{cifar}: The full name of the CIFAR-FS dataset is the CIFAR100 Few-Shots dataset, which is derived from the CIFAR 100 dataset and contains 100 categories, with 600 images in each category, for a total of 60,000 images. In use, it is usually divided into training set (64 types), validation set (16 types) and test set (20 types), and the image size is unified to $32 \times 32$.
    \item FGVC-Aircraft\cite{aircraft}: FGVC-Aircraft dataset includes 10200 aircraft images (102 aircraft models, 100 images per model). The image resolution is about 1-2 Mpixels.
    \item Describable Textures (DTD)\cite{dtd}: DTD is a texture database, consisting of 5640 images, organized according to a list of 47 terms (categories) inspired from human perception. There are 120 images for each category. Image sizes range between 300x300 and 640x640, and the images contain at least 90% of the surface representing the category attribute.
    \item Quick Draw\cite{draw1}: The Quick Draw Dataset is a collection of 50 million drawings across 345 categories, contributed by players of the game Quick, Draw!
    \item Fungi\cite{fungi}: This datasets contains 100000 fungi images belong to 1394 different categories, which is all fungi classes that have been spotted by the general public in Denmark.
    \item VGG Flower\cite{vgg}: VGG Flower dataset contains 8189 flower images belong to 102 categories. The flowers chosen to be flower commonly occuring in the United Kingdom. Each class consists of between 40 and 258 images.
    \item Traffic Signs\cite{traffic}: Traffic Signs dataset consists of 50,000 images of German road signs in 43 classes.
    \item MSCOCO\cite{mscoco}: The images in MSCOCO dataset are collected from Flickr with 1.5 million object instances belonging to 80 classes labelled and localized using bounding boxes.
\end{itemize}

\begin{table}
\tiny
\centering
\caption{Related datasets.}
\begin{tabular}{|c|c|c|c|c|c|c|c|c|}
\hline
Datasets & Derived from & \makecell[c]{Number of \\ images} & Image size & \makecell[c]{Number of \\ categories} & Content & Fields & Reference     \\ 
 \hline
\textit{mini}ImageNet & ImageNet & 60000 & $84 \times 84$ & 100 & objects classification & natural scene & \cite{miniimagenet}    \\  
\hline
\textit{tiered}ImageNet & ImageNet & 779165 & $84 \times 84$ & 608 & objects classification & natural scene & \cite{tieredimagenet}     \\
\hline
Plantae  & iNat2017 & 196613 & varying & 2101 & plants \& animals classification & natural scene & \cite{plantae}  \\
\hline
Places     & N/A & 10 million & $200 \times 200$ & 400+ & scene classification  & natural scene & \cite{places}   \\
\hline
Stanford Dogs     & ImageNet & 20000+ & varying & 120 & dogs fine-grained classification & natural scene &  \cite{dogs}   \\
\hline
Stanford Cars     & N/A & 16185 & varying & 196 & cars fine-grained classification & natural scene &  \cite{cars}   \\
\hline
CUB   & ImageNet & 11788 & $84 \times 84$ & 200 & birds fine-grained classification & natural scene &  \cite{cub}   \\  
\hline
CropDiseases   & N/A & 87000 & $256 \times 256$ & 38 & crop leaves classification & natural scene &  \cite{cropdiseases}  \\
 \hline
EuroSAT   & Sentinel-2 satellite & 27000 & $64 \times 64$ & 10 & land classification & remote sensing &  \cite{eurosat}   \\
\hline
ISIC 2018   & N/A & 11720 & $600 \times 450$ & 7 & dermoscopic lesion classification & medical &  \cite{isic1}   \\
\hline
ChestX   & N/A & 100K & $1024 \times 1024$ & 15 & lung diseases classification & medical & \cite{chestx}  \\
\hline
Omniglot   & N/A & 25260 & $28 \times 28$ & 1623 & characters classification & character &  \cite{omniglot}   \\
\hline
CIFAR-FS   & CIFAR 100 & 60000 & $32 \times 32$ & 100 & objects classification & natural scene &  \cite{cifar}   \\
 \hline
 
FGVC-Aircraft   & N/A & 10200 & varying & 100 & Aircraft fine-grained classification & natural scene &  \cite{aircraft}   \\
\hline

DTD   & N/A & 5640 & varying & 47 & textures classification & natural scene &  \cite{dtd}   \\
\hline
Quick Draw   & Quick draw! & 50 million & $128 \times 128$ & 345 & drawing images classification & Art &  \cite{draw1}   \\
\hline

Fungi   & N/A & 100000 & varying & 1394 & fungi fine-grained classification & natural scene &  \cite{fungi}   \\
\hline

VGG Flower   & N/A & 8189 & varying & 102 & flowers fine-grained classification & natural scene &  \cite{vgg}   \\
\hline

Traffic Signs   & N/A & 50000 & varying & 43 & Traffic signs classification & natural scene &  \cite{traffic}   \\
\hline

MSCOCO   & N/A & 1.5 million & varying & 80 & objects classification & natural scene &  \cite{mscoco}   \\
\hline
\end{tabular}
\end{table}

\subsection{Benchmarks} \label{benchmark}
This section mainly introduces the benchmarks of CDFSL problem, including \textit{miniImageNet \& CUB} (\textit{mini-CUB}), \textit{Meta-Dataset}, a standard fine-gained classification benchmark \textit{(FGCB)}, \textit{BSCD-FSL}.

\textbf{mini-CUB}.
In the early stage of CDFSL development, researchers usually use \textit{mini-CUB} to verify the performance of the proposed CDFSL algorithm. The images in \textit{mini}Imagenet~\cite{miniimagenet} are all natural images, and \textit{CUB}~\cite{cub} is a fine-grained image dataset about birds. Therefore, as one of the most commonly used benchmarks, it aims to verify the fine-grain based cross-domain performance of the CDFSL algorithm.

\textbf{DomainNet}~\cite{domainnet}
DomainNet is originally proposed to address UDA problem. Some approaches utilize it to evaluate the CDFSL performance. It contains 6 domains, with each domain containing 345 categories of common objects. The domains include Clipart, Infograph, Painting, Quickdraw, Real, and Sketch.

\textbf{Office-Home}~\cite{officehome}
Office-Home is a benchmark dataset for domain adaptation which contains 4 domains where each domain consists of 65 categories. The four domains are: Art, Clipart, Product, and Real-World. It contains 15,500 images, with an average of around 70 images per class and a maximum of 99 images in a class.

\textbf{Meta-Dataset}~\cite{meta-dataset}.
Meta-Dataset is a large-scale diverse benchmark, for measuring various image classification models in realistic and challenging few-shot contexts such as CDFSL. This dataset consists of 10 publicly available natural image datasets, handwritten characters, and graffiti datasets, including ILSVRC-2012 \cite{imagenet}, Omniglot \cite{omniglot}, Aircraft \cite{aircraft}, CUB-200-2011 \cite{cub}, Describable Textures \cite{dtd}, Quick Draw \cite{draw1}, Fungi \cite{fungi}, VGG Flower \cite{vgg}, Traffic Signs \cite{traffic} and MSCOCO \cite{mscoco}. These datasets were chosen because they are free and easy to obtain, span a variety of visual concepts (natural and human-made) and vary in how fine-grained the class definition is. %Figure \ref{meta-dataset} is a visual display of Meta-Dataset.

\textbf{FGCB}.
In the early stage of CDFSL development, researchers focused on fine-grain based cross-domain, so a conventional benchmark was derived: fine-grained benchmark. It contains five datasets such as \textit{mini}ImageNet, Plantae~\cite{plantae}, Places~\cite{places}, Cars~\cite{cars}, and CUB~\cite{cub}. \textit{mini}ImageNet is regard as source domain and other datasets as target domain. All images in the five datasets are natural images. The source domain include 60000 common objects, and the target domains are usually the fine-grained classification dataset, such as the images of Plantae, Cars, CUB are plants and animals, cars, and birds, respectively. This means that for this benchmark, the main challenge of across domains comes is how to transfer the category information from coarse to fine. %The visual display is shown in Figure\ref{fine-grained}.

\textbf{BSCD-FSL}~\cite{bscd-fsl}.
As a more challenging benchmrak in CDFSL, BSCD-FSL includes five datasets consists of  \textit{mini}ImageNet, CropDisease~\cite{cropdiseases}, EuroSAT~\cite{eurosat}, ISIC~\cite{isic1,isic2}, ChestX~\cite{chestx}. CropDisease is a fine-grained datasets of crop leaves, all images in which are industrial natural images. EuroSAT, ISIC, and ChestX have the different imaging ways with natural image, they are satellite images, dermatology images, and radiology images, respectively.

\begin{table}
\tiny
\centering
\caption{Related benchmarks.}
\begin{tabular}{|c|c|c|c|c|c|c|c|}
\hline
Benchmarks & Number of datasets & Source domain & Target domain & Related fields & Reference   \\ 
 \hline
mini-CUB & 2 & \textit{mini}ImageNet & CUB & natural scene & N/A   \\  
\hline
DomainNet & 6 & \textit{mini}ImageNet & \makecell[c]{ Clipart \& Infograph \& Painting \& Quickdraw \& Real \& \\ Sketch } & \makecell[c]{natural scene \\ art} & \cite{domainnet}   \\  
\hline
Office-Home & 4 & Real-World & \makecell[c]{ Art \& Clipart \& Product} & \makecell[c]{natural scene \\ art} & \cite{officehome}   \\  
\hline
Meta-Dataset & 10 & ILSVRC-2012 & \makecell[c]{ Aircraft \& CUB \& DTD \& Fungi \& VGGFlower \& \\ Traffic signs \& MSCOCO \& Omniglot \& Quick draw } & \makecell[c]{natural scene \\ character \\ art} & \cite{meta-dataset}       \\
\hline
FGCB & 5 & \textit{mini}ImageNet & \makecell[c]{Plantae \& Places \& Cars \& CUB } & natural scene &  N/A   \\
\hline
BSCD-FSL & 5 & \textit{mini}ImageNet & \makecell[c]{CropDisease \& EuroSAT \& ISIC \& ChestX } & \makecell[c]{natural scene \\ remote sensing \\ medical} & \cite{bscd-fsl}    \\
\hline
\end{tabular}
\end{table}
% \subsection{Metrics} \label{Sec:Protect_before}

% \subsection{Show Your Evidence} \label{Sec:Protect_Evidence}

\subsection{Summary} \label{sum}
In this section, we introduce the 20 datasets and 6 benchmarks that are commonly used in CDFSL. The currently proposed CDFSL algorithms can be divided into three taxonomies based on these datasets and benchmarks, which will be discussed in next section.