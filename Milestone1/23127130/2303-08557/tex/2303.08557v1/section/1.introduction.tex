\section{Introduction}
During the past decade, under the joint driving force of big image data and the availability of powerful computing hardware, Machine Learning techniques, in particular Deep Learning~\cite{dl}, have brought revolutionary progress for various computer vision tasks including fundamental ones like image classification~\cite{ic1,ic2}, segmentation~\cite{sg1,app-segmentation}, and synthesis~\cite{is1,is2}, and object detection~\cite{od1,app-detection}. For instance, deep learning has achieved (91.10$\%$ top-1 and 99.02$\%$ top-5) accuracy on the ImageNet image classification challenge, exceeding the cognitive abilities of human beings at 95$\%$ top-5. These capabilities are impressive and unprecedented, especially considering the intrinsic advantages of automation, such as processing data at a much larger scale and efficiency than humans. While it appears that the issue has been resolved, it is important to note that this is merely an experimental outcome within a closed dataset. These huge achievements have been credited to supervised deep learning demanding adequate data and labeling, which, however, remains a substantial disparity from the practical implementation. Firstly, data labeling is an expensive and time-consuming process in many fields, including industrial inspection, endangered species identification, and underwater scene analysis. To address this issue, researchers have explored the use of semi-supervised learning algorithms. However, these algorithms often require strict assumptions, such as the smoothness assumption, cluster assumption, manifold assumption, \etc., and have high requirements for training data, such as the need for unlabeled data to be from the same category as labeled data and be evenly distributed. These limitations make them challenging to apply in practice. Furthermore, in certain fields, such as medical imaging, military applications, and remote sensing, data privacy concerns can make it difficult to collect large samples, resulting in only a few available samples.

Solving problems with limited supervised information using few-shot learning (FSL) is feasible based on biological evidence~\cite{human-like}. Humans have an excellent ability to recognize a new object with only a few samples. For instance, children can easily distinguish between a "cat" and a "dog" with only a few pictures, a capability that machines are yet to attain human-like performance. Additionally, in certain scenes such as natural scene images, it is relatively easy to acquire large amounts of data. Researchers are inspired by the rapid learning ability of humans and transfer learning, and hope that deep learning models can quickly learn new categories with only a small number of samples after learning a large amount of data of a certain category. Therefore, the goal of FSL is to leverage prior knowledge to learn new tasks with only a few labeled samples, which has attracted significant attention due to its crucial industrial and academic applications. Since the introduction of this problem in 2006~\cite{fsl-1}, numerous research methods have been proposed~\cite{fslsurvey,fslsurvey22,fslsurvey20,fsl18,fslsurvey1}.

With the development of FSL, limited training data, domain variations, and task modifications make FSL more challenging, leading to the emergence of variants such as semi-supervised FSL~\cite{semifsl}, unsupervised FSL~\cite{ufsl1,ufsl2}, zero-shot learning (ZSL)~\cite{zsl1}, cross-domain FSL (CDFSL)~\cite{feature-wise,bscd-fsl}, and more. These variants are regarded as distinctive cases of FSL tasks in terms of both samples and domain learning. CDFSL addresses the performance degradation in FSL due to domain gaps between auxiliary data that provide prior knowledge and the data in FSL tasks. Figure~\ref{int} illustrates the difference between FSL and CDFSL. It has practical applications in many fields with limited supervision information, such as rare cancer detection, video event detection~\cite{multi-task}, object tracking~\cite{osl}, and gesture recognition~\cite{dad}. For instance, in the rare cancer detection, obtaining high-quality supervised cancer samples is typically a challenging and expensive process, and there are legal concerns related to patient privacy. In this case, CDFSL can be used to detect rare cancers by utilizing the prior knowledge acquired from a large amount of natural scene images. Therefore, CDFSL has significant practical implications for solving real-world problems. However, it combines the challenges of both transfer learning and FSL, namely the existence of domain gaps and class shift between the auxiliary and target data, and the scarcity of sample sizes in the target domain, making it a more challenging task. Therefore, after researchers evaluated the cross-domain problem in FSL approaches in 2019~\cite{clc,rffl},~\cite{feature-wise} introduced the concept of CDFSL for the first time and proposed corresponding solutions in 2020. Since then, CDFSL has gained widespread attention as a branch of FSL, and numerous related works have been published in top publications. Figure~\ref{imaging} presents the milestones of CDFSL technologies from 2020 to the present, including representative CDFSL methods and related benchmarks.
\begin{figure}%[b]
	\centering
  \vspace{-0.3cm}
 	\includegraphics[width=\linewidth]{CDFSLProblem-10.pdf}
  \vspace{-0.5cm}
	\caption{\textcolor{black}{The difference of few-shot learning and cross-domain few-shot learning.}}
 \vspace{-0.5cm}
	\label{int}
\end{figure}

So far, several existing surveys have made detailed summaries and prospects for FSL~\cite{fsl18,fslsurvey,fslsurvey1,fslsurvey20,fslsurvey22}.~\cite{fsl18} divides FSL into experience learning and concept learning, discussing how to use data from other domains to augment small sample data or rectify existing knowledge. More recently,~\cite{fslsurvey} investigates the minimization of empirical risk and define FSL in terms of experience, task, and performance, while also introducing CDFSL as one of the branches of FSL. Both~\cite{fslsurvey1} and~\cite{fslsurvey20} introduce CDFSL as a variation of FSL.~\cite{fslsurvey1} discusses meta-learning, non-meta-learning, and hybrid meta-learning approaches to FSL, and briefly outlines the pioneering work~\cite{feature-wise} in CDFSL, while~\cite{fslsurvey20} discusses benchmarks and additional works in CDFSL. Furthermore, a taxonomy is provided from the perspective of prior knowledge in~\cite{fslsurvey22}. This paper represents the task shift in FSL as a cross near-domain problem and indicates that existing work cannot address the cross distance-domain problem. All the aforementioned works envision cross-domain issues in FSL as a potential direction. However, there is currently a lack of systematic literature that summarizes and discusses the various related works for CDFSL. Hence, in this period of rapid development, and to stimulate future research and enable newcomers to better understand this challenging problem, this paper presents, for the first time, a comprehensive review of the CDFSL problem. Firstly, this paper collects and analyzs a large body of literature on the topic. The analysis of the reference index reveals that prior to the formal proposal of CDFSL, some works had already focused on the cross-domain issues in the field of FSL~\cite{clc, rffl}. Immediately afterward, its introduction as a branch topic of FSL, CDFSL has gained significant attention and been widely explored. In addition, we define CDFSL using the machine learning definition~\cite{ml,fml} and transfer learning theory~\cite{tltheory}. Secondly, the analysis of a large number of related papers shows that the unique issue of CDFSL is the unreliable two-stage empirical risk minimization problem, which stems from the combination of two factors: (1) a significant discrepancy between the source and target domains(both in terms of the tasks they perform and the domains themselves), (2) the limited amount of supervised information available in the target domain. The details are discussed in Section~\ref{background}. Hence, all the CDFSL work requires organization through a scientific taxonomy to address its specific challenges. Next, with regard to the question of how to transfer knowledge in CDFSL, this paper provides a comprehensive overview of existing approaches and systematically categorizes them into four distinct categories: instance-guided, parameter-based, feature post-processing and hybrid approaches. To facilitate the understanding of CDFSL and provide a comprehensive evaluation of existing methods, the paper also compiles and introduces a comprehensive collection of relevant datasets and benchmarks. The information related to these datasets and benchmarks is presented in detail, providing valuable insights for researchers and practitioners alike. The paper then goes on to analyze and compare the performance of the different approaches, providing a comprehensive understanding of the state-of-the-art in CDFSL, as discussed in Section~\ref{methods} and Section~\ref{performance}. Finally, we explore future research directions for CDFSL by considering three perspectives, including problem set-ups, applications, and theories, which provide a comprehensive understanding of the field and its potential for future growth. Contributions of this survey can be summarized as follows:
\begin{figure}%[b]
	\centering
  %\vspace{-0.3cm}
	\includegraphics[width=\linewidth]{cdfsl-pipeline-9.pdf}
 \vspace{-0.5cm}
	\caption{\textcolor{black}{Chronological milestones on CDFSL from 2019 to the present, including representative CDFSL approaches and the related benchmarks. CDFSL was first noticed as a topic in 2020 when two related benchmarks, Meta-Dataset~\cite{meta-dataset} and BSCD-FSL~\cite{bscd-fsl}, were released for CDFSL. The pioneering CDFSL work~\cite{feature-wise} is proposed simultaneously. And~\cite{feature_reweight_1,lscdfsl} are followed proposed, which are the few CDFSL works in 2020. Subsequently,~\cite{st,dynamic,hybrid_1,hybrid_4,hybrid_2} explored many new setups for CDFSL like cross multi-domain few-shot learning, etc. And \cite{boosting,ata,data_target_1,feature_reweight_5,parameter_weight_2,confess,feature_reweight_9} try to improve the CDFSL performance by utilizing different manners. Please see Section~\ref{methods} for details.}}
 \vspace{-0.5cm}
	\label{imaging}
\end{figure}

\begin{itemize}
    \item We analyzed existing CDFSL papers and provided a comprehensive survey, a first of its kind. We also defined CDFSL formally, connecting it to classic ML~\cite{ml,fml} and transfer learning theory~\cite{tltheory}. This helps guide future research in the field. % It is general enough to include existing CDFSL works while specific sufficient to clarify the goal of CDFSL. This definition helps set future research targets in the CDFSL area.
    \item We listed relevant learning problems for CDFSL with examples, clarifying their relation and differences. This helps position CDFSL among various learning problems. We also analyzed unique issues and challenges of CDFSL, helping to explore a scientific taxonomy for CDFSL work.
    \item We conducted an extensive literature review, organizing it into a unified taxonomy based on instance-guided, parameter-based, feature post-processing, and hybrid approaches. We introduced applicable scenarios for each taxonomy, which can help to discuss its pros and cons. We also presented datasets and benchmarks for CDFSL, summarizing insights from performance results, and discussing each category's pros and cons, improving understanding of CDFSL methods.
    \item We proposed promising future directions for CDFSL in problem set-ups, applications, and theories, based on current weaknesses and potential improvements.
\end{itemize}

The remainder of this survey is organized as follows. Section \ref{background} provides an overview of CDFSL, including its formal definition, relevant learning problems, unique issue and challenges, and a taxonomy of existing works in terms of instance, parameter, feature, and hybrid. Section \ref{methods} deals with various approaches to CDFSL problems in detail. Section \ref{performance} presents performance results followed by the pros and cons of approaches from each category. And Section~\ref{future} discusses future directions for CDFSL in terms of set-ups, applications, and theories. Finally, the survey provides conclusions in Section \ref{conclusion}.