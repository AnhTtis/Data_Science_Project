\section{Introduction}
Deep learning~\cite{dl} has been highly successful in computer vision~\cite{sg1,od1,app-detection,zhou2024diffdet4sar,li2024predicting,yang2024saratr,LiSARATRX25}, largely due to the availability of large-scale labeled datasets. However, in many practical scenarios, obtaining such large amounts of labeled data is difficult or costly. To address this challenge, Few-shot learning (FSL) aims to enable models to learn new tasks with only a limited number of labeled samples. Consequently, this problem has garnered significant attention in both academia and industry due to its broad real-world applications. While humans can easily distinguish between objects after seeing only a few examples, machines struggle to achieve similar efficiency. In domains such as natural scene images, large datasets are readily available, but FSL is crucial in scenarios where collecting large amounts of data is difficult. Since the problem was first introduced in 2006~\cite{fsl-1}, numerous methods have been proposed to tackle the challenges of FSL~\cite{fslsurvey,fslsurvey22,fslsurvey20,fsl18,fslsurvey1}.

With the development of FSL, challenges such as limited training data, domain variations, and task modifications have led to the emergence of various FSL variants, including semi-supervised FSL~\cite{semifsl}, unsupervised FSL~\cite{ufsl1,ufsl2}, zero-shot learning (ZSL)\cite{zsl1}, and cross-domain FSL (CDFSL)~\cite{feature-wise,bscd-fsl}, among others. These variants represent distinctive cases of FSL in terms of sample availability and domain learning. This paper focuses specifically on CDFSL variants. The traditional FSL problem assumes that both prior knowledge and target tasks come from the same domain, which is often restrictive in real-world applications. CDFSL addresses this issue by overcoming the domain gap between auxiliary data (which provides prior knowledge) and the target data in FSL tasks, as show in Figure~\ref{int}. For instance, in art image recognition tasks involving scribble, cartoon, or sketch images, FSL could theoretically leverage prior knowledge from related domains like cartoons and sketches. However, such data is often scarce due to copyright restrictions and the high cost of collection. As a result, researchers have turned to data-rich domains, such as natural scene images, to address the challenges of few-shot image recognition in the field of art.
However, the significant domain gap between these domains often leads to performance degradation in FSL. CDFSL faces challenges from both transfer learning and FSL, including domain gaps, class shifts, and the scarcity of labeled samples in the target domain, making it a more complex task. Since its formal introduction in 2020~\cite{feature-wise}, CDFSL has garnered widespread attention, with numerous methods published in top venues~\cite{bscd-fsl,st,dynamic,hybrid_1,feature_reweight_6}. Figure~\ref{imaging} presents the milestones of CDFSL technologies from 2019 to the present, showcasing representative CDFSL methods and related benchmarks.
\begin{figure}%[b]
	\centering
  \vspace{-0.3cm}
 	\includegraphics[width=0.9\linewidth]{CDFSLProblem-10.pdf}
  \vspace{-0.3cm}
	\caption{\textcolor{black}{The difference of few-shot learning and cross-domain few-shot learning.}}
 \vspace{-0.4cm}
	\label{int}
\end{figure}


So far, several surveys have provided comprehensive overviews and future directions for FSL~\cite{fsl18,fslsurvey,fslsurvey1,fslsurvey20,fslsurvey22}. For example,\cite{fsl18} categorizes FSL into experiential and conceptual learning, while\cite{fslsurvey} focuses on empirical risk minimization and defines FSL by experience, task, and performance, introducing CDFSL as a branch of FSL. Both~\cite{fslsurvey1} and~\cite{fslsurvey20} highlight CDFSL as a variant of FSL, discussing meta-learning approaches and benchmarks. Lastly,~\cite{fslsurvey22} offers a taxonomy based on prior knowledge and emphasizes that current methods have yet to fully tackle cross-domain challenges. Collectively, these works point to cross-domain learning as a promising area for future FSL research. Currently, there are two elementary surveys on CDFSL~\cite{wang2023survey,deng2023survey}. \cite{wang2023survey} classifies methods into benchmark, single source, and multiple source categories, while~\cite{deng2023survey} categorizes algorithms into data augmentation and feature alignment paradigms. In contrast, to stimulate future research and help newcomers better understand this challenging problem, this paper offers the first classification grounded in theoretical analysis and provides a comprehensive review, offering deeper insights into the core principles of CDFSL. Firstly, this paper compiles and analyzes a broad range of literature on the topic. An analysis of the reference index reveals that even before the formal introduction of CDFSL, some works had already tried to solve cross-domain issues within the FSL framework~\cite{clc, rffl}. Following its formal introduction as a branch of FSL, CDFSL has garnered significant attention. Additionally, we define CDFSL using both machine learning theory~\cite{ml,erm1} and transfer learning principles~\cite{tltheory}. Secondly, our analysis highlights that the unique challenge in CDFSL lies in the unreliable nature of two-stage empirical risk minimization. The details are discussed in Section~\ref{background}. To address these challenges, the paper organizes CDFSL research into four categories: $\mathcal{D}$-Extension, $\mathcal{H}$-Constraint, $\Delta$-Adaptation, and hybrid approaches. We also compile relevant datasets and benchmarks to evaluate the methods, and analyze their performance, as discussed in Sections~\ref{methods} and~\ref{performance}. Finally, we explore future research directions for CDFSL by considering three perspectives, including problem set-ups, applications, and theories, which provide a comprehensive understanding of the field and its potential for future growth. Contributions of this survey can be summarized as follows:

\begin{itemize}
    \item We analyzed existing CDFSL papers and provided a comprehensive survey. We also defined CDFSL formally, connecting it to classic ML~\cite{ml,erm1} and transfer learning theory~\cite{tltheory}. This helps guide future research in the field.
    \item We listed relevant learning problems for CDFSL with examples, clarifying their relation and differences. This helps position CDFSL among various learning problems. We also analyzed unique issues and challenges of CDFSL, helping to explore a scientific taxonomy for CDFSL work.
    \item We conducted an extensive literature review, organizing it into a unified taxonomy based on $\mathcal{D}$-Extension, $\mathcal{H}$-Constraint, $\Delta$-Adaptation, and hybrid approaches. We introduced applicable scenarios for each taxonomy to help discuss its pros and cons. We also presented datasets and benchmarks for CDFSL, summarizing insights from performance results to improve understanding of CDFSL methods.
    \item We proposed promising future directions for CDFSL in problem set-ups, applications, and theories, based on current weaknesses and potential improvements.
\end{itemize}

\begin{figure}
	\centering
  \vspace{-0.3cm}
        \includegraphics[width=\linewidth]{response/crop_fig2.pdf}
 \vspace{-0.5cm}
	\caption{Chronological milestones of CDFSL from 2019 to the present, including representative CDFSL approaches and related benchmarks. Key events include the release of Meta-Dataset~\cite{meta-dataset} and BSCD-FSL~\cite{bscd-fsl} in 2020, the introduction of pioneering works such as~\cite{feature-wise}, and subsequent contributions like~\cite{feature_reweight_1,lscdfsl}. Later works~\cite{st,dynamic,hybrid_1,hybrid_4,hybrid_2} explored new setups, while~\cite{boosting,ata,data_target_1,feature_reweight_5,parameter_weight_2,confess,feature_reweight_9} focused on improving performance. Please see Section~\ref{methods} for details.}
 \vspace{-0.3cm}
	\label{imaging}
\end{figure}

The remainder of this survey is organized as follows: Section \ref{background} provides an overview of CDFSL, including its definition, challenges, and taxonomy. Section \ref{methods} covers approaches to CDFSL in detail, while Section \ref{performance} presents performance results and evaluates methods. Section \ref{future} explores future directions in set-ups, applications, and theories. Finally, Section \ref{conclusion} concludes the survey.