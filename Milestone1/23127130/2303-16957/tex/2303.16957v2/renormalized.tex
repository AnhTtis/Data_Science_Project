\documentclass[aps,prl,twocolumn]{revtex4-1}
\usepackage{hyperref}
\hypersetup{colorlinks=true, citecolor=blue, urlcolor=blue, linkcolor=blue}

\pdfoutput=1
\usepackage[utf8]{inputenc}
\usepackage[T1]{fontenc}
\usepackage{amssymb}
\usepackage{amsmath}
\usepackage{graphicx}
\usepackage{subcaption}
\usepackage{color}
\usepackage{calc}
\usepackage{mathtools}
\usepackage{tabularx}

\newcommand{\Tr}{\operatorname{Tr}}
\renewcommand{\d}{\textrm{d}}
\renewcommand{\vec}[1]{\mathbf{#1}}
\newcommand{\e}{{\mathrm{e}}}
\newcommand{\im}{\mathrm{i}}
\newcommand{\diagramBox}[2][0.5]{\raisebox{0.5ex-#1\height}{#2}}
\newcommand{\diagramBoxBorder}[4][0.5]{
  \raisebox{0.5ex-#1\height}[0.5ex+\height-#1\height+#2][#1\height-0.5ex+#3]{#4}
}
\newcommand{\TODO}[1]{\textbf{<TODO: #1>}}
\newcommand{\AI}[1]{\textcolor{red}{#1}}
\usepackage[normalem]{ulem}
\newcommand{\SE}{Schr\"odinger equation}
\newcommand{\Eq}[1]{Eq.~\eqref{#1}}
\newcommand{\Fig}[1]{Fig.~\ref{#1}}

\begin{document}

\title{
Averting the infrared catastrophe in the gold standard of quantum chemistry
}
\author{Nikolaos Masios}
\author{Andreas Irmler}
\author{Tobias Sch\"afer}
\author{Andreas Gr\"uneis}
\email{andreas.grueneis@tuwien.ac.at}
\affiliation{
  Institute for Theoretical Physics, TU Wien,\\
  Wiedner Hauptstraße 8-10/136, 1040 Vienna, Austria
}
\date{\today, PREPRINT}

\keywords{particle-particle ladder diagram;
ring diagrams; coupled cluster;
random phase approximation}


\begin{abstract}
Coupled-cluster theories can be used to compute \emph{ab initio} electronic correlation energies
of real materials with systematically improvable accuracy.
However, the widely-used coupled cluster singles and doubles plus perturbative
triples (CCSD(T)) method is only applicable to insulating materials.
For zero-gap materials the truncation of the underlying many-body perturbation expansion
leads to an infrared catastrophe.
Here, we present a novel
perturbative triples formalism that yields convergent correlation energies
in metallic systems. Furthermore, the computed correlation energies for the
three dimensional uniform electron gas at metallic densities are in good agreement with
quantum Monte Carlo results. At the same time the newly proposed method
retains all desirable properties of CCSD(T) such as its accuracy for insulating systems
as well as its low computational cost compared to a full inclusion of the triples.
This paves the way for \emph{ab initio} calculations of real metals
with chemical accuracy.
\end{abstract}

\maketitle

%\section{Introduction}

\emph{Introduction.} -- \emph{Ab initio} methods that achieve
systematically improvable accuracy
for metallic systems are urgently needed to understand chemical reactions on metal surfaces
or to compute the thermodynamic stability of materials.
Currently available exchange and correlation energy density functionals
often fail to achieve the desired
level of accuracy compared to experiment. A prominent failure includes the
incorrect prediction of molecular adsorption sites
on metal surfaces~\cite{Feibelman2001}.  %ref feibelman, alesandro, kresse, etc.
As an alternative, more accurate Quantum Monte Carlo (QMC) calculations can be applied to
metals~\cite{Pozzo2008,Doblhoff2017}. 
However, even diffusion QMC calculations exhibit a strong dependence on the
fixed node approximation~\cite{Nemec2010}.
Compared to QMC and density functional theory, many-electron perturbation theories offer a conceptually different approach to
solve the many-electron problem with high accuracy.  In
particular, coupled-cluster (CC) theory offers a systematically improvable ansatz
for the many-electron wave function employing a series of
higher order particle-hole excitation operators.
While systems exhibiting strong correlation, e.g. stretched covalent bonds,
require high-order excitation operators or multi reference approaches,
single reference systems are already described accurately using low orders~\cite{bartlett2007}.
In particular, at the truncation level of  single, double and
perturbative triple particle–hole excitation operators, CCSD(T) theory predicts
atomization and reaction energies for a wide class of single reference molecules with an
accuracy of approximately 1~kcal/mol~\cite{bartlett2007}.
As such, CCSD(T) is often referred to as the `gold standard' of molecular quantum chemistry.
This also motivated recent applications of coupled-cluster theory to study
solids~\cite{Booth2013,McClain2017,Yang640,grueneis2015b,Gruber2018,Weiler2022}, where
highly accurate predictions of, for example, pressure-temperature phase diagrams~\cite{gruber18b}
could be achieved.
However, it should be noted that such calculations require a careful convergence
with respect to the employed basis sets and system size, which 
is much more complicated than for lower levels of theory~\cite{McClain2017,Gruber2018,Irmler2021}.

Moreover, metallic systems still constitute a major challenge for currently available CC theories.
Although CCSD can be applied to metals, recently obtained results for a number
of metallic systems indicate that CCSD falls short of achieving chemical accuracy in metals,
which is expected and agrees with findings for molecules and insulating solids~\cite{Mihm2021,Neufeld2022}.
The inclusion of the perturbative triples correction (T) though,
is not possible due to the infrared catastrophe
caused by the truncation of the many-electron perturbation expansion~\cite{Shepherd2013}.
The infrared catastrophe leads to a divergence of the CCSD(T) correlation energy per electron for metals
for increasing simulation cell sizes also referred to as the thermodynamic limit (TDL).
A full non-perturbative inclusion of the triple excitation operator in the CC method
is convergent but computationally too expensive and can only be applied to few relatively small systems~\cite{Neufeld2017}.
Here, we present a modification to
the perturbative triples theory that is applicable to metals and
retains all desirable properties including its accuracy for insulating systems
and low computational cost compared to a full inclusion of the triples.

\emph{Theory} --
To better understand the infrared catastrophe of perturbation theories denoted as $(X)$,
we  make use of the following three quantities:
the correlation energy $E_\text{c}^{(X)}$, the electronic transition structure factor $S^{(X)}(\mathbf{q})$
and the quantity $T^{(X)}_i(\mathbf{q})$.
The correlation energy is defined as
%
\begin{equation}
\label{eq:ccdCorr}
        E_\text{c}^{(X)} = \sum_{\mathbf{q}}\upsilon(\mathbf{q})\underbrace{ \Biggl[\, 
               \frac{\delta\upsilon_{ij}^{ab}}{\delta\upsilon( \mathbf{q} )}
               \left(2 t_{ij}^{ab} - t_{ji}^{ab} \right)_{(X)} \Biggr] }_{\coloneqq S^{(X)}(\mathbf{q})}.
\end{equation}
%
%where the Coulomb integrals are given by
%
%\begin{equation}
%\label{eq:coulomb}
%    \upsilon_{ij}^{ab} = \frac{4 \pi}{\Omega \left| \mathbf{k}_i-\mathbf{k}_a \right|^2}
%                         \delta_{\mathbf{k}_i-\mathbf{k}_a, \mathbf{k}_b - \mathbf{k}_j}.
%\end{equation}
%
The indices $i$, $j$ and $a$, $b$ denote occupied and virtual spatial
orbitals, respectively. Einstein summation convention applies to repeated indices throughout this work.
We will firstly focus on second-order perturbation theory,
direct ring CC doubles (rCCD) theory, which is closely related to the random-phase approximation (RPA), 
and CC doubles theory (CCD).
We note that, due to the symmetry of the  uniform electron gas (UEG) Hamiltonian, single excitations
are absent.
Furthermore, in the UEG, the one-electron orbitals are plane waves with wave vectors $\vec
k_i$, $\vec k_j$ and $\vec k_a$, $\vec k_b$.  This allows to write the
two-electron repulsion integral as
%
\begin{math}\upsilon_{ij}^{ab} = \upsilon(\mathbf{q})
\delta_{\mathbf{k}_i-\mathbf{k}_a, \mathbf{k}_b - \mathbf{k}_j}\end{math},
%
with the momentum transfer vector ${\mathbf{q} = \mathbf{k}_i - \mathbf{k}_a}$,
${\upsilon(\mathbf{q}) = \frac{4 \pi}{\Omega \left| \mathbf{q} \right|^2}}$
and $\Omega$ being the volume of the simulation cell. 
The functional derivative
$  \frac{\delta \upsilon_{ij}^{ab}}{\delta \upsilon(\mathbf{q})} =\delta_{\mathbf{q}, \mathbf{k}_b - \mathbf{k}_j} \delta_{\mathbf{q}, \mathbf{k}_i - \mathbf{k}_a} $
enables a concise notation.
The amplitudes $t_{ij}^{ab}$ are obtained by solving the amplitude equations of the employed many-electron
perturbation theory.

\Eq{eq:ccdCorr} introduces the electronic transition structure factor $S(\mathbf{q})$,
which gives access to the dependence of the
correlation energy on the interelectronic interaction distance.
An additional quantity of significance for the present work is given by
\begin{equation}
T^{(X)}_i(\mathbf{q})=
\Biggl[\, 
               \frac{\delta\upsilon_{ij}^{ab}}{\delta\upsilon( \mathbf{q} )}
               {t_{ij}^{ab}}_{(X)} \Biggr].
\end{equation}
Similar to the structure factor, $T^{(X)}_i(\mathbf{q})$ depends on the momentum transfer vector $\mathbf{q}$.

\emph{The infrared catastrophe} --
Having introduced the most important quantities needed for our analysis,
we turn to the case of second-order perturbation theory, which
is a textbook example for the infrared catastrophe. In particular,
we focus on second-order M\o ller-Plesset perturbation theory (MP2), which employs the Hartree--Fock (HF) Hamiltonian
as the unperturbed reference system~\cite{moller_note_1934}.

The MP2 correlation energy is given by Eq.~(\ref{eq:ccdCorr}) using $t_{ij}^{ab} =
\upsilon_{ij}^{ab} / \Delta_{ab}^{ij}$, where
$\Delta^{ij}_{ab}=\varepsilon_i+\varepsilon_j-\varepsilon_a-\varepsilon_b$.
The correlation energy for the UEG diverges
due to the summation over
elements in the amplitudes with both the occupied orbital $i$ and the
virtual orbital $a$ close to the Fermi surface.
The rate of divergence is $\log(q)$~\cite{mattuck} and $\log(\log(q))$~\cite{Harris}
for Hartree or HF orbital
energies, respectively. $q$ refers to the lower spherical cutoff radius in the
analytical integration over $\mathbf{q}$.
Here, we employ HF orbital energies only.
The comparison between analytic and numerical results is slightly complicated
by the fact that $\Delta^{ij}_{ab}$ approaches the analytic behaviour
of $\lim_{\mathbf{q} \to 0}\Delta^{ij}_{ab}\propto |\mathbf q| \ln(|\mathbf q|)$ only slowly
with respect to the studied system size.
However, our numerical findings for $t_{ij}^{ab}$ shown in \Fig{fig:sofg}(c)
% for $S(\mathbf{q})$ and $t_{ij}^{ab}$ 
agree reasonably well with
the analytic result of $\propto |\mathbf q|^{-3}/\log(|\mathbf q|)$.
The infrared catastrophe due to the singularity of $S(\mathbf{q})$ at $|\mathbf{q}|=0$
can be inferred from the plot for MP2 theory shown in \Fig{fig:sofg}(a).

The numerical procedure follows the description in
Ref.~\cite{Shepherd2014}. We use twist-averaging which
helps to reduce the fluctuations due to discretization errors of the finite
simulation cell~\cite{Gruber2018,Drummond2008,Lin2001}.

\begin{figure*}
  \centering
% \includegraphics[width=0.9\linewidth]{fig1.png}
 \includegraphics[width=0.23\linewidth]{StructureFactors_ccd_rccd_rs20.png}
 \includegraphics[width=0.21\linewidth]{StructureFactors_triples_rs20.png}
 \includegraphics[width=0.24\linewidth]{tij_ccd_drccd_rs20.png}
 \includegraphics[width=0.24\linewidth]{Energies_final_twisted.png}
  \caption {
           (a)\&(b) UEG transition structure factor contributions at different levels of theory
           for 246 electrons with 2178 spatial orbitals and $r_{s}=20\,$a.u.
           (c) Contributions to $t_{ij}(\mathbf{q})$ at different levels of theory for 730 electrons with
           6254 spatial orbitals and $r_{s}=20$~a.u.
           (d)\&(e) CBS correlation energies per electron at $r_s=3\,$a.u. for the UEG retrieved as $N^{-2/3}$, where $N$ is the electron number.
           (d) MP2, rCCD, and CCD correlation energies. (e) (T) and (cT) energies. 
           (e) (T) and (cT) correlation energy per electron for Li denoted as Li-(T) and Li-(cT).
           }
            \label{fig:sofg}
\end{figure*}


\emph{The ring summation} --
As already demonstrated by Macke in 1950~\cite{macke_uber_1950},
the divergence in second-order perturbation theory can be averted
by including carefully selected higher-order contributions of the many-electron
perturbation expansion, corresponding to ring diagrams.
Algebraically, this can be implemented by solving the ring-coupled-cluster amplitude equation given by
\begin{equation}
%\begin{split}
\label{eq:t-ring}
t_{ij}^{ab} = \left( \upsilon_{ij}^{ab}
             + 2  \upsilon_{ic}^{ak} t_{kj}^{cb}
             + 2  t_{ik}^{ac} \upsilon_{cj}^{kb} %\right. \\
            + 4 t_{il}^{ad} \upsilon_{dc}^{lk} t_{kj}^{cb}
              \right) / \Delta_{ab}^{ij},
%\end{split}
\end{equation}
which is formally equivalent to solving the Casida equations~\cite{scuseria_2008}.
For the ring-coupled-cluster correlation energy to converge in the long wavelength limit,
the first term on the right-hand side of Eq.(\ref{eq:t-ring}),
which is equivalent to MP2 theory, needs to be partially cancelled by the additional
linear ring (lr) and quadratic ring (qr) terms in the amplitudes.

We now address the
question: How do the terms on the right-hand-side in Eq.~(\ref{eq:t-ring}) cancel each other in the limit $|\mathbf{q}| \rightarrow 0$?
To this end, we re-write Eq.~(\ref{eq:t-ring}) in the following way
\begin{equation}
\begin{split}
\label{eq:t-ring-2nd}
t^\text{rCCD}_{ij}(\mathbf{q}) = \upsilon(\mathbf{q})
  &\left(1 + 2\sum_k t^\text{rCCD}_{kj}(\mathbf{q}) + 2 \sum_k t^\text{rCCD}_{ik}(\mathbf{q}) \right. \\
          &+ \left. 4 \sum_{kl} t^\text{rCCD}_{il}(\mathbf{q}) t^\text{rCCD}_{kj}(\mathbf{q}) \right) / \Delta_{ab}^{ij} \textrm{,}
\end{split}
\end{equation}
which is possible because of momentum conservation and the fact that ring diagrams do not couple different $\mathbf{q}$.
As was shown by Freeman in Ref.~\cite{freeman_coupled-cluster_1977},
the solution of the above equation in the long wave limit leads to
$\lim_{\mathbf{q} \to 0} T^\text{rCCD}_i(\mathbf{q}) = -1/2$.
Consequently, the terms in the parenthesis in Eq.(\ref{eq:t-ring-2nd}) vanish such that
$\lim_{\mathbf{q} \to 0} t^\text{rCCD}_{ij}(\mathbf{q}) \propto {|\mathbf{q}|^{-1}}$~\cite{freeman_coupled-cluster_1977}.
We note that amplitudes diverging as $|\mathbf{q}|^{-1}$ yield convergent correlation energies
and vanishing $S(\mathbf{q})$'s for small $|\mathbf{q}|$.
Furthermore we stress that the above findings are identical for Hartree and
Hartree--Fock orbital energies.
\Fig{fig:sofg}(c) depicts our numerical results for $t^\text{rCCD}_{ij}(\mathbf{q})$,
which diverges with the analytically known behaviour of ${|\mathbf{q}|^{-1}}$ in the long wavelength limit.
The  Supplemental Material (SM)
contains numerical results confirming that $\lim_{\mathbf{q} \to 0} T^\text{rCCD}_i(\mathbf{q}) = -1/2$.

We complement the above discussion
by numerically studying the rCCD transition
structure factor and its individual diagrammatic contributions~\cite{Irmler2019}
given by
\begin{equation}
\label{eq:srccd}
S^\text{rCCD}(\mathbf{q}) = S^\text{MP2}(\mathbf{q})
       + S^\text{lr}(\mathbf{q}) + S^\text{qr}(\mathbf{q}) \text.
\end{equation}
%
The calculated contributions to $S(\mathbf{q})$ are depicted in
Fig.~\ref{fig:sofg}(a).  Although the individual contributions
diverge, the total rCCD transition structure factor converges towards zero in the
limit of $\mathbf{q} \rightarrow 0$ (see Fig.~\ref{fig:sofg}(a)).
We arrive at the first important insight of this work.
The singularity at $|\mathbf{q}|=0$ in $S^\text{MP2}(\mathbf{q})$ is
cancelled by one half of the linear ring terms, whereas the singularity of
the quadratic ring term is cancelled by the `other' half of the linear term.
We also stress that the leading-order behaviour of
$S^\text{rCCD}(\mathbf{q})$ in the limit $|\mathbf{q}| \rightarrow 0$
is $S^\text{rCCD}(\mathbf{q})\propto |\mathbf{q}|$. This was shown by
Bishop and L\"{u}hrmann~\cite{bishop_1982} (see Ref.~\cite{Mihm2023} for a detailed discussion).
For simulation cells with finite electron numbers this leads to a finite size error scaling
of $N^{-2/3}$, where $N$ is the number of electrons. %ref Spencer?
In passing, we note that insulating systems have a different leading order behaviour
around $\mathbf{q} \rightarrow 0$ given by
$S^\text{rCCD}(\mathbf{q})\propto |\mathbf{q}|^2$, implying a finite
size error scaling of $N^{-1}$~\cite{Liao2016,Martin2016}.


\emph{The coupled-cluster doubles method} --
Having established a framework to understand and analyse
the divergence and convergence of correlation energies in MP2 and rCCD,
we now turn to CCD theory.
In addition to ring diagrams,
CCD theory includes further terms, linear and quadratic in $t^\text{CCD}_{ij}(\mathbf{q})$
(the full set of CCD equations can be found in the SM).
This makes an analytic solution impossible even for the simple UEG model.
Note that the contributions of the linear terms are even identical to those from third-order
perturbation theory, if $t^\text{CCD}_{ij}(\mathbf{q})$ is replaced by
$\upsilon_{ij}^{ab} / \Delta_{ab}^{ij}$.
As discussed by Mattuk~\cite{mattuck} for any finite-order
perturbation theory, ring terms yield the most divergent contributions at small $|\mathbf{q}|$.
We stress that the dominance of the ring terms originates from the ``piling-up
of factors $1/q^2$'', as the greatest piling-up occurs for the ring terms when
only a single momentum transfer is involved~\cite{gell-mann_correlation_1957}.
Therefore, this dominance prevails irrespectively of the usage of
$t^\text{CCD}_{ij}(\mathbf{q})$ instead of $\upsilon_{ij}^{ab} /
\Delta_{ab}^{ij}$ at small $|\mathbf{q}|$.
Consequently, it follows, in agreement with the work of Emrich and Zabolitzky~\cite{Emrich1984},
that for the long wavelength limit: (i) in the CCD
amplitude equations the most divergent contributions are the ones
given in Eq.(\ref{eq:t-ring-2nd}), and (ii)
these ring contributions to the CCD amplitudes must therefore cancel each other
precisely as for the rCCD amplitude equations.
This leads us to one pivotal conclusion of the
present work:
$\lim_{|\mathbf{q}| \to 0} T^\text{CCD}_i(\mathbf{q}) = \lim_{|\mathbf{q}| \to 0} T^\text{rCCD}_i(\mathbf{q}) = -1/2 $.


We corroborate the above paragraph with numerical results for the individual contributions
to $t^\text{CCD}_{ij}(\mathbf{q})$.
\Fig{fig:sofg}(c) depicts that the lr and qr contributions
diverge as $\propto {|\mathbf{q}|^{-3}/\log(q)}$. Note that these lr and qr contributions are evaluated using
CCD amplitudes.
Moreover, it is shown that the remaining linear and quadratic contributions
to the CCD amplitudes
denoted as rest-linear and rest-quadr, respectively, 
diverge with a weaker power for $\mathbf{q} \to 0$.
This underpins the conclusions drawn above that in the long wavelength limit: (i) the ring
contributions to CCD amplitudes dominate, and (ii) the ring terms cancel each other
precisely as in rCCD.
Furthermore, we find that $t^\text{CCD}_{ij}(\mathbf{q})\propto {|\mathbf{q}|^{-1}}$,
which leads to a transition structure
factor  $S^\text{CCD}(\mathbf{q})$ depicted in
\Fig{fig:sofg}(a) that approaches zero in the limit of  $\mathbf{q} \to 0$.

\emph{Triple particle-hole excitation operators} --
We now turn to CC theories that approximate the triple particle-hole
excitation operator in a perturbative manner.
In these cases the post-CCSD correlation energy and transition structure factor
contributions are given by
\begin{equation}
\label{eq:corr_triples}
        E_\text{c}^{(J)} = \sum_{\mathbf{q}}\upsilon(\mathbf{q})\underbrace{ \Biggl[\, \\
                            \frac{\delta W_{ijk}^{abc}}{\delta\upsilon( \mathbf{q} )}
                           \left( A_{abc}^{ijk} \right)_{(J)} \Biggr] }_{\coloneqq S^{(J)}(\mathbf{q})}.
\end{equation}
Here, ($J$) refers to the employed approximation.
In the case of  (T), $A_{abc}^{ijk}=\bar{W}_{abc}^{ijk}/\Delta_{abc}^{ijk}$ with
$\bar{X}^{ijk}_{abc}=\frac{4}{3}{X}^{ijk}_{abc} -2{X}^{ijk}_{acb}  +\frac{2}{3}{X}^{ijk}_{bca}  $.
$\Delta_{abc}^{ijk}$ refers to the difference in HF orbital energies for the occupied and unoccupied
states labelled by the respective indices.
$W^{abc}_{ijk}$ is defined as
\begin{equation}
\label{eq:triples_residuum}
               W_{ijk}^{abc} = P_{ijk}^{abc} \left( t_{ij}^{ae}\upsilon_{ek}^{bc} - t_{im}^{ab}\upsilon_{jk}^{mc} \right).
\end{equation}
The permutation operator $P_{ijk}^{abc}$ is defined as $P_{ijk}^{abc}X_{ijk}^{abc}= 
X_{ijk}^{abc} + X_{ikj}^{acb} + X_{kij}^{cab} + X_{kji}^{cba} + X_{jki}^{bca} + X_{jik}^{bac}$.
We employ the following functional derivative for a concise notation used to define the structure factor:
${\frac{\delta W_{ijk}^{abc}}{\delta\upsilon( \mathbf{q} )} = P_{ijk}^{abc}
                           \left( t_{ij}^{ae}\delta_{\mathbf{q},\mathbf{k}_k-\mathbf{k}_c}
                          - t_{im}^{ab}\delta_{\mathbf{q},\mathbf{k}_k-\mathbf{k}_c} \right)}$.
\Fig{fig:sofg}(b) depicts $S^\text{(T)}(\mathbf{q})$, which shares many
similarities with $S^\text{MP2}(\mathbf{q})$.
Both $S(\mathbf{q})$'s exhibit a singularity for ${|\mathbf{q}| \to 0}$
and yield correlation energies per electron that diverge as the system size increases.
The infrared catastrophe of (T) is
caused by the unscreened Coulomb interactions included in the approximation to $A_{abc}^{ijk}$~\cite{Shepherd2013}.

Here, we introduce a novel correlation energy expression
that yields convergent energies for the UEG at the level of triple particle-hole
excitation operators 
 without depending on any ad-hoc parameters.
We refer to the method as (cT) because it includes the \emph{complete} set of terms 
present in the triples amplitude equations in a non-iterative manner.
Naturally, the coupling of triples amplitudes with each other is disregarded.
Thus, in (cT) we use the following approximation to $A_{abc}^{ijk}=\bar{M}_{abc}^{ijk}/\Delta_{abc}^{ijk}$,
where
\begin{equation}
\label{eq:M_piecuch}
               M_{abc}^{ijk} = P_{ijk}^{abc} \left( t_{ij}^{ae}\upsilon_{ek}^{bc} + 2t_{ij}^{ae} \upsilon_{ef}^{bm} t_{mk}^{fc} + \ldots \right).
\end{equation}
For brevity we show only a selection of terms that exhibit a divergent behaviour for
${\mathbf{q} \to 0}$.
Note that the (T) and (cT) approximation to $A_{abc}^{ijk}$ agrees for all terms linear in $t_{ij}^{ab}$.
To cancel the divergence in (T), (cT) includes additional ring terms 
such as the second term in the above equation. 
We show the diagrams corresponding to both terms from Eq.~(\ref{eq:M_piecuch}) in \Fig{fig:sofg}(b).
For these terms, it follows that
$\upsilon_{ek}^{bc} + 2 \upsilon_{ef}^{bm} t_{mk}^{fc}=\upsilon_{ek}^{bc}
(1-2T_m(\mathbf{q}') )$, with $\mathbf{q}' = \mathbf{k}_k - \mathbf{k}_c$.
Note that ${\mathbf{q}' \to 0}: 1-2T_m(\mathbf{q}')=0 $,
which is formally identical to our results for the cancellation of the
divergence between the linear ring and the MP2 terms in the long wavelength
limit.
We stress that (cT) also includes
additional \textit{rest} terms that are not related to the divergence of (T).
The complete expression of $M_{abc}^{ijk}$ including all contributions can be found in
the SM and Ref.~\cite{Piecuch2002}.

\Fig{fig:sofg}(b) confirms that $S^\text{(cT)}(\mathbf{q})$
converges in the long wavelength limit.
Furthermore, $S^\text{(cT)-ring}(\mathbf{q})$ is shown to cancel the divergence of $S^\text{(T)}(\mathbf{q})$.
For completeness, \Fig{fig:sofg}(b) also depicts $S^\text{(cT)-rest}(\mathbf{q})$, which includes all
contributions to Eq.~(\ref{eq:M_piecuch}) that are not included in (T) or (cT)-ring,
e.g. exchange-like terms. Our findings show that $S^\text{(cT)-rest}(\mathbf{q})$ converges to zero as ${\mathbf{q} \to 0}$.
This shows that the (cT) correlation energy expression averts the infrared catastrophe of the
(T) approximation for the UEG without requiring an iterative solution of the triple amplitudes
as recently proposed in Ref.~\cite{Neufeld2023}.

\emph{Results.} -- We now turn to the discussion of numerical results for correlation energies
obtained for electron gas simulation cells with various electron numbers at different levels of theory.
\Fig{fig:sofg}(d-e) displays the behavior of the correlation energy per electron
as a function of $N^{-2/3}$ for MP2, rCCD, CCD, (T) and (cT).
The employed system sizes vary from $N=38$ electrons to $342$ electrons.
All presented correlation energies have been extrapolated to the complete basis set (CBS) limit.
The infrared catastrophe in MP2 theory becomes visible
on approach to the TDL ($N\to \infty$) for \Fig{fig:sofg}(d).
In contrast, rCCD and CCD correlation energies deviate considerably from the MP2 counterpart,
converging as $N^{-2/3}$ to the TDL.
In an analogous way, \Fig{fig:sofg}(e)
verifies that the (T) correlation energy contribution
also diverges as we move to the TDL, while its counterpart, (cT), exhibits a behaviour
that more closely resembles that of rCCD and CCD theories.


Table~\ref{tab:energies} summarizes correlation energies obtained from CCD, CCD(T) and 
CCD(cT) methods compared to i-FCIQMC, DMC and CCDT (CCD with full triple excitations) results.
The energies were extrapolated to the CBS limit, and the systems were parametrized by a range of
Wigner-Seitz radii ($r_{s}=1, 2, 3, 5, 10$ a.u.) and $N=14, 54$ electrons.
We first discuss the results obtained for the 14 electron system.
In this case i-FCIQMC can be viewed as an exact reference, whereas CCDT serves as reference
for any approximate triples theory.
CCDT results from Ref.~\cite{Neufeld2017} are in good agreement with i-FCIQMC in the high density limit and
differ at low densities. Higher levels of CC theory are needed to capture all
important correlation effects for increasing $r_{s}$~\cite{Neufeld2017}.
CCD(T) and CCD(cT) correlation estimates are in good agreement with CCDT.
We note that the agreement between CCD(T) and CCDT is
fortuitous and only valid for small $N$,
as can be seen from the divergence of CCD(T) as a function of $N$ in
\Fig{fig:sofg}(e).

We now discuss the results obtained for the 54 electron system summarized in Table~\ref{tab:energies}.
Here, we compare to DMC reference results.
In this case CCD(T) is fortuitously close to DMC
even at relatively low densities corresponding to $r_{s}=10$~a.u.
As can be seen from \Fig{fig:sofg}(b) and \Fig{fig:sofg}(e), this agreement is due to
error cancellation between two effects. On the one hand, (T) overestimates long range correlation effects.
On the other hand, higher-order cluster operators are missing in CCD(T), which underestimates correlation
effects at lower densities.
This error cancellation fails for larger electron numbers.
CCD(cT) averts the infrared catastrophe and obtains accurate correlation energy results compared to
DMC for all densities up to $r_{s}=5$~a.u. Only at lower densities CCD(cT)
starts to deviate significantly from DMC due to the neglect of higher-order cluster operators.

 \begin{table}
\centering
       \caption{CBS limit correlation energies per electron of the UEG in mHa. r\textsubscript{s} is given in atomic units. }
       \resizebox{\columnwidth}{!}{
            \begin{tabular}{ c c c c c c c c c c}
            \hline\hline
                      & \multicolumn{4}{c}{14 electrons} &                     &  \multicolumn{4}{c}{54 electrons} \\
                        \cline{2-5}                                             \cline{7-10} 
             Method & $r\textsubscript{s}=1$ & $r\textsubscript{s}=2$ & $r\textsubscript{s}=3$ & $r\textsubscript{s}=5$  &   & $r\textsubscript{s}=1$ & $r\textsubscript{s}=2$ & $r\textsubscript{s}=5$ & $r\textsubscript{s}=10$ \\

            \hline
%             MP2                 & -35.7 & -28.5 & -23.9 & -18.3    &              & -39.5 & -32.0 & -20.8 & -13.4 \\ 
%             rCCD                & -29.0 & -21.5 & -17.4 & -12.8    &              & -31.9 & -23.9 & -14.2 & -8.8 \\
             CCD                 & -36.7 & -29.2 & -24.2 & -18.1    &              & -38.4 & -30.2 & -18.5 & -11.3 \\
             CCD(T)              & -37.9 & -31.5 & -27.1 & -21.4    &              & -39.9 & -33.1 & -22.6 & -15.0 \\
             CCD(cT)             & -37.8 & -31.3 & -26.9 & -21.1    &              & -39.8 & -32.8 & -22.1 & -14.5 \\
            \cline{1-10}
            CCDT\footnote{CCDT data are from the work of Neufeld and Thom~\cite{Neufeld2017}.}  & -37.9 & -31.5 & -27.0 & -21.2  \\
            i-FCIQMC\footnote{i-FCIQMC data are from the work of Shepherd \emph{et al.}~\cite{shepherd2012}}  & -38.0 & -31.8 & \textemdash & -21.9 \\
            DMC\footnote{These DMC data are from the work of L\'opez R{\'i}os \emph{et al}.~\cite{LopezRios2006}}  &&&&&& -39.0 & -32.6 & -22.8 & -15.6 \\

              \hline\hline
              \end{tabular}
     }
\label{tab:energies}
\end{table}

%BACKUP TABLE!!!!!!!
%
%\begin{table}
%\centering
%       \caption{CBS limit correlation energies per electron of the UEG with 54 electrons in mHa. r\textsubscript{s} is given in atomic units. }
%            \begin{tabular}{ c c c c c}
%            \hline\hline
%             Method & $r\textsubscript{s}=1$ & $r\textsubscript{s}=2$ & $r\textsubscript{s}=5$ & $r\textsubscript{s}=10$ \\
%
%            \hline
%             CCD                          & -38.4 & -30.2 & -18.5 & -11.3 \\
%             CCD(T)                       & -39.9 & -33.1 & -22.6 & -15.0 \\
%             CCD(cT)                      & -39.8 & -32.8 & -22.1 & -14.5 \\
%            \hline
%            DMC\footnote{These DMC data are from the work of L\'opez R{\'i}os \emph{et al}.~\cite{LopezRios2006}}  & -39.0 & -32.6 & -22.8 & -15.6 \\
%
%              \hline\hline
%              \end{tabular}
%\label{tab:energies_BACKUP}
%\end{table}




As an important test, we apply CCSD(cT) theory to a set of molecules.
In our benchmark we use a set of 26 different molecules that 
give access to 23 different closed-shell reaction energies. This selection of
molecules was already used in a previous work~\cite{Irmler2021}.
% and is a subset
%from the testset employed by Knizia \emph{et al.}~\cite{Knizia2009}.
As reference we
use energies from a converged CCSDT calculation. The standard deviation of the
reaction energy for the 23 reactions is 0.9~kJ/mol for both CCSD(T) and
CCSD(cT) (the maximum error is 2.1~kJ/mol and 3.3~kJ/mol for CCSD(T) and
CCSD(cT), respectively). This illustrates that both, CCSD(T) and CCSD(cT),
are very accurate approximations for the full CCSDT energy.

Finally, we present results for the lithium bcc metal.
We find that CCSD(cT) predicts a cohesive energy of $60.1\,$mHa/atom,
which is in excellent agreement with the experimental estimate
corrected for zero-point vibrations of $60.9\,$mHa/atom~\cite{Schimka2011}.
Our estimate includes a HF, CCSD, (cT) and core-valence MP2 contribution
of $20.5$, $30.4$, $8$ and $1.2\,$mHa/atom, respectively. The convergence of
the (cT) correlation energy is depicted in \Fig{fig:sofg}(e).
Numerical details can be found in the SM.

\emph{Summary and conclusions.} --
We have introduced the highly accurate and computationally
efficient CCSD(cT) theory, which paves the way for achieving chemical accuracy
in \emph{ab initio} calculations of real metals.
Although the presented approach was applied to paramagnetic systems only,
it can also be generalized to ferromagnetic systems using an unrestricted formalism.

Several far reaching conclusions must be drawn from the present work.
The RPA in the electron gas
is formally identical to CCSD in the long wavelength limit. This explains why the embedding
of CCSD into the RPA yields rapidly convergent results~\cite{Schaefer2021a, Schaefer2021}.
Furthermore, the contributions of single particle-hole excitation operators
is not taken into account for the UEG, which could also play an important role
in real \emph{ab initio} systems~\cite{Stoehr2021}.

Our work also sheds new light on potential shortcomings of CCSD(T) theory.
This could explain (part of) the discrepancies observed between DMC and CCSD(T) interaction
energies of large molecules~\cite{Al-Hamdani2021}.

\emph{Acknowledgements.} --The authors thankfully acknowledge support and funding from the
European Research Council (ERC) under the European Union’s Horizon 2020 research and innovation
program (Grant Agreement No. 715594). We gratefully
acknowledge many fruitful discussions with Felix Hummel, Alejandro Gallo and James Shepherd. 
The computational results presented have been achieved in part using the Vienna Scientific Cluster (VSC).

%dislodge, undergo
\bibliography{renormalized}

\end{document}

