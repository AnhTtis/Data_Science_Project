\documentclass[aps,prl,twocolumn]{revtex4-1}
\usepackage{hyperref}
\hypersetup{colorlinks=true, citecolor=blue, urlcolor=blue, linkcolor=blue}

\pdfoutput=1
\usepackage[utf8]{inputenc}
\usepackage[T1]{fontenc}
\usepackage{amssymb}
\usepackage{amsmath}
\usepackage{graphicx}
\usepackage{subcaption}
\usepackage{color}
\usepackage{calc}
\usepackage{mathtools}
\usepackage{tabularx}

\newcommand{\Tr}{\operatorname{Tr}}
\renewcommand{\d}{\textrm{d}}
\renewcommand{\vec}[1]{\mathbf{#1}}
\newcommand{\e}{{\mathrm{e}}}
\newcommand{\im}{\mathrm{i}}
\newcommand{\diagramBox}[2][0.5]{\raisebox{0.5ex-#1\height}{#2}}
\newcommand{\diagramBoxBorder}[4][0.5]{
  \raisebox{0.5ex-#1\height}[0.5ex+\height-#1\height+#2][#1\height-0.5ex+#3]{#4}
}
\newcommand{\TODO}[1]{\textbf{<TODO: #1>}}
\newcommand{\AI}[1]{\textcolor{red}{#1}}
\usepackage[normalem]{ulem}
\newcommand{\SE}{Schr\"odinger equation}
\newcommand{\Eq}[1]{Eq.~\eqref{#1}}
\newcommand{\Fig}[1]{Fig.~\ref{#1}}

\begin{document}

\title{
Averting the infrared catastrophe in the gold standard of quantum chemistry
}
\author{Nikolaos Masios}
\author{Andreas Irmler}
\author{Andreas Gr\"uneis}
\email{andreas.grueneis@tuwien.ac.at}
\affiliation{
  Institute for Theoretical Physics, TU Wien,\\
  Wiedner Hauptstraße 8-10/136, 1040 Vienna, Austria
}
\date{\today, PREPRINT}

\keywords{particle-particle ladder diagram;
ring diagrams; coupled cluster;
random phase approximation}


\begin{abstract}
Coupled-cluster theories can be used to compute electronic correlation energies
of \emph{ab initio} systems with systematically improvable accuracy.
However, the widely-used coupled cluster singles and doubles plus perturbative
triples (CCSD(T)) method is only applicable to insulating systems.
For zero-gap materials the truncation of the underlying many-body perturbation expansion
leads to an infrared catastrophe.
Here, we present a novel
perturbative triples formalism that yields convergent correlation energies
in metallic systems. Furthermore, the computed correlation energies for the
three dimensional uniform electron gas at metallic densities are in good agreement with
quantum Monte Carlo results. At the same time the newly proposed method
retains all desirable properties of CCSD(T) such as its accuracy for insulating systems
as well as its low computational cost compared to a full inclusion of the triples.
This paves the way for \emph{ab initio} calculations of real metals
with chemical accuracy.
\end{abstract}

\maketitle

%\section{Introduction}

\emph{Introduction.} -- \emph{Ab initio} methods that achieve
systematically improvable accuracy
for metallic systems are urgently needed to understand chemical reactions on metal surfaces
or to compute the thermodynamic stability of materials.
Currently available exchange and correlation energy density functionals
often fail to achieve the desired
level of accuracy compared to experiment. A prominent failure includes the
incorrect prediction of molecular adsorption sites
on metal surfaces~\cite{Nie2010}.  %ref feibelman, alesandro, kresse, etc.
As an alternative, more accurate Quantum Monte Carlo (QMC) calculations can be applied to
metals~\cite{Pozzo2008,Doblhoff2017}. 
However, even diffusion QMC calculations exhibit a strong dependence on the
fixed node approximation~\cite{Nemec2010}.
Compared to QMC and DFT, many-electron perturbation theories offer a conceptually different approach to
solve the many-electron problem with high accuracy.  In
particular, coupled-cluster (CC) theory offers a systematically improvable ansatz
for the many-electron wave function employing a series of
higher orders of particle-hole excitation operators that converge rapidly
to the exact result.
At the truncation level of  single, double and
perturbative triple particle–hole excitation operators, CCSD(T) theory predicts
atomization and reaction energies for a wide class of molecules with an
accuracy of approximately 1~kcal/mol~\cite{bartlett2007}. As
such, CCSD(T) is often referred to as the `gold standard' of quantum chemistry.
Coupled-cluster theories are very successful in molecular quantum
chemistry and also become more widely-used to study
solids~\cite{Booth2013,McClain2017,Yang640,grueneis2015b,Gruber2018},
where they can achieve highly accurate predictions of, for example,
pressure-temperature phase diagrams~\cite{gruber18b}.

But metallic systems still constitute a major challenge for currently available CC theories.
Although CCSD can be applied to metals, recently obtained results for a number
of metallic systems indicate that CCSD falls short of achieving chemical accuracy in metals~\cite{Mihm2021,Neufeld2022},
which is expected and agrees with findings for molecules and insulating solids.
However, the inclusion of the perturbative triples correction (T)
is not possible due to the infrared catastrophe
caused by the truncation of the many-electron perturbation expansion~\cite{Shepherd2013}.
The infrared catastrophe leads to a divergence of CCSD(T) correlation energies for metals
in the thermodynamic limit of the employed simulation cells.
A full non-perturbative inclusion of the triple excitation operator in the CC method
is convergent but computationally too expensive and can only been applied to few relatively small systems~\cite{Neufeld2017}.
Here, we present a modification to
the perturbative triples theory that is applicable to metallic systems and
retains all desirable properties including its accuracy for insulating systems
and low computational cost compared to a full inclusion of the triples.
Our findings provide new insight into the cause of the infrared catastrophe and how it can be averted
in higher-order perturbation theories.

%aG: lack of space. cant include structure of this work
%We first discuss the divergence of the correlation energy in second-order perturbation theory
%also referred to as infrared catastrophe.
%Next we turn to the random-phase-approximation (RPA),
%which performs a resummation of all `ring' diagrams to infinite order, leading to a
%convergent correlation energy.
%We use analytic as well as numeric results for the uniform electron gas model
%to carefully analyse the behavior of these perturbation theories.
%Finally we turn to the more complex CC methods and introduce a
%highly efficient modification to the perturbative triples correlation energy
%expression that yields convergent correlation energies for metals.

\emph{Theory} --
To better understand the infrared catastrophe of perturbation theories denoted as $(X)$,
we  make use of the following three quantities:
the correlation energy $E_\text{c}^{(X)}$, the electronic transition structure factor $S^{(X)}(\mathbf{q})$
and the quantity $T^{(X)}_i(\mathbf{q})$.
The correlation energy is defined as
%
\begin{equation}
\label{eq:ccdCorr}
        E_\text{c}^{(X)} = \sum_{\mathbf{q}}\upsilon(\mathbf{q})\underbrace{ \Biggl[\, \sum_{ijab}
               \frac{\delta\upsilon_{ij}^{ab}}{\delta\upsilon( \mathbf{q} )}
               \left(2 t_{ij}^{ab} - t_{ji}^{ab} \right)_{(X)} \Biggr] }_{\coloneqq S^{(X)}(\mathbf{q})},
\end{equation}
%
%where the Coulomb integrals are given by
%
%\begin{equation}
%\label{eq:coulomb}
%    \upsilon_{ij}^{ab} = \frac{4 \pi}{\Omega \left| \mathbf{k}_i-\mathbf{k}_a \right|^2}
%                         \delta_{\mathbf{k}_i-\mathbf{k}_a, \mathbf{k}_b - \mathbf{k}_j}.
%\end{equation}
%
The indices $i$, $j$ and $a$, $b$ label occupied and virtual spatial
orbitals, respectively.
At first we focus on second-order perturbation theory, %ref
ring CC doubles (rCCD) theory, which is closely related to the random-phase approximation (RPA) %ref
and CC doubles theory (CCD). %ref
We note that, due to the symmetry of the UEG Hamiltonian, single excitations
are absent.
Furthermore, in the UEG, the one-electron orbitals are plane waves with wave vectors $\vec
k_i$, $\vec k_j$ and $\vec k_a$, $\vec k_b$.  This allows to write the
two-electron repulsion integral as
%
\begin{math}\upsilon_{ij}^{ab} = \upsilon(\mathbf{q})
               = \frac{4 \pi}{\Omega \left| \mathbf{k}_i-\mathbf{k}_a \right|^2}
\delta_{\mathbf{k}_i-\mathbf{k}_a, \mathbf{k}_b - \mathbf{k}_j}\end{math},
%
with the momentum transfer vector $\mathbf{q} = \mathbf{k}_i - \mathbf{k}_a$
and $\Omega$ being the volume of the simulation cell. 
The functional derivative
$  \frac{\delta \upsilon_{ij}^{ab}}{\delta \upsilon(\mathbf{q})} =\delta_{\mathbf{q}, \mathbf{k}_b - \mathbf{k}_j} \delta_{\mathbf{q}, \mathbf{k}_i - \mathbf{k}_a} $
enables a concise notation.
%
%\begin{equation}
%  \frac{\delta \upsilon_{ij}^{ab}}{\delta \upsilon(\mathbf{q})} =
%  \delta_{\mathbf{q}, \mathbf{k}_b - \mathbf{k}_j}
%  \delta_{\mathbf{q}, \mathbf{k}_i - \mathbf{k}_a}.
%\end{equation}
%
The amplitudes $t_{ij}^{ab}$ are obtained by solving the amplitude equations of the employed many-electron
perturbation theory.

\Eq{eq:ccdCorr} introduces the electronic transition structure factor $S(\mathbf{q})$,
which gives access to the dependence of the
correlation energy on the interelectronic interaction distance.
An additional quantity of significance for the present work is given by
\begin{equation}
T^{(X)}_i(\mathbf{q})=
\Biggl[\, \sum_{jab}
               \frac{\delta\upsilon_{ij}^{ab}}{\delta\upsilon( \mathbf{q} )}
               {t_{ij}^{ab}}_{(X)} \Biggr].
\end{equation}
Similar to the structure factor, $T^{(X)}_i(\mathbf{q})$ depends on the momentum transfer vector $\mathbf{q}$.

\emph{The infrared catastrophe} --
Having introduced the most important quantities needed for our analysis,
we now turn to the case of second-order perturbation theory, which
is a textbook example for the infrared catastrophe. In particular,
we focus on second-order M\o ller-Plesset perturbation theory (MP2) theory, which employs the Hartree--Fock (HF) Hamiltonian
as the unperturbed reference system.

The MP2 energy is given by Eq.~(\ref{eq:ccdCorr}) and using $t_{ij}^{ab} =
\upsilon_{ij}^{ab} / \Delta_{ab}^{ij}$, where
$\Delta^{ij}_{ab}=\varepsilon_i+\varepsilon_j-\varepsilon_a-\varepsilon_b$.
The correlation energy for the UEG diverges
due to summation over
elements in the amplitudes with both the occupied orbital $i$ and the
virtual orbital $a$ close to the Fermi surface.
The rate of divergence is $\log(q)$~\cite{mattuck} and $\log(\log(q))$~\cite{Harris}
for Hartree or Hartree--Fock orbital
energies $\varepsilon_p$, respectively.
%It is important to note that this divergence occurs using both Hartree or Hartree--Fock orbital
%energies $\varepsilon_p$.
%However, the presence of the Fock exchange contribution ``softens'' the divergence of the second-order
%correlation energy from $\log(q)$~\cite{mattuck} to $\log(\log(q))$~\cite{Harris}.
In the main article we will only employ Hartree--Fock orbital
energies $\varepsilon_p$. A numerical comparison for the two different reference systems
can be found in the Supplemental Material. 
In agreement with analytic results, our numerical findings exhibit the
infrared catastrophe, as indicated by a singularity in $S(\mathbf{q})$ at $|\mathbf{q}|=0$
shown in \Fig{fig:sofg}(a).
The numerical procedure follows the description in
Ref.~\cite{Shepherd2014}. We use twist-averaging in all our results which
helps to reduce the fluctuations due to discretization errors of the finite
simulation cell~\cite{Gruber2018,Drummond2008,Lin2001}.
%This confirms
%the well-known fact that the infrared catastrophe originates from overestimated
%long-range correlation effects.

\begin{figure*}
  \centering
%\includegraphics[width=\linewidth]{fig1_test.png}
  \includegraphics[width=0.245\linewidth]{StructureFactors_rccd_decomposition_rs20.png}
  \hfill
  \includegraphics[width=0.245\linewidth]{StructureFactors_ccd_drccd_rs20.png}
  \hfill
  \includegraphics[width=0.245\linewidth]{StructureFactors_triples_rs20.png}
  \hfill
  \includegraphics[width=0.242\linewidth]{T_ccd_drccd_rs20.png}
  \caption {Results from uniform electron gas calculations for $N=246$ electrons with 2178 spatial orbitals for 
           a density of $r_{s}=20$~a.u.. For the rCCD 266 twists were performed, while for CCD,
           (T) and (cT) 144 twists. (a) Decomposed transition structure factors for MP2, linear and quadratic terms in rCCD;
           (b) CCD, rCCD and their difference; (c) (T) and (cT). (d) shows the minimum value for all $i$ of
            $T_i(\mathbf{q})$, on the CCD and rCCD level of theory. Further results with different values of
            $r_s$ can be found in the Supplemental Material.}
            \label{fig:sofg}
\end{figure*}


\emph{The ring summation} --
As already demonstrated by Macke in 1950~\cite{macke_uber_1950},
the divergence in second-order perturbation theory can be averted
by including carefully selected higher-order contributions of the many-electron
perturbation expansion, corresponding to ring diagrams.
%that lift the divergence of $S(\mathbf{q})$ at $|\mathbf{q}|=0$.
Algebraically this can be implemented by solving the ring-coupled-cluster amplitude equation given by
\begin{equation}
\begin{split}
\label{eq:t-ring}
t_{ij}^{ab} &= \left( \upsilon_{ij}^{ab}
             + 2 \sum_{kc} \upsilon_{ic}^{ak} t_{kj}^{cb}
             + 2 \sum_{kc} t_{ik}^{ac} \upsilon_{cj}^{kb} \right. \\
            &+ \left. 4 \sum_{klcd} t_{il}^{ad} \upsilon_{dc}^{lk} t_{kj}^{cb}
              \right) / \Delta_{ij}^{ab},
\end{split}
\end{equation}
which formally agrees with solving the Cassida equations~\cite{scuseria_2008}.
For the ring-coupled-cluster correlation energy to converge in the long wavelength limit,
the first term on the right-hand side of Eq.(\ref{eq:t-ring}),
which is equivalent to MP2 theory, needs to be partly cancelled by the additional
linear ring (lr) and quadratic ring (qr) terms in the amplitudes.
Note that for small values of $\mathbf{q}$ the sign of the MP2 and quadratic
terms is negative, whereas the linear term is positive. This implies that,
although all three terms diverge individually with the same power law as MP2,
there exists a non-trivial cancellation in the long wavelength limit.
We complement the above discussion
by numerically studying the rCCD transition
structure factor and its individual diagrammatic contributions~\cite{Irmler2019}
given by
%\AI{Note: we have to introduce ring and especially qr, unfortunately}
%
\begin{equation}
\label{eq:srccd}
S^\text{rCCD}(\mathbf{q}) = S^\text{MP2}(\mathbf{q})
       + S^\text{lr}(\mathbf{q}) + S^\text{qr}(\mathbf{q}) \text.
\end{equation}
%
The calculated contributions to $S(\mathbf{q})$ are depicted in
Fig.~\ref{fig:sofg}(a).  Although the individual contributions
diverge, the total rCCD transition structure factor converges towards zero in the
limit of $\mathbf{q} \rightarrow 0$ (see Fig.~\ref{fig:sofg}(b)).
We also stress that the leading order behaviour of
$S^\text{rCCD}(\mathbf{q})$ in the limit $|\mathbf{q}| \rightarrow 0$
is $S^\text{rCCD}(\mathbf{q})\propto |\mathbf{q}|$. This was shown by
Bishop and L\"{u}hrmann~\cite{bishop_1982} (see Ref.~\cite{Mihm2023} for a detailed discussion).
%This is not so good right now as it implies that Bishop is already discussing the 
%structure factor, which is not the case at it is explained by shepherd.
%as well as Freeman (A.I. really? where? not in the 1977 paper)
%aG: at least there is a figure that clearly shows its linear
For simulation cells with finite electron numbers this leads to a finite size error scaling
of $N^{-2/3}$, where $N$ is the number of electrons. %ref Spencer?
In passing, we note that insulating systems have a different leading order behaviour
around $\mathbf{q} \rightarrow 0$ given by
$S^\text{rCCD}(\mathbf{q})\propto |\mathbf{q}|^2$, implying a finite
size error scaling of $N^{-1}$~\cite{Liao2016,Martin2016}.

We now address the
question, how the terms on the right-hand-side in Eq.~(\ref{eq:srccd}) cancel each other in the limit $|\mathbf{q}| \rightarrow 0$?
To this end we re-write Eq.~(\ref{eq:t-ring}) in the following way
\begin{equation}
\begin{split}
\label{eq:t-ring-2nd}
%t_{ij}^{ab} &= \left( \upsilon_{ij}^{ab}
t^\text{rCCD}_{ij}(\mathbf{q}) = \upsilon(\mathbf{q})
  &\left(1 + 2\sum_k t^\text{rCCD}_{kj}(\mathbf{q}) + 2 \sum_k t^\text{rCCD}_{ik}(\mathbf{q}) \right. \\
          &+ \left. 4 \sum_{kl} t^\text{rCCD}_{il}(\mathbf{q}) t^\text{rCCD}_{kj}(\mathbf{q}) \right) / \Delta_{ij}^{ab} \textrm{,}
\end{split}
\end{equation}
which is possible because ring diagrams do not couple different $\mathbf{q}$.
We point to the important analytic finding by Freeman in Ref.~\cite{freeman_coupled-cluster_1977}, who demonstrated
that the solution of the above equation in the long wave limit leads to $\lim_{\mathbf{q} \to 0} T^\text{rCCD}_i(\mathbf{q}) = -1/2$. 
We stress that these findings are identical for Hartree and
Hartree--Fock orbital energies.
Consequently, we arrive at the first important insight of this work.
The singularity at $|\mathbf{q}|=0$ in the MP2 term in Eq.(\ref{eq:t-ring}) is
cancelled by one half of the linear ring terms, whereas the singularity of
the quadratic ring term is cancelled by the `other' half of the linear term.
\Fig{fig:sofg}(d) depicts our numerical results for $T^\text{rCCD}_i(\mathbf{q})$,
which converges to the analytically known value of $-1/2$ in the long wavelength limit.
Unfortunately, however, the convergence is relatively slow  and requires
excruciatingly large systems containing more than 400 electrons.


\emph{The coupled-cluster doubles method} --
Having established a firm framework to understand and analyse
the divergence and convergence of correlation energies in MP2 and rCCD theories,
we now turn to the significantly more complex CCD theory.
In addition to the ring diagrams included in rCCD,
CCD theory also includes a number of ladder and exchange diagrams, which makes an exact
analytic solution impossible even for the simple uniform electron gas model.
%We note, however, that several approximate attempts have been made in the
%past.

Here, we use a numerical approach to investigate the behavior
of $T^\text{CCD}_i(\mathbf{q})$ and $S^\text{CCD}(\mathbf{q})$ in the long wavelength limit.
\Fig{fig:sofg}(b) depicts that $S^\text{CCD}(\mathbf{q})$
and $S^\text{rCCD}(\mathbf{q})$ differ markedly in the region around the minimum.
However, in the limit of  $\mathbf{q} \to 0$, both structure factors approach another
such that their difference
%as emphasized by the depicted difference between $S^\text{rCCD}(\mathbf{q})$
%and $S^\text{CCD}(\mathbf{q})$, which 
decays to zero rapidly.
Furthermore, \Fig{fig:sofg}(d) shows a similar agreement between $T^\text{CCD}_i(\mathbf{q})$
and $T^\text{rCCD}_i(\mathbf{q})$, which leads us to one central finding of the
present work:
$\lim_{|\mathbf{q}| \to 0} T^\text{CCD}_i(\mathbf{q}) = \lim_{|\mathbf{q}| \to 0} T^\text{rCCD}_i(\mathbf{q}) = -1/2 $.
This can be understood by the fact that the ring diagrams are the most divergent terms within 
each order of perturbation theory. A careful inspection of the full CCD amplitude equations
suggests that an identical cancellation between the terms present in rCCD is required to converge the
CCD amplitude equations.

\emph{Triple particle-hole excitation operators} --
We now turn to CC theories that approximate the triple particle-hole
excitation operator in a perturbative manner.
In these cases the post-CCSD correlation energy and transition structure factor
contributions are given by
\begin{equation}
\label{eq:corr_triples}
        E_\text{c}^{(J)} = \sum_{\mathbf{q}}\upsilon(\mathbf{q})\underbrace{ \Biggl[\, \sum_{\substack{{ijk} \\
                           {abc}}}  \frac{\delta W_{ijk}^{abc}}{\delta\upsilon( \mathbf{q} )}
                           \left( A_{abc}^{ijk} \right)_{(J)} \Biggr] }_{\coloneqq S^{(J)}(\mathbf{q})}.
\end{equation}
($J$) refers on the employed approximation.
In the case of  (T), $A_{abc}^{ijk}$ corresponds to $\bar{W}_{abc}^{ijk}/\Delta_{abc}^{ijk}$ with
$\bar{X}^{ijk}_{abc}=\frac{4}{3}{X}^{ijk}_{abc} -2{X}^{ijk}_{acb}  +\frac{4}{3}{X}^{ijk}_{bca}  $.
$\Delta_{abc}^{ijk}$ refers to the difference in HF orbital energies for the occupied and unoccupied
states labelled by the respective indices.
$W^{abc}_{ijk}$ is defined as
\begin{equation}
\label{eq:triples_residuum}
               W_{ijk}^{abc} = P_{ijk}^{abc} \left( \sum_e t_{ij}^{ae}\upsilon_{ek}^{bc} - \sum_m t_{im}^{ab}\upsilon_{jk}^{mc} \right).
\end{equation}
We employ the following functional derivative for a concise notation used to define the structure factor.
\begin{equation}
\label{eq:triples_bare}
            \frac{\delta W_{ijk}^{abc}}{\delta\upsilon( \mathbf{q} )} = P_{ijk}^{abc}
                           \left( \sum_e t_{ij}^{ae}\delta_{\mathbf{q},\mathbf{k}_k-\mathbf{k}_c}
                          -\sum_m t_{im}^{ab}\delta_{\mathbf{q},\mathbf{k}_k-\mathbf{k}_c} \right).
\end{equation}
The permutation operator $P_{ijk}^{abc}$ is defined as $P_{ijk}^{abc}X_{ijk}^{abc}= 
X_{ijk}^{abc} + X_{ikj}^{acb} + X_{kij}^{cab} + X_{kji}^{cba} + X_{jki}^{bca} + X_{jik}^{bac}$.
%\AI{Remove?: Next, we briefly recapitulate the infrared catastrophe in (T) theory, which was already discussed
%in Ref.~\cite{Shepherd2013}.}
Next, we employ the transition structure factor to briefly recapitulate the infrared catastrophe in (T) theory~\cite{Shepherd2013}.
\Fig{fig:sofg}(c) depicts $S^\text{(T)}(\mathbf{q})$, which shares many
similarities with $S^\text{MP2}(\mathbf{q})$.
Both methods exhibit a divergence for ${|\mathbf{q}| \to 0}$
and yield correlation energies per electron that diverge as the system size increases.
The infrared catastrophe of (T) is
caused by the unscreened Coulomb interactions included in Eq.~(\ref{eq:triples_residuum}).
%\AI{Remove? : 
%In Ref.~\cite{Shepherd2013}, it was shown that the correlation energy divergence 
%can be averted by replacing the unscreened Coulomb interaction with an approximate
%parameter-dependent screened interaction.

Here, we introduce a novel correlation energy expression
that yields convergent energies for the uniform electron gas at the level of triple particle-hole
excitation operators and is exact to fourth-order perturbation theory without
depending on any ad-hoc parameters.
We refer to the method as (cT) because it includes the \emph{complete} set of terms 
%\AI{\sout{that couple the double
%with the triple amplitudes in a non-iterative manner.} 
present in the triples amplitude equations in a non-iterative manner.
Naturally, the coupling of triples amplitudes with each other is disregarded.
%\AI{\sout{We stress, however, that the coupling of the triples amplitudes with each other and higher-order amplitudes is disregarded.}}
Thus, in (cT) we use the following approximation to $A_{abc}^{ijk}=\bar{M}_{abc}^{ijk}/\Delta_{abc}^{ijk}$,
where
\begin{equation}
\label{eq:M_piecuch}
               M_{abc}^{ijk} = P_{ijk}^{abc} \left(\sum_e t_{ij}^{ae}\upsilon_{ek}^{bc} + \sum_e 2t_{ij}^{ae} \sum_{fm} \upsilon_{ef}^{bm} t_{mk}^{fc} + \ldots \right).
\end{equation}
For brevity we show only a selection of representative terms that exhibit a divergent behaviour for
${\mathbf{q} \to 0}$. The complete expression of $M_{abc}^{ijk}$ including the singles contributions can be found in
Ref.~\cite{Piecuch2002}. For the terms in the parenthesis in Eq.~(\ref{eq:M_piecuch}), it
follows for ${\mathbf{q}' \to 0}$ that
$\upsilon_{ek}^{bc} + 2 \sum_{fm} \upsilon_{ef}^{bm} t_{mk}^{fc}=\upsilon_{ek}^{bc}
(1-2T_m(\mathbf{q}') )=0$, with $\mathbf{q}' = \mathbf{k}_k - \mathbf{k}_c$,
which is formally identical to our results for the cancellation of the
divergence between the linear ring and the MP2 terms in the long wavelength
limit.
%save space by not including the sentence below
%Although only the terms written in Eq.~(\ref{eq:M_piecuch}) are required to lift
%the divergence, we prefered to include all terms appearing in the full triples amplitude
%equations. This guarantees the correct treatment of antisymmetry of the wavefuntion.
This shows that the (cT) correlation energy expression averts the infrared catastrophe of the
(T) approximation for the UEG without requiring an iterative solution of the triple amplitudes
as recently proposed in Ref.~\cite{Neufeld2023}.
Furthermore the (cT) expression is computationally not more expensive than
(T) and requires no additional quantities. From this we conclude that (cT) is
currently the most efficient \emph{ab initio} post-CCSD correlation energy correction that can be applied
to a wide range of systems including insulators, semiconductors and metals.
%The full set of equations can be found in Ref.~\cite{shavitt_2009,Noga1987,Noga1988}
%The complete expression of $M_{abc}^{ijk}$ including the singles contributions
%can be found in Ref.~\cite{Piecuch2002}.
\\
As an additional confirmation and to demonstrate that (cT) yields indeed convergent correlation 
energies per electron, \Fig{fig:sofg}(c) shows the numerical findings for $S^\text{(cT)}(\mathbf{q})$.
In contrast to $S^\text{(T)}(\mathbf{q})$, $S^\text{(cT)}(\mathbf{q})$ does not exhibit
a divergence for $\lim_{\mathbf{q} \to 0}$. Instead $S^\text{(cT)}(\mathbf{q})$
shows a minimum and converges to zero in the long wavelength limit, which is
very similar to rCCD and CCD.

\emph{Results.} -- We now turn to the discussion of numerical results for correlation energies
obtained for electron gas simulation cells with various electron numbers at different levels of theory.
%These results serve to demonstrate the convergence or divergence of
%different methods for the UEG. Furthermore we assess the accuracy of the presented
%methods by comparing to currently available
%benchmark results obtained using Diffusion Monte Carlo (DMC) and Full Configuration Interaction
%Quantum Monte Carlo (FCIQMC).
\Fig{fig:UegEnergies} displays the behavior of the correlation energy per electron
as a function of electron number $N$ for MP2, rCCD, CCD, (T) and (cT).
%The depicted energies exhibit a very smooth convergence to the thermodynamic limit,
%which is made possible by the twist-averaging method.
%In this manner shell-filling and finite-size error effects are significantly ameliorated.
The employed system sizes vary from $N=14$ electrons to $342$ electrons, using finite-$M$
basis sets and extrapolating to the CBS limit by an $M^{-1}$ power-law.
%The systems are parameterized by a Wigner-Seitz radius of $r_{s}=3\alpha_{0}$, in atomic units (a.u.) and a simple cubic
%simulation unit cell in reciprocal space was used. 
The infrared catastrophe in MP2 theory becomes visible
on approach to the TDL for \Fig{fig:UegEnergies}(a).
In contrast, rCCD and CCD correlation energies deviate markedly from the MP2 counterpart,
reaching a plateau as expected for convergent theories.
In an analogous way, \Fig{fig:UegEnergies}(b)
verifies that the (T) correlation energy contribution, for the reasons we have already discussed,
also diverges as we move to the TDL, while its counterpart, (cT), exhibits a behaviour
that closely resembles that of rCCD and CCD theories.

\begin{figure}
\centering
%\includegraphics[width=\columnwidth]{fig2_test.png}
\includegraphics[width=0.49\columnwidth]{Energies_twisted.png}
\includegraphics[width=0.49\columnwidth]{Triples_twisted.png}
        \caption{CBS correlation energies per electron retrieved as a function of electron number N.
                 %extrapolated values to the complete basis set limit (CBS) by a $M^{-1}$ power-law. Each point 
                 %was obtained for a pair of N and M, with their values varying between $N=14-342$ and $M=454-2430$,
                 %correspondingly. The closed-shell configurations of electrons derive from a simple cubic reciprocal-
                 %space lattice.
                 The systems were parametrized by a Wigner-Seitz radius of $r_{s}=3$
                 and 40 twists were performed for each system. (a) MP2, rCCD and CCD energies. (b) (T) and 
                 (cT) energies. All units in (a.u.). 
}
\label{fig:UegEnergies}
\end{figure}

Table~\ref{tab:energies} summarizes correlation energies obtained from MP2, rCCD, CCD, CCD(T) and 
CCD(cT) methods compared to i-FCIQMC, DMC and CCDT (CCD with full triple excitations) results.
The energies were extrapolated to the CBS limit, and the systems were parametrized by a range of
Wigner-Seitz radii ($r_{s}=1, 2, 3, 5, 10$ a.u.) and $N=14, 54$ electrons.
We first discuss the results obtained for the 14 electron system.
In this case i-FCIQMC can be viewed as an exact reference, whereas CCDT serves as reference
for any approximate triples theory.
CCDT results from Ref.~\cite{Neufeld2017} are in good agreement with i-FCIQMC in the high density limit and
differ at low densities. Higher levels of CC theory are needed to capture all
important correlation effects for increasing $r_{s}$~\cite{Neufeld2017}.
This can also be confirmed by comparing  CCDT to CCD.
To test the role of higher-order ring contributions, we compare MP2 to rCCD.
Our findings show that the ring contributions are significant even for 14 electrons.
Furthermore, we note that the difference between rCCD and CCD is larger than the difference
between MP2 and CCD.
CCD(T) and CCD(cT) correlation estimates are in good agreement with CCDT.
We note that the agreement between CCD(T) and CCDT is
fortuitous and only valid for calculations with a small number of electrons,
as can be seen from the divergence of CCD(T) as a function of $N$ in
\Fig{fig:UegEnergies}(b).

We now discuss the results obtained for the 54 electron system summarized in Table~\ref{tab:energies}.
Here we compare to DMC reference results.
The trends observed for MP2, rCCD and CCD are qualitatively similar to those for $N=14$.
However, in this case CCD(T) is fortuitously close to DMC
even at relatively low densities corresponding to $r_{s}=10$~a.u..
As can be seen from \Fig{fig:UegEnergies}(b) and \Fig{fig:sofg}(c), this agreement is due to
error cancellation between two effects. On the one hand, (T) overestimates long range correlation effects.
On the other hand, higher-order cluster operators are missing in CCD(T), which underestimates correlation
effects at lower densities.
We expect that this error cancellation fails for larger electron numbers.
CCD(cT) averts the infrared catastrophe and obtains accurate correlation energy results compared to
DMC for all values of $r_{s}$ up to $r_{s}=5$~a.u.. Only at lower densities CCD(cT)
starts to deviate significantly from DMC due to the neglect of higher-order cluster operators.

 \begin{table}
\centering
       \caption{CBS limit correlation energies per electron of the UEG in mHa. r\textsubscript{s} is given in atomic units. }
       \resizebox{\columnwidth}{!}{
            \begin{tabular}{ c c c c c c c c c c}
            \hline\hline
                      & \multicolumn{4}{c}{14 electrons} &                     &  \multicolumn{4}{c}{54 electrons} \\
                        \cline{2-5}                                             \cline{7-10} 
             Method & $r\textsubscript{s}=1$ & $r\textsubscript{s}=2$ & $r\textsubscript{s}=3$ & $r\textsubscript{s}=5$  &   & $r\textsubscript{s}=1$ & $r\textsubscript{s}=2$ & $r\textsubscript{s}=5$ & $r\textsubscript{s}=10$ \\

            \hline
             MP2                 & -35.7 & -28.5 & -23.9 & -18.3    &              & -39.5 & -32.0 & -20.8 & -13.4 \\ 
             rCCD                & -29.0 & -21.5 & -17.4 & -12.8    &              & -31.9 & -23.9 & -14.2 & -8.8 \\
             CCD                 & -36.7 & -29.2 & -24.2 & -18.1    &              & -38.4 & -30.2 & -18.5 & -11.3 \\
             CCD(T)              & -37.9 & -31.5 & -27.1 & -21.4    &              & -39.9 & -33.1 & -22.6 & -15.0 \\
             CCD(cT)             & -37.8 & -31.3 & -26.9 & -21.1    &              & -39.8 & -32.8 & -22.1 & -14.5 \\
            \cline{1-10}
            CCDT\footnote{CCDT data are from the work of Neufeld and Thom~\cite{Neufeld2017}.}  & -37.9 & -31.5 & -27.0 & -21.2  \\
            i-FCIQMC\footnote{i-FCIQMC data are from the work of Shepherd \emph{et al.}~\cite{shepherd2012}}  & -38.0 & -31.8 & \textemdash & -21.9 \\
            DMC\footnote{These DMC data are from the work of L\'opez R{\'i}os \emph{et al}.~\cite{LopezRios2006}}  &&&&&& -39.0 & -32.6 & -22.8 & -15.6 \\

              \hline\hline
              \end{tabular}
     }
\label{tab:energies}
\end{table}

%\begin{table}
%\centering
%       \caption{CBS limit correlation energies per electron of the 14 and 54 electron uniform electron gas
%                in Hartree. }
%       \resizebox{\columnwidth}{!}{
%            \begin{tabular}{ c c c c c c }
%            \hline\hline
%                    $N$ & Methods & $r\textsubscript{s}=1\alpha_{0}$ & $r\textsubscript{s}=2\alpha_{0}$ & $r\textsubscript{s}=3\alpha_{0}$ & $r\textsubscript{s}=5\alpha_{0}$ \\
%
%            \hline
%                                              & MP2                 & -0.0357 & -0.0285 & -0.0239 & -0.0183 \\
%                                              & rCCD                & -0.0290 & -0.0215 & -0.0174 & -0.0128 \\
%                                              & CCD                 & -0.0367 & -0.0292 & -0.0242 & -0.0181 \\
%            \raisebox{0.0ex}{$14$ electrons}  & CCD(T)              & -0.0379 & -0.0315 & -0.0271 & -0.0214 \\
%                                              & CCD(cT)             & -0.0378 & -0.0313 & -0.0269 & -0.0211 \\
%            \cline{2-6}
%                                              &CCDT\footnote{CCDT data are from the work of Neufeld and Thom~\cite{Neufeld2017}.}                                                                                                                                               & -0.0379 & -0.0315 & -0.0270 & -0.0212  \\
%                                              &i-FCIQMC\footnote{\emph{i}FCIQMC data are from the work of Shepherd \emph{et al.}~\cite{shepherd2012}}                                                                                                                             & -0.0380 & -0.0318 & \textemdash & -0.0219 \\
%
%            \hline
%
%                     &  & $r\textsubscript{s}=1\alpha_{0}$ & $r\textsubscript{s}=2\alpha_{0}$ & $r\textsubscript{s}=5\alpha_{0}$ & $r\textsubscript{s}=10\alpha_{0}$ \\
%            \hline
%
%                                              & MP2                 & -0.0395 & -0.0320 & -0.0208 & -0.0134 \\
%                                              & rCCD                & -0.0319 & -0.0239 & -0.0142 & -0.0088 \\
%                                              & CCD                 & -0.0384 & -0.0302 & -0.0185 & -0.0113 \\[-1.25ex]
%           \raisebox{1.5ex}{$54$ electrons}   & CCD(T)              & -0.0399 & -0.0331 & -0.0226 & -0.0150 \\
%                                              & CCD(cT)             & -0.0398 & -0.0328 & -0.0221 & -0.0145 \\
%            \cline{2-6}
%                                              & DMC\footnote{These DMC data are from the work of L\'opez R{\'i}os \emph{et al}.~\cite{LopezRios2006}}                                                                                                                                            & -0.0390 & -0.0326 & -0.0228 & -0.0156 \\
%
%              \hline\hline
%              \end{tabular}
%     }
%\label{tab:energies}
%\end{table}
%\emph{Molecules} -- 
As a final important test, we apply CCSD(cT) theory to a set of molecules.
In our benchmark we use a set of 26 different molecules that 
give access to 23 different closed-shell reaction energies. This selection of
molecules was already used in a previous work~\cite{Irmler2021} and is a subset
from the testset employed by Knizia \emph{et al.}~\cite{Knizia2009}. As reference we
use energies from a converged CCSDT calculation. The standard deviation of the
reaction energy for the 23 reactions is 0.9~kJ/mol for both CCSD(T) and
CCSD(cT) (the maximum error is 2.1~kJ/mol and 3.3~kJ/mol for CCSD(T) and
CCSD(cT), respectively). This illustrates that both, CCSD(T) and CCSD(cT),
are very accurate approximations for the full CCSDT energy.



\emph{Summary and conclusions.} --
We have introduced the highly accurate and computationally
efficient CCSD(cT) theory, which paves the way for achieving chemical accuracy
in \emph{ab initio} calculations of real metallic systems.

%Our analysis of the infrared catastrophe for a number of perturbation theories
%was based on analytical and numerical findings obtained for one of
%the most fundamental model systems in condensed matter physics, the UEG.
%Initially, we verified the divergence of second-order perturbation
%theory, as well as the convergence of rCCD and CCD
%with respect to the system size $N$, by studying
%structure factors and correlation energies per electron.
%Furthermore, we demonstrated that (T) correlation energies per electron diverge,
%validating that the CCSD(T) method is inapplicable to metallic systems, whilst its newly
%introduced  counterpart, (cT), yields convergent correlation energies per electron.
Several far reaching conclusions must be drawn from the present work.
The RPA to the electron gas employing the resummation of ring diagrams
is formally identical to CCSD theory in the long wavelength limit. This explains why the embedding
of CCSD theory into the RPA yields rapidly convergent results~\cite{Schaefer2021a, Schaefer2021}.
However, we note that this equivalence only holds for amplitudes at $\mathbf{q}=0$. A Taylor series expansion of
$S^\text{rCCD}(\mathbf{q})$ and $S^\text{CCD}(\mathbf{q})$ around $\mathbf{q}=0$ still is expected
to yield different expansion coefficients for terms at least linear 
in $\mathbf{q}$. Furthermore, the contributions of single particle-hole excitation operators
is not taken into account for the UEG, which could also play an important role
in real \emph{ab initio} systems~\cite{Stoehr2021}.

Our work also sheds new light on potential shortcomings of CCSD(T) theory.
We expect that even small gap systems with significant long range correlation
effects are potentially not described with sufficient accuracy if the terms included in (cT)
are missing.
This could explain (part) of the discrepancies observed between DMC and CCSD(T) interaction
energies of large molecules~\cite{Al-Hamdani2021}.

\emph{Acknowledgements.} --The authors thankfully acknowledge support and funding from the
European Research Council (ERC) under the European Union’s Horizon 2020 research and innovation
program (Grant Agreement No. 715594). We gratefully
acknowledge many fruitful discussions with Felix Hummel, Tobias Sch\"afer and James Shepherd. 
The computational results presented have been achieved in part using the Vienna Scientific Cluster (VSC).

%dislodge, undergo
\bibliography{renormalized}

\end{document}

