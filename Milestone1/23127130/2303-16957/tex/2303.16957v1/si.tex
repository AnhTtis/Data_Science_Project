\documentclass[aps,prl,onecolumn,notitlepage]{revtex4-1}
\usepackage{hyperref}
\usepackage[margin=1cm]{geometry}
\usepackage{threeparttable}
\usepackage{graphicx}
\usepackage{booktabs}
\usepackage[dvipsnames]{xcolor}
\usepackage{float}
%\address{}

\begin{document}

\title{
Supplemental Material for: Averting the infrared catastrophe in the gold standard of quantum chemistry
}
\author{Nikolaos Masios}
\author{Andreas Irmler}
\author{Andreas Gr\"uneis}
\email{andreas.grueneis@tuwien.ac.at}
\affiliation{
  Institute for Theoretical Physics, TU Wien,\\
  Wiedner Hauptstraße 8-10/136, 1040 Vienna, Austria
}

\maketitle

\section{Structure factor for different values of $r_s$}

The discussion in the main article was based on a structure factor obtained for a density of
$r_s = 20$. Fig.~\ref{fig:rs} shows structure factors for different values of
$r_s$. This illustrates that the discussed behaviour of the structure factor not only holds
for $r_s = 20$, but is also true for the important range of values between 1 and 5.
However, for larger values of $r_s$ the minimum of the structure factor is already observable
for smaller electron numbers. This was the reason for choosing a rather high value of $r_s$ in
the main article. In both calculations, roughly 18 virtual spacial orbitals per occupied spacial
orbital are used.

\begin{figure*}
  \centering
  \includegraphics[width=0.8\linewidth]{rs.png}
  \caption{For two different number of electrons $N$, twist-averaged structure factors for rCCD are shown.
(a) shows results for $r_s = 3$, (b) shows results for $r_s = 20$.}
  \label{fig:rs}
\end{figure*}

\section{rCCD results for larger electron numbers}

%For rCCD we study systems with more than 246 electrons.  Results
%are presented for the largest electron numbers, which led to stable
%calculations using our implementation.
Figure~\ref{fig:hf} shows rCCD results using $N=1598$ electrons together with data for $N=246$ electrons.
Similar to the numerical results in the main article, a Hartree--Fock
reference is used in this calculations. Fig.~\ref{fig:hf}(a) shows a
very slow convergence of $\textrm{min}_i \left[T_i(\mathbf{q}) \right] \to -0.5$ for
$\mathbf{q} \to 0$. Fig.~\ref{fig:free} shows results obtained using a
Hartree reference. Importantly, similar to the calculations using a Hartree--Fock reference,
$S(\mathbf{q})$ and $\textrm{min}_i \left[T_i(\mathbf{q})\right]$ show the same fundamental limit for
$\mathbf{q} \to 0$.
In all calculations, roughly 18 virtual spacial orbitals per occupied spacial
orbital are used.
%Fig.~\ref{fig:hf}(b) and Fig.~\ref{fig:free}(b) show the corresponding drCCD structure factors which exhibit
%a very similar behaviour and both appear to converge to zero for $\mathbf{q} \to 0$.

\begin{figure*}
  \centering
  \includegraphics[width=0.8\linewidth]{hf.png}
  \caption{Results obtained using a Hartree--Fock reference for two different electron numbers $N$.
           (a) shows $\textrm{min}_i \left[T_i(\mathbf{q})\right]$, whereas (b) shows $S(\mathbf{q})$ as
           defined in the main article.  }
  \label{fig:hf}
\end{figure*}

\begin{figure*}
  \centering
  \includegraphics[width=0.8\linewidth]{free.png}
  \caption{Similar results as in Fig.~\ref{fig:hf}, but using a Hartree reference.}
  \label{fig:free}
\end{figure*}
%This document presents additional data for the manuscript.
%Additional data, input files and post-processing tools
%
%Since the geometries for all considered systems can be found in the supplementary
%
%\begin{figure*}
%  \centering
%  \includegraphics[width=0.31\linewidth]{T_ccd_drccd_rs1.png}
%  \hfill
%  \includegraphics[width=0.31\linewidth]{T_ccd_drccd_rs5.png}
%  \hfill
%  \includegraphics[width=0.31\linewidth]{T_ccd_drccd_rs20.png}
%  \caption{T for three different rs. 1, 5, and 20.}
%  \label{fig:sofg}
%\end{figure*}
%
\section{Molecular testset}

Table~\ref{tab:mol} summarizes
correlation energy contributions
obtained using different triple particle-hole excitation approximations. The
geometries can be found in the work of Knizia et al. \cite{Knizia2009}. For
these calculations the Hartree--Fock ground state was obtained with the NWChem
package~\cite{Valiev2010} and interfaced to cc4s~\cite{cc4s} as described in
Ref.~\cite{Gallo2021}.  The here employed testset was already used in
a previous work \cite{Irmler2021}.  This is where Hartree--Fock and CCSD
energies for the given molecules can be found.

\begin{table}
\centering
\caption{
Triple particle-hole excitation correlation energy contributions
calculated with an aug-cc-pVTZ basis set.
(T) and (cT) are comuted using a one-shot approach as
described in the main article. The energy for the full T, given in the last column, is evaluated by
$E_T = E(CCSDT) - E(CCSD)$. All energies in atomic units.}
\label{tab:mol}
\begin{tabular}{l| c| c| c}
Molecule &   (T)  &  (cT)&   T    \\
\hline
C2H2     & -0.01709& -0.01575& -0.01709 \\
C2H4     & -0.01550& -0.01440& -0.01588 \\
CH3Cl    & -0.01657& -0.01537& -0.01728 \\
CH3OH    & -0.01587& -0.01471& -0.01613 \\
CH3SH    & -0.01663& -0.01547& -0.01745 \\
CH4      & -0.00653& -0.00616& -0.00693 \\
CO       & -0.01789& -0.01642& -0.01800 \\
CO2      & -0.03031& -0.02769& -0.03001 \\
CS2      & -0.03663& -0.03330& -0.03702 \\
Cl2      & -0.01952& -0.01802& -0.02046 \\
ClF      & -0.01867& -0.01727& -0.01924 \\
F2       & -0.01966& -0.01818& -0.01967 \\
H2       &  0.00000&  0.00000&  0.00000 \\
H2O      & -0.00867& -0.00805& -0.00874 \\
H2O2     & -0.02022& -0.01861& -0.02018 \\
H2S      & -0.00867& -0.00814& -0.00941 \\
HCHO     & -0.01757& -0.01620& -0.01771 \\
HCN      & -0.01889& -0.01732& -0.01867 \\
HCOOH    & -0.02799& -0.02569& -0.02795 \\
HCl      & -0.00876& -0.00816& -0.00933 \\
HF       & -0.00755& -0.00702& -0.00758 \\
HNCO     & -0.03044& -0.02783& -0.03027 \\
N2       & -0.01981& -0.01810& -0.01944 \\
N2H4     & -0.01760& -0.01627& -0.01776 \\
NH3      & -0.00833& -0.00777& -0.00854 \\
SO2      & -0.03679& -0.03346& -0.03645 \\
\hline
\end{tabular}
\end{table}

\newpage
\bibliography{renormalized}

\end{document}


