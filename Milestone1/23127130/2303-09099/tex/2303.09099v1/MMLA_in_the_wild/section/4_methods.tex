\section{Methods}

\subsection{Research "in-the-wild"}
The phrase "research in-the-wild" is used in HCI studies to differentiate between research conducted in lab-based environments and research that involves embedding new technology interventions in everyday situations \citep{crabtree2013introduction}. Arguably \citep{rooksby2013wild}, a key tenet of studies conducted in-the-wild is that they can provide more ecologically valid findings compared with typical measures collected under controlled conditions \citep{balestrini2020moving}. Indeed, in-the-wild studies often have to deal with a number of practical and ethical challenges and uncertainties rarely discussed in published studies \citep{rogers2017research}. Carefully identifying these can reveal the biases and the logistics that researchers and designers should consider to design and deploy emerging technologies in a specific context. Based on \citeauthor{rogers2017research}'s proposed framework \cite{rogers2017research} to design for and analyse in-the-wild research, other authors, such as \citet{balestrini2020moving}, have suggested analysis approaches to identify practical challenges that can arise in an in-the-wild study. Inspired by this work \citep{balestrini2020moving}, we explored the main logistic, privacy and ethical challenges when designing and deploying MMLA research in-the-wild, organised around five main themes: 1) \textit{space and place} -- what is the impact of the computational system on the existing setting?; 2) \textit{technology} -- how was the technology used? -- in this case, the data and the analytics; 3) \textit{design} -- what was the impact of the design approach? -- in this case, human-centred design; 4) \textit{social factors} -- what are the stakeholders' concerns and expectations?; and 5) \textit{sustainability} -- what is needed for the new technology to be used continuously over time? 

\subsection{Sources of evidence}
Table \ref{table:evidence} summarises the sources of evidence captured from students, teachers, and the researcher team, using a set of interviews and surveys. This section describes how the questions asked in these cover the five themes introduced above.% Details on these are presented next.

In the first iteration of the study, all participating students were asked to rate their perception of the intrusiveness of the sensing technology in the learning space (theme 1) using a seven-point Likert scale. In addition, twenty volunteering students (18 females, avg. age: 22.21, std. dev.: 4.40 - S1-S20) who participated in the first iteration also participated in 1-hour post-hoc individual interviews to explore their perceptions and experiences more in-depth  in relation to the themes \textit{1- space and place} (i.e., their perceptions on the intrusiveness of devices used and the data collection); \textit{2- data and analytics} (their perceptions on the MMLA dashboard); and \textit{4- social factors} (their concerns regarding the deployment). Moreover, students were presented with early prototypes of the visualisations that ended up in the MMLA dashboard in the second iteration using their own data. Each interview was recorded using an online video conferencing platform (i.e., Zoom) and lasted about 60 minutes. 


% Please add the following required packages to your document preamble:
% \usepackage{multirow}
\begin{table}[b]
\caption{Sources of evidence and themes explored inspired by \citeauthor{balestrini2020moving}'s approach \cite{balestrini2020moving} to identify practical challenges that can arise in an in-the-wild study}
\label{table:evidence}
\begin{tabular}{|l|l|l|l|}
\hline
\textbf{Iteration}       & \textbf{Participants} & \textbf{Sources of evidence} & \textbf{Themes explored}                                                                                                                           \\ \hline
\multirow{1}{*} & Students              & Survey (N=253)        & 1) Space and place                                                                                                                                 \\ \cline{2-4} 
                    & Students              & Interviews (N=20)            & \begin{tabular}[c]{@{}l@{}}1) Space and place\\ 2) Data and analytics\\ 4) Social factors\end{tabular}                                             \\ \hline
\multirow{2}{*} & Students              & Survey (N=47)         & 2) Data and analytics                                                                                                                              \\ \cline{2-4} 
& Teachers              & Survey (N=11)         & 2) Data and analytics                                                                                                                              \\ \cline{2-4}
                    & Teachers              & Interviews (N=4)             & \begin{tabular}[c]{@{}l@{}}1) Space and place\\ 2) Data and analytics\\ 3) Human-centredness\\ 4) Social factors\\ 5) Sustainability\end{tabular} \\ \cline{2-4} 
                    & Researchers           & Survey (N=5)          & \begin{tabular}[c]{@{}l@{}}1) Space and place\\ 3) Human-centredness\\ 4) Social factors\\ 5) Sustainability\end{tabular}                         \\ \hline
\end{tabular}
\end{table}


In the second iteration, the four senior teaching team members (T1-T4, all females) and a total of 7 additional supporting teachers (T5-11, also all females) led the debriefs after each simulation across all classes. All teachers who led the debrief using the MMLA dashboard were asked to complete a survey immediately after each class. The survey comprised questions related to the second theme of the study in-the-wild (\textit{technology - data and analytics}) to inquire about the helpfulness (e.g.\textit{ How the visualisations assisted you in the reflective debrief?}) and integration of the MMLA dashboard into the learning experience (\textit{e.g., How did you integrate the visualisations into the debrief?)}. Moreover, the four senior teaching team members, who were also part of the teachers leading the debrief (T1-T4), participated in a post-hoc reflective interview session to gain insights into their experiences and challenges in using the MMLA dashboard in an authentic setting. We asked questions related to the five themes presented above to explore their perceptions of the study in-the-wild (e.g., regarding \textit{1- space and place}: \textit{How intrusive was it to equip students, teaching staff and the simulation space with various sensors? What kind of unexpected issues did you face that may have affected the learning goals of the simulations?}; \textit{2- technology}: \textit{How and for what purpose did you use the tool during the debrief?}; \textit{3- design and human-centredness}: \textit{To what extent do you value the collaboration with researchers to design this technology with them?}; \textit{4- social factors}: \textit{Did you perceive or hear any concerns from students regarding the study?}; and \textit{5- sustainability}: \textit{What steps would be needed to make the current system into a real-world application without the help of researchers behind it?}). Two interviews were conducted with two teachers at a time, each lasting about 40 minutes. Video recordings of all the interviews were fully transcribed for further analysis. 

In addition, students who consented to participate in the second iteration were invited to participate in a survey to gather their perceptions on the trust of the MMLA dashboard information and their comments about their experience when navigating the information presented in the MMLA dashboard(2 - data and analytics). 
The survey showed the visualisations that were included in the dashboard using students' own data. Students were asked to rate their perception of trust using a five-point Likert scale (1= \textit{I would completely trust this information} - 5: \textit{I would not trust this information}). They were also asked to explain their rate and give comments on their experience with the MMLA dashboard. A total of 47 students (40 females, avg. age: 23.81, std. dev: 5.61 - S21-67) completed this survey.
 
Finally, five members of the research team (R1-R5), who were the ones mostly involved in the deployment, were asked to fill in a survey to reflect on and document their challenges and experiences during both the first and second data collection. The questions were related to four themes (except technology since it is about how teachers and students used the technology) and were similar to those asked to teachers as presented above. The complete protocols for teachers' post-reflection sessions and the research team survey can be found in the supplementary material (see appendices A.1-5, in the order the surveys and interviews were described above).  


\subsection{Analysis}
%Analysis

We synthesised a set of lessons learnt by following a hybrid deductive and inductive thematic analysis approach \citep{fereday2006demonstrating, thomas2008methods}. The \textit{researcher team} and the \textit{senior teaching team} met several times to discuss challenges and concerns around the logistics and consenting strategies that may have impacted the learning experience of students and the overall teaching experience. The first deductive step involved using the literature on research in-the-wild \citep{rogers2017research, balestrini2020moving} to identify the initial five themes presented above that also helped to scaffold the collection of further evidence about the MMLA deployment. Then, the sources of evidence were inductively coded and triangulated by three researchers together to find emerging topics for each main theme \citep{braun2012thematic}. None of these researchers was involved in responding the research team survey. Decisions were made simultaneously and through consensus \citep{mcdonald2019reliability}. All the authors then met to further discuss the lessons learnt and how the evidence can illustrate the challenges that were faced, considering the perspectives of the various educational stakeholders. 
%Table \ref{table:themesandtopics} summarises the resulting emerging topics. The following section reports the results categorised by themes and emerging topics.
