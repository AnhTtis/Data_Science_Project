

\section{Survey and interview guide}
In this section, we provide the survey questions and interview guide for our study. 

\subsection{Student's interview}
\label{ap:students-interview}
%This interview was responded by x students in iteration X of the study.
A total of 20 students participated in the interview during Iteration 1 of the study. Students were asked the following questions:
\begin{flushleft}
\textbf{Theme 1: Space and place}

%  
     - What data do you recall was collected during the simulation?
     
     - Why do you think the data was collected during the simulation? 
     
     - How did you feel wearing sensors during the simulation?
     
     - Did you have any concerns about wearing additional sensors and the microphone?
%   

\textbf{Theme 2: Data and Analytics}
  
     - Would you trust the information presented in the visualisations? 
   

\textbf{Theme 4: Social Factors}

  
     - Who do you think should look at these data?
     
     - Would you be willing to share this data for academic purposes? E.g.: for the teacher to guide the debriefing session?
   
\end{flushleft}
%\pagebreak 


\subsection{ Teaching team survey}
\label{ap:teacher-survey}

\begin{flushleft}
The following survey questions were handled to the teaching team at the end of the debrief session. A total of 11 teachers answered the survey in the Iteration 2 of the study.

\textbf{Theme 2: Data and Analytics}
  
     - How did you integrate the data into the debrief?
     
     - Did you think the data assisted student learning?
     
     - Did you find using the data during the debrief challenging? Why?
   
\end{flushleft}

\subsection{Senior teaching team interview}
\label{ap:teacher-interview}
In this interview, a total of 4 teachers participated in Iteration 2 of the study.
Teachers were asked the following questions:

\begin{flushleft}
\textbf{Theme 1: Space and place}
  
 - How intrusive was it to equip students, teaching staff and the simulation space with various sensors?
 
 - What kind of unexpected (technical and/or logistic) issues did you face that may have affected the learning goals of the simulations?
 
 - How do you think those unexpected issues (technical and/or logistic) can be minimised in the future?
 
 - Do you think you (or other teachers) may need some training with the tool beforehand?
   

\textbf{Theme 2: Data and Analytics}
  
 - How did you use the “tool” (slides) during the debrief? For what purposes?
 
 - To what extent do you think that using the tool may have been helpful for you or the students during the debrief?
 
 - Do you think any of the data presented may have been misleading? If yes, explain how?

 - Did you trust the information presented to you through the tool during the debrief? 
 
 Did you also use the visualisations for cases where the data were incomplete (for example, when not all the students were tracked)?
 
 - How can we improve the system so you can trust more on the data?
   

\textbf{Theme 3: Human-Centredness}
  
 - To what extent do you value the collaboration with learning analytics researchers to design this technology with them? What motivates you to do that? 
   

\textbf{Theme 4: Social Factors}
  
 - What do you think about the consenting strategy from last year's study (iteration 1) in comparison to the one for this year's study (iteration 2)?
 
 - What do you think can be done differently regarding the consenting strategies for a future study?
 
 - Did you perceive or hear any concerns from students regarding the study? If so, can you explain?
 
 - Besides the students and the teacher leading the debrief, who do you think would benefit from looking at the visualisations shown in the tool used in the debrief?
   


\textbf{Theme 5: Sustainability}
  
 - How can we run our research studies in the future in a more sustainable way? 
 
 - What steps would be needed to make our current system into a real-world application without the help of a team of researchers behind it?
   
\end{flushleft}

\subsection{Researchers' survey}
\label{ap:researchers-survey}
At the end of Iteration 2 of the study, five researchers from our team were asked to fill in a survey, comprised of the following questions:

\begin{flushleft}
\textbf{Theme 1: Space and Place}
  
 - How complex was it to transport, install and configure the equipment before the data collection location? 
 
 - Did you face any specific challenges/problems?

 - What kind of unexpected technical challenges/problems did you face regarding the sensors/equipment during the study/data collection?

 - What kind of unexpected logistic issues did you face that could affect the study?

 - How do you think those unexpected issues (technical and/or logistic) can be minimised in the future?

\textbf{Theme 3: Human-Centredness}
  
 - What do you value the most when collaborating or interacting with teachers or students to plan, analyse or validate your research ideas?
   

\textbf{Theme 4: Social Factors}
  
 - What do you think about the consenting strategy from last year's data collection (iteration 1)? 

 - What do you think about the consenting strategy from this year's data collection (iteration 2)?

 - Please, share your ideas on what you think can be done differently regarding the consenting strategies for a future study/data collection.
   

\textbf{Theme 5: Sustainability}
  
- How can we run our research studies in the future in a more sustainable way? 

- How can we recover from failure and debug issues to provide reliability to our systems? 

- What steps would be needed to make our current system into a real-world application without the help of a team of researchers behind it?

\end{flushleft}

\subsection{Student's survey}
\label{ap:students-survey}
%This interview was responded by x students in iteration X of the study.
A total of 47 students completed a survey during Iteration 2 of the study. Students were asked the following questions about the visualisations presented in the MMLA dashboard:
\begin{flushleft}
\textbf{Theme 2: Data and Analytics}

\textbf{\textit{Visualisation 1: Team Communication}}

Please review the following visualisation that represents the data we collected during your simulation. Reflect on what it may represent and answer the questions below. 

\begin{figure}[h!]
\includegraphics[width=12cm]{figures/Vis1.png}
\end{figure}


\textit{Note: this visualisation IS representing YOUR data}

- To what extent you would trust this visualisation to judge or reflect on your own performance?

        \begin{enumerate}
        \item I’d absolutely trust on it
        
        \item I’d trust it to some extent
        
        \item neutral/borderline
        
        \item I’d not trust it to some extent
        
        \item I’d absolutely not trust on it
        \end{enumerate}
        
- Please, briefly explain your response.

- Do you have any comments on how this visualisation can be improved, or would you add something to the visualisation to make it more straightforward?   
\pagebreak 

\textbf{\textit{Visualisation 2: Team Speaking and Positioning}}

Please review the following visualisation that represents the data we collected during your simulation. Reflect on what it may represent and answer the questions below. 

\begin{figure}[h!]
\includegraphics[width=12cm]{figures/Vis2.png}
\end{figure}


\textit{Note: this visualisation IS representing YOUR data}

- To what extent you would trust this visualisation to judge or reflect on your own performance?

        \begin{enumerate}
        \item I’d absolutely trust on it
        
        \item I’d trust it to some extent
        
        \item neutral/borderline
        
        \item I’d not trust it to some extent
        
        \item I’d absolutely not trust on it
        \end{enumerate}
        
- Please, briefly explain your response.

- Do you have any comments on how this visualisation can be improved, or would you add something to the visualisation to make it more straightforward?   
\pagebreak 


\textbf{\textit{Visualisation 3: Team Prioritisation}}

Please review the following visualisation that represents the data we collected during your simulation. Reflect on what it may represent and answer the questions below. 

\begin{figure}[h!]
\includegraphics[width=12cm]{figures/Vis3.png}
\end{figure}


\textit{Note: this visualisation IS representing YOUR data}

- To what extent you would trust this visualisation to judge or reflect on your own performance?

        \begin{enumerate}
        \item I’d absolutely trust on it
        
        \item I’d trust it to some extent
        
        \item neutral/borderline
        
        \item I’d not trust it to some extent
        
        \item I’d absolutely not trust on it
        \end{enumerate}
        
- Please, briefly explain your response.

- Do you have any comments on how this visualisation can be improved, or would you add something to the visualisation to make it more straightforward?   
\end{flushleft}

\pagebreak 


\textbf{Authors' statement}

The content and contribution of the manuscript are unique in relation to our previous publications. In the manuscript, we cite our own papers where details that are not related to the main contribution of the current manuscript can be found. More specifically, these are two other papers, cited in Section 3, where some details about the human-centred design approach we followed can be consulted:

\textit{ Vanessa Echeverria, Roberto Martinez-Maldonado, Lixiang Yan, Linxuan Zhao, Gloria Fernandez-Nieto, Dragan Gašević, and Simon Buckingham Shum. 2022. HuCETA: A Framework for Human-Centered Embodied Teamwork Analytics. IEEE Pervasive Computing (2022), 1–11.}

\textit{Gloria Milena Fernandez Nieto, Kirsty Kitto, Simon Buckingham Shum, and Roberto Martinez-Maldonado. 2022. Beyond the Learning Analytics Dashboard: Alternative Ways to Communicate Student Data Insights Combining Visualisation, Narrative and Storytelling. In 12th International Learning Analytics and Knowledge Conference (Online, USA) (LAK22). ACM New York, NY, USA, 219–229.}


\begin{comment}
\section{Multimodal analytics Dashboard}
%figures are here in case you want to edit them
%https://docs.google.com/presentation/d/1Ro3TlyYltKfrrKTi-NFcPIHWpDwp3OCp/edit?usp=sharing&ouid=100865265970408855218&rtpof=true&sd=true

In this section, we present the three visualisations that were part of the dashboard utilised during the simulation debrief.

\textbf{Team speaking interaction}


\begin{figure}[h!]
\includegraphics[width=8cm]{figures/Vis1.png}
\end{figure}

\textbf{Team speaking and positioning}
\begin{figure}[h!]
\includegraphics[width=8cm]{figures/Vis2.png}
\end{figure}

\textbf{Team prioritisation}
\begin{figure}[h!]
\includegraphics[width=8cm]{figures/Vis3.png}
\end{figure}

\end{comment}