\section{Related Work}

 %Most MMLA studies so far have primarily focused on developing prototypes and testing the functionality of different combinations of sensors and analytics approaches \citep{shankar2018review, mu2020multimodal, noroozi2020multimodal}. 
In this section, we review the most recent systematic literature reviews on MMLA and related works that have identified several prominent logistical, privacy and ethical challenges that need to be addressed for this promising area to remain relevant and have an actual impact on educational practices.

%MMLA and multimodal data have received increased attention from the learning analytics and educational technology communities as a promising research direction that holds the potential to generate meaningful insights about teaching, and learning in partially and non-computer mediated educational contexts \citep{sharma2020multimodal, chango2022review}. 
% \citep{alwahaby2021evidence, yan2022scalability}. 

%\subsection{Practical Challenges}

%The practical challenges of MMLA are mainly associated with the lack of well-reported and large-scale studies that structurally assess the influence of MMLA innovations on actual educational practice. For example, \citet{alwahaby2021evidence} reviewed 100 MMLA articles and concluded that most of the empirical evidence presented in prior studies remains descriptive, correlational, or anecdotal, with little strong causal evidence regarding the impacts of MMLA innovations on real-world educational practices. 


%Consequently, although the alignment between MMLA innovations and learning design should be one of the foundations for developing MMLA innovations \citep{cukurova2020promise, ochoa_multimodal_2022}, such alignments are rarely considered or reported in the existing literature, as evidenced in recent reviews \citep{sharma2020multimodal, praharaj2021literature}. 
%Likewise, the lack of MMLA studies that closed the LA loop by providing feedback to students or insights to teachers during educational practices instead through post-hoc research-focused interviews or surveys also hindered the understanding of MMLA innovations' actual impacts on learning and teaching outcomes \citep{yan2022scalability}. %Therefore, in-depth insights on aligning MMLA innovations with learning designs, relevant theories, and educational stakeholders are urgently needed to ensure future MMLA studies do not deviate from the ultimate goals of learning analytics \citep{gavsevic2015let}.

\subsection{Logistical Challenges}
Most MMLA studies so far have primarily focused on developing prototypes and testing the functionality of different combinations of sensors and analytics approaches \citep{shankar2018review, mu2020multimodal, noroozi2020multimodal}. Yet, many concerns have been raised regarding the logistical challenges that can emerge when moving from controlled settings to in-the-wild MMLA deployments such as the added intrusiveness of sensing devices and complexity in their installation and orchestration \citep{chua2019technologies}. \citet{yan2022scalability} systematically reviewed these logistical issues and identified a relatively low level of technology readiness regarding existing MMLA innovations, resulting in heavy reliance on the onsite support of researchers or technicians. This  undermines the sustainability of these systems and unnecessarily increases the complexity of the learning situation from the teachers' perspective. While most of the sensing technologies used in MMLA research can be purchased off-the-shelf, implementing these technologies in authentic physical learning spaces often requires extensive technical background for tasks such as physical installation, system integration, and modalities synchronisation \citep{crescenzi2020multimodal, shankar2018review, mu2020multimodal}. There is also a trade-off between data quality and affordability as most of the MMLA innovations that rely on mature sensing technologies, such as location sensors, eye-trackers, and biometric sensors, can be financially unscalable due to the high unit prices \citep{yan2022scalability}. Although low-cost alternatives are emerging \citep[e.g.,][]{ochoa2018rap, saquib2018sensei}, these technologies remain in the prototype and validation stages and often sacrifice accuracy or portability for affordability. 

Likewise, the lack of MMLA studies that have closed the LA loop by providing some form of end-user interface to students or insights to teachers make it harder for educational stakeholders to weigh the benefits against the potential added complexity to their already rich educational ecologies \citep{yan2022scalability}. Although the alignment between MMLA innovations and learning design should be one of the foundations for developing MMLA innovations \citep{cukurova2020promise, ochoa_multimodal_2022}, such alignments are rarely considered or reported in the existing literature, as noted in recent literature reviews \citep{sharma2020multimodal, praharaj2021literature}. This can undermine teacher and student confidence, if they do not understand how the MMLA system aligns with their teaching practices or learning outcomes. 

All of these challenges are hallmarks of emerging HCI infrastructures that must be co-evolved with work practices. This in-the-wild MMLA deployment offered the opportunity to study how both educational and technical stakeholders learnt to work together to address the challenges.%Gaining insights regarding this trade-off between data quality and affordability could benefit educational researchers and practitioners when evaluating their budgets against the type of educational insights they are trying to capture. 

\subsection{Privacy Challenges}
As a research area that benefits from the data collection opportunities enabled by various sensing technologies, the privacy issues surrounding the adoption of MMLA innovations are the focus of critical debate. \citet{crescenzi2020multimodal} emphasised the need to consider the privacy implications of using sensing technologies to generate analytics about children's activity. Such implications have also been identified by students and teachers who have expressed concerns regarding the security of their data \citep{mangaroska2021challenges, kasepalu2021teachers}. These privacy implications of MMLA innovations have been under-investigated in the literature \citep{Alwahaby2022, yan2022scalability, Oviatt2018challenges}. Specifically, while most works published in MMLA  mention that informed consent was obtained from participants, none of the existing works has elaborated on the consenting strategies they adopted, which could contribute valuable insights regarding data security measures for protecting individual privacy and maximising data autonomy (e.g., individuals' autonomy of removing their data from the database) \citep{beardsley2020enhancing}. Additionally, while most of MMLA innovations endeavour to provide dashboards and visualisations for supporting educational practices, privacy issues regarding who has the right to see these visualisations  remain unclear, especially in the contexts of collaborative learning where, in most cases, individuals' personal trace data, even anonymised (e.g., masking students' identity with numbers or colours), could remain identifiable when used for provoking reflections at a group-level, since other students typically have the contextual knowledge to decode anonymised representations \citep{mangaroska2021challenges, Alwahaby2022}. Providing additional empirical evidence on educational stakeholders' perspectives of these privacy-related issues could potential benefit the on-going development of MMLA, and is a particular focus of this study.

\subsection{Ethical Challenges}
Beyond logistical and privcy issues, the potential biases in analytics, and cognitive dissonances that may be caused by the inconsistency between individuals' observations and generated insights, could also undermine the potential benefits of MMLA innovations \citep{ferguson2016guest,Oviatt2018challenges}. Such issues are vital as the accuracy of the existing MMLA-based predictive models and early-warning systems are far from suitable for practical deployment (e.g., rarely above 80\% accuracy), and these models have mostly been developed and evaluated based on relatively small sample sizes (i.e., with n < 50) \citep{yan2022scalability}. These small sample sizes combined with the poor reporting standards found in the existing MMLA literature could also mask potential algorithmic biases that may disadvantage certain minority groups of students as replicating these studies remain difficult without adequately reported methodologies \citep{luzardo2014estimation, yan2022scalability}. Additionally, \citet{Alwahaby2022} also highlighted the significant concerns regarding the need to enhance trust and data transparency within MMLA systems and \citet{yan2021footprints} suggested that more research needs to be done to assess the potential risk of making decisions with incomplete multimodal data.%Additionally, using unsupervised machine learning techniques to cluster and label students may also induce the potential risk of discrimination, where certain labels (e.g., at-risk)  could negatively impact learners' self-esteem and educators' expectations \cite{higgins2002stages}. 
Consequently, understanding the ethical practices of using these analytics is also essential but rarely considered in prior literature \citep{selwyn2019s} and requires the participation of key educational stakeholders such as students and educators \citep{Oviatt2018challenges}.  A large-scale in-the-wild study opens new opportunities to study approaches to these ethical challenges under more authentic conditions than has been reported to date. 

\subsection{Contribution to HCI and Research Question}
Against the literature reviewed above we formulate the following research question (RQ) that guided our study: 

\textit{\textbf{RQ:} What logistical, privacy and ethical challenges emerge from a complex MMLA, in-the-wild study that closes the analytics loop by providing direct feedback to students?}

In addressing this question, the contribution of this paper is a set of lessons learnt regarding how such challenges were, or could have been, addressed in the context of a two-year deployment of a MMLA system in an authentic educational scenario. The implications of this study should assist researchers, developers and designers in making informed decisions about the effective deployment of innovations that involve the use of ubiquitous computing technologies, sensing devices and artificial intelligence (AI) algorithms to augment teaching and learning in physical spaces. 

% Reviews I reckon you already cited in the Scalability paper and which I used in the intro
% \cite{sharma2020multimodal}
% \cite{crescenzi2020multimodal}
% \cite{chua2019technologies}
% \cite{alwahaby2021evidence}

% Note for Jimmie - Other SLR on MMLA reviews to be inlcuded: 
% \cite{noroozi2020multimodal}
% \cite{shankar2018review}
% \cite{praharaj2021literature}
% \cite{mu2020multimodal}
% \cite{yan2022scalability} %of course!
% \cite{chango2022review}
