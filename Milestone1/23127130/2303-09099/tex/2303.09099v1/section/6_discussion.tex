\section{Discussion}
In this section, we summarise the lessons learnt from our MMLA in-the-wild deployment; then discuss the implications of these findings for practice, identify various limitations of our in-the-wild study, and suggest some potential directions for future research and development.

\subsection{Summary of lessons learnt}

This paper provides a summary of some of the key logistical, privacy and ethical challenges that emerged from our complex MMLA, in-the-wild study. These can be listed as follows: \hfill \break


\emph{Space and place}
\begin{itemize}
\item \textbf{Intrusiveness} -- While students did not report discomfort in wearing sensors, teachers can still get concerned about their potential \textit{distracting factor} and some students can feel \textit{stressed} about being monitored. 

\item \textbf{MMLA Technology readiness} -- The lack of MMLA technology readiness can severely impact the lesson plan. Teachers need to play an active role to create \textit{strategies to moderate} the sensing/analytics technologies, and minimise potential disruptions and setup time.  

\item \textbf{Unexpected issues during the MMLA deployment} -- While several technical issues that can emerge during the MMLA deployment are beyond the control of the research team, reducing the number of devices used can minimise potential technical failures. Some high-end sensors may need to be replaced with less expensive sensors, that may capture coarser data, if the change increases \textit{reliability}. 

\item \textbf{Multimodal data quality, portability of sensors and affordability} -- At least currently, a trade-off may exist between capturing \textit{high quality} data and the portability and affordability of the sensing technology.
\end{itemize}

\emph{Technology: data and analytics}
\begin{itemize}
\item \textbf{Purpose of capturing multimodal data} -- If communicated clearly, students are willing to participate in a complex MMLA study and contribute their data for the purpose of helping their teachers or future students. Teachers can and need to develop strategies to optimise the use of multimodal data to support students.  

\item \textbf{Multimodal data incompleteness and trustworthiness} -- Although multimodal data is required to build analytical representations of an embodied learning experience, multimodal sensor data are intrinsically incomplete and subject to bias. Thus, mechanisms to ensure MMLA systems are \textit{trustworthy} and designing for data incompleteness are required. 

\item \textbf{Emerging issues related to visualising multimodal data} -- Teachers need to be supported to develop relevant \textit{data literacy skills} to understand the basic inner-workings of specific MMLA systems and for them to develop pedagogical \textit{strategies around the effective use} of the intrinsically complex MMLA visual interfaces. Students may also require visualisation guidance or explanatory features for them to the meaning of the data in educational terms.
\end{itemize}

\emph{Design: human-centredness}
\begin{itemize}
\item \textbf{Human-centred MMLA and students' learning} -- Teachers' appreciation of partnering with researchers in the design process can lead to creating MMLA systems aligned with teaching practices and learning goals. 

\item \textbf{Human-centred MMLA and research innovation} -- Involving teachers and students in the design process contributes to the validation of the MMLA interfaces according to the learning design and to the improvement of the logistics of the MMLA research study. 
\end{itemize}

\emph{Social factors}
\begin{itemize}
\item \textbf{Consenting and participation strategies} -- It is challenging to explain to students what a complex MMLA study entails. Providing too many technical details about the sensors and the analytics in advance does not necessarily contribute to clarity. Explaining the complexity of the MMLA deployment \textit{in person} can enable students to ask clarification questions and then provide informed consent.  

\item \textbf{Data privacy and sharing} -- Students were willing to share their multimodal data with others if their privacy is preserved and the purpose is limited to supporting learning. While most students see their multimodal data as only beneficial to themselves, some students can see the potential benefit to make their data available to other students to learn from their experiences or for teachers to improve the design of the learning tasks. 
\end{itemize}

\emph{Sustainability}
\begin{itemize}
\item \textbf{Technological sustainability} -- A potential strategy to maximise long-term technical sustainability is a lightweight \textit{microservices-based architecture} that can enable attaching and detaching heterogeneous sensors as required.

\item \textbf{MMLA appropriation in the classroom} -- A potential strategy to maximise adoption and technology appropriation includes embedding sensing capabilities into the classroom, providing a high degree of user control, providing training to teachers on system usage and data interpretation, and keeping the need for support from a technical actor to a minimum extent.
\end{itemize}

\subsection{Implications for practice}
The lessons learnt from our in-the-wild MMLA study have several implications. We summarise these into the following three recommendations to provide guidance for researchers, developers and designers to make informed decisions about the effective deployment of MMLA in-the-wild. 

\textbf{\textit{Forging design partnerships with teachers and students}.} The more sensors are used to capture activity in complex educational scenarios that involve non-computer mediated interactions, or ill-defined, open tasks such as in teamwork, the more complex the meaning-making process becomes to move from data to insights  \citep{echeverria19towards}. Thus, as rich data infrastructures become more commonplace in educational contexts \citep{guzman2021learning}, it is also becoming critical to forge strong partnership relationships among teachers, students, educational decision-makers, researchers and developers. This has the potential to ensure that algorithmic outputs and data representations are meaningful and aligned to local learning objectives and pedagogical values \citep{Ahn2019}. Indeed, some educational researchers have started to utilise the body of knowledge and practice from design communities, such as participatory design and co-design, in data-intensive educational contexts \cite{BuckinghamShum2019}. However, following human-centred design approaches is yet to be seen in MMLA according to the most recent review \citep{yan2022scalability}. 

In our study, several practical challenges in the MMLA deployment demanded expertise from a wide range of areas (such as learning analytics, interaction design, and information visualisation), plus knowledge from stakeholders contributing insights and evidence from their lived experiences. By giving an active voice to students and involving teachers in the design process we were able to identify the key practical challenges that can easily undermine adoption if they are not addressed in a timely manner. Teacher/student involvement was also critical to give meaning to the complex multimodal data streams both for research purposes, and to design the MMLA dashboard aimed at end-users. An indicator of the success of the teachers' partnering experience, is that once they reflected on the value of the MMLA deployment, they wanted to move the deployment to happen as a part of their regular classes, potentially making the transition from research to practice an immediate possibility. 

Yet, much work is still required to develop specific guidelines to create human-centred MMLA systems. For example, the rapidly growing human-centred AI \citep{shneiderman2021human} movement within and beyond HCI has much to offer to the design and development of MMLA systems to ensure that novel AI tools are effectively in service of students and teachers. Moreover, researchers and developers may want to address the complexity of visual interfaces of multimodal data by grounding their designs in key Information Visualisation principles aimed at scaffolding the interpretation of large amounts of data by non-technical users (e.g., by applying data visualisation guidance \citep{ceneda2016characterizing} or data storytelling \citep{martinez20} principles).  


\textbf{\textit{Designing MMLA considering data imperfection and teacher control}.} 
% Depending on the context, empatica may work when students do not move a lot, otherwise, we need more feasible sensors/devices
In Jeffrey Heer's view \citep{heer2019agency}, \textit{"AI methods can be applied to helpfully reshape, rather than replace, human labor"}. In our study, the ultimate aim is not to replace the teacher but augment their repertoire of tools they can use to support students' reflective thinking through data interfaces. Yet, the data captured from the physical world through sensing devices are often incomplete, noisy, and unreliable \citep{bamgboye2018towards}. Moreover, beyond the use of multimodal data in education, it has been  reported that there is commonly a disconnection between logged data and higher-order educational constructs \citep{echeverria19towards, mangaroska2018learning}. This means that the design of effective MMLA interfaces needs to deal with data incompleteness and partial models of the actual learning activity. Creating MMLA systems that perform fully automated actions based on these incomplete data can thus be risky, and cannot be recommended at this level of MMLA maturity.

A primary finding from our MMLA in-the-wild study is that teachers see that a key requirement to maximise the sustainability of the complex computational system is to provide a high degree of user control. The debate around the balance between human agency and AI automation is not new in HCI \citep[e.g.][]{shneiderman1997direct}, yet, it is nascent in the context of MMLA. Nonetheless, \citet{Ogan19} suggested that once sensing technologies mature to the extent that they enable capturing a variety of behaviours in the classroom, we should let teachers empower themselves to use data for making informed decisions and improving their own classroom practices. 

Moreover, we learnt that if the MMLA interface does not provide any visual cue about potential data incompleteness, both teachers and students can attempt to make potentially misleading inferences from the data. More problematically, decisions can be made and actions can be taken without sufficient recognition that logged student data is, by definition, imperfect \citep{Kitto18Imperfection}. In the long term, this can damage their trust in the system. 

Future work can consider at least two potential ways to address these challenges. First, as suggested by some of the teachers in our study, it may be possible to identify gaps in teachers' knowledge around the use of data in their practice such as whether they are aware of how the multimodal data are collected, what educational constructs are being modelled, the limitations of algorithmic outputs, and the kinds of insights that can be derived from them. Professional development programs can be created to increase teachers' AI literacy \citep{long2020ai} and visualisation literacy \citep{pozd2023} for them to understand, to some extent, how they can integrate the MMLA interfaces into their existing practices or how they can adapt their current practices to the new possibilities enabled by the use of such multimodal data. Alternatively or in parallel, the teachers in our study also suggested that the MMLA user interface can be designed to provide visual cues that alert teachers about the reliability of the data so they can make informed data interpretations or decide not to use the MMLA system for a session with uncertain data. To address this, researchers and developers of this kind of innovations may want to consider elements from the emerging literature on the human aspects of AI explainability \citep{JIANG2022102839,khosravi2022explainable} to design MMLA systems that, for example, reveal their assumptions and biases in ways that make sense to non-specialist users so they can keep in control of the potential pedagogical actions that can be taken \citep{selwyn2019s}. 



\textbf{\textit{Ensuring teachers' and students' safety}. }
Enhancing physical learning spaces with rich sensing capabilities unavoidably raises critical questions about the potentially harmful effects of excessive surveillance and potential threats to students' and teachers' privacy rather than supporting learning. Preserving human safety in increasingly autonomous smart environments has been identified as one of the main HCI grand challenges \citep{stephanidis2019seven}. \citet{selwyn2019s} explains that even learning analytics systems intended to only support students' learning run the risk of being utilised for broader purposes: \textit{"the concern here lies with the secondary (re)uses of learning analytics data by institutions and other `third parties'”} (p.3). Multimodal learning data can raise particular concerns since analysing a combination of on-skin and under-skin sensor data can lead to richer user models that could be used for student profiling or for performance measurement of teachers which may have negative consequences for the individuals concerned \citep{selwyn2018doing}. Unfortunately, the ethical implications of using MMLA systems have been seldom mentioned in the literature, as has been flagged in recent scoping works \citep{cukurova2020promise,worsley2021new} and reviews \citep{Alwahaby2022,crescenzi2020multimodal}. 

 Our findings flagged some further concerns. Teachers and students may not easily grasp all the potential ways in which their data can be exploited. Yet, they had sufficient awareness to confirm that their data should only be used by themselves or by other educational stakeholders to support other students. Strict guidelines about data privacy and data ownership should be established for systems that use students' multimodal data since some of these data can be highly sensitive. For example, designers could explore ways in which end-users can indicate to the MMLA system to forget their multimodal data totally or partially after it has been used for educational purposes \citep{Muller2022}. Visualising multimodal data also raised another set of potential concerns. In our second iteration, students' inclinations to participate in a MMLA study changed as they seemed to be more willing to participate in a study that only involved data collection but were not sure about all the implications related to having a user interface showing their data in front of their classmates. In this regard, future MMLA work aimed at closing the learning analytics loop by providing end-user data interfaces would benefit from building upon the long-standing HCI research focused on designing for sharing personal data through group interfaces \citep{greenberg1999pdas}. Moreover, although some preliminary work has attempted to discuss ways to effectively write up consent forms for MMLA studies \citep{beardsley2020enhancing}, further work is needed to understand how students can make informed decisions regarding their participation in MMLA studies or in terms of data sharing based on the types of data used in a particular MMLA innovation. 

%One suggestion along these lines is to give students ownership of their own data — what could be termed “personal data sovereignty” (Jarchow & Estermann, 2015). 

\subsection{Limitations}
Our study has various limitations. First, the lessons learnt are not generalisable as MMLA studies cannot be treated as a generic type of analytics. Our study involved the use of video, physiological wristbands, audio, and indoor positioning sensing. Although these cover most types of sensors used in  MMLA studies \citep{yan2022scalability}, students' and teachers' perceptions towards sensing technologies can vary across learning situations and technical setups. For example, in other studies where laboratory-grade EEG headsets have been worn by students, their perceptions towards potential negative effects related to sensor intrusiveness have been more prominent compared to those of the students in our study \citep{mangaroska2021challenges}. 

A second limitation is that the teachers and students in our study were, to some extent, accustomed to technology-equipped learning spaces, such as the simulation rooms. Thus, our MMLA sensors were added to an existing ecology of devices and educational practices that involve the use of technologies of various kinds. Nonetheless, most of the existing technologies are not used for the purpose of monitoring and data-intensive reflection thus the lived experiences of the educational stakeholders were novel in relation to the MMLA innovation. 

A third limitation is that the students who participated in the study and the interviews were those who were more willing to participate and often highly motivated as participation was optional. We could not interview participants who were less inclined to experience the MMLA study which prevented us from gaining a deeper understanding of the factors considered by non-consenting students or potential further concerns about the deployment. 

Besides the comments from students and teachers, we also reported some of the lessons learnt from a researcher's perspective with the aim of sharing the particular experiences and insights we gained from this in-the-wild experience. Readers are encouraged to interpret these as such rather than as generalisable claims. 

Finally, evidence was captured heterogeneously from iterations 1 and 2 of our study (e.g., students were interviewed about intrusiveness in iteration 1 but not in iteration 2). This was a consequence of conducting the study under authentic conditions in which the research aims adapted to the needs and availability of the teachers, the students and the planned educational activities. Yet, we did not want to challenge these to preserve the in-the-wild nature of the study.
