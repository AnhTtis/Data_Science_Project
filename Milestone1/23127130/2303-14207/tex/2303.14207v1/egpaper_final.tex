\documentclass[10pt,twocolumn,letterpaper]{article}

\usepackage{iccv}
\usepackage{times}
\usepackage{epsfig}
\usepackage{graphicx}
\usepackage{amsmath}
\usepackage{amssymb}
\usepackage{xcolor}         % colors
\usepackage{float}
\usepackage{caption}
%\usepackage{natbib}
\usepackage{amsfonts,amssymb}
\usepackage{mathrsfs} 
\usepackage{tabularx}
\usepackage{wrapfig}
\usepackage{verbatim}
\usepackage{dsfont}
\usepackage{tablefootnote}
\usepackage[stable]{footmisc}
\usepackage{booktabs}
\usepackage{multicol}
\usepackage{multirow}
\usepackage{color}
\usepackage{subcaption}
\usepackage{pythonhighlight}

%\newcommand\rev[1]{{\color{red}{#1}}}
\newcommand{\rulesep}{\color{black} \unskip\ \vrule\ }

\newcommand\rev[1] { {}{#1} }

\newcommand\todo[1]{\textbf{TODO: }{\color{red}{#1}}}

\definecolor{todocolor}{RGB}{255,0,00}
\newcommand\TODO[1] {\PackageWarning{}{Unprocessed todo}\emph{\textcolor{todocolor}{TODO: #1}}}

\newcommand{\red}[1]{{\color{red}#1}}
\newcommand{\blue}[1]{{\color{blue}#1}}

\definecolor{jiapeng}{rgb}{0.2, 0.4,0.9}
\newcommand{\jiapeng}[1]{\textcolor{jiapeng}{\emph{Jiapeng:~{#1}}}}
\renewcommand{\jiapeng}{}
\renewcommand{\blue}{}

\renewcommand{\vec}[1]{\mathbf{#1}}
\newcommand{\expnumber}[2]{{#1}\mathrm{e}{#2}}
\DeclareMathOperator{\IoU}{IoU}
\DeclareMathOperator{\SCA}{SCA}
\DeclareMathOperator{\CKL}{CKL}
\DeclareMathOperator{\CD}{CD}
\DeclareMathOperator{\KL}{KL}
% Include other packages here, before hyperref.

% If you comment hyperref and then uncomment it, you should delete
% egpaper.aux before re-running latex.  (Or just hit 'q' on the first latex
% run, let it finish, and you should be clear).
\usepackage[breaklinks=true,pagebackref=true,colorlinks,letterpaper=true,bookmarks=false]{hyperref}

\iccvfinalcopy % *** Uncomment this line for the final submission

\def\iccvPaperID{****} % *** Enter the ICCV Paper ID here
\def\httilde{\mbox{\tt\raisebox{-.5ex}{\symbol{126}}}}

% Pages are numbered in submission mode, and unnumbered in camera-ready
\ificcvfinal\pagestyle{empty}\fi

\begin{document}
%%%%%%%%% TITLE
%\title{DiffuScene: Indoor Scene Synthesis via Diffusion Models}
\title{DiffuScene: Scene Graph Denoising Diffusion Probabilistic Model for Generative Indoor Scene Synthesis}


\author{
    Jiapeng Tang$^1$ \quad Yinyu Nie$^1$ \quad Lev Markhasin$^2$ \quad Angela Dai$^1$ \quad Justus Thies$^3$ \quad Matthias Nie{\ss}ner$^1$ \\
    \\
     $^{1}$ Technical University of Munich \quad
     $^{2}$ Sony Europe RDC Stuttgart \\ %Sony Europe Ltd. \\
     $^{3}$ Max Planck Institute for Intelligent Systems, Tübingen, Germany \\
     \url{https://tangjiapeng.github.io/projects/DiffuScene} \\
}

\twocolumn[{%
	\renewcommand\twocolumn[1][]{#1}%
	\maketitle
        %\vspace{-0.3cm}
	\begin{center}
            \vspace{-6mm}
		\centerline{
                 \includegraphics[width=\linewidth]{./figs/teaser_diffuscene.pdf}        
		}
        %\vspace{-0.1cm}
        \captionof{figure}{ We present \emph{DiffuScene}, a diffusion model for diverse and realistic indoor scene synthesis, which can facilitate various down-stream applications: scene completion from partial scenes (left); scene arrangements of given objects (middle); \blue{scene generation from a text prompt describing partial scene configurations.} (right).}
        \label{fig:teaser}
	\end{center}
}]

% Remove page # from the first page of camera-ready.
\ificcvfinal\thispagestyle{empty}\fi


%%%%%%%%% ABSTRACT
\begin{abstract}


Over the past few years, there has been a significant amount of research focused on studying the ReLU activation function, with the aim of achieving neural network convergence through over-parametrization. However, recent developments in the field of Large Language Models (LLMs) have sparked interest in the use of exponential activation functions, specifically in the attention mechanism.

Mathematically, we define the neural function $F: \R^{d \times m} \times  \mathbb{R}^d \rightarrow \mathbb{R}$ using an exponential activation function. Given a set of data points with labels $\{(x_1, y_1), (x_2, y_2), \dots, (x_n, y_n)\} \subset \mathbb{R}^d \times \mathbb{R}$ where $n$ denotes the number of the data. Here $F(W(t),x)$ can be expressed as $F(W(t),x) := \sum_{r=1}^m a_r \exp(\langle w_r, x \rangle)$, where $m$ represents the number of neurons, and $w_r(t)$ are weights at time $t$. It's standard in literature that $a_r$ are the fixed weights and it's never changed during the training. We initialize the weights $W(0) \in \mathbb{R}^{d \times m}$ with random Gaussian distributions, such that $w_r(0) \sim \mathcal{N}(0, I_d)$ and initialize $a_r$ from random sign distribution for each $r \in [m]$.

Using the gradient descent algorithm, we can find a weight $W(T)$ such that $\| F(W(T), X) - y \|_2 \leq \epsilon$ holds with probability $1-\delta$, where $\epsilon \in (0,0.1)$ and $m = \Omega(n^{2+o(1)}\log(n/\delta))$. To optimize the over-parametrization bound $m$, we employ several tight analysis techniques from previous studies [Song and Yang arXiv 2019, Munteanu, Omlor, Song and Woodruff ICML 2022]. 

 

\end{abstract}

%%%%%%%%% BODY TEXT

\section{Introduction}

The increasing complexity of source code poses a key challenge to the reliability of large-scale software systems. Software bugs in these systems can lead to safety issues~\cite{bug_safety} for users around the world as well as cause non-negligible financial losses~\cite{bug_loss}. As such, developers have to spend a large amount of time and effort on bug fixing. Consequently, \aprfull (\apr), designed to automatically generate patches to fix software bugs, has attracted wide attention from both academia and industry~\cite{long2016prophet, legoues2012genprog, long2015spr, lou2020can, tufano2018empstudy}. 


To achieve \apr, one popular approach is known as Generate-and-Validate (G\&V)~\cite{qi2015gv, ghanbari2019prapr, lou2020can, le2016hdrepair, legoues2012genprog, wen2018capgen, hua2018sketchfix, martinez2016astor, koyuncu2020fixminder, liu2019tbar, liu2019avatar}, which is typically based on the following pipeline: First, fault localization techniques~\cite{wong2016fl, abreu2007ochiai, zhang2013injecting, papadakis2015metallaxis, li2019deepfl, li2017transforming} are applied to determine the suspicious locations in programs where bugs are likely to exist. Then, the buggy locations are used by the \apr tools to generate a list of patches that replace buggy lines with correct lines. Afterward, each patch is validated against the original test suite to identify any \emph{plausible patches} (i.e., passing all tests in the test suite). Finally, to determine the \emph{correct patches}, developers examine the list of plausible patches to see if any of them can correctly fix the bug. 

Traditional \apr tools can mainly be categorized into heuristic-based~\cite{legoues2012genprog, le2016hdrepair, wen2018capgen}, constraint-based~\cite{mechtaev2016angelix, le2017s3, demacro2014nopol, long2015spr} and \template~\cite{ghanbari2019prapr, hua2018sketchfix, martinez2016astor, liu2019tbar, liu2019avatar}. Among these traditional tools, \template \apr tools~\cite{ghanbari2019prapr, liu2019tbar, benton2020effectiveness} have been able to achieve state-of-the-art results. \Template \apr tools typically leverage pre-defined templates (e.g., adding a nullness check) for bug fixing. However, since these fix templates are typically handcrafted, the number and types of bugs they are able to fix can be limited. 



To address the limitations of traditional \apr, researchers have proposed various \learning \apr tools~\cite{li2020dlfix, chen2018sequencer, jiang2021cure, lutellier2020coconut, zhu2021recoder, ye2022rewardrepair} based on the \nmtfull (\nmt) architecture~\cite{sutskever2014mt} where the input is the buggy code snippets and the goal is to translate the buggy code snippets into a fixed version. To accomplish this, \learning \apr tools require supervised training datasets with pairs of both buggy and fixed code snippets in order to learn how to perform this translation step. These training data are usually obtained by mining historical bug fixes using heuristics/keywords~\cite{dallmeier2007benchmark}, which can be imprecise for identifying bug-fixing commits; even the actual bug-fixing commits can include irrelevant code changes, leading to further pollution in the dataset~\cite{xia2022alpharepair}.
% 
Moreover, it can be hard for such \apr tools to generalize and fix bug types unseen during training. 



To better leverage recent advances in \plmfull{s} (\plm{s}), researchers~\cite{xia2022alpharepair, xia2023repairstudy, kolak2022patch, prenner2021codexws} have directly applied \plm{s} to generate patches without bug-fixing datasets. These \llm-based \apr tools work by either directly generating a complete code function~\cite{prenner2021codexws, xia2023repairstudy} or predict/infill the correct code snippet given its surrounding context~\cite{xia2022alpharepair, xia2023repairstudy}. By directly using \llm{s} that are pre-trained on billions of open-source code snippets, \llm-based \apr tools can achieve state-of-the-art performance on many repair datasets~\cite{xia2022alpharepair}. 


% 
%
%

Traditional \apr tools have long used the insight of the \emph{plastic surgery hypothesis}~\cite{barr2014plastic} where it states that the code ingredients to fix a bug already exist within the same project. Traditional \apr tools have manually designed pattern-~\cite{ghanbari2019prapr, saha2017elixir} or heuristic-based~\cite{jiang2018simfix, legoues2012genprog} approaches to finding and using such relevant code ingredients to generate fixes for bugs. However, the plastic surgery hypothesis has been largely ignored in \llm-based \apr. In fact, \llm provides a unique opportunity to fully automate the plastic surgery hypothesis idea via fine-tuning (learning project-specific information via model updates from the buggy project) and prompting (directly providing relevant code ingredients to the model), and make it directly applicable to different languages (since the \llm{s} are typically multi-lingual).%
Moreover, despite the intensive manual efforts involved, traditional \apr tools still cannot fully leverage project-specific information due to large search space for leveraging/composing existing code ingredients. In contrast, the project-specific information can effectively leveraged by \llm{s} due to their power in code understanding/vectorization, e.g., even partial/imprecise information may still guide \llm{s} in correct patch generation!
 To this end, we ask the question: \emph{How useful is the plastic surgery hypothesis in the era of \plm{s}}?








\mypara{Our Work.} To answer the question, we present \ourtech{\xspace} -- a \llm-based approach that automatically utilizes the plastic surgery hypothesis by systematically combining multiple fine-tuning and prompting strategies for \apr. \ourtech fine-tunes \plm{s} using two novel domain-specific training strategies: \textbf{\epfinetune} -- we fine-tune using the original buggy project by aggressively masking out a high percentage of tokens, which allows \plm to learn project-specific code tokens and programming styles; and \textbf{\rofinetune} -- which only masks out a single continuous code sequence per training sample, allowing the model to get used to the final \csapr task of predicting a single continuous code sequence. Furthermore, we directly leverage the ability for \plm{s} to understand natural language instructions and introduce a novel prompting strategy, \textbf{\idprompting}, which uses information retrieval and static analysis to obtain a list of relevant identifiers for the buggy lines. While such relevant identifiers are critical for fixing some difficult bugs, they may not be seen by the \llm during inference due to limited context window size. Through the use of prompting, we directly tell the model to use these extracted identifiers (relevant code ingredients) to generate the correct code. Finally, to perform repair, we combine all four model variants (including the base model, both fine-tuned models and the base model with prompting) for the final repair.





While our insight of leveraging the plastic surgery hypothesis for \llm-based \apr is generalizable across different types of \plm{s}, to implement \ourtech, we choose a recent \plm{\xspace}, \ctfive~\cite{wang2021codet5}, which is pre-trained on millions of open-source code snippets. \ctfive is an encoder-decoder model trained using \mspfull (\msp) objective where a percentage of tokens are masked out and each continuous masked token sequence is referred to as a masked span. Also, although we only extract relevant identifiers from the current buggy project (since this paper focuses on the plastic surgery hypothesis), our work can be easily extended to obtain other code information (such as relevant statements or functions) from other sources, such as  the massive pre-training corpora~\cite{husain2020codesearchnet} or historical bug-fixing datasets~\cite{jiang2019infer}, which can provide more coding knowledge for \llm{s}. Besides, although we mainly focus on using traditional string comparison algorithms for information retrieval in this paper, these techniques can be easily replaced by other frequency-based retrieval~\cite{robertson2009probabilistic} and neural search (or embedding-based search)~\cite{reimers2019sentence}.
  In summary, this paper makes the following contributions:


%


\begin{itemize}[noitemsep, leftmargin=*, topsep=0pt]
    \item \textbf{Dimension.} This paper is the first to revisit the important plastic surgery hypothesis in the era of \llm{s}. It opens up a new dimension for \llm-based \apr to incorporate previously neglected information from the buggy project itself to boost \apr performance. Furthermore, it demonstrates the promising future of retrieval-based prompting for modern \llm-based \apr.
    \item \textbf{Implementation.} We implement \ourtech based on the recent \ctfive model. We augment the model using two novel fine-tuning strategies: \epfinetune and \rofinetune, along with a novel prompting strategy based on information retrieval and static analysis: \idprompting. We combine the patches generated by all four models together and perform patch ranking to speed up \apr.% 
    \item \textbf{Evaluation Study.} We conduct an extensive evaluation against state-of-the-art \apr tools. On the widely studied \dfj 1.2 and 2.0 datasets~\cite{just2014dfj}, \ourtech is able to achieve the new state-of-the-art results of 89 and 44 correct bug fixes (15 and 8 more than best baseline) respectively.  Furthermore, we perform a broad ablation study to justify our design. \ourtech demonstrates for the first time that the plastic surgery hypothesis can substantially boost \llm-based \apr and advance state-of-the-art \apr, while being fully automated and general. Moreover, even partial/imprecise code ingredients may still effectively guide \llm{s} for \apr!
\end{itemize}


\section{Related work}
\noindent \textbf{Video foundation models.}
With sufficient computational power and an abundant source of data, there have been attempts to build a single large-scale foundation model that can be adapted to diverse downstream tasks.
Along with the success of foundations models in the natural language processing domain~\cite{brown2020language,chen2021evaluating,devlin2019bert} and in computer vision~\cite{bertasius2021space,jia2021scaling,radford2021learning}, video data has become another data type of interest, as it has grown in scale due to numerous internet video-sharing platforms.
Accordingly, several methods to train a video foundation model have been proposed.
Due to the innate multi-modality of video data, \textit{i.e.}, a combination of visual $\cdot$ vocal $\cdot$ textual context, most works have centered around the variations of the cross-modal attention mechanism \cite{akbari2021vatt,bertasius2021space,gabeur2020multi,luo2020univl,neimark2021video,tan2021look,wei2020multi,yang2021taco}.
In addition, as most video data lack proper labels or descriptions, contrastive learning methods were studied to learn meaningful feature representations or enhance video-text alignment in a self-supervised manner \cite{akbari2021vatt,kuang2021video,luo2020univl,yang2021taco}.

More specifically, MERLOT \cite{zellers2021merlot} proposed a multi-modal representation learning method for visual commonsense reasoning, which also performed well in twelve video reasoning tasks.
VATT \cite{akbari2021vatt} introduced a multi-modal learning method via contrastive learning. 
The pre-trained model performed well in a variety of vision tasks from image classification to video action recognition and zero-shot video retrieval.
Another representative work, UniVL \cite{luo2020univl} proposed a straightforward pre-training method with auxiliary loss functions. 
After fine-tuning on a specific task, the pre-trained model performed outstandingly in a wide range of tasks of text-to-video retrieval, action segmentation, action step localization, video sentiment analysis, and video captioning.
Other foundation models for multiple video tasks include \cite{li2020hero,sun2019learning,sun2019videobert,zhu2020actbert,fu2021violet,wang2022all}. 

\noindent \textbf{Auxiliary learning.}
In order to enhance the performance of one or a multitude of primary tasks, auxiliary learning methods can be incorporated.
\cite{ruder2017overview} introduced Multi-task learning (MTL) to the deep neural networks by training a single model with multiple task losses to assist learning on the main task.
Such a method is generally adapted to pre-train the foundation models in the self-supervised manner~\cite{li2020hero,sun2019learning,sun2019videobert,zhu2020actbert,fu2021violet,wang2022all}.
However, these various pretext task losses used in the pre-training phase are ignored in the fine-tuning phase, and only the primary task loss is minimized.

Recently, meta-learning methods have been introduced for auxiliary learning.
\cite{liu2019self,navon2020auxiliary,shu2019meta} proposed a meta-learning method in which the model learns auxiliary tasks to generalize well to unseen data. 
In these settings, a separate subset of data is held out as the primary task, while the others are used as auxiliary tasks that aid the primary task's performance.
Similar methods were adopted for computer vision tasks such as semantic segmentation \cite{xu2021leveraging}.
Other domain applications include navigation tasks with reinforcement learning \cite{ye2021auxiliary}, or self-supervised learning methods on graph data \cite{hwang2020self}.

\subsection{Approach}
This paper describes three concepts with the aim 
of a robust, energy-efficient robot control. 
While these concepts are rather straightforward and intuitive, they 
are not yet utilised in mainstream manipulator control.
It is not argued that all of these principles 
need to be used, but if the (sub)task allows it, using any 
of these principles can have a positive impact on the energy-efficiency.

\vspace{-4mm}
\subsubsection{Contextual prior knowledge:}
When humans perform a transportation task, they do  
not perform strict PTP motions such as traditional industrial 
robots. Instead, movements with a certain tolerance on the position 
are performed. 
This allows the natural dynamics of the system to be exploited, 
as will be explained in the following subsection. Typically, the 
tolerances come from 
knowledge about both the environment and the task context. For example, 
the spatial constraints, 
fragility of the payload, 
if a certain part of the task requires a higher precision, 
etc. 
It is clear that this knowledge precedes the task execution and 
determines how the human will perform the task.
The execution is generally done in multiple states, e.g., picking up 
the payload, moving and placing near the target position, making small 
adjustments when necessary.

This knowledge is used to split up the task in multiple 
subtasks and identify the different requirements. Robust 
controllers and monitors are then developed to perform and coordinate 
between these subtasks. 
Examples of such requirements are crane like operations such as:
 lifting the load to a certain 
height, transporting it without colliding, and lowering the load 
until contact is made.

The task also does not require high control precision throughout, 
but only for the initial grasping and final placement. 
In addition, this does not need to come only from the 
control.
Geometric constraints such as the environment or a previously placed
payload can be used to achieve this accuracy by sliding against them. 
This is further explained in section \ref{sec:discrete_control}.

By using this knowledge, lower-cost (and often also lower-weight) 
hardware can be used, so that a more robust, 
energy efficient execution can be developed. 
Thus, for a repetitive task, the cost of 
designing and implementing a task-specific controller is not 
necessarily higher than a generic, less energy-efficient controller.\\

\vspace{-8mm}
\subsubsection{Exploiting natural dynamics}
In this work, the natural dynamics of the system are used to inject 
as little energy as possible, resulting in energy-efficient 
motions.
However, precise control of the timing is lost when the system freely
follows its natural dynamics.

Due to the layout of the used cable robot (Fig.1), when the end effector is
in a fully constrained position, releasing the power of one (or more) 
of the motors, will result in a pendulum-like swing around the 
cables that are still powered, or braked. 
This swing is used in the control strategy to cover the horizontal 
distance while consuming a minimal amount of energy. 

\vspace{-4mm}
\subsubsection{Active use of brakes}
Based on the context, certain subtasks may occur where a joint 
does not need to move. Instead of producing a constant standstill
torque, it can also be opted to brake the joint.
Another case occurs when the demanded motion is in line with 
external forces such as gravity. In case of a continuous brake,
the brake force can be directly controlled to achieve a certain 
resulting force. 
With a discrete brake, a tolerance region can be determined between  
which the brake switches on-and-off to achieve a similar effect.
Section \ref{sec:continuous_control} utilises this concept to drop the
payload without driving the motor. The brakes can also be used to stop 
the natural dynamics, if necessary.
\section{Experimental Results}
\label{sec:experiments}
\subsection{Training Details}
\cite{Kalantari2017DeepHD} provides the first dataset specifically designed for multi-exposure HDR fusion under large motion. It consists of 74 training sets, which we use to supervise the training of our model. We crop the input images to patches of size \(256 \times 256\) at a step size of 64. This totally generates 20128 training samples. To augment training samples, we randomly rotate and flip the training images. The training adopts Adam optimizer. The learning rate is initialized to \(10^{-4}\) and is reduced to \(10^{-5}\) after 20 epochs. It is observed that 40 epochs are sufficient for the training to converge.    

\subsection{Numerical Evaluation}
We numerically measure the performance of our method on the 15 test sets of \cite{Kalantari2017DeepHD}, by Peak Signal-to-Noise Ratio (PSNR) and Structure Similarity, computed in both tonemapping domain (-\(\mu\)) and HDR linear domain (-L). Visual difference metric HDR-VDP-2 is also adopted, where the parameters are set as same as in previous works \cite{wu2018end} and \cite{niu2021hdrgan}. 

Table \ref{table_metrics} compares our model with state-of-the-art models. For \cite{yan2020nonlocal} and \cite{xiong2021hierarchical}, we use the results reported in the publications. Note that \cite{sen2012robust} and \cite{hu2013hdr} are not machine learning based methods. Moreover,  \cite{Kalantari2017DeepHD} and \cite{wu2018end} apply optical flow and homography transformation to preprocess the input images respectively, and hence entail extra computation. 

Table \ref{table_metrics} shows that our method outperforms competing method in terms of PSNR-L, SSIM-$\mu$, SSIM-L and HDR-VDP-2. It ranks the second best in PSNR-$\mu$, being slightly (0.1dB) inferior to \cite{xiong2021hierarchical}. Note that \cite{xiong2021hierarchical} utilizes a pretrained model to detect ghosting regions for training, whereas our method does not require any pretrained model. The high PSNR and SSIM scores varify that our model has strong HDR reconstruction ability and can accurately restore the radiance and structure of the scene in both tonemapping domain and HDR linear domain. Furthermore, its high performance in term of HDR-VDP-2\cite{mantiuk2011hdr} performance indicates that our method can generate HDR image visually close to the target image.

\begin{table*}[ht]
\centering
\begin{tabular}{l|c|c|c|c|c}
\hline
& PSNR-$\mu$ & PSNR-L & SSIM-$\mu$ & SSIM-L & HDR-VDP-2 \\
\hline
\bfseries Sen & 40.97 & 38.36 & 0.9830 & 0.9746 & 60.60\\
\hline
\bfseries Hu  & 35.65 & 30.80 & 0.9725 & 0.9491 & 58.34\\
\hline
\bfseries Kalantari & 42.69 & 41.22 & 0.9888 & 0.9845 & 65.05\\
\hline
\bfseries DeepHDR& 41.99 & 41.22 & 0.9878 & 0.9859 & \underline{65.91}\\
\hline
\bfseries AHDR & 43.62 & 41.03 & 0.9900  &\underline{0.9883} & 63.85 \\
\hline 
\bfseries NHDRRNet& 42.414 & - & 0.9887 & - & 61.21 \\
\hline 
\bfseries HDR-GAN &43.92 & \underline{41.57} &\underline{0.9905} &0.9865 & 65.45\\
\hline 
\bfseries HFNet & \textbf{44.28} & 41.47 & - & - & - \\
\hline 
\bfseries Ours & \underline{44.18} & \textbf{42.19}&\textbf{0.9912} & \textbf{0.9883}& \textbf{67.07} \\
\hline
\end{tabular}
\caption{Numerical performance of the proposed model, evaluated on the dataset by Kalantari-Ramamoorthi. The best and second best results for each metric are marked in \textbf{bold} and \underline{underlined}, respectively}
\label{table_metrics}
\end{table*}

\subsection{Visual Performance Evaluation}

\begin{figure*}[!htb]
\centering
\includegraphics[width=\textwidth]{experiments/kalantari_test.png}
\caption{Visual comparison on the test set of Kalantari-Ramamoorthi dataset. Zoom-in views of reconstruction by each method are presented on the saturated regions that contain moving objects. Our network built with gated Swin Transformer yields noticeably better visual results than other methods.}
\label{fig_kalantari_test}
\end{figure*}
Fig. \ref{fig_kalantari_test} present the visual performance of our method and comparable methods on two examples from \cite{Kalantari2017DeepHD}. We present the zoom-in views of two challenging cases, where large saturated regions contain substantial non-rigid motion in the reference image. The two patch-based methods do not reconstruct the missing details in the saturated regions, as they heavily rely on the details provided by the reference image for registration. Image reconstructed by the optical flow based method \cite{Kalantari2017DeepHD} suffers motion blur artifacts. This is because the convolutions of DeepHDR and HDR-GAN have limited receptive fields, and hence are hampered to repair missing content in misaligned regions by aligned regions. The gating mechanism of AHDR is only applied to low-level features, so the high-level outliers may deteriorate the HDR fusion. In contrast to comparable methods, our model remarkably overcomes the ghosting artifacts.

\begin{figure}[ht]
\centering
\includegraphics[width=\columnwidth]{experiments/sen_test.pdf}
\caption{Visual performance comparison on example images from the dataset by Sen et al. Zoom in views on challenging areas are presented. Although the ground truth is unavailable, it can be clearly observed that our method visually performs better than comparable methods.}
\label{sen_test}
\end{figure}

\begin{figure}[ht]
\centering
\includegraphics[width=\columnwidth]{experiments/tursun_test.pdf}
\caption{Visual performance comparison on example images from the dataset by Tursun et al. Compared to state of the art methods, our method suffers less ghosting artifact.}
\label{tursun_test}
\end{figure}

Fig.\ref{sen_test} and Fig.\ref{tursun_test} present visual performance of our method on two examples from benchmark datasets \cite{sen2012robust} and \cite{tursun2016objective}. As these test datasets   do not provide ground truth image. we mark the visual difference on the results generated by different methods. It can be seen that our method suffers less artifacts than other methods in various scenes with various motion patterns, achieving better visual results. Our method creates high-quality HDR more robustly and generalizes well. 

\subsection{Ablation Study}

\begin{table}[h]
\centering
\resizebox{\columnwidth}{!}{
\begin{tabular}{l|c|c|c|c|c}
\hline
                         & PSNR-$\mu$ & PSNR-l & SSIM-$\mu$ & SSIM-l & HDR-VDP-2 \\ \hline
restormer(w/o ssim loss) & 44.00  & 41.5   & 0.9906 & 0.9873 & 64.72  \\ \hline
Ours(w/o ssim loss)      & 44.07  & 41.83  & 0.9909 & 0.9879 &  64.78  \\ \hline
Ours                     & 44.18  & 42.19  & 0.9912 & 0.9883 & 67.07      \\ \hline
\end{tabular}
}
\caption{Experimental results of ablation study. We compare using Gated Swin Transformer v.s. Gated Transformer, and the combined loss function v.s. the traditional $l_{1}$ norm loss function.}
\label{table_ablation_block_loss}
\end{table}

We verify various components of our method, including Swin Transformer, loss function, and gating mechanism by ablation study.

\subsubsection{Ablation Study on Block Design}
Our model has similar architecture to Restormer, which uses modified Transformer, whereas we use modified Swin Transformer as the building unit. For comparison, we replace the residual modules in each block in our model with multiple transformer layers as in Restormer, with same number of transformer layers. Table \ref{table_ablation_block_loss} presents the results, which show that using Swin Transformer achieves superior performance in all measures. The reason is that the attention module of Restormer is computed channel-wise, but forgoes the cross-exposure spatial dependency to repair the non-aligned area. 

\subsubsection{Ablation Study on Loss Function}
We trained our model under different loss function configurations, as shown in \ref{table_ablation_block_loss}. The results validate that the SSIM loss benefits detail reconstruction.

\subsubsection{Ablation Study on Gating Mechanism}
\begin{table}[h]
\resizebox{\columnwidth}{!}{
\begin{tabular}{l|c|c|c|c|c}
\hline
           & PSNR-$\mu$ & PSNR-l & SSIM-$\mu$ & SSIM-l & HDR-VDP-2 \\ \hline
w/o gating & 43.14  & 41.03  & 0.9904 & 0.9868 &     64.88      \\ \hline
one gating & 43.44  & 41.42  & 0.9909 & 0.9882 &    67.13   \\ \hline
Ours       & 43.61  & 41.74  & 0.9909 & 0.9881 & 66.96     \\ \hline
\end{tabular}
}
\caption{Ablation experimental results to verify the effectiveness of the gating mechanism}
\label{table_ablation_gating}
\end{table}

The gating mechanism is an important component in our model. Ablation study is conducted in the gating mechanism as follows.

\textbf{w/o gating}: The gating mechanism is not used in the feed forward network of all transformer layers in the model, that it, our GST unit degenerate to the vanilla Swin Transformer.

\textbf{one gating}: The gating mechanism is only used in the first Swin Transformer layers subsequent to the embedding layer, but not used for other layers. 

 Table \ref{table_ablation_gating} shows the results of the ablation experiments, where the model is trained for 20 epochs. By removing the gating mechanism, the network relies on self-attention for image alignment, resulting in the lowest performance. On top of it, adding gates to low level layers notably improves the HDR reconstruction. Furthermore, by integrating the gating mechanism with all Swin Transformer layers, the model effectively inpaints information in non-aligned regions and obtains the highest HDR reconstruction results, thus validates the effectiveness of the gating mechanism in our model.

\section{Conclusion}\label{sec:conclusion}
In this work, we focus on addressing the fundamental challenge of OOD detection tasks, which is how to fully understand the semantic discrepancy between the ID/OOD samples. We reveal that the key to success in the realistic SCOOD task is to allocate as many ID samples in the unlabeled set correctly as possible. To this end, we propose a novel uncertainty-aware optimal transport scheme that introduces class-specific energy scores as guidance for effective label assignment. Experimental results show that our method achieves better performance than previous state-of-the-art methods on SCOOD benchmarks.

\textbf{Limitations.} In addition to temperature scaling, other techniques such as feature clipping applied in ReAct~\cite{sun2021react} also enhance the performance of energy score, so how to obtain an OOD score that best fits the SCOOD task can be further explored. Moreover, a setting highly related to SCOOD has been proposed in \cite{katz2022training} and formulated as a constrained optimization problem. We will also theoretically analyze these practical OOD settings in our feature work.

% \section*{Acknowledgments}
\textbf{Acknowledgments.} 
This work is supported by National Key R\&D Program of China under Grant 2020AAA0105701, National Natural Science Foundation of China (NSFC) under Grants 61872327, Major Special Science and Technology Project of Anhui, National Natural Science Foundation of China (62033012) and Ant Group through Ant Research Intern Program.



{\small
\bibliographystyle{ieee_fullname}
\bibliography{egbib}
}

\clearpage
%\newpage
%\cleardoublepage
\appendix
In this supplemental material, we provide details for our implementation in Sec.~\ref{SecImple}, dataset pre-processing and text prompt generation in Sec.~\ref{SecData}, baseline implementations in Sec.~\ref{SecBaseline}, additional results in Sec.~\ref{SecAddRes}, and user studies in Sec.~\ref{SecUser}.

\section{Implementations}
\label{SecImple}

\subsection{Shape Auto-Encoder}
\label{SubSecShapeAE}

We adopt a pre-trained shape auto-encoder to extract a set of latent shape codes for CAD models from the 3D-FUTURE~\cite{fu20213dm} dataset. The network architecture of the shape auto-encoder is shown in Fig.~\ref{fig:shapeae}. It is a variational auto-encoder, similar to FoldingNet~\cite{yang2018foldingnet}.
Specifically, a point cloud $\mathbf{P}_{in}$ of size 2,048 is fed into a graph encoder based on PointNet~\cite{qi2017pointnet} with graph convolutions~\cite{wang2019dynamic} to extract a global latent code of dimension 512, which is used to predict the mean $\mathbf{\mu}$ and variance $\mathbf{\sigma}$ of a low-dimensional latent space of size 64.
Subsequently, a compressed latent is sampled from $\mathcal{N}(\mathbf{\mu}, \mathbf{\sigma})$.
%\TODO{maybe stupid question, but what is reparametrization sampling. we should explain that}
Finally, the compressed latent is mapped back to the original space and passed to the FoldingNet decoder to recover a point cloud $\mathbf{P}_{rec}$ of size 2,025.
The used training objective is a weighted combination of Chamfer distance (\ie CD) and KL divergence.
\begin{equation}
    \label{EquaShapeAE}
    L_{vae} = \CD(\mathbf{P}_{in}, \mathbf{P}_{rec}) + \omega_{kl} *\KL(\mathcal{N}(\mathbf{\mu}, \mathbf{\sigma}) || \mathcal{N}(\mathbf{0}, \mathbf{I})) ,
\end{equation}
where $\omega_{kl}$ is set to 0.001.
The latent compression and KL regularization leads to a compact and structured latent space, focusing on global shape structures.
The shape autoencoder is trained on a single RTX 2080 with a batch size of 16 for 1,000 epochs.
The learning rate is initialized to $lr=\expnumber{1}{-4}$ and then gradually decreases with the decay rate of 0.1 in every 400 epochs.
\begin{figure}
    \centering
    \includegraphics[width=\linewidth]{./figs/shapeautoencoder.pdf}
    \caption{\textbf{Shape Auto-encoder.}}
    \label{fig:shapeae}
\end{figure}

\subsection{Shape Code Diffusion}
\label{SubSecShapeDiffu}

We use the extracted latent codes to train shape code diffusion.
While we apply KL regularization, the value range of latent codes is still unbound.
To make it easier to diffuse, we scale the latent codes to $[-1, 1]$ by using the statistical minimum and maximum feature values over the whole set.
During inference, we rescale generated shape codes.

\subsection{Shape Retrieval}
\label{SubSecRetrieval}

During inference, we use shape retrieval as the post-processing procedure to acquire object surface geometries for generated scene graphs.
Concretely, for each graph node, we perform the nearest neighbor search in the 3D-FUTURE~\cite{fu20213dm} dataset to find the CAD model with the same class label, the closest bounding box size, and the closest geometry feature.
Previous works~\cite{wang2021sceneformer, paschalidou2021atiss} only use object semantics and bounding box sizes during shape retrieval, we consider the similarity of geometry descriptors. Thus, our method can retrieve more accurate shape geometries. After the object retrieval, we place the retrieved CAD models into the scene based on the predicted locations, orientation angles, and sizes.  

% \subsection{Loss function}
% \label{SubSecLoss}

\section{Dataset}
\label{SecData}

\paragraph{Preprocessing}
The dataset preprocessing is based on the setting of ATISS~\cite{paschalidou2021atiss}.
We start by filtering out those scenes with problematic object arrangements such as severe object intersections or incorrect object class labels, e.g., beds are misclassified as wardrobes in some scenes.
Then, we remove those scenes with unnatural sizes. The floor size of a natural room is within $6m \times 6m$ and its height is less than $4m$. Subsequently, we ignore scenes that have too few or many objects.
The number of objects in valid bedrooms is between 3 and 13. As for dining and living rooms, the minimum and maximum numbers are set to 3 and 21 respectively. Thus, the number of scene graph nodes is $N=13$ in bedrooms and $N=21$ in dining and living rooms. In addition, we delete scenes that have objects out of pre-defined categories. After pre-processing, we obtained 4,041 bedrooms, 900 dining rooms, and 813 living rooms.

For the semantic class diffusion, we have an additional class of  `empty' to define the existence of an object. Combining with the object categories that appeared in each room type, we have $L=22$ object categories for bedrooms, and
$L=25$ object categories for dining and living rooms in total. The category labels 
are listed as follows.

\begin{python}
# 22 3D-Front bedroom categories
['empty', 'armchair', 'bookshelf', 'cabinet',
'ceiling_lamp', 'chair', 'children_cabinet',
'coffee_table', 'desk', 'double_bed',
'dressing_chair', 'dressing_table', 'kids_bed',
'nightstand', 'pendant_lamp', 'shelf',
'single_bed', 'sofa', 'stool', 'table',
'tv_stand', 'wardrobe']

# 25 3D-Front dining or living room categories
['empty', 'armchair', 'bookshelf', 'cabinet', 
'ceiling_lamp', 'chaise_longue_sofa', 
'chinese_chair', 'coffee_table', 'console_table',  
'corner_side_table',  'desk', 'dining_chair', 
'dining_table', 'l_shaped_sofa', 'lazy_sofa', 
'lounge_chair', 'loveseat_sofa', 
'multi_seat_sofa', 'pendant_lamp', 
'round_end_table', 'shelf', 'stool', 
'tv_stand', 'wardrobe', 'wine_cabinet']
\end{python}

\paragraph{Text Prompt Generation}
We follow the SceneFormer~\cite{wang2021sceneformer} to generate text prompts describing partial scene configurations. Each text prompt contains one to three sentences. We explain the details of text formulation process by using the text prompt 'The room has a dining table, a pendant lamp, and a lounge chair. The pendant lamp is above the dining table. There is a stool to the right of the lounge chair.` as an example. First, we randomly select three objects from a scene, get their class labels, and then count the number of appearances of each selected object category. As such, we can get the first sentence. Then, we find all valid object pairs associated with the selected three objects. An object pair is valid only if the distance between two objects is less than a certain threshold that is set to 1.5 in our method. Next, we calculate the relative orientations and translations, from which we can determine the relationship type of the valid object pair from the candidate pool: 'is above to`, 'is next to`, 'is left of`, 'is right of`, ' surrounding`, 'inside`, 'behind`, 'in front of`, and 'on`. In this way, we can acquire some relation-describing sentences like the second and third sentences in the example. Finally, we randomly sampled zero to two relation-describing sentences.

\section{Baselines}
\label{SecBaseline}

\paragraph{DepthGAN} 
DepthGAN~\cite{yang2021indoor} adopts a generative adversary network to train 3D scene synthesis using both semantic maps and depth images. The generator network is built with 3D convolution layers, which decode a volumetric scene with semantic labels. A differentiable projection layer is applied to project the semantic scene volume into depth images and semantic maps under different views, where a multi-view discriminator is designed to distinguish the synthesized views from ground-truth semantic maps and depth images during the adversarial training.


\paragraph{Sync2Gen} 
Sync2Gen~\cite{yang2021scene} represents a scene arrangement as a sequence of 3D objects characterized by different attributes (e.g., bounding box, class category, shape code). The generative ability of their method relies on a variational auto-encoder network, where they learn objects' relative attributes. Besides, a Bayesian optimization stage is used as a post-processing step to refine object arrangements based on the learned relative attribute priors.

\paragraph{ATISS}
ATISS~\cite{paschalidou2021atiss} considers a scene as an unordered set of objects and then designs a novel autoregressive transformer architecture to model the scene synthesis process. During training, based on the previously known object attributes, ATISS utilizes a permutation-invariant transformer to aggregate their features and  predicts the location, size, orientation, and class category of the next possible object conditioned on the fused feature. 
The original version of ATISS~\cite{paschalidou2021atiss} is conditioned on a 2D room mask from the top-down orthographic projection of the 3D floor plane of a scene. To ensure fair comparisons, we train an unconditional ATISS without using a 2D room mask as input, following the same training strategies and hyperparameters as the original ATISS.


% \section{Evaluation Metrics}
% \label{SecEval}

% \paragraph{Fr{\'e}chet Inception Distance}

% \paragraph{Kernel Inception Distance}

% \paragraph{Scene Classification Accuracy}

% \paragraph{Category KL Divergenece}



\section{Additional Results}
\label{SecAddRes}

\paragraph{Unconditional Scene Synthesis}
\begin{figure*}[!htbp]
	\centering
 	\begin{subfigure}[t]{0.23\textwidth}
		\includegraphics[width=\textwidth]
	{figs/experiments/uncond_gallery/SecondBedroom-35821_15_393.jpg}
 
  	\includegraphics[width=\textwidth]{./figs/experiments/uncond_gallery/LivingRoom-41893_126_445.jpg}
   
		\includegraphics[width=\textwidth]{./figs/experiments/uncond_gallery/LivingDiningRoom-86944_84_103.jpg}

            \includegraphics[width=\textwidth]{./figs/experiments/unconditional/ours/living/LivingRoom-71071_8_074.jpg}
    
	\end{subfigure}
	\rulesep
        %
	\begin{subfigure}[t]{0.23\textwidth}
		\includegraphics[width=\textwidth]
	{figs/experiments/uncond_gallery/SecondBedroom-52584_24_964.jpg}
 
  	\includegraphics[width=\textwidth]{./figs/experiments/uncond_gallery/LivingRoom-50084_76_718.jpg}
   
		\includegraphics[width=\textwidth]{./figs/experiments/uncond_gallery/LivingDiningRoom-99518_174_183.jpg}

            \includegraphics[width=\textwidth]{./figs/experiments/uncond_gallery/LivingDiningRoom-163914_165_492.jpg}

	\end{subfigure}
	\rulesep
        %
        \begin{subfigure}[t]{0.23\textwidth}
		\includegraphics[width=\textwidth]
	{figs/experiments/uncond_gallery/SecondBedroom-86888_75_920.jpg}
 
  	\includegraphics[width=\textwidth]{./figs/experiments/uncond_gallery/LivingRoom-68491_81_650.jpg}
   
		\includegraphics[width=\textwidth]{./figs/experiments/uncond_gallery/LivingDiningRoom-109935_48_096.jpg}

        \includegraphics[width=\textwidth]{./figs/experiments/uncond_gallery/LivingRoom-71071_8_997.jpg}
        
	\end{subfigure}
	\rulesep
         %
	\begin{subfigure}[t]{0.23\textwidth}
		\includegraphics[width=\textwidth]
	{figs/experiments/uncond_gallery/SecondBedroom-258160_63_131.jpg}
 
  	\includegraphics[width=\textwidth]{./figs/experiments/uncond_gallery/LivingRoom-71071_8_485.jpg}
   
		\includegraphics[width=\textwidth]{./figs/experiments/uncond_gallery/LivingDiningRoom-126918_10_090.jpg}

  \includegraphics[width=\textwidth]{./figs/experiments/uncond_gallery/LivingRoom-88425_55_025.jpg}
	\end{subfigure}
	\caption{Diverse and plausible results of unconditional scene synthesis from our method. }
\label{fig:uncond_gallery}
%\vspace{2mm}
\end{figure*}

 In Fig.~\ref{fig:uncond_gallery}, we provide more visualization results of our unconditional scene synthesis model. 

\paragraph{Scene Completion}
\begin{figure*}[t]
    %\vspace{-2mm}
	\centering
	\begin{subfigure}[t]{0.14\textwidth}
            \includegraphics[width=\textwidth]{./figs/experiments/scene_completion_supple/partial/Bedroom-15797_117_075.jpg}
            \includegraphics[width=\textwidth]{./figs/experiments/scene_completion_supple/partial/Bedroom-17102_150_930.jpg}
            \includegraphics[width=\textwidth]{./figs/experiments/scene_completion_supple/partial/LivingDiningRoom-233_45_129.jpg}
            \includegraphics[width=\textwidth]{./figs/experiments/scene_completion_supple/partial/LivingDiningRoom-69704_153_931.jpg}
        \caption{Partial Scenes}
	\end{subfigure}
        \rulesep
        %
        \begin{subfigure}[t]{0.41\textwidth}
    	\includegraphics[width=0.33\textwidth]{./figs/experiments/scene_completion_supple/atiss/Bedroom-15797_075.jpg}%
            \hfill
    	\includegraphics[width=0.33\textwidth]{./figs/experiments/scene_completion_supple/atiss/Bedroom-15797_344.jpg}%
            \hfill
      	\includegraphics[width=0.33\textwidth]{./figs/experiments/scene_completion_supple/atiss/Bedroom-15797_439.jpg} 
       %%%%%%
    	\includegraphics[width=0.33\textwidth]{./figs/experiments/scene_completion_supple/atiss/Bedroom-17102_180.jpg}%
            \hfill
        \includegraphics[width=0.33\textwidth]{./figs/experiments/scene_completion_supple/atiss/Bedroom-17102_713.jpg}%
        \hfill
        \includegraphics[width=0.33\textwidth]{./figs/experiments/scene_completion_supple/atiss/Bedroom-17102_930.jpg}
        %%%%%
        \includegraphics[width=0.33\textwidth]{./figs/experiments/scene_completion_supple/atiss/LivingDiningRoom-233_45_000.jpg}%
        \hfill
        \includegraphics[width=0.33\textwidth]{./figs/experiments/scene_completion_supple/atiss/LivingDiningRoom-233_45_002.jpg}%
        \hfill
        \includegraphics[width=0.33\textwidth]{./figs/experiments/scene_completion_supple/atiss/LivingDiningRoom-233_45_003.jpg}
        %%%%%
        \includegraphics[width=0.33\textwidth]{./figs/experiments/scene_completion_supple/atiss/LivingDiningRoom-69704_153_000.jpg}%
        \hfill
        \includegraphics[width=0.33\textwidth]{./figs/experiments/scene_completion_supple/atiss/LivingDiningRoom-69704_153_001.jpg}%
        \hfill
        \includegraphics[width=0.33\textwidth]{./figs/experiments/scene_completion_supple/atiss/LivingDiningRoom-69704_153_005.jpg}
        \caption{ATISS~\cite{paschalidou2021atiss}}
	\end{subfigure}
        \rulesep
        %
	\begin{subfigure}[t]{0.41\textwidth}
    	\includegraphics[width=0.33\textwidth]{./figs/experiments/scene_completion_supple/ours/Bedroom-15797_117_010.jpg}%
            \hfill
            \includegraphics[width=0.33\textwidth]{./figs/experiments/scene_completion_supple/ours/Bedroom-15797_117_007.jpg}%
    	\hfill
    	\includegraphics[width=0.33\textwidth]{./figs/experiments/scene_completion_supple/ours/Bedroom-15797_117_015.jpg}
        %%%%%%%%%%%%%%%%%%%%%%%
    	\includegraphics[width=0.33\textwidth]{./figs/experiments/scene_completion_supple/ours/Bedroom-17102_150_004.jpg}%
            \hfill
            \includegraphics[width=0.33\textwidth]{./figs/experiments/scene_completion_supple/ours/Bedroom-17102_150_011.jpg}%
    	\hfill
    	\includegraphics[width=0.33\textwidth]{./figs/experiments/scene_completion_supple/ours/Bedroom-17102_150_013_2.jpg}
             %%%%%%%%%%%%%%%%%%%%%%%
    	\includegraphics[width=0.33\textwidth]{./figs/experiments/scene_completion_supple/ours/LivingDiningRoom-233_45_002.jpg}%
            \hfill
            \includegraphics[width=0.33\textwidth]{./figs/experiments/scene_completion_supple/ours/LivingDiningRoom-233_45_012.jpg}%
    	\hfill
    	\includegraphics[width=0.33\textwidth]{./figs/experiments/scene_completion_supple/ours/LivingDiningRoom-233_45_017.jpg}
             %%%%%%%%%%%%%%%%%%%%%%%
    	\includegraphics[width=0.33\textwidth]{./figs/experiments/scene_completion_supple/ours/LivingDiningRoom-69704_016.jpg}%
            \hfill
            \includegraphics[width=0.33\textwidth]{./figs/experiments/scene_completion_supple/ours/LivingDiningRoom-69704_019.jpg}%
    	\hfill
    	\includegraphics[width=0.33\textwidth]{./figs/experiments/scene_completion_supple/ours/LivingDiningRoom-69704_251.jpg}
		\caption{Ours}
	\end{subfigure}
	\caption{\textbf{Scene completion} from partial scenes with only three objects given as inputs. Compared to ATISS, our method produced more diverse completion results with higher fidelity.}
    \label{fig:completion_supple}
    %\vspace{-2mm}
\end{figure*}
We present more qualitative comparisons on the task of scene completion in Fig.~\ref{fig:completion_supple}.

\paragraph{Scene Arrangement}
\begin{figure*}[!ht]
	\centering
	\begin{subfigure}[t]{0.14\textwidth}
            \includegraphics[width=\textwidth]{./figs/experiments/arrangement_supple/noisy/LivingDiningRoom-270_12_012.jpg}
            \includegraphics[width=\textwidth]{./figs/experiments/arrangement_supple/noisy/LivingDiningRoom-13177_88_442.jpg}
            \includegraphics[width=\textwidth]{./figs/experiments/arrangement_supple/noisy/LivingDiningRoom-64631_169_700.jpg}
            \includegraphics[width=\textwidth]{./figs/experiments/arrangement_supple/noisy/LivingDiningRoom-79295_184_184.jpg}
        \caption{Noisy Scenes}
	\end{subfigure}
        \rulesep
        %%%%
        \begin{subfigure}[t]{0.41\textwidth}
    	\includegraphics[width=0.33\textwidth]{./figs/experiments/arrangement_supple/atiss/LivingDiningRoom-270_12_012.jpg}%
            \hfill
    	\includegraphics[width=0.33\textwidth]{./figs/experiments/arrangement_supple/atiss/LivingDiningRoom-270_12_204.jpg}%
            \hfill
      	\includegraphics[width=0.33\textwidth]{./figs/experiments/arrangement_supple/atiss/LivingDiningRoom-270_12_396.jpg} 
            %%%%%%%%%%%%%%%%%%%%%%%
    	\includegraphics[width=0.33\textwidth]{./figs/experiments/arrangement_supple/atiss/LivingDiningRoom-13177_88_088.jpg}%
            \hfill
            \includegraphics[width=0.33\textwidth]{./figs/experiments/arrangement_supple/atiss/LivingDiningRoom-13177_88_796.jpg}%
    	\hfill
            \includegraphics[width=0.33\textwidth]{./figs/experiments/arrangement_supple/atiss/LivingDiningRoom-13177_88_973.jpg}
               %%%%%%%%%%%%%%%%%%%%%%%
    	\includegraphics[width=0.33\textwidth]{./figs/experiments/arrangement_supple/atiss/LivingDiningRoom-64631_169_523.jpg}%
            \hfill
            \includegraphics[width=0.33\textwidth]{./figs/experiments/arrangement_supple/atiss/LivingDiningRoom-64631_169_700.jpg}%
    	\hfill
            \includegraphics[width=0.33\textwidth]{./figs/experiments/arrangement_supple/atiss/LivingDiningRoom-64631_169_877.jpg}
            %%%%%%%%%%%%%%%%%%%%%%%
    	\includegraphics[width=0.33\textwidth]{./figs/experiments/arrangement_supple/atiss/LivingDiningRoom-79295_184_184.jpg}%
            \hfill
            \includegraphics[width=0.33\textwidth]{./figs/experiments/arrangement_supple/atiss/LivingDiningRoom-79295_184_376.jpg}%
    	\hfill
            \includegraphics[width=0.33\textwidth]{./figs/experiments/arrangement_supple/atiss/LivingDiningRoom-79295_184_568.jpg}
        \caption{ATISS~\cite{paschalidou2021atiss}}
	\end{subfigure}
        \rulesep
        %%%
	\begin{subfigure}[t]{0.41\textwidth}
    	\includegraphics[width=0.33\textwidth]{./figs/experiments/arrangement_supple/ours/LivingDiningRoom-270_12_014.jpg}
            \hfill
            \includegraphics[width=0.33\textwidth]{./figs/experiments/arrangement_supple/ours/LivingDiningRoom-270_12_017.jpg}%
    	\hfill
    	\includegraphics[width=0.33\textwidth]{./figs/experiments/arrangement_supple/ours/LivingDiningRoom-270_12_020.jpg}
      %%%%%%%%%%%%%%%%%%%%%%%
    	\includegraphics[width=0.33\textwidth]{./figs/experiments/arrangement_supple/ours/LivingDiningRoom-13177_88_005.jpg}%
            \hfill
            \includegraphics[width=0.33\textwidth]{./figs/experiments/arrangement_supple/ours/LivingDiningRoom-13177_88_021.jpg}%
    	\hfill
    	\includegraphics[width=0.33\textwidth]{./figs/experiments/arrangement_supple/ours/LivingDiningRoom-13177_88_796.jpg}
      %%%%%%%%%%%%%%%%%%%%%%%
      \includegraphics[width=0.33\textwidth]{./figs/experiments/arrangement_supple/ours/LivingDiningRoom-64631_169_017.jpg}%
            \hfill
            \includegraphics[width=0.33\textwidth]{./figs/experiments/arrangement_supple/ours/LivingDiningRoom-64631_169_169.jpg}%
    	\hfill
    	\includegraphics[width=0.33\textwidth]{./figs/experiments/arrangement_supple/ours/LivingDiningRoom-64631_169_346.jpg}
           %%%%%%%%%%%%%%%%%%%%%%%
      \includegraphics[width=0.33\textwidth]{./figs/experiments/arrangement_supple/ours/LivingDiningRoom-79295_184_018.jpg}%
            \hfill
            \includegraphics[width=0.33\textwidth]{./figs/experiments/arrangement_supple/ours/LivingDiningRoom-79295_184_006.jpg}%
    	\hfill
    	\includegraphics[width=0.33\textwidth]{./figs/experiments/arrangement_supple/ours/LivingDiningRoom-79295_184_017.jpg}
		\caption{Ours}
	\end{subfigure}
	\caption{Scene re-arrangements of collections of random objects.  Compared to ATISS, our method generates various object placement options
        with better plausibility.}
    \label{fig:arrangement_supple}
    %\vspace{-4mm}
\end{figure*}

% \begin{figure*}[!ht]
% 	\centering
% 	\begin{subfigure}[t]{0.14\textwidth}
%             % \includegraphics[width=\textwidth]{./figs/experiments/arrangment/noisy/Bedroom-32051_73_000.jpg}
%             \includegraphics[width=\textwidth]{./figs/experiments/arrangment/noisy/LivingDiningRoom-37405_17_004.jpg}
%         \caption{Noisy Scenes}
% 	\end{subfigure}
%         \rulesep
%         \begin{subfigure}[t]{0.41\textwidth}
%     	% \includegraphics[width=0.33\textwidth]{./figs/experiments/arrangment/atiss/Bedroom-32051_73_001.jpg}%
%      %        \hfill
%     	% \includegraphics[width=0.33\textwidth]{./figs/experiments/arrangment/atiss/Bedroom-32051_73_002.jpg}%
%      %        \hfill
%      %  	\includegraphics[width=0.33\textwidth]{./figs/experiments/arrangment/atiss/Bedroom-32051_73_003.jpg} 
%     	\includegraphics[width=0.33\textwidth]{./figs/experiments/arrangment/atiss/LivingDiningRoom-37405_17_005.jpg}%
%             \hfill
%             \includegraphics[width=0.33\textwidth]{./figs/experiments/arrangment/atiss/LivingDiningRoom-37405_17_003.jpg}%
%     	\hfill
%             \includegraphics[width=0.33\textwidth]{./figs/experiments/arrangment/atiss/LivingDiningRoom-37405_17_015.jpg}
%         \caption{ATISS~\cite{paschalidou2021atiss}}
% 	\end{subfigure}
%         \rulesep
% 	\begin{subfigure}[t]{0.41\textwidth}
%     	% \includegraphics[width=0.33\textwidth]{./figs/experiments/arrangment/ours/Bedroom-32051_73_084.jpg}%
%      %        \hfill
%      %        \includegraphics[width=0.33\textwidth]{./figs/experiments/arrangment/ours/Bedroom-32051_73_105.jpg}%
%     	% \hfill
%     	% \includegraphics[width=0.33\textwidth]{./figs/experiments/arrangment/ours/Bedroom-32051_73_150.jpg}
%     	% \includegraphics[width=0.33\textwidth]{./figs/experiments/arrangment/ours/LivingDiningRoom-37405_17_002.jpg}%
%             \hfill
%             \includegraphics[width=0.33\textwidth]{./figs/experiments/arrangment/ours/LivingDiningRoom-37405_17_014.jpg}%
%     	\hfill
%     	\includegraphics[width=0.33\textwidth]{./figs/experiments/arrangment/ours/LivingDiningRoom-37405_17_024.jpg}
% 		\caption{Ours}
% 	\end{subfigure}
% 	\caption{Scene re-arrangements of collections of random objects.  \jiapeng{Compared to ATISS, our diffusion-based method generates various object placement options
%         with better plausibility.}}
%     \label{fig:arrangement}
%     \vspace{-4mm}
% \end{figure*}

\begin{figure*}[!ht]
	\centering
    	\begin{subfigure}[t]{0.23\textwidth}
            \includegraphics[width=\textwidth]{./figs/experiments/arrange_lego/noisy.jpg}
            \caption{Noisy Scenes}
    	\end{subfigure}
        \rulesep
        \begin{subfigure}[t]{0.23\textwidth}
            \includegraphics[width=\textwidth]{./figs/experiments/arrange_lego/atiss.jpg}
            \caption{ATISS~\cite{paschalidou2021atiss}}
	   \end{subfigure}
        \rulesep
    	\begin{subfigure}[t]{0.23\textwidth}
            \includegraphics[width=\textwidth]{./figs/experiments/arrange_lego/lego.jpg}
    		\caption{LEGO~\cite{wei2023lego}}
    	\end{subfigure}
     \rulesep
     \begin{subfigure}[t]{0.23\textwidth}
            \includegraphics[width=\textwidth]{./figs/experiments/arrange_lego/ours.jpg}
    		\caption{Ours}
    	\end{subfigure}
	\caption{Scene re-arrangements of collections of random objects. Compared to ATISS and LEGO, our method generates more favourable object placements with more symmetric pairs.}
    \label{fig:arrangement}
    \vspace{-4mm}
\end{figure*}

\begin{abstract}
In state estimation algorithms that use feature tracks as input, it is customary to assume that the errors in feature track positions are zero-mean Gaussian. Using a combination of calibrated camera intrinsics, ground-truth camera pose, and depth images, it is possible to compute ground-truth positions for feature tracks extracted using an image processing algorithm. We find that feature track errors are not zero-mean Gaussian and that the distribution of errors is conditional on the type of motion, the speed of motion, and the image processing algorithm used to extract the tracks.
\end{abstract}


\section{Introduction}

Many state estimation algorithms assume that measurements are zero-mean Gaussian. This is an explicit assumption in the Kalman Filter and its nonlinear variants \cite{thrun_probabilistic_2005, barrau_invariant_2018} and implicitly built-into the optimization problem of bundle adjustment algorithms \cite{mur-artal_orb-slam:_2015} and outlier-rejection algorithms \cite{civera_1-point_2009}. With extensive calibration with respect to temperature and mechanical alignment, the zero-mean Gaussian assumption is sufficient for the measurements of sensors such as inertial measurement units (IMUs) \cite{vectornav_imu_calibration, tedaldi_robust_2014}, even if it is still not completely true: Extended Kalman Filters (EKFs) that rely on these IMUs are deployed on safety-critical systems actively in use.

Even though several well-known algorithms for Simultaneous Localization and Mapping (SLAM) rely on the often-deployed EKF (e.g. \cite{jones_visual-inertial_2011,Geneva2020ICRA,bloesch_iterated_2017}), SLAM is still an active area of research. The existence of recently-released and actively used research benchmark datasets \cite{hilti_benchmark, tartanair2020iros} indicate that the robotics and computer vision communities still believe that performance of SLAM and an understanding of its failure cases are still insufficient, even after three decades of development \cite{early_slam_tutorial}. This motivates an examination into the fundamental assumptions of SLAM.

This manuscript visits the assumption that feature tracks, the ``measurements" of any indirect visual SLAM algorithm, contain only zero-mean Gaussian error. The covariance of the feature tracks is typically a tuning parameter to for all features at all times. We show that the feature track errors are not zero-mean Gaussian and furthermore, that the errors are conditional on the type of motion, the speed of motion, and the type of feature tracker used to extract the feature tracks. To our knowledge, this is the first study of the mean and covariance of feature tracks \emph{conditional} on the factors that affect them.

The organization of the paper is as follows. Section \ref{sec:feature_track_uq} details the methods. Section \ref{sec:feature_tracker_experiment_details} presents some key figures, and summarizes the error distribution of feature trackers. Section \ref{sec:discussion} ends with some concluding remarks. Additional figures from the experiment are given in the Appendix.



\subsection{Related Work}

\paragraph{Performance of feature detectors and descriptors conditional on nuisances.} The main metric used to evaluate feature detectors is \emph{repeatability} \cite{mikolajczyk_comparison_2005}, or the probability that a feature detector will detect the same feature across multiple images of the same scene under different illuminations and viewpoints. Other metrics are \emph{entropy} \cite{heinly_comparative_2012}, the spread of detected features over an image, and \emph{recall} \cite{aanaes_interesting_2012}, the number of features that are likely ``matchable'' to features in another image of the same scene. On the other hand, the primary metrics used to evaluate feature descriptors are \emph{precision} and \emph{recall}, calculated using pairs of ``matches'' that are found using the descriptor \cite{mikolajczyk_performance_2005}. The evaluation of feature detectors requires multiple images of the same scene. The evaluation of descriptors originally used the same datasets as the evaluation of detectors. To disentangle the problem of detecting features from the evaluation of feature description, two comprehensive datasets of image patches was released in 2017 \cite{balntas_hpatches_2017, maier_ground_2017}. At around the same time, \cite{schonberger_comparative_2017} evaluated both learned and handcrafted feature detectors and descriptors. Of most interest to us are \cite{heinly_comparative_2012}, which used a small dataset containing pure rotation, pure scaling, and illumination changes to evaluate the performance of various detector/descriptor combinations condition on each, \cite{zhao_image_2020}, which extended the datasets used in \cite{heinly_comparative_2012}, and \cite{aanaes_interesting_2012}, which evaluated the performance of feature detectors conditional on change in view angle and lighting condition. Tangentially interesting are \cite{hauagge_image_2012}, which released a dataset of image pairs that are geometrically consistent, but contain large changes in style (e.g. summer vs. winter) and lighting; and \cite{sattler_benchmarking_2018}, which contains groups of image sequences with similar motions, but large outdoor illumination changes.


\paragraph{Learning or Fitting a Covariance Matrix to Feature Tracks.} Early works sought to compute covariance of feature location using information in the RGB image. \cite{kanazawa_we_2001} approximated the covariance with the Hessian of the image centered at the feature point was the covariance of a detected feature -- the idea is that the sharper the curvature given by the Hessian, the more likely a convolutional filter will find the correct location of the feature. \cite{nickels_estimating_2002} contains a sum-of-squared-differences formula for computing feature track covariance. \cite{zeisl_estimation_2009} contains a formula for computing the covariance matrix of SIFT and SURF features. Later on, \cite{sheorey_uncertainty_2015} and \cite{wong_uncertainty_2017} present two methods to model the mean and covariance of Lucas-Kanade feature tracks. With the exception  of \cite{sheorey_uncertainty_2015}, which assumes that the location of a feature track could be a Gaussian Mixture Model, all other models assume that uncertainty is zero-mean Gaussian.



\section{Method}
\label{sec:feature_track_uq}

We wish to characterize the dependence of \textbf{mean error}, \textbf{mean absolute error}, \textbf{covariance}, \textbf{outlier ratio}, and \textbf{feature lifetime} on motion type, speed, tracker type, and when available, lighting. The types of motion investigated are:
\begin{itemize}
\item \textbf{Sideways motion} -- Linear movement with no rotation.
\item \textbf{Fixating motion} -- Moving in a constant radius around a central object. The camera is always pointed directly at the central object, creating some rotation.
\item \textbf{Forwards motion} -- Driving-like motion. The primary change frame-to-frame is scale. Points near the center of an image will stay near the center in subsequent frames.
\item \textbf{AR/VR motion} -- Movement that consists of mostly rotations around a persistent scene.
\end{itemize}
To vary speed, we skip frames at regular intervals from the image sequences. Nominal speed, or a speed of 1.00, means that all frames are used. A speed of 2.00 means that the feature tracker will only see every other frame, and a speed of 3.00 means that the feature tracker will only see one in every three frames. We do not test speeds below 1.00. The exact speeds tested depends on dataset. Finally, we also investigate the effect of two types of feature trackers:
\begin{itemize}
\item \textbf{Lucas-Kanade Sparse Optical Flow} \cite{lucas_iterative_1981}
\item \textbf{Correspondence Tracker} using the SIFT descriptor \cite{lowe_object_1999}. Although computationally expensive, the SIFT descriptor was chosen because of its availability and its performance when used in state estimation tasks \cite{schonberger_comparative_2017}. The descriptor of a feature track is set at the first frame it is detected and never updated.
\end{itemize}

We have chosen \emph{not} to study lens distortion, since this would require multiple similar datasets collected with different cameras. All images in all datasets either have been preprocessed to remove lens distortions, or simulated without lens distortions. Since the Lucas-Kanade tracker is differential, we also choose not to study a differential correspondence tracker that updates the descriptor of a feature track at every frame.



\subsection{Equations}

Consider a feature $i$ that was first detected at time $t^i_0$. If a depth image is available at time $t^i_0$ and $g_{sc}(t^i_0)$ is known, we may fix the feature's position in the spatial frame, $X_s^i$:
\begin{equation}
\begin{aligned}
    X^i_c(t^i_0) &= \pi^{-1}_K(x_p(t^i_0), Z^i_0) \\
    X^i_s &= g_{sc}(t^i_0) \circ X_c(t^i_0) \\
    \label{eq:fixing_Xs}
\end{aligned}
\end{equation}
In the above equation, $Z^i_c(t^i_0)$ is the third coordinate, or depth, of $X^i_c(t^i_0)$. Once, $X_s^i$ is fixed, we can then calculate the \textbf{``ground-truth feature track"} $\bar x_p^i(t)$:
\begin{equation}
    \bar x^i_p(t) = \pi_K(g_{sc}^{-1}(t) \circ X_s^i).
    \label{eq:gt_tracks}
\end{equation}
Some datasets provide a ground-truth point-cloud generated by a single lidar scan rather than a stream of depth images. A lidar scan is a point cloud with $M \sim 10^7$ points in the lidar frame $L$, which is defined as the camera frame at a particular time $t_L$: $\mathbf P_L = \{ P^0_L, P^1_L, \dots, P^M_L \}$. We can calculate the pixel coordinates of each point $j$ in $\mathbf P_L$: 
\begin{equation}
\pi_K(\mathbf P_L) = \{ \pi_K(P^0_L), \pi_K(P^1_L), \dots, \pi_K(P^M_L) \}
\label{eq:laser_scan_proj}
\end{equation}
Feature tracks visible at time $t_L$ can be associated with the nearest point in $\pi_K(\mathbf P_L)$. Suppose the nearest point in $\pi_K(\mathbf P_L)$ to feature $i$ is $P^j_L$. Then, the ground-truth track of feature $i$ is
\begin{equation}
\begin{aligned}
    X^i_s &= g_{sc}(t_L) \circ P^j_L \\
    \bar x^i_p(t) &= \pi_K(g_{sc}^{-1}(t) \circ X^i_s).
    \label{eq:dtu_px_groundtruth}
\end{aligned}
\end{equation}
Once we have a ground-truth feature track for feature $i$, we can calculate the error signal for that feature:
\begin{equation}
    e^i(t) = x_p^i(t) - \bar x_p^i(t)
    \label{eq:px_error_def}
\end{equation}
where $x_p^i(t)$ is the observed track. 


For datasets that provide a ground-truth point cloud at a single frame, the \textbf{mean error at timestep $t$} is
\begin{equation}
    \mu(t) = \frac{1}{M(t)} \sum_{i=1}^{M(t)} e^i(t)
    \label{eq:mean_error_at_time}
\end{equation}
where $M(t)$ is the number of tracked features at time $t$. The \textbf{mean absolute error at timestep $t$}
\begin{equation}
    \kappa(t) = \frac{1}{M(t)} \sum_{i=1}^{M(t)} |e^i(t)|.
    \label{eq:mean_abs_error_at_time}
\end{equation}
Similarly, the \textbf{covariance at timestep $t$} is calculated by
\begin{equation}
    \Sigma(t) = \frac{1}{M(t)-1} \sum_{i=1}^{M(t)} e^i(t) e^i(t)^T.
    \label{eq:cov_at_time}
\end{equation}
It is only possible to compute $\mu(t)$, $\kappa(t)$, and $\Sigma(t)$ for features that are visible at time $t_L$, when the laser scan was acquired.


For datasets that provide a stream of depth images, we use different definitions of mean error, mean absolute error, and covariance. We can also use all features and not just those visible in a particular frame. The \textbf{mean error after $k$ timesteps} is
\begin{equation}
    \nu(k) = \frac{1}{\Psi(k)} \sum_{i=1}^{\Phi(k)} e^i(t^i_0+k\delta_t)
    \label{eq:mean_error_after_timesteps}
\end{equation}
where $\Psi(k)$ is the number of features in the entire dataset tracked for at least $k$ timesteps and $\delta_t$ is the length of each timestep. The \textbf{mean absolute error after $k$ timesteps is:}
\begin{equation}
    \eta(k) = \frac{1}{\Psi(k)} \sum_{i=1}^{\Psi(k)} |e^i(t^i_0+k\delta_t)|
    \label{eq:mean_abs_error_after_timesteps}
\end{equation}
where $\Phi(k)$ is the number of features tracked for at least $k$ timesteps and $\delta_t$ is the length of each timestep. Finally, the \textbf{covariance after $k$ timesteps} is given by
\begin{equation}
    \Phi(k) = \frac{1}{\Psi(k)-1} \sum_{i=1}^{\Psi(k)} e^i(t^i_0+k\delta_t) e^i(t^i_0+k\delta_t)^T.
    \label{eq:cov_after_timesteps}
\end{equation}
When depth data is available at all frames, we define the feature's 3D location at the frame it is first detected and use equations \eqref{eq:mean_error_after_timesteps}, \eqref{eq:mean_abs_error_after_timesteps}, \eqref{eq:cov_after_timesteps}. %

At each frame, a feature tracker will attribute some features in one frame to the features in the previous frame. Let $F(t)$ be the total number of features in the frame at time $t$. The features in each frame will consist of $f_0(t)$ correct attributions, $f_1(t)$ incorrect attributions, and $f_2(t)$ new features, where $f_0(t) + f_1(t) + f_2(t) = F(t)$ and $f_0(t) + f_1(t) \leq F(t-1)$. Outlier rejection algorithms are used to determine $f_0(t)$ and $f_1(t)$ in real-time. The \textbf{outlier ratio} is defined as:
\begin{equation}
\frac{f_1(t)}{F(t-1)}.
\end{equation}

Finally, the \textbf{feature lifetime} of a feature track is the total number of consecutive frames in which it found and successfully attributed. A feature is ``born" at the frame it is first detected and ``dies" if a feature is not found for a single frame.


\section{Experiment Details}
\label{sec:feature_tracker_experiment_details}

\subsection{Feature Tracker Configuration}
\label{sec:feature_tracker_configuration}

We used the feature tracker is the \texttt{Tracker} object integrated with XIVO, our in-house SLAM system. The tracker is configured to use the AGAST corner detector \cite{mair_adaptive_2010}, and to track between 1000 and 1200 features at a time. The AGAST corner detector was chosen for its speed and because it detects a large number of features in most scenes. The feature tracker was configured to track up to 1200 features per scene. We use RANSAC with $p=0.995$ and an error threshold of 3 pixels to reject outliers. More details on the \texttt{Tracker} object and XIVO can be found in Appendix \ref{chapter:about_xivo}.

Since the tracker software was programmed to be part of a larger system and not specifically for these experiments, the implementation of the Correspondence Tracker is not ideal. If a feature is visible in frames 0-5, but is not detected in frame 2, the tracker will drop the feature at frame 2 and initialize a new one in frame 3. This behavior is consistent with the definition of feature lifetime given in the previous section, but is not the ideal implementation for a Correspondence Tracker because there is always a possibility that a corner detector will not find the corner in one frame, or that a descriptor will be just a little too different in one particular frame because of lighting. A more ideal implementation of the Correspondence Tracker would drop frames after a $N_m$ missed frames, where $N_m > 1$ is an experimentally determined number. The definition of feature lifetime would also be changed to accommodate this more complex behavior. As a result of this choice, the distribution of feature lifetimes for the Correspondence Tracker are shorter than they otherwise would be. Furthermore, our experiments will fail to characterize trends that only appear at higher speeds.


\subsection{Dataset-Specific Details}

\paragraph{DTU Point Features Dataset.}

The DTU Point Features Dataset \cite{aanaes_interesting_2012} consists of sixty scenes of fixating motion. In the dataset, one or more objects is placed at the center of stage lit with up to 19 LEDs. A camera is mounted on a robot arm and moved in a precise manner at the stage. At each of 119 fixed locations, the camera acquires an image lit with one of the 19 LEDs, enabling lighting experiments using image-based relighting. The dataset contains a laser scan of the scene at a single frame, called the Key Frame. The original image size is 1600 $\times$ 1200. For speed, we use 800 $\times$ 600 px. grayscale versions of the images instead of the full resolution images.

We make use of the first 49 frames of each scene, or Arc 1 (see Figure \ref{fig:dtu_light_stage}). The Key Frame is Frame 25. We calculate mean error $\mu$, mean absolute error $\kappa$, and covariance $\Sigma$ using equations \eqref{eq:mean_error_at_time}, \eqref{eq:mean_abs_error_at_time}, and \eqref{eq:cov_at_time}. Since 3D data is only available at the Key Frame, calculation of errors and covariances only includes features that exist in Frame 25. Therefore, there is a bias towards longer tracks, as all short tracks that don't exist in Frame 25 are all tossed out. Since the ``ground-truth" position of each feature in 3D is defined by its position in Frame 25, all results will therefore show that Frame 25 has zero covariance and the lowest errors. Statistics on feature lifetime and outlier rejection, however, do include features that do not exist in Frame 25.

To compute the ground-truth location of a feature track, we must associate a feature track to a point in a laser scan point cloud (eq. \eqref{eq:dtu_px_groundtruth}). Since the point cloud does not cover every pixel in the image, associations between features and laser scan points are only made if the pixel value of the laser scan point (eq. \eqref{eq:laser_scan_proj}) is less than 0.25 pixels from the feature.  Associating a pixel to a laser scan point with the incorrect depth measurement will result in a very large calculated means in equation \eqref{eq:mean_error_at_time}. Even with the low 0.25 pixel threshold, this bad association can still happen around edges and corners of objects. So that our analyses do not include very many of these poor depth associations, we throw out feature tracks whose maximum error is greater than the 90th percentile.

Since the DTU Point Features dataset was designed to enable image-based relighting, we also investigated the effects of directional light in addition to speed and the tracker used. We tested the same directional lights as  \cite{aanaes_interesting_2012}. The position of each directional light is shown in Figure \ref{fig:dtu_light_stage}.




\begin{figure}
    \centering
    \includegraphics[width=3.2in]{feature_tracker_uq/annotated_paths_of_interest.png}
    \includegraphics[width=3.2in]{feature_tracker_uq/annotated_light_stage_setup.png}
    \caption{\textbf{An Illustration of the Light Stage Setup in the DTU Point Features Dataset.}  \textbf{Left:} The locations at which images were acquired in the DTU Point Features dataset form three arcs and a linear path. Laser scans of the scenes were collected at the Key Frame (front and center). Frames from Arc 1 (circled in blue) are used for this experiment. \textbf{Right:} Red circles depict the location of 19 physical LEDs used to light the scene, which are spaced out over the scene. At each camera position in the left figure, the authors of the DTU Point Features dataset acquired 19 images. In each image, exactly one of the 19 LEDs is switched on. Acquiring 19 images in each location this way facilitates experiments in lighting using image-based relighting. Diffuse lighting can be simulated by using all 19 photographs from each position equally. More intense directional lighting can be simulated by weighting some LEDs more than others. In our experiments, we vary lighting from back-to-front (BF0-BF7) and left-to-right (LR0-LR9) as the camera follows the motion of Arc 1. Lights LR0 - LR9 and BF0 - BF7 are calculated by using Gaussian-weights on the 19 lights with $\sigma=20$cm; Light LR6 is highlighted in green. Figures are reprinted and annotated with permission.}
    \label{fig:dtu_light_stage}
\end{figure}



\paragraph{KITTI Vision Suite.} The raw data \cite{Geiger2012CVPR} in the KITTI Vision Suite consists RGB, GPS, IMU, and Lidar data captured from a moving vehicle. The motion captured in the images is predominantly forwards. The Lidar data was then processed into a separate benchmark dataset of depth images for single-image depth prediction and depth completion \cite{uhrig_sparsity_2017}. We make use stream \texttt{Image02}. Sequences containing ``still frames" (e.g. significant amount of waiting at a traffic light), are excluded. Excluding sequences containing still frames leaves 28 scenes for our experiments. Although this is fewer scenes than the DTU dataset, it is still more frames because most sequences are longer than 49 frames.

Since 3D data is available at every frame, we define a feature's 3D position using the depth image from the very first frame where it was detected. Therefore, we use mean error $\nu$ (eq. \eqref{eq:mean_error_after_timesteps}), absolute error $\eta$ (eq. \eqref{eq:mean_abs_error_after_timesteps}), and covariance $\Phi$ (eq. \eqref{eq:cov_after_timesteps}). To avoid errors due to bad depth measurements, we throw out the tracks whose maximum L2 error are above the 90th percentile and only calculate $\nu$, $\eta$, and $\Phi$ at timesteps where there are at least 100 features (see Fig. \ref{fig:kitti_avg_feats}).



\paragraph{Simulated Supplementary Data.} For AR/VR motions and sideways motions, we collected simulated RGB-D data in Gazebo. The simulation consisted of a Microsoft Kinect, modified so that RGB and depth data would be co-located, mounted on a Hector quadrotor \cite{hector_quadrotor} in ROS Melodic. The scene consisted of large objects from the Open Source Robotics Foundation's Gazebo Model Library. Images have a resolution of 800 $\times$ 600 pixels. In the subsequent sections, we refer to these datasets as ``Gazebo Linear" and ``Gazebo AR/VR". The AR/VR trajectory used to collect data is shown in Figure \ref{fig:gazebo_arvr_traj}.

In the Gazebo Linear dataset, we throw out tracks whose errors are above the 80-th percentile due to drift that naturally occurs when using the Lucas-Kanade Tracker in an environment containing straight and crisp edges parallel to the direction of motion. More details are given in Figure \ref{fig:gazebo_linear_error_throwout}. In the Gazebo AR/VR dataset, the we throw out tracks whose errors are above the 90-th percentile, as motions are no longer parallel to the straight edges.


\begin{figure}
\centering
\includegraphics[width=0.48\textwidth]{feature_tracker_uq/gazebo_arvr_figs/ARVR_translation_gt.pdf}
\includegraphics[width=0.48\textwidth]{feature_tracker_uq/gazebo_arvr_figs/ARVR_rotation_gt.pdf}
\caption{\textbf{The trajectory generated for the AR/VR scenario.} The commanded trajectory used to collected the AR/VR data was generated from the translation (left) and rotation (right) plotted above. Translation is generated point-to-point using haversines and rotation is generated from slerping.}
\label{fig:gazebo_arvr_traj}
\end{figure}



\subsection{Results}

Overall, we find that mean error, mean absolute error, covariance, feature lifetime, and outlier ratio are all dependent on the type of motion, the tracker used, and the speed. For the DTU Point Features dataset, we found no dependence on the existence of directional light unless the directional light happened to cause tracking failure at high speeds. In Tables \ref{tab:dtu_summary_table} - \ref{tab:gazebo_arvr_summary_table}, we list the exact dependence of mean error, mean absolute error, feature lifetime, covariance, and outlier ratio on each independent variable. Differences in Tables \ref{tab:dtu_summary_table} - \ref{tab:gazebo_arvr_summary_table} lead us to conclude that feature tracks are dependent on motion, tracker, and speed, but not the existence of directional light.

One notable difference between the Lucas-Kanade and Correspondence Trackers is that feature tracks produced by the Lucas-Kanade Tracker drift steadily while the Correspondence Tracker does not. This is because the Lucas-Kanade Tracker is differential, i.e. the characterization of a feature will slightly change frame to frame. For the Correspondence Tracker, this is not true. Therefore, the location of the feature track will drift, and the direction and magnitude of drift is dependent on the direction of motion. With left-to-right fixating motion, drift is positive (see Figure \ref{fig:dtu_diffuse_1.00_meanerror}). With left-to-right linear motion, drift is negative, and also larger (see Figure \ref{fig:gazebo_linear_LK_meanerror}). In AR/VR motion, the direction of drift changes with motion (see Figure \ref{fig:gazebo_arvr_LK_meanerror}). The flipside is that the Lucas-Kanade tracker generates features with a longer lifetime (see Figures \ref{fig:dtu_track_lifetime}, \ref{fig:kitti_feature_lifetime}, \ref{fig:gazebo_linear_feature_lifetime}, \ref{fig:gazebo_arvr_feature_lifetime}). When motion is fixating, the Correspondence Tracker also drifts about the direction of motion (see Figure \ref{fig:dtu_mean_error_sideways}).

Finally, we note that the zero-mean Gaussian assumption holds when motion is predominantly forwards and we are using the Correspondence Tracker (see Figures \ref{fig:kitti_match_meanerror} and \ref{fig:kitti_match_cov}). All figures supporting the assertions in this section are given in the Appendix.






\begin{table}[htp]
    \centering
    \begin{tabular}{p{1in}|p{1.0in}|p{1.0in}|p{2.5in}}
                & \textbf{Tracker} & \textbf{Lighting} & \textbf{Speed} \\
    \hline
    $\mu(t)$ & No (fig. \ref{fig:dtu_diffuse_1.00_meanerror}) & No (figs. \ref{fig:dtu_lighting_mu_LK}, \ref{fig:dtu_lighting_mu_match}) & No (figs. \ref{fig:dtu_match_diffuse_mean_error_varyspeed}, \ref{fig:dtu_LK_mean_varyspeed}) \\
    \hline
    $\kappa(t)$ & Yes (fig. \ref{fig:dtu_diffuse_1.00_MAE_cov}) & No (fig. \ref{fig:dtu_diffuse_1.00_MAE_cov}) & Yes for Correspondence Tracker (fig. \ref{fig:dtu_match_diffuse_MAE_varyspeed}), No for Lucas-Kanade Tracker (fig. \ref{fig:dtu_LK_MAE_varyspeed})\\
    \hline
    $\Sigma(t)$ & Yes (fig. \ref{fig:dtu_diffuse_1.00_MAE_cov}) & No (fig. \ref{fig:dtu_lighting_sigma_LK}, \ref{fig:dtu_lighting_sigma_match}) & Yes for Correspondence Tracker (fig. \ref{fig:dtu_match_diffuse_cov_varyspeed}), No for Lucas-Kanade Tracker (fig. \ref{fig:dtu_LK_cov_varyspeed})\\
    \hline
    Feature Lifetime & Yes (fig. \ref{fig:dtu_track_lifetime}) & No  (fig. \ref{fig:dtu_lighting_feature_lifetimes}) & Yes (fig.  \ref{fig:dtu_active_features}) \\
    \hline
    Outlier Ratio & Yes (figs. \ref{fig:dtu_track_outliers_lights}, \ref{fig:dtu_track_outliers_speed}) & No (fig.  \ref{fig:dtu_track_outliers_lights}) & Yes  (fig. \ref{fig:dtu_track_outliers_speed}) \\
    \end{tabular}
    \caption{\textbf{DTU Point Features Results Summary.} Cells contain whether or not the dependent variables in the left column are affected by the independent variables listed in the top row. Entries also contain figure numbers containing justification. The ``Tracker" and ``Lighting" columns contain references to figures containing plots at nominal speed. Although not indicated in the table, Figures \ref{fig:dtu_speed2.00_percent_outlier} - \ref{fig:dtu_LK_cov_speed12.00} in the Appendix show that the existence of directional lighting continues to not affect outlier ratio, mean error, mean absolute error, and covariance at higher speeds for both the Lucas-Kanade and Correspondence Trackers.}
    \label{tab:dtu_summary_table}
\end{table}



\begin{table}[htp]
    \centering
    \begin{tabular}{p{1in}|p{1.5in}|p{2.50in}}
                & \textbf{Tracker}  & \textbf{Speed} \\
    \hline
    $\nu(t)$ & No (fig. \ref{fig:kitti_1.00_meanerror}) & Yes (figs. \ref{fig:kitti_LK_meanerror}, \ref{fig:kitti_match_meanerror}) \\
    \hline
    $\eta(t)$ & Yes (fig. \ref{fig:kitti_1.00_error_cov}) & No for Correspondence Tracker (fig. \ref{fig:kitti_match_MAE}), Yes for Lucas-Kanade Tracker (figs. \ref{fig:kitti_LK_MAE}) \\
    \hline
    $\Phi(t)$ & Yes (fig. \ref{fig:kitti_1.00_error_cov}) & No for Correspondence Tracker (fig. \ref{fig:kitti_match_cov}), Yes for Lucas-Kanade Tracker (fig. \ref{fig:kitti_LK_cov}) \\ 
    \hline
    Feature Lifetime & Yes (fig. \ref{fig:kitti_feature_lifetime}) & Yes (fig. \ref{fig:kitti_avg_feats}) \\
    \hline
    Outlier Ratio & Yes (fig. \ref{fig:kitti_outlier_ratio}) & Yes (fig. \ref{fig:kitti_outlier_ratio})\\
    \end{tabular}
    \caption{\textbf{KITTI Results Summary.} Cells contain whether or not the dependent variables in the left column are affected by the independent variables listed in the top row. Entries also contain figure numbers containing justification.}
    \label{tab:kitti_summary_table}
\end{table}



\begin{table}[htp]
    \centering
    \begin{tabular}{p{1in}|p{1.5in}|p{2.5in}}
                & \textbf{Tracker}  & \textbf{Speed} \\
    \hline
    $\nu(t)$ & Yes (fig. \ref{fig:gazebo_linear_1.00_meanerror}) & No for Correspondence Tracker (fig. \ref{fig:gazebo_linear_match_meanerror}), Yes for Lucas-Kanade Tracker   (fig. \ref{fig:gazebo_linear_LK_meanerror}) \\
    \hline
    $\eta(t)$ & Yes (fig. \ref{fig:gazebo_linear_1.00_error_cov}) & Yes (figs. \ref{fig:gazebo_linear_LK_MAE}, \ref{fig:gazebo_linear_match_MAE}) \\
    \hline
    $\Phi(t)$ & Yes (fig. \ref{fig:gazebo_linear_1.00_error_cov}) & Yes (figs.  \ref{fig:gazebo_linear_match_cov}, \ref{fig:gazebo_linear_LK_cov}) \\ 
    \hline
    Feature Lifetime & Yes (fig. \ref{fig:gazebo_linear_feature_lifetime}) & Yes (fig. \ref{fig:gazebo_linear_avg_feats}) \\
    \hline
    Outlier Ratio & Yes (fig. \ref{fig:gazebo_linear_outlier_ratio}) &  No for Correspondence Tracker, Yes for Lucas-Kanade Tracker (fig. \ref{fig:gazebo_linear_outlier_ratio})\\
    \end{tabular}
    \caption{\textbf{Gazebo Linear Results Summary.} Cells contain whether or not the dependent variables in the left column are affected by the independent variables listed in the top row. Entries also contain figure numbers containing justification.}
    \label{tab:gazebo_linear_summary_table}
\end{table}


\begin{table}[htp]
    \centering
    \begin{tabular}{p{1in}|p{1.5in}|p{2.5in}}
                & \textbf{Tracker}  & \textbf{Speed} \\
    \hline
    $\nu(t)$ & Yes (fig. \ref{fig:gazebo_arvr_1.00_meanerror}) & No (fig. \ref{fig:gazebo_arvr_LK_meanerror}, \ref{fig:gazebo_arvr_match_meanerror}) \\
    \hline
    $\eta(t)$ & Yes (fig. \ref{fig:gazebo_arvr_1.00_error_cov}) & Yes for Correspondence   Tracker (fig. \ref{fig:gazebo_arvr_match_MAE}), No for Lucas-Kanade  Tracker (fig. \ref{fig:gazebo_arvr_LK_MAE})  \\
    \hline
    $\Phi(t)$ & Yes (fig. \ref{fig:gazebo_arvr_1.00_error_cov}) &  Yes for Correspondence   Tracker (fig. \ref{fig:gazebo_arvr_match_cov}), No for Lucas-Kanade Tracker (fig. \ref{fig:gazebo_arvr_LK_cov})  \\ 
    \hline
    Feature Lifetime & Yes (fig. \ref{fig:gazebo_arvr_feature_lifetime}) & Yes (fig. \ref{fig:gazebo_arvr_avg_feats}) \\
    \hline
    Outlier Ratio & Yes (fig. \ref{fig:gazebo_arvr_outlier_ratio}) & No for Correspondence Tracker, Yes for Lucas-Kanade Tracker (fig. \ref{fig:gazebo_arvr_outlier_ratio})\\
    \end{tabular}
    \caption{\textbf{Gazebo AR/VR Results Summary.} Cells contain whether or not the dependent variables in the left column are affected by the independent variables listed in the top row. Entries also contain figure numbers containing justification.}
    \label{tab:gazebo_arvr_summary_table}
\end{table}


\section{Discussion}
\label{sec:discussion}

Other than the caveat about the Correspondence Tracker noted in Section \ref{sec:feature_tracker_configuration}, the main limitation of this work is that there are more variables we could have tested, but chose not to. Examples of variables we chose not to test are the choice of feature detector and descriptor, and characteristics in the scene. For example, would the Correspondence Tracker have as little drift when moving forwards in an indoor environment and comparing BRIEF descriptors? Testing for conditionality on more variables inevitably leads to an unmanageable experiment, so we chose to lock in the feature detector and descriptor to well-performing available options and let the dataset dictate available scenes. Nevertheless, our work is a first step in characterizing the dependence of mean error, mean absolute error, covariance, feature lifetime, and outlier ratio on motion, tracker, speed, and the existence of directional lighting. The main conclusion is that the common zero-mean Gaussian assumption is rarely true. This conclusion motivates a few areas of future work.

The most immediate direction of future work is to continue to use the Extended Kalman Filter and dynamically adapt filter parameters, such as covariance estimates and the number of tracked features, to the scene. Since feature tracks are not zero-mean, covariance estimates will have to be enlarged so that feature tracks containing the extra bias are not outliers. Machine learning approaches to adapting the covariance already exist \cite{vega-brown_cello_2013, liu_deep_2018}. Since statistical methods are not often desirable in safety critical systems, it is of interest to compare performance when covariance is adjusted by a learned model to when covariance is adjusted by a finite state machine. While this approach is the most immediate, it does not address the fact that it brings no convergence guarantees in a downstream state estimation process and will therefore require extensive testing for each application.

The second area of future work is to adapt existing state estimation algorithms to accommodate feature tracks that are not zero-mean Gaussian. It may not be possible, however, to design a filter that is both computationally tractable, guaranteed to converge, and simple enough to implement on a complex, realistic system. This motivates the study in the next chapter, and the third area of future work.

The third direction of future work is to adjust individual feature tracks \emph{before} they are used by a state estimation algorithm that assumes that measurements are zero-mean Gaussian. This is the approach used for IMUs: errors in IMUs measurements are primarily dependent on temperature and mechanical alignment errors, so IMU measurements are adjusted for temperature and known mechanical misalignments before they are passed to a downstream computer. For feature tracks, the calibration table would be more complex, as it is dependent on speed, motion type, and the type of tracker used.



We visualize additional qualitative comparisons on the task of scene arrangement in Fig.~\ref{fig:arrangement_supple}. Also, the quantitative results are shown in Tab.~\ref{tab:arrange}. 

\paragraph{Text-conditioned Scene Synthesis}
\begin{figure*}[!htbp]
	\centering
	\begin{subfigure}[t]{0.24\textwidth}
            \includegraphics[width=\textwidth]{././figs/experiments/text2scene_supple/LivingDiningRoom-107_text.jpg}
            \vspace{2mm}
            \includegraphics[width=\textwidth]{././figs/experiments/text2scene_supple/LivingDiningRoom-1744_text.jpg}
            \vspace{2mm}
            \includegraphics[width=\textwidth]{././figs/experiments/text2scene_supple/LivingDiningRoom-2106_text.jpg}
            \vspace{2mm}
            \includegraphics[width=\textwidth]{././figs/experiments/text2scene_supple/LivingDiningRoom-3483_text.jpg}
        \caption{Input text}
	\end{subfigure}%
        \hfill
 	\begin{subfigure}[t]{0.215\textwidth}
            \includegraphics[width=\textwidth]{././figs/experiments/text2scene_supple/LivingDiningRoom-107_41_041_gt.jpg}
            \includegraphics[width=\textwidth]{././figs/experiments/text2scene_supple/LivingDiningRoom-1744_61_061_gt.jpg}
            \includegraphics[width=\textwidth]{././figs/experiments/text2scene_supple/LivingDiningRoom-2106_83_083_gt.jpg}
            \includegraphics[width=\textwidth]{././figs/experiments/text2scene_supple/LivingDiningRoom-3483_163_163_gt.jpg}
        \caption{Reference}
	\end{subfigure}%
        \hfill
 	\begin{subfigure}[t]{0.215\textwidth}
            \includegraphics[width=\textwidth]{././figs/experiments/text2scene_supple/LivingDiningRoom-107_41_041_atiss.jpg}
            \includegraphics[width=\textwidth]{././figs/experiments/text2scene_supple/LivingDiningRoom-1744_61_445_atiss.jpg}
            \includegraphics[width=\textwidth]{././figs/experiments/text2scene_supple/LivingDiningRoom-2106_83_467_atiss.jpg}
            \includegraphics[width=\textwidth]{././figs/experiments/text2scene_supple/LivingDiningRoom-3483_163_355_atiss.jpg}
        \caption{ATISS~\cite{paschalidou2021atiss}}
	\end{subfigure}%
        \hfill
 	\begin{subfigure}[t]{0.215\textwidth}
            \includegraphics[width=\textwidth]{././figs/experiments/text2scene_supple/LivingDiningRoom-107_41_018_ours.jpg}
            \includegraphics[width=\textwidth]{././figs/experiments/text2scene_supple/LivingDiningRoom-1744_61_002_ours.jpg}
            \includegraphics[width=\textwidth]{././figs/experiments/text2scene_supple/LivingDiningRoom-2106_83_003_ours.jpg}
            \includegraphics[width=\textwidth]{././figs/experiments/text2scene_supple/LivingDiningRoom-3483_163_010_ours.jpg}
        \caption{Ours}
	\end{subfigure}
	\caption{\textbf{Text-conditioned scene synthesis}. The input text describes only a partial scene configuration. Our method generates more plausible scenes matched with the texts.}
    \label{fig:text2scene_supple}
    %\vspace{-6mm}
\end{figure*}

We provide additional qualitative comparisons on the text-conditioned scene synthesis in Fig.~\ref{fig:text2scene_supple}. 
As observed, in the first and third rows, ATISS has object intersection issues while ours does not. In the second row, our method can correctly generate a corner side table on the left of the armchair. However, ATISS generates a corner side table on the right of the armchair.
 In the fourth row, our method can generate four dining chairs that are consistent with the text description, but ATISS can only generate two dining chairs.
The quantitative results evaluated by FID, KID, and SCA are reported in Tab.~\ref{tab:text}. Our method consistently outperforms ATISS in all used metrics.

\section{User Study}
\label{SecUser}

We conducted a perceptual user study to evaluate the quality of our method against ATISS on the application of text-conditioned scene synthesis.
As shown in Fig.~\ref{fig:user_study}, we provide the visualization of a ground-truth scene used to generate a text prompt as a reference. For each pair of results, a user needs to answer ``which of the generated scene can better match the text prompt?" and ``Which of the generated scene is more reasonable and realistic?".
%needs to decide which of the generated scene can better match the text prompt and judge which of the synthesized scene is more plausibly realistic than the other.
%\TODO{what is plausibly realistic? write down the question that you asked in the study}
We collect the answers of 225 scenes from 45 users and calculate the statistics. 62$\%$ of the user answers prefer our method to ATISS in realism.  55$\%$ of answers think our method is more consistent with the text prompt.

\begin{figure*}[!htbp]
    \centering
    \includegraphics[width=\linewidth]{./figs/experiments/user_study/question_reference.jpg}
    
    \includegraphics[width=\linewidth]{./figs/experiments/user_study/question_match.jpg}

    \includegraphics[width=\linewidth]{./figs/experiments/user_study/question_realism.jpg}

    \caption{\textbf{User Study UI}. Based on the reference scene used to generate text prompts, users are asked which of the synthesized scene is more matched with the text prompt and more realistic. Note that the results from ATISS and our method are randomly shuffled to avoid bias.}
    \label{fig:user_study}
    
\end{figure*}


\end{document}