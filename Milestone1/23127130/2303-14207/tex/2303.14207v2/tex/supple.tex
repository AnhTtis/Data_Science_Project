In this supplemental material, we provide details for our implementation in Sec.~\ref{SecImple}, dataset pre-processing and text prompt generation in Sec.~\ref{SecData}, baseline implementations in Sec.~\ref{SecBaseline}, additional results in Sec.~\ref{SecAddRes}, and user studies in Sec.~\ref{SecUser}.

\section{Implementations}
\label{SecImple}

\subsection{Shape Auto-Encoder}
\label{SubSecShapeAE}

We adopt a pre-trained shape auto-encoder to extract a set of latent shape codes for CAD models from the 3D-FUTURE~\cite{fu20213dm} dataset. The network architecture of the shape auto-encoder is shown in Fig.~\ref{fig:shapeae}. It is a variational auto-encoder, similar to FoldingNet~\cite{yang2018foldingnet}.
Specifically, a point cloud $\mathbf{P}_{in}$ of size 2,048 is fed into a graph encoder based on PointNet~\cite{qi2017pointnet} with graph convolutions~\cite{wang2019dynamic} to extract a global latent code of dimension 512, which is used to predict the mean $\mathbf{\mu}$ and variance $\mathbf{\sigma}$ of a low-dimensional latent space of size 32.
Subsequently, a compressed latent is sampled from $\mathcal{N}(\mathbf{\mu}, \mathbf{\sigma})$.
%\TODO{maybe stupid question, but what is reparametrization sampling. we should explain that}
Finally, the compressed latent is mapped back to the original space and passed to the FoldingNet decoder to recover a point cloud $\mathbf{P}_{rec}$ of size 2,025.
The used training objective is a weighted combination of Chamfer distance (\ie CD) and KL divergence.
\begin{equation}
    \label{EquaShapeAE}
        L_{vae} = \CD(\mathbf{P}_{in}, \mathbf{P}_{rec}) + \omega_{kl} *\KL(\mathcal{N}(\mathbf{\mu}, \mathbf{\sigma}) || \mathcal{N}(\mathbf{0}, \mathbf{I})) ,
\end{equation}
where $\omega_{kl}$ is set to 0.001.
The latent compression and KL regularization leads to a compact and structured latent space, focusing on global shape structures.
The shape autoencoder is trained on a single RTX 2080 with a batch size of 16 for 1,000 epochs.
The learning rate is initialized to $lr=\expnumber{1}{-4}$ and then gradually decreases with the decay rate of 0.1 in every 400 epochs.
\begin{figure}[!htp]
    \centering
    \includegraphics[width=\linewidth]{./figs/shapeautoencoder.pdf}
    \caption{\textbf{Shape Auto-encoder.}}
    \label{fig:shapeae}
\end{figure}

\subsection{Shape Code Diffusion}
\label{SubSecShapeDiffu}

We use the extracted latent codes to train shape code diffusion.
While we apply KL regularization, the value range of latent codes is still unbound.
To make it easier to diffuse, we scale the latent codes to $[-1, 1]$ by using the statistical minimum and maximum feature values over the whole set.
During inference, we rescale generated shape codes.

\subsection{Shape Retrieval}
\label{SubSecRetrieval}

During inference, we use shape retrieval as the post-processing procedure to acquire object surface geometries for generated scenes.
Concretely, for each instance, we perform the nearest neighbor search in the 3D-FUTURE~\cite{fu20213dm} dataset to find the CAD model with the same class label and the closest geometry feature.

% \subsection{Loss function}
% \label{SubSecLoss}

\section{Dataset}
\label{SecData}

\paragraph{Preprocessing}
The dataset preprocessing is based on the setting of ATISS~\cite{paschalidou2021atiss}.
We start by filtering out those scenes with problematic object arrangements such as severe object intersections or incorrect object class labels, e.g., beds are misclassified as wardrobes in some scenes.
Then, we remove those scenes with unnatural sizes. The floor size of a natural room is within $6m \times 6m$ and its height is less than $4m$. Subsequently, we ignore scenes that have too few or many objects.
The number of objects in valid bedrooms is between 3 and 13. As for dining and living rooms, the minimum and maximum numbers are set to 3 and 21 respectively. Thus, the number of objects is $N=13$ in bedrooms and $N=21$ in dining and living rooms. In addition, we delete scenes that have objects out of pre-defined categories. After pre-processing, we obtained 4,041 bedrooms, 900 dining rooms, and 813 living rooms.

For the semantic class diffusion, we have an additional class of  `empty' to define the existence of an object. Combining with the object categories that appeared in each room type, we have $L=22$ object categories for bedrooms, and
$L=25$ object categories for dining and living rooms in total. The category labels 
are listed as follows.

\begin{python}
# 22 3D-Front bedroom categories
['empty', 'armchair', 'bookshelf', 'cabinet',
'ceiling_lamp', 'chair', 'children_cabinet',
'coffee_table', 'desk', 'double_bed',
'dressing_chair', 'dressing_table', 'kids_bed',
'nightstand', 'pendant_lamp', 'shelf',
'single_bed', 'sofa', 'stool', 'table',
'tv_stand', 'wardrobe']

# 25 3D-Front dining or living room categories
['empty', 'armchair', 'bookshelf', 'cabinet', 
'ceiling_lamp', 'chaise_longue_sofa', 
'chinese_chair', 'coffee_table', 'console_table',  
'corner_side_table',  'desk', 'dining_chair', 
'dining_table', 'l_shaped_sofa', 'lazy_sofa', 
'lounge_chair', 'loveseat_sofa', 
'multi_seat_sofa', 'pendant_lamp', 
'round_end_table', 'shelf', 'stool', 
'tv_stand', 'wardrobe', 'wine_cabinet']
\end{python}

\paragraph{Text Prompt Generation}
We follow the SceneFormer~\cite{wang2021sceneformer} to generate text prompts describing partial scene configurations. Each text prompt contains one to three sentences. We explain the details of text formulation process by using the text prompt 'The room has a dining table, a pendant lamp, and a lounge chair. The pendant lamp is above the dining table. There is a stool to the right of the lounge chair.` as an example. First, we randomly select three objects from a scene, get their class labels, and then count the number of appearances of each selected object category. As such, we can get the first sentence. Then, we find all valid object pairs associated with the selected three objects. An object pair is valid only if the distance between two objects is less than a certain threshold that is set to 1.5 in our method. Next, we calculate the relative orientations and translations, from which we can determine the relationship type of the valid object pair from the candidate pool: 'is above to`, 'is next to`, 'is left of`, 'is right of`, ' surrounding`, 'inside`, 'behind`, 'in front of`, and 'on`. In this way, we can acquire some relation-describing sentences like the second and third sentences in the example. Finally, we randomly sampled zero to two relation-describing sentences.

\section{Baselines}
\label{SecBaseline}

\paragraph{DepthGAN} 
DepthGAN~\cite{yang2021indoor} adopts a generative adversary network to train 3D scene synthesis using both semantic maps and depth images. The generator network is built with 3D convolution layers, which decode a volumetric scene with semantic labels. A differentiable projection layer is applied to project the semantic scene volume into depth images and semantic maps under different views, where a multi-view discriminator is designed to distinguish the synthesized views from ground-truth semantic maps and depth images during the adversarial training.


\paragraph{Sync2Gen} 
Sync2Gen~\cite{yang2021scene} represents a scene arrangement as a sequence of 3D objects characterized by different attributes (e.g., bounding box, class category, shape code). The generative ability of their method relies on a variational auto-encoder network, where they learn objects' relative attributes. Besides, a Bayesian optimization stage is used as a post-processing step to refine object arrangements based on the learned relative attribute priors.

\paragraph{ATISS}
ATISS~\cite{paschalidou2021atiss} considers a scene as an unordered set of objects and then designs a novel autoregressive transformer architecture to model the scene synthesis process. During training, based on the previously known object attributes, ATISS utilizes a permutation-invariant transformer to aggregate their features and  predicts the location, size, orientation, and class category of the next possible object conditioned on the fused feature. 
The original version of ATISS~\cite{paschalidou2021atiss} is conditioned on a 2D room mask from the top-down orthographic projection of the 3D floor plane of a scene. To ensure fair comparisons, we train an unconditional ATISS without using a 2D room mask as input, following the same training strategies and hyperparameters as the original ATISS.


\section{Ablation Studies}
\label{SecAbla}
 %We provide more detailed explanations for ablation studies.
In main paper, we investigated the effectiveness of each design in our DiffuScene, including network architecture, loss function, and geometry feature diffusion. We present more implementation details of each method variant.

\noindent \textbf{What is the effect of UNet-1D+Attention as the denoiser? } 
We advocate the use of UNet-1D with attention layers as the denoising network. 
The self-attention layers within this architecture effectively aggregate all object features and explore inter-object relationships, facilitating the learning of a global context that aids in distinguishing different objects within the scene.
An alternative choice is to use a pure transformer network, like the one adopted in DALLE-2~\cite{ramesh2022hierarchical}. However, our comparisons revealed a marginal degradation in performance metrics such as FID, KID, SCA, and CKL. It demonstrates that UNet-1D with attention layers is more adept at capturing accurate scene distributions than networks solely composed of transformation layers.

\noindent \textbf{What is the effect of multiple prediction heads in the denoiser?} 
In our denoiser architecture, we employ three distinct encoding and prediction heads tailored for specific object properties, including bounding box parameters, semantic class labels, and geometry codes. By utilizing multiple diffusion heads with individual loss functions for each attribute (e.g., bbox, class, geometry), we mitigate the risk of bias towards any single attribute within a single encoding and prediction head. This approach ensures that our denoiser effectively captures and processes diverse object properties without favoring one over the others. The consistent improvement in each evaluation metric verifies the effectiveness of multiple prediction heads.

\noindent \textbf{What is the effect of the IoU loss? }
In scene diffusion models, we employ noise prediction loss as the primary supervision, focusing on attribute denoising of individual object instances. However, this loss does not address object intersections within a scene. To alleviate the issue, we augment it with pair-wise bounding box IoU loss. Quantitative comparisons indicate that incorporating IoU loss results in the synthesis of scenes with improved symmetry and enhanced plausibility, as evidenced by lower FID, KID, SCA, PIoU and higher Sym.

\noindent \textbf{What is the effect of geometry feature diffusion?}
To evaluate our method's performance without geometry feature diffusion, we eliminate the geometry feature encoding and prediction heads from our denoiser network. Consequently, this method only produces bounding boxes and class labels for objects within a scene.
During inference, 
 for each generated object, we conduct shape retrieval in the 3D-FUTURE~\cite{fu20213dm} dataset to find the CAD model with the same class label and the closest 3D bounding box sizes.
 %
Fig. 5 of the main paper shows that our model can find symmetric nightstands by beds due to the geometry awareness of the diffusion process and shape retrieval. Table 3 in the main paper presents the comparison in the formation of symmetric pairs: 0.72 (w/ shape diffusion) vs. 0.50 (w/o shape diffusion).
This highlights the effectiveness of geometry feature diffusion in achieving symmetric placements and semantically coherent arrangements. Improved plausibility in synthesis results is reflected in lower FID, KID, and SCA evaluations. Additionally, the decrease in CKL suggests that the joint diffusion of geometry code and object layout facilitates learning more similar object class distributions.
% \section{Evaluation Metrics}
% \label{SecEval}

% \paragraph{Fr{\'e}chet Inception Distance}

% \paragraph{Kernel Inception Distance}

% \paragraph{Scene Classification Accuracy}

% \paragraph{Category KL Divergenece}



\section{Additional Results}
\label{SecAddRes}

\paragraph{Diversity Analysis.} 
%
The qualitative comparisons in Fig. 7 of the main paper and Fig.~\ref{fig:completion_supple} %of the supplementary 
illustrate that our diffusion-based method can produce more diverse results than the baseline methods.
%
Following ATISS and LEGO, we use FID and KID to quantitatively evaluate the result diversity.
%
We compare both the mean and covariance of generated and reference scene distribution.
%
Additionally, we include Precision / Recall commonly used to evaluate generative models~\cite{kynkaanniemi2019improved}. 
%
Precision is the probability that a randomly generated scene falls within the support of real scene distribution.
%
Recall is the probability that a random scene from the datasets falls within the generated scene distribution.
%
Tab.~\ref{tab:pre_rec} shows that our approach outperfoms all baselines in both metrics, which demonstrates better diversity, plausiblity, and mode coverage. 
%
\documentclass[10pt,twocolumn,letterpaper]{article}

\usepackage{iccv}
\usepackage{times}
\usepackage{epsfig}
\usepackage{graphicx}
\usepackage{amsmath}
\usepackage{amssymb}

\usepackage{multirow}
\usepackage{tabularx}
\usepackage{listings}
\usepackage{xcolor}
\usepackage{setspace}
\usepackage{float}
\lstdefinelanguage{json}{
  basicstyle=\small\ttfamily,
  numbers=left,
  numberstyle=\tiny,
  stepnumber=1,
  numbersep=8pt,
  showstringspaces=false,
  breaklines=true,
  frame=lines,
  backgroundcolor=\color{gray!10},
  morestring=[b]{"},
  stringstyle=\color{blue},
  commentstyle=\color{red},
  keywordstyle=\color{cyan},
  morekeywords={true,false,null}
}

\usepackage{ifthen}
\usepackage[export]{adjustbox}
\usepackage{subcaption,graphicx}


% Include other packages here, before hyperref.

% If you comment hyperref and then uncomment it, you should delete
% egpaper.aux before re-running latex.  (Or just hit 'q' on the first latex
% run, let it finish, and you should be clear).
\usepackage[breaklinks=true,bookmarks=false]{hyperref}

\iccvfinalcopy % *** Uncomment this line for the final submission

\def\iccvPaperID{****} % *** Enter the ICCV Paper ID here
\def\httilde{\mbox{\tt\raisebox{-.5ex}{\symbol{126}}}}

% Pages are numbered in submission mode, and unnumbered in camera-ready
\ificcvfinal\pagestyle{empty}\fi

\begin{document}

%%%%%%%%% TITLE
\title{





%%% Use this command to specify your EasyChair submission number.
%%% In anonymous mode, it will be printed on the first page.

\acmSubmissionID{???}

%%% Use this command to specify the title of your paper.

% \title[Team Sport Analytics is a Multiagent Systems Problem]{Team Sport Analytics is a Multiagent Systems Problem}
\title[Presenting Multiagent Challenges in Team Sports Analytics]{Presenting Multiagent Challenges in Team Sports Analytics}
\subtitle{Blue Sky Ideas Track}
% \title[How Multiagent Systems and Sports Analytics can Help Each Other]{How Multiagent Systems and Sports Analytics can Help Each Other}

%%% Provide names, affiliations, and email addresses for all authors.

% \author{Paper: \#1506}

\author{David Radke}
\email{dradke@blackhawks.com}
\affiliation{%
  \institution{Chicago Blackhawks}
	\city{Chicago, IL}
	\country{USA \\ University of Waterloo}
}
% \email{dtradke@uwaterloo.ca}
% \affiliation{%
%   \institution{University of Waterloo}
% 	% \city{Waterloo, ON}
% 	% \country{Canada}
% }


\author{Alexi Orchard}
\email{alexi.orchard@uwaterloo.ca}
\affiliation{%
  \institution{University of Waterloo}
	\city{Waterloo, ON}
	\country{Canada}
}

% \author {
%     % Authors
%     David Radke\textsuperscript{\rm 1,2},
%     Alexi Orchard\textsuperscript{\rm 2}
% }
% \affiliation {
%     % Affiliations
%     \textsuperscript{\rm 1}Chicago Blackhawks,
%     \textsuperscript{\rm 2}University of Waterloo\\
%     dradke@blackhawks.com, 
%     alexi.orchard@uwaterloo.ca
% }}
%%%%%%%%% AUTHORS


\author{
    Dmitrii Torbunov, Yi Huang, Huan-Hsin Tseng, Haiwang Yu, \\Jin Huang, Shinjae Yoo, Meifeng Lin, Brett Viren, Yihui Ren\\
    Brookhaven National Laboratory, Upton, NY, USA\\
    % Upton, NY, USA\\
    {\tt\small \{dtorbunov,yhuang2,htseng,hyu,jhuang,sjyoo,mlin,bviren,yren\}@bnl.gov}
}

\maketitle
% Remove page # from the first page of camera-ready.
\ificcvfinal\thispagestyle{empty}\fi

\newcommand{\imagenetcbar}{ImageNet-$\overline{\mbox{C}}$ }
\newcommand{\mrq}[1]{\textbf{#1}}
\newcommand{\rqspace}{}

%%%%%%%%% ABSTRACT
\begin{abstract}


Over the past few years, there has been a significant amount of research focused on studying the ReLU activation function, with the aim of achieving neural network convergence through over-parametrization. However, recent developments in the field of Large Language Models (LLMs) have sparked interest in the use of exponential activation functions, specifically in the attention mechanism.

Mathematically, we define the neural function $F: \R^{d \times m} \times  \mathbb{R}^d \rightarrow \mathbb{R}$ using an exponential activation function. Given a set of data points with labels $\{(x_1, y_1), (x_2, y_2), \dots, (x_n, y_n)\} \subset \mathbb{R}^d \times \mathbb{R}$ where $n$ denotes the number of the data. Here $F(W(t),x)$ can be expressed as $F(W(t),x) := \sum_{r=1}^m a_r \exp(\langle w_r, x \rangle)$, where $m$ represents the number of neurons, and $w_r(t)$ are weights at time $t$. It's standard in literature that $a_r$ are the fixed weights and it's never changed during the training. We initialize the weights $W(0) \in \mathbb{R}^{d \times m}$ with random Gaussian distributions, such that $w_r(0) \sim \mathcal{N}(0, I_d)$ and initialize $a_r$ from random sign distribution for each $r \in [m]$.

Using the gradient descent algorithm, we can find a weight $W(T)$ such that $\| F(W(T), X) - y \|_2 \leq \epsilon$ holds with probability $1-\delta$, where $\epsilon \in (0,0.1)$ and $m = \Omega(n^{2+o(1)}\log(n/\delta))$. To optimize the over-parametrization bound $m$, we employ several tight analysis techniques from previous studies [Song and Yang arXiv 2019, Munteanu, Omlor, Song and Woodruff ICML 2022]. 

 

\end{abstract}
%%%%%%%%% BODY TEXT
\section{Introduction}

    In the context of Deep Learning (DL), it has been empirically discovered that the use of ensembles can improve the model's accuracy in tasks such as regression and classification.
    It has been speculated~\cite{deepEnsemLoss} that the main reason behind these improvements is the implicit diversity in the solutions found that when aggregated as an ensemble obtain better predictions.
    In this work, we evaluate the resiliency of diverse deep learning classifiers and the role that the different kinds of diversity play in improving them.

    \subsection{The case for design diversity}
    	
        Design diversity~\cite{littlewood2001modeling} is a technique to increase the resilience of safety critical systems.
        It is established as a best practice in standards such as in vehicle functional safety ~\cite{ISO26262} to prevent dependent failures, safety of the intended functionality ~\cite{SOTIF} to address system limitations of machine-learning-based components, and avionic software~\cite{euroCAE}.
    
        A common pitfall is to misunderstand \textit{independent development} as \textit{design diversity}.
        In~\cite{boeing777} the designers of a safety-critical system preferred to let multiple teams collaborate, although the purpose of having multiple teams is to produce multiple designs of a single specification.
        This was justified with the claim that specification problems can be better mitigated with such collaboration but at the cost of the sought \textit{independence}.
    
        The key problem is, that independent development can (and will) produce designs with common failures mainly due to the fact that independent designs do not enforce diverse design choices.
        In fact, it has been statistically proven that independently developed software results in dependent failure behavior on randomly selected inputs~\cite{ELmodel}.
    
        In~\cite{LMmodel}, it has been shown that what is needed to reduce dependent failure behavior is diversity in design \textit{choices}.
        If the choices are made satisfying certain properties, it can be expected (in the average case) to obtain negatively correlated failure behavior, i.e., better than independent.
    
        In DL, however, design choices are not made by the human designers explicitly but are a result of the architecture, data, and optimization approach.
        Furthermore, existing diversity metrics are not directly related to the DL model's \textit{design choices} and are known to have a diversity-accuracy trade-off~\cite{KunchevaW03}.
        We investigate if a diversity metric closer to the model \textit{design choices} can improve the model resilience compared to existing metrics.
    
        \newcounter{reqno}
        \newcommand{\rrq}[2]{\textbf{RQ#1:}\refstepcounter{reqno}\label{req:#2}}
        \newcommand{\refreq}[1]{\textbf{RQ\ref{req:#1}}}

    \subsection{Main research questions}
    
        We aim to answer the following research questions:\\
        \rrq{1}{accRes} Is model accuracy, size, or architecture the main explanatory factor of resilience against natural image corruptions?\\
        \rqspace
        \rrq{2}{att} Can a diversity metric closer to design choices improve the known accuracy-diversity trade-off?\\
        \rqspace
        \rrq{3}{bestEnf} Which diversity enforcement heuristic produces the most resilient models?\\
        \rqspace
        \rrq{4}{divNCL} How diverse are the predictions, activations, and input feature attributions of models created with a diversity enforcement learning approach?\\    
    
        The rest of the paper is structured as follows: Section~\ref{sota} provides a brief overview of the current state-of-the-art in diversity enforcement and measurement.
        Next, the methodology is stated in Section~\ref{labelDesignDiversity}.
        Our experiments are then presented in Section~\ref{labelExperiments}, followed by a discussion of the outcomes and conclusions in Section~\ref{labelDiscussion}.

\section{Related work}
\label{sota}

    \subsection{Ensemble creation techniques} 
    
        The most relevant techniques for ensemble creation are:
        a) Ensembles of independently trained models where diversity originates from the training process randomness, e.g., seed.
        Each ensemble member loss is a function notated as:
    
        \begin{equation}
        \label{eq:ind}
            l(h^{i},y)
        \end{equation}
        where $y$ is the ground-truth label, and $h^{i}$ is the output of the $i$th single ensemble member.
        \cite{ensResilience} presents an analysis of the resilience of independently trained ensembles.
        b) Bagging~\cite{breiman1996bagging}, reduces the variance of multiple models by averaging the outcomes of models created with different training data subsets.
        c) Boosting~\cite{schapire1999brief} sequentially trains models to reduce bias by sampling incorrectly classified inputs more often in the next model.
        d) Negative Correlation Learning (NCL)~\cite{LIU} trains models \textit{in parallel} with a shared penalty term in their loss function to enforce prediction diversity.
        Generalized NCL (GNCL)~\cite{gncl} proposes two extensions for NCL: i) a generalized loss function for each member:
        \begin{equation}
            \label{eq:gncl_i}
            \sum_{i=1}^{M}{l(h^{i},y)-\frac{\lambda}{2M}\sum_{i=1}^{M}{{d_{i}}^{T}\mathfrak{D}d_{i}}}
        \end{equation}
        where $M$ is the total number of members in the ensemble, $d$ is the difference $h^i-f$ with $f$ as the ensemble prediction, $\mathfrak{D}$ is the 2nd derivative of the loss function, and $\lambda$ is a weighting hyper-parameter.
        ii) An implicit enforcement of diversity by balancing the ensemble and the individual loss:
        \begin{equation}
            \label{eq:balanced}
            \frac{1}{M}\sum_{i=1}^{M}{\big(\lambda{l(f,y)} + \frac{1-\lambda}{M}\sum_{i=1}^{M}{l(h^{i},y)}\big)}
        \end{equation}
    
    \subsection{Diversity metrics in DL}
    
        There are many proposed metrics for diversity.
        \cite{BrownWHY05} presents a survey and taxonomy for diversity metrics and \cite{GongZH19} presents a survey of diversity for ML.
        In this work, we focus on behavior diversity metrics of a DL model.
        Input data diversity such as different modalities or implementation aspects such as the number of layers is not considered.
    
        \begin{figure}
    	\includegraphics[width=\linewidth]{figures/div_types.eps}\caption{Model behavior diversity.
            a) Invariant decision boundary \& diverse sample representation.
            b) Diverse decision boundary and measured by prediction errors.
            c) Diverse decision boundary and measured by feature relevance. } \label{figDivTypes}
        \end{figure}
    
        \textbf{Prediction (output,  failures) diversity}
        Multiple prediction-based diversity metrics have been proposed. \cite{KunchevaW03} presents a comprehensive evaluation of this class of metrics. Pair-wise measures based on the correct and incorrect statistics of two models include the Q-statistics, correlation coefficient $\rho$, and the disagreement measure:
        \begin{equation}
            \label{eq:disagreement}
            D_{p,q}=\frac{N^{01}+N^{10}}{N^{11}+N^{10}+N^{01}+N^{00}}
        \end{equation}
        where the indexes $a,b$ of the binary vectors $N^{ab}$ indicate the correctness of the classifiers, i.e.,  $(h^{i=p}=y)\Rightarrow (a=1) \wedge (h^{i=p}\neq{y})\rightarrow(a=0)$.        
        Non-pairwise measures that evaluate non-binary diversity include entropy, coincident failure diversity, cosine similarity, Kullback–Leibler divergence~\cite{DvornikMS19,NamYLL21}, and the the Shannon equitability index~\cite{peet1975relative,ensResilience}:
        \begin{equation}
            \label{eq:shannon}
            E = -\sum_{i=1}^{S}{p_{i}\ln{p_{i}}}\big/\ln{S}
        \end{equation}
        %which measures \textit{species} diversity
        where $S$ is the total number of prediction species/classes and $p_i$ is the proportion of observed species $i$.
        %In~\cite{Ben-CohenZBFZ21}, the term semantic diversity is used to refer to multiple tags or classes present in a single image which can lead to different correct predictions.
    
        \textbf{Representation (activation) diversity}
        The intermediate representations (IR) can also be used to measure diversity.
        \cite{deepEnsemLoss} compares the diversity in representation space from independently created ensembles and ensembles from variational approaches.
        Measuring IR diversity is challenging due to space size and semantic ambiguity, i.e., the same semantic concept can be represented in many different ways.
        A naive use of any diversity metric such as cosine similarity could give semantically irrelevant diversity scores.
        In~\cite{Kornblith0LH19}, the Centered Kernel Alignment (CKA) metric is proposed to obtain a statistical measure across a dataset on the similarity of any two layers of a DL model:
        \begin{equation}
            \label{eq:cka}
            \text{CKA}(K,L) = \frac{\text{HSIC}(K,L)}{\sqrt{\text{HSIC}(K,K)\text{HSIC}(L,L)}}
        \end{equation}
        where $K$ and $L$ are similarity matrices of the two feature maps being compared and HSIC is the Hilbert-Schmidt Independence Criterion which measures statistical independence.
        The feature maps may be layer activations or attention maps such as Saliency, Integrated Gradients, and Grad-CAM~\cite{SimonyanVZ13, SelvarajuCDVPB17, SundararajanTY17}.
        
        \cite{QiKL21} proposed the use of the pull-away loss term from generative adversarial networks to induce diversity of such activations.
        Self-attention~\cite{VaswaniSPUJGKP17} (not related to attention maps) is one of the key techniques in the transformer architecture.
        In~\cite{pacmac} the embeddings used to feed attention heads are masked in such a way as to enforce diversity of activations.
        In zero-shot learning, the \textit{attribute} concept is used to enable training models that can, later on, predict unseen labels.
        These attributes can be considered for IR diversity, as well ~\cite{ZhaoSWZ22}.
        Closely related to NCL, the self-supervised approach of contrastive learning~\cite{SchroffKP15,ChenK0H20} trains two models to produce latent features that are diverse for false positive cases and similar in true positives through a loss function such as the triplet loss that enforces the models to learn the similarity metric.
    
        \begin{figure}
            \includegraphics[width=\linewidth]{figures/nutshell2.eps}
    	\caption{Attribution map diversity: Two models may predict the same outcome but based on different \textit{evidence}.}\label{fig2}
        \end{figure}
    
        \textbf{Input feature attribution diversity}
        The importance of input features can be used to measure behavior diversity that to the best of our knowledge has not been explored.
        Figure~\ref{figDivTypes} shows the relationship of an attribution-based metric w.r.t. prediction and representation diversity.
        Note, that attribution is not the same as attention or attributes in the context of zero-shot learning.
        Attention maps, such as those obtained from the activation of intermediate layers of CNNs, reflect the excitation of a network given an input.
        This activation however is not necessarily correlated with the final prediction, e.g., it could be an inhibitory factor.
        Attribution, on the other hand, indicates the importance of a feature to the final decision. See Figure~\ref{fig2} where \textit{Saliency} is used to display the original pixels masked by the attribution scores from each model. A change to a pixel with high attribution (brighter) will have a stronger influence on the model prediction than a change to a pixel with low attribution.
    
    \subsection{Other diversity-based resilience approaches}
        Augmenting the training data by applying affine transformations such as rotations and scaling, geometric distortions such as blurring, and texture transfer help DL models to generalize better with a limited training data~\cite{dataAugMiko}.
        Adversarial training increases the robustness to intended attacks with adversarial samples to limit the model vulnerability to input perturbations~\cite{GoodfellowSS14}.
        Such training data approaches are effective and complementary to the design diversity approaches of this study that address the model diversity.
    
        Modality and point of view diversity~\cite{multimodal} is an approach to address the failure modes of sensors such as cameras, radar, and lidar.
        The design diversity of DL models explored in this study is orthogonal to this approach, as model diversity can be applied to every single modality.

\section{Methodology}
\label{labelDesignDiversity}

    We propose three sets of experiments:
        
    \begin{enumerate}
        \item Evaluate resiliency of diverse architectures and training approaches.
        \item Measure diversity of prediction and diversity of attribution from independently created models of diverse architectures and evaluate robustness correlation.
        \item Enforce diversity with NCL, evaluate the resulting robustness, and inspect three kinds of diversity: prediction, representation, and attribution.
    \end{enumerate}	
    
    In addition to addressing the main research questions, we put the following hypothesis to test: that attribution-based diversity (Equation~\ref{eq:attMetric}) can be positively correlated with ensemble resiliency if a better accuracy trade-off is achieved compared to prediction-based diversity.
    \begin{equation}
        A = {\sum_{c=1}^{C}{\sum_{p=1}^{P}{Var(a_{c,p})}}}
        \label{eq:attMetric}
    \end{equation}
    where $a$ is the input attribution score of a model at color channel $c$ and pixel coordinate $p$.
    The computation of the input attribution scores $a$ is performed with an attribution method, such as Saliency. %: $a_{c,p}=\frac{d}{dp}h$.

    \begin{figure}
	\includegraphics[width=\linewidth]{figures/formal_model_viz.eps}
	\caption{Relation between diverse design methodologies and the difficulty function in the LM Model~\cite{LMmodel}} 
        \label{fig_designdiv_space}
    \end{figure}
        
    This hypothesis is inspired by the theoretical result of the Littlewood and Miller (LM) model~\cite{LMmodel} that diverse design choices can produce less common failures. Diverse attribution maps of correct classifications imply that the models make predictions based on independent factors, which is not the case in prediction diversity.

    \subsection{Probability model for design diversity (LM model)}
            
        The Littlewood and Miller (LM) model~\cite{LMmodel} defines a probabilistic framework to analyze the impact of methodological diversity in the expected failure behavior.
        The model defines: 1) an input space $\mathcal{X}=\{x_1,x_2,...\}$, representing all possible inputs $x$ to a program and 2) a program space $\mathcal{P}=\{(\pi_1, \pi_2)\}$, for all possible programs $\pi$ that could implement a program specification.
        A given design methodology will determine the probability to come up with a program $\pi$ and is denoted as $S_{A}(\pi)$.
        Another design methodology $S_{B}(\pi)$ will assign a different probability to the same program.
        The model uses the concept of a difficulty function  $\theta_{M}(x)$ that measures the probability that a randomly chosen program $\pi$ from a given methodology distribution $S_{M}(\pi)$ will fail on a particular input $x \in \mathcal{X}$.
        The key insight consists in noticing that $\theta_{A}(x)$ can be different for a different methodology $\theta_{B}(x)$, i.e., for some methodology, a certain input may be difficult, but for another, it may be easy.
        See Figure~\ref{fig_designdiv_space} for a visual representation of these spaces.
        An analysis of this model concludes that if the design methodologies produce different difficulty functions $\theta$, then the expected failure behavior on a random input will be negatively correlated due to the fact that the covariance of the $\theta$'s can be negative.
            
        With this model, it is finally shown that a design methodology with diverse design choices that satisfy the following three properties will result in less common failures: 1) logically unrelated (one decision is independent of the other), 2) common failures of a decision are due to different factors, and 3) indifference to the selection of each methodology (no methodology is superior).

    \subsection{Loss function to enforce attribution diversity}
	\label{labelLossFunction}
    
        We perform a first attempt to enforce attribution diversity with the following loss:
        \begin{equation}
            \frac{1}{M}\sum_{i=1}^{M}{l(h^{i},y)}-\lambda{A}
            \label{eq:attDivEnf}
        \end{equation}
    
        This loss computes attribution scores variance in an ensemble and uses it as a penalty term weighted by $\lambda$.

    \subsection{Failures addressed}
    
        In this study, we evaluate resilience to covariate dataset distribution shifts, i.e., when the distribution of input features of the test dataset does not match the distribution of the training dataset. 
        We use four natural image perturbations from the \imagenetcbar dataset~\cite{MintunKX21}   that are sensible to occur in vision application domains, such as obstructions or liquid contaminants.
        Our scope is not to evaluate robustness against adversarial attacks, label shift variations, or resiliency to noise variations such as Gaussian, brown, etc.

\section{Experimental results}
\label{labelExperiments}

    \subsection{ML resiliency to data corruptions}
    \label{labelExpMlresiliency}

        To understand the relationship between accuracy, size, and resiliency to natural corruptions of DL models we evaluate a set of architectures (convolutional NNs, transformers, and subnetworks from neural architecture search (NAS)) and training approaches (supervised, self-supervised, and knowledge distillation) on both the ImageNet validation dataset and on the corrupted version \imagenetcbar "Lines" (strength of 1.6). See Table~\ref{tab:archCompa}.

        
        \begin{table}
            \footnotesize
            \caption{Resiliency of architectures to natural image corruption (\imagenetcbar Lines(1.6 strength))}\label{tab:archCompa}
            \resizebox{\linewidth}{!}{%
            \begin{tabular}{|c|c|c|c|c|c|c|c|c|}
            \hline
            Training & Arch. & DL model & \#  & \multicolumn{2}{c|}{ImageNet}  & \multicolumn{2}{c|}{\imagenetcbar}  \\
            approach & class &  & Params  &  \multicolumn{2}{c|}{} &  \multicolumn{2}{c|}{Lines1.6} \\
             &  & & & top1 & top5 & top1 & top5\\
            \hline
            Self-sup. & CNN &  DINO ResNet50~\cite{CaronTMJMBJ21} & 25.6M & 75.30 & 92.61 & 21.80 & 40.66\\
            Superv. & CNN &  ResNet50~\cite{HeZRS16} & 25.6M & 75.85 & 92.88 & 34.95 & 55.88\\
            Superv. & CNN &  ResNext50\_32\_4d~\cite{XieGDTH17} & 25.0M & 77.49 & 93.57 & 39.46 & 60.61\\
            Superv. & CNN &  ResNet152~\cite{HeZRS16} & 60.2M & 78.25 & 93.96 & 40.17 & 62.09\\
            Superv. & NAS &  EfficientNet\_b7~\cite{TanL19} & 66.3M & 74.82 & 92.13 & 51.11 & 73.81\\
            Superv. & CNN &  ResNext101\_64x4d~\cite{XieGDTH17} & 83.4M & 82.90 & 96.22 & 51.89 & 72.46\\
            Superv. & ViT &  Swin tiny~\cite{LiuL00W0LG21} & 28.3M & 81.34 & 95.612 & 55.29 & 78.45 \\
            Self-sup. & ViT &  DINO ViT\_b\_8~\cite{CaronTMJMBJ21} & 87.3M & 80.06 & 95.02 & 55.75 & 78.37\\
            Superv. & ViT &  Swin base~\cite{LiuL00W0LG21} & 87.8M & 83.30 & 96.46 & 58.71 & 81.05\\
            Superv. & ViT &  ViT base p16~\cite{DosovitskiyB0WZ21} & 86.6M & 80.88 & 95.28 & 61.28 & 82.26\\
            K. distill. & ViT &  DeiT-base~\cite{TouvronCDMSJ21} & 87.3M & 83.34 & 96.50 & 64.07 & 84.66\\
            Self-sup. & ViT &  SwinV2~\cite{Liu0LYXWN000WG22} & 87.9M & 83.34 & 96.44 & 59.976 & 81.70\\
            \hline
            \end{tabular}
        }
        \end{table}
        
        \textbf{Observations to Table~\ref{tab:archCompa}:} Although the model size is highly correlated with the final accuracy and resilience in the corrupted dataset, the architecture seems to play a more determining factor.
        The smallest transformer with only 28M parameters is superior to other CNNs with 2 or 3x more parameters.
        Self-supervision slightly decreases both metrics as appreciated in the comparison of ResNet50 and ViT models using supervised learning.
        SwinV2 is an exception but this model introduced more architectural innovations too.
        Knowledge distillation from a CNN teacher shows a slight improvement over supervised ViTs.

        To understand the effect of other corruptions, we evaluate a ResNet50 model\footnote{In the remainder of this paper, we select ResNet50 as the baseline architecture due to its common use as a reference.} on six different data sets: ImageNet~\cite{DengDSLL009}, ImageNetv2~\cite{RechtRSS19}, and four corruptions on a fixed perturbation strength from the \imagenetcbar dataset~\cite{MintunKX21}: Plasma (4.0), Checkerboard(4.0), Waterdrop(7.0) and Lines (1.6). See Figure~\ref{fig:indivAcc}.        
        The first two are in-distribution, i.e., the covariates (input features) and labels of the validation set follow a similar distribution to the training data set.
        The last four are out-of-distribution, as the model has never seen such corruptions of the input images during training.
            
        \begin{figure}
            \includegraphics[width=\linewidth] {figures/individual_accuracy2.eps}
		\caption{Top-1 accuracy of ResNet50 on in-distribution data sets (ImageNet and ImageNetv2\cite{RechtRSS19}) and out-of-distribution datasets (\imagenetcbar\kern-0.5em). The resilience of ResNet50 drops significantly against natural corruptions.} \label{fig:indivAcc}
	\end{figure}

        \textbf{Observations to Figure~\ref{fig:indivAcc}:} Different corruptions have different effects on the model performance, and a typical ``good" classifier with accuracy close to 80\% can have a tremendous performance decrease in situations where a human would probably not.
		
    \subsection{Diversity of ensembles from heterogeneous architectures}
    \label{labelExpHetero}

        To understand the diversity/accuracy trade-off of the attribution-based metric in comparison to the established prediction-based diversity approach we perform two different experiments: First, we create multiple ensembles of independently trained models with a wide diversity in architecture. Second, we create multiple ensembles using models discovered in a weight-sharing super-network~\cite{CaiGWZH20}, i.e., models whose architecture has been found using neural architecture search (NAS) and not by manual design.

        The architectures explored here are CNNs (ResNext~\cite{XieGDTH17} \& SqueezeNet~\cite{IandolaMAHDK16}), Vision transformers (DeiT~\cite{TouvronCDMSJ21}) and NAS (MNASNET~\cite{TanCPVSHL19} \& BootstrapNAS~\cite{munoz2022enabling}) using supervised or self-supervised training\footnote{Details on the architecture and optimization hyper-parameters can be found in Section A of the supplementary material.}. In total 14 models were trained with different hyperparameters to create three-member ensembles of all possible combinations.

        Figure~\ref{fig:heteroDisLinear} shows the ensemble performance of all 364 ensembles created from these 14 models using an averaging consensus mechanism, i.e., the logit output of all ensemble members is averaged first and then the highest score is used to make the prediction.
        The left-hand side shows on the X-axis the attribution-based diversity metric (Eq.~\ref{eq:attMetric}).
        The right-hand side shows the disagreement prediction diversity metric (Eq.~\ref{eq:disagreement}).
        Each point is an ensemble evaluated on the entire validation dataset of ImageNet.
        The color indicates the final average accuracy of the ensemble.
        The Y-axis indicates the average benefit of creating an ensemble: $Y= A_{ens}-A_{top}$, where $A_{ens}$ is the ensemble accuracy and $A_{top}$ is the accuracy of its most accurate member, i.e., how much accuracy improvement was obtained in comparison to a single model (the most accurate in the ensemble).
        In this way, it can be appreciated when an ensemble makes sense: it has to lay above the zero line (dashed).
        The ensemble cost is measured by the number of parameters which has a direct influence on the memory and the number of operations required.
        The ideal ensemble is one with the brightest color, smallest radius, and residing above zero.
        
        \begin{figure}[htp]
            \subcaptionbox{Attribution diversity\label{fig:hLa}}{\includegraphics[width=0.5\linewidth]{figures/fig_hetero3_exp_linear_combination_DIFF_imagenet.eps}}\hspace{0em}%
            \subcaptionbox{Prediction disagreement\label{fig:hLb}}{\includegraphics[width=0.5\linewidth]{figures/fig_hetero3_exp_linear_combination_OUT_DIS_imagenet.eps}}
            \caption{Evaluation on ImageNet validation dataset of 364 three-member ensembles from heterogeneous architectures using \textbf{averaging as consensus mechanism}. Y-axis: Improvement of the ensemble against its own top ensemble member. X-axis: Normalized diversity metric. Color: absolute ensemble accuracy. Bubble size: Model parameter size. The attribution diversity metric is not negatively correlated with the ensemble improvement as disagreement diversity is.}
        \label{fig:heteroDisLinear}
        \end{figure}

        \textbf{Observations to Figure~\ref{fig:heteroDisLinear}:} The figure shows how attribution diversity is positively correlated with ensemble improvement, while it corroborates the known fact that prediction-based diversity is negatively correlated~\cite{KunchevaW03}.

        Figure~\ref{fig:heteroDisVoting} shows the same ensemble combinations, but this time using a majority voting consensus mechanism, i.e., the prediction with the highest number of votes wins. Draws are randomly resolved.

        \begin{figure}[htp]
            \subcaptionbox{Attribution diversity\label{fig:hDVa}}{\includegraphics[width=0.5\linewidth]{figures/fig_hetero3_exp_voter_DIFF_imagenet.eps}}\hspace{0em}%
            \subcaptionbox{Prediction disagreement\label{fig:hDVb}}{\includegraphics[width=0.5\linewidth]{figures/fig_hetero3_exp_voter_OUT_DIS_imagenet.eps}}
            \caption{Evaluation on ImageNet validation dataset of the same ensembles as in Figure~\ref{fig:heteroDisLinear} but using \textbf{voting as consensus mechanism}. In contrast to averaging, voting produces mostly ensembles that decrease the final performance instead of improving it.}
            \label{fig:heteroDisVoting}
        \end{figure}

        \textbf{Observations to Figure~\ref{fig:heteroDisVoting}:} The same correlation trends can be observed with majority voting. However, the most interesting aspect is that the vast majority of the ensembles here reside under the zero line.
        This means that majority voting with three ensembles tends on average to produce less accurate models. This corroborates the findings of~\cite{KunchevaW03}.

        We evaluate the same ensembles on five more validation datasets and verify that the observed trend in the validation dataset applies to natural corruptions.
        In addition, we compare the two diversity metrics to a simple validation accuracy metric.
        See Figure~\ref{fig:heteroODD}.

        \begin{figure}[htp]
            \subcaptionbox{\centering Attr. ImageNetv2\label{fig3:odda}}{\includegraphics[width=0.2\linewidth]{figures/fig_hetero3_exp_linear_combination_DIFF_imagenetv2.eps}}%
            \hspace{0em}%
            \subcaptionbox{\centering Attr. Waterdrop\label{fig3:oddb}}{\includegraphics[width=0.2\linewidth]{figures/fig_hetero3_exp_linear_combination_DIFF_waterdrop_7.eps}}%
            \hspace{0em}%
            \subcaptionbox{\centering Attr. Lines\label{fig3:oddc}}{\includegraphics[width=0.2\linewidth]{figures/fig_hetero3_exp_linear_combination_DIFF_lines.eps}}%
            \hspace{0em}%
            \subcaptionbox{\centering Attr. Plasma\label{fig3:oddd}}{\includegraphics[width=0.2\linewidth]{figures/fig_hetero3_exp_linear_combination_DIFF_plasma_noise.eps}}%
            \hspace{0em}%
            \subcaptionbox{\centering Attr. Checkerb.\label{fig3:odde}}{\includegraphics[width=0.2\linewidth]{figures/fig_hetero3_exp_linear_combination_DIFF_checkboard.eps}}%
            \hspace{0em}%
            \subcaptionbox{\centering Disagr. ImageNetv2\label{fig3:oddf}}{\includegraphics[width=0.2\linewidth]{figures/fig_hetero3_exp_linear_combination_OUT_DIS_imagenetv2.eps}}%
            \hspace{0em}%
            \subcaptionbox{\centering Disagr. Waterdrop\label{fig3:oddg}}{\includegraphics[width=0.2\linewidth]{figures/fig_hetero3_exp_linear_combination_OUT_DIS_waterdrop_7.eps}}%
            \hspace{0em}%
            \subcaptionbox{\centering Disagr. Lines\label{fig3:oddh}}{\includegraphics[width=0.2\linewidth]{figures/fig_hetero3_exp_linear_combination_OUT_DIS_lines.eps}}%
            \hspace{0em}%
            \subcaptionbox{\centering Disagr. Plasma\label{fig3:oddi}}{\includegraphics[width=0.2\linewidth]{figures/fig_hetero3_exp_linear_combination_OUT_DIS_plasma_noise.eps}}%
            \hspace{0em}%
            \subcaptionbox{\centering Disagr. Checkerb.\label{fig3:oddj}}{\includegraphics[width=0.2\linewidth]{figures/fig_hetero3_exp_linear_combination_OUT_DIS_checkboard.eps}}%
            \hspace{0em}%
            \subcaptionbox{\centering Acc. ImageNetv2\label{fig3:oddk}}{\includegraphics[width=0.2\linewidth]{figures/fig_hetero3_exp_justAcc_imagenetv2.eps}}%
            \hspace{0em}%
            \subcaptionbox{\centering Acc. Waterdrop\label{fig3:oddl}}{\includegraphics[width=0.2\linewidth]{figures/fig_hetero3_exp_justAcc_waterdrop_7.eps}}%
            \hspace{0em}%
            \subcaptionbox{\centering Acc. Lines\label{fig3:oddm}}{\includegraphics[width=0.2\linewidth]{figures/fig_hetero3_exp_justAcc_lines.eps}}%
            \hspace{0em}%
            \subcaptionbox{\centering Acc. Plasma\label{fig3:oddn}}{\includegraphics[width=0.2\linewidth]{figures/fig_hetero3_exp_justAcc_plasma_noise.eps}}%
            \hspace{0em}%
            \subcaptionbox{\centering Acc. Checkerb.\label{fig3:oddo}}{\includegraphics[width=0.2\linewidth]{figures/fig_hetero3_exp_justAcc_checkboard.eps}}
            \caption{Comparison of \textbf{trend lines}, i.e., correlation of improvement to three different metrics on five validation datasets (consensus: averaging). Columns: Datasets. Rows: Metrics. The attribution metric (a to e) has a much lower accuracy trade-off compared to the disagreement metric (f to j) but is higher than the average accuracy metric (k to o).
            Bigger plots to appreciate individual ensembles are presented in Section A.2 of the supplemental material.
            }
            \label{fig:heteroODD}
        \end{figure}

        \textbf{Observations to Figure~\ref{fig:heteroODD}:} Attribution-based diversity is better correlated as well.
        These results serve as evidence to confirm that the diversity-accuracy trade-off is better for attribution than for prediction diversity.
        However, the metric of averaging the individual accuracies of the ensemble members is more strongly correlated with the ensemble improvement in corruptions.

        Next, Figure~\ref{fig:heteroAutoML} presents the results of the second experiment on architectures created with NAS.
        We used the open-source framework BootstrapNAS \cite{munoz2022enabling, munoz2022automated} to create a weight-sharing super-network.
        The super-network is trained from an initial ResNet50 model.
        We then sample 11 subnetworks with different configurations but similar complexity by varying the width and depth of the CNN.

        \begin{figure}[htp]
            \subcaptionbox{Attribution\label{fig:Autoa}}{\includegraphics[width=0.33\linewidth]{figures/fig_BNAS_exp_linear_combination_DIFF_imagenet.eps}}\hspace{0em}%
            \subcaptionbox{Disagreement\label{fig:Autob}}{\includegraphics[width=0.33\linewidth]{figures/fig_BNAS_exp_linear_combination_OUT_DIS_imagenet.eps}}
            \subcaptionbox{Accuracy\label{fig:Autoc}}{\includegraphics[width=0.33\linewidth]{figures/fig_BNAS_exp_justAcc_imagenet.eps}}        
            \caption{Comparison of 165 heterogeneous ensembles of \textbf{architectures automatically created with weight-sharing neural architecture search (NAS)}. 11 models with different architectures were selected to be very close in complexity, i.e., number of parameters. The correlation between the two diversity metrics is highly positive but the increment in performance is very small.} \label{fig:heteroAutoML}
        \end{figure}

        \textbf{Observations to Figure~\ref{fig:heteroAutoML}:} Although the correlations seem strong for all metrics, the actual ensemble improvement is very low, i.e., less than 0.04\%.

        To identify the effect of the complexity of the chosen attribution method, we evaluated the pair-wise diversity on the entire validation set on six subnetworks using \textit{Saliency} and \textit{Integrated Gradients} attribution methods using 1, 2, 10, and 50 backpropagation passes. See Figure~\ref{fig:attMethodsComp}.
        The average correlation coefficient of the normalized diversity scores of all methods is 0.998.
        Using Saliency-based attribution is then justified as it provides the lowest performance penalty as the computational overhead grows linearly with the number of backpropagation passes.

        \begin{figure}[htbp]
            \centering
            \includegraphics[width=\linewidth]{figures/att_methods.eps}
            \caption{Comparison of Saliency and Integrated Gradients with 2, 10 \& 50 samples.
            The X-axis represents different pairs of models created with BootstrapNAS.
            The Y-axis is the normalized diversity score of each method.
            }\label{fig:attMethodsComp}
        \end{figure}
		
    \subsection{Enforcing diversity in homogeneous ensembles}
    \label{labelExpHomo}

        We perform a set of training experiments to enforce diversity into the ensembles through the loss function via the Negative Correlation Learning paradigm.
        We use ResNet50 for all ensemble members, and evaluate different heuristics: \\
        a) Independently trained members using cross-entropy as loss in Equation~\ref{eq:ind}.
        Four different consensus approaches in GNCL using Equation~\ref{eq:gncl_i}:
        b) average,
        c) median,
        d) geometric mean,
        e) majority vote,
        f) GNCL and averaging consensus but masking the penalty term for incorrect classifications, i.e., $ (h^{i} \neq y) \Rightarrow (\lambda=0)$, and
        g) Balancing a loss function between the team and individual members (Equation~\ref{eq:balanced}).
        The optimization method in all cases was AdaBelief~\cite{ZhuangTDTDPD20} for 100 epochs with a learning rate of 1e-3 decaying 10\% every 30 epochs, epsilon of 1e-8, betas: (0.9,0.999), batch size of 64 and a $\lambda$ factor of 0.2.
        In ImageNet classification, we empirically observed that bigger $\lambda$ values in Equation~\ref{eq:gncl_i} fail to learn.
        Results for the six heuristics are presented in Figure~\ref{fig:gncl}.
		
        \begin{figure}[htp]
            {\includegraphics[width=\linewidth]{figures/Aggregate_accuracy.eps}}\hspace{0em}%
            \caption{The resulting accuracy and \textbf{resiliency to natural corruptions of diversity enforcement} based on Negative Correlation Learning with different consensus mechanisms.
            Each ensemble has three members with a ResNet50 architecture.
            The heuristic of balancing the loss of the ensemble and the loss of the individual member produced the most resilient ensemble to corruptions.
            } \label{fig:gncl}
        \end{figure}

        \textbf{Observations to Figure~\ref{fig:gncl}:} Explicit enforcement of prediction diversity does not result in improved resilience. However the balanced loss (Eq.~\ref{eq:balanced}) provides a significant advantage in 3 out of 4 natural corruptions.

    \subsubsection{First attempt at enforcing attribution diversity}

        We perform a first attempt to enforce attribution diversity using the loss of Equation~\ref{eq:attDivEnf} and the same optimization parameters used in GNCL.
        The computational overhead to calculate the attributions is 2x using the Saliency method.
        Empirically, we tried five different lambda weights values: \{10, 1,0.1,0.01,0.001\} but found training instabilities.
        The smallest $\lambda$ value resulted in convergence up to epoch 21 for 63.7\% top1 accuracy.
        We believe that the penalty term of Eq.~\ref{eq:attDivEnf} is in conflict with the original loss and it would be more appropriate to investigate a better penalty term than to optimize this hyper-parameter in future work.

    \subsubsection{Diversity of NCL ensembles}

        Figures~\ref{fig:attmapsAll},~\ref{fig:cka} together with Table~\ref{tab:shanon} show three types of diversity (attribution, prediction, and intermediate representation) for three models created through independently created heterogeneous architectures, prediction diversity enforcement and attribution diversity enforcement with the following top1 accuracies on the ImageNet validation dataset: 78.2\%, 76.1\%, and 63.7\%.

        \textbf{Prediction diversity.}
        In Table~\ref{tab:shanon}, the Shannon equitability index metric (Eq.~\ref{eq:shannon})  is shown for correctly and incorrectly classified samples for three ensembles: attribution diversity (Eq.~\ref{eq:attDivEnf}), prediction diversity (Eq.~\ref{eq:disagreement}) and heterogeneous architectures on all six datasets.
        The heterogeneous ensemble produces more diverse predictions in general.
		
	\begin{table}
            \footnotesize
            \caption{\textbf{Diversity of predictions} from all members of three ensembles as measured by the Shanon equitability index $H$. $H_{corr}$ and $H_{inco}$ indicate the metric computed on all samples that were correctly or incorrectly classified. The six subcolumns correspond to the six validation datasets.}\label{tab:shanon}
            \resizebox{\linewidth}{!}{%
            \begin{tabular}{|l|c|c|c|c|c|c|c|c|c|c|c|c|}
            \hline
             &\multicolumn{6}{c|}{$H_{corr}$}& 
             \multicolumn{6}{c|}{$H_{inco}$}\\
            &IN&I2&WD&LI&PL&CB& IN&I2&WD&LI&PL&CB\\
            \hline
            Att. div.&\underline{0.13}&0.16&0.29&0.32&0.23&0.23&0.46&0.49&0.62&0.60&0.57&0.58\\
            Pred. div.&0.10&0.13&0.29&0.31&0.21&0.23&0.46&0.49&0.67&0.69&0.59&0.63\\    Hetero.&0.12&\underline{0.17}&\underline{0.35}&\underline{0.37}&\underline{0.25}&\underline{0.29}&\underline{0.51}&\underline{0.55}&\underline{0.74}&\underline{0.77}&\underline{0.67}&\underline{0.71}\\
            \hline
            \end{tabular}
            }
        \end{table}

        \textbf{Attribution diversity.}
        We present a few resulting attribution maps in Figure~\ref{fig:attmapsAll} for the NCL-based prediction-diversity enforcement, attribution-diversity enforcement (at epoch 21), and independently trained architectures.
    

        \begin{figure}
		\includegraphics[width=\linewidth]{figures/att_maps_all.eps}
		\caption{\textbf{Attribution map diversity} of different diversity-inducing techniques on 8 ImageNet val. dataset samples.} \label{fig:attmapsAll}
	\end{figure}

        \textbf{Observations to Figure~\ref{fig:attmapsAll}:} Independently trained heterogeneous architectures and attribution-diversity enforcement produce more diverse attribution maps than homogeneous models trained to have diverse prediction outcomes.

        \textbf{Representation diversity.}
        In Figure~\ref{fig:cka}, we investigate the resulting diversity/similarity of the internal layers via CKA (Equation~\ref{eq:cka}) of two ensemble members for three different diversity enforcing techniques.

        \begin{figure}[htp]
            \centering
            \subcaptionbox{\centering Attribution diversity enf.\label{fig:ckaa}}{\includegraphics[width=0.26\linewidth]{figures/cka_att_ImageNet.eps}}\hspace{1em}%
            \subcaptionbox{\centering Prediction diversity enf.\label{fig:ckab}}{\includegraphics[width=0.26\linewidth]{figures/cka_out_ImageNet.eps}}\hspace{1em}%
            \subcaptionbox{\centering Heterog. architectures\label{fig:ckac}}{\includegraphics[width=0.26\linewidth]{figures/cka_hetero_ImageNet.eps}}\hspace{1em}%
            %\subcaptionbox{Scale\label{fig:ckacolor}}
            {\includegraphics[width=0.08\linewidth]{figures/cka_colorbar.eps}}
            \caption{Centered Kernel Alignment maps to visualize the resulting \textbf{similarity of model layers} on different diversity enforcing techniques.} \label{fig:cka}
        \end{figure}

        \textbf{Observations to Figure~\ref{fig:cka}:} The CKA visualization reflects that the enforcement of attribution diversity produces less similarity in the layers than output diversity enforcement or by independent heterogeneous architectures.

\section{Discussion and conclusions}
\label{labelDiscussion}

    In this section, we discuss and interpret the results of Section~\ref{labelExperiments} and summarize the research questions' answers.

    \subsection{Answers to research questions}
        \refreq{accRes}: In our experiments, it is observed that model architecture is more important to resiliency than model accuracy or size.
        On \refreq{att}, we consistently observed that attribution-based diversity is more positively correlated with accuracy than prediction-based disagreement diversity.
        Answering \refreq{bestEnf}, balancing the loss of the individual members and the ensemble provided a significant advantage in 3 out of 4 natural corruptions when compared to the prediction diversity enforcement variants.
        \refreq{divNCL}: Prediction diversity was higher for heterogeneous architectures trained independently than NCL on prediction or attribution diversity.
        Attribution diversity is significantly lower when enforcing prediction diversity compared to heterogeneous architectures trained independently.
        Activation diversity is low at the last layers for both prediction and attribution diversity enforcement, while for heterogeneous architectures trained independently, the middle layers showed less diversity.
        
    \subsection{Results discussions}

        \textbf{Advantage of Transformers vs CNNs:} The superiority of the transformer architecture in terms of resilience against natural corruptions could be attributed to their capability to pay attention globally instead of locally as CNNs do, and thus they may be able to construct more useful intermediate representations that suffer less from perturbations.

        \textbf{NAS ensembles in our experiments are not diverse enough:} Only small improvements are obtained from ensembles created from sub-networks.
        These are jointly trained as part of a single weight-sharing super-network and thus have little diversity. Larger search/design spaces should be explored in future work.

        \textbf{Balancing member vs ensemble accuracy:} Explicitly enforcing prediction diversity is outperformed by implicit enforcement through the balance of ensemble and individual accuracy (Equation~\ref{eq:balanced}).
        This could be due to the use of two less conflicting objectives than prediction diversity.

        \textbf{Diversity-accuracy trade-off improvement:} In contrast to the disagreement metric, attribution diversity does not require models to deviate from a correct prediction.
        The correlation is however not very strong as models tend to find similar features for prediction, and implicit attribution deviations may be the product of imperfect learning.
        The enforcement of attribution diversity with the proposed loss proved however to be insufficient and better heuristics need to be explored.
        
        \textbf{Diverse architectures produced more diversity than NCL-based methods}
        An ensemble selected from a combination of independently trained heterogeneous architectures and training approaches resulted in higher levels of diversity in prediction, attribution, and intermediate layers.
        However, mixing different architectures does not consistently produce good ensembles as observed in the many ensembles under the zero line in Figure~\ref{fig:heteroODD}, as a good model may not benefit from ensembling with a less good one, as their common modes are not negatively correlated.

    \subsection{Conclusions and next steps}

        In this study, we explored different approaches to measure and enforce diversity in ensembles and evaluated their impact on natural data corruption resiliency.
        The key takeaways are:
        1) model architecture is more important for resiliency than model size or model accuracy,
        2) attribution-based diversity is less negatively correlated to the ensemble accuracy than prediction-based diversity,
        3) a balanced loss function of individual and ensemble accuracy creates more resilient ensembles for image natural corruptions, and
        4) architecture diversity produces more diversity in all explored diversity metrics: predictions, attributions, and activations.
        
        In addition, other valuable findings are:
        a) Saliency attribution can be sufficient to measure input attribution diversity,
        b) Ensembles created from models of similar complexity that were discovered by weight-sharing Neural Architecture Search for our experiments barely provide any accuracy improvement, and
        c) Enforcing attribution-based diversity during training through a variance-based penalty term is not stable and needs further research.
        
        In future work, several experiments could be done to understand the
        complexity-resiliency trade-off, e.g., through knowledge distillation,
        improved heuristics to enforce attribution diversity, and compare diversity approaches in tasks beyond image classification.

\paragraph{Acknowledgements}

This work was partially funded by the Federal Ministry for Economic Affairs and Climate Action of Germany, as part of the research project SafeWahr (Grant Number: 19A21026C).

We would also like to thank Professors Lorenzo Strigini and Peter Popov for the fruitful conversations and feedback on this work.

{\small
\bibliographystyle{ieee_fullname}
\bibliography{egbib}
}

%%%%%%%%% SUPLEMENTAL MATERIAL
\onecolumn

\renewcommand{\thesection}{\Alph{section}}
\renewcommand{\thetable}{\roman{table}}
\renewcommand{\thefigure}{\Alph{figure}}

\setcounter{section}{0}
\setcounter{table}{0}
\setcounter{figure}{0}


\section{Supplemental material}

\subsection{Training parameters for heterogeneous architectures}

Table~\ref{tab:tParams} shows the architecture and optimization hyper-parameters used for training the  models used in our experiments.

\begin{table}[H]
\centering
\resizebox{\linewidth}{!}{
\begin{tabular}{|c|c|c|c|c|c|c|}
\hline
id & Architecture & Optimizer & Parameters & Scheduler & Epochs &
BatchSize\\
\hline
ResNext50\_32\_2&ResNext50 cardinality=32, blockWidth=2 layers=[3;4;6;3], dropout=0.2 & SGD & lr=0.1 decay=0.0001, momentum=0.9 & Step: $\gamma$=0.1, epochs=30 & 100 & 32\\
\hline
ResNext50\_32\_4l&ResNext50 cardinality=32, blockWidth=4 layers=[2;2;3;2], dropout=0.2 & SGD & lr=0.1 decay=0.0001, momentum=0.9 & Step: $\gamma$=0.1, epochs=30 & 100 & 32\\
\hline
ResNext50\_16\_4&ResNext50 cardinality=16, blockWidth=4 layers=[3;4;6;3], dropout=0.2 & SGD & lr=0.1 decay=0.0001, momentum=0.9& Step: $\gamma$=0.1, epochs=30& 100 & 32\\
\hline
MNASNET\_1&MNASNET ratio=1, dropout=0.2 & SGD & lr=0.1 decay=0.0001, momentum=0.9& Step: $\gamma$=0.1, epochs=30& 100 & 32\\
\hline
MNASNET\_1p&MNASNET ratio=1, dropout=0.2 & RMSprop & lr=0.256, decay=0.9, momentum=0.9& Step: $\gamma$=0.97, epochs=2.4& 100 & 32\\
\hline
Squeezenet\_512 & Squeezenet version=1.1 & SGD & lr=0.01 decay=0.0002, momentum=0.9& Step: $\gamma$=0.1, epochs=30& 100 & 512\\
\hline
Squeezenet & Squeezenet version=1.1 & SGD & lr=0.01 decay=0.0002, momentum=0.9& Step: $\gamma$=0.1, epochs=30& 100 & 128\\
\hline
\multirow{2}{*}{bootstrapNAS-B\_0} & ResNet50 depth=[0, 0, 0, 0, 1], width=[0, 0, 0, 2, 2, 2]&\multirow{2}{*}{SGD} &\multirow{2}{*}{See Listing~\ref{lst:progShr}} &\multirow{2}{*}- &\multirow{2}{*}-&\multirow{2}{*}-\\
& expansion=[0.2, 0.2, 0.2, 0.25, 0.2, 0.25, 0.25, 0.25,0.2, 0.25, 0.25, 0.25] & & & & &\\
\hline
\multirow{2}{*}{bootstrapNAS-B\_1} & ResNet50 d=[0, 0, 0, 0, 1], w=[0, 0, 0, 2, 2, 2]&\multirow{2}{*}{SGD} &\multirow{2}{*}{See Listing~\ref{lst:progShr}} &\multirow{2}{*}- &\multirow{2}{*}- & \multirow{2}{*}-\\
& expansion=[0.25, 0.2, 0.25, 0.25, 0.2, 0.25, 0.25, 0.25, 0.2, 0.25, 0.25, 0.25] & & & & &\\
\hline
deit\_tiny\_p16\_224 & DeiT size=tiny(3 heads), patchSize=16, embedding=192& AdamW & lr=0.0005 & Cosine & 300 & 256\\
\hline
\multirow{2}{*}{dino\_deit\_tiny1} & DINO DeiT size=tiny, patchSize=16, embedding=192, & \multirow{2}{*}{AdamW} & \multirow{2}{*}{lr=0.0005} & \multirow{2}{*}{Cosine} & Backbone=300 & \multirow{2}{*}{256}\\
&seed=0, localCropsScale=\{0.05,0.2\}, globalCropsScale=\{0.4, 1.\}&&&& Classifier=100&\\
\hline
\multirow{2}{*}{dino\_deit\_tiny2} & DINO DeiT size=tiny, patchSize=16, embedding=192, & \multirow{2}{*}{AdamW} & \multirow{2}{*}{lr=0.0005} & \multirow{2}{*}{Cosine} & Backbone=300 & \multirow{2}{*}{256}\\
&seed=7, localCropsScale=\{0.05,0.2\}, globalCropsScale=\{0.4, 1.\}&&&& Classifier=100&\\
\hline
\multirow{2}{*}{dino\_resnet1} & DINO ResNet50 backboneSeed=7 localCropsScale=\{0.05,0.14\} & \multirow{2}{*}{SGD} & \multirow{2}{*}{decay=0.0001, lr=0.03} & \multirow{2}{*}{-} & Backbone=300 & \multirow{2}{*}{256}\\
&globalCropsScale=\{0.14, 1.\}, classifierSeed=5&&&& Classifier=100&\\
\hline
\multirow{2}{*}{dino\_resnet2} & DINO ResNet50 backboneSeed=7 localCropsScale=\{0.05,0.14\} & \multirow{2}{*}{SGD} & \multirow{2}{*}{decay=0.0001, lr=0.03} & \multirow{2}{*}{-} & Backbone=300 & \multirow{2}{*}{256}\\
&globalCropsScale=\{0.14, 1.\}, classifierSeed=15&&&& Classifier=100&\\
\hline
\end{tabular}
            }
\caption{Training architectures and parameters used to create ensembles of heterogeneous architectures}
\label{tab:tParams}
\end{table}

Listing~\ref{lst:progShr} shows an example configuration file to create a super-network using a pre-trained model from Torchvision.
The configuration parameters are used by BootstrapNAS in the Neural Network Compression Framework (NNCF) and specify the size and elasticity of the super-network.  Once the super-network has been trained, the user can extract models of different sizes and performances.

{\setstretch{0.5}\tiny
\begin{lstlisting}[label=lst:progShr, linewidth=\columnwidth, language=json, caption=Super-network configuration example for NNCF's BootstrapNAS]
# Insert here model and dataset fields
# Insert here optimizer fields
"bootstrapNAS":{
    "training": {
        "algorithm":"progressive_shrinking",   
        "progressivity_of_elasticity": ["depth", "width"], 
        "batchnorm_adaptation": {
            "num_bn_adaptation_samples": 1500},
    "schedule": { 
        "list_stage_descriptions": [
        {"train_dims": ["depth"], "epochs": 25, 
        "depth_indicator": 1, "init_lr": 2.5e-6, 
        "epochs_lr": 25},
        {"train_dims": ["depth"], "epochs": 40, "depth_indicator": 2, "init_lr": 2.5e-6, "epochs_lr": 40},
        {"train_dims": ["depth", "width"], "epochs": 50, "depth_indicator": 2, "reorg_weights": true, "width_indicator": 2, "bn_adapt": true, "init_lr": 2.5e-6, "epochs_lr": 50},
        {"train_dims": ["depth", "width"], "epochs": 50, "depth_indicator": 2, "reorg_weights": true, "width_indicator": 3, "bn_adapt": true, "init_lr": 2.5e-6, "epochs_lr": 50}
        ]
    }, 
    "elasticity": {            
        "available_elasticity_dims": ["width", "depth"],
        "width": {
        "max_num_widths": 3,
        "min_width": 32,
        "width_step": 32, 
        "width_multipliers": [1, 0.80, 0.60]
        }
    }    
},
    "search": {
        "algorithm": "NSGA2",
        "num_evals": 1000,
        "population": 50,
        "ref_acc": 93.65
    }
}
\end{lstlisting}
}

The parameters shown in Table~\ref{tab:nasParam} indicate the configuration of the subnetworks extracted from the super-network in our
experiments. We used a previous version of BootstrapNAS in our experiments, which extends Once-for-all (OFA) super-networks from Cai et al. [4] and follows its conventions to describe the search space. In newer versions of BootstrapNAS, expansion ratios are handled by an elastic width handler, and the search space description follows a different convention. 

\begin{table}[H]
\centering
\resizebox{\linewidth}{!}{
\begin{tabular}{|c|c|}
\hline
id & Subnetwork Configurations \\\hline
B\_0 & depth: [0, 0, 0, 0, 1], expansion: [0.2, 0.2, 0.2, 0.25, 0.2, 0.25, 0.25, 0.25, 0.2, 0.25, 0.25, 0.25], width: [0, 0, 0, 2, 2, 2], \\\hline
nB\_0 & depth: [0, 0, 0, 0, 1], expansion: [0.2, 0.2, 0.2, 0.25, 0.2, 0.25, 0.25, 0.25, 0.2, 0.25, 0.25, 0.25], width : [0, 0, 0, 2, 2, 2]\\\hline
nB\_1 & depth: [0, 0, 0, 0, 1], expansion:  [0.25, 0.2, 0.25, 0.25, 0.2, 0.25, 0.25, 0.25, 0.2, 0.25, 0.25, 0.25], width : [0, 0, 0, 2, 2, 2]\\\hline
nB\_2 & depth: [0, 0, 0, 0, 1], expansion:  [0.25, 0.2, 0.25, 0.2, 0.2, 0.25, 0.25, 0.25, 0.2, 0.25, 0.25, 0.25], width : [0, 0, 0, 2, 2, 2]\\\hline
nB\_3 & depth: [0, 0, 0, 0, 1], expansion:  [0.25, 0.2, 0.25, 0.25, 0.2, 0.2, 0.25, 0.25, 0.2, 0.25, 0.25, 0.25], width : [0, 0, 0, 2, 2, 2]\\\hline
dB\_1a & depth: [0, 0, 1, 0, 0], expansion: [0.25, 0.2, 0.25, 0.25, 0.2, 0.25, 0.25, 0.25, 0.2, 0.25, 0.25, 0.25], width : [0, 0, 0, 2, 2, 2]\\\hline
dB\_1b & depth: [1, 0, 0, 0, 0], expansion: [0.25, 0.2, 0.25, 0.25, 0.2, 0.25, 0.25, 0.25, 0.2, 0.25, 0.25, 0.25], width : [0, 0, 0, 2, 2, 2],\\\hline
wB\_1a & depth: [0, 0, 0, 0, 1], expansion: [0.25, 0.2, 0.25, 0.25, 0.2, 0.25, 0.25, 0.25, 0.2, 0.25, 0.25, 0.25], width : [0, 0, 1, 1, 2, 2],\\\hline
wB\_1b & depth: [0, 0, 0, 0, 1], expansion: [0.25, 0.2, 0.25, 0.25, 0.2, 0.25, 0.25, 0.25, 0.2, 0.25, 0.25, 0.25], width : [0, 1, 1, 1, 1, 2],\\\hline
wB\_1c & depth: [0, 0, 0, 0, 1], expansion: [0.25, 0.2, 0.25, 0.25, 0.2, 0.25, 0.25, 0.25, 0.2, 0.25, 0.25, 0.25], width : [1, 1, 1, 1, 1, 1],\\\hline
    \end{tabular}
    }
    \caption{Configuration of subnetworks extracted for the creation of ensembles of heterogeneous architectures}
    \label{tab:nasParam}
\end{table}


Table~\ref{tab:trainingTrans} shows the transformations used during training for all individual models.

    \begin{table}[H]
        \centering
        \begin{tabular}{|c|c|}
            \hline
            Transform & Parameters\\
             \hline
             RandomResizedCrop & size=(224, 224), scale=(0.08, 1.0), ratio=(0.75, 1.3333), interpolation=bilinear \\
             \hline
             RandomHorizontalFlip & p=0.5 \\
             \hline
             Normalize & mean=[0.485, 0.456, 0.406], std=[0.229, 0.224, 0.225]\\
             \hline
        \end{tabular}
        \caption{Training data set transforms used for the training of all models}
        \label{tab:trainingTrans}
    \end{table}

Figure~\ref{tab:indAccNAS} shows the final top1 accuracy scores of each model defined by Table~\ref{tab:tParams}.

\begin{figure}[htbp]
    \begin{subtable}[t]{0.45\textwidth}
        \centering
        \begin{tabular}{|c|c|}
            \hline
            Model id & Top 1\% accuracy \\
            \hline
            ResNext50\_32\_2 & 75.198  \\
            ResNext50\_32\_4l & 75.072 \\
            ResNext50\_16\_4 & 75.432 \\
            MNASNET\_1 & 54.408 \\
            MNASNET\_1p & 54.0 \\
            Squeezenet\_512 & 41.918 \\
            Squeezenet & 56.558 \\
            bootstrapNAS-B\_0 & 76.342 \\
            bootstrapNAS-B\_1 & 76.282 \\
            deit\_tiny\_p16\_224 & 71.654 \\
            dino\_deit\_tiny1 & 67.932 \\
            dino\_deit\_tiny2 & 67.52 \\
            dino\_resnet1 & 67.236 \\
            dino\_resnet2 & 67.216 \\
            \hline
        \end{tabular}
    \end{subtable}
    \quad
    \begin{subtable}[t]{0.45\textwidth}
        \centering
        \begin{tabular}{|c|c|}
            \hline
            Model id & Top 1\% accuracy \\
            \hline
            B\_0 & 76.342 \\
            B\_1 & 76.282 \\
            nB\_0 & 76.301\\
            nB\_1 & 76.318\\
            nB\_2 & 76.191\\
            nB\_3 & 76.142\\
            dB\_1a & 76.138\\
            dB\_1b & 76.084\\
            wB\_1a & 76.170\\
            wB\_1b & 75.804\\
            wB\_1c & 75.852\\
            \hline
        \end{tabular}
        
    \end{subtable}
    \caption{Individual accuracies of individual trained models on ImageNet validation dataset}
        \label{tab:indAccNAS}
\end{figure}

\newpage

\subsection{Comparison of prediction-based disagreement, input attribution diversity and average accuracy metrics in ensembles of heterogeneous architectures}

    \begin{figure}[htp]
        \subcaptionbox{\centering Attr. ImageNetv2\label{figB:odda}}{\includegraphics[width=0.2\linewidth]{figures/fig_hetero3_exp_linear_combination_DIFF_imagenetv2.eps}}%
        \hspace{0em}%
        \subcaptionbox{\centering Attr. Waterdrop\label{figB:oddb}}{\includegraphics[width=0.2\linewidth]{figures/fig_hetero3_exp_linear_combination_DIFF_waterdrop_7.eps}}%
        \hspace{0em}%
        \subcaptionbox{\centering Attr. Lines\label{figB:oddc}}{\includegraphics[width=0.2\linewidth]{figures/fig_hetero3_exp_linear_combination_DIFF_lines.eps}}%
        \hspace{0em}%
        \subcaptionbox{\centering Attr. Plasma\label{figB:oddd}}{\includegraphics[width=0.2\linewidth]{figures/fig_hetero3_exp_linear_combination_DIFF_plasma_noise.eps}}%
        \hspace{0em}%
        \subcaptionbox{\centering Attr. Checkerb.\label{figB:odde}}{\includegraphics[width=0.2\linewidth]{figures/fig_hetero3_exp_linear_combination_DIFF_checkboard.eps}}%
        \hspace{0em}%
        \subcaptionbox{\centering Disagr. ImageNetv2\label{figB:oddf}}{\includegraphics[width=0.2\linewidth]{figures/fig_hetero3_exp_linear_combination_OUT_DIS_imagenetv2.eps}}%
        \hspace{0em}%
        \subcaptionbox{\centering Disagr. Waterdrop\label{figB:oddg}}{\includegraphics[width=0.2\linewidth]{figures/fig_hetero3_exp_linear_combination_OUT_DIS_waterdrop_7.eps}}%
        \hspace{0em}%
        \subcaptionbox{\centering Disagr. Lines\label{figB:oddh}}{\includegraphics[width=0.2\linewidth]{figures/fig_hetero3_exp_linear_combination_OUT_DIS_lines.eps}}%
        \hspace{0em}%
        \subcaptionbox{\centering Disagr. Plasma\label{figB:oddi}}{\includegraphics[width=0.2\linewidth]{figures/fig_hetero3_exp_linear_combination_OUT_DIS_plasma_noise.eps}}%
        \hspace{0em}%
        \subcaptionbox{\centering Disagr. Checkerb.\label{figB:oddj}}{\includegraphics[width=0.2\linewidth]{figures/fig_hetero3_exp_linear_combination_OUT_DIS_checkboard.eps}}%
        \hspace{0em}%
        \subcaptionbox{\centering Acc. ImageNetv2\label{figB:oddk}}{\includegraphics[width=0.2\linewidth]{figures/fig_hetero3_exp_justAcc_imagenetv2.eps}}%
        \hspace{0em}%
        \subcaptionbox{\centering Acc. Waterdrop\label{figB:oddl}}{\includegraphics[width=0.2\linewidth]{figures/fig_hetero3_exp_justAcc_waterdrop_7.eps}}%
        \hspace{0em}%
        \subcaptionbox{\centering Acc. Lines\label{figB:oddm}}{\includegraphics[width=0.2\linewidth]{figures/fig_hetero3_exp_justAcc_lines.eps}}%
        \hspace{0em}%
        \subcaptionbox{\centering Acc. Plasma\label{figB:oddn}}{\includegraphics[width=0.2\linewidth]{figures/fig_hetero3_exp_justAcc_plasma_noise.eps}}%
        \hspace{0em}%
        \subcaptionbox{\centering Acc. Checkerb.\label{figB:oddo}}{\includegraphics[width=0.2\linewidth]{figures/fig_hetero3_exp_justAcc_checkboard.eps}}
        \caption{Comparison of \textbf{improvement correlation of three different metrics on five validation datasets} using averaging as consensus mechanism. Columns: Datasets. Rows: Metrics.}
        \label{figB:supHeteroODD}
        \end{figure}



\end{document}


%
\begin{figure}[!htp]
    %\vspace{-4mm}
    \centering
\includegraphics[width=.86\linewidth]{./figs/experiments/real_completion/real_completion.pdf}
    %\vspace{-3mm}
    \caption{
    Scene completion of a real scene. We select an sofa and perform CAD retrieval to obtain a partial scene as input.
    }
    \label{fig:real_world_completion}
    %\vspace{-3mm}
\end{figure}
%
\begin{figure}[!htp]
    %\vspace{-0.6cm}
    \centering
\includegraphics[width=.9\linewidth]{./figs/experiments/text_edit/text_edit.pdf}
%\vspace{-0.5cm}
    \caption{
    Text-guided (a) object suggestion (b) scene editing.
    }
    \label{fig:text_editing}
%\vspace{-0.6cm}
\end{figure}
%

\paragraph{Unconditional Scene Synthesis}
%\subsection{Unconditional Scene Synthesis}
\begin{figure*}[!ht]
	\centering
 	\begin{subfigure}[t]{0.23\textwidth}
            \includegraphics[width=\textwidth]{figs/experiments/unconditional/depthGAN/bed/404_scene.jpg}
		%\includegraphics[width=\textwidth]{figs/experiments/unconditional/depthGAN/bed/083_scene.jpg}
		%%% used in main paper \includegraphics[width=\textwidth]{figs/experiments/unconditional/depthGAN/bed/134_scene.jpg}
            \includegraphics[width=\textwidth]{figs/experiments/unconditional/depthGAN/bed/176_scene.jpg}
            \includegraphics[width=\textwidth]{figs/experiments/unconditional/depthGAN/dining/005_scene.jpg}
		%%% used in main paper\includegraphics[width=\textwidth]{figs/experiments/unconditional/depthGAN/dining/006_scene.jpg}
            \includegraphics[width=\textwidth]{figs/experiments/unconditional/depthGAN/dining/052_scene.jpg}
		\includegraphics[width=\textwidth]{figs/experiments/unconditional/depthGAN/dining/161_scene.jpg}
            \includegraphics[width=\textwidth]{figs/experiments/unconditional/depthGAN/living/001_scene.jpg}
		%%% used in main paper\includegraphics[width=\textwidth]{figs/experiments/unconditional/depthGAN/living/110_scene.jpg}
            \includegraphics[width=\textwidth]{figs/experiments/unconditional/depthGAN/living/152_scene.jpg}
		%%% used in main paper \includegraphics[width=\textwidth]{figs/experiments/unconditional/depthGAN/living/180_scene.jpg}
		\caption{DepthGAN~\cite{yang2021indoor}}
	\end{subfigure}
	\rulesep
	\begin{subfigure}[t]{0.23\textwidth}
		\includegraphics[width=\textwidth] {figs/experiments/unconditional/sync2gen/bed/005_scene.jpg}
		%%% used in main paper\includegraphics[width=\textwidth]{figs/experiments/unconditional/sync2gen/bed/042_scene.jpg}
            %\includegraphics[width=\textwidth]{figs/experiments/unconditional/sync2gen/bed/049_scene.jpg}
            \includegraphics[width=\textwidth]{figs/experiments/unconditional/sync2gen/bed/067_scene.jpg}
            %%%%%%%%%%%%%%%%
            %%% used in main paper\includegraphics[width=\textwidth]{figs/experiments/unconditional/sync2gen/dining/024_scene.jpg}
		%%% used in main paper\includegraphics[width=\textwidth]{figs/experiments/unconditional/sync2gen/dining/043_scene.jpg}
            \includegraphics[width=\textwidth]{figs/experiments/unconditional/sync2gen/dining/078_scene.jpg}
		\includegraphics[width=\textwidth]{figs/experiments/unconditional/sync2gen/dining/978_scene.jpg}
            %%%%%%%%%%%%%%%%
            %%% used in main paper\includegraphics[width=\textwidth]{figs/experiments/unconditional/sync2gen/living/000_scene.jpg}
		\includegraphics[width=\textwidth]{figs/experiments/unconditional/sync2gen/living/006_scene.jpg}
            \includegraphics[width=\textwidth]{figs/experiments/unconditional/sync2gen/living/088_scene.jpg}
		\includegraphics[width=\textwidth]{figs/experiments/unconditional/sync2gen/living/047_scene.jpg}
		\caption{Sync2Gen~\cite{yang2021scene}}
	\end{subfigure}
	\rulesep
        \begin{subfigure}[t]{0.23\textwidth}
            \includegraphics[width=\textwidth]{figs/experiments/unconditional/atiss/bed/Bedroom-11202_83_524.jpg}
		\includegraphics[width=\textwidth]{figs/experiments/unconditional/atiss/bed/Bedroom-691_85_234.jpg}
		%\includegraphics[width=\textwidth]{figs/experiments/unconditional/atiss/bed/Bedroom-5215_101_850.jpg}
		%%% used in main paper\includegraphics[width=\textwidth]{figs/experiments/unconditional/atiss/bed/Bedroom-47437_39_531.jpg}
            %%%%
            \includegraphics[width=\textwidth]{figs/experiments/unconditional/atiss/dining/DiningRoom-164_131_676.jpg}
		%%% used in main paper\includegraphics[width=\textwidth]{figs/experiments/unconditional/atiss/dining/DiningRoom-2432_86_381.jpg}
            \includegraphics[width=\textwidth]{figs/experiments/unconditional/atiss/dining/DiningRoom-2817_15_455.jpg}
		%%% used in main paper\includegraphics[width=\textwidth]{figs/experiments/unconditional/atiss/dining/DiningRoom-2817_15_792.jpg}
            \includegraphics[width=\textwidth]{figs/experiments/unconditional/atiss/living/LivingDiningRoom-270_12_095.jpg}
		%%% used in main paper\includegraphics[width=\textwidth]{figs/experiments/unconditional/atiss/living/LivingDiningRoom-1419_47_031.jpg}
            \includegraphics[width=\textwidth]{figs/experiments/unconditional/atiss/living/LivingDiningRoom-9530_100_053.jpg}
		\includegraphics[width=\textwidth]{figs/experiments/unconditional/atiss/living/LivingRoom-1097_67_061.jpg}
		\caption{ATISS~\cite{paschalidou2021atiss}}
	\end{subfigure}
	\rulesep
	\begin{subfigure}[t]{0.23\textwidth}
		%%% used in main paper\includegraphics[width=\textwidth]{figs/experiments/unconditional/ours/bed/Bedroom-2719_8_130.jpg}
		%\includegraphics[width=\textwidth]{figs/experiments/unconditional/ours/bed/Bedroom-22570_139_160.jpg}
            %\includegraphics[width=\textwidth]{figs/experiments/unconditional/ours/bed/MasterBedroom-19531_93_061.jpg}
		%\includegraphics[width=\textwidth]{figs/experiments/unconditional/ours/bed/SecondBedroom-35274_130_141.jpg}
        \includegraphics[width=\textwidth]{./figs/experiments/unconditional/ours/bed_new_ours/Bedroom-22495_98_561.jpg}
        \includegraphics[width=\textwidth]{./figs/experiments/unconditional/ours/bed_new_ours/Bedroom-13858_160_293.jpg}
            %%%%%%%%%%%%%%%%%%%%%%%
        \includegraphics[width=\textwidth]{figs/experiments/unconditional/ours/dining/DiningRoom-164_131_119.jpg}
		\includegraphics[width=\textwidth]{figs/experiments/unconditional/ours/dining/DiningRoom-12376_156_049.jpg}
            %%% used in main paper\includegraphics[width=\textwidth]{figs/experiments/unconditional/ours/dining/DiningRoom-9982_116_095.jpg}
		%%% used in main paper\includegraphics[width=\textwidth]{figs/experiments/unconditional/ours/dining/DiningRoom-15534_1_169.jpg}
             %%%%%%%%%%%%%%%%%%%%%%%
            %%% used in main paper\includegraphics[width=\textwidth]{figs/experiments/unconditional/ours/living/LivingDiningRoom-1625_110_034.jpg}
		\includegraphics[width=\textwidth]{figs/experiments/unconditional/ours/living/LivingDiningRoom-14432_148_037.jpg}
            \includegraphics[width=\textwidth]{figs/experiments/unconditional/ours/living/LivingDiningRoom-34678_2_215.jpg}
		\includegraphics[width=\textwidth]{figs/experiments/unconditional/ours/living/LivingRoom-3540_79_071.jpg}
		\caption{Ours}
	\end{subfigure}
	\caption{\textbf{Additional results of unconditional scene synthesis}. We compare our method with the state-of-the-art by generating from random noises, where our results present higher diversity and better plausibility with fewer penetration issues and more symmetric pairs.}
    \label{fig:uncond_comparison_supple}
\end{figure*}
\begin{figure*}[!htbp]
	\centering
 	\begin{subfigure}[t]{0.23\textwidth}
		\includegraphics[width=\textwidth]
	{figs/experiments/uncond_gallery/SecondBedroom-35821_15_393.jpg}
 
  	\includegraphics[width=\textwidth]{./figs/experiments/uncond_gallery/LivingRoom-41893_126_445.jpg}
   
		\includegraphics[width=\textwidth]{./figs/experiments/uncond_gallery/LivingDiningRoom-86944_84_103.jpg}

            \includegraphics[width=\textwidth]{./figs/experiments/unconditional/ours/living/LivingRoom-71071_8_074.jpg}
    
	\end{subfigure}
	\rulesep
        %
	\begin{subfigure}[t]{0.23\textwidth}
		\includegraphics[width=\textwidth]
	{figs/experiments/uncond_gallery/SecondBedroom-52584_24_964.jpg}
 
  	\includegraphics[width=\textwidth]{./figs/experiments/uncond_gallery/LivingRoom-50084_76_718.jpg}
   
		\includegraphics[width=\textwidth]{./figs/experiments/uncond_gallery/LivingDiningRoom-99518_174_183.jpg}

            \includegraphics[width=\textwidth]{./figs/experiments/uncond_gallery/LivingDiningRoom-163914_165_492.jpg}

	\end{subfigure}
	\rulesep
        %
        \begin{subfigure}[t]{0.23\textwidth}
		\includegraphics[width=\textwidth]
	{figs/experiments/uncond_gallery/SecondBedroom-86888_75_920.jpg}
 
  	\includegraphics[width=\textwidth]{./figs/experiments/uncond_gallery/LivingRoom-68491_81_650.jpg}
   
		\includegraphics[width=\textwidth]{./figs/experiments/uncond_gallery/LivingDiningRoom-109935_48_096.jpg}

        \includegraphics[width=\textwidth]{./figs/experiments/uncond_gallery/LivingRoom-71071_8_997.jpg}
        
	\end{subfigure}
	\rulesep
         %
	\begin{subfigure}[t]{0.23\textwidth}
		\includegraphics[width=\textwidth]
	{figs/experiments/uncond_gallery/SecondBedroom-258160_63_131.jpg}
 
  	\includegraphics[width=\textwidth]{./figs/experiments/uncond_gallery/LivingRoom-71071_8_485.jpg}
   
		\includegraphics[width=\textwidth]{./figs/experiments/uncond_gallery/LivingDiningRoom-126918_10_090.jpg}

  \includegraphics[width=\textwidth]{./figs/experiments/uncond_gallery/LivingRoom-88425_55_025.jpg}
	\end{subfigure}
	\caption{Diverse and plausible results of unconditional scene synthesis from our method. }
\label{fig:uncond_gallery}
%\vspace{2mm}
\end{figure*}

\begin{figure*}[t]
    %\vspace{-2mm}
	\centering
	\begin{subfigure}[t]{0.14\textwidth}
            \includegraphics[width=\textwidth]{./figs/experiments/scene_completion_supple/partial/Bedroom-15797_117_075.jpg}
            \includegraphics[width=\textwidth]{./figs/experiments/scene_completion_supple/partial/Bedroom-17102_150_930.jpg}
            \includegraphics[width=\textwidth]{./figs/experiments/scene_completion_supple/partial/LivingDiningRoom-233_45_129.jpg}
            \includegraphics[width=\textwidth]{./figs/experiments/scene_completion_supple/partial/LivingDiningRoom-69704_153_931.jpg}
        \caption{Partial Scenes}
	\end{subfigure}
        \rulesep
        %
        \begin{subfigure}[t]{0.41\textwidth}
    	\includegraphics[width=0.33\textwidth]{./figs/experiments/scene_completion_supple/atiss/Bedroom-15797_075.jpg}%
            \hfill
    	\includegraphics[width=0.33\textwidth]{./figs/experiments/scene_completion_supple/atiss/Bedroom-15797_344.jpg}%
            \hfill
      	\includegraphics[width=0.33\textwidth]{./figs/experiments/scene_completion_supple/atiss/Bedroom-15797_439.jpg} 
       %%%%%%
    	\includegraphics[width=0.33\textwidth]{./figs/experiments/scene_completion_supple/atiss/Bedroom-17102_180.jpg}%
            \hfill
        \includegraphics[width=0.33\textwidth]{./figs/experiments/scene_completion_supple/atiss/Bedroom-17102_713.jpg}%
        \hfill
        \includegraphics[width=0.33\textwidth]{./figs/experiments/scene_completion_supple/atiss/Bedroom-17102_930.jpg}
        %%%%%
        \includegraphics[width=0.33\textwidth]{./figs/experiments/scene_completion_supple/atiss/LivingDiningRoom-233_45_000.jpg}%
        \hfill
        \includegraphics[width=0.33\textwidth]{./figs/experiments/scene_completion_supple/atiss/LivingDiningRoom-233_45_002.jpg}%
        \hfill
        \includegraphics[width=0.33\textwidth]{./figs/experiments/scene_completion_supple/atiss/LivingDiningRoom-233_45_003.jpg}
        %%%%%
        \includegraphics[width=0.33\textwidth]{./figs/experiments/scene_completion_supple/atiss/LivingDiningRoom-69704_153_000.jpg}%
        \hfill
        \includegraphics[width=0.33\textwidth]{./figs/experiments/scene_completion_supple/atiss/LivingDiningRoom-69704_153_001.jpg}%
        \hfill
        \includegraphics[width=0.33\textwidth]{./figs/experiments/scene_completion_supple/atiss/LivingDiningRoom-69704_153_005.jpg}
        \caption{ATISS~\cite{paschalidou2021atiss}}
	\end{subfigure}
        \rulesep
        %
	\begin{subfigure}[t]{0.41\textwidth}
    	\includegraphics[width=0.33\textwidth]{./figs/experiments/scene_completion_supple/ours/Bedroom-15797_117_010.jpg}%
            \hfill
            \includegraphics[width=0.33\textwidth]{./figs/experiments/scene_completion_supple/ours/Bedroom-15797_117_007.jpg}%
    	\hfill
    	\includegraphics[width=0.33\textwidth]{./figs/experiments/scene_completion_supple/ours/Bedroom-15797_117_015.jpg}
        %%%%%%%%%%%%%%%%%%%%%%%
    	\includegraphics[width=0.33\textwidth]{./figs/experiments/scene_completion_supple/ours/Bedroom-17102_150_004.jpg}%
            \hfill
            \includegraphics[width=0.33\textwidth]{./figs/experiments/scene_completion_supple/ours/Bedroom-17102_150_011.jpg}%
    	\hfill
    	\includegraphics[width=0.33\textwidth]{./figs/experiments/scene_completion_supple/ours/Bedroom-17102_150_013_2.jpg}
             %%%%%%%%%%%%%%%%%%%%%%%
    	\includegraphics[width=0.33\textwidth]{./figs/experiments/scene_completion_supple/ours/LivingDiningRoom-233_45_002.jpg}%
            \hfill
            \includegraphics[width=0.33\textwidth]{./figs/experiments/scene_completion_supple/ours/LivingDiningRoom-233_45_012.jpg}%
    	\hfill
    	\includegraphics[width=0.33\textwidth]{./figs/experiments/scene_completion_supple/ours/LivingDiningRoom-233_45_017.jpg}
             %%%%%%%%%%%%%%%%%%%%%%%
    	\includegraphics[width=0.33\textwidth]{./figs/experiments/scene_completion_supple/ours/LivingDiningRoom-69704_016.jpg}%
            \hfill
            \includegraphics[width=0.33\textwidth]{./figs/experiments/scene_completion_supple/ours/LivingDiningRoom-69704_019.jpg}%
    	\hfill
    	\includegraphics[width=0.33\textwidth]{./figs/experiments/scene_completion_supple/ours/LivingDiningRoom-69704_251.jpg}
		\caption{Ours}
	\end{subfigure}
	\caption{\textbf{Scene completion} from partial scenes with only three objects given as inputs. Compared to ATISS, our method produced more diverse completion results with higher fidelity.}
    \label{fig:completion_supple}
    %\vspace{-2mm}
\end{figure*}
In Fig.~\ref{fig:uncond_comparison_supple}, we provide additional qualitative comparisons against state-of-the-art methods on the unconditional scene synthesis model. Also,  more visualization results of our unconditional scene synthesis model are presented in Fig.~\ref{fig:uncond_gallery}.

\paragraph{Scene Arrangement}
%\subsection{Scene Arrangement}
%\begin{abstract}
In state estimation algorithms that use feature tracks as input, it is customary to assume that the errors in feature track positions are zero-mean Gaussian. Using a combination of calibrated camera intrinsics, ground-truth camera pose, and depth images, it is possible to compute ground-truth positions for feature tracks extracted using an image processing algorithm. We find that feature track errors are not zero-mean Gaussian and that the distribution of errors is conditional on the type of motion, the speed of motion, and the image processing algorithm used to extract the tracks.
\end{abstract}


\section{Introduction}

Many state estimation algorithms assume that measurements are zero-mean Gaussian. This is an explicit assumption in the Kalman Filter and its nonlinear variants \cite{thrun_probabilistic_2005, barrau_invariant_2018} and implicitly built-into the optimization problem of bundle adjustment algorithms \cite{mur-artal_orb-slam:_2015} and outlier-rejection algorithms \cite{civera_1-point_2009}. With extensive calibration with respect to temperature and mechanical alignment, the zero-mean Gaussian assumption is sufficient for the measurements of sensors such as inertial measurement units (IMUs) \cite{vectornav_imu_calibration, tedaldi_robust_2014}, even if it is still not completely true: Extended Kalman Filters (EKFs) that rely on these IMUs are deployed on safety-critical systems actively in use.

Even though several well-known algorithms for Simultaneous Localization and Mapping (SLAM) rely on the often-deployed EKF (e.g. \cite{jones_visual-inertial_2011,Geneva2020ICRA,bloesch_iterated_2017}), SLAM is still an active area of research. The existence of recently-released and actively used research benchmark datasets \cite{hilti_benchmark, tartanair2020iros} indicate that the robotics and computer vision communities still believe that performance of SLAM and an understanding of its failure cases are still insufficient, even after three decades of development \cite{early_slam_tutorial}. This motivates an examination into the fundamental assumptions of SLAM.

This manuscript visits the assumption that feature tracks, the ``measurements" of any indirect visual SLAM algorithm, contain only zero-mean Gaussian error. The covariance of the feature tracks is typically a tuning parameter to for all features at all times. We show that the feature track errors are not zero-mean Gaussian and furthermore, that the errors are conditional on the type of motion, the speed of motion, and the type of feature tracker used to extract the feature tracks. To our knowledge, this is the first study of the mean and covariance of feature tracks \emph{conditional} on the factors that affect them.

The organization of the paper is as follows. Section \ref{sec:feature_track_uq} details the methods. Section \ref{sec:feature_tracker_experiment_details} presents some key figures, and summarizes the error distribution of feature trackers. Section \ref{sec:discussion} ends with some concluding remarks. Additional figures from the experiment are given in the Appendix.



\subsection{Related Work}

\paragraph{Performance of feature detectors and descriptors conditional on nuisances.} The main metric used to evaluate feature detectors is \emph{repeatability} \cite{mikolajczyk_comparison_2005}, or the probability that a feature detector will detect the same feature across multiple images of the same scene under different illuminations and viewpoints. Other metrics are \emph{entropy} \cite{heinly_comparative_2012}, the spread of detected features over an image, and \emph{recall} \cite{aanaes_interesting_2012}, the number of features that are likely ``matchable'' to features in another image of the same scene. On the other hand, the primary metrics used to evaluate feature descriptors are \emph{precision} and \emph{recall}, calculated using pairs of ``matches'' that are found using the descriptor \cite{mikolajczyk_performance_2005}. The evaluation of feature detectors requires multiple images of the same scene. The evaluation of descriptors originally used the same datasets as the evaluation of detectors. To disentangle the problem of detecting features from the evaluation of feature description, two comprehensive datasets of image patches was released in 2017 \cite{balntas_hpatches_2017, maier_ground_2017}. At around the same time, \cite{schonberger_comparative_2017} evaluated both learned and handcrafted feature detectors and descriptors. Of most interest to us are \cite{heinly_comparative_2012}, which used a small dataset containing pure rotation, pure scaling, and illumination changes to evaluate the performance of various detector/descriptor combinations condition on each, \cite{zhao_image_2020}, which extended the datasets used in \cite{heinly_comparative_2012}, and \cite{aanaes_interesting_2012}, which evaluated the performance of feature detectors conditional on change in view angle and lighting condition. Tangentially interesting are \cite{hauagge_image_2012}, which released a dataset of image pairs that are geometrically consistent, but contain large changes in style (e.g. summer vs. winter) and lighting; and \cite{sattler_benchmarking_2018}, which contains groups of image sequences with similar motions, but large outdoor illumination changes.


\paragraph{Learning or Fitting a Covariance Matrix to Feature Tracks.} Early works sought to compute covariance of feature location using information in the RGB image. \cite{kanazawa_we_2001} approximated the covariance with the Hessian of the image centered at the feature point was the covariance of a detected feature -- the idea is that the sharper the curvature given by the Hessian, the more likely a convolutional filter will find the correct location of the feature. \cite{nickels_estimating_2002} contains a sum-of-squared-differences formula for computing feature track covariance. \cite{zeisl_estimation_2009} contains a formula for computing the covariance matrix of SIFT and SURF features. Later on, \cite{sheorey_uncertainty_2015} and \cite{wong_uncertainty_2017} present two methods to model the mean and covariance of Lucas-Kanade feature tracks. With the exception  of \cite{sheorey_uncertainty_2015}, which assumes that the location of a feature track could be a Gaussian Mixture Model, all other models assume that uncertainty is zero-mean Gaussian.



\section{Method}
\label{sec:feature_track_uq}

We wish to characterize the dependence of \textbf{mean error}, \textbf{mean absolute error}, \textbf{covariance}, \textbf{outlier ratio}, and \textbf{feature lifetime} on motion type, speed, tracker type, and when available, lighting. The types of motion investigated are:
\begin{itemize}
\item \textbf{Sideways motion} -- Linear movement with no rotation.
\item \textbf{Fixating motion} -- Moving in a constant radius around a central object. The camera is always pointed directly at the central object, creating some rotation.
\item \textbf{Forwards motion} -- Driving-like motion. The primary change frame-to-frame is scale. Points near the center of an image will stay near the center in subsequent frames.
\item \textbf{AR/VR motion} -- Movement that consists of mostly rotations around a persistent scene.
\end{itemize}
To vary speed, we skip frames at regular intervals from the image sequences. Nominal speed, or a speed of 1.00, means that all frames are used. A speed of 2.00 means that the feature tracker will only see every other frame, and a speed of 3.00 means that the feature tracker will only see one in every three frames. We do not test speeds below 1.00. The exact speeds tested depends on dataset. Finally, we also investigate the effect of two types of feature trackers:
\begin{itemize}
\item \textbf{Lucas-Kanade Sparse Optical Flow} \cite{lucas_iterative_1981}
\item \textbf{Correspondence Tracker} using the SIFT descriptor \cite{lowe_object_1999}. Although computationally expensive, the SIFT descriptor was chosen because of its availability and its performance when used in state estimation tasks \cite{schonberger_comparative_2017}. The descriptor of a feature track is set at the first frame it is detected and never updated.
\end{itemize}

We have chosen \emph{not} to study lens distortion, since this would require multiple similar datasets collected with different cameras. All images in all datasets either have been preprocessed to remove lens distortions, or simulated without lens distortions. Since the Lucas-Kanade tracker is differential, we also choose not to study a differential correspondence tracker that updates the descriptor of a feature track at every frame.



\subsection{Equations}

Consider a feature $i$ that was first detected at time $t^i_0$. If a depth image is available at time $t^i_0$ and $g_{sc}(t^i_0)$ is known, we may fix the feature's position in the spatial frame, $X_s^i$:
\begin{equation}
\begin{aligned}
    X^i_c(t^i_0) &= \pi^{-1}_K(x_p(t^i_0), Z^i_0) \\
    X^i_s &= g_{sc}(t^i_0) \circ X_c(t^i_0) \\
    \label{eq:fixing_Xs}
\end{aligned}
\end{equation}
In the above equation, $Z^i_c(t^i_0)$ is the third coordinate, or depth, of $X^i_c(t^i_0)$. Once, $X_s^i$ is fixed, we can then calculate the \textbf{``ground-truth feature track"} $\bar x_p^i(t)$:
\begin{equation}
    \bar x^i_p(t) = \pi_K(g_{sc}^{-1}(t) \circ X_s^i).
    \label{eq:gt_tracks}
\end{equation}
Some datasets provide a ground-truth point-cloud generated by a single lidar scan rather than a stream of depth images. A lidar scan is a point cloud with $M \sim 10^7$ points in the lidar frame $L$, which is defined as the camera frame at a particular time $t_L$: $\mathbf P_L = \{ P^0_L, P^1_L, \dots, P^M_L \}$. We can calculate the pixel coordinates of each point $j$ in $\mathbf P_L$: 
\begin{equation}
\pi_K(\mathbf P_L) = \{ \pi_K(P^0_L), \pi_K(P^1_L), \dots, \pi_K(P^M_L) \}
\label{eq:laser_scan_proj}
\end{equation}
Feature tracks visible at time $t_L$ can be associated with the nearest point in $\pi_K(\mathbf P_L)$. Suppose the nearest point in $\pi_K(\mathbf P_L)$ to feature $i$ is $P^j_L$. Then, the ground-truth track of feature $i$ is
\begin{equation}
\begin{aligned}
    X^i_s &= g_{sc}(t_L) \circ P^j_L \\
    \bar x^i_p(t) &= \pi_K(g_{sc}^{-1}(t) \circ X^i_s).
    \label{eq:dtu_px_groundtruth}
\end{aligned}
\end{equation}
Once we have a ground-truth feature track for feature $i$, we can calculate the error signal for that feature:
\begin{equation}
    e^i(t) = x_p^i(t) - \bar x_p^i(t)
    \label{eq:px_error_def}
\end{equation}
where $x_p^i(t)$ is the observed track. 


For datasets that provide a ground-truth point cloud at a single frame, the \textbf{mean error at timestep $t$} is
\begin{equation}
    \mu(t) = \frac{1}{M(t)} \sum_{i=1}^{M(t)} e^i(t)
    \label{eq:mean_error_at_time}
\end{equation}
where $M(t)$ is the number of tracked features at time $t$. The \textbf{mean absolute error at timestep $t$}
\begin{equation}
    \kappa(t) = \frac{1}{M(t)} \sum_{i=1}^{M(t)} |e^i(t)|.
    \label{eq:mean_abs_error_at_time}
\end{equation}
Similarly, the \textbf{covariance at timestep $t$} is calculated by
\begin{equation}
    \Sigma(t) = \frac{1}{M(t)-1} \sum_{i=1}^{M(t)} e^i(t) e^i(t)^T.
    \label{eq:cov_at_time}
\end{equation}
It is only possible to compute $\mu(t)$, $\kappa(t)$, and $\Sigma(t)$ for features that are visible at time $t_L$, when the laser scan was acquired.


For datasets that provide a stream of depth images, we use different definitions of mean error, mean absolute error, and covariance. We can also use all features and not just those visible in a particular frame. The \textbf{mean error after $k$ timesteps} is
\begin{equation}
    \nu(k) = \frac{1}{\Psi(k)} \sum_{i=1}^{\Phi(k)} e^i(t^i_0+k\delta_t)
    \label{eq:mean_error_after_timesteps}
\end{equation}
where $\Psi(k)$ is the number of features in the entire dataset tracked for at least $k$ timesteps and $\delta_t$ is the length of each timestep. The \textbf{mean absolute error after $k$ timesteps is:}
\begin{equation}
    \eta(k) = \frac{1}{\Psi(k)} \sum_{i=1}^{\Psi(k)} |e^i(t^i_0+k\delta_t)|
    \label{eq:mean_abs_error_after_timesteps}
\end{equation}
where $\Phi(k)$ is the number of features tracked for at least $k$ timesteps and $\delta_t$ is the length of each timestep. Finally, the \textbf{covariance after $k$ timesteps} is given by
\begin{equation}
    \Phi(k) = \frac{1}{\Psi(k)-1} \sum_{i=1}^{\Psi(k)} e^i(t^i_0+k\delta_t) e^i(t^i_0+k\delta_t)^T.
    \label{eq:cov_after_timesteps}
\end{equation}
When depth data is available at all frames, we define the feature's 3D location at the frame it is first detected and use equations \eqref{eq:mean_error_after_timesteps}, \eqref{eq:mean_abs_error_after_timesteps}, \eqref{eq:cov_after_timesteps}. %

At each frame, a feature tracker will attribute some features in one frame to the features in the previous frame. Let $F(t)$ be the total number of features in the frame at time $t$. The features in each frame will consist of $f_0(t)$ correct attributions, $f_1(t)$ incorrect attributions, and $f_2(t)$ new features, where $f_0(t) + f_1(t) + f_2(t) = F(t)$ and $f_0(t) + f_1(t) \leq F(t-1)$. Outlier rejection algorithms are used to determine $f_0(t)$ and $f_1(t)$ in real-time. The \textbf{outlier ratio} is defined as:
\begin{equation}
\frac{f_1(t)}{F(t-1)}.
\end{equation}

Finally, the \textbf{feature lifetime} of a feature track is the total number of consecutive frames in which it found and successfully attributed. A feature is ``born" at the frame it is first detected and ``dies" if a feature is not found for a single frame.


\section{Experiment Details}
\label{sec:feature_tracker_experiment_details}

\subsection{Feature Tracker Configuration}
\label{sec:feature_tracker_configuration}

We used the feature tracker is the \texttt{Tracker} object integrated with XIVO, our in-house SLAM system. The tracker is configured to use the AGAST corner detector \cite{mair_adaptive_2010}, and to track between 1000 and 1200 features at a time. The AGAST corner detector was chosen for its speed and because it detects a large number of features in most scenes. The feature tracker was configured to track up to 1200 features per scene. We use RANSAC with $p=0.995$ and an error threshold of 3 pixels to reject outliers. More details on the \texttt{Tracker} object and XIVO can be found in Appendix \ref{chapter:about_xivo}.

Since the tracker software was programmed to be part of a larger system and not specifically for these experiments, the implementation of the Correspondence Tracker is not ideal. If a feature is visible in frames 0-5, but is not detected in frame 2, the tracker will drop the feature at frame 2 and initialize a new one in frame 3. This behavior is consistent with the definition of feature lifetime given in the previous section, but is not the ideal implementation for a Correspondence Tracker because there is always a possibility that a corner detector will not find the corner in one frame, or that a descriptor will be just a little too different in one particular frame because of lighting. A more ideal implementation of the Correspondence Tracker would drop frames after a $N_m$ missed frames, where $N_m > 1$ is an experimentally determined number. The definition of feature lifetime would also be changed to accommodate this more complex behavior. As a result of this choice, the distribution of feature lifetimes for the Correspondence Tracker are shorter than they otherwise would be. Furthermore, our experiments will fail to characterize trends that only appear at higher speeds.


\subsection{Dataset-Specific Details}

\paragraph{DTU Point Features Dataset.}

The DTU Point Features Dataset \cite{aanaes_interesting_2012} consists of sixty scenes of fixating motion. In the dataset, one or more objects is placed at the center of stage lit with up to 19 LEDs. A camera is mounted on a robot arm and moved in a precise manner at the stage. At each of 119 fixed locations, the camera acquires an image lit with one of the 19 LEDs, enabling lighting experiments using image-based relighting. The dataset contains a laser scan of the scene at a single frame, called the Key Frame. The original image size is 1600 $\times$ 1200. For speed, we use 800 $\times$ 600 px. grayscale versions of the images instead of the full resolution images.

We make use of the first 49 frames of each scene, or Arc 1 (see Figure \ref{fig:dtu_light_stage}). The Key Frame is Frame 25. We calculate mean error $\mu$, mean absolute error $\kappa$, and covariance $\Sigma$ using equations \eqref{eq:mean_error_at_time}, \eqref{eq:mean_abs_error_at_time}, and \eqref{eq:cov_at_time}. Since 3D data is only available at the Key Frame, calculation of errors and covariances only includes features that exist in Frame 25. Therefore, there is a bias towards longer tracks, as all short tracks that don't exist in Frame 25 are all tossed out. Since the ``ground-truth" position of each feature in 3D is defined by its position in Frame 25, all results will therefore show that Frame 25 has zero covariance and the lowest errors. Statistics on feature lifetime and outlier rejection, however, do include features that do not exist in Frame 25.

To compute the ground-truth location of a feature track, we must associate a feature track to a point in a laser scan point cloud (eq. \eqref{eq:dtu_px_groundtruth}). Since the point cloud does not cover every pixel in the image, associations between features and laser scan points are only made if the pixel value of the laser scan point (eq. \eqref{eq:laser_scan_proj}) is less than 0.25 pixels from the feature.  Associating a pixel to a laser scan point with the incorrect depth measurement will result in a very large calculated means in equation \eqref{eq:mean_error_at_time}. Even with the low 0.25 pixel threshold, this bad association can still happen around edges and corners of objects. So that our analyses do not include very many of these poor depth associations, we throw out feature tracks whose maximum error is greater than the 90th percentile.

Since the DTU Point Features dataset was designed to enable image-based relighting, we also investigated the effects of directional light in addition to speed and the tracker used. We tested the same directional lights as  \cite{aanaes_interesting_2012}. The position of each directional light is shown in Figure \ref{fig:dtu_light_stage}.




\begin{figure}
    \centering
    \includegraphics[width=3.2in]{feature_tracker_uq/annotated_paths_of_interest.png}
    \includegraphics[width=3.2in]{feature_tracker_uq/annotated_light_stage_setup.png}
    \caption{\textbf{An Illustration of the Light Stage Setup in the DTU Point Features Dataset.}  \textbf{Left:} The locations at which images were acquired in the DTU Point Features dataset form three arcs and a linear path. Laser scans of the scenes were collected at the Key Frame (front and center). Frames from Arc 1 (circled in blue) are used for this experiment. \textbf{Right:} Red circles depict the location of 19 physical LEDs used to light the scene, which are spaced out over the scene. At each camera position in the left figure, the authors of the DTU Point Features dataset acquired 19 images. In each image, exactly one of the 19 LEDs is switched on. Acquiring 19 images in each location this way facilitates experiments in lighting using image-based relighting. Diffuse lighting can be simulated by using all 19 photographs from each position equally. More intense directional lighting can be simulated by weighting some LEDs more than others. In our experiments, we vary lighting from back-to-front (BF0-BF7) and left-to-right (LR0-LR9) as the camera follows the motion of Arc 1. Lights LR0 - LR9 and BF0 - BF7 are calculated by using Gaussian-weights on the 19 lights with $\sigma=20$cm; Light LR6 is highlighted in green. Figures are reprinted and annotated with permission.}
    \label{fig:dtu_light_stage}
\end{figure}



\paragraph{KITTI Vision Suite.} The raw data \cite{Geiger2012CVPR} in the KITTI Vision Suite consists RGB, GPS, IMU, and Lidar data captured from a moving vehicle. The motion captured in the images is predominantly forwards. The Lidar data was then processed into a separate benchmark dataset of depth images for single-image depth prediction and depth completion \cite{uhrig_sparsity_2017}. We make use stream \texttt{Image02}. Sequences containing ``still frames" (e.g. significant amount of waiting at a traffic light), are excluded. Excluding sequences containing still frames leaves 28 scenes for our experiments. Although this is fewer scenes than the DTU dataset, it is still more frames because most sequences are longer than 49 frames.

Since 3D data is available at every frame, we define a feature's 3D position using the depth image from the very first frame where it was detected. Therefore, we use mean error $\nu$ (eq. \eqref{eq:mean_error_after_timesteps}), absolute error $\eta$ (eq. \eqref{eq:mean_abs_error_after_timesteps}), and covariance $\Phi$ (eq. \eqref{eq:cov_after_timesteps}). To avoid errors due to bad depth measurements, we throw out the tracks whose maximum L2 error are above the 90th percentile and only calculate $\nu$, $\eta$, and $\Phi$ at timesteps where there are at least 100 features (see Fig. \ref{fig:kitti_avg_feats}).



\paragraph{Simulated Supplementary Data.} For AR/VR motions and sideways motions, we collected simulated RGB-D data in Gazebo. The simulation consisted of a Microsoft Kinect, modified so that RGB and depth data would be co-located, mounted on a Hector quadrotor \cite{hector_quadrotor} in ROS Melodic. The scene consisted of large objects from the Open Source Robotics Foundation's Gazebo Model Library. Images have a resolution of 800 $\times$ 600 pixels. In the subsequent sections, we refer to these datasets as ``Gazebo Linear" and ``Gazebo AR/VR". The AR/VR trajectory used to collect data is shown in Figure \ref{fig:gazebo_arvr_traj}.

In the Gazebo Linear dataset, we throw out tracks whose errors are above the 80-th percentile due to drift that naturally occurs when using the Lucas-Kanade Tracker in an environment containing straight and crisp edges parallel to the direction of motion. More details are given in Figure \ref{fig:gazebo_linear_error_throwout}. In the Gazebo AR/VR dataset, the we throw out tracks whose errors are above the 90-th percentile, as motions are no longer parallel to the straight edges.


\begin{figure}
\centering
\includegraphics[width=0.48\textwidth]{feature_tracker_uq/gazebo_arvr_figs/ARVR_translation_gt.pdf}
\includegraphics[width=0.48\textwidth]{feature_tracker_uq/gazebo_arvr_figs/ARVR_rotation_gt.pdf}
\caption{\textbf{The trajectory generated for the AR/VR scenario.} The commanded trajectory used to collected the AR/VR data was generated from the translation (left) and rotation (right) plotted above. Translation is generated point-to-point using haversines and rotation is generated from slerping.}
\label{fig:gazebo_arvr_traj}
\end{figure}



\subsection{Results}

Overall, we find that mean error, mean absolute error, covariance, feature lifetime, and outlier ratio are all dependent on the type of motion, the tracker used, and the speed. For the DTU Point Features dataset, we found no dependence on the existence of directional light unless the directional light happened to cause tracking failure at high speeds. In Tables \ref{tab:dtu_summary_table} - \ref{tab:gazebo_arvr_summary_table}, we list the exact dependence of mean error, mean absolute error, feature lifetime, covariance, and outlier ratio on each independent variable. Differences in Tables \ref{tab:dtu_summary_table} - \ref{tab:gazebo_arvr_summary_table} lead us to conclude that feature tracks are dependent on motion, tracker, and speed, but not the existence of directional light.

One notable difference between the Lucas-Kanade and Correspondence Trackers is that feature tracks produced by the Lucas-Kanade Tracker drift steadily while the Correspondence Tracker does not. This is because the Lucas-Kanade Tracker is differential, i.e. the characterization of a feature will slightly change frame to frame. For the Correspondence Tracker, this is not true. Therefore, the location of the feature track will drift, and the direction and magnitude of drift is dependent on the direction of motion. With left-to-right fixating motion, drift is positive (see Figure \ref{fig:dtu_diffuse_1.00_meanerror}). With left-to-right linear motion, drift is negative, and also larger (see Figure \ref{fig:gazebo_linear_LK_meanerror}). In AR/VR motion, the direction of drift changes with motion (see Figure \ref{fig:gazebo_arvr_LK_meanerror}). The flipside is that the Lucas-Kanade tracker generates features with a longer lifetime (see Figures \ref{fig:dtu_track_lifetime}, \ref{fig:kitti_feature_lifetime}, \ref{fig:gazebo_linear_feature_lifetime}, \ref{fig:gazebo_arvr_feature_lifetime}). When motion is fixating, the Correspondence Tracker also drifts about the direction of motion (see Figure \ref{fig:dtu_mean_error_sideways}).

Finally, we note that the zero-mean Gaussian assumption holds when motion is predominantly forwards and we are using the Correspondence Tracker (see Figures \ref{fig:kitti_match_meanerror} and \ref{fig:kitti_match_cov}). All figures supporting the assertions in this section are given in the Appendix.






\begin{table}[htp]
    \centering
    \begin{tabular}{p{1in}|p{1.0in}|p{1.0in}|p{2.5in}}
                & \textbf{Tracker} & \textbf{Lighting} & \textbf{Speed} \\
    \hline
    $\mu(t)$ & No (fig. \ref{fig:dtu_diffuse_1.00_meanerror}) & No (figs. \ref{fig:dtu_lighting_mu_LK}, \ref{fig:dtu_lighting_mu_match}) & No (figs. \ref{fig:dtu_match_diffuse_mean_error_varyspeed}, \ref{fig:dtu_LK_mean_varyspeed}) \\
    \hline
    $\kappa(t)$ & Yes (fig. \ref{fig:dtu_diffuse_1.00_MAE_cov}) & No (fig. \ref{fig:dtu_diffuse_1.00_MAE_cov}) & Yes for Correspondence Tracker (fig. \ref{fig:dtu_match_diffuse_MAE_varyspeed}), No for Lucas-Kanade Tracker (fig. \ref{fig:dtu_LK_MAE_varyspeed})\\
    \hline
    $\Sigma(t)$ & Yes (fig. \ref{fig:dtu_diffuse_1.00_MAE_cov}) & No (fig. \ref{fig:dtu_lighting_sigma_LK}, \ref{fig:dtu_lighting_sigma_match}) & Yes for Correspondence Tracker (fig. \ref{fig:dtu_match_diffuse_cov_varyspeed}), No for Lucas-Kanade Tracker (fig. \ref{fig:dtu_LK_cov_varyspeed})\\
    \hline
    Feature Lifetime & Yes (fig. \ref{fig:dtu_track_lifetime}) & No  (fig. \ref{fig:dtu_lighting_feature_lifetimes}) & Yes (fig.  \ref{fig:dtu_active_features}) \\
    \hline
    Outlier Ratio & Yes (figs. \ref{fig:dtu_track_outliers_lights}, \ref{fig:dtu_track_outliers_speed}) & No (fig.  \ref{fig:dtu_track_outliers_lights}) & Yes  (fig. \ref{fig:dtu_track_outliers_speed}) \\
    \end{tabular}
    \caption{\textbf{DTU Point Features Results Summary.} Cells contain whether or not the dependent variables in the left column are affected by the independent variables listed in the top row. Entries also contain figure numbers containing justification. The ``Tracker" and ``Lighting" columns contain references to figures containing plots at nominal speed. Although not indicated in the table, Figures \ref{fig:dtu_speed2.00_percent_outlier} - \ref{fig:dtu_LK_cov_speed12.00} in the Appendix show that the existence of directional lighting continues to not affect outlier ratio, mean error, mean absolute error, and covariance at higher speeds for both the Lucas-Kanade and Correspondence Trackers.}
    \label{tab:dtu_summary_table}
\end{table}



\begin{table}[htp]
    \centering
    \begin{tabular}{p{1in}|p{1.5in}|p{2.50in}}
                & \textbf{Tracker}  & \textbf{Speed} \\
    \hline
    $\nu(t)$ & No (fig. \ref{fig:kitti_1.00_meanerror}) & Yes (figs. \ref{fig:kitti_LK_meanerror}, \ref{fig:kitti_match_meanerror}) \\
    \hline
    $\eta(t)$ & Yes (fig. \ref{fig:kitti_1.00_error_cov}) & No for Correspondence Tracker (fig. \ref{fig:kitti_match_MAE}), Yes for Lucas-Kanade Tracker (figs. \ref{fig:kitti_LK_MAE}) \\
    \hline
    $\Phi(t)$ & Yes (fig. \ref{fig:kitti_1.00_error_cov}) & No for Correspondence Tracker (fig. \ref{fig:kitti_match_cov}), Yes for Lucas-Kanade Tracker (fig. \ref{fig:kitti_LK_cov}) \\ 
    \hline
    Feature Lifetime & Yes (fig. \ref{fig:kitti_feature_lifetime}) & Yes (fig. \ref{fig:kitti_avg_feats}) \\
    \hline
    Outlier Ratio & Yes (fig. \ref{fig:kitti_outlier_ratio}) & Yes (fig. \ref{fig:kitti_outlier_ratio})\\
    \end{tabular}
    \caption{\textbf{KITTI Results Summary.} Cells contain whether or not the dependent variables in the left column are affected by the independent variables listed in the top row. Entries also contain figure numbers containing justification.}
    \label{tab:kitti_summary_table}
\end{table}



\begin{table}[htp]
    \centering
    \begin{tabular}{p{1in}|p{1.5in}|p{2.5in}}
                & \textbf{Tracker}  & \textbf{Speed} \\
    \hline
    $\nu(t)$ & Yes (fig. \ref{fig:gazebo_linear_1.00_meanerror}) & No for Correspondence Tracker (fig. \ref{fig:gazebo_linear_match_meanerror}), Yes for Lucas-Kanade Tracker   (fig. \ref{fig:gazebo_linear_LK_meanerror}) \\
    \hline
    $\eta(t)$ & Yes (fig. \ref{fig:gazebo_linear_1.00_error_cov}) & Yes (figs. \ref{fig:gazebo_linear_LK_MAE}, \ref{fig:gazebo_linear_match_MAE}) \\
    \hline
    $\Phi(t)$ & Yes (fig. \ref{fig:gazebo_linear_1.00_error_cov}) & Yes (figs.  \ref{fig:gazebo_linear_match_cov}, \ref{fig:gazebo_linear_LK_cov}) \\ 
    \hline
    Feature Lifetime & Yes (fig. \ref{fig:gazebo_linear_feature_lifetime}) & Yes (fig. \ref{fig:gazebo_linear_avg_feats}) \\
    \hline
    Outlier Ratio & Yes (fig. \ref{fig:gazebo_linear_outlier_ratio}) &  No for Correspondence Tracker, Yes for Lucas-Kanade Tracker (fig. \ref{fig:gazebo_linear_outlier_ratio})\\
    \end{tabular}
    \caption{\textbf{Gazebo Linear Results Summary.} Cells contain whether or not the dependent variables in the left column are affected by the independent variables listed in the top row. Entries also contain figure numbers containing justification.}
    \label{tab:gazebo_linear_summary_table}
\end{table}


\begin{table}[htp]
    \centering
    \begin{tabular}{p{1in}|p{1.5in}|p{2.5in}}
                & \textbf{Tracker}  & \textbf{Speed} \\
    \hline
    $\nu(t)$ & Yes (fig. \ref{fig:gazebo_arvr_1.00_meanerror}) & No (fig. \ref{fig:gazebo_arvr_LK_meanerror}, \ref{fig:gazebo_arvr_match_meanerror}) \\
    \hline
    $\eta(t)$ & Yes (fig. \ref{fig:gazebo_arvr_1.00_error_cov}) & Yes for Correspondence   Tracker (fig. \ref{fig:gazebo_arvr_match_MAE}), No for Lucas-Kanade  Tracker (fig. \ref{fig:gazebo_arvr_LK_MAE})  \\
    \hline
    $\Phi(t)$ & Yes (fig. \ref{fig:gazebo_arvr_1.00_error_cov}) &  Yes for Correspondence   Tracker (fig. \ref{fig:gazebo_arvr_match_cov}), No for Lucas-Kanade Tracker (fig. \ref{fig:gazebo_arvr_LK_cov})  \\ 
    \hline
    Feature Lifetime & Yes (fig. \ref{fig:gazebo_arvr_feature_lifetime}) & Yes (fig. \ref{fig:gazebo_arvr_avg_feats}) \\
    \hline
    Outlier Ratio & Yes (fig. \ref{fig:gazebo_arvr_outlier_ratio}) & No for Correspondence Tracker, Yes for Lucas-Kanade Tracker (fig. \ref{fig:gazebo_arvr_outlier_ratio})\\
    \end{tabular}
    \caption{\textbf{Gazebo AR/VR Results Summary.} Cells contain whether or not the dependent variables in the left column are affected by the independent variables listed in the top row. Entries also contain figure numbers containing justification.}
    \label{tab:gazebo_arvr_summary_table}
\end{table}


\section{Discussion}
\label{sec:discussion}

Other than the caveat about the Correspondence Tracker noted in Section \ref{sec:feature_tracker_configuration}, the main limitation of this work is that there are more variables we could have tested, but chose not to. Examples of variables we chose not to test are the choice of feature detector and descriptor, and characteristics in the scene. For example, would the Correspondence Tracker have as little drift when moving forwards in an indoor environment and comparing BRIEF descriptors? Testing for conditionality on more variables inevitably leads to an unmanageable experiment, so we chose to lock in the feature detector and descriptor to well-performing available options and let the dataset dictate available scenes. Nevertheless, our work is a first step in characterizing the dependence of mean error, mean absolute error, covariance, feature lifetime, and outlier ratio on motion, tracker, speed, and the existence of directional lighting. The main conclusion is that the common zero-mean Gaussian assumption is rarely true. This conclusion motivates a few areas of future work.

The most immediate direction of future work is to continue to use the Extended Kalman Filter and dynamically adapt filter parameters, such as covariance estimates and the number of tracked features, to the scene. Since feature tracks are not zero-mean, covariance estimates will have to be enlarged so that feature tracks containing the extra bias are not outliers. Machine learning approaches to adapting the covariance already exist \cite{vega-brown_cello_2013, liu_deep_2018}. Since statistical methods are not often desirable in safety critical systems, it is of interest to compare performance when covariance is adjusted by a learned model to when covariance is adjusted by a finite state machine. While this approach is the most immediate, it does not address the fact that it brings no convergence guarantees in a downstream state estimation process and will therefore require extensive testing for each application.

The second area of future work is to adapt existing state estimation algorithms to accommodate feature tracks that are not zero-mean Gaussian. It may not be possible, however, to design a filter that is both computationally tractable, guaranteed to converge, and simple enough to implement on a complex, realistic system. This motivates the study in the next chapter, and the third area of future work.

The third direction of future work is to adjust individual feature tracks \emph{before} they are used by a state estimation algorithm that assumes that measurements are zero-mean Gaussian. This is the approach used for IMUs: errors in IMUs measurements are primarily dependent on temperature and mechanical alignment errors, so IMU measurements are adjusted for temperature and known mechanical misalignments before they are passed to a downstream computer. For feature tracks, the calibration table would be more complex, as it is dependent on speed, motion type, and the type of tracker used.



We visualize additional qualitative comparisons on the task of scene arrangement in Fig.~\ref{fig:arrangement_lego_supple}.  
%
LEGO~\cite{wei2023lego} aims to predict 2D object locations and orientations, taking the input of a floor plane, object semantics and geometries. It does not handle objects like lamps that could hang from the ceiling. 
%
In contrast, DiffuScene is a scene-generative model that predicts 3D instance properties from random noise, including 3D locations and orientations, semantics, and geometries.  
%
Compared to ATISS and LEGO, our method generates various object placement options with better plausibility and more symmetries.
\begin{figure*}[b]
	\centering
        %\scalebox{0.8}{
 	\begin{subfigure}[t]{0.23\textwidth}
            \includegraphics[width=\textwidth]{./figs/experiments/arrange_lego_supple/noisy/Bedroom-430_9_657.jpg}
            \includegraphics[width=\textwidth]{./figs/experiments/arrange_lego_supple/noisy/Bedroom-18055_16_016.jpg}
            \includegraphics[width=\textwidth]{./figs/experiments/arrange_lego_supple/noisy/LivingDiningRoom-1625_110_110.jpg}
		%%% used in main paper\includegraphics[width=\textwidth]{figs/experiments/unconditional/depthGAN/dining/006_scene.jpg}
            \includegraphics[width=\textwidth]{./figs/experiments/arrange_lego_supple/noisy/LivingDiningRoom-6843_60_060.jpg}
            \includegraphics[width=\textwidth]{./figs/experiments/arrange_lego_supple/noisy/LivingDiningRoom-17657_95_287.jpg}
            \includegraphics[width=\textwidth]{./figs/experiments/arrange_lego_supple/noisy/LivingDiningRoom-20097_16_016.jpg}
		\caption{Noisy Scene}
	\end{subfigure}
         %%%%%%%%%%%%%%%%%
	\rulesep
	\begin{subfigure}[t]{0.23\textwidth}
            \includegraphics[width=\textwidth]{./figs/experiments/arrange_lego_supple/atiss/Bedroom-430_9_009.jpg}
            \includegraphics[width=\textwidth]{./figs/experiments/arrange_lego_supple/atiss/Bedroom-18055_16_340.jpg}
		\includegraphics[width=\textwidth]{./figs/experiments/arrange_lego_supple/atiss/LivingDiningRoom-1625_110_110.jpg}
            \includegraphics[width=\textwidth]{./figs/experiments/arrange_lego_supple/atiss/LivingDiningRoom-6843_60_060.jpg}
            \includegraphics[width=\textwidth]{./figs/experiments/arrange_lego_supple/atiss/LivingDiningRoom-17657_95_095.jpg}
            \includegraphics[width=\textwidth]{./figs/experiments/arrange_lego_supple/atiss/LivingDiningRoom-20097_16_016.jpg}
        \caption{ATISS~\cite{paschalidou2021atiss}}
	\end{subfigure}
         %%%%%%%%%%%%%%%%%
	\rulesep
        \begin{subfigure}[t]{0.23\textwidth}
            \includegraphics[width=\textwidth]{./figs/experiments/arrange_lego_supple/lego/Bedroom-430_0_220.jpg}
		\includegraphics[width=\textwidth]{./figs/experiments/arrange_lego_supple/lego/Bedroom-18055_3_851.jpg}
            \includegraphics[width=\textwidth]{./figs/experiments/arrange_lego_supple/lego/LivingDiningRoom-1625_1_543.jpg}
            \includegraphics[width=\textwidth]{./figs/experiments/arrange_lego_supple/lego/LivingDiningRoom-6843_0_457.jpg}
            \includegraphics[width=\textwidth]{./figs/experiments/arrange_lego_supple/lego/LivingDiningRoom-17657_0_245.jpg}
            \includegraphics[width=\textwidth]{./figs/experiments/arrange_lego_supple/lego/LivingDiningRoom-20097_0_49.jpg}
		\caption{LEGO~\cite{wei2023lego}}
	\end{subfigure}
         %%%%%%%%%%%%%%%%%
	\rulesep
	\begin{subfigure}[t]{0.23\textwidth}
            \includegraphics[width=\textwidth]{./figs/experiments/arrange_lego_supple/ours/Bedroom-430_0_009.jpg}
        \includegraphics[width=\textwidth]{./figs/experiments/arrange_lego_supple/ours/Bedroom-18055_16_016.jpg}
        \includegraphics[width=\textwidth]{./figs/experiments/arrange_lego_supple/ours/LivingDiningRoom-1625_110_110.jpg}
		\includegraphics[width=\textwidth]{./figs/experiments/arrange_lego_supple/ours/LivingDiningRoom-6843_60_060.jpg}
        \includegraphics[width=\textwidth]{./figs/experiments/arrange_lego_supple/ours/LivingDiningRoom-17657_95_863.jpg}
		\includegraphics[width=\textwidth]{./figs/experiments/arrange_lego_supple/ours/LivingDiningRoom-20097_16_976.jpg}
		\caption{Ours}
	\end{subfigure}
    %}
        %%%%%%%%%%%%%%%%%
	\caption{\textbf{Scene re-arrangements} of collections of random objects.  Compared to ATISS and LEGO, our method generates various object placement options
        with better plausibility and more symmetries.}
    \label{fig:arrangement_lego_supple}
    %\vspace{-4mm}
\end{figure*}


\paragraph{Scene Completion}
%\subsection{Scene Completion}
We present more qualitative comparisons on the task of scene completion in Fig.~\ref{fig:completion_supple}.
Also, the quantitative results are shown in Tab.~\ref{tab:completion}.  Compared to ATISS, our method produced more diverse completion results with higher fidelity. Our method can consistently outperform ATISS  in all listed metrics.
\begin{table}[!hbt]
    % \captionsetup[table]{
    % %labelfont=bf,
    % %textfont=normalfont,
    % textfont=bf,
    % singlelinecheck=off,
    % %justification=raggedright
    % }
    \renewcommand\arraystretch{1.2}
    \setlength{\tabcolsep}{2.4pt}
	\begin{center}
            \begin{tabular}{*{6}{c}}
			\toprule
			{Room} & {Method} & FID $\downarrow$ & KID $\downarrow$ %& SCA $\%$   
                   & \#Sym. & PIoU \\  %& Overlap   
			\midrule
			\midrule
            \multirow{2}*{Bed} &  ATISS & 30.54  & 2.38 %& 26.73   
                                        &  0.01 & 0.84  \\  
                                  
                                  &  Ours  & \textbf{27.32} & \textbf{1.92} % &\textbf{54.16} 
                                      %& \textbf{0.70} & \textbf{0.61} \\ 
                                & \textbf{0.47} & \textbf{0.61} \\

        \midrule
        \multirow{2}*{Dining} & ATISS & 42.65  & 8.32 %& 43.99  
                                        &  1.42  & 1.73 \\ 
                    &  Ours  & \textbf{40.99} & \textbf{6.31} %&\textbf{49.06} 
                              %& \textbf{3.69} & \textbf{0.95} \\ 
                               & \textbf{2.57} & \textbf{0.84} \\
                \midrule
        \multirow{2}*{Living} & ATISS & 43.30  & 5.22  %&   
                                        &  0.16  &  0.87\\ 
                              
                              &  Ours  & \textbf{40.49} & \textbf{4.59}  %& \textbf{} 
                              & \textbf{2.24} & \textbf{0.58} \\ 

        \bottomrule
        \end{tabular}
        \caption{Quantitative comparisons on the task of \textbf{scene completion} on 3D-FRONT bedrooms, dining rooms, and living rooms. Only 3 objects are given in the partial scenes. }
        \label{tab:completion}
        \end{center}
        \vspace{-6mm}
\end{table}

\paragraph{Real-world Scene Generalization}
%\subsection{Real-world Scene Generalization}
%
While trained on synthetic dataset, our method can be evaluated on real-world scenes without finetuning, e.g. for scene completion as shown in Fig.~\ref{fig:real_world_completion}.
%
Compared to ATISS, our method produces a more favourable scene.

\paragraph{Text-conditioned Scene Synthesis}
%\subsection{Text-conditioned Scene Synthesis}
\begin{figure*}[!htbp]
	\centering
	\begin{subfigure}[t]{0.24\textwidth}
            \includegraphics[width=\textwidth]{././figs/experiments/text2scene_supple/LivingDiningRoom-107_text.jpg}
            \vspace{2mm}
            \includegraphics[width=\textwidth]{././figs/experiments/text2scene_supple/LivingDiningRoom-1744_text.jpg}
            \vspace{2mm}
            \includegraphics[width=\textwidth]{././figs/experiments/text2scene_supple/LivingDiningRoom-2106_text.jpg}
            \vspace{2mm}
            \includegraphics[width=\textwidth]{././figs/experiments/text2scene_supple/LivingDiningRoom-3483_text.jpg}
        \caption{Input text}
	\end{subfigure}%
        \hfill
 	\begin{subfigure}[t]{0.215\textwidth}
            \includegraphics[width=\textwidth]{././figs/experiments/text2scene_supple/LivingDiningRoom-107_41_041_gt.jpg}
            \includegraphics[width=\textwidth]{././figs/experiments/text2scene_supple/LivingDiningRoom-1744_61_061_gt.jpg}
            \includegraphics[width=\textwidth]{././figs/experiments/text2scene_supple/LivingDiningRoom-2106_83_083_gt.jpg}
            \includegraphics[width=\textwidth]{././figs/experiments/text2scene_supple/LivingDiningRoom-3483_163_163_gt.jpg}
        \caption{Reference}
	\end{subfigure}%
        \hfill
 	\begin{subfigure}[t]{0.215\textwidth}
            \includegraphics[width=\textwidth]{././figs/experiments/text2scene_supple/LivingDiningRoom-107_41_041_atiss.jpg}
            \includegraphics[width=\textwidth]{././figs/experiments/text2scene_supple/LivingDiningRoom-1744_61_445_atiss.jpg}
            \includegraphics[width=\textwidth]{././figs/experiments/text2scene_supple/LivingDiningRoom-2106_83_467_atiss.jpg}
            \includegraphics[width=\textwidth]{././figs/experiments/text2scene_supple/LivingDiningRoom-3483_163_355_atiss.jpg}
        \caption{ATISS~\cite{paschalidou2021atiss}}
	\end{subfigure}%
        \hfill
 	\begin{subfigure}[t]{0.215\textwidth}
            \includegraphics[width=\textwidth]{././figs/experiments/text2scene_supple/LivingDiningRoom-107_41_018_ours.jpg}
            \includegraphics[width=\textwidth]{././figs/experiments/text2scene_supple/LivingDiningRoom-1744_61_002_ours.jpg}
            \includegraphics[width=\textwidth]{././figs/experiments/text2scene_supple/LivingDiningRoom-2106_83_003_ours.jpg}
            \includegraphics[width=\textwidth]{././figs/experiments/text2scene_supple/LivingDiningRoom-3483_163_010_ours.jpg}
        \caption{Ours}
	\end{subfigure}
	\caption{\textbf{Text-conditioned scene synthesis}. The input text describes only a partial scene configuration. Our method generates more plausible scenes matched with the texts.}
    \label{fig:text2scene_supple}
    %\vspace{-6mm}
\end{figure*}
We provide additional qualitative comparisons on the text-conditioned scene synthesis in Fig.~\ref{fig:text2scene_supple}. 
As observed, in the first and third rows, ATISS has object intersection issues while ours does not. In the second row, our method can correctly generate a corner side table on the left of the armchair. However, ATISS generates a corner side table on the right of the armchair.
 In the fourth row, our method can generate four dining chairs that are consistent with the text description, but ATISS can only generate two dining chairs.
% The quantitative results evaluated by FID, KID, and SCA are reported in Tab.~\ref{tab:text}. Our method consistently outperforms ATISS in all used metrics.
%
\paragraph{Scene editing via texts.} 
In Fig.~\ref{fig:text_editing}, we show that our method can support text-guided object suggestion and scene editing, without changing the attributes of other objects.

\section{User Study}
\label{SecUser}

We conducted a perceptual user study to evaluate the quality of our method against ATISS on the application of text-conditioned scene synthesis.
As shown in Fig.~\ref{fig:user_study}, we provide the visualization of a ground-truth scene used to generate a text prompt as a reference. For each pair of results, a user needs to answer ``which of the generated scene can better match the text prompt?" and ``Which of the generated scene is more reasonable and realistic?".
%needs to decide which of the generated scene can better match the text prompt and judge which of the synthesized scene is more plausibly realistic than the other.
%\TODO{what is plausibly realistic? write down the question that you asked in the study}
We collect the answers of 225 scenes from 45 users and calculate the statistics. 62$\%$ of the user answers prefer our method to ATISS in realism.  55$\%$ of answers think our method is more consistent with the text prompt.

\begin{figure*}[!htbp]
    \centering
    \includegraphics[width=\linewidth]{./figs/experiments/user_study/question_reference.jpg}
    
    \includegraphics[width=\linewidth]{./figs/experiments/user_study/question_match.jpg}

    \includegraphics[width=\linewidth]{./figs/experiments/user_study/question_realism.jpg}

    \caption{\textbf{User Study UI}. Based on the reference scene used to generate text prompts, users are asked which of the synthesized scene is more matched with the text prompt and more realistic. Note that the results from ATISS and our method are randomly shuffled to avoid bias.}
    \label{fig:user_study}
    
\end{figure*}
