\setlength{\tabcolsep}{3.5pt}
\begin{table}[!htbp]
    % \captionsetup[table]{
    % %labelfont=bf,
    % %textfont=normalfont,
    % textfont=bf,
    % singlelinecheck=off,
    % %justification=raggedright
    % }
	\renewcommand\arraystretch{1.2}
	\begin{center}
		\begin{tabular}{*{5}{c}}
			\toprule
			%\multirow{2}*{Method} & \multicolumn{4}{c}{Bedroom} \\ \cmidrule(lr){2-5}
                Method & FID $\downarrow$ & KID $\downarrow$ & SCA $\%$  & CKL $\downarrow$ \\ % $\times100$ 
			\midrule
			\midrule
            % Ours-Grid-$8^3$
            %       & 31.28  & 14.81 & 84.55  & 6.23 \\

            % Ours-Plane-$32^2$
            %       & 28.17  & 11.12  & 83.80  & 2.67 \\
        
            Ours-PVCNN 
                  & 25.81  & 8.55 & 72.98  & 1.53 \\
                  
            % Ours-Transformer
            %       &   &  &   &  0.17\\

            %Causal 
            Ours-Transformer
                  & 29.08  & 4.59 & 73.63  & 0.36\\

            % Ours-Auto. diff.
            %       &   &  &   & \\

            Ours-Single head
                    & 19.78  & 2.07 & 54.53  & 0.69 \\

            Ours-w/o pos. emb. 
                    &   &   &   & \\ 
            Ours-w/o iou reg. 
                    &   &   &   & \\
                    
            Ours-w/o geometry code  
                   & 18.40  & 1.55  & 55.42 & 0.66 \\

            Ours-final
                   & \textbf{18.29} & \textbf{1.42}  & \textbf{53.52}  & \textbf{0.35}  \\

        \bottomrule
        \end{tabular}
        \caption{Quantitative ablation studies on the task of unconditional scene synthesis on the 3D-FRONT bedrooms.} % Note that for the Scene Classification Accuracy (SCA), the score close to 0.5 is better.
        \label{tab:ablation}
        \end{center}
        \vspace{-6mm}
\end{table}