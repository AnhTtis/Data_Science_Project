 \def\isarxiv{1} %%% for icml submission version, we comment this line

\ifdefined\isarxiv
\documentclass[11pt]{article}

\usepackage[numbers]{natbib}

\else
%\documentclass[nohyperref]{article}
\documentclass{article}
%\usepackage{icml2022}
\usepackage{neurips_2022}
\fi


\usepackage{amsmath}
\usepackage{amsthm}
\usepackage{amssymb}
\usepackage{algorithm}
\usepackage{subfig}
\usepackage{algpseudocode}
\usepackage{graphicx}
\usepackage{grffile}
\usepackage{wrapfig,epsfig}
\usepackage{url}
\usepackage{xcolor}
\usepackage{epstopdf}


\usepackage{bbm}
\usepackage{dsfont}

 %%% print refs in table of contents
%\displaybreak
\allowdisplaybreaks

%\usepackage[lmargin=1in,rmargin=1in,tmargin=0.8in,bmargin=0.8in]{geometry}

\ifdefined\isarxiv

\let\C\relax
\usepackage{tikz}
\usepackage{hyperref}  %%% arxiv don't allow this.
\hypersetup{colorlinks=true,citecolor=blue,linkcolor=blue} %%% Zhao : maybe we should comment this in submission.
\usetikzlibrary{arrows}
\usepackage[margin=1in]{geometry}

\else

\usepackage{microtype}
\usepackage{hyperref}
\definecolor{mydarkblue}{rgb}{0,0.08,0.45}
\hypersetup{colorlinks=true, citecolor=mydarkblue,linkcolor=mydarkblue}
%\usepackage[capitalize,noabbrev]{cleveref}
%\usepackage{colortbl}

\fi
%\linespread{1}
%\newcommand{\QED}{\hfill$\qed$}
%\graphicspath{{./figs/}}


\newtheorem{theorem}{Theorem}[section]
\newtheorem{lemma}[theorem]{Lemma}
\newtheorem{definition}[theorem]{Definition}
\newtheorem{notation}[theorem]{Notation}
%\newtheorem{proof}[theorem]{Proof}
\newtheorem{proposition}[theorem]{Proposition}
\newtheorem{corollary}[theorem]{Corollary}
\newtheorem{conjecture}[theorem]{Conjecture}
\newtheorem{assumption}[theorem]{Assumption}
\newtheorem{observation}[theorem]{Observation}
\newtheorem{fact}[theorem]{Fact}
\newtheorem{remark}[theorem]{Remark}
\newtheorem{claim}[theorem]{Claim}
\newtheorem{example}[theorem]{Example}
\newtheorem{problem}[theorem]{Problem}
\newtheorem{open}[theorem]{Open Problem}
\newtheorem{property}[theorem]{Property}
\newtheorem{hypothesis}[theorem]{Hypothesis}

\newcommand{\wh}{\widehat}
\newcommand{\wt}{\widetilde}
\newcommand{\ov}{\overline}
\newcommand{\N}{\mathcal{N}}
\newcommand{\R}{\mathbb{R}}
\newcommand{\RHS}{\mathrm{RHS}}
\newcommand{\LHS}{\mathrm{LHS}}
\renewcommand{\d}{\mathrm{d}}
\renewcommand{\i}{\mathbf{i}}
\renewcommand{\tilde}{\wt}
\renewcommand{\hat}{\wh}
\newcommand{\Tmat}{{\cal T}_{\mathrm{mat}}}

\DeclareMathOperator*{\E}{{\mathbb{E}}}
\DeclareMathOperator*{\var}{\mathrm{Var}}
\DeclareMathOperator*{\Z}{\mathbb{Z}}
\DeclareMathOperator*{\C}{\mathbb{C}}
\DeclareMathOperator*{\D}{\mathcal{D}}
\DeclareMathOperator*{\median}{median}
\DeclareMathOperator*{\mean}{mean}
\DeclareMathOperator{\OPT}{OPT}
\DeclareMathOperator{\supp}{supp}
\DeclareMathOperator{\poly}{poly}
\DeclareMathOperator{\asy}{asy}

\DeclareMathOperator{\nnz}{nnz}
\DeclareMathOperator{\sparsity}{sparsity}
\DeclareMathOperator{\rank}{rank}
\DeclareMathOperator{\diag}{diag}
\DeclareMathOperator{\dist}{dist}
\DeclareMathOperator{\cost}{cost}
\DeclareMathOperator{\vect}{vec}
\DeclareMathOperator{\tr}{tr}
\DeclareMathOperator{\dis}{dis}
\DeclareMathOperator{\cts}{cts}
\DeclareMathOperator{\Var}{Var}


\makeatletter
\newcommand*{\RN}[1]{\expandafter\@slowromancap\romannumeral #1@}
\makeatother
\newcommand{\Zhao}[1]{{\color{red}[Zhao: #1]}}
\newcommand{\Danyang}[1]{{\color{purple}[Danyang: #1]}}
\newcommand{\Yeqi}[1]{{\color{blue}[Yeqi: #1]}} %%%Change to intern name


\usepackage{lineno}
\def\linenumberfont{\normalfont\small}





\begin{document}

\ifdefined\isarxiv

\date{}


\title{An Over-parameterized Exponential Regression}
\author{
Yeqi Gao\thanks{\texttt{a916755226@gmail.com}. The University of Washington.}
\and
Sridhar Mahadevan\thanks{\texttt{smahadev@adobe.com}. Adobe Research.}
\and
Zhao Song\thanks{\texttt{zsong@adobe.com} Adobe Research.}
}




\else

\title{Intern Project} 
\maketitle 
\iffalse
\icmltitlerunning{????}
%\linenumbers

\twocolumn[

\icmltitle{???}
% It is OKAY to include author information, even for blind
% submissions: the style file will automatically remove it for you
% unless you've provided the [accepted] option to the icml2019
% package.

% List of affiliations: The first argument should be a (short)
% identifier you will use later to specify author affiliations
% Academic affiliations should list Department, University, City, Region, Country
% Industry affiliations should list Company, City, Region, Country

% You can specify symbols, otherwise they are numbered in order.
% Ideally, you should not use this facility. Affiliations will be numbered
% in order of appearance and this is the preferred way.
\icmlsetsymbol{equal}{*}

\begin{icmlauthorlist}
\icmlauthor{Aeiau Zzzz}{equal,to}
\icmlauthor{Bauiu C.~Yyyy}{equal,to,goo}
\icmlauthor{Cieua Vvvvv}{goo}
\icmlauthor{Iaesut Saoeu}{ed}
\icmlauthor{Fiuea Rrrr}{to}
\icmlauthor{Tateu H.~Yasehe}{ed,to,goo}
\icmlauthor{Aaoeu Iasoh}{goo}
\icmlauthor{Buiui Eueu}{ed}
\icmlauthor{Aeuia Zzzz}{ed}
\icmlauthor{Bieea C.~Yyyy}{to,goo}
\icmlauthor{Teoau Xxxx}{ed}\label{eq:335_2}
\icmlauthor{Eee Pppp}{ed}
\end{icmlauthorlist}

\icmlaffiliation{to}{Department of Computation, University of Torontoland, Torontoland, Canada}
\icmlaffiliation{goo}{Googol ShallowMind, New London, Michigan, USA}
\icmlaffiliation{ed}{School of Computation, University of Edenborrow, Edenborrow, United Kingdom}

\icmlcorrespondingauthor{Cieua Vvvvv}{c.vvvvv@googol.com}
\icmlcorrespondingauthor{Eee Pppp}{ep@eden.co.uk}

% You may provide any keywords that you
% find helpful for describing your paper; these are used to populate
% the "keywords" metadata in the PDF but will not be shown in the document
\icmlkeywords{Machine Learning, ICML}

\vskip 0.3in
]

\printAffiliationsAndNotice{\icmlEqualContribution} 
\fi
\fi





\ifdefined\isarxiv
\begin{titlepage}
  \maketitle
  \begin{abstract}


Over the past few years, there has been a significant amount of research focused on studying the ReLU activation function, with the aim of achieving neural network convergence through over-parametrization. However, recent developments in the field of Large Language Models (LLMs) have sparked interest in the use of exponential activation functions, specifically in the attention mechanism.

Mathematically, we define the neural function $F: \R^{d \times m} \times  \mathbb{R}^d \rightarrow \mathbb{R}$ using an exponential activation function. Given a set of data points with labels $\{(x_1, y_1), (x_2, y_2), \dots, (x_n, y_n)\} \subset \mathbb{R}^d \times \mathbb{R}$ where $n$ denotes the number of the data. Here $F(W(t),x)$ can be expressed as $F(W(t),x) := \sum_{r=1}^m a_r \exp(\langle w_r, x \rangle)$, where $m$ represents the number of neurons, and $w_r(t)$ are weights at time $t$. It's standard in literature that $a_r$ are the fixed weights and it's never changed during the training. We initialize the weights $W(0) \in \mathbb{R}^{d \times m}$ with random Gaussian distributions, such that $w_r(0) \sim \mathcal{N}(0, I_d)$ and initialize $a_r$ from random sign distribution for each $r \in [m]$.

Using the gradient descent algorithm, we can find a weight $W(T)$ such that $\| F(W(T), X) - y \|_2 \leq \epsilon$ holds with probability $1-\delta$, where $\epsilon \in (0,0.1)$ and $m = \Omega(n^{2+o(1)}\log(n/\delta))$. To optimize the over-parametrization bound $m$, we employ several tight analysis techniques from previous studies [Song and Yang arXiv 2019, Munteanu, Omlor, Song and Woodruff ICML 2022]. 

 


  \end{abstract}
  \thispagestyle{empty}
\end{titlepage}

{\hypersetup{linkcolor=black}
\tableofcontents
}
\newpage

\else

\begin{abstract}


Over the past few years, there has been a significant amount of research focused on studying the ReLU activation function, with the aim of achieving neural network convergence through over-parametrization. However, recent developments in the field of Large Language Models (LLMs) have sparked interest in the use of exponential activation functions, specifically in the attention mechanism.

Mathematically, we define the neural function $F: \R^{d \times m} \times  \mathbb{R}^d \rightarrow \mathbb{R}$ using an exponential activation function. Given a set of data points with labels $\{(x_1, y_1), (x_2, y_2), \dots, (x_n, y_n)\} \subset \mathbb{R}^d \times \mathbb{R}$ where $n$ denotes the number of the data. Here $F(W(t),x)$ can be expressed as $F(W(t),x) := \sum_{r=1}^m a_r \exp(\langle w_r, x \rangle)$, where $m$ represents the number of neurons, and $w_r(t)$ are weights at time $t$. It's standard in literature that $a_r$ are the fixed weights and it's never changed during the training. We initialize the weights $W(0) \in \mathbb{R}^{d \times m}$ with random Gaussian distributions, such that $w_r(0) \sim \mathcal{N}(0, I_d)$ and initialize $a_r$ from random sign distribution for each $r \in [m]$.

Using the gradient descent algorithm, we can find a weight $W(T)$ such that $\| F(W(T), X) - y \|_2 \leq \epsilon$ holds with probability $1-\delta$, where $\epsilon \in (0,0.1)$ and $m = \Omega(n^{2+o(1)}\log(n/\delta))$. To optimize the over-parametrization bound $m$, we employ several tight analysis techniques from previous studies [Song and Yang arXiv 2019, Munteanu, Omlor, Song and Woodruff ICML 2022]. 

 

\end{abstract}

\fi


\section{Introduction}

The increasing complexity of source code poses a key challenge to the reliability of large-scale software systems. Software bugs in these systems can lead to safety issues~\cite{bug_safety} for users around the world as well as cause non-negligible financial losses~\cite{bug_loss}. As such, developers have to spend a large amount of time and effort on bug fixing. Consequently, \aprfull (\apr), designed to automatically generate patches to fix software bugs, has attracted wide attention from both academia and industry~\cite{long2016prophet, legoues2012genprog, long2015spr, lou2020can, tufano2018empstudy}. 


To achieve \apr, one popular approach is known as Generate-and-Validate (G\&V)~\cite{qi2015gv, ghanbari2019prapr, lou2020can, le2016hdrepair, legoues2012genprog, wen2018capgen, hua2018sketchfix, martinez2016astor, koyuncu2020fixminder, liu2019tbar, liu2019avatar}, which is typically based on the following pipeline: First, fault localization techniques~\cite{wong2016fl, abreu2007ochiai, zhang2013injecting, papadakis2015metallaxis, li2019deepfl, li2017transforming} are applied to determine the suspicious locations in programs where bugs are likely to exist. Then, the buggy locations are used by the \apr tools to generate a list of patches that replace buggy lines with correct lines. Afterward, each patch is validated against the original test suite to identify any \emph{plausible patches} (i.e., passing all tests in the test suite). Finally, to determine the \emph{correct patches}, developers examine the list of plausible patches to see if any of them can correctly fix the bug. 

Traditional \apr tools can mainly be categorized into heuristic-based~\cite{legoues2012genprog, le2016hdrepair, wen2018capgen}, constraint-based~\cite{mechtaev2016angelix, le2017s3, demacro2014nopol, long2015spr} and \template~\cite{ghanbari2019prapr, hua2018sketchfix, martinez2016astor, liu2019tbar, liu2019avatar}. Among these traditional tools, \template \apr tools~\cite{ghanbari2019prapr, liu2019tbar, benton2020effectiveness} have been able to achieve state-of-the-art results. \Template \apr tools typically leverage pre-defined templates (e.g., adding a nullness check) for bug fixing. However, since these fix templates are typically handcrafted, the number and types of bugs they are able to fix can be limited. 



To address the limitations of traditional \apr, researchers have proposed various \learning \apr tools~\cite{li2020dlfix, chen2018sequencer, jiang2021cure, lutellier2020coconut, zhu2021recoder, ye2022rewardrepair} based on the \nmtfull (\nmt) architecture~\cite{sutskever2014mt} where the input is the buggy code snippets and the goal is to translate the buggy code snippets into a fixed version. To accomplish this, \learning \apr tools require supervised training datasets with pairs of both buggy and fixed code snippets in order to learn how to perform this translation step. These training data are usually obtained by mining historical bug fixes using heuristics/keywords~\cite{dallmeier2007benchmark}, which can be imprecise for identifying bug-fixing commits; even the actual bug-fixing commits can include irrelevant code changes, leading to further pollution in the dataset~\cite{xia2022alpharepair}.
% 
Moreover, it can be hard for such \apr tools to generalize and fix bug types unseen during training. 



To better leverage recent advances in \plmfull{s} (\plm{s}), researchers~\cite{xia2022alpharepair, xia2023repairstudy, kolak2022patch, prenner2021codexws} have directly applied \plm{s} to generate patches without bug-fixing datasets. These \llm-based \apr tools work by either directly generating a complete code function~\cite{prenner2021codexws, xia2023repairstudy} or predict/infill the correct code snippet given its surrounding context~\cite{xia2022alpharepair, xia2023repairstudy}. By directly using \llm{s} that are pre-trained on billions of open-source code snippets, \llm-based \apr tools can achieve state-of-the-art performance on many repair datasets~\cite{xia2022alpharepair}. 


% 
%
%

Traditional \apr tools have long used the insight of the \emph{plastic surgery hypothesis}~\cite{barr2014plastic} where it states that the code ingredients to fix a bug already exist within the same project. Traditional \apr tools have manually designed pattern-~\cite{ghanbari2019prapr, saha2017elixir} or heuristic-based~\cite{jiang2018simfix, legoues2012genprog} approaches to finding and using such relevant code ingredients to generate fixes for bugs. However, the plastic surgery hypothesis has been largely ignored in \llm-based \apr. In fact, \llm provides a unique opportunity to fully automate the plastic surgery hypothesis idea via fine-tuning (learning project-specific information via model updates from the buggy project) and prompting (directly providing relevant code ingredients to the model), and make it directly applicable to different languages (since the \llm{s} are typically multi-lingual).%
Moreover, despite the intensive manual efforts involved, traditional \apr tools still cannot fully leverage project-specific information due to large search space for leveraging/composing existing code ingredients. In contrast, the project-specific information can effectively leveraged by \llm{s} due to their power in code understanding/vectorization, e.g., even partial/imprecise information may still guide \llm{s} in correct patch generation!
 To this end, we ask the question: \emph{How useful is the plastic surgery hypothesis in the era of \plm{s}}?








\mypara{Our Work.} To answer the question, we present \ourtech{\xspace} -- a \llm-based approach that automatically utilizes the plastic surgery hypothesis by systematically combining multiple fine-tuning and prompting strategies for \apr. \ourtech fine-tunes \plm{s} using two novel domain-specific training strategies: \textbf{\epfinetune} -- we fine-tune using the original buggy project by aggressively masking out a high percentage of tokens, which allows \plm to learn project-specific code tokens and programming styles; and \textbf{\rofinetune} -- which only masks out a single continuous code sequence per training sample, allowing the model to get used to the final \csapr task of predicting a single continuous code sequence. Furthermore, we directly leverage the ability for \plm{s} to understand natural language instructions and introduce a novel prompting strategy, \textbf{\idprompting}, which uses information retrieval and static analysis to obtain a list of relevant identifiers for the buggy lines. While such relevant identifiers are critical for fixing some difficult bugs, they may not be seen by the \llm during inference due to limited context window size. Through the use of prompting, we directly tell the model to use these extracted identifiers (relevant code ingredients) to generate the correct code. Finally, to perform repair, we combine all four model variants (including the base model, both fine-tuned models and the base model with prompting) for the final repair.





While our insight of leveraging the plastic surgery hypothesis for \llm-based \apr is generalizable across different types of \plm{s}, to implement \ourtech, we choose a recent \plm{\xspace}, \ctfive~\cite{wang2021codet5}, which is pre-trained on millions of open-source code snippets. \ctfive is an encoder-decoder model trained using \mspfull (\msp) objective where a percentage of tokens are masked out and each continuous masked token sequence is referred to as a masked span. Also, although we only extract relevant identifiers from the current buggy project (since this paper focuses on the plastic surgery hypothesis), our work can be easily extended to obtain other code information (such as relevant statements or functions) from other sources, such as  the massive pre-training corpora~\cite{husain2020codesearchnet} or historical bug-fixing datasets~\cite{jiang2019infer}, which can provide more coding knowledge for \llm{s}. Besides, although we mainly focus on using traditional string comparison algorithms for information retrieval in this paper, these techniques can be easily replaced by other frequency-based retrieval~\cite{robertson2009probabilistic} and neural search (or embedding-based search)~\cite{reimers2019sentence}.
  In summary, this paper makes the following contributions:


%


\begin{itemize}[noitemsep, leftmargin=*, topsep=0pt]
    \item \textbf{Dimension.} This paper is the first to revisit the important plastic surgery hypothesis in the era of \llm{s}. It opens up a new dimension for \llm-based \apr to incorporate previously neglected information from the buggy project itself to boost \apr performance. Furthermore, it demonstrates the promising future of retrieval-based prompting for modern \llm-based \apr.
    \item \textbf{Implementation.} We implement \ourtech based on the recent \ctfive model. We augment the model using two novel fine-tuning strategies: \epfinetune and \rofinetune, along with a novel prompting strategy based on information retrieval and static analysis: \idprompting. We combine the patches generated by all four models together and perform patch ranking to speed up \apr.% 
    \item \textbf{Evaluation Study.} We conduct an extensive evaluation against state-of-the-art \apr tools. On the widely studied \dfj 1.2 and 2.0 datasets~\cite{just2014dfj}, \ourtech is able to achieve the new state-of-the-art results of 89 and 44 correct bug fixes (15 and 8 more than best baseline) respectively.  Furthermore, we perform a broad ablation study to justify our design. \ourtech demonstrates for the first time that the plastic surgery hypothesis can substantially boost \llm-based \apr and advance state-of-the-art \apr, while being fully automated and general. Moreover, even partial/imprecise code ingredients may still effectively guide \llm{s} for \apr!
\end{itemize}

 %%% Section 1. Introduction

\section{Technique Overview}\label{sec:tech_overview}
This paper presents a proof demonstrating that a two-layer neural network employing the exponential activation function which can achieve a desired small loss value after sufficient iterations, given a large enough number of neurons $m$, an appropriate learning rate $\eta$, and the initialization method specified in Definition~\ref{def:duplicate_weights}.

By bounding the difference of the weights over the training and choosing a proper learning rate $\eta$, we bound the loss by induction. We will introduce how we bound the loss under the assumption for the small perturbation on the weights. And then we will introduce how we bound the  weights and gradients respectively.

\paragraph{Bounding the loss by induction}

To establish this result, we begin by bounding the summation of the differences between the weights at the current step and their initial values, assuming that $w$ is in a small range such that $\Delta w_r(t)\leq R\in (0,0.01)$ where $t$ denote the step here. 

To establish an upper bound on the prediction loss $\|y-F(t+1)\|_2^2$, we first assume that the weight $w$ is within a small range, namely $\Delta w_r(t)\leq R\in (0,0.01)$. We decompose the loss into four parts: 
\begin{itemize}
    \item The loss at the previous step  $\|y-F(t)\|_2^2$
    \item $C_1:=-2m \eta (F(t)-y)^\top H(t) (F(t)-y)$
    \item $C_2:=2 (F(t)-y)^\top H_{\asy} (t) (F(t)-y)$
    \item $C_3:=\| F(t+1) - F(t) \|_2^2$
\end{itemize}

Using terms that involve the parameters $m$, $\eta$, $n$, and $B$ multiplied by $\|y-F(t)\|_2^2$, we can compute the upper bounds $C_1$, $C_2$, and $C_3$ respectively. Then, by induction and choosing a proper value for $m$, we can bound $\|y-F(t)\|_2^2$ where the loss is $\|y-F(0)\|_2^2= \|y\|_2^2$ at initialization under the given assumption.

\paragraph{Bounding the weights by induction}
Finally, to complete the proof, we establish an upper bound for $\Delta w_r(t)=\|w_r(t)-w_r(0)\|_2$. To do this, we transform $\Delta w_r(t)$ into a form that is a multiple of $\exp(B+R)\sqrt{n}$ and the current loss $\|y-F(t)\|_2$. 

\paragraph{Bounding the gradients by induction}
 Based on the result above, we will continue our work on bounding the changing of the weights over the training. By appropriately choosing the learning rate $\eta$, we can ensure that $\Delta w_r(t)$ is small enough, namely $0.01$. We have the following definition 
\begin{align*}
    H(w)_{i,j} :=  \frac{1}{m} \langle x_i,x_j\rangle \sum_{r\in [m]} \exp(\langle w_r,x_i\rangle)\cdot \exp( \langle w_r,x_j\rangle)
\end{align*}
where $r\in [m]$ as the index the neurons, and $x_i$ denotes the data where $i\in [n]$ and $j\in [n]$.

Next, we will constrain the changes of $H$ under the assumption that $w$ is located within a small ball. In addition, we must bound the discrepancy between discrete and continuous functions. Drawing upon the conclusions reached regarding perturbations in weight $w$, we can guarantee the convergence of the over-parametrized neural network with an exponential activation function. 





%\vspace{-2mm}
\section{Preliminary}\label{sec:preli}
\paragraph{Notations.}For a positive integer, we use $[n]$ to denote set $\{ 1,2,\cdots,n\}$. 
We use $\cosh(x) =\frac{1}{2}( e^x + e^{-x})$ and $\sinh(x) = \frac{1}{2}(e^x - e^{-x} )$.
For a square matrix, we use $\tr[A]$ to denote the trace of $A$.
An $n \times n$ symmetric real matrix $A$ is said to be positive-definite if $x^{\top} A x > 0$ for all non-zero $x \in \R^n$.
An $n \times n$ symmetric real matrix $A$ is said to be positive-semidefinite if $x^{\top} A x \geq 0$ for all non-zero $x \in \R^n$. For any function $f$, we use $\wt{O}(f) = f \cdot \poly(\log f)$.



\subsection{Matrix hyperbolic functions}
\begin{definition}[Matrix function]
Let $f:\R \rightarrow \R$ be a real function and $A\in \R^{n\times n}$ be a real symmetric function with eigendecomposition 
\begin{align*} 
A=Q\Lambda Q^{-1}
\end{align*}
where $\Lambda\in \R^{n\times n}$ is a diagonal matrix. Then, we have
\begin{align*}
    f(A):=Qf(\Lambda) Q^{-1},
\end{align*}
where $f(\Lambda)\in \R^{n\times n}$ is the matrix obtained by applying $f$ to each diagonal entry of $\Lambda$.
\end{definition}

We have the following lemma to bound $\cosh(A)$ and delay the proof to Appendix~\ref{sec:cosh_bound_proof}.
\begin{lemma}\label{lem:cosh_bound}
Let $A$ be a real symmetric matrix, then we have
\begin{align*}
    \|\cosh(A)\| = \cosh(\|A\|) \leq \tr[\cosh(A)].
\end{align*}
We also have 
\begin{align*}
\|A\| \leq 1+\log(\tr[\cosh(A)]).
\end{align*}
\end{lemma}



\subsection{Properties of \texorpdfstring{$\sinh$}{} 
and \texorpdfstring{$\cosh$}{}
}


We have the following lemma for properties of $\sinh$ and $\cosh$. 
\begin{lemma}[Scalar version]\label{lem:property_sinh_cosh_scalar}
Given a list of numbers $x_1, \cdots x_n$, we have
\begin{itemize}
    \item $( \sum_{i=1}^n \cosh^2(x_i) )^{1/2} \leq \sqrt{n} + ( \sum_{i=1}^n \sinh^2(x_i) )^{1/2}$,
    \item $(\sum_{i=1}^n \sinh^2(x_i) )^{1/2} \geq \frac{1}{\sqrt{n}} (\sum_{i=1}^n \cosh(x_i) - n)$.
\end{itemize}
\end{lemma}
\begin{proof}
For the first equation, we can bound $( \sum_{i=1}^n \cosh^2(x_i) )^{1/2}$ by:
\begin{align*}
     ( \sum_{i=1}^n \cosh^2(x_i) )^{1/2} 
    = &~ (n + \sum_{i=1}^n \sinh^2(x_i))^{1/2} \\
    \leq &~\sqrt{n} + ( \sum_{i=1}^n \sinh^2(x_i) )^{1/2}
\end{align*}
where the first step comes from fact~\ref{fact:cosh_sinh_1}, and the second step follows from $\sqrt{a + b} \leq \sqrt{a} + \sqrt{b}$.

For the second equation, we can bound $(\sum_{i=1}^n \sinh^2(x_i) )^{1/2}$ by:
\begin{align*}
    (\sum_{i=1}^n \sinh^2(x_i) )^{1/2} 
    \geq &~ \frac{1}{\sqrt{n}}(\sum_{i=1}^{n} \sinh(x_i)) \\
    \geq &~ \frac{1}{\sqrt{n}}(\sum_{i=1}^{n} \cosh(x_i) -n)
\end{align*}
where the first step follows that $\sqrt{\frac{\sum_{i=1}^{n} x_i^2}{n}} \geq \frac{\sum_{i=1}^{n} x_i}{n}$,
and the second step follows from fact~\ref{fact:cosh_sinh_1} and $\sqrt{x^2 -1} \geq \sqrt{x} - 1$.
\end{proof}


We also have a lemma for the matrix version. 
\begin{lemma}[Matrix version]\label{lem:property_sinh_cosh_matrix}
For any real symmetric matrix $A$, we have
\begin{itemize}
    \item $ (\tr[\cosh^2(A)])^{1/2} \leq \sqrt{n} + \tr[ \sinh^2(A) ]^{1/2}$,
    \item $(\tr[ \sinh^2(A) ])^{1/2} \geq \frac{1}{\sqrt{n}} ( \tr[ \cosh(A) ] - n ) $.
\end{itemize}
\end{lemma}

\begin{proof}

{\bf Part 1.}
We have
\begin{align*}
    (\tr[\cosh^2(A)])^{1/2} = & ~ ( n+  \tr[\sinh^2(A)] )^{1/2} \\
    \leq & ~ \sqrt{n} + \tr[ \sinh^2(A) ]^{1/2}.
\end{align*}
where the first step follows from $ \cosh^2(A) - \sinh^2(A) = I$.

{\bf Part 2.}
Let $\sigma_i$ denote the singular value of $\cosh(A)$
\begin{align*}
    ( \tr[  \sinh^2(A) ] )^{1/2} 
    = & ~  ( \tr[ \cosh^2(A) ] - n )^{1/2} \\
    = & ~  ( \sum_{i=1}^n \sigma_i^2 - 1 )^{1/2} \\ 
    \geq & ~   \frac{1}{\sqrt{n}} \sum_{i=1}^n \sqrt{ \sigma_i^2  -1 } \\
    \geq & ~    \frac{1}{\sqrt{n}} (\sum_{i=1}^n \sigma_i - 1 ) \\
    = & ~ \frac{1}{\sqrt{n}} ( \tr[  \cosh(A) ] -  n ) 
\end{align*}
where the second step follows from $\| \cdot \|_2 \geq \frac{1}{\sqrt{n}} \| \cdot \|_1$, the third step follows from $\sigma_i \geq 1$.

\end{proof}





\newpage
\usetikzlibrary{quotes,arrows.meta}

\usepackage{Box}











 



\section{Convergence}\label{sec:convergence}
In Section~\ref{sec:convergence}, when the neural network is excessively over-parametrized, we observe a linear decrease in the training error, leading to its ultimate convergence to 0.
In Section~\ref{sec:converge:mainresult}, we present our paper's main result. Section~\ref{sec:converge:induction_weights} outlines the induction lemma for weights, while Section~\ref{sec:converge:induction_loss} introduces the induction lemma for loss. Finally, in Section~\ref{sec:coverge:induction_gradient}, we provide the induction lemma for gradient.


\subsection{Main Result}\label{sec:converge:mainresult}

\begin{theorem}[Main result, formal version of Theorem~\ref{thm:informal}]\label{thm:formal}
If the following conditions hold
\begin{itemize}
    \item Let  $\lambda=\lambda_{\min}(H^{\cts})>0$
    \item  $m = \Omega( \lambda^{-2} n^2 \exp(4 B) \log^2(n/\delta) )$
    \item Let $w_r$ and $a_r$ be defined as Definition~\ref{def:duplicate_weights}.
    \item Let $\eta=0.01 \lambda / ( m n^2 \exp(4 B) )$
    \item Let $T = \Omega( (m \eta \lambda)^{-1} \log(n/\epsilon)  ) = \Omega( \lambda^{-2} n^2 \exp(4B) \cdot \log(n/\epsilon) )$
\end{itemize}
Then, we have running algorithm with $T$ iterations
\begin{align*}
    \| F(T) - y \|_2^2 \leq \epsilon
\end{align*}
\end{theorem}
\begin{proof}

We choose $\sigma = 1$.

We have proved $\| F(0) - y \|_2^2 \leq n$.

Using the choice of $T$, then it directly follows from alternatively applying Lemma~\ref{lem:induction_part_2_loss} and Lemma~\ref{lem:induction_part_1_weights}.

Since $\exp(\Theta(B)) = n^{o(1)}$, then in Theorem~\ref{thm:informal}, we can simplify the $n^2 \exp(\Theta(B)) = n^{2+o(1)}$.
\end{proof}


\subsection{Induction Part 1. For Weights}\label{sec:converge:induction_weights}

\begin{lemma}[Induction Part 1 for weights]\label{lem:induction_part_1_weights}
If the following condition hold
\begin{itemize}
    \item General Condition 1. Let $\lambda = \lambda_{\min} (H^{\cts}) > 0$
    \item General Condition 2. $\eta=0.01\lambda /(mn^2 \exp{(4 B)})$
    \item General Condition 3. Let $D$ be defined as Definition~\ref{def:D}
    \item General Condition 4. Let $w_r$ and $a_r$ be defined as Definition~\ref{def:duplicate_weights}.
    \item General Condition 5. $D < R$
    
    \item {\bf Weights Condition.} $\| w_r(i) - w_r(0)\|_2 \leq  R$ for all $i \in [t]$
    \item {\bf Loss Condition.}  $\| F(i) - y \|_2^2 \leq \| F(0) - y \|_2^2\cdot (1-m\eta \lambda/2)^i$,  $\forall i \in [t]$
    \item {\bf Gradient Condition.} $\eta \| \Delta w_r(i) \|_2 \leq 0.01$ for all $r \in [m]$, for all $i
    \in [t]$
\end{itemize}

For $t+1$ and $\forall r\in [m]$, it holds that:
\begin{align*}
\| w_r(t+1) - w_r(0) \|_2 \leq D.
\end{align*}
\end{lemma}
 

\begin{proof}


We have
\begin{align} \label{eq:upper_bound_etasum}
& ~ \eta \sum_{i=0}^\infty (1-n\lambda/2)^{i/2}\notag\\
= & ~ \eta \sum_{i=0}^\infty (1-\eta \lambda/4)^i \notag \\
\leq & ~ \eta \frac{1}{ \eta \lambda / 4 } \notag \\
\leq & ~ 8/\lambda
\end{align}
where the 1st step is due to Fact~\ref{fac:taylor}, the 2nd step is due to Fact~\ref{fac:taylor}, the last step is due to simple algebra.

Our approach involves utilizing the gradient's norm as a means of constraining the distance as follows:
\begin{align*}
& ~ \|w_r(0)-w_r(t+1)\|_2\\
\le & ~\eta \sum_{i=0}^{t} \| \Delta w_r(i) \|_2 \\
\le & ~ \eta \sum_{i=0}^{t} \exp(B+R) \cdot \sqrt{n}  \cdot \|F(i)-y\|_2 \\
\le & ~ \eta \sum_{i=0}^{t} (1-{\eta \lambda}/{2})^{i/2} \cdot \exp(B +  R) \cdot \sqrt{n} \cdot \|F(0)-y\|_2 \\
\le & ~ 8 \sqrt{n} \cdot \lambda^{-1} \cdot \exp(B + R)  \|F(0)-y\|_2 \\
= & ~ D
\end{align*}
where the 1st is from $w_r(s+1)-w_r(s)=\eta \cdot \Delta w_r(s)$, the 2nd step is due to Lemma~\ref{lem:bound_Delta_w_at_time_s} for $m t$ times, the 3rd step is due to Condition 3 in Lemma statement, the forth step is due to simple algebra, and the forth step is due to Eq.~\eqref{eq:upper_bound_etasum}, the last step is due to Condition 2 in Lemma statement.
\end{proof}




\subsection{Induction Part 2. For Loss}\label{sec:converge:induction_loss}
Now, we present our next induction lemma.
\begin{lemma}[Induction Part 2. For Loss]\label{lem:induction_part_2_loss}
Let $t$ be a fixed integer. 

If the following conditions hold
\begin{itemize}
   \item General Condition 1. Let $\lambda = \lambda_{\min} (H^{\cts}) > 0$
    \item General Condition 2. $\eta=0.01\lambda /(mn^2 \exp{(4 B)})$
    \item General Condition 3. Let $D$ be defined as Definition~\ref{def:D}
    \item General Condition 4. Let $w_r$ and $a_r$ be defined as Definition~\ref{def:duplicate_weights}.
    \item General Condition 5. $D < R$
    \item {\bf Weight Condition.} $\| w_r(t) - w_r(0) \|_2 \leq D < R$, $\forall r \in [m]$
    \item {\bf Loss Condition.} $\| F(i) - y \|_2^2 \leq (1-m\eta \lambda/2)^i \cdot \| F(0) - y \|_2^2$, for all $i \in [t]$
    \item {\bf Gradient Condition.} $\eta \| \Delta w_r(i) \|_2 \leq 0.01$  $\forall r \in [m]$, $\forall i\in [t]$
\end{itemize}
Then we have
\begin{align*}
\| F (t+1) - y \|_2^2 \leq ( 1 - m \eta \lambda / 2 )^{t+1} \cdot \| F (0) - y \|_2^2.
\end{align*}
\end{lemma}
\begin{proof}
 



 
Recall the update rule (Definition~\ref{def:update}),
\begin{align*}
w_{r}(t+1) = w_r(t) - \eta \cdot \Delta w_{r}(t)
\end{align*}

$\forall i \in [n]$, it follows that

\begin{align*}
& ~ F_i(t+1) - F_i(t) \\
= & ~   \sum_{r\in [m]} a_r \cdot ( \exp( \langle w_r(t+1),x_i \rangle) - \exp(\langle w_r(t),x_i \rangle) )  \\
= & ~  \sum_{r\in [m]} a_r \cdot \exp(\langle w_r(t),x_i\rangle) \cdot ( \exp(- \eta \langle \Delta w_r(t),x_i\rangle) - 1 ) \\
= & ~  \sum_{r\in [m]} a_r \cdot \exp(w_r(t)^\top x_i) \cdot (-\eta \langle \Delta w_r(t), x_i \rangle + \Theta(1) \eta^2 \langle \Delta w_r(t), x_i \rangle^2 ) \\
= & ~ v_{1,i} + v_{2,i}
\end{align*}
where the third step follows from $|\eta \Delta w_r(t)^\top x_i| \leq 0.01$ and Fact~\ref{fac:taylor},  the last step is from
\begin{align*}
v_{1,i}:= & ~ \sum_{r=1}^m a_r \cdot \exp(\langle w_r(t),x_i\rangle) \cdot (-\eta \langle \Delta w_r(t), x_i \rangle ) \\
v_{2,i}:= & ~ \sum_{r=1}^m a_r \cdot \exp(\langle w_r(t),x_i\rangle) \cdot \Theta(1) \cdot \eta^2 \cdot  \langle \Delta w_r(t), x_i \rangle^2
\end{align*}
Here $v_{1,i}$ is linear in $\eta$ and $v_{2,i}$ is quadratic in $\eta$. Thus, $v_{1,i}$ is the first order term, and $v_{2,i}$ is the second order term.
 


Recall the definition of $H$ over timestamp $t$ (see Definition~\ref{def:H_s})
\begin{align*}
H(t)_{i,j} = & ~ \frac{1}{m} \sum_{r\in [m]} x_i^\top x_j \exp( \langle w_r(t),x_i\rangle) \cdot \exp( \langle w_r(t),x_j\rangle) ,  
\end{align*}
Further, we define $C_1, C_2, C_3$
\begin{align*}
C_1 = & ~ -2 \eta (F(t)- y )^\top H(t) ( F(t)-y) , \\ 
C_2 = & ~ - 2 (F(t)- y  )^\top v_2 , \\
C_3 = & ~ \| F (t+1) - F(t) \|_2^2 . 
\end{align*}
Then we can rewrite
\begin{align*}
\| y -F(t+1) \|_2^2 = \| y - F(t) \|_2^2 + C_1 + C_2 + C_3
\end{align*}



We have
\begin{align*}
  \|F(t)-y\|_2^2 
 \leq  \|F(t-1)-y\|_2^2 \cdot (1-m\eta \lambda/2) 
\end{align*}
 where the first step follows is due to Lemma~\ref{lem:loss_one_step_shrinking}. 

 
 
Thus, we complete the proof.
\end{proof}
 

\subsection{Induction Part 3. For Gradient }\label{sec:coverge:induction_gradient}


\begin{lemma}[Induction Part 3. For Loss]\label{lem:induction_part_3_gradient}
Let $t$ be a fixed integer. 

If the following conditions hold
\begin{itemize}
   \item General Condition 1. Let $\lambda = \lambda_{\min} (H^{\cts}) > 0$
    \item General Condition 2. $\eta=0.01\lambda /(mn^2 \exp{(4 B)})$
    \item General Condition 3. Let $D$ be defined as Definition~\ref{def:D}
    \item General Condition 4. Let $w_r$ and $a_r$ be defined as Definition~\ref{def:duplicate_weights}.
    \item General Condition 5. $D < R$
    \item {\bf Weight Condition.} $\| w_r(t) - w_r(0) \|_2 \leq D < R$, $\forall r \in [m]$
    \item {\bf Loss Condition.} $\|  F(i) - y \|_2^2 \leq \| F(0) - y \|_2^2 \cdot (1-m\eta \lambda/2)^i $, $\forall i \in [t]$
    \item {\bf Gradient Condition.} $\eta \| \Delta w_r(i) \|_2 \leq 0.01$  $\forall r \in [m]$, $\forall i \in [t]$
\end{itemize}
Then we have
\begin{align*}
\eta \| \Delta w_r(t+1) \|_2 \leq 0.01, \forall r \in [m]
\end{align*}
\end{lemma}
\begin{proof}

We have
\begin{align*}
\eta \| \Delta w_r(t+1) \|_2 
= & ~ \eta \left\| \sum_{i=1}^n  a_r x_i \cdot (y_i - F_i(t+1))   \cdot \exp(  \langle w_r(t+1), x_i\rangle ) \right\|_2 \notag\\
\leq & ~ \eta \exp (B + R) \cdot    \sum_{i=1}^n | y_i - F_i(t+1) | \notag\\
\leq & ~ \eta \exp (B + R)  \cdot  \sqrt{n} \cdot \| y - F(s) \|_2  \\
\leq & ~ \eta \exp (B + R)  \cdot  \sqrt{n} \cdot \| y - F(0) \|_2  \\
\leq & ~ \eta \exp (B + R)  \cdot n \\
\leq & ~ 0.01
\end{align*}
where the 1st step follows from Definition~\ref{def:Delta_w_r_at_time_t}, the 2nd step is due to Lemma~\ref{lem:bound_on_exp_w_and_perturb_w}, the 3rd step is due to Cauchy-Schwartz inequality, the 4th step follows is due to {\bf Loss Condition}, the 5th step follows from $\| y-F(0) \|_2 = \sqrt{n}$, the sixth step is due to the choice of $\eta$.

\end{proof}

\newpage
\section{Induction Part 1: For Weight}\label{sec:induction_for_weight}
In Section~\ref{sec:induction_for_weight}, we present the weight bound, which helps us complete the proof. Section~\ref{sec:induction:definition} introduces various definitions used throughout the paper, while Section~\ref{sec:induction:bound_gradient} proposes the bounding gradient lemma and its corresponding proof.


\subsection{Definition of \texorpdfstring{$D$}{}}\label{sec:induction:definition}

To simplify the notation, we present the definition as follows.
\begin{definition}\label{def:D}
We define $D_{\cts}$ 
\begin{align*}
D := 8 \cdot \lambda^{-1} \cdot \exp( B + R ) \cdot \frac{ \sqrt{n} }{ m } \cdot \| y - F(0) \|_2 .
\end{align*}
\end{definition}

We define the kernel with respect to timestample $s$.
\begin{definition}\label{def:H_s}
Let $H(s) \in \R^{n \times n}$ be a matrix defined for any $s$ in the interval $[0,t]$.
\begin{align*}
H(s)_{i,j} := \frac{1}{m} \sum_{r\in [m]} x_i^\top x_j \cdot \exp{(\langle w_r(s),x_i\rangle)} \cdot \exp{(\langle w_r(s),x_j\rangle)}.
\end{align*} 
\end{definition}
\begin{definition} \label{def:H_asy}
For any matrix $P \in [-1,1]^{m \times n}$, we define asymmetric matrix $H_{\asy}$

\begin{align*}
H_{\asy}(s)_{i,j} := \frac{1}{m} \sum_{r\in [m]} x_i^\top x_j \cdot p_{i,r} \cdot \exp{(\langle w_r(s),x_i\rangle)} \cdot \exp{(\langle w_r(s),x_j\rangle)}.
\end{align*} 

\end{definition}
\begin{claim} \label{cla:upper_bound_HP}
We have
\begin{align*}
\| H_{\asy} (s) \|_{\infty} \leq \exp( 2 (B + R) ).
\end{align*}
holds with probability $1-\delta$.
\end{claim}
\begin{proof}

\begin{align*}
     \|H(P)\|_\infty
=  \max_{i\in [n], j\in [n]}\{ \frac{1}{m} \sum_{r\in [m]} x_i^\top x_j \cdot p_{i,r} \cdot \exp{(w_r(s)^\top x_i)} \cdot \exp{(w_r(s)^\top x_j)} \} 
\end{align*}
where the first step is from Definition~\ref{def:H_asy}.

It is sufficient to make a bound for each $i \in [n]$ and $j \in [n]$.

We have
\begin{align*}
  & ~ \frac{1}{m} \sum_{r\in [m]} x_i^\top x_j \cdot p_{i,r} \cdot \exp{(\langle w_r(s), x_i\rangle)} \cdot \exp{(\langle w_r(s), x_j\rangle)} \} \\
 \leq ~ &  \frac{1}{m} \sum_{r\in [m]}  \exp{(\langle w_r(s),x_i\rangle)} \cdot \exp{(\langle w_r(s), x_j\rangle)}\\
\leq ~ &  \frac{1}{m} \sum_{r\in [m]}  \exp{(2 (R+B))}\\
= ~ & \exp(2(B+R))
\end{align*}
 the 1st step is from $\| x_i \|_2 \leq 1$ and $|p_{i,r}| \leq 1$, the second step is due to Lemma~\ref{lem:bound_on_exp_w_and_perturb_w}.
\end{proof}

 

\subsection{Bounding the gradient at any time}\label{sec:induction:bound_gradient}
In this section, we bound the gradient at any time.
\begin{lemma}\label{lem:bound_Delta_w_at_time_s}
It the following condition hold,
\begin{itemize}
    \item $\| w_r(s) - w_r(0) \|_2 \leq R$
\end{itemize}
For any timestamp at time $s$, we have
\begin{align*}
\| \Delta w_r(s) \|_2 
\leq \exp(B+R )  \sqrt{n}  \| y - F(s) \|_2
\end{align*}
\end{lemma}
\begin{proof}

We have
\begin{align*}
\| \Delta w_r(s) \|_2 
= & ~  \left\| \sum_{i=1}^n (y_i - F_i)   a_r x_i \cdot \exp(  w_r(s)^\top x_i ) \right\|_2 \notag\\
\leq & ~ \exp (B + R) \cdot    \sum_{i=1}^n | y_i - F_i(s) | \notag\\
\leq & ~ \exp (B + R)  \cdot  \sqrt{n} \cdot \| y - F(s) \|_2 
\end{align*}
where the first step follows from Definition~\ref{def:Delta_w_r_at_time_t}, the second step follows from Lemma~\ref{lem:bound_on_exp_w_and_perturb_w}, the third step follows from Cauchy-Schwartz inequality.
\end{proof}

\begin{lemma}\label{lem:bound_Delta_w_times_eta}
It the following condition hold,
\begin{itemize}
    \item $\eta = 0.01 \lambda /(m n^2 \exp(4B))$
    \item $\| w_r(s) - w_r(0) \|_2 \leq R$
\end{itemize}
For any timestamp at time $s$, we have
\begin{align*}
\eta \| \Delta w_r(s) \|_2 
\leq 0.01
\end{align*}
\end{lemma}
\begin{proof}
This trivially follows from choice of $\eta$.
\end{proof}

\newpage
\begin{table*}[tb]
  \small
  \centering
  \begin{tabular}{l|c|c}
    \toprule
    % \hline
    Models& $\mathcal{L}_{\text{con}}$ & $\mathcal{L}_{\text{div}}$ \\
    \midrule
    \textit{Contrastive learning:}\\
    SimCLR & $-f(\mathbf{x}^i)^Tf(\hat{\mathbf{x}}^i)$ & $\log\left(e^{f(\mathbf{x}^i)^Tf(\hat{\mathbf{x}}^i)} + \sum_{\mathbf{x}^j \in \mathbf{B}}e^{f(\mathbf{x}^i)^Tf(\mathbf{x}^j)}\right)$\\
    MoCo & $-f(\mathbf{x}^i)^Tf(\hat{\mathbf{x}}^i)$ & $\log\left(e^{f(\mathbf{x}^i)^Tf(\hat{\mathbf{x}}^i)} + \sum_{\mathbf{x}^j \in \mathbf{M}}e^{f(\mathbf{x}^i)^Tf(\mathbf{x}^j)}\right)$ \\
    \midrule
    \textit{Feature decorrelation:}\\
    Barlow Twins & $\sum_{j=1}^d \left(1 - \sum_{i=1}^N f_j(\mathbf{x}^i)f_j(\hat{\mathbf{x}}^i)\right)^2$ & $\lambda \sum_{k \not = j} \left(\sum_{i=1}^Nf_j(\mathbf{x}^i)f_k(\hat{\mathbf{x}}^i) \right)^2$\\
    \midrule
    \textit{Asymmetric network:}\\
    BYOL (DirectPred) & $-f(\mathbf{x}^i)^T\mathbf{W}_p^Tf(\hat{\mathbf{x}}^i)$ & $\sqrt{\sum_{\mathbf{x}^j \in \mathbf{T}} \rho_j (f(\mathbf{x}^i)^Tf(\mathbf{x}^j))^2 + \epsilon^2 \lambda_{\text{max}} f(\mathbf{x}^i)^Tf(\mathbf{x}^i)}  \ \|f(\hat{\mathbf{x}}^i)\|_2$\\
    \midrule
    \textit{Reconstruction:}\\
    MAE & $\|g(f(\hat{\mathbf{x}}^i)) - \mathbb{E}_{\hat{\mathbf{x}}^i \sim \mathcal{T}(\mathbf{x}^i)}[g(f(\hat{\mathbf{x}}^i))]\|^2$& $\|\mathbb{E}_{\hat{\mathbf{x}}^i \sim \mathcal{T}(\mathbf{x}^i)}[g(f(\hat{\mathbf{x}}^i))] - \mathbf{x}^i\|^2$\\
    \bottomrule
\end{tabular}
\caption{Consistency \wh{and} diversity \wh{losses} of different pretraining \wh{models}.}
\label{tab:loss}
\end{table*}

\begin{table}[tb]
  \small
  \centering
  \begin{tabular}{l|c|c|c}
    \toprule
    % \hline
    Models& $\mathcal{C}$ & $\mathcal{D}$ & Linear Eval\\
    \midrule
    BYOL & 0.99 & 0.08 & collapse\\
    \midrule
    SkeletonCLR & 0.11 & 36.8 & 78.9\\
    AimCLR & 0.18 & 36.4 & 79.7\\
    \midrule
    \textbf{F4F} & 0.41 & 8.06 & 83.0\\
    \bottomrule
\end{tabular}
  \caption{
  Consistency and diversity of different pretraining methods on NTU 60 xview dataset with \wh{the} joint stream. $\mathcal{C}$ means the consistency metrics and $\mathcal{D}$ is the diversity metrics.
  }
  \label{tab:cd}
\end{table}

\ifdefined\isarxiv
%\section*{Acknowledgments}
\bibliographystyle{alpha}
\bibliography{ref}
\else
\bibliography{ref}
%\bibliographystyle{icml2022}
\bibliographystyle{alpha}

\fi



\newpage
\onecolumn
\appendix





%%%% Cut-line between first 10 pages and appendix







%%% some writing rules

%% Writing rule for creating tags.
%% Tags :
%% Theorem    \ref{thm:bla_bla}
%% Lemma      \ref{lem:bla_bla}
%% Claim      \ref{cla:bla_bla}
%% Corollary  \ref{cor:bla_bla}
%% Fact       \ref{fac:bla_bla}
%% Definition \ref{def:bla_bla}
%% Section    \ref{sec:bla_bla}
%% Subsection \ref{sub:bla_bla}
%% Equation   \ref{eq:bla_bla}



\end{document}



%%%%%%%%%%%%%%%%%%%%%%%%%%%%%%%%%%%%%%%%%%%%%%%%%%%%%%%%%%%%%%%%%%%%%%%%%%%%%%%%%%%%%%%%%%%%%%%%%%%%%%%%%%%%%%%%%%%%%%%%%%%%%%%%%%%%%%%%%%%%%%%%%%%%%%%%%%%%%%%%%%%%%%%%%%%%%%%%%%%%%%%%%%%%%%%%%%%%%%%%%%%%%%%%%%%%%%%%%%%%%%%%%%%%%%%%%%%%%%%%%%%%%%%%%%%%%%%%%%%%%%%%%%%%%%%%%%%%%%%%%%%%%%%%%%%%%%%%%%%%%%%%%%%%%%%%%%%%%%%%%%%%%%%%%%%%%%%%%%%%%%%%%%%%%%%%%%%%%%%%%%%%%%%%%%%%%%%%%%%%%%%%%%%%%%%%%%%%%%%%%%%%%%%%%%%%%%%%%%%%%%%%%%%%%%%%%%%%%%%%%%%%%%%%%%%%%%%%%%%%%%
