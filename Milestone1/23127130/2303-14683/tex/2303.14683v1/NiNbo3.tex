%%%%%%%%%%%%%%%%%%%%%%%%%%%%%%%%%%%%%%%%%%%%%%%%%%%%%%%%%%%%%%%%%%%%%%%%
%    INSTITUTE OF PHYSICS PUBLISHING                                   %
%                                                                      %
%   `Preparing an article for publication in an Institute of Physics   %
%    Publishing journal using LaTeX'                                   %
%                                                                      %
%    LaTeX source code `ioplau2e.tex' used to generate `author         %
%    guidelines', the documentation explaining and demonstrating use   %
%    of the Institute of Physics Publishing LaTeX preprint files       %
%    `iopart.cls, iopart12.clo and iopart10.clo'.                      %
%                                                                      %
%    `ioplau2e.tex' itself uses LaTeX with `iopart.cls'                %
%                                                                      %
%%%%%%%%%%%%%%%%%%%%%%%%%%%%%%%%%%
%
%
% First we have a character check
%
% ! exclamation mark    " double quote  
% # hash                ` opening quote (grave)
% & ampersand           ' closing quote (acute)
% $ dollar              % percent       
% ( open parenthesis    ) close paren.  
% - hyphen              = equals sign
% | vertical bar        ~ tilde         
% @ at sign             _ underscore
% { open curly brace    } close curly   
% [ open square         ] close square bracket
% + plus sign           ; semi-colon    
% * asterisk            : colon
% < open angle bracket  > close angle   
% , comma               . full stop
% ? question mark       / forward slash 
% \ backslash           ^ circumflex
%
% ABCDEFGHIJKLMNOPQRSTUVWXYZ 
% abcdefghijklmnopqrstuvwxyz 
% 1234567890
%
%%%%%%%%%%%%%%%%%%%%%%%%%%%%%%%%%%%%%%%%%%%%%%%%%%%%%%%%%%%%%%%%%%%
%
\documentclass[12pt]{iopart}
\usepackage{graphicx}
\usepackage{subfigure}
% \usepackage{amsmath,amsthm,mathrsfs,amsfonts,dsfont}
\usepackage{color}
\usepackage[10pt]{moresize}
\usepackage{amssymb}
\usepackage{amscd}
\usepackage{enumerate}
\usepackage{epsfig}
\usepackage{bm}
\usepackage{color}
\usepackage{threeparttable}
\usepackage[square, comma, sort&compress, numbers]{natbib}
\usepackage{url}
\usepackage{hyperref}
\hypersetup{colorlinks,linkcolor=blue, anchorcolor=blue, urlcolor=blue, citecolor=blue}


\newcommand{\gguide}{{\it Preparing graphics for IOP Publishing journals}}
%Uncomment next line if AMS fonts required
\usepackage{iopams}  
\begin{document}

\title{Effect of light injection on the security of practical quantum key distribution}

\author{Liying Han$^{1,2,3}$, Yang Li$^{1,2,3,*}$, Hao Tan$^{2,3}$, Weiyang Zhang$^{2,3}$, Wenqi Cai$^{1,2,3}$, Juan Yin$^{1,2,3}$, Jigang Ren$^{1,2,3}$, Feihu Xu$^{1,2,3}$, Shengkai Liao$^{1,2,3,*}$, Chengzhi Peng$^{1,2,3}$}

\address{
$^1$ Hefei National Research Center for Physical Sciences at the Microscale and School of Physical Sciences, University of Science and Technology of China, Hefei 230026, China

$^2$ Shanghai Research Center for Quantum Science and CAS Center for Excellence in Quantum Information and Quantum Physics, University of Science and Technology of China, Shanghai 201315, China

$^3$ Hefei National Laboratory, University of Science and Technology of China, Hefei 230088, China

$^*$ Authors to whom any correspondence should be addressed.
}
\ead{\mailto{liyang9@ustc.edu.cn}, \mailto{skliao@ustc.edu.cn}}
\vspace{10pt}
\begin{indented}
\item[]December 2022
\end{indented}

\begin{abstract}
Quantum key distribution (QKD) based on the fundamental laws of quantum physics can allow the distribution of secure keys between distant users. 
However, the imperfections in realistic devices may lead to potential security risks, which must be accurately characterized and considered in practical security analysis.
High-speed optical modulators, being as one of the core components of practical QKD systems, can be used to prepare the required quantum states. 
Here, we find that optical modulators based on $LiNbO_3$ are vulnerable to photorefractive effect caused by external light injection. 
By changing the power of external light, eavesdroppers can control the intensities of the prepared states, posing a potential threat to the security of QKD.
We have experimentally demonstrated the influence of light injection on $LiNbO_3$-based optical modulators and analyzed the security risks caused by the potential green light injection attack, along with the corresponding countermeasures.
\end{abstract}

\noindent{\it Keywords}: quantum cryptography, quantum key distribution, quantum security, quantum communication
\maketitle
%
%
% Uncomment for keywords
%\vspace{2pc}
%\noindent{\it Keywords}: XXXXXX, YYYYYYYY, ZZZZZZZZZ
%
% Uncomment for Submitted to journal title message
%\submitto{\JPA}
%
% Uncomment if a separate title page is required
%\maketitle
% 
% For two-column output uncomment the next line and choose [10pt] rather than [12pt] in the \documentclass declaration
%\ioptwocol
%



\section{Introduction}

Since the first protocol proposed in 1984 \cite{Bennett1984}, quantum key distribution (QKD), which can enable two distant parties Alice and Bob to share the same secret keys with information-theoretical-proven security, has made tremendous developments in the past few decades \cite{Bennett1989ACS,Peng2007RPL,rosenberg2007longPRL,wang2022twin,Buttler1998PRL,Ursin2007NP,Liao2017NP,yin2020,chen2021,RN73}. 
However, in practical QKD systems, there is a gap between the devices and the theoretical model, and imperfect devices could be exploited by an eavesdropper to obtain information about the secret keys \cite{Xu2020RMP,THA1,THA2,THA3,sun2015,pang2020,huang2019,huang2020,Ponosova2022,Xu_2010,Wu:20,qi2007,Vadim2011,Tan_2022}. 
Many methods have been proposed to defeat the attacks. 
Security patching is a practical approach, that is, once one discovers a new type of attack, corresponding countermeasures against this attack can be proposed and realized in the existing QKD systems. The second approach is to fully characterize the devices using in a QKD system and describe the devices accurately in mathematical models.  
Moreover, measurement-device-independent QKD \cite{lo2012} and twin-field QKD \cite{TF2018} have been recently proposed, which are immune all the potential detection-side loopholes. 
The source side becomes the more vulnerable part of any QKD setup and attracts more attention.
% Attacks at practical QKD systems include Trojan horse attack \cite{THA1,THA2,THA3}, laser seeding attack \cite{sun2015,pang2020,huang2019}, laser-damage attack \cite{huang2020,Ponosova2022},phase-remapping attack \cite{Xu_2010}, detector blinding attack \cite{Wu:20}, time-shift attack \cite{qi2007}, channel calibration attack \cite{Vadim2011}, etc.

There have been several researches on the potential loopholes at the source-side components of practical QKD systems \cite{THA1,THA2,sun2015,pang2020,huang2019,huang2020,Ponosova2022}, including laser diodes (LDs), attenuators, optical modulators, isolators and so forth. 
By injecting intense external light into the LDs, the phase, wavelength or intensity of the output light may be changed and some side-channel information can be extracted by the eavesdroppers \cite{sun2015,pang2020,huang2019}. 
There are also some attacks against optical modulators, such as Trojan horse attack \cite{THA1,THA2,THA3} and phase-remapping attack \cite{Xu_2010}, where phase information can be extracted from a phase modulator (PM) using optical-frequency-domain reflectometry or from the imperfection of electrical pulses for modulation. 
Meanwhile, the recent work demonstrates that it is not foolproof to adopt optical isolators and attenuators to resist external light \cite{Tan_2022}, as their performance deteriorates in the presence of external magnetic field and external strong light. 
% There is no doubt that optical modulators are key device of QKD systems, so research on its attack is very significant to the security of practical QKD.

Currently, high-speed optical modulators are widely applied in practical QKD systems to prepare the required quantum states for different protocols, and $LiNbO_3$-based optical modulators are the most  used type due to the wide transparent window, high refractive index, high second-order nonlinearity and stable physical and chemical characteristics \cite{QiLi2020}.
Lithium niobate is a kind of photorefractive material, whose refractive index distribution can be controlled by external illumination \cite{ma2012}. And the light-induced $\Delta$n produces beam degradation during propagation and light intensity limitation effects \cite{Jubera2014,Villarroel:10}. 
Based on this characteristic, we propose and demonstrate a method of using external green light injection to affect the security of QKD system. 
Our experiment demonstrates that all the tested modulators show increased insertion loss after green light irradiation, with maximum losses of 7.19 dB for the PM samples and 1.31 dB for the intensity modulator (IM) samples.
These insertion loss variation can be further recovered and restored to initial state by shining weaker green light on the modulator.
Taking advantage of this ability to proactively control the intensity of the emitted quantum light, the security of practical QKD systems may be compromised by the eavesdroppers.
Based on our experimental results, we then analyze the security risks caused by the potential green light injection attack and discuss about the corresponding countermeasures.

This paper is organized as follows. In section 2, we give an introduction to $LiNbO_3$-based modulators in QKD systems. In section 3, we show the experimental setup of green light injection, and the  results with several modulator samples. In section 4, we analyze the effect of this phenomenon on the security of QKD. In section 5, we discuss about some potential countermeasures against the hacking strategy proposed. Finally, we make conclusions in section 6.



\section{LiNbO3-based modulators in quantum key distribution}
$LiNbO_3$-based optical modulators take advantage of linear electro-optic effect to realize optical phase shift, which is an optically second-order non-linear effect, also known as the Pockels-Effect. 
This effect describes the change in the refractive index of an optical material when an external electric field is applied. 
The amount of change in refractive index is proportional to the strength of the electric field, as described in Fig. \ref{fig1}(b).
Based on the characteristic of optical phase shift, $LiNbO_3$-based optical modulators are widely applied in QKD applications, including both decoy state encoding \cite{wang2005,lo2005} and quantum state encoding.
The decoy state encoding can be realized either with a direct $LiNbO_3$ IM \cite{Xu2020RMP}, or with an intensity modulator built using a Sagnac interferometer and a $LiNbO_3$ PM \cite{Roberts:2018}.
The quantum state encoding $LiNbO_3$ using phase modulators includes phase encoding \cite{PhysRevA.101.032319}, time-bin phase encoding \cite{PhysRevLett.121.190502}, polarization encoding \cite{Li:19} and Gaussian discrete modulations \cite{ma2014}, etc.

\begin{figure}
\centering
\includegraphics[width=1\linewidth]{fig1.pdf}
\caption{
(a) Internal structure and components of phase modulators, including a substrate, a $LiNbO_3$-based waveguide and electrodes that provide an external electric field. 
(b) Phase modulator characteristic curve.
(c) Internal structure and components of intensity modulators, which uses a Mach–Zehnder interferometer configuration with two branches. 
(d) Intensity modulator characteristic curve. 
Applied voltage will cause a phase difference between the two branches, resulting in a change in output power.
}
\label{fig1}
\end{figure}

Photorefractive is another optical non-linear effect of $LiNbO_3$, which results from photovoltaic effect caused by external illumination \cite{kosters2009,zhang1996}. 
The generation process of photorefractive effect is as follows: external light drives photoexcited charges into adjacent bands and forms the photogenerated charge carriers; the charges leave the light area and settle in the dark area driven by the carrier concentration distribution, extra electric field or photovoltaic effect thus; the Space-charge distribution induces a space-charge field, resulting in a change in the refractive index distribution. For $LiNbO_3$ crystals, the photovoltaic effect is the main mechanism for the movement of photoexcited carriers in the absence of an applied electric field.
Under the exposure of the photoelectric field, the refractive index distribution of the waveguide undergoes a permanent change.
This change will not disappear immediately in the absence of external light, and can be recovered by uniform illumination, heating or placing in the dark. 
The photorefractive effect can be preserved for quite a long time - days or even months - which is why $LiNbO_3$ can be used for data storage. 
The biggest difference between photorefraction and optical damage is whether the change can be recovered or not.
The light power required to produce photorefractive effects is much weaker than other nonlinear effects,and  milliwatt-level light is sufficient for photorefractive crystal. 
Based on the above features, an eavesdropper can attempt to use external light to change refractive index distribution, thereby compromising the security of practical QKD. 
% We therefore propose and demonstrate a method for using external green light to influence the safety of QKD systems. 
% The details of the experiments and results can be found in Section 3.


\section{Experimental measurements on green light injection} 

\subsection{Experimental setup}
In our experiments, we select several $LiNbO_3$ modulator samples to test their properties before and after green light irradiation, the basic information of which is summarized in Table \ref{tab1}. 
Among them, two of the PMs are from the same manufacturer, Conquer, but from different batches. 
The parameters to be tested include insertion loss, half-wave voltage (V$\pi$) and extinction ratio (only for IM).
For PM, the insertion loss is the ratio of the output power to the input power, and the V$\pi$ is the voltage required to increase the phase by $\pi$.
For IM, the insertion loss here is defined as the sum of maximum transmitted power and minimum transmitted power in the increases cycle, where V$\pi$ is the voltage required to change the output power from minimum to maximum, and the extinction ratio is the result of dividing the maximum and minimum power.
% As for PMs, insertion loss and half-wave voltage (V$\pi$) are main parameters to be concerned in use. 
% We focus on testing their changes and whether the changes can be restored or not. 
% As for IMs, the extinction ratio is also tested except for insertion loss and V$\pi$. 

\begin{table*}
\caption{\label{tab1}
Basic information on all the modulators under test (4 phase modulators and 2 intensity modulators), including manufacturer, waveguide process and Doping process. 
$LiNbO_3$-based modulators are usually fabricated using two processes, Ti diffusion or proton exchange, along with a selective doping process to increase the threshold of photorefractive resistance.} 

\begin{indented}
\lineup
\item[]\begin{tabular}{@{}*{4}{l}}
\br                              
Number&Manufacturer&Waveguide process&Doping process\cr 
\mr
PM-1&Conquer&Ti diffusion&-\cr
PM-2&Conquer&Ti diffusion&-\cr
PM-3&Ixblue&Unspecified&MgO\cr
PM-4&Eospace&Unspecified&Unspecified\cr
IM-1&Conquer&Proton exchange&-\cr
IM-2&Eospace&Unspecified&Unspecified\cr
\br
\end{tabular}
\end{indented}
\end{table*}

The test of insertion loss is performed using the setup shown in Fig. \ref{fig2}(a). 
The test laser is a fiber-pigtailed 1550 nm laser diode, combined with the 532 nm injection laser using a wavelength division multiplexer (WDM).
Power meter A monitors the power of the 532 nm laser after a 50:50 fiber beam-splitter (BS), while power meter B is used to check the loss of modulator before and after green light injection. 
In order to test V$\pi$, we use the PM to construct a Mach-Zehnder-interferometer configuration, which can be replaced when test the IMs, as shown in Fig \ref{fig2}(b). 
The electronic pulse generator produces two sawtooth waves of the same frequency, one for triggering the oscilloscope and the other for modulating the PM/IM. 
The electrical signal output by the photodetector is also connected to the oscilloscope, similar to the curve in Fig. \ref{fig1}(d). 
V$\pi$ can be obtained by calculating the period of the measured sinusoidal curve. 
The extinction ratio of the IM can be obtained by applying accretive DC voltage and measuring the maximum and minimum power within a half-wave voltage cycle.

\begin{figure}
\centering 
\includegraphics[width=1\linewidth]{fig2.pdf}
\caption{Simplified diagram of experimental setup of (a) insertion loss, with the PM/IM as a replaceable device under test.
BS: beam splitter; WDM: wavelength division multiplexer.
(b) V$\pi$, with the PM in a MZI configuration or IM as a replaceable device under test.
PG: electronic pulse generator; MZI: Mach-Zehnder interferometer.
}
\label{fig2}
\end{figure}



\subsection{Experimental results}
\subsubsection{Phase modulator}
We first test the insertion loss of four PM samples before and after green light irradiation, and the test results are shown in Fig. \ref{fig3}. 
We increase the power of the incident green light to 2 mW with a step size of 200 $\mu$W and an exposure time of 5 minutes per step.
All four PMs show an increase in insertion loss with increasing green light power.
Particularly, PM-1 exhibits a maximum insertion loss increase of about 7.19 dB, while PM-2 and PM-3 exhibit similar performance with insertion loss increases of slightly less than 1 dB. 
At the same time, PM-4 does not seem to be greatly affected by the green light of 2 mW, and its insertion loss increases by 0.5 dB as the green light power continues increases to 8 mW. 
We will not continue to increase the optical power in case irreversible photo-damage occurs.
To recover the insertion loss, we inject 50 uW of green light into the PMs with an exposure time of 30 seconds per step, and record the insertion loss over time. 
All four PMs quickly recover their insertion loss within 4 minutes within the error range.
The above tests demonstrate that insertion loss of $LiNbO_3$-based PMs can be controlled by external green light, and the whole process takes less than 10 minutes.

\begin{figure}
\centering 
\includegraphics[width=1\linewidth]{fig3.pdf}
\caption{(a) Test results of increased insertion loss of four PMs. 
We increase the power of the incident green light to 2 mW with a step size of 200 $\mu$W and an exposure time of 5 minutes per step.
The PMs show an increase in insertion loss with a maximum of 7.19 dB.
(b)  Test results of recovered insertion loss of four PMs.
We inject 50 $\mu$W of green light to restore the insertion loss of modulators with an exposure time of 30 seconds per step.
All four PMs rapidly restore insertion loss to original state within 4 minutes.
}
\label{fig3}
\end{figure}

We also test the V$\pi$ of four PMs before, after green light irradiation and after recovery, and the test results are shown in Table \ref{tab2}. 
Particularly, the half wave voltage of PM-1 show an maximum increase of 1.53 V after green light irradiation, and can also be recovered to original state within the error range.

\begin{table*}
\caption{\label{tab2}
Test results of half wave voltage of PMs before green light irradiation (V$\pi_{before}$), after green light irradiation (V$\pi_{after}$) and after recovery (V$\pi_{recovery}$). 
$\Delta$V$\pi$ is the increase of V$\pi$ after green light irradiation. 
$\Delta$Loss is the maximum increase of insertion loss in the process of green light irradiation. 
The half wave voltage of the PMs show an maximum increase of 1.53 V after green light irradiation, and can also be recovered to original state.
% In order to test V$\pi$ before and after green light irradiation, we use the electronic pulse generator to produces two sawtooth waves. 
% The one for triggering is 500k HZ sawtooth wave with Vpp (trigger) of 1 V, and the other one for modulation is of the same frequency with Vpp (signal) of 10 V, that is higher than V$\pi$. 
% V$\pi$ can be calculated as below: V$\pi$ = 1/2*Vpp (signal)*(T$_2$/T$_1$), T$_2$ means the period of the sine curve, and T$_1$ is the period of the sawtooth wave.
} 

\begin{indented}
\lineup
\item[]\begin{tabular}{@{}*{6}{l}}
\br                              
Number&V$\pi_{before}$ (V)&V$\pi_{after}$ (V)&V$\pi_{recovery}$ (V)&$\Delta$V$\pi$ (V)&$\Delta$Loss (dB)\cr 
\mr
PM-1&4.04&5.57&4.03&1.53&7.19\cr
PM-2&3.90&4.06&3.91&0.16&0.75\cr
PM-3&4.68&5.10&4.70&0.42&0.91\cr
PM-4&2.79&2.79&2.78&0.00&0.50\cr
\br
\end{tabular}
\end{indented}
\end{table*}

From the above test results, it can be seen that the four PMs mainly exhibit the phenomenon of increased insertion loss and fast recovery under weaker green light injection, which is obviously consistent with the photorefractive characteristics.
Despite being from the same manufacturer, PM-2 shows better performance than PM-1, indicating variability between batches. 
We also briefly test another PM from the same batch with PM-1, and the results shows that the maximum loss increased by about 10 dB after green light irradiation. 
Therefore, we suspect that the batch of lithium niobate crystals used for PM-2 are of better quality and have fewer defects, resulting in less variation of insertion loss. 
As for PM-3, we learned from the manufacturer that PM is doped with an optical-damage-resistant Impurities-MgO, which can effectively increase the optical damage threshold \cite{lengyel2015growth} and may explain the smaller variation of insertion loss than PM-1. 
Unfortunately, we did not get useful information about PM-4 from its manufacturer, such as doping process and waveguide process.
Therefore, we cannot draw any conclusion about this. 

There are other recovery methods, such as placing the modulator in the dark and heating it \cite{zhang1996}. 
We then place the PM-1 in the dark for 3 days after irradiation with 2 mW green light, and the insertion loss recovered less than 10\% without additional irradiation. 
However, the insertion loss does not restore to its initial value after few days, complete recovery will take longer. 
This works in Eve's favor because the injection green laser doesn't need to be turned on all the time, making Eve less likely to be found.

% The other phenomenon is that half wave voltage of PMs shows increase correlated with the variation of insertion loss. 
% Photorefractive causes change of the refractive index distribution, that can lead to the increase of half wave voltage. 
% The variation of insertion loss reflects the degree of refractive index change from the side, so there is positive correlation between half wave voltage and insertion loss.

\subsubsection{Intensity modulator}
The test results of insertion loss and extinction ratio of two IM samples are shown in Fig. \ref{fig4}. 
Similar to PMs, the insertion loss of IMs generally increases with the power of injection green light, with a insertion loss increase of about 0.30 dB for IM-1 and 1.31 dB for IM-2. 
As for the recovery process, it takes about 60 minutes for IM-1 about 330 minutes for IM-2 to restore to original states, which are much longer than above PMs.
The extinction ratio of IMs also decrease as the power of injection green light increase, where IM-1 decreases from 44.39 dB to 23.16 dB and IM-2 decreases from 24.27 dB to 17.77 dB.
The decrease of extinction ratio can also be recovered with the injection of 50 $\mu$W of green light.
The test results of the half wave voltage of two IMs are shown in Table \ref{tab3}, and the maximum change is less than 0.1 V.
The above tests demonstrate that insertion loss of $LiNbO_3$-based IMs can also be controlled by external green light, but may take more time than above PMs.
% Compared with the recovery process of PMs, the time consumed by IMs is much longer. 
% However, all the tests above can still demonstrate that insertion loss of $LiNbO_3$-based IMs can be controlled by external green light. 
% test results of decreased  increases in general, and returns to the initial values after the recovery process. 

\begin{figure}
\centering 
\includegraphics[width=1\linewidth]{fig4.pdf}
\caption{(a) Test results of increased insertion loss of two IMs. 
We increase the power of the incident green light to 2 mW with a step size of 200 $\mu$W and an exposure time of 5 minutes per step.
The IMs show an increase in insertion loss with a maximum of 1.31 dB.
(b) Test results of decreased extinction ratio of two IMs.
The IMs show an decrease in extinction ratio with a maximum of 21.23 dB.
(c) Test results of recovered insertion loss of two IMs.
All two IMs restore insertion loss to original state within 330 minutes.
(d) Test results of recovered extinction ratio of two IMs.
IM-1 restores insertion loss to original state within 60 minutes, while IM-2 does not restore to original state after 330 minutes.
}
\label{fig4}
\end{figure}

\begin{table*}
\caption{\label{tab3}Half wave voltage of IMs before green light irradiation (V$\pi_{before}$), after green light irradiation (V$\pi_{after}$) and after recovery (V$\pi_{recovery}$). 
$\Delta$V$\pi$ is the increase of V$\pi$ after green light irradiation.
$\Delta$Loss is the maximum increase of insertion loss in the process of green light irradiation.} 

\begin{indented}
\lineup
\item[]\begin{tabular}{@{}*{6}{l}}
\br                              
Number&V$\pi_{before}$ (V)&V$\pi_{after}$ (V)&V$\pi_{recovery}$ (V)&$\Delta$V$\pi$ (V)&$\Delta$Loss (dB)\cr 
\mr
IM-1&5.00&5.06&5.06&0.06&0.39\cr
IM-2&4.04&4.11&4.04&0.07&1.31\cr
\br
\end{tabular}
\end{indented}
\end{table*}

From the above test results, IM-1 outperfoms IM-2 with less increase of insertion loss. 
We learned from the manufacturer that PM-1 uses waveguide process of Ti diffusion, while IM-1 uses waveguide process of proton exchange, and its photorefractive sensitivity is much lower than that of Ti-indiffused waveguide \cite{Kondo:94}. 
At the same time, we do not have process information from which to draw conclusions explaining the poor performance of IM-2. 
Compared with the PMs, IM exhibit different phenomenons in terms of extinction ratio decrease.
Here, the splitting ratio of 1550 nm Mach–Zehnder modulator is not 50 : 50 at 532 nm, which results in a different insertion loss increase of the two arms.
This will in turn causes the splitting ratio to be no longer 50:50 and results in a decreasing extinction ratio.
% Compared with the PMs, IM exhibit different phenomenons in terms of extinction ratio decrease and longer time taken to cover insertion loss. 
% \textcolor{red}{Meanwhile, the recovery rate of the attenuation of the two arms is different, that results from the different power of 532 nm light divided into two arms to restore the insertion loss. 
% Thus, the time consumed by IMs is much longer than PMs.}




\section{Security risk evaluation of green light injection attack}
The basic schematic of the green light injection attack is shown in Fig. \ref{fig5}.
We put no assumption on the quantum channel used for QKD, and the channel is completely open to the eavesdroppers. 
The eavesdropper, Eve, can launch a green light injection attack in three steps: Firstly, Eve injects strong green light from quantum channel into the QKD transmitter (Alice) to increase the insertion loss of modulators without the owner noticing; secondly, the owner of the QKD system calibrates the QKD system, which means Alice and QKD receiver (Bob) accept the fact that the calibrated QKD system is in the initial state and the owners start quantum communication; thirdly, Eve injects weaker green light to restore the insertion loss of modulators. 
By repeating the first and third steps to control insertion loss of modulators, Eve can actively control the output photon numbers of the QKD transmitter without drawing attention.
% As the experimental results shown in section 3, the output power of QKD transmitter becomes controllable for Eve after the green light injection. 

\begin{figure}
\centering 
\includegraphics[width=0.6\linewidth]{fig5.pdf}
\caption{Basic schematic of the green light injection attack.
An eavesdropper can actively control the output photon numbers of the QKD transmitter through external green light injection. 
}
\label{fig5}
\end{figure}

We adopt security analysis of QKD in Ref. \cite{huang2019} to the prepare-and-measure decoy-state BB84 protocol. 
The typical implementation is evaluated where Alice and Bob use three different intensities, $\mu_s$, $\nu_1$, and $\nu_2$ that satisfy $\mu_s > \nu_1 > \nu_2$ and $\nu_2 = 0$.
Secret keys can be extracted from those events employing the signal intensity $\mu_s$ in the Z basis and X basis, while the decoy $\nu_1$ intensity events are used for parameter estimation. 
In the asymptotic limit of an infinite number of transmitted signals, the secret key rate can be lower bounded by \cite{ma2005}
\begin{equation}
    R_L\geq\frac{1}{2}(-Q_\mu f_eH_2(E_\mu)+Q_1(1-H_2(e_1))),
\end{equation}
where Q$_\mu$ is the gain  of signal states, E$_\mu$ is the overall quantum bit error rate, Q$_1$ is the gain of single-photon states, e$_1$ is the error rate of single-photon states. Q$_\mu$ and E$_\mu$ can be measured directly from the experiment, and Q$_1$ and e$_1$ need to be estimated by preset experimental parameters. Here, for each given value of total loss, we select the optimal values of the intensities $\mu_s$, $\nu_1$ that maximize R$_L$. 

The $Q_1$ and e$_1$ can be estimated by   
\begin{equation}
    Q_1\geq\frac{\mu^2e^{-\mu}}{\mu\nu_1-\mu\nu_2-\nu_1^2+\nu_2^2}(Q_{\nu_1}e^{\nu_1}-Q_{\nu_2}e^{\nu_2}-\frac{\nu_1^2-\nu_2^2}{\mu^2}(Q_\mu e^\mu-Y_0)),
\label{2}
\end{equation}
\begin{equation}
    e_1\leq\frac{E_{\nu_1}Q_{\nu_1}e^{-\nu_1}-E_{\nu_2}Q_{\nu_1}e^{-\nu_2}}{(\nu_1-\nu_2)Y_1},
\label{3}
\end{equation}
where $\nu_1$ and $\nu_2$ are number of photons in strong and weak decoy states. Here, we take $\nu_2$ as 0. 

In the presence of green light injection attack, $\mu$ ({$\mu$ $\in$ \{$\mu_s$, $\nu_1$, $\nu_2$\}}) becomes $k\mu$ for a certain k that depends on the attack. 
If Alice and Bob fail to notice the green light injection, here they will estimate the parameters, Q$_1$ and e$_1$, with the observed quantities Q$_{k\mu}$ and E$_{k\mu}$ and the original intensities $\mu$. 
If they are aware of this attack, they estimate the parameters, Q$_1$ and e$_1$, with the changed intensities $k\mu$ and get true secure key rate. 
In order to correspond to the loss mentioned above, we convert k to $\Delta$Loss. 
For simulation purposes we use the experimental parameters as bellow: background rate, Y$_0$, is 2.6*10$^{-5}$; total misalignment error, e$_d$, is 0.01; error correction efficiency, f$_0$, is 1.12; detection efficiency, $\eta_d$, is 0.6. 
We take $\Delta$ = 3 dB as the typical value to evaluate the impact of insertion loss changes on the key rate, which matches the variation of insertion loss measured by our experiment.

\begin{figure}
\centering 
\includegraphics[width=1\linewidth]{fig6.pdf}
\caption{(a) Simulated secure key rate as a function of the total loss. We take $\Delta$ = 3 dB as the typical value to to evaluate the impact of insertion loss changes on the key rate. (b) Lower (R$_L$) bounds on the secret key rate as a function of the parameter $\Delta$Loss for a fixed total loss of 12.22 dB, that contains 10 dB of link loss and 0.6 of detection efficiency.
}
\label{fig6}
\end{figure}

The resulting lower bounds on the secret key rate are shown in Fig. \ref{fig6}. 
The black line indicates the lower bound $R_{L}$ given in the absence of the attack. 
The red solid line shows the value of $R_{L}$ estimated by Alice and Bob if they're not aware of the attack. 
The blue line, on the other hand, illustrates the correct secure value of $R_{L}$ in the presence of attack.
As we can see in Fig. \ref{fig6}(a), the secure $R_{L}$ given by the blue line is significantly lower than the $R_{L}$ actually estimated by Alice and Bob. 
More precisely, in the presence of the attack, the security proof introduced in Ref. \cite{ma2005} cannot guarantee the security of the key obtained by Alice and Bob. 
Furthermore, as shown in Fig. \ref{fig6}(b), the secure $R_{L}$ drops dramatically as $\Delta$ increases, and it turns out that a larger $\Delta$ may be a compromise of practical QKD. 
In summary, legitimate users of the system may significantly overestimate the secure key rate provided by appropriate security proofs in the presence of the attack.



\section{Countermeasures}
Our experimental results show that none of the $LiNbO_3$-based optical modulators is confidently robust against the green light injection attack.
An eavesdropper can actively control the output photon numbers of the QKD transmitter through external green light injection, which will affect the practical security of QKD systems.
Therefore, countermeasures against such attacks need to be developed. 

According to the test results, modulators from different manufacturers have different performances, and users should carefully choose $LiNbO_3$-based optical modulators, such as those doped with optical-damage-resistant Impurities-MgO. 
Modulators based on other materials can also be chosen, but similar effects of these modulators should be carefully tested before being applied to QKD systems.
Another option for Alice to protect the QKD transmitter is to apply optical isolators, which can significantly isolate Eve's injection light and make Eve more difficult to attack.
However, the reverse transmission isolation of short wavelengths is rarely studied, and we only note the work of researchers testing the isolation of 1550-nm isolators in the 1500–2100 nm range \cite{nasedkin2022quantum}. 
A wider wavelength range needs to be considered, such as all wavelength bands that the optical fiber can transmit.
We also note another recent work \cite{Tan_2022}, where the reverse transmission isolation of isolatosr and circulators is significantly reduced under the impact of an external magnetic fields. 
This work indicates that the performance of isolators may be compromised when other attacks are launched at the same time.
The third option for Alice to discover eavesdroppers might be to use an incident-light monitor to detect the injected light, including light of short wavelength. 
However, because the insertion loss does not restore to its initial value even after few days, it will be more difficult for the legitimate users to detect Eve without keeping the external laser on at all times. 
In order to discover Eve's existence, Alice needs to record all anomalous events and pick out the cases that belong to green light injection attack, which may be a difficult task. 
But overall, it's a feasible way to add a monitor. 
% There is no doubt that it would to be a lot harder for Eve to launch an attack if the isolator worked at short wavelength. 
% Additionally, doping $LiNbO_3$ with optical-damage-resistant impurities can lead to a higher photorefractive resistance threshold \cite{kong2012}.



\section{Conclusion and discussion}
In summary, we experimentally investigate the impact of photorefraction induced by injected green light on $LiNbO_3$-based optical modulators to enhance the security of practical QKD systems. 
We select several $LiNbO_3$ modulator samples to test their properties before and after green light irradiation, and they all exhibit increased insertion loss after green light irradiation. 
At the same time, the half-wave voltage and extinction ratio of the modulators also tend to increase and decrease, respectively.
We also experimentally investigate that all the changes can be recovered by shining weaker green light, providing convincing evidence for a hard-to-find green light injection attack, since Eve can actively control the output photon numbers of the QKD transmitter through external green light injection.
Based on our experimental results, we analyze the security risks of the this attack and find that the legitimate users of the system, when attacked, might significantly overestimate the secret key rate provided by proper security proofs. 
Our work highlights the significance of selecting optical modulators, adding optical isolators and monitoring the injection light power to enhance the security of the practical QKD system.
% A detailed analysis of its effect on decoy-state BB84 and MDI QKD protocols is given in Ref. \cite{huang2019}. 

% \section{Discussion and outlook}
In this work, we use 532 nm green light for this light injection attack. 
Ref \cite{luo2012} shows that Zr:Fe:$LiNbO_3$ crystals have larger refractive index change, higher recording sensitivity and larger dual-wave coupling gain coefficient at 473 nm wavelength than at 532 nm wavelength under the same experimental conditions. 
That means that more wavelengths of light can be utilized to carry out such attack, which makes defenses much more difficult. 
Which wavelength of light works best at the lowest light intensity is also a subject worth studying.
As for free-space QKD, the green light is often used as a beacon light, which will mix and overlap with the green light injected into the attack, making it more difficult to detect.

Because all the modulators exhibit obvious increase in insertion loss, so we accounted for this effect in our security risk evaluation. 
It is worth noting one of the PM samples shows a half-wave voltage increase of 1.53 V, which Eve can also taken advantage of. 
The effect of V$\pi$ increase is similar to the phase-remapping attack \cite{Xu_2010}, causing the phase difference between coded states to change. 
With the green light injection, Eve convinces Alice that the V$\pi$ has increased, which hadn't changed at all. 
Eve can remap the encoded phase information from \{0, $\pi$/2, $\pi$, 3$\pi$/2 \} to \{0, $\delta$/2, $\delta$, 3$\delta$/2\}, where $\delta$ \textgreater $\pi$. 
This means distinguishing between different states at the cost of a lower bit error rate. 
The green light injection makes the phase-remapping attack not only suitable for bidirectional QKD systems, such as the "plug-and-play" system, but also has a broader range of  applications. 
Thus, this phenomenon should be be studied in more detail, which can be carried out in follow-up works.   




\ack
This work was supported by National Key Research and Development Program of China (2020YFA0309701, 2020YFA0309703, 2020YFE0200600), Shanghai Municipal Science andTechnology MajorProject under Grant 2019SHZDZX01, Anhui Initiative in Quantum Information Technologies.


\section*{References}
\bibliographystyle{iopart-num}
\bibliography{NiNbo3}



\end{document}

