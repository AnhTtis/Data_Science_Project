



\onecolumn
\appendix



\section{asyrp}

\begin{figure}
\centering
\includegraphics[width=\linewidth]{tex/paper/figs/asyrp/asyrp_glasses.jpg}
\caption{Effect of Asyrp with Glasses direction  anycost}
% \label{fig:my_label}
\end{figure}


\section{PCA Results}


% \subsection{Small scale 10 inference steps, 250 samples for PCA}


\begin{figure}[h!]
\centering
\includegraphics[width=\textwidth]{tex/paper/figs/pca/pixel-pca_test_all10steps250samples.jpg}
\caption{PCA is calculated across timesteps simultaneously. Applied to two samples}
\end{figure}

\begin{figure}[h!]
\centering
\includegraphics[width=\textwidth]{tex/paper/figs/pca/pixel-pca_all_vs_indv10steps250samples.jpg}
\caption{Comparison of the effect of calculating PCA across all timesteps vs for each timestep individually. The resultant PCA directions are applied across all timesteps. We see small differences, mostly in the lesser dominant eigenvectors.}
\end{figure}

\begin{figure}[htb]
\centering
\begin{subfigure}[b]{\linewidth}
\includegraphics[width=\textwidth]{tex/paper/figs/pca/pixel-pca_test_all10steps250samples-eta1.jpg}
\caption{250 Samples}
\end{subfigure}
\begin{subfigure}[b]{\linewidth}
\includegraphics[width=\textwidth]{tex/paper/figs/pca/pixel-pca_test_all10steps500samples-eta1.jpg}
\caption{500 Samples}
\end{subfigure}

\begin{subfigure}[b]{\linewidth}
\includegraphics[width=\textwidth]{tex/paper/figs/pca/pixel-pca_test_all10steps1000samples-eta1.jpg}
\caption{1000 Samples}
\end{subfigure}
\begin{subfigure}[b]{\linewidth}
\includegraphics[width=\textwidth]{tex/paper/figs/pca/pixel-pca_test_all10steps2000samples-eta1.jpg}
\caption{2000 Samples}
\end{subfigure}

\caption{Pixel PCA, varying the number of samples}

\end{figure}
\begin{figure}[h!]
\centering
\includegraphics[width=\textwidth]{tex/paper/figs/pca/pixel-pca_test_all10steps5000samples-eta1.jpg}
\caption{
Pixel PCA 5000 samples}
\end{figure}


\begin{figure}[h!]
\centering
\includegraphics[width=\textwidth]{tex/paper/figs/pca/ldm-pca_test_all10steps250samples.jpg}
\caption{
 For the compvis LDM model.
 PCA is calculated across timesteps simultaneously. 
 Applied to two samples}
\end{figure}


% \subsection{Medium scale 50 inference steps, 250 samples for PCA}


\begin{figure}[h!]
\centering
\includegraphics[width=\textwidth]{tex/paper/figs/pca/pixel-pca_test_all50steps500samples-eta1.jpg}
\caption{Pixel CelebA PCA 50 inference steps 500 samples}
\end{figure}

\section{Power Iteration Results}
% \subsection{Small scale, 10 inference steps}

\begin{figure}[h!]
\centering
\includegraphics[width=\textwidth]{tex/paper/figs/poweriter/PowerIter10steps99window.jpg}
\caption{ Power Iteration 10 steps DDIM}
\end{figure}

\begin{figure}[h!]
\centering
\includegraphics[width=\textwidth]{tex/paper/figs/poweriter/PowerIter50steps99window.jpg}
\caption{ Power Iteration 50 steps DDIM}
\end{figure}


% \begin{figure}[h!]
% \centering
% \includegraphics[width=\linewidth]{tex/paper/figs/pca/init_plot.jpg}
% \caption{  \red{TODO: placeholder, maybe interesting maybe move to appendix}}
% \label{fig:pcainit}
% \end{figure}


% \section{Synthetic Data set}
% \begin{figure}[h!]
% \centering
% \includegraphics[width=0.75\textwidth]{figs/data_samples.jpg}
% \caption{Samples from the syntehtic data set created with TensorGAN \cite{Haas2022tensorGAN2}. In each block, top row is the neutral example and bottom has an attribute. From left to right the attributes are first the 6 prototypical emotions: 
% 'anger' 'disgust' 'fear', 'happiness' 'sadness' and  'surprise' and two rotations "left rot", "right rot"
% }
% \label{fig:sgdata}
% \end{figure}


% \section{Expression Editing}

% \begin{figure}[h!]
% \centering
% \includegraphics[width=0.8\textwidth]{figs/expr_edit_combined.jpg}
% \caption{Expression editing using pixel and latent space models with DDIM, DDPM as well as $h$-space with and without asyrp. In all cases we define the edit as $\mathbf{q}+\mathbf{n}$ based on directions $\mathbf{n}$ calculated from the synthetic dataset.
% \red{[NOTE] asyrp is *not* implemented for DDPM and results there defaults to symmetric reverse process}
% }
% \label{fig:expr-editing}
% \end{figure}

% \clearpage
% \section{Reconstruction vs inference steps}

% \begin{figure}[h!]
% \centering
% \begin{subfigure}[b]{\linewidth}
% \includegraphics[width=\linewidth]{figs/renconstruction_vs_inferencesteps.png}
% \caption{Reconstruction MSE as a function of the number of inference steps}
% % \vspace{10pt}
% % \label{fig:exp1-smile-compare}
% \end{subfigure}
% \begin{subfigure}[b]{\linewidth}
% \includegraphics[width=\linewidth]{figs/renconstruction_vs_inferencesteps_imgs.png}
% \caption{Recontructions for DDIM sampling different numbers of inference steps}
% % \label{fig:exp1-yaw-rot-compare}
% \end{subfigure}
% \caption{Reconstruction quality vs number of inference steps for DDIM/DDPM sampling}
% \end{figure}

% \clearpage
% \section{Visualization of noise directions}
% \begin{figure}[h!]
% \centering
% \begin{subfigure}[b]{\linewidth}
% \includegraphics[width=\linewidth]{figs/directions.jpg}
% \caption{Visualisation of expression noise maps ($\mathbf{x}^T$s) for DDIM}
% % \vspace{10pt}
% % \label{fig:exp1-smile-compare}
% \end{subfigure}

% \begin{subfigure}[b]{\linewidth}

% \includegraphics[width=\linewidth]{figs/directions-ddpm-zs.jpg}
% \caption{Visualisation of expression noise maps ($\mathbf{z}_t$s) for DDPM}
% % \label{fig:exp1-yaw-rot-compare}
% \end{subfigure}
% \caption{Visualisation of noise maps for DDIM and DDPM}
% \end{figure}
    
% \begin{figure}[h!]
% \centering
% \includegraphics[width=0.6\textwidth]{figs/renconstruction_vs_inferencesteps.png}
% \caption{Samples from the syntehtic dataset. IN each block, top row is the neutral example and bottom has an attribute.
% From left to right the attributes are first the 6 prototypical emotions: 
% 'anger' 'disgust' 'fear', 'happiness' 'sadness' and  'surprise' and two rotations "left rot", "right rot"
% }
% \label{fig:}
% \end{figure}