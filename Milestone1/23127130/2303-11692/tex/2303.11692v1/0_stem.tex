% Template for ICASSP-2020 paper; to be used with:
%          spconf.sty  - ICASSP/ICIP LaTeX style file, and
%          IEEEbib.bst - IEEE bibliography style file.
% --------------------------------------------------------------------------
\documentclass[9pt]{article}
\usepackage{spconf,amsmath,graphicx, cite}

% Example definitions.
% --------------------
\def\x{{\mathbf x}}
\def\L{{\cal L}}
\newcommand{\tabref}[1]{\mbox{Table~\ref{#1}}}
\newcommand{\figref}[1]{\mbox{Figure~\ref{#1}}}
\newcommand{\secref}[1]{\mbox{Section~\ref{#1}}}
\newcommand{\ssecref}[1]{\mbox{Subsection~\ref{#1}}}
%\newcommand{\eqref}[1]{\mbox{Eq~\ref{#1}}}
\usepackage{amsfonts}
\usepackage{url}
\usepackage{color}
%\usepackage{tablestyles}
\usepackage[english]{babel}
\usepackage[T1]{fontenc}
\usepackage[utf8]{inputenc}
\usepackage{booktabs} % To thicken table lines
\usepackage{algorithm}
\usepackage{algorithmic} 
\usepackage{nicefrac}
\usepackage{overpic}
\usepackage{tikz}
\usepackage{multirow}
 \usepackage{diagbox}


\newcommand{\equref}[1]{\mbox{Equation~\ref{#1}}}
\newcommand{\algref}[1]{\mbox{Algorithm~\ref{#1}}}
\newcommand{\sy}[1]{\textcolor{blue}{#1}} 


\DeclareRobustCommand\onedot{\futurelet\@let@token\@onedot}
\def\@onedot{\ifx\@let@token.\else.\null\fi\xspace}
\def\eg{\emph{e.g}\onedot} \def\Eg{\emph{E.g}\onedot}
\def\ie{\emph{i.e}\onedot} \def\Ie{\emph{I.e}\onedot}
\def\cf{\emph{c.f}\onedot} \def\Cf{\emph{C.f}\onedot}
\def\etc{\emph{etc}\onedot} \def\vs{\emph{vs}\onedot}
\def\wrt{w.r.t\onedot} \def\dof{d.o.f\onedot}
\def\etal{\emph{et al}\onedot}

\newcommand{\ourname}{ByteCover3}
\newcommand{\mname}{LAL}
\newcommand{\Vone}{ByteCover}
\newcommand{\chk}{{\centering\checkmark}}
\newcommand\kechen[1]{\textcolor{red}{#1}}



\newcommand\blfootnote[1]{%
  \begingroup
  \renewcommand\thefootnote{}\footnote{#1}%
  \addtocounter{footnote}{-1}%
  \endgroup
}


\newcommand\specparen[2]{%
  \def\Krn{\kern1ex}%
  \def\useanchorwidth{T}%
  \setbox0=\hbox{\Krn\stackengine{0pt}{\scriptstyle#1}{\scriptstyle#2}{O}{c}{F}{F}{S}}%
  \stackon[2pt]{\stackunder[2pt]{)}{\makebox[\wd0][l]{\Krn$\scriptstyle#1$}}}%
                                   {\makebox[\wd0][l]{\Krn$\scriptstyle#2$}}%
}
% https://tex.stackexchange.com/questions/165058/typesetting-sequence-notation



% Title.
% ------
\title{ByteCover3: Accurate Cover Song Identification on Short Queries}
%
% Single address.
% ---------------
\name{Xingjian Du$^1$, Zijie Wang$^{2\dagger}$, Xia Liang$^{1\dagger}$,  Huidong Liang$^3$, Bilei Zhu$^1$, Zejun Ma$^1$}
\address{$^1$ByteDance\quad $^2$Zhejiang University\quad$^3$University of Oxford}
%
% For example:
% ------------
%\address{School\\
%	Department\\
%	Address}
%
% Two addresses (uncomment and modify for two-address case).
% ----------------------------------------------------------
%\twoauthors
%  {A. Author-one, B. Author-two\sthanks{Thanks to XYZ agency for funding.}}
%	{School A-B\\
%	Department A-B\\
%	Address A-B}
%  {C. Author-three, D. Author-four\sthanks{The fourth author performed the work
%	while at ...}}
%	{School C-D\\
%	Department C-D\\
%	Address C-D}
%
\begin{document}
%\ninept
%
\maketitle
%

\begin{abstract}
Deep learning based methods have become a paradigm for cover song identification (CSI) in recent years, where the ByteCover systems have achieved state-of-the-art results on all the mainstream datasets of CSI. However, with the burgeon of short videos, many real-world applications require matching short music excerpts to full-length music tracks in the database, which is still under-explored and waiting for an industrial-level solution. In this paper, we upgrade the previous ByteCover systems to \emph{\ourname{}} that utilizes local features to further improve the identification performance of short music queries. \ourname{} is designed with a local alignment loss (LAL) module and a two-stage feature retrieval pipeline, allowing the system to perform CSI in a more precise and efficient way. We evaluated \ourname{} on multiple datasets with different benchmark settings, where \ourname{} beat all the compared methods including its previous versions. 
\blfootnote{$\dagger$ These authors contributed equally.}
\end{abstract}
%


\begin{keywords}
Cover song identification, ByteCover, local alignment loss, MaxMean similarity, short queries.
% classification loss, 
%triplet loss.
\end{keywords}

\begin{figure*}
	\centering
	\vspace{-2em}
	\includegraphics[width=0.9\textwidth]{figures/main.pdf}
	\vspace{-4em}
        \caption{The architecture of \ourname{}. \ourname{} uses an embedding extractor to extract local features from the CQT chunks and optimizes the model using the LAL loss. The LAL loss consists of a LAL classification loss and a LAL triplet loss.}
	\vspace{-1em}
	\label{fig:overall}

\end{figure*}


\section{Introduction}
% \begin{figure}[t]
% \centering
% \includegraphics[width=0.98\columnwidth]{figures/intro.jpg} 
% \caption{Illustration .} 
% \label{fig intro}
% \end{figure}

Recent years have witnessed a successful use of deep learning methods in the task of cover song identification (CSI), i.e., finding cover versions of a given music track in a music database. These methods generally formulate CSI as either a classification problem~\cite{yu2019temporal, yu2020learning, xu2018key} or a metric learning problem~\cite{yesiler2020accurate}, or a combination of both~\cite{du2021bytecover,du2022bytecover2,hu2022wideresnet}, and then train deep neural networks to learn low-dimensional features from different representations of audio. The features are then indexed and retrieved, where the distances between features are used to measure the similarities of songs. Deep learning models have proved their capability in learning discriminative and robust features, boosting the accuracy of CSI by a large margin compared to traditional methods based on handcrafted features.

% different audio representations such as constant-Q transform (CQT)~\cite{brown1991calculation}, melodic line~\cite{doras2020combining} and the harmonic structure~\cite{yesiler2019datacos}

 % that are invariant to irrelevant changes in timbre, tempo, structure and lyrics among various music versions

Despite of the promising progress above, one challenge remains for CSI in real-world applications, which is that most existing methods only consider situations when the query and the database item are both full-length music tracks with a typical duration of several minutes. Nevertheless, in many real-world scenarios, the task is to identify short music queries which are, for example, tens of seconds long, against full-length database songs. For instance, massive short videos with less-than-one-minute length have been created and uploaded to short video platforms such as TikTok in the past years, where a large proportion of these videos are accompanied by a carefully-selected music track that may be a remixed or cover fragment of an original music. For copyright management and reporting purpose, platforms need to identify these cover fragments. Unfortunately, as shown in section~\ref{sec:exp}, most existing CSI systems failed in the experiments of identifying short queries.  
An explanation for such incapability is that current CSI methods are mostly equipped with a global pooling layer to aggregate the information from all time sections, and then generate a global embedding for each song as the CSI feature. However, when matching a short music clip to a full-length recording, there will be some irrelevant sections in the full-length recording that may create noises in the embedding, which pose a negative impact on the similarity measurement between features. \par
% To solve this problem, an intuitive idea is to extract local features from both queries and database items, and calculate the similarity between two songs based only on the local features that have been matched, thus avoiding the interference from irrelevant sections. This idea has been considered in a very few traditional CSI works~\cite{rafii2014audio, cai2016two}, which, however, all have their limitations. For example, although capable of dealing with short queries, \cite{cai2016two} can only identify live versions of songs from known artists. \cite{cai2016two} performed music segmentation first and used segment-wise features for similarity measurement, but its performance has a noticeable gap with respect to SOTA CSI methods. Moreover, there is no report in~\cite{cai2016two} about the adaption of the method to short queries.
To solve this problem, an intuitive idea is to extract local features from both queries and database items, and calculate the similarity between two songs based only on the local features that have high matching score, thus avoiding the interference from irrelevant sections. This idea has been considered in a few traditional CSI works~\cite{muller2005audio, casey2008analysis, grosche2012toward, rafii2014audio, cai2016two}. However, due to the limited discriminative capacity and robustness of handcrafted features, as well as the high complexity in performing efficient indexing and retrieval, traditional CSI methods are generally only applicable for small databases (e.g., the database used in \cite{grosche2012toward} contains only 2,484 recordings) or simplified scenarios (e.g., \cite{rafii2014audio} can only identify live versions of songs from known artists), while their performances for the general CSI task against large databases are poor (or have not been reported). In \cite{zalkow2021efficient}, Zalkow \emph{et al.} made the first attempt to employ deep learning for short-query CSI and used convolutional neural networks (CNNs) to compress the input features. However, the databases used in \cite{zalkow2021efficient} are still limited to thousands of audio recordings or even less, and its usability in real-world applications is unverified.



% , which, however, all have their limitations. For example, although capable of dealing with short queries, \cite{cai2016two} can only identify live versions of songs from known artists. \cite{cai2016two} performed music segmentation first and used segment-wise features for similarity measurement, but its performance has a noticeable gap with respect to SOTA CSI methods. Moreover, there is no report in~\cite{cai2016two} about the adaption of the method to short queries.

In this paper, we extend our previous works of ByteCover~\cite{du2021bytecover} and ByteCover2~\cite{du2022bytecover2}, and present the new version of our CSI system, \emph{ByteCover3}, to solve the problem of identifying short music queries against a industry-scale database of full-length recordings. Different from existing works that rely on global features~\cite{yesiler2020accurate,yu2020learning,doras2019cover,du2021bytecover,du2022bytecover2}, ByteCover3 is designed to learn a set of deep local embeddings (or features) from each audio and uses the matching of local embeddings to accomplish the identification of short queries against full songs. To optimize the matching of local features, we propose a new loss termed \emph{local alignment loss} (LAL) and apply it in our multi-loss paradigm first introduced in ByteCover \cite{du2021bytecover}. LAL constitutes one of the major contributions of ByteCover3, and by using the LAL loss, the performance of CSI can be significantly improved. Moreover, to improve the efficiency of feature matching, a two-stage feature retrieval pipeline, which consists of an approximate nearest neighbor (ANN) stage and a re-ranking stage, is also designed. This consists of our second contribution.



% . LAL is specifically designed to optimize the similarity measurement between local features of music tracks with different length, and by using LAL, the performance of CSI can be significantly improved. To optimize feature indexing and retrieval, we also build for ByteCover3 a two-stage feature retrieval pipeline, which consists an approximate nearest neighbor search stage and a re-ranking stage to further improves the efficiency and effectiveness of ByteCover3.


\vspace{-0.2cm}
\section{ByteCover3}
\label{sec:approach}
\vspace{-0.8em}
The overall architecture of \ourname{} is illustrated in Fig. \ref{fig:overall}. \ourname{} is inherited from ByteCover and ByteCover2 and adopts a multi-loss learning paradigm for CSI. One of the major settings that differs it from previous works lies in the use of local features. In this section, we first describe the extraction of local features and then introduce our main contributions, i.e., the LAL loss for local feature matching and the two-stage feature retrieval pipeline. 
\vspace{-1em}
\subsection{Local Feature Extraction}
To extract local features from each recording, we first resample the audio to $22,050$ Hz and split it into $N$ short chunks with a length of $20$ seconds and a overlap of $10$ seconds. For each chunk, we calculate a constant-Q transform (CQT) spectrogram with the number of bins per octave and the hop size set to $12$ and $512$ respectively, using Hann window as the window function. The CQT spectrograms are then downsampled with an averaging factor of $100$ along the time axis to reduce the computation cost. Therefore, the input audio is processed into a compressed $N$-chunk CQTs $\mathbf{S} \in \mathbb{R}^{N \times F \times T}$, where $N$ is the number of chunks, $F$ is the number of CQT bins ($84$ in our setting), and $T$ is the number of frames in each chunk.

The ResNet-IBN model \cite{du2021bytecover}, which replaces the residual connection blocks of ResNet50~\cite{he2016deep} with the instance-batch normalization (IBN) blocks, is then applied as the backbone to extract local embeddings from the input CQTs. In ByteCover3, our ResNet-IBN follows the original ByteCover setting, except that a 3-D input $\mathbf{S} \in \mathbb{R}^{N \times F \times T}$ is taken instead of the original 2-D input. The output of ResNet-IBN before the global generalized mean (GeM) pooling layer is hence a 4-D embedding ${\mathbf{Z}} \in \mathbb{R}^{N \times C \times H \times W}$, where $C$ is the number of output channels, $H$ and $W$ are the spatial sizes along the frequency and time axes respectively. In practice, we set $C = 2048$, $H = 6$ and $W=\nicefrac{T}{8}$. Finally, the temporal and frequency axes of $\mathbf{Z}$ are integrated by the GeM pooling operation and a dimensionality reduction module, i.e., PCA-FC \cite{du2022bytecover2}, is utilized on the channel dimension to obtain the compacted final embedding ${\mathbf{X}} \in \mathbb{R}^{N\times 512}$, which contains $N$ local features from the original audio, as opposed to ByteCover and ByteCover2 that only adopted a single global embedding.
\vspace{-1em}
\subsection{The Local Alignment Loss}
Existing CSI methods generally employ either a classification loss (e.g., softmax loss) or a metric learning loss (e.g., triplet loss) or a combination of them as the optimization objective during training. Nevertheless, as these methods only rely on a single global vector for each music track, their loss functions are limited to measuring similarities between two vectors (e.g. dot product or cosine similarity). Whereas in our case, we wish to compare two sequences of vectors that contain different number of local features, which requires a new similarity measure. To address this problem, we propose a novel loss design called Local Alignment Loss (LAL) $\mathcal{L}_{\operatorname{lal}}$ consisting of a classification loss $\mathcal{L}_{\operatorname{lac}}$ and a triplet loss $\mathcal{L}_{\operatorname{lat}}$.


% two loss objectives for training: the classification approach that classifies different versions of a music into the same class, and the metric learning approach that minimizes the distance between versions while maximizing the distance between different songs. The previous ByteCover~\cite{du2021bytecover} took a multi-loss training approach that combined both classification and triplet loss, which significantly improved the performance in CSI. 

We first introduce a similarity measure termed \emph{MaxMean} inspired by~\cite{cai2016two}: let $\mathbf{X} \in \mathbb{R}^{M\times C}$ and $\mathbf{Y} \in \mathbb{R}^{N\times C}$ denote two $C$-dimensional feature sequences that each contain $M$ and $N$ local features ($M$ and $N$ could be highly different). For each local feature $\{\mathbf{x}_i\}_{i=1}^M \in \mathbb{R}^{1\times C}$ in $\mathbf{X}$, we calculate the cosine similarity between $\mathbf{x}_i$ and all the local features $\{\mathbf{y}_i\}_{j=1}^N \in \mathbb{R}^{1\times C}$ in $\mathbf{Y}$, and regard the maximal value as the similarity measure $s_i$ for $\mathbf{x}_i$:
\begin{equation}
s_i = \max(\cos(\mathbf{x}_i, \mathbf{y}_j)), ~j=1,\dots,N,
\end{equation}
and the final similarity score is obtained by taking average over all the similarity measures, i.e., $\operatorname{MaxMean}(\mathbf{X}, \mathbf{Y})=\frac{1}{M}\sum_{i=1}^M s_i$. The shorter local feature is always regraded as the first operand, because the \textit{MaxMean} operator is non-commutative. Since only the maximal value of all the matching scores from $\mathbf{x}_i$ to $\mathbf{Y}$ is considered, we can avoid the distractions of local features in $\mathbf{Y}$ that are irrelevant to $\mathbf{x}_i$. 

With the \emph{MaxMean} measure described above, we then illustrate how the original classification loss in previous ByteCover~\cite{du2021bytecover} is transformed to the novel LAL. Recall in ByteCover, the classification loss $\mathcal{L}_{cls}$ is defined as:
\begin{align}
\mathcal{L}_{cls} = \operatorname{CE}(\sigma(\mathbf{W}\mathbf{f}^\mathrm{T}),~ y) = \operatorname{CE}(\sigma(\{\mathbf{w}_k\mathbf{f}^\mathrm{T}\}_{k=1}^{K}),~ y),
\end{align}
where $\operatorname{CE}(\cdot,\cdot)$ is the cross entropy and $\sigma(\cdot)$ is the softmax function. We denote $y$ as the ground-truth label, $\mathbf{f} \in \mathbb{R}^{1\times C}$ as the global feature extracted from ResNet-IBN, and $\mathbf{W}\in\mathbb{R}^{K\times C}$ as the weight matrix in the linear layer before softmax that contains $K$ weight vectors $\{{\bf w}_k\}_{k=1}^K$ for classification.

To adapt $\mathcal{L}_{cls}$ to the novel \emph{MaxMean} measure with the local features, we draw inspiration from \cite{sun2020circle} and consider $\mathbf{w}_k$ as a proxy feature representation of the $k^\mathrm{th}$ class. In this sense, the result of $\mathbf{w}_k\mathbf{f}^\mathrm{T}$ can be interpreted as the similarity score between two features $\mathbf{w}_k$ and $\mathbf{f}$ based on dot product, which we argue can be replaced by the \emph{MaxMean} metric. Specifically, our new local alignment classification loss $\mathcal{L}_{lac}$ is written as:
\begin{align}
\mathcal{L}_{lac}  & = \operatorname{CE}(\sigma(\{\operatorname{logit}_k\}_{k=1}^K), y),\\
\operatorname{logit}_k  & = \operatorname{MaxMean}(\mathbf{X}, \mathbf{W}_k),
\end{align}
where ${\bf X} \in \mathbb{R}^{N \times C}$ is the final embedding with $N$ local features extracted by ResNet-IBN, $\mathbf{W} \in \mathbb{R}^{K \times L \times C}$ is a trainable weight matrix in the linear layer before softmax and ${\bf W}_k \in \mathbb{R}^{L \times C}$ denotes the proxy representation for class $k$.

In addition to the classification loss, a triplet loss was also used in ByteCover, which is simply modified in ByteCover3 by replacing the Euclidean distance with \emph{MaxMean} metric:
\begin{equation}
\mathcal{L}_{lat} = [\operatorname{MaxMean}({\bf X}_n,{\bf X}) - \operatorname{MaxMean}({\bf X}_p,{\bf X})]_{+}.
\end{equation}
Finally, our overall loss $\mathcal{L}_{lal}$ is given by $\mathcal{L}_{lal} = \mathcal{L}_{lac} + \mathcal{L}_{lat}$.
\vspace{-0.4em}
\subsection{Two-Stage Feature Retrieval}
\vspace{-0.2em}
An efficient feature retrieval pipeline is also critical for constructing a practical industrial-strength CSI system. Previous methods usually use an all-pairs strategy that includes computing the similarity between the query sample and each item in the database, which is time consuming. Moreover, there is a significant leap in complexity from the vector similarity measure ($\mathcal{O}(1)$) to the local alignment measure \textit{MaxMean} ($\mathcal{O}(n^2)$)~\cite{cai2016two}, which makes the all-pairs strategy even worse for ByteCover3. To solve this problem, we propose a two-stage pipeline with a hierarchical searching strategy for the retrieval of deep local embeddings. 

Given a query sample with $M$ local features, the first stage is to eliminate the database recordings that are highly unlikely to be a match. Specifically, for each local feature in the query, we search for its Top-$K$ nearest neighbors in the gallery of local features extracted in advance from all the database recordings, using the hierarchical navigable small world (HNSW) graphs \cite{malkov2018efficient}, with $K$ set to $50$. This results in a candidate set of $M\times K$ local matches for the given query, based on which our second stage of feature retrieval is further performed. Suppose that the $M \times K$ local matches originate from $D$ database recordings ($D \leq M\times K$ since some local matches may original from the same recordings), and thus our second stage is to compare the given query with each of the $D$ candidate recordings, based on the \emph{MaxMean} measure introduced above. The candidate recordings with the highest \emph{MaxMean} similarities are finally outputted as the retrieval results.

In practical use, the query is typically less than 60s, and thus we have $M \leq 5$ as our local features are extracted every 20s with overlap of 10 seconds. Therefore, in the second stage we only need to calculate the \emph{MaxMean} similarity for $M \times K \leq 250$ times, which is significantly less then the calculation needed in the all-pairs strategy.

% The first stage of recalling is conducted by vector similarity search that aims to shrink the size of candidates sets. In detail, every vector in the local embedding sequence with length $n$ of query sample searches for its Top-$K$ nearest vector in the database. These $n \times K$ results yield a candidate set which has $n \times K$ recording at the most, because the different results may belong to the same recording. In practice, the $K$ is set to $50$ and the $n$ ranges from $1$~$6$ while the duration query is shorter than 1 minute.  Moreover, the implementation of vector similarity search can be accelerated dramatically with ANN search methods, as the error of Top-$K$ caused by ANN decreases when $K$ increases from $1$ ~\cite{malkov2018efficient}. Thus, the size of candidate set is shrunk from millions to $50~300$ by the first recall layer. Further, the second layer conducts fine-grained ranking under MinMax measure between the query sample and other samples in this compact candidate set to get the final most matched recording. As shown in Tab. \ref{tab:time}, by incorporating the two-layer searching pipeline and ANN method, the new ByteCover3 system with local embeddings achieves a similar speed with the previous ByteCover2 system that uses global embedding.

\section{Experiments}
\label{sec:exp}
\begin{figure*}
\begin{center}
\includegraphics[width=.95\linewidth]{img/novelview.pdf}
\end{center}
\vspace{-0.4cm}
   \caption{\textbf{Qualitative comparison with baseline methods on Novel View Synthesis. } We compare our novel view synthesis on transparent objects with the methods that we identify as most relevant to ours, NeRF~\cite{mildenhall2020nerf}, Eikonal Field~\cite{bemana2022eikonal}, IDR~\cite{yariv2020multiview}, and PhySG~\cite{zhang2021physg}. Our method outperforms the others on the high-frequency details caused by ray refraction. }
   \vspace{-0.4cm}
\label{fig:novelviewsyn}
\end{figure*}

\subsection{Synthetic Data Evaluation}
\vspace{-0.2cm}
\medskip
\noindent\textbf{Datasets. }We use the 4 mesh objects of kitty, cow, bear, and key-mouse from~\cite{Xing2022drot, zhang2021physg}, and render each object with the smooth dielectric BSDF model with Mitsuba 3 \cite{jakob2022mitsuba3} under an environmental light emitter. For synthetic dataset evaluations we set interior IOR to 1.4723 for glass and exterior IOR to 1.00028 for air. We also create datasets with interior IOR set to 1.2, and 2.4 for ablation studies. We uniformly sample 200 camera poses on the upper hemisphere around each object following the Fibonacci lattice and randomly assign 100 each for training and testing. We obtain object masks through data pre-processing~\cite{remove_bg}.    

\vspace{-0.2cm}
\medskip
\noindent\textbf{Baseline. }As discussed in Sec.~\ref{sec:related} and shown in Tab.~\ref{tab:baseline}, no other work studies the same problem as ours, i.e., modeling refraction for transparent objects with complex geometry by neural networks for novel view and relighting synthesis. We, therefore, classify our baselines based on different tasks: \textbf{NeRF}~\cite{mildenhall2020nerf} and \textbf{Eikonal Fields}~\cite{bemana2022eikonal} on novel view synthesis; \textbf{IDR}~\cite{yariv2020multiview} on novel view synthesis and geometry reconstruction; \textbf{PhySG}~\cite{zhang2021physg} on novel view and relighting synthesis, and geometry reconstruction.  As geometry is not our aimed task to improve, extracted mesh quality is only to show that RBN effectively disentangles geometry and ray refraction on appearance. We do not include comparisons with volume-based neural relighting methods~\cite{boss2021nerd, verbin2022refnerf} as they share the same appearance model with PhySG.

\begin{figure}
\begin{center}
\includegraphics[width=0.85\linewidth]{img/relight-cow.pdf}
\end{center}
\vspace{-0.4cm}
   \caption{\textbf{Qualitative results on Relighting for synthetic datasets. } We show that our network can faithfully relight the object with unseen environment illumination, unlike PhySG~\cite{zhang2021physg}. }
\label{fig:relightsyn}
\vspace{-0.3cm}
\end{figure}

\vspace{-0.2cm}
\medskip
\noindent\textbf{Novel View Synthesis.} 
A qualitative comparison of our method and baseline methods is shown in Fig.~\ref{fig:novelviewsyn}. NeRF and Eikonal Fields model object appearance as MLP-based volume and cannot distill radiance properly around the object surface. However, when modeling refractive objects with complex geometry, it is important to locate the surface for accurate refraction direction estimation. Eikonal fields relies on user-defined bounding boxes, resulting in failure cases where the opaque scene and the refractive part cannot be separated. Meanwhile, IDR and PhySG are surface-based methods, but IDR models appearance as a light field~\cite{wood2000surface} and cannot correctly interpolate the high-frequency change of the refracted environment illumination on object appearance. PhySG uses Disney BSDF~\cite{burley2012physically} which does not work for non-opaque objects and therefore fails to correctly disentangle geometry and appearance. 

We report quantitative evaluation on novel view synthesis with metrics including PSNR, SSIM, and LPIPS~\cite{zhang2018perceptual} through testing on held-out images in Tab.~\ref{tab:quanti}. Our method significantly outperforms all of our baselines on synthesizing novel views for accurately modeling the refraction direction of each ray intersected with geometry. 

% \vspace{-0.2cm}
\medskip
\noindent\textbf{Relight Synthesis.} 
We provide a qualitative comparison of relighting synthesis in Fig.~\ref{fig:relightsyn}. As the environment map used during training is natural and unstructured unlike in prior works~\cite{lyu2020differentiable, wu2018full}, many pixels share similar radiance, but our learned refractions are not overfitted on the training illumination; they are aligned with the true refractions. We relight each scene with an unseen environment map to test the correctness of the object refraction. PhySG fails on this task as it does not model refractive material, resulting in incorrect appearance decomposition~\cite{burley2012physically}.  We report quantitative evaluation w.r.t ground truth relighting in Tab.~\ref{tab:quanti}. 



\begin{table}
  \centering
  \scalebox{0.9}{
  \begin{tabular}{@{}lcccc@{}}
    \toprule
    Synthetic & \multicolumn{4}{c}{$\downarrow$Chamfer $L_1(10^{-3})$} \\
    \midrule
    Method & Kitty & Bear & Key Mouse & Cow\\
    \cmidrule{2-5}
    IDR \cite{yariv2020multiview}& 4.30 & 3.66 & 3.70 & 11.66 \\
    PhySG \cite{zhang2021physg} & 87.67 & 67.43 & 31.61 & 52.17 \\
    \textbf{NEMTO} & \textbf{2.22} & \textbf{1.71} & \textbf{2.27} & \textbf{2.60} \\
    \bottomrule
    \vspace{-0.2cm}
  \end{tabular}
    }
  \caption{\textbf{Quantitative evaluation on recovered meshes of synthetic datasets.} We report the chamfer distance metric~\cite{cdcode} on g.t. mesh versus extracted meshes as a quantitative measure for reconstructed geometry quality. NEMTO achieves better results than baseline methods that models object surfaces. }
  \label{tab:chamfersyn}
  \vspace{-0.4cm}
\end{table}


\vspace{-0.2cm}
\medskip
\noindent\textbf{Disentanglement on Geometry and Appearance.}
We evaluate our extracted geometry on synthetic datasets with ground truth mesh through the Chamfer distance metric and compare our geometry with those of surfaced-based methods. In Fig~\ref{fig:relightsyn}, the geometry of PhySG is entangled with surface appearance, i.e. the appearance under the original illumination is imprinted on the surface and raised geometry. Tab.~\ref{tab:chamfersyn} shows that IDR does better than PhySG, though still worse than ours. Our geometry and refracted appearance are better separated due to our modeling of ray refraction and optimizations. 

\begin{figure}[t]
\begin{center}
   \includegraphics[width=0.8\linewidth]{img/ablation-ior.pdf}
\end{center}
   \caption{\textbf{Experiments on different transparent media. }We show that NEMTO works for different transparent media other than glass. The learned $\eta_\mathbf{t}$ is adaptive to different media and allows our model to synthesize faithful results.}
\label{fig:ablation-ior}
\vspace{-0.4cm}
\end{figure}



\vspace{-0.2cm}
\medskip
\noindent\textbf{Robustness to different refractive indices.}
We conducted experiments on transparent objects rendered with various IORs to showcase the robustness of our framework to IOR changes. Our approach is suitable for different types of refractive materials, as demonstrated in Fig.~\ref{fig:ablation-ior}. Note that our predicted $\eta_\mathbf{t}$ for the blending of ray refraction and reflection is also adaptive to different IOR, as shown in the case for IOR = 2.4, the reflected radiance is adequately brighter than in IOR = 1.4723 and 1.2. 

\begin{table}
\scalebox{0.75}{
  \centering
  \begin{tabular}{@{}lcccccc@{}}
  
    \toprule
     & \multicolumn{3}{c}{ Novel View } 
     & \multicolumn{3}{c}{Relighting} \\
     
    \cmidrule(lr){2-4}\cmidrule(lr){5-7}
    
    Method 
    & PSNR $\uparrow$ & SSIM $\uparrow$ & LPIPS $\downarrow$ 
    & PSNR $\uparrow$ & SSIM $\uparrow$ & LPIPS $\downarrow$ \\
    
    \cmidrule(lr){2-4}\cmidrule(lr){5-7}
    
    NeRF~\cite{mildenhall2020nerf} 
    & 21.274 & 0.837 & 0.171
    & - & - & - 
    \\ 

    Eikonal~\cite{bemana2022eikonal} 
    & 15.866 & 0.452 & 0.589
    & - & - & - 
    \\ 
    
    
    IDR~\cite{yariv2020multiview} 
    & 22.695 & 0.851 & 0.152
    & - & - & - 
    \\ 
    
    PhySG~\cite{zhang2021physg}
    & 19.981 & 0.791 & 0.203
    & 15.412 & 0.749 & 0.237
    \\ 
        
    \midrule
    
    SDF-A 
    & 21.758 & 0.828 & 0.145
    & 17.846 & 0.787 & 0.192
    \\ 

    w/o $\mathcal{L}_{\textrm{rg}}$
    & 15.659 & 0.746 & 0.221
    & 14.585 & 0.713 & 0.238
    \\
    
    w/o $\mathcal{L}_{\textrm{rs}}$
    & 21.623 & 0.811 & 0.163
    & 19.026 & 0.823 & 0.149
    \\
    
    \textbf{NEMTO} 
    & \textbf{26.582} & \textbf{0.924} & \textbf{0.083}
    & \textbf{25.147} & \textbf{0.918} & \textbf{0.098}
    \\
    
    \bottomrule
  \end{tabular}
  }
  \vspace{0.05cm}
  \caption{\textbf{Quantitative Evaluations. } We present the average result on all synthetic datasets. The first three methods are not capable of relighting. Our method performs significantly better on both novel view and relighting synthesis than all of our baseline methods and ablation experiments. }
  \label{tab:quanti}
  \vspace{-0.4cm}
\end{table}



\begin{figure}[t]
\begin{center}
   \includegraphics[width=0.9\linewidth]{img/sdf-a.pdf}
\end{center}
\vspace{-0.3cm}
   \caption{\textbf{Qualitative ablation on SDF-A}. SDF-A shows that jointly optimizing refraction and geometry is prone to error. Our approach performs significantly better than this naive approach.}
\label{fig:ablation-sdfa}
\vspace{-0.4cm}
\end{figure} 




\begin{figure*}
\begin{center}
\includegraphics[width=.95\linewidth]{img/rw.pdf}
\end{center}
   \caption{\textbf{Qualitative results on image synthesis and extracted geometry for real-world data. } We compare our extracted geometry, novel view synthesis, and relighting with the extracted geometry and rendering layer of TLG~\cite{li2020through}, which restricts the light bounce within transparent media to only two bounces. }
\label{fig:relightrw}
\vspace{-0.5cm}
\end{figure*}


\begin{figure}[t]
\begin{center}
   \includegraphics[width=\linewidth]{img/ablation-loss.pdf}
\end{center}
\vspace{-0.4cm}
   \caption{\textbf{Ablation on losses for ray refraction optimizations.} Each experiment is trained with a frozen geometry network to demonstrate the effect of each loss term on ray bending. }
\label{fig:ablation-loss}
\vspace{-0.4cm}
\end{figure}

\vspace{-0.2cm}
\subsection{Ablation Studies}
\vspace{-0.2cm}
We perform the following ablation studies to demonstrate the effectiveness of Our RBN, $\mathcal{L}_{\textrm{rg}}$, and $\mathcal{L}_{\textrm{rs}}$. 

\vspace{-0.2cm}
\medskip
\noindent\textbf{Ablation on ray bending network.} We implemented a naive version of our method~\textbf{SDF-A} without using RBN. It renders transparent objects with \textit{analytical} refraction to demonstrates the effectiveness of our RBN and neural environment matting method over the use of a physically-based differentiable renderer on transparent objects. As shown in Fig.~\ref{fig:ablation-sdfa}, our method synthesizes more accurate results when jointly optimizing for geometry and light refraction, which are better disentangled. This is evident from the smoother surfaces of our method due to $\mathcal{L}_{\textrm{rs}}$. NEMTO estimated smoother surface normal than SDF-A, and gives much more faithful ray refractions.%SDF-A shows that our method is able to disentangle geometry and appearance during joint optimization of the networks.

\vspace{-0.2cm}
\medskip
\noindent\textbf{Ablation on $\mathcal{L}_{\textrm{rg}}$ and $\mathcal{L}_{\textrm{rs}}$.} For experiments on the refraction guiding and refraction smoothness loss, we fix the optimized geometry and only show different optimization results for refraction prediction. The lower part of Tab.~\ref{tab:quanti} shows quantitative evaluation that our complete architecture performs better than without each of these two loss terms. Fig.~\ref{fig:ablation-loss} compares the learned refraction from each ablation experiment: in column (b) without $\mathcal{L}_{\textrm{rg}}$, the model cannot learn the correct direction; in column (c) without $\mathcal{L}_{\textrm{rs}}$, the optimized ray refraction is around the true scope but shows wrong wave-patterned artifacts.

\vspace{-0.2cm}
\subsection{Real World Data Results}
\noindent\textbf{Datasets.} For real-world evaluation, we use 4 sets of captured images and environment maps on dog, monkey, pig, and mouse shapes from TLG ~\cite{li2020through}. As TLG only provides 10-12 images for each real-world object, we render training data with the ground truth CT-scanned meshes following the steps of synthetic datasets generation detailed in the supplementary. We do evaluations on released real-world images. 

\vspace{-0.3cm}
\medskip
\noindent\textbf{Edited Scene Synthesis. }Fig.~\ref{fig:relightrw} shows our synthesis for real-world images. Our method is able to predict accurate ray refraction for transparent objects and produces a smoother surface normal prediction on geometry extraction than TLG. TLG designed a novel differentiable rendering layer for physically-based transparent object modeling, but it only renders up to two bounces of refraction, whereas our method does not pose an upper bound on the number of ray bounces. Moreover, TLG does not work with an unknown IOR for transparent objects. Note that, although TLG claims to require only 10-12 images for testing, it requires rendering a large-scale synthetic dataset with 1.5k HDR (High Dynamic Range) environment maps for training, which is unnecessary in our case.% However, the generalization ability of TLG remains an isse.
\vspace{-0.2cm}


\vspace{-1.0em}
\section{Conclusion}
\vspace{-0.5em}
In this paper, we propose to combine local feature matching and two-stage feature retrieval for efficient CSI of short music queries. A new loss termed LAL is designed to optimize the similarity measurement between songs with different length. Experimental results show that \ourname\ outperforms all benchmark models on three synthetic datasets for short-query CSI, while being highly efficient in local embedding extraction and hierarchical retrieval. For future work, we are currently studying to apply \ourname{} to other real-world applications such as set list identification, music matching with accurate timestamp and humming recognition.




% Furthermore, the ByteCover2 provides two very interesting insights. First, initializing the fully connected layer with the PCA-transformation matrix may encourage the model to learn less redundant vector representation of music more efficiently. Second, for modern large-scale CSI systems, vector retrieval tends to be more time-consuming than model inference. Dimensionality reduction will largely reduce this type of latency of CSI system.
%In this paper, we have proposed a simple and efficient design of the cover song identification system. With one network for representation extraction and one cosine distance metric for retrieval, our \ourname\ system outperforms all previous state-of-the-art cover detection methods by a significant margin on four public benchmark datasets.  
%The combination of classification and metric learning training schemes encourages our model to learn a discriminative feature embedding. The instance-batch normalization module's utilization leads the \ourname\ model to be robust against the musical variations. \par
%As future work, the interpretability of our model is needed to improve further. It is worthwhile to study how this powerful system identifies covers and learn transposition, tempo, timing, or structure-invariance.

\vfill\pagebreak



% References should be produced using the bibtex program from suitable
% BiBTeX files (here: strings, refs, manuals). The IEEEbib.bst bibliography
% style file from IEEE produces unsorted bibliography list.
% -------------------------------------------------------------------------
\bibliographystyle{IEEEbib}
\bibliography{6_refs}

\end{document}
