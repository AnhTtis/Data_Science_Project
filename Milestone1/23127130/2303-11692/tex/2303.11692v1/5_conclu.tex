\vspace{-1.0em}
\section{Conclusion}
\vspace{-0.5em}
In this paper, we propose to combine local feature matching and two-stage feature retrieval for efficient CSI of short music queries. A new loss termed LAL is designed to optimize the similarity measurement between songs with different length. Experimental results show that \ourname\ outperforms all benchmark models on three synthetic datasets for short-query CSI, while being highly efficient in local embedding extraction and hierarchical retrieval. For future work, we are currently studying to apply \ourname{} to other real-world applications such as set list identification, music matching with accurate timestamp and humming recognition.




% Furthermore, the ByteCover2 provides two very interesting insights. First, initializing the fully connected layer with the PCA-transformation matrix may encourage the model to learn less redundant vector representation of music more efficiently. Second, for modern large-scale CSI systems, vector retrieval tends to be more time-consuming than model inference. Dimensionality reduction will largely reduce this type of latency of CSI system.
%In this paper, we have proposed a simple and efficient design of the cover song identification system. With one network for representation extraction and one cosine distance metric for retrieval, our \ourname\ system outperforms all previous state-of-the-art cover detection methods by a significant margin on four public benchmark datasets.  
%The combination of classification and metric learning training schemes encourages our model to learn a discriminative feature embedding. The instance-batch normalization module's utilization leads the \ourname\ model to be robust against the musical variations. \par
%As future work, the interpretability of our model is needed to improve further. It is worthwhile to study how this powerful system identifies covers and learn transposition, tempo, timing, or structure-invariance.
