\label{sec:introduction}

% Introduction
On large scales, and at early times, matter fluctuations are small and can be described using linear perturbation theory; the evolution of small perturbations can be solved analytically.  Once fluctuations become more developed, however, their properties can no longer be explained by linearised equations, and instead a full non-linear treatment is needed.  The halo model provides an intuitive way to approximate the matter distribution in the non-linear regime.  It posits that all matter resides in haloes, which are $\mathcal{O}(100)$ times denser than the cosmological average -- a view that has been largely corroborated by numerical simulations.  Once the properties and distribution of these haloes are known, one can estimate the statistical properties of the matter distribution in the cosmos.  To be concrete,  power spectra for matter and its tracers can be understood as the sum of two components: inter-halo (two-halo) and intra-halo (one-halo) clustering.  The halo model, therefore, can (and has) been used in analysing cosmological data from various probes of large-scale structure.  

%tracers
The statistical properties of any tracer of matter can also be modelled,  provided that the connection between the tracer and host haloes is known.  If haloes are taken to be the sites of galaxy formation, all that is needed to model the galaxy clustering signal is how galaxies occupy haloes of different masses. The problem then can be split into how galaxies cluster within the same halo and how different haloes, which might include varying numbers of galaxies, cluster with respect to each other. In principle, the same logic can be applied to \emph{any} tracer of the large-scale structures,  as long as the signal from the tracer emanates from haloes.  For example,  the thermal Sunyaev-Zel'dovich (tSZ) effect is sourced by electron pressure,  which is at its most intense within haloes, and so reasonable models for tSZ clustering, and its cross-correlation with other tracers, may be derived using the halo model.  

The halo properties that are required to make a prediction using the halo model are the halo bias (how haloes cluster relative to matter), halo mass function (number density of haloes with different masses), and halo profile (how matter or its tracers are distributed within a halo). These ingredients are most often extracted from numerical simulations and (sometimes) calibrated across a range of cosmological parameters.  It is usual to assume that haloes are linearly biased,  spherical objects with properties that are only a function of the halo mass,  although these restrictions can be relaxed.  We call this method of using the halo model the ``analytical approach".

There is a second approach to using the halo model, the ``simulation-based approach":  Here, haloes are identified in a simulation and then `painted' with a specific tracer (\eg galaxies), such that the desired clustering properties can be directly measured.  While the analytical approach is quicker and more flexible, the simulation-based approach is potentially more accurate,  but it is slower and requires \nbody simulations.  This can become a problem in cosmological analyses where a wide range of  parameters and/or cosmological models need to be explored.  Once the analytical approach is adjusted to reach a desired accuracy,  it can be extended more readily to other cosmological parameters and/or models compared to the simulation based approach.

The halo model has been used in one form or another to analyse data from weak gravitational lensing by large-scale structure,  known as cosmic shear.  This is because cosmic shear relies on information from non-linear matter distribution.  \halofit \citep{Smith2003,Takahashi2012} and \hmcode \citep{Mead2015a,Mead2021a} which have their roots in the halo-model approach (see section~\ref{sec:modelling_matter}) form the basis of all primary analysis of recent cosmic shear data: Canada France Hawaii Telescope lensing Survey \citep[CFHTLenS,][]{Heymans2013,Joudaki2017},  Deep Lens Survey \citep[DLS,][]{Jee2013},  Kilo Degree Survey  \citep[KiDS,][]{Hildebrandt2017, Hildebrandt2020, Asgari2021},  Dark Energy Survey \citep[DES,][]{Troxel2018,Amon2022,Secco2022} and Hyper Suprime-Cam \citep[HSC,][]{Hikage2019,Hamana2020}.

Data from galaxy clustering and the cross-correlation between weak lensing and galaxy clustering,  known as galaxy--galaxy lensing,  have been analysed with a flexible halo model approach to capture information from smaller scales \citep[for example][]{Cacciato2013, More2015, Miyatake2022b, Dvornik2022}.  \cite{Troester2022} applied the halo model formalism of \cite{Mead2020} to the cross-correlation between tSZ and weak lensing in a cosmological analysis.  The halo model can also predict the intrinsic alignments of galaxies,  which is a prominent astrophysical systematic in cosmic-shear studies \citep{Schneider2010,Fortuna2021}. 

While data from modern \emph{galaxy} surveys is most often modelled using a halo model applied to, or directly calibrated against simulations,  most analyses concerning cross-correlations of different tracers use the 'vanilla' halo model as a first modelling effort \citep[\eg][]{Komatsu2002, Hill2014, Hurier2014, Battaglia2015, Troester2017, Feng2017, Osato2018, Wolz2019, Koukoufilippas2020, Yan2021}.   The limitations of the analytical halo-model approach are appreciated and accounted for when it comes to modelling the galaxy and matter cross-correlation (galaxy--galaxy lensing), but less well appreciated for other cross correlations where the modelling and data are less mature.  As we will discuss later, these limitations are a strong function of exactly what one is trying to model, particularly of the relationship between tracer strength and halo mass. For these reasons, we feel that a pedagogical review is useful and timely.

% Uses of the halo model: Galaxy and halo connection and all other none cosmological parameter uses
Extracting information about cosmological parameters and galaxy-occupation statistics via the halo model has been historically challenging. The simplicity of the analytical halo model, which is appealing from the perspective of understanding, can become a problem as the data to which it is exposed increases in quality. The halo model should be calibrated against simulations to check its accuracy, and a failure to do so may result in incorrect parameter constraints where real signal is mistaken for some modelling deficiency. There have been many attempts to use the halo model to constrain cosmological and galaxy--halo parameters \citep[\eg][]{Tinker2005, vandenBosch2013, More2013, Cacciato2013, More2015a, Leauthaud2017,Zacharegkas2022}. Recent attempts to include non-linear halo bias \citep[][]{Nishimichi2019, Mead2021b} within the halo model provide a promising way to improve the accuracy of the halo model to the extent that it can be trusted to recover cosmological parameters and information about the galaxy--halo connection from current \citep[\eg][]{Mahony2022,Miyatake2022b, Dvornik2022} and forthcoming survey data.  Another powerful property of the halo model is that its ingredients can be directly calibrated against real observations,  by allowing for parameters to vary freely in likelihood analyses \citep[see for example][]{Gu2023}. 

% Reason for this review
The halo model first came to cosmological prominence at the beginning of the millennium \citep{Seljak2000, Ma2000b, Peacock2000} and, after a flurry of related publications, was first reviewed by \cite{Cooray2002}. As far as we know, there has been no subsequent attempt to review the halo model. Here, we attempt to provide a pedagogical summary of the modern uses of the halo model with a focus on its cosmological applications and the modelling of the auto and cross power spectra of different fields.

% Structure of this review
The structure of this review is as follows: In Section~\ref{sec:basics} we provide a comprehensive derivation of the halo-model power spectrum in order to highlight the assumptions lurking behind the model. In Section~\ref{sec:ingredients} we discuss the ingredients that are necessary to make the halo model predictive in the case of the matter distribution and how it is modelled in practice, while in Section~\ref{sec:tracers} we see how the halo model can be extended for tracers of matter, such as galaxies. In Section~\ref{sec:non_standard} we discuss some non-standard approaches and improvements to halo modelling that have appeared over the years. In Section~\ref{sec:altcosmo} we discuss applications to cosmologies beyond the standard \LCDM. In Section~\ref{sec:software} we detail the publicly-available software for performing halo-model calculations, and finally we summarise in Section~\ref{sec:summary}.