\label{sec:tracers}

So far, we have discussed the application of the halo model for calculating the power spectrum of matter and galaxies, but we noted in Section~\ref{sec:derivation} that our initial derivation was applicable to \emph{any} diffuse tracer of large-scale structure whose halo profile can be specified. In this Section we review work where the halo model has been used to calculate these other spectra. Most authors simply replace the halo profiles in equations~(\ref{eq:one_halo_term}) and (\ref{eq:two_halo_term}; ignoring $\Bnl$) with those relevant for the new tracer. There is no need to specify new mass functions or halo-bias relations since, in a model where all signal originates from haloes, this is already included self consistently. Generally, the spectrum of the new tracer will have the linear shape at large scales, with an amplitude determined by the tracer occupation statistics. At smaller scales the shape of the one-halo term will be governed by the shape of the tracer profiles. Using the halo model in this way might be accurate, but the accuracy should be assessed on a case-by-case basis and ideally should be confirmed by comparing the results of calculations to measurements from simulations. Some of the additions that will be discussed in Section~\ref{sec:non_standard} may be more or less important for spectra of different tracers. Finally, we note that any significant contribution to a signal that is genuinely diffuse, such that it cannot be tied to a halo, is difficult to include self consistently within a model where all signal originates from haloes (although see Section~\ref{sec:smooth}).

\subsection{Galaxies}
\label{sec:HOD}

% Introduction
We touched on modelling galaxy power spectra in Section~\ref{sec:discrete_tracers}.  Here we go into more detail and explore the relation between halo masses and the distribution of galaxies within those haloes, the so-called halo occupation distribution (HOD).  Note that, if a stochastic relationship between galaxy occupation and halo mass is assumed then properties of the statistical distribution of galaxies also need to be specified.

% Central galaxies
A commonly-used HOD is the five-parameter model of \cite{Zheng2005}, who measured the relation between haloes and galaxies from a smoothed particle hydrodynamics simulation and a semi-analytic galaxy formation model.  They find that the mean number of central galaxies given a halo mass, $M$, can be well described by
%
\begin{equation}
\average{N_\mathrm{c}(M)} = \frac{1}{2}
\left[1+\mathrm{erf}\left(\frac{\log_{10}(M/M_\mathrm{min})}{\sigma_{\log_{10}M}}\right)\right]\ ,
\label{eq:HOD_Nc}
\end{equation}
%
where $\mathrm{erf}(x)$,  is the error function. 
Haloes can never host more than a single central galaxy, as enforced by the error function ranging from -1 to 1.  
$M_{\rm min}$ is the characteristic minimum halo mass, which means that for $M\ll M_\mathrm{min}$ a halo is unlikely to host a central galaxy while for $M\gg M_\mathrm{min}$ haloes are almost certain to host a single central galaxy, 
with the width of the transition around $M_\mathrm{min}$ governed by $\sigma_{\log_{10}M}$. 
Any random process whose outcome can only be zero or one is governed by Bernoulli statistics; the statistical properties of this distribution are given in Table~\ref{tab:HOD}.

% Satellite galaxies
In the  \cite{Zheng2005} model the mean number of satellites galaxies is
%
\begin{equation}
\average{N_\mathrm{s}(M)} = \Theta(M-M_0)\left(\frac{M-M_0}{M_1}\right)^\alpha\ ,
\label{eq:HOD_Ns}
\end{equation}
%
where $M_0$ is the truncation mass, below which a halo is not expected to host any satellites ($ \Theta(M-M_0)=1$ for $M>M_  0$ and 0 otherwise).
Haloes with $M=M_0+M_1$ host a single satellite galaxy on average. If $\alpha=1$ then the number of satellite galaxies scales linearly with halo mass. Satellite galaxies are often assumed to follow Poisson statistics\footnote{Although, as we will discuss below, this condition usually does not hold.}; the statistical properties of this distribution are given in Table~\ref{tab:HOD}. A simpler three-parameter HOD is that of \cite{Zehavi2004}, which maps to that of \cite{Zheng2005} in the limit that $\sigma_{\log_{10}M}=0$ in equation~(\ref{eq:HOD_Nc}) and $M_0=0$ in equation~(\ref{eq:HOD_Ns}). A more comprehensive HOD model has been given in \cite{Cacciato2012} which links the distribution of galaxies with their luminosity and stellar mass functions; quantities that can be more readily connected to observations.  Emulator based approaches have also been proposed to model HOD,  allowing for extra free parameters to be included: for example those that capture assembly bias \citep{Salcedo2022}.

% Table of occupation statistics
\begin{table}
\caption{Halo-occupation properties for central and satellite galaxies assuming that central galaxies are Bernoulli distributed with mean $p$ (equation~\ref{eq:HOD_Nc}) and satellite galaxies are Poisson distributed with mean $\lambda$ (equation~\ref{eq:HOD_Ns}). Also shown is how the statistics of satellite galaxies are modified if the central condition is imposed. In this case the satellite galaxy distribution is no longer Poisson and $\lambda$ is no longer the mean, even if $\lambda$ would usually be obtained from (something similar to) equation~(\ref{eq:HOD_Ns}). If $p=1$ then the central condition has no impact and the expressions in the last two rows of the table are equal. Note also that $\average{N_\mathrm{c}N_\mathrm{s}}=\average{N_\mathrm{s}}=\lambda$, or equivalently $\mathrm{Cov}(N_\mathrm{c}, N_\mathrm{s})=\lambda(1-p)$.}
\begin{center}
\begin{tabular}{c c c c c c}
\hline
Galaxy type & $\average{N}$ & $\average{N^2}$ & $\average{N(N-1)}$ & $\mathrm{Var}(N)$ \\
\hline
Centrals & $p$ & $p$ & $0$ & $p(1-p)$ \\
Satellites & $\lambda$ & $\lambda(1+\lambda)$ & $\lambda^2$ & $\lambda$ \\
cen. cond. & $p\lambda$ & $p\lambda(1+\lambda)$ & $p\lambda^2$ & $p\lambda[1+\lambda(1-p)]$ \\
\hline
\end{tabular}
\end{center}
\label{tab:HOD}
\end{table}

% Central condition
If the parameters $N_\mathrm{c}$ and $N_\mathrm{s}$ were independent then $\average{N_\mathrm{c}N_\mathrm{s}}=\average{N_\mathrm{c}}\average{N_\mathrm{s}}$ in equation~(\ref{eq:one_halo_central_satellite}), but it would be possible for a halo to host a satellite galaxy without hosting a central. To avoid this, the `central condition' is often imposed such that the number of satellite galaxies is fixed to zero if there is no central galaxy.
Imposing this additional constraint affects the (initially assumed) statistics of $N_\mathrm{s}$, especially for halo masses that contain $\sim1$ galaxy, where $N_\mathrm{c}\sim 0.5$. In this case, the central condition distorts the initial distribution,  resulting in fewer haloes containing satellites, and therefore more haloes containing zero satellites than would otherwise be assumed \citep[\eg][]{Beutler2013}. 
This means that the assumption of Bernoulli statistics for central occupation, Poisson statistics for satellite occupation, and the central condition are mutually incompatible. Something has to give, and traditionally it is the Poisson assumption for satellite galaxies that is modified to retain consistency.
If $\mathcal{P}(N_\mathrm{s})$ is the original probability for a halo to host $N_\mathrm{s}$ satellite galaxies, then this is modified according to
\begin{equation}
\mathcal{P}'(N'_\mathrm{s}) = \average{N_\mathrm{c}} \mathcal{P}(N_\mathrm{s})+(1-\average{N_\mathrm{c}})\delta_{0 N_\mathrm{s}}\ ,
\end{equation}
where $\delta_{ij}$ is the Kroenecker delta. If $\average{N_\mathrm{c}}=1$ then the satellite distribution is unchanged; if $\average{N_\mathrm{c}}=0$ then $\mathcal{P}'(N'_\mathrm{s}=0) = 1$ and $\mathcal{P}'(N'_\mathrm{s}\geq 1)=0$ automatically. These transformations of the probability distribution modify $\average{N_\mathrm{s}}$, $\average{N^2_\mathrm{s}}$ and $\average{N_\mathrm{s}(N_\mathrm{s}-1)}$ in a calculable way, with each being lowered by a multiplicative factor of $\average{N_\mathrm{c}}$ relative to the values calculated for the initially assumed distribution. Therefore, if applying the central condition, the replacements in the final row of Table~\ref{tab:HOD} should be made when evaluating equations~(\ref{eq:one_halo_satellite_satellite}) and (\ref{eq:one_halo_central_satellite}). Note that this implies that $\average{N_\mathrm{s}}$ calculated from equation~(\ref{eq:HOD_Ns}; or any similar equation) will \emph{not} be the true mean number of satellites, and that a covariance between $N_\mathrm{c}$ and $N_\mathrm{s}$ is generated. Recent studies also show that the distribution of galaxies can be significantly non-Poissonian in a mass-dependant manner, and if ignored this can potentially bias cosmological results \citep{Dvornik2018, Beltz-Mohrmann2020,Hadzhiyska2022, Dvornik2022}.


\begin{figure}
\begin{center}
\includegraphics[width=\columnwidth]{plots/HOD.pdf}
\end{center}
\caption{Example HODs for typical surveys that targets red galaxies (upper) and blue galaxies (lower). The solid lines show the mean numbers of central and satellite galaxies as a function of halo mass, while the bounded regions show the expected scatter ($\pm\sigma$) about this. The darker dashed line shows the number of satellite galaxies when imposing the central condition while the lighter dashed lines indicate their expected scatter.  The number of satellites is generally lowered when this condition is imposed.  This is a small effect, and it induces a covariance between central and satellite galaxies in the region where $\average{N_\mathrm{c}}\sim0.5$ and the scatter is also affected. When $\average{N_\mathrm{c}}\sim1$ this difference vanishes.  }
\label{fig:HOD}
\end{figure}

% Example HOD plot explanation
Example HODs are shown in Fig.~\ref{fig:HOD} where the small difference to the mean and variance of the satellite galaxy distribution when imposing the central condition can be seen. We show example HODs from surveys that target red (\eg BOSS; parameters taken from \citealt{Zhai2017}) and blue (\eg GAMA; parameters taken from \citealt{Smith2017}) galaxies. Red galaxies are mainly centrals, with a few satellites, while blue galaxies are mainly satellites at high halo mass.


\subsection{Intrinsic Alignments of Galaxies}
% Introduction
Galaxy formation processes are expected to imprint a correlation between the ellipticity of a galaxy and its environment, hence also with its neighbours.  As the observed correlation between galaxy shapes is the core measurement to detect weak gravitational lensing by large-scale structure, it is critical to determine the correlations that arise purely from the intrinsic alignment (IA) of galaxies.  Without accurate IA models, robust cosmological constraints can not be extracted from weak lensing observations \citep[see][for a review]{joachimi/etal:2015}.  

Luminous red galaxies have been observed to align with their local density field \citep[see for example][]{mandelbaum/etal:2006,joachimi/etal:2011}, but the intrinsic alignment between blue galaxies has yet to be detected \citep{johnston/etal:2019}.  This suggests that different shape-formation mechanisms may be in place.  One hypothesis is that the ellipticity of red/elliptical galaxies is determined by the linear tidal field \citep{catelan/etal:2001, hirata/seljak:2004}, with blue/spiral galaxies gaining their shape through a tidal-torque mechanism \citep{schafer:2009}.  \citet{Schneider2010} suggest that satellite galaxies may also be subject to an infall alignment mechanism where their ellipticity points towards the centre of their parent halo.  Observations find a more complex scenario with satellite radial alignment detected on small scales, changing to a random alignment on large scales \citep{georgiou/etal:2019}, with the alignment strength also sensitive to luminosity \citep{singh/etal:2015,huang/etal:2018}.

The halo model framework provides a route to encode this complexity to define a flexible IA cosmological model \citep{Schneider2010}.  \citet{Fortuna2021} adopt the commonly-used \citet{bridle/king:2007} Non-linear Linear Alignment model (NLA) to describe the alignment between central galaxies.  The NLA model is based on the \citet{hirata/seljak:2004} linear tidal field alignment model, replacing the linear matter power spectrum with a non-linear version such as \hmcode.  This modification was found to improve the agreement of the model with IA numerical simulations \citep{heymans/etal:2006}.  An HOD is then used to determine the fraction of blue and red centrals to determine the `two-halo' part of the matter-intrinsic ellipticity power spectrum, commonly referred to as `GI', with
\begin{equation}
P^{\rm 2h}_{\rm GI} (k) = f^{\rm red}_{\rm cen} P^{\rm NLA, red}_{\rm GI}(k)  + f^{\rm blue}_{\rm cen} P^{\rm NLA, blue}_{\rm GI}(k) \, .
\end{equation}
Here different amplitudes for the NLA model of the red and blue central populations are facilitated, although \citet{Fortuna2021} show that this may not be necessary as the colour dependence of the IA signal is not seen when restricting the sample to central galaxies.   

As the halo population is, on average, spherical, the average inter-halo satellite-central alignment is zero. The `one-halo' IA term then derives from the alignment of satellites with each other and the local matter field with
\begin{equation}
P^{\rm 1h}_{\rm GI} (k) = \int\, \mathrm{d}M \, n(M) \frac{M}{\bar{\rho}_{\rm m}} f_{\rm s} \frac{\langle N_{\rm s}|M \rangle}{\bar{n}_{\rm g}}\, |\hat{\gamma}^{\rm I}({\bf k}|M)| \, \hat{U}(M,k) \, .
\end{equation}
Here $n(M)$ is the halo mass function (Section~\ref{sec:halo_mass_function}), $f_{\rm s}$ is the fraction of satellites which may vary as a function of redshift, $\langle N_{\rm s}|M \rangle$ is the halo occupation distribution of satellites (equation~\ref{eq:HOD_Ns}), ${\bar{n}_{\rm g}}$ is the mean number density of galaxies (equation~\ref{eqn:ng}), and $\hat{U}(M,k)$ is the Fourier transform of the normalised matter density profile (equation~\ref{eq:normU}).  The alignment strength depends on $\hat{\gamma}^{\rm I}$, the density weighted average of the projected satellite ellipticity, assuming all satellites point towards the halo centre.  This term can also include a radial dependance, to decorrelate the alignment on large scales.

In this section we have outlined the halo model for the correlation between intrinsic galaxy ellipticity and the density field, the `GI' term.  We refer the reader to \citet{Schneider2010} for the equivalent terms for the correlation between intrinsic shapes, also known as the `II' term.  Both terms contaminate a tomographic weak-lensing analysis.  \citet{Fortuna2021} argues that this halo model approach is preferable to the most often used NLA model, as it provides a natural route to include observations of the changing fraction of red and blue galaxies across the redshift range of the weak lensing survey, in addition to direct measurements of satellite alignments within groups.  
  
\subsection{Thermal Sunyaev-Zeldovich effect}
\label{sec:tSZ}

% Introduction
The thermal Sunyaev-Zeldovich (tSZ) effect arises when CMB photons are scattered by free electrons, predominantly those hot electrons found in galaxy clusters. This results in a spectral distortion of the CMB black-body spectrum with a characteristic frequency dependence, and this (Compton-$y$ signal) can be extracted from CMB temperature data. The strength of the $y$ signal depends on the product of the free electron temperature and density, a quantity that has units of pressure, and it is therefore the electron-pressure profile that is relevant for tSZ halo-model calculations. The number of free electrons in a halo scales with $M$, and the temperature scales as $\sim M^{2/3}$, so the overall profiles scales like $\sim M^{5/3}$, which means that the shape and amplitude of spectra involving $tSZ$ are determined by more massive haloes than either matter or galaxies, whose profiles scale exactly as $M$ and $\sim M$ (if satellite dominated) respectively. This in turn implies that spectra involving tSZ are relatively sensitive to $\sigma_8$ \citep{Refregier2002, Komatsu2002}, which arises because the high-mass end of the halo mass function is sensitive to $\sigma_8$. This also means that the one-halo term is comparatively high amplitude, and the transition region in the power spectrum occurs at a relatively larger scale \citep[\eg][]{Mead2020}. The dependence on gas temperature makes tSZ an interesting direct probe of baryonic feedback \citep[\eg][]{McCarthy2014, Hojjati2015}. A good pedagogical discussion of the halo model in the context of tSZ is provided by \cite{Hill2013b}, as well as the idea of masking massive low-redshift clusters in order to boost the signal-to-noise (see also \citealt{Hill2018}). 

% Electron pressure profiles
Electron pressure profiles can be derived from theoretical arguments \citep[\eg][]{Komatsu2001, Ostriker2005} or from fitting to observational data and simulations \citep[\eg][the so-called \emph{universal} pressure profile]{Arnaud2010}. It is clear that concepts like non-thermal pressure support and baryonic feedback affect the pressure distribution with galaxy clusters \citep[\eg][]{Shaw2010}, and therefore that models based on hydrostatic equilibrium are overly simplified.

% Pressure bias and hydrostatic mass bias
While the tSZ auto spectrum can be measured, the cosmological constraints from this are in disagreement with those from more developed probes \citep[\eg][]{Planck2015XXII}, possibly due to the self-correlation of residual systematics in the $y$ maps (although see \citealt{McCarthy2014, Horowitz2017, Bolliet2018}). With the auto spectrum suspect, tSZ halo models have been used in cross correlation by: \citeauthor{Addison2012} (\citeyear{Addison2012}; Cosmic Infrared Background CIB);  \citeauthor{Hajian2013} (\citeyear{Hajian2013}; $X$-ray clusters); \citeauthor{Hill2014} (\citeyear{Hill2014}; CMB lensing); \citeauthor{Ma2015} (\citeyear{Ma2015}; galaxy lensing); \citeauthor{Vikram2016} (\citeyear{Vikram2016}; galaxy groups); \citeauthor{Tanimura2019b} (\citeyear{Tanimura2019b}; galaxy clustering); \citeauthor{Koukoufilippas2020} (\citeyear{Koukoufilippas2020}; galaxy clustering); \citeauthor{Yan2021} (\citeyear{Yan2021}; CIB, galaxy clustering); \citeauthor{Maniyar2021} (\citeyear{Maniyar2021}; CIB). Much of the focus is on measuring the pressure bias from the large-scale portion of the power spectrum, which can be thought of as the $k\to0$ limit of the pressure integral that contributes to the two-halo term
\begin{equation}
\average{bP_\mathrm{e}} = \int_0^\infty P_\mathrm{e}(M)b(M) n(M)\,\diff M\ ,
\end{equation}
where $P_\mathrm{e}(M)$ is the mean electron pressure in a halo of mass $M$. Other focus is on the so-called \emph{hydrostatic-mass bias}, which arises due to deviations of the gas from hydrostatic equilibrium, and this biases inferred cluster masses compared to more direct (\eg weak lensing) measurements.

\subsection{$X$-rays}
\label{sec:X-ray}

% Single X-ray paragraph
$X$-rays are emitted via the bremsstrahlung process, when the direction-of-travel of free electrons is modified by an interaction with a proton. Since this is a two-body scattering process the total contribution of a halo profile scales like $ M^2\times T \sim M^{8/3}$, even more strongly than electron pressure. This ensures that the $X$-ray auto spectrum will be completely dominated by the one-halo contribution (unless masking is applied), which in turn leads to a strong dependence on $\sigma_8$ \citep{Diego2003} via the dependence of the high-mass tail of the halo mass function. Halo models of $X$-rays maps have been used in cross-correlation analyses by: \citeauthor{Hurier2015} (\citeyear{Hurier2014}; \citeyear{Hurier2015}; tSZ); \citeauthor{Singh2017} (\citeyear{Singh2017}; AGN galaxy clustering); \citeauthor{Hurier2019} (\citeyear{Hurier2019}; CMB lensing).

\subsection{Cosmic Infrared Background}
\label{sec:CIB}

% Single CIB paragraph
Warm dust in star-forming galaxies emits infrared radiation, which can be detected by space-based telescopes. The halo signal will be proportional to the amount of dust in a halo, which in simple models can be taken to scale with the number of galaxies \citep[\eg][]{Xia2012}, although different populations of galaxies (\eg spirals, star-burst, proto-spheroids) can also be considered as long as an occupation model for each is specified. Given the galactic origin, one may expect the halo profile of CIB emission to be similar to that of galaxies. Note that CIB flux is typically measured in frequency bins, and CIB emission from distant galaxies will be redshifted, so the resulting angular power spectra will mix galaxy populations and redshifts in unintuitive ways. \cite{Xia2012} considered the CIB power spectrum while \cite*{Addison2012} considered the cross spectrum of CIB and tSZ. \cite*{Addison2013} suggested that evolution of the dust spectral energy distribution and scale-dependent halo bias (\ie $\Bnl$) may be required to self consistently understand one- and two-point functions of the CIB within the same theoretical framework. More recently, \cite{Maniyar2021} has presented a halo model where CIB emission is tied to the halo-mass accretion history.

\subsection{Neutral hydrogen}
\label{sec:HI}

% Single HI paragraph
Neutral hydrogen (\textsc{hi}) emits characteristic $21\cm$ wavelength radiation when the electron and proton spin align or misalign. The profile signal will scale with $M$ and the fraction of \textsc{hi} in a halo, but \textsc{hi} is eroded by heat and AGN activity, so the signal will be dominated by comparatively low-mass haloes that still retain significant \textsc{hi} reservoirs. Ingredient lists for the \textsc{hi} abundance and halo profiles can be found in \cite{Padmanabhan2017a} and \cite{Villaescusa-Navarro2018}, and clustering calculations are presented in \cite{Padmanabhan2017b}, \citeauthor{Feng2017} (\citeyear{Feng2017}; in cross-correlation with Lyman-$\alpha$) and \cite*{Schneider2021}. \cite{Wolz2019} investigates shot noise in the power spectrum of \textsc{hi} under differing assumptions about the source of \textsc{hi} emission -- either co-located with galaxies or with dark matter.

%\subsection{ISW}
%\label{sec:ISW}
%\Mead{Maybe some other Peacock papers here?}
%\cite{Hang2021} % with galaxies

\subsection{$\gamma$-rays}
\label{sec:gamma}

% Single gamma-ray paragraph
If dark matter has a significant self-interaction cross section then dark matter--dark matter annihilation events are expected to produce a potentially detectable flux of $\gamma$-rays. This signal scales with the square of the dark-matter density, with the result that halo cores, low-mass haloes and subhaloes are expected to produce the most significant contributions. The cross correlation of gamma ray maps with large-scale structure has been investigated by \cite{Shirasaki2014} and \cite{Troester2017} using halo models to generate theory curves and constrain the cross section.
