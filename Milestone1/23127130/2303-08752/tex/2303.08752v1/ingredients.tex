\label{sec:ingredients}

% Introduction - halo-mass definitions
When using the halo model it is necessary to make choices for the bias, halo mass function and halo profiles. Due to the lack of a fundamental theory for non-linear gravitational clustering, it is common to calibrate these ingredients via \nbody simulations, or even via data. Within the halo model, haloes are treated as discrete entities, although real haloes never have clear boundaries. When defining haloes in simulations it is necessary to make a choice of boundary, and this choice must be consistent when using collections of simulation-calibrated ingredients within a halo model. The fundamental choice is how to identify a halo from the \nbody particle distribution. Two algorithms are in common usage: friends-of-friends (FoF; \citealt{Huchra1982}) and spherical-overdensity (SO; \citealt{Lacey1994}). 

%FoF
The FoF scheme is simpler, with the only user-specified parameter being the `linking length', which defines the maximum distance between two particles that are considered to be part of the same halo.  All particles within the linking length of at least one other particle in the halo are joined to that halo. Typically the linking length is taken to be $b=0.2$ times the mean-inter-particle separation.  A FoF finder with this linking length applied to particles following an isothermal distribution ($\rho\propto r^{-2}$), will define a halo boundary such that the halo has a mean overdensity close to the analytical spherical-collapse result (see Section~\ref{sec:spherical_collapse}) in an Einstein-de Sitter model ($\Delta_\mathrm{v}\simeq178$ for $\Omega_{\rm m}=1$).

%SO
SO algorithms, on the other hand, first choose halo centres (usually from minima in the gravitational potential, but sometimes in the density) and then grow spheres out from these peaks until a fixed overdensity threshold has been reached. With SO there are several choices to be made: most obviously the value for the overdensity threshold ($200\bar\rho$ is common) but also exactly how to define the halo centres (how are continuous fields defined from the discrete particle distribution?) and how to count haloes as distinct entities (so-called percolation). While FoF is conceptually simpler, and is a mathematically unambiguous operation, SO is more common because it relates more closely to how halo formation is thought to occur and to how haloes are identified in data sets. SO haloes are, by definition, spherical (although the particle distribution that contributes to them may not be), whereas FoF haloes can be elongated structures, some of which may look to be distinct objects joined by a bridge. 

% 'Correct' definition % Cosmology dependence of halo identification
It should be noted that there is no single `correct' halo definition, and the best choice will depend on the observable that one is attempting to model. It is also important to be consistent, and to use sets of relations that have been calibrated on haloes identified using the same definition.  However, we note that it may be prudent to identify haloes using a cosmology-dependent definition, which accounts for the fact that halo formation happens at different rates, with different end results, in different cosmologies. Indeed, \cite{Courtin2011}, \cite{Despali2016} and \cite{Mead2017} have all noted that more `universal' (cosmology independent) behaviour is observed when haloes are identified with an overdensity threshold derived from the spherical-collapse model, with the general trend that haloes become denser the more dark energy takes hold of the expansion. Useful fitting functions can be found in \cite{Bryan1998} and \cite{Mead2017}.

\subsection{Preliminary definitions}
\label{sec:definitions}

Let us define a few useful quantities before we introduce the different ingredients.
The variance in the linear matter overdensity field when smoothed on comoving scale $R$ is
%
\begin{equation}
\sigma^2(R) =\int_0^\infty 4\pi \left(\frac{k}{2\pi}\right)^3 P^\mathrm{lin}(k)T^2(kR)\,\diff \ln k\ ,
\label{eq:sigmaR}
\end{equation}
%
where $T(kR)$ is the filter window function, which is almost exclusively taken to be a real-space top hat; the Fourier transform of which is
%
\begin{equation}
T(x)=\frac{3}{x^3}\left(\sin x - x\cos x\right)\ .
\end{equation}
%
The Lagrangian comoving scale,  $R$,  is the comoving radius of a sphere in a homogeneous Universe which contains a given mass of $M$,
%
\begin{equation}
M=\frac{4}{3}\pi R^3\bar\rho\ ,
\end{equation}
%
where $\bar\rho$ is the mean comoving matter density. This relation allows us to write $\sigma(R)$ in terms of mass. The `peak height', 
%
\begin{equation}
\nu(M)=\delta_\mathrm{c}/\sigma(M)\ ,
\label{eq:peak_height}
\end{equation}
%
is a useful quantity that increases monotonically with the halo mass. Here $\delta_\mathrm{c}(z)\simeq 1.686$ is the critical linear overdensity needed for haloes to collapse under the spherical-collapse model at redshift $z$.

\subsection{Halo mass function}
\label{sec:halo_mass_function}


% Table of mass functions
\begin{table*}
\caption{A non-exhaustive list of popular halo mass functions, usually presented either as $f(\sigma)$ or $f(\nu)$. For each mass function we list the halo finder and definition, and we note that the mass function is a strong function of these choices: FoF haloes are uniquely defined by the linking length, but SO haloes need the halo-centre-finding and percolation scheme to be defined, as well as the overdensity threshold. SO haloes are then defined as spherical objects that are bounded such that they have a certain overdensity relative to either the background (\eg $200$) or critical (\eg $200$c) densities. Where a virial definition is used, this is most commonly evaluated using the fitting formula of \protect\cite{Bryan1998}. For every mass function, a linear halo bias can be derived using the peak-background split argument (equation~\ref{eq:peak_background_split}) but this may not be an accurate description of the large-scale bias \citep[\eg][]{Tinker2010, Manera2010}. The calibrated bias models of \protect\cite{Sheth2001} and \protect\cite{Tinker2010} are based on the mass functions of \protect\cite{Sheth1999} and \protect\cite{Tinker2008} respectively, but do not use the peak-background split. We also note whether each mass function is normalised such that all mass is in haloes (equation~\ref{eq:mass_normalisation}).}
\begin{center}
\begin{tabularx}{\textwidth}{lcccX}
\hline
Reference & Finder & Definition & Normalised & Notes \\
\hline
\cite{Press1974} & -- & -- & Yes & Purely analytical argument using the spherical collapse model, not connected to a specific mass definition, cosmology or redshift. \\
\cite{Sheth1999} & SO & virial & Yes & Original paper calculates the halo bias via the peak-background split; \cite*{Sheth2001} use an ellipsoidal-collapse argument for a more accurate bias. Cosmology dependence of $\delta_\mathrm{c}$ accounted for via spherical collapse. \\%(Tormen, private communication). \\
\cite{Jenkins2001} & FoF & $0.2$ & No & First accurate parameterisation for FoF haloes.\\
\cite{Warren2006} & FoF & $0.2$ & No & Argument presented for resolution-dependent conversion between FoF and SO masses. Correction to FoF masses for low-particle haloes. \\
\cite{Reed2007} & FoF & $0.2$ & No & Depends on effective power spectrum index at the collapse scale, $n_\mathrm{eff}$, as well as $\nu$.\\
\cite{Peacock2007} & FoF & $0.2$ & Yes & Based on fit to model of \cite{Warren2006}. \\
\cite{Tinker2008} & SO & $200$--$3200$ & Both & Parametrised in terms of $\sigma$. Principle result is unnormalised and has redshift-dependent (non-universal) parameters for $\Delta_\mathrm{h}=200$. However, redshift-independent results are presented for a variety of other SO halo definitions, and these can be interpolated between (appendix B) for a virial halo definition. A normalised mass function is also presented (appendix C). \\
\cite{Tinker2010} & SO & $200$--$3200$ & Yes & Mass function is the same as the normalised (appendix C) version from \cite{Tinker2008} but recast in terms of $\nu$. Once again, redshift-dependent parameters are presented only for $\Delta_\mathrm{h}=200$. A calibrated halo bias is presented that fulfils equation~(\ref{eq:bias_normalisation}) without using the peak-background split argument. \\
%\cite{Manera2010} & FoF & $0.15$, $0.168$, $0.2$ & Yes & \\
\cite{Crocce2010} & FoF & $0.2$ & No & Uses functional form of \cite{Warren2006}. \\
\cite{Bhattacharya2011} & FoF & $0.2$ & No & Consider $w$CDM dark energy. \\
\cite{Courtin2011} & FoF & virial & No & Results demonstrate that virial definitions are `more universal'; semi-analytical relation to convert $\Delta_\mathrm{h}$ to FoF linking length. \\
\cite{Watson2013} & FoF/SO & various & No & Consider wide redshift range, from $z=30$ to $0$.\\
%\cite{Angulo2013b} & SO & $200$c & & \\
\cite{Despali2016} & SO & virial & No & Argues that the mass function is `more universal' when a cosmology-dependent virial criterion is used to identify haloes, rather than a fixed overdensity threshold. \\
\cite{McClintock2019a} & SO & $200$ & Yes & Parameters of a universal, $\sigma$-dependent, fitting function are emulated to encompass cosmology and redshift dependence. \\
\cite{Bocquet2020} & SO & $200$c & No & Mass function principle components are directly emulated. \\
\hline
\end{tabularx}
\end{center}
\label{tab:massfunctions}
\end{table*}

% Halo mass function
\begin{figure}
\begin{center}
\includegraphics[width=\columnwidth]{plots/hmf_and_bias.pdf}
\end{center}
\caption{Upper panel: Dimensionless multiplicity function, $M^2 n(M)/\bar\rho$, as a function of halo mass for three popular halo-mass functions \protect\citep{Sheth1999, Tinker2010, Despali2016} at $z=0$ (solid) and $z=1$ (dashed) for the virial halo definition. The top axis shows the $\nu$ values corresponding to halo mass at $z=0$ \emph{only} (the mapping is $z$-dependent). For normalised mass functions, integrating the multiplicity function over $\ln M$ will equal unity, and therefore the shape of the function determines the contribution of haloes in a logarithmic mass range to the total mass in the cosmos, with the peak determining those most important. As the universe evolves, more haloes of higher mass are created, but this is done at the expense of those of lower mass via mergers. Lower panel: the associated linear halo bias, low mass haloes are anti-biased ($0<b<1$) with a constant asymptotic value at low mass. The transition to biased, $b>1$, objects occurs around the non-linear mass ($\nu=1$). At fixed halo mass, haloes are comparatively rarer, and more highly biased, at higher $z$. The bias shown for \citecaption{Tinker2010} is from their calibrated fitting function, whereas in the other two cases it is from the peak-background split. \citecaption{Tinker2010} predicts fewer, but more highly biased, high-mass haloes compared to the other models.}
\label{fig:multiplicity}
\end{figure}

The halo mass function is usually parametrised in terms of either $\sigma$ or $\nu$, rather than $M$ directly, because it has been shown that the halo mass function (and also bias) exhibit close-to-universal behaviour as a function of cosmology and redshift in terms of these variables \citep[\eg][]{Press1974, Bond1991, Sheth1999, Tinker2008}. Analytical approaches for calculating (approximately) the halo mass function rely on either peaks theory or excursion sets. These methods start from the initial matter density field and relate some of its properties to the haloes that form later. For peaks theory, the focus is on peaks in the primordial matter density field \citep{Bardeen1986}, while excursion sets look at overdense regions \citep[\eg][]{Bond1991, Bond1996, Stein2019}.  The peak height, $\nu$, has been shown to be the relevant quantity to consider when calculating halo formation via peaks theory and excursion sets. For reference, for a vanilla \LCDM cosmology at $z=0$, $\nu=0.5$, $1$, $2$, and $3$ correspond to $\simeq 10^{10.4}$, $10^{12.5}$, $10^{14.2}$, and $10^{14.9}\Msun$. Some common mass functions are shown in Fig.~\ref{fig:multiplicity}, and the (generally non-linear) mapping between $M$ and $\nu$ can be read off the axes.

% f(nu)
Now we can write the halo mass function in terms of the peak height,  by defining 
%
\begin{equation}
f(\nu)\,\diff\nu=\frac{M}{\bar\rho}n(M)\,\diff M\ .
\label{eq:nu_M_conversion}
\end{equation}
%
If all mass is to be contained in haloes, then $f(\nu)$ integrated over all $\nu\in[0,\infty]$ should equal unity, which derives from mass conservation (equation~\ref{eq:mass_normalisation}). Note that this condition is only imposed on some fitting functions. Other than the constraint imposed by mass conservation, the shape of the low-mass end of the halo-mass function is difficult to access through \nbody simulations due to finite particle resolution. Commonly-used fitting functions should be interpreted with caution in this regime. A common form of $f(\nu)$ to be found in the literature is that of \cite{Sheth1999}:
%
\begin{equation}
f(\nu)=A\left[1+(q\nu^2)^{-p}\right]\mathrm{e}^{-q\nu^2/2}\ ,
\label{eq:ST_massfunction}
\end{equation}
%
where $p$, $q$ and (sometimes) $A$ are fitted to simulated data. If $A$ is fitted independently of $p$ and $q$ then the mass function will not be normalised. If $A$ is not fitted then it depends on $p$ and $q$ via the normalisation condition. In Table~\ref{tab:massfunctions} we list some popular mass functions together with the halo finder and definition on which they were calibrated.

Finally we note that $\delta_\mathrm{c}\simeq 1.686$ is usually assumed; a value that corresponds to a universe with an Einstein-de Sitter background\footnote{A spatially flat cosmology with the matter density parameter $\Omega_{\rm m}=1$.}.  
Although,  $\delta_\mathrm{c}$ has a weak cosmology dependence \citep[\eg][]{Lacey1993} that can be calculated using the spherical-collapse model (fitting formulae: \citealt{Nakamura1997, Mead2017}). This cosmology dependence is often ignored in the conversion between $\nu$ and $M$. However, the general exponential form of the halo mass functions (\eg equation~\ref{eq:ST_massfunction}) can make this weak dependence have a larger impact than one might first assume.  For example,  spherical collapse predicts that $\delta_\mathrm{c}\simeq 1.676$ for $\Om=0.3$ \LCDM, a small decrease from the canonical $1.686$. However, this small difference results in a $\sim4$ per cent \emph{increase} in the abundance of rare ($\nu=4$; $M\simeq\sform{2}{15}\Msun$; $z=0$) haloes if the mass function of \cite{Sheth1999} applies. \cite{Courtin2011} and \cite{Mead2017} have suggested that retaining this cosmology dependence of $\delta_\mathrm{c}$ may improve the cosmological universality of halo mass functions and halo-model calculations.

%%%%%%%%%%%%%%%%%%%%%%%%%%%%%%%%%%%%%%%%%%%%%%%%%%%%%%%
\subsection{Linear halo bias}
\label{sec:lin_bias}

% Peak-background split
On scales large enough to comfortably encompass the largest haloes, the overdensity of haloes of any mass can be approximated by the (unconditional) halo mass function via the peak-background split argument \citep{Cole1989, Mo1996, Sheth2001}. The density field is thought of as a sum of large- and small-scale waves with haloes forming at global peaks; more peaks are pushed over the formation threshold when large and small-scale waves constructively interfere, which will be in regions of large-scale overdensity, leading to biased clustering. The peak-background split argument can be used to calculate an approximate linear halo bias from any mass function:
%
\begin{equation}
b(\nu) = 1-\frac{1}{\delta_\mathrm{c}}\left[1+\frac{\diff\ln f(\nu)}{\diff\ln\nu} \right]\ .
\label{eq:peak_background_split}
\end{equation}
%
If a particular $f(\nu)$ satisfies the mass-normalisation condition in equation~(\ref{eq:mass_normalisation}), then the combination of $f(\nu)$ with $b(\nu)$ calculated this way automatically satisfies the bias-normalisation condition in equation~(\ref{eq:bias_normalisation}). However, if these normalisation conditions are not important, then equation~(\ref{eq:peak_background_split}) can be still applied to any mass function to get an expression for the bias (although it may not be accurate). In practice, not satisfying the bias-normalisation condition is only a fundamental problem when calculating the matter spectrum, and only then if one explicitly integrates over all halo masses, which is not normally done in halo-model codes\footnote{See the discussion in Appendix A of \cite{Mead2020}}.

Note that the peak-background split is \emph{not} the only way to satisfy the bias normalisation condition, and popular bias relations (\citealt*{Sheth2001}; \citealt{Tinker2010}) satisfy the normalisation condition using other schemes. The accuracy of the peak-background split has been disputed \citep[\eg][]{Manera2010} and calibrated bias relations may therefore be preferred.


%%%%%%%%%%%%%%%%%%%%%%%%%%%%%%%%%%%%%%%%%%%%%%%%%%%%%%
\subsection{Non-linear halo bias}
\label{sec:non_linear_bias}

% Introduction
As discussed in Subsection~\ref{sec:two_halo_term}, the beyond-linear portion of the halo bias is not often considered in halo-model calculations: It is common to set $\beta^\mathrm{nl}=0$ in equation~(\ref{eq:Bnl_def}), and therefore implicitly $I^\mathrm{nl}=0$ in equation~(\ref{eq:Inl}). This means that the `standard' two-halo term at large scales for any tracer is always the linear power multiplied by some scaling (bias) factors, which arise jointly through the linear halo bias and halo occupation. As shown by \cite{Mead2021b}, the lack of beyond-linear bias is mainly responsible for the poor performance of the standard halo model in the transition region between the two- and one-halo terms. At smaller scales the standard two-halo term is suppressed by the halo window functions, but this effect is often not visible in the total halo-model spectrum since it occurs on scales where the one-halo term tends to dominate the power. Note well that the presence of the window functions in the standard two-halo term is \emph{not} accounting for halo exclusion (the fact that spatially-exclusive haloes should not overlap), but instead is a blurring of correlation between points in different haloes caused by the fact that these points are not at the exact halo centre. In reality, the two-halo term should be further suppressed by halo exclusion (Section~\ref{sec:exclusion}), which is also absent in simple linear bias models.

% Cross-correlation coefficient figure
\begin{figure}
\begin{center}
\includegraphics[width=\columnwidth]{plots/Rhh.pdf}
\end{center}
\caption{Halo--halo correlation coefficient as a function of scale taken from the \DQ emulator. At large scales this is consistent with unity, but it drops below unity at smaller scales. This implies that there is a non-zero covariance in the clustering between haloes in different mass bins, which cannot be captured by a simple linear bias model.}
\label{fig:Rhh}
\end{figure}

% Discussion of cross correlation
The two-halo term accounts for inter-halo clustering and therefore the fundamentally-correct spectrum to include within the two-halo term is the halo power spectrum, for which no fitting function exists in the literature. In our notation (equation~\ref{eq:two_halo_term}) the beyond-linear portion of this is factored out into $\beta^\mathrm{nl}$. To illustrate some properties of this, we define the halo--halo cross correlation in Fourier space as 
%
\begin{align}
\begin{split}
R_\mathrm{hh}(M_1, &M_2, k)= \\
&\frac{P_\mathrm{hh}(M_1, M_2, k)}{\sqrt{P_\mathrm{hh}(M_1, M_1, k) P_\mathrm{hh}(M_2, M_2, k)}}\ ,
\label{eq:cross_correlation_coefficient}
\end{split}
\end{align}
%
and show this for various halo masses in Fig.~\ref{fig:Rhh}, where the correlation is calculated using the \DQ emulator of \cite{Nishimichi2019, Miyatake2022a}. The fact that this departs from unity at small scales indicates a non-zero covariance between the clustering of haloes in different mass bins \citep[][]{Hamaus2010, Baldauf2013, Schmidt2016}. This indicates that any model for the non-linear halo bias where the bias is separable,  
%
\begin{equation}
P_\mathrm{hh}(M_1, M_2, k)\simeq b(M_1, k)b(M_2, k) P_\mathrm{mm}(k)
\end{equation}
%
fails to describe the covariant structure, even if a scale dependent $b(M, k)$ \citep[\eg][]{Fedeli2014b} or if a non-linear $P_\mathrm{mm}(k)$ is used: the non-linear halo-bias is fundamentally a non-separable function! The structure of the halo power spectrum ensures that the clustering of haloes in one mass bin respond to the clustering of haloes in other mass bins. \cite{Tinker2005} use a fitting function in real space for the radial dependence of the non-linear halo bias and compute that as a function of the non-linear matter correlation function. This fitting function has no dependence on halo masses other than through the linear halo bias, and is thus difficult to interpret for different galaxy populations that may exist in very different haloes. 

% Non-linear power in two-halo term
Some authors \citep[\eg][]{Cacciato2012, vandenBosch2013}, particularly those interested in using the halo model to compute galaxy spectra, replace the linear spectrum that appears in equation~(\ref{eq:two_halo_term}) with the full non-linear matter spectrum\footnote{Usually from a fitting function, for example \halofit \citep{Smith2003, Takahashi2012} or \hmcode \citep{Mead2015b, Mead2021a}, although this could also come from an emulator \citep[\eg][]{Lawrence2017, Knabenhans2019, Angulo2020}.}. We note that this is inconsistent with the halo-model ethos, since in principle the non-linear matter power should be computable via the halo model. However, for galaxies, it has been demonstrated that using the non-linear power provides a better approximation at quasi-linear ($k\simeq0.1\iMpc$) scales compared to using the linear power. Despite this, we advise an abundance of caution: the non-linear matter power contains its own one-halo term, which arises due to the auto-convolution of the matter profiles at small scales. This feature has no analogue in the halo spectrum (which is dominated by exclusion at such scales), even though the shape of the halo spectrum may be super-linear at quasi-linear scales. This means that using the non-linear matter spectrum can be extremely wrong at small scales, and one virtue of the linear approximation is that it will be significantly less wrong. Incorporating a small-scale halo-exclusion model may ameliorate the problem induced when using the non-linear matter spectrum, with the `exclusion' term performing the joint job of dampening the excess one-halo power and accounting for the genuine spatially exclusivity of haloes, but does not escape the physical incorrectness of employing the non-linear matter spectrum in this role. Finally, the replacement $P^\mathrm{lin}(k)\to P^\mathrm{nl}(k)$, with $I^\mathrm{nl}(k)=0$, in equation~(\ref{eq:two_halo_term}) suffers from the same problem demonstrated in Fig.~\ref{fig:Rhh}; the value of $R_\mathrm{hh}=1$ always, contrary to measurements and theoretical expectations. 

% Actual non-linear bias
Few authors have tackled the issue of non-linear halo bias in detail. \cite{Smith2007, Ginzburg2017} used the combination of perturbation theory for both the matter field and halo bias to demonstrate that improved predictions could be made for quasi-linear scales. Unfortunately, because perturbation theory fails at smaller scales, so does this method. \cite{Nishimichi2019} emulated the halo power spectrum directly, so that it could be incorporated within halo-model calculations of the galaxy power spectrum. This emulated power contains both classical non-linearity in the bias together with halo exclusion, since both effects are present in simulations and in the measured halo overdensity fields. Finally, \cite{Mead2021b} showed that incorporating the non-linear halo bias (measured from \nbody simulations) within halo-model calculations dramatically improves accuracy in the transition region. The $\beta^\mathrm{nl}$ defined in that work (equation~\ref{eq:Bnl_def}) is related to the halo stochasticity matrix defined by \cite{Hamaus2010} and the halo stochasticity covariance defined by \cite{Schmidt2016}.

%%%%%%%%%%%%%%%%%%%%%%%%%%%%%%%%%%%%%%%%%%%%%%%%%%%%%%

\subsection{Dark matter halo profiles}
\label{sec:dark_matter_haloes}

% NFW profile
By far the most common form taken for the density profile of collisionless matter is that of \citeauthor*{Navarro1997} (NFW; \citeyear{Navarro1997})
%
\begin{equation}
\rho(r)=\frac{\rho_\mathrm{s}}{r/r_\mathrm{s}(1+r/r_\mathrm{s})^2}\ ,
\label{eq:NFW}
\end{equation}
%
where $\rho_\mathrm{s}$ and $r_\mathrm{s}$ are the scale radius and density, both of which depend on the halo mass. The profile is usually truncated at the halo radius $r_\mathrm{h}$ and if this truncation is not imposed then it should be noted that the total mass of the profile is formally infinite. The halo radius (which need not necessarily be the `virial' radius\footnote{In the context of halo definitions, the `virial' radius is often used interchangeably with `halo' radius. It need not have anything to do with virialised haloes or the virial theorem.}) is calculated via
%
\begin{equation}
M=\frac{4}{3}\pi r_\mathrm{h}^3\Delta_\mathrm{h}\bar\rho\ ,
\label{eq:virial_radius}
\end{equation}
%
where $\Delta_\mathrm{h}$ is the halo overdensity with respect to the background matter density (usually either $200$, or $200$ times the critical density, or else the virial definition). 
A less common, but possibly more accurate, choice for the halo profile is that of \cite{Einasto1984}, 
%
\begin{equation}
\rho(r)=\rho_{\rm s}{\rm exp}\left(- \frac{2}{\alpha} \left[\frac{r}{r_{\rm s}} -1 \right]^\alpha\right) \ ,
\end{equation}
%
which has an extra `shape' parameter, $\alpha$, as well as an $r_\mathrm{s}$ similar to the NFW profile, \citep[also see][for more details]{Navarro2004,Gao2008}.
A recent update to the \cite{Einasto1984} halo profiles has been given by \cite{Diemer2022}, which includes two characteristic scales and fits to numerical simulations more accurately.  

% Table of mass functions
\begin{table*}
\caption{A non-exhaustive list of popular NFW profile concentration--mass relations that have been fitted to data from \nbody simulations. Note that different samples of haloes may have been used in fitting each relation (\eg relaxed vs. all haloes) and different criteria may be employed to isolate unique halo centres, and different cosmologies may have been considered. We encourage the reader to carefully read the papers below to ensure that they understand the details of the relation they are using and to ensure that it is appropriate for their use case.}
\begin{center}
\begin{tabularx}{\textwidth}{lcX}
\hline
Reference & Definition & Notes \\
\hline
\cite{Navarro1997} & $200$c & Depends on a cosmology-dependent halo-collapse redshift that is calculated semi-analytically. \\
\cite{Bullock2001} & virial & Two relations presented in paper: a simple model where $c$ is a power-law in $M$ (although scaled by a cosmology-dependent non-linear mass) and a more complicated model where $c$ is related to a cosmology-dependent halo formation redshift, which is calculated semi-analytically. \\
\cite{Eke2001} & virial & Depends on a cosmology-dependent halo-collapse redshift that is calculated semi-analytically. \\
\cite{Neto2007} & $200$c & Only considered the Millennium \citet{Springel2005a} cosmology at $z=0$. \\
\cite{Maccio2008} & virial & Modified version of the \cite{Bullock2001} algorithm. \\
\cite{Duffy2008} & $200$, $200$c, virial & Simple $c(M)$ power-law relations are presented that are fitted to simulations of \wmap5 cosmology. Explicit $z$ dependence. Separate relations for `relaxed' and `full' samples of haloes. \\
\cite{Prada2012} & $200$c & `Cosmology dependent' relation presented as a function of $\sigma(M, a)$. Upturn in halo concentration for high-mass haloes. \\
\cite{Kwan2013} & $200$c & Emulated relation for a variety of $w$CDM cosmologies. \\ 
\cite{Ludlow2014} & $200$c & Relates halo concentration to mass-accretion history. \\
\cite{Klypin2014} & $200$c & Parametrised in terms of $\nu$. \\
\cite{Diemer2015} & $200$c & Present a semi-analytical, cosmology-dependent model parametrised in terms of $\nu$ and $n_\mathrm{eff}$ -- the effective slope of the power spectrum on collapse scales. Demonstrates that concentration--mass relation is `most universal' when masses are defined via $200$c. \\
\cite{Correa2015} & $200$c & Relates halo concentration to mass-accretion history. Only applies to relaxed haloes. \\
\cite{Okoli2016} & $200$c & Focusses on relaxed low-mass haloes using analytical arguments. Cosmology dependence incorporated via $\nu$ dependence. \\
\cite{Ludlow2016} & $200$c & Applies for WDM as well as for CDM cosmologies. Depends on a collapse redshift that is calculated semi-analytically. \\
\cite{Child2018} & $200$c & Power-law relation but scaled via the cosmology-dependent non-linear mass. Also consider Einasto profiles. Individual and stacked halo profiles considered separately. \\
\cite{Diemer2019} & $200$c & Improved version of \cite{Diemer2015} with additional dependence on the logarithmic linear growth rate to capture non-standard expansion histories. \\
\hline
\end{tabularx}
\end{center}
\label{tab:concentration}
\end{table*}

% Concentration
To fully specify the halo profile in equation~(\ref{eq:NFW}) we need to know the scale radius, $r_\mathrm{s}$, which is usually related to $r_\mathrm{h}$ via a concentration--mass relation: $c=r_\mathrm{h}/r_\mathrm{s}$. These are always calibrated to haloes measured in \nbody simulations, and once again we stress that the relations will depend on the halo definition, as well as the details of precisely how the concentration was inferred from the measured halo sample. For example: Is the relation fitted to the mean or median halo profile in a mass bin, or to individual haloes? Is the cumulative profile fitted or the raw density? Are certain haloes discarded from the sample? Is the concentration inferred from the circular velocity profile? In Table~\ref{tab:concentration} we list some concentration--mass relations that are in common usage. We also note that a scatter in the concentration parameter at fixed halo mass is seen in haloes identified in \nbody simulations \citep[\eg][]{Jing2000, Bullock2001} with an approximate log-normal distribution with $\sigma_{\ln c}\simeq 0.3$. This scatter can be included in halo-model calculations (see Section~\ref{sec:scatter}).

%This scatter is usually neglected in halo-model calculations, but can be included relatively straightforwardly using the method outlined by \cite{Giocoli2010} in the limit that the scatter is not correlated with the environment (unlikely to be true in detail). For the matter--matter power spectrum, uncorrelated scatter in concentration modifies the power by a few per cent at $k\sim 10\iMpc$.

% Normalisation
The constant of proportionality from equation~(\ref{eq:NFW}) can be found by ensuring that integrating the density profile over the halo volume gives the correct enclosed mass:
%
\begin{equation}
\rho_\mathrm{s} = \frac{M}{4\pi r_\mathrm{h}^3}\left[\frac{c^3}{\ln(1+c)-c/(1+c)}\right]\ .
\end{equation}
%
In passing, we note that the mass enclosed at a given radius by an NFW profile is
%
\begin{equation}
M(r) = M\left[\frac{\ln(1+r/r_\mathrm{s})-(r/r_\mathrm{s})/(1+r/r_\mathrm{s})}{\ln(1+c)-c/(1+c)}\right]\ .
\end{equation}
%

% Halo window profiles
\begin{figure}
\begin{center}
\includegraphics[width=\columnwidth]{plots/NFW_U.pdf}
\end{center}
\caption{NFW profile normalised Fourier transforms. We show a range of halo masses log-spanning $10^{13}$ to $10^{15}\Msun$ with parameters corresponding to a \LCDM cosmology at $z=0$. This takes in a range of virial radii from $0.44$ to $2.1\Mpc$ and concentrations from $6.9$ to $4.7$, with central $M=10^{14}\Msun$, $r_\mathrm{h}=0.95\Mpc$ and $c=5.7$. The vertical lines show approximate wavenumbers corresponding to the central halo virial and scale radius.}
\label{fig:window}
\end{figure}

% Window profiles
Once the real-space halo profile has been specified it must be Fourier transformed (equation~\ref{eq:window_function}) for use in the power spectrum calculation (equations~\ref{eq:one_halo_term} and \ref{eq:two_halo_term}). Example normalised profile Fourier transforms are shown in Fig.~\ref{fig:window}. Note that the shape dependence of smaller haloes only affects the power at higher $k$. In the power spectrum calculation the normalised windows are multiplied by halo mass, which boosts the contribution from higher-mass haloes, but are also multiplied by the mass function, which reduces the contribution.

% Overdensity choice
The choice of halo overdensity to use in the halo model, $\Delta_\mathrm{h}$, is fixed if using ingredients that have been calibrated on haloes identified via a SO finder, but it is less obvious what to choose for $\Delta_\mathrm{h}$ when dealing with FoF-identified haloes. Many schemes have been proposed to relate a linking length, $b$, to $\Delta_\mathrm{h}$, the simplest involve imagining halo profiles sampled by discrete particles, and then calculating the corresponding linking length at the halo boundary. This has the unattractive property that linking length depends on the halo profile \citep[\eg][]{Lukic2009}, so often a simple isothermal halo is assumed when performing this conversion, which leads to the approximate relation \citep{Lacey1994}:
%
\begin{equation}
\left(\frac{b}{0.2}\right)^{-3} \simeq \frac{\Delta_\mathrm{h}}{180}\ .
\end{equation}
%
The correspondence with the matter dominated spherical-collapse result is precisely why the linking length $b=0.2$ is often chosen. \cite{Warren2006} demonstrated that FoF linking would underestimate the masses of haloes with low particle number, and proposed a correction that is sometimes applied to boost the halo masses of FoF identified haloes. Other relations between linking length and overdensity have been proposed in the literature \cite[\eg][]{Courtin2011, More2011}. It should be noted that SO finders can also underestimate the halo mass function at low mass due to particle-resolution issues, see \citealt{Nishimichi2019}.


% Baryonic feedback
\subsection{Baryonic feedback}
\label{sec:baryons}

% Introduction
When modelling the matter power spectrum via the halo model it is common to employ NFW profiles, which provide a reasonable match to gravity-only simulated data at small scales. However, in reality `matter' in the universe is comprised of CDM, gas and stars/dust, each of which occupies haloes in a unique way. The halo model can be used to gauge the effect of the presence of gas and stars,  which alter the matter power spectrum compared to the form it would have were gravity to be the only significant force in structure formation. Larger-scale effects ($0.5\iMpc \lesssim k \lesssim 10\iMpc$) on the power spectrum arise from redistributed gas,  mainly due to the Active Galactic Nuclei (AGN) expelling gas from halo centres\footnote{On smaller scales supernovae explosions can also contribute to baryon feedback.}.   Due to these effects, and it its intrinsic pressure, gas that remains bound to a halo may have a different profile from NFW.  Smaller-scale contributions ($k \gtrsim 10\iMpc$) to the power deviation arise primarily from stars clustering densely in halo cores \citep[see][for a review of feedback in cosmology]{Chisari2019b}.

% Approaches
Originally, \cite{White2004b} showed that the small-scale matter spectrum could be changed at the $\order{10\%}$ level by reasonable changes to the halo structure that could be calculated theoretically in a spherical-halo scenario with angular-momentum-conserving gas collapse. \cite*{Rudd2008} and \cite*{Zentner2008} show that a decrease in small-scale power was expected due to AGN activity, and that this could be captured by changing the concentration--mass relation in the NFW profile that enters the halo-model power spectrum calculation. The exact impact that feedback has on the matter spectrum is still uncertain \citep[\eg][]{vanDaalen2011, McCarthy2017, vanDaalen2020}, but is certainly at least $\order{10\%}$ for reasonable feedback scenarios. 

\cite{Semboloni2011, Semboloni2013, Fedeli2014a, Fedeli2014b} show that total-matter power spectra can be constructed by taking separate profiles for each component of the matter, and that the impact of AGN feedback could be captured if the gas content of haloes was assumed to be decreased. \cite{Mead2020} showed that this could be extended to modelling all combinations of auto/cross spectra that can be extracted from the hydrodynamic simulations. This modelling can then form the basis of effective models of feedback that attempt to model the response (or reaction) in power spectrum only \citep[\eg][]{Mead2020, Mead2021a}, thus circumventing the difficult issue of the general inaccuracy of the halo-model calculation. 

In approaches such as \cite{Mohammed2014a, Mohammed2014b, Sullivan2021} where the one-halo term is reduced to a series expansion, it has been shown that the series can be fitted to power spectra for a range of feedback scenarios with similar performance to the gravity-only case. \cite{Debackere2020} suggested that the mass-dependent halo baryon fraction could be measured using external data (\eg thermal or kinetic Sunyaev-Zeldovich or X-ray observations), and this could be used to provide an external constraint on the impact that feedback may have on the power. However, such arguments rely on the halo model providing a perfect mapping between the properties of haloes and their power spectra.

% Tracers other than matter-matter
How baryonic feedback alters the spectrum of tracers other than matter has not received significant attention.   In most cases, galaxy--galaxy lensing and galaxy clustering studies have limited themselves to large scales where the impact of baryon feedback and non-linear galaxy bias are small.  Although, feedback has been accounted for in studies that push to smaller scales by allowing for a variable halo-concentration amplitude (accounting for matter redistribution) and a separate concentration amplitude for the satellite galaxies \citep[see for example][]{Cacciato2013,Viola2015, vanUitert2016, Dvornik2018, Debackere2020,Dvornik2022,Amon2023}.

\subsection{Modelling the matter power spectrum}
\label{sec:modelling_matter}

Modelling the matter power spectrum is particularly useful for weak-lensing studies, where the lensing signal is sourced by the distribution of \emph{all} matter in the universe. However, it has long been recognised that the accuracy of the halo model prediction is poor compared to what is required by contemporary lensing data (see Fig.~\ref{fig:hm_comp}). This has led to several attempts to develop fitting functions specifically for the matter power.

% HALOFIT
\subsubsection{\halofit}
\label{sec:halofit}
Originally presented by \cite{Smith2003}, \halofit is a halo-model-inspired fitting function with $\sim 30$ free parameters that was fitted to \nbody simulation data. It does not use the halo model directly, but the power spectrum is broken down as the sum of a `quasi-linear' and `halo' term, which are analogues of the two- and one-halo terms. Further inspiration from the halo model is used in that the fitting functions are parameterised in terms of $\sigma(R)$ (and its derivatives), as opposed to random functions of the cosmological parameters. \halofit was updated in accuracy by \cite{Takahashi2012} and a prescription for massive neutrinos was added by \cite{Bird2012}. \halofit is accurate at around the $\sim5\%$ level for $k<10\iMpc$ and $z<2$ for a wide range of cosmologies. Note well that \halofit cannot be used to predict any spectra other than that of matter.

% HMcode
\subsubsection{\hmcode}
\label{sec:hmcode}
Originally presented by \cite{Mead2015b} and then updated by \cite{Mead2016} and \cite{Mead2021a}, \hmcode is a version of the halo model that has been augmented to produce accurate matter power spectra. While the backbone of the calculation is the vanilla halo model described in Subsection~\ref{sec:matter}, there are several additions and tweaks that were necessary in order to enhance accuracy. These tweaks ensure that the model pertains to a population of `effective haloes' the physical reality of which should not be taken too seriously. \hmcode is accurate at the $\sim2.5\%$ level for $k<10\iMpc$ and $z<2$ across a wide range of cosmologies. Note that \hmcode cannot be used to predict any spectra other than that of matter. It is also not obvious that the same tweaks that are required to provide accurate matter spectra would work, or would even be appropriate, if one wanted to extend the method to other tracers.
