\label{sec:summary}

%intro
We have presented a pedagogical review of the halo model and of its uses in analysing cosmological data.  The flexibility of this framework, and its ability to model non-linear scales, makes it very attractive for analysing data from a multitude of probes of large-scale cosmological structure, and to extract information from smaller scales where linear perturbation theory no longer applies.  

% summary of what was discussed

% basics and ingredients
We began with a derivation of the power spectrum and introduced the concept of two- and one-halo terms.  We continued by describing how the power spectrum is modified for discrete tracers of matter,  such as galaxies (Section~\ref{sec:basics}).  To make a prediction with the halo model one needs estimates of its ingredients: the abundance of haloes of different masses (halo mass function,  HMF),  the relation between the halo distribution and the underlying linear matter field (halo bias) and the distribution of matter or its tracers within the haloes (halo profile).  Aside from a few exceptions, these ingredients are measured and calibrated against simulations.  In Section~\ref{sec:ingredients} we focused on modelling matter power spectra and the ingredients needed to do so,  including the modelling of baryon feedback and versions of the halo model that are directly calibrated against simulations,  such as \halofit and \hmcode.    

% tracers
We then turned our focus to the modelling of tracers of matter in Section~\ref{sec:tracers}.  Once we know how a tracer populates the haloes we can predict its distribution using the same halo model formalism.  We began with galaxies and halo occupation distribution (HOD) and how central and satellite galaxies need to be treated separately.  Aside from galaxy positions,  large-scale structure impacts the intrinsic alignments of galaxies which can be modelled in a similar manner.  While galaxies are discrete tracers for matter,  other tracers, such as hot electrons in galaxy clusters, seen for example through the tSZ effect,  are diffused tracers.  We further discussed a few examples of such diffuse signals. 

%extesions to vanilla
The standard `vanilla' halo model makes simplifying assumptions,  such as that all haloes are spherical and that they are linearly biased with respect to matter.  In Section~\ref{sec:non_standard} we relax some of these assumptions and discuss what improvements are expected when adding more complexity to the base halo model formalism.  We note that some of these improvements are more important for two-point functions of specific tracer combinations. 

%altcosmo
While the majority of the review is focused on solutions within the standard \LCDM cosmologies,  in Section~\ref{sec:altcosmo} we show that the same approach can be applied to exotic cosmological models and extensions to \LCDM; for example,  models with massive neutrinos,  $f(R)$ gravity and dynamical or interacting dark energy models.  In these cases the spherical-collapse model can provide useful insights.  In particular,  we discuss that although the halo model can be inaccurate in predicting power spectra,  it can estimate the ratios of non-linear power spectra at percent level,  when the linear power spectra are matched between the two models.  The spherical-collapse solutions can then be applied to include the cosmology dependence of key inputs for the halo model.  The \react formalism, which we briefly introduce at the end of Section~\ref{sec:altcosmo}, takes full advantage of this property of the halo model. 

%software
Finally, we release a halo model code including \textsc{jupyter} notebooks for (almost) all figures in this paper and many demo examples: \pyhalomodel. In Section~\ref{sec:software} we also listed other halo model related codes that we are aware of.

%forward look
The use of the halo model for cosmological analysis has seen a recent boost \citep[for example][]{Miyatake2022b,Troester2022, Dvornik2023,Amon2023},  thanks to creative applications of corrections and adjustments to this flexible framework.  The halo model can be applied in a more analytic versus a more simulation based approach.  While simulation-based approaches can be more accurate, they are limited to the range of models that are simulated, and therefore lose some of the flexibility of the halo model.  Many recent analyses, however, opt for a half-way method that inherits the best of both worlds.  New methods of applying the halo model, for example \react or the inclusion of non-linear halo bias have the potential to address  most or all of the inaccuracies in the vanilla halo model predictions, while maintaining flexibility.  With these advancements,  we expect the halo model to be applied with more rigour to upcoming cosmological survey data,  which was the main motivation for writing this review.  Finally, we note that under the halo-model approach there is a pathway for uniting astrophysics and cosmology.