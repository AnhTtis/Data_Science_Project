\label{sec:altcosmo}

% Introduction
The halo model provides a framework for modelling the matter distribution and its tracers in a way that is not limited to a specific cosmological model. So far, we have assumed that the underlying cosmological model is standard \LCDM. However, the methods discussed can be applied to a variety of alternative cosmologies, as long as we have an estimate of their ingredients: the halo mass function, halo bias, halo profile (and HOD in the case of galaxies). 
The halo model has been used to make predictions for non-linear cosmological observables in a variety of non-standard scenarios. This is achieved either by using physically-motivated arguments to calibrate ingredients that were fitted to \LCDM to the new scenario, or else by running simulations and extracting the required ingredients (hopefully in a form that permits generalisation).  In Section~\ref{sec:ingredients} we saw that the ingredients are usually calibrated against simulations, with the exception of a few cases that use theoretical arguments.  

\subsection{Ingredients measured from simulations}
\label{sec:alt_ingridents}

Simulations have demonstrated that the density profile of dark-matter haloes follow a double power law profile across a wide range of cosmological scenarios.  This is perhaps due to the different modes of mass accretion throughout their life time,  where their initial steady mass accretion phase creates a single power law profile and their secondary violent merger phase flattens their profile in the inner regions \citep[\eg][]{Angulo2017}. Depending on the cosmology, however, different concentration, $c(M)$, relations are observed. These can be extracted from simulations, or somewhat understood via arguments that relate halo concentration to halo formation time \citep[][]{Baldi2010,Diemer2015, Correa2015, Ludlow2016, Diemer2019}. 

We can divide the alternative scenarios into several general categories, ranging from models that modify the properties of dark matter or dark energy, such as warm dark matter (WDM) or dynamical dark energy,  models where gravity is modified,  such as $f(R)$ gravity and scenarios where extra components are included, for example massive neutrinos. 

%WDM
A number of WDM simulations have been analysed with a focus on measuring halo-model ingredients \citep[for example][]{Schneider2012,Schneider2013,Lovell2014,Ludlow2016}.  WDM particles are expected to wash out structure at smaller scales relative to CDM due to their free streaming, with a characteristic scale dependence linked to the particle mass.  This washing-out effect is seen in the halo-mass function in \nbody simulations. For high-mass haloes, the abundance follows  \LCDM, while there are fewer low-mass haloes.  In addition, while massive haloes follow the same halo profile as in \LCDM, low-mass halos show a relatively steeper profile in their centre (concentration is non-monotonic with mass), suggesting that they have endured fewer mergers.   \cite{Smith2011b} and \cite{Viel2012} consider the halo model in WDM models,  and find that predictions are improved if a genuinely smooth component of the matter is assumed and accounted for in the bias and mass function (see Subsection~\ref{sec:smooth}).  There are indications of an enhanced linear halo bias from both simulations and theoretical work.  Adding all of these new components into a halo model approach, \cite{Schneider2012} find a 10\% agreement with the measured power spectra from simulations.  This may be further improved with the addition of non-linear halo bias (see Section~\ref{sec:non_linear_bias}).  Aside from WDM, other dark matter models have been explored in the literature, for example \cite{Simon2021, Dome2023} simulated fuzzy dark matter.  

%Dynamical dark energy models,  minimally coupled
Dynamical dark energy is perhaps the most well-known extension to \LCDM. As a result many simulations include one or two equation-of-state parameters that describe the density variation of dark energy with time \citep[for example][]{Heitmann2016} and can capture the behaviour of a wide range of such models, where dark energy is minimally coupled  \cite[see][for quintessence with Ratra-Peebles and SUGRA potentials]{DEUSS}. These models generally only affect the background cosmology,  but their impact can be seen in the growth of structures.  A dark energy with a negative equation-of-state parameter,  acts as an anti-gravity force emptying voids and creating a higher contrast cosmic web.  Simulations find a relative universality relation for the halo mass function, at the  $10\%$ level,  once the linear growth function is adjusted to include the impact of these dark energy models \citep{Courtin2011,Heitmann2016}. Other dark-energy models, such as clustering dark energy have also been explored \citep[see ][which focuses on cluster counts]{Batista2017}.

%Interacting dark energy-dark matter models
Dark energy may also interact with dark matter through an energy exchange which couples these fields.  The energy can flow either direction,  but most interesting are models were dark matter is transformed into dark energy,  as they can potentially explain the dominance of dark energy in the current epoch.  \cite{Baldi2010,Baldi2023} simulations show that, as expected, the energy transfer from dark matter to dark energy reduces the abundance of haloes as well as halo concentration, across all masses.   Such models have also been explored via spherical collapse in the literature,  finding qualitatively consistent results \citep{Wintergerst2010,Tarrant2012, Barros2019,Barros2020}.

%Modified Gravity
Alternative gravity models that were originally introduced to account for the origins of $\Lambda$, have more recently been explored to test the validity of general relativity at cosmological scales.  These models require a screening mechanism so that they reduce to the general relativity solutions at smaller scale and/or denser regions in the Universe.  Simulations for the more popular models exist today.  For example $f(R)$ has been simulated in \cite{Schmidt2009a,Li2012},  showing that the abundance of rare massive halos increases while their bias is decreased and their profile remains unchanged (this is preserved by the screening mechanism),  as a result power spectra are enhanced in a scale dependant manner. \citeauthor*{Dvali2000} (DGP; \citeyear{Dvali2000}) gravity was explored in \cite{Schmidt2010a, Schmidt2010b} who find qualitative agreement between spherical collapse and simulations.  \cite{Barreira2014} simulate two types of Galileon gravity and explored their impact on halo model ingredients.

%Massive neutrinos
Massive neutrinos are a form of (subdominant) hot dark matter that can wash out structures on larger scales compared to WDM.  The impact of massive neutrinos on structures has been studied in simulations assuming varying levels of model complexity, starting from background only effects to including neutrino particles and evolving them alongside the cold dark matter particles \citep{Brandbyge2008, Brandbyge2009,Viel2010, Agarwal2011, Bird2012,  Massara2014,  Upadhye2014}.  Since modelling the behaviour of the massive neutrino particles is challenging, these studies disagree on the best methodology and their associated accuracy. 

%hmcode
The impact of these models can be included in the halo model to reach percent-level agreement for power spectra when compared to simulation results.  \cite{Mead2016}, for example, models the effect of several alternative cosmologies within the \hmcode formalism: dynamical dark energy (calibrated against their own simulations),   coupled dark matter-dark energy (calibrated against \citealt{Baldi2010}),  massive neutrinos (calibrated against \citealt{Massara2014}), $f(R)$ gravity (calibrated against \citealt{Li2013}).  They find a few percent agreement in all cases, aside from the \cite{Vainshtein1972} screening scenario, which achieves a $10\%$ accuracy.   \cite{Dentler2022} also used the \hmcode formalism to set constraints on fuzzy dark matter from data.  Note that \hmcode only provides matter power, and cannot easily be generalised to other tracers. 

%Hydro sims
Most simulations of alternative models do not include baryonic effects such as AGN feedback. There are, however, a limited number of studies that have explored hydrodynamical simulations for alternative models.   \cite{Puchwein2013b} introduce a code for hydrodynamical simulation of modified gravity models and show some degeneracies on the power spectrum between the effects of baryonic feedback and $f(R)$ gravity.  The impact of $f(R)$ gravity on abundance and other properties of clusters is further investigated by \cite{Arnold2014}. Other authors have included baryons as a separate component in their simulations to account for the differing coupling mechanisms between baryons and other particles, although they do not account for the full hydrodynamics of baryons \citep[for example][]{Baldi2010,Hammami2015,Baldi2022}.

\subsection{Spherical collapse}
\label{sec:spherical_collapse}

% Introduction
The most accurate understanding of how structure formation proceeds in non-standard cosmologies derives from simulations. However, considering the collapse of a single spherical perturbation in an otherwise featureless universe, the so-called `spherical-collapse model', has provided valuable insights about the formation process, effective linear-collapse threshold, and the virial radius of collapsed haloes.

% Equations
The differential equation governing the collapse of a perturbation of uniform overdensity $\delta$ is
%
\begin{equation}
\ddot\delta+2H\dot\delta-\frac{4}{3}\frac{\dot\delta^2}{1+\delta}=\frac{3}{2}\Omega_\mathrm{m}(a)H^2\delta(1+\delta)\ ,
\label{eq:spherical_collapse}
\end{equation}
%
where dots denote time derivatives and it has been assumed that an initially uniform perturbation remains uniform throughout its evolution. $H$ is the (time-dependent) Hubble parameter and $\Omega_\mathrm{m}(a)$ is the (time-dependent) value of $\Omega_\mathrm{m}$. Usually (but not always\footnote{Some perturbations can be prevented from collapsing as dark energy, or similar, comes to dominate the expansion.}) solving equation~(\ref{eq:spherical_collapse}) for an initially small perturbation reveals that the perturbation grows, reaches a maximum size, and then collapses (due to the spherical symmetry this collapse is to an infinite-density singularity). Equation~(\ref{eq:spherical_collapse}) can be linearised to give
\begin{equation}
\ddot\delta+2H\dot\delta=\frac{3}{2}\Omega_\mathrm{m}(a)H^2\delta\ ,
\label{eq:linear_theory}
\end{equation}
which is the familiar differential equation governing the evolution of matter perturbations on sub-horizon scales. 

% delta_c
Equations~(\ref{eq:spherical_collapse}) and (\ref{eq:linear_theory}) can be solved in tandem, for the same initial condition, to reveal what value linear theory would predict when the non-linear theory reaches collapse. In an Einsten--de Sitter universe this value is the oft-quoted $\delta_\mathrm{c}\simeq1.686$, but this value has some cosmology dependence, for example in an $\Omega_\mathrm{m}=0.3$ \LCDM universe $\delta_\mathrm{c}\simeq1.676$, lower by $\sim 0.5\%$ \citep[\eg][]{Nakamura1997}.  As we discussed at the end of Section~\ref{sec:halo_mass_function},  including the cosmology dependence of $\delta_\mathrm{c}$ can improve the mass-function universality and halo-model predictions. 

% Delta_v
Of course, real perturbations are not perfectly spherical, and even if they were, inhomogeneity in the true gravitational field would distort them from this idealised form. No perturbation will collapse to a singularity but instead will equilibrate into a roughly spherical structure with a density gradient and with kinetic and potential energy split as per the virial theorem (a virialized halo). If one takes the point of halo formation as given by the time the spherical-model would predict a singularity to form, one can then derive a rough estimate for the density of the resulting structure by applying the virial theorem at the point of collapse to derive a radius. In an Einstein--de Sitter universe this gives the result that the halo overdensity should be $\Delta_\mathrm{v}\simeq178$, independent of halo mass. This is the origin of the $\sim 200$ overdensity criterion that is often used when identifying haloes in simulations. In an $\Omega_\mathrm{m}=0.3$ model $\Delta_\mathrm{v}\simeq 330$ or $\simeq310$ depending on whether the contribution of dark energy is included within the virial theorem or not.  Fitting functions for the cosmology dependence of $\Delta_\mathrm{v}$ can be found in \citeauthor{Bryan1998} (\citeyear{Bryan1998}; \LCDM) and \citeauthor{Mead2017} (\citeyear{Mead2017}; homogeneous dark energy, ignoring dark-energy contributions to virialization).

% Alternative cosmologies
Analogues to equation~(\ref{eq:spherical_collapse}) can also be derived and solved in any fully-specified cosmological model.  This is done for dark energy models \citep{Mota2004, Bartelmann2006, Abramo2007, Pace2010, Wintergerst2010, Mead2017, Barros2020},  but care must be taken with dark energy perturbations which can be important if its sound speed is smaller than speed of light \citep{Batista2021,Batista2023}. In early dark energy scenarios the initial conditions for the equations need to be set in the correct linear growing mode. In coupled dark energy--dark matter models the coupling needs to be incorporated, as well as the back reaction of this on the local dark energy. Spherical collapse in massive-neutrino cosmologies was investigated by \cite{LoVerde2014}. In viable modified gravity models \citep[\eg][]{Brax2010, Schmidt2010a, Schmidt2010b, Borisov2012, Lombriser2013a, Kopp2013, Barreira2013, Taddei2014} screening mechanisms must be incorporated. Spherical collapse in generalised dark matter has been investigated by \cite{Pace2020}.

\subsection{\react}
\label{sec:react}

While the raw halo model predictions for the matter spectrum have been shown to be inaccurate at the $\sim 10\%-30\%$ level \citep[\eg][]{Mead2015b}, it has been demonstrated that the ratio of halo model predictions for different cosmologies can more accurately predict the same ratio measured in simulations \citep[\eg][]{Schmidt2010a, Schmidt2010b, Mead2017,Gupta2023}. This is particularly true if the two cosmologies are chosen such that they share a linear spectrum in both shape and amplitude, and if the spherical-collapse model is used to include the cosmology dependence of $\delta_\mathrm{c}$ and $\Delta_\mathrm{v}$. 

Originally presented by \cite{Cataneo2019}, the \react method uses this insight to produce accurate spectra for dynamical dark energy and modified gravity models ($f(R)$ and DGP), but also incorporates perturbation theory.  Within the \react formalism a pseudo power spectrum needs to be defined, which has the same shape and amplitude as the linear power spectrum in the target cosmology,
%
\begin{equation}
P^{\rm psuedo}_{\rm linear}(k) = P^{\rm target}_{\rm linear}(k)\;.
\end{equation}
%
Although there may not be a match for the pseudo linear power spectrum under the flat-\LCDM model,  its evolution to non-linear power spectrum assumes a flat-\LCDM model without neutrinos.  This evolution can be calculated using \hmcode or emulators \citep[\eg][]{Giblin2019}.  The non-linear power spectrum is then constructed via
%
\begin{equation}
P^{\rm target}_{\rm non-linear}= R(k) P^{\rm psuedo}_{\rm non-linear}(k)\;,
\end{equation}
where $R(k)$ is the `reaction'.  To predict the reaction a combined halo model and perturbation theory approach is employed \citep[see][for a concise review]{Cataneo2022}.  Note that \react currently only targets the matter spectra and only predicts the reaction of the power spectrum to a particular ingredient, and it therefore requires accurate non-linear spectra for \LCDM as an ingredient.

Updates to \react have been presented by \citeauthor{Cataneo2020} (\citeyear{Cataneo2020}; massive neutrinos), \citeauthor{Bose2020} (\citeyear{Bose2020}; modified gravity), and \citeauthor{Bose2021} (\citeyear{Bose2021}; massive neutrinos and baryonic feedback).  \cite{Srinivasan2021} have applied the reaction methodology to time dependent $\mu$,  which captures the phenomenology of a range of modified gravity models. The \react formalism was applied to cosmological analysis of the KiDS weak lensing data to set constraints on $f(R)$ gravity \citep{Troester2021}.
