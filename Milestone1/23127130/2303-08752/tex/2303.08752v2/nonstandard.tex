\label{sec:non_standard}

% Comparison of halo model and Hmcode to the DarkEmu
\begin{figure}
\begin{center}
\includegraphics[width=\columnwidth]{plots/hm_comp.pdf}
\end{center}
\caption{The performance of the analytical halo model calculation compared to state-of-the-art semi-analytical calculations from \darkemu at $z=0$ for matter, galaxy and matter-galaxy power spectra.  \darkemu is accurate at the $4\%$ level relative to simulations (shaded region).  The upper panel shows the total $P(k)$ (solid) and the one-halo (dotted) and two-halo (dot-dashed) terms.  The lower panel shows the ratio between the halo model and the \darkemu power spectra. The vanilla halo model predictions (solid curves) show errors of between  $\sim10\%$ and  $\sim15\%$ with respect to the reference model.  When the non-linear halo bias,  calculated using the same emulator, is included (dashed lines) the error is reduced to $<10\%$ for intermediate scales.  Including this effect in the halo model works best for the galaxy power spectrum bringing its  error to within  $\sim5\%$ of the reference model.  This is because the dark emulator can only estimate the halo-halo power spectrum for more massive haloes, which host most of the galaxies that contribute to the signal.  When modelling matter or matter-galaxy power spectra,  an estimate of the clustering of lower mass haloes is also needed.  As a result we fail to achieve the same level of accuracy for these power spectra.  The \hmcode prediction on the other hand provides a good agreement with the \darkemu matter power spectrum.  \hmcode is accurate at the $2.5\%$ level compared to the simulations it was calibrated against.}
\label{fig:hm_comp}
\end{figure}

% Introduction and figure
The standard derivation outlined in Section~\ref{sec:derivation}, its extension to discrete tracers outlined in Section \ref{sec:discrete_tracers} and sets of ingredients like those mentioned in Section~\ref{sec:ingredients}, constitutes most of what is commonly called `the halo model' in the literature. In Fig.~\ref{fig:hm_comp} we demonstrate the performance of the standard calculation (\LCDM; $z=0$; $\Delta_\mathrm{h}=200$; \citealt{Tinker2010} mass function and bias; \citealt{Duffy2008} concentration--mass relation; \citealt{Zheng2005} satellite-dominated HOD with $M_\mathrm{min}=10^{13}\Msun$; $\sigma=0.5$; $M_0=M_\mathrm{min}$; $M_1=10^{14}\Msun$; $\alpha=1$) compared to state-of-the-art semi-analytical theoretical calculations for matter, galaxy and matter-galaxy power from the \darkemu \citep{Nishimichi2019}, which is accurate at the $\sim4\%$ level relative to simulations.  We see that the analytical calculation is in error at the $\sim15\%$ level, with a similarly-shaped residual for each power spectra. 
At $k\sim0.03\iMpc$ we see good agreement, where differences between the halo model and \darkemu originate from differences in linear halo bias.  At $k\sim0.1\iMpc$ we see the standard halo models \emph{over} predict the power, which is due to perturbative effects suppressing the true power at these scales, but this not being included in the halo model. Including the non-linear halo bias ameliorates this because the non-linear halo clustering contains these non-linear effects. At $k\sim0.7\iMpc$ all halo models under-predict the power, which is somewhat corrected once the non-linear halo bias is added.  For $k>2\iMpc$ the match is quite good, but degrades at smaller scales where the shapes of the halo profiles in the one-halo term become important. For the matter power we also compare  \hmcode \citep{Mead2021a} with  \darkemu and find them in good agreement.  \hmcode is accurate at the $\sim2.5\%$ level relative to simulations. 

% 
To improve these discrepancies, many authors have considered extensions to the standard model, and we discuss some of these here. These extensions will usually require some new ingredient to be calibrated from simulations. The following subsections here are loosely ordered in terms of the scale they have an impact on, from largest to smallest. Some of the non-standard approaches only apply to power spectra for a specific tracer combination, while others offer more general potential improvements.

% Smooth matter
\subsection{Smooth matter}
\label{sec:smooth}

% Introduction
When a halo finder is run on \nbody simulation output, it commonly allocates $\sim$half of the simulated mass to haloes. The remaining half is either in structures of low particle number (a threshold of $\sim200$ particles is often adopted) or else looks to be approximately smoothly distributed in the inter-halo medium. If the simulation is re-run with more \nbody particles low-mass haloes are often identified in regions that were previously thought to be devoid of haloes. It remains an unsolved problem in cosmology to determine whether all mass in the Universe is bound in haloes of ever lower masses, although there are compelling theoretical arguments \citep[\eg][]{Press1974, Bond1991}, and simulations suggest that this should be so \citep[\eg][]{Angulo2017, Wang2020}, at least for perfectly cold dark matter particles\footnote{ Although note that any realistic particle-based dark-matter model has at least some streaming velocities, which translates to some unbound particles}.

% Some references
Halo mass functions are fitted to data from \emph{measured} haloes, above some mass threshold, and sometimes the fitting forms are constrained such that all mass in the universe \emph{is} contained in haloes if the fitting function is interpreted literally \citep[\eg][]{Sheth1999}. Some authors have considered how a genuinely smooth component of matter could be added to the halo model, either in the context of WDM \citep{Smith2011b}, or in the context of gas expelled from haloes by AGN feedback \citep{Fedeli2014a, Debackere2020, Mead2020}. In all cases the smooth matter is taken to be distributed as per the linear matter perturbations, via a biased linear power spectrum. In the case of matter spectra, the bias of the smooth component is constrained by the fact that an unbiased linear spectrum must be returned at large scales.

% N-body experiements
\cite{vanDaalen2015} used \nbody experiments to test the contribution of smooth (or non-halo) matter to the overall matter power spectrum. If only matter in FoF haloes (sometimes called groups) is used, then essentially all power for $k>3\iMpc$ can be accounted for, but if SO haloes are used then this drops to between $80$ and $95\%$ of the power, depending on the overdensity threshold. In the SO case, presumably the remaining power arises from dense material just outside the (artificial) spherical halo boundary, which is contained within FoF groups. It is less clear how this extra-halo material contributes at intermediate scales, or if is better to think of it as genuinely smooth matter, or matter that would resolve itself into low-mass haloes were the simulation to be run with higher resolution.

% Voids
\subsection{Voids}
\label{sec:voids}

% Single paragraph
In the canonical (linear halo bias) halo model, only `linear' voids will be present, which enter through the lack of  haloes in low-density regions. However, it is clear from visual inspection of the particle distribution in \nbody simulations that the distribution of voids in the Universe is not a Gaussian random field in detail. \cite*{Voivodic2020} create a version of the halo model that explicitly accounts for voids and which therefore requires a void abundance, linear bias and profile to be specified. Voids are essentially considered in the same way as haloes, but with under-density profiles rather than over-density profiles. This then allows for the calculation of correlations such as halo--void, void--void etc. In addition, `dust' (non-halo, non-void) matter is also accounted for in the same manner as described in Subsection~\ref{sec:smooth}. Because voids are physically larger than haloes, their `one-halo' (one-void) term peaks at larger scales than the standard one-halo term, and therefore adds some (much-needed) power into the transition region of the matter power spectrum. Note that in the non-linear halo bias approach advocated in Section~\ref{sec:non_linear_bias}, the two-point contribution voids will be included in the $\beta^\mathrm{nl}$ function, since voids are really a spatially-coherent lack of haloes, and it is the two-point function of this spatial coherence that is captured by $\beta^\mathrm{nl}$.

% Perturbation theory
\subsection{Perturbation theory}
\label{sec:perturbation}

% Single paragraph
To go beyond linear halo bias and linear power \cite{Valageas2011, Mohammed2014a, Seljak2015, Schmidt2016, Hand2017, Philcox2020} have all considered replacing the linear theory spectrum with something higher-order that can, in principle, be calculated via perturbation theory. These approaches have been demonstrated to improve the halo-model prediction for the matter spectrum in the quasi-linear regime, but it is not obvious how to translate their results to power spectra of tracers other than the matter -- if linear bias is employed the problem indicated in Fig.~\ref{fig:Rhh} remains. \cite{Smith2007} jointly consider a perturbative bias expansion with perturbation theory for the matter, getting good results in the mildly non-linear regime. Recently, \cite*{Sullivan2021} have looked at biased tracers together with a perturbative model at large scales, with an exclusion term incorporated for halo and galaxy clustering. However, the bias model is still effectively linear and fitting parameters mask the underlying physical relationship between halo, matter, and galaxy clustering.

% TODO: Weave
\cite{Valageas2011} combined perturbation theory with the halo model to improve accuracy on quasi-linear scales.
Other schemes of which we are aware that are halo-model inspired and that can be used for matter spectra are the halo-Zel'dovich perturbation theory of \cite{Mohammed2014a}, \cite{Seljak2015}, and \cite{Sullivan2021}. Here the two-halo term is modelled using the \cite{Zeldovich1970} approximation, while a series expansion is used for the one-halo term with halo-model insight used to constrain the allowed terms in the series. Finally, there is the effective halo model of \cite{Philcox2020}, which uses the effective field theory of large-scale structure for a two-halo term combined with a standard one-halo term.

% Halo compensation
\subsection{Excess large-scale power}
\label{sec:excess_large_scale_power}

% k -> 0 one-halo problem
On large scales, the Fourier halo profiles $\hat{W}_{u}(M,k\to0) \propto k^0$ (constant) and therefore the same is true of the scale-dependence of the one-halo term (equation~\ref{eq:one_halo_term}): $P_{uv}^\mathrm{1h}(k\to0) \propto k^0$, a shot-noise-like contribution. In contrast, in the same limit the two-halo term (equation~\ref{eq:two_halo_term}) always has the linear spectrum shape at large scales: $P_{uv}^\mathrm{2h}(k\to0) \propto k^{n_s}$. It is then inevitable that the one-halo term has more power than the two-halo term at some sufficiently large scale, with the exact scale being governed by the redshift and the fields $u$ and $v$. This problem manifests differently in configuration space, since terms $\propto k^0$ contribute only to the $r=0$ part of the correlation function. However, this can cause the integral over the correlation function to be non-zero when $r=0$ is included, contrary to the requirement that the mean overdensity be zero.

% P(k) problems
The standard halo model thus predicts that the total power spectrum transitions from being dominated by the \emph{one-halo} term at ultra-large scales (typically $k\sim10^{-4}\iMpc$), to two-halo power at large scales ($k\sim10^{-2}$) and back to one-halo power at small scales ($k\sim1\iMpc$). For the autospectrum of a discrete tracer this might be the correct manifestation of shot noise dominating the signal at very large scales \citep[see for example][]{Seljak2009}.  For matter--matter, or matter--galaxies this is wrong \citep[\eg][]{Ginzburg2017}, since we would expect intra-halo power to dominate only at small scales, leaving only inter-halo power at large scales\footnote{In practice this rarely affects halo-model predictions since the scales on which this problem is manifest are so large.}. This problem can be seen at the largest scales shown in Fig.~\ref{fig:power}.  Some authors have suggested corrections to halo exclusion (see section~\ref{sec:exclusion},  \citealt{Smith2011a, Schneider2023}) or the interpretation of the 1-halo term \citep{Valageas2011} as possible solutions to this problem. 

% Explanation of problem and ad-hoc solutions
\cite{Zeldovich1970} noted that any local process that satisfies mass and momentum conservation must have a large-scale spectrum that decreases faster than $k^4$. Galaxies do not obey these conservation laws, so can have a shot-noise contribution at large scales. However, matter should obey these conservation laws, but they have not been imposed in the standard formulation of the halo model, thus leading to the large-scale breakage \citep{Seljak2000}. Indeed, any shot-noise contribution to the matter one-halo term at large scales is \emph{not} observed in simulations \citep{Crocce2008}. In the effective models of \cite{Mohammed2014a, Seljak2015, Hand2017, Sullivan2021} an ad-hoc `compensation kernel' multiplies the standard one-halo term to enforce mass conservation, $k^2$, and to ensure that predictions from perturbation theory are recovered on large scales. $k^2$ scaling is sufficient to make the one-halo term subdominant at large scales, and it was found that enforcing momentum conservation spoiled predictions in the transition region. Presumably $k^4$ scaling could be enforced with a different form of the compensation kernel if necessary.

% Theoretical solutions
\cite{Valageas2011} demonstrated that mass conservation could be imposed in the halo model if a Lagrangian framework was adopted from the outset, resulting in a one-halo term that decays like $k^2$ at large scales, while \cite{Schmidt2016} enforced mass and momentum conservation and was able to get $k^4$ scaling. \cite{Hamaus2010} showed that the large-scale power of massive haloes is sub-Poisson. \cite{Ginzburg2017} was able to incorporate this within the halo model using a halo-fluctuation field and the general bias expansion.

\subsection{Halo compensation}
\label{sec:compensation}

% Compensated profiles
\cite{Cooray2002} suggest using `compensated' matter profiles as a way of avoiding excess large-scale power in matter spectra. These profiles are compensated in the sense that they have both a positive and a negative overdensity that exactly cancel, in contrast to standard profiles that only have positive overdensity. For a compensated matter profile $\hat{W}_{m}(M,k\to0) \propto k^2$ (rather than constant) and therefore $P_\mathrm{mm}^\mathrm{1h}(k\to0) \propto k^4$, in line with the expectation from mass and momentum conservation \citep[\eg][]{Smith2003}. This occurs naturally with compensated profiles, since each is massless and therefore placing and moving these perturbations incurs no mass or momentum cost. However, the compensation also affects the halo-profile that appears in the two-halo term, which therefore also removes the large-scale, linear power from the halo model: $P_\mathrm{mm}^\mathrm{2h}(k\to0) \propto k^{4+n_s}$ rather than $k^{n_s}$; the baby is thrown out with the bath water. This can be understood since compensated profiles have no overall density perturbation when smoothed on scales much larger than their radii. The two-halo term then just re-arranges these zero-density perturbation features, and when smoothed on a sufficiently large scale this will represent an unperturbed density field. A simple fix for this was suggested by \cite{Chen2020, Chen2022}, who view the density field as a sum of the linear perturbations and the haloes\footnote{In the standard halo model the haloes themselves \emph{are} the linear density field.}, if these haloes are compensated then this two-halo term returns to linear form at large scales.

% Halo exclusion
\subsection{Halo exclusion}
\label{sec:exclusion}

% Problem
One non-linear feature of the halo spectrum that has received considerable attention in the literature is that of `halo exclusion'  \citep[\eg][]{Takada2003, Zehavi2004, Tinker2005, Smith2007, Schneider2012, vandenBosch2013, Baldauf2013}: the fact that haloes have finite radii means that the halo correlation must drop to zero on scales below the radii of the two haloes in question. The details of how this operates depends on the exact definition of haloes \citep[\eg][]{Garcia2019}.  For example, different algorithms may choose to discard halo centres if they are too close.  The eventual masses of haloes may include double counted particles that lie within the boundary of two proximate haloes. 

Exclusion is automatically included if using $P_\mathrm{hh}$ or $\Bnl$ calibrated from simulations \citep[\eg][]{Nishimichi2019, Mead2021b}, but must be incorporated by hand if using a simpler prescription for the two-halo term (\eg linear halo bias).  The scale at which halo exclusion becomes relevant depends on halo size and, by proxy, halo mass.  It will kick in at larger scales for more massive haloes compared to those less massive, and consequently smaller. Since exclusion affects the two-halo term at small scales, it is a subdominant contribution to the total power spectrum for some tracer combinations (\eg matter--matter), but is important for others (\eg galaxy--halo).

% Solutions
Halo exclusion can be approximately included in Fourier space by multiplying the halo power spectrum by some suppressing window function \citep[\eg][]{Schneider2013}. However, most often halo exclusion is included by fixing the halo correlation function to zero below some radius. This strongly suppresses the two-halo term on scales around the halo radius, but ignores the fact that one might expect a more gentle truncation for scales near the halo boundary.  

To allow for a smoother transition, \cite{Zehavi2004} and \cite{Tinker2005} proposed to alter the integration upper limits in equation~(\ref{eq:two_halo_term}) to be dependent on the halo virial radii and the scale in question, in such a way that the halo-halo correlation is fixed to zero if it arises from haloes within some distance of each other: if $r< r_{\mathrm{h},1}+r_{\mathrm{h},2}$. A similar scheme was followed by \cite{Smith2011a}, and this could be expanded to any other function of the halo virial radii. Unfortunately, this makes the halo-model prediction for the overall number density of galaxies deviate from the true value, so the end result needs to be corrected by hand to account for this. In addition, this constraint must be imposed in configuration space, which can make it numerically inefficient if working in Fourier space \citep[][]{Murray2020}.

% Halo asphericity
\subsection{Halo asphericity and substructure}
\label{sec:asphericity}

% Problem
The standard halo model takes haloes to be perfect spheres with a hard boundary. However, the reality is that haloes have no clear boundary and are generally triaxial (aspherical) with a typical minor-to-major axis ratio of $\sim0.6$ \citep{Jing2002}, furthermore this halo asphericity may be spatially correlated due to gravitational tidal fields. This also applies to galaxies (or other fields) that may exist within haloes, with the galaxy--galaxy and galaxy--matter intrinsic alignments being a particularly important contaminant in weak gravitational lensing \citep[\eg][]{hirata/seljak:2004}. The lowest-order clustering statistics average over all directions, and so a spherical approximation for the halo profile may be reasonable, but recall that $\average{W_u W_v}$ appears in the one-halo term, and therefore any scatter (or covariance) about the mean profile will contribute power. Particular configurations of higher-order statistics are likely to be more sensitive to profile shape than two-point statistics \citep[\eg][]{Takada2003}.

% Smith & Watts theoretical basis
\cite{Smith2005} were the first to lay the theoretical foundations for including asphericty within the halo model. A triaxial profile is required, with distribution functions for the axis ratios, these profiles are then integrated over all orientations in the one- and two-halo terms. An alignment correlation function is required to capture spatial alignments and appears in the two-halo term. In \cite{Smith2005}, halo-model matter power spectra were found to be lower at the $\sim5\%$ level for $k \sim 1$ -- $10\iMpc$ relative to a model with no triaxiality, with the difference mainly driven by high-mass haloes. An alternative approach to environmental dependence is presented by \cite*{Gil-Marin2011} where galaxies are split by environment and which therefore allows cross-correlations between different populations of galaxies to be computed. Tangentially related to these methods are papers that use the halo model to estimate the intrinsic alignment signal \citep[\eg][]{Schneider2010, Fortuna2021} for weak lensing. In these cases, inter-halo intrinsic alignment is driven by large-scale tidal alignment, whereas intra-halo tides align halo satellite galaxies with the central galaxy.

% Substructure
Realistic haloes contain clumpy substructure, but this is ignored when a smooth halo profile is adopted for use in a halo-model calculation. Substructure can be included by breaking the halo into smooth and clumpy components and including a substructure mass function \citep[\eg][]{Sheth2003, Dolney2004, Giocoli2010}. Halo models with substructure generally predict a strong increase in power at small scales ($k\gtrsim10\iMpc$) relative to a model with a smooth halo. Additionally, matter can be broken down into that in clumps and that in the smooth halo, and it is then possible to calculate cross correlations between the clumps (or galaxies therein) and the surrounding matter, which may be more appropriate for galaxy--galaxy lensing.

% N-body experiments
The impact of halo alignment can also be investigated using simulations: Halo particles (or galaxies) can be randomly rotated about the halo centre to artificially `spherize' the halo; alternatively, individual haloes can be coherently, but randomly, rotated to break intrinsic correlations in the cosmic web. Such numerical experiments were performed by \cite*{vanDaalen2012} for the case of galaxy clustering, where breaking intrinsic alignments was shown to decrease the power spectrum by $\sim2\%$ at $0.1\lesssim k/\iMpc \lesssim1$ and breaking alignments within individual haloes was shown to decrease the power, starting with a $1\%$ level effect at $k\gtrsim0.2\iMpc$ and reaching a more dramatic $\sim20\%$ for $k\simeq10\iMpc$. These are quite significant effects on the power, but clearly the overall impact of these effects will depend on the galaxy sample. To investigate the impact of substructure on the matter power spectrum \cite{Pace2015} performed a similar experiments to \cite{vanDaalen2012}, but for matter profiles, showing that power at $k\gtrsim0.3\iMpc$ was subdued at the $\gtrsim 0.5\%$ level for coherent rotation and $\gtrsim 1\%$ level for spherizing rotation (which removes halo substructure). It was also demonstrated that the two types of rotations have similar ($\gtrsim 1\%$ for $k\gtrsim 0.3\iMpc$) level effects on the cross spectrum between those particles inside and outside of haloes (a proxy for the substructure--matter cross spectrum). These differences increase at smaller scales and are likely to be strongly dependent on the combination of tracers considered. Note well that halo profiles measured from simulations are often subject to the `spherizing' technique discussed in this paragraph if they are measured from the stacked profiles.

% Scatter in halo properties
\subsection{Scatter in halo properties and assembly bias}
\label{sec:scatter}

% Single paragraph
When matter profiles are measured from simulations a significant scatter is observed in the relation between halo concentration and mass. For concentration measured for spherical NFW haloes, this scatter is approximately log-normal with $\sigma_{\ln c}\sim 0.3$ \citep[\eg][]{Jing2000, Bullock2001}. The scatter may contain cosmological information if it relates to the scatter in halo-formation times, which itself depends on the power spectrum and can be estimated using extended-Press-Schechter theory \citep[\eg][]{Bond1991}. 

However, this scatter is often ignored in the halo model, and  instead the mean halo profile and the mean concentration--mass relation are used. This is not correct in detail because it is the expectation value of the squared halo profile that appears in the one-halo term (equation~\ref{eq:one_halo_term}). In the limit that this scatter is not spatially correlated, its impact may be accounted for by introducing a second integral within equation~(\ref{eq:one_halo_term}) over the probability distribution of halo concentrations \citep[\eg][]{Cooray2001, Dolney2004, Giocoli2010}. 

Realistic scatter boosts the matter power at the few per-cent level for $k\gtrsim 10\iMpc$, but may have a larger impact for other spectra or higher-order spectra. It may be that the effect of this can be captured by using the standard deviation of the halo concentration relation, rather than the mean or median. In the spatially-correlated case this scatter is difficult to account for (although the methods of \citealt{Smith2005} could be employed), and falls under the general heading of `assembly bias', requiring its own power spectrum. In contrast, for galaxy power spectra the spatially-uncorrelated scatter in halo-occupation number is usually accounted for (via the variance that emerges from the expectation values in equations~\ref{eq:one_halo_central_central} and \ref{eq:one_halo_satellite_satellite}). If a cross spectrum is calculated there is also a possible covariance between the halo profiles of the two tracers: \cite{Koukoufilippas2020} consider this in the tSZ--galaxy cross correlation as a mass-independent term that then affects the amplitude of the one-halo term. More complicated would be to consider the profile--structure covariance as a function of halo mass.  Finally, we note that for SO haloes there will be no scatter in halo boundary as this is fixed by the halo mass. However, the realistic density field contains triaxial structures and the scatter in the `boundary' of these structures may well contribute additional power in a realistic calculation that accounted for non-spherical haloes.
