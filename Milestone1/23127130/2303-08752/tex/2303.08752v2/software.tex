\label{sec:software}

% Advert for the code included with this review
Together with this paper, we provide a \textsc{python} package that can be used to implement the calculations described here: \pyhalomodel \footnote{\link{https://github.com/alexander-mead/pyhalomodel}}. Using the same software we have also written a pure-\python implementation of \href{https://github.com/alexander-mead/HMcode-python}{\sc hmcode}\footnote{\link{https://github.com/alexander-mead/HMcode-python}}, although this has not been used to make figures for this paper. There are several other software packages that are publicly available and that we have investigated while writing this review: 
\begin{itemize}

% Dark emulator
\item The \textsc{dark emulator}\footnote{\link{https://darkquestcosmology.github.io/}} of \cite{darkemu} emulates the halo mass function and power spectrum, and has the capacity to perform galaxy--galaxy and galaxy--matter halo-model calculations on top of the emulated quantities. Uniquely, because the halo power spectrum is emulated, the non-linear halo bias (called $\Bnl$ in this paper) is automatically included.

% halomod
\item Recently, the \halomod\footnote{\link{https://github.com/halomod/halomod}} package of \cite{halomod} has been released, which can be used as either a \python package or as a browser application\footnote{\link{https://thehalomod.app}}. Currently, this can be used for matter and galaxy power spectra, but the code is extendable to other tracers in principle.

% Colossus
\item \textsc{colossus}\footnote{\link{https://bdiemer.bitbucket.io/colossus/index.html}} \citep{Diemer2018} is a \python toolkit for calculations pertaining to cosmology, the large-scale structure of the universe, and the properties of dark matter haloes. It does not perform halo-model calculations, but does provide the necessary ingredients from a variety of different sources.

% CCL
\item The LSST Core Cosmology Library \citep[\textsc{ccl}\footnote{\link{https://github.com/LSSTDESC/CCL}};][]{CCL} contains a standard halo model calculator written in \python.

% HMcode
\item The \hmcode\footnote{\link{https://github.com/alexander-mead/HMcode}} software, developed by \citep{HMcode} performs a version of the standard halo model calculation for the matter power spectrum. The calculation is augmented for enhanced accuracy. The original source code is written in \textsc{fortran}, but a \python wrapper\footnote{\link{https://pypi.org/project/pyhmcode/}} around the \textsc{fortran} is also available. \hmcode is also included within \camb\footnote{\link{camb.readthedocs.io}}.


\end{itemize}

% Other software that I don't really know about (also CHOMP)
Other public codes that we are aware of, but that we have not had a chance to thoroughly investigate are \textsc{halogen}\footnote{\link{https://github.com/EmmanuelSchaan/HaloGen}} \citep[][]{halogen},  \textsc{aum}\footnote{\link{https://github.com/surhudm/aum}} \citep{More2015a} and \textsc{class\textunderscore sz} \footnote{\link{https://github.com/CLASS-SZ/class\textunderscore sz}}  \citep{Bolliet2018, Bolliet2023}.
