%Basics of the halo model 
\label{sec:basics}

% Introductory figure
\begin{figure*}
\begin{center}
\includegraphics{plots/halo_model.png}
\end{center}
\caption{A schematic visualisation of the halo-model process. The left-hand panel shows the matter density field in a $25\times25\times5\,(h^{-1}\mathrm{Mpc})^3$ region of an \nbody simulation, centred on a massive ($\sim10^{14.5}\Msun$) halo identified at $z=0$. The central panel shows the result of isolating all haloes identified in the simulation and replacing these with idealised spherical haloes of the same mass. The right-hand panel shows the result of populating these haloes with galaxies according to a simple galaxy-occupation prescription.}
\label{fig:visualisation}
\end{figure*}

At the core of the halo model is the approximation that we can fully describe the complex structure of the cosmic web simply as a sum of its individual components: dark matter, gas, and galaxies, all distributed in haloes. The complexity then translates into the problem of how to model the profile, mass distribution and bias of these haloes relative to the underlying (linear-theory) matter distribution, and how to accurately populate haloes with gas and galaxies. This `mapping information' can be motivated by, or extracted from, numerical simulations, leading to a flexible analytical model of the statistical properties of the cosmic web. By confronting this model with observations, the halo-model approach can provide constraints on the underlying cosmology of the Universe, in addition to providing unique insight into the halo--galaxy connection, and furthering our understanding of galaxy formation and evolution. A schematic of the halo-model approach is shown in Fig.~\ref{fig:visualisation}, where the first panel shows the density field in an \nbody simulation, the second panel shows this replaced by a spherical halo approximation, and the final panel shows a possible galaxy distribution.

\onecolumngrid
\hrulefill

\subsection{Standard derivation}
\label{sec:derivation}

We start by defining a field in real space, $\theta_{\rm u}(\bold{x})$, where  $\bold{x}$ is the three-dimensional comoving position and the label ${\rm u}$ stands for the field we are interested in modelling. Fields are also a function of time, usually parameterised via $z$, but we suppress this argument here and throughout this paper to make the notation less cluttered. Examples of such fields would be `matter', `halo' or `galaxy' over-densities that vary from place-to-place in the Universe.  
We make the assumption that everything in our field is contained within haloes distributed throughout the space with a spherically symmetric profile, $W_{{\rm u},i}$, centred at position $\bold{x}_i$, such that
%
\begin{equation}
\theta_{\rm u}(\mathbf{x})=\sum_i N_i\;W_{{\rm u},i}(|\mathbf{x}-\mathbf{x}_i|)\;,
\label{eq:theta_haloes}
\end{equation}
%
where the sum runs over all volume elements and $N_i=\lbrace0,1\rbrace$ determines whether there is a halo centre in that volume element.  
The Fourier transform of the field, in terms of comoving wavenumber $\bold{k}$, is given by
%
\begin{equation}
\hat{\theta}_{\rm u}(\bold{k})=\ft{\theta_{\rm u}(\bold{x})}{\bold{k}}{\bold{x}}\ ,
\end{equation}
%
With the variable change $\mathbf{r}=\bold{x}-\bold{x}_i$,
%
\begin{align}
\label{eq:ffield}
\hat{\theta}_{\rm u}(\bold{k})=\sum_i\mathrm{e}^{-{\rm i}\dotp{k}{x}_i}\; N_i \int  W_{{\rm u},i}(|\bold{r}|) \mathrm{e}^{-{\rm i} |\bold{k}||\bold{r}|\cos\theta} d^3 \bold{r} 
= \sum_i\mathrm{e}^{-{\rm i}\dotp{k}{x}_i} N_i\; \hat{W}_{\mathrm{u},i}(k)\ ,
\end{align}
%
where we have recognised the integral as the Fourier transform of the halo profile, $\hat{W}_{{\rm u},i}(k)$.  The spherical symmetry of the halo profile allows us to integrate over the angular co-ordinates such that the Fourier transform of the halo profile is given by
%
\begin{equation}
\hat{W}_{{\rm u},i}(k)=\int_0^\infty\frac{\sin(kr)}{kr}\;W_{{\rm u},i}(r)\;4\pi r^2\,\diff r\ .
\label{eq:window_function}
\end{equation}
%
We assume that the properties of each halo $i$ are defined solely by its mass $M_i$, such that $W_{{\rm u},i}= W_{\rm u}(M_i,r)$, and that the halo masses are distributed according to the halo-mass-distribution function $n(M)$, where $n(M)\,\diff M$ is the number density of haloes with masses between $M$ and $M+\diff M$.  With these assumptions we can find the mean value of $\theta_{\rm u}(\mathbf{x})$ by averaging over all haloes,
%
\begin{align}
\label{eq:meantheta}
\average{\hat{\theta}_{\rm u}(\mathbf{x})} = \left \langle \sum_i N_i\; W_{{\rm u}}(M_i,|\mathbf{x}-\mathbf{x}_i|) \right \rangle 
= \int_0^{\infty}  \diff M\; n(M) \int \diff^3 x'\; W_{\rm u}(M,|\mathbf{x}-\mathbf{x}'|)
=\int_0^{\infty}  \diff M\; W_{\rm u}(M) n(M)\;.
\end{align}
%
where we have translated the ensemble average, including $N_i$, into $\int  \diff M\; n(M)\; \Delta V_i$ with $\Delta V_i$ denoting the volume element.  The sum over $i$ can then be converted into an integral over the volume, $ \int \diff^3 x'$. To obtain the last equality in equation~(\ref{eq:meantheta}) we have separated the halo shape information through $W_{\rm u}(M,x)= W_{\rm u}(M) U_{\rm u}(M,x)$,  where $U_{\rm u}(M,x)$ is the normalised halo profile,
%
\begin{equation}
\int \diff^3 x\; U_{\rm u}(M,x) = 1\;.
\label{eq:normU}
\end{equation}
%
All the information about the amplitude of the halo profile is contained in $W_{\rm u}(M)$. Similarly we can define $\hat{W}_{\rm u}(M,k)= W_{\rm u}(M) \hat{U}_{\rm u}(M,k)$, where $\hat{U}_{\rm u}(M,k)$ is the Fourier transform of $ U_{\rm u}(M,x)$. From equation~(\ref{eq:normU}) we can conclude that $\hat{U}_{\rm u}(M,k\to0)=1$, which implies that $\hat{W}_{\rm u}(M,k\to0)=\hat{W}_{\rm u}(M)$. We can understand this result by considering that at large scales a halo acts as a point mass, which translates to a constant in Fourier space.  

Next we consider the correlation between two fields, $\theta_u$ and $\theta_v$, which could be identical, for example matter--matter, or different, for example matter--galaxies.  
Our fields are real and translationally invariant such that 
%
\begin{equation}
\average{\theta_{\rm u}(\bold{x})\;\theta_{\rm v}(\bold{x}')}=\xi_{\rm uv}(|\bold{x}-\bold{x}'|) \, ,
\label{eq:correlationuv}
\end{equation}
%
where $\xi_{\rm uv}$ is the two point correlation function between the two fields and it only depends on the separation $|\bold{x}-\bold{x}'|$. 
In Fourier space we have an equivalent relation for the power spectrum, $P_{\rm uv}(k)$,
\begin{equation}
\average{\hat{\theta}_{\rm u}(\bold{k})\hat{\theta}^*_{\rm v}(\bold{k}')}=(2\pi)^3 \delta_{\rm D}(\mathbf{k} -\mathbf{k}') P_{\rm uv}(k) \, ,
\label{eq:poweruv}
\end{equation}
%
where $\delta_{\rm D}$ is the Dirac delta function. The dimension of the power spectrum in equation~(\ref{eq:poweruv}) is volume times the dimensions of $\theta_{\rm u}$ and $\theta_{\rm v}$. Therefore we will sometimes use this definition of power spectrum instead
%
\begin{equation}
\Delta^2_{\rm uv}(k)= 4\pi\left(\frac{k}{2\pi}\right)^3 P_{\rm uv}(k)\ ,
\end{equation}
%
which removes the dependence on a volume dimension. 

We can find the power spectrum by inserting for the fields from equation~(\ref{eq:ffield}),
%
\begin{equation}
\average{\hat{\theta}_{\rm u}(\bold{k})\hat{\theta}^*_{\rm v}(\bold{k'})} = \left\langle\sum_{i,j} \mathrm{e}^{-{\rm i}\mathbf{k}\cdot\mathbf{x}_i}\; \mathrm{e}^{{\rm i}\mathbf{k}'\cdot\mathbf{x}_j}\; N_i\; N_j\; \hat{W}_{{\rm u},i}(M,k) \;\hat{W}_{{\rm v},j}(M,k') \right\rangle\ .
\label{eq:fullpuv}
\end{equation}
%
We can separate the sums above into two parts: when $i=j$, we measure field correlations within a single halo,  corresponding to the {\it one-halo} term,  $P^\mathrm{1h}_{\rm uv}(k)$;  when $i\neq j$, we measure field correlations between distinct haloes, called the {\it two-halo} term, $P^\mathrm{2h}_{\rm uv}(k)$.  


\subsubsection{The one-halo term}
\label{sec:one_halo_term}
The one-halo term, where $i=j$ in equation~(\ref{eq:fullpuv}), is given by
%
\begin{equation}
\average{\hat{\theta}_{\rm u}(\bold{k})\hat{\theta}^*_{\rm v}(\bold{k'})} = \left\langle\sum_i \mathrm{e}^{-{\rm i}(\mathbf{k}-\mathbf{k}')\cdot\mathbf{x}_i} N_i\; \hat{W}_{{\rm u},i}(M,k)\; \hat{W}_{{\rm v},i}(M,k') \right\rangle\ ,
\label{eq:one_halo_term_sum}
\end{equation}
%
where we have used $N_i^2=N_i$.  We take similar steps to what was done in equation~(\ref{eq:meantheta}) to turn the ensemble average and the sum into continuous integrals,
%
\begin{align}
\average{\hat{\theta}_{\rm u}(\bold{k})\hat{\theta}^*_{\rm v}(\bold{k'})} = \int_0^{\infty}  \diff M\; n(M) \int \diff^3 x\; \mathrm{e}^{-{\rm i}(\mathbf{k}-\mathbf{k}')\cdot\mathbf{x}_i}\;  \hat{W}_{\rm u}(M,k)\;\hat{W}_{\rm v}(M,k')\;.
\label{eq:one_halo_term_theta}
\end{align}
%
The integral over the volume results in a $(2\pi)^3 \delta_{\rm D}(\mathbf{k}-\mathbf{k}')$.  Comparing equations~\eqref{eq:one_halo_term_theta} to \eqref{eq:poweruv} we arrive at the one-halo power spectrum,
%
\begin{equation}
P^\mathrm{1h}_{\rm uv}(k)=\int_0^\infty \hat{W}_{\rm u}(M,k)\;\hat{W}_{\rm v}(M,k) \; n(M)\,\diff M\ .
\label{eq:one_halo_term}
\end{equation}
%
The $k\to0$ limit of the one-halo term is independent of the shape of the halo profile as discussed after equation~(\ref{eq:normU}), $W_{\rm u}(M,k\to0)\to W_{\rm u}(M)$. For this reason at large scales the one-halo term only contributes as a constant $P(k)$, so-called shot noise. 

\subsubsection{The two-halo term}
\label{sec:two_halo_term}

The two-halo term is defined when $i\neq j$ in equation~(\ref{eq:fullpuv}).   This time to turn the sums and the ensemble average into integrals we need to also consider the correlation between the positions of haloes,  $\average{N_i\;N_j}=\xi_{\rm hh}^{ij}(|\bold{x}_i-\bold{x}_j|)$.  
We then follow the same steps as the one-halo term and obtain two integrals over halo masses and two over the volume elements where these haloes reside,
%
\begin{align}
\label{eq:two_halo_term_theta}
\average{\hat{\theta}_{\rm u}(\bold{k})\hat{\theta}^*_{\rm v}(\bold{k'})} &= \int_0^\infty  \diff M_1 \int_0^\infty\diff M_2\; n(M_1)\; n(M_2)\;\hat{W}_{\rm u}(M_1,k)\; \hat{W}_{\rm v}(M_2,k) \\ \nonumber
& \times \int \diff x^3 \int \diff x'^3 \;\mathrm{e}^{-{\rm i}\mathbf{k}\cdot\mathbf{x}}\; \mathrm{e}^{{\rm i}\mathbf{k}'\cdot\mathbf{x}'} \average{N(M_1,\mathbf{x})\;N(M_2,\mathbf{x}')}\,  .
\end{align}
%
We also know that
%
\begin{equation}
\int \diff x^3 \int \diff x'^3 \;\mathrm{e}^{-{\rm i}\mathbf{k}\cdot\mathbf{x}}\; \mathrm{e}^{{\rm i}\mathbf{k}'\cdot\mathbf{x}'} \average{N(M_1,\mathbf{x})\;N(M_2,\mathbf{x}')} = \average{\hat{N}(M_1,\mathbf{k})\;\hat{N}^*(M_2,\mathbf{k}')}= (2\pi)^3 \delta_{\rm D}(\mathbf{k}-\mathbf{k}') P_\mathrm{hh}(M_1,M_2,k)\, ,
\label{eq:hh_power}
\end{equation}
%
where $P_\mathrm{hh}(M_1,M_2,k)$ is the power spectrum of the halo centres with the shot-noise contribution subtracted.  Inserting for the two volume integrals in equation~(\ref{eq:two_halo_term_theta}) from equation~(\ref{eq:hh_power}) we find the two-halo power spectrum,
%
\begin{equation}
P^\mathrm{2h}_{\rm uv}(k)=\int_0^\infty\int_0^\infty\diff M_1\;\diff M_2\; P_\mathrm{hh}(M_1,M_2,k)\;
\hat{W}_{\rm u}(M_1,k)\;\hat{W}_{\rm v}(M_2,k)\;n(M_1)\;n(M_2)\ .
\label{eq:full_two_halo_term}
\end{equation}
%
As haloes are biased tracers of the underlying matter field, we can approximate the power spectrum of the halo centres as
%
\begin{equation}
P_\mathrm{hh}(M_1,M_2,k)=b(M_1)b(M_2)P^\mathrm{lin}_\mathrm{mm}(k)[1+\Bnl(M_1,M_2,k)]\ ,
\label{eq:Bnl_def}
\end{equation}
%
where $b(M)$ is the linear bias of haloes with mass $M$, and $P^\mathrm{lin}_\mathrm{mm}(k)$ is the linear-theory matter power spectrum.  The function $\Bnl$ then models all non-linear effects that are missing from the linear-bias--linear-field model,  vanishing on large-scales: $\Bnl(M_1,M_2,k\to0)=0$.  With this model we arrive at the two-halo power spectrum
%
\begin{equation}
P^\mathrm{2h}_{\rm uv}(k)=P^\mathrm{lin}_\mathrm{mm}(k)I^\mathrm{nl}_{\rm uv}(k)+
P^\mathrm{lin}_\mathrm{mm}(k)\prod_{n={\rm u,v}}\left[\int_0^\infty \hat{W}_n(M,k)b(M)n(M)\mathrm{d}M\right]\ ,
\label{eq:two_halo_term}
\end{equation}
%
where the non-linear halo bias modelling is captured in the term
%
\begin{equation}
I^{\mathrm{nl}}_{\rm uv}(k)=\int_0^\infty\int_0^\infty \Bnl(M_1,M_2,k)
\hat{W}_{\rm u}(M_1,k)\hat{W}_{\rm v}(M_2,k)b(M_1)b(M_2)n(M_1)n(M_2)\,\mathrm{d}M_1\mathrm{d}M_2\ .
\label{eq:Inl}
\end{equation}
%
It is common to set $I^\mathrm{nl}_{\rm uv}(k)=0$ for all $k$ and therefore assume that halo bias is linear,  as there is no analytical solution for this term.  We will discuss this term in more detail in section~\ref{sec:non_linear_bias}.

\subsection{Discrete tracers}
\label{sec:discrete_tracers}

With some small modifications, the theory described in the previous subsection can be applied to discrete tracers, such as galaxies. Modifications are necessary for two reasons: the first is that there can be a non-negligible scatter in the number of tracers that occupy haloes of the same mass; the second is that when computing the autocorrelation of a discrete tracer field, there is an automatic correlation of the field with itself at zero separation, the so-called shot noise. In configuration space this manifests at $r=0$ in the correlation function, but in Fourier space this is spread evenly over all wavenumbers, resulting in a constant shot noise power spectrum, $P^\mathrm{sn}=1/\bar{n}$, where $\bar{n}$ is the mean tracer number density.

In the following we consider a field of galaxies and in particular the galaxy number density contrast,  $\theta_{\rm u}= \delta_{\rm g} = (n_{\rm g}-\bar{n}_{\rm g})/\bar{n}_{\rm g}$,  although all calculations apply to any discrete tracer.  The mean number density of galaxies, $\bar{n}_{\rm g}$, is defined as
%
\begin{equation}
\bar{n}_{\rm g} = \average{n_{\rm g}} =  
\int_0^\infty \diff M \; n(M)\; N_{\rm g}(M) \int \diff^3 x \;U_{\rm g}(M,|\mathbf{x}-\mathbf{x}'|)= \int_0^\infty \diff M \; n(M)\; N_{\rm g}(M),
\label{eqn:ng}
\end{equation}
%
where we have followed the same logic as in equation~(\ref{eq:meantheta}). $N_{\rm g}(M)$ is the number of galaxies occupying a halo of mass $M$ and $U_\mathrm{g}(M,k)$ is the normalised distribution of the galaxies within the halo. 

Let us first assume that there is no scatter in the halo-occupation distribution (HOD), $N_{\rm g}(M)$.  The temptation when converting equations~(\ref{eq:one_halo_term}) and (\ref{eq:full_two_halo_term}) or (\ref{eq:two_halo_term}) to be appropriate for discrete tracers is to set
%
\begin{equation}
\hat{W}_\mathrm{g}(M,k) = \frac{N_\mathrm{g}(M)}{\bar{n}_\mathrm{g}}\hat{U}_\mathrm{g}(M,k)\ ,
\label{eq:discrete_substitution}
\end{equation}
%
%where $N_\mathrm{g}(M)$ is the number of tracers occupying a halo of mass $M$ and $\hat{U}_\mathrm{g}(M,k)$ is the normalised Fourier transform of the distribution of those tracers within the halo. 
This substitution works for the two-halo term, and for cross spectra of discrete tracer populations, but fails when computing the autospectrum of a discrete tracer field because it fails to take into account the fact that the field necessarily self correlates.  In this case $N_{\rm g}$ tracers create $N_{\rm g}$ contributions to the self correlation or the shot noise.  Importantly,  these shot noise terms are independent of the distribution of the tracers,  $U_{\rm g}$. To arrive at the correct one-halo expression,  we separate the cross galaxy term from the auto galaxy term by subtracting an $N_{\rm g}(M)\;U^2_{\rm g}(M)\;n(M)$ from the integrand and replacing it with the shot noise contribution, $N_{\rm g}(M)n(M)$\footnote{Another way to think of this is that if a halo contains only one galaxy, the one-halo term should be pure shot noise, with no dependence on the halo profile; the $N(N-1)$ term ensures the cancellation of this part of the one-halo term in that case.},
%
\begin{equation}
P^\mathrm{1h}_\mathrm{gg}(k)=\frac{1}{\bar{n}^2_\mathrm{g}}
\int_0^\infty \left[N_\mathrm{g}(M)(N_\mathrm{g}(M)-1)\hat{U}^2_\mathrm{g}(M,k)
+N_\mathrm{g}\right]n(M)\,\diff M\ ,
\label{eq:one_halo_term_discrete}
\end{equation}
%
this is often written as
%
\begin{equation}
%P^\mathrm{1h}_\mathrm{gg}(k)=\frac{1}{\bar{n}^2_\mathrm{g}}
%\int_0^\infty N_\mathrm{g}(N_\mathrm{g}-1)\hat{U}^2_\mathrm{g}(M,k) n(M)\,\diff M+P^\mathrm{sn}_\mathrm{gg}\ ,
P^\mathrm{1h}_\mathrm{gg}(k)=\frac{1}{\bar{n}^2_\mathrm{g}}
\int_0^\infty N_\mathrm{g}(M)(N_\mathrm{g}(M)-1)\hat{U}^2_\mathrm{g}(M,k) n(M)\,\diff M+P^\mathrm{sn}_\mathrm{gg}\ ,
\label{eq:one_halo_term_discrete_shotnoise}
\end{equation}
%
where
\begin{equation}
%P^\mathrm{sn}_\mathrm{gg} = \frac{1}{\bar{n}^2_\mathrm{g}}
%\int_0^\infty N_\mathrm{g} n(M)\,\diff M\ = \frac{1}{\bar{n}_\mathrm{g}}\ .
P^\mathrm{sn}_\mathrm{gg} = \frac{1}{\bar{n}^2_\mathrm{g}}
\int_0^\infty N_\mathrm{g}(M) n(M)\,\diff M\ = \frac{1}{\bar{n}_\mathrm{g}}\ .
\label{eq:shot_noise}
\end{equation}
%
This shot-noise term is often, but not always, subtracted from measured spectra, although it shows up again in the covariance matrix.  Note that in the $N\to\infty$ limit $N^2\gg N$ and equation~(\ref{eq:one_halo_term_discrete}) returns to the form obtainable from the substitution in equation~(\ref{eq:discrete_substitution}). This limit is appropriate for a general emissive profile, for example matter haloes taken to be composed from sub-atomic dark-matter particles, or even those from comparatively massive simulation particles. In this latter case, though the shot noise contribution is a real contribution, it is usually subtracted from simulation measurements because it is an artefact that arises due to the discretisation-techniques necessarily employed by \nbody simulations.

 In contrast, if we consider the cross spectrum between two different discrete populations ($\mathrm{g}$ and $\mathrm{g}'$; which may live in the same haloes), the one-halo term would be
% 
\begin{equation}
 P^\mathrm{1h}_\mathrm{gg'}(k)=
 \frac{1}{\bar{n}_\mathrm{g}\bar{n}_\mathrm{g'}}
 \int_0^\infty N_\mathrm{g}(M)N_\mathrm{g'}(M)\hat{U}_\mathrm{g}(M,k)\hat{U}_\mathrm{g'}(M,k) n(M)\,\diff M\ ,
 \label{eq:one_halo_term_discrete_cross}
 \end{equation}
% 
 which is exactly the generalization of the result from the continuous emissivity case. Note that evaluating equation~(\ref{eq:one_halo_term_discrete}) for haloes that contain $N_\mathrm{g}=1$ generates a pure shot-noise contribution. It is only when $N_\mathrm{g}>1$ that terms that involve the halo profile are generated, as one would expect.

When considering galaxy populations, it is usual that the galaxy population is broken down into separate contributions from central galaxies, with occupation number either $0$ or $1$, and satellite galaxies. It is also usual that some scatter is accounted for in the occupation numbers at fixed halo mass, which means we must keep the expectation value from equation~(\ref{eq:one_halo_term_sum}) to give:
%
\begin{equation}
%P^\mathrm{1h}_\mathrm{cc}(k)=\frac{1}{\bar{n}^2_\mathrm{g}}
%\int_0^\infty \average{N_\mathrm{c}} n(M)\,\diff M\ ;
P^\mathrm{1h}_\mathrm{cc}(k)=\frac{1}{\bar{n}^2_\mathrm{g}}
\int_0^\infty \average{N_\mathrm{c}(M)} n(M)\,\diff M\ ;
\label{eq:one_halo_central_central}
\end{equation}
%
\begin{equation}
%P^\mathrm{1h}_\mathrm{ss}(k)=\frac{1}{\bar{n}^2_\mathrm{g}}
%\int_0^\infty \left[\average{N_\mathrm{s}(N_\mathrm{s}-1)}\hat{U}^2_\mathrm{s}(M,k)+\average{N_\mathrm{s}}\right] n(M)\,\diff M\ ;
P^\mathrm{1h}_\mathrm{ss}(k)=\frac{1}{\bar{n}^2_\mathrm{g}}
\int_0^\infty \left[\average{N_\mathrm{s}(M)(N_\mathrm{s}(M)-1)}\hat{U}^2_\mathrm{s}(M,k)+\average{N_\mathrm{s}(M)}\right] n(M)\,\diff M\ ;
\label{eq:one_halo_satellite_satellite}
\end{equation}
%
\begin{equation}
%P^\mathrm{1h}_\mathrm{cs}(k)=
%\frac{1}{\bar{n}^2_\mathrm{g}}
%\int_0^\infty \average{N_\mathrm{c}N_\mathrm{s}}\hat{U}_\mathrm{c}(M,k)\hat{U}_\mathrm{s}(M,k) n(M)\,\diff M\ .
P^\mathrm{1h}_\mathrm{cs}(k)=
\frac{1}{\bar{n}^2_\mathrm{g}}
\int_0^\infty \average{N_\mathrm{c}(M)N_\mathrm{s}(M)}\hat{U}_\mathrm{c}(M,k)\hat{U}_\mathrm{s}(M,k) n(M)\,\diff M\ .
\label{eq:one_halo_central_satellite}
\end{equation}
%
We have used the fact that the occupation number of the central galaxies is $N_\mathrm{c}=0,1$ to eliminate the $\average{N_\mathrm{c}(N_\mathrm{c}-1)}$ term from equation~(\ref{eq:one_halo_central_central}), which leaves a pure shot-noise contribution. It is often taken that the central galaxies lie at the exact halo centre, in which case $\hat{U}_\mathrm{c}=1$, but we have left this term, which only enters the central--satellite cross spectrum, for completeness (some authors consider mis-centring, which can be included via this term). It is often taken that the central and satellite occupation numbers are independent: $\average{N_\mathrm{c}N_\mathrm{s}}=\average{N_\mathrm{c}}\average{N_\mathrm{s}}$ or otherwise the `central condition' is imposed such that satellites can only exist if there is a central galaxy: $\average{N_\mathrm{c}N_\mathrm{s}}=\average{N_\mathrm{s}}$. It is often assumed that the satellite occupation is determined by Poisson statistics: $\average{N_\mathrm{s}(N_\mathrm{s}-1)} = \average{N_\mathrm{s}}^2$. If shot-noise is subtracted this eliminates $P^\mathrm{1h}_\mathrm{cc}$ entirely, and also eliminates the term $\propto\average{N_\mathrm{s}}$ in the square bracket in equation~(\ref{eq:one_halo_satellite_satellite}); this is the most common form of these equations to be found in the literature. The full galaxy autospectrum can be constructed from these constituents via
%
\begin{equation}
P_\mathrm{gg}(k)=P_\mathrm{cc}(k)+2P_\mathrm{cs}(k)+P_\mathrm{ss}(k)\ .
\label{eq:galaxy_power_sum}
\end{equation}
%
Some expressions similar to equation~(\ref{eq:galaxy_power_sum}) contain pre-factors of the fraction of galaxies that are centrals or satellites. We avoid this here by defining the central and satellite fields as overdensity with respect to the total number of galaxies.
Note that it is \emph{not} possible to arrive at the same result for $P_\mathrm{gg}$ by replacing $N_\mathrm{g}\to N_\mathrm{c}+N_\mathrm{s}$ and $U_\mathrm{g}\to U_\mathrm{s}$ in equation~(\ref{eq:one_halo_term_discrete}) because this cannot account for the unique clustering and occupation properties of two distinct galaxy populations. A common \emph{approximate} \citep[\eg][]{Seljak2000} expression can be obtained by replacing the profile power of $2$ in equation~(\ref{eq:one_halo_term_discrete}) with $1$ if $\average{N_\mathrm{g}}\lesssim 2$ and retaining $2$ otherwise. This  roughly accounts for the fact that if more than one satellite is present the one-halo contribution to the power is dominated by the satellite auto correlation (equation~\ref{eq:one_halo_satellite_satellite}), whereas if a single satellite is present then it is dominated by the central--satellite cross correlation (equation~\ref{eq:one_halo_central_satellite}). Another reasonable approximation would be to use equation~(\ref{eq:one_halo_term_discrete}) with an occupation-number-weighted halo profile
%
\begin{equation}
\hat{U}_\mathrm{g}(M,k)\simeq
\frac{\average{N_\mathrm{c}}\hat{U}_\mathrm{c}(M,k)+\average{N_\mathrm{s}}\hat{U}_\mathrm{s}(M,k)}
{\average{N_\mathrm{c}}+\average{N_\mathrm{s}}}\ ,
\label{eq:approx_gal_window}
\end{equation}
%
together with
%
\begin{equation}
\average{N_\mathrm{g}(N_\mathrm{g}-1)}=\average{N_\mathrm{s}(N_\mathrm{s}-1)}+2\average{N_\mathrm{c}N_\mathrm{s}}\ .
\label{eq:Ng_exp}
\end{equation}
%
This last equation is always true as long as $\average{N_\mathrm{c}(N_\mathrm{c}-1)}=0$ (\ie $0$ or $1$ central galaxy only). The two terms on the right-hand side of equation~(\ref{eq:Ng_exp}) can be evaluated following the logic in the paragraph after equation~(\ref{eq:one_halo_central_satellite}). Using equation~(\ref{eq:approx_gal_window}) in equation~(\ref{eq:one_halo_term_discrete}) is not perfect, but the relative error can be small compared to the proper evaluation of equations~(\ref{eq:one_halo_central_central}--\ref{eq:galaxy_power_sum}) depending on the halo-occupation model.

\hrulefill
\vspace{0.5cm}
\twocolumngrid
\subsection{Matter}
\label{sec:matter}

% Normalisation of relations
Another special case is to evaluate the halo model solutions for the `matter' distribution in order to evaluate the matter power spectrum, but this involves some unique considerations. First, note that the adopted halo mass function and linear halo bias \emph{must} satisfy the following properties for any power spectrum involving the matter to have the correct large-scale limit
%
\begin{equation}
\int_0^\infty Mn(M)\,\mathrm{d}M=\bar\rho\ ,
\label{eq:mass_normalisation}
\end{equation}
%
\begin{equation}
\int_0^\infty Mb(M)n(M)\,\mathrm{d}M=\bar\rho\ .
\label{eq:bias_normalisation}
\end{equation}
%
where $\bar\rho$ is the mean comoving cosmological matter density. 
In other words, these equations enforce that all matter is contained in haloes and that, on average, matter is unbiased with respect to itself. Achieving these limits is difficult numerically because of the large amount of mass contained in low-mass haloes according to most popular mass functions.  Therefore, special care must be taken with the two-halo integral in the case of power spectra that involve the matter field to ensure that equation~(\ref{eq:bias_normalisation}) holds \citep[see appendix A of][]{Mead2020}.  Some popular mass function and bias relations enforce these consistency relations while others do not. If they do not, it might be possible to manually enforce these relations, but this normally involves manipulating the low-mass halo population.

\begin{figure}
\begin{center}
\includegraphics[width=\columnwidth]{plots/power_HOD_all.pdf}
\end{center}
\caption{The upper panel shows example power spectra computed using the halo model at $z=0$: matter, galaxy and matter--galaxy. The linear spectrum is shown for comparison as well as the breakdown into two- and one-halo terms where that does not clutter the plots. The middle panel shows the ratio of each spectrum to linear while the lower panel shows the ratio to the non-linear halo model matter spectrum. The galaxy sample can be seen to be positively biased relative to the matter. This bias is scale-independent at large scales, but becomes scale dependent at intermediate scales where the difference in the ways that galaxies and matter occupy haloes becomes important.}
\label{fig:power}
\end{figure}

% Example for matter P(k)
In the special case of the power spectrum for matter density contrast, $\delta_{\rm m} = (\rho-\bar{\rho})/\bar{\rho}$,  we use $W_\mathrm{m}(M,k)=MU_\mathrm{m}(M,k)/\bar\rho$ and equation~(\ref{eq:one_halo_term}) becomes
%
\begin{equation}
P^\mathrm{1h}_\mathrm{mm}(k)=\frac{1}{\bar\rho^2}\int_0^\infty M^2\hat{U}^2_\mathrm{m}(M,k) n(M)\,\diff M\ .
\label{eq:one_halo_term_matter}
\end{equation}
%
while equation~(\ref{eq:two_halo_term}) becomes
%
\begin{align}
\begin{split}
P^\mathrm{2h}_\mathrm{mm}(k)&=P^\mathrm{lin}_\mathrm{mm}(k) \\
&\times\left[\frac{1}{\bar\rho}\int_0^\infty M \hat{U}_\mathrm{m}(M,k)b(M)n(M)\,\mathrm{d}M\right]^2\ ,
\end{split}
\end{align}
%
(ignoring non-linear halo biasing). We see that $P^\mathrm{2h}_\mathrm{uv}(k\to0)=P^\mathrm{lin}_\mathrm{mm}(k\to0)$ automatically as the term in the square brackets equals unity (equation~\ref{eq:bias_normalisation}) in this limit. For spectra other than matter this is no longer true, and in general the large-scale limit of the two halo term will be equal to the linear spectrum multiplied by amplitude factors (so-called bias) that account for the field content that arises from how the field populates haloes (e.g.  galaxy bias when the field is galaxy overdensity) and the halo bias.  Example power spectra at $z=0$ for a \LCDM model are shown in Fig.~\ref{fig:power} for matter, matter--galaxies and galaxies\footnote{Non-linear halo bias is ignored, the mass function is taken from \cite{Sheth1999}, halo concentration from \cite{Duffy2008} and the HOD from \cite{Zheng2005}; discussed in Sections~\ref{sec:halo_mass_function}, \ref{sec:dark_matter_haloes} and \ref{sec:HOD} respectively. The HOD parameters are $M_\mathrm{min}=M_0=M_1=10^{13}\Msun$, $\sigma_{\log_{10}M}=0.3$ and $\alpha=1$ (equations~\ref{eq:HOD_Nc} and \ref{eq:HOD_Ns}). Satellite galaxies are taken to trace matter. Shot noise is subtracted from the galaxy autospectra.}. The sample of galaxies chosen can be seen to be positively biased ($b\sim 1.3$) relative to the matter at large scales. At smaller scales the spectra have different shapes, a consequence of galaxy and matter occupying haloes in different ways.

