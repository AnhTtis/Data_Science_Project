%%%%%%%%%%%%%%%%%%%%%%%%%%%%%%%%%%%%%%%%%%%%%%%
%%%%%% APPENDIX
%%%%%%%%%%%%%%%%%%%%%%%%%%%%%%%%%%%%%%%%%%%%%%%
%\appendix

%\begin{table}[h!]
  \centering
  \begin{tabular}{|c|c|c|}
    \hline
    Description & Symbol & Value \\
    \hline
    \hline
    $\element[][Ni]{56}$ Lifetime & $\tau_\text{Ni}$ & $8.77$~d\\
    $\element[][Co]{56}$ Lifetime & $\tau_\text{Co}$ & $111.45$~d\\
    $\element[][Ni]{56}$ Decay Lum. & $q_\text{Ni}$ & $6.45\times 10^{43}$~$\text{erg}~M_\odot^{-1}~\text{s}^{-1}$ \\
    $\element[][Co]{56}$ Gamma Decay Lum. & $q_{\text{Co},\gamma}$ & $1.38\times 10^{43}$~$\text{erg}~M_\odot^{-1}~\text{s}^{-1}$ \\
    $\element[][Co]{56}$ Positron Decay Lum. & $q_{\text{Co},e^+}$ & $4.64\times 10^{41}$~$\text{erg}~M_\odot^{-1}~\text{s}^{-1}$ \\
    \hline
    
  \end{tabular}
  \caption{Nuclear decay constants from \cite{2019MNRAS.484.3941W} used in the radioactive decay chain heating function.}
  \label{tab:nucdata}
\end{table}


%% \section{Data Products}

%% We publish all unprocessed input data products, analysis code, notebooks and products.

%% \todo[inline]{Insert citation once published}
%% All data products as well as the used codes are published at XXX (citation).

%\FloatBarrier

\section{Photometric Models}
\label{appendix:photmodel}

\begin{table}[h]
  \begin{tabular}{|c|c|c|}
    \hline
    Transient & Band & Model \\
    \hline
    \hline
    \supernova{2019odp} & ugri & plateau-contardo \\
    \supernova{2019odp} & zJHK & linear \\
    \hline
    iPTF13bvn & Ugriz & plateau-contardo \\
    \supernova{2008D} & UBVri & prebump-contardo \\
    \supernova{1998bw} & UBV $\text{R}_C\,\text{I}_C$ & plateau-conardo \\
    \supernova{2002ap} & UBVRI & plateau-contardo \\
    \hline
  \end{tabular}
  \caption{Used photometric models per supernova and band}
  \label{tab:a:photmodel:assignment}
\end{table}

In this section we describe the used analytic photometry light curve models and priors for the models.
We use different models depending on the lightcurve coverage and the shape of the early excess (if there was one detected) with the decision being specific to the photometric band.
The assignment of photometric model to a given transient and band is listed in \autoref{tab:a:photmodel:assignment}.
All analytic functions use the phase relative to the prior peak estimate: $\Delta t \equiv t - t_\text{peak}$.

Our (complex) lightcurve models are based on the lightcurve model by \cite{2000A&A...359..876C} (hereafter C20):
\begin{equation}
  m_\text{C20}(\Delta t) = \delta(\Delta t) \left( \alpha + \beta \Delta t + A_\text{DF} \exp{\frac{-(\Delta t - t_{0,\text{DF}})}{2 \sigma_\text{DF}^2}} \right),
\end{equation}
where $\Delta t$ is the phase relative to the main peak prior time, $\delta(t)$ is the explosion scaling function, $\alpha$ is the linear intercept, $\beta$ is the linear slope, $A_\text{DF}$ is the amplitude of the main difussion peak gaussian, $t_{0,\text{DF}}$ is the phase offset of the main diffusion peak gaussian and $\sigma_\text{DF}$ is the width of the gaussian.
The rise scaling function is given as:
\begin{equation}
  \delta(\Delta t)^{-1} = 1 - \exp{\frac{-(\Delta t - t_{0,\text{rise}})}{\tau_\text{rise}}},
\end{equation}
where $t_{0,\text{rise}}$ is the phase offset of the rise scaling function, $\tau_\text{rise}$ is the rise timescale.

We derive two modified lightcurve models from the C20 depending on the shape of the early excess:
\begin{enumerate}
\item The ``plateau-contardo'' model adds a smoothing function $g(t)$ to interpolate between the plateau magnitude $m_\text{plat}$ and the $m_\text{C20}$ function:
  \begin{eqnarray}
    g(\Delta t) =\left(
    \arctan\left( \frac{\Delta t - t_{0,\text{plat}}}{\tau_\text{smooth}} \right) \frac{1}{\pi} + 0.5
    \right)^2 \\
    m_\text{PLC}(\Delta t) = m_\text{plat} + (m_\text{C20}(\Delta t)-m_\text{plat}) g(\Delta t)
  \end{eqnarray}
  This introduces three additional parameters to the C20 model that we allow to vary in a reasonable range: plateau magnitude $m_\text{plat}$, plateau end time $t_{0,\text{plat}}$ and the smoothing timescale $\tau_\text{smooth}$.
\item The ``prebump-contardo'' model adds a secondary gaussian peak at a peak relative to the main peak:
  \begin{equation}
    %m_\text{PBC}(t) = p_\text{C20} + \frac{A_\text{PB}}{1-\frac{\exp{-(\Delta t - t_{0,\text{rise}})}}{\tau_\text{rise}}} \exp{ \frac{-(\Delta t - t_{0,\text{PB}})}{2 \sigma_\text{PB}^2} }
    m_\text{PBC}(\Delta t) = p_\text{C20} + \delta(t) A_\text{PB} \exp{ \frac{-(\Delta t - t_{0,\text{PB}})}{2 \sigma_\text{PB}^2} }
  \end{equation}
  This introduces three additional parameters to the model: pre-bump amplitude $A_\text{PB}$, pre-bump width $\sigma_\text{PB}$ and center time of the pre-bump $t_{0,\text{PB}}$.
\end{enumerate}

If the data only covers a small time range before/after peak we use a linear model instead:
\begin{equation}
  m_\text{lin}(\Delta t) = \alpha + \frac{\beta}{1000} \Delta t
\end{equation}

%% \begin{equation}
%%   m(t) = m_\text{peak} + A \cdot \exp \left( \frac{-(t-t_0)^2}{2\,\sigma^2} \right)
%% \end{equation}

%% For the GP kernel hyper-parameters we use a common prior for the kernel amplitude.
%% \begin{table}
%%   \begin{tabular}{|c|c|c|c|}
%%     \hline
%%     Parameter & Symbol & Unit & Prior \\
%%     \hline
%%     \hline
%%     Kernel Amplitude & $\log K_A$ & & $\mathcal{U}(-10, -1)$ \\
%%     %Kernel Lengthscale & $$
%%     \hline
%%   \end{tabular}
%%   \caption{Common priors used for all photometric lightcurve models.}
%%   \label{tab:a:photmodel:common:prior}
%% \end{table}




%% \subsection{Plateau Contardo}
%% \label{appendix:photmodel:plateaucontardo}

%% This model is based on the empirical lightcurve model by \cite{2000A&A...359..876C}.


%% \subsection{Prebump Contardo}
%% \label{appendix:photmodel:prebumpcontardo}

%% \subsection{Linear}
%% \label{appendix:photmodel:linear}

%% \begin{table}
%%   \begin{tabular}{|c|c|c|c|}
%%     \hline
%%     Parameter & Symbol & Unit & Prior \\
%%     \hline
%%     \hline
%%     Intercept & B & mag & $\mathcal{U}(10, 30)$ \\
%%     Slope & A & mag~$\text{d}^{-1}$ & $\mathcal{U}(0,100)$ \\
%%     \hline
%%   \end{tabular}
%%   \caption{Priors for the linear photometric model}
%%   \label{tab:a:photmodel:linear:prior}
%% \end{table}

%% In cases where only a short time frame pre- or post-peak has been observed we use a linear model.
%% The used priors are listed in \autoref{tab:a:photmodel:linear:prior}.

%% \begin{equation}
%%   m(t) = B + \frac{A}{1000} \cdot \Delta t
%% \end{equation}

%% % Priors:
%% % linear intercept B -> 10 - 30 [mag]
%% % linear slope A -> 0 - 100
%% % Kernel log amplitude -> -10 to -1



%% \subsection{Gaussian}
%% \label{appendix:photmodel:gaussian}

%% \begin{equation}
%%   m(t) = m_\text{peak} + A \cdot \exp \left( \frac{-(t-t_0)^2}{2\,\sigma^2} \right)
%% \end{equation}


% unused stuff:
%The Gaussian Process parameters, offsets between the photometric instruments and the ZTF band, 

%% In most cases we use a modified variant of the \cite{2000A&A...359..876C} function:
%% \begin{equation}
%%   m(t) = \begin{cases}
%%     \frac{m_0 + \lambda t + A\cdot \exp(\frac{-t^2}{2 \sigma^2})}{1-\exp(\frac{t_{0,r} -  t}{\tau_0})} & t \ge t_1 \\
%%     m_p & t \le t_1
%%   \end{cases},
%% \end{equation}
%% where $t$ is the phase from peak $g$-band magnitude, $\sigma$ is the width of the main peak gaussian, $t_{0,r}$ is the zero-point time of the rise, $\tau_0$ is the rise time constant, $t_1$ is the plateau cut-off time and $m_p$ is the magnitude during the plateau phase.

%% For some bands where only very limited phase coverage is available we use a simple linear model \footnote{And we expect this to be the best model for this phase window}:
%% \begin{equation}
%%   m(t) = m_0 + \lambda t
%% \end{equation}

%% For \supernova{2008D} we use the unmodified version of the \cite{2000A&A...359..876C} function with the first peak centered near the first detection and the second peak denoting the main peak.

\FloatBarrier

%% \section{Lightcurve Construction}
%% \label{appendix:lc:construction}

%% Since the photometry was obtained with a few different  instruments we have to first correct for slight shifts between the photometric systems of the different instruments.
%% We correct all photometry to the ZTF system since the largest fraction of the photometry dataset comes from the ZTFCam.
%% Since the used filter systems are a quite reasonable match between the different instruments and we expect photometric uncertainties from image subtraction to dominate, we only consider a static offset between the systems.
%% The lightcurves are interpolated using Gaussian Process regression \citep[see][for a review]{gortler2019a}.
%% % TODO: ref to supplemental materials?


%% \section{Flux Calibration}

%% As basis of truth we use the integrated and interpolated $r$-band lightcurve to provide the foundation of the spectra.

%% We compute synthetic photometry for the ZTF filter band for all observed spectra and use the offset between the synthetic photometry and the observed lightcurve to compute the correction factors.

%% As additional crosscheck we compute the same correction factors for the $g$ and $i$ bands and plot them in \autoref{fig:fluxcal:comparison}.

%% \begin{figure}
%%   \missingfigure{Comparison of the correction for the spectra based on the different ZTF bands}
%%   \caption{Comparison of the correction for the spectra based on the different ZTF bands}
%%   \label{fig:fluxcal:comparison}
%% \end{figure}


% XXX: or maybe should be section instead of subsection of LC?
%% \section{Pseudobolometric Lightcurve}
%% \label{sec:a:qbol}

%% \todo[inline]{Will anything else actually go here?}

%% %% METHOD COMPARISON
%% \subsection{Method comparison}
%% \label{sec:a:qbol:comparison}

%% Since we have wideband photometric observations for roughly a period of one month post-peak we can do a crude comparison of different methods for estimating the pseudobolometric luminosity.
%% However since we don't have any observations that are shorter in wavelength than optical or longer than NIR (no far-IR or radio observations) we have to make a assumption about the shape of the spectral energy distribution (SED) outside our observation window.
%% For this we take the common assumption of the SED being a blackbody, which we will try to constrain using our optical and NIR observations.

%% In this small comparison we compare the pseudobolometric luminosity estimates by the empirical \cite{2014MNRAS.437.3848L} method that is widely used for core-collapse supernovae against both direct integration of the GROND photometry (using the trapezoid method) as well as the SuperBoL estimator \citep{2017PASP..129d4202L}.
%% %The time evolution of the relative ratio is shown \autoref{fig:lc:qbol:comparison:ratio} and one can see that all the different methods are in fairly good agreement.
%% %While there is some disagreement between the two different Lyman methods it is well within the 3-sigma uncertainties.


%% For the early plateau we use a different strategy since this behavior starts to deviate significantly from commonly observed behavior and thus the usual empirical methods may not be applicable.
%% We fit a blackbody to the reddest filters of the photometry, since we expect that these filter bands suffer from the least amount of line blanketing.
%% We only integrate the SED from the reddest filter to infinity.
%% We fit a second blackbody to the bands affected by the line-blanking.
%% Integrating all this we get a luminosity for the plateau of $10^{x \pm y}$ erg/s in the x to y keV band.

\section{Oxygen Mass Modelling}
\label{sec:a:oxygen}


%% \begin{table}
%%   \centering
%%   \begin{tabular}{|c|c|c|c|}
%%     \hline
%%     Parameter & Symbol & Unit & Prior \\
%%     \hline
%%     \hline
%%     Oxygen Mass & $\log M(\ion{O}{I})$ & $M_\odot$ & $\mathcal{U}(-18,2)$ \\
%%     LTE Temperature & $T$ & K & $\mathcal{U}(1000, 11000)$ \\
%%     Line Width & $W$ & $\AA$ & $\mathcal{U}(30, 230)$ \\
%%     Distance & $D$ & cm & $\mathcal{U}(D_\text{min}, D_\text{max})$ \\
%%     \hline
%%     Continuum Level & $\beta_i$ & & $\mathcal{U}(1, 10000)\cdot 10^{-18}$ \\
%%     Continuum Slope & $\alpha_i$ & & $\mathcal{U}(-0.01, 0.01)\cdot 10^{-18}$ \\
%%     \hline
%%   \end{tabular}
%%   \caption{Priors for the oxygen mass estimate model}
%% \end{table}

%\subsection{Validation}
%\label{sec:a:oxygen:validation}

\subsection{NLTE Deviation Factor}
\label{sec:a:oxygen:nlte}

The LTE departure coefficient is defined as follows:
\begin{equation}
  d_2 = \frac{n_2}{n_2^\text{LTE}}
\end{equation}

We estimate the sensitivity to deviations from LTE conditions by running the fitting procedure for a spectrum multiple times in a sequence where we vary the allowed maximum LTE departure coefficient $d_2$ from 0.01 to 1.0 (where 1.0 means LTE conditions).
The resulting change due to the change of the LTE departure can be seen in \autoref{fig:a:oxygen:nlte:sens}.

\begin{figure}
  \includegraphics[width=\linewidth]{plots/tmp/oxygen_nlte_deviation_sensitivity.png}
  \caption{Plot showing the oxygen mass percentiles as a function of the minimum LTE departure coefficient $d_2$.}
  \label{fig:a:oxygen:nlte:sens}
\end{figure}

%\subsubsection{NLTE Deviation Factor}

We get a rough estimate for the LTE departure coefficient $d_2$ by estimating the electron density $n_e$ from the $\ion{O}{I}\,\lambda 7774$ recombination line and using the following relation:
\begin{equation}
  d_2 \approx \left( 1 + 1.44 \left( \frac{T}{1000\,\text{K}} \right)^{-0.034} \left( \frac{n_e}{10^8\,\text{cm}^{-3}} \right)^{-1} \right)^{-1}
  \label{eqn:a:oxygen:nlte:ne}
\end{equation}
which we got by dividing eqn. 2 from \cite{1996ApJ...456..811H} by eqn. 2 from \cite{2015A&A...573A..12J}.
%Since the collision rates $C_{ij}$ depend on the electron density $n_e$ we need an estimator for this quantity.
To approximate the electron density we use the oxygen recombination lines that are visible in the earlier spectra.
We can use the following relation from \cite{2015A&A...573A..12J} to relate the line luminosity to the electron density (their eqn. 3):
\begin{equation}
  L_\text{rec} = \frac{4\pi}{3} \left( V_\text{core} t \right)^3 \Psi \alpha_\text{eff} f_O n_e^2 h \nu,
\end{equation}
where $V_\text{core}$ is the line width, $t$ is the time since explosion, $\Psi$ is the fraction of electrons provided by oxygen ionizations and $f_O$ is the oxygen zone filling factor.

We assume $\Psi$ to be of order unity (from \citet{2015A&A...573A..12J}).
We estimate $V_\text{core}$ from the measured gaussian line width.
%We use the $\ion{O}{I}\,\lambda 7774$ recombination line.
Since $\alpha_\text{eff}$ depends on the temperature, we use the values from \citet[their sect. C.1]{2015A&A...573A..12J} for three different temperatures.
Using $\alpha(T=2500)=2.8 \times 10 ^{-13}$, $\alpha(T=5000)=1.6 \times 10^{-13}$ and $\alpha(T=7500)=1.1 \times 10^{-13}$ and a crudely estimated line luminosity of $L \simeq 27 \cdot 10^{38}$ erg\,$\text{s}^{-1}$ as well as the line width estimated velocity of 2464 \kms  (both from the $+158$\,d Keck spectrum) we estimate $n_e \sqrt{f_O}$ to be:
$2.2 \cdot 10^{8}$\,$\text{cm}^{-3}$ (2500~K), $2.9 \cdot 10^{8}$\,$\text{cm}^{-3}$ (5000~K) and $3.5 \cdot 10^{8}$\,$\text{cm}^{-3}$ (7500~K).
This is in line to the values seen in \cite{2015A&A...573A..12J}.
Using \autoref{eqn:a:oxygen:nlte:ne} we get a range of $0.6$ to $0.72$ for LTE departure coefficient $d_2$ for an assumed zone filling factor of one.
Higher filling factors would yield even higher values.
However the recombination lines can be emitted from higher density regions than the nebular lines we are using in the main analysis and are thus not really suitable for quantitative analysis and we thus do not use the derived $d_2$ value/range.



%% For the NLTE model we assume the first excited state $u_1$ to be still in LTE.

%% Thus we have the following balance equation:
%% \begin{eqnarray}
%%   N_g\,C_{g2} + N_1\,C_{12} = N_2 \left( A_{21}\beta_S + C_{21} + C_{2g} \right)
%% \end{eqnarray}

%% We approximate $C$ as follows:
%% \begin{equation}
%%   C_{ul} (T, n_e) = Q_{ul} n_e = n_e \cdot 8.6 \cdot 10^-6 \frac{Y(T)}{T^{1/2} g_u}
%% \end{equation}
%% For the inverse factors we use the following relation:
%% \begin{equation}
%%   c_{ul} = C_{lu} \exp{(E_l-E_u)/k_B T} \frac{g_l}{g_u}
%% \end{equation}
%% \todo[inline]{Ref for that relation.. and does it apply? or the Saha version?}

%% Since we approximate $N_g \approx N$ and $N_1 \approx N_1^\text{LTE}$ we can rewrite above equation as:
%% \begin{equation}
%%   N_2 = \frac{N\,Q_{2g} (T) \exp(-E_2/kT) n_e \frac{g_2}{g_g} + N_1(N,T)\,Q_{21}(T) \exp(-\Delta_{21}/kT) n_e\frac{g_2}{g_1}}{A_{21} \beta_S + Q_{21}(T)n_e + Q_{2g}(T)n_e}
%% \end{equation}

%% We assume $\beta_S$ to be 1 in this case, any reduction in the probability only moves the ratio closer to LTE.

%% Comparing this to $N_2^\text{LTE}$ we show this as a function of temperature and fill factor in \autoref{fig:nlte:deviation}.

%% \begin{figure}
%%   \includegraphics[width=\linewidth]{plots/tmp/oxygen_nlte_deviation.png}
%%   \caption{NLTE Deviation fraction $d_2$ as a function of temperature and fill factor $f_O$}
%%   \label{fig:nlte:deviation}
%% \end{figure}

%% Next we take restate the problem in terms of fill factor and compare the optical depth derived oxygen mass with the line luminosity derived oxygen mass.
%% For this we take 0.5 slices of $f_O$ in log space and compare them.



%% \subsubsection{Comparison with Jerkstrand (2014) method}

%% Here we perform forward modelling of spectra and then applying the method from J14.

\subsection{Spectral Fitting Results}
\label{appendix:oxygen:spectralfits}

In this section we show the best fit results from the spectral line profile models.
In \autoref{fig:a:oxygen:spec:diag:notlate}, \autoref{fig:a:oxygen:spec:diag:keckearly} and \autoref{fig:a:oxygen:spec:diag:kecklate} we show the relevant spectral fitting regions with model spectra drawn from the posterior distribution overplotted.
The corresponding corner plots visualizing the posterior distributions can be found in \autoref{fig:a:oxygen:spec:corner:notlate}, \autoref{fig:a:oxygen:spec:corner:keckearly}, \autoref{fig:a:oxygen:spec:corner:kecklate}.

\begin{figure}[h]
  \includegraphics[width=\linewidth]{plots/specs/model_diag_oxygen_not_late_7774.png}
  \caption{NOT/ALFOSC spectrum at 128 days post-peak with model spectra drawn from the posterior distribution of the model overlaid in grey.}
  \label{fig:a:oxygen:spec:diag:notlate}
\end{figure}

\begin{figure}[h]
  \includegraphics[width=\linewidth]{plots/specs/model_diag_oxygen_keck_early_7774.png}
  \caption{Early Keck/LRIS spectrum at 138 days post-peak with model spectra drawn from the posterior distribution of the model overlaid in grey.}
  \label{fig:a:oxygen:spec:diag:keckearly}
\end{figure}

\begin{figure}[h]
  \includegraphics[width=\linewidth]{plots/specs/model_diag_oxygen_keck_late_gauss.png}
  \caption{Late Keck/LRIS spectrum at 358 days post-peak with model spectra drawn from the posterior distribution of the model overlaid in grey.}
  \label{fig:a:oxygen:spec:diag:kecklate}
\end{figure}


\begin{figure*}[h]
  \includegraphics[width=\linewidth]{plots/specs/model_corner_not_late_7774.png}
  \caption{Corner plot of the $\ion{O}{I}\lambda~7774$ spectral line profile model for the NOT/ALFOSC spectrum at 128 days post-peak.}
  \label{fig:a:oxygen:spec:corner:notlate}
\end{figure*}

\begin{figure*}[h]
  \includegraphics[width=\linewidth]{plots/specs/model_corner_keck_early_7774.png}
  \caption{Corner plot of the $\ion{O}{I}\lambda~7774$ spectral line profile model for the early Keck/LRIS spectrum at 138 days post-peak.}
  \label{fig:a:oxygen:spec:corner:keckearly}
\end{figure*}

\begin{figure*}[h]
  \includegraphics[width=\linewidth]{plots/specs/model_corner_keck_late_gauss.png}
  \caption{Corner plot of the \texttt{gaussian} spectral line profile model for the late Keck/LRIS spectrum at 358 days post-peak.}
  \label{fig:a:oxygen:spec:corner:kecklate}
\end{figure*}



%% \begin{figure*}[h]
%%   \includegraphics[width=\linewidth]{plots/tmp/2021-07-06-oxygen-mass-estimate-corner-keck_early.png}
%%   \caption{Corner Plot of the nested sampling fit to the early Keck LRIS spectrum at 138 days post peak}
%%   \label{fig:mod:neb:oxygen:corner:early}
%% \end{figure*}

%% \begin{figure*}[h]
%%   \includegraphics[width=\linewidth]{plots/tmp/2021-07-06-oxygen-mass-estimate-corner-keck_late.png}
%%   \caption{Corner Plot of the nested sampling fit to the late Keck LRIS spectrum at 358 days post peak}
%%   \label{fig:mod:neb:oxygen:corner:late}
%% \end{figure*}

\FloatBarrier

\subsection{Physical Fitting Results}
\label{appendix:oxygen:physfits}

The resulting corner plots of the fits of the second stage of the fitting routine (outlined in \autoref{sec:oxygen:fitting}) are shown in \autoref{fig:a:oxygen:phys:corner:notlate}, \autoref{fig:a:oxygen:phys:corner:keckearly}, \autoref{fig:a:oxygen:phys:corner:kecklate}.
The used input line fluxes (converted to luminosities) are given in \autoref{tab:mod:neb:oxygen:fluxes}.

\begin{figure}[h]
  \includegraphics[width=\linewidth]{plots/specs/oxygen_phys_not_late_7774.png}
  \caption{Corner plot of the physical oxygen model for the NOT/ALFOSC spectrum at 128 days post-peak.}
  \label{fig:a:oxygen:phys:corner:notlate}
\end{figure}

\begin{figure}[h]
  \includegraphics[width=\linewidth]{plots/specs/oxygen_phys_keck_early_7774.png}
  \caption{Corner plot of the physical oxygen model for the early Keck/LRIS spectrum at 138 days post-peak.}
  \label{fig:a:oxygen:phys:corner:keckearly}
\end{figure}

\begin{figure}[h]
  \includegraphics[width=\linewidth]{plots/specs/oxygen_phys_keck_late_gauss.png}
  \caption{Corner plot of the physical oxygen model for the late Keck/LRIS spectrum at 358 days post-peak.}
  \label{fig:a:oxygen:phys:corner:kecklate}
\end{figure}

\FloatBarrier
\newpage


\section{Photometric Blackbody Fitting}
\label{appendix:bbfit}

We use the interpolated and pre-processed lightcurve constructed by the method described in \autoref{sec:obs:interpolated}.
We construct the time grid to perform the fitting by selecting all observations between the first time all selected filter bands had at least one detection and the peak and then selecting all days post-peak that had at least one observation.
This ensures we are only interpolating the lightcurve and never have to rely on extrapolating it (which is highly uncertain at the very early epochs).

For each time on the time grid we estimate the extinction-corrected magnitude in all filter bands.
Next we fit a three-parameter Bayesian model to these observations: temperature $\log\,T$, radius $\log\, R$ and the distance $D$.
We follow the suggestion by \cite{2022ApJ...937...75A} and use a log-uniform prior for the temperature (although we use nested sampling instead of MCMC).
The distance is a nuisance parameter, which is constrained by the redshift uncertainty.
In the likelihood function we generate a blackbody SED using the temperature and radius.
We then perform synthetic photometry using the corresponding filter curves to compare to the lightcurve.
We perform the posterior calculation using the nested sampling code \textit{dynesty}.

\subsection{Validation}

We compare the resulting temperature and radius between three different filter combinations: $gri$, $ri$ and $rizJH$.
The comparison is shown in \autoref{fig:bbfit:validation:comparison}.
We only have a full spectral coverage from optical to NIR between 40 and 80 days.

\begin{figure}
  \includegraphics[width=\linewidth]{plots/phot/bb_fit_validation_filter_comparison.png}
  \caption{Comparison between three filter sets.}
  \label{fig:bbfit:validation:comparison}
\end{figure}

%% \section{Spectral Fitting}
%% \label{appendix:specfit}

%% Here we describe the spectral fitting algorithm we employ.

%% We use the following likelihood function:
%% \begin{equation}
%%   \log \mathcal{L}(\theta|\lambda,F_\lambda,\sigma) = - \frac{1}{2} \sum_{i=0}^{N} \left( \frac{\left(F_i - M(\lambda_i) \right)^2}{\sigma^2} + \log \sigma^2 \right)
%% \end{equation}

%% \begin{equation}
%%   \sigma^2 = \sigma_F^2 + \sigma_0^2
%% \end{equation}

%% We use dynesty to estimate the posterior distribution.


