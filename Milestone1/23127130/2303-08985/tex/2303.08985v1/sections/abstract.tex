% This document is a model and instructions for \LaTeX.
% This and the IEEEtran.cls file define the components of your paper [title, text, heads, etc.]. *CRITICAL: Do Not Use Symbols, Special Characters, Footnotes, 
% or Math in Paper Title or Abstract.
Sentiment Classification is a fundamental task in the field of Natural Language Processing, and has very important academic and commercial applications. It aims to automatically predict the degree of sentiment present in a text that contains opinions and subjectivity at some level, like product and movie reviews, or tweets. This can be really difficult to accomplish, in part, because different domains of text contains different words and expressions. In addition, this difficulty increases when text is written in a non-English language due to the lack of databases and resources. As a consequence, several cross-domain and cross-language techniques are often applied to this task in order to improve the results. In this work we perform a study on the ability of a classification system trained with a large database of product reviews to generalize to different Spanish domains. Reviews were collected from the MercadoLibre website from seven Latin American countries, allowing the creation of a large and balanced dataset. Results suggest that generalization across domains is feasible though very challenging when trained with these product reviews, and can be improved by pre-training and fine-tuning the classification model.

\begin{IEEEkeywords}
Spanish Sentiment Classification, Natural Language Processing, Cross-domain
\end{IEEEkeywords}