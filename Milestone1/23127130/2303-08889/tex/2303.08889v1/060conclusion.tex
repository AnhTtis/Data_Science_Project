\section{Discussion and Conclusion}



In this paper, we studied the tweet and user-level properties of misinformation tweets that get countered versus those that do not. The in-depth analysis shows that misinformation tweets expressing negative emotion, strong emotion, third-person pronouns, and strategies associated with impoliteness are more likely to result in more countering replies from users. Our result also shows that tweets that get countered have a higher amount of reply engagement in proportion to like, retweet, and quote tweet engagement. Moreover, we develop well-performing classifiers to predict whether a misinformation tweet will be countered or not, and if so, to what degree they will be countered (i.e. the proportion of its replies that end up being counter-replies).



Given the statistical significance of our analysis and the high performance of our classifiers, we demonstrate that it is possible to identify tweets that are more or less likely to get countered. In particular, nearly all of these attributes (tweet linguistic attributes and user attributes) are readily available as soon as the tweet is posted, allowing for the quantity of future counter-misinformation (or the lack thereof) to be reasonably forecast. This can have major implications in times of breaking news or other such events in which large quantities of (mis-)information are posted to online platforms at a rapid rate; in conjunction with state-of-the-art misinformation detection approaches, the counter-reply prediction approach presented in this paper can be used to identify tweets that are less likely to be countered, possibly necessitating additional platform-level approaches to control the spread of misinformation for these tweets. One of these approaches may be adding or increasing interventions to draw attention towards accuracy, an approach that has been shown to be effective in discouraging users from spreading misinformation~\cite{pennycook2020}.



A limitation of this work that it focuses on only one platform: Twitter. On other online platforms, different mechanisms of post and user engagement, as well as information exchange, may be present~\cite{micallef2022cross}, possibly influencing the types of misinformation tweets and posters users will choose to counter. Another limitation is that it studies only one topic (COVID-19 vaccines), which has become one of the most widely discussed topics in our society due to the universal effects of the COVID-19 pandemic. On misinformation-related topics that might be more obscure or less widely discussed (e.g. flat earth theories), it could be possible that the more specific demographics of misinformation and/or counter-reply posters may affect the ways in which they interact. In addition, we only study text in the English language; the dynamics and discussion in other languages and other modalities (images, videos) may differ~\cite{verma2022overcoming}. 



For future work, similar analysis can be performed on the user network surrounding the misinformation poster and counter-reply poster (e.g. their followers and those they follow, how much misinformation these accounts spread, etc.) in order to assess if there are any network-related attributes that may increase the likelihood of counter-replies. In addition, given that we can reasonably determine which tweets will and will not be countered, it would also be valuable to perform user studies or field studies to evaluate if certain characteristics about online encounters with misinformation can increase (or decrease) the likelihood of a user posting a counter-reply. Also, while we explore it in Section~\ref{sec:ineq}, further studies can be done to understand the inequities surrounding counter-reply targets along additional demographic, social, political, and/or geographic dimensions; this can allow further exploration of the greater societal implications surrounding counter-misinformation.


