\section{Example Usage}
\label{sec:usage}

In the following, we show how the new features in \HEJ~2.2 can be used
in practice. For concreteness, we will generate
leading-order events with \textsc{Sherpa}~2.2 and analyse the output with \textsc{Rivet}~3~\cite{Bierlich:2019rhm}. However, we stress that any leading-order
generator that can produce event files in the LHEF
format~\cite{Alwall:2006yp} is supported. In addition to the direct
\textsc{Rivet} interface, \HEJ can write its output to event files in
various formats and allows arbitrary custom analyses via
plugins. Since these options are not new, we refer to the \HEJ
documentation on \url{https://hej.hepforge.org} for details.

\subsection{Same-sign W pair production with jets}
\label{sec:example_WW}

We first consider the process $pp \to (W^- \to e \bar{\nu}_e) (W^- \to
\mu \bar{\nu}_\mu) + \geq 2$ jets with the parameters shown in table~\ref{tab:WW_param}.
\begin{table}[htb]
  \centering
  \renewcommand{\arraystretch}{1.1}
  \begin{tabular}{lp{5mm}l}
    \toprule
    Collider energy && $\sqrt{s} = 13$\,TeV\\[.3em]
    Scales && $\mu_r = \mu_f = \frac{H_T}{2}$\\[.3em]
    PDF set && CT14nlo\\[.3em]
    Electroweak input parameters && $\alpha = 1/132.3572$\\
                    && $m_W = 80.385$\,GeV\\
                    && $\Gamma_W = 2.085$\,GeV\\
                    && $m_Z = 91.1876$\,GeV\\
                    && $\Gamma_Z = 2.4952$\,GeV\\[.3em]
    Jet definition && anti-$k_t$~\cite{Cacciari:2008gp} \\
                    && $R = 0.4$ \\
                    && $p_\perp > 20$\,GeV\\
    \bottomrule
  \end{tabular}
  \caption{Parameters for the production of multiple jets together with a same-sign W boson pair decaying to charged leptons and neutrinos.}
  \label{tab:WW_param}
\end{table}

\subsubsection{Generating leading-order input}
\label{sec:input_WW}

To produce the required leading-order input, we can use \textsc{Sherpa} with the following runcard.
\lstinputlisting[language=sherpa,title=\texttt{Run.dat}]{examples/WW/Run.dat}
The various settings are explained in more detail in the
\textsc{Sherpa} manual. Since \HEJ treats all quarks as massless, we
have to set the charm and bottom quark masses to zero for
consistency. As explained in section~\ref{sec:low_pt}, leading-order
events containing particles with transverse momenta below the analysis
cuts can still contribute to the resummed prediction. For this reason,
we accept jets with transverse momenta as low as 15\,GeV, despite
having an analysis cut of 20\,GeV. In section~\ref{sec:low_pt_ex}, we
will discuss a computationally more efficient way to incorporate this
contribution from leading-order events that do not pass the analysis
transverse momentum cuts.

We can now generate input events for the case of two jets by running
\begin{lstlisting}[language=sh]
Sherpa
\end{lstlisting}
in the directory containing \texttt{Run.dat}. We should then also
produce event files for higher jet multiplicities after adjusting the
\lstinline!EVENT_OUTPUT!, \lstinline!Process!, and
\lstinline!FastjetFinder! entries in \texttt{Run.dat}.

To avoid the creation of large intermediate event files, we can use
named pipes instead, i.e.~we run \textsc{Sherpa} with
\begin{lstlisting}[language=sh]
mkfifo events_WW2j.lhe
Sherpa &
\end{lstlisting}
Previous \HEJ versions only accepted input from pipes if the total
cross section was equal to the sum of event weights, which is not the
case for \textsc{Sherpa} event files. This restriction is lifted in the new
version $2.2$, which removes a potential bottleneck in time and storage.

\subsubsection{\HEJ resummation}
\label{sec:resum_WW}

In addition to the leading-order event input, \HEJ needs a
configuration file. Adapting the template \texttt{config.yml}
distributed with the \HEJ source code to the parameters listed in
table~\ref{tab:WW_param} we get
\lstinputlisting[language=yaml,title=\texttt{config\_WWjets.yml}]{examples/WW/config_WWjets.yml}
Here, we choose to pass the resummed events directly to the
\textsc{Rivet} analyses\footnote{The   \texttt{MC\_WWINC} and \texttt{MC\_WWJETS} analyses are written for opposite-sign W boson pair production but can also be used for same-sign W boson pairs.} \texttt{MC\_WWINC} and \texttt{MC\_WWJETS}. Using the Docker
virtualisation software, we can run \HEJ~\cite{docker} with the
following command.
\begin{lstlisting}[language=sh]
docker run -v $PWD:$PWD -w $PWD hejdock/hej HEJ config_WWjets.yml events_WW2j.lhe
\end{lstlisting}
Alternatively, after compiling and installing \HEJ and its dependencies we can use
\begin{lstlisting}[language=sh]
HEJ config_WWjets.yml events_WW2j.lhe
\end{lstlisting}
This produces the \textsc{Rivet} analysis output file
\texttt{WW2j.yoda}. We produce resummed predictions for the
higher-multiplicity event files \texttt{events\_WWnj.lhe} in the same
way, after changing the \lstinline!analyses! entry in
\texttt{config\_WWjets.yml} to
\begin{lstlisting}[language=yaml]
analyses:
  - rivet: [MC_WWINC, MC_WWJETS]
    output: WWnj
\end{lstlisting}
and adjusting the event file name when running \HEJ. To guarantee
statistically independent output, it is also recommended to change the
\lstinline!seed! entry for each run.

We then combine the results for the different jet multiplicities with
\begin{lstlisting}[language=sh]
yodastack -o HEJ_WWjets.yoda HEJ_WW*j.yoda
\end{lstlisting}
and produce plots with
\begin{lstlisting}[language=sh]
rivet-mkhtml HEJ_WWjets.yoda
\end{lstlisting}
As examples, we show the inclusive jet multiplicities and the
distribution of the rapidity difference between the two W bosons
obtained from resumming fixed-order predictions with two and three jets
in figure~\ref{fig:distr}.
\begin{figure}[htb]
  \centering
  \includegraphics[width=0.47\linewidth]{examples/WW/jet_multi_inclusive}
  \includegraphics[width=0.47\linewidth]{examples/WW/WW_deta}
  \caption{%
    Inclusive $N$-jet cross sections (left) and rapidity difference
    between the two W bosons (right) obtained with \textsc{Sherpa} and
    \HEJ 2.2 for the production of two leptonically decaying W$^-$ bosons
    with at least two jets.
  }
  \label{fig:distr}
\end{figure}

\subsubsection{Dedicated low transverse momentum runs}
\label{sec:low_pt_ex}

So far, we have generated the leading-order events with significantly
looser transverse momentum cuts than wanted for the final analysis. As
argued in section~\ref{sec:low_pt}, it is more efficient to
split the generation into a high-statistics run with the strict cuts
used in the final analysis and a low-statistics run with loose cuts
where in each leading-order event there is at least one particle that
does not pass the final cuts. For the jet transverse momentum cuts,
this separation is facilitated by a new option in \HEJ 2.2 which
ensures the presence of at least one ``soft'' jet in the input for the
low-statistics run. Note that this option only refers to jets; any
other particles should be generated according to the loose transverse
momentum cuts in both samples.

In detail, one should go through the following steps for the present
example of same-sign W pair production with jets:
\begin{enumerate}
\item Generate the high-statistics sample.
\begin{enumerate}
  \item Change the minimum transverse momentum in the \lstinline!FastjetFinder! entry in \texttt{Run.dat} from 15 to 20.
  \item Correspondingly, change
\begin{lstlisting}[language=yaml]
fixed order jets:
  min pt: 15
\end{lstlisting}
    to
\begin{lstlisting}[language=yaml]
fixed order jets:
  min pt: 20
\end{lstlisting}
    in \texttt{config\_WWjets.yml}
  \item Run \textsc{Sherpa} and \HEJ as before.
  \item Revert the changes to both configuration files.

  \end{enumerate}
\item Generate the low-statistics sample.
  \begin{enumerate}
  \item Reduce the number of events generated by Sherpa, for example
by setting \lstinline!EVENTS! entry in \texttt{Run.dat} to 1000.
  \item Add the entry
\begin{lstlisting}[language=yaml]
require low pt jet: true
\end{lstlisting}
to \texttt{config\_WWjets.yml}.
\item In the \lstinline!event treatment! entry in \texttt{config\_WWjets.yml},
  change all \lstinline!keep! values to \lstinline!discard!. Specifically, change
\begin{lstlisting}[language=yaml]
event treatment:
  FKL: reweight
  unordered: keep
  extremal qqbar: keep
  central qqbar: keep
  non-resummable: keep
\end{lstlisting}
to
\begin{lstlisting}[language=yaml]
event treatment:
  FKL: reweight
  unordered: discard
  extremal qqbar: discard
  central qqbar: discard
  non-resummable: discard
\end{lstlisting}
\item Change the name of the output file, e.g.
\begin{lstlisting}[language=yaml]
analyses:
  - rivet: [MC_WWINC, MC_WWJETS]
    output: WW2j_lowpt
\end{lstlisting}
  \item Run \textsc{Sherpa} and \HEJ.
  \end{enumerate}
\end{enumerate}
After repeating these steps for higher jet multiplicities, the
samples can again be combined with
\begin{lstlisting}[language=sh]
yodastack -o HEJ_WWjets.yoda HEJ_WW*j.yoda HEJ_WW*j_lowpt.yoda
\end{lstlisting}
to reproduce the results obtained in section~\ref{sec:resum_WW} with better statistics.

Note that the \HEJFOG is already optimised for generating event
samples with a small fraction of low transverse momentum events using
its \lstinline!peak pt! option. Here, the separate generation of two
samples as above is unnecessary and in fact inferior to the built-in
generation.

\subsection{Higgs boson production with one or more jets}
\label{sec:example_H}

% \todo[inline]{%
%   Sherpa does not write the photons to the Les Houches file for $H \to
%   \gamma\gamma$.  I think we can only do $H \to \gamma\gamma$ because
%   the Sherpa that's installed on the grid is a custom version hacked by
%   Marian.  So we have to talk about stable Higgs production here.
% }

We now consider the production of a Higgs boson together with one or
more jets. We use the parameters listed in
table~\ref{tab:H_param}. For the sake of simplicity, we first consider
the limit of an infinitely heavy top quark.

\begin{table}[htb]
  \centering
  \renewcommand{\arraystretch}{1.1}
  \begin{tabular}{lp{5mm}l}
    \toprule
    Collider energy && $\sqrt{s} = 13$\,TeV\\[.3em]
    Scales && $\mu_r = \mu_f = \frac{H_T}{2}$\\[.3em]
    PDF set && CT14nlo\\[.3em]
    Jet definition && anti-$k_t$ \\
                    && $R = 0.4$ \\
                    && $p_\perp > 20$\,GeV\\
    \bottomrule
  \end{tabular}
  \caption{Parameters for the production of a Higgs boson together with at least one jet.}
  \label{tab:H_param}
\end{table}

In close analogy with section~\ref{sec:example_WW}, we first generate
leading-order input events for Higgs boson production with a single
jet. We use \textsc{Sherpa} with the following run card
\lstinputlisting[language=sherpa,title=\texttt{Run.dat}]{examples/H/Run.dat}
For the resummation, we use a similar \HEJ configuration file as
before. Anticipating further runs with higher multiplicities, we
enable resummation for the supported NLL configurations. In the
present case this concerns configurations involving an unordered
gluon, which first contribute to Higgs boson plus dijet
production, cf.~section~\ref{sec:HEJ_summary}. Since there is no
standard \textsc{Rivet} analysis for stable Higgs boson production,
we use the generic \texttt{MC\_JETS} analysis.
\lstinputlisting[language=sherpa,title=\texttt{config\_Hjets.yml}]{examples/H/config_Hjets.yml}
As described in section~\ref{sec:example_WW}, we then add predictions
for higher jet multiplicities.

\subsubsection{Quark mass corrections}
\label{sec:H_mt}

For accurate predictions in the high-energy region, we have to take
into account the finite top quark mass. In the following, we assume a
mass of 174\,GeV. On the \textsc{Sherpa} side, we can add \textsc{AMEGIC}~\cite{Krauss:2001iv}
 and \textsc{OpenLoops}~\cite{Cascioli:2011va} to
\lstinline!ME_SIGNAL_GENERATOR! and insert the following lines into the
\lstinline[language=sherpa]!(run)! block:
\begin{lstlisting}[language=sherpa]
  # finite top mass effects
  KFACTOR GGH
  OL_IGNORE_MODEL 1
  OL_PARAMETERS preset 2 allowed_libs pph2,pphj2,pphjj2 psp_tolerance 1.0e-7
\end{lstlisting}
\HEJ needs to be compiled with support for
\textsc{QCDLoop}~\cite{Carrazza:2016gav} to incorporate quark mass
corrections in Higgs boson production. We can include them by adding
\begin{lstlisting}[language=yaml]
Higgs coupling:
   use impact factors: false
   mt: 174
\end{lstlisting}
to the configuration file.

For higher jet multiplicities, we face the problem that it is no
longer feasible to compute the leading-order input with exact
dependence on the top quark mass $m_t$. However, we can still retain this
dependence and also include the dependence on the bottom-quark mass $m_b$ in
the high-energy resummation.

As in equation~(\ref{eq:weight}), the weight $w$ of a leading-order matched
resummation event has the following dependence on the leading-order
matrix element $\mathcal{M}_{\text{LO}}$ and the all-order \HEJ matrix
element $\mathcal{M}_{\HEJ}(m_b, m_t)$:
\begin{equation}
  \label{eq:ME_wt}
  w \propto  \frac{\lvert\mathcal{M}_{\text{LO}}(m_b, m_t)\rvert^2 \lvert\mathcal{M}_{\HEJ}(m_b, m_t)\rvert^2}{\lvert\mathcal{M}_{\HEJ, \text{LO}}(m_b, m_t)\rvert^2}.
\end{equation}
% To avoid double counting, we have to divide by the modulus square of
% the leading-order truncation $\mathcal{M}_{\HEJ, \text{LO}}$ of the
% \HEJ matrix element, calculated for the kinematics of the
% leading-order input event.
For consistency, the values for the quark
masses have to match those used in
$\mathcal{M}_{\text{LO}}$. Therefore, if the leading-order input is
only known for $m_b \to 0, m_t \to \infty$ the correct reweighting
factor is
\begin{equation}
  \label{eq:ME_wt_mtinf}
  w \propto  \frac{\lvert\mathcal{M}_{\text{LO}}(0, \infty)\rvert^2 \lvert\mathcal{M}_{\HEJ}(m_b, m_t)\rvert^2}{\lvert\mathcal{M}_{\HEJ, \text{LO}}(0, \infty)\rvert^2} =  \frac{\lvert\mathcal{M}_{\text{LO}}(0, \infty)\rvert^2 \lvert\mathcal{M}_{\HEJ}(m_b, m_t)\rvert^2}{\lvert\mathcal{M}_{\HEJ, \text{LO}}(m_b, m_t)\rvert^2} \times \frac{\lvert\mathcal{M}_{\HEJ, \text{LO}}(m_b, m_t)\rvert^2}{\lvert\mathcal{M}_{\HEJ, \text{LO}}(0, \infty)\rvert^2}.
\end{equation}
Currently, there is no built-in \HEJ option for choosing different
quark mass values in $\mathcal{M}_{\HEJ}$ and $\mathcal{M}_{\HEJ,
\text{LO}}$. However, \HEJ supports flexible custom analyses, which
allow us to manually reweight by the correction factor
$\lvert\mathcal{M}_{\HEJ, \text{LO}}(m_b, m_t)\rvert^2 / \lvert\mathcal{M}_{\HEJ, \text{LO}}(0,
\infty)\rvert^2$ in equation~\eqref{eq:ME_wt_mtinf}. We can also use this
opportunity to include bottom quark mass corrections in the case where
only the exact leading-order dependence on the top quark mass is available.

Custom analyses are described in the \HEJ user documentation on
\url{https://hej.hepforge.org}, where also a template is provided. The
reweighting can be implemented as shown here:
\lstinputlisting[language=C++,title=\texttt{higgs\_matching\_analysis.cc}]{examples/H/higgs_matching_analysis.cc}
We can then compile the analysis into a shared object library with a
compiler supporting C++17, for instance a recent version of
\texttt{g++}:
\begin{lstlisting}[language=sh]
g++ -Wall -Wextra $(HEJ-config --cxxflags) -fPIC -shared -O2 \
  -fvisibility=hidden \
  -Wl,-soname,libhiggs_matching_analysis.so \
  -o libhiggs_matching_analysis.so higgs_matching_analysis.cc
\end{lstlisting}
To use our custom analysis, we adjust the \HEJ configuration file. We
use YAML anchors (starting with \lstinline!&!) and references
(starting with \lstinline!*!) to ensure that the settings passed to
the analysis are consistent. The following code listing is for the
case of a leading-order prediction in the infinite top-quark mass
limit.
\lstinputlisting[language=yaml,title=\texttt{config\_Hjets\_mbmt.yml}]{examples/H/config_Hjets_mbmt.yml}
If the leading order prediction includes the exact dependence on the
top-quark mass, but not the dependence on the bottom-quark mass, one
should replace the \lstinline!LO Higgs coupling! entry by
\begin{lstlisting}[language=yaml]
    LO Higgs coupling:
      use impact factors: false
      mt: 174
\end{lstlisting}

\subsubsection{Matching to next-to-leading order}
\label{sec:match_NLO_ex}

Using \HEJ 2.2, we can extend the fixed-order accuracy of
distributions from LO to NLO cf.~section~\ref{sec:match_NLO}. In the
following, we consider NLO matching for the production of a Higgs
boson together with a single jet in the limit of an infinitely heavy
top quark. We apply the matching bin-by-bin in the resulting
histograms. We can generate an NLO prediction using \textsc{Sherpa} and
\textsc{OpenLoops} by adjusting the run card as follows.
\lstinputlisting[language=sherpa,title=\texttt{Run.dat}]{examples/H/Run_NLO.dat}
We obtain histograms by running \textsc{Rivet} on the event output:
\begin{lstlisting}[language=sh]
  rivet -o Hj_NLO.yoda --ignore-beams -a MC_JETS -q events_H1j_NLO.hepmc
\end{lstlisting}
For the \HEJ prediction, we add the option
\begin{lstlisting}[language=yaml]
NLO truncation:
  enabled: true
  nlo order: 1    # number of jets
\end{lstlisting}
to the configuration file from section~\ref{sec:example_H}, change the
name of the \textsc{Rivet} output file to \texttt{Hj\_HEJ\_NLO.yoda},
and run \HEJ on the previously generated leading-order input file
assuming an infinitely heavy top quark. To obtain the final NLO matched
prediction, we should multiply each histogram in the original \HEJ
output by the ratio of the corresponding histograms in
\texttt{Hj\_NLO.yoda} and \texttt{Hj\_HEJ\_NLO.yoda}. The following
script gives an example of how this reweighting can be
implemented. For the sake of brevity we have omitted the error
handling code, which is of course essential in actual applications.
\lstinputlisting[language=python,title=\texttt{reweight\_NLO.py}]{examples/H/reweight_NLO.py}
Note that the \texttt{MC\_WJETS} \textsc{Rivet} analysis employs
adaptive binning for a number of distributions, causing the
division of the respective histograms to fail. This problem can of
course be circumvented by using a custom analysis.

\subsection{Charged lepton pair production with many jets}
\label{sec:HEJFOG_ex}

The production of two charged leptons with jets was first implemented
in \HEJ~2.1~\cite{Andersen:2021qma} for at most moderate jet
multiplicities, where exact fixed-order matching is feasible. In
\HEJ~2.2, approximate high-multiplicity events can be generated with
the \HEJ Fixed Order Generator.

In the following example, we consider the process $pp \to \mu^+ \mu^-
+ \geq$ 2 jets with the same parameters as in previous examples, see
table~\ref{tab:WW_param}. We assume that predictions including up to 4
jets have already been produced as described in
sections~\ref{sec:example_WW} and~\ref{sec:example_H}. To generate
input events with 5 jets, we adapt the configuration file
\texttt{configFO.yml} distributed together with \HEJ:
\lstinputlisting[language=yaml,title=\texttt{configFO\_Zjets.yml}]{examples/Z/configFO_Zjets.yml}
By setting the \lstinline!peak pt! option we ensure that most events
are generated above the analysis cut of 20\,GeV. This means that there
is no need for two separate runs with different transverse momentum
cuts as described in section~\ref{sec:low_pt_ex} for conventional
fixed-order generators. We can now generate events with
\begin{lstlisting}
docker run -v $PWD:$PWD -w $PWD hejdock/hej HEJFOG config_HEJFOG.yml
\end{lstlisting}
when using the \HEJ Docker container or
\begin{lstlisting}
HEJFOG config_HEJFOG.yml
\end{lstlisting}
when using a local \HEJ installation. Afterwards, we run \HEJ and
combine the results from different multiplicities as described in
section~\ref{sec:resum_WW}.

%%% Local Variables:
%%% mode: latex
%%% TeX-master: "main"
%%% End:
