\section{Features of \HEJ 2.2}
\label{sec:features}

\subsection{\HEJ in a nutshell}
\label{sec:HEJ_summary}

Before describing the changes in \HEJ 2.2, let us briefly review the
general formalism and program structure. As input, \HEJ requires
leading-order (LO) events, generated with
e.g.~\textsc{Sherpa}~\cite{Bothmann:2019yzt} or
\textsc{MadGraph5\_aMC@NLO}~\cite{Alwall:2014hca}. For higher jet
multiplicities exact fixed-order generation becomes increasingly time
consuming. To address this problem, \HEJ includes the fast \HEJ
fixed-order generator \HEJFOG based on the high-energy approximation
of the leading-order matrix elements.

Using the kinematics of each (approximate or exact) input event, we
identify whether resummation is possible. For each event that permits
resummation, \HEJ generates a number of matching events in the
resummation phase space, which include real and virtual corrections to
all orders in the high-energy limit. Details are given
in~\cite{Andersen:2018tnm}. Together with the unchanged non-resummable
input events, the generated resummation events are then passed on to
any number of output event files and/or analyses. This standard control flow is
depicted in figure~\ref{fig:flow}. It can be modified through \HEJ
options, such that e.g.~non-resummable events are discarded.
\begin{figure}[htb]
  \centering
  \includegraphics{flow}
  \caption{Standard \HEJ control flow.}
  \label{fig:flow}
\end{figure}

The first type of event kinematics for which resummation is
implemented are leading-logarithmic (LL) configurations, which for
pure multijet production have to fulfil the following constraints:
\begin{enumerate}
\item The flavour of the most backward outgoing parton has to match
the flavour of the backward incoming parton.
\item The flavour of the most forward outgoing parton has to match
the flavour of the forward incoming parton.
\item All other outgoing partons have to be gluons.
\end{enumerate}
These criteria remain the same in processes involving virtual photons
and/or Z bosons. For virtual W bosons, the incoming and outgoing
flavours in criteria 1 and 2 only have to match up to the change
induced by W boson couplings. In the case of a final-state Higgs
boson, configurations where the backward (forward) incoming parton is
a quark or antiquark and the most backward (forward) outgoing particle
is the Higgs boson are formally subleading. Nevertheless, we also
implement resummation for such configurations. Depending on the
process, resummation is also implemented for two further types of
next-to-leading-logarithmic (NLL) configurations. These configurations
differ from LL ones as follows.
\begin{itemize}
\item \emph{Unordered gluon:} Either the most forward or most backward
outgoing parton is a gluon, and the next outgoing parton in rapidity
order is a quark or antiquark whose flavour matches the one of the
respective incoming parton.
\item \emph{Quark-antiquark:} A pair of final-state gluons that are
adjacent in rapidity is replaced by a quark-antiquark pair.
\end{itemize}
The current status of the implemented resummation is summarised in
table~\ref{tab:summary}.

\begin{table}[tb]
  \centering
  \begin{tabular}{lcccc}
    \toprule
    Process & pure LL & LO + LL & \multicolumn{2}{c}{NLL} \\
    \cmidrule(lr){4-5}
    &&&unordered gluon & quark-antiquark\\
    \midrule
    $\geq$2 jets & \HEJ 2.0 & \HEJ 2.0 & \HEJ 2.0 & \HEJ 2.1\\
    H + 1 jet & --- & \HEJ 2.2 & N/A & N/A \\
    H + $\geq$2 jets & \HEJ 2.0 & \HEJ 2.0 & \HEJ 2.0 & ---\\
    W + $\geq$2 jets & \HEJ 2.1 & \HEJ 2.1 & \HEJ 2.1 & \HEJ 2.1\\
    Z/$\gamma$ + $\geq$2 jets & \HEJ 2.2 & \HEJ 2.1 & \HEJ 2.1 & ---\\
    W$^\pm$W$^\pm$ + $\geq$2 jets & --- & \HEJ 2.2 & --- & ---\\
    \bottomrule
  \end{tabular}
  \caption{%
    Implemented processes and higher-order logarithmic corrections in
    \HEJ. The ``pure LL'' column lists processes implemented in the
    \HEJFOG. The NLL columns include both pure NLL and NLL matched to LO.
  }
  \label{tab:summary}
\end{table}

The resummation events generated for the LL and supported NLL
configurations are given a final matrix element weight of
\begin{equation}
  \label{eq:weight}
  \lvert\mathcal{M}_{\HEJ}\rvert^2 \frac{\lvert\mathcal{M}_{\text{LO}}\rvert^2}{\lvert\mathcal{M}_{\HEJ, \text{LO}}\rvert^2},
\end{equation}
where
$\mathcal{M}_{\HEJ}$ is the all-order scattering matrix element in the
high-energy approximation, $\mathcal{M}_{\HEJ, \text{LO}}$ its
leading-order truncation, calculated for the kinematics of the input
event and $\lvert\mathcal{M}_{\text{LO}}\rvert^2$ is taken from the LO input.

To illustrate the structure of the \HEJ matrix element, we first focus on LL
configurations in pure multijet production. We denote these configurations as $f_a f_b \to f_a \cdots f_b$, where $f_a$ is the flavour of the incoming parton in the backward direction with momentum
$p_a$. Correspondingly, we use $f_b$ and $p_b$ for the flavour and
momentum of the forward incoming parton. The final state contains $n$
partons with momenta $p_1,\dots,p_n$, which we order by rapidity,
viz.~$y_1 < \dots < y_n$. The most backward outgoing parton has
flavour $f_{a}$, the most forward one flavour $f_{b}$, and all other
outgoing partons are gluons. Using this notation, we can write the general form of the squared \HEJ
matrix element as
\begin{equation}
  \label{eq:ME_fact}
  \begin{split}
    \overline{\left\lvert\mathcal{M}_{\HEJ}^{f_a f_b \to f_a \cdots f_b}\right\rvert}^2 ={}&\mathcal{B}_{f_a,f_b}(p_a, p_b, p_1, p_n)\\
    &\cdot \prod_{i=1}^{n-2} \mathcal{V}(p_a,p_b,p_1,p_n, q_i, q_{i+1})\\
    &\cdot \prod_{i=1}^{n-1} \mathcal{W}(q_{i\perp}, y_i, y_{i+1}),
  \end{split}
\end{equation}
where $q_i=p_a - \sum_{j=1}^i p_j$ is the $t$-channel momentum after
the emission of parton $i$. $\mathcal{B}_{f_a,f_b}$ is derived from the modulus
square of the Born-level matrix element for the process $f_a f_b \to
f_a f_b$, $\mathcal{V}$ accounts for the real emission of the $n-2$
gluons in between $f_a$ and $f_b$, and $\mathcal{W}$ incorporates the
virtual and unresolved real corrections.

The Born-level function $\mathcal{B}_{f_a,f_b}$ is given by
\begin{equation}
  \label{eq:B}
  {\cal B}_{f_a,f_b}(p_a, p_b, p_1, p_n) = \frac{(4\pi\alpha_s)^n}{4 (N_C^2 - 1)} \frac{K_{f_a}}{q_1^2} \frac{K_{f_b}}{q_{n-1}^2} \|S_{f_a f_b \to f_{a}\cdots f_{b}}\|^2 ,
\end{equation}
where $\alpha_s$ is the strong coupling constant and $N_C = 3$ the number of colours. $K_{f_a}$ and $K_{f_b}$ are
generalised colour factors depending on the respective parton flavour
and, in the case of gluons, also the parton momentum. For quarks and
antiquarks one finds $K_f = C_F = \frac{N_C^2-1}{2N_C}$; the factor $K_g$ for gluons is
derived in~\cite{Andersen:2009he}. $S_{f_a f_b \to f_{a}\cdots f_{b}}$
denotes the contraction of two currents:
\begin{equation}
  \label{eq:S_QCD}
  \| S_{f_a f_b \to f_a \cdots f_b}\|^2 \equiv \| j^a \cdot j^b\|^2 = \sum_{\substack{\lambda_a=+,-\\ \lambda_b=+,-}}\lvert j^{\mu,\lambda_a}(p_1, p_a) j_\mu^{\lambda_b}(p_n, p_b)\rvert^2,
\end{equation}
where $j_\mu^\lambda$ is the current
\begin{equation}
  \label{eq:j}
  j_\mu^\lambda(p,q) = \bar{u}^\lambda(p) \gamma_\mu u^\lambda(q)
\end{equation}
for helicity $\lambda$. \HEJ employs the symbolic manipulation
language \texttt{FORM}~\cite{Vermaseren:2000nd} to generate compact
symbolic expressions for current contractions.

The real corrections are given by contractions of Lipatov vertices~\cite{Andersen:2017kfc}:
\begin{align}
  \label{eq:V}
  \mathcal{V}(p_a,p_b,p_1,p_n, q_i, q_{i+1}) ={}& -\frac{C_A}{q_i^2 q_{i+1}^2} V_\mu(p_a,p_b,p_1,p_n, q_i, q_{i+1}) V^\mu(p_a,p_b,p_1,p_n, q_i, q_{i+1}),\\
  \label{eq:V_Lipatov}
  V^\mu(p_a,p_b,p_1,p_n, q_i, q_{i+1})={}& -(q_i+q_{i+1})^\mu \nonumber\\
  &+ \frac{p_a^\mu}{2} \left( \frac{q_i^2}{p_{i+1}\cdot p_a} +
  \frac{p_{i+1}\cdot p_b}{p_a\cdot p_b} + \frac{p_{i+1}\cdot p_n}{p_a\cdot p_n}\right) +
p_a \leftrightarrow p_1 \nonumber\\
  &- \frac{p_b^\mu}{2} \left( \frac{q_{i+1}^2}{p_{i+1} \cdot p_b} + \frac{p_{i+1}\cdot
      p_a}{p_b\cdot p_a} + \frac{p_{i+1}\cdot p_1}{p_b\cdot p_1} \right) - p_b
  \leftrightarrow p_n,
\end{align}
with $C_A = N_C$.

Finally, the virtual and unresolved real corrections $\mathcal{W}$ can
be expressed in terms of the regularised Regge trajectory $\omega^0$:
\begin{equation}
  \label{eq:W}
  \mathcal{W}(q_{j\perp},y_j,y_{j+1}) = \exp[\omega^0(q_{j\perp}) (y_{j+1}- y_j)].
\end{equation}
For a detailed discussion and an explicit expression for $\omega^0$
see~\cite{Andersen:2018tnm}.

The generalisations to NLL configurations and additional non-partonic
final state particles are derived
in~\cite{Andersen:2012gk,Andersen:2016vkp,Andersen:2017kfc,Andersen:2018kjg,Andersen:2020yax,Andersen:2021vnf,Andersen:2022zte}. In
all cases one finds a factorisation into a Born-level function,
resolved real emissions, and virtual and unresolved real
corrections. In the absence of interference, one recovers the same
structure as in equation~\eqref{eq:ME_fact}. In particular,
the functions $\mathcal{V}$ and $\mathcal{W}$ comprising the all-order
corrections are universal, whereas the Born-level function
$\mathcal{B}$ is process dependent.

\subsection{High-energy resummation for W pair production}
\label{sec:HEJ_WW}

Based on the pure-QCD LL configurations $f_a f_b \to f_a \cdots f_b$,
additional W bosons can be produced via emission off the partons $f_a$
and $f_b$. In \HEJ 2.2, we consider LL configurations with two
leptonically decaying W bosons with equal charge. For definiteness, we
discuss configurations $f_a f_b \to (W^- \to e \bar{\nu}_e) (W^-
\to \mu \bar{\nu}_\mu) f_{a'} \cdots f_{b'}$, where the rapidities of
the final-state charged and neutral leptons do not necessarily respect
any rapidity ordering. Note that the couplings to the W bosons induce
flavour changes $f_a \to f_{a'}$ and $f_b \to f_{b'}$. The production
of two positively charged W bosons and the decay of the two W bosons
into the same lepton flavours is completely analogous.

We identify two contributions to the amplitude. Parton $f_a$ can
either couple to the W boson that decays into an electron and its antineutrino or
to the W boson decaying into a muon and its antineutrino. In the first case, the $t$-channel momenta are given by
\begin{equation}
  \label{eq:t-channel_e}
  q_{i,e} = p_a - p_e - p_{\bar{\nu}_e} - \sum_{j=1}^i p_j,
\end{equation}
where $p_e$ is the momentum of the electron and $p_{\bar{\nu}_e}$ the
momentum of its antineutrino. In the second case the $t$-channel momenta
are
\begin{equation}
  \label{eq:t-channel_mu}
  q_{i,\mu} = p_a - p_\mu - p_{\bar{\nu}_\mu} - \sum_{j=1}^i p_j
\end{equation}
with the muon momentum $p_\mu$ and the corresponding antineutrino momentum
$p_{\bar{\nu}_\mu}$. The resulting modulus square of the matrix element including interference is~\cite{Andersen:2021vnf}
\begin{equation}
\label{eq:ME_Z}
\begin{split}
  \left\lvert\mathcal{M}_{\HEJ}^{f_a f_b\to e\bar{\nu}_e \mu \bar{\nu}_\mu f_{a'} \cdots f_{b'}}\right\rvert^2 &=\ \frac{(4 \pi \alpha_s)^n}{4(N_c^2-1)}\ K_{f_a}K_{f_b} C_A^{n-2}\\
  \times \Bigg( &{\frac{\| j^a_{W,e}\cdot
    j^b_{W,\mu}\|^2}{q^2_{1,e}q^2_{n-1,e}}}
\prod^{n-2}_{i=1} {\frac{-V^2(q_{i,e},
  q_{i+1,e})}{q^2_{i,e} q^2_{i+1,e}}} \prod_{i=1}^{n-1}
  \mathcal{W}(q_{i,e\perp},y_i,y_{i+1})\\
+\ &{\frac{\|j^a_{W,\mu} \cdot j^b_{W,e} \|^2}{q^2_{1,\mu}q^2_{n-1,\mu}}}
     \prod^{n-2}_{i=1}{\frac{-V^2(q_{i,\mu}, q_{i+1,\mu})}{q^2_{i,\mu} q^2_{i+1,\mu}}} \prod_{i=1}^{n-1}
\mathcal{W}(q_{i,\mu\perp},y_i,y_{i+1}) \\
-\ &{\frac{2\Re\{ (j^a_{W,e}\cdot j^b_{W,\mu})(\overline{j^a_{W,\mu} \cdot
      j^b_{W,e}})\}}{\sqrt{q^2_{1,e}q^2_{1,\mu}}\sqrt{q^2_{n-1,e} q^2_{n-1,\mu}}}}\\
&\times \prod^{n-2}_{i=1}{\frac{V(q_{i,e}, q_{i+1,e})\cdot V(q_{i,\mu},
         q_{i+1,\mu})}{\sqrt{q^2_{i,e}q^2_{i,\mu}} \sqrt{q^2_{i+1,e}q^2_{i+1,\mu}}}} \prod_{i=1}^{n-1}
         \mathcal{W}(\sqrt{q_{i,e\perp}q_{i,\mu\perp}},y_i,y_{i+1})\Bigg).
\end{split}
\end{equation}
Here, we have introduced contractions between generalised currents
$j^c_{W,l}$ accounting for the coupling between a parton with flavour
$f_c$ and a W boson decaying into a charged lepton $l$ and the
corresponding antineutrino. The contractions are
\begin{align}
  \label{eq:j_contr_sq_1}
  \| j^a_{W,e} \cdot j^b_{W,\mu}\|^2 ={}& \left\lvert j_V^{\rho,\lambda_a \lambda_e}(p_1, p_a, p_e, p_{\bar{\nu}_e}) j_{V}^{\sigma,\lambda_b\lambda_\mu}(p_{n+2}, p_b, p_\mu, p_{\bar{\nu}_\mu}) g_{\rho\sigma}\right\rvert^2,\\
  \label{eq:j_contr_sq_2}
  \| j^a_{W,\mu} \cdot j^b_{W,e}\|^2 ={}& \left\lvert j_V^{\rho,\lambda_a \lambda_\mu}(p_1, p_a, p_\mu, p_{\bar{\nu}_\mu}) j_{V}^{\sigma,\lambda_b\lambda_e}(p_{n+2}, p_b, p_e, p_{\bar{\nu}_e})g_{\rho\sigma}\right\rvert^2,\\
  \label{eq:j_contr_mixed}
  \begin{split}
  (j^a_{W,e}\cdot j^b_{W,\mu})(\overline{j^a_{W,\mu} \cdot
      j^b_{W,e}})={}&
    j_V^{\rho,\lambda_a \lambda_e}(p_1, p_a, p_e, p_{\bar{\nu}_e}) j_{V}^{\sigma,\lambda_b\lambda_\mu}(p_{n+2}, p_b, p_\mu, p_{\bar{\nu}_\mu}) g_{\rho\sigma} \\
    &\qquad\times \overline{j_{V}^{\alpha\lambda_a \lambda_\mu}(p_1, p_a, p_\mu,
      p_{\bar{\nu}_\mu}) j_V^{\beta,\lambda_b \lambda_e}(p_{n+2}, p_b, p_e,
      p_{\bar{\nu}_e})}g_{\alpha\beta} ,
  \end{split}
\end{align}
where the parton helicities $\lambda_a$ and $\lambda_b$ are determined
by the flavour of the respective parton, namely $\lambda_c=-$ if $f_c$
is a quark and $\lambda_c=+$ if $f_c$ is an antiquark. The electron
helicity $\lambda_e$ and the muon helicity $\lambda_\mu$ correspond to
the charge sign of the parent W boson, i.e.\ $\lambda_e = \lambda_\mu
= -$ in the present case. We have introduced a generalised current
$j_V^{\rho,\lambda_a \lambda_\ell}$ for the coupling of a parton with
helicity $\lambda$ to a leptonically decaying vector
boson\footnote{Here, we assume a W boson. However, the same expression
holds for neutral vector bosons after replacing the antineutrino
momentum $p_{\bar{\nu}_{\ell}}$ by the antilepton momentum
$p_{\bar{\ell}}$.} $V$ with lepton helicity $\lambda_{\ell}$. It is
given by~\cite{Andersen:2020yax}
\begin{equation}
  \label{eq:j_V}
  \begin{split}
  j_V^{\rho,\lambda_a \lambda_\ell}(p_1,p_a,p_{\ell},p_{\bar{\nu}_{\ell}}) =&\ \frac{g_V^2}{2}\
     \frac1{p_V^2-M_V^2+i\ \Gamma_V M_V}\ \bar{u}^{\lambda_{\ell}}(p_\ell) \gamma_\alpha
                                               v^{\lambda_{\ell}}(p_{\bar{\nu}_\ell}) \\
& \cdot \left( \frac{ \bar{u}^{\lambda_a}(p_1) \gamma^\alpha (\slashed{p}_V +
  \slashed{p}_1)\gamma^\rho u^{\lambda_a}(p_a)}{(p_V+p_1)^2} +
\frac{ \bar{u}^{\lambda_a}(p_1)\gamma^\rho (\slashed{p}_a - \slashed{p}_V)\gamma^\alpha u^{\lambda_a}(p_a)}{(p_a-p_V)^2} \right).
\end{split}
\end{equation}
$p_V=p_{\ell}+p_{\bar{\nu}_\ell}$ is the vector boson momentum, $g_V$ its
coupling to the fermion $f_a$, $M_V$ its mass, and $\Gamma_V$ the
width.

\subsection{Higgs boson production with a single jet}
\label{sec:HEJ_Hj}

In the gluon-fusion production of a Higgs boson together with one or
more jets new LL configurations beyond those derived from pure
multijet production (c.f.~section~\ref{sec:HEJ_summary}) arise. In
these configurations, one of the incoming partons is a gluon while the
corresponding most forward or most backward outgoing particle is the
Higgs boson, i.e.~$g f_b \to H\cdots f_b$ or $f_a g \to f_a \cdots
H$. Without loss of generality we consider the former
configuration. The modulus square of the \HEJ matrix element reads~\cite{Andersen:2022zte}
\begin{align}
  \label{eq:ME_fact_outer}
  \begin{split}
    \overline{\left\lvert\mathcal{M}_{\HEJ}^{g f_b \to H \cdots f_b}\right\rvert}^2 ={}&\mathcal{B}_{H, f_b}(p_a, p_b, p_1, p_n)\\
    &\cdot \prod_{i=1}^{n-2} \mathcal{V}(p_a, p_b, p_a, p_n, q_i, q_{i+1})\\
    &\cdot \prod_{i=1}^{n-1} \mathcal{W}(q_{i\perp}, y_i, y_{i+1}),
  \end{split}
\end{align}
with the universal real and virtual correction factors $\mathcal{V}$
and $\mathcal{W}$ defined in equations~\eqref{eq:V} and
\eqref{eq:W}. The only differences to the pure QCD case in
equation~\eqref{eq:ME_fact} are the replacement $p_1 \to p_a$ in the
third argument of $\mathcal{V}$ and the adjustment of the
process-dependent Born function to~\cite{Andersen:2022zte}
\begin{align}
  \label{eq:B_Hq}
  \mathcal{B}_{H,f_b} ={}& \frac {(4\pi\alpha_s)^{n-1}} {4(N_c^2-1)}
                           \frac {1} {q_1^2}
                           \frac{K_{f_b}}{q_{n-1}^2}
                           \left\|S_{g f_b \to H f_b}\right\|^2,\\
  \label{eq:S_Hf}
  \left\|S_{g f_b \to  H f_b}\right\|^2 ={}&  \sum_{
  \substack{
  \lambda_{a} = +,-\\
  \lambda_{b} = +,-
  }}
 \left\lvert\epsilon_\mu^{\lambda_a}(p_a)\ V_H^{\mu\nu}(p_a, p_a-p_1)\ j_\nu^{\lambda_b}(p_n, p_b)\right\rvert^2,
\end{align}
where $\epsilon^{\lambda_a}(p_a)$ is the polarisation vector of the
incoming gluon and $V_H$ the effective vertex coupling the Higgs boson to two
gluons at one-loop, including finite quark-mass dependence.  The structure of
equation~(\ref{eq:ME_fact_outer}) then allows finite quark-mass dependence to be
applied for arbitrarily high numbers of jets.

\subsection{Spillover from small transverse momenta}
\label{sec:low_pt}

As described in section~\ref{sec:HEJ_summary}, a number of all-order
resummation events is generated for each resummable input event.
Since the resummation events include real corrections, the resulting
kinematics differ slightly from the kinematics of the input
events. While jet rapidities are always preserved, this is
generally neither true for transverse momenta nor for the rapidities
of any other particles. This implies that cuts imposed on the
leading-order generation should be significantly looser than the final
analysis cuts. Empirically, the difference in transverse momentum cuts
should be about 10-20\%, with a slight increase towards larger
multiplicities.

Simply adjusting the cuts in the leading-order generation is correct,
but inefficient: events with small transverse momenta dominate the
leading-order prediction, but only give a small contribution to the
final resummed results. It is therefore more efficient to split up the leading-order
generation. One first generates a high-statistics sample in which all
particles fulfil the transverse momentum cuts of the analysis. Then,
one generates a second low-statistics sample where in each event there
is at least one particle with small transverse momentum that does not
pass the final cuts. Since the two samples are disjoint, one can
separately apply \HEJ resummation to each sample and add up the results.

However, implementing the requirement of at least one ``soft'' particle is often
not straightforward with standard fixed-order generators. To
facilitate resummation for the low transverse momentum sample,
\HEJ~2.2 introduces a new option for discarding events in which all
jets are above the analysis threshold. An example is given in section~\ref{sec:low_pt_ex}.

\subsection{Matching to Next-to-Leading Order}
\label{sec:match_NLO}

To improve the accuracy of the obtained total cross section to
next-to-leading order (NLO), one can obviously multiply the \HEJ prediction by a
flat factor of $\sigma_{\text{NLO}} / \sigma_\HEJ$, where
$\sigma_\HEJ$ is the leading-order accurate total cross section
according to \HEJ and $\sigma_{\text{NLO}}$ the total cross section at
NLO. \HEJ 2.2 enables us to achieve NLO accuracy also in
differential distributions. Considering a distribution $d\sigma/d
\mathcal{O}$ in some observable $\mathcal{O}$, we can combine NLO and
\HEJ resummation through the reweighting
\begin{equation}
  \label{eq:match_NLO}
  \left(\frac{d\sigma}{d \mathcal{O}}\right)_{\HEJ + \text{NLO}} =
  \frac{(d\sigma/d\mathcal{O})_{\text{NLO}}}{(d\sigma/d\mathcal{O})_{\HEJ,\text{NLO}}} \left(\frac{d\sigma}{d \mathcal{O}}\right)_{\HEJ}.
\end{equation}
Here, the subscript \HEJ denotes the prediction before reweighting,
\text{NLO} the NLO-accurate prediction, and $\HEJ,$NLO the truncation
of the \HEJ prediction to NLO. In section~\ref{sec:match_NLO_ex}, we
show in an example how to truncate the \HEJ prediction and obtain
NLO-reweighted distributions.

\subsection{Predictions without Fixed-Order Matching}
\label{sec:HEJFOG}

The computational cost for generating the fixed-order input events
rises steeply with the jet multiplicity. For this reason, \HEJ
includes the \HEJFOG, a fast generator based on the leading-order
truncation of the \HEJ matrix element given in
equation~\eqref{eq:ME_fact}. The intended use is that one will generate exact
low-multiplicity input events with a conventional generator and
supplement them with approximate high-multiplicity events using the
\HEJFOG. In \HEJ~2.2, the \HEJFOG includes charged lepton pair
production with jets as a new process. Furthermore, the generation
efficiency for the production of a W boson with jets has been improved
by aligning the rapidity of the W boson with its emitter, reducing the
Monte Carlo uncertainty by a factor of up to 2.


%%% Local Variables:
%%% mode: latex
%%% TeX-master: "main"
%%% End:
