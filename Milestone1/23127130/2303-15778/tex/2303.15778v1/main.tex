\documentclass[preprint,a4paper,12pt]{elsarticle}
\pdfoutput=1
\unitlength=1mm

\usepackage{amsmath,amssymb,graphicx,todonotes,xspace,hyperref,slashed,subcaption,float,tabulary}
\usepackage{booktabs}
\usepackage{listings}
\usepackage[absolute]{textpos}
\usepackage[section]{placeins}
\setlength{\oddsidemargin}{0cm}
\setlength{\evensidemargin}{0cm}
\setlength{\textwidth}{16cm}
\setlength{\topmargin}{0cm}
\setlength{\textheight}{23cm}

\hypersetup{
  pdftitle={HEJ 2.2},
  colorlinks,
  urlcolor={blue},
  linkcolor={blue},
}

\graphicspath{{figures/}{build/figures/}}

\lstloadlanguages{[11]C++}

\lstset{ %
 backgroundcolor=\color{lightgray!50},   % choose the background color; you must add \usepackage{color} or \usepackage{xcolor}
 basicstyle=\usefont{T1}{DejaVuSansMono-TLF}{m}{n}\fontsize{9pt}{11pt}\selectfont,        % the size of the fonts that are used for the code
 breakatwhitespace=true,          % sets if automatic breaks should only happen at whitespace
 breaklines=false,                 % sets automatic line breaking
 prebreak=\mbox{\textcolor{red}{$\hookrightarrow$}\space},
 postbreak=\mbox{\textcolor{red}{$\hookrightarrow$}\space},
 captionpos=t,                    % sets the caption-position to bottom
 commentstyle=\color{red},        % comment style
 deletekeywords={...},            % if you want to delete keywords from the given language
 escapeinside={\%*}{*)},          % if you want to add LaTeX within your code
 extendedchars=true,              % lets you use non-ASCII characters; for 8-bits encodings only, does not work with UTF-8
 frame=single,                    % adds a frame around the code
 framerule=1pt,
 keepspaces=true,                 % keeps spaces in text, useful for keeping indentation of code (possibly needs columns=flexible)
 keywordstyle=\color{blue},       % keyword style
 otherkeywords={override},                % if you want to add more keywords to the set
 numbers=none,                    % where to put the line-numbers; possible values are (none, left, right)
 numbersep=5pt,                   % how far the line-numbers are from the code
 rulecolor=\color{black},         % if not set, the frame-color may be changed on line-breaks within not-black text (e.g. comments (green here))
 showspaces=false,                % show spaces everywhere adding particular underscores; it overrides 'showstringspaces'
 showstringspaces=false,          % underline spaces within strings only
 showtabs=false,                  % show tabs within strings adding particular underscores
 stepnumber=2,                    % the step between two line-numbers. If it's 1, each line will be numbered
 stringstyle=\color{gray},     % string literal style
 tabsize=2,                    % sets default tabsize to 2 spaces
 emph={},
 emphstyle=\color{darkgreen}
}

\lstdefinelanguage{sherpa}{%
  morekeywords={run,processes,selector,analysis},
  sensitive=true,
  morecomment=[l]{\#},
}

\lstdefinelanguage{yaml}{%
  morekeywords={true,false},
  sensitive=false,
  morecomment=[l]{\#},
}

\newcommand{\HEJ}{{\tt HEJ}\xspace}
\newcommand{\HEJFOG}{{\tt HEJFOG}\xspace}
\newcommand{\HEJone}{{\tt HEJ~1}\xspace}
\newcommand{\HEJtwo}{{\tt HEJ~2}\xspace}
\newcommand{\HIGHEJ}{\emph{High Energy Jets}\xspace}
\newcommand{\tinyspace}{\mkern 1mu}

\newcounter{bla}
\newenvironment{refnummer}{%
\list{[\arabic{bla}]}%
{\usecounter{bla}%
 \setlength{\itemindent}{0pt}%
 \setlength{\topsep}{0pt}%
 \setlength{\itemsep}{0pt}%
 \setlength{\labelsep}{2pt}%
 \setlength{\listparindent}{0pt}%
 \settowidth{\labelwidth}{[9]}%
 \setlength{\leftmargin}{\labelwidth}%
 \addtolength{\leftmargin}{\labelsep}%
 \setlength{\rightmargin}{0pt}}}
 {\endlist}

%\journal{Computer Physics Communications}

\begin{document}

\begin{frontmatter}
\title{\HEJ 2.2: W boson pairs and Higgs boson plus jet production at high energies}

\author[a]{Jeppe~R.~Andersen}
\author[b]{Bertrand~Duclou\'e}
\author[a]{Hitham~Hassan}
\author[b]{Conor~Elrick}
\author[c]{Andreas~Maier}
\author[b]{Graeme~Nail}
\author[b]{J\'er\'emy~Paltrinieri}
\author[d]{Andreas~Papaefstathiou}
\author[b]{Jennifer~M.~Smillie}

\address[a]{Institute for Particle Physics Phenomenology, University of Durham,
Durham, DH1 3LE, UK}
\address[b]{Higgs Centre for Theoretical Physics, University of Edinburgh,
  Peter Guthrie Tait Road, Edinburgh, EH9 3FD, UK}
\address[c]{Deutsches Elektronen-Synchrotron DESY, Platanenallee 6,
  15738 Zeuthen, Germany}
\address[d]{Department of Physics, Kennesaw State University, Kennesaw, GA 30144, USA}

\begin{abstract}
We present version 2.2 of the \HIGHEJ (\HEJ) Monte Carlo event
generator for hadronic scattering processes at high energies. The new
version adds support for two further processes of central
phenomenological interest, namely the production of a W boson pair
with equal charge together with two or more jets and the production of
a Higgs boson with at least one jet.
Furthermore, a new prediction for charged lepton pair production with
high jet multiplicities is provided in the high-energy limit. The
accuracy of \HEJ 2.2 can be increased further through an enhanced
interface to standard predictions based on conventional perturbation
theory. We describe all improvements and
provide extensive usage examples. \HEJ 2.2 can be obtained from
\url{https://hej.hepforge.org}.
\end{abstract}

\begin{keyword}
  Collider Physics; Monte Carlo Event Generation; Resummation
\end{keyword}

\begin{textblock*}{10cm}(100mm,22mm)
IPPP/23/18, DCPT/23/36, DESY-23-038
\end{textblock*}
\end{frontmatter}

{\bf NEW VERSION PROGRAM SUMMARY}

\begin{small}
\noindent
{\em Program Title:} \HEJ.                                          \\
% {\em CPC Library link to program files:} (to be added by Technical Editor) \\
% {\em Code Ocean capsule:} (to be added by Technical Editor)\\
{\em Licensing provisions:} GPLv2 or later.\\
{\em Programming language:} C++.                               \\
{\em Journal reference of previous version:} Comput.Phys.Commun. 278 (2022) 108404.\\
{\em Does the new version supersede the previous version?:} Yes.  \\
{\em Reasons for the new version:} Support for further scattering processes and improved combination with fixed-order predictions.\\
{\em Summary of revisions:} The new release adds the ability to predict the QCD
component in the high-energy production of two leptonically decaying W bosons with
equal charge, together with two or more jets. High-energy resummation
is now also implemented for the production of a Higgs boson together
with a single jet, whereas before only processes involving at least
two jets had been considered. Pure resummed predictions for lepton
pair production via a virtual photon or Z boson together with jets are
now provided for high jet multiplicities, where fixed-order matching is
no longer feasible. The interface to fixed-order generators has been
extended significantly, including options for differential reweighting
to next-to-leading order, filtering of jets with low transverse
momentum, and the capability to stream a wider range of input event
files.  \\
{\em Nature of problem:}
Hadronic scattering processes at high energies, i.e.~with large invariant
masses between jets, are of great phenomenological interest. This is
in particular the case for measurements of weak-boson scattering and
weak-boson fusion production of a Higgs boson. In the high-energy
region, standard perturbation theory exhibits poor convergence for the QCD
contributions, which limits the predictive power of conventional Monte
Carlo generators.\\
{\em Solution method:}
The poor convergence of the perturbative series can be traced to the
appearance of large high-energy logarithms. \HEJ is a fully flexible
event generator combining fixed-order accuracy with the all-order
resummation of such logarithms, based on the \HIGHEJ framework.
The new version \HEJ 2.2 provides accurate predictions
for a range of processes of central phenomenological interest,
including the QCD component of same-sign W boson pair production with multiple jets and Higgs
boson production in association with one or more jets.
% \\
% {\em Additional comments including restrictions and unusual features:}\\
\end{small}

\newpage
\tableofcontents

% Importance and appeal of children's drawings
Children's depictions of the human figure are highly expressive and varied.
As one of the very first subjects children attempt to draw, the representation begins as an almost unintelligible cloud of scribbles. 
As the child grows, their representation of the human figure becomes more developed and is extended to graphically represent many different types of characters: people, animals, and even personified objects (see Figure 1).

Who among us has not wished, either as a child or as an adult, to see such figures come to life and move around on the page?
Sadly, while it is relatively fast to produce a single drawing, creating the sequence of images necessary for animation is a much more tedious endeavor, requiring discipline, skill, patience, and sometimes complicated software.
As a result, most of these figures remain static upon the page.

% We built a system to animate them.
Inspired by the importance and appeal of the drawn human figure, we design and build a system to automatically animate it given an in-the-wild photograph of a child's drawing. 
Our system is fast, intuitive, and robust to much of the variation present in these types of drawings, making it well-suited to allow our target audience--children--to see their own characters coming to life.
The system is comprised of four stages: figure detection, segmentation masking, pose estimation/rigging, and animation. 
We describe each stage and identify common causes of failure in each. 
For object detection and pose estimation, we make use of existing computer vision models designed to detect human figures and joints in photographs; we fine-tune these models for use with children's drawings.
For segmentation, we present a straightforward, image processing-based method that, for animation purposes, is more useful and accurate than segmentation masks obtained from a fine-tuned object detection model.
During the animation step, we take advantage of the \textit{twisted perspective} commonly seen in children’s drawings to retarget motion capture data onto the character in a novel and appealing way.

% We use existing machine learning models. However, given the wide domain gap it's not clear how much fine-tuning data was needed. So we ran some experiments to find out and report it.
While our system leverages existing models and techniques, most are not directly applicable to the task due to the many differences between photographic images and simple pen and paper representations. 
To this end, we couple the presentation of our system with a set of experiments exploring the relationship between fine-tuning training set size and success rates.
We also include a perceptual study validating viewer preference for incorporating \textit{twisted perspective} into the motion retargeting step.

We validate the desirability and appeal of our system by building and publicly releasing a version of it as the \AD Demo \,\cite{animateddrawings}.
Launched in December 2021, this demo has been used by millions of people around the world to animate their children's drawings.
Inspired by this reception, our second contribution is The Amateur Drawings Dataset: \hjs{180,000 drawings and user-accepted annotations collected, with consent, through the demo. See Section \ref{sec:UI} for a description of how the annotations were generated.}
We believe this dataset will be a resource to researchers from various fields seeking to better understand the space of amateur drawings, evaluate new algorithms in this domain, or develop new drawing-based tools in general.

To summarize, our contributions are as follows:
\begin{enumerate}
    \item 
    We explore the problem of automatic sketch-to-animation for children's drawings of human figures and present a framework that achieves this effect. We also present a set of experiments determining the amount of training data necessary to achieve high levels of success and a perceptual study validating the usefulness of our motion retargeting technique.
    \item To encourage additional research in the domain of amateur drawings, we present a first-of-its-kind dataset of 180,000 user-submitted amateur drawings, along with user-accepted bounding box, segmentation mask, and joint location annotations.
\end{enumerate}

Upon acceptance of this paper, we plan to publicly release the Amateur Drawings Dataset, project code, and fine-tuned model weights.

\section{Features of \HEJ 2.2}
\label{sec:features}

\subsection{\HEJ in a nutshell}
\label{sec:HEJ_summary}

Before describing the changes in \HEJ 2.2, let us briefly review the
general formalism and program structure. As input, \HEJ requires
leading-order (LO) events, generated with
e.g.~\textsc{Sherpa}~\cite{Bothmann:2019yzt} or
\textsc{MadGraph5\_aMC@NLO}~\cite{Alwall:2014hca}. For higher jet
multiplicities exact fixed-order generation becomes increasingly time
consuming. To address this problem, \HEJ includes the fast \HEJ
fixed-order generator \HEJFOG based on the high-energy approximation
of the leading-order matrix elements.

Using the kinematics of each (approximate or exact) input event, we
identify whether resummation is possible. For each event that permits
resummation, \HEJ generates a number of matching events in the
resummation phase space, which include real and virtual corrections to
all orders in the high-energy limit. Details are given
in~\cite{Andersen:2018tnm}. Together with the unchanged non-resummable
input events, the generated resummation events are then passed on to
any number of output event files and/or analyses. This standard control flow is
depicted in figure~\ref{fig:flow}. It can be modified through \HEJ
options, such that e.g.~non-resummable events are discarded.
\begin{figure}[htb]
  \centering
  \includegraphics{flow}
  \caption{Standard \HEJ control flow.}
  \label{fig:flow}
\end{figure}

The first type of event kinematics for which resummation is
implemented are leading-logarithmic (LL) configurations, which for
pure multijet production have to fulfil the following constraints:
\begin{enumerate}
\item The flavour of the most backward outgoing parton has to match
the flavour of the backward incoming parton.
\item The flavour of the most forward outgoing parton has to match
the flavour of the forward incoming parton.
\item All other outgoing partons have to be gluons.
\end{enumerate}
These criteria remain the same in processes involving virtual photons
and/or Z bosons. For virtual W bosons, the incoming and outgoing
flavours in criteria 1 and 2 only have to match up to the change
induced by W boson couplings. In the case of a final-state Higgs
boson, configurations where the backward (forward) incoming parton is
a quark or antiquark and the most backward (forward) outgoing particle
is the Higgs boson are formally subleading. Nevertheless, we also
implement resummation for such configurations. Depending on the
process, resummation is also implemented for two further types of
next-to-leading-logarithmic (NLL) configurations. These configurations
differ from LL ones as follows.
\begin{itemize}
\item \emph{Unordered gluon:} Either the most forward or most backward
outgoing parton is a gluon, and the next outgoing parton in rapidity
order is a quark or antiquark whose flavour matches the one of the
respective incoming parton.
\item \emph{Quark-antiquark:} A pair of final-state gluons that are
adjacent in rapidity is replaced by a quark-antiquark pair.
\end{itemize}
The current status of the implemented resummation is summarised in
table~\ref{tab:summary}.

\begin{table}[tb]
  \centering
  \begin{tabular}{lcccc}
    \toprule
    Process & pure LL & LO + LL & \multicolumn{2}{c}{NLL} \\
    \cmidrule(lr){4-5}
    &&&unordered gluon & quark-antiquark\\
    \midrule
    $\geq$2 jets & \HEJ 2.0 & \HEJ 2.0 & \HEJ 2.0 & \HEJ 2.1\\
    H + 1 jet & --- & \HEJ 2.2 & N/A & N/A \\
    H + $\geq$2 jets & \HEJ 2.0 & \HEJ 2.0 & \HEJ 2.0 & ---\\
    W + $\geq$2 jets & \HEJ 2.1 & \HEJ 2.1 & \HEJ 2.1 & \HEJ 2.1\\
    Z/$\gamma$ + $\geq$2 jets & \HEJ 2.2 & \HEJ 2.1 & \HEJ 2.1 & ---\\
    W$^\pm$W$^\pm$ + $\geq$2 jets & --- & \HEJ 2.2 & --- & ---\\
    \bottomrule
  \end{tabular}
  \caption{%
    Implemented processes and higher-order logarithmic corrections in
    \HEJ. The ``pure LL'' column lists processes implemented in the
    \HEJFOG. The NLL columns include both pure NLL and NLL matched to LO.
  }
  \label{tab:summary}
\end{table}

The resummation events generated for the LL and supported NLL
configurations are given a final matrix element weight of
\begin{equation}
  \label{eq:weight}
  \lvert\mathcal{M}_{\HEJ}\rvert^2 \frac{\lvert\mathcal{M}_{\text{LO}}\rvert^2}{\lvert\mathcal{M}_{\HEJ, \text{LO}}\rvert^2},
\end{equation}
where
$\mathcal{M}_{\HEJ}$ is the all-order scattering matrix element in the
high-energy approximation, $\mathcal{M}_{\HEJ, \text{LO}}$ its
leading-order truncation, calculated for the kinematics of the input
event and $\lvert\mathcal{M}_{\text{LO}}\rvert^2$ is taken from the LO input.

To illustrate the structure of the \HEJ matrix element, we first focus on LL
configurations in pure multijet production. We denote these configurations as $f_a f_b \to f_a \cdots f_b$, where $f_a$ is the flavour of the incoming parton in the backward direction with momentum
$p_a$. Correspondingly, we use $f_b$ and $p_b$ for the flavour and
momentum of the forward incoming parton. The final state contains $n$
partons with momenta $p_1,\dots,p_n$, which we order by rapidity,
viz.~$y_1 < \dots < y_n$. The most backward outgoing parton has
flavour $f_{a}$, the most forward one flavour $f_{b}$, and all other
outgoing partons are gluons. Using this notation, we can write the general form of the squared \HEJ
matrix element as
\begin{equation}
  \label{eq:ME_fact}
  \begin{split}
    \overline{\left\lvert\mathcal{M}_{\HEJ}^{f_a f_b \to f_a \cdots f_b}\right\rvert}^2 ={}&\mathcal{B}_{f_a,f_b}(p_a, p_b, p_1, p_n)\\
    &\cdot \prod_{i=1}^{n-2} \mathcal{V}(p_a,p_b,p_1,p_n, q_i, q_{i+1})\\
    &\cdot \prod_{i=1}^{n-1} \mathcal{W}(q_{i\perp}, y_i, y_{i+1}),
  \end{split}
\end{equation}
where $q_i=p_a - \sum_{j=1}^i p_j$ is the $t$-channel momentum after
the emission of parton $i$. $\mathcal{B}_{f_a,f_b}$ is derived from the modulus
square of the Born-level matrix element for the process $f_a f_b \to
f_a f_b$, $\mathcal{V}$ accounts for the real emission of the $n-2$
gluons in between $f_a$ and $f_b$, and $\mathcal{W}$ incorporates the
virtual and unresolved real corrections.

The Born-level function $\mathcal{B}_{f_a,f_b}$ is given by
\begin{equation}
  \label{eq:B}
  {\cal B}_{f_a,f_b}(p_a, p_b, p_1, p_n) = \frac{(4\pi\alpha_s)^n}{4 (N_C^2 - 1)} \frac{K_{f_a}}{q_1^2} \frac{K_{f_b}}{q_{n-1}^2} \|S_{f_a f_b \to f_{a}\cdots f_{b}}\|^2 ,
\end{equation}
where $\alpha_s$ is the strong coupling constant and $N_C = 3$ the number of colours. $K_{f_a}$ and $K_{f_b}$ are
generalised colour factors depending on the respective parton flavour
and, in the case of gluons, also the parton momentum. For quarks and
antiquarks one finds $K_f = C_F = \frac{N_C^2-1}{2N_C}$; the factor $K_g$ for gluons is
derived in~\cite{Andersen:2009he}. $S_{f_a f_b \to f_{a}\cdots f_{b}}$
denotes the contraction of two currents:
\begin{equation}
  \label{eq:S_QCD}
  \| S_{f_a f_b \to f_a \cdots f_b}\|^2 \equiv \| j^a \cdot j^b\|^2 = \sum_{\substack{\lambda_a=+,-\\ \lambda_b=+,-}}\lvert j^{\mu,\lambda_a}(p_1, p_a) j_\mu^{\lambda_b}(p_n, p_b)\rvert^2,
\end{equation}
where $j_\mu^\lambda$ is the current
\begin{equation}
  \label{eq:j}
  j_\mu^\lambda(p,q) = \bar{u}^\lambda(p) \gamma_\mu u^\lambda(q)
\end{equation}
for helicity $\lambda$. \HEJ employs the symbolic manipulation
language \texttt{FORM}~\cite{Vermaseren:2000nd} to generate compact
symbolic expressions for current contractions.

The real corrections are given by contractions of Lipatov vertices~\cite{Andersen:2017kfc}:
\begin{align}
  \label{eq:V}
  \mathcal{V}(p_a,p_b,p_1,p_n, q_i, q_{i+1}) ={}& -\frac{C_A}{q_i^2 q_{i+1}^2} V_\mu(p_a,p_b,p_1,p_n, q_i, q_{i+1}) V^\mu(p_a,p_b,p_1,p_n, q_i, q_{i+1}),\\
  \label{eq:V_Lipatov}
  V^\mu(p_a,p_b,p_1,p_n, q_i, q_{i+1})={}& -(q_i+q_{i+1})^\mu \nonumber\\
  &+ \frac{p_a^\mu}{2} \left( \frac{q_i^2}{p_{i+1}\cdot p_a} +
  \frac{p_{i+1}\cdot p_b}{p_a\cdot p_b} + \frac{p_{i+1}\cdot p_n}{p_a\cdot p_n}\right) +
p_a \leftrightarrow p_1 \nonumber\\
  &- \frac{p_b^\mu}{2} \left( \frac{q_{i+1}^2}{p_{i+1} \cdot p_b} + \frac{p_{i+1}\cdot
      p_a}{p_b\cdot p_a} + \frac{p_{i+1}\cdot p_1}{p_b\cdot p_1} \right) - p_b
  \leftrightarrow p_n,
\end{align}
with $C_A = N_C$.

Finally, the virtual and unresolved real corrections $\mathcal{W}$ can
be expressed in terms of the regularised Regge trajectory $\omega^0$:
\begin{equation}
  \label{eq:W}
  \mathcal{W}(q_{j\perp},y_j,y_{j+1}) = \exp[\omega^0(q_{j\perp}) (y_{j+1}- y_j)].
\end{equation}
For a detailed discussion and an explicit expression for $\omega^0$
see~\cite{Andersen:2018tnm}.

The generalisations to NLL configurations and additional non-partonic
final state particles are derived
in~\cite{Andersen:2012gk,Andersen:2016vkp,Andersen:2017kfc,Andersen:2018kjg,Andersen:2020yax,Andersen:2021vnf,Andersen:2022zte}. In
all cases one finds a factorisation into a Born-level function,
resolved real emissions, and virtual and unresolved real
corrections. In the absence of interference, one recovers the same
structure as in equation~\eqref{eq:ME_fact}. In particular,
the functions $\mathcal{V}$ and $\mathcal{W}$ comprising the all-order
corrections are universal, whereas the Born-level function
$\mathcal{B}$ is process dependent.

\subsection{High-energy resummation for W pair production}
\label{sec:HEJ_WW}

Based on the pure-QCD LL configurations $f_a f_b \to f_a \cdots f_b$,
additional W bosons can be produced via emission off the partons $f_a$
and $f_b$. In \HEJ 2.2, we consider LL configurations with two
leptonically decaying W bosons with equal charge. For definiteness, we
discuss configurations $f_a f_b \to (W^- \to e \bar{\nu}_e) (W^-
\to \mu \bar{\nu}_\mu) f_{a'} \cdots f_{b'}$, where the rapidities of
the final-state charged and neutral leptons do not necessarily respect
any rapidity ordering. Note that the couplings to the W bosons induce
flavour changes $f_a \to f_{a'}$ and $f_b \to f_{b'}$. The production
of two positively charged W bosons and the decay of the two W bosons
into the same lepton flavours is completely analogous.

We identify two contributions to the amplitude. Parton $f_a$ can
either couple to the W boson that decays into an electron and its antineutrino or
to the W boson decaying into a muon and its antineutrino. In the first case, the $t$-channel momenta are given by
\begin{equation}
  \label{eq:t-channel_e}
  q_{i,e} = p_a - p_e - p_{\bar{\nu}_e} - \sum_{j=1}^i p_j,
\end{equation}
where $p_e$ is the momentum of the electron and $p_{\bar{\nu}_e}$ the
momentum of its antineutrino. In the second case the $t$-channel momenta
are
\begin{equation}
  \label{eq:t-channel_mu}
  q_{i,\mu} = p_a - p_\mu - p_{\bar{\nu}_\mu} - \sum_{j=1}^i p_j
\end{equation}
with the muon momentum $p_\mu$ and the corresponding antineutrino momentum
$p_{\bar{\nu}_\mu}$. The resulting modulus square of the matrix element including interference is~\cite{Andersen:2021vnf}
\begin{equation}
\label{eq:ME_Z}
\begin{split}
  \left\lvert\mathcal{M}_{\HEJ}^{f_a f_b\to e\bar{\nu}_e \mu \bar{\nu}_\mu f_{a'} \cdots f_{b'}}\right\rvert^2 &=\ \frac{(4 \pi \alpha_s)^n}{4(N_c^2-1)}\ K_{f_a}K_{f_b} C_A^{n-2}\\
  \times \Bigg( &{\frac{\| j^a_{W,e}\cdot
    j^b_{W,\mu}\|^2}{q^2_{1,e}q^2_{n-1,e}}}
\prod^{n-2}_{i=1} {\frac{-V^2(q_{i,e},
  q_{i+1,e})}{q^2_{i,e} q^2_{i+1,e}}} \prod_{i=1}^{n-1}
  \mathcal{W}(q_{i,e\perp},y_i,y_{i+1})\\
+\ &{\frac{\|j^a_{W,\mu} \cdot j^b_{W,e} \|^2}{q^2_{1,\mu}q^2_{n-1,\mu}}}
     \prod^{n-2}_{i=1}{\frac{-V^2(q_{i,\mu}, q_{i+1,\mu})}{q^2_{i,\mu} q^2_{i+1,\mu}}} \prod_{i=1}^{n-1}
\mathcal{W}(q_{i,\mu\perp},y_i,y_{i+1}) \\
-\ &{\frac{2\Re\{ (j^a_{W,e}\cdot j^b_{W,\mu})(\overline{j^a_{W,\mu} \cdot
      j^b_{W,e}})\}}{\sqrt{q^2_{1,e}q^2_{1,\mu}}\sqrt{q^2_{n-1,e} q^2_{n-1,\mu}}}}\\
&\times \prod^{n-2}_{i=1}{\frac{V(q_{i,e}, q_{i+1,e})\cdot V(q_{i,\mu},
         q_{i+1,\mu})}{\sqrt{q^2_{i,e}q^2_{i,\mu}} \sqrt{q^2_{i+1,e}q^2_{i+1,\mu}}}} \prod_{i=1}^{n-1}
         \mathcal{W}(\sqrt{q_{i,e\perp}q_{i,\mu\perp}},y_i,y_{i+1})\Bigg).
\end{split}
\end{equation}
Here, we have introduced contractions between generalised currents
$j^c_{W,l}$ accounting for the coupling between a parton with flavour
$f_c$ and a W boson decaying into a charged lepton $l$ and the
corresponding antineutrino. The contractions are
\begin{align}
  \label{eq:j_contr_sq_1}
  \| j^a_{W,e} \cdot j^b_{W,\mu}\|^2 ={}& \left\lvert j_V^{\rho,\lambda_a \lambda_e}(p_1, p_a, p_e, p_{\bar{\nu}_e}) j_{V}^{\sigma,\lambda_b\lambda_\mu}(p_{n+2}, p_b, p_\mu, p_{\bar{\nu}_\mu}) g_{\rho\sigma}\right\rvert^2,\\
  \label{eq:j_contr_sq_2}
  \| j^a_{W,\mu} \cdot j^b_{W,e}\|^2 ={}& \left\lvert j_V^{\rho,\lambda_a \lambda_\mu}(p_1, p_a, p_\mu, p_{\bar{\nu}_\mu}) j_{V}^{\sigma,\lambda_b\lambda_e}(p_{n+2}, p_b, p_e, p_{\bar{\nu}_e})g_{\rho\sigma}\right\rvert^2,\\
  \label{eq:j_contr_mixed}
  \begin{split}
  (j^a_{W,e}\cdot j^b_{W,\mu})(\overline{j^a_{W,\mu} \cdot
      j^b_{W,e}})={}&
    j_V^{\rho,\lambda_a \lambda_e}(p_1, p_a, p_e, p_{\bar{\nu}_e}) j_{V}^{\sigma,\lambda_b\lambda_\mu}(p_{n+2}, p_b, p_\mu, p_{\bar{\nu}_\mu}) g_{\rho\sigma} \\
    &\qquad\times \overline{j_{V}^{\alpha\lambda_a \lambda_\mu}(p_1, p_a, p_\mu,
      p_{\bar{\nu}_\mu}) j_V^{\beta,\lambda_b \lambda_e}(p_{n+2}, p_b, p_e,
      p_{\bar{\nu}_e})}g_{\alpha\beta} ,
  \end{split}
\end{align}
where the parton helicities $\lambda_a$ and $\lambda_b$ are determined
by the flavour of the respective parton, namely $\lambda_c=-$ if $f_c$
is a quark and $\lambda_c=+$ if $f_c$ is an antiquark. The electron
helicity $\lambda_e$ and the muon helicity $\lambda_\mu$ correspond to
the charge sign of the parent W boson, i.e.\ $\lambda_e = \lambda_\mu
= -$ in the present case. We have introduced a generalised current
$j_V^{\rho,\lambda_a \lambda_\ell}$ for the coupling of a parton with
helicity $\lambda$ to a leptonically decaying vector
boson\footnote{Here, we assume a W boson. However, the same expression
holds for neutral vector bosons after replacing the antineutrino
momentum $p_{\bar{\nu}_{\ell}}$ by the antilepton momentum
$p_{\bar{\ell}}$.} $V$ with lepton helicity $\lambda_{\ell}$. It is
given by~\cite{Andersen:2020yax}
\begin{equation}
  \label{eq:j_V}
  \begin{split}
  j_V^{\rho,\lambda_a \lambda_\ell}(p_1,p_a,p_{\ell},p_{\bar{\nu}_{\ell}}) =&\ \frac{g_V^2}{2}\
     \frac1{p_V^2-M_V^2+i\ \Gamma_V M_V}\ \bar{u}^{\lambda_{\ell}}(p_\ell) \gamma_\alpha
                                               v^{\lambda_{\ell}}(p_{\bar{\nu}_\ell}) \\
& \cdot \left( \frac{ \bar{u}^{\lambda_a}(p_1) \gamma^\alpha (\slashed{p}_V +
  \slashed{p}_1)\gamma^\rho u^{\lambda_a}(p_a)}{(p_V+p_1)^2} +
\frac{ \bar{u}^{\lambda_a}(p_1)\gamma^\rho (\slashed{p}_a - \slashed{p}_V)\gamma^\alpha u^{\lambda_a}(p_a)}{(p_a-p_V)^2} \right).
\end{split}
\end{equation}
$p_V=p_{\ell}+p_{\bar{\nu}_\ell}$ is the vector boson momentum, $g_V$ its
coupling to the fermion $f_a$, $M_V$ its mass, and $\Gamma_V$ the
width.

\subsection{Higgs boson production with a single jet}
\label{sec:HEJ_Hj}

In the gluon-fusion production of a Higgs boson together with one or
more jets new LL configurations beyond those derived from pure
multijet production (c.f.~section~\ref{sec:HEJ_summary}) arise. In
these configurations, one of the incoming partons is a gluon while the
corresponding most forward or most backward outgoing particle is the
Higgs boson, i.e.~$g f_b \to H\cdots f_b$ or $f_a g \to f_a \cdots
H$. Without loss of generality we consider the former
configuration. The modulus square of the \HEJ matrix element reads~\cite{Andersen:2022zte}
\begin{align}
  \label{eq:ME_fact_outer}
  \begin{split}
    \overline{\left\lvert\mathcal{M}_{\HEJ}^{g f_b \to H \cdots f_b}\right\rvert}^2 ={}&\mathcal{B}_{H, f_b}(p_a, p_b, p_1, p_n)\\
    &\cdot \prod_{i=1}^{n-2} \mathcal{V}(p_a, p_b, p_a, p_n, q_i, q_{i+1})\\
    &\cdot \prod_{i=1}^{n-1} \mathcal{W}(q_{i\perp}, y_i, y_{i+1}),
  \end{split}
\end{align}
with the universal real and virtual correction factors $\mathcal{V}$
and $\mathcal{W}$ defined in equations~\eqref{eq:V} and
\eqref{eq:W}. The only differences to the pure QCD case in
equation~\eqref{eq:ME_fact} are the replacement $p_1 \to p_a$ in the
third argument of $\mathcal{V}$ and the adjustment of the
process-dependent Born function to~\cite{Andersen:2022zte}
\begin{align}
  \label{eq:B_Hq}
  \mathcal{B}_{H,f_b} ={}& \frac {(4\pi\alpha_s)^{n-1}} {4(N_c^2-1)}
                           \frac {1} {q_1^2}
                           \frac{K_{f_b}}{q_{n-1}^2}
                           \left\|S_{g f_b \to H f_b}\right\|^2,\\
  \label{eq:S_Hf}
  \left\|S_{g f_b \to  H f_b}\right\|^2 ={}&  \sum_{
  \substack{
  \lambda_{a} = +,-\\
  \lambda_{b} = +,-
  }}
 \left\lvert\epsilon_\mu^{\lambda_a}(p_a)\ V_H^{\mu\nu}(p_a, p_a-p_1)\ j_\nu^{\lambda_b}(p_n, p_b)\right\rvert^2,
\end{align}
where $\epsilon^{\lambda_a}(p_a)$ is the polarisation vector of the
incoming gluon and $V_H$ the effective vertex coupling the Higgs boson to two
gluons at one-loop, including finite quark-mass dependence.  The structure of
equation~(\ref{eq:ME_fact_outer}) then allows finite quark-mass dependence to be
applied for arbitrarily high numbers of jets.

\subsection{Spillover from small transverse momenta}
\label{sec:low_pt}

As described in section~\ref{sec:HEJ_summary}, a number of all-order
resummation events is generated for each resummable input event.
Since the resummation events include real corrections, the resulting
kinematics differ slightly from the kinematics of the input
events. While jet rapidities are always preserved, this is
generally neither true for transverse momenta nor for the rapidities
of any other particles. This implies that cuts imposed on the
leading-order generation should be significantly looser than the final
analysis cuts. Empirically, the difference in transverse momentum cuts
should be about 10-20\%, with a slight increase towards larger
multiplicities.

Simply adjusting the cuts in the leading-order generation is correct,
but inefficient: events with small transverse momenta dominate the
leading-order prediction, but only give a small contribution to the
final resummed results. It is therefore more efficient to split up the leading-order
generation. One first generates a high-statistics sample in which all
particles fulfil the transverse momentum cuts of the analysis. Then,
one generates a second low-statistics sample where in each event there
is at least one particle with small transverse momentum that does not
pass the final cuts. Since the two samples are disjoint, one can
separately apply \HEJ resummation to each sample and add up the results.

However, implementing the requirement of at least one ``soft'' particle is often
not straightforward with standard fixed-order generators. To
facilitate resummation for the low transverse momentum sample,
\HEJ~2.2 introduces a new option for discarding events in which all
jets are above the analysis threshold. An example is given in section~\ref{sec:low_pt_ex}.

\subsection{Matching to Next-to-Leading Order}
\label{sec:match_NLO}

To improve the accuracy of the obtained total cross section to
next-to-leading order (NLO), one can obviously multiply the \HEJ prediction by a
flat factor of $\sigma_{\text{NLO}} / \sigma_\HEJ$, where
$\sigma_\HEJ$ is the leading-order accurate total cross section
according to \HEJ and $\sigma_{\text{NLO}}$ the total cross section at
NLO. \HEJ 2.2 enables us to achieve NLO accuracy also in
differential distributions. Considering a distribution $d\sigma/d
\mathcal{O}$ in some observable $\mathcal{O}$, we can combine NLO and
\HEJ resummation through the reweighting
\begin{equation}
  \label{eq:match_NLO}
  \left(\frac{d\sigma}{d \mathcal{O}}\right)_{\HEJ + \text{NLO}} =
  \frac{(d\sigma/d\mathcal{O})_{\text{NLO}}}{(d\sigma/d\mathcal{O})_{\HEJ,\text{NLO}}} \left(\frac{d\sigma}{d \mathcal{O}}\right)_{\HEJ}.
\end{equation}
Here, the subscript \HEJ denotes the prediction before reweighting,
\text{NLO} the NLO-accurate prediction, and $\HEJ,$NLO the truncation
of the \HEJ prediction to NLO. In section~\ref{sec:match_NLO_ex}, we
show in an example how to truncate the \HEJ prediction and obtain
NLO-reweighted distributions.

\subsection{Predictions without Fixed-Order Matching}
\label{sec:HEJFOG}

The computational cost for generating the fixed-order input events
rises steeply with the jet multiplicity. For this reason, \HEJ
includes the \HEJFOG, a fast generator based on the leading-order
truncation of the \HEJ matrix element given in
equation~\eqref{eq:ME_fact}. The intended use is that one will generate exact
low-multiplicity input events with a conventional generator and
supplement them with approximate high-multiplicity events using the
\HEJFOG. In \HEJ~2.2, the \HEJFOG includes charged lepton pair
production with jets as a new process. Furthermore, the generation
efficiency for the production of a W boson with jets has been improved
by aligning the rapidity of the W boson with its emitter, reducing the
Monte Carlo uncertainty by a factor of up to 2.


%%% Local Variables:
%%% mode: latex
%%% TeX-master: "main"
%%% End:

\section{A Minimax Theorem for Robust Reinforcement Learning}\label{sec:application}



In this section, we discuss the reward poisoning attack considered in \citet{banihashem2021defense}, which can be formulated as a convex-nonconcave minimax optimization program. We show that the minimax equality holds in this optimization problem in the tabular policy setting and under policies represented by a class of neural networks, as a consequence of our results in Sections~\ref{sec:tabular} and \ref{sec:NN}. To our best knowledge, the existence of the Nash equilibrium for this robust RL problem has not been established before even in the tabular case.


We again consider the infinite horizon, average reward MDP $\Mcal=(\Scal,\Acal,\Pcal,r)$ introduced in Section~\ref{sec:tabular}, where $r$ is the true, unpoisoned reward function. 
% We use $\pi^{\star}$ to denote an optimal policy, which is a (not necessarily unique) solution to \eqref{eq:obj}. 
% To reflect the dependency of the value function on the reward, we denote
% \begin{align*}
%     J_r(\pi)&=\lim_{K \rightarrow \infty} \frac{\sum_{k=0}^{K} r(s_k, a_k)}{K}=\mathbb{E}_{s\sim\mu_{\pi}, a\sim \pi}[r(s_k,a_k)]=r^{\top}\widehat{\mu}_{\pi}.
%     % \pi_r^{\star}&=\argmax_{\pi\in\Delta(\Acal)^{\Scal}}J_r(\pi).
% \end{align*}
Let $\Pi^{\text{det}}$ denote the set of deterministic policies from $\Scal$ to $\Acal$. With the perfect knowledge of this MDP, an attacker has a target policy $\pi_{\dagger}\in\Pi^{\text{det}}$ and tries to make the learning agent adopt the policy by manipulating the reward function. 
Mathematically, the goal of the attacker can be described by the function $\operatorname{Attack}(r,\pi_{\dagger},\epsilon_{\dagger})$ which returns a poisoned reward under the true reward $r$, the target policy $\pi_{\dagger}$, and a pre-selected margin parameter $\epsilon_{\dagger}\geq0$. $\operatorname{Attack}(r,\pi_{\dagger},\epsilon_{\dagger})$ is the solution to the following optimization problem
\begin{align}
    \begin{aligned}
    \operatorname{Attack}(r,\pi_{\dagger},\epsilon_{\dagger})\quad=\quad\argmin_{r'}\quad&\sum_{s\in\Scal,a\in\Acal}\left(r'(s,a)-r(s,a)\right)^2\\
    \operatorname{s.t.}\quad& J_{r'}(\pi_{\dagger})\geq J_{r'}(\pi)+\epsilon_{\dagger},\quad\forall \pi\in\Pi^{\text{det}}\backslash\pi_{\dagger}.
    \end{aligned}\label{eq:attack_obj}
\end{align}
In other words, the attacker needs to minimally modify the reward function to make $\pi_{\dagger}$ the optimal policy under the poisoned reward. This optimization program minimizes a quadratic loss under a finite number of linear constraints and is obviously convex.

The learning agent observes the poisoned reward $r_{\dagger}=\operatorname{Attack}(r,\pi_{\dagger},\epsilon_{\dagger})$ rather than the original reward $r$. As noted in \citet{banihashem2021defense}, without any defense, the learning agent solves the policy optimization problem under $r_{\dagger}$ to find $\pi_{\dagger}$, which may perform arbitrarily badly under the original reward. One way to defend against the attack is to maximize the performance of the agent in the worst possible case of the original reward, which leads to solving a minimax optimization program of the form
\begin{align}
    \max_{\pi\in\Delta_{\Acal}^{\Scal}}\min_{r'} J_{r'}(\pi)\quad\operatorname{s.t.}\,\, \operatorname{Attack}(r',\pi_{\dagger},\epsilon_{\dagger})=r_{\dagger}.\label{eq:robustrl_obj}
\end{align}
% When we fix the reward $r'$, \eqref{eq:robustrl_obj} reduces to a standard policy optimization problem, which we have shown is non-convex but has connected superlevel sets. On the other hand, when the policy $\pi$ is fixed, \eqref{eq:robustrl_obj} reduces to 
% \begin{align}
%     \min_{r'} J_{r'}(\pi)\quad\operatorname{s.t.}\,\, \operatorname{Attack}(r',\pi_{\dagger},\epsilon_{\dagger})=r_{\dagger},\label{eq:robustrl_convexside}
% \end{align}
% which has a linear objective function and a convex (and compact) constraint set and is therefore a convex program\footnote{We justify this claim in Section~\ref{sec:robustrl_convexside_proof} of the appendix.}.
When the policy $\pi$ is fixed, \eqref{eq:robustrl_obj} reduces to 
\begin{align}
    \min_{r'} J_{r'}(\pi)\quad\operatorname{s.t.}\,\, \operatorname{Attack}(r',\pi_{\dagger},\epsilon_{\dagger})=r_{\dagger}.\label{eq:robustrl_convexside}
\end{align}
With the justification deferred to Appendix~\ref{sec:robustrl_convexside_proof}, we point out that \eqref{eq:robustrl_convexside} consists of a linear objective function and a convex (and compact) constraint set, and is therefore a convex program. On the other hand, when we fix the reward $r'$, \eqref{eq:robustrl_obj} reduces to a standard policy optimization problem.

We are interested in investigating whether the following minimax equality holds.
\begin{align}
    \max_{\pi\in\Delta_{\Acal}^{\Scal}}\min_{r':\operatorname{Attack}(r',\pi_{\dagger},\epsilon_{\dagger})=r_{\dagger}} J_{r'}(\pi) = \min_{r':\operatorname{Attack}(r',\pi_{\dagger},\epsilon_{\dagger})=r_{\dagger}} \max_{\pi\in\Delta_{\Acal}^{\Scal}}J_{r'}(\pi).
    \label{eq:robustrl_minimax}
\end{align}
This is a fundamental question to ask in minimax optimization, as it is is an important characterization of the optimization landscape. The minimax equality implies the existence of a Nash equilibrium. At the Nash equilibrium solution pair, neither player can achieve a better function value by changing its strategy, which provides a strong notion of equilibrium and global optimality.
The knowledge of the existence of the Nash equilibrium may be useful for designing and analyzing algorithms for solving the problem \citep{kim2008minimax,ricceri2008recent}. 

% It is known that the minimax inequality always holds
% \begin{align*}
%     \max_{\pi}\min_{r':\operatorname{Attack}(r',\pi_{\dagger},\epsilon_{\dagger})=r_{\dagger}} J_{r'}(\pi) \leq \min_{r':\operatorname{Attack}(r',\pi_{\dagger},\epsilon_{\dagger})=r_{\dagger}} \max_{\pi}J_{r'}(\pi).
% \end{align*}
% However, as the program is not convex-concave, it is unclear whether this condition holds as an equality.

% In the rest of the section, we show that the connectedness of the superlevel sets in reinforcement learning implies the minimax equality
% \begin{align}
%     \max_{\pi}\min_{r':\operatorname{Attack}(r',\pi_{\dagger},\epsilon_{\dagger})=r_{\dagger}} J_{r'}(\pi) = \min_{r':\operatorname{Attack}(r',\pi_{\dagger},\epsilon_{\dagger})=r_{\dagger}} \max_{\pi}J_{r'}(\pi).
%     \label{eq:robustrl_minimax}
% \end{align}
% Equivalently, this equality means that there exists a Nash equilibrium solution pair $(\pi^{\star},r^{\star})$.


It is well-known that the minimax equality 
\begin{align}
    \min_{y \in \Ycal} \max_{x \in \Xcal} f(x, y)=\max_{x \in \Xcal} \min_{y \in \Ycal} f(x, y)
\end{align}
holds for function $f:\Xcal\times \Ycal\rightarrow\mathbb{R}$ if 1) $\Xcal,\Ycal$ are finite-dimensional simplexes \citep{neumann1928theorie}, or 2) $f$ is continuous quasiconvex-quasiconcave and $\Xcal,\Ycal$ are convex compact sets \citep{sion1958general,kindler2005simple}. However, the function $J_r(\pi)$ does not fall under either category. This makes the validity of equation \eqref{eq:robustrl_minimax} unclear from the existing literature.




% In the rest of the section, we prove that the connectedness of the superlevel sets in reinforcement learning implies a minimax theorem for \eqref{eq:robustrl_obj}
% \begin{align}
%     \sup_{\theta\in\Omega}\min_{r\in \Rcal}\mathbb{E}[J_r(\pi_{\theta})] = \min_{r\in \Rcal}\sup_{\theta\in\Omega}\mathbb{E}[J_r(\pi_{\theta})].
%     \label{eq:robustrl_minimax}
% \end{align}
% The reason of using supremum rather than the maximum is that the optimal policy may be deterministic. Under a softmax policy parameterization, a deterministic policy can only be achieved in the limit by sending certain parameters to infinity.

In the rest of this section, we establish the equality \eqref{eq:robustrl_minimax} and show that it is a simple consequence of the connectedness of the superlevel sets in reinforcement learning and a minimax theorem adapted from \citet{simons1995minimax} on a special class of convex-nonconcave functions. We now state this minimax theorem and note that this is essentially a simplified version of \citet{simons1995minimax}[Theorem 4].
% specialized to the Euclidean space.


\begin{thm}\label{thm:minimax_simplified}
Consider a separately continuous function $f:\Xcal\times \Ycal\rightarrow\mathbb{R}$, with $\Ycal$ being a convex, compact set. 
Suppose that $f(x,\cdot)$ is convex for all $x\in\Xcal$. Also suppose that the collection of functions $\{f(\cdot,y)\}_{y\in\Ycal}$ is equiconnected. Then, we have
\begin{align}
    \sup_{x \in \Xcal} \min_{y\in \Ycal} f(x,y)=\min_{y\in \Ycal} \sup_{x \in \Xcal} f(x,y).
\end{align}
\end{thm}
Theorem~\ref{thm:minimax_simplified} states that the minimax equality holds under two main conditions (other than the continuity condition, which can easily be verified to hold for $J_{r}(\pi)$). First, the function $f(x,y)$ needs to be convex with respect to the variable $y$ within a convex, compact constraint set. Second, $f(x,y)$ needs to have a connected superlevel set with respect to $x$, and the path function constructed to prove the connectedness of the superlevel set is independent of $y$. As we have shown in this section and earlier in the paper, if we model $J_r(\pi)$ by $f(x,y)$ with $\pi$ and $r$ corresponding to $x$ and $y$, both conditions are observed in the optimization problem \eqref{eq:robustrl_obj}, which allows us to state the following corollary.
\begin{cor}\label{cor:minimax_robustrl_tabular}
Suppose that the Markov chain $\Mcal$ satisfies Assumption~\ref{assump:ergodicity} on ergodicity. Then, the minimax equality \eqref{eq:robustrl_minimax} holds.
\end{cor}

When the neural network presented in Section~\ref{sec:NN} is used to represent the policy, the collection of functions $\{J_{r,\Omega}\}_{r}$ is also equiconnected. This allows us to extend the minimax equality above to the neural network policy class. Specifically, consider the poisoned reward defense problem \eqref{eq:robustrl_obj} where the policy $\pi_{\theta}$ is represented by the parameter $\theta\in\Omega$ as described in Section~\ref{sec:NN}. 
Using $f(x,y)$ to model $J_r(\pi_{\theta})$ with $x$ and $y$ mirroring $\theta$ and $r$, we can easily establish the minimax theorem in this case as a consequence of Theorem~\ref{thm:connected_firstlayer2S} and \ref{thm:minimax_simplified}.

\begin{cor}\label{cor:minimax_robustrl_NN}
Suppose that the Markov chain $\Mcal$ satisfies Assumption \ref{assump:ergodicity} on ergodicity and that the neural policy class satisfies Assumptions~\ref{assump:sigma}-\ref{assump:network_dimension}. Then, we have
\begin{align}
    \sup_{\theta\in\Omega}\min_{r':\operatorname{Attack}(r',\pi_{\dagger},\epsilon_{\dagger})=r_{\dagger}} J_{r'}(\pi_{\theta}) = \min_{r':\operatorname{Attack}(r',\pi_{\dagger},\epsilon_{\dagger})=r_{\dagger}} \sup_{\theta\in\Omega}J_{r'}(\pi_{\theta}).
\end{align}
\end{cor}

Corollary~\ref{cor:minimax_robustrl_tabular} and \ref{cor:minimax_robustrl_NN} establish the minimax equality (or equivalently, the existence of the Nash equilibrium) for the robust reinforcement learning problem under adversarial reward attack for the tabular and neural network policy class, respectively. To our best knowledge, these results are both novel and previously unknown in the existing literature. The Nash equilibrium is an important global optimality notion in minimax optimization, and the knowledge on its existence can provide strong guidance on designing and analyzing algorithms for solving the problem.




\section{Conclusions}
\label{sec:conclusions}

We have demonstrated that the fraction of negative event weights in
existing large high-multiplicity samples can be reduced by more than
an order of magnitude, whilst preserving predictions for observables
within statistical uncertainties. Concretely, we have employed the cell
resampling method proposed in~\cite{Andersen:2021mvw} with NLO event
samples for Z boson production with up to three jets
and W boson production with five jets produced with \textsc{Sherpa}
and \textsc{BlackHat}.

For the first time, cell resampling has been applied to samples with
up to several billions of events. This was made possible by
algorithmic improvements leading to a speed-up by several orders of
magnitude. Our updated implementation can be retreived from
\url{https://cres.hepforge.org/}.

The advances in the development of the cell resampling method
presented in this work pave the way for future applications to processes with
high-multiplicities, in particular including parton showered
predictions. It will be necessary to quantify the uncertainty
introduced by the weight smearing. Variations in the maximum cell size
parameter and different prescriptions for weight redistribution within
a cell can serve as handles to assess this uncertainty. Another
promising avenue for further exploration is the analysis of the
information on weight distribution within phase space collected during
cell resampling. Regions with insufficient Monte Carlo statistics
could be identified by their accumulated negative weight, thereby
guiding the event generation. We leave the investigation of these
questions to future work.

\section*{Acknowledgements}

AM thanks Zahari Kassabov for encouragement to reconsider the use of nearest
neighbour search trees. The work of JRA and DM is supported by the STFC under
grant ST/P001246/1.

%%% Local Variables:
%%% mode: latex
%%% TeX-master: "main"
%%% End:


% \newpage
% \appendix

\bibliographystyle{elsarticle-num}
\bibliography{papers}

\end{document}

%%% Local Variables:
%%% mode: latex
%%% TeX-master: t
%%% End:
