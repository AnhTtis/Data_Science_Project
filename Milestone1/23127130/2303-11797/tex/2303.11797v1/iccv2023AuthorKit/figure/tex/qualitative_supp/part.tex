
% \begin{figure*}[t]
%   \centering
%   \renewcommand{\thesubfigure}{}
%      \subfigure[(a) Image]
%     {\includegraphics[width=0.21\linewidth]{iccv2023AuthorKit/figure/pdf/qualitative_supp/part_image.pdf}}
%      \subfigure[(b) OVSeg~\cite{liang2022open}]
%     {\includegraphics[width=0.21\linewidth]{iccv2023AuthorKit/figure/pdf/qualitative_supp/part_ovseg.pdf}}
%     \subfigure[(c) \textbf{\ours (ours)}]
%     {\includegraphics[width=0.21\linewidth]{iccv2023AuthorKit/figure/pdf/qualitative_supp/part_ours.pdf}}\

% \vspace{-10pt}
% \caption{\textbf{Visualization of part segmentaiton.} }
% \vspace{-10pt}
%   \label{fig:partseg-supple}
% \end{figure*}

\begin{figure*}
    \centering
    \includegraphics[width=0.95\linewidth]{iccv2023AuthorKit/figure/pdf/qualitative_supp/part_supp.pdf}\hfill\\
    \vspace{-5pt}
    \caption{\textbf{Qualitative results on human part segmentation.} We compare CAT-Seg with OVSeg~\cite{liang2022open}.}
    \label{fig:partseg-supple}\vspace{-5pt}
\end{figure*}