\twocolumn[{%
\centering
\vspace{-20pt}
\begin{figure}[H]
    \centering
    \hsize=1.0\textwidth
    % \vspace{-20pt}
    \includegraphics[width=\textwidth]{iccv2023AuthorKit/figure/pdf/architecture.pdf}
    \vspace{-5pt}
    \caption{\textbf{More architectural details of \ours:} (a) overall architecture. (b) embedding guidance. Note that a generalized embedding guidance is illustrated to include different attention designs, \ie shifted window attention~\cite{liu2021swin} or linear attention~\cite{katharopoulos2020transformers}. (c) upsampling decoder layer. GN: Group Normalization~\cite{wu2018group}. LN: Layer Normalization~\cite{ba2016layer}.}
    \label{fig:guidance-architecture}
    \vspace{-5pt}
\end{figure}%
}]

\renewcommand{\thesection}{\Alph{section}}
\setcounter{section}{0}
\section*{Appendix}
In the following, we first provide more implementation details in Section~\ref{A}. We then provide additional experimental results and ablation study in Section~\ref{B}. Finally, we present qualitative results for all the benchmarks and human part segmentation in Section~\ref{C} and a discussion of limitations in Section~\ref{D}.


\section{More Details}\label{A}
\subsection{Architectural Details}

In the following, we provide more architectural details. Our overall architecture is first illustrated in Fig.~\ref{fig:guidance-architecture} (a).

\smallbreak
\noindent\textbf{Embedding guidance.}
In this paragraph, we provide more details of embedding guidance, which is designed to facilitate the cost aggregation process by exploiting its rich semantics for a guidance.  We first extract visual and text embeddings from feature backbone and frozen CLIP text encoder~\cite{radford2021learning}, respectively. The embeddings then undergo linear projection and concatenated to the cost volume before query and key projections in aggregation layer. The design is illustrated in Fig.~\ref{fig:guidance-architecture}~(b). In each case, for employing Swin Transformer~\cite{liu2021swin} as a feature backbone, the stage 3 output is used, whereas for ResNet-101~\cite{he2016deep}, the output from the last conv layer of \texttt{conv4\_x} is used as our guidance.

\smallbreak
\noindent\textbf{Upsampling decoder.}
The detailed architecture is illustrated in Fig.~\ref{fig:guidance-architecture}~(c). For multi-level features $E^V_{Dec}$, we leverage the features of last layers from \texttt{conv2\_x} and \texttt{conv3\_x} when ResNet-101~\cite{he2016deep} is used, whereas the output features of stages 1 and 2 are leveraged when Swin Transformer~\cite{liu2021swin} is used. 
%\twocolumn[{%
\centering
\vspace{-20pt}
\begin{figure}[H]
    \centering
    \hsize=1.0\textwidth
    % \vspace{-20pt}
    \includegraphics[width=\textwidth]{iccv2023AuthorKit/figure/pdf/architecture.pdf}
    \vspace{-5pt}
    \caption{\textbf{More architectural details of \ours:} (a) overall architecture. (b) embedding guidance. Note that a generalized embedding guidance is illustrated to include different attention designs, \ie shifted window attention~\cite{liu2021swin} or linear attention~\cite{katharopoulos2020transformers}. (c) upsampling decoder layer. GN: Group Normalization~\cite{wu2018group}. LN: Layer Normalization~\cite{ba2016layer}.}
    \label{fig:guidance-architecture}
    \vspace{-5pt}
\end{figure}%
}]


% \vspace{-10pt}
% \paragraph{Motivation for embedding guidance.}
% As previously stated, direct optimization of the CLIP image encoder can harm open-vocabulary capability of CLIP and cost aggregation can serve an effective way to transfer the knowledge of CLIP from image-level to pixel-level. However, during the cost computation stage, the projection of high-dimensional embeddings into restricted low-dimensional space inevitably results in the loss of semantic information related to class or fine-grained details of the input image. One solution for the loss of information is to aid the cost aggregation with the readily available embedding used for cost computation, extracted from the CLIP image and text encoder. However, as Table.~\ref{tab:feature-vs-cost} implies, direct utilization of such embeddings can lead to performance degradation. To detour this utilization, 


\begin{figure}[t]
    \centering
    \includegraphics[width=0.9\linewidth]{iccv2023AuthorKit/figure/pdf/fine-tuning2.pdf}\hfill\\
    \vspace{-5pt}
    \caption{\textbf{Illustration of the proposed fine-tuning approach.} To fine-tune the CLIP image encoder, we optimize the attention layers and position embeddings exclusively. This approach not only improves efficiency but also enhances performance. Here, $N$ refers to the number of layers in the CLIP image encoder.}
    \label{fig:fine-tune}\vspace{-5pt}
\end{figure}

\subsection{Other Implementation Details}
\smallbreak
\noindent\textbf{Training details.}
In Fig.~\ref{fig:fine-tune}, we visualize the proposed approach to fine-tune the CLIP image encoder, which was introduced in Section 3.5 in the main paper. A cost volume resolution of $H=W=24$ is used for training. The position embeddings of the CLIP image encoder is initialized with bicubic interpolation~\cite{touvron2021training}, and we set training resolution as $384\times 384$. For ViT-B and ViT-L variants, we initialize CLIP~\cite{radford2021learning} with official weights of ViT-B/16 and ViT-L/14@336px respectively. For ViT-H and ViT-G variants, we initialize with OpenCLIP~\cite{cherti2022reproducible} weights trained with LAION-2B~\cite{schuhmann2022laion}. The CLIP text encoder remains frozen across all experiments. The feature backbone networks are pre-trained on ImageNet-1k~\cite{deng2009imagenet} in $224\times224$ resolution for ResNet-101~\cite{he2016deep}, or on ImageNet-21k in $384\times384$ resolution for Swin Transformer~\cite{liu2021swin}. All hyperparameters are kept constant across the evaluation datasets.  

The {\tt Baseline} module is by design the core of the {\tt pygwb} stochastic analysis. %
Its main role is to manage the cross-correlation between {\tt Interferometer} data products, combine these into a single cross-spectrum, which represents the point estimate of the analysis, and calculate the associated error, as introduced in Sec.~\ref{sec: GWB analysis}.

The standard initialization of a {\tt Baseline} object simply requires a pair of {\tt Interferometer} objects. %
\begin{Verbatim}[commandchars=\\\{\},frame=leftline,framesep=1.5ex,framerule=0.8pt,fontsize=\small]
\PY{k+kn}{from} \PY{n+nn}{pygwb} \PY{k+kn}{import} \PY{n}{baseline}
\PY{n}{H2H2\PYZus{}baseline} \PY{o}{=} \PY{n}{baseline}\PY{o}{.}\PY{n}{Baseline}\PY{p}{(}\PY{l+s+s2}{\PYZdq{}}\PY{l+s+s2}{H1\PYZhy{}H2}\PY{l+s+s2}{\PYZdq{}}\PY{p}{,} \PY{n}{H1}\PY{p}{,} \PY{n}{H2}\PY{p}{)}
\end{Verbatim}

Here {\tt H1} and {\tt H2} are {\tt Interferometer} objects. % 
It is also possible to load a previously stored {\tt Baseline} object in {\tt pickle} format by calling the relevant class method. %

The data loaded into the {\tt Interferometer} objects are automatically imported into the {\tt Baseline} object upon initialization. % 
The {\tt Baseline} object relies on the {\tt spectral} module to calculate cross-correlations between the data streams, following the methodology shown in Sec.~\ref{sec:spectral}. %
Similarly, it relies on the {\tt postprocessing} module to obtain the point estimate $\hat{\Omega}^\alpha_{\rm ref}$ and its variance $\sigma^\alpha_{\rm ref}$, as described in Eqs.~(\ref{eq:ptest_postpoc}--\ref{eq:sigma_postpoc}). %
The user may choose to calculate point estimate and sigma spectra or point values; in the latter case, the spectra are automatically stored to facilitate subsequent analyses. %

Calculating $\hat{\Omega}^\alpha_{\rm ref}$, as well as performing parameter estimation on the GWB spectrum, requires the two-detector \gls{orf}, $\gamma_{IJ}$, shown in Eq.~\eqref{eq:orf}. %
The \gls{orf} is calculated at {\tt Baseline} object initialization, then stored as an attribute. %
By default, we assume \gls{gr}, which presents two independent degrees of freedom for the strain field, typically $A=\{+, \times\}$ in the transverse-traceless gauge. For a precise derivation of this function and detector response definitions, see for example~\cite{Romano_2017}. %

The {\tt Baseline} object is also equipped to probe circularly polarized backgrounds~\cite{Seto:2007tn}, and non-\gls{gr} polarizations in the \gls{gwb}, such as scalar and vector backgrounds~\cite{TeVeS}. %
This requires selecting a different choice of $A$, according to the chosen polarization type, which can be declared when calculating $\hat{\Omega}^\alpha_{\rm ref}$ or the \gls{orf} directly. %
Details on the expressions for non-\gls{gr} $\gamma_{IJ}$ functions may be found in the appendix of~\cite{TeVeS}.



\smallbreak
\noindent\textbf{Text prompt templates.}
To obtain text embeddings from the text encoder, we follow CLIP~\cite{radford2021learning} and form sentences with the class names, such as \texttt{"a photo of a \{class\}"}. We ensemble 80 text prompts originally used in CLIP for ImageNet classification, without additionally curating text prompts or synonyms.

\smallbreak
\noindent\textbf{Feature and cost aggregation baselines.}
In this paragraph, we provide more details of the architecture of two models introduced in Fig.~\ref{fig:conceptual_qual}: One is feature aggregation method and the other is cost aggregation method. As shown in Fig.~\ref{fig:baseline} (a), the feature aggregation method directly leverages the features extracted from CLIP image encoder by feeding the concatenated image and text embeddings into the upsampling decoder. Fig.~\ref{fig:baseline} (b) shows the cost aggregation approach that constructs cost volume instead, and subsequent embedding layer processes it to feed into upsampling decoder. 
%

\subsection{Patch Inference}
The practicality of Vision Transformer (ViT)~\cite{dosovitskiy2020image} for high-resolution image processing has been limited due to its quadratic complexity with respect to the sequence length. As our model leverages ViT to extract image embeddings, \ours may struggle to output to the conventional image resolutions commonly employed in semantic segmentation literature, such as $640\times 640$~\cite{cheng2021per,ghiasi2022scaling}, without sacrificing some accuracy made by losing some fine-details. Although we can adopt the same approach proposed in~\cite{zhou2022extract} to upsample the positional embedding~\cite{zhou2022extract}, we ought to avoid introducing excessive computational burdens, and thus adopt an effective inference strategy without requiring additional training which is illustrated in Fig.~\ref{fig:patch-inference}.

To this end, we begin by partitioning the input image into overlapping patches of size $\frac{H}{N_P} \times \frac{W}{N_P}$. Intuitively, given an image size of $640 \times 640$, we partition the image to sub-images of size $384 \times 384$, which matches the image resolution at training phase, and each sub-images has overlapping regions $128 \times 128$. Subsequently, we feed these sub-images and the original image that is resized to $384 \times 384$ into the model. Given the results for each patches and the image, we merge the obtained prediction, while the overlapping regions are averaged to obtain the final prediction. In practice, we employ $N_P=2$,  while adjusting the overlapping region to match the effective resolution of $640\times 640$.

\begin{figure}[t]
    \centering
    \includegraphics[width=1.0\linewidth]{supple/patch_inference_pdf.pdf}\hfill\\
    \vspace{-5pt}
    \caption{\textbf{Illustration of the patch inference.} During inference, we divide the input image into patches, thereby increasing the effective resolution.}
    \label{fig:patch-inference}
    \vspace{-10pt}
\end{figure}


\section{Additional Ablation Study}\label{B}
\subsection{Comparison of Aggregation Baselines}\vspace{-10pt}
\begin{table}[H]
    \centering
    \resizebox{0.48\textwidth}{!}{%
    \begin{tabular}{cl|cccccc}
    \toprule
        & Methods & A-847 & PC-459 & A-150 & PC-59 & PAS-20 & $\textnormal{PAS-20}^b$
        \\
        \midrule\midrule
        \textbf{(I)} & Feature agg. + Freeze & 3.1 & 8.7 & 16.6 & 46.8 & 92.3 & 69.7\\
        \textbf{(II)} & Feature agg. + F.T. & \underline{3.8} & \underline{10.9} & \underline{19.1} & \underline{53.5} & \underline{96.2} & \underline{74.2}\\
        \midrule
        \textbf{(III)} & Cost agg. + Freeze& 2.7 & 5.4 & 11.0 & 27.7 & 65.1 & 44.9\\
        \textbf{(IV)} & Cost agg. + F.T. & \textbf{9.0} & \textbf{17.2} & \textbf{26.9} & \textbf{57.0} & \textbf{96.9} & \textbf{76.8}\\
        \bottomrule
    \end{tabular}%
    }
    \vspace{-5pt}
    \caption{\textbf{Quantitative comparison between feature and cost aggregation.} Cost aggregation acts as an effective alternative to direct fine-tuning of CLIP image encoder. \textit{F.T.: Fine-Tuning.}
    }
    \label{tab:feature-vs-cost}
    \vspace{-10pt}

\end{table}

In this ablation study, we provide a quantitative comparison of two aggregation baselines, feature aggregation and cost aggregation, in Table~\ref{tab:feature-vs-cost}. We freeze the CLIP image encoder and only optimize the upsampling decoder, and the results are summarized in \textbf{(I)} and \textbf{(III)}. Subsequently, in \textbf{(II)} and \textbf{(IV)}, we fine-tune the CLIP image encoder on top of \textbf{(I)} and \textbf{(III)}. Our results show that feature aggregation  can benefit from fine-tuning, but the gain is only marginal. On the other hand, cost aggregation benefits significantly from fine-tuning, highlighting the effectiveness of cost aggregation to transfer knowledge in CLIP encoder. 

\subsection{Ablation Study of the Number of Layers $N_B$ }\vspace{-10pt}
\begin{table}[H]
    \centering
    \resizebox{0.48\textwidth}{!}{%
    \begin{tabular}{c|cccccc}
    \toprule
        \# of layers & A-847 & PC-459 & A-150 & PC-59 & PAS-20 & $\textnormal{PAS-20}^b$
        \\
        \midrule\midrule
        1 & \underline{10.7} & 16.3 & \underline{30.6} & 61.2 & \textbf{96.6} & 81.7\\
        2 & \textbf{10.8} & \textbf{20.4} & \textbf{31.5} & \underline{62.0} & \textbf{96.6} & \underline{81.8}\\
        3 & \textbf{10.8} & \underline{20.0} & \textbf{31.5} & 61.9 & \underline{96.5} & \underline{81.8}\\
        4 & \textbf{10.8} & \textbf{20.4} & \textbf{31.5} & \textbf{62.1} & \textbf{96.6} & \textbf{82.2}\\
        \bottomrule
    \end{tabular}%
    }
    \vspace{-5pt}
    \caption{\textbf{Effects of varying $N_B$.}
    }
    \label{tab:layer-ablation}
    \vspace{-10pt}

\end{table}
Table~\ref{tab:layer-ablation} summarizes the effects of varying $N_B$. From the results, we find that choosing higher $N_B$ does not always lead to performance improvements.
% ; rather, as the model gets heavier, it risks overfitting, which is confirmed by the results when $N_B = 4$. 
Note that we chose $N_B = 2$ to balance between performance and model capacity.

\subsection{Ablation Study of Inference Strategy}\vspace{-10pt}
\begin{table}[H]
    \centering
    \resizebox{\linewidth}{!}{
   \begin{tabular}{l|cccccc}
        \toprule
        Methods & A-847 & PC-459 & A-150 & PC-59 & PAS-20 & $\textnormal{PAS-20}^b$
         \\
        \midrule\midrule
        \ours w/ training reso. & \underline{10.1}& \underline{18.9}& 29.8& \underline{59.8}& \underline{96.1}& 79.7\\
        \ours$\dagger$ & OOM & 14.8& \underline{30.4}& 59.7& 95.9& \underline{80.3}\\
        \hlrow\ours (ours) & \textbf{10.8}& \textbf{20.4}& \textbf{31.5}& \textbf{62.0}& \textbf{96.6}& \textbf{81.8}\\
        \bottomrule
\end{tabular}}%
\vspace{-5pt}
        \caption{\textbf{Ablation study of inference strategy.} CLIP with ViT-L is used for ablation. $\dagger$: Process $640\times 640$ by upsampling position embedding of CLIP ViT. \textit{OOM: Out-of-memory.}}
    \label{tab:inference-ablation}
    \vspace{-10pt}
\end{table}
Table~\ref{tab:inference-ablation} presents effects of different inference strategies for our model. The first row shows the results using the training resolution at inference time. The second row represents a variant that upsamples the positional embedding within the CLIP image encoder, allowing the encoder to process higher resolution images. Finally, the last row adopts the proposed patch inference strategy. It is shown that increasing the positional embedding introduces substantial memory overhead. Moreover, for some datasets, using training resolution yields better performance. On the other hand, our proposed approach can bring large performance gains, while also ensuring high efficiency.

\subsection{Effects of Upsampling Decoder}\vspace{-10pt}
\begin{table}[H]
    \centering
    \resizebox{\linewidth}{!}{
   \begin{tabular}{l|cccccc}
        \toprule
        Methods & A-847 & PC-459 & A-150 & PC-59 & PAS-20 & $\textnormal{PAS-20}^b$
         \\
        \midrule\midrule
        \ours w/o upsampling decoder & \underline{10.1}& \underline{18.9}& \underline{29.8}& \underline{59.8}& \underline{96.1}& \underline{79.7}\\
        \hlrow\ours (ours) & \textbf{10.8} & \textbf{20.4}& \textbf{31.5}& \textbf{62.0}& \textbf{96.6}& \textbf{81.8}\\
        \bottomrule
\end{tabular}}%
\vspace{-5pt}
        \caption{\textbf{Ablation study of upsampling decoder.} CLIP with ViT-L is used for ablation.}
    \label{tab:conv-decoder}
    \vspace{-10pt}
\end{table}
We provide an quantitative results of adopting the proposed upsampling decoder in Table~\ref{tab:conv-decoder}. The results show consistent improvements across all the benchmarks.

\subsection{Comparison of Inference Time}\vspace{-10pt}
\begin{figure}[t]
    \centering
  \renewcommand{\thesubfigure}{}
     \subfigure[(a)]
    {\includegraphics[width=0.497\linewidth]{iccv2023AuthorKit/figure/pdf/training_time.pdf}}\hfill
     \subfigure[(b)]
    {\includegraphics[width=0.497\linewidth]{iccv2023AuthorKit/figure/pdf/inference_time.pdf}}\hfill\\
\vspace{-10pt}
\caption{\textbf{Run-time comparison.} (a) Comparison of training time among different fine-tuning choices. (b) Comparison of inference time to~\cite{ding2022decoupling}. \textit{F.T.: Fine-Tuning.}}
\vspace{-10pt}
  \label{fig:inference-time}
\end{figure}
In Table~\ref{tab:inference-time}, we show run-time comparison at inference phase to the recent two-stage approach~\cite{ding2022decoupling}. We report the results with different VLM, \textit{e.g.,} ViT-B, L, H and G.   Note that the inference speed of the proposed method differs depending on the number of categories, resulting in different run-times across the evaluation datasets. In comparison to~\cite{ding2022decoupling}, we report the mean run-time, as its performance demonstrates minimal variation with respect to the number of categories.

When utilizing ViT-B for vision-language models, our method exhibits a relatively slower inference time for some scenarios, including A-847 and PC-459.  However, as we scale the VLM to larger variants, such as ViT-L, H, and G, our approach enjoys significantly faster inference by almost over 20 times faster, highlighting the efficiency of the proposed method.

% \vspace{-10pt}
% \paragraph{Result with additional prompt engineering.}
% In Table~\ref{tab:prompt-engineering}, we study the effect of prompt engineering introduced in~\cite{ghiasi2022scaling}. This includes augmenting the class description with synonyms or adding detailed description for solving ambiguity of polysemy. During inference, .As the result shows, our method also can benefit from manual prompt engineering.

\section{More Qualitative Results}\label{C}
We provide more qualitative results on A-847~\cite{zhou2019semantic} in Fig.~\ref{fig:ade847}, PC-459~\cite{mottaghi2014role} in Fig.~\ref{fig:pc459}, A-150~\cite{zhou2019semantic} in Fig.~\ref{fig:ade150}, and PC-59~\cite{mottaghi2014role} in Fig.~\ref{fig:pc59}. We also further compare the results in A-847~\cite{zhou2019semantic} with other methods~\cite{ding2022decoupling, xu2022simple, liang2022open} in Fig.~\ref{fig:ade847-comparison}.

\smallbreak
\noindent\textbf{Qualitative results on part segmentation.}
We further compare human part segmentation results with \cite{liang2022open} in Fig.~\ref{fig:partseg-supple}. 
% Specifically, we provide 5 classes for part segmentation as follows: \{``background", ``head of person", ``torso of person", ``arm of person", ``leg of person"\}.

\section{Limitations}\label{D}
To evaluate open-vocabulary semantic segmentation results, we follow ~\cite{ghiasi2022scaling, liang2022open} and compute the metrics using the other segmentation datasets. However, since the ground-truth segmentation maps involve some ambiguities, the reliability of  the evaluation dataset is somewhat questionable. For example, the last row of Fig.~\ref{fig:qualitative}~(e) exemplifies how our predictions in the mirror, ``sky" and ``car", as well as ``plant" in between the ``fence", are classified as wrong segmentation as the ground truth classes are ``mirror" and ``fence". Constructing a more reliable dataset including ground-truths accounting for above issue for accurate evaluation is an intriguing topic. 
%but this is beyond the scope of our work.

\begin{figure*}
    \centering
    \includegraphics[width=0.95\linewidth]{supple/ade847_pdf.pdf}\hfill\\
    \vspace{-5pt}
    \caption{\textbf{Qualitative results on ADE20K~\citep{zhou2019semantic} with 847 categories.}}
    \label{fig:ade847}\vspace{-5pt}
\end{figure*}

\begin{figure*}
    \centering
    \includegraphics[width=0.95\linewidth]{iccv2023AuthorKit/figure/pdf/qualitative_supp/pc459.pdf}\hfill\\
    \vspace{-5pt}
    \caption{\textbf{Qualitative results on PASCAL Context~\cite{mottaghi2014role} with  459 categories.}}
    \label{fig:pc459}\vspace{-5pt}
\end{figure*}
\begin{figure*}
    \centering
    \includegraphics[width=0.95\linewidth]{iccv2023AuthorKit/figure/pdf/qualitative_supp/ade150.pdf}\hfill\\
    \vspace{-5pt}
    \caption{\textbf{Qualitative results on ADE20K~\cite{zhou2019semantic} with 150 categories.}}
    \label{fig:ade150}\vspace{-5pt}
\end{figure*}
\begin{figure*}
    \centering
    \includegraphics[width=0.95\linewidth]{iccv2023AuthorKit/figure/pdf/qualitative_supp/pc59.pdf}\hfill\\
    \vspace{-5pt}
    \caption{\textbf{Qualitative results on PASCAL Context~\cite{mottaghi2014role} with 59 categories.}}
    \label{fig:pc59}\vspace{-5pt}
\end{figure*}
\begin{figure*}
    \centering
    \includegraphics[width=1.0\linewidth]{supple/supp_comparison_pdf.pdf}\hfill\\
    \vspace{-5pt}
    \caption{\textbf{Comparison of qualitative results on ADE20K~\citep{zhou2019semantic} with 847 categories.} We compare CAT-Seg with ZegFormer~\citep{ding2022decoupling}, ZSseg~\citep{xu2022simple}, and OVSeg~\citep{liang2022open} on A-847 dataset.}
    \label{fig:ade847-comparison}\vspace{-5pt}
\end{figure*}

\begin{figure*}[h] 
\centering
\includegraphics[width=1\textwidth]{Figures/part.pdf}
\caption{Qualitative results on SP-Part.}
\label{fig:part_seg}
% \vspace{-10pt}
\end{figure*}


\clearpage
