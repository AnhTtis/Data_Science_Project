
\begin{figure*}[t]
    \centering
    \Large \textbf{CAT-Seg: Cost Aggregation for Open-Vocabulary Semantic Segmentation}\\[5pt]
    \large \textbf{-- Supplementary Material --}

    
    \hsize=1.0\textwidth
    \includegraphics[width=\textwidth]{supple/architecture.pdf}
    \vspace{-10pt}
    \caption{\textbf{More architectural details of CAT-Seg:} (a) overall architecture. (b) embedding guidance. Note that a generalized embedding guidance is illustrated to include different attention designs, \ie, shifted window attention~\cite{liu2021swin} or linear attention~\cite{katharopoulos2020transformers}. (c) upsampling decoder layer. GN: Group Normalization~\cite{wu2018group}. LN: Layer Normalization~\cite{ba2016layer}.}
    \label{fig:guidance-architecture}
    \vspace{-10pt}
\end{figure*}

\renewcommand{\thesection}{\Alph{section}}
\setcounter{section}{0}


In the following, we provide the full results from MESS~\cite{blumenstiel2023mess} in Section~A. We further provide implementation details in Section~B. We then provide additional experimental results and ablation study in Section~C. Finally, we present qualitative results for the benchmarks in Section~D and a discussion of limitations in Section~E.





\section{More Results}\label{A}
\begin{table*}[h]
    \begin{center}
    

    
    \resizebox{\textwidth}{!}{
    \begin{tabular}{l|cccccc|ccccc|cccc|cccc|ccc|c}
    \toprule
 & \multicolumn{6}{c|}{General} & \multicolumn{5}{c|}{Earth Monitoring} & \multicolumn{4}{c|}{Medical Sciences} & \multicolumn{4}{c|}{Engineering} & \multicolumn{3}{c|}{Agri. and Biology} & \\
 & \rotatebox[origin=l]{90}{BDD100K} & \rotatebox[origin=l]{90}{Dark Zurich} & \rotatebox[origin=l]{90}{MHP v1} & \rotatebox[origin=l]{90}{FoodSeg103} & \rotatebox[origin=l]{90}{ATLANTIS} & \rotatebox[origin=l]{90}{DRAM} & \rotatebox[origin=l]{90}{iSAID} & \rotatebox[origin=l]{90}{ISPRS Pots.} & \rotatebox[origin=l]{90}{WorldFloods} & \rotatebox[origin=l]{90}{FloodNet} & \rotatebox[origin=l]{90}{UAVid} & \rotatebox[origin=l]{90}{Kvasir-Inst.} & \rotatebox[origin=l]{90}{CHASE DB1} & \rotatebox[origin=l]{90}{CryoNuSeg} & \rotatebox[origin=l]{90}{PAXRay-4} & \rotatebox[origin=l]{90}{Corrosion CS} & \rotatebox[origin=l]{90}{DeepCrack} & \rotatebox[origin=l]{90}{PST900} & \rotatebox[origin=l]{90}{ZeroWaste-f} & \rotatebox[origin=l]{90}{SUIM} & \rotatebox[origin=l]{90}{CUB-200} & \rotatebox[origin=l]{90}{CWFID} & \rotatebox[origin=l]{90}{Mean} \\
\midrule\midrule
\textit{Random (LB)} & \phantom{0}\textit{1.48} & \phantom{0}\textit{1.31} & \phantom{0}\textit{1.27} & \phantom{0}\textit{0.23} & \phantom{0}\textit{0.56} & \phantom{0}\textit{2.16} & \phantom{0}\textit{0.56} & \phantom{0}\textit{8.02} & \textit{18.43} & \phantom{0}\textit{3.39} & \phantom{0}\textit{5.18} & \textit{27.99} & \textit{27.25} & \textit{31.25} & \textit{31.53} & \textit{9.3} & \textit{26.52} & \phantom{0}\textit{4.52} & \phantom{0}\textit{6.49} & \phantom{0}\textit{5.3}\phantom{0} & \phantom{0}\textit{0.06} & \textit{13.08} & \textit{10.27} \\
\textit{Best sup. (UB)} & \textit{44.8}\phantom{0} & \textit{63.9}\phantom{0} & \textit{50.0}\phantom{0} & \textit{45.1}\phantom{0} & \textit{42.22} & \textit{45.71} & \textit{65.3}\phantom{0} & \textit{87.56} & \textit{92.71} & \textit{82.22} & \textit{67.8}\phantom{0} & \textit{93.7}\phantom{0} & \textit{97.05} & \textit{73.45} & \textit{93.77} & \textit{49.92} & \textit{85.9}\phantom{0} & \textit{82.3}\phantom{0} & \textit{52.5}\phantom{0} & \textit{74.0}\phantom{0} & \textit{84.6}\phantom{0} & \textit{87.23} & \textit{70.99} \\
\midrule
ZSSeg-B & 32.36 & 16.86 & \phantom{0}7.08 & \phantom{0}8.17 & 22.19 & 33.19 & \phantom{0}3.8\phantom{0} & 11.57 & 23.25 & 20.98 & 30.27 & 46.93 & \underline{37.0}\phantom{0} & \textbf{38.7} & \underline{44.66} & \phantom{0}3.06 & 25.39 & 18.76 & \phantom{0}8.78 & \underline{30.16} & \phantom{0}4.35 & 32.46 & 22.73 \\
ZegFormer-B & 14.14 & \phantom{0}4.52 & \phantom{0}4.33 & 10.01 & 18.98 & 29.45 & \phantom{0}2.68 & 14.04 & 25.93 & 22.74 & 20.84 & 27.39 & 12.47 & 11.94 & 18.09 & \phantom{0}4.78 & 29.77 & 19.63 & \underline{17.52} & 28.28 & \underline{16.8}\phantom{0} & 32.26 & 17.57 \\
X-Decoder-T & \underline{47.29} & 24.16 & \phantom{0}3.54 & \phantom{0}2.61 & 27.51 & 26.95 & \phantom{0}2.43 & 31.47 & 26.23 & \phantom{0}8.83 & 25.65 & 55.77 & 10.16 & 11.94 & 15.23 & \phantom{0}1.72 & 24.65 & 19.44 & 15.44 & 24.75 & \phantom{0}0.51 & 29.25 & 19.8\phantom{0} \\
SAN-B & 37.4\phantom{0} & 24.35 & \phantom{0}8.87 & \underline{19.27} & \underline{36.51} & 49.68 & \phantom{0}4.77 & \underline{37.56} & 31.75 & \underline{37.44} & \textbf{41.65} & \underline{69.88} & 17.85 & 11.95 & 19.73 & \phantom{0}3.13 & \underline{50.27} & 19.67 & \textbf{21.27} & 22.64 & \textbf{16.91} & \phantom{0}5.67 & 26.74 \\
OpenSeeD-T & \textbf{47.95} & \underline{28.13} & \phantom{0}2.06 & \phantom{0}9.0\phantom{0} & 18.55 & 29.23 & \phantom{0}1.45 & 31.07 & 30.11 & 23.14 & 39.78 & 59.69 & \textbf{46.68} & 33.76 & 37.64 & 13.38 & 47.84 & \phantom{0}2.5\phantom{0} & \phantom{0}2.28 & 19.45 & \phantom{0}0.13 & 11.47 & 24.33 \\
Gr.-SAM-B & 41.58 & 20.91 & \textbf{29.38} & 10.48 & 17.33 & \underline{57.38} & \underline{12.22} & 26.68 & \underline{33.41} & 19.19 & 38.34 & 46.82 & 23.56 & \underline{38.06} & 41.07 & \textbf{20.88} & \textbf{59.02} & \textbf{21.39} & 16.74 & 14.13 & \phantom{0}0.43 & \underline{38.41} & \underline{28.52} \\
\hlrow CAT-Seg-B & 46.71 &\textbf{28.86} &\underline{23.74} &\textbf{26.69} &\textbf{40.31} &\textbf{65.81} &\textbf{19.34} &\textbf{45.36} &\textbf{35.72} &\textbf{37.57} &\underline{41.55} &48.2\phantom{0} &16.99 &15.7\phantom{0} &31.48 &12.29 &31.67 &\underline{19.88} &\underline{17.52} &\textbf{44.71} &10.23 &\textbf{42.77} & \textbf{31.96} \\

\midrule
OVSeg-L & \underline{45.28} & 22.53 & \phantom{0}6.24 & 16.43 & 33.44 & 53.33 & \phantom{0}8.28 & 31.03 & 31.48 & 35.59 & 38.8 & \underline{71.13} & \underline{20.95} & \underline{13.45} & 22.06 & \phantom{0}6.82 & 16.22 & \underline{21.89} & 11.71 & 38.17 & 14.0\phantom{0} & 33.76 & 26.94 \\
SAN-L & 43.81 & \underline{30.39} & \phantom{0}9.34 & \underline{24.46} & \underline{40.66} & \underline{68.44} & 11.77 & \textbf{51.45} & \textbf{48.24} & \underline{39.26} & \textbf{43.41} & \textbf{72.18} & \phantom{0}7.64 & 11.94 & \underline{29.33} & \phantom{0}6.83 & \underline{23.65} & 19.01 & \underline{18.32} & \underline{40.01} & \underline{19.3}\phantom{0} & \phantom{0}1.91 & \underline{30.06} \\
Gr.-SAM-L & 42.69 & 21.92 & \underline{28.11} & 10.76 & 17.63 & 60.8\phantom{0} & \underline{12.38} & 27.76 & 33.4\phantom{0} & 19.28 & 39.37 & 47.32 & \textbf{25.16} & \textbf{38.06} & \textbf{44.22} & \textbf{20.88} & \textbf{58.21} & 21.23 & 16.67 & 14.3\phantom{0} & \phantom{0}0.43 & \underline{38.47} & 29.05 \\
\hlrow CAT-Seg-L & \textbf{47.87} &\textbf{34.96} &\textbf{32.54} &\textbf{33.31} &\textbf{45.61} &\textbf{73.82} & \textbf{20.58} & \underline{50.81} & \underline{46.42} & \textbf{41.36} & \underline{40.79} &61.13 &3.72 &11.94 &22.02 &\underline{11.03} &19.9\phantom{0} &\textbf{22.0}\phantom{0} &\textbf{27.87} &\textbf{53.0}\phantom{0} &\textbf{22.93} &\textbf{39.91} &\textbf{34.7}\phantom{0}\\
        \bottomrule
    \end{tabular}
    }
    \end{center}
    \vspace{-15pt}
    \caption{\textbf{Full results of quantitative evaluation on MESS~\citep{blumenstiel2023mess}.} 
    }\label{tab:mess_full}
    \vspace{-10pt}
\end{table*}

\paragraph{Full quantitative results on MESS benchmark.}
In Table~\ref{tab:mess_full}, we provide the results of all 22 datasets within MESS~\cite{blumenstiel2023mess}, including results from Grounded-SAM~\cite{Grounded-SAM_Contributors_Grounded-Segment-Anything_2023}.


\section{More Details}\label{B}
\subsection{Architectural Details}

In the following, we provide more architectural details. Our detailed overall architecture is illustrated in Fig.~\ref{fig:guidance-architecture} (a).

\smallbreak
\noindent\textbf{Embedding guidance.}
In this paragraph, we provide more details of embedding guidance, which is designed to facilitate the cost aggregation process by exploiting its rich semantics for a guidance.  We first extract visual and text embeddings from CLIP encoders~\cite{radford2021learning}. The embeddings then undergo linear projection and concatenated to the cost volume before query and key projections in aggregation layer. The design is illustrated in Fig.~\ref{fig:guidance-architecture}~(b).

\smallbreak
\noindent\textbf{Upsampling decoder.}
The detailed architecture is illustrated in Fig.~\ref{fig:guidance-architecture}(c). In our upsampling decoder, we start by taking high-resolution features from the CLIP ViT model~\cite{dosovitskiy2020image}. We then apply a single transposed convolution layer to these extracted features to generate an upsampled feature map. Initially, the extracted feature maps have a resolution of $24\times 24$ pixels. However, after processing them with the transposed convolution operation, we increase their resolution to $48\times 48$ pixels for the first feature map, denoted as $E^V_{Dec,1}$, and to $96\times 96$ pixels for the second feature map, denoted as $E^V_{Dec,2}$.

To obtain $E^V_{Dec,1}$, we utilize the output of the 8th layer for the ViT-B/16 model, and for the ViT-L/14 model, we use the output of the 16th layer. For the extraction of $E^V_{Dec,2}$, we employ shallower features: the output of the 4th layer for the ViT-B/16 model as a VLM, and the output of the 8th layer for the ViT-L/14 model. These features are employed to enhance cost embeddings with fine details using a U-Net-like architecture~\cite{ronneberger2015u}.


\subsection{Other Implementation Details}
\smallbreak
\noindent\textbf{Training details.}
A resolution of $H=W=24$ is used during training for constructing cost volume. The position embeddings of the CLIP image encoder is initialized with bicubic interpolation~\cite{touvron2021training}, and we set training resolution as $384\times 384$. For ViT-B and ViT-L variants, we initialize CLIP~\cite{radford2021learning} with official weights of ViT-B/16 and ViT-L/14@336px respectively. All hyperparameters are kept constant across the evaluation datasets.  

\smallbreak
\noindent\textbf{Text prompt templates.}
To obtain text embeddings from the text encoder, we form sentences with the class names, such as \texttt{"A photo of a \{class\}"}. We do not explore handcrafted prompts in this work, but it is open for future investigation. %


\begin{table*}[h]
    \begin{center}
       
    \resizebox{\textwidth}{!}{
    \begin{tabular}{lllccl}
\toprule
Dataset & Link & Licence & Split & \# of classes & Classes \\
\midrule\midrule
BDD100K \cite{yu2020bdd100k} & \href{https://bdd-data.berkeley.edu}{berkeley.edu} & \href{https://doc.bdd100k.com/license.html}{custom} & val & 19 & [road; sidewalk; building; wall; fence; pole; traffic light; traffic sign; ...] \\
Dark Zurich \cite{sakaridis2019guided} & \href{https://www.trace.ethz.ch/publications/2019/GCMA_UIoU/}{ethz.ch} & custom & val & 20 & [unlabeled; road; sidewalk; building; wall; fence; pole; traffic light; ...] \\
MHP v1 \cite{li2017multiple} & \href{https://github.com/ZhaoJ9014/Multi-Human-Parsing}{github.com} & \href{https://lv-mhp.github.io/}{custom} & test & 19 & [others; hat; hair; sunglasses; upper clothes; skirt; pants; dress; ...] \\
FoodSeg103 \cite{wu2021large} & \href{https://xiongweiwu.github.io/foodseg103.html}{github.io} & Apache 2.0 & test & 104 & [background; candy; egg tart; french fries; chocolate; biscuit; popcorn; ...] \\
ATLANTIS \cite{erfani2022atlantis} & \href{https://github.com/smhassanerfani/atlantis}{github.com} & Flickr (images) & test & 56 & [bicycle; boat; breakwater; bridge; building; bus; canal; car; ...] \\
DRAM \cite{cohen2022semantic} & \href{https://faculty.runi.ac.il/arik/site/artseg/Dram-Dataset.html}{ac.il} & custom (in download) & test & 12 & [bird; boat; bottle; cat; chair; cow; dog; horse; ...] \\
\hline
iSAID \cite{waqas2019isaid} & \href{https://captain-whu.github.io/iSAID/dataset.html}{github.io} & Google Earth (images) & val & 16 & [others; boat; storage tank; baseball diamond; tennis court; bridge; ...] \\
ISPRS Potsdam \cite{DatasetISPRSPotsdam} & \href{https://www.isprs.org/education/benchmarks/UrbanSemLab/default.aspx}{isprs.org} & no licence provided\footnote{Upon request, the naming of the data provider and project is required.} & test & 6 & [road; building; grass; tree; car; others] \\
WorldFloods \cite{mateo2021towards} & \href{https://github.com/spaceml-org/ml4floods/blob/main/jupyterbook/content/worldfloods_dataset.md}{github.com} & CC NC 4.0 & test & 3 & [land; water and flood; cloud] \\
FloodNet \cite{rahnemoonfar2021floodnet} & \href{https://github.com/BinaLab/FloodNet-Supervised_v1.0}{github.com} & \href{https://cdla.dev/permissive-1-0/}{custom} & test & 10 & [building-flooded; building-non-flooded; road-flooded; water; tree; ...] \\
UAVid \cite{lyu2020uavid} & \href{https://uavid.nl}{uavid.nl} & CC BY-NC-SA 4.0 & val & 8 & [others; building; road; tree; grass; moving car; parked car; humans] \\
\hline
Kvasir-Inst. \cite{jha2021kvasir} & \href{https://datasets.simula.no/kvasir-instrument/}{simula.no} & \href{https://datasets.simula.no/kvasir-instrument/}{custom} & test & 2 & [others; tool] \\
CHASE DB1 \cite{fraz2012ensemble} & \href{https://blogs.kingston.ac.uk/retinal/chasedb1/}{kingston.ac.uk} & CC BY 4.0 & test & 2 & [others; blood vessels] \\
CryoNuSeg \cite{mahbod2021cryonuseg} & \href{https://www.kaggle.com/datasets/ipateam/segmentation-of-nuclei-in-cryosectioned-he-images}{kaggle.com} & CC BY-NC-SA 4.0 & test & 2 & [others; nuclei in cells] \\
PAXRay-4 \cite{seibold2022detailed} & \href{https://constantinseibold.github.io/paxray/}{github.io} & \href{https://constantinseibold.github.io/paxray/}{custom} & test & 4x2 & [others, lungs], [others, bones], [others, mediastinum], [others, diaphragm] \\
\hline
Corrosion CS \cite{bianchi2021corrosion} & \href{https://figshare.com/articles/dataset/Corrosion_Condition_State_Semantic_Segmentation_Dataset/16624663}{figshare.com} & CC0 & test & 4 & [others; steel with fair corrosion; ... poor corrosion; ... severe corrosion] \\
DeepCrack \cite{liu2019deepcrack} & \href{https://github.com/yhlleo/DeepCrack/tree/master}{github.com} & \href{https://github.com/yhlleo/DeepCrack/tree/master}{custom} & test & 2 & [concrete or asphalt; crack] \\
PST900 \cite{shivakumar2020pst900} & \href{https://github.com/ShreyasSkandanS/pst900_thermal_rgb}{github.com} & GPL-3.0 & test & 5 & [background; fire extinguisher; backpack; drill; human] \\
ZeroWaste-f \cite{bashkirova2022zerowaste} & \href{http://ai.bu.edu/zerowaste/}{ai.bu.edu} & CC-BY-NC 4.0 & test & 5 & [background or trash; rigid plastic; cardboard; metal; soft plastic] \\
\hline
SUIM \cite{islam2020semantic} & \href{https://irvlab.cs.umn.edu/resources/suim-dataset}{umn.edu} & MIT & test & 8 & [human diver; reefs and invertebrates; fish and vertebrates; ...] \\
CUB-200 \cite{welinder2010caltech} & \href{https://www.vision.caltech.edu/datasets/cub_200_2011/}{caltech.edu} & \href{https://www.vision.caltech.edu/datasets/cub_200_2011/}{custom} & test & 201 & [background; Laysan Albatross; Sooty Albatross; Crested Auklet; ...] \\
CWFID \cite{haug2015crop} & \href{https://github.com/cwfid/dataset}{github.com} & \href{https://github.com/cwfid/dataset}{custom} & test & 3 & [ground; crop seedling; weed] \\
\bottomrule
    \end{tabular}
}
    \end{center}
    \vspace{-15pt}
        \caption{\textbf{Details of the datasets in the MESS benchmark~\cite{blumenstiel2023mess}.} 
    }\label{tab:mess-detail}
    \vspace{-10pt}
\end{table*}


\begin{figure}[t]
    \centering
    \includegraphics[width=1.0\linewidth]{supple/patch_inference_pdf.pdf}\hfill\\
    \vspace{-5pt}
    \caption{\textbf{Illustration of the patch inference.} During inference, we divide the input image into patches, thereby increasing the effective resolution.}
    \label{fig:patch-inference}
    \vspace{-10pt}
\end{figure}

\subsection{Patch Inference}
The practicality of Vision Transformer (ViT)~\cite{dosovitskiy2020image} for high-resolution image processing has been limited due to its quadratic complexity with respect to the sequence length. As our model leverages ViT to extract image embeddings, \ours may struggle to output to the conventional image resolutions commonly employed in semantic segmentation literature, such as $640\times 640$~\cite{cheng2021per,ghiasi2022scaling}, without sacrificing some accuracy made by losing some fine-details. Although we can adopt the same approach proposed in~\cite{zhou2022extract} to upsample the positional embedding~\cite{zhou2022extract}, we ought to avoid introducing excessive computational burdens, and thus adopt an effective inference strategy without requiring additional training which is illustrated in Fig.~\ref{fig:patch-inference}.

To this end, we begin by partitioning the input image into overlapping patches of size $\frac{H}{N_P} \times \frac{W}{N_P}$. Intuitively, given an image size of $640 \times 640$, we partition the image to sub-images of size $384 \times 384$, which matches the image resolution at training phase, and each sub-images has overlapping regions $128 \times 128$. Subsequently, we feed these sub-images and the original image that is resized to $384 \times 384$ into the model. Given the results for each patches and the image, we merge the obtained prediction, while the overlapping regions are averaged to obtain the final prediction. In practice, we employ $N_P=2$,  while adjusting the overlapping region to match the effective resolution of $640\times 640$.

\subsection{More Details of MESS Benchmark}

In Table~\ref{tab:mess-detail}, we provide details of the datasets in the MESS benchmark~\cite{blumenstiel2023mess}.

\section{Additional Ablation Study}\label{C}

\subsection{Ablation Study of Inference Strategy}\vspace{-10pt}
\begin{table}[H]
    \centering
    \resizebox{\linewidth}{!}{
   \begin{tabular}{l|cccccc}
        \toprule
        Methods & A-847 & PC-459 & A-150 & PC-59 & PAS-20 & $\textnormal{PAS-20}^b$
         \\
        \midrule\midrule
        \ours w/ training reso. & \underline{10.1}& \underline{18.9}& 29.8& \underline{59.8}& \underline{96.1}& 79.7\\
        \ours$\dagger$ & OOM & 14.8& \underline{30.4}& 59.7& 95.9& \underline{80.3}\\
        \hlrow\ours (ours) & \textbf{10.8}& \textbf{20.4}& \textbf{31.5}& \textbf{62.0}& \textbf{96.6}& \textbf{81.8}\\
        \bottomrule
\end{tabular}}%
\vspace{-5pt}
        \caption{\textbf{Ablation study of inference strategy.} CLIP with ViT-L is used for ablation. $\dagger$: Process $640\times 640$ by upsampling position embedding of CLIP ViT. \textit{OOM: Out-of-memory.}}
    \label{tab:inference-ablation}
    \vspace{-10pt}
\end{table}
Table~\ref{tab:inference-ablation} presents effects of different inference strategies for our model. The first row shows the results using the training resolution at inference time. The last row adopts the proposed patch inference strategy. It is shown that our proposed approach can bring large performance gains, compared to using the training resolution.

\subsection{Ablation on VLM}

\begin{table}[H]
\centering
\resizebox{\linewidth}{!}{
\begin{tabular}{l|cccccc}
        \toprule
        VLM & A-847 & PC-459 & A-150 & PC-59 & PAS-20 & $\textnormal{PAS-20}^b$
         \\
        \midrule\midrule
        EVA-02-CLIP-L/14~\cite{sun2023eva} & \underline{16.4} & \underline{24.5} & 37.8 & \underline{62.7} & \textbf{97.9} & \textbf{83.7}\\
        SigLIP-ViT-L/16~\cite{zhai2023sigmoid} & \textbf{18.0}& \textbf{26.1} & \textbf{39.1} & 60.9& \underline{97.2}&80.8\\
        \hlrow CLIP-ViT-L/14 & 16.0 & 23.8 & \underline{37.9} & \textbf{63.3} & 97.0 & \underline{82.5} \\
        \bottomrule
\end{tabular}}
    \vspace{-5pt}
\caption{\textbf{Results on various VLMs.}}
    \vspace{-10pt}
    \label{tab:vlms}
\end{table}

Table~\ref{tab:vlms} shows the results with various VLMs. We found that \ours can be applied to various VLMs, and better results can be obtained when a more powerful model is applied.

\section{More Qualitative Results}\label{D}
We provide more qualitative results on A-847~\cite{zhou2019semantic} in Fig.~\ref{fig:ade847}, PC-459~\cite{mottaghi2014role} in Fig.~\ref{fig:pc459}, A-150~\cite{zhou2019semantic} in Fig.~\ref{fig:ade150}, and PC-59~\cite{mottaghi2014role} in Fig.~\ref{fig:pc59}. We also further compare the results in A-847~\cite{zhou2019semantic} with other methods~\cite{ding2022decoupling, xu2022simple, liang2022open} in Fig.~\ref{fig:ade847-comparison}.


\section{Limitations}\label{E}
To evaluate open-vocabulary semantic segmentation results, we follow ~\cite{ghiasi2022scaling, liang2022open} and compute the metrics using the other segmentation datasets. However, since the ground-truth segmentation maps involve some ambiguities, the reliability of  the evaluation dataset is somewhat questionable. %
Constructing a more reliable dataset including ground-truths accounting for above issue for accurate evaluation is an intriguing topic. 


\begin{figure*}
    \centering
    \includegraphics[width=0.95\linewidth]{supple/ade847_pdf.pdf}\hfill\\
    \vspace{-5pt}
    \caption{\textbf{Qualitative results on ADE20K~\citep{zhou2019semantic} with 847 categories.}}
    \label{fig:ade847}\vspace{-5pt}
\end{figure*}

\begin{figure*}
    \centering
    \includegraphics[width=0.95\linewidth]{iccv2023AuthorKit/figure/pdf/qualitative_supp/pc459.pdf}\hfill\\
    \vspace{-5pt}
    \caption{\textbf{Qualitative results on PASCAL Context~\cite{mottaghi2014role} with  459 categories.}}
    \label{fig:pc459}\vspace{-5pt}
\end{figure*}
\begin{figure*}
    \centering
    \includegraphics[width=0.95\linewidth]{iccv2023AuthorKit/figure/pdf/qualitative_supp/ade150.pdf}\hfill\\
    \vspace{-5pt}
    \caption{\textbf{Qualitative results on ADE20K~\cite{zhou2019semantic} with 150 categories.}}
    \label{fig:ade150}\vspace{-5pt}
\end{figure*}
\begin{figure*}
    \centering
    \includegraphics[width=0.95\linewidth]{iccv2023AuthorKit/figure/pdf/qualitative_supp/pc59.pdf}\hfill\\
    \vspace{-5pt}
    \caption{\textbf{Qualitative results on PASCAL Context~\cite{mottaghi2014role} with 59 categories.}}
    \label{fig:pc59}\vspace{-5pt}
\end{figure*}
\begin{figure*}
    \centering
    \includegraphics[width=1.0\linewidth]{supple/supp_comparison_pdf.pdf}\hfill\\
    \vspace{-5pt}
    \caption{\textbf{Comparison of qualitative results on ADE20K~\citep{zhou2019semantic} with 847 categories.} We compare CAT-Seg with ZegFormer~\citep{ding2022decoupling}, ZSseg~\citep{xu2022simple}, and OVSeg~\citep{liang2022open} on A-847 dataset.}
    \label{fig:ade847-comparison}\vspace{-5pt}
\end{figure*}


\clearpage
\clearpage
