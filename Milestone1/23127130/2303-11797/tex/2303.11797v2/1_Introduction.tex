\section{Introduction}

Open-vocabulary semantic segmentation aims to assign each pixel in an image to a class label from an unbounded range, defined by text descriptions. To handle the challenge of associating an image with a wide variety of text descriptions, pre-trained vision-language foundation models, \eg, CLIP~\cite{radford2021learning} and ALIGN~\cite{jia2021scaling}, have drawn attention as they exerted strong open-vocabulary recognition capabilities achieved through training on extensive image-text datasets. Nonetheless, these foundation models primarily receive image-level supervision during training, which introduces a notable disparity when applying them to the pixel-level segmentation tasks~\cite{zhou2022extract}. 

To address this gap, recent works~\cite{ding2022decoupling,ghiasi2022scaling,xu2022simple, liang2022open, xu2023open, xu2023side, yu2023convolutions} have reformulated the task into a region-level problem by utilizing mask proposal generators. While this partially bridges the discrepancy between the pre-training and the downstream task, a discernible gap persists between the conceptualization of regions and the entire image for CLIP. 

\section{Threat Model and Advantages of Our Hardware-based Adversarial Detector} \label{sec: motivation}
\ry{In this part, I want to highlight the comparison between hardware and software attacks}
%Normally, software-based adversarial detectors are easier to implement, cheaper to develop and more well-studied than those based on hardware computational signals.
% We would like to stress that our goal for investigating hardware-based adversarial detectors is not to achieve better performance in detection than the conventional white-box software based methods.  
\subsection{Threat Model} \label{sec: threat model}
\ry{This section is threat model: attack is `white-box', detector is `black-box'}
The victim is a DNN classifier, which is pre-trained with a public dataset. The testing dataset may be kept private.
We assume the strongest `white-box' attack model, where the attacker has full knowledge of the victim model and training dataset in order to generate adversarial samples with minimum perturbations. 
On the contrary, the detection system assumes the most limited scenario, under a `black-box' view of the victim, without access to the victim's inputs, parameters, and intermediate outputs or execution details. 
The only information available to the detector to distinguish adversarial samples is the EM side-channel measurement and the victim model's prediction class.
For training the adversarial detector with EM traces, a public benign dataset is used. 

\if false 
\ry{In this part, we discuss more settings of the detector especially the data used in two phases.}
In general, the detecting process can be summed up into two phases, training phase and detecting phase.
To begin with, we train an Out-of-Distribution(OOD) detector on a public benign dataset of the same classification task, which should be distinct from the victim's training dataset.
For each query, the detector will obtain the classification result and an EM trace along with the model execution to fit its EM classifiers and anomaly detectors.  
During the detection phase, the victim model is in operation and under attack when the pre-trained detector decides whether the current input is adversarial or not, only based on the victim model output and its EM trace.
\fi 

\subsection{Advantages}
Compared to software-based adversarial detection methods, our hardware-based detector, EMShepherd, has three distinct advantages: privacy-preserving, portability, and robustness.

\begin{itemize}[leftmargin=*]
    \item \ry{Add a new motivation here. The motivation is that using \name can help the user protect their privacy.} 
    \name protects the DNN model user's data privacy as it is agnostic to the model's inputs, which instead are always required by prior reconstruction-based detection methods~\cite{meng2017magnet, yang2022you}. 
    %Most model users are benign whose inputs may be sensitive and should not be shared with \textit{third-party detectors}. 
    The sensitive inputs should not be shared with \textit{third-party detectors}. 
    Our design only requires the output class labels and the EM signals, which are passively leaked to common acquisition equipment. 
    %    Our design is suitable for such cases as it only requires the EM signals and the inference outputs during the model execution. Generally speaking, EM signals and labels have less private information leakage.
    \item \ry{The second motivation is still related to privacy. This time we consider model privacy when the model structure or parameters should be kept private.}
   \name also protects the model confidentiality.  No model information, including %Using hardware-based detectors can prevent the third-party defender from accessing some confidential model information such as  
   hyper-parameters, parameters, and logits, is needed, in stark contrast to the previous software-based detection methods~\cite{ma2019nic,feinman2017detecting}.
    %Our \name only acquires the EM traces during model inference in a passive and noninvasive manner, 
    The EM data processing and the adversarial detector training process are both victim model-agnostic. 
    Therefore, our method has more general usage, applicable to closed-source DNN applications, which are pervasive in edge devices where the user only queries the models for the final prediction output. 
    \item \ry{The third motivation is portability.}  
    Owing to the model-agnostic feature, EMShepherd can be easily ported for wide-range hardware devices with different DNN implementations for diverse applications. It can be used as a `plug and play' (PnP) device, aside from the target system, to work automatically without user intervention or contact with the victim system. 
    \item \ry{The last motivation is about adaptive attacks, we should propose that EM signal is hard to imitate, so it is hard for adaptive attacks to generate sample fraud both detector and victim.} 
    Adaptive attack~\cite{adaptive} is a threat to most software defense methods where the attacker adjusts the adversarial perturbations to mislead both the victim models and defense systems.
   %  The hardware-based detection method can provide a double protection on top of most software defense methods such as adversarial training.
   %  Although the adptive adversarial example fools the robust model, its computation patterns during the DNN model execution are still well kept in the EM traces and our EMShepherd framework still works well for detecting the new type of adversarial examples.  
   %  Meanwhile, due to the high complexity of EM signals and non-explicit dependency of the EM signals on computations, it is extremely hard to have an adaptive attack on our detection method, i.e., adversarial examples whose EM signals are deliberately controlled to evade the EM-based detector.
   However, due to the high complexity and non-explicit dependency of the EM signals on computations and data, 
   it is extremely hard to have an adaptive attack on our detection method, 
   i.e., adversarial examples whose EM signals are deliberately controlled to evade the EM-based detector. 
\end{itemize}






In this work, we investigate methods to transfer the holistic understanding capability of images to the pixel-level task of segmentation. 
While a straightforward approach would be to fine-tune the encoders of CLIP, existing methods struggle in such attempt~\cite{zhou2022extract,yu2023convolutions,xu2023side} as they encounter significant overfitting problems to the seen classes.
This results in the misalignment of the joint embedding space for unseen classes, as the CLIP features undergo decoder modules for aggregating them into segmentation masks, hence losing their alignment.
Consequently, most methods~\cite{ding2022decoupling,ghiasi2022scaling,xu2022simple, liang2022open, xu2023open, xu2023side, yu2023convolutions} opt for freezing the encoders of CLIP instead, remaining the challenge underexplored.

In this regard, we extend the exploration of adapting CLIP for open-vocabulary semantic segmentation and introduce a novel cost-based framework. We propose to aggregate the cosine similarity between image and text embeddings of CLIP, \textit{i.e.}, the matching cost, drawing parallels to the visual correspondence literature~\cite{kendall2017end}. Surprisingly, we find that fine-tuning CLIP upon this framework effectively adapts CLIP to the downstream task of segmentation for both seen and unseen classes, as shown in Fig.~\ref{fig:motivation}. Noticing this, we delve into better aggregating the cost volume between image and text for segmentation.

Intuitively, the cost volume can be viewed as rough semantic masks grounded to their respective classes, as illustrated in Fig.~\ref{fig:intuition}. Subsequently, these rough masks can be further refined to obtain accurate predictions, being the cost aggregation process. In light of this, we aim to effectively aggregate the cost volume and configure the process into spatial and class aggregation, regarding its multi-modal nature from being established between image and text. Furthermore, by observing the effectiveness of fine-tuning CLIP for its adaptation to semantic segmentation, we explore various methods to facilitate this process efficiently.

We analyze our cost aggregation framework to be advantageous in two aspects for adapting CLIP to dense prediction: \textit{i}) the robustness of cost aggregation against overfitting, and \textit{ii}) the direct construction of the cost volume from image and text embeddings of CLIP. For cost aggregation, the aggregation layers operate upon similarity scores, preventing them from overfitting to the features~\cite{cai2020matching,song2021adastereo,liu2022graftnet}. Moreover, as opposed to existing methods where they often employ decoder layers upon the image embeddings of CLIP~\cite{zhou2022extract, yu2023convolutions}, we do not introduce additional layers that can potentially project the embeddings to a different embedding space. 

Our framework, dubbed \ours, combines our cost aggregation-based framework consisting of spatial and class aggregation, with our optimal approach for fine-tuning the encoders of CLIP. We achieve state-of-the-art results on every standard open-vocabulary benchmark with large margins, gaining +3.6 mIoU in A-847 and +8.1 mIoU in PC-459 compared to the recent state-of-the-art. Not only \ours it is effective, but is also efficient both for training and inference compared to region-text methods, being over $\times$3.7 faster for inference. Furthermore, even in the extreme scenario~\citep{blumenstiel2023mess} where the domain of the image and text description differs significantly from the training dataset, our model outperforms existing state-of-the-art methods with a large margin, paving the way for various domain-specific applications. 

We summarize our contribution as follows:
\begin{itemize}
  \item We propose a cost aggregation-based framework for open-vocabulary semantic segmentation, effectively adapting CLIP to the downstream task of segmentation by fine-tuning its encoders. 
  \item To aggregate the image-text cost volume, we consist of our framework with spatial and class aggregation to reason the multi-modal cost volume and explore various methods to enhance our cost aggregation framework.
  \item Our framework, named \ours, establishes state-of-the-art performance for standard open-vocabulary benchmarks, as well as for extreme case scenarios~\cite{blumenstiel2023mess}, demonstrating versatility and practicality.
\end{itemize}

\begin{figure}[t]
  \centering
    \subfloat[CLIP]
{\includegraphics[width=0.33\linewidth]{figures/fig_cost/vanilla.pdf}}\hfill
    \subfloat[Fine-tuned CLIP]
{\includegraphics[width=0.33\linewidth]{figures/fig_cost/fig_finetune.pdf}}\hfill
     \subfloat[Aggregated Cost]
 {\includegraphics[width=0.33\linewidth]{figures/fig_cost/fig_agg.pdf}}\hfill\\
    
\vspace{-5pt}
\caption{\textbf{Visualization of the cost volume.} We visualize the raw cost volume obtained from frozen CLIP in (a) and fine-tuned CLIP in (b), and the aggregated cost in (c) through \ours. The top row correspond to the seen class ``chair" and the bottom row correspond to the unseen class ``sofa".} 
\label{fig:intuition}
\vspace{-10pt}
\end{figure}

