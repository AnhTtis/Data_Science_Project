

\subsection{Comparison to zero-shot open-vocab baseline} \label{sec:groupvit}
% \vspace{-1.em}
As a reference, we provide a comparison with GroupVit \cite{xu2022groupvit}, which solves a related but different segmentation problem. GroupVit \cite{xu2022groupvit} requires a text query from the test set to find each segment in the input image, and the number of segments is fixed to 8 per image by its architecture. Table \ref{tab:groupvit} shows that  GroupVit obtains a better IoU on PAS VOC with 20 classes; however, our method is significantly closing the gap on the more challenging PAS Context-59, especially with human-threshold IoU, despite using no text queries to help find semantic segments.
%, \todo{ espicailly with IoU scores with human thresholding (IoU$_\text{h}$). 
Figure \ref{fig:groupvit} shows that our method can discover more objects and provide more specific labeling, while GroupVit labels only a few objects and does not label every part of the input image.
%.can label objects more specifically.


\begin{table}[]
\centering
\caption{Comparison to GroupVit \cite{xu2022groupvit}, which solves a related but different segmentation problem and requires input text queries. *denotes scores computed on 1,000 random test images. IoU$_\text{c}$ and IoU$_\text{h}$ are IoU with constant thresholding or human verification.}
% \vspace{-0.5em} IoU
\label{tab:groupvit}
\resizebox{\columnwidth}{!}{
\setlength{\tabcolsep}{10pt}
\begin{tabular}{lcc@{\extracolsep{6pt}}cc}
\toprule
                            & \multicolumn{3}{c}{\textbf{PC-59}}                               & \multicolumn{1}{c}{\textbf{PAS-20}}                                  \\ \cline{2-4} \cline{5-5} \\[-1em]
 \multicolumn{1}{c}{Method} &  IoU$_\text{c}$  & IoU$_{\text{h } \geq 2}$  & IoU$_{\text{h } \geq 1}$ & $\text{IoU}_\text{c}$  \\ \midrule
GroupVit \cite{xu2022groupvit} & 22.4      & -     & -      & 52.3      \\
Ours                           & 19.6      & 20.9*      & 22.7*     & 20.1     \\ \bottomrule
\end{tabular}}
% \vspace{-0.5em}
\end{table}



\begin{figure}
\centering
% ablation_4 1.1
  \includegraphics[scale=0.43]{./figs/groupvit.pdf}
%   \includegraphics[scale=1.1]{./figs/ablation_4.pdf}
\vspace{-1.5em}
  \caption{\textbf{Qualitative comparison to GroupVit \cite{xu2022groupvit}.} 
  Despite lower IoU score, our method discovers more objects in input images, including objects not shown in the dataset (hay, mirror). Our method can offer more specific labels, such as `stool'.
  }
  % \vspace{-1.em}
  \label{fig:groupvit}
%   \vspace{-1.5em}
\end{figure}