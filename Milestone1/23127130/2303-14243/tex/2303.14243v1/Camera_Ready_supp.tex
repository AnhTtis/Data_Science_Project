% CVPR 2023 Paper Template
% based on the CVPR template provided by Ming-Ming Cheng (https://github.com/MCG-NKU/CVPR_Template)
% modified and extended by Stefan Roth (stefan.roth@NOSPAMtu-darmstadt.de)

\documentclass[10pt,twocolumn,letterpaper]{article}

%%%%%%%%% PAPER TYPE  - PLEASE UPDATE FOR FINAL VERSION
% \usepackage[review]{cvpr}      % To produce the REVIEW version
\usepackage{cvpr}              % To produce the CAMERA-READY version
%\usepackage[pagenumbers]{cvpr} % To force page numbers, e.g. for an arXiv version

% Include other packages here, before hyperref.
\usepackage{graphicx}
\usepackage{amsmath}
\usepackage{amssymb}
\usepackage{booktabs}
\usepackage{bm}
\usepackage{arydshln}
\usepackage{siunitx}
\sisetup{
    group-separator = {,},
    group-minimum-digits = 3
}
\usepackage[accsupp]{axessibility}

% It is strongly recommended to use hyperref, especially for the review version.
% hyperref with option pagebackref eases the reviewers' job.
% Please disable hyperref *only* if you encounter grave issues, e.g. with the
% file validation for the camera-ready version.
%
% If you comment hyperref and then uncomment it, you should delete
% ReviewTempalte.aux before re-running LaTeX.
% (Or just hit 'q' on the first LaTeX run, let it finish, and you
%  should be clear).
\usepackage{bibentry}
\makeatletter\let\saved@bibitem\@bibitem\makeatother
%\usepackage{hyperref}
\usepackage[pagebackref,breaklinks,colorlinks]{hyperref}
\makeatletter\let\@bibitem\saved@bibitem\makeatother

% Support for easy cross-referencing
\usepackage[capitalize]{cleveref}
\crefname{section}{Sec.}{Secs.}
\Crefname{section}{Section}{Sections}
\Crefname{table}{Table}{Tables}
\crefname{table}{Tab.}{Tabs.}


%%%%%%%%% PAPER ID  - PLEASE UPDATE
\def\cvprPaperID{8412} % *** Enter the CVPR Paper ID here
\def\confName{CVPR}
\def\confYear{2023}





%%START
\usepackage{xpatch} % also loads expl3
\makeatletter
\xpatchcmd{\@bibitem}
  {\item}
  {\item[\@biblabel{\changekey{#1}}]}
  {}{}
\xpatchcmd{\@bibitem}
  {\the\value{\@listctr}}
  {\changekey{#1}}
  {}{}
\makeatother

\ExplSyntaxOn
\cs_new:Npn \changekey #1
 {
  \str_case:nVF {#1} \g_changekey_list_tl { ?? }
 }
\cs_new_protected:Npn \setchangekey #1 #2
 {
  \tl_gput_right:Nn \g_changekey_list_tl { {#1}{#2} }
 }
\tl_new:N \g_changekey_list_tl
\cs_generate_variant:Nn \str_case:nnF { nV }
\ExplSyntaxOff

\setchangekey{mildenhall2021nerf}{23} %nerf
\setchangekey{SunSC22}{33} %directvoxgo
\setchangekey{fridovich2022plenoxels}{9} %plenoxels
\setchangekey{pumarola2021d}{29} %d-nerf
\setchangekey{tineuvox}{8} %tineuvox
\setchangekey{Lombardi:2019}{19} %nv
\setchangekey{li2020neural}{16} %nsff
\setchangekey{park2021nerfies}{27} %nerfies
\setchangekey{park2021hypernerf}{28} %hypernerf
\setchangekey{kania2022conerf}{13} %conerf

\usepackage[hide=true]{nobibprint}%default
%%END

\begin{document}

%%%%%%%%% TITLE - PLEASE UPDATE
\title{DyLiN: Making Light Field Networks Dynamic\\ Supplementary Material
}

\author{Heng Yu$^{1}$ \quad Joel Julin$^{1}$  \quad Zolt\'{a}n \'{A}. Milacski$^{1}$\quad Koichiro Niinuma$^{2}$ \quad L\'{a}szl\'{o} A. Jeni$^{1}$ \vspace{4pt}\\
	$^1$Robotics Institute, Carnegie Mellon University \quad
    $^2$Fujitsu Research of America \\
    {\tt\small \{hengyu, jjulin, zmilacsk\}@andrew.cmu.edu} \quad {\tt\small kniinuma@fujitsu.com} \quad {\tt\small laszlojeni@cmu.edu} \\
}
\maketitle

\section{Overview}
In this supplementary material, we provide detailed quantitative and additional qualitative results, showcasing the benefits of our proposed DyLiN and CoDyLiN methods. Furthermore, we also provide the training times one should expect given our current setup.
%:
%\begin{itemize}
    %\item the detailed per-scene quantitative results for the synthetic (\cref{tab:synth2}) and real (\cref{tab:real2}) dynamic scenes, extending Tab.~\textcolor{red}{1} and Tab.~\textcolor{red}{2} in the main paper, respectively;
    %\item more qualitative results for reconstruction quality on synthetic dynamic scenes (\cref{fig:qual_synth}), extending Fig.~\textcolor{red}{6} in the main paper;
    %\item qualitative results for ablation on the synthetic Standup scene (\cref{fig:qual-real-ablation}), complementing Fig.~\textcolor{red}{8} in the main paper;
    %\item qualitative results for controllable dynamic scenes (\cref{fig:qual-real-ablation}), complementing Tab.~\textcolor{red}{4} in the main paper.
%\end{itemize}

\section{Per-Scene Quantitative Results}
%\label{s:per_scene_quant}
For the sake of completeness, we provide the detailed per-scene quantitative results for reconstruction quality (PSNR, SSIM, MS-SSIM, LPIPS) on the synthetic (\cref{tab:synth2}) and real (\cref{tab:real2}) dynamic scenes, extending Tab.~\textcolor{red}{1} and Tab.~\textcolor{red}{2} in the main paper that average these numbers across the scenes.
Accordingly, we found that our DyLiN performs the best with respect to the SSIM and LPIPS metrics, generating perceptually better images, yet it sometimes falls behind in terms of PSNR and MS-SSIM that may prefer blurred results.
Knowledge distillation improves a lot, our deformation and hyperspace MLPs yield slightly better results, while fine-tuning on the original training data gives a considerable boost.



\setcounter{table}{4}
\setcounter{figure}{8}

\begin{table*}[ht]
\centering
\caption{Per-scene quantitative results on synthetic dynamic scenes. Notations: Multi-Layer Perceptron (MLP), PD (pointwise deformation), FT (fine-tuning). We utilized D-NeRF as the teacher model for our DyLiNs. The winning numbers are highlighted in bold.}
\resizebox{\textwidth}{!}{%
\begin{tabular}{lcccccccccccc}
\toprule
                                 & \multicolumn{3}{c}{Hell Warrior}                              & \multicolumn{3}{c}{Mutant}                                     & \multicolumn{3}{c}{Hook}                                          & \multicolumn{3}{c}{Bouncing Balls}                                     \\
\cmidrule(lr){2-4} \cmidrule(lr){5-7} \cmidrule(lr){8-10} \cmidrule(lr){11-13}
Method                           & PSNR$\uparrow$           & SSIM$\uparrow$ & LPIPS$\downarrow$ & PSNR$\uparrow$            & SSIM$\uparrow$ & LPIPS$\downarrow$ & PSNR$\uparrow$               & SSIM$\uparrow$ & LPIPS$\downarrow$ & PSNR$\uparrow$                    & SSIM$\uparrow$ & LPIPS$\downarrow$ \\ \midrule
NeRF\cite{mildenhall2021nerf}    & 13.52                    & 0.8100           & 0.2500              & 20.31                     & 0.9100           & 0.0900              & 16.65                        & 0.8400           & 0.1900              & 20.26                             & 0.9100           & 0.2000              \\
DirectVoxGo\cite{SunSC22}        & 13.51                    & 0.7500           & 0.2500              & 19.45                     & 0.8900           & 0.1200              & 16.16                        & 0.8000           & 0.2100              & 20.20                             & 0.8700           & 0.2200              \\
Plenoxels\cite{fridovich2022plenoxels}& 15.19               & 0.7800           & 0.2700              & 21.44                     & 0.9100           & 0.0900              & 17.90                        & 0.8100           & 0.2100              & 21.30                             & 0.8900           & 0.1800              \\
T-NeRF\cite{pumarola2021d}       & 23.19                    & 0.9300           & 0.0800              & 30.56                     & 0.9600           & 0.0400              & 27.21                        & 0.9400           & 0.0600              & 37.81                             & 0.9800           & 0.1200              \\
D-NeRF\cite{pumarola2021d}       & 25.10                    & 0.9500           & 0.0600              & 31.29                     & 0.9700           & 0.0200              & 29.25                        & 0.9600           & 0.1100              & 38.93                             & 0.9800           & 0.1000              \\
TiNeuVox-S\cite{tineuvox}        & 27.00                    & 0.9500           & 0.0900              & 31.09                     & 0.9600           & 0.0500              & 29.30                        & 0.9500           & 0.0700              & 39.05                             & 0.9900           & 0.0600              \\
TiNeuVox-B\cite{tineuvox}        & \textbf{28.17}                   & 0.9700           & 0.0700              & 33.61                     & 0.9800           & 0.0300              & \textbf{31.45}                        & 0.9700           & 0.0500              & 40.73                             & 0.9900           & 0.0400              \\ \midrule
DyLiN, w/o two MLPs, w/o FT (ours)           &           26.81         &       0.9885         &       0.0363              &                     32.13      &         0.9961       &        0.0186          &                         
         29.89  &             0.9922      &          0.0297                         &    
        39.78 &       0.9997 &        0.0099                   \\
DyLiN, w/o two MLPs (ours)     &         27.73         &       0.9893         &       0.0317           &                     33.26  &        0.9971        &        0.0101           &              
        30.20        &    0.9928            &     0.0187              &            
        41.13 &       \textbf{0.9998} &      0.0064                 \\
DyLiN, PD MLP only, w/o FT (ours)           &          26.82      &           0.9886      &           0.0362        &                    32.13       &     0.9963           &      0.0185             &      
        29.94     &         0.9923       &     0.0296       &                 
        39.70 &           0.9996 &          0.0096              \\
DyLiN, PD MLP only (ours)     &            27.75      &      0.9896      &      0.0302           &                     33.47      &      0.9972          &       0.0102            &      
        30.39     &        0.9930         &         \textbf{0.0186}          &       
        41.52      &         \textbf{0.9998}      &         \textbf{0.0062}                \\
DyLiN, w/o FT (ours)      &             26.90        &       0.9887        &       0.0360            &                      32.17     &        0.9963        &        0.0182           &         
        29.99     &      0.9923          &         0.0289          &         
        40.02      &         0.9997      &         0.0098               \\
DyLiN (ours)  &               27.79      &        \textbf{0.9898}     &        \textbf{0.0298}        &                      \textbf{33.80}    &        \textbf{0.9974}        &         \textbf{0.0086}          &                       30.49       &      \textbf{0.9931}          &       \textbf{0.0186}            &    
        \textbf{41.59}       &     \textbf{0.9998}        &     \textbf{0.0062}                \\\midrule
                                 & \multicolumn{3}{c}{Lego} &                                    \multicolumn{3}{c}{T-Rex} &                                    \multicolumn{3}{c}{Stand Up} &                                   \multicolumn{3}{c}{Jumping Jacks}                                    \\
\cmidrule(lr){2-4} \cmidrule(lr){5-7} \cmidrule(lr){8-10} \cmidrule(lr){11-13}
Method                           & PSNR$\uparrow$           & SSIM$\uparrow$ & LPIPS$\downarrow$ & PSNR$\uparrow$            & SSIM$\uparrow$ & LPIPS$\downarrow$ & PSNR$\uparrow$               & SSIM$\uparrow$ & LPIPS$\downarrow$ & PSNR$\uparrow$                    & SSIM$\uparrow$ & LPIPS$\downarrow$ \\ \midrule
NeRF\cite{mildenhall2021nerf}    & 20.30                    & 0.7900           & 0.2300              & 24.29                     & 0.9300           & 0.1300              & 18.19                        & 0.8900           & 0.1400              & 18.28                             & 0.8800           & 0.2300              \\
DirectVoxGo\cite{SunSC22}        & 21.13                    & 0.9000           & 0.1000              & 23.27                     & 0.9200           & 0.0900              & 17.58                        & 0.8600           & 0.1600              & 17.80                             & 0.8400           & 0.2000              \\
Plenoxels\cite{fridovich2022plenoxels}& 21.97                    & 0.9000           & 0.1100              & 25.18                     & 0.9300           & 0.0800              & 18.76                        & 0.8700           & 0.1500              & 20.18                             & 0.8600           & 0.1900              \\
T-NeRF\cite{pumarola2021d}       & 23.82                    & 0.9000           & 0.1500              & 30.19                     & 0.9600           & 0.1300              & 31.24                        & 0.9700           & 0.0200              & 32.01                             & 0.9700          & 0.0300              \\
D-NeRF\cite{pumarola2021d}       & 21.64                    & 0.8300           & 0.1600              & 31.75                     & 0.9700           & 0.0300              & 32.79                        & 0.9800           & 0.0200              & 32.80                             & 0.9800           & 0.0300              \\
TiNeuVox-S\cite{tineuvox}        & 24.35                    & 0.8800           & 0.1300              & 29.95                     & 0.9600           & 0.0600              & 32.89                        & 0.9800           & 0.0300              & 32.33                             & 0.9700           & 0.0400              \\
TiNeuVox-B\cite{tineuvox}        & \textbf{25.02}                   & 0.9200           & 0.0700              & 32.70                     & 0.9800           & 0.0300              & 35.43                        & 0.9900           & 0.0200              & \textbf{34.23}                             & 0.9800           & 0.0300              \\ \midrule
DyLiN, w/o two MLPs, w/o FT (ours)         &    22.11                &        0.9747        &   0.0612                &     
31.35    &      0.9978   &       0.0290              &            
33.98     &        0.9973     &        0.0140            &                   33.24 & 0.9981 & 0.0260              \\
DyLiN, w/o two MLPs (ours)    &    22.42                 &       0.9761        &   0.0493           &        
32.80      &      0.9984      &      0.0170               &               35.31     &      0.9980     &      0.0084            &                 33.67   &  0.9984   &  0.0155                   \\
DyLiN, PD MLP only, w/o FT (ours)          &    22.13               &       0.9748         &      0.0618          &          
32.18     &      0.9982     &      0.0282          &                    33.97    &      0.9973    &      0.0140            &                 33.19 &  0.9982 &  0.0257                \\
DyLiN, PD MLP only (ours)    &    22.76                &       0.9775     &        0.0452       &        
32.77     &          \textbf{0.9985}     &          0.0176              &          35.56      &  0.9981      &    0.0082               &                    33.68 &  0.9984 &  0.0152                 \\
DyLiN, w/o FT (ours)      &     22.24              &       0.9754        &        0.0600      &           
32.24     &      0.9982     &      0.0276           &                    34.15    &     0.9974    &     0.0141             &                              33.23  &   0.9983  &   0.0256              \\
DyLiN (ours)  &       23.10         &     \textbf{0.9791}   & \textbf{0.0443}  &          
\textbf{32.91}  &  \textbf{0.9985}   &  \textbf{0.0168}    &  \textbf{35.95}   &    \textbf{0.9983}   &    \textbf{0.0074}         &                                  33.84 &  \textbf{0.9985} &  \textbf{0.0151}   \\    
\bottomrule
\end{tabular}%
}
%extended table 1
\label{tab:synth2}
\end{table*}



%%%%%%%%%%%%%%%%%%%%%%%%%%%%%%%%%%%%%%%%%%%%%%%%%%%%%%%%%%%%

\begin{table*}[ht]
\centering
\caption{Per-scene quantitative results on real dynamic scenes. Notations: Multi-Layer Perceptron (MLP), PD (pointwise deformation), FT (fine-tuning), N/A (not available in the cited research paper). We utilized HyperNeRF as the teacher model for our DyLiNs. The winning numbers are highlighted in bold.}
\resizebox{\textwidth}{!}{%
\begin{tabular}{lcccccc}
\toprule
                                                 & \multicolumn{2}{c}{Broom}          & \multicolumn{2}{c}{3D Printer}     & \multicolumn{2}{c}{Chicken}        \\
\cmidrule(lr){2-3} \cmidrule(lr){4-5} \cmidrule(lr){6-7}                    Method                                           & PSNR$\uparrow$ & MS-SSIM$\uparrow$ & PSNR$\uparrow$ & MS-SSIM$\uparrow$ & PSNR$\uparrow$ & MS-SSIM$\uparrow$ \\ \midrule
NeRF\cite{mildenhall2021nerf}   & 19.90           & 0.653             & 20.70           & 0.780             & 19.90           & 0.777             \\
NV    \cite{Lombardi:2019}                                             & 17.70           & 0.623             & 16.20           & 0.665             & 17.60           & 0.615             \\
NSFF     \cite{li2020neural}                                        & \textbf{26.10}          & \textbf{0.871}             & \textbf{27.70}           &\textbf{0.947}             & 26.90           & 0.944             \\
Nerfies     \cite{park2021nerfies}                                        & 19.20           & 0.567             & 20.60           & 0.830             & 26.70           & 0.943             \\
HyperNeRF   \cite{park2021hypernerf}                                          & 19.30           & 0.591             & 20.00           & 0.821             & 26.90           & 0.948             \\
TiNeuVox-S\cite{tineuvox}       & 21.90           & 0.707             & 22.70           & 0.836             & 27.00           & 0.929             \\
TiNeuVox-B\cite{tineuvox}       & 21.50           & 0.686             & 22.80           & 0.841             & \textbf{28.30}           & 0.947             \\ \midrule
DyLiN, w/o two MLPs, w/o FT (ours)                            &           21.98 & 0.808            &          22.99 & 0.899            &  26.89    &    0.948               \\
DyLiN, w/o two MLPs (ours)                    &        22.04  &  0.811         &            23.16 & 0.905           &  27.35    &   0.954            \\
DyLiN, PD MLP only, w/o FT (ours)                          &       22.02 & 0.805       &                23.04 & 0.903          &  26.88    & 0.948                  \\
DyLiN, PD MLP only (ours)                      &      22.14 & 0.815           &              23.19 & 0.906             &  27.53     & 0.955              \\
DyLiN, w/o FT (ours)                       &     22.04 & 0.809            &               23.06 & 0.902         &  26.91   &   0.948              \\
DyLiN (ours)                &      22.14 & 0.823            &              23.21 & 0.906          &  27.62   &  \textbf{0.956}       \\ \midrule
                                                 & \multicolumn{2}{c}{Peel Banana}    & \multicolumn{2}{c}{Americano}      & \multicolumn{2}{c}{Expressions}    \\
\cmidrule(lr){2-3} \cmidrule(lr){4-5} \cmidrule(lr){6-7}
Method                                           & PSNR$\uparrow$ & MS-SSIM$\uparrow$ & PSNR$\uparrow$ & MS-SSIM$\uparrow$ & PSNR$\uparrow$ & MS-SSIM$\uparrow$ \\ \midrule
NeRF\cite{mildenhall2021nerf}   & 20.00           & 0.769             &       N/A         &       N/A          &          N/A    &         N/A        \\
NV \cite{Lombardi:2019}                                                  & 15.90           & 0.380             &      N/A        &        N/A         &          N/A    &         N/A        \\
NSFF  \cite{li2020neural}                                           & 24.60           & 0.902             &      N/A        &       N/A          &          N/A    &         N/A        \\
Nerfies   \cite{park2021nerfies}                                         & 22.40           & 0.872             &      N/A        &       N/A          &          N/A    &         N/A        \\
HyperNeRF   \cite{park2021hypernerf}                                     & 23.30           & 0.896             &     18.42           &        0.720           &         25.40       &   0.958                \\
TiNeuVox-S\cite{tineuvox}       & 22.10           & 0.780             &      N/A        &       N/A          &           N/A   &         N/A        \\
TiNeuVox-B\cite{tineuvox}       & 24.40           & 0.873             &      N/A        &       N/A          &           N/A   &         N/A        \\ \midrule
DyLiN, w/o two MLPs, w/o FT (ours)    &       23.38 & 0.872        &   18.45     &  0.722                  &       25.36 & 0.950    \\
DyLiN, w/o two MLPs (ours)                      &         24.35 & 0.906        &          30.85    &           0.977     &           26.33 & 0.967            \\
DyLiN, PD MLP only, w/o FT (ours)                            &          23.70 & 0.882       &         18.47 &  0.722            &            25.55 & 0.960          \\
DyLiN, PD MLP only (ours)                     &       25.72 & 0.936           &        31.01 & 0.978        &           26.33 & 0.967           \\
DyLiN, w/o FT (ours)                       &        23.97 & 0.886            &         18.48 & 0.722            &           26.51 & 0.969         \\
DyLiN (ours)                &         \textbf{27.36} & \textbf{0.952}           &          \textbf{31.56} & \textbf{0.982}         &        \textbf{26.91} & \textbf{0.974}  \\
\bottomrule
\end{tabular}%
}
%extended table 2
\label{tab:real2}
\end{table*}


\section{More Qualitative Results}
We provide additional qualitative results for 3 experiments.

First, \cref{fig:qual_synth} depicts more qualitative results for reconstruction quality on synthetic dynamic scenes, extending Fig.~\textcolor{red}{6} in the main paper.
Specifically, the Standup scene includes buttons on the shirt of the avatar (\cref{fig:hook_gt}), and the baselines are all missing them (\cref{fig:hook_dnerf,fig:hook_tineuvox}), whereas our full method is capable of reconstructing such details (\cref{fig:hook_ours2}).
Furthermore, the Bouncing Ball scene involves shadow casting (\cref{fig:jumping_gt}).
Inside the shadowed area, D-NeRF \cite{pumarola2021d} produces horizontal artifacts (\cref{fig:jumping_dnerf}), while TiNeuVox \cite{tineuvox} predicts an inaccurate boundary (\cref{fig:jumping_tineuvox}).
Again, our full model outputs the correct shadow (\cref{fig:jumping_ours2}).

Second, \cref{fig:qual-real-ablation} shows qualitative results for ablation on the synthetic Standup scene using a D-NeRF teacher model, complementing Fig.~\textcolor{red}{8} in the main paper that is restricted to real scenes and distilling from HyperNeRF \cite{park2021hypernerf}.
D-NeRF gives an oversmoothed prediction (\cref{fig:exp2-hyper}), whereas the two MLPs of our DyLiN gradually reduce the blurriness of the face (\cref{fig:exp2-ours1,fig:exp2-ours2,fig:exp2-ours3}).

Lastly, \cref{fig:cor2l_res} illustrates qualitative results for the real controllable Transformer scene, complementing the numbers of Tab.~\textcolor{red}{4} in the main paper.
We portray the effects of altering the attribute input $\alpha_i\in[-1,1]$, which encodes the body pose of the character.
%We compare our CoDyLiN with its CoNeRF \cite{kania2022conerf} teacher model.
We found that the CoNeRF \cite{kania2022conerf} teacher model produces yellow color artifacts outside the boundary of the character (see, e.g., top row 1\textsuperscript{st} inset), whereas our CoDyLiN student model captures the boundary well.
\section{Training Times}
On a single NVIDIA A100 GPU, the full process takes $\approx 38\text{--}\SI{43}{\hour}$, including $5\text{--}\SI{10}{\hour}$ to train the teacher, $\SI{13}{\hour}$ for drawing $S=\SI{10000}{}$ training samples for KD, and $\SI{20}{\hour}$ for training the student via KD.

\begin{figure*}[!ht]
     \centering
     \begin{subfigure}[b]{0.15\textwidth}
         \centering
         \includegraphics[width=0.99\textwidth]{imgs/supp/standup/ori.png}
         \caption*{\textbf{Standup}}
         \label{fig:hook_box}
     \end{subfigure}     
     \setcounter{subfigure}{0}
     \begin{subfigure}[b]{0.15\textwidth}
         \centering
         \includegraphics[width=0.99\textwidth]{imgs/supp/standup/016_gt.png}
         \caption{Ground Truth}
         \label{fig:hook_gt}
     \end{subfigure}
    %  \hfill
     \begin{subfigure}[b]{0.15\textwidth}
         \centering
         \includegraphics[width=0.99\textwidth]{imgs/supp/standup/016_nerf.png}
         \caption{D-NeRF \cite{pumarola2021d}}
         \label{fig:hook_dnerf}
     \end{subfigure}
    \begin{subfigure}[b]{0.15\textwidth}
         \centering
         \includegraphics[width=0.99\textwidth]{imgs/supp/standup/016_tineuvox_resized.png}
         \caption{TiNeuVox \cite{tineuvox}}
         \label{fig:hook_tineuvox}
     \end{subfigure}
    \begin{subfigure}[b]{0.15\textwidth}
         \centering
         \includegraphics[width=0.99\textwidth]{imgs/supp/standup/016_step1.png}
         \caption{Ours-1}
         \label{fig:hook_ours1}
     \end{subfigure}
    \begin{subfigure}[b]{0.15\textwidth}
         \centering
         \includegraphics[width=0.99\textwidth]{imgs/supp/standup/016_step2.png}
         \caption{Ours-2}
         \label{fig:hook_ours2}
     \end{subfigure}    
     
     \begin{subfigure}[b]{0.15\textwidth}
         \centering
         \includegraphics[width=0.99\textwidth]{imgs/supp/bouncingball/ori.png}
         \caption*{\textbf{Bouncing Ball}}
         \label{fig:jumping_box}
     \end{subfigure}     
     %\setcounter{subfigure}{0}
     \begin{subfigure}[b]{0.15\textwidth}
         \centering
         \includegraphics[width=0.99\textwidth]{imgs/supp/bouncingball/015_gt.png}
         \caption{Ground Truth}
         \label{fig:jumping_gt}
     \end{subfigure}
    %  \hfill
     \begin{subfigure}[b]{0.15\textwidth}
         \centering
         \includegraphics[width=0.99\textwidth]{imgs/supp/bouncingball/015_dnerf.png}
         \caption{D-NeRF \cite{pumarola2021d}}
         \label{fig:jumping_dnerf}
     \end{subfigure}
    \begin{subfigure}[b]{0.15\textwidth}
         \centering
         \includegraphics[width=0.99\textwidth]{imgs/supp/bouncingball/015_tineuvox_resized.png}
         \caption{TiNeuVox \cite{tineuvox}}
         \label{fig:jumping_tineuvox}
     \end{subfigure}
    \begin{subfigure}[b]{0.15\textwidth}
         \centering
         \includegraphics[width=0.99\textwidth]{imgs/supp/bouncingball/015_ours1.png}
         \caption{Ours-1}
         \label{fig:jumping_ours1}
     \end{subfigure}
    \begin{subfigure}[b]{0.15\textwidth}
         \centering
         \includegraphics[width=0.99\textwidth]{imgs/supp/bouncingball/015_ours2.png}
         \caption{Ours-2}
         \label{fig:jumping_ours2}
     \end{subfigure}    


        \caption{More qualitative results on synthetic dynamic scenes. We compare our DyLiN (Ours-1, Ours-2) with the ground truth, the D-NeRF teacher model and TiNeuVox.
        %We utilized D-NeRF as the teacher model for our DyLiNs.
        Ours-1 and Ours-2 were trained without and with fine-tuning on the original data, respectively.}
        \label{fig:qual_synth}
        %extended fig6
%\end{figure*}
\vspace{\intextsep}
%\begin{figure*}[!htb]
%     \centering
     \begin{subfigure}[b]{0.15\textwidth}
         \centering
         \includegraphics[width=0.99\textwidth,height=2.492cm,keepaspectratio]{imgs/supp/standup123/ori.png}
         \caption*{\textbf{Standup}}
         \label{fig:exp2}
     \end{subfigure}     
     \setcounter{subfigure}{0}
     \begin{subfigure}[b]{0.15\textwidth}
         \centering
         \includegraphics[width=0.99\textwidth]{imgs/supp/standup123/gt.png}
         \caption{Ground Truth}
         \label{fig:exp2-gt}
     \end{subfigure}
    %  \hfill
     \begin{subfigure}[b]{0.15\textwidth}
         \centering
         \includegraphics[width=0.99\textwidth]{imgs/supp/standup123/dnerf.png}
         \caption{D-NeRF \cite{pumarola2021d}}
         \label{fig:exp2-hyper}
     \end{subfigure}
    \begin{subfigure}[b]{0.15\textwidth}
         \centering
         \includegraphics[width=0.99\textwidth]{imgs/supp/standup123/1.png}
         \caption{Ours-1}
         \label{fig:exp2-ours1}
     \end{subfigure}
    \begin{subfigure}[b]{0.15\textwidth}
         \centering
         \includegraphics[width=0.99\textwidth]{imgs/supp/standup123/2.png}
         \caption{Ours-2}
         \label{fig:exp2-ours2}
     \end{subfigure}
     \begin{subfigure}[b]{0.15\textwidth}
         \centering
         \includegraphics[width=0.99\textwidth]{imgs/supp/standup123/3.png}
         \caption{Ours-3}
         \label{fig:exp2-ours3}
     \end{subfigure}     
%     \begin{subfigure}[b]{0.15\textwidth}
%         \centering
%         \includegraphics[width=0.99\textwidth,height=2.152cm,keepaspectratio]{imgs/expressions/expression2_box.png}
%         \caption*{\textbf{Expression}}
%         \label{fig:exp3}
%     \end{subfigure}     
%     \setcounter{subfigure}{0}
%     \begin{subfigure}[b]{0.15\textwidth}
%         \centering
%         \includegraphics[width=0.99\textwidth]{imgs/expressions/exp2_gt.png}
%         \caption{Ground Truth}
%         \label{fig:exp3-gt}
%     \end{subfigure}
    %  \hfill
%     \begin{subfigure}[b]{0.15\textwidth}
%         \centering
%         \includegraphics[width=0.99\textwidth]{imgs/expressions/exp2_hyper.png}
%         \caption{HyperNeRF \cite{park2021hypernerf}}
%         \label{fig:exp3-hyper}
%     \end{subfigure}
%    \begin{subfigure}[b]{0.15\textwidth}
%         \centering
%         \includegraphics[width=0.99\textwidth]{imgs/expressions/exp2_ours1.png}
%         \caption{Ours-1}
%         \label{fig:exp3-ours1}
%     \end{subfigure}
%    \begin{subfigure}[b]{0.15\textwidth}
%         \centering
%         \includegraphics[width=0.99\textwidth]{imgs/expressions/exp2_ours2.png}
%         \caption{Ours-2}
%         \label{fig:exp3-ours2}
%     \end{subfigure}
%     \begin{subfigure}[b]{0.15\textwidth}
%         \centering
%         \includegraphics[width=0.99\textwidth]{imgs/expressions/exp2_ours3.png}
%         \caption{Ours-3}
%         \label{fig:exp3-ours3}
%     \end{subfigure}
%     \begin{subfigure}[b]{0.15\textwidth}
%         \centering
%         \includegraphics[width=0.99\textwidth,height=2.48cm,keepaspectratio]{imgs/supp/americano123/ori.png}
%         \caption*{\textbf{Americano}}
%         \label{fig:banana}
%     \end{subfigure}     
     %\setcounter{subfigure}{0}
%     \begin{subfigure}[b]{0.15\textwidth}
%         \centering
%         \includegraphics[width=0.99\textwidth]{imgs/supp/americano123/gt.png}
%         \caption{Ground Truth}
%         \label{fig:banana-gt}
%     \end{subfigure}
    %  \hfill
%     \begin{subfigure}[b]{0.15\textwidth}
%         \centering
%         \includegraphics[width=0.99\textwidth]{imgs/supp/americano123/hyper.png}
%         \caption{HyperNeRF \cite{park2021hypernerf}}
%         \label{fig:banana-hyper}
%     \end{subfigure}
%    \begin{subfigure}[b]{0.15\textwidth}
%         \centering
%         \includegraphics[width=0.99\textwidth]{imgs/supp/americano123/1.png}
%         \caption{Ours-1}
%         \label{fig:banana-ours1}
%     \end{subfigure}
%    \begin{subfigure}[b]{0.15\textwidth}
%         \centering
%         \includegraphics[width=0.99\textwidth]{imgs/supp/americano123/2.png}
%         \caption{Ours-2}
%         \label{fig:banana-ours2}
%     \end{subfigure}
%    \begin{subfigure}[b]{0.15\textwidth}
%         \centering
%         \includegraphics[width=0.99\textwidth]{imgs/supp/americano123/3.png}
%         \caption{Ours-3}
%         \label{fig:banana-ours3}
%     \end{subfigure}    


        \caption{Qualitative results for ablation on the synthetic Standup scene. We compare our DyLiN (Ours-1, Ours-2, Ours-3) with the ground truth and the D-NeRF teacher model.
        %We utilized HyperNeRF as the teacher model for our DyLiNs.
        Ours-1 was trained without our two MLPs.
        Ours-2 was trained with pointwise deformation MLP only.
        Ours-3 is our full model with both of our proposed two MLPs.}
        %Qualitative comparison between our DyLiN network (Ours-1, Ours-2), D-NeRF, and TiNeuVox on synthetic scenes. Ours-1 is trained solely on pseudo data and Ours-2 is trained on both pseudo+real data.}
        \label{fig:qual-real-ablation}
        %extended fig8
\end{figure*}


\begin{figure*}[!htb]
\centering
 \includegraphics[width=0.99\linewidth]{imgs/conerf_transformer_zoom.pdf}
  \captionof{figure}{Qualitative results on the real controllable Transformer scene.
  We utilized CoNeRF \cite{kania2022conerf} as the teacher model for our CoDyLiN. 
  Red circles indicate regions enlarged in insets.
  Best viewed zoomed in.}
 %\caption{Qualitative results on real controllable scenes.}
 \label{fig:cor2l_res}
\end{figure*}







%%%%%%%%% REFERENCES
\bibliographystyle{ieee_fullname}
\bibliography{egbib}

\end{document}






