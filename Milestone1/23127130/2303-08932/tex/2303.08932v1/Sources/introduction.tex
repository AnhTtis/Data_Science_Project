\section{Introduction}


Data
spaces are a cutting-edge solution to the difficulties encountered in data exchange and integration. They act as a federated platform for sharing and exchanging data among various entities, providing the necessary tools and security measures to ensure data can be safely shared and consumed. The ultimate aim of data spaces is to enhance the accessibility and usage of data by a wide range of stakeholders, fostering innovation and research in the digital economy. These innovative infrastructures empower all users by making data integration simpler, more adaptable, and more efficient. conventional data exchange systems, data spaces have the advantage of easily onboarding new members.  

%Unlike conventional data exchange systems, dataspaces offer an incremental approach to data integration, allowing for a gradual implementation of advanced features as the need arises, starting with basic functionality.


The theoretical importance of data spaces lies in their ability to seamlessly integrate data sources that have diverse schemas and structures, without the requirement of a shared schema or extensive upfront effort to standardize the data. This is made possible through the extensive use of semantic technologies~\cite{alexiev2022data}, which help align the data and make it usable in a consistent manner. As a result, using data spaces reduces the time and effort needed to establish a data integration system, while offering increased flexibility and scalability. However, achieving satisfactory levels of semantic interoperability in data spaces is a complex and ongoing process. It requires ongoing development of standards, tools, and techniques for data integration and processing. %Moreover, the need for privacy and security in data spaces can pose additional challenges to achieving semantic interoperability, as entities may be reluctant to share their data if they are concerned about data privacy and security.


International Data Spaces\footnote{\url{https://internationaldataspaces.org/}} and Gaia-X\footnote{\url{https://gaia-x.eu/}} are two prominent initiatives in Europe and globally that are designed to bring together various stakeholders to create a secure, interoperable, and decentralized data infrastructure. The ultimate objective of IDS and Gaia-X is to establish a trustworthy, secure, and efficient data infrastructure that supports data-driven innovation while preserving privacy and security. These initiatives focus on providing infrastructure services that increase trust among entities, advance data sovereignty, and protect data privacy. As a result, they are expected to simplify the process of accessing and utilizing data securely and transparently for businesses and organizations.

As the number of organizations and individuals participating in various data spaces continues to increase, the demand for effective data management and exchange solutions will also rise. The key to successful data exchange lies in the mutual understanding of the data being shared, which is referred to as \emph{semantic interoperability}. Thus, finding ways to enhance semantic interoperability is of utmost importance as the number of members joining different data spaces continues to grow.

Although machine learning has shown promise in addressing the above-mentioned interoperability aspects of data exchange in general, its application to Data Spaces has not been fully explored. This paper aims to shed light on the potential of using machine learning to improve semantic interoperability in Data Spaces, making it the go-to solution for data exchange.

In this paper, we present our perspective on how machine learning solutions can be utilized to improve the semantic interoperability of a data space, using IDS as a concrete example. Although there are numerous aspects of semantic interoperability, we concentrate on six key challenges that are prevalent in data exchange. It is crucial to note that this paper does not make any definitive claims, but instead offers a comprehensive overview and framework for integrating machine learning solutions directly into data spaces.



