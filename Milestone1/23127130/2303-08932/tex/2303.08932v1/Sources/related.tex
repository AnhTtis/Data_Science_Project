\section{Related Work}

%Data Spaces as a research project has started in 2015. Therefore, there the focus was on the legal, technical and metadata interoperability aspects and only a few studies have addressed the data semantic aspect~\cite{alexiev2022data}. 

The concept of Data Spaces has been in existence for several decades, but in recent times, it has gained significant attention, and considerable effort is being devoted to facilitating data exchange in today's data-driven ecosystem, such as International Data Spaces (IDS)\footnote{\url{https://www.fraunhofer.de/en/research/lighthouse-projects-fraunhofer-initiatives/international-data-spaces.html}} and the Common European Data Spaces~\cite{scerri2022common}. 
So far, the primary focus in practical data spaces has been on legal, technical, and metadata interoperability, with little attention given to the semantic aspect of data, as only a few studies have been conducted in this area~\cite{alexiev2022data}. This means that in terms of semantic interoperability, the current focus is on metadata only with the assumption that it exists. However, data semantic interoperability has been studied for decades. For example, Ouksel et al.~\cite{ouksel1999semantic} discussed the issue of finding accurate information in a complex, heterogeneous information system like the Internet and Web. They proposed a framework for interoperability that involves relating information to real-world entities and acknowledges the changing nature of semantics. More than one decade later, Kiljander et al.~\cite{kiljander2014semantic} discussed the need for common approaches to enable high-level interoperability between heterogeneous IoT devices to realize pervasive computing and IoT visions. It divides the interoperability challenge into two levels: connectivity and semantics. The connectivity level covers traditional Open System Interconnection (OSI) model layers from the physical to the transport layer. The semantic level covers technologies needed for enabling meaning-sharing between communicating parties. The authors stated that semantic level interoperability has been identified as a main goal in the Semantic Web and that semantic web technology can be used to represent knowledge about the physical world in IoT-related projects.

%EU Data Spaces

Semantic interoperability in data exchange has been also addressed in specific domains. Lin et al.~\cite{lin2011investigating} evaluated the usage of Logical Observation Identifiers Names and Codes (LOINC) and its impact on the interoperability of laboratory data from different institutions that use LOINC codes. Heterogeneous data formats have been discovered among different institutions for the same laboratory tests using LOINC codes. After investigating the common problems that arise when aggregating such data, they suggest that more guidance on best practices in coding laboratory results is needed to achieve greater interoperability.

