%\section{Discussion}
\subsection*{Discussion}

%As the number of organizations and individuals participating in various Data Spaces continues to increase, the demand for effective data management and exchange solutions will also rise. The key to successful data exchange lies in the mutual understanding of the data being shared, which is referred to as semantic interoperability. Thus, finding ways to enhance semantic interoperability is of utmost importance as the number of members joining different Data Spaces continues to grow.

%Although machine learning has shown promise in addressing the above-mentioned interoperability aspects of data exchange in general, its application to Data Spaces has not been fully explored. This paper aims to shed light on the potential of using machine learning to improve semantic interoperability in Data Spaces, making it the go-to solution for data exchange.


The enhancement of semantic interoperability of data spaces is a complex task that involves different facets and approaches. In this paper, we have focused on specific aspects that can be improved through the use of machine learning in the context of International Data Spaces. To achieve this, we propose the development of machine learning-powered software that can be easily integrated into the Data Spaces connectors as smart data apps. This will make the software more user-friendly and accessible, allowing for seamless integration into the existing system.

In addition, with the growing popularity of Gaia-X in Europe and beyond, this software can also be provided as a service within the Gaia-X framework, offering members a valuable resource for improving semantic interoperability. By integrating machine learning into the data spaces, organizations can ensure that their data is properly structured, and their systems can effectively communicate and exchange information with other systems, resulting in more efficient and effective data management and exchange.
