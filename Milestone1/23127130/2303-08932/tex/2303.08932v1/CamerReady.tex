%%
%% This is file `sample-sigconf.tex',
%% generated with the docstrip utility.
%%
%% The original source files were:
%%
%% samples.dtx  (with options: `sigconf')
%% 
%% IMPORTANT NOTICE:
%% 
%% For the copyright see the source file.
%% 
%% Any modified versions of this file must be renamed
%% with new filenames distinct from sample-sigconf.tex.
%% 
%% For distribution of the original source see the terms
%% for copying and modification in the file samples.dtx.
%% 
%% This generated file may be distributed as long as the
%% original source files, as listed above, are part of the
%% same distribution. (The sources need not necessarily be
%% in the same archive or directory.)
%%
%% Commands for TeXCount
%TC:macro \cite [option:text,text]
%TC:macro \citep [option:text,text]
%TC:macro \citet [option:text,text]
%TC:envir table 0 1
%TC:envir table* 0 1
%TC:envir tabular [ignore] word
%TC:envir displaymath 0 word
%TC:envir math 0 word
%TC:envir comment 0 0
%%
%%
%% The first command in your LaTeX source must be the \documentclass command.
\documentclass[sigconf]{acmart}

\usepackage{enumitem}
%% NOTE that a single column version may be required for 
%% submission and peer review. This can be done by changing
%% the \doucmentclass[...]{acmart} in this template to 
%% \documentclass[manuscript,screen]{acmart}
%% 
%% To ensure 100% compatibility, please check the white list of
%% approved LaTeX packages to be used with the Master Article Template at
%% https://www.acm.org/publications/taps/whitelist-of-latex-packages 
%% before creating your document. The white list page provides 
%% information on how to submit additional LaTeX packages for 
%% review and adoption.
%% Fonts used in the template cannot be substituted; margin 
%% adjustments are not allowed.
%%
%%
%% \BibTeX command to typeset BibTeX logo in the docs
\AtBeginDocument{%
  \providecommand\BibTeX{{%
    \normalfont B\kern-0.5em{\scshape i\kern-0.25em b}\kern-0.8em\TeX}}}

%% Rights management information.  This information is sent to you
%% when you complete the rights form.  These commands have SAMPLE
%% values in them; it is your responsibility as an author to replace
%% the commands and values with those provided to you when you
%% complete the rights form.



%
%  Uncomment \acmBooktitle if th title of the proceedings is different
%  from ``Proceedings of ...''!
%
%\acmBooktitle{Woodstock '18: ACM Symposium on Neural Gaze Detection,
%  June 03--05, 2018, Woodstock, NY} 


\usepackage{tikz}
\newcommand*\circled[1]{\tikz[baseline=(char.base)]{
            \node[shape=circle,draw,inner sep=2pt] (char) {#1};}}
%
%  Uncomment \acmBooktitle if th title of the proceedings is different
%  from ``Proceedings of ...''!
%
%\acmBooktitle{Woodstock '18: ACM Symposium on Neural Gaze Detection,
%  June 03--05, 2018, Woodstock, NY} 



%%
%% Submission ID.
%% Use this when submitting an article to a sponsored event. You'll
%% receive a unique submission ID from the organizers
%% of the event, and this ID should be used as the parameter to this command.
%%\acmSubmissionID{123-A56-BU3}

%%
%% For managing citations, it is recommended to use bibliography
%% files in BibTeX format.
%%
%% You can then either use BibTeX with the ACM-Reference-Format style,
%% or BibLaTeX with the acmnumeric or acmauthoryear sytles, that include
%% support for advanced citation of software artefact from the
%% biblatex-software package, also separately available on CTAN.
%%
%% Look at the sample-*-biblatex.tex files for templates showcasing
%% the biblatex styles.
%%

%%
%% The majority of ACM publications use numbered citations and
%% references.  The command \citestyle{authoryear} switches to the
%% "author year" style.
%%
%% If you are preparing content for an event
%% sponsored by ACM SIGGRAPH, you must use the "author year" style of
%% citations and references.
%% Uncommenting
%% the next command will enable that style.
%%\citestyle{acmauthoryear}

%%
%% end of the preamble, start of the body of the document source.
\begin{document}


%%
%% The "title" command has an optional parameter,
%% allowing the author to define a "short title" to be used in page headers.
\title{Enhancing Data Space Semantic Interoperability through Machine Learning: a Visionary Perspective}

%%
%% The "author" command and its associated commands are used to define
%% the authors and their affiliations.
%% Of note is the shared affiliation of the first two authors, and the
%% "authornote" and "authornotemark" commands
%% used to denote shared contribution to the research.
%\authornote{Both authors contributed equally to this research.}


\definecolor{idcolor}{HTML}{A6CE39}
\newcommand{\orcidlink}[1]{\href{https://orcid.org/#1}{\color{idcolor}\faOrcid}}

\author{Zeyd Boukhers}
\affiliation{%
  \institution{Fraunhofer Institute for Applied Information Technology FIT and University Hospital Cologne, Institute for Biomedical Informatics}
  \streetaddress{ Konrad-Adenauer-Straße, 53757}
  %\city{Sankt Augustin}
  %\state{Ohio}
  \country{Germany}
  \postcode{43017-6221}
}
\email{zeyd.boukhers@fit.fraunhofer.de}
\orcid{0000-0001-9778-9164}


\author{Christoph Lange}
\affiliation{%
  \institution{Fraunhofer Institute for Applied Information Technology FIT and RWTH Aachen University}
  \streetaddress{ Konrad-Adenauer-Straße, 53757}
  %\city{Sankt Augustin}
  %\state{Ohio}
  \country{Germany}
  \postcode{43017-6221}
}
\email{christoph.lange-bever@fit.fraunhofer.de}
\orcid{0000-0001-9879-3827}


\author{Oya Beyan}
\affiliation{%
  \institution{University of Cologne, Faculty of Medicine and University Hospital Cologne, Institute for Biomedical Informatics}
  \streetaddress{Kerpener Straße 62 50937}
  %\city{Cologne}
  \country{Germany}}
\email{oya.beyan@uni-koeln.de}
\orcid{0000-0001-7611-3501}

%%
%% By default, the full list of authors will be used in the page
%% headers. Often, this list is too long, and will overlap
%% other information printed in the page headers. This command allows
%% the author to define a more concise list
%% of authors' names for this purpose.
\renewcommand{\shortauthors}{Trovato and Tobin, et al.}

%%
%% The abstract is a short summary of the work to be presented in the
%% article.
\begin{abstract}

Our vision paper outlines a plan to improve the future of semantic interoperability in data spaces through the application of machine learning. The use of data spaces, where data is exchanged among members in a self-regulated environment, is becoming increasingly popular. However, the current manual practices of managing metadata and vocabularies in these spaces are time-consuming, prone to errors, and may not meet the needs of all stakeholders. By leveraging the power of machine learning, we believe that semantic interoperability in data spaces can be significantly improved. This involves automatically generating and updating metadata, which results in a more flexible vocabulary that can accommodate the diverse terminologies used by different sub-communities. Our vision for the future of data spaces addresses the limitations of conventional data exchange and makes data more accessible and valuable for all members of the community.

\end{abstract}

%%
%% The code below is generated by the tool at http://dl.acm.org/ccs.cfm.
%% Please copy and paste the code instead of the example below.
%%
\begin{CCSXML}
<ccs2012>
   <concept>
       <concept_id>10002951.10002952.10003219.10003217</concept_id>
       <concept_desc>Information systems~Data exchange</concept_desc>
       <concept_significance>500</concept_significance>
       </concept>
   <concept>
       <concept_id>10002951.10002952.10002971.10003450</concept_id>
       <concept_desc>Information systems~Data access methods</concept_desc>
       <concept_significance>500</concept_significance>
       </concept>
   <concept>
       <concept_id>10002951.10003260.10003309.10003315</concept_id>
       <concept_desc>Information systems~Semantic web description languages</concept_desc>
       <concept_significance>500</concept_significance>
       </concept>
 </ccs2012>
\end{CCSXML}

\ccsdesc[500]{Information systems~Data exchange}
\ccsdesc[500]{Information systems~Data access methods}
\ccsdesc[500]{Information systems~Semantic web description languages}

%%
%% Keywords. The author(s) should pick words that accurately describe
%% the work being presented. Separate the keywords with commas.
\keywords{data spaces, semantic interoperability, machine learning}


\iffalse
%% A "teaser" image appears between the author and affiliation
%% information and the body of the document, and typically spans the
%% page.
\begin{teaserfigure}
  \includegraphics[width=\textwidth]{sampleteaser}
  \caption{Seattle Mariners at Spring Training, 2010.}
  \Description{Enjoying the baseball game from the third-base
  seats. Ichiro Suzuki preparing to bat.}
  \label{fig:teaser}
\end{teaserfigure}
\fi
\received{20 February 2007}
\received[revised]{12 March 2009}
\received[accepted]{5 June 2009}

%%
%% This command processes the author and affiliation and title
%% information and builds the first part of the formatted document.
\maketitle

\section{Introduction}
\label{sec:introduction}
% \begin{itemize}
%     % Diffusion of FL
%     \item {\st{Diffusion of FL}}
%     % Security threats to FL
%     \item {\st{Security threats to FL with particular focus on model poisoning}}
%     % Limitations of existing countermeasures
%     \item {\st{Current countermeasures (e.g., KRUM) and their limitations}}
%     % Proposed method and its advantages
%     \item {\st{Intuitive description of the proposed method and its difference (i.e., advantages) w.r.t. state of the art}}
%     % Main contributions
%     \item {\st{Summary of the main contributions of this work}}
%     % Paper's structure and organization
%     \item {\st{Paper's structure and organization}}
% \end{itemize}

% Diffusion of FL
Recently, {\em federated learning} (FL) has emerged as the leading paradigm for training distributed, large-scale, and privacy-preserving machine learning (ML) systems~\cite{mcmahan2017googleai,mcmahan2017aistats}. 
The core idea of FL is to allow multiple edge clients to collaboratively train a shared, global model without disclosing their local private training data.
%Specifically, an FL system consists of a central server and many edge clients; 
A typical FL round involves the following steps: {\em(i)} the server randomly picks some clients and sends them the current, global model; {\em(ii)} each selected client locally trains its model with its own private data; then, it sends the resulting local model to the server;\footnote{Whenever we refer to global/local model, we mean global/local model {\em parameters}.} {\em(iii)} the server updates the global model by computing an \emph{aggregation function}, usually the average (FedAvg), on the local models received from clients.
% \begin{enumerate}
%     \item[{\em(i)}] the server sends the current, global model to the clients and appoints some of them for training;
%     \item[{\em(ii)}] each selected client locally trains its copy of the global model with its own private data; then, it sends the resulting local model back to the server;\footnote{Whenever we refer to global/local model, we mean global/local model {\em parameters}.}
%     \item[{\em(iii)}] the server updates the global model by computing an \emph{aggregation function} on the local models received from clients (by default, the average, also referred to as FedAvg~\cite{mcmahan2017aistats}).
% \end{enumerate}
This process goes on until the global model converges. %(e.g., after a certain number of rounds or other similar stopping criteria).
%\\
% The advantages of FL over the traditional, centralized learning paradigm are undoubtedly clear in terms of flexibility/scalability (clients can join/disconnect from the FL network dynamically), network communications (only model weights\footnote{We will use \textit{parameters} and \textit{weights} interchangeably.} are exchanged between clients and server), and privacy (each client's private training data is kept local at the client's end and not uploaded to the server).
\\
% Security threats to FL
%However, the growing adoption of FL also raises security concerns~\cite{costa2022covert}, particularly about its confidentiality, integrity, and availability.
Although its advantages over standard ML, FL also raises security concerns~\cite{costa2022covert}. %, particularly about its confidentiality, integrity, and availability~\cite{costa2022covert}.
% OLD, LONG VERSION
% Indeed, some work deals with privacy leakage that may expose the local data of some clients~\cite{melis2019sp}. 
% A large body of work, instead, investigates attacks that usually aim to detriment the predictive accuracy of the learned global model. For instance, \emph{data poisoning} attacks achieve this goal by letting an adversary pollute the training set of some corrupt FL clients with maliciously crafted examples~\cite{jagielski2018sp}.
% Similarly, in \emph{model poisoning} the attacker attempts to tweak the global model weights~\cite{bhagoji2019pmlr} by directly perturbing the local model's weights of some infected FL clients before these are sent to the central server for aggregation, usually via so-called Byzantine attacks. 
% It turns out that Byzantine model poisoning attacks severely impact standard FedAvg; therefore, more robust aggregation functions must be designed to make FL systems secure.
Here, we focus on \emph{untargeted model poisoning} attacks~\cite{bhagoji2019pmlr}, where an adversary attempts to tweak the global model weights %\footnote{We will use the terms \textit{parameters} and \textit{weights} interchangeably.} 
by directly perturbing the local model's parameters of some infected clients before these are sent to the central server for aggregation.
In doing so, the adversary aims to jeopardize the global model \textit{indiscriminately} at inference time.
Such model poisoning attacks severely impact standard FedAvg; therefore, more robust aggregation functions must be designed to secure FL systems.
\\
% In this paper, we focus on designing a novel robust aggregation scheme at the server's end to contrast the effect of Byzantine model poisoning attacks.
%
% Current countermeasures and their limitations
%Several countermeasures have been proposed in the literature to combat model poisoning attacks on FL systems.
% Some methods use simple statistics more robust than plain average to smooth the impact of malicious updates (e.g., Trimmed Mean and FedMedian~\cite{yin2018icml}). 
% Other defenses implement outlier detection techniques to discard malicious updates from the aggregation performed at the server's end. Those are either based on heuristics (e.g., Krum/Multi-Krum~\cite{blanchard2017nips} and Bulyan~\cite{mhamdi2018pmlr}) or data-driven approaches (e.g., K-means clustering~\cite{shen2016acm} or DnC via spectral analysis~\cite{shejwalkar2021ndss}). 
% Finally, some strategies rely on a centralized ``source of trust'' to spot potential malicious updates (e.g., FLTrust~\cite{cao2020fltrust}).
% Several countermeasures have been proposed in the literature to combat model poisoning attacks on FL systems, i.e., to discard possible malicious local updates from the aggregation performed at the server's end. 
% These techniques range from simple statistics more robust than plain average (e.g., Trimmed Mean and FedMedian~\cite{yin2018icml}) to outlier detection heuristics (e.g., Krum/Multi-Krum~\cite{blanchard2017nips} and Bulyan~\cite{mhamdi2018pmlr}) or data-driven approaches (e.g., spectral analysis via K-means clustering~\cite{shen2016acm} or spectral analysis), or methods based on ``source of trust'' (e.g., FLTrust~\cite{cao2020fltrust}).
% OLD, LONG VERSION
%Several countermeasures have been proposed in the literature to combat Byzantine model poisoning attacks on FL systems.
% Descriptive statistics
% For example, Trimmed Mean and FedMedian aggregate local model updates using more robust statistics than standard average~\cite{yin2018icml}.
%
% % Heuristics for outlier detection
% Many existing Byzantine-resilient strategies implement some outlier detection heuristics to discard the model updates sent by potentially malicious clients from the input of the aggregation function.
% One of the most popular heuristics is Krum~\cite{blanchard2017nips}.
% This strategy tries to mitigate the impact of Byzantine attacks by selecting as a global model the local model with the smallest sum of Euclidean distances to {\em all} the other local models.
% Although powerful, Krum requires the server to know (or, at least, estimate) the number of malicious FL clients upfront, which is generally impossible in a realistic attack scenario. %
% Moreover, Krum may become ineffective for complex, high-dimensional model parameter spaces due to the curse of dimensionality.
% Bulyan~\cite{mhamdi2018pmlr} tries to overcome this issue by combining Krum with a variant of Trimmed Mean.
% % Data-driven outlier detection
% Other strategies use data-driven outlier detection techniques -- e.g., via K-means clustering~\cite{shen2016acm} -- to spot potential malicious local model updates. 
% %For instance, Shen et al. propose to cluster local model updates with K-means and thus identify outliers.
%
% % Other techniques
% As far as the server is concerned, any local model received can be from a potential malicious client. 
% FLTrust~\cite{cao2020fltrust} assumes the server acts as a client, i.e., trains a local model on an additional {\em trustworthy} dataset at the server's end and compares it against all the local models from other clients. 
% This way, the server can rely on some ``source of trust'' when discarding potentially malicious clients.
%\\
% Limitations of existing Byzantine-resilient strategies
Unfortunately, existing defense mechanisms either rely on simple heuristics (e.g., Trimmed Mean and FedMedian by~\cite{yin2018icml}) or need strong and unrealistic assumptions to work effectively (e.g., foreknowledge or estimation of the number of malicious clients in the FL system, as for Krum/Multi-Krum~\cite{blanchard2017nips} and Bulyan~\cite{mhamdi2018pmlr}, which, however, cannot exceed a fixed threshold).
Furthermore, outlier detection methods using K-means clustering~\cite{shen2016acm} or spectral analysis like DnC~\cite{shejwalkar2021ndss} do not directly consider the temporal evolution of local model updates received.
Finally, strategies like FLTrust~\cite{cao2020fltrust} require the server to collect its own dataset and act as a proper client, thereby altering the standard FL protocol.
\\
% OLD, LONG VERSION
% Overall, existing Byzantine-resilient strategies are either simple heuristics (e.g., FedMedian) or, if they are more complex, they rely on strong and unrealistic assumptions to work effectively (e.g., knowing the number of malicious clients in the FL system in advance, as for Krum and alike).
% Furthermore, data-driven outlier detection methods do not consider the temporary evolution of local model updates received (e.g., K-means clustering). 
% Finally, strategies like FLTrust requires the server to collect its own dataset and act as a proper client, thereby altering the standard FL protocol.
%
% Description of the proposed method
This work introduces a novel pre-aggregation \textit{filter} robust to untargeted model poisoning attacks. Notably, this filter $(i)$ operates without requiring prior knowledge or constraints on the number of malicious clients and $(ii)$ inherently integrates temporal dependencies. 
The FL server can employ this filter as a preprocessing step before applying \textit{any} aggregation function, be it standard like FedAvg or robust like Krum or Bulyan.
Specifically, we formulate the problem of identifying corrupted updates as a multidimensional (i.e., matrix-valued) time series anomaly detection task. 
The key idea is that legitimate local updates, resulting from well-calibrated iterative procedures like stochastic gradient descent (SGD) with an appropriate learning rate, show \textit{higher predictability} compared to malicious updates. This hypothesis stems from the fact that the sequence of gradients (thus, model parameters) observed during legitimate training exhibit regular patterns, as validated in Section~\ref{subsec:intuition}. %until convergence. 
%This regularity may be more pronounced for smooth convex loss functions, but it can still be captured within an appropriate time window, even for more complex and convoluted loss surfaces. 
%We provide evidence of this claim in Appendix~B, where we show that the average mutual information (i.e., ``predictability''), calculated over pairs of legitimate model updates sent at different FL rounds, is significantly higher than the corresponding computation for a malicious client.
\\
Inspired by the matrix autoregressive (MAR) framework for multidimensional time series forecasting~\cite{chen2021je}, we propose the FLANDERS ({\em \textbf{F}ederated \textbf{L}earning meets \textbf{AN}omaly \textbf{DE}tection for a \textbf{R}obust and \textbf{S}ecure}) filter.
The main advantages of FLANDERS over existing strategies like FLDetector~\cite{zhao2020multivariate} are its resilience to large-scale attacks, where $50\%$ or more FL participants are hostile, and the capability of working under realistic non-iid scenarios.
We attribute such a capability to two key factors: $(i)$ FLANDERS works without knowing a priori the ratio of corrupted clients, and $(ii)$ it embodies temporal dependencies between intra- and inter-client updates, quickly recognizing local model drifts caused by evil players. Below, we summarize our main contributions:

\begin{itemize}
\item[{\em(i)}]
We provide empirical evidence that the sequence of models sent by legitimate clients is more predictable than those of malicious participants performing untargeted model poisoning attacks.
\\
\item[{\em(ii)}] 
We introduce FLANDERS, the first pre-aggregation filter for FL robust to untargeted model poisoning based on multidimensional time series anomaly detection.
\\
\item[{\em(iii)}] 
We integrate FLANDERS into Flower,\footnote{\scriptsize{\url{https://flower.dev/}}} a popular FL simulation framework for reproducibility.
\\
\item[{\em(iv)}] 
We show that FLANDERS improves the robustness of the existing aggregation methods under multiple settings: different datasets, client's data distribution (non-iid), models, and attack scenarios.
\\
\item[{\em(v)}] 
We publicly release all the implementation code of FLANDERS along with our experiments.\footnote{\scriptsize{\url{https://anonymous.4open.science/r/flanders_exp-7EEB}}}
\end{itemize}

% Paper's structure and organization
The remainder of the paper is structured as follows. %some related work and the current state-of-the-art solutions to security issues that FL entails. 
Section~\ref{sec:background} covers background and preliminaries. 
In Section~\ref{sec:related}, we discuss related work.
Section~\ref{sec:problem} and Section~\ref{sec:method} describe the problem formulation and the method proposed. % to tackle it. 
Section~\ref{sec:experiments} gathers experimental results. %, and Section~\ref{sec:limitations} discusses some limitations of this work.
Finally, we conclude in Section~\ref{sec:conclusion}.
 %discusses the limitations of this work and draws future research directions.
%reports conclusions and draws perspectives for future research directions.

%%%%%%% OLD %%%%%%%
%to overcome the resilience of Byzantine failures in distributed Stochastic Gradient Descent computations. 
% The strength of Krum is its time complexity, which is linear in the gradient dimension. 
% However, the robustness of the approach is guaranteed for gradient-based learning applications only when the majority of the clients are not compromised. 
% Besides, the aggregation mechanism of Krum, as well as that of similar methods, is robust from a coarse-grained perspective and does not provide solutions to errors and perturbations that may occur at inference time.
%A related approach to~\cite{blanchard2017nips} is the work of Su et al.~\cite{su2016dc}. Here, the authors propose an iterated approximate agreement to tackle a multi-layer scenario attacked by Byzantine agents. 
%However, the method works efficiently on the sole discrete context and it is inapplicable to continuous state environments.
%\gabri{Maybe, we should just talk about the main limitations of existing countermeasures without digging into their details (or, we can just mention Krum as this is the most popular one). I will move the description of all these methods to the Related Work section.}
\section{Related work}
\noindent \textbf{Video foundation models.}
With sufficient computational power and an abundant source of data, there have been attempts to build a single large-scale foundation model that can be adapted to diverse downstream tasks.
Along with the success of foundations models in the natural language processing domain~\cite{brown2020language,chen2021evaluating,devlin2019bert} and in computer vision~\cite{bertasius2021space,jia2021scaling,radford2021learning}, video data has become another data type of interest, as it has grown in scale due to numerous internet video-sharing platforms.
Accordingly, several methods to train a video foundation model have been proposed.
Due to the innate multi-modality of video data, \textit{i.e.}, a combination of visual $\cdot$ vocal $\cdot$ textual context, most works have centered around the variations of the cross-modal attention mechanism \cite{akbari2021vatt,bertasius2021space,gabeur2020multi,luo2020univl,neimark2021video,tan2021look,wei2020multi,yang2021taco}.
In addition, as most video data lack proper labels or descriptions, contrastive learning methods were studied to learn meaningful feature representations or enhance video-text alignment in a self-supervised manner \cite{akbari2021vatt,kuang2021video,luo2020univl,yang2021taco}.

More specifically, MERLOT \cite{zellers2021merlot} proposed a multi-modal representation learning method for visual commonsense reasoning, which also performed well in twelve video reasoning tasks.
VATT \cite{akbari2021vatt} introduced a multi-modal learning method via contrastive learning. 
The pre-trained model performed well in a variety of vision tasks from image classification to video action recognition and zero-shot video retrieval.
Another representative work, UniVL \cite{luo2020univl} proposed a straightforward pre-training method with auxiliary loss functions. 
After fine-tuning on a specific task, the pre-trained model performed outstandingly in a wide range of tasks of text-to-video retrieval, action segmentation, action step localization, video sentiment analysis, and video captioning.
Other foundation models for multiple video tasks include \cite{li2020hero,sun2019learning,sun2019videobert,zhu2020actbert,fu2021violet,wang2022all}. 

\noindent \textbf{Auxiliary learning.}
In order to enhance the performance of one or a multitude of primary tasks, auxiliary learning methods can be incorporated.
\cite{ruder2017overview} introduced Multi-task learning (MTL) to the deep neural networks by training a single model with multiple task losses to assist learning on the main task.
Such a method is generally adapted to pre-train the foundation models in the self-supervised manner~\cite{li2020hero,sun2019learning,sun2019videobert,zhu2020actbert,fu2021violet,wang2022all}.
However, these various pretext task losses used in the pre-training phase are ignored in the fine-tuning phase, and only the primary task loss is minimized.

Recently, meta-learning methods have been introduced for auxiliary learning.
\cite{liu2019self,navon2020auxiliary,shu2019meta} proposed a meta-learning method in which the model learns auxiliary tasks to generalize well to unseen data. 
In these settings, a separate subset of data is held out as the primary task, while the others are used as auxiliary tasks that aid the primary task's performance.
Similar methods were adopted for computer vision tasks such as semantic segmentation \cite{xu2021leveraging}.
Other domain applications include navigation tasks with reinforcement learning \cite{ye2021auxiliary}, or self-supervised learning methods on graph data \cite{hwang2020self}.
\section{ML-enhanced Data Spaces}


Semantic interoperability in data spaces is a complex issue that involves multiple aspects, as illustrated in Figure~\ref{fig:aspects}. While machine learning has the potential to improve each of these aspects, traditional approaches have primarily utilized machine learning techniques in isolation, rather than within the broader context of data spaces. It is vital to consider the full spectrum of semantic interoperability aspects and integrate machine learning in a comprehensive and holistic manner within the data space environment. 

\begin{figure}
    \centering
    \includegraphics[width=\linewidth]{Figures/aspects.pdf}
    \caption{Semantic interoperability aspects in data spaces that machine learning can enhance}
    \label{fig:aspects}
\end{figure}




Figure~\ref{fig:overview} presents an overview of the ML-enhanced data space in the International Data Spaces environment, showcasing six key aspects of data management among three stakeholders, including data providers and consumers and service providers. These aspects are:
\begin{itemize}[leftmargin=*]
\item  \textbf{Automatic Metadata Extraction (\circled{1})}: A machine learning model can automatically extract essential attribute values from the data if metadata is not already available, helping data providers to prepare their data for exchange and consumption without the need for manual metadata preparation.

\item \textbf{Ontology and Vocabulary Alignment (\circled{2})}: The vocabulary of the data space is aligned with the vocabulary of the data provider, enabling data consumers to understand the data being exchanged. This eliminates the need for members in the data space to adopt the same internal vocabulary, which can often be a challenging task.

\item \textbf{FAIRness Evaluation (\circled{3})}: The FAIRness level of the data is assessed based on provided or extracted metadata, allowing the data provider to improve the FAIRness of their data and allowing the data consumer to understand the ease of use of the data.

\item \textbf{Data Quality Assessment \& Enhancement (\circled{4})}: The quality of the data is evaluated and improved if possible, based on the format of the data. Machine learning can be used to evaluate and enhance structured and tabular data, however, it's important to recognize that the quality metrics may vary depending on the format of the data. For example, it might be challenging to assess the quality of unstructured data (e.g., a corpus of documents).

%not all formats support quality assessment.

\item \textbf{Privacy Preserving (\circled{5})}: ML-based anonymization and masking techniques can be applied to data that contains private, sensitive, or personal information to make it shareable. Sensitive data can be automatically detected or provided by the data provider, allowing data providers to share their data without any privacy concerns.

\item \textbf{Compatibility Improvement (\circled{6})}: The data is transformed into a readable format for the data consumer. In cases where data is being merged with the consumer's data, the consumer will communicate the structure and format, enabling the data to be transformed accordingly. This allows the consumer to make use of the received data without having to put in additional effort to read and understand it.
\end{itemize}


\begin{figure*}[h]
    \centering
    \includegraphics[width=\linewidth]{Figures/overview.pdf}
    \caption{An overview of an ML-enhanced Data Space with three members. (1) Automatic Metadata Extraction, (2) Ontology and Vocabulary Alignment, (3) FAIRness Evaluation, (4) Data Quality Assessment \& Enhancement, (5) Privacy Preserving, (6) Compatibility Improvement.}
    \label{fig:overview}
\end{figure*}



In the following, we discuss each of these aspects: 


\subsection{Automatic Metadata Extraction}
Metadata plays a vital role in data exchange as it enables data consumers to understand the data and determine if it meets their needs. However, many data providers may be hesitant to provide the necessary metadata due to a lack of capacity or knowledge to prepare it for their resources. This can be a significant obstacle in data exchange, as it limits the ability of consumers to access and utilize the data they need.

To overcome this challenge, machine learning can be leveraged to (semi) automatically extract metadata from resources. Machine learning algorithms can be trained on a dataset of resources and their corresponding metadata, allowing them to learn the patterns and relationships between the data and the metadata. These algorithms can then be applied to new resources to extract the relevant metadata. This approach has the advantage of being able to handle complex and nuanced relationships between the data and the metadata. It can also be easily updated and adapted as the data and its needs evolve. However, it is important to note that a typical challenge in data spaces is that the resources have different, heterogeneous formats.




%Machine learning can also assist in the creation of metadata for resources by using natural language processing to extract information from the resource's title or description to automatically generate the metadata. The metadata properties can vary based on the resource being exchanged and it's necessary to use different ML models for different sources.


Different resources being exchanged in data spaces can have varying metadata properties, and it may be necessary to utilize different machine learning (ML) models for different resources and metadata attributes. For instance, in the case of document corpora, Natural Language Processing (NLP) techniques can be employed to extract titles and descriptions. Specifically, automatic metadata extraction techniques such as those in~\cite{boukhers2022vision, tkaczyk2017new} can be utilized to extract metadata from each document, such as \emph{Publication Date}, \emph{Author}, \emph{Language}, etc. This metadata can then be used to derive the metadata for the entire collection, such as \emph{Publication Range}, \emph{Authors}, \emph{Languages}, etc.



\subsection{Ontology and Vocabulary Alignment}
\label{sec:onto}

The International Data Spaces Reference Architecture\footnote{\url{https://internationaldataspaces.org/use/reference-architecture/}} highlights the importance of common vocabularies for effective data exchange within a data space. However, in practice, data providers may have their own unique vocabularies, making it difficult to align them with the vocabulary used in the data space. This can be due to the cost and effort involved in mapping their existing vocabularies to the data space vocabulary, or due to the fact that a data provider may participate in multiple data spaces with different vocabularies.

To tackle these challenges, machine learning algorithms can be utilized to support automatic mapping between the local vocabulary of a data provider and the vocabulary used in the data space. This allows for seamless and interoperable data exchange, without requiring data providers to adopt a new vocabulary.

Machine learning-based methods for ontology alignment~\cite{nezhadi2011ontology} and ontology matching~\cite{doan2004ontology} can be applied to automatically map concepts and terms from one ontology or vocabulary to another. These algorithms use techniques such as semantic similarity measures~\cite{sousa2022supervised}, graph-based methods~\cite{shenoy2013secured}, and deep learning models~\cite{khoudja2018ontology, iyer2020veealign, bento2020ontology} to identify correspondences between concepts in different ontologies or vocabularies. The goal is to produce a mapping that enables data exchange between systems using different ontologies or vocabularies while preserving the meaning of the data.



\subsection{FAIRness Evaluation}

%Recently, the FAIR principles (i.e. \textbf{F}indable, \textbf{A}ccessible, \textbf{I}nteroperable and \textbf{R}esuable) are highly regarded in data exchange as compliance with them increases the likelihood of resources being reused. Therefore, evaluating the FAIRness of a resource can assess its fitness for use. For instance, even though the data might be suitable for a particular use case, the accompanying license might not be appropriate. Identifying this in advance can save time and resources, as the cost of negotiating for data exchange or waiting for access to data that does not meet all the requirements (including FAIRness, licensing, and access conditions) can take several months. Thus, evaluating the FAIRness level of a resource beforehand can help reduce the time and effort spent on obtaining data that may not be suitable for the intended use.


The FAIR (i.e., \textbf{F}indable, \textbf{A}ccessible, \textbf{I}nteroperable and \textbf{R}eusable) principles are becoming increasingly important in data exchange and sharing. These principles aim to ensure that data resources are easily discoverable, accessible, can be easily integrated with other data sources, and can be reused for multiple purposes. Compliance with these principles makes it more likely that data will be used and reused, as it increases the overall quality and usability of the resource.


Evaluating the FAIRness of a resource is a crucial step in determining its fitness for use, as it helps to identify any potential barriers to reuse. This can include issues such as licensing restrictions, data access conditions, and data interoperability issues. Conducting this evaluation in advance can save valuable time and resources, as it helps to avoid the need for costly negotiations or lengthy wait times for access to data that may not be suitable for the intended use.

As discussed in Section~\ref{sec:onto}, the use of shared vocabularies, such as ontologies, is important for increasing the findability and interoperability of resources. However, only using mapping techniques (see Section~\ref{sec:onto}) may not be enough, as internal ontologies that describe the metadata may not be represented using common classes. To address this issue, machine learning techniques, such as BERTmap~\cite{he2022bertmap}, can be used to assess the level of compatibility between the provider's ontology and the data space's ontology. Additionally, rule-based and semantic web technologies can be used to evaluate the structure of the metadata, further increasing the overall FAIRness of the resource.


\subsection{Data Quality Assessment \& Enhancement}
Data quality is a crucial concern for data consumers, as it impacts the trustworthiness and usefulness of the data. Unfortunately, metadata alone cannot provide any indication of the quality of the data. To ensure the quality of data, various dimensions must be considered, including accuracy, completeness, correctness, validity, integrity, and uniqueness. The importance of each dimension may vary depending on the intended use of the data and the needs of the data consumer.


Accuracy refers to how closely the data reflects the real-world phenomenon it represents. Completeness refers to the extent to which all necessary data is present. Correctness pertains to the degree to which the data adheres to established rules, such as those related to syntax, semantics, or data constraints. Validity refers to the degree to which the data follows the predefined format, structure, and domain. Integrity is the degree to which the data is protected against unauthorized changes. Lastly, uniqueness refers to the degree to which each data item is distinct and identifiable.

To ensure data quality, data providers must take steps to assess and improve the quality of their data. This can include implementing data validation and quality checks, using techniques like data profiling and data cleaning, and implementing data governance policies and procedures. Data consumers should also take steps to assess the quality of the data they receive, such as evaluating the data's source and provenance, performing data quality checks, and monitoring the data for anomalies.


Machine learning algorithms can play a crucial role in ensuring the quality and accuracy of data. One way they achieve this is by comparing the data to other sources to validate its accuracy. Additionally, machine learning algorithms can be trained to identify patterns and anomalies in the data~\cite{agrawal2015survey, pang2021deep}, helping to flag any potential inaccuracies or errors.

Another benefit of using machine learning algorithms is the ability to complete missing data. By analyzing patterns and relationships in the data, machine learning models can make predictions about missing values and fill them in~\cite{thomas2021systematic, raja2020missing, hasan2021missing}. This is especially useful in cases where it would be time-consuming or challenging to manually fill in missing data.

Furthermore, machine learning techniques can also be applied to identify and remove duplicates in data, improving the overall uniqueness and consistency of the data~\cite{park2022deepsketch,tarun2021scheme, christen2019towards}.


\subsection{Privacy Preserving}

Private and sensitive data, such as personal information, medical records, and financial data, is often subject to strict regulations and guidelines for protection and access.
% These regulations.
In order for different systems to exchange and use private data, they must be able to accurately interpret and understand the meaning and context of the data, and ensure that it is being used in compliance with applicable laws and regulations. ensuring semantic interoperability for private data requires a combination of technical solutions, such as secure data exchange protocols and data anonymization techniques, and strict governance and compliance mechanisms.

To achieve this, data providers can use machine learning techniques to automatically detect private and sensitive data in their systems~\cite{ray2021sensitive, ahmed2021automated} and take appropriate actions to mask~\cite{torra2022privacy} or anonymize~\cite{majeed2020anonymization} the data. This can help protect individuals' privacy while enabling data sharing and interoperability. For example, techniques such as data de-identification, data masking, and differential privacy can be used to remove identifying information from data while preserving its usefulness for analysis. 



\subsection{Compatibility Improvement}
Also, when the same vocabulary and ontology are used by the data provider and consumer, resources are not semantically interoperable if they are not compatible with the consumer system of their resource to be integrated with. To overcome the incompatibility of resources in data exchange, solutions include data mapping and data transformation. Machine learning techniques have shown great performance in these tasks. 


Resources are not semantically interoperable when they cannot be understood or used by the systems that need to access them. This can occur when the resources have different data formats or structures, making it difficult for systems to integrate and make use of the information.

To overcome the incompatibility of resources in data exchange, solutions include data mapping~\cite{li2018mfecnn} and data transformation~\cite{sajid2019predictive}. Data mapping is the process of aligning the data elements from one resource to the corresponding elements in another resource. Data transformation is the process of converting data from one format or structure to another. Both of these solutions can help to make resources compatible and enable data exchange. Machine learning can also be used to convert data from one format to another, such as natural language text to structured data~\cite{verma2020unstructured}. 







\iffalse

Figure~\ref{fig:overview} illustrates an overview of an ML-enhances data space with three members and service providers. The ML-enhanced approach to data space management has six aspects, as outlined below:

Automatic Metadata Extraction (\circled{1}): If the metadata is not already created, the machine learning technique can extract the most important attributes from the data.

Ontology and Vocabulary Alignment (\circled{2}): The data provider's vocabulary is mapped to the vocabulary of the data space, to ensure that the data consumers can understand the received data.

FAIRness Evaluation (\circled{3}): The FAIRness level of the data is assessed based on the provided or extracted metadata.

Data Quality Assessment & Enhancement (\circled{4}): Quality assessment and improvement of data are possible for data with a format that allows it. Structured and tabular data can be evaluated and enhanced using machine learning techniques, but it is important to note that not all data formats support quality assessment.

Privacy-Preserving (\circled{5}): If the data contains private, sensitive, or personal information, anonymization and masking techniques are applied to make the data suitable for sharing. The sensitive data can be automatically detected or provided by the data provider.

Compatibility Improvement (\circled{6}): The data is transformed into a readable format for the data consumer. In cases where the data is to be merged with the data consumer's data, the latter communicates its structure and format to the data provider, allowing the data to be transformed accordingly.





First, if the metadata is not already created, it can be automatically extracted from the data (see \circled{1}). The capability of the machine learning technique can differ according to the type of data but in general, the values of the most important attributes can be extracted. Second, the vocabulary of the data provider can be mapped to the vocabulary of the data space to ensure that the data consumers can understand the received data (see \circled{2}). Third, the FAIRness level is assessed given the provided or extracted metadata (see \circled{3}). Fourth, The quality of the data is assessed and improved if the format of the data allows it (see \circled{4}). Specifically, The quality of structured and tabular data can theoretically be assessed and improved using machine learning techniques. Fifth, if the data contains private, sensitive or personal data, anonymization and masking techniques can be used to make the data ready to be shared (see \circled{5}). The sensitive data can be either detected automatically or provided by the data provider. Finally, the data is transformed in such a way that it can be readable by the data consumer. In case the data is planned to be merged with the data consumer's data, the latter communicates its structure and format with the data provider so that the data is transformed accordingly. 

\fi
We provide some comments on the growth conditions which constituted the majority of our analysis in sections \ref{sec:Hmixing} and \ref{sec:Hsigma}. In the simplest cases of Lemma \ref{lemma:unstableGrowth}, growth was established in an analogous fashion to the old one-step expansion condition (\ref{eq:oldOneStepExpansion}), finding the relevant Jacobians $M_j$ and checking that their expansion factors $K(M_j)$ satisfy
\begin{equation}
    \label{eq:discussionOneStep}
    \sum_j \frac{1}{K(M_j)} <1.
\end{equation}
For the more complicated cases, the inductive method used to establish growth near the accumulation points in Lemma \ref{lemma:unstableGrowth} and the weakened one-step expansion condition (\ref{eq:oneStep}) both address the same fundamental issue: the splitting of unstable curves by singularities into an unbounded number of small components. They circumvent this obstacle in rather different ways, however. While (\ref{eq:oneStep}) generalises (\ref{eq:discussionOneStep}) to ensure an growth of unstable curves `on average' (see \cite{chernov_statistical_2009} for a precise statement), our inductive method is a more direct adaptation of (\ref{eq:discussionOneStep}), using it to generate contradictory geometric conditions which a hypothetical non-growing unstable curve must satisfy. It may be possible to prove Theorem \ref{sec:Hmixing} using (\ref{eq:oneStep}) as the basis for growth. Since we required (\ref{eq:oneStep}) anyway for proving Theorem \ref{thm:HsigmaExp}, this could potentially condense our analysis, but only to a minor extent. A convenience of the method used in section \ref{sec:Hmixing} is that, by way of the `simple intersection' property, it naturally gives geometric information on the images of manifolds, useful for proving the property \textbf{(M)} of Theorem \ref{thm:katok-strelcyn}.

We expect that essentially analogous analysis can be applied to establish mixing properties in a wide class of piecewise linear non-uniformly hyperbolic maps, including those (like the OTM) which sit on the boundary of ergodicity and beyond. While we have relied on the precise partition structure of $H_\sigma$, its fundamental feature (self-similar sequences of elements $A^k$, sharing boundaries with its neighbours $A^{k-1},A^{k+1}$ and accumulating onto some point $p$) is quite typical to return map systems. See, for example, those of various stadium billiards \cite{chernov_chaotic_2006,chernov_improved_2008,chernov_statistical_2009} and LTMs \cite{springham_polynomial_2014}. Indeed, the same method can be used to prove the Bernoulli property for non-monotonic LTMs \cite{myers_hill_mixing_2022}, where monotonicity of the manifold images cannot be assumed and the classical argument \cite{sturman_mathematical_2006} fails. The OTM is the pointwise limit of these maps as the boundary shrinks to null measure. It further has utility in proving growth conditions for maps which are uniformly hyperbolic but possess regions $A_j$ where the hyperbolicity is very weak, signified by $K(M_j) \approx 1$, so that (\ref{eq:discussionOneStep}) fails. Typically this leads to suboptimal bounds on mixing windows, see e.g. \cite{wojtkowski_model_1981,przytycki_ergodicity_1983,myers_hill_family_2022}. The map $H_{(\eta,\eta)}$ for $\eta \approx 1/2$ is another example, possessing weak hyperbolicity over $A_2, A_3$. Letting $\varepsilon = |\eta-1/2|>0$, there is an upper bound $N = N(\varepsilon)$ on escape times from the intersections $A_2\cap \sigma, A_3 \cap \sigma$. The growth lemma then follows by applying the inductive step roughly $N$ times and can be established for arbitrarily small $\varepsilon$, opening the door to establishing optimal mixing windows.

The above gives two examples of piecewise linear perturbations to $H$ where mixing with respect to Lebesgue is preserved and our methods can be applied. Nonlinear perturbations to the shear profiles complicate the analysis in several ways. Firstly as the map's Jacobians takes on a broader range of values, cone invariance becomes an increasingly harder condition to establish. Cones must be widened, giving looser bounds on expansion factors, which may already be weak due to new regions of weaker stretching. This, together with the change from polygonal to curvilinear return time partition elements and nonlinear local manifolds, adds some complexity to showing growth conditions. This does not rule out certain (small) nonlinear perturbations however. There is some leeway in the inequalities which govern cone invariance and growth of local manifolds, the latter of which is not too dissimilar from the piecewise linear setting (see Lemmas \ref{lemma:piecewiseApprox}, \ref{lemma:componentLength}). Certain small perturbations would not alter the \emph{topological} structure of the return time partition, i.e. which elements share boundaries, the key information needed for setting up the induction. Finally while the partition elements would no longer be polygonal, only coarse geometric information is required for verifying each inductive step. Following the above, a potential perturbation could be to replace the linear portions of each shear by a cubic, perturbing the tent profile
\[  f(t) = \begin{cases} 2t & 0 \leq t \leq 1/2, \\ 2(1-t) & 1/2 \leq t \leq 1 ,\end{cases} \]
of the OTM shears to
\[  f_a(t) = \begin{cases} \frac{1}{8} t \left(16 - a + 6at - 8at^{2} \right) & 0 \leq t \leq 1/2, \\ \frac{1}{8}\left(1-t\right)\left( 16 - a + 6a\left(1-t\right) - 8a\left(1-t\right)^{2}\right)  & 1/2 \leq t \leq 1, \end{cases}   \]
for $a>0$. For small enough $a$ the gradient range $f'(t)$ is restricted to small neighbourhoods of $\{ 2, -2\}$ and the escape time partition retains a similar structure. We illustrate this in Figure \ref{fig:perturbations}, showing escapes from the square $S_3$ under the map $G \circ F$, equivalent to escapes from the perturbed $A_3$ under the $G \circ F$, but with a cleaner geometry for comparison. When $a$ is too large the analogy to the OTM breaks down. At $a=16$ the map is twice differentiable everywhere and features a new source of slowed mixing, the Jacobian is the identity at the corner points $x,y \in \{  0, 1/2 \}$ giving locally parabolic behaviour (visible in the escape time partition). 

\begin{figure}
    \centering
    \includegraphics[width=0.24 \linewidth]{0.png}
    \includegraphics[width=0.24 \linewidth]{4.png}
    \includegraphics[width=0.24 \linewidth]{8.png}
    \includegraphics[width=0.24 \linewidth]{16.png}
    \caption{Partition of escape times from $S_3$ under the mapping $F \circ G$ for $a= 0,4,8,16$. }
    \label{fig:perturbations}
\end{figure}
\section{Conclusion}\label{sec:conclusion}
In this work, we focus on addressing the fundamental challenge of OOD detection tasks, which is how to fully understand the semantic discrepancy between the ID/OOD samples. We reveal that the key to success in the realistic SCOOD task is to allocate as many ID samples in the unlabeled set correctly as possible. To this end, we propose a novel uncertainty-aware optimal transport scheme that introduces class-specific energy scores as guidance for effective label assignment. Experimental results show that our method achieves better performance than previous state-of-the-art methods on SCOOD benchmarks.

\textbf{Limitations.} In addition to temperature scaling, other techniques such as feature clipping applied in ReAct~\cite{sun2021react} also enhance the performance of energy score, so how to obtain an OOD score that best fits the SCOOD task can be further explored. Moreover, a setting highly related to SCOOD has been proposed in \cite{katz2022training} and formulated as a constrained optimization problem. We will also theoretically analyze these practical OOD settings in our feature work.

% \section*{Acknowledgments}
\textbf{Acknowledgments.} 
This work is supported by National Key R\&D Program of China under Grant 2020AAA0105701, National Natural Science Foundation of China (NSFC) under Grants 61872327, Major Special Science and Technology Project of Anhui, National Natural Science Foundation of China (62033012) and Ant Group through Ant Research Intern Program.




%%
%% The next two lines define the bibliography style to be used, and
%% the bibliography file.
\bibliographystyle{ACM-Reference-Format}
\bibliography{sample-base}



\end{document}
\endinput
%%
%% End of file `sample-sigconf.tex'.
