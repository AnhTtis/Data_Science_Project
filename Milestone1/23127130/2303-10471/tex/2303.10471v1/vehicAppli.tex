\section{Literature Review: Vehicular Applications}
\label{sec:vehicApp}

In this section we present our review of  the literature  and categorize the results of over 100 research papers based on both application and algorithm type that are concerned with the applications of Koopman operator theory to the field of smart mobility and vehicular engineering. Table~\ref{tab1} details the specific Koopman operator-based system identification method/algorithm used by the studies referenced hereafter. Given the vast number of algorithms and variants thereof, the reader is encouraged to refer to the respective studies to obtain their technical details. The following presents the surveyed literature organized by vehicle category. 

%Additionally, Figure~\ref{fig:TimeLine}  provides a  timeline of utilization of Koopman operator theory in various vehicular applications. The %timeline presented in Figure~\ref{fig:TimeLine} also demonstrates in which year a given Koopman operator-based algorithm has first emerged (i.e., %the first emergence of the algorithms provided in Table~\ref{tab1}) in the reviewed literature. 

\subsection{Aerospace}
%\noindent\textcolor{White}{empty}  
\subsubsection*{\textbf{Drones/Quadrotors}}   Many of the aerospace applications falling under this review are concerned with \emph{unmanned aerial vehicles} (UAVs), usually of the quadcopter variety.  Specific studies also focus on particular maneuvers, for example, Koopman Eigenfunction Dynamic Mode Decomposition (KEEDMD) has been used for general quadrotor model generation \cite{folkestad:2020b} and, more specifically, to learn the nonlinear ground-effect to improve the speed and quality with which a multi-rotor aircraft may land \cite{folkestad:2020a}.   

Optimization of UAV flight has been explored using Dynamic Mode Decomposition (DMD) \cite{vasisht:2020} and DMD with Control (DMDc) \cite{jin:2020} in optimal control, and by the adjoint Koopman operator \cite{meyers:2019} for expected state propagation, with demonstrated advantages over stochastic control schemes. The adjoint Koopman operator in this literature refers to the left adjoint of the Frobenius-Perron operator.    

Several methods including DMD, Extended DMD (EDMD), bilinear EDMD (biEDMD) and Koopman Canonical Transform have been compared against each other on a planar quadrotor flight testbed, where the superiority of the Koopman Canonical Transform has been demonstrated in handling affine dynamics for nonlinear model predictive control (NMPC).   

Path planning using Robust Koopman Model Predictive Control (RK-MPC) has also been demonstrated in a quadrotor simulation \cite{mamakoukas:2022}. Optimal control for quadcopter stabilization has been demonstrated with models identified using EDMD \cite{berrueta:2020}. Finally, an artificial neural network (ANN)-based approach called Split Koopman Autoencoder has been used in the context of remote state monitoring of UAVs \cite{girgis:2021}, where the communication aspect of flight pertaining to radio frequency signal processing has been addressed.  

\subsubsection*{\textbf{Missiles/Hypersonic Regime}} A few other studies pertaining to the aeronautical domain have also been found to utilize Koopman operator-based methods. This includes the application of \emph{ballistic airdrop}, where the adjoint Koopman operator is used to determine the optimal air release point for ariel delivery to a specified ground target under parametric uncertainty \cite{leonard:2019}. Modeling of \emph{missile dynamics} from noisy data for model predictive control has also been undertaken using Sparse Identification of Nonlinear Dynamics (SINDy) and Stepwise Akaike Information Criteria (SAIC), where it has been shown to be superior  in comparison with state-feedback control \cite{matpan:2021}. 

Remaining the theme of \emph{supersonic flight}, model generation for aerodynamic flutter has been performed using Higher Order DMD (HODMD) to extract frequencies and damping from tests with reduced manual interaction and more robust aeroelastic analysis \cite{le:2019}. Further, in this flow regime, \emph{supersonic combustion ramjet} (ScramJet) engines are susceptible to an ``unstart condition'', which is when the airflow in a duct violently breaks down.  This phenomenon occurs when the pre-combustion shock train (PSCT) location translates upstream beyond the front of the inlet, causing flow separation within the engine and resultant shear layer oscillations. The detection and characterization of this condition relies on accurate modeling of flow characteristics which has been the objective of a study through the use of multi-resolution DMD (mrDMD), demonstrating the inadequacy of the regular DMD approach \cite{sullivan:2021}.  

For \emph{autonomous aircraft} and for \emph{rocket combustion} instability control, principle orthogonal decomposition-based DMD (POD-DMD) was used to simplify equations of motion with a reduced number of variables and selective sensitivity \cite{otto:2022}. POD-DMD is also found to have been used in the computational fluid dynamics (CFD) analysis of flow around an \emph{airfoil} in the sub/transonic regime, with the method increasing computational efficiency by three orders of magnitude while accuracy remained within 5\% as compared to other methods \cite{renganathan:2018}.  

Transitioning between air and space flight, upper and trans-atmospheric dynamics have posed a challenge due to the many environmental factors involved as well as vehicle controllability in what is usually the \emph{hypersonic flight regime}. For this situation, EDMD has been employed to identify a system model for optimal attitude control \cite{mi:2022}.  Similarly, the adjoint Koopman operator has also been used \cite{meyers:2021} to identify equations of motion through a linearly-constrained quadratic program to model atmospheric reentry.  

\subsubsection*{\textbf{Space Systems}} In terms of space system applications, the categories can again be divided into \emph{dynamics-related} (including traversing and maneuvering) and \emph{subsystem-related} (including propulsion). We have found the most application to be the minimum-fuel \emph{orbital rendezvous}.  One approach employed Koopman Map Inversion  to obtain a linearized model for optimal control \cite{servadio:2022a}, while another approach demonstrated Neural Koopman Lyapunov Control for linearizing a generalized affine system \cite{zinage:2022b}. Minimization of the Frobenius norm was performed on a similarly affine \emph{thrust-vectoring} application using the pseudo-inverse to directly solve for the Koopman operator \cite{zinage:2021}.   

In the study \cite{arnas:2022}, a linear model for \emph{zonal harmonics} around the moon was derived using Schur decomposition, rather than a singular value decomposition (SVD) approach to approximate solutions to perturbed ordinary differential equations. A related problem is \emph{lunar station-keeping}, namely for Lyapunov and Halo orbits in the circular-restricted three-body problem (CR3BP).  One study creatively obtained the Koopman operator approximation of the system matrix through direct computation using Legendre polynomials, which are already by their nature a complete and finite set of orthonormal basis in Hilbert space \cite{servadio:2022b}. See \cite{manzoor:2011} for more information on the CR3BP.  

Low thrust trajectory optimization in \emph{underactuated orbital flight} was addressed by a projection method onto vector fields defined by the input matrix \cite{hofmann:2022}. The authors of \cite{manzoor:2022b} demonstrated the extraction of system equations for a tethered subsatellite deployment maneuver subjected to unmodelled dynamics and disturbances using DMD and DMDc. The same objective was achieved by directly using Koopman-invariant observables \cite{chen:2020}, but is not always possible or practical for most problems. Further,  system equations of a \emph{lander} modeled as an inverted pendulum with stabilization thrusters below its center of gravity were derived  using EDMD \cite{bhandari:2021}.  EDMD was also used for a lunar lander in the dynamic allocation of control between a human driver and robot~\cite{broad:2020},~\cite{broad:2018}, referred to as model-based shared control (MbSC).  

On the subsystem side of space applications, DMD was used to find resonant frequencies, damping coefficient and mode shapes in a CFD simulation of a rocket engine's \emph{cryogenic swirl injector} \cite{luong:2021}, the critical flow rate at which vibration occurs, or \emph{``garden hose instability''} (commonly encountered in rocket engines), was investigated using Arnoldi iteration to attain Koopman modes \cite{canham:2017}. 

\begin{remark}[Identified Gap in the Literature]
Although the objective of the aforementioned aerospace-related studies has been limited to system identification, the ultimate goal of almost 37\% of the studies within this vehicle category (which is almost 46\% above all vehicle categories combined together) was to obtain equations of motion in a linear form for the purposes of control using model predictive control (MPC) or other state-space methods. As it can be seen from Fig. \ref{fig:VehCat},  Aerospace has been the largest vehicle class employing emerging Koopman operator-based methods as compared to any other vehicle class.  Despite this, the variety of aerospace vehicle types was found to be quite limited, mostly being small multi-rotor type UAVs.  For this vehicle class, we have not been able to find from the surveyed literature studies pertaining to helicopters and balloons.  
\label{remGap1} 
\end{remark} 

%\begin{figure*}[!t]
%	\centering
%		\includegraphics[height=0.6\textwidth]{figures/TimeLineWaqas.png}
%	\caption{Timeline of utilization of Koopman operator theory in various vehicular applications. The timeline also demonstrates in which year a given Koopman operator-based algorithm has first emerged (i.e., the first emergence of the algorithms provided in Table~\ref{tab1}) in the reviewed literature.}
%	\label{fig:TimeLine}   % This is the figure name for pointing throughout the document 
%\end{figure*}


\subsection{Automotive}

\subsubsection*{\textbf{Automobile Engines}} In terms of vehicle subsystems for combustion instability in internal combustion engines (ICE), the Hankel Alternative View of Koopman (HAVOK) method was employed for the prediction of \emph{pre-ignition and super-knock} from real-time peak-pressure data \cite{manzoor:2022a}, while the authors of \cite{ma:2022} employed a portion of the same method (although not by name) to describe the thermoacoustic oscillation characterizing the transition between chaotic states and limit cycles. Also for ICE engines, \emph{turbine dynamics} were investigated (e.g., in superchargers) using EDMD to model and predict turbulent and steady-state behavior \cite{zinage:2022a}.    

\subsubsection*{\textbf{EV Applications}} In terms of \emph{electric vehicles} (EVs), a linear model for motor control was extracted using DMD to actuate a permanent magnet synchronous motor through switching insulated-gate bipolar transistors (IGBTs) \cite{hanke:2018}. IGBTs are a common means to convert direct current (DC) from a battery to the appropriate coils within a motor to control speed in modern EVs. Similarly, an artificial neural network approach has been used to linearize a \emph{DC-DC converter model} for switching control \cite{maksakov:2020}.   Furthermore, linear data-driven predictors afforded by Koopman operator formalism have been utilized to formulate the eco-driving problem for electric vehicles  in a constrained quadratic program setting~\cite{gupta2022koopman},~\cite{shen2022data}. Additionally, data-driven design methods based on Koopman operator theory have been utilized to design X-in-the-loop environments for electrical vehicles~\cite{marco2021data}. Finally, there is a recent body of literature on Koopman operator-based state estimation/prediction and fault diagnosis for batteries that are widely used in electric vehicles~\cite{kanbur2020thermal},~\cite{moreno2023low}.  

\subsubsection*{\textbf{Automotive Model Identification and Control}} Model identification for \emph{nonlinear tire dynamics} using EDMD is investigated in both \cite{cibulka:2019} and \cite{vsvec:2021a}, with the former utilizing a \emph{single-track model} (making it applicable to motorcycles), while the latter further applies MPC control. Similarly, \cite{cibulka:2020} develops an MPC controller with a single-track model obtained by learning Koopman-invariant observables directly from data to recover the vehicle from a nonlinear state (e.g., skidding), when present. MPC is also used to minimize bounce by means of adjusting propulsive force in \cite{buzhardt:2022} while using EDMD for their model generation, thus becoming an alternative method for suspension control. Finally, handling and stability control (with linear time-varying MPC, or LTV-MPC) using \emph{torque vectoring} is explored by \cite{vsvec:2021b} using EDMD for model identification.   

\subsubsection*{\textbf{Autonomous  Vehicle Motion Control/ADAS Systems}} The remaining studies in this vehicle category pertain to motion control in autonomous or \emph{advanced driver-assist systems} (ADAS). In this group, artificial neural network methods were most prevalent.  Control for vehicle motion planning was enabled using Deep Direct Koopman (DDK) or variants in \cite{xiao:2021_i13}, \cite{xiao:2022} and \cite{wang:2021}, with the latter-most specifically applied to a case study dealing with optimal trajectory prediction in racing. Deep learning-based EDMD was employed in \cite{xiao:2022}, also for system identification in path tracking.  For vehicle-to-vehicle related optimized management of traffic comprised of autonomous vehicles, data-driven MPC (DMPC) was employed for coordinated movement \cite{zhan:2022} (e.g., through a controlled intersection) which they term as ``autonomous vehicle platooning''; here their focus is also on the comparison between centralized versus distributed controllers.  The final studies in this category all aim to also obtain a linear model for MPC design and have to do with lane-keeping employing Bilinear Koopman Realization \cite{yu:2022}, Koopman Tracking MPC (KTMPC) \cite{wang:2022}, and WOEDMD \cite{guo:2022} for Operator-AV shared control.   

\begin{remark}[Identified Gap in the Literature]
For this vehicle class, we have not been able to find from the surveyed literature studies pertaining to tracked (including tanks) and screw-propelled vehicles, vehicles otherwise specialized for travel over multiple terrains (e.g., snow, sand, grass, or semi-aquatic environments), as well as tractors, emerging e-mobility devices and other specialized vehicles. Relevant information for applications concerning rovers may be found in the robotics literature, presented in an upcoming section.   
\label{remGap2}  
\end{remark}

\subsection{Marine}


\subsubsection*{\textbf{Autonomous Marine Vehicles}} This vehicle category included some items which could have been categorized instead in the section for robotics, however, where the application dealt specifically with guidance, navigation or propulsion in water, it was considered to be a marine vehicle. This includes a robotic fish, where an EDMD-like algorithm was employed using high-order derivatives of physics-based functions of the state to linearize affine dynamics \cite{sinha:2016}. Similarly, the adjoint Koopman operator was used to improve the efficiency of a \emph{swimming robot} in a flow-like environment \cite{salam:2022}, where it could learn the dynamics of its environment. Finally, a robot was shown to follow a simulated river while avoiding probable locations of unsafe areas (navigation with probabilistic safety constraints) using Naturally Structured DMD (NSDMD)~\cite{pan:2021}, which is actually a modified EDMD algorithm.  

\subsubsection*{\textbf{Oceanic Applications}} The next common theme relates to \emph{oceanic applications}. In the context of an oil spill, oceanic flow was modeled using the adjoint Koopman operator to determine the optimal location for ships to release dispersant to control contaminants in a double-gyre fluid flow field \cite{sinha:2016}. Prediction of wind and oceanic flow patterns was also included in a review that surveyed the use of DMD and its variants \cite{schmid:2022}, including EDMD, Exact DMD, Debiased DMD (also known as forward- and backward-DMD, multiresolution DMD, Hankel DMD (also known as HAVOK), higher-order DMD (which includes derivatives of observable functions) and the adjoint Koopman operator.  However, there are not all applied to vehicular applications, yet is a valuable resource for one who seeks to find an appropriate Koopman operator-based method for a potential vehicular application. Finally, a dissertation by \cite{pan:2021} presents Time-delay DMD (TD-DMD), EDMD, Kernel DMD (KDMD) and Sparsity Promoting DMD (spDMD), and includes the application of model identification for the 3D turbulent air-wake of a ship.  

A unique study has also been found relating to the \emph{measurement of sea ice concentration}, which aims to detect exponentially decaying spatial modes in the Arctic and Antarctic oceans \cite{hogg:2020}. This study is an example of one explicitly relating to satellite data processing, however, there may be many others using Koopman operator-based methods extending into the area of remote sensing and geographic information systems (GIS) which are not within the scope of this survey. 

\begin{remark}[Identified Gap in the Literature]
For this vehicle class, we have not been able to find from the surveyed literature studies pertaining to hovercraft, submarines (including autonomous or remotely piloted underwater vehicles) and offshore platforms. Relevant information for applications concerning hydrofoils may be found in the CFD literature. 
\label{remGap3}  
\end{remark} 

\subsection{Mining}

\subsubsection*{\textbf{Hydraulic Fracturing}} For this category, the most common studies employing Koopman operator-based methods were found for the application of \emph{hydraulic fracturing}, which included \cite{narasingam:2020a}, \cite{klie:2019}, \cite{narasingam:2020b} and \cite{bao:2019}. However, these studies were deemed to fall outside the scope of this review due to their non-vehicular nature.  This is due to the fact that they focused largely on the detection of shale deposits with fixed drilling infrastructure. On the other hand, from a subsystem perspective, it may be somewhat appropriate to include processes enabling natural resource extraction through pipelines.  In that sense, Hankel-based DMD (HDMD) was used to model the \emph{multiphase flow dynamics} of an oil-gas slug and forecast hold-up time profiles \cite{ali:2021}. At the very least, this could have relevance to the operation of inspection/health-monitoring and cleaning vehicles that are typically used in pipelines.  

\subsubsection*{\textbf{Autonomous Excavation}}
Amongst other studies within this category, an \emph{autonomous excavation} application was found where Koopman operator-based system identification was performed using Duel Faceted Linearization for the selection of Koopman invariant observable variables \cite{sotiropoulos:2021}, whereafter an MPC control strategy was applied. Koopman Mode Decomposition (KMD) was used for identifying growing or decaying modes from traffic data and was shown to be superior in performance as compared to artificial intelligence methods \cite{avila:2017}. Finally, the aforementioned study on autonomous vehicle platooning \cite{zhan:2022} can arguably also belong in this category.  

\begin{remark}[Identified Gap in the Literature]
For this vehicle class, we have not been able to find from the surveyed literature studies pertaining to subterranean machines, such as those used for tunnel boring or directional drilling, landships, and elevators. 
\label{remGap4} 
\end{remark} 

\subsection{Traffic}
Traffic management was found to be an area of research where Koopman operator-based system identification techniques are being used. Given its distinctness from physical road vehicles, it has been assigned its own category. The majority of applications in this class of vehicle pertained to \emph{traffic signal phase timing}. In the studies by \cite{ling:2020} and \cite{ling:2018}, DMD was utilized for early identification of unstable queue growth, with the latter further proposing an adaptive traffic control system. The same objective was sought by \cite{lehmberg:2021}, but instead using EDMD to predict pedestrian traffic and an MPC controller for vehicle signaling in response to it.

\subsection{Robotics}


\subsubsection*{\textbf{Robotic Arms}} The operation of \emph{robotic arms} was found to be the most common application in this category and has been treated as a ``vehicle'' for the purposes of this survey as such devices are usually employed to spatially transport a payload from one point to another. For this purpose, EDMD was used (which they refer to as Koopman-MPC) to actuate the arm under voltage disturbance \cite{zhu:2022}. An aforementioned study from the Aerospace category \cite{jin:2020} also demonstrates DMDc and other approaches including ANN and Reinforcement Learning on a robotic arm.  

\subsubsection*{\textbf{Human-Robot Collaboration}} A modified form of EDMD using Autodidact Stiffness Learning was used for detection and adaptivity in applied torque for the human-machine interface of a manipulator (i.e., yoke controller) \cite{goyal:2022}. Similarly, end-effector motion of industrial robotic arms around humans requires environmental state prediction for safe path planning. This was done in one study where the Koopman operator was directly solved for by taking the pseudo-inverse involved in minimization of the Frobenius norm (a computationally expensive operation) \cite{sinha:2022}. The same objective of \emph{safe path planning} was also achieved in \cite{gutow:2020} with the use of the adjoint Koopman operator and in \cite{bujorianu:2021} using the Stochastic Koopman Operator. The latter study also cited an interesting application of their method for automated air traffic management but was not selected for inclusion in the Aerospace category given the lack of demonstration (i.e., simulation, physical experiment or substantive formulation).  

\subsubsection*{\textbf{Soft Robotics}} Other robotics-related applications found in the literature included a \emph{pneumatically actuated soft manipulator} which used EDMD for pick \& place operations for objects of unknown mass \cite{bruder:2021}. Underactuated control of the same type of robot was explored in \cite{haggerty:2020} using Hankel DMD (HDMD). An aforementioned study from the Aerospace category \cite{berrueta:2020} also involved control of a \emph{robotic ball} (called ``Sphero SPRK'') rolling in level sand to follow a predetermined trajectory, EDMD was used here. Similarly, \cite{folkestad:2020b} from the same category also included an example of a wheeled robot using a modified KEEDMD algorithm for mode unknown dynamics and improving computational efficiency.   

\subsubsection*{\textbf{Wheeled/Legged/Swimming Robots}} A unique study involving a wheeled robot utilized EDMD for increased computational efficiency and real-time implementation \cite{shi:2021}. Using \emph{jointed legs} to locomote is another means by which a robot may traverse; the authors of \cite{wu:2021} investigated such crawling and used DMD to expose the method's limitations. Marine robotic applications, as previously mentioned, include the robotic fish \cite{mamakoukas:2021} and obstacle-avoiding river-traversing robot using NSDMD \cite{moyalan:2022}. The former used an MPC controller to make the fish swim in a line or circle, with the linearized model obtained directly from an adaptive error minimization approach using Taylor series error bounds.  

\begin{remark}[Identified Gap in the Literature]
For this vehicle class, we have not been able to find from the surveyed literature studies pertaining to climbing or jumping vehicles, vehicles relying on peristaltic locomotion (see, e.g.,~\cite{Scheraga:2020} for such a robot prototype), attack or surveillance platforms, and robot swarms. Although, a unique biolocomotion study was indeed found to employ DMD to enable mapping between an upper limb and its contra-lateral lower limb while walking forward at constant speed \cite{boudali:2017}. This may arguably qualify as a mode of transportation (i.e., walking), and may very well apply to bipedal robots which are designed to walk like humans. 
\label{remGap5} 
\end{remark} 

\subsection{Rail}
Only one single study was found pertaining to this vehicle category, which was for an MPC application of a \emph{high-speed train} whose linearized model was obtained via EDMD \cite{chen:2021}.  

\begin{remark}[Identified Gap in the Literature]
For this vehicle class, we have not been able to find from the surveyed literature studies pertaining to trams, cable cars and roller coasters. It is important to note that factors surrounding the operation of vehicles or their subsystems were not discounted in the literature search.  For
example, the HAVOK algorithm’s predictive qualities may
 have potential in the areas of environmental forecasting (e.g.,
passenger load, wind and earthquake) and health monitoring
 (e.g., component mean time between failures), such that
vehicles could be operated with appropriate constraints during
 times of expected adverse conditions. 
\label{remGap6} 
\end{remark}