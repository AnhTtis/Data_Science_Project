\section{Introduction}
\label{sec:intro}

\IEEEPARstart{K}{oopman} operator theory is named after Bernard Koopman, who in the 1930s proved the premise that linear transformations of nonlinear dynamical systems exist when represented in Hilbert [function] space \cite{koopman:1931}. Historically, determining a Koopman-invariant subspace was accomplished by trial and error despite being unsuccessful for most dynamical systems. The enabling  engine for  modern-day data-driven applications of the Koopman operator theory are due to the profound insights on geometrical and statistical properties of dynamical systems in the Ph.D. dissertation~\cite{mezic1994geometrical} and the foundational line of work~\cite{mezic2005spectral,mezic2004comparison} on harmonic  analysis of the  Koopman operator.  Mezi\'{c}'s pioneering work~\cite{mezic1994geometrical,mezic2005spectral,mezic2004comparison} along  with   the  computational breakthroughs  based on singular value decomposition (SVD) enabled approximation of the Koopman operator from large amounts of data without relying on the pseudo-inversion of large non-square matrices. It is remarked that \emph{the first modern-day engineering application} of Koopman operator theory has been due to Mezi\'{c} and Banaszuk~\cite{mezic2004comparison} for model parameter identification in combustion rigs. 

\begin{figure}[t]
\centering
\includegraphics[width=0.4\textwidth]{figures/Timeline_area.png}
\caption{\small Publication timeline of the surveyed literature. The number of studies incorporating Koopman operator-based methods in smart mobility and vehicle engineering applications has been increasing nearly exponentially over an eight-year span thus far.}
\label{fig:Timeline_area}   % This is the figure name for pointing throughout the document 
\end{figure}


As demonstrated in Figure~\ref{fig:Timeline_area}, \emph{the first vehicular applications} started to emerge only six years after the computational breakthrough due to the Dynamic Mode Decomposition (DMD) technique was achieved in 2008~\cite{schmid:2008}. Among vehicular applications alone, it is evident from Figure~\ref{fig:Timeline_area} that the number of studies incorporating Koopman operator-based methods has been increasing nearly exponentially over an eight-year time span since 2014. However, with a mere maximum of 30 such studies published so far in a given year, the likely trend is further exponential growth as more researchers become aware of the associated algorithms, and as the applicability of such algorithms simultaneously evolves. 

\noindent \textbf{Features of the Current Survey Paper.} The purely data-driven nature of Koopman operators holds the promise of capturing unknown and complex dynamics for reduced-order model generation and system identification, through which the rich machinery of linear control techniques can be utilized. The emergent nature of the smart mobility and vehicular-related applications, where  the Koopman operator  in each particular application needs to be approximated, implies that the development of various Koopman operator approximation  algorithms is expected to grow along with the vehicular problems they aim to solve.  Given the ongoing development of this research area and the many existing open problems in the fields of smart mobility and vehicle engineering, a survey of techniques and open challenges of applying Koopman operator theory to this vibrant area is warranted.  To the best of our knowledge, this survey paper is the \emph{first of its kind} reviewing the applications of Koopman operator theory within a focused research area, namely, smart mobility and vehicle engineering applications. A \emph{notable feature} of our survey paper is reviewing and categorizing the results of over 100 research papers based on both application and algorithm type  (see Table~\ref{tab1} and Section~\ref{sec:vehicApp}) that are concerned with the applications of Koopman operator theory to the field of smart mobility and vehicular engineering. Such a \emph{comprehensive and  detailed categorization} will be beneficial to the research practitioners working in the field.  Furthermore, this review paper discusses theoretical aspects of Koopman operator theory that have been largely neglected by the smart mobility and vehicle engineering community and yet have large potential for contributing to solving open problems in these areas. Additionally, our survey paper seeks to \emph{identify gaps} in the smart mobility and vehicle engineering research where new and existing Koopman operator-based methods have the potential to further develop and address unsolved problems  potentially benefiting from the perspectives of nonlinear system identification, control, global linearization, and the predictive powers that Koopman operator theory has to offer (see, e.g., Remarks~\ref{remGap1}--\ref{remGap6}). 

\begin{table}[t]

\renewcommand{\arraystretch}{1.5} 
\resizebox{\columnwidth}{!}{
\begin{tabular}{c|ccccc}
\hline
\textbf{Language}                                                                     & \textbf{En} & \textbf{De} & \textbf{Es} & \textbf{Fr} & \textbf{Ja} \\ \hline
\textbf{Datasize}                                                                     & 1,000        & 3,000        & 3000        & 3,000        & 3,000        \\ \hline
\textbf{\begin{tabular}[c]{@{}c@{}}Sentence length\end{tabular}}         & 23.42      & 19.42      & 23.21      & 22.09      & 48.39       \\ \hline
\textbf{\begin{tabular}[c]{@{}c@{}}\# of critical objects \end{tabular}} & 2.87       & 3.56       & 4.17       & 3.11       & 4.42       \\ \hline
\textbf{\begin{tabular}[c]{@{}c@{}}\# of backgrounds\end{tabular}}      & 1.25        & 1.27       & 1.39       & 1.25       & 1.56       \\ \hline
\textbf{\begin{tabular}[c]{@{}c@{}}\# of relationships\end{tabular}}    & 1.57       & 2.05       & 1.78       & 1.81       & 2.46       \\ \hline
\end{tabular}
}
\caption{\dataset~statistics}
\label{table1}
\end{table}



The rest of this paper is organized as follows.  After presenting the relevant taxonomy in  Section~\ref{sec:genTax}, we provide a brief overview  the basic underpinnings of the Koopman operator theory in Section~\ref{sec:briefOver}. The literature review with categorized vehicular applications is presented in  Section~\ref{sec:vehicApp}, where each subsection concludes with a list of open research questions for the application of Koopman operator-based methods in terms of vehicle types not encountered in the literature.   Other relevant applications, which are not explicitly vehicular in nature, and theoretical/algorithmic variations are reviewed in  Section~\ref{sec:vehicApp2}.   

\begin{remark}
A survey paper on variants of DMD authored by Chen \emph{et al.}~\cite{chen:2012} had been published in 2012. However, being conducted a decade ago, it was well before the emergence of Koopman operator vehicular applications in the literature. Nevertheless, a very interesting discussion on the optimal application of DMD can be found in that paper. Another influential review in the field of Koopman operator theory is due to Budi{\v{s}}i{\'c} \emph{et al.}~\cite{budivsic2012applied}.   Over the course of developing this paper, Schmid~\cite{schmid:2022}, who is the pioneer of the original mode decomposition method, has also published his own survey on the variants of his method found in the literature. However, Schmid's survey is not focused on the specific topic of smart mobility and vehicle engineering applications. Finally, we remark that while Schmid's survey paper is focused on DMD and its variants, a complete review on Koopman operator methods in fluid mechanics due to Mezi\'{c} has been presented in~\cite{mezic2013analysis}.
\label{rem:features}
\end{remark}