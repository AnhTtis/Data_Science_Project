\documentclass[journal]{IEEEtran}

\usepackage{graphicx}           % this is required for specific image positioning commands
\usepackage[dvipsnames]{xcolor} % this is required for \textcolor to work
\usepackage{amsmath}         % this is required for symbol used to represent Koopman operator
\usepackage{amsfonts}        % this is required for \mathbb{R} to work 
\usepackage{float}           % this is required for better table positioning
\usepackage{xcolor,colortbl} % this is required for \definecolor to work
\newtheorem{remark}{Remark}  % this is required for numbered Remarks
\usepackage{arydshln}        % this is for the \hline alternative in the table

\hyphenation{appli-cations app-lied} % correct bad hyphenation here

% \def\thesubsubsectiondis{\unskip\arabic{subsubsection}.}    % this is required to disable indent in \subsubsection

\begin{document}

\title{Vehicular Applications of Koopman Operator Theory -- A Survey}

\author{\IEEEauthorblockN{Waqas A. Manzoor$^{1,2}$}
	\and
	\IEEEauthorblockN{Samir A. Rawashdeh$^{2}$}
	\and
	\IEEEauthorblockN{Alireza Mohammadi$^{2}$}\\
	Ford Motor Company, University of Michigan
	%
	\thanks{$^1$Dept. of Systems Engineering \& Validation, Ford Motor Company, Dearborn, MI, USA. $^2$ Dept. of Electrical and Computer Engineering, University of Michigan, Dearborn, MI, 
	USA. Corresponding Author: A. Mohammadi, e-mail: \texttt{amohmmad@umich.edu}}
	
	%Waqas A. Manzoor$^{1,2}$, %~\IEEEmembership{Member,~IEEE,} % ORCID ID: 0000-0002-7080-3998
        %Samir Rawashdeh$^2$,
        %and~Alireza Mohammadi$^{\dagger,2}$
%\thanks{$^1$Dept. of Systems Engineering \& Validation, Ford Motor Company, \\20901 Oakwood, Dearborn, MI, USA 48124\\$^2$Dept. of Electrical and Computer Engineering, %University of Michigan - Dearborn, 4901 Evergreen Rd, Dearborn, MI, USA 48128\\
%$^\dagger$Corresponding author's e-mail: amohmmad@umich.edu\\
%Manuscript received Month 00, 2022; revised Month 00, 2022.}
}

% The paper headers
\markboth{IEEE Access,~Vol.~XX, No.~X, January~2023}%
{Shell \MakeLowercase{\textit{et al.}}: Koopman Vehicle Survey}  % Focuses are on land, airborne, and maritime

\maketitle

\begin{abstract}
Koopman operator theory has proven to be a promising approach to nonlinear system identification and global linearization. For nearly a century, there had been no efficient means of calculating the Koopman operator for applied engineering purposes. The introduction of a recent computationally efficient method in the context of fluid dynamics, which is based on the system dynamics decomposition to a set of normal modes in descending order,  has overcome this long-lasting computational obstacle. The purely data-driven nature of Koopman operators holds the promise of capturing unknown and complex dynamics for reduced-order model generation and system identification, through which the rich machinery of linear control techniques can be utilized. Given the ongoing development of this research area and the many existing open problems in the fields of smart mobility and vehicle engineering, a survey of techniques and open challenges of applying Koopman operator theory to this vibrant area is warranted. This review focuses on the various solutions of the Koopman operator which have emerged in recent years, particularly those focusing on mobility applications, ranging from characterization and component-level control operations to vehicle performance and fleet management. Moreover, this comprehensive review of over 100 research papers highlights  the breadth of ways Koopman operator theory has been applied to various vehicular applications with a detailed categorization  of the applied Koopman operator-based algorithm type. Furthermore, this review paper discusses theoretical aspects of Koopman operator theory that have been largely neglected by the smart mobility and vehicle engineering community and yet have large potential for contributing to solving open problems in these areas. 

%Koopman operator theory has proven to be a promising approach to nonlinear system identification and global linearization. For nearly a century, there had been no efficient means of calculating the Koopman operator, which can effectively be represented by a state transition matrix, for general applications. Recently, a computationally efficient method based on the decomposition of a system’s dynamics to a set of normal modes in descending order has emerged in the context of fluid dynamics. The purely data-driven nature of Koopman operators holds the promise that they can capture unknown and complex dynamics for reduced-order model generation and system identification, through which the rich machinery of linear control techniques can be utilized. Given the ongoing development of this research area and the many existing open problems in the fields of smart mobility and vehicle engineering, a comprehensive survey of techniques and open challenges of applying Koopman operator theory to this vibrant area is warranted. This review focuses on the various solutions of the Koopman operator which have emerged in recent years, particularly those focusing on vehicular applications, ranging from characterization and control at the component-level operations to vehicle performance and fleet management. Furthermore, this comprehensive review highlights the breadth of ways Koopman operator theory has been applied to mobility applications, ranging from the original mode decomposition algorithm to predictive and neural network-based techniques. Furthermore, we discuss certain theoretical aspects of Koopman operator theory that have been largely neglected by the smart mobility and vehicle engineering community and yet have large potential for contributing to solving open problems in these areas. A notable feature of our survey paper is categorizing the results of over 100 research papers that are concerned with the applications of Koopman operator theory to the field of smart mobility and vehicular engineering. 
\end{abstract}

\begin{IEEEkeywords}
Koopman operator, global linearization, system identification, nonlinear control
\end{IEEEkeywords}

\IEEEpeerreviewmaketitle