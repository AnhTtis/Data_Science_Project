\section{General Taxonomy and Vehicle Type Categorization}
\label{sec:genTax}

For the purposes of this survey, we generally  consider a \emph{vehicle} to be any man-made instrument that can carry a payload, including occupants, equipment, sensors or any other items. In some contexts, the item may be its own presence. Most generally, vehicle has also been defined as any mechanized equipment. All these ideas have been captured in Merriam-Webster's formal definitions \cite{merriamwebster:vehicle}. In this paper, the physical type of vehicle according to Merriam-Webster  is considered. This physical type includes systems and processes associated with automotive, aerospace, marine, rail, robotic, and subterranean classes of vehicles, along with some interfacing or noteworthy classes such as biolocomotion, communication, and traffic management, amongst others.

%\begin{table}[t]
\caption{We compare our featuremetric refinement method using the proposed \textbf{NeFeS} network with photometric-based refinement baselines on \textit{Cambridge Hospital}.}
\label{supp:table:photometric-refinement}
\centering
% \resizebox{\linewidth}{!}{
\begin{tabular}{lc}
\toprule
Methods & Hospital \\
\midrule
DFNet                                                   & 2.00m/2.98$\degree$\\
DFNet + Sparse NeRF photometric$_{50}$                         & 1.19m/1.52$\degree$ \\
DFNet + Dense NeRF photometric$_{50}$                         & 0.80m/1.12$\degree$ \\
\textbf{DFNet + }$\textbf{NeFeS}_{\textbf{50}}$         & \boldred{0.55m}/\boldred{0.90$\degree$} \\
\bottomrule
\end{tabular}
% }
\end{table}




The \emph{motivation} to focus on vehicular applications stems from the fact that many processes and subsystems are not easily modeled to a sufficient level of fidelity and/or are subject to a significant level of noise/disturbances.  Such conditions pose limitations on the performance, control and overall utilization of the processes and subsystems that transduce energy into motion. Vehicles of different types, such as aircraft and automobiles, may further share common subsystems (e.g., combustion chamber or pump) or types of maneuvers (e.g., braking or collision avoidance).  Consequently, from a dynamics and controls standpoint, it is sufficient to maintain the scope of this study to include all major vehicle categories.   



Of all the studies found in this survey, Figure \ref{fig:VehCat} (the pie chart on the left) illustrates their proportions in terms of the  type of vehicles represented.  These include aerospace, automotive, marine, mining, traffic, robotics and rail vehicles. Also included are some studies that are theoretical only in terms of presenting a novel Koopman operator-based algorithm or those of general relevance (e.g., pertaining to a generic subsystem or component of multiple possible vehicles) without substantially demonstrating it on any vehicle, whether in simulation, in-vehicle, or on a hardware-in-the-loop test bench.  


Figure \ref{fig:VehCat} (the pie chart on the right) further breaks down the proportions of studies that pertain to specific types of functions, rather than vehicle platforms.  These include traverse, maneuver, subsystem and guidance, as defined in what follows. 
 
\begin{itemize}
    \item \textbf{Traverse:} refers to the macro-scale function of a vehicle moving from one point to another (e.g., orbiting). 
    \item \textbf{Maneuver:} refers to a specific mission, operation, reconfiguration or change in situation a vehicle may undertake within its journey (e.g., docking). 
    \item \textbf{Subsystem:} refers to the subject of the study focusing on a component or set of components and their specific operation (e.g., a battery). 
    \item \textbf{Guidance:} refers to the navigational aspect of vehicle path-finding and maintenance, correction or modification of trajectory (e.g., obstacle avoidance). 
\end{itemize}


Finally, ``Traffic Management" is concerned with the coordinated motion of multiple vehicles. Given the uniqueness of certain problems arising in traffic management, we have decided to segregate such studies into their own category.  

\begin{remark}[Machine Learning Community Taxonomy]
With the mainstreaming of Koopman operator-based methods, there also seems to be a linguistic generalization in that the term ``Koopman" or ``Koopman model" is increasingly used to describe any finite state transition matrix approximated for an unmodeled or nonlinear system. This is especially true in the artificial intelligence and machine learning community (see, e.g., \cite{balakrishnan:2021}). Such use may continue to uphold validity since Koopman operator theory remains one of the main formal justifications for utilizing linear state transition matrices for closely capturing the behavior of nonlinear dynamical systems by means of proper linearization. 
\label{rem:MLkoopman}
\end{remark}

%
\begin{table*}[h]
\tabcolsep7.5pt
\caption{Binding energy (BE) of carbon-chain species}
\label{tab:Ebind}
\begin{center}
\begin{tabular}{cc|cc|cc}
\hline
Species & BE (K) & Species& BE (K) & Species& BE (K) \\
\hline
%HC$_3$N &\\
%HC$_3$N &\\
C$_2$& 10000& C$_8$N&7200&C$_3$O&2750 \\
%C$_{2}$&10000$^{b}$\\
C$_3$&2500&C$_9$N&8000&C$_5$O&4350 \\
%C$_{3}$&2500$^{b}$&&\\
C$_{4}$&3200&C$_{10}$N&8800 & C$_7$O&5950  \\ 
C$_5$&4000&C$_2$H$_2$&2587 &C$_9$O&7550 \\
C$_6$&4800&C$_2$H$_4$&2500 & HC$_2$O&2400 \\
C$_7$&5600&C$_2$H$_5$&3100 &SiC$_2$&4300 \\
%C$_{5}$&4000$^{b}$&&\\
%C$_{6}$&4800$^{b}$&&\\
%C$_{7}$&5600$^{b}$&&\\
C$_{8}$&6400&C$_2$H$_6$&1600 &SiC$_3$&5100 \\
C$_{9}$&7200&C$_4$H$_2$&4187& SiC$_4$&5900 \\
C$_{10}$&8000&C$_5$H$_2$&4987 \\
C$_{11}$&9600&C$_6$H$_2$&5787\\
C$_2$H&3000&C$_7$H$_2$&6587\\
$l$-C$_3$H&4000&C$_2$P&4300 & \\
$c$-C$_3$H&5200&C$_3$P&5900\\
C$_4$H&3737&C$_4$P&7500\\
C$_5$H&4537&C$_2$S&2700\\
C$_6$H&5337&C$_3$S&3500\\
C$_7$H&6137&C$_4$S&4300\\
C$_8$H&6937&HC$_3$N&4580\\
$c$-$\rm{C_3H_2}$&5900&HC$_4$N&5380\\
C$_2$N&2400&HC$_5$N&6180\\
C$_3$N&3200&HC$_6$N&7780\\
C$_4$N&4000&HC$_7$N&7780\\
C$_5$N&4800&HC$_8$N&9380\\
C$_6$N&5600&HC$_9$N&9380\\
C$_7$N&6400&C$_2$O&1950\\

%HC$_3$O&3111$^{b}$

%C$_7$N&6400$^{a}$&H$_2$C$_3$N&3133$^{b}$\\
%C$_8$N&7200$^{b}$&\\
%C$_9$N&8000$^{b}$&\\
%C$_2$O&1950$^{b}$&&\\
%C$_3$O&4208$^{a}$&&\\
%C$_5$O&4350$^{b}$&&\\
%C$_7$O&5950$^{b}$&&\\
%C$_9$O&7550$^{b}$&&\\
% &&C$_2$S&2943$^{a}$\\
% &&C$_3$S&3500$^{b}$\\
% &&C$_4$S&4300$^{b}$\\
% &&HC$_3$N&3475$^{a}$\\
% &&HC$_4$N&5380$^{b}$\\
% &&HC$_5$N&6180$^{b}$\\
% &&HC$_6$N&7780$^{b}$\\
% &&HC$_7$N&7780(it should change)$^{b}$\\
% &&HC$_8$N&9380$^{b}$\\
% &&HC$_9$N&9380(it should change)$^{b}$\\
% &&HC$_2$O&2400$^{b}$\\
% &&HC$_3$O&3111$^{a}$\\
% &&SiC$_2$&4300$^{b}$\\
% &&SiC$_3$&5100$^{b}$\\
% &&SiC$_4$&5900$^{b}$\\
% %&&\\
\hline
\end{tabular}
\end{center}
%\begin{tabnote}
{Taken from the KIDA (\url{https://kida.astrochem-tools.org/}), and also see \citet{wake17}, \citet{pent17}, \citet{das18}.}
% higher-order-chain values are estimated with lower order chain plus one carbon atom's BE (KIDA has used 800 K for these estimations since updated binding energy of C atom is 1300 K, thus need to update those values. HC$_{2n+1}$ show different binding energy, however, we think the BE of HC$_4$N, HC$_6$N, and HC$_8$N needs to revisit. Especially for C$_n$P group, C$_n$+P is the rule for the estimation of binding energy. Also, include BE values of HC$_n$O ($n>4$) following a similar carbon addition method. Check some estimation if possible for recently detected higher-order carbon chains.
%\end{tabnote}
\end{table*} 


\begin{remark}[A Brief Note on CFD studies]
The inclusion of computational fluid dynamics (CFD) studies has generally been avoided in this review. The only exceptions are the studies containing an explicit vehicular application or proposing a new type of Koopman operator-based system identification method/variant. This is because the DMD solution to the Koopman operator was itself first derived in the very context of CFD (see the seminal work by  Schmid~\cite{schmid:2010}) and has since had the most time to mature in the CFD literature. Many such CFD-centric studies involve a generic case study of flow past a cylinder or airfoil, which may have relevance to, e.g., lifting surfaces, screw propellers, or more generically, pumps and turbines.  Thus, the inclusion of literature from the CFD domain poses a vast grey area, often with speculative applicability.  For example, residual DMD (resDMD) was used in a CFD-focused study \cite{colbrook:2022} for supersonic plasma discharge but has significant relevancy to satellite propulsion systems. To narrow the scope of the search, all literature pertaining to the fluid dynamics realm has been excluded, other than those with explicit vehicular applications, or those which identify a novel algorithm (in which case the corresponding study was grouped into the `Theory' category). This takes away a major source of ambiguity, given that much of the pure fluid dynamics literature is generalized (e.g., flow past a cylinder) such that it may or may not be relevant to vehicle motion.  
\label{rem:CFD}
\end{remark}

\begin{figure*}[t]
	\centering
	\begin{minipage}{0.3\textwidth}
		\includegraphics[width=0.95\textwidth]{figures/VehCat.png}
	\end{minipage}
    %
    \hspace{6ex}
    \begin{minipage}{0.3\textwidth}
    	\vspace{-2ex}
    	\includegraphics[width=0.95\textwidth]{figures/FcnCat.png}
    \end{minipage}
	%
	\caption{Vehicle categories (left) and function categories (right) of the surveyed smart mobility and 
		vehicular engineering literature. The area of each piece in the pie charts is proportional to the ratio of 
		the number of the conducted studies within each particular area to the number of total studies.}
	\label{fig:VehCat}   % This is the figure name for pointing throughout the document 
\end{figure*}

%\begin{figure}[t]
%\centering
%\includegraphics[width=0.25\textwidth]{figures/}
%\caption{Function categories of the surveyed literature.}
%\label{fig:FcnCat}   % This is the figure name for pointing throughout the document 
%\end{figure}
%  