\section{Literature Review: Vehicle-related \& Other Relevant Studies}
\label{sec:vehicApp2}

In this section we provide an overview of other relevant applications, which are not explicitly vehicular in nature, and theoretical/algorithmic variations of the Koopman operator framework that might be beneficial for future applications in the area of smart mobility and vehicular engineering. 

\subsection{General Studies Applicable to Vehicles}
Studies focused on fluid flow are among the most common vehicle-related research topics where Koopman operator theory has played an integral role. Using DMD, the modeling of nonlinear oscillations due to vortex shedding was investigated in \cite{le:2015} and is highly relevant for aeronautical applications.  Also relevant to aeronautical engines that operate in the transonic regime is the use of DMD to model separated and turbulent flow within a convergent-divergent nozzle \cite{leask:2021}, which manifests as the gas path of gas turbine engines. Most internal and external combustion engines also rely on liquid [fuel] injection, for which \cite{leask:2021} is highly relevant as it demonstrates the modeling, prediction and control of nonlinear flow associated with atomization dynamics, enables by SMS and deep convolutional Koopman network (CKN). Similarly, MPC control of nonlinear fluid flow was demonstrated in \cite{bieker:2020} using a deep learning approach. 

The second most common research area found applicable to this category were studies pertaining to motor control, which is especially relevant to UAVs and robots, but potentially also to other types of vehicles when examining them from a subsystem perspective.  This was achieved in \cite{yingzhao:2020} using Gaussian process-based Koopman operator in robust controller design, and in \cite{calderon:2022} using EDMD for current control for the synchronous operation of motors. Finally, a power management study used Stochastic Adversarial Koopman Operator with Auxillary Neural Network for the quick learning of reduced order models that measure the state of charge of Lithium-ion batteries. Potentially applicable to some aircraft and specialized ground vehicles, one study used DMD in the diagnostics of natural gas rotating detonation engines \cite{journell:2020}. 

\subsection{Theoretical Issues with Potential Applications to Smart Mobility and Vehicular Engineering}
\label{subsec:theorIssue}
This section introduces some studies which are theoretically focused on the derivation of unique Koopman operator-based techniques but have not been utilized in the application-focused literature. They are included in this review due to their potential for any future applications the reader may be motivated towards. Firstly, \cite{cohen:2021} uses DMD and rescaled DMD (rDMD) for image processing. Also relating to images, \cite{xiao:2021_i31} used Deterministic and Convolutional Koopman Networks (DCKNet and CKNet, respectively) to predict a suitable trajectory from a provided topography to solve the standard Mountain Car Problem.  This may have relevance to energy-limited adaptive cruise control applications in the automotive category. Linearized, reduced order models in \cite{baddoo:2021} are identified using Physics-informed DMD (piDMD), while \cite{qian:2020} similarly makes a case for physics preservation but utilizes a projection-based model reduction approach. Examples in the former include channel flow and flow past a cylinder, which may be relevant to Marine vehicles, but does not explicitly specify such. The inverted pendulum model is generated in \cite{han:2020} using Deep Neural Network-based Koopman (Koopman DNN), with \cite{bakker:2020} also employing an ANN approach, and \cite{songy:2021} combining ANN with accelerated learning using Deep Koopman Reinforcement Learning (DKRL). Finally, an ANN-based Exact DMD approach is also presented in \cite{li:2021} in comparison with EDMD and LIR-DMD for the development of multi-scale models from coarse data with long-range prediction.  

EDMD and Stochastic Koopman Operator (SKO) are used in \cite{williams:2015}, while \cite{williams:2016} presents a modified EDMD. It has been noted that throughout this survey, EDMD was the most used method in a modified form. Sparse Identification of Nonlinear Dynamics with Control (SINDYc) has been developed in \cite{williams:2015}, and can potentially be applied anywhere DMDc has been used (e.g., in \cite{manzoor:2022b} for tethered subsatellite deployment), although the paper demonstrates its application on a predator-prey model and the Lorenz system. An example in \cite{zhang:2022} applies Robust Tube-based MPC with Koopman (r-KMPC) on the Van der Pol Oscillator, which may have relevance to applications in wireless communication, among other areas. Finally, an interesting application of auto-tuning (i.e., model evaluation) using DMD was presented in \cite{avila:2021} in the context of a zero-sum game.  This may have potential applications in the balancing of parameters and fuzzy criteria in the realm of AV aggressiveness and wargaming.  