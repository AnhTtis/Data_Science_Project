\documentclass[./../../paper.tex]{subfiles}
\graphicspath{{\subfix{./../../figures/}}}

\begin{document}
    The remainder of the thesis is outlined as follows: In \autoref{ch:prereq}, we introduce all of the important concepts that are crucial to this thesis. Most importantly, we introduce the main research discipline \Gls{PM} and the subject of our research: \emph{Counterfactuals}. Furthermore we cover some necessary background required to understand the methods, we employ.
    

    The \autoref{ch:methods}, introduces our methodological framework in further detail. The chapter explains all the important components and methods, we apply, to answer the research question. Among these methods, we introduce the methodological architecture, a modified version of the \Gls{damerau_levenshtein}.

    \autoref{ch:evaluation} covers the main approach behind our experimental setup. We discuss how we attempt to answer our research questions and introduce the datasets we are using and how we conduct the preprocessing. 

    In \autoref{ch:results} we report on the results and insights we gain from executing our research approach. 
    
    All the results are summarised in \autoref{ch:discussion}. Here, we summarize and interpret our results. We discuss limitations and possible improvements. We also discuss implications for future research endeavors. 

    The \autoref{ch:conclusion} summarizes the thesis and the implications for the \Gls{PM} research field.


\end{document}