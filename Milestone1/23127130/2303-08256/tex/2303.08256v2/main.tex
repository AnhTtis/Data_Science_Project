%\documentclass[sigconf]{acmart}[9pt]
\documentclass[conference,9pt]{IEEEtran}

\AtBeginDocument{%
  \providecommand\BibTeX{{%
    \normalfont B\kern-0.5em{\scshape i\kern-0.25em b}\kern-0.8em\TeX}}}
\usepackage{cite}
%\setcopyright{none}
%\settopmatter{printacmref=false} % Removes citation information below abstract
%\renewcommand\footnotetextcopyrightpermission[1]{} % removes footnote with conference 

% \setcopyright{acmcopyright}
% \copyrightyear{2018}
% \acmYear{2018}
% \acmDOI{XXXXXXX.XXXXXXX}

\usepackage{url}
\usepackage{enumitem}

\usepackage[utf8]{inputenc} % allow utf-8 input
\usepackage[T1]{fontenc}    % use 8-bit T1 fonts
\usepackage{url}            % simple URL typesetting
\usepackage{booktabs}       % professional-quality tables
\usepackage{amsfonts}       % blackboard math symbols
\usepackage{amsmath}
\usepackage{nicefrac}       % compact symbols for 1/2, etc.
\usepackage{microtype}      % microtypography
\usepackage{xcolor}         % colors
\usepackage{colortbl}
%\usepackage{subfigure}
\usepackage{subcaption}
\usepackage{graphicx}
\usepackage{svg}
\usepackage{multirow}
\usepackage{wrapfig}
\usepackage{threeparttable}
\usepackage{tikz}
\usepackage{makecell}
\usepackage{tablefootnote}
\usepackage[hidelinks]{hyperref}
\usepackage{dblfloatfix}

\usepackage{color}
%\settopmatter{printacmref=False}
\pagenumbering{gobble}


\begin{document}
\pagestyle{plain}
%%
%% The "title" command has an optional parameter,
%% allowing the author to define a "short title" to be used in page headers.
\title{\textsc{Gamora}: \underline{G}raph Le\underline{a}rning based Sy\underline{m}bolic Reas\underline{o}ning for La\underline{r}ge-Scale Boole\underline{a}n Networks}

%%
%% The "author" command and its associated commands are used to define
%% the authors and their affiliations.
%% Of note is the shared affiliation of the first two authors, and the
%% "authornote" and "authornotemark" commands
%% used to denote shared contribution to the research.

\author{Nan Wu$^1$, Yingjie Li$^2$, Cong Hao$^3$, Steve Dai$^4$, Cunxi Yu*$^2$, Yuan Xie$^5$
\\$^1$University of California, Santa Barbara, $^2$University of Utah, $^3$Georgia Institute of Technology,\\ $^4$NVIDIA, $^5$Alibaba DAMO Academy\\ 
nanwu@ucsb.edu, yingjie.li@utah.edu, callie.hao@gatech.edu, sdai@nvidia.com, yuanxie@gmail.com\\
*correspondence: cunxi.yu@utah.edu}

%\affiliation{%
%   \institution{The Th{\o}rv{\"a}ld Group}
%   \streetaddress{1 Th{\o}rv{\"a}ld Circle}
%   \city{Hekla}
%   \country{Iceland}}
% \email{larst@affiliation.org}

%\renewcommand{\shortauthors}{Trovato and Tobin, et al.}
\newcommand{\red}[1]{\textcolor{red}{#1}}


\newcommand{\cy}[1]{\textcolor{violet}{#1}}


\maketitle

\begin{abstract}
Reasoning high-level abstractions from bit-blasted Boolean networks (BNs) such as gate-level netlists can significantly benefit functional verification, logic minimization, datapath synthesis, malicious logic identification, etc. 
Mostly, conventional reasoning approaches leverage structural hashing and functional propagation, suffering from limited scalability and inefficient usage of modern computing power.
In response, we propose a novel symbolic reasoning framework exploiting graph neural networks (GNNs) and GPU acceleration to reason high-level functional blocks from gate-level netlists, namely \textsc{Gamora},
%\footnote{Gamora is a fictional character in American comic books, with superhuman strength and agility.}, 
which offers {\bf high reasoning performance} w.r.t exact reasoning algorithms, strong {\bf scalability} to BNs with over 33 million nodes, and {\bf generalization capability} from simple to complex designs. 
To further demonstrate the capability of \textsc{Gamora}, we also evaluate its reasoning performance after various technology mapping options, since technology-dependent optimizations are known to make functional reasoning much more challenging.
Experimental results show that (1) \textsc{Gamora} reaches almost 100\% and over 97\% reasoning accuracy for carry-save-array (CSA) and Booth-encoded multipliers, respectively, with up to six orders of magnitude speedups compared to the state-of-the-art implementation in the ABC framework; 
(2) \textsc{Gamora} maintains high reasoning accuracy ($>$92\%) in finding functional modules after complex technology mapping, and we comprehensively analyze the impacts on \textsc{Gamora} reasoning from technology mapping.
\textsc{Gamora} is available at \textcolor{teal}{\url{https://github.com/Yu-Utah/Gamora}}.

%\footnote{This work will appear at 60th Design Automation Conference (DAC'23)}.
% Gamora is further evaluated with various technology mapping options, where technology-dependent optimizations are known to make functional reasoning much more challenging. Experimental results offer comprehensive insights of impacts to Gamora reasoning w.r.t technology mapping, and maintains high reasoning accuracy ($>$92\%) in finding functional modules after complex technology mapping. 
\end{abstract}

%%
%% The code below is generated by the tool at http://dl.acm.org/ccs.cfm.
%% Please copy and paste the code instead of the example below.
%%
% \begin{CCSXML}

% \end{CCSXML}

% \ccsdesc[500]{Computer systems organization~Embedded systems}
% \ccsdesc[300]{Computer systems organization~Redundancy}
% \ccsdesc{Computer systems organization~Robotics}
% \ccsdesc[100]{Networks~Network reliability}

%%
%% Keywords. The author(s) should pick words that accurately describe
%% the work being presented. Separate the keywords with commas.
%\keywords{datasets, neural networks, gaze detection, text tagging}



\section{Introduction}
\label{sec:intro}
\begin{figure}[t]
\begin{center}
    \includegraphics[width=1\linewidth]{figures/teaser.pdf}
\end{center}
\vspace{-0.1in}
\caption{\textbf{{\em Foggy} vs {\em Clear} NeRF.} Our \ournerf gets rid of reconstruction errors manifested as foggy ``floaters" in the density volume without additional input or significant computational overhead. 
%
Below are density profiles along a given ray before and after our geometry correction procedure, where we discard density peaks corresponding to floaters.
}
\label{fig:teaser}
\vspace{-0.2in}
\end{figure}



%The emergence of 
Neural Radiance Fields (NeRFs)~\cite{mildenhall2020nerf}  %and its variants 
have made revolutionary contributions in %photo-realistic 
novel view synthesis~\cite{barron2021mip,barron2022mip}, 
autonomous driving~\cite{rematas2022urban,tancik2022block}, digital human~\cite{hong2022headnerf,zhao2022humannerf}, and 3D content generation~\cite{eg3d,poole2022dreamfusion,lin2022magic3d}.
%by leveraging a multi-layer perceptron (MLP) to implicitly model the mapping from input 5D coordinates (i.e., 3D coordinates $\mathbf{x} = (x,y,z)$ and 2D viewing directions $\mathbf{d}=(\theta,\phi)$) to volume density $\sigma$ and view-dependent emitted radiance color $\mathbf{c} = (r,g,b)$. 
%
%They then use traditional volume rendering mechanisms on the obtained continuous 5D function (i.e., MLP) to generate novel views. 
To date, unfortunately, most NeRF-based methods encounter challenges when tackling large-scale cluttered scenes (e.g., Fig.~\ref{fig:teaser}):
\begin{enumerate}[leftmargin=0.16in, topsep=2pt,itemsep=-1ex,partopsep=1ex,parsep=1ex]
\item Input observations used for NeRF are often too sparse  compared to forward-facing or synthetic looking-inward scenes;
%\item Recovering fine-grained objects within a large volume is challenging for NeRF; %in capturing details accurately.
\item View-dependent visual effects give rise to ambiguity, resulting in a ``foggy" density field as shown in Fig.~\ref{fig:teaser}. 
%
Such artifacts are particularly pronounced in indoor scenes strewn with view-dependent appearances, such as specular highlights, glossy surface reflections from man-made objects. 
\end{enumerate}

Despite attempts to enhance NeRF's rendering quality given suboptimal input, such as using 3D conical frustums~\cite{barron2021mip,barron2022mip}, physically-grounded augmentations~\cite{chen2022aug}, and misalignment correction~\cite{jiang2022alignerf},  these challenges have yet to be fully resolved.
%
Depth supervision~\cite{deng2022depth, wei2021nerfingmvs} or proxy geometry~\cite{xu2021scalable,wu2022scalable} images can help alleviate the challenges in handling large-scale with sparse input, at the expense of %but they come at the cost of requiring 
expensive pre-processing or additional input.
%
Another line of work~\cite{wang2021neus, oechsle2021unisurf, wang2022neuris} achieves better reconstruction of surface geometry by using signed distances instead of volume density as scene representation. However, they sacrifice the ability to synthesize photo-realistic novel views.

%We observe that NeRF has been suffering from foggy ``floater" artifacts in large-scale cluttered scenes.
%
%Such artifacts are particularly pronounced in indoor scenes strewn with view-dependent appearances from man-made objects. 
%
To address the above issues, we propose an extension to NeRF, dubbed as {\bf \ournerf}, which enforces effective {\em appearance} and {\em geometry} constraints conducive to accurate colors and 3D densities estimation. We believe \ournerf can contribute beyond novel view synthesis, such as NeRF object detection~\cite{hu2022nerf}, NeRF object segmentation~\cite{zhi2021place, liu2022unsupervised, fan2022nerf,ren2022neural}, and NeRF registration~\cite{goli2022nerf2nerf}, where the rooms for improvement are substantial if more accurate color and density estimation are available.

Correspondingly, there are two steps in \ournerf. First, for appearance correction, the view-independent and view-dependent color components are predicted from the underlying 3D scene, which is combined to produce the final color estimation (Fig.~\ref{fig:toaster}).
%
The view-independent component (diffuse color and shading) captures the overall scene color, while the view-dependent component (highlights or reflections) captures color variations due to changes in viewing angle.
%
\ournerf then discards these view-dependent appearances in the training views to prevent them from interfering with the density estimation.
%
Second, a simple and effective geometry correction procedure will be performed to further eliminate the foggy ``floaters" or density errors. This geometry correction procedure is based on an assumption in line with traditional ray tracing in computer graphics.
\begin{comment}
% xh: basically copying method
On the other hand, ClearNeRF performs a geometric correction procedure performed on each traced ray during inference to refine the density estimation and better tackle the floater artifacts. 
%
The geometry correction procedure assumes that there should only be one salient peak along each traced ray during NeRF inference. 
Only the salient peak closest to the ray origin (the camera center) corresponds to  true geometry while the others will be manifested as foggy floaters hovering in the density volume. 
%
This assumption is in line with traditional ray tracing in computer graphics where in the absence of noise, only one intersection per ray should be returned to indicate the closest ray-object intersection.
%
\end{comment}
%%%%%%%%%%%
%As shown in Fig.~\ref{fig:teaser}, when reconstructing an indoor scene with sparse input and highly view-dependent objects, NeRF produces severe floating artifacts due to its attempt to explain view-dependent appearances.
%
Experiments verify that our proposed \ournerf can effectively get rid of floater artifacts without additional input.% or significant computational overhead. 


In summary, our contributions include the following:
\begin{itemize}[leftmargin=0.16in, topsep=2pt,itemsep=-1ex,partopsep=1ex,parsep=1ex]
    \item We propose a concise method for decomposing view-independent and view-dependent appearance during NeRF training and eliminate the interference of view-dependent appearance.
    \item We propose a geometric correction procedure performed on each traced ray during inference to refine the density estimation and better tackle the floater artifacts.
    \item Extensive experiments and ablations verify the effectiveness of our core designs and results in improvements over the vanilla NeRF and other state-of-the-art alternatives.
    %without additional computational resources or other inputs.
\end{itemize}




\section{Preliminary}
\label{Preliminary}
We provide the background on the adversarial attack, diffusion models, and adversarial purification in this section.

\subsection{Adversarial Attacks}
Adversarial attacks aim to manipulate or trick machine learning models by adding imperceptible perturbations to input data that can cause the model to misclassify or produce incorrect outputs. The adversarial attacks can be categorized into black-box, grey-box, and white-box attacks. The black-box attack assumes that the attacker knows nothing about the internal structure of the classifier and defender. The white-box attack assumes that the attacker can obtain any information about the defender and the target classifier, including the architecture and parameter weights. The grey-box lies between the white- and black-box attacks, where the attacker partially knows the target model. In this work, we only focus on the performance of purification in the white-box attack since the white-box attack is the most difficult to defend from the defender's perspective.

The Projected Gradient Descent (PGD)~\citep{Madry2017TowardsDL} method is a common white-box attack. PGD is a gradient-based attack that iteratively updates an adversarial example using the following rule
\begin{equation}
\mathbf{x}_{i + 1} = \Pi_{\mathcal{X}} \left(\mathbf{x}_i + \alpha_i \text{sign} \nabla_\mathbf{x} \mathcal{L}(f_\phi(\mathbf{x}), y) |_{\mathbf{x}=\mathbf{x}_i}\right),
\end{equation}
where $f_\phi$ represents a classifier, and $\Pi_\mathcal{X}$ indicates a projection operation onto $\mathcal{X}$. PGD can only be applied for the differentiable defense methods. For non-differentiable defense methods, the Backward Pass Differentiable Approximation (BPDA)~\cite{obfuscated-gradients} is widely used, which computes the gradient of the non-differentiable function by using a differentiable approximation. Expectation over Transformation (EOT)~\citep{Athalye2017SynthesizingRA} can be additionally employed for randomized defenses, which optimizes the expectation of the randomness. AutoAttack~\citep{Croce2020ReliableEO} is an ensemble of four different types of attacks. In this work, we measure the robustness of purification methods against these attack methods. 

\subsection{Diffusion Models}
Recently, diffusion-based models~\citep{Ho2020DenoisingDP, Song2020ScoreBasedGM} have gained increasing attention in generative models. Unlike the VAEs and GANs, the diffusion-based models produce samples by gradually removing noise from random noise. The training of diffusion-based models consists of two processes, the forward process, and the reverse denoising process.
The forward process adds Gaussian noise over $T$ steps to the observed input $\mathbf{x}_0$ with a predefined variance scheduler $\beta_t$, whose joint distribution is defined as 
\begin{equation}
    q(\mathbf{x}_{1:T} \vert \mathbf{x}_0) = \prod^T_{t=1} q(\mathbf{x}_t \vert \mathbf{x}_{t-1}),
\end{equation}
where $q(\mathbf{x}_t \vert \mathbf{x}_{t-1})$ is a Gaussian transition kernel from $\mathbf{x}_{t-1}$ to $\mathbf{x}_t$ 
\begin{equation}
    q(\mathbf{x}_{t} | \mathbf{x}_{t-1}) := \mathcal{N}(\mathbf{x}_{t}; \sqrt{1 - \beta_t} \mathbf{x}_{t-1}, \beta_t\textbf{I}).
\end{equation}
The reverse process denoises the random noise $\mathbf{x}_{T}$ over $T$ times, whose joint distribution is defined as
\begin{equation}
    p_\theta(\mathbf{x}_{0:T}) = p(\mathbf{x}_T) \prod^T_{t=1} p_\theta(\mathbf{x}_{t-1} | \mathbf{x}_t).
\end{equation}
The transition distribution from $\mathbf{x}_t$ to $\mathbf{x}_{t-1}$ is often modeled by Gaussian distribution
\begin{equation}
    p_\theta(\mathbf{x}_{t-1} | \mathbf{x}_t) = \mathcal{N}(\mathbf{x}_{t-1}; \boldsymbol{\mu}_\theta(\mathbf{x}_t, t), \sigma^2_t \textbf{I}),
\end{equation}
where $\sigma_t$ is a variance, and $\boldsymbol{\mu}_\theta$ is a predicted mean of $\mathbf{x}_{t-1}$ derived from a learnable denoising model $\boldsymbol{\epsilon}_\theta$. The denoising model is often trained by predicting a random noise at each time step via following objective
\begin{equation}
L(\theta) = \mathbb{E}_{t, \mathbf{x}_0, \boldsymbol{\epsilon}} \Big[\|\boldsymbol{\epsilon} - \boldsymbol{\epsilon}_\theta(\mathbf{x}_t, t)\|^2 \Big], \\
\end{equation}
where $\boldsymbol\epsilon$ is a Gaussian noise, i.e., $\boldsymbol\epsilon \sim \mathcal{N}(0, \textbf{I})$. 
The model $\boldsymbol{\epsilon}_\theta$ takes the noisy input $\mathbf{x}_t$ and the time step $t$ to predict the actual noise $\boldsymbol\epsilon$ at time $t$.
In the Denoising Diffusion Probabilistic Model (DDPM)~\citep{Ho2020DenoisingDP}, the reverse denoising process is performed over $T$ steps through random sampling, resulting in a slower generation of samples compared with GANs and VAEs.

Based on the fact that the multiple denoising steps can be performed at a single step via a non-Markovian process, \citet{Song2020DenoisingDI} proposes a new sampling strategy, which we call Denoising Diffusion Implicit Model (DDIM) sampler, to accelerate the reverse denoising process. In this work, we compare the performances of DDPM and DDIM samplers in the diffusion-based purification approach.

\subsection{Adversarial Purification}
Adversarial purification via generative models is a technique used to improve the robustness of machine learning models against adversarial attacks~\citep{Samangouei2018DefenseGANPC}. The idea behind this technique is to use a generative model to learn the underlying distribution of the clean data and use it to purify the adversarial examples.

Diffusion-based generative models can be used as a purification process if we assume that the imperceptible adversarial signals as noise~\citep{Nie2022DiffusionMF}.
To do so, the purification process adds noise to the adversarial example via the forward process with $t^*$ steps, and it removes noises via the denoising process. The choice of the number of forward steps $t^*$ is essential since too much noise can remove the semantic information of the original example, or too little noise cannot remove adversarial perturbation. 
In theory, as we add more noise to the adversarial example, the distributions over the noisy adversarial example and the true example become close to each other~\citep{Nie2022DiffusionMF}. Therefore, the denoised examples are likely to be similar.

\cutsectionup
\section{Approach}
\cutsectiondown

We study the phenomena outlined in the introduction by creating and measuring the performance of classifiers trained to detect images sampled from \emph{unseen} generators and subsequently training new generators to fool them, in sequential rounds, forming a chain of generators and classifiers. We do this in one of two settings, first with low dimensional images (MNIST), a simplistic DCGAN, and a basic classifier architecture. In the second setting, we use higher dimensional images (FFHQ), and perform experiments using the unmodified StyleGAN2 (SG2) architecture. Seeking to  minimize sources of variance as much as possible, we limit to a single GAN architecture and a fixed dataset in both settings. We also do not use the ``truncation'' trick \citecustom{karras2019style}, a sample-time heuristic commonly used with the SG2 architecture to improve the output visual quality at the expense of diversity (\emph{see} Supplement for more discussion on this). In the SG2 setting, we test three different widely-used classifier architectures: ResNet-50, Inception-v3, and MobileNetV2. These architectures were chosen for their architectural diversity. All classifiers and generators are trained from scratch, without any pre-training. Supplement provides details about the model architectures and training parameters.

\cutsubsubsectionup
\subsection{A note on terminology} \label{sec:terminology}
\cutsubsubsectiondown

Because our procedure involves both GANs and classifiers, there is potential ambiguity in terminology as GANs themselves are trained with a subnetwork designed to distinguish generated images from natural images, which is commonly called the ``discriminator'', ``adversarial network'', or ``critic'', among others. To keep the text clear, we will refer to subnetworks co-trained with a generator which together comprise a GAN as ``\textbf{discriminators}'', denoted $D$. The networks trained on samples from multiple, independently trained generators are referred to as ``\textbf{classifiers}'', $C$. Each sequential round of training a pool of GANs followed by training classifiers is an ``\textbf{iteration}'' (detailed in Sec. \ref{sec:overview_setup}, and Figs. \ref{fig:experiment_setup_1} and \ref{fig:experiment_setup_2}) and is indexed with a superscript. Iterations are distinct from training steps: during a single iteration, GANs are fully trained, then classifiers are fully trained using those GAN generators. Broadly speaking an ``\textbf{artifact}'' is any property of a generated image that distinguishes it from a real image. By ``\textbf{knowledge gaps}'', we are referring to a specific class of artfacts that reliably occur \emph{across} samples from a generator. Since this class of artifacts is the only one studied in this work, we use artifact and knowledge gap interchangeably.

\cutsubsubsectionup
\subsection{Overview of setup and iterations} \label{sec:overview_setup}
\cutsubsubsectiondown
\begin{figure}[h]
	\centering
    \begin{subfigure}{\linewidth}
        \centering
        \includegraphics[width=0.7\linewidth]{images/gan_training_zeroth.pdf}
        \caption{Stage 1 at iteration 0: GAN training with standard loss function} \label{fig:gan_training_zeroth}
    \end{subfigure}%
    \hspace{0.05\textwidth}
    % \hspace*{\fill}   % maximize separation between the subfigures
    \begin{subfigure}{\linewidth}
    \centering
        \includegraphics[width=0.7\linewidth]{images/classifier_training.pdf}
        \caption{Stage 2 at iteration $i$: Classifier training}\label{fig:classifier_training}
    \end{subfigure}%
	\caption{\textbf{Experimental setup \& training classifiers.} Generators $G$ are \textbf{\textcolor[HTML]{93C47C}{green}}, co-trained discriminators $D$ are \textbf{\textcolor[HTML]{A4C2F4}{blue}} and classifiers $C$ trained using multiple, frozen generators are \textbf{\textcolor[HTML]{C27BA0}{purple}}. Dashed borders indicate that the subnetwork is not being updated during this stage of the iteration. \subref{fig:gan_training_zeroth} Generators trained in iteration 0 are trained in the typical way. \subref{fig:classifier_training} Classifiers are trained in the second stage of all iterations, on samples drawn from subsets of the generators trained in the first stage.}
\label{fig:experiment_setup_1}
% \vspace{-0.05in}
\end{figure}

Our experiments consist of sequential rounds (``iterations''), each with two stages: first, a pool of GAN generators initialized randomly is trained, then classifiers are trained to detect samples from the generators trained in the first stage. In the first stage of the first iteration ($i = 0$), a number of GANs (DCGAN in the first setting, SG2 in the second setting) are trained independently on the training images (MNIST in the first setting, FFHQ in the second setting), as shown in \cref{fig:gan_training_zeroth}. This setup is modified slightly in later iterations (\emph{see} \cref{fig:experiment_setup_2}) as detailed below. Classifier training follows in the second stage (\cref{fig:classifier_training}) as a standard classification task where each classifier is trained on a balanced dataset of real images and images sampled from a subset of generators trained in the first stage. The second stage is the same in every iteration, always sampling images from generators trained in the first stage of the iteration.
The first stage of subsequent iterations ($i > 0$) proceeds like the first stage of the first iteration but with a modified generator loss function: generators are trained to fool not only the discriminator they are co-trained with, but also frozen classifiers from preceding iterations. To do this we modify the ``classical'' GAN generator loss function $\mathbf{\mathcal{L}}$:
\begin{dmath}
\mathbf{\mathcal{L}_{G^{(i)}}} = -\log(D^{(i)}(G^{(i)}(w)))
\label{eq:orig_gan_loss}
\end{dmath}
in one of two ways. In the first, $\mathbf{\mathcal{L}^\Sigma_{G^{(i)}}}$, generators must fool a classifier from \emph{every} preceding iteration: 
\begin{dmath}
\mathbf{\mathcal{L}^\Sigma_{G^{(i)}}} = -[\log(D^{(i)}(G^{(i)}(w))) + \phi \sum_{j=0}^{i-1}\log(C^{(j)}_0(G^{(i)}(w)))]
\label{eq:fool_all_gan_loss}
\end{dmath}
A graphical depiction of a single generator using this loss function is shown in \cref{fig:gan_training_modified}. $\phi$ is a used to weight the relative influence of classifiers. Because a classifier from each previous iteration must be fooled in order to minimize this function, we refer to it as the ``fool-all'' loss function.

The other generator loss function variation, $\mathbf{\mathcal{L}^*_{G^{(i)}}}$, relies purely on a classifier from the iteration immediately preceding the current one, rather than all preceding iterations: 
\begin{align}
\mathbf{\mathcal{L}^*_{G^{(i)}}} = -[\log(D^{(i)}(G^{(i)}(w))) + \phi \log(C^{(i-1)}_0(G^{(i)}(w)))]
\label{eq:memoryless_gan_loss}
\end{align}
This is depicted in \cref{fig:gan_training_modified_memoryless}. Because $\mathbf{\mathcal{L}^*_{G^{(i)}}}$ depends only on the current iteration and the preceding iteration, we refer to this as the ``memoryless'' loss function.

The two modifications result in markedly different training dynamics. Reported results will generally be for the ``fool-all'' $\mathcal{L}^\Sigma$ variation (\cref{fig:gan_training_modified}). When results are based on experiments using the ``memoryless'' variation $\mathcal{L}^*$ (\cref{fig:gan_training_modified_memoryless}), they will be explicitly noted as such. Classifiers are frozen (i.e., their weights are not updated) during the first stage of every iteration.

\begin{figure}[h]
	\centering
    \begin{subfigure}{\linewidth}
        \centering
        \includegraphics[width=0.7\linewidth]{images/gan_training_modified.pdf}
        \caption{Stage 1 at iteration $i$: GAN training with ``fool-all'' modified loss function}
        \label{fig:gan_training_modified}
    \end{subfigure}%
    \hspace{0.05\textwidth}
    % \hspace*{\fill}   % maximize separation between the subfigures
    \begin{subfigure}{\linewidth}
        \centering
        \includegraphics[width=0.7\linewidth]{images/gan_training_modified_memoryless.pdf}
        \caption{Stage 1 at iteration $i$: GAN training with ``memoryless'' modified loss function} \label{fig:gan_training_modified_memoryless}
    \end{subfigure}%
	\caption{\textbf{GANs trained in higher iterations.} In subsequent iterations ($i > 0$), stage 1 GAN training is modified from the first iteration ($i=0$, \emph{see} \cref{fig:gan_training_zeroth}) such that the generator $G^{(i)}_k$ learns to fool not only its co-trained discriminator $D^{(i)}_k$ but also \subref{fig:gan_training_modified} $i$ classifiers, one from each preceding iteration (\cref{eq:fool_all_gan_loss}) or \subref{fig:gan_training_modified_memoryless} a single classifier from the immediately preceding iteration (\cref{eq:memoryless_gan_loss}). At $i=1$, these two approaches are equivalent.}
\label{fig:experiment_setup_2}
\end{figure}
The classifier subscript $0$, used in Figs. \ref{fig:gan_training_modified} and \ref{fig:gan_training_modified_memoryless} (e.g., $C^{(i-1)}_0$), is purely to distinguish classifiers within the same iteration. In each iteration, multiple classifiers are trained that are initialized randomly and trained independently. When testing a GAN trained to fool the previous iteration's classifiers, classifiers used for training and testing are trained on disjoint subsets of generators, to measure generalization. For example, if $G^{(i)}_k$ is trained to fool $C^{(i-1)}_0$, and is evaluated against $C^{(i-1)}_1$, then $C^{(i-1)}_0$ and $C^{(i-1)}_1$ are trained on disjoint subsets of iteration $i-1$ generators.
\begin{table}[h]
\begin{center}
\small
% \setlength{\tabcolsep}{2pt}
\begin{tabular}{l | c c c }
\whline
\multirow{2}{*}{\makecell[c]{Model}} &  \multirow{2}{*}{\makecell[c]{Params \\ (M)}} & \multirow{2}{*}{\makecell[c]{MACs \\ (G)}}  & \multirow{2}{*}{\makecell[c]{Top-1 \\ (\%)}} \\
~ & ~ & ~ &  \\
\whline
DeiT-S \cite{deit} & 22 & 4.6 & \textbf{79.8} \\ 
MetaNeXt-Attn & 22 & 4.6 & 3.9  \\ 
ConvNeXt-S (\textit{iso.}) \cite{convnext} & 22 & 4.3 & 79.7 \\
\modelname{}-S (\textit{iso.}) & 22 & 4.2 & 79.7 \\
\hline
DeiT-B \cite{deit} & 87 & 17.6 & 81.8 \\
ConvNeXt-S (\textit{iso.}) \cite{convnext} & 87 & 16.9 & 82.0 \\
\modelname{}-S (\textit{iso.}) & 86 & 16.8 & \textbf{82.1}  \\
\whline
\end{tabular}
\end{center}
\vspace{-2mm}
\caption{\textbf{Comparison among ViT, isotropic ConvNeXt and \modelname{}.}  MetaNeXt-Attn is instantiated from MetaNeXt with token mixer of self-attention \cite{transformer}.}
\label{tab:iso}
\end{table}

\begin{table*}[h]
    \centering
    \setlength{\tabcolsep}{5pt}
    \begin{tabular}{l | c | c | c c | c c | c}
% \toprule
\whline
\multirow{2}{*}{\makecell[c]{Model}}   &  \multirow{2}{*}{\makecell[c]{Mixing \\ Type}}     & 
\multirow{2}{*}{\makecell[c]{Image \\ (size)}} & \multirow{2}{*}{\makecell[c]{Params \\ (M)}}  & \multirow{2}{*}{\makecell[c]{MACs \\ (G)}} & \multicolumn{2}{c|}{Throughput (img/second)} & \multirow{2}{*}{\makecell[c]{Top-1 \\ (\%)}}
\\
~ &  ~ & ~ & ~ & ~ & Train & Inference & ~ \\
\whline
DeiT-S \cite{deit}  & Attn  & $224^2$ & 22 & 4.6 & 1227 & 3781 & 79.8 \\
T2T-ViT-14 \cite{t2t}  & Attn   & $224^2$ & 22 & 4.8 & -- & -- & 81.5 \\
TNT-S \cite{tnt} & Attn & $224^2$ & 24 & 5.2 & -- & -- & 81.5 \\
Swin-T \cite{swin}  & Attn   & $224^2$ &  29 & 4.5 & 564 & 1768 & 81.3  \\
Focal-T \cite{focal_transformer} & Attn & $224^2$ & 29 & 4.9 & -- & -- & 82.2 \\
\hdashline
ResNet-50 \cite{resnet, resnetsb}   & Conv   & $224^2$ & 26 & 4.1 & 969 &  3149 & 78.4 \\
RSB-ResNet-50 \cite{resnet, resnetsb}   & Conv   & $224^2$ &  26 & 4.1 & 969 &  3149 & 79.8 \\
RegNetY-4G \cite{regnet, resnetsb}   & Conv   & $224^2$ & 21 & 4.0 & 670 & 2694 & 81.3 \\
FocalNet-T \cite{focalnet} & Conv & $224^2$ & 29 & 4.5 & -- & -- & 82.3 \\
\br 
ConvNeXt-T \cite{convnext}   & Conv   & $224^2$ & 29 &  4.5 & 575 & 2413 \textcolor{gray}{(1943)} & 82.1 \\
\br 
\modelname{}-T (Ours) & Conv & $224^2$ & 28 &  4.2 & 901 (+57\%) & 2900 (+20\%) & 82.3 (+0.2) \\
\whline
T2T-ViT-19 \cite{t2t}  & Attn   & $224^2$ & 39 & 8.5 & -- & -- & 81.9 \\
PVT-Medium \cite{pvt} & Attn & $224^2$ & 44 & 6.7 & -- & -- & 81.2 \\
Swin-S \cite{swin}  & Attn   & $224^2$ & 50 &  8.7 & 359  & 1131 & 83.0  \\
Focal-S \cite{focal_transformer} & Attn & $224^2$ & 51 & 9.1 & -- & -- & 83.5 \\
\hdashline
RSB-ResNet-101 \cite{resnet, resnetsb}   & Conv   & $224^2$ &  45 & 7.9 & 620 & 2057 & 81.3 \\
RegNetY-8G \cite{regnet, resnetsb}   & Conv   & $224^2$ & 39 &  8.0 & 689 & 1326 & 82.1 \\
FocalNet-S \cite{focalnet} & Conv & $224^2$ & 50 & 8.7 & -- & -- & 83.5 \\
\br 
ConvNeXt-S \cite{convnext}   & Conv   & $224^2$ & 50 & 8.7 & 361 & 1535 \textcolor{gray}{(1275)} & 83.1 \\
\br 
\modelname{}-S (Ours) & Conv & $224^2$ & 49 & 8.4 & 521 (+44\%) & 1750 (+14\%) & 83.5 (+0.4) \\
\whline
DeiT-B \cite{deit}  & Attn  & $224^2$ & 86 & 17.5 & 541 & 1608 & 81.8 \\
T2T-ViT-24 \cite{t2t}  & Attn & $224^2$ & 64 &  13.8 & -- & -- & 82.3 \\
TNT-B \cite{tnt} & Attn &  $224^2$ & 66 & 14.1 & -- & -- & 82.9 \\
PVT-Large \cite{pvt} & Attn &  $224^2$ & 62 & 9.8 & -- & -- & 81.7 \\
Swin-B \cite{swin}  & Attn  & $224^2$ & 88 &  15.4 & 271 & 843 & 83.5 \\
Focal-B \cite{focal_transformer} & Attn & $224^2$ & 90 & 16.0 & -- & -- & 83.8 \\
\hdashline
RSB-ResNet-152 \cite{resnet, resnetsb}   & Conv  & $224^2$ & 60 & 11.6 & 437 & 1457 & 81.8 \\
RegNetY-16G \cite{regnet, resnetsb}   & Conv  & $224^2$ & 84 &15.9 & 322 & 1100 & 82.2 \\
RepLKNet-31B \cite{replknet}  & Conv & $224^2$ & 79  & 15.3 & -- & -- & 83.5 \\
FocalNet-B \cite{focalnet} & Conv & $224^2$ & 89 & 15.4 & -- & -- & 83.9 \\
\br 
ConvNeXt-B \cite{convnext}   & Conv   & $224^2$ & 89  & 15.4 & 267 & 1122 \textcolor{gray}{(969)} & 83.8  \\
\br
\modelname{}-B (Ours) & Conv & $224^2$ & 87 & 14.9 & 375 (+40\%) & 1244 (+11\%) & 84.0 (+0.2) \\
\hline
ViT-Base/16 \cite{vit} & Attn & $384^2$ & 87 & 55.4 & 130 & 359 & 77.9 \\
DeiT-B \cite{deit} & Attn & $384^2$ & 86 & 55.4 & 131 & 361 & 83.1 \\
Swin-B \cite{swin} & Attn & $384^2$ & 88 & 47.1 & 104 & 296 & 84.5 \\
\hdashline
RepLKNet-31B \cite{replknet} & Conv & $384^2$ & 79 & 45.1 & -- & -- & 84.8 \\
\br
ConvNeXt-B \cite{convnext} & Conv & $384^2$ & 89 & 45.0 & 95 & 393 (\textcolor{gray}{337}) & 85.1 \\
\br
\modelname{}-B (Ours) & Conv & $384^2$ & 87 & 43.6 & 139 (+46\%) & 428 (+9\%) & 85.2 (+0.1) \\

\whline
\end{tabular}


    \caption{\label{tab:imagenet}
    \textbf{Performance of models trained on ImageNet-1K.} The throughputs are measured on an A100 GPU with batch size of 128 and full precision (FP32). Our environment is PyTorch 1.13.0 and NVIDIA CUDA 11.7.1. The better results of ``Channel First" and ``Channel Last"  memory layouts are reported. The numbers in \textcolor{gray}{gray} color are reported by ConvNeXt \cite{convnext}. In our environment, ConvNeXt achieves much higher throughput than the values reported in the paper \cite{convnext}.}
\end{table*}

\section{Experiment}
%%%%%%%%% Table: ADE20K uper
\begin{table}[t]
\centering
\setlength{\tabcolsep}{3.5pt}
\begin{tabular}{l|cccc}
\whline
\multirow{2}{*}{Backbone} & \multicolumn{3}{c}{UperNet}\\
\cline{2-5}
& Params (M) & MACs (G) & FPS & mIoU (\%) \\
    \whline
    Swin-T~\cite{swin}                & 60 & 945 & 20.6 & 45.8 \\
    ConvNeXt-T ~\cite{convnext}          & 60 & 939 & 20.6 & 46.7 \\
    \modelname{}-T  & 56 & 933 & 22.7 & \textbf{47.9} \\
	\hline
    Swin-S~\cite{swin}                & 81 & 1038 & 16.2 & 49.5 \\
    ConvNeXt-S ~\cite{convnext}          & 82 & 1027 & 16.8 & 49.6 \\
    \modelname{}-S & 78 & 1020 & 17.6 & \textbf{50.0} \\
	\hline
    Swin-B~\cite{swin}                & 121 & 1188 & 16.2 & 49.7 \\
     ConvNeXt-B ~\cite{convnext}          & 122 & 1170 & 15.7 & 49.9 \\
    \modelname{}-B  & 115 & 1159 & 17.5 & \textbf{50.6} \\
\whline
\end{tabular}

\caption{\textbf{Performance of Semantic segmentation with UperNet \cite{upernet} on ADE20K~\cite{ade20k} validation set.} Images are cropped to $512 \times 512$ for training. The MACs are measured with input size of $512 \times 2048$. }
\label{tab:upernet}
\normalsize
\end{table}

%%%%%%%%% Table: ADE20K
\begin{table}[h!]
\small
\centering
\setlength{\tabcolsep}{3.5pt}
\begin{tabular}{l|ccc}
\whline
\multirow{2}{*}{Backbone} & \multicolumn{3}{c}{Semantic FPN}\\
\cline{2-4}
& Params (M) & MACs (G) & mIoU (\%) \\
    \whline
    ResNet-50~\cite{resnet}                & 29 & 46 & 36.7 \\
    PVT-Small~\cite{pvt}          & 28 & 45 & 39.8 \\
	PoolFormer-S24  \cite{metaformer}                  & 23 & 39 & 40.3 \\
    \modelname{}-T & 28 & 44 & \textbf{43.1} \\
    \hline
    ResNet-101~\cite{resnet}      & 48 & 65 &  38.8\\
    ResNeXt-101-32x4d~\cite{resnext} & 47 & 65 & 39.7 \\
    PVT-Medium~\cite{pvt}         & 48 & 61 & 41.6 \\
     PoolFormer-S36 \cite{metaformer}                   & 35 & 48 & 42.0 \\
    PoolFormer-M36 \cite{metaformer}                   & 60 & 68 & 42.4 \\
     \modelname{}-S & 50 & 65 & \textbf{45.6} \\
    \hline
    PVT-Large~\cite{pvt}          & 65 & 80 & 42.1 \\
    % \hline
    ResNeXt-101-64x4d~\cite{resnext} & 86 & 104 & 40.2 \\
     PoolFormer-M48 \cite{metaformer}                  & 77 & 82 &  42.7 \\
     \modelname{}-B & 85 & 100 & \textbf{46.4} \\
\whline
\end{tabular}
\caption{\textbf{Performance of Semantic segmentation with Semantic FPN \cite{fpn} on ADE20K~\cite{ade20k} validation set.} Images are cropped to $512 \times 512$ for training. The MACs are measured with input size of $512 \times 512$. }
\label{tab:fpn}
\normalsize
\end{table}


\subsection{Image classification}
\myPara{Setup}
For the image classification task, ImageNet-1K \cite{imagenet_cvpr, imagenet_ijcv} is one of the most commonly-used benchmarks, which contains around 1.3 million images in the training set and 50 thousand images in the validation set. To fairly compared with the widely-used baselines, \eg  Swin \cite{swin} and ConvNeXt \cite{convnext}, we mainly follow the training hyper-parameters from DeiT \cite{deit} without distillation. Specifically, the models are trained by AdamW \cite{adamw} optimizer with a learning rate $lr = 0.001 \times \mathrm{batch size} / 1024$ ($lr=4e-3$ and $\mathrm{batch size} = 4096$ are used in this paper the same as ConvNeXt). Following DeiT, data augmentation includes standard random resized crop, horizontal flip, RandAugment \cite{randaugment}, Mixup \cite{mixup}, CutMix \cite{cutmix}, Random Erasing \cite{random_erasing} and color jitter. For regularization, label smoothing \cite{inception_v3}, stochastic depth \cite{stochastic_depth}, and weight decay are adopted. Like ConvNeXt, we also use LayerScale \cite{layerscale}, a technique to help train deep models. Our code is based on PyTroch \cite{pytorch} and timm \cite{rw2019timm} libraries. 


\myPara{Results} 
We compare \modelname{} with various state-of-the-art models, including attention-based and convolution-based models. As can be seen in Table \ref{tab:imagenet}, \modelname{} achieves highly competitive performance as well as enjoys higher speed. With similar model sizes and MACs, \modelname{} consistently outperforms ConvNeXt in terms of top-1 accuracy, and also exhibits higher throughput. For example, \modelname{}-T not only surpasses ConvNeXt-T by 0.2\%, but also enjoys $1.6 \times$/$1.2 \times $ training/inference throughputs than ConvNeXts, similar to those of ResNet-50. That is to say, \modelname{}-T enjoys both ResNet-50's speed and ConvNeXt-T's accuracy. 
Moreover,  following Swin and ConvNeXt, we also finetuned the model trained at the resolution of $224 \times 224$ to $384 \times 384$ for 30 epochs.  We can see that \modelname{} still obtains promising performance similar to that of ConvNeXt.


Besides the 4-stage framework \cite{vgg, resnet, swin}, another notable one is ViT-style \cite{vit} isotropic architecture which has only one stage. To match the parameters and MACs of DeiT, we construct \modelname{} (\textit{iso.}) following ConvNeXt \cite{convnext}. Specifically, for the small/base model, we set the embedding dimension as 384/768 and the block number as 18/18. Besides, we build a model called MetaNeXt-Attn which is instantiated from MetaNeXt block by specifying self-attention as token mixer. The aim of this model is to investigate whether it is possible to merge two residual sub-blocks of the Transformer block into a single one. The experiment results are shown in Table \ref{tab:iso}. It can be seen that \modelname{} can also perform well with the isotropic architecture, demonstrating \modelname{} exhibits good generalization across different frameworks. It is worth noting that MetaNeXt-Attn could not be trained to converge and only achieved an accuracy of 3.9\%. This result suggests that, unlike the token mixer in MetaFormer, the token mixer in MetaNeXt cannot be too complex. If it is, the model may not be trainable.


%%%%%%%%% Table: Ablation
\begin{table*}[h]
\centering
\setlength{\tabcolsep}{2pt}
\scalebox{1.0}{\begin{figure}
       \centering
        \setlength{\tabcolsep}{1pt}
        {\scriptsize
        \begin{tabular}{c c c c c c c }
            { Original } &
            \multicolumn{2}{c}{  } &
            \multicolumn{4}{c}{$\longleftarrow$ Object level variations $\longrightarrow$} \\
            \includegraphics[width=0.185\linewidth]{images/ablation/chair.jpg} &
            \multicolumn{2}{c}{  } &
            \includegraphics[width=0.185\linewidth]{images/ablation/1_only_prompt_mixing/bench.jpg} &
            \includegraphics[width=0.185\linewidth]{images/ablation/1_only_prompt_mixing/stool.jpg} &
            \includegraphics[width=0.185\linewidth]{images/ablation/1_only_prompt_mixing/armchair.jpg} &
            \includegraphics[width=0.185\linewidth]{images/ablation/1_only_prompt_mixing/saddle.jpg} \\
            \multicolumn{3}{c}{  } &
            \multicolumn{4}{c}{ Only Prompt Mixing } \\
            \multicolumn{3}{c}{ } &
            \includegraphics[width=0.185\linewidth]{images/ablation/2_with_self_attn_injection/bench.jpg} &
            \includegraphics[width=0.185\linewidth]{images/ablation/2_with_self_attn_injection/stool.jpg} &
            \includegraphics[width=0.185\linewidth]{images/ablation/2_with_self_attn_injection/armchair.jpg} &
            \includegraphics[width=0.185\linewidth]{images/ablation/2_with_self_attn_injection/saddle.jpg} \\
            \multicolumn{3}{c}{  } &
            \multicolumn{4}{c}{ + Attention-Based Shape Localization } \\
            \multicolumn{3}{c}{ } &
            \includegraphics[width=0.185\linewidth]{images/ablation/3_background_blending/bench.jpg} &
            \includegraphics[width=0.185\linewidth]{images/ablation/3_background_blending/stool.jpg} &
            \includegraphics[width=0.185\linewidth]{images/ablation/3_background_blending/armchair.jpg} &
            \includegraphics[width=0.185\linewidth]{images/ablation/3_background_blending/saddle.jpg} \\
            \multicolumn{3}{c}{  } &
            \multicolumn{4}{c}{ + Controllable Background Preservation } \\
        \end{tabular}
        }
    \vspace{1mm}
    \captionof{figure}{
    Ablating our full object variations pipeline. Original image was crated using the prompt ``A \emph{chair} with a dog on it''. 
    }
    \vspace{-10pt}
    \label{fig:ablation}
\end{figure}
}
\caption{\textbf{Ablation for \modelname{} on ImageNet-1K classification benchmark.} \modelname{}-T is utilized as the baseline for the ablation study. Top-1 accuracy on the validation set is reported. 
}
\label{tab:ablation}
\normalsize
\end{table*}


\subsection{Semantic segmentation}
\myPara{Setup} 
ADE20K~\cite{ade20k}, one of the commonly used scene parsing benchmarks, is used to evaluate our models on semantic segmentation task. ADE20K includes 150 fine-grained semantic categories, containing twenty thousand and two thousand images in the training set and validation set, respectively.
The checkpoints trained on ImageNet-1K \cite{imagenet_cvpr} at the resolution of $224^2$ are utilized to initialize the backbones. Following Swin \cite{swin} and ConvNeXt \cite{convnext}, we firstly evaluate \modelname{} with UperNet \cite{upernet}. The models are trained with AdamW \cite{adamw} optimizer with learning rate of 6e-5 and batch size of 16 for 160K iterations. Following PVT \cite{pvt} and PoolFormer \cite{metaformer}, \modelname{} is also evaluated with Semantic FPN \cite{fpn}. 
In common practices~\cite{fpn,chen2017deeplab}, for the setting of 80K iterations, the batch size is 16. Following PoolFormer \cite{metaformer}, we increase the batch size to 32 and decrease the iterations to 40K, to speed up training. The AdamW \cite{adam, adamw} optimized is adopted with a learning rate of 2e-4 and a polynomial decay schedule of 0.9 power. Our code is based on PyTorch \cite{pytorch} and mmsegmentation library \cite{mmseg2020}.

\myPara{Results}
For segmentation with UpNet \cite{upernet}, the results are shown in Table \ref{tab:upernet}. As can be seen, 
\modelname{} consistently outperforms Swin \cite{swin} and ConvNeXt \cite{convnext} for different model sizes. On the setting of Semantic FPN \cite{fpn} as shown in Table \ref{tab:fpn}, \modelname{} significantly surpasses other backbones, like PVT \cite{pvt} and PoolFormer \cite{metaformer}. These results show that \modelname{} also has a high potential for dense prediction tasks.   





\subsection{Ablation studies}
We conduct ablation studies on ImageNet-1K \cite{imagenet_cvpr, imagenet_ijcv} using \modelname{}-T as baseline from the following aspects.


\myPara{Branch} Inception depthwise convolution includes four branches, three convolutional ones, and identity mapping. When removing any branch of horizontal or vertical band kernel, performance significantly drops from 82.3\% to 81.9\%, demonstrating the importance of these two branches. This is because these two branches with band kernels can enlarge the receptive field of the model. For the branch of small square kernel size of $3\times 3$, removing it can also achieve up to 82.0\% top-1 accuracy and bring higher throughput. This inspires us that if we attach more importance to the model speed, the simple version of \modelname{} without the square kernel of $3\times 3$ can be adopted.  For the band kernel, Inception v3 mostly equips them in a sequential way. We find that this assembling method can also obtain similar performance and even a little speed up the model. A possible reason is that PyTorch/CUDA may have optimized sequential convolutions well, and we only implement the parallel branches at a high level (see Algorithm \ref{alg:code}). We believe the parallel method will be faster when it is optimized better. Thus, parallel method for the band kernels is adopted by default. 


\myPara{Band kernel size} It is found the performance can be improved from kernel size 7 to 11, but it drops when the band kernel size increases to 13. This phenomenon may result from the optimization difficulty and can be solved by methods like structural re-parameterization \cite{repvgg, replknet}. For simplicity, we just set the band kernel size as 11 by default. 

\myPara{Convolution branch ratio} When the ratio increases from $1/8$ to $1/4$, performance improvement can not be observed. Ma \etal \cite{shufflenet_v2} also point out that it is not necessary for all channels to conduct convolution. But when the ratio decreases to $1/16$, it brings a serious performance drop. It is because a smaller ratio would limit the degree of token mixing, resulting in performance drop. Thus, we set the convolution branch ratio as $1/8$ by default.








\section{Discussion and Limitations}

Although we can ablate concepts efficiently for a wide range of object instances, styles, and memorized images, our method is still limited in several ways. First, while our method overwrites a target concept, this does not guarantee that the target concept cannot be generated through a different, distant text prompt. We show an example in \reffig{limitation} (a), where after ablating {\menlo Van Gogh}, the model can still generate {\menlo starry night painting}. However, upon discovery, one can resolve this by explicitly ablating the target concept {\menlo starry night painting}. Secondly, when ablating a target concept, we still sometimes observe slight degradation in its surrounding concepts, as shown in \reffig{limitation} (c). 

\nupur{Our method does not prevent a downstream user with full access to model weights from re-introducing the ablated concept~\cite{ruiz2022dreambooth,kumari2022multi,gal2022image}. Even without access to the model weights, one may be able to iteratively optimize for a text prompt with a particular target concept. Though that may be much more difficult than optimizing the model weights, our work does not guarantee that this is impossible.}

Nevertheless, we believe every creator should have an ``opt-out'' capability. We take a small step towards this goal, creating a computational tool to remove copyrighted images and artworks from large-scale image generative models.



%%
%% The next two lines define the bibliography style to be used, and
%% the bibliography file.
%\vspace{-5pt}
\bibliographystyle{plain}
\bibliography{ref}

\end{document}
\endinput
%%
%% End of file `sample-authordraft.tex'.
