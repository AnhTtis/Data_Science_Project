\section{More on Sparsifiers}

In order to synthetically generate metric sparse depth for TartanAir, we use a sparsifier that samples ground truth at locations determined via feature tracking. This is a realistic sparsifier to consider, as it simulates how a real VIO system would be used to generate sparse metric depth. We target low densities of sparse points---around 150 points---in line with the quantity that would be tracked by the frontend of VINS-Mono. While it is possible to track higher densities, doing so would require more computation and would be less suitable for real-time applications.

The feature tracker \cite{Lusk2018} we use with TartanAir enforces a minimum distance between landmarks, which leads to high coverage in the sparsity map given a sufficiently textured scene. In contrast, the sparse points provided in the VOID dataset tend to be more clustered, as illustrated in Figure \ref{fig:sparsity_coverage}. Clustering leads to regions of smoother interpolation within the convex hull of the scale map scaffolding, which makes learning dense scale easier and improves metric depth prediction. This is supported by our observation that experiments with SML see a greater reduction in error on VOID samples when compared to TartanAir samples: e.g., a 39\% reduction in iAbsRel on VOID versus a 26\% reduction in iAbsRel on TartanAir. Given these differences in sparsity patterns and coverage, our demonstration of successful zero-shot transfer from TartanAir to VOID is particularly impressive and highlights the robustness of our scale map learning approach.

\begin{figure}[!htb]
\centering
  \begin{tabular}{@{}*{3}{c@{\hspace{1mm}}}c@{}}
    \scriptsize{RGB Image} & \scriptsize{Sparse Depth Locations} & \scriptsize{Scale Map Scaffolding} \\
    % \vspace{-0.75mm}
    \includegraphics[width=0.33\linewidth]{suppl/sparsifiers/sparse_tartanair_ex_rgb.pdf}&
    \includegraphics[width=0.33\linewidth]{suppl/sparsifiers/sparse_tartanair_ex_conf.pdf}&
    \includegraphics[width=0.33\linewidth]{suppl/sparsifiers/sparse_tartanair_ex_scales.pdf}\\
    % \vspace{-0.75mm}
    \includegraphics[width=0.33\linewidth]{suppl/sparsifiers/sparse_void_ex_rgb.pdf}&
    \includegraphics[width=0.33\linewidth]{suppl/sparsifiers/sparse_void_ex_conf.pdf}&
    \includegraphics[width=0.33\linewidth]{suppl/sparsifiers/sparse_void_ex_scales.pdf}\\
  \end{tabular}
  \caption{Differences in sparsity and coverage of known metric sparse depth between TartanAir (top) and VOID (bottom) samples, shown alongside the interpolated scale map scaffolding. For TartanAir, sparse depth locations are determined through feature tracking via a VINS-Mono frontend. For VOID, sparse depth is obtained using XIVO~\cite{Fei2019xivo}.}
  \label{fig:sparsity_coverage}
  % \vspace{-12pt}
\end{figure}