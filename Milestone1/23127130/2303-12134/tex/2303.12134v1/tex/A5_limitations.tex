\section{Limitations}

When integrated into a real-world system, our approach would rely on successful VIO to produce sparse depth; hence, the limitations of our work are driven by limitations of visual-based odometry. In cases of rapid motion, poor lightning, or lack of sufficient texture, VIO will have difficulty tracking landmarks and output very few or even no points. Fewer sparse depth points will result in less reliable estimates for global scale and shift, and without any points, global alignment will fail. For scale map scaffolding, at least three non-colinear points are needed so that interpolation does not occur over a convex hull that is a point or line. Whenever fewer sparse depth points are provided, the resulting scale map scaffolding becomes an identity map and loses meaning. This hinders how well our SML network can regress dense scales. Similarly to how motion and depth estimation can be alternated to boost accuracy in both tasks \cite{Teed2020DeepV2D}, exploring conditions under which odometry-guided metric depth alignment could help recovery from odometry failure would be interesting future work.