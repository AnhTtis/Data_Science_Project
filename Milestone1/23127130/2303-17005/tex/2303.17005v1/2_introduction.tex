% !TEX root = main.tex

%%%%%%%%%%%%%%%%%%%%%%%%%%%%%%%%%%%%%%%%%%%%%%%%%%%%%%%%%%%%%%%%%%%%%%%%%%%%%%%%
%%%%%                              Introduction                            %%%%%
%%%%%%%%%%%%%%%%%%%%%%%%%%%%%%%%%%%%%%%%%%%%%%%%%%%%%%%%%%%%%%%%%%%%%%%%%%%%%%%%

%%%% motivations
%%%% based on existing challenges, step by step talk about why we do this work
\section{Introduction}
\label{sec:introduction}

%% Motivation:
%%  - scientific meaning of polar region
%%  - AUV is a potential tool for ice-water interface
%%  - underwater state estimation is an open problem
The ocean in polar regions plays a vital role, affecting the global biogeochemical cycle~\cite{SeaIceRole-Vancoppenolle-2013}, and ice-water boundary is an active region where transports between ice and water happen.  
Previously, underwater vehicles have been successfully deployed there for physical and biogeochemical measurement about sea ice thickness~\cite{IceThickness_Wadhams_2012} and micro algae communities~\cite{AlgalCommunities_Robison_2011}.
More recently, more and more under-ice AUV deployments ~\cite{Icefin-Spears-2016,Autosub2000_Phillips_2020,Icemap-Williams-2014} have been conducted since AUVs are becoming a promising candidate to produce a higher spatial coverage to fill the observation gaps left by traditional ice-anchored instruments and ice coring.
% 
However, the localization (or called state estimation) is particularly challenging in the underwater environment due to the lack of GPS~\cite{AUV-Localization-Review_Paull_2014}.
Therefore, the collected measurements are hard to georeferencing to study the spatial extents of the processes, e.g., the algae patchness, and sea-ice surface topography.
Under-ice AUV operation is an extreme case that AUVs could not return to the surface to obtain the GPS fixes for bounding the localization drift \cite{UnderIce-VehicleDesign_Barker_2020}. 

%% Challenge 1: challenging motion
%   - acoustic beacon
%   		- installation and calibration	
% 	- DVL: 
% 		- drift
% 		- more drift for low speed (SNR higher)  
Acoustic transducer arrays, such as Long Baseline (LBL) \cite{LBL-Underice_Jakuba_2008} and Ultra-short Baseline (USBL) \cite{USBL-Underice_Kukulya_2010}, are commonly used for under-ice navigation.
But, they will require installation and extrinsic calibration for transducers.
In the under-ice environment, the performance of acoustic communication devices may degrade due to up-bending sound propagation and ice keel blockage~\cite{UnderIce-AcousticCom_Freitag_2017}.
In contrast, self-contained underwater navigation methods are also researched as it requires fewer logistic operations.
The most common one is the dead-reckoning navigation~\cite{DVL-Nav_Whitcomb_1999} which fuses velocity measured by a DVL and an inertial measurement sensor.
However, due to sensor noise, this method suffers from unbounded position errors as small as 12 m per hour from the DVL-Inertial Odometry(DIO)~\cite{INS-DVL-Pressure_Zhao_2022}. 
We also have observed that if an AUV is experiencing irregular motions such as low-speed cruising, quick turns and hovering, the navigation drift will grow rapidly possibly due to the increased signal-to-noise (SNR) ratio in the IMU and DVL measurements. 

%% Challenge 2: challenging images
%%  - Visual:
%%    - dynamic illumination
%%		- motion blur
%%		- light attenuation 
%%    - limited visibility
%%  - VIO:
%% 		- more detail, workng with imu
%%		- still not stable
For the ice-water boundary exploration, vehicles would maintain a close distance (1 to 2 m) in order to collect vital measurements (such as, sea-ice roughness, light penetration and water density) to study the ice-water exchanges. 
The close distance makes it possible to use camera images to aid underwater localization for AUVs.
In recent years, Visual-Odometry(VO)/Simultaneous Localization and Mapping (SLAM) has drawn increased attention, serving the rapid development in robot autonomy.
%
To improve the robustness of state estimation, Visual-Inertial-Odometry (VIO)~\cite{VIO-Review_Huang_2019} has been widely used. 
However, underwater visual-based SLAM is more challenging because of the dynamic illumination, limited visibility, light obstruction, texture-less area and motion blur \cite{Underwater-VIO-Survey_Joshi_2019}. 
Also, VIO has a well-known issue that the metric scale is not observable if there is no acceleration excitation. 
In such case, additional range sensors (e.g., a scanning sonar)~\cite{SVIn2_Rahman_2019}) are needed for reliable and accurate SLAM solutions in the underwater environment.

%% Challenge 3: 
%% 	- realtime running onboard
The algorithm development always has to consider the trade-off between accuracy and computational cost when deploying on the robotics platform. 
Non-linear optimization based VIO (e.g. OKVIS~\cite{OKVIS_Leutenegger_2015} and VINS-Mono~\cite{VINS-Mono_Qin_2018}) allows for the reduction of error through relinearization but with a high computational cost. 
Filtering-based VIO (e.g. MSCKF~\cite{MSCKF_Mourikis_2007}, ROVIO~\cite{ROVIO_Bloesch_2017}) is proven to be efficient and accurate in resources constraint applications, e.g, the state-of-the-art ARCore~\cite{ARCore} running on mobile devices.
To deal with the degraded performance that may exist when only using the one-time linearization, observability-based methodology~\cite{OC-MSCKF_Huang_2008} can be applied to improve the consistency. 
More details about the relevant works can be found in Section II.

%% our work DVL-Pressure-VIO
In this paper, we present a multi-modal sensor fusion framework to address the two challenges in underwater environments, degenerate motion and challenging image conditions.
Our method fuses the measurements from DVL, IMU, camera and pressure sensor in the MSCKF framework to improve the robustness of state estimation.
Our main contributions\footnote{code: \url{https://github.com/GSO-soslab/msckf_dvio}} are:
\begin{itemize}
  \item A tightly-couple multi-sensor fusion method using DVL, IMU, camera and pressure sensors with resilience to sensor failures (e.g. lost visual track), presented in Section III.
  \item A new Keyframe-based state clone and marginalization strategies for better triangulation result, presented in Section III.F.
  \item A DVL point cloud enhanced feature position recovery with a new data association and estimation approach, presented in Section IV.
  \item Algorithm validation with a challenging under-ice data set, presented in Section V.
\end{itemize} 

%% Paper structure
The remaining paper is organized as follows: the next section reviews the related work on multi-sensor fused state estimation with an emphasis on underwater environments.
Section~\ref{sec:filter_description} introduces the filter implementation with multi-sensor setup and the implementation of the keyframe method. 
Section~\ref{sec:enhancement} presents our DVL-aided feature enhancement approach.
Section~\ref{sec:experiment} presents the experiment results, and we will conclude the paper and discuss our future plans in Section~\ref{sec:conclusions}.