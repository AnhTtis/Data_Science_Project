% !TEX root = main.tex

%%%%%%%%%%%%%%%%%%%%%%%%%%%%%%%%%%%%%%%%%%%%%%%%%%%%%%%%%%%%%%%%%%%%%%%%%%%%%%%%
%%%%%                                Method                                %%%%%
%%%%%%%%%%%%%%%%%%%%%%%%%%%%%%%%%%%%%%%%%%%%%%%%%%%%%%%%%%%%%%%%%%%%%%%%%%%%%%%%

\section{Filter Description}
\label{sec:filter_description}

In this section, we will present the tightly-coupled multi-sensors fusion based on the state-of-art MSCKF \cite{MSCKF_Mourikis_2007} framework.
For underwater robots, we further present the measurement update equations for the DVL and pressure sensor, followed by our keyframe selection strategy.

\subsection{State Vector} 
We follow the notation in~\cite{OpenVINS_Geneva_2020} and define the system state to be $\mathbf{x}_k$ at the time step, $k$.
As shown in Eq.~\ref{eq:states} to \ref{eq:states_3}, the system state consists of the current IMU state, $\mathbf{x}_{IMU}$ and the clone state, $\mathbf{x}_{C}$, which contains $n$ past IMU poses.
In Eq.~\ref{eq:states_2} and \ref{eq:states_3}, ${^{I_k}_{G}}\bar{q}$ \cite{IndirectKF-Trawny-2005} is the unit quaternion representing the rotation from the global frame $\{G\}$ to the IMU frame $\{{I_k}\}$ at time $k$,
${^G}\mathbf{p}_{I_k}$ and ${^G}\mathbf{v}_{I_k}$ are the IMU position and velocity with respect to $\{G\}$, 
$\mathbf{b}_g$ and $\mathbf{b}_a$ describe the biases of the angular velocity and linear acceleration measured by the gyro and accelerometer in an IMU. 
The cloned IMU poses are denoted by $\{{^{I_i}_{G}}\bar{q}$ and $ {^G}\mathbf{p}_{I_i}\}, i \in [1,n]$.
\vspace{-2ex}
\begin{align}
\mathbf{x}_k  &= 
        \left[ 
               \begin{array}{cc} 
                 \mathbf{x}_{IMU}^\top  & 
                 \mathbf{x}_{Clone}^\top 
               \end{array} 
        \right]^\top 
        \label{eq:states} \\
\mathbf{x}_{IMU} &= 
        \left[ 
               \begin{array}{ccccc} 
                 {^{I_k}_{G}}\bar{q}^\top  & 
                 {^G}\mathbf{p}_{I_k}^\top &
                 {^G}\mathbf{v}_{I_k}^\top &
                 \mathbf{b}_g^\top     &
                 \mathbf{b}_a^\top 
               \end{array} 
        \right]^\top 
        \label{eq:states_2} \\
\mathbf{x}_{Clone} &= 
        \left[ 
               \begin{array}{ccccc} 
                 {^{I_1}_{G}}\bar{q}^\top  & 
                 {^G}\mathbf{p}_{I_1}^\top &
                 \cdots &
                 {^{I_n}_{G}}\bar{q}^\top     &
                 {^G}\mathbf{p}_{I_n}^\top 
               \end{array} 
        \right]^\top 
        \label{eq:states_3}
\end{align}

In this paper, we define $\mathbf{x}=\hat{\mathbf{x}} \boxplus \tilde{\mathbf{x}}$, 
where $\mathbf{x}$ is the true state, $\hat{\mathbf{x}}$ is estimation, $\tilde{\mathbf{x}}$
is the error state, and the $\boxplus$ operation maps the vector to a given manifold
\cite{Manifold-Fusion_Hertzberg_2013}. For quaternions, we define the quaternion boxplus 
operation using the left quaternion error in Eq.~\ref{eq:q_manifold} where $\boldsymbol{\theta}$ is the Euler angles.
\begin{equation}
\label{eq:q_manifold}
    \bar{q} \boxplus \delta \boldsymbol{\theta} 
        \triangleq 
    \left[ \begin{array}{c} \dfrac{1}{2}\delta  \boldsymbol{\theta}  \\ 1 
           \end{array} 
    \right] \otimes \bar{q}
\end{equation}

\subsection{IMU Propagation}
The state is propagated from $k-1$ to $k$ time step using the generic nonlinear IMU kinematics model  \cite{Fundamental-Inertial-Nav_Chatfield_1997} with IMU measurements, including linear accelerations (${^I}\mathbf{a}_m$) and angular velocities (${^I}\bm{\omega}_m$).
\begin{align}\label{eq:prop_nonlinear}
    \mathbf{x}_k = f(\mathbf{x}_{k-1}, {^I}\mathbf{a}_m, {^I}\boldsymbol{\omega}_m, \mathbf{n}_I)
\end{align}
where $\mathbf{n}_I = \left[ 
                        \begin{array}{cccc} 
                            \mathbf{n}_g^\top    & 
                            \mathbf{n}_a^\top    &
                            \mathbf{n}_{\omega g}^\top &
                            \mathbf{n}_{\omega a}^\top        
                        \end{array} 
                    \right]^\top $,
including the zero-mean Gaussian noise ($\mathbf{n}_g$ and $\mathbf{n}_a$) and the random walk bias noise ($\mathbf{n}_{\omega g}$ and $\mathbf{n}_{\omega a}$) for the gyroscope and accelerometer.
The estimated state and propagated covariance are:
\begin{align}\label{eq:propagated}
    \mathbf{\hat{x}}_{k|k-1} & = 
        f(\mathbf{\hat{x}}_{k-1|k-1}, {^I}\mathbf{a}_m, {^I}\boldsymbol{\omega}_m, \mathbf{0}) \\
    \mathbf{P}_{k|k-1}       & = 
        \boldsymbol{\Phi}_{k|k-1} \mathbf{P}_{k-1|k-1} \boldsymbol{\Phi}_{k|k-1}^\top +
        \mathbf{G}_{k-1} \mathbf{Q} \mathbf{G}_{k-1}^\top
\end{align}
where $\mathbf{\hat{x}}_{k|k-1}$ denotes the estimated state at time $k$ given the measurements at time $k-1$, $\boldsymbol{\Phi}_{k|k-1}$ and $\mathbf{G}_{k-1}$ are system Jacobian and noise Jacobian of the nonlinear system \cite{MSCKF_Mourikis_2007}, and $\mathbf{Q}$ is a discrete-time covariance matrix of IMU noise $\mathbf{n}_I$.

% \subsection{State Augmentation}

% When the image keyframe is determined, IMU state will propagate forward to the timestep of the image. 
% The propagated inertial state is cloned to $\mathbf{x}_C$, and the covariance matrix is augmented using the equation below.

% \begin{align}\label{eq:cov_aug}
% \mathbf{P}_{k|k}     = \left[ 
%                           \begin{array}{c} 
%                              \mathbf{I}_{(15+6n)}       \\ 
%                              \mathbf{J}
%                           \end{array} 
%                          \right] 
%                          \mathbf{P}_{k|k}  
%                          \left[ 
%                           \begin{array}{c} 
%                              \mathbf{I}_{(15+6n)}       \\ 
%                              \mathbf{J} 
%                           \end{array} 
%                          \right]^\top                   
% \end{align}
% where the Jacobian $\mathbf{J}$ is given by:
% \begin{align}\label{eq:cov_aug_J}
% \mathbf{J} = 
%   \left[ 
%     \begin{array}{cccc} 
%        \mathbf{I}_{3\times3} & \mathbf{0}_{3\times3}  &
%        \mathbf{0}_{3\times9} & \mathbf{0}_{3\times6n} \\ 
%        \mathbf{0}_{3\times3} & \mathbf{I}_{3\times3}  &
%        \mathbf{0}_{3\times9} & \mathbf{0}_{3\times6n}
%     \end{array} 
%   \right]                 
% \end{align}

\subsection{DVL Velocity Measurement Update}

% measurement function
The DVL velocity measurement is defined in Eq.~\ref{eq:dvl_measurement} which is a function of the linear and angular velocity in the IMU frame, the relative transformation (${^I_D}\mathbf{R}$ and ${^I}\mathbf{p}_D$) between the IMU frame and the DVL frame, and the rotation (${^{I_k}_G}\mathbf{R}$) from the global frame to the IMU frame.
We also have the measurement noise $\mathbf{n}_D \thicksim \mathcal{N}(\mathbf{0}, \mathbf{R}_D)$, and the skew-symmetric matrix of the IMU's angular velocity denoted by
$\lfloor {^{I_k}}\bm{\omega} \rfloor_\times$.
For EKF update, the Jacobian matrix $\mathbf{H}_{D,k}$ with respect to the state, $\mathbf{x}_k)$ can be found in \cite{INS-DVL-Pressure_Zhao_2022}.
\begin{align}\label{eq:dvl_measurement}
    \mathbf{z}_{D,k} = h_D(\mathbf{x}_k)+ \mathbf{n}_D 
                     = {^I_D}\mathbf{R}^\top 
                     ({^{I_k}_G}\mathbf{R} {^G}\mathbf{v}_{I_k} + \lfloor {^{I_k}}\bm{\omega} \rfloor_{\times} {^I}\mathbf{p}_D)
                     + \mathbf{n}_D
\end{align}

\subsection{Pressure Measurement Update}
The pressure measurement can be written in Eq.~\ref{eq:z_measurement} where $\mathbf{s}=[\begin{array}{ccc} 0 & 0 & 1 \end{array}]$ used for selecting the third dimension, ${^P}\mathbf{P}_{in} = [\begin{array}{ccc} 0 & 0 & {^P}p_{in} \end{array}]^\top$ and ${^P}p_{in}$ are the pressure measurement at the initial position, ${^P}\mathbf{P}_{k} = [\begin{array}{ccc} 0 & 0 & {^P}p_{k} \end{array}]^\top$ and ${^P}p_k$ is the pressure measurement at timestamp $k$, the three rotation matrices (${^D_P}\mathbf{R}$, ${^I_D}\mathbf{R}$, and ${^{I_k}_G}\mathbf{R}^\top$) are used to transform the pressure measurement into the global frame, and $n_{p_z}$ is a zero-mean white Gaussian noise. 
For EKF update, the Jacobian matrix $\mathbf{H}_{p_z}$ with respect to the state, $\mathbf{x}_k)$, can be found in \cite{INS-DVL-Pressure_Zhao_2022}.
\begin{align}\label{eq:z_measurement}
    z_{p_z,k} 
        = h_{p_z}(\mathbf{x}_k)+ n_P
        = \mathbf{s} {^{I_k}_G}\mathbf{R}^\top {^I_D}\mathbf{R} {^D_P}\mathbf{R} 
           ({^P}\mathbf{P}_{in} - {^P}\mathbf{P}_{k}) + n_{p_z}
\end{align}

\subsection{Visual Measurement Update}

We perform point feature tracking on a selected image and use the feature tracking results in multiple sequential images (or keyframe images) to update the system state and covariance. 
For a well-calibrated camera, the measurement of a feature in the camera frame $\{C_i\}$ is the perspective projection of its 3D position 
$\prescript{C_i}{}{\mathbf{p}}_{f} =[\prescript{C_i}{}{x},\prescript{C_i}{}{y},
\prescript{C_i}{}{z}]^\top$ onto the normalized plane, which is given by Eq.~\ref{eq:cam_measurement} where $\prescript{C_i}{}{\mathbf{p}}_{f}$ is given by Eq.~\ref{eq:pf}.
In Eq.~\ref{eq:cam_measurement} and \ref{eq:pf}, $\pi(\cdot)$ is the perspective projection function, function $\tau(\cdot)$ transforms a point based on the given transformation matrix,
$\mathbf{n}_C$ is the zero-mean white Gaussian noise for camera measurement,
$\{ \prescript{C}{I}{\mathbf{R}}, \prescript{C}{}{\mathbf{p}}_{I} \}$ is the extrinsic calibration between IMU and camera,
$\{ \prescript{I_k}{G}{\mathbf{R}}, \prescript{G}{}{\mathbf{p}}_{I_k} \}$ 
is the cloned IMU pose at image time $k$,
$\prescript{G}{}{\mathbf{p}}_{f}$ is the feature's 3D position in the global frame. 
\begin{align}\label{eq:cam_measurement}
\mathbf{z}_{C,i} 
  &= \pi(\prescript{C_i}{}{\mathbf{p}}_{f}) + \mathbf{n}_C 
   = \left[ 
        \begin{array}{c} 
           \prescript{C_i}{}{x}/\prescript{C_i}{}{z} \\
           \prescript{C_i}{}{y}/\prescript{C_i}{}{z}
        \end{array} 
     \right] + \mathbf{n}_C   
\end{align}
\vspace{-3ex}
\begin{align}\label{eq:pf}
\prescript{C_i}{}{\mathbf{p}}_{f}
  = \tau(\prescript{G}{}{\mathbf{p}}_{f},  
     \prescript{C}{I}{\mathbf{T}} \prescript{I_k}{G}{\mathbf{T}} ) 
  = \prescript{C}{I}{\mathbf{R}} \prescript{I_k}{G}{\mathbf{R}} 
     (\prescript{G}{}{\mathbf{p}}_{f} - \prescript{G}{}{\mathbf{p}}_{I_k}) 
     + \prescript{C}{}{\mathbf{p}}_{I}
\end{align}

For each feature, we compute the residual using the Jacobian matrices of the measurement function (Eq.~\ref{eq:cam_measurement}) with respect to the state and feature ($\mathbf{H}_{x}$ and $\mathbf{H}_{f}$)
\begin{align}\label{eq:res_cam_linearized}
\mathbf{\tilde{z}}_{C_k} 
  = \mathbf{H}_{x} \tilde{\mathbf{x}}_k +
    \mathbf{H}_{f} \prescript{G}{}{\tilde{\mathbf{p}}}_{f} +
    \mathbf{n}_C              
\end{align}
Next, we apply the left nullspace~\cite{MSCKF_Mourikis_2007} of the feature Jacobian to convert Eq.~\ref{eq:res_cam_linearized} to \ref{eq:res_cam_nullspace}, and used it for EKF update.
\begin{align}\label{eq:res_cam_nullspace}
\mathbf{N}^\top\mathbf{\tilde{z}}_{C_k} 
  = \mathbf{N}^\top\mathbf{H}_{x} \tilde{\mathbf{x}}_k +
    \mathbf{N}^\top \mathbf{n}_C              
\end{align}

\subsection{Keyframe Marginalization}

During underwater exploration, the vehicle may hover or move closer for detailed visual investigation.
While hovering will cause a small translation between two consecutive image frames, the vertical movements will cause a small feature disparity within two successful tracking steps.
Both cases will lead to degraded triangulation results, which brings in bad visual updates for state estimation.
In~\cite{MSCKF-Hovering_Kottas_2013}, the authors propose a marginalization strategy that updates features in a consistent way for the hovering case.
However, in a more general way, we intend to use keyframe marginalization to handle more degenerate motions.
Inspired by the keyframe-based visual SLAM~\cite{ORB-SLAM3_Campos_2021}, we have implemented a strategy to insert the IMU clones based on three criteria, feature numbers, motion constraint, and scene constraint.
If a new image has more than 50 features detected, its translation (estimated from DVL-IMU fused odometry) has exceeded 0.1m, and more than 10\% of the features from the previous frame have disappeared, this new image will be treated as a keyframe.

When we reach the maximum number of keyframes, the algorithm will perform a marginalization similar to the one in S-MSCKF~\cite{S-MSCKF_Sun_2018}. 
While the standard MSCKF \cite{MSCKF_Mourikis_2007} uses the feature measurements in several poses (one-third of slide window evenly spaced), we will only use the feature measurements in two keyframes (the oldest one and the second-latest shown in Fig.~\ref{fig:feat_marg}) for standard MSCKF-feature update to reduce the computational cost.
We also implemented another change during marginalization.
Instead of removing two poses that were used in marginalization from the state matrix, our method will keep the second-latest pose even though it was used for marginalization.
The second-latest pose is saved because of the following reasons.
Any non-max-tracked features (i.e., $f_{j+1}$ in Fig.~\ref{fig:feat_marg}) detected in second-latest pose will have minimum impact on the update because these features only have one measurement when doing the marginalization.
However, keeping these features and the pose will allow us to use them in the future for sufficient MSCKF-feature updates.

\vspace{-3ex}
\textbf{\begin{figure}[h] \centering
\includegraphics[width=0.32\textwidth,height=0.16\textwidth]{images/feat_marg.png}
    \caption{The feature marginalization when clones at maximum.} \label{fig:feat_marg}
\end{figure}}
