% !TEX root = main.tex

%%%%%%%%%%%%%%%%%%%%%%%%%%%%%%%%%%%%%%%%%%%%%%%%%%%%%%%%%%%%%%%%%%%%%%%%%%%%%%%%
%%%%%                           		Experiment                         		 %%%%%
%%%%%%%%%%%%%%%%%%%%%%%%%%%%%%%%%%%%%%%%%%%%%%%%%%%%%%%%%%%%%%%%%%%%%%%%%%%%%%%%

\begin{figure}[h]
    \centering
    \includegraphics[scale=0.35]{images/rov}
    \caption{The modified BlueROV-2 used in the  experiment.}
    \label{fig:rov}
\end{figure}
% \vspace{-2ex}

\section{Experiment Results}
\label{sec:experiment}


\begin{figure*}[t] \centering
 \vspace{-5ex}
 \includegraphics[width=0.99\textwidth,height=0.2\textwidth]{images/metashape.png}
    \vspace{-6ex}
    \caption{The Metashape reconstructed result. The largest ice-hole on the left side is the starting point for the vehicle, the total length of this reconstructed result is roughly 40 meters} \label{fig:metashape}
\end{figure*}

\subsection{Experiment data set}
% hardware 
In March 2021, we have conducted an under-ice experiment under the frozen Keweenaw Waterway in Michigan using a modified BlueROV2~\cite{UnderIce-ROV_Zhao_2021} with a suite of sensors shown in Fig.~\ref{fig:rov}. 
The ice thickness is about 30 cm.
An ice hole (about 1m by 1m) was cut for deploying the ROV while several small ice holes (shown in Fig. \ref{fig:metashape}) were also drilled along the transect. They are spaced at 10 meters except the second one.
During the experiment, the ROV is remotely controlled by the pilot to drive along a straight line multiple times (roughly 40 meters each way) from a position at 4 meters deep, resulting in a total traveling distance of about 200 m and a total duration of about 20 minutes.

We used the experimental data set to validate our proposed sensor fusion framework.
In the data set, the up-looking stereo-camera is running at 15 Hz with a raw image size of 1616 by 1240 pixels. 
Only one camera was used for this experiment. The upward-looking DVL is pinging at 4 Hz, the IMU is running at 100 Hz, and the pressure sensor on the DVL is sampling at 2 Hz.
The standard deviation (SD) of the DVL single ping at 3 m/s is about 0.005 m/s from Norteck technical specification. During the data collection, the vehicle moves at about 0.4 m/s and the transformation between DVL and IMU is roughly measured. Therefore, we set the velocity SD to between 0.0375 - 0.1 m/s for the experiment.
The original result shown in \cite{UnderIce-ROV_Zhao_2021} used the robot localization without correcting the time delays (about 10 seconds) between the IMU and DVL due to the DVL driver issue.
Even though the localization in \cite{UnderIce-ROV_Zhao_2021} shows a low drift, it may be a coincidence.
In this data set, we have corrected the delays during the validation process.
One unique feature in this data set is that, occasionally, the ROV is controlled to hover in place.
Such maneuvers will challenge the visual SLAM performance since during the hovering no significant translation is available for feature triangulation. 
In application, hovering may be needed in several key locations during an under-ice exploration to collect more measurements on abnormal biogeochemical processes, e.g., a salt brine injection and algae bloom.

\subsection{Results}
The ground truth vehicle path is generated using Agisoft Metashape based on SfM technique, a rendering of the ice surface is shown in Fig.~\ref{fig:metashape}.
For comparison, we use the evo~\cite{evo_Grupp_2017} toolbox to align (recovery orientation and scale) Metashape ground truth and estimated odometry created from different sensor fusion methods.
We only selected a short amount of time (90 seconds about 10 meters) at the beginning for alignment.
Herein, we compare the localization results from 10 settings, as shown in Table~\ref{tab:odom_setting} against the ground truth path.
\begin{table}[h]
    \caption{Setup with different sensor suites and features. "Y" means used and "N" means not used.}
    \label{tab:odom_setting}
    \centering
    \begin{tabular}{c|c|c|c|c|c|c|c|c|c|c}
    \hline
    \textbf{Case \#} &\textbf{1} &\textbf{2} &\textbf{3} &\textbf{4} &\textbf{5} 
                     &\textbf{6} &\textbf{7} &\textbf{8} &\textbf{9} &\textbf{10} \\
    \hline
    Visual      &Y  &Y  &Y  &Y  &Y  &N  &\textit{Y}  &\textit{Y}  &\textit{Y}  &\textit{Y}  \\    
    \hline
    DVL         &Y  &Y  &Y  &Y  &Y  &Y  &\textit{N}  &\textit{N}  &\textit{N}  &\textit{N}  \\
    \hline
    IMU         &Y  &Y  &Y  &Y  &Y  &Y  &\textit{Y}  &\textit{Y}  &\textit{Y}  &\textit{Y}  \\
    \hline
    Pressure    &Y  &Y  &Y  &Y  &N  &Y  &\textit{N}  &\textit{N}  &\textit{Y}  &\textit{Y}  \\
    \hline
    Enhancement &Y  &N  &Y  &N  &Y  &N  &\textit{N}  &\textit{N}  &\textit{N}  &\textit{N}  \\
    \hline
    Keyframe    &Y  &Y  &N  &N  &Y  &N  &\textit{Y}  &\textit{N}  &\textit{Y}  &\textit{N}  \\
    \hline
    \end{tabular}
\end{table} 
% \vspace{-2ex}

For all tracks, we used identical parameters in the MSCKF and system initialization is conducted using the method from~\cite{INS-DVL-Pressure_Zhao_2022}. 
We used CLAHE~\cite{CLAHE_Pizer_1987} with KLT~\cite{KLT_Lucas_1981} method for the front-end feature tracking because the descriptor based methods, such as the ORB and KAZE, didn't provide us with a consistent tracking result. 
We found that the descriptor-based method can be confused by the air bubbles in the ice which appear in similar shapes and sizes.
We present all the resulting vehicle paths estimated from case $1\thicksim6$ in Fig.~\ref{fig:plot_result}(a) with different colors. Noted that VIO options (case $7\thicksim10$) are not visualized since those runs failed quickly at the beginning because of the hovering maneuvers.  
From Fig.~\ref{fig:plot_result}(a), we can easily observe that the odometry generated without the visual assist (case 6) is drifting away from the ground truth. 
In contrast, the paths generated with visual assistant stay closer to the ground truth path, especially, during the first and the second transects.
We believe that the angle offset between tracks and the ground truth during the third and fourth segments may due to the hovering maneuverings (2-3 minutes) near the ROV deployment hole at the end of the second segment. 

% \vspace{-2ex}
% \textbf{\begin{figure}[h] \centering
% \includegraphics[width=0.5\textwidth,height=0.3\textwidth]{images/aligned_traj.png}
%     \caption{The resulting trajectories for all the cases.} \label{fig:aligned_traj}
% \end{figure}}
% \vspace{-2ex}

\begin{figure*}[t] \centering
 % \vspace{-1ex}
    \makebox[0.49\textwidth]{\small (a) }
    \makebox[0.48\textwidth]{\small (b) }
    % \vspace{+2mm}
    % \\
    \includegraphics[width=0.48\textwidth]{images/plot_trajectories.png} 
    \hspace{+2mm}
    \includegraphics[width=0.48\textwidth]{images/plot_errors.png}
    \caption{The evaluation result. (a) The aligned trajectories for case 1 to case 6. (b) The ATE translation RMSE for X-Y axis. } 
    \label{fig:plot_result}
\end{figure*}

To further compare the performance in different cases, we computed the Absolute Trajectory Error (ATE)~\cite{VIO-EVO_Zhang_2018} in X and Y and the X-Y plane between the ground truth path and aligned each odometry.
The statistical values are listed in Table~\ref{tab:rmse}
and the X-Y errors are presented in Fig.~\ref{fig:plot_result}(b).
Based on that, we could see that the integration of visual measurement into the MSCKF will help with reducing errors. 
Overall, the drift in the Y direction (transversal to the vehicle transects) is higher than in the x direction. 
This may be the fact that the vehicle's sway velocity is slightly small than its surge speed.
Therefore, a lower SNR may be expected in the transversal direction, causing the increased drift.

\begin{table}[h!]
  \begin{center}
    \caption{ATE with RMSE metric for different cases.}
    \label{tab:rmse}
    \begin{tabular}{c|c|c|c|c|c|c} 
      \hline
      \textbf{Case \#} &\textbf{1} &\textbf{2} &\textbf{3} &\textbf{4} &\textbf{5} &\textbf{6}\\
      \hline
      RMSE(X)   &\textbf{0.39}  &0.42  &1.34  &1.23   &1.45  &3.54  \\
      RMSE(Y)   &\textbf{1.82}  &1.86  &2.09  &2.29   &1.45  &2.87  \\
      RMSE(X-Y) &\textbf{1.11}  &1.14  &1.71  &1.76   &1.45  &3.21  \\
      \hline
    \end{tabular}
  \end{center}
\end{table}
\vspace{-1ex}

When comparing the statistical values in Table~\ref{tab:rmse}, we have several findings.
First, visual-fused solutions are better than case 6 which only used the DVL, IMU, and pressure measurements.
Second, case 1 and 2 are better than case 3 and 4. 
This comparison allows us to highlight the benefit of having keyframe selection mechanism which allows a longer translation for a better result in feature triangulation, ultimately affecting the localization.
Third, case 1 is better than case 5 means pressure update actually helped the 2D pose estimation.
Fourth, our method (case 1 with DVL-aided feature enhancement and keyframe selection enabled) produced the lowest RMSE. 
However, case 2 (with DVL-aided feature enhancement disabled but keyframe selection enabled) is only slightly worse than case 1.
This small improvement may be mainly caused by two following reasons.
First, the detected visual features are too close to the vehicle (roughly between 1-2 meters). 
Therefore, the feature's position in z corrected by the DVL measurements is relatively small (even though the improvement is visible in Fig.~\ref{fig:depth_enhance}), resulting in a small impact on the state estimation.
Second, the feature measurements noise is set to 0.09 pixel which is relatively high compared to 0.0035 we set when testing the VIO on simulated data from OpenVINS.
We also tried 0.01 for our data, the localization error was larger than the shown result.
Therefore, we think there are still room for improvement, especially, in the front-end feature tracking.

% \begin{figure}[h]
%     \centering
%     \includegraphics[scale=0.35]{images/error.png}
%     \caption{The errors in X-axis, Y-axis and X-Y plane.}
%     \label{fig:errors}
% \end{figure}
% \vspace{-2ex}




% init alignment RMSE:
%   - DIPO:  0.04372366500998966
%   - VDIPO: 0.042506709947047125
%   - VDIPO-wo-enhance: 0.041838946871800786
%   - VDIPO-wo-enhance-keyframe: 0.04293661481755104
%   - VDIPO-wo-keyframe: 0.04353596573555296 
