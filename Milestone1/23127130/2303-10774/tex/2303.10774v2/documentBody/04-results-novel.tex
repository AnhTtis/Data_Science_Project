\begin{figure}[!tb]
    \centering
    \includegraphics[width=1.0\linewidth]{figures/novel.pdf}
    \caption{
    Visualizing novel attributes in different client GANs characterized by challenging distribution shifts with respect to the reference GAN (CelebA or AFHQ Cat-GANs). In each case, we show image manipulation in the attribute direction for two random sample from the latent space. 
    }
    \vspace{-2mm}
    \label{fig:cats_vs_dogs}
\end{figure}

\subsection{Evaluation: Novel/Missing Attribute Discovery}
In this section, we study the effectiveness of xGA in discovering novel (only present in the client) and missing (only present in the reference model) attributes. 

\myparagraph{Qualitative results} 
We first show results for novel attribute discovery for different client GANs in Figure \ref{fig:cats_vs_dogs}. xGA produces highly intuitive results by identifying attributes that are unlikele to occur in the reference GAN. For example, ``cartoon eyes'' and ``sculptures'' are found to be unique to Disney and Met Faces GANs, when compared to CelebA. Next, we performed missing attribute discovery from the controlled CelebA experiments, where we know precisely which attribute is not encoded by the client GAN w.r.t the reference (standard CelebA StyleGANv2). As described earlier, the client models are always trained on a subset of data used by the reference model and by design, there are no novel attributes. Figure \ref{fig:leave_out_celeba} shows examples for the different missing attributes. We find that xGA successfully reveals each of the missing client attributes, even though the data distributions $P_c(\mathrm{x})$ and $P_r(\mathrm{x})$ are highly similar (except for a specific missing attribute). 


\myparagraph{Quantitative results } 
To benchmark xGA in missing attribute discovery, we use the $7$ controlled CelebA client models and audit with respect to the reference CelebA GAN. We denote the set of attributes (one or more) which are explicitly excluded in each client model by ${\mathcal{M}}$. In order to evaluate how well xGA identifies the excluded attributes, we introduce a metric based on mean reciprocal rank (MRR) \cite{radev2002evaluating,voorhees1999trec}. For each of the $N_m$ missing attributes from xGA, we compute the average semantic discrepancy from the ``oracle'' attribute classifier as, 
$$\mathrm{a}^n = \mathbb{E}_i[|\mathcal{C}(\mathcal{G}_c(\mathrm{z}_i, \delta_n)) - \mathcal{C}(\mathcal{G}_c(\mathrm{z}_i))|].$$Denoting the rank of a missing attribute $m \in \mathcal{M}$ in the difference vector $\mathrm{a}^n$ as $\text{rank}(m,\mathrm{a}^n)$, we can define the attribute recovery (for both missing/novelty) score as:
\begin{equation}
\mathcal{R}_{\text{Score}} = \mathbb{E}_m  \bigg[\underset{n} {\max} \left( \frac{1}{\text{rank}(m,\mathrm{a}^n)} \right)\bigg]
\label{eq:score_unique}
\end{equation}In Table \ref{tbl:celeba_metrics_common_novel}, we show results for missing attribute discovery based on this score. We observe that xGA significantly outperforms all baselines in identifying the missing attribute across the suite of client GANs.

\begin{figure}[!t]
    \centering
    \includegraphics[width=0.99\linewidth]{figures/leave_n_out_CelebA.pdf}
    \caption{Using multiple clients trained with different subsets of CelebA data (one of the face attributes explicitly dropped), we find that, in all cases, xGA accurately recovers the missing attribute.}
    \vspace{-4mm}
    \label{fig:leave_out_celeba}
\end{figure}