% CVPR 2023 Paper Template
% based on the CVPR template provided by Ming-Ming Cheng (https://github.com/MCG-NKU/CVPR_Template)
% modified and extended by Stefan Roth (stefan.roth@NOSPAMtu-darmstadt.de)
\documentclass[10pt,twocolumn,letterpaper]{article}
%%%%%%%%% PAPER TYPE  - PLEASE UPDATE FOR FINAL VERSION
%\usepackage[review]{cvpr}      % To produce the REVIEW version
%\usepackage{cvpr}              % To produce the CAMERA-READY version
\usepackage[pagenumbers]{cvpr} % To force page numbers, e.g. for an arXiv version
%
% These are recommended to typeset algorithms but not required. See the subsubsection on algorithms. Remove them if you don't have algorithms in your paper.
\usepackage{graphicx}
\usepackage{algorithm}
%\usepackage{algorithmic}
%\usepackage{svg}
%\usepackage{tikz}
\usepackage{comment}
\usepackage{amsmath,amssymb} % define this before the line numbering.
%\newcommand*{\xdash}[1][3em]{\rule[0.5ex]{#1}{0.55pt}}
%\newcommand{\sZ}{\ensuremath{\scalebox{.86}{Z}
%\kern-.41em\raisebox{.07em}{$\xdash[.3em]$}}}
%\newcommand{\sz}{\ensuremath{{z}
%\kern-.4em\raisebox{-.045em}{$\xdash[.3em]$}}}
\newcommand{\myparagraph}[1]{\vspace{.01ex} \noindent  \textbf{#1}}
\usepackage{dsfont}
\usepackage{booktabs}
\usepackage{color}
\usepackage{bm}



% It is strongly recommended to use hyperref, especially for the review version.
% hyperref with option pagebackref eases the reviewers' job.
% Please disable hyperref *only* if you encounter grave issues, e.g. with the
% file validation for the camera-ready version.
%
% If you comment hyperref and then uncomment it, you should delete
% ReviewTempalte.aux before re-running LaTeX.
% (Or just hit 'q' on the first LaTeX run, let it finish, and you
%  should be clear).
\usepackage[pagebackref,breaklinks,colorlinks]{hyperref}


% Support for easy cross-referencing
\usepackage[capitalize]{cleveref}
\crefname{section}{Sec.}{Secs.}
\Crefname{section}{Section}{Sections}
\Crefname{table}{Table}{Tables}
\crefname{table}{Tab.}{Tabs.}


\newcommand{\ours}[0]{xGA}

%%%%%%%%% PAPER ID  - PLEASE UPDATE
\def\cvprPaperID{3939} % *** Enter the CVPR Paper ID here
\def\confName{CVPR}
\def\confYear{2023}


\begin{document}


\title{Cross-GAN Auditing: Unsupervised Identification of Attribute Level Similarities and Differences between Pretrained Generative Models}


\author{Matthew L. Olson\textsuperscript{\rm 1}, Shusen Liu\textsuperscript{\rm 2}, Rushil Anirudh\textsuperscript{\rm 2}, Jayaraman J. Thiagarajan\textsuperscript{\rm 2}, Peer-Timo Bremer\textsuperscript{\rm 2}, \\ and Weng-Keen Wong\textsuperscript{\rm 1} \\
\textsuperscript{\rm 1} Oregon State University - EECS, \textsuperscript{\rm 2}Lawrence Livermore National Laboratory 
 - CASC\\
{\tt\small \{olsomatt,wongwe\}@oregonstate.edu, \{liu42,anirudh1,jayaramanthi1,bremer5\}}@llnl.gov \\
}



\twocolumn[{%
\renewcommand\twocolumn[1][]{#1}%
\maketitle
\begin{center}
        \centering
        \includegraphics[width=0.90\linewidth]{figures/teaser.pdf}
        \captionof{figure}{
        We introduce (\textbf{xGA}) an approach for fully unsupervised cross-GAN auditing and validation. Given two pre-trained GANs (Reference \& Client), xGA evaluates the client by identifying three types of semantic attributes -- (a) Common: those that exist in both models, (b) Novel: those only present in the client and (c) Missing: those that only exist in the reference.  On the right, we show results across multiple studies, that among others include notable shifts in distribution between the Reference (CelebA) to Client (Toon, Disney, Met Faces). xGA also lends itself easily to comparing models with different properties on the same dataset as shown on the bottom right for StyleGAN3-T vs.\ StyleGAN-R. And CelebA-M is a control dataset we create that does not contain glasses, ties and smiles.
        }
        \label{fig:teaser}
\end{center}%
}]




\begin{abstract}
Generative Adversarial Networks (GANs) are notoriously difficult to train especially for complex distributions and with limited data. This has driven the need for tools to audit trained networks in human intelligible format, for example, to identify biases or ensure fairness. Existing GAN audit tools are restricted to coarse-grained, model-data comparisons based on summary statistics such as FID or recall. In this paper, we propose an alternative approach that compares a newly developed GAN against a prior baseline. To this end, we introduce \emph{Cross-GAN Auditing} ({x}GA) that, given an established ``reference" GAN and a newly proposed ``client" GAN, jointly identifies intelligible attributes that are either \emph{common} across both GANs, \emph{novel} to the client GAN, or \emph{missing} from the client GAN. This provides both users and model developers an intuitive assessment of similarity and differences between GANs. We introduce novel metrics to evaluate attribute-based GAN auditing approaches and use these metrics to demonstrate quantitatively that {x}GA outperforms baseline approaches. We also include qualitative results that illustrate the common, novel and missing attributes identified by {x}GA from GANs trained on a variety of image datasets\footnote{Source code is available at \url{https://github.com/mattolson93/cross_gan_auditing}}.  
\end{abstract}

%%%%%%%%%%%%%%%%%
%
%   Approach it from data 
%
%%%%%%%%%%%%%%%%%%



Humans are multimodal learners. We communicate with each other about things that we have experienced and knowledge we have gained using our senses---most commonly including sight as well as hearing, touch, smell, and taste. Our communication channel is limited to a single modality---spoken language, signed language, or text---but a reader or listener is expected to use his or her imagination to visualize and reason about the content being described. In general, language is used to describe scenes, events, and images; the words used to describe these are used to conjure up a visual impression in the listener. Therefore, it is natural to consider the types of visual reasoning used in understanding language, and to ask how well we can currently model them with computational methods.

Consider, for instance, the questions in Figure \ref{fig:teaser}. Concreteness is typically correlated with how well a concept can be visually imagined. For example, a concrete word such as \emph{present} often has a unique visual representation. In addition, common associations such as \emph{ocean}$\rightarrow$\emph{blue} (color) and \emph{corn chip}$\rightarrow$\emph{triangle} (shape) reflect properties of an imagined visual representation of the item in question. These properties may be difficult to infer from text alone without prior knowledge gained from visual input; for instance, a number of studies have investigated the partial ability of blind English speakers to predict color associations and how it differs from the intuition of sighted speakers\footnote{This phenomenon is illustrated in \href{https://www.youtube.com/watch?v=59YN8_lg6-U}{this interview} with Tommy Edison, a congenitally blind man, in which he describes his understanding and frequent confusion regarding color associations.}~\cite{van2021blind, saysani2021seeing, saysani2018colour, shepard1992representation, marmor1978age}. 

There has been a wealth of recent research vision-and-language (V\&L) tasks involving both text and image data, and the use of vision-language pretraining (VLP) to create models that are able to reason jointly about both of these modalities together~\cite{chen2020uniter,kim2021vilt,li2021align,chen2022vlp}. Notable in this regard is CLIP~\cite{radford2021learning}, consisting of paired text and image encoders jointly trained on a contrastive objective, that learns to align text and image embeddings in a shared semantic space. On the other hand, text encoder models such as BERT~\cite{devlin2018bert} learn to reason about text in a unimodal vacuum, with knowledge derived from pretraining tasks that only involve textual data.




Prior work has investigated the performance of multimodally trained text encoders on various natural language understanding (NLU) tasks with mixed results, sometimes finding that they are outperformed by unimodal models~\cite{iki2021effect} and at other times suggesting improved performance~\cite{wang2021simvlm}. However, these works fine-tune the models under consideration on NLU tasks before evaluation, making it difficult to disentangle the effects of multimodal pretraining and fine-tuning configuration on the observed performance. Additionally, these works do not address the distinction between NLU tasks requiring implicit visual reasoning and ones that are purely non-visual. We refer to natural language inference involving implicit visual reasoning as \emph{visual language understanding} (VLU) and propose a suite of VLU tasks that may be used to evaluate visual reasoning capabilities of pretrained text encoders, focusing primarily on zero-shot methods.

We compare multimodally trained text encoders such as that of CLIP to BERT and other unimodally trained text encoders, evaluating their performance on our suite of VLU tasks. We evaluate these models in without modifying their internal weights in order to probe their knowledge obtained during pretraining. A key design aspect of these tests is the probing method used to evaluate knowledge. Previous work has probed the knowledge of BERT and similar models using a masked language modelling (MLM) paradigm~\cite{petroni2019language, rogers2020primer}, but this cannot be directly applied to CLIP since it was not pretrained with MLM. We therefore propose a new zero-shot probing method that we term \emph{Stroop probing}. This is based on the psychological Stroop effect~\cite{macleod1991half} (described in Section \ref{sec:probing}), which suggests that salient items should have a stronger interference effect on the representation of their context.

Strikingly, we find that the multimodally trained text encoders under consideration outperform unimodally trained text encoders on VLU tasks, both when comparing to much larger encoders as well as ones of comparable size. We also compare these models on baseline NLU tasks that do not involve visual reasoning and find that models such as CLIP underperform on these tasks, demonstrating that they do not have a global advantage on NLU tasks. We conclude that exposure to images during pretraining improves performance on text-only tasks that require visual reasoning. Furthermore, our findings isolate the effect of the text component of multimodal models for tasks such as text to image generation, providing principled guidelines for understanding the knowledge that such models inject into downstream vision tasks.



\section{Background}
\label{sec:background}

\subsubsection{Plagiarism in Programming Contests}

Source code plagiarism is a well-researched area of studies~\cite{marins2014survey}, however, the developed solutions are usually focused on finding plagiarism in homework assignments~\cite{novak2016academia, plugiarismreview2019},  
there are only a few works devoted to plagiarism in competitive programming~\cite{contest_bangladesh2021, contest_warsaw2019}. Even then, they are mainly focused on integrating a particular plagiarism detection tool into the online judging system, and not on comparing different existing tools. 

Unlike some other cases, source code in programming contests has its own specifics that directly relate to finding plagiarism in it. In addition to the usual plagiarism hiding techniques~\cite{comparison2009}, the solutions will have a lot of similar code that is natural for contests. Firstly, there is \textit{template} code, \textit{i.e.}, some common implementations of popular algorithms that the contestant copies into every solution. Secondly, there is code that will be almost the same in every solution to the given problem but not actually plagiarized, \textit{e.g.}, reading the input or printing the answer.
Currently, active research is underway on how to find similar code and reduce its influence on the similarity~\cite{common2016, common2020}.

The template code is particularly difficult to take into account, because it can be completely different for different contestants in both size and implementation. Also, functions from the template code can be actively used or ignored completely by the contestant, depending on a particular task, making it very difficult to automatically preprocess all submissions by simply removing all template code from them. It is clear that template code can easily become a weak spot for many algorithms, and this must be taken into account when using plagiarism detection tools on the contest code and when building a benchmark for their comparison.

\subsubsection{Source Code Plagiarism Detection Tools}

Many different tools were developed aimed at detecting source code plagiarism~\cite{plugiarismreview2019}. \textit{Text-based} algorithms treat a program as a simple text, without taking into account its programming language. The advantages of such approaches are language-independence and high performance~\cite{comparison2009}, however, this comes at the expense of lower accuracy. A popular text-based tool is Sherlock~\cite{sherlock}. This tool converts the file into a sequence of string tokens, hashes it, and extracts a subsample of hashes. To determine the similarity of two programs, Sherlock calculates the similarity of the sequences of hashes.

\textit{Token-based} algorithms run a specific lexer on the program and compare token streams. Such approaches are still fast but consider a deeper representation of the program. One of the earliest plagiarism detection tools, SIM~\cite{sim1999}, applies an algorithm for finding the maximum sequence alignment to the resulting token sequence of given programs. The similarity of two programs is then defined as their alignment score. Another popular token-based tool, JPlag~\cite{jplag2003}, defines the similarity as the percent of tokens from the first sequence that can be covered by tokens from the second one. A different token-based tool, MOSS~\cite{moss2003}, is based on comparing the fingerprints of programs. A fingerprint is constructed in three steps: (1) all the \textit{n}-grams for the token stream are built, (2) these \textit{n}-grams are hashed, and (3) to avoid comparing big sets of hashes, MOSS uses a \textit{winnowing} algorithm to select a certain subset of hashes for each program. The idea of winnowing has got popular, and several tools appeared that are based on it. One such tool is Dolos~\cite{dolos2022}. Unlike MOSS, Dolos is open-source, supports more programming languages, and provides powerful visualizations of the results.

Finally, \textit{graph-based} algorithms build a graph (usually, a program dependence graph) of the program, which shows the dependencies of the data within the program. To avoid the naive solving of an NP-hard problem, they use certain heuristics. For example, BPlag~\cite{bplag2021} uses the idea of a Greedy-String-Tiling algorithm to find similar parts in the graphs of two programs.

Overall, it can be seen that there exist a lot of approaches for finding plagiarism in the source code, however, given the specifics of programming contests, it is not clear how well they perform in such a setting. To evaluate the existing tools, to help researchers further improve them for competitive programming, as well as to provide a benchmark for future solutions, in this work, we aim to collect the first dedicated dataset of programming contest plagiarism.

\begin{figure*}[htbp]
\centering
    \includegraphics[width=\textwidth]{figures/pipeline.pdf}
    \centering
    \vspace{-0.5cm}
    \caption{The pipeline of the proposed approach for collecting the dataset.}
    \label{fig:pipeline}
    \vspace{-0.5cm}
\end{figure*}
\section{Methods}
We approach GAN auditing as performing attribute-level comparison to a reference GAN. For simplicity, we consider the setup where there is a single reference and client model to perform auditing, though xGA can be used even with multiple reference or client models (see experiments). Let us define the reference and client generators as $\mathcal{G}_r: \mathcal{Z}_r \mapsto \mathcal{X}_r$ and $\mathcal{G}_c: \mathcal{Z}_c \mapsto \mathcal{X}_c$ respectively. Here, $\mathcal{Z}_{r}$ and $\mathcal{Z}_{c}$ refer to the corresponding latent spaces and the generators are trained to approximate the data distributions $P_r(\mathrm{x})$ and $P_c(\mathrm{x})$. Our formulation encompasses the scenario where $P_r(\mathrm{x}) = P_c(\mathrm{x})$ but the model architectures are different, or the challenging setting of $P_r(\mathrm{x}) \neq P_c(\mathrm{x})$ (e.g., CelebA faces vs Met Faces datasets). 


\begin{figure}[t]
\centering
         
         \includegraphics[width=1.00\linewidth]{figures/xga_explained.png}
        \caption{
        A table showing the proposed xGA modifications to typical contrastive loss with a simple two attribute model.  
        }
        \label{fig:pos_neg_table}
\end{figure}

The key idea of xGA is to audit a client model $\mathcal{G}_c$ via attribute (i.e., directions in the latent space) comparison to a reference model, in lieu of computing summary scores (e.g., FID, recall) from the synthesized images. In order to enable a fine-grained, yet interpretable, analysis of GANs, xGA performs automatic discovery and categorization of latent attributes: (i) \textit{common}: attributes that are shared by both the models; (ii) \textit{missing}: attributes that are captured by $\mathcal{G}_r$, but not $\mathcal{G}_c$; (iii) \textit{novel}: attributes that are encoded in $\mathcal{G}_c$ but not observed in the reference model. We express this new categorization scheme in figure \ref{fig:pos_neg_table}. Together, these latent attributes can provide a holistic characterization of GANs, while circumventing the need for customized metrics or human-centric analysis.

\noindent \textbf{Latent attributes}: Following state-of-the-art approaches such as LatentCLR \cite{yuksel2021latentclr}, we define attributes as direction vectors in the latent space of a GAN. For any sample $\mathrm{z} \in \mathcal{Z}_c$ and a direction vector $\mathrm{\delta}_n$, we can induce attribute-specific manipulation to the corresponding image as
\begin{equation}
\label{eq:direction}
    \mathcal{D}: (\mathrm{z}, \mathrm{\delta}_n)\rightarrow \mathrm{z} + \alpha \mathrm{\delta}_n,\mbox{ where }\mathrm{\delta}_n = \frac{\mathbf{M}_n\mathrm{z}}{\lVert \mathbf{M}_n\mathrm{z}\rVert},
\end{equation}for a scalar $\alpha$, and a learnable matrix $\mathbf{M}_n$. In other words, we consider the attribute change to be a linear model defined by the learnable direction $\mathrm{\delta}_n$. The manipulated image can then be obtained as $\mathcal{G}_c(\mathcal{D}(\mathrm{z},\mathrm{\delta}_n))$, or in shorter notation $\mathcal{G}_c(\mathrm{z},\mathrm{\delta}_n)$. Note that these latent attributes are not pre-specified and are discovered as part of the auditing process.


\subsection{Common Attribute Discovery}
Identifying common attributes between the client and reference GAN models is challenging, since it requires that the latent directions are \emph{aligned}, i.e., the exact same semantic change must be induced in unrelated latent spaces. When distilling from a parent model, i.e., training Toons from  Faces, attributes appear to align naturally, even under severe distribution shifts~\cite{wu2021stylealign}.
However, this does not hold true when the two models are trained independently, which requires us to  solve the joint problem of identifying the attributes as well as explicitly aligning them.

Formally, for a common attribute, we want the semantic change (in the generated images) induced by manipulating any sample $\mathrm{z} \in \mathcal{Z}_c$ along a direction $\delta$ in the client GAN's latent space to match the change in the direction $\bar{\delta}$ from the reference GAN's latent space for any $\bar{\mathrm{z}} \in \mathcal{Z}_r$. In other words, $\mathrm{S}(\mathcal{G}_c(\mathrm{z},\delta), \mathcal{G}_c(\mathrm{z})) \approx \mathrm{S}(\mathcal{G}_r(\bar{\mathrm{z}}, \bar{\delta}),\mathcal{G}_r(\bar{\mathrm{z}})), \forall~z \in \mathcal{Z}_c, \bar{\mathrm{z}} \in \mathcal{Z}_r$. Here, $\mathrm{S}$ denotes an \textit{oracle} detector (e.g., human subject test) which measures the semantic changes between the original sample and that obtained by manipulating the common attribute.

However, in practice, such a semantic change detector is not accessible and we need to construct a surrogate mechanism to quantify the alignment, i.e., 
\begin{multline}
\label{eq:alignment_1}
     \min_{\mathrm{\delta}_n, \bar{\mathrm{\delta}}_n} \mathcal{L}\bigg(\mathcal{G}_c(\mathrm{z},\mathrm{\delta}_n), \mathcal{G}_r(\bar{\mathrm{z}}, \bar{\mathrm{\delta}}_n)\bigg), \forall \mathrm{z}\in \mathcal{Z}_c, \forall \bar{\mathrm{z}}\in \mathcal{Z}_r,
\end{multline}for a common attribute pair $(\mathrm{\delta}_n,\bar{\mathrm{\delta}}_n)$. Any choice of the loss function $\mathcal{L}$ must satisfy two key requirements: (a) identify high-quality, latent 
directions within each of the latent spaces; 
(b) encourage cross-GAN alignment such that similar attributes end up being strongly correlated under the loss function. For example, in the case of a single GAN, the LatentCLR~\cite{yuksel2021latentclr} approach learns distinct directions using a contrastive objective that defines positive samples as those that have all been perturbed in the same direction, while manipulations in all other directions are considered negative\footnote{Other single GAN methods could be adapted, but LatentCLR's flexible loss requires less computation without the need to enforce orthogonality at every learning step.}. However, this approach is not suitable for our setting because of a key limitation -- alignment requires us to operate in a common feature space so that semantics across the two models are comparable. To address this, we first modify the objective  to operate in the latent space of an external, pre-trained feature extractor $\mathcal{F}$. In order to support alignment even in the scenario where $P_c(\mathrm{x}) \neq P_r(\mathrm{x})$, we can choose $\mathcal{F}$ that is robust to commonly occurring distributional shifts. 

Our approach works on mini-batches of size $B$ samples each, randomly drawn from $\mathcal{Z}_c$ and $\mathcal{Z}_r$ respectively. For the $i^{\text{th}}$ sample in a mini-batch from $\mathcal{Z}_c$, let us define the vector $\mathrm{h}_i^n$ as the divergence between the output of the GAN before and after perturbing along the $n^{\text{th}}$ latent direction, computed in the feature space of $\mathcal{F}$, \textit{i.e.}, 
$h_i^n = \mathcal{F}(\mathcal{G}_c(\mathrm{z}_i,\mathrm{\delta}_n)) - \mathcal{F}(\mathcal{G}_c(\mathrm{z}_i)).$ Similarly, we define the divergence $\bar{\mathrm{h}}_j^n = \mathcal{F}(\mathcal{G}_r(\bar{\mathrm{z}}_j,\bar{\mathrm{\delta}}_n)) - \mathcal{F}(\mathcal{G}_r(\bar{\mathrm{z}}_j))$ for the reference GAN. Next, we measure the semantic similarity between the divergence vectors as 
$g(\mathrm{h}_i^n,\bar{\mathrm{h}}_j^n) = \exp(\mathrm{cos}(\mathrm{h}_i^n,\bar{\mathrm{h}}_j^n)/\tau),$ where $\tau$ is the temperature parameter, and $\mathrm{cos}$ refers to cosine similarity. Now, the loss function for inferring a common attribute can be written as

\begin{equation}
\begin{split}
\label{eq:xent_1}
    \displaystyle &\mathcal{L}_{\text{xent}}(\mathrm{\delta}_n,\bar{\mathrm{\delta}}_n, \lambda_a) = \\ 
    & -\log \frac{\sum\limits_{i=1}^B \sum\limits_{j\neq i}^B 
    g(\mathrm{h}_i^n,\mathrm{h}_j^n) + g(\bar{\mathrm{h}}_i^n,\bar{\mathrm{h}}_j^n)  + \lambda_{\text{a}}g(\bar{\mathrm{h}}_i^n,\mathrm{h}_j^n)
    }{\sum\limits_{
    \substack{
       i=1 \\
       j=1
      }
    }^B\sum\limits_{l=1}^{N}\mathds{1}_{[l\neq n]}\bigg( 
    g(\mathrm{h}_i^l,\mathrm{h}_j^n) + g(\bar{\mathrm{h}}_i^l,\bar{\mathrm{h}}_j^n) + g(\bar{\mathrm{h}}_i^l,\mathrm{h}_j^n) \bigg)}
\end{split}
\end{equation}
Here $N$ denotes the total number of attributes. While the first two terms in the numerator are aimed at identifying distinct attributes from $\mathcal{G}_c$ and $\mathcal{G}_r$, the third term enforces the pair $(\delta_n, \bar{\delta}_n)$ to induce similar semantic change. When the $\lambda_a$ parameter is set to $0$, this optimization reinforces self-similarity of the attributes, without cross-similar semantics.  The terms in the denominator are based on the negative pairs (divergences from different latent directions) to enable contrastive training.

\subsection{Novel \& Missing Attribute Discovery}
A key component of our GAN auditing framework is the discovery of interpretable attributes that are unique to or missing from the client GAN's latent space. This allows practitioners to understand the novelty and limitations of a GAN model with respect to a well-established reference GAN. To this end, we exploit the key intuition that images synthesized by manipulating an attribute specific to the client model can manifest as out-of-distribution (OOD) to the reference model (and vice versa).

In order to characterize the OOD nature of such realizations, we define a likelihood score in the feature space from $\mathcal{F}$, which indicates whether a given sample is out of distribution. More specifically, we use the Density Ratio Estimation (DRE) \cite{sugiyama2012density,Nam15} method that seeks to approximate the ratio: $\mathrm{\gamma}(\mathrm{x}) = \frac{P(\mathrm{x})}{Q(\mathrm{x})}$ for any sample $\mathrm{x}$. When the ratio is low, it is likely that $\mathrm{x}$ is from the distribution $Q$ and hence OOD to $P$. We choose DRE, specifically the Kullbeck-Liebler Importance Estimation Procedure (KLIEP) \cite{sugiyama2008}, over other scoring functions because it is known to be highly effective at accurately detecting outliers \cite{menon16}.

\begin{figure}[t]
\centering
         
         \includegraphics[width=1.00\linewidth]{figures/xga_diagram_detailed.png}
        \caption{
        A diagram of our xGA model. $\mathcal{G}_{r}$, $\mathcal{G}_{c}$, and $\mathcal{F}$ are fixed pretrained models. $\delta_n$ and $\bar{\delta}_n$ are direction models trained to learn aligned attributes between the two Generators using the features of $\mathcal{F}$, and $f_{dre}$ are regularization models for unique attributes. 
        }
        \label{fig:latentCLR}
\end{figure}


We pre-train two separate DRE models to approximate $\mathrm{\gamma}_c(\mathrm{z})$, and $\mathcal{\gamma}_r(\bar{\mathrm{z}})$, wherein we treat data from $\mathcal{F}(\mathcal{G}_c(\mathrm{z}))$ as $P$ and $\mathcal{F}(\mathcal{G}_r(\bar{\mathrm{z}}))$ as $Q$ for the former, and vice versa for the latter. These DRE models are implemented as 2-layer MLP networks, $f^c_{\text{dre}}(.), f^r_{\text{dre}}(.)$, such that 
\begin{equation}
\label{eq:DRE_models}
    \hat{\mathrm{\gamma}}_c(\mathrm{z}) = f^c_{\text{dre}}(\mathcal{F}(\mathcal{G}_c(\mathrm{z}))) \mbox{ and }  \hat{\mathrm{\gamma}}_r(\bar{\mathrm{z}}) = f^r_{\text{dre}}(\mathcal{F}(\mathcal{G}_r(\bar{\mathrm{z}}))),
\end{equation}where $\mathcal{F}$ is the same feature extractor from \eqref{eq:xent_1}. We pass the output of the MLPs through a softplus ($\varphi(\mathrm{x}) = \log(1+e^{\mathrm{x}})$) function to ensure non-negativity. As stated previously, we use the KLIEP method to train DRE models. Using Section 4.1 of \cite{menon16}, the KLIEP loss used for training is defined as:
\begin{equation}
\label{eq:dre_train}
\mathcal{L}^c_{\text{KLIEP}} = \frac{1}{T_{2}} \sum_{j=1}^{T_{2}} \hat{\mathrm{\gamma}}_c\left(\mathrm{\bar{\mathrm{z}}}_j\right) - \frac{1}{T_{1}} \sum_{i=1}^{T_{1}} \ln \hat{\mathrm{\gamma}}_c(\mathrm{z}_i),
\end{equation}where $\bar{\mathrm{z}}_j$ and $\mathrm{z}_i$ are random samples drawn from the latent spaces $\mathcal{Z}_r$ and $\mathcal{Z}_c$ respectively (with $T_1$ and $T_2$ total samples). Similarly, we can define the KLIEP loss term for the reference model as:
\begin{equation}
\label{eq:dre_train2}
\mathcal{L}^r_{\text{KLIEP}} = \frac{1}{T_{1}} \sum_{i=1}^{T_{1}} \hat{\mathrm{\gamma}}_r\left(\mathrm{\mathrm{z}}_i\right) - \frac{1}{T_{2}} \sum_{j=1}^{T_{2}} \ln \hat{\mathrm{\gamma}}_r(\bar{\mathrm{z}}_j).
\end{equation}

We also investigated using log-loss functions to train the DRE model, but found it to be consistently inferior to the KLIEP losses (see supplement for details). Finally, we use the pre-trained DRE models from the client and reference GAN data to identify novel and missing attributes, where for a given attribute $n$ in the reference GAN, we can enforce its uniqueness by utilizing the client DRE model to give us 
$ \mathcal{L}_{\text{Unique}}^r(\mathrm{\delta}_n) = \hat{\mathrm{\gamma_c}}(\mathrm{z}, \mathrm{\delta}_n ) $ 
and similarly for the client GAN we can use the reference DRE model 
$\mathcal{L}_{\text{Unique}}^c(\mathrm{\bar{\delta}}_n) = \hat{\mathrm{\gamma_r}}(\mathrm{\bar{z}}, \mathrm{\bar{\delta}}_n ) $ 
Note, we interpret the novel attributes from the reference GAN as the missing attributes for the client GAN. 

\subsection{Overall Objective}
We now present the overall objective of xGA to identify $N_c$ common, $N_n$ novel and $N_m$ missing attributes simultaneously. Denoting the total number of attributes $N = N_c + \text{max}(N_n, N_m)$, the total loss can be written as:
\begin{equation}
\begin{split}
\label{eq:full_xga}
    \nonumber \mathcal{L}_{\text{xGA}} = & \sum_{n=1}^{N} \mathcal{L}_{\text{xent}}(\delta_n, \bar{\delta}_n, \mathds{1}_{[n\leq N_c]} \lambda_a) \\
        \nonumber & + \lambda_b \bigg[ 
        \sum_{p=N_c+1}^{N_c + N_n} 
        \mathcal{L}_{\text{Unique}}^c(\mathrm{\bar{\delta}}_p)  
          + \sum_{q=N_c + 1}^{N_c + N_m} \mathcal{L}_{\text{Unique}}^r(\mathrm{\delta}_q)  
          \bigg] 
\end{split}
\end{equation}
Here, the hyper-parameter $\lambda_b$ is the penalty for enforcing the attributes between the two latent spaces to be disparate (missing/novel). And we set $g( . , . )=0$ in $L_{xent}$ if one of the directions vectors does not exist (i.e. when $N_n \neq N_m$).




 


\section{Experiments}
In order to systematically evaluate the efficacy of our proposed GAN audit approach, we consider a suite of GAN models trained using several benchmark datasets. In this section, we present both qualitative and quantitative assessments of xGA, and additional results are included in the Supplementary Material.

\subsection{Datasets and GAN Models}
\label{subsec:data_gans}
For most experiments, we used a StyleGANv2~\cite{karras2019style} trained on the CelebA \cite{liu2015faceattributes} dataset as our reference GAN model. This choice is motivated both by its wide-spread use as well as the availability of fine-grained, ground truth attributes for each of the face images in CelebA, and to ensure that this model is fully independent from other client GANs (e.g., ToonGAN is finetuned from FFHQ GAN). In one experiment for the AFHQ dataset, we used a StyleGANv2 trained using only \textit{cat} images from AFHQ as the reference. Also, we considered FFHQ-trained StyleGANv3~\cite{karras2017progressive} and non-StyleGAN architectures such as GANformer~\cite{hudson2021ganformer2} for defining the reference (see supplement).

In our empirical study, we constructed a variety of (StyleGANv2) client models and performed xGA: (i) $5$ trained with different CelebA subsets constructed by excluding images specific to a chosen attribute (hat, glasses, male, female and beard); (ii) $2$ trained with CelebA subsets constructed by excluding images containing any of a chosen set of attributes (beards$\mid$hats, smiles$\mid$glasses$\mid$ties); (iv) $3$ transferred GANs for Met Faces, cartoons~\cite{cartoonStyleGan22}, and Disney images \cite{cartoonStyleGan22} respectively.

\subsection{Training Settings}
In all our experiments, xGA training is carried out for $10,000$ iterations with random samples drawn from $\mathcal{Z}_c$ and $\mathcal{Z}_r$. We fixed the desired number of attributes to be $N_c = 12$, $N_n = 4$ and $N_m = 4$. Note, this choice was to enable training xGA on a single 15GB Tesla T4 GPU. With the StyleGAN2 models, our optimization takes $4$ hours; StyleGAN3 takes $12$ hours due to gradient check-pointing. For all latent directions $\{\delta_n\}$ and $\{\bar{\delta}_n\}$, we set $\alpha=3$ and this controls how far we manipulate each sample in a given direction. In each iteration, the effective batch size was $10$, wherein $2$ samples were used to construct a positive pair and a subset of $5$ directions were randomly chosen for updating (enforced due to memory constraints).  We used the Adam \cite{kingma2014adam} optimizer with learning rate $0.001$ to update the latent direction parameters. Note, all other model parameters (generators, feature extractor, DRE models) were fixed and never updated. Following common practice with StyleGANs, the attributes are modeled in the style space and the generator's outputs are appropriately resized to fit the size requirements of the chosen feature extractor.

For our optimization objective, we set the hyper-parameter $\lambda_a = 0.1$ in $\mathcal{L}_{\text{xGA}}$. To perform $\text{DRE}$ training, we used 2-layer MLPs trained via the Adam optimizer for $1000$ iterations to minimize the KLIEP losses specified in \eqref{eq:dre_train} and \eqref{eq:dre_train2}. At each step, we constructed batches of $32$ samples from both reference and client GANs, and projected them into the feature space of $\mathcal{F}$. Lastly, we set $\lambda_b = 1.0$; we explore tuning this parameter in the supplement, finding it to be relatively insensitive.



\subsection{Evaluation: Common Attribute Discovery}
We begin by evaluating the ability of xGA in recovering common attributes across reference and client models. As mentioned earlier, for effective alignment, the choice of the feature extractor is critical. More specifically, $\mathcal{F}$ must be sufficiently expressive to uncover aligned attributes from both client and reference models. Furthermore, it is important to handle potential distribution shifts across the datasets used to train the GAN models. Hence, a feature extractor that can be robust to commonly occurring distribution shifts is expected to achieve effective alignment via \eqref{eq:xent_1}. In fact, we make an interesting observation that performing attribute discovery in such an external feature space leads to improved disentanglement in the inferred latent directions. For all results reported here, we used a robust variant of ResNet that was trained to be adversarially robust to style variations~\cite{shu2021encoding}. Please refer to the ablation in Section \ref{sec:subsec_ablation} for a comparison of different choices.

\myparagraph{Qualitative results } In Figure \ref{fig:metface}, we show several examples of common attributes identified by xGA for different client-reference pairs, we observe that xGA finds non-trivial attributes. For example, the ``sketchify'' attribute which naturally occurs in Met Faces (a  dataset of paintings), is surprisingly encoded even in the reference CelebA GAN (which only consists of photos of people). We also show examples of other interesting attributes such as ``orange fur'' in the case of dog-GAN $\times$ cat-GAN or ``blonde hair'' in the case of Disney-GAN $\times$ CelebA-GAN.
%In Figure \ref{fig:teaser}, we show other examples across client and reference models that include a large distribution shift (Toon, CelebA) where the attribute corresponds to change in hair color across both domains. Finally, in Figure \ref{fig:cats_vs_dogs}, we show that this works even across seemingly unrelated datasets like Cats and Dogs.
These results indicate that our proposed alignment objective, when coupled with a robust feature space, can effectively reveal common semantic directions across the client and reference models. We include several additional examples in the supplement. 

\myparagraph{Quantitative results }
To perform more rigorous quantitative comparisons, we setup a controlled experiment using $7$ client models corresponding to different CelebA subsets (obtained by excluding images pertinent to specific characteristics). As discussed earlier, we use a standard CelebA StyleGANv2 as the common reference model across all $7$ experiments. Next, we introduce a score of merit for common attribute discovery based on the intuition that images perturbed along the same attribute will result in similar prediction changes, when measured through an ``oracle" attribute classifier \cite{liu2015faceattributes}. 

We first generate a batch of random samples from the latent spaces of client and reference GANs, and manipulate them along a common attribute direction $(\delta_n, \bar{\delta}_n)$ inferred using xGA. In other words, we synthesize pairs of original and attribute-manipulated images from the two GANs and for each pair, we measure the discrepancy in the predictions from an ``oracle' attribute classifier. Mathematically, this can be expressed as $\mathrm{a}^n_i = |\mathcal{C}(\mathcal{G}_c(\mathrm{z}_i, \delta_n)) - \mathcal{C}(\mathcal{G}_c(\mathrm{z}_i))|$ and $\bar{\mathrm{a}}^n_j = |\mathcal{C}(\mathcal{G}_r(\bar{\mathrm{z}}_j, \bar{\delta}_n)) - \mathcal{C}(\mathcal{G}_r(\bar{\mathrm{z}}_j))|$, where $\mathcal{C}$ is the attribute classifier trained using the labeled CelebA dataset. Finally, we define an alignment score that compares the expected prediction discrepancy across the two GANs using cosine similarity (higher value indicates alignment).
\begin{align}
\label{eq:cocosine}
\mathcal{A}_{\text{score}} = & \mathbb{E}_n \bigg[\cos\bigg(\mathbb{E}_i[\mathrm{a}^n_i], \mathbb{E}_j [\bar{\mathrm{a}}^n_j] \bigg)\bigg],
%\bm{\bar{a}}_{(n,1)}, \bm{\bar{a}}_{(n,2)}
%=  \frac {\bm{\bar{a}}_{(n,1)} \cdot \bm{\bar{a}}_{(n,2)}}{||\bm{\bar{a}}_{(n,1)}|| \cdot ||\bm{\bar{a}}_{(n,2)}||}
\end{align}where the inner expectations are w.r.t. the batch of samples and the outer expectation is w.r.t. the $N_c$ common attributes.
\begin{figure}[!htbp]
    \centering
    \includegraphics[width=1.0\linewidth]{figures/common.pdf}
    \caption{Visualizing common attributes discovered using xGA for different client-reference GAN pairs. For each case, we illustrate one common attribute (indicated by our description in green) with two random samples from the GAN latent space. } 
    \label{fig:metface}
\end{figure}


We implement $5$ baseline approaches that apply state-of-the-art attribute discovery methods to the client and reference GANs (independently), and subsequently peform greedy, post-hoc alignment. In particular, we consider SeFa \cite{shen2021closed}, Voynov \cite{voynov2020unsupervised}, LatentCLR \cite{yuksel2021latentclr}, Jacobian \cite{wei2021jacobian}, and Hessian \cite{peebles2020hessian} methods for attribute discovery. Given the attributes for the two GANs, we use predictions from the ``oracle'' attribute classifier to measure the degree of alignment between every pair of directions. For example, the pair with the highest cosine similarity score is selected as the first common attribute. Next, we use the remaining latent directions to greedily pick the next attribute, and this process is repeated until we obtain $N_c=12$ attributes. We compute the alignment score from \eqref{eq:cocosine} for all the methods and report results from the $7$ controlled experiments in Table \ref{tbl:celeba_metrics_common_novel}. Interestingly, we find that, despite using the ``oracle'' classifier for alignment, the performance of the baseline methods is significantly inferior to xGA. This clearly evidences the efficacy of our optimization strategy.

\begin{table}[!tb]
\centering
\begin{tabular}{rcc}
Method & $\mathcal{A}_{\text{score}}$ ($\uparrow$)& $\mathcal{R}_{\text{score}}$ ($\uparrow$) \\ \hline
SeFa + G. S        & $0.382 \pm{0.042}$           &  $0.167	\pm{0.165}$ \\
Voynov + G. S      & $0.544	\pm{0.033}$           & $0.254	\pm{0.246}$ \\
LatentCLR + G. S   & $0.543	\pm{0.031}$           & $0.297	\pm{0.326}$ \\
Hessian + G. S     & $0.567	\pm{0.065}$           & $0.224	\pm{0.273}$ \\
Jacobian + G. S    & $0.502	\pm{0.024}$           & $0.233	\pm{0.201}$ \\ \midrule
\ours & $\mathbf{0.660} \pm{0.147}$               & $\bm{0.411} \pm{0.193}$ \\
% \ours + Attr. Cls. & $0.338 \pm{0.208}$\\ \hline
\end{tabular}
\caption{\textbf{Common and Missing attribute discovery}. The average alignment scores from the $7$ controlled CelebA experiments. Note, we report both the mean and standard deviations ($\pm{\text{ std}}$) for each case, and ``G. S'' refers to the greedy strategy that we use for alignment.} 
\label{tbl:celeba_metrics_common_novel}
\end{table}





\begin{figure}[!tb]
    \centering
    \includegraphics[width=1.0\linewidth]{figures/novel.pdf}
    \caption{
    Visualizing novel attributes in different client GANs characterized by challenging distribution shifts with respect to the reference GAN (CelebA or AFHQ Cat-GANs). In each case, we show image manipulation in the attribute direction for two random sample from the latent space. 
    }
    \vspace{-2mm}
    \label{fig:cats_vs_dogs}
\end{figure}

\subsection{Evaluation: Novel/Missing Attribute Discovery}
In this section, we study the effectiveness of xGA in discovering novel (only present in the client) and missing (only present in the reference model) attributes. 

\myparagraph{Qualitative results} 
We first show results for novel attribute discovery for different client GANs in Figure \ref{fig:cats_vs_dogs}. xGA produces highly intuitive results by identifying attributes that are unlikele to occur in the reference GAN. For example, ``cartoon eyes'' and ``sculptures'' are found to be unique to Disney and Met Faces GANs, when compared to CelebA. Next, we performed missing attribute discovery from the controlled CelebA experiments, where we know precisely which attribute is not encoded by the client GAN w.r.t the reference (standard CelebA StyleGANv2). As described earlier, the client models are always trained on a subset of data used by the reference model and by design, there are no novel attributes. Figure \ref{fig:leave_out_celeba} shows examples for the different missing attributes. We find that xGA successfully reveals each of the missing client attributes, even though the data distributions $P_c(\mathrm{x})$ and $P_r(\mathrm{x})$ are highly similar (except for a specific missing attribute). 


\myparagraph{Quantitative results } 
To benchmark xGA in missing attribute discovery, we use the $7$ controlled CelebA client models and audit with respect to the reference CelebA GAN. We denote the set of attributes (one or more) which are explicitly excluded in each client model by ${\mathcal{M}}$. In order to evaluate how well xGA identifies the excluded attributes, we introduce a metric based on mean reciprocal rank (MRR) \cite{radev2002evaluating,voorhees1999trec}. For each of the $N_m$ missing attributes from xGA, we compute the average semantic discrepancy from the ``oracle'' attribute classifier as, 
$$\mathrm{a}^n = \mathbb{E}_i[|\mathcal{C}(\mathcal{G}_c(\mathrm{z}_i, \delta_n)) - \mathcal{C}(\mathcal{G}_c(\mathrm{z}_i))|].$$Denoting the rank of a missing attribute $m \in \mathcal{M}$ in the difference vector $\mathrm{a}^n$ as $\text{rank}(m,\mathrm{a}^n)$, we can define the attribute recovery (for both missing/novelty) score as:
\begin{equation}
\mathcal{R}_{\text{Score}} = \mathbb{E}_m  \bigg[\underset{n} {\max} \left( \frac{1}{\text{rank}(m,\mathrm{a}^n)} \right)\bigg]
\label{eq:score_unique}
\end{equation}In Table \ref{tbl:celeba_metrics_common_novel}, we show results for missing attribute discovery based on this score. We observe that xGA significantly outperforms all baselines in identifying the missing attribute across the suite of client GANs.

\begin{figure}[!t]
    \centering
    \includegraphics[width=0.99\linewidth]{figures/leave_n_out_CelebA.pdf}
    \caption{Using multiple clients trained with different subsets of CelebA data (one of the face attributes explicitly dropped), we find that, in all cases, xGA accurately recovers the missing attribute.}
    \vspace{-4mm}
    \label{fig:leave_out_celeba}
\end{figure}

\subsection{Analysis}
\label{sec:subsec_ablation}
In this section, we examine the key components of \ours{} to understand its behavior better. 

\myparagraph{Impact of the Choice of $\mathcal{F}$ } We start by studying the choice of the external, feature space used to perform attribute discovery. For this analysis, we consider the case where we assume $\mathcal{G}_r = \mathcal{G}_c$, wherein xGA simplifies to the standard setting of attribute discovery with a single GAN model (set $\lambda_b = 0$), such as SeFA and latentCLR. We make an interesting observation that, using a robust latent space  leads to improved diversity and disentanglement in the inferred attributes, when compared to the native latent space of StyleGAN. To quantify this behavior we consider two evaluation metrics based on the predictions for a batch of synthesized images $\mathcal{G}_c(\mathrm{z}, \delta_n)$ from the ``oracle'' attribute classifier. First, for each latent direction $\delta_n$, the average prediction entropy $\mathcal{H}_{\text{score}}$~\cite{liu2015faceattributes} is defined as:

\begin{equation}
\label{eq:h_score}
\mathcal{H}_{\text{score}} =  \mathbb{E}_n \bigg[\mathbb{E}_i \bigg[\texttt{Entropy}(\mathcal{C}(\mathcal{G}_c(\mathrm{z}_i, \delta_n)))\bigg]\bigg] 
\end{equation}

Second, the deviation in the predictions across all latent directions $\mathcal{D}_{\text{score}}$ is defined in \eqref{eq:d_score}, where $K$ is the total number of attributes in the ``oracle'' classifier $\mathcal{C}$:

\begin{equation}
\label{eq:d_score}
\mathcal{D}_{\text{score}} =  \sum_{k=1}^K \texttt{Variance}\bigg[ \bigg\{\mathbb{E}_i[\mathcal{C}(\mathcal{G}_c(\mathrm{z}_i, \delta_n)]\bigg\}_{n=1}^N \bigg]_k
\end{equation}

When the entropy is low, it indicates that the semantic manipulation is concentrated to a specific attribute, and hence disentangled. On the other hand, when the deviation is high, it is reflective of the high diversity in the inferred latent directions.
\begin{table}[!tb]
\centering
\resizebox{1.0 \columnwidth}{!}{
\begin{tabular}{rll}
Method            & $\mathcal{H}_{\text{score}}$ ($\downarrow$) & $\mathcal{D}_{\text{score}}$ ($\uparrow$) \\ \hline
SeFa  \cite{shen2021closed}                          & $4.006 \pm{0.259}$ & $1.031 \pm{0.077}$  \\
	
% SeFa (n=100) \cite{shen2021closed}                 & $5.811 \pm{0.327}$ &  $12345 \pm{0.1234}$   \\
LatentCLR \cite{yuksel2021latentclr}                  & $2.348 \pm{0.203}$ &   $0.749 \pm{0.929}$   \\ 
% LatentCLR (n=100)           & $3.818 \pm{0.084}$  & $12345 \pm{0.1234}$    \\ 
Voynov \cite{voynov2020unsupervised}                    & $2.508 \pm{0.069}$ &  $0.585 \pm{0.725}$    \\ 
Hessian \cite{peebles2020hessian}                    & $2.707 \pm{0.145}$ &  $0.642 \pm{0.795}$    \\ 
Jacobian \cite{wei2021jacobian}                    & $2.675 \pm{0.070}$ &  $0.661 \pm{0.826}$    \\ \midrule
\ours~(ViT)           &   $1.988 \pm{0.068}$                        &   $3.072 \pm{3.845}$   \\ 
\ours~(MAE ViT)     &   $2.102 \pm{0.035}$                          &   $3.103 \pm{3.814}$   \\ 
\ours~(CLIP ViT)     &   $2.091 \pm{0.041}$                         &   $3.135 \pm{3.901}$   \\
\ours~(ResNet-50)                  & $1.901 \pm{0.060}$             &   $3.111 \pm{3.852}$  \\ 
\ours~(Clip ResNet-50)               & $2.033 \pm{0.038}$           &   $3.121 \pm{3.863}$  \\ 
\ours~(advBN ResNet-50)           & $\mathbf{1.881} \pm{0.057}$     &  $\mathbf{3.153} \pm{3.904}$    \\ \hline
% \midrule
% \ours + Attr. Cls. (oracle)           & $1.858 \pm{0.054}$ &   $2.824	 \pm{3.497}$   \\
\end{tabular}
}
\caption{ 
\textbf{Choice of the feature space for attribute discovery}. Using an external feature space is superior to GAN's native style space, in terms of both entropy ($\times 100$) and deviation metrics. In this experiment, we set $\mathcal{G}_r = \mathcal{G}_c$, and aggregate the metrics from the set of controlled CelebA StyleGANs.
%The experiment uses the optimization described by LatentCLR, where we additionally use a variety of encoders rather than the GAN's own feature space. 
%While each encoder does relatively well, the CelebA attribute classifier, unsurprisingly, enables finding the most disentangled attributes, as its feature space is directly tuned for CelebA attributes.
}
%\vspace{-5mm}
\label{tbl:entropy_results}
\end{table}



For this analysis, we considered the following feature extractors for implementing xGA: (i) vanilla ResNet-50 trained on ImageNet \cite{he2016deep}; (ii) robust variant of ResNet-50 trained with advBN\cite{shu2021encoding}; (iii) ResNet-50 trained via CLIP \cite{radford2021learning}. Table \ref{tbl:entropy_results} shows the performance of the three feature extractors on attribute discovery with our $7$ CelebA GANs trained using different data subsets. Note, we scale all entropy and diversity scores by $100$ for ease of readability. We make a striking finding that, in terms both the entropy and deviation scores, performing attribute discovery in an external feature space is significantly superior to carrying out the optimization in the native style space (all baselines). As expected, LatentCLR produces the most disentangled attributes among the baselines, and regardless of the choice of $\mathcal{F}$, xGA leads to significant improvements. More importantly, the key benefit of xGA becomes more apparent from the improvements in the deviation score over the baselines. In the supplement, we include examples for the attributes inferred using all the methods. Finally, among the different choices for $\mathcal{F}$, the advBN ResNet-50 performs the best in terms of both metrics and hence it was used in all our experiments.

\begin{figure}[!tb]
     \centering
     \includegraphics[width=0.99\linewidth]{figures/1gan_example.pdf}
     \caption{Comparing xGA on single GAN attribute discovery with existing approaches, we find that more diverse and novel attributes can be found simply by using an external feature space. We exploit this for effective alignment across two GAN models. 
     }
     \label{fig:1gan_examples}
 \end{figure}
 
\myparagraph{Single GAN Qualitative Results}
Figure \ref{fig:1gan_examples} visualizes a shortened example of the top $3$ attributes (induce most changes in the ``oracle'' classifier predictions). This example show a clear improvement by using a pretrained feature extractor, as xGA identifies the most diverse semantic changes. Complete results, all discovered attributes for all methods, are shown in the supplement.


\myparagraph{Extending xGA to compare multiple GANs } Though all our experiments used a client model w.r.t a reference, our method can be readily extended to perform comparative analysis of multiple GANs, with the only constraint arising from GPU memory since all generators need to be loaded into memory for optimization. We performed a proof-of-concept experiment by discovering common attributes across $3$ different independently trained StyleGANs as shown in Figure \ref{fig:3gan}. For this setup, we expanded the cost function outlined in \eqref{eq:xent_1} to include $3$ pairwise alignment terms from the $3$ GANs to perform contrastive training, in addition to an extra independent term from the third model. While beyond scope for the current work, scaling xGA is an important direction for future work. 

\begin{figure}[!tb]
    \centering
    \includegraphics[width=0.99\linewidth]{figures/3gan.pdf}
    \caption{Common attributes identified using xGA with three different StyleGANs.} 
    \vspace{-2mm}
     \label{fig:3gan}
\end{figure}



While the results obtained are an encouraging first step towards open-set 3D semantic segmentation there are still many open questions to improve such approaches, some of which we discuss in the following.

Currently, the largest factor limiting segmentation performance is the quality of the vision-language features. While LSeg uses natural language features from CLIP trained on a very large dataset, the visual encoder is trained on the small closed-set ADE20K dataset. If we were able to compute dense pixel-aligned visual-language features from open-set web scraped data without requiring any human annotations, we believe that results could eventually surpass supervised learning methods. \cite{ranasinghe2022perceptual} presented some promising initial results on learning pixel aligned features without using segmentation masks or other expert annotations. 

In real-time experiments, our system relied on poses coming from a SLAM system. If many bad poses are computed by the SLAM system, the 3D representation could become corrupted by bad updates. Possible solutions include treating the sparse SLAM poses as initial guesses and optimizing the poses jointly with scene geometry, as in \cite{sucar2021imap, zhu2022nice}, or bad poses could be filtered out by analyzing the photometric or geometric error across frames. 

In robotics, downstream modules, such as motion planners and high-level planning systems, might benefit from a more explicit and principled representation of geometry than what we presented in this paper. For example, signed distance function based approaches \cite{wang2021neus} might provide better surface and occupancy reconstruction and have other favorable properties, such as the ability to compute the normal of a surface by differentiating through the distance function. For the time being, our method is limited to static scenes. Dealing with moving objects within scenes remains an open problem, but promising recent research \cite{kong2023vmap} suggests that extending neural implicit representations to dynamic scenes might be feasible.


%%%%%%%%% REFERENCES
{\small
\bibliographystyle{ieee_fullname}
\bibliography{00-references}
}
\newpage
\appendix

\section{Ablation study}
First, we investigate the effect of the $\lambda_b$ parameter on KLIEP loss that allows us to discover novel attributes. In addition to KLIEP loss presented in the main text, we analyze a model trained with simple log loss used to train binary classifiers to predict the likelihood of a given sample. 

\subsection{Log Loss Model}
 This model, we denote as LOG, is nearly identical to the DRE models except instead of a softplus final activation, it uses a sigmoid function $\sigma(x) = \frac{1}{1 + e^{-x}}$. 
These models are used to classify whether a given feature belongs to $\mathcal{G}_c(z)$ or $\mathcal{G}_r(\bar{z})$.  We pre-train two separate LOG models to approximate $\hat{\gamma}_c(x) = \hat{p}_c(x)$, and $\hat{\gamma}_r(x) =  \hat{p}_r(x)$, where we treat $\mathcal{G}_c$ as $P(\mathrm{x} | Y = 1)$ and  $\mathcal{G}_r$ as $P(\mathrm{x} | Y = 0)$. These LOG models are learned using simple 2-layer MLPs, $f^c_{LOG}(~), f^r_{LOG}(~)$, such that 
\begin{equation}
\label{eq:log_models}
    \hat{\mathrm{\gamma}}_c(\mathrm{z}) = f^c_{\text{LOG}}(\mathcal{F}(\mathcal{G}_c(\mathrm{z}))) \mbox{ and }  \hat{\mathrm{\gamma}}_r(\bar{\mathrm{z}}) = f^r_{\text{LOG}}(\mathcal{F}(\mathcal{G}_r(\bar{\mathrm{z}}))),
\end{equation} where $\mathcal{F}$ is the same Encoder model used in main paper's equation 3.

The loss used for training the LOG models is defined as follows:
\begin{equation}
\mathcal{L}^c_{\text{Log}} = \frac{1}{T_{2}} \sum_{j=1}^{T_{2}} - \log (1 - \hat{\gamma}_c\left(\mathrm{\bar{z}}_j\right) )
                           + \frac{1}{T_{1}} \sum_{i=1}^{T_{1}} - \log (\hat{\gamma}_c\left(\mathrm{z  }_i)\right)
\end{equation}



where $\mathrm{\bar{z}}_j$ and $\mathrm{z}_i$ are random samples drawn from the latent space of each generator. The loss term for the second model LOG model is 
\begin{equation}
\mathcal{L}^r_{\text{Log}} = \frac{1}{T_{1}} \sum_{j=1}^{T_{1}} - \log (1 - \hat{\gamma}_r\left(\mathrm{\bar{z}}_j\right) )
                           + \frac{1}{T_{2}} \sum_{i=1}^{T_{2}} - \log (\hat{\gamma}_r\left(\mathrm{z  }_i)\right)
\end{equation}

The LOG models $f^1_{LOG}(~), f^2_{LOG}(~)$ are  trained to minimize $\mathcal{L}^1_{\text{Log}},\mathcal{L}^2_{\text{Log}}$ respectively. 

Finally, the trained LOG models are used to minimize the loss in equation 7 (rather than DRE models); the objective in equation 7 remains the same.



\begin{table}[!tb]
\centering
\small{
\begin{tabular}{ccc}
$\lambda$ & $\mathcal{R}_{\text{Score}}$ (DRE loss) ($\uparrow$) & $\mathcal{R}_{\text{Score}}$ (Log loss )($\uparrow$)\\ \hline
0                                          & $0.42  \pm{0.38}$           & $0.42 \pm{0.38}$     \\
0.1                                       & $\bm{0.61} \pm{0.35}$       & $0.37 \pm{0.41}$     \\
0.2                                       & $0.54 \pm{0.33}$             & $0.44 \pm{0.39}$     \\
0.5                                       & $0.57 \pm{0.40}$            & $0.45 \pm{0.38}$     \\
1                                         & $\bm{0.61} \pm{0.33}$       & $0.40 \pm{0.40}$     \\
5                                         & $0.57 \pm{0.39}$            & $0.34 \pm{0.32}$    \\                     \hline
\end{tabular}
\caption{ The effect on the unique direction score when modifying the regularization $\lambda$ on the average $\mathcal{R}_{\text{Score}}$ ($\pm{\text{ std}}$) for the the 7 CelebA pairwise leave-attribute-out experiments using a Robust ResNet-50 encoder.} 
%\vspace{-2.5mm}
\label{tbl:dre_lambda_abl}
}
\end{table}

\subsection{Missing attribute ablation study results} 

Table \ref{tbl:dre_lambda_abl} illustrates the missing attribute discovery score for each CelebA split versus full CelebA. With $\lambda=0$ (i.e. ignoring the DRE loss), the attribute discovery process has difficulty capturing some missing attributes. When using a regularization model trained with Log-loss, the results are consistently worse than DRE, sometimes even worse than with $\lambda=0$. The KLIEP loss model, on the other hand, performs consistently better for all lambda values $>0$.




\section{Same dataset, different architecture}
\begin{figure}[!tb]
    \centering
    \includegraphics[width=1.0\linewidth]{figures/pggan_v_gansformer.pdf}
    \caption{An example of applying our method to two generative models trained on the same dataset (FFHQ). We find ProgGAN and GANformer are able to find some alignment, and that the newer model (GANformer) is better at capturing the full data distribution of FFHQ (Missing) whereas ProgGAN is prone to generating non-realistic images (Novel).}
     \label{fig:proggan_v_gansformer}
\end{figure}

\begin{figure}[!tb]
    \centering
    \includegraphics[width=1.0\linewidth]{figures/sg3.pdf}
    \caption{A few examples from our experiment applying xGA between two StyleGAN3 models, both trained on FFHQ, but with different model configurations. As expected both models having translation equivarience, and the rotation equivariance is missing from the translation model. }
     \label{fig:sg3}
\end{figure}

To verify the effectiveness of xGA at comparing models trained on the same dataset with different configurations, we perform two sets of experiments. We use Prog-GAN \cite{karras2017progressive} (client) and GANformer \cite{hudson2021ganformer2} (reference) trained on the FFHQ dataset. Figure \ref{fig:proggan_v_gansformer} shows an example of how these two GANs can be aligned, and how the novel/missing attribute reflects each GAN's capacity to learn the data distribution. 
We also use two configurations of a StyleGAN3 \cite{karras2021alias} trained on FFHQ. Figure \ref{fig:sg3} shows how translation equivarience is preserved in both models, whereas only the StyleGAN3-r model is rotationally equivarient.



\begin{figure}[!tb]
     \centering
     \includegraphics[width=0.99\linewidth]{figures/1gan_example.pdf}
     \caption{Comparing xGA on single GAN attribute discovery with existing approaches, we find that more diverse and novel attributes can be found simply by using an external feature space. We exploit this for effective alignment across two GAN models. Complete examples for all methods are provided below. 
     }
     \label{fig:1gan_examples}
 \end{figure}


\section{Single GAN results}
Here we present the full training of all learned directions for each of our methods using the same starting point from CelebA GAN.
Figure \ref{fig:1gan_examples} visualizes a shortened example of the top $3$ attributes (induce most changes in the ``oracle'' classifier predictions) and it is clear that xGA identifies the most diverse semantic changes. Complete results can be seen as follows:
\begin{enumerate}
    \item Sefa \cite{shen2021closed}: Figure \ref{fig:supp-singlegan-sefa}
    \item LatentCLR \cite{yuksel2021latentclr}: Figure \ref{fig:supp-singlegan-latentclr}
    \item Voynov \cite{voynov2020unsupervised}: Figure \ref{fig:supp-singlegan-voynov}
    \item Hessian \cite{peebles2020hessian}: Figure \ref{fig:supp-singlegan-hessian}
    \item Jacobian \cite{wei2021jacobian}: Figure \ref{fig:supp-singlegan-jacobian}
    \item xGA (ImageNet ResNet-50): figure \ref{fig:supp-singlegan-resnet}
    \item xGA (advBN ResNet-50): figure \ref{fig:supp-singlegan-advbn}
    \item xGA (CLIP ResNet-50): figure  \ref{fig:supp-singlegan-clip}
\end{enumerate}
We visualize both positive and negative directions for every model. Even though xGA and LatentCLR are not directly trained for negative directions, we find these attributes to be semantically meaningful and interesting.


Next we present an example where we compare two LatentCLR models trained on different GANs where the reference GAN is CelebA and client GAN is CelebA without Hats. We sort all the directions by most similar (as described in the main paper) and show an example of the results in Figure \ref{fig:supp-2singleGANs-sorted}, finding many similarities, but no dedicated Hat attribute in the reference GAN. Showing how without the dedicated constraint of the DRE models, finding missing attributes is difficult.

\section{Expanded Qualitative Results}
Here we present many additional examples of shared directions between two GANs,
and novel/missing directions from a few different GAN pairs that contain subset of the CelebA dataset. 
We introduce a new GAN (anime), as it produces interesting common, missing, and novel attributes, though the GAN itself produces lower quality images than other models, and as such we leave it here in the supplement.
The figures are arranged as follows:
\begin{enumerate}
    \item Common attributes: CelebA (reference) and Metface (client) sketch (\ref{fig:supp-metface_sketch}), formal (\ref{fig:supp-metface_formal}), and curly hair  (\ref{fig:supp-metface_curly})
    \item Common attributes: Anime (client) and Toon (reference) purple hair (\ref{fig:supp-anime_purplehair}), orange/brown hair (\ref{fig:supp-anime_orangehair}), open mouths (\ref{fig:supp-anime_mouths}), and smiling (\ref{fig:supp-anime_smiles}); missing attributes of green hair / lipstick (\ref{fig:supp-anime_uniques})
    \item Common attributes: CelebA (reference) and Disney (client) blonde hair (\ref{fig:supp-disney_blonde}), and brown hair (\ref{fig:supp-disney_brown}); novel Disney attributes of turning green / cartoonish eyes (\ref{fig:supp-disney_uniques})
    \item Additional missing attributes from different CelebA client GANs, with CelebA reference GAN (\ref{fig:supp-celeba_uniques})
\end{enumerate}

\section{Expanded Quantitative Results}

First we present the results for using ViT-based feature extractors in table \ref{tbl:supp-vit-experiments}. We include 3 different pretrained models: one original trained on ImageNet, CLIP, and MAE. While ViT does well for entropy metric, it performs poorly for cross model based experiments.

Next, we present the entire results for our missing attribute quantitative experiments. To recap these experiments, we use the $7$ controlled CelebA models which are missing one or more attributes (hat, glasses, male, female, beard, beards|hats, and smiles|glasses|ties) and treat them as the client model; we audit these models with respect to the reference CelebA GAN. The $7$ missing attribute experiments are shown in table \ref{tbl:supp-full-recovery}, where we can see xGA performs well (e.g., easily finding the missing glasses attribute). The $7$ attribute alignment experiments are shown in table \ref{tbl:supp-full-alignment}, where again we see xGA with a robust resnet performs well, especially when the client GAN is missing multiple attributes (e.g., client GAN is missing beards and hats).

For completion's sake, we run pairwise experiments between each GAN, treating each GAN as reference versus the other $7$ GANs, which results in a total of $56$ client/reference paired experiments. We report these comprehensive results in the following tables (where rows are reference GAN and columns are the client): \ref{tbl:supp-unique-voynov}, \ref{tbl:supp-unique-latentclr}, \ref{tbl:supp-unique-hessian}, \ref{tbl:supp-unique-jacobian}, \ref{tbl:supp-unique_vanilla} , \ref{tbl:supp-unique_att} ,  \ref{tbl:supp-unique_robust} ,  \ref{tbl:supp-unique_rnclip} ,  \ref{tbl:supp-unique_vit} ,  \ref{tbl:supp-unique_clipvit} , and  \ref{tbl:supp-unique_mae}. 
We also compute the the common attribute results experiments in the following tables: \ref{tbl:supp-cosine-voynov},\ref{tbl:supp-cosine-latentclr},\ref{tbl:supp-cosine-hessian},\ref{tbl:supp-cosine-jacobian},  \ref{tbl:supp-cosine-vanilla}, \ref{tbl:supp-cosine-att}, \ref{tbl:supp-cosine-robust}, \ref{tbl:supp-cosine-clipRN}, \ref{tbl:supp-cosine-vit}, \ref{tbl:supp-cosine-clipvit}, and \ref{tbl:supp-cosine-mae}.

\begin{table*}[!tb]\centering
\begin{tabular}{llllllll}
                       & Female & Male  & No Hats & No Glasses & No Beards & \begin{tabular}[c]{@{}l@{}}No Beard\\ No Hats\end{tabular} & \begin{tabular}[c]{@{}l@{}}No Glasses\\ No Smiles\\ No Ties\end{tabular} \\ \hline
SeFa                    & 0.143  & 0.143 & 0.045   & 0.111      & 0.063     & 0.278       & 0.189  \\
jacobian                & 0.478  & 0.536 & 0.086   & 0.390      & 0.388     & 0.120       & 0.287  \\
Hessian                 & 1.000  & 1.000 & 0.056   & 0.167      & 0.167     & 0.096       & 0.407  \\
LatentCLR               & 1.000  & 1.000 & 0.250   & 0.333      & 0.200     & 0.153       & 0.537  \\
Voynov                  & 0.500  & 1.000 & 0.333   & 0.050      & 0.500     & 0.153       & 0.259  \\
xGA (ResNet-50)         & 1.000  & 1.000 & 0.333   & 0.500      & 0.200     & 0.167       & 0.465  \\
xGA (Clip ResNet-50)    & 1.000  & 1.000 & 1.000   & 0.200      & 0.200     & 0.108       & 0.383  \\
xGA (advBN ResNet-50)   & 1.000  & 1.000 & 0.250   & 1.000      & 0.063     & 0.183       & 0.401  \\ \hline
\end{tabular}
\caption{The full results for the recovery scores ($\mathcal{R}_{\text{score}}$), where CelebA GAN is the reference.}
 \label{tbl:supp-full-recovery}
\end{table*}




\begin{table*}[!tb] \centering
\begin{tabular}{llllllll}
              & Female & Male  & No Hats & No Glasses & No Beards & \begin{tabular}[c]{@{}l@{}}No Beard\\ No Hats\end{tabular} & \begin{tabular}[c]{@{}l@{}}No Glasses\\ No Smiles\\ No Ties\end{tabular} \\ \hline
SeFa                    & 0.413  & 0.458 & 0.372   & 0.374      & 0.314     & 0.387    & 0.355  \\
Hessian                 & 0.475  & 0.489 & 0.652   & 0.598      & 0.618     & 0.615    & 0.525  \\
LatentCLR               & 0.519  & 0.511 & 0.556   & 0.533      & 0.512     & 0.593    & 0.579  \\
Voynov                  & 0.566  & 0.477 & 0.567   & 0.555      & 0.570     & 0.562    & 0.513  \\
Jacobian                & 0.523  & 0.452 & 0.505   & 0.528      & 0.519     & 0.491    & 0.495  \\
xGA (ResNet-50)         & 0.457  & 0.403 & 0.740   & 0.792      & 0.461     & 0.643    & 0.489  \\
xGA (Clip ResNet-50)    & 0.753  & 0.451 & 0.772   & 0.791      & 0.894     & 0.580    & 0.656  \\
xGA (advBN ResNet-50)   & 0.615  & 0.357 & 0.750   & 0.825      & 0.619     & 0.803    & 0.649  \\ \hline
\end{tabular}
\caption{The full results for the alignment scores ($\mathcal{A}_{\text{score}}$), where CelebA GAN is the reference}
 \label{tbl:supp-full-alignment}
\end{table*}

\begin{table*}[!tb]
\centering
\begin{tabular}{llll}
Method / Model      &   $\mathcal{H}_{\text{score}}$ ($\downarrow$)    &  $\mathcal{A}_{\text{score}}$ ($\uparrow$) & $\mathcal{R}_{\text{score}}$ ($\uparrow$) \\ \hline
xGA + ViT         &   $1.988 \pm{0.068}$    & $0.377 \pm{0.090} $    & $0.249 \pm{0.217}$          \\
xGA + ViT + MAE   &   $2.102 \pm{0.035}$    & $0.349 \pm{0.089}$          & $0.194 \pm{0.197}$         \\ 
xGA + ViT + Clip  &   $2.091 \pm{0.041}$    & $0.397 \pm{0.122}$          & $0.268 \pm{0.195}$          \\ 
\hline
\end{tabular}
\caption{ ViT-based extractors results. The average entropy scores for all 8 CelebA experiments, the average alignment scores ($\mathcal{A}_{\text{score}}$) for the CelebA pairwise experiments, and the average recovery scores ($\mathcal{R}_{\text{score}}$) for the CelebA pairwise leave-attribute-out experiments ($\pm{\text{ std}}$)} 
\label{tbl:supp-vit-experiments}
\end{table*}



\begin{figure*}
\centering
         \includegraphics[width=0.89\linewidth]{figures/supp/1gan_celeba_vs_hats.png}
        \caption{All 16 directions of two single GAN LatentCLR models trained on different GANs where the reference GAN is CelebA and client GAN is CelebA without Hats. The attributes are sorted by most similar ($k=0$) to least similar ($k=15$). While there are some major similarities (short hair $k=0$, eyeglasses $k=5$), the lack of a dedicated constraint for finding hats shows the flaws with this approach. This example is a clear demonstration that without a dedicated cross-model constraint (i.e., using the DRE models), finding missing attributes is difficult.}
        \label{fig:supp-2singleGANs-sorted}
\end{figure*}





\begin{figure*}
\centering
         \includegraphics[width=0.89\linewidth]{figures/supp/singleGAN/resnet.png}
         
        \caption{All 16 learned attributes of a Vanilla ResNet model.}
        \label{fig:supp-singlegan-resnet}
\end{figure*}

\begin{figure*}
\centering
         \includegraphics[width=0.89\linewidth]{figures/supp/singleGAN/advbn.png}
         
        \caption{All 16 learned attributes of a robust ResNet model.}
        \label{fig:supp-singlegan-advbn}
\end{figure*}

\begin{figure*}
\centering
         \includegraphics[width=0.89\linewidth]{figures/supp/singleGAN/clip.png}
         
        \caption{All 16 learned attributes of a clip ResNet model.}
        \label{fig:supp-singlegan-clip}
\end{figure*}

\begin{figure*}
\centering
         \includegraphics[width=0.89\linewidth]{figures/supp/singleGAN/att.png}
         
        \caption{All 16 learned attributes of a Attribute Classifier ResNet model.}
        \label{fig:supp-singlegan-att}
\end{figure*}


\begin{figure*}
\centering
         \includegraphics[width=0.89\linewidth]{figures/supp/singleGAN/globalclr.png}
         
        \caption{All 16 learned attributes of the original LatentCLR model (using global directions, rather than conditional).}
        \label{fig:supp-singlegan-globalclr}
\end{figure*}


\begin{figure*}
\centering
         \includegraphics[width=0.89\linewidth]{figures/supp/singleGAN/hessian.png}
         
        \caption{All 16 learned attributes of the Hessian method.}
        \label{fig:supp-singlegan-hessian}
\end{figure*}


\begin{figure*}
\centering
         \includegraphics[width=0.89\linewidth]{figures/supp/singleGAN/jacobian.png}
         
        \caption{All 16 learned attributes of the Jacobian method.}
        \label{fig:supp-singlegan-jacobian}
\end{figure*}

\begin{figure*}
\centering
         \includegraphics[width=0.89\linewidth]{figures/supp/singleGAN/latentclr.png}
         
        \caption{All 16 learned attributes of the original LatentCLR model with conditional directions.}
        \label{fig:supp-singlegan-latentclr}
\end{figure*}

\begin{figure*}
\centering
         \includegraphics[width=0.89\linewidth]{figures/supp/singleGAN/voynov.png}
         
        \caption{All 16 learned attributes of the Voynov method.}
        \label{fig:supp-singlegan-voynov}
\end{figure*}

\begin{figure*}
\centering
         \includegraphics[width=0.89\linewidth]{figures/supp/singleGAN/sefa.png}
         
        \caption{The top 16 learned attributes of the SeFa method.}
        \label{fig:supp-singlegan-sefa}
\end{figure*}



\begin{figure*}[]
\centering
         \includegraphics[width=0.66\linewidth]{figures/supp/1sketch.png}
         
         \includegraphics[width=0.66\linewidth]{figures/supp/2sketch.png}

        \caption{Examples of common sketch attribute between CelebA (Top) and Metfaces (Bottom).}
        \label{fig:supp-metface_sketch}
\end{figure*}

\begin{figure*}[]
\centering
         \includegraphics[width=0.66\linewidth]{figures/supp/1formal.png}
         %\hskip 2ex
         \includegraphics[width=0.66\linewidth]{figures/supp/2formal.png}

        \caption{Examples of common formal-wear attribute between CelebA (Top) and Metfaces (Bottom). }
         \label{fig:supp-metface_formal}
\end{figure*}



\begin{figure*}[]
\centering
         \includegraphics[width=0.66\linewidth]{figures/supp/1curly.png}
         %\hskip 2ex
         \includegraphics[width=0.66\linewidth]{figures/supp/2curly.png}

        \caption{Examples of common  white, curly hair attribute between CelebA (Top) and Metfaces (Bottom).}
        \label{fig:supp-metface_curly}
\end{figure*}


\begin{figure*}[]
\centering
         \includegraphics[width=0.66\linewidth]{figures/supp/1purplehair.png}
         %\hskip 2ex
         \includegraphics[width=0.66\linewidth]{figures/supp/2purplehair.png}

        \caption{Examples of common purple hair attribute between Anime (Top) and Toon (Bottom).}
        \label{fig:supp-anime_purplehair}
\end{figure*}


\begin{figure*}[]
\centering
         \includegraphics[width=0.66\linewidth]{figures/supp/1otherhair.png}
         
         \includegraphics[width=0.66\linewidth]{figures/supp/2otherhair.png}

        \caption{Examples of common orange/brown hair attribute between Anime (Top) and Toon (Bottom).}
        \label{fig:supp-anime_orangehair}
\end{figure*}



\begin{figure*}[]
\centering
         \includegraphics[width=0.66\linewidth]{figures/supp/1mouth.png}
         
         \includegraphics[width=0.66\linewidth]{figures/supp/2mouth.png}

        \caption{Examples of common open-mouth attribute between Anime (Top) and Toon (Bottom).}
        \label{fig:supp-anime_mouths}
\end{figure*}


\begin{figure*}[]
\centering
         \includegraphics[width=0.66\linewidth]{figures/supp/1smile.png}
         
         \includegraphics[width=0.66\linewidth]{figures/supp/2smile.png}

        \caption{Examples of common smiling attribute between Anime (Top) and Toon (Bottom).}
        \label{fig:supp-anime_smiles}
\end{figure*}



\begin{figure*}[]
\centering
         \includegraphics[width=0.66\linewidth]{figures/supp/1unique_green.png}

         \includegraphics[width=0.66\linewidth]{figures/supp/2unique_lipstick.png}

        \caption{Examples of novel green hair attribute from Anime (Top) and the missing lipstick attribute from Toon (Bottom).}
        \label{fig:supp-anime_uniques}
\end{figure*}

\begin{figure*}[]
\centering
         \includegraphics[width=0.66\linewidth]{figures/supp/1blonde.png}

         \includegraphics[width=0.66\linewidth]{figures/supp/2blonde.png}

        \caption{Examples of common blonde attribute between CelebA (Top) and Disney (Bottom). }
         \label{fig:supp-disney_blonde}
\end{figure*}


\begin{figure*}[]
\centering
         \includegraphics[width=0.66\linewidth]{figures/supp/1brown.png}

         \includegraphics[width=0.66\linewidth]{figures/supp/2brown.png}

        \caption{Examples of common brown hair attribute between CelebA (Top) and Disney (Bottom). }
         \label{fig:supp-disney_brown}
\end{figure*}



\begin{figure*}[]
\centering
         \includegraphics[width=0.66\linewidth]{figures/supp/disney_unique.png}

         \includegraphics[width=0.66\linewidth]{figures/supp/disney_unique2.png}

        \caption{Two examples of novel Disney attributes: making princesses ogre-like and large cartoonish eyes.}
         \label{fig:supp-disney_uniques}
\end{figure*}

\begin{figure*}[]
\centering
         \includegraphics[width=0.33\linewidth]{figures/supp/glass.png}
         \vspace{1mm}
         \includegraphics[width=0.33\linewidth]{figures/supp/hats.png}
         \vspace{1mm}
         \includegraphics[width=0.33\linewidth]{figures/supp/smile_glass_tie.png}
        \caption{Examples of various missing attributes from full CelebA GAN against three different attribute splits: \textbf{(Left 3)} the missing eyeglass attribute, \textbf{(Middle 3)} the missing hats attribute and \textbf{(Right 3)} the missing eyeglass, smiling, and attempting to identify the necktie attribute.}
        \label{fig:supp-celeba_uniques}
\end{figure*}

\newpage
\newpage
\begin{table*}[]\centering
\vspace{25mm}
\begin{tabular}{lllllllll}
                       & Full CelebA & Female & Male  & No Hats & No Glasses & No Beards & \begin{tabular}[c]{@{}l@{}}No Beard\\ No Hats\end{tabular} & \begin{tabular}[c]{@{}l@{}}No Glasses\\ No Smiles\\ No Ties\end{tabular} \\ \hline
Full CelebA              &       & 0.143  & 0.143 & 0.045 & 0.111   & 0.063   & 0.278        & 0.189               \\
Female              & 0.000 &        & 0.167 & 1.000 & 0.167   & 0.200   & 0.170        & 0.417               \\
Male                & 0.000 & 0.059  &       & 0.250 & 0.111   & 0.250   & 0.188        & 0.303               \\
No Hats               & 0.000 & 0.056  & 0.111 &       & 0.083   & 0.067   & 0.188        & 0.267               \\
No Glasses            & 0.000 & 0.059  & 0.125 & 0.333 &         & 0.042   & 0.306        & 0.203               \\
No Beards             & 0.000 & 0.333  & 0.083 & 0.043 & 0.063   &         & 0.185        & 0.107               \\
\begin{tabular}[c]{@{}l@{}}No Beard\\ No Hats\end{tabular}         & 0.000 & 0.143  & 0.100 & 0.125 & 0.143   & 0.125   &              & 0.511               \\
\begin{tabular}[c]{@{}l@{}}No Glasses\\ No Smiles\\ No Ties\end{tabular} & 0.000 & 0.053  & 0.333 & 0.250 & 0.500   & 0.059   & 0.096        &     \\ \hline               
\end{tabular}
\caption{The full results for the recovery scores ($\mathcal{R}_{\text{score}}$) of the SeFa method.}
 \label{tbl:supp-unique-sefa}
\end{table*}



\begin{table*}[]\centering
\begin{tabular}{lllllllll}
                       & Full CelebA & Female & Male  & No Hats & No Glasses & No Beards & \begin{tabular}[c]{@{}l@{}}No Beard\\ No Hats\end{tabular} & \begin{tabular}[c]{@{}l@{}}No Glasses\\ No Smiles\\ No Ties\end{tabular} \\ \hline
Full CelebA            & -           & 0.478  & 0.536 & 0.086   & 0.390      & 0.388     & 0.120                                                      & 0.287                                                                    \\
Female                 & 0           & -      & 0.246 & 0.048   & 0.050      & 0.046     & 0.048                                                      & 0.279                                                                    \\
Male                   & 0           & 0.586  & -     & 0.124   & 0.333      & 0.733     & 0.384                                                      & 0.405                                                                    \\
No Hats                & 0           & 0.251  & 0.240 & -       & 0.097      & 0.180     & 0.108                                                      & 0.313                                                                    \\
No Glasses             & 0           & 0.585  & 0.542 & 0.048   & -          & 0.114     & 0.100                                                      & 0.218                                                                    \\
No Beards              & 0           & 0.290  & 0.583 & 0.069   & 0.080      & -         & 0.066                                                      & 0.271                                                                    \\
\begin{tabular}[c]{@{}l@{}}No Beard\\ No Hats\end{tabular}          & 0           & 0.373  & 0.396 & 0.034   & 0.189      & 0.049     & -                                                          & 0.297                                                                    \\
\begin{tabular}[c]{@{}l@{}}No Glasses\\ No Smiles\\ No Ties\end{tabular} & 0           & 0.448  & 0.667 & 0.055   & 0.033      & 0.537     & 0.269                                                      & -  \\  \hline                                                                    
\end{tabular}
\caption{The full results for the recovery scores ($\mathcal{R}_{\text{score}}$) of the Jacobian loss.}
 \label{tbl:supp-unique-jacobian}
\end{table*}

\begin{table*}[]\centering
\begin{tabular}{lllllllll}
                       & Full CelebA & Female & Male  & No Hats & No Glasses & No Beards & \begin{tabular}[c]{@{}l@{}}No Beard\\ No Hats\end{tabular} & \begin{tabular}[c]{@{}l@{}}No Glasses\\ No Smiles\\ No Ties\end{tabular} \\ \hline
Full CelebA            & -           & 1.000  & 1.000 & 0.056   & 0.167      & 0.167     & 0.096                                                      & 0.407                                                                    \\
Female                 & 0           & -      & 0.200 & 0.056   & 0.045      & 0.036     & 0.057                                                      & 0.364                                                                    \\
Male                   & 0           & 0.333  & -     & 0.077   & 1.000      & 1.000     & 0.300                                                      & 0.300                                                                    \\
No Hats                & 0           & 0.500  & 0.250 & -       & 0.083      & 0.071     & 0.042                                                      & 0.370                                                                    \\
No Glasses             & 0           & 0.333  & 0.250 & 0.083   & -          & 0.111     & 0.071                                                      & 0.150                                                                    \\
No Beards              & 0           & 0.125  & 0.167 & 0.071   & 0.063      & -         & 0.046                                                      & 0.218                                                                    \\
\begin{tabular}[c]{@{}l@{}}No Beard\\ No Hats\end{tabular}          & 0           & 0.167  & 0.333 & 0.032   & 0.071      & 0.053     & -                                                          & 0.375                                                                    \\
\begin{tabular}[c]{@{}l@{}}No Glasses\\ No Smiles\\ No Ties\end{tabular} & 0           & 1.000  & 0.333 & 0.200   & 0.048      & 0.143     & 0.102                                                      & -    \\  \hline                                                                   
\end{tabular}
\caption{The full results for the recovery scores ($\mathcal{R}_{\text{score}}$) of the Hessian loss.}
 \label{tbl:supp-unique-hessian}
\end{table*}

\begin{table*}[]\centering
\begin{tabular}{lllllllll}
                       & Full CelebA & Female & Male  & No Hats & No Glasses & No Beards & \begin{tabular}[c]{@{}l@{}}No Beard\\ No Hats\end{tabular} & \begin{tabular}[c]{@{}l@{}}No Glasses\\ No Smiles\\ No Ties\end{tabular} \\ \hline
Full CelebA            & -           & 1.000  & 1.000 & 0.250   & 0.333      & 0.200     & 0.153                                                      & 0.537                                                                    \\
Female                 & 0           & -      & 0.250 & 0.048   & 0.034      & 0.050     & 0.044                                                      & 0.137                                                                    \\
Male                   & 0           & 1.000  & -     & 0.143   & 0.500      & 0.333     & 0.267                                                      & 0.242                                                                    \\
No Hats                & 0           & 0.250  & 1.000 & -       & 0.125      & 0.167     & 0.077                                                      & 0.158                                                                    \\
No Glasses             & 0           & 0.333  & 1.000 & 0.038   & -          & 0.250     & 0.525                                                      & 0.153                                                                    \\
No Beards              & 0           & 0.125  & 0.500 & 0.045   & 0.063      & -         & 0.049                                                      & 0.381                                                                    \\
\begin{tabular}[c]{@{}l@{}}No Beard\\ No Hats\end{tabular}          & 0           & 1.000  & 1.000 & 0.043   & 0.167      & 0.091     & -                                                          & 0.389                                                                    \\
\begin{tabular}[c]{@{}l@{}}No Glasses\\ No Smiles\\ No Ties\end{tabular} & 0           & 1.000  & 0.500 & 0.067   & 0.033      & 0.125     & 0.072                                                      & - \\  \hline                                                                      
\end{tabular}
\caption{The full results for the recovery scores ($\mathcal{R}_{\text{score}}$) of the LatentCLR loss.}
 \label{tbl:supp-unique-latentclr}
\end{table*}

\begin{table*}[]\centering
\begin{tabular}{lllllllll}
                       & Full CelebA & Female & Male  & No Hats & No Glasses & No Beards & \begin{tabular}[c]{@{}l@{}}No Beard\\ No Hats\end{tabular} & \begin{tabular}[c]{@{}l@{}}No Glasses\\ No Smiles\\ No Ties\end{tabular} \\ \hline
Full CelebA            & -           & 0.500  & 1.000 & 0.333   & 0.050      & 0.500     & 0.153                                                      & 0.259                                                                    \\
Female                 & 0           & -      & 0.143 & 0.045   & 0.167      & 0.038     & 0.061                                                      & 0.410                                                                    \\
Male                   & 0           & 0.500  & -     & 0.250   & 0.333      & 0.333     & 0.306                                                      & 0.256                                                                    \\
No Hats                & 0           & 0.500  & 0.143 & -       & 0.091      & 0.500     & 0.139                                                      & 0.511                                                                    \\
No Glasses             & 0           & 0.333  & 1.000 & 0.050   & -          & 0.167     & 0.094                                                      & 0.377                                                                    \\
No Beards              & 0           & 0.167  & 0.250 & 0.167   & 0.091      & -         & 0.052                                                      & 0.370                                                                    \\
\begin{tabular}[c]{@{}l@{}}No Beard\\ No Hats\end{tabular}          & 0           & 0.500  & 0.500 & 0.034   & 0.050      & 0.034     & -                                                          & 0.386                                                                    \\
\begin{tabular}[c]{@{}l@{}}No Glasses\\ No Smiles\\ No Ties\end{tabular} & 0           & 0.143  & 0.500 & 0.143   & 0.029      & 1.000     & 0.286                                                      & - \\  \hline                                                                      
\end{tabular}
\caption{The full results for the recovery scores ($\mathcal{R}_{\text{score}}$) of the voynov loss.}
 \label{tbl:supp-unique-voynov}
\end{table*}


\begin{table*}[] \centering
\begin{tabular}{lllllllll}
                       & Full CelebA & Female & Male  & No Hats & No Glasses & No Beards & \begin{tabular}[c]{@{}l@{}}No Beard\\ No Hats\end{tabular} & \begin{tabular}[c]{@{}l@{}}No Glasses\\ No Smiles\\ No Ties\end{tabular} \\ \hline
Full CelebA            & -           & 1.000  & 1.000 & 0.333   & 0.500      & 0.200     & 0.167         & 0.465                  \\
Female                 & 0           & -      & 0.333 & 0.053   & 0.067      & 0.034     & 0.046         & 0.381                  \\
Male                   & 0           & 1      & -     & 0.059   & 0.250      & 0.083     & 0.082         & 0.521                  \\
No Hats                & 0           & 1      & 0.5   & -       & 0.143      & 0.071     & 0.062         & 0.460                  \\
No Glasses             & 0           & 1      & 0.5   & 0.059   & -          & 0.091     & 0.081         & 0.377                  \\
No Beards              & 0           & 1      & 1     & 1       & 0.091      & -         & 0.154         & 0.378                  \\
\begin{tabular}[c]{@{}l@{}}No Beard\\ No Hats\end{tabular}          & 0           & 1      & 1     & 0.033   & 0.05       & 0.042     & -             & 0.382                  \\
\begin{tabular}[c]{@{}l@{}}No Glasses\\ No Smiles\\ No Ties\end{tabular} & 0           & 1      & 1     & 0.083   & 0.029      & 0.333     & 0.122         & -                      \\ \hline
\end{tabular}
\caption{The full results for the recovery scores ($\mathcal{R}_{\text{score}}$) of the ImageNet ResNet.}
 \label{tbl:supp-unique_vanilla}
\end{table*}




\begin{table*}[] \centering
\begin{tabular}{lllllllll}
                       & Full CelebA & Female & Male  & No Hats & No Glasses & No Beards & \begin{tabular}[c]{@{}l@{}}No Beard\\ No Hats\end{tabular} & \begin{tabular}[c]{@{}l@{}}No Glasses\\ No Smiles\\ No Ties\end{tabular} \\ \hline
Full CelebA            & -           & 1.000  & 0.333 & 1.000   & 0.091      & 1.000     & 0.525         & 0.390                  \\
Female                 & 0           & -      & 0.200 & 0.042   & 0.059      & 0.038     & 0.036         & 0.198                  \\
Male                   & 0           & 1      & -     & 0.063   & 0.143      & 0.056     & 0.306         & 0.492                  \\
No Hats                & 0           & 1      & 0.333 & -       & 0.333      & 1.000     & 0.516         & 0.365                  \\
No Glasses             & 0           & 1      & 0.333 & 0.1     & -          & 0.500     & 0.270         & 0.357                  \\
No Beards              & 0           & 1      & 1     & 0.053   & 0.111      & -         & 0.042         & 0.211                  \\
\begin{tabular}[c]{@{}l@{}}No Beard\\ No Hats\end{tabular}          & 0           & 1      & 1     & 0.033   & 0.071      & 0.048     & -             & 0.212                  \\
\begin{tabular}[c]{@{}l@{}}No Glasses\\ No Smiles\\ No Ties\end{tabular} & 0           & 1      & 1     & 1       & 0.03       & 0.5       & 0.274         & -                      \\ \hline
\end{tabular}
\caption{The full results for the recovery scores ($\mathcal{R}_{\text{score}}$) of the Attribute Classifier ResNet.}
 \label{tbl:supp-unique_att}
\end{table*}



\begin{table*}[] \centering
\begin{tabular}{lllllllll}
                       & Full CelebA & Female & Male  & No Hats & No Glasses & No Beards & \begin{tabular}[c]{@{}l@{}}No Beard\\ No Hats\end{tabular} & \begin{tabular}[c]{@{}l@{}}No Glasses\\ No Smiles\\ No Ties\end{tabular} \\ \hline
Full CelebA            & -           & 1.000  & 1.000 & 0.250   & 1.000      & 0.063     & 0.183         & 0.401                  \\
Female                 & 0           & -      & 0.333 & 0.500   & 0.042      & 0.048     & 0.047         & 0.363                  \\
Male                   & 0           & 0.5    & -     & 0.077   & 0.500      & 0.111     & 0.563         & 0.419                  \\
No Hats                & 0           & 1      & 1     & -       & 1.000      & 0.143     & 0.517         & 0.423                  \\
No Glasses             & 0           & 1      & 0.333 & 0.034   & -          & 0.333     & 0.200         & 0.361                  \\
No Beards              & 0           & 1      & 1     & 0.143   & 0.045      & -         & 0.113         & 0.370                  \\
\begin{tabular}[c]{@{}l@{}}No Beard\\ No Hats\end{tabular}          & 0           & 1      & 1     & 0.028   & 0.5        & 0.038     & -             & 0.376                  \\
\begin{tabular}[c]{@{}l@{}}No Glasses\\ No Smiles\\ No Ties\end{tabular} & 0           & 1      & 1     & 0.056   & 0.033      & 1         & 0.556         & -                      \\ \hline
\end{tabular}
\caption{The full results for the recovery scores ($\mathcal{R}_{\text{score}}$) of the Robust ResNet.}
 \label{tbl:supp-unique_robust}
\end{table*}




\begin{table*}[] \centering
\begin{tabular}{lllllllll}
                       & Full CelebA & Female & Male  & No Hats & No Glasses & No Beards & \begin{tabular}[c]{@{}l@{}}No Beard\\ No Hats\end{tabular} & \begin{tabular}[c]{@{}l@{}}No Glasses\\ No Smiles\\ No Ties\end{tabular} \\ \hline
Full CelebA            & -           & 1.000  & 1.000 & 1.000   & 0.200      & 0.200     & 0.108         & 0.383                  \\
Female                 & 0           & -      & 0.250 & 0.028   & 0.200      & 0.063     & 0.131         & 0.194                  \\
Male                   & 0           & 0.143  & -     & 0.028   & 0.200      & 0.063     & 0.131         & 0.055                  \\
No Hats                & 0           & 1      & 1     & -       & 0.500      & 0.125     & 0.133         & 0.231                  \\
No Glasses             & 0           & 1      & 1     & 0.037   & -          & 0.100     & 0.148         & 0.397                  \\
No Beards              & 0           & 1      & 1     & 0.083   & 0.333      & -         & 0.098         & 0.373                  \\
\begin{tabular}[c]{@{}l@{}}No Beard\\ No Hats\end{tabular}          & 0           & 1      & 1     & 0.1     & 0.2        & 0.029     & -             & 0.371                  \\
\begin{tabular}[c]{@{}l@{}}No Glasses\\ No Smiles\\ No Ties\end{tabular} & 0           & 1      & 1     & 0.063   & 0.125      & 0.167     & 0.205         & -                      \\ \hline
\end{tabular}
\caption{The full results for the recovery scores ($\mathcal{R}_{\text{score}}$) of the CLIP ResNet.}
 \label{tbl:supp-unique_rnclip}
\end{table*}




\begin{table*}[] \centering
\begin{tabular}{lllllllll}
                    & Full CelebA & Female & Male  & No Hats & No Glasses & No Beards & \begin{tabular}[c]{@{}l@{}}No Beard\\ No Hats\end{tabular} & \begin{tabular}[c]{@{}l@{}}No Glasses\\ No Smiles\\ No Ties\end{tabular} \\ \hline
Full CelebA            & -           & 1.000  & 1.000 & 0.200   & 0.091      & 0.050     & 0.276         & 0.363                  \\
Female                 & 0           & -      & 0.333 & 0.040   & 0.045      & 0.056     & 0.054         & 0.140                  \\
Male                   & 0           & 1      & -     & 0.500   & 0.111      & 0.063     & 0.140         & 0.199                  \\
No Hats                & 0           & 0.5    & 1     & -       & 0.200      & 0.040     & 0.076         & 0.369                  \\
No Glasses             & 0           & 0.333  & 1     & 0.111   & -          & 0.063     & 0.042         & 0.388                  \\
No Beards              & 0           & 0.143  & 0.143 & 0.125   & 0.1        & -         & 0.086         & 0.209                  \\
\begin{tabular}[c]{@{}l@{}}No Beard\\ No Hats\end{tabular}          & 0           & 0.143  & 0.5   & 0.031   & 0.053      & 0.032     & -             & 0.192                  \\
\begin{tabular}[c]{@{}l@{}}No Glasses\\ No Smiles\\ No Ties\end{tabular} & 0           & 1      & 1     & 0.033   & 0.125      & 0.125     & 0.108         & -                      \\ \hline
\end{tabular}
\caption{The full results for the recovery scores ($\mathcal{R}_{\text{score}}$) of the ImageNet ViT.}
 \label{tbl:supp-unique_vit}
\end{table*}






\begin{table*}[] \centering
\begin{tabular}{lllllllll}
                       & Full CelebA & Female & Male  & No Hats & No Glasses & No Beards & \begin{tabular}[c]{@{}l@{}}No Beard\\ No Hats\end{tabular} & \begin{tabular}[c]{@{}l@{}}No Glasses\\ No Smiles\\ No Ties\end{tabular} \\ \hline
Full CelebA            & -           & 0.500  & 1.000 & 0.027   & 1.000      & 0.067     & 0.264         & 0.387                  \\
Female                 & 0           & -      & 0.200 & 0.034   & 0.032      & 0.036     & 0.048         & 0.188                  \\
Male                   & 0           & 0.333  & -     & 0.091   & 0.125      & 0.045     & 0.229         & 0.232                  \\
No Hats                & 0           & 1      & 1     & -       & 0.063      & 0.067     & 0.170         & 0.365                  \\
No Glasses             & 0           & 1      & 0.5   & 0.067   & -          & 0.053     & 0.066         & 0.189                  \\
No Beards              & 0           & 0.5    & 1     & 0.091   & 0.333      & -         & 0.073         & 0.127                  \\
\begin{tabular}[c]{@{}l@{}}No Beard\\ No Hats\end{tabular}          & 0           & 0.2    & 0.5   & 0.029   & 0.056      & 0.038     & -             & 0.369                  \\
\begin{tabular}[c]{@{}l@{}}No Glasses\\ No Smiles\\ No Ties\end{tabular} & 0           & 1      & 1     & 0.032   & 0.125      & 0.037     & 0.107         & -                      \\ \hline
\end{tabular}
\caption{The full results for the recovery scores ($\mathcal{R}_{\text{score}}$) of the CLIP ViT.}
 \label{tbl:supp-unique_clipvit}
\end{table*}



\begin{table*}[] \centering
\begin{tabular}{lllllllll}
                       & Full CelebA & Female & Male  & No Hats & No Glasses & No Beards & \begin{tabular}[c]{@{}l@{}}No Beard\\ No Hats\end{tabular} & \begin{tabular}[c]{@{}l@{}}No Glasses\\ No Smiles\\ No Ties\end{tabular} \\ \hline
Full CelebA            & -           & 0.200  & 0.250 & 0.100   & 0.077      & 0.077     & 0.156         & 0.068                  \\
Female                 & 0           & -      & 0.200 & 0.038   & 0.056      & 0.033     & 0.044         & 0.119                  \\
Male                   & 0           & 1      & -     & 0.053   & 0.143      & 0.042     & 0.104         & 0.511                  \\
No Hats                & 0           & 0.091  & 1     & -       & 0.063      & 0.028     & 0.046         & 0.194                  \\
No Glasses             & 0           & 0.25   & 1     & 0.067   & -          & 0.034     & 0.047         & 0.081                  \\
No Beards              & 0           & 0.5    & 1     & 0.045   & 0.083      & -         & 0.072         & 0.356                  \\
\begin{tabular}[c]{@{}l@{}}No Beard\\ No Hats\end{tabular}          & 0           & 0.25   & 0.333 & 0.067   & 0.059      & 0.034     & -             & 0.068                  \\
\begin{tabular}[c]{@{}l@{}}No Glasses\\ No Smiles\\ No Ties\end{tabular} & 0           & 1      & 0.5   & 0.027   & 0.091      & 0.043     & 0.163         & -                      \\ \hline
\end{tabular}
\caption{The full results for the recovery scores ($\mathcal{R}_{\text{score}}$) of the MAE ViT.}
 \label{tbl:supp-unique_mae}
\end{table*}

%%%%%%%%%%%%%%%%%%%% COSINE SCORES %%%%%%%%%%%%%%%%%%%%%%%%%%

\begin{table*}[] \centering
\begin{tabular}{llllllll}
              & Female & Male  & No Hats & No Glasses & No Beards & \begin{tabular}[c]{@{}l@{}}No Beard\\ No Hats\end{tabular} & \begin{tabular}[c]{@{}l@{}}No Glasses\\ No Smiles\\ No Ties\end{tabular} \\ \hline
Full CelebA        & 0.413 & 0.458 & 0.3715 & 0.374 & 0.314 & 0.387     & 0.355            \\
Female       & -        & 0.329 & 0.3615 & 0.320 & 0.352 & 0.360     & 0.334            \\
Male         & -        & -        & 0.3732 & 0.426 & 0.364 & 0.352     & 0.381            \\
No Hats      & -        & -        & -        & 0.400 & 0.300 & 0.325     & 0.379            \\
No Glasses   & -        & -        & -        & -        & 0.336 & 0.305     & 0.343            \\
No Beards    & -        & -        & -        & -        & -        & 0.302     & 0.295            \\
\begin{tabular}[c]{@{}l@{}}No Beard\\ No Hats\end{tabular} & -        & -        & -        & -        & -        & -            & 0.341            
\end{tabular}
\caption{The full results for the alignment scores ($\mathcal{A}_{\text{score}}$) of the SeFa method.}
 \label{tbl:supp-cosine-sefa}
\end{table*}

\begin{table*}[]\centering
\begin{tabular}{llllllll}
              & Female & Male  & No Hats & No Glasses & No Beards & \begin{tabular}[c]{@{}l@{}}No Beard\\ No Hats\end{tabular} & \begin{tabular}[c]{@{}l@{}}No Glasses\\ No Smiles\\ No Ties\end{tabular} \\ \hline
Full CelebA   & 0.475  & 0.489 & 0.652   & 0.598      & 0.618     & 0.615                                                      & 0.525                                                                    \\
Female        & -      & 0.367 & 0.523   & 0.457      & 0.559     & 0.559                                                      & 0.345                                                                    \\
Male          & -      & -     & 0.466   & 0.401      & 0.372     & 0.431                                                      & 0.450                                                                    \\
No Hats       & -      & -     & -       & 0.512      & 0.620     & 0.505                                                      & 0.518                                                                    \\
No Glasses    & -      & -     & -       & -          & 0.556     & 0.502                                                      & 0.503                                                                    \\
No Beards     & -      & -     & -       & -          & -         & 0.563                                                      & 0.477                                                                    \\
\begin{tabular}[c]{@{}l@{}}No Beard\\ No Hats\end{tabular} & -      & -     & -       & -          & -         & -                                                          & 0.439                                                                    \\  \hline
\end{tabular}

\caption{The full results for the alignment scores ($\mathcal{A}_{\text{score}}$) of the Hessian loss.}
 \label{tbl:supp-cosine-hessian}
\end{table*}

\begin{table*}[]\centering
\begin{tabular}{llllllll}
              & Female & Male  & No Hats & No Glasses & No Beards & \begin{tabular}[c]{@{}l@{}}No Beard\\ No Hats\end{tabular} & \begin{tabular}[c]{@{}l@{}}No Glasses\\ No Smiles\\ No Ties\end{tabular} \\ \hline
Full CelebA   & 0.519  & 0.511 & 0.556   & 0.533      & 0.512     & 0.593                                                      & 0.579                                                                    \\
Female        & -      & 0.412 & 0.452   & 0.513      & 0.550     & 0.613                                                      & 0.388                                                                    \\
Male          & -      & -     & 0.373   & 0.474      & 0.425     & 0.448                                                      & 0.426                                                                    \\
No Hats       & -      & -     & -       & 0.460      & 0.485     & 0.576                                                      & 0.482                                                                    \\
No Glasses    & -      & -     & -       & -          & 0.484     & 0.630                                                      & 0.491                                                                    \\
No Beards     & -      & -     & -       & -          & -         & 0.550                                                      & 0.501                                                                    \\
\begin{tabular}[c]{@{}l@{}}No Beard\\ No Hats\end{tabular} & -      & -     & -       & -          & -         & -                                                          & 0.496                                                                    \\ \hline
\end{tabular}

\caption{The full results for the alignment scores ($\mathcal{A}_{\text{score}}$) of the LatentCLR loss.}
 \label{tbl:supp-cosine-latentclr}
\end{table*}


\begin{table*}[]\centering
\begin{tabular}{llllllll}
              & Female & Male  & No Hats & No Glasses & No Beards & \begin{tabular}[c]{@{}l@{}}No Beard\\ No Hats\end{tabular} & \begin{tabular}[c]{@{}l@{}}No Glasses\\ No Smiles\\ No Ties\end{tabular} \\ \hline
Full CelebA   & 0.566  & 0.477 & 0.567   & 0.555      & 0.570     & 0.562                                                      & 0.513                                                                    \\
Female        & -      & 0.430 & 0.606   & 0.609      & 0.577     & 0.592                                                      & 0.505                                                                    \\
Male          & -      & -     & 0.508   & 0.503      & 0.429     & 0.456                                                      & 0.453                                                                    \\
No Hats       & -      & -     &         & 0.596      & 0.553     & 0.583                                                      & 0.553                                                                    \\
No Glasses    & -      & -     & -       & -          & 0.572     & 0.559                                                      & 0.564                                                                    \\
No Beards     & -      & -     & -       & -          & -         & 0.616                                                      & 0.567                                                                    \\
\begin{tabular}[c]{@{}l@{}}No Beard\\ No Hats\end{tabular} & -      & -     & -       & -          & -         & -                                                          & 0.499                                                                    \\ \hline
\end{tabular}
 \caption{The full results for the alignment scores ($\mathcal{A}_{\text{score}}$) of the Voynov method.}
 \label{tbl:supp-cosine-voynov}
\end{table*}



\begin{table*}[]\centering
\begin{tabular}{llllllll}
              & Female & Male  & No Hats & No Glasses & No Beards & \begin{tabular}[c]{@{}l@{}}No Beard\\ No Hats\end{tabular} & \begin{tabular}[c]{@{}l@{}}No Glasses\\ No Smiles\\ No Ties\end{tabular} \\ \hline
Full CelebA   & 0.523  & 0.452 & 0.505   & 0.528      & 0.519     & 0.491                                                      & 0.495                                                                    \\
Female        & -      & 0.413 & 0.481   & 0.522      & 0.490     & 0.511                                                      & 0.449                                                                    \\
Male          & -      & -     & 0.488   & 0.451      & 0.443     & 0.409                                                      & 0.384                                                                    \\
No Hats       & -      & -     & -       & 0.498      & 0.515     & 0.523                                                      & 0.465                                                                    \\
No Glasses    & -      & -     & -       & -          & 0.497     & 0.517                                                      & 0.503                                                                    \\
No Beards     & -      & -     & -       & -          & -         & 0.544                                                      & 0.484                                                                    \\
\begin{tabular}[c]{@{}l@{}}No Beard\\ No Hats\end{tabular} & -      & -     & -       & -          & -         & -                                                          & 0.512                                                                    \\ \hline
\end{tabular}
 \caption{The full results for the alignment scores ($\mathcal{A}_{\text{score}}$) of the Jacobian loss.}
 \label{tbl:supp-cosine-jacobian}
\end{table*}




\begin{table*}[] \centering
\begin{tabular}{llllllll}
              & Female & Male  & No Hats & No Glasses & No Beards & \begin{tabular}[c]{@{}l@{}}No Beard\\ No Hats\end{tabular} & \begin{tabular}[c]{@{}l@{}}No Glasses\\ No Smiles\\ No Ties\end{tabular} \\ \hline
Full CelebA   & 0.457  & 0.403 & 0.740   & 0.792      & 0.461     & 0.643         & 0.489                  \\
Female        & -      & 0.409 & 0.651   & 0.720      & 0.599     & 0.483         & 0.444                  \\
Male          & -      & -     & 0.508   & 0.377      & 0.328     & 0.360         & 0.390                  \\
No Hats       & -      & -     & -       & 0.611      & 0.778     & 0.414         & 0.560                  \\
No Glasses    & -      & -     & -       & -          & 0.698     & 0.659         & 0.649                  \\
No Beards     & -      & -     & -       & -          & -         & 0.652         & 0.556                  \\
\begin{tabular}[c]{@{}l@{}}No Beard\\ No Hats\end{tabular} & -      & -     & -       & -          & -         & -             & 0.492                  \\ \hline
\end{tabular}
\caption{The full results for the alignment scores ($\mathcal{A}_{\text{score}}$) of the ImageNet Trained ResNet.}
 \label{tbl:supp-cosine-vanilla}
\end{table*}


\begin{table*}[] \centering
\begin{tabular}{llllllll}
              & Female & Male  & No Hats & No Glasses & No Beards & \begin{tabular}[c]{@{}l@{}}No Beard\\ No Hats\end{tabular} & \begin{tabular}[c]{@{}l@{}}No Glasses\\ No Smiles\\ No Ties\end{tabular} \\ \hline
Full CelebA   & 0.069  & 0.272 & 0.419   & 0.494      & 0.556     & 0.543         & 0.014                  \\
Female        & -      & 0.155 & 0.411   & 0.230      & 0.191     & 0.354         & 0.056                  \\
Male          & -      & -     & 0.248   & 0.392      & 0.086     & 0.338         & 0.231                  \\
No Hats       & -      & -     & -       & 0.370      & 0.279     & 0.346         & 0.569                  \\
No Glasses    & -      & -     & -       & -          & 0.494     & 0.405         & 0.447                  \\
No Beards     & -      & -     & -       & -          & -         & 0.331         & 0.510                  \\
\begin{tabular}[c]{@{}l@{}}No Beard\\ No Hats\end{tabular} & -      & -     & -       & -          & -         & -             & 0.403                  \\ \hline
\end{tabular}
\caption{The full results for the alignment scores ($\mathcal{A}_{\text{score}}$) of the Attribute Classifier ResNet.}
 \label{tbl:supp-cosine-att}
\end{table*}



\begin{table*}[] \centering
\begin{tabular}{llllllll}
              & Female & Male  & No Hats & No Glasses & No Beards & \begin{tabular}[c]{@{}l@{}}No Beard\\ No Hats\end{tabular} & \begin{tabular}[c]{@{}l@{}}No Glasses\\ No Smiles\\ No Ties\end{tabular} \\ \hline
Full CelebA   & 0.615  & 0.357 & 0.750   & 0.825      & 0.619     & 0.803         & 0.649                  \\
Female        & -      & 0.439 & 0.643   & 0.644      & 0.404     & 0.565         & 0.525                  \\
Male          & -      & -     & 0.384   & 0.444      & 0.417     & 0.175         & 0.289                  \\
No Hats       & -      & -     & -       & 0.508      & 0.310     & 0.744         & 0.641                  \\
No Glasses    & -      & -     & -       & -          & 0.717     & 0.682         & 0.557                  \\
No Beards     & -      & -     & -       & -          & -         & 0.568         & 0.495                  \\
\begin{tabular}[c]{@{}l@{}}No Beard\\ No Hats\end{tabular} & -      & -     & -       & -          & -         & -             & 0.474                  \\ \hline
\end{tabular}
\caption{The full results for the alignment scores ($\mathcal{A}_{\text{score}}$) of the Robust ResNet.}
 \label{tbl:supp-cosine-robust}
\end{table*}





\begin{table*}[] \centering
\begin{tabular}{llllllll}
              & Female & Male  & No Hats & No Glasses & No Beards & \begin{tabular}[c]{@{}l@{}}No Beard\\ No Hats\end{tabular} & \begin{tabular}[c]{@{}l@{}}No Glasses\\ No Smiles\\ No Ties\end{tabular} \\ \hline
Full CelebA   & 0.753  & 0.451 & 0.772   & 0.791      & 0.894     & 0.580         & 0.656                  \\
Female        & -      & 0.283 & 0.733   & 0.684      & 0.639     & 0.474         & 0.337                  \\
Male          & -      & -     & 0.371   & 0.435      & 0.396     & 0.337         & 0.314                  \\
No Hats       & -      & -     & -       & 0.815      & 0.679     & 0.715         & 0.505                  \\
No Glasses    & -      & -     & -       & -          & 0.748     & 0.608         & 0.623                  \\
No Beards     & -      & -     & -       & -          & -         & 0.706         & 0.507                  \\
\begin{tabular}[c]{@{}l@{}}No Beard\\ No Hats\end{tabular} & -      & -     & -       & -          & -         & -             & 0.723                  \\ \hline
\end{tabular}
\caption{The full results for the alignment scores ($\mathcal{A}_{\text{score}}$) of the CLIP ResNet.}
 \label{tbl:supp-cosine-clipRN}
\end{table*}




\begin{table*}[] \centering
\begin{tabular}{llllllll}
              & Female & Male  & No Hats & No Glasses & No Beards & \begin{tabular}[c]{@{}l@{}}No Beard\\ No Hats\end{tabular} & \begin{tabular}[c]{@{}l@{}}No Glasses\\ No Smiles\\ No Ties\end{tabular} \\ \hline
Full CelebA   & 0.308  & 0.417 & 0.433   & 0.486      & 0.455     & 0.409         & 0.416                  \\
Female        & -      & 0.251 & 0.348   & 0.476      & 0.468     & 0.495         & 0.365                  \\
Male          & -      & -     & 0.168   & 0.275      & 0.304     & 0.331         & 0.363                  \\
No Hats       & -      & -     & -       & 0.409      & 0.484     & 0.499         & 0.390                  \\
No Glasses    & -      & -     & -       & -          & 0.347     & 0.363         & 0.417                  \\
No Beards     & -      & -     & -       & -          & -         & 0.344         & 0.164                  \\
\begin{tabular}[c]{@{}l@{}}No Beard\\ No Hats\end{tabular} & -      & -     & -       & -          & -         & -             & 0.363                  \\ \hline
\end{tabular}
\caption{The full results for the alignment scores ($\mathcal{A}_{\text{score}}$) of the ImageNet ViT.}
 \label{tbl:supp-cosine-vit}
\end{table*}






\begin{table*}[] \centering
\begin{tabular}{llllllll}
              & Female & Male  & No Hats & No Glasses & No Beards & \begin{tabular}[c]{@{}l@{}}No Beard\\ No Hats\end{tabular} & \begin{tabular}[c]{@{}l@{}}No Glasses\\ No Smiles\\ No Ties\end{tabular} \\ \hline
Full CelebA   & 0.418  & 0.373 & 0.496   & 0.358      & 0.579     & 0.539         & 0.499                  \\
Female        & -      & 0.169 & 0.343   & 0.399      & 0.368     & 0.504         & 0.258                  \\
Male          & -      & -     & 0.183   & 0.381      & 0.213     & 0.187         & 0.225                  \\
No Hats       & -      & -     & -       & 0.448      & 0.577     & 0.422         & 0.513                  \\
No Glasses    & -      & -     & -       & -          & 0.529     & 0.385         & 0.440                  \\
No Beards     & -      & -     & -       & -          & -         & 0.509         & 0.440                  \\
\begin{tabular}[c]{@{}l@{}}No Beard\\ No Hats\end{tabular} & -      & -     & -       & -          & -         & -             & 0.368                  \\ \hline
\end{tabular}
\caption{The full results for the alignment scores ($\mathcal{A}_{\text{score}}$) of the CLIP ViT.}
 \label{tbl:supp-cosine-clipvit}
\end{table*}





\begin{table*}[] \centering
\begin{tabular}{llllllll}
              & Female & Male  & No Hats & No Glasses & No Beards & \begin{tabular}[c]{@{}l@{}}No Beard\\ No Hats\end{tabular} & \begin{tabular}[c]{@{}l@{}}No Glasses\\ No Smiles\\ No Ties\end{tabular} \\ \hline
Full CelebA   & 0.513  & 0.385 & 0.439   & 0.294      & 0.469     & 0.417         & 0.255                  \\
Female        & -      & 0.203 & 0.456   & 0.313      & 0.312     & 0.521         & 0.342                  \\
Male          & -      & -     & 0.371   & 0.286      & 0.238     & 0.201         & 0.306                  \\
No Hats       & -      & -     & -       & 0.322      & 0.411     & 0.378         & 0.302                  \\
No Glasses    & -      & -     & -       & -          & 0.459     & 0.427         & 0.286                  \\
No Beards     & -      & -     & -       & -          & -         & 0.301         & 0.324                  \\
\begin{tabular}[c]{@{}l@{}}No Beard\\ No Hats\end{tabular} & -      & -     & -       & -          & -         & -             & 0.236                  \\ \hline
\end{tabular}
\caption{The full results for the alignment scores ($\mathcal{A}_{\text{score}}$) of the MAE ViT.}
 \label{tbl:supp-cosine-mae}
\end{table*}

\end{document}
