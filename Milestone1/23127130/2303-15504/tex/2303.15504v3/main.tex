\documentclass[aps,prx,notitlepage,superscriptaddress,showpacs,twocolumn,pdflatex]{revtex4-2}
\usepackage{graphicx,subfigure,epsfig}
 \usepackage{array,multirow}
\usepackage{dcolumn}
\usepackage{amssymb,amsmath,amsfonts,mathrsfs}
\usepackage{array}
\usepackage{times,setspace}
\usepackage{latexsym}
\usepackage{float,flafter,bm,bbm}
\usepackage{epstopdf,color,multirow}
\usepackage[colorlinks,linkcolor=blue,anchorcolor=blue,urlcolor=blue,citecolor=blue]{hyperref}
\usepackage{footnote}
\usepackage{booktabs}
\usepackage{titletoc}

\hypersetup{
    colorlinks=true,
    linkcolor=blue,
    filecolor=magenta,
    urlcolor=blue,
}

\begin{document}
\title{The Ginzburg-Landau theory of flat band superconductors with quantum metric}
\author{Shuai A. Chen}
\email{chsh@ust.hk}

\affiliation{Department of Physics, Hong Kong University of Science and Technology,
Clear Water Bay, Hong Kong, China}
\author{K. T. Law}
\email{phlaw@ust.hk}
\affiliation{Department of Physics, Hong Kong University of Science and Technology,
Clear Water Bay, Hong Kong, China}
\date{\today}



\begin{abstract}
Recent experimental study unveiled highly unconventional phenomena in the superconducting twisted bilayer graphene (TBG) with ultra flat bands, which cannot be described by the conventional BCS theory. For example, given the small Fermi velocity of the flat bands, the predicted superconducting coherence length accordingly to BCS theory is more than 20 times shorter than the measured values. A new theory is needed to understand many of the unconventional properties of flat band superconductors. In this work, we  establish a Ginzburg-Landau (GL) theory from a microscopic flat band Hamiltonian. The GL theory shows how the properties of the physical quantities such as the critical temperature, the superconducting coherence length, the upper critical field and the superfluid density are governed by the quantum metric of the Bloch states. One key conclusion is that the superconducting coherence length is not determined by the Fermi velocity but by the size of the optimally localized Wannier functions which is limited by quantum metric. Applying the theory to TBG, we calculated the superconducting coherence length and the upper critical fields. The results match the experimental ones well without fine tuning of parameters. The established GL theory provides a new and general theoretical framework for understanding flat band superconductors with quantum metric.
\end{abstract}


\maketitle

\emph{\color{blue}Introduction.---} 
Our understanding of quantum states of matter has been greatly deepened by the study of the geometric properties of Bloch states in crystals. Specifically, the imaginary and real parts of the quantum geometric tensor of Bloch states, which are the Berry curvature and the quantum metric respectively, greatly influence the properties of the quantum state \cite{1980CMaPh76289P,berry1989quantum}.
The Berry curvature arises from the phase difference between two neighboring Bloch states and characterizes the band topology of states such as the quantum Hall and the Chern insulating states \cite{prlIQH,1982Thoulessprl,1984Berry,1994JMP355373B,TIRMP,TISCRMP,berry1989quantum,2023arXiv230302180B}. On the other hand, the quantum metric measures the distance between two adjacent Bloch states \cite{1990AAgeometry,2011EPJB79121R}. It describes the wave function extension and quantifies the level of obstructions of an exponentially localized Wannier basis \cite{PhysRevB.56.12847}. The quantum metric property is important for the formation of fractional quantum Hall and fractional Chern insulating states \cite{PhysRevLett.107.116801,PhysRevB.33.2481,2002LNP59598F,2013CRPhy14816P,PhysRevB.88.115117,PhysRevB.90.165139,2022arXiv221013487W}. More recent studies have shown the fundamental roles of quantum metric in various physical phenomena, including quantum transport and electromagnetic responses \cite{PhysRevB.87.245103,PhysRevLett.115.166802,PhysRevLett112166601,PhysRevB.94.134423,PhysRevResearch.3.L042018,PhysRevB.104.L100501,PhysRevLett.126.156602,2021NatPh..18..290A,2022PhRvB.105h5154M}, superfluidity and superconductivity in flat bands \cite{2015NatCo68944P,PhysRevB.95.024515,PhysRevA.97.033625,PhysRevB.98.220511,2020arXiv200716205J,PhysRevB.102.201112,PhysRevA.101.053631,PhysRevLett.117.045303,PhysRevLett.127.170404,PhysRevLett.128.087002,PhysRevB.106.014518,PhysRevB.105.L140506,PhysRevB.106.104514,2022arXiv220900007H,2022arXiv220402994H,2022arXiv221109846J}, and quantum phase transitions \cite{2006PhRvB..74w5111T,PhysRevLett99095701,PhysRevLett99100603,2021PNAS11806744V,2023PNAS..12017816M}. In particular, the effect of quantum metric on the properties of moir\'e materials has attracted much attention in recent years \cite{2018Natur.556...80C,2018Natur55643C,2019Natur.574..653L,PhysRevX.9.031049,PhysRevLett.123.237002,PhysRevB.101.060505,PhysRevResearch.2.023237,2020PhRvL124p7002X,PhysRevResearch4013164,PhysRevB.104.115160,PhysRevResearch.4.013209,2021arXiv211100807T,PhysRevLett.127.246403,PhysRevLett.128.176403}. 


The quantum metric effect on superconductivity in flat band systems with vanishing Fermi velocity $v_F$ is particularly interesting. On one hand, according to BCS theory, a large pairing gap $\Delta$ and a high critical temperature $T_c$ are expected due to the large density of states of flat bands. Moreover, the relation $\xi = \frac{\hbar v_{F}}{\Delta}$ seemingly implies a vanishing short coherence length $\xi$ such that electrons are tightly bound to form Cooper pairs.  These BCS relations point to a very robust superconducting state in flat band superconductors. On the other hand, the diverging effective mass implies a vanishing superfluid weight $\mathsf{D}_{s}$ as $\mathsf{D}_{s}\!\propto\!1/m^{*}\!\!\rightarrow0$. This implies the absence of supercurrents and the absence of Meissner effect which define superconductivity. Recently, Peotta and {T{\"o}rm{\"a}} \cite{2015NatCo68944P} shed light on the problem by pointing out that a supercurrent is indeed achievable and the superfluid weight is proportional to the quantum metric for the flat bands \cite{2015NatCo68944P,PhysRevB.95.024515,PhysRevB.106.014518}.  

Very recently, the superconducting properties of twisted bilayer graphene (TBG) with an extremely low Fermi velocity of $v_F \approx 1,000$m/s were studied experimentally. It was shown that many of the superconducting properties deviate greatly from the conventional BCS predictions \cite{2023Natur.614..440T}. For example, the coherence length is estimated to be around $2.6$nm according to the BCS relation $\xi = \frac{\hbar v_{F}}{\Delta}$, which is much shorter than the estimated value of 55nm (at optimal doping) according to the upper critical measurements. Due to the large effective mass of the electrons, it is also expected that the superfluid stiffness, which is proportional $\frac{1}{m^{*}}$, is low.  The  Berezinskii-Kosterlitz-Thouless (BKT) transition temperature is estimated to be about 0.05K which is much lower than the measured $T_c = 2.2$K at optimal doping. In short, BCS relations which connect physical quantities with $v_F$ or $m^{*}$ failed to provide a proper description of superconductivity in TBG. A new theory is needed to understand flat band superconductivity.

\begin{table}[]
\caption{Comparison between the BCS theory and the GL theory of flat band superconductors. The results for the
superfluid weight $\mathsf{D}_s$ at temperature $T$, superconducting transition temperature $T_c$, superconducting coherence length $\xi$, 
and upper critical field $H_{c2}$ are summerized. The quantum metric $\gamma_2^{ab}$ is defined in Eq.~\eqref{eq:gab} and $\bar\gamma_2^{ab}$ is averaged over the Brillouin zone as in Eq.~(\ref{eq:aver_gamma}).
 % $\mathrm{det}(\bar{\gamma}_{2}^{ab})$ is defined in Eq.~\eqref{eq:gab}. 
Here $\Delta_0(T)$ denotes the mean-field order parameter at temperature $T$, $T_\mathrm{MF}$ denotes the mean field critical temperature determined by Eq.~\eqref{eq:self-eq}, $\mathcal A_\mathrm{uc}$ is the area of a unit cell and $\Phi_0=hc/2e$ is the flux quantum.
For a BCS superconductor, $n_s$ is the superfluid density. 
The experimental values are adopted from Ref.~\onlinecite{2023Natur.614..440T}.
}
\label{tab:summary}
\begin{tabular}{ccccc}
\toprule
                  & BCS      & Flatband & TBG(Exp.) & Theory \\ 
\midrule
$\mathsf{D}_s$ & $\frac{n_{s}}{m}$       &  $\frac{2g\Delta_{0}(T)}{ \mathcal A_\mathrm{uc} }\sqrt{\mathrm{det}(\bar{\gamma}_{2}^{ab})}$ &    \\
$T_c$               & $ 1.75^{-1}g\Delta_{0}$        & $\frac{\pi g\Delta_0(T_{\mathrm{BKT}})}{8\mathcal A_\mathrm{uc}}\sqrt{\mathrm{det}(\bar{\gamma}_{2}^{ab})}$    &    $2.2$K  & $1.6$K  \\
$\xi$               & $\frac{\hbar v_{F}}{g\Delta_0}$     &  $\sqrt{\frac{T_\mathrm{MF}}{\vert T- T_\mathrm{MF}\vert}  }[\mathrm{det}(\bar{\gamma}_{2}^{ab})]^{\frac{1}{4}}$  &  $55$nm &$ 35$nm  \\
$H_{c2}$            &  $2\pi (\frac{T_{c}}{v_{F}})^{2}$      &   $\frac{\vert T-T_\mathrm{MF}\vert}{T_\mathrm{MF}}\frac{\Phi_0}{2\pi\sqrt{\det(\bar{\gamma}_{2}^{ab})}}$  &   $0.10$T &  $ 0.26$T     \\
\bottomrule
\end{tabular}
\end{table}

In this work, we develop the Ginzburg-Landau (GL) theory of flat band superconductors by incorporating the quantum geometric properties of the Bloch electrons. 
Besides reproducing previous results concerning the BKT transition temperature \cite{2015NatCo68944P} and the superfluid weight \cite{2015NatCo68944P}, the GL theory allows us to determine the coherence length and the upper critical field and their dependence on the quantum metric. The results are summarized in Table.~\ref{tab:summary}. Applying our theory to TBG with a small Fermi velocity, we estimated the coherence length, and the upper critical field which match the experimental measurements very well without the fine tuning of parameters as shown in Table.~\ref{tab:summary} and Fig.~\ref{Fig:TBG}. A striking result concerning $\xi$ is that it is independent of interaction strength at zero temperature and purely determined by the quantum metric effect (See also Eq.~(\ref{eq:cohlen})). Contrary to the conventional understanding that a stronger interaction will bind electrons closer together to reduce the Cooper pair size (which is measured by~$\xi$), the quantum metric limits the size of the Cooper pairs. The Cooper pair size cannot be smaller than the size of optimally localized Wannier functions \cite{PhysRevB.56.12847} constructed by the Bloch states. In the case of TBG, the quantum metric limits $\xi$ to be tens of nanometers as observed in the experiment \cite{2023Natur.614..440T}. 

In the following, we first derive the GL free energy which incorporates the quantum geometry effects of Bloch electrons. Second, the superfluid weight, the upper critical field and the superconducting coherence length are derived. Finally, we apply the GL theory to explain the unconventional behaviors of superconducting TBG.


%%%%%%%%%%%%%%%%%%%%%%%%%%%%%%%%%%%%%%%%
\emph{\color{blue} The Ginzburg-Landau Free Energy.---} 
We start with a model Hamiltonian $H=H_{0}+H_{\mathrm{int}}$ defined on a lattice containing $N$ sites. It is assumed that $H_{0}$ possesses an isolated flat band at the Fermi energy. In general, the flat band has Bloch states which take the form $e^{-i\mathbf{k}\cdot\mathbf{r}}g_{\mathbf{k}\xi}(\alpha)$, where $\alpha$ indexes the orbital degrees of freedom. Even though the band is completely flat, there can be nontrivial quantum geometry effects encoded by $g_{\mathbf{k}\xi}(\alpha)$. An example of a nontrivial flat band is a band with a none zero Chern number for which it is not possible to find a complete set of exponentially localized Wannier basis \cite{PhysRevB.56.12847}.
The occurrence of a superconducting phase is associated with the presence of an attractive interaction in a $d$-spatial dimensions, 
\begin{equation}
H_{\mathrm{int}}=-g\int d\mathbf r  a_{+}^{\dagger}(\mathbf{r})a_{-}^{\dagger}(\mathbf{r})a_{-}(\mathbf{r})a_{+}(\mathbf{r})~,\label{eq:Hint}
\end{equation}
with $a_{\xi}(\mathbf{r})$ as the electron annihilation operator carrying two flavors $\xi=\pm$. To resolve the role of quantum geometry with interactions, we project the electron operators to the Bloch electrons of the targeted flat band 
\begin{equation}
a_{\xi}(\mathbf{r})\rightarrow\frac{1}{\sqrt{N}}\sum_{\mathbf{q}}e^{i\mathbf{q}\cdot\mathbf{r}}g_{\mathbf{q}\xi}^{*}(\alpha)c_{\mathbf{q}\xi}~,\label{eq:projectiongk}
\end{equation}
where $c_{\mathbf{q}\xi}$ annihilates an electron with momentum $\mathbf{q}$ over the first Brillouin  on the targeted band  
% with $-\pi\leq q_{x,y}<\pi$ 
and $\alpha$ represents other quantum numbers of the Bloch state. The expansion in Eq.~(\ref{eq:projectiongk})
projects out other bands while $g_{\mathbf{k}\xi}$ encodes the quantum geometry effect \cite{PhysRevLett.93.206602,2017AnPhy.377..345C}.
We proceed with the Hubbard-Stratonovich transformation by introducing
a bosonic field $\Delta(\mathbf{r})$, 
\begin{align}
\Delta(\mathbf{r}) & =a_{-}(\mathbf{r})a_{+}(\mathbf{r})~,
\end{align}
Then, the Lagrangian density $\mathcal{L}$ is obtained through the path integral approach (see Supplementary Materials(SM) \cite{SM}) such that
\begin{align}
\mathcal{L}= & (-i\omega-\mu)(\bar{c}_{\mathbf{k},+}c_{\mathbf{k},+}+\bar{c}_{\mathbf{k},-}c_{\mathbf{k},-})\nonumber \\
 & -g\sum_{\mathbf{q}}[\Gamma(\mathbf{q},\mathbf{k})\Delta(\mathbf{k})\bar{c}_{\mathbf{q}+\mathbf{\frac{k}{2}},+}\bar{c}_{\mathbf{-q}+\frac{\mathbf{k}}{2},-}+h.c.],
\end{align}
where $c_{\mathbf{q},\xi}$ denotes the Grassmann fields, $\mu$ is
the chemical potential and  $\Delta(\mathbf{k})\equiv \sum_{\mathbf{r}}\Delta(\mathbf{r})e^{i\mathbf{k}\cdot\mathbf{r}}$ is the Fourier component of the bosonic field $\Delta(\mathbf{r})$. 
The projection in Eq.~\eqref{eq:projectiongk} introduces the form factor  $\Gamma(\mathbf{q},\mathbf{k})$ that 
modifies the coupling constant $g$. The form factor is defined as $\Gamma(\mathbf{q},\mathbf{k})\equiv\sum_{\alpha}g_{\mathbf{\mathbf{-q}+\mathbf{k}/2},+}(\alpha)g_{\mathbf{\mathbf{q}+\mathbf{k}/2},-}(\mathbf{\alpha})$, and it plays a crucial role in the context of superconductivity.
Formally, the GL free energy $F[\Delta]$ is obtained by integrating out the fermion fields
at a finite temperature $T$ such that, 
\begin{equation}
F[\Delta]=\sum_{\mathbf{k}}\!g\bar{\Delta}(\mathbf{k})\Delta(\mathbf{k})-T\ln\int\mathcal{D}[c,\bar{c}]e^{-\int_{0}^{\beta}d\tau\sum_{\mathbf{q}}\mathcal{L}}~.\label{eq:GL_S}
\end{equation}
To calculate $F[\Delta]$, we perform an expansion $\Delta(\mathbf{k})=\Delta_{0}\delta_{\mathbf{k},\mathbf{0}}+\delta\Delta(\mathbf{k})$
around the extremum of $F[\Delta]$. Here, $\Delta_{0}$ represents the mean-field value at temperature $T$, while $\delta\Delta(\mathbf{k})$ represents the fluctuations of the order parameter.
By minimizing the GL free energy $\frac{\partial F[\Delta]}{\partial\Delta_{0}}=\frac{\partial F[\Delta]}{\partial\bar{\Delta}_{0}}=0$, the mean field order parameter $\Delta_{0}$ can be determined from the self-consistent
 gap equation 
\begin{equation}
1=\frac{1}{N}\sum_{\mathbf{q}}\frac{g|\Gamma(\mathbf{q})|^2}{2\epsilon(\mathbf{q})}\tanh\frac{\beta\epsilon(\mathbf{q})}{2}, \label{eq:self-eq}
\end{equation}
where $\epsilon(\mathbf{q})=\sqrt{\left|g\Gamma(\mathbf{q})\Delta_{0}\right|^{2}+\mu^{2}}$
denotes the dispersion of Bogoliubov quasiparticles. In the presence
of time-reversal symmetry $g_{-\mathbf{k},-}=g_{\mathbf{k},+}^{*}$,
$\Gamma(\mathbf{q})=1$ and under a uniform ansatz $\Delta(\mathbf{r})=\Delta_{0}$,
from Eq.~\eqref{eq:self-eq} one may extract a relation $g\Delta_{0}/T_{\mathrm{MF}}=2$  at half-filling 
with $T_\mathrm{MF}$ as the critical temperature and $g\Delta_0$ as the paring gap (see SM \cite{SM}),
which is larger
than the ratio ($\sim1.7$) from a conventional BCS theory.

Going beyond mean field and include the fluctuations, we have $F[\Delta]=F_{0}+F_{2}+\mathcal{O}(|\delta\Delta|^{4})$
up to the second order of $\delta\Delta(\mathbf{k})$. In particular,
$F_{0}$ recovers the grand potential 
\begin{equation}
F_{0}=\sum_{\mathbf{k}}\left[g\vert\Delta_{0}\vert^{2}-2\ln(1+e^{-\beta\epsilon(\mathbf{k})})+\epsilon(\mathbf{k})\right].\label{eq:grandpot}
\end{equation}
The second order $F_{2}\equiv\sum_{\mathbf{k}}\mathcal{L}[\delta\Delta]$
describes the Gaussian fluctuations with 
\begin{equation}
\mathcal{L}[\delta\Delta]=\left[g-g^{2}\chi(\mathbf{k})\right]\delta\bar{\Delta}(\mathbf{k})\delta\Delta(\mathbf{k})~,\label{eq:L_Delta}
\end{equation}
where $\chi(\mathbf{k})$ is the four-point correlation function,
\begin{align}
\chi(\mathbf{k})\equiv & \frac{1}{N}\sum_{q}\vert\Gamma(\mathbf{q},\mathbf{k})\vert^{2}[\mathcal{G}(q+k/2)\mathcal{G}(-q+k/2)\nonumber \\
 & +\mathcal{F}(q+k/2)\mathcal{F}(-q+k/2)]~,\label{eq:chi}
\end{align}
with Gor'kov's normal and anomalous Green functions $\mathcal{G}(q)$
and $\mathcal{F}(q)$ ($q=(\mathbf{q},\omega)$) (see SM \cite{SM}). 
In contrast to conventional superconductors, the Bloch wavefunctions play a significant role in both the
effective interaction and the quasiparticle dispersion. The prefactor $|\Gamma(\mathbf{q},\mathbf{k})|^{2}$ 
in Eq.~(\ref{eq:chi}) highlights the importance of the wavefunctions of the Bloch wavefunctions. The significance of $|\Gamma(\mathbf{q},\mathbf{k})|^{2}$ is that, the pairing strength of a finite momentum Cooper pair is weighed by $\Gamma(\mathbf{q},\mathbf{k})$, such that $\chi(\mathbf{k})$ is $\mathbf{k}$-dependent. This generates a finite superfluid weight to account for the Meissner effect, even though the effective mass of electrons diverges for a completely flat band. As we show below, the form factor encodes the quantum metric effects. 

%%%%%%%%%%%%%
\emph{\color{blue}Superfluid weight, BKT transition and quantum metric.---} In general, the form factor can be expanded as a function of $\mathbf{k}$ up to the second order:
\begin{equation}
\vert\Gamma(\mathbf{q},\mathbf{k})\vert^{2}=\gamma_{0}(\mathbf{q})-\sum_{ab}\gamma_{2}^{ab}(\mathbf{q})k_{a}k_{b},\label{eq:chi2expan}
\end{equation}
where the absence of a linear term is due to the stability of the mean-field ansatz. 




For the time-reversal invariant system where $g_{\mathbf{q},+}=g_{\mathbf{-q},-}^{*}\equiv g_{\mathbf{q}}$,
$\gamma_{0}(\mathbf{q})$ becomes the inner product $\gamma_{0}(\mathbf{q})\equiv\vert\langle g_{\mathbf{q}}\vert g_{\mathbf{q}}\rangle\vert^{2}=1$
and $\gamma_{2}(\mathbf{q})$ is the Fubini-Study
metric \citep{1980CMaPh76289P,2010arXiv1012.1337C} with components
\begin{equation}
\gamma_{2}^{ab}(\mathbf{q})\equiv\mathrm{Re}\langle\partial_{\mathbf{q}_{a}}g_{\mathbf{q}}\vert(1-|g_{\mathbf{q}}\rangle\langle g_{\mathbf{q}}|)\vert\partial_{\mathbf{q}_{b}}g_{\mathbf{q}}\rangle,\label{eq:gab}
\end{equation}
which measures the Bures distance between two quantum states. The quantum metric $\gamma_{2}^{ab}(\mathbf{q})$ characterizes
how the Bloch states interfere with each other.
The appearance of the quantum metric
in Eq.~\eqref{eq:chi}  and Eq.~\eqref{eq:chi2expan} clearly illustrates the crucial
role of the quantum geometric effect in determining superconductivity
fluctuations. In contrast, this effect is not evident at the mean-field level, as demonstrated by Eq.~\eqref{eq:grandpot}.
After integrating out
the Matsubara frequency along with the expansion in Eq.~(\ref{eq:chi2expan}),
we have  
$\chi(\mathbf{k})=\chi_{0}-\frac{1}{8}\sum_{ab}\mathsf{\chi}_2^{ab}k_{a}k_{b}$,
with the explicit form as 
\begin{align}
\chi_{0} & =\frac{1}{N}\sum_{\mathbf{q}}\frac{\gamma_{0}(\mathbf{q})}{2}\frac{1}{\epsilon(\mathbf{q})},\label{eq:chi0}\\
\chi_{2}^{ab} & =\frac{2g^{2}\Delta_{0}^{2}}{N}\sum_{\mathbf{q}}\frac{\tanh\left(\frac{\beta\epsilon(\mathbf{q})}{2}\right)}{\epsilon(\mathbf{q})}\gamma_{2}^{ab}(\mathbf{q}).\label{eq:chiab}
\end{align}
Thus in the continuum limit $F_2= \int d^d \mathbf r \mathcal L[\delta\Delta]$, we reach an effective theory  $\mathcal L[\delta\Delta]$ 
\begin{equation}
\mathcal{L}[\delta\Delta]=\frac{1}{8}\sum_{ab}\mathsf{D}_{s}^{ab}\partial_{a}\delta\bar{\Delta}\partial_{b}\delta\Delta.\label{eq:LdeltaD}
\end{equation}
The presence of the factor $\mathsf{D}_{s}^{ab}\equiv  \mathcal A_\mathrm{uc}^{-1} \chi_{s}^{ab}$, with $\mathcal A_\mathrm{uc}$ being area of the unit cell,
which depends on $\gamma_{2}^{ab}(\mathbf{q})$, indicates that the dynamics of the fluctuation of the order parameter is governed by the quantum geometry. The superfluid weight is measured by the factor $\gamma_{2}^{ab}(\mathbf{q})$. Conversely, in the absence of quantum geometry ($\gamma_{2}^{ab}=0$), as indicated by $\mathsf{D}_{s}^{ab}=0$ in Eq.~\eqref{eq:LdeltaD}, there will be no fluctuations.
Examine the phase fluctuations 
$\delta\Delta(\mathbf{r})=\Delta_{0}e^{2i\theta(\mathbf{r})}-\Delta_{0}\simeq2i\theta(\mathbf{r})\Delta_{0}$
(and ignore the amplitude fluctuations),  and we obtain the effective Lagrangian 
\begin{equation}
\mathcal{L}[\theta]=\frac{1}{2}\text{\ensuremath{\sum_{ab}}}\mathsf{D}_{s}^{ab}\partial_{a}^{}\theta\partial_{b}^{}\theta~,
\end{equation}
and the supercurrent $j_{b}^{}=\sum_{a}\mathsf{D}_{s}^{ab}\partial_{a}^{}\theta$.
We can now determine the BKT transition temperature $T_{\mathrm{BKT}}=\pi\sqrt{\mathrm{det}\mathsf{D}_{s}^{ab}}/8$ \cite{PhysRevLett.39.1201}, by identifying the factor $\mathsf{D}_{s}^{ab}$ in Eq.~(\ref{eq:chiab}) as the intrinsic superfluid weight, which is consistent with previous studies \cite{2015NatCo68944P,PhysRevB.95.024515,PhysRevB.106.014518}. 
Basically, the order parameter $\Delta_0$ depends on temperature $T$, and thus $T_\mathrm{BKT}$ should be solved self-consistently.
For an isotropic superconductor with a flat Bogoliubov quasiparticle band, a simple relation $T_{\mathrm{BKT}}/g\Delta_{0}=\frac{\pi\sqrt{\mathrm{det}\bar{\gamma}_{2}^{ab}}}{8\mathcal A_\mathrm{uc}}$ holds around filling $\mu=0$, where the average of the quantum metric over the Brillouin zone is:
\begin{equation}
\bar{\gamma}_{2}^{ab}= \frac{1}{N}\sum_\mathbf q\gamma_{2}^{ab}(\mathbf{q}). 
\label{eq:aver_gamma}
\end{equation}
Interesting, when the $\bar{\gamma}_{2}^{ab}$ is tuned to be sufficiently large, $T_\mathrm{BKT}$ will approaches to $T_\mathrm{MF}$ that is obtained from Eq.~\eqref{eq:self-eq} (see SM \cite{SM}). 
It is important to emphasize that in conventional superconductors, $\mathsf D_{s} = n_s/m^*$  with $n_s$ being the superfluid density at temperature $T$, such that $T_{\mathrm{BKT}}=\pi n_s/(8m^*)$ which goes to zero for flat bands. 



\emph{\color{blue} Upper critical field $H_{c2}$.---} 
Another important physical quantity of a superconductor is the (orbital) upper critical field which is expected to be infinite according to BCS theory for a flat band. As the mean-field order parameter is suppressed by vortex excitations around $H_{c2}$, we can derive the GL free energy from Eq.~\eqref{eq:GL_S}  by assuming a vanishing mean field, $\Delta_{0}=0$. An effective Lagrangian can be obtained after integrating out the fermion field, and for an isotropic system, we have in the continuum limit $F_2= \int d^d \mathbf r \mathcal L[\delta\Delta]$,
\begin{equation}
\mathcal{L}[\delta\Delta]\!=\!\frac{1}{2m^{*}}\vert\nabla\delta\Delta\vert^{2}+a(T)\vert\delta\Delta\vert^{2}+\mathcal{O}(\vert\delta\Delta\vert^{4}),\label{eq:Lhc2}
\end{equation}
with 
\begin{align}
\frac{1}{2m^{*}} & =\frac{\beta g^{2}\bar{\nu}}{4\mathcal A_\mathrm{uc}}\sqrt{\mathrm{det}(\bar{\gamma}_{2}^{ab})},\\
a(T) & =  \frac{ 4g - g^2 \beta\bar{\nu}\bar{\gamma}_{0}}{4\mathcal A_\mathrm{uc} } ,
\end{align}
where $\bar{\gamma}_{0}=\frac{1}{N}\sum_{\mathbf{q}}\gamma_{0}(\mathbf{q})$ and $\bar{\nu}=\frac{2(1-2\nu)}{\ln(\nu^{-1}-1)}$ with $\nu$ being the filling factor.
\emph{Hereafter}, we focus on a system of time reversal symmetry to simply have $\bar{\gamma}_{0}=1$.
We discover that the quantum metric $\sqrt{\mathrm{det}(\bar{\gamma}_{2}^{ab})}$ gives rise to a finite effective mass of Cooper pairs, whereas the change in sign of $a(T)$ gives rise to the mean field critical temperature $T_{\mathrm{MF}}=g\frac{\bar{\nu}}{4}$.
The magnetic field can be included in the free energy by the minimal coupling $-i\nabla\rightarrow-i\nabla+2e\mathbf{A}$ in Eq.~(\ref{eq:Lhc2}). Then, the upper critical field $H_{c2}$ can be determined using the standard GL approach \cite{altland2010condensed} and we have:
\begin{equation}
H_{c2}=\frac{\Phi_0}{2\pi\sqrt{\det(\bar{\gamma}_{2}^{ab})}}\frac{\vert T-T_\mathrm{MF}\vert}{T_\mathrm{MF}}. \label{eq:hc2}
\end{equation}
where $\Phi_0=hc/2e$ is a flux quantum.
It is clear that a finite average quantum metric $\bar{\gamma}_{2}^{ab}$ gives rise to a finite $H_{c2}$. From the inequality, $\sqrt{\det\gamma_{2}^{ab}(\mathbf{k})}\geq \frac{1}{2}\vert\mathrm{Tr}\mathcal{B}(\mathbf{k})\vert$
for two dimensional systems \citep{1980CMaPh76289P,2015NatCo68944P}, we have
$\sqrt{\det(\bar{\gamma}_{2}^{ab})}\geq 1/N\sum_\mathbf{q}\sqrt{\mathrm{det}(\gamma_2^{ab})}\geq 1/N\sum_\mathbf{q}\frac{1}{2}\vert\mathrm{Tr}\mathcal{B}(\mathbf{k})\vert\geq \mathcal A_\mathrm{uc}\frac{|C|}{2\pi},$
where $\mathcal{B}(\mathbf{k})$ is the Berry curvature and $C$ is the Chern number. This leads to two important observations. First, the upper critical field $H_{c2}$ is bound
by $H_{c2}\leq\frac{\Phi_0}{\mathcal A_\mathrm{uc}|C|}\frac{\vert T- T_\mathrm{MF}\vert}{T_\mathrm{MF}}$
for a topological band. Second, even for
a topological trivial band, as long as the Berry curvature is locally finite, the quantum geometry is relevant and is bounded by the averaged 
$\vert\mathrm{Tr}\mathcal{B}(\mathbf{k})\vert$.
Moreover, using the condition $H_{c2}\xi^{2}= \Phi_{0}/2\pi$, 
we find that  the superconducting coherence length to be :
\begin{equation}
\xi=\sqrt{\frac{T_\mathrm{MF}}{\vert T- T_\mathrm{MF}\vert}  }[\mathrm{det}(\bar{\gamma}_{2}^{ab})]^{\frac{1}{4}}.
\label{eq:cohlen}
\end{equation}
As expected, the expression of $\xi$ is dramatically different from the BCS relation $\xi = \frac{\hbar v_{F}}{\Delta}$ which vanishes as $v_F$ goes to zero. It is important to note that at zero temperature, the coherence length is reduced to $\xi (T=0)=[\mathrm{det}(\bar{\gamma}_{2}^{ab})]^{\frac{1}{4}}$ which is the size of optimally localized wannier functions constructed from the Bloch states of the flat band \cite{PhysRevB.56.12847,2015NatCo68944P} (also see SM \cite{SM}). This is a very interesting result that $\xi$ is the shortest at $T=0$ and independent of the interaction strength $g$. It means that the minimal Cooper pair size is purely determined by the quantum metric and a stronger interaction cannot bound the electrons closer to each other. When temperature increases, the interaction energy is incorporated into $T_\mathrm{MF}$ and affects $\xi$.


\begin{figure}[th]
\centering \includegraphics[scale=0.68]{metric} \caption{
(a) The band structure and distribution of the quantum metric $\sqrt{\mathrm{det}(\gamma_{2}^{ab})}$, as shown in Eq.~\eqref{eq:gab}, over the moir\'e BZ in a TBG. The quantum metric diverges at the two Dirac points at $\mathrm{K}$ and $\mathrm{K}^{\prime}$, while it significantly contributes at $\Gamma$.
(b) The superconducting coherence length $\xi$ as a function of chemical potential $\mu$ in the low-temperature limit. Initially, as doping moves away from $\mu=-0.1\mathrm{meV}$, $\xi$ decreases, but it then increases around $\mu=-0.3$meV. Our calculations use a temperature of $\beta=200\mathrm{meV}^{-1}$, and we only display one spin-valley species for simplification.
}
\label{Fig:TBG} 
\end{figure}
%%%%%%%%%%%%%%%%%%%%%%%%%%%%%%%%

\emph{\color{blue} Application to TBG---} 
Recently, superconductivity was observed in TBG with extremely small $v_F$, estimated to be around $1,000$m/s. In this section, we employ the GL theory developed above to explain the observed superconducting coherence length which is about 20 times larger than the one estimated from BCS theory. 

As shown in Eq.~(\ref{eq:cohlen}), to calculate the coherence length $\xi$ at zero temperature, we calculate the quantum metric of the Bloch states of the flat bands using the Bistrizer-Macdonald model to describe the moir\'e band structure of TBG near the magic angle \cite{2011PNAS..10812233B}. The details of the model is given in the SM \cite{SM}.  In Fig.~\ref{Fig:TBG}(a), the band structure at twisted angle $\theta=1.08^{\circ}$ is shown and the bandwidth is extremely narrow which is on the order of 1meV. For simplicity, only the moir\'e bands originated from one valley is shown.


It is clear that the valence (solid line) and the conduction (dashed line) bands touch at the Dirac points at $\mathsf{K}$ and $\mathsf{K}^{\prime}$ of the moir\'e Brillouin zone. Here, the energy at the Dirac points is set to have chemical potential $\mu=0$, corresponding to the filling factor of $\nu=0$.
To model the superconducting phase, we assume a simple singlet pairing potential \cite{PhysRevB.101.060505,PhysRevLett.123.237002}  
$H_{\mathrm{int}}=-g\int d^2\mathbf r\sum_{\ell\xi}\psi_{\uparrow\rho,\ell\xi}^{\dagger}(\mathbf{r})\psi_{\downarrow\bar{\rho},\ell\xi}^{\dagger}(\mathbf{r})\psi_{\downarrow\bar{\rho},\ell\xi}(\mathbf{r})\psi_{\uparrow\rho,\ell\xi}(\mathbf{r})$
with the valley indices $\rho=\pm$, sublattices $\xi=A,B$, and layer
$\ell$, while ignoring other correlation effects induced by interactions between electrons \cite{2018Natur.556...80C,PhysRevLett.124.097601}.  
Here $\psi_{\sigma\rho,\ell\xi}$ denotes the Fermion operators in the continuum limit. 
It is important to note that the pairing needs not be in this form in TBG but we focus on the quantum metric effect here regardless of the details  of the pairing.


We can determine the superconducting coherence length within the GL theory using Eq.~\eqref{eq:cohlen}. At the low temperature limit $\beta\rightarrow\infty$, the coherence length originating from quantum metric is essentially determined by the quantum metric $\xi=[\mathrm{(}\mathrm{det}\bar{\gamma}_{2}^{ab})]^{1/4}$, which is independent of the pairing coupling $g$. The quantum metric as a function of momentum is shown in Fig.~\ref{Fig:TBG}(a). It is interesting to note that the quantum metric diverges around the two Dirac points and takes a relatively large value at $\Gamma$. When evaluating the coherence length $\xi$, we take into account the finite bandwidth of TBG in Eq.~\eqref{eq:chi} and the details are shown in SM \cite{SM}. Fig.~\ref{Fig:TBG}(b) shows the variation of $\xi$ with different chemical potentials $\mu$ which is relevant to the experimental regime where the filling factor is below $-1/2$ at $\mu=-0.12\mathrm{meV}$.  In the regime  $-0.28\mathrm{meV} < \mu < -0.2\mathrm{meV}$, we see a decrease and then increase of $\xi$ as the chemical potential lowers. The increase of $\xi$ at lower chemical potential is due to the increase of quantum metric near the $\Gamma$-point of the model. Amazingly,   similar $\xi$ dependence on the chemical potential was observed in the experiment \cite{2023Natur.614..440T}. Without fine tuning of parameters, we obtained $\xi \approx 30 \mathrm{nm}$ at zero temperature which is comparable with the experimental values of about $55 \mathrm{nm}$. The estimated $H_{c2}$ are also plotted in Fig.~\ref{Fig:TBG}(b). The optimal $H_{c2}\sim0.26\mathrm{T}$ which is comparable with the experimental value of $0.1\mathrm{T}$. It is important to note that the deviation of the theoretical results at $T=0$ from the experimental results can be due to the finite temperature effects which increase $\xi$ and reduce $H_{c2}$. 

\emph{\color{blue} Conclusion.---} 
We developed a GL theory for flat band superconductors which includes the quantum metric effects. The GL theory allows us to derive many of the important physical quantities of superconductors in terms of the quantum metric of the Bloch electrons as summerized in Table I. Importantly, we found that the coherence length, which is expected to be zero from conventional BCS theory, is finite for flat bands with quantum metric. Physically, the size of the optimally localized Wannier funcions, which is governed by the quantum metric, determines the superconducting coherence length at zero temperature. By calculating the quantum metric of TBG, we explained the coherence length dependence on the charge density in the experiment. This GL theory provides a general framework to understand the unconventional properties of flat band superconductors with quantum metric.

\emph{Acknowledgments.---} We thank Jeanie Lau for informing us their experimental results which inspired this work. We acknowledge valuable discussions with Tai-Kai Ng, Adrian Po, Wen Huang and Yan-bin Yang. K.T.L. acknowledges the support of the Ministry of Science and Technology, China, and the Hong Kong Research Grants Council through Grants No. 2020YFA0309600, No. RFS2021-6S03, No. C6025-19G, No. AoE/P-701/20, No. 16310520, No. 16310219, No. 16307622, and No. 16309718.


% \bibliography{ref}
%apsrev4-2.bst 2019-01-14 (MD) hand-edited version of apsrev4-1.bst
%Control: key (0)
%Control: author (8) initials jnrlst
%Control: editor formatted (1) identically to author
%Control: production of article title (0) allowed
%Control: page (0) single
%Control: year (1) truncated
%Control: production of eprint (0) enabled
\begin{thebibliography}{77}%
\makeatletter
\providecommand \@ifxundefined [1]{%
 \@ifx{#1\undefined}
}%
\providecommand \@ifnum [1]{%
 \ifnum #1\expandafter \@firstoftwo
 \else \expandafter \@secondoftwo
 \fi
}%
\providecommand \@ifx [1]{%
 \ifx #1\expandafter \@firstoftwo
 \else \expandafter \@secondoftwo
 \fi
}%
\providecommand \natexlab [1]{#1}%
\providecommand \enquote  [1]{``#1''}%
\providecommand \bibnamefont  [1]{#1}%
\providecommand \bibfnamefont [1]{#1}%
\providecommand \citenamefont [1]{#1}%
\providecommand \href@noop [0]{\@secondoftwo}%
\providecommand \href [0]{\begingroup \@sanitize@url \@href}%
\providecommand \@href[1]{\@@startlink{#1}\@@href}%
\providecommand \@@href[1]{\endgroup#1\@@endlink}%
\providecommand \@sanitize@url [0]{\catcode `\\12\catcode `\$12\catcode
  `\&12\catcode `\#12\catcode `\^12\catcode `\_12\catcode `\%12\relax}%
\providecommand \@@startlink[1]{}%
\providecommand \@@endlink[0]{}%
\providecommand \url  [0]{\begingroup\@sanitize@url \@url }%
\providecommand \@url [1]{\endgroup\@href {#1}{\urlprefix }}%
\providecommand \urlprefix  [0]{URL }%
\providecommand \Eprint [0]{\href }%
\providecommand \doibase [0]{https://doi.org/}%
\providecommand \selectlanguage [0]{\@gobble}%
\providecommand \bibinfo  [0]{\@secondoftwo}%
\providecommand \bibfield  [0]{\@secondoftwo}%
\providecommand \translation [1]{[#1]}%
\providecommand \BibitemOpen [0]{}%
\providecommand \bibitemStop [0]{}%
\providecommand \bibitemNoStop [0]{.\EOS\space}%
\providecommand \EOS [0]{\spacefactor3000\relax}%
\providecommand \BibitemShut  [1]{\csname bibitem#1\endcsname}%
\let\auto@bib@innerbib\@empty
%</preamble>
\bibitem [{\citenamefont {{Provost}}\ and\ \citenamefont
  {{Vallee}}(1980)}]{1980CMaPh76289P}%
  \BibitemOpen
  \bibfield  {author} {\bibinfo {author} {\bibfnamefont {J.~P.}\ \bibnamefont
  {{Provost}}}\ and\ \bibinfo {author} {\bibfnamefont {G.}~\bibnamefont
  {{Vallee}}},\ }\bibfield  {title} {\bibinfo {title} {{Riemannian structure on
  manifolds of quantum states}},\ }\href {https://doi.org/10.1007/BF02193559}
  {\bibfield  {journal} {\bibinfo  {journal} {Communications in Mathematical
  Physics}\ }\textbf {\bibinfo {volume} {76}},\ \bibinfo {pages} {289}
  (\bibinfo {year} {1980})}\BibitemShut {NoStop}%
\bibitem [{\citenamefont {Berry}(1989)}]{berry1989quantum}%
  \BibitemOpen
  \bibfield  {author} {\bibinfo {author} {\bibfnamefont {M.~V.}\ \bibnamefont
  {Berry}},\ }\bibfield  {title} {\bibinfo {title} {The quantum phase, five
  years after},\ }in\ \href@noop {} {\emph {\bibinfo {booktitle} {Geometric
  phases in physics}}},\ \bibinfo {editor} {edited by\ \bibinfo {editor}
  {\bibfnamefont {A.}~\bibnamefont {Shapere}}\ and\ \bibinfo {editor}
  {\bibfnamefont {F.}~\bibnamefont {Wilczek}}}\ (\bibinfo  {publisher} {World
  scientific},\ \bibinfo {address} {Singapore},\ \bibinfo {year} {1989})\
  Chap.~\bibinfo {chapter} {1}, pp.\ \bibinfo {pages} {1--28}\BibitemShut
  {NoStop}%
\bibitem [{\citenamefont {Klitzing}\ \emph {et~al.}(1980)\citenamefont
  {Klitzing}, \citenamefont {Dorda},\ and\ \citenamefont {Pepper}}]{prlIQH}%
  \BibitemOpen
  \bibfield  {author} {\bibinfo {author} {\bibfnamefont {K.~v.}\ \bibnamefont
  {Klitzing}}, \bibinfo {author} {\bibfnamefont {G.}~\bibnamefont {Dorda}},\
  and\ \bibinfo {author} {\bibfnamefont {M.}~\bibnamefont {Pepper}},\
  }\bibfield  {title} {\bibinfo {title} {New method for high-accuracy
  determination of the fine-structure constant based on quantized hall
  resistance},\ }\href {https://doi.org/10.1103/PhysRevLett.45.494} {\bibfield
  {journal} {\bibinfo  {journal} {Phys. Rev. Lett.}\ }\textbf {\bibinfo
  {volume} {45}},\ \bibinfo {pages} {494} (\bibinfo {year} {1980})}\BibitemShut
  {NoStop}%
\bibitem [{\citenamefont {Thouless}\ \emph {et~al.}(1982)\citenamefont
  {Thouless}, \citenamefont {Kohmoto}, \citenamefont {Nightingale},\ and\
  \citenamefont {den Nijs}}]{1982Thoulessprl}%
  \BibitemOpen
  \bibfield  {author} {\bibinfo {author} {\bibfnamefont {D.~J.}\ \bibnamefont
  {Thouless}}, \bibinfo {author} {\bibfnamefont {M.}~\bibnamefont {Kohmoto}},
  \bibinfo {author} {\bibfnamefont {M.~P.}\ \bibnamefont {Nightingale}},\ and\
  \bibinfo {author} {\bibfnamefont {M.}~\bibnamefont {den Nijs}},\ }\bibfield
  {title} {\bibinfo {title} {Quantized hall conductance in a two-dimensional
  periodic potential},\ }\href {https://doi.org/10.1103/PhysRevLett.49.405}
  {\bibfield  {journal} {\bibinfo  {journal} {Phys. Rev. Lett.}\ }\textbf
  {\bibinfo {volume} {49}},\ \bibinfo {pages} {405} (\bibinfo {year}
  {1982})}\BibitemShut {NoStop}%
\bibitem [{\citenamefont {{Berry}}(1984)}]{1984Berry}%
  \BibitemOpen
  \bibfield  {author} {\bibinfo {author} {\bibfnamefont {M.~V.}\ \bibnamefont
  {{Berry}}},\ }\bibfield  {title} {\bibinfo {title} {{Quantal Phase Factors
  Accompanying Adiabatic Changes}},\ }\href
  {https://doi.org/10.1098/rspa.1984.0023} {\bibfield  {journal} {\bibinfo
  {journal} {Proceedings of the Royal Society of London Series A}\ }\textbf
  {\bibinfo {volume} {392}},\ \bibinfo {pages} {45} (\bibinfo {year}
  {1984})}\BibitemShut {NoStop}%
\bibitem [{\citenamefont {{Bellissard}}\ \emph {et~al.}(1994)\citenamefont
  {{Bellissard}}, \citenamefont {{van Elst}},\ and\ \citenamefont
  {{Schulz-Baldes}}}]{1994JMP355373B}%
  \BibitemOpen
  \bibfield  {author} {\bibinfo {author} {\bibfnamefont {J.}~\bibnamefont
  {{Bellissard}}}, \bibinfo {author} {\bibfnamefont {A.}~\bibnamefont {{van
  Elst}}},\ and\ \bibinfo {author} {\bibfnamefont {H.}~\bibnamefont
  {{Schulz-Baldes}}},\ }\bibfield  {title} {\bibinfo {title} {{The
  noncommutative geometry of the quantum Hall effect}},\ }\href
  {https://doi.org/10.1063/1.530758} {\bibfield  {journal} {\bibinfo  {journal}
  {Journal of Mathematical Physics}\ }\textbf {\bibinfo {volume} {35}},\
  \bibinfo {pages} {5373} (\bibinfo {year} {1994})},\ \Eprint
  {https://arxiv.org/abs/cond-mat/9411052} {arXiv:cond-mat/9411052 [cond-mat]}
  \BibitemShut {NoStop}%
\bibitem [{\citenamefont {Hasan}\ and\ \citenamefont {Kane}(2010)}]{TIRMP}%
  \BibitemOpen
  \bibfield  {author} {\bibinfo {author} {\bibfnamefont {M.~Z.}\ \bibnamefont
  {Hasan}}\ and\ \bibinfo {author} {\bibfnamefont {C.~L.}\ \bibnamefont
  {Kane}},\ }\bibfield  {title} {\bibinfo {title} {Colloquium: Topological
  insulators},\ }\href {https://doi.org/10.1103/RevModPhys.82.3045} {\bibfield
  {journal} {\bibinfo  {journal} {Rev. Mod. Phys.}\ }\textbf {\bibinfo {volume}
  {82}},\ \bibinfo {pages} {3045} (\bibinfo {year} {2010})}\BibitemShut
  {NoStop}%
\bibitem [{\citenamefont {Qi}\ and\ \citenamefont {Zhang}(2011)}]{TISCRMP}%
  \BibitemOpen
  \bibfield  {author} {\bibinfo {author} {\bibfnamefont {X.-L.}\ \bibnamefont
  {Qi}}\ and\ \bibinfo {author} {\bibfnamefont {S.-C.}\ \bibnamefont {Zhang}},\
  }\bibfield  {title} {\bibinfo {title} {Topological insulators and
  superconductors},\ }\href {https://doi.org/10.1103/RevModPhys.83.1057}
  {\bibfield  {journal} {\bibinfo  {journal} {Rev. Mod. Phys.}\ }\textbf
  {\bibinfo {volume} {83}},\ \bibinfo {pages} {1057} (\bibinfo {year}
  {2011})}\BibitemShut {NoStop}%
\bibitem [{\citenamefont {{Bouhon}}\ \emph {et~al.}(2023)\citenamefont
  {{Bouhon}}, \citenamefont {{Timmel}},\ and\ \citenamefont
  {{Slager}}}]{2023arXiv230302180B}%
  \BibitemOpen
  \bibfield  {author} {\bibinfo {author} {\bibfnamefont {A.}~\bibnamefont
  {{Bouhon}}}, \bibinfo {author} {\bibfnamefont {A.}~\bibnamefont {{Timmel}}},\
  and\ \bibinfo {author} {\bibfnamefont {R.-J.}\ \bibnamefont {{Slager}}},\
  }\bibfield  {title} {\bibinfo {title} {{Quantum geometry beyond projective
  single bands}},\ }\href {https://doi.org/10.48550/arXiv.2303.02180}
  {\bibfield  {journal} {\bibinfo  {journal} {arXiv e-prints}\ ,\ \bibinfo
  {eid} {arXiv:2303.02180}} (\bibinfo {year} {2023})},\ \Eprint
  {https://arxiv.org/abs/2303.02180} {arXiv:2303.02180 [cond-mat.mes-hall]}
  \BibitemShut {NoStop}%
\bibitem [{\citenamefont {Anandan}\ and\ \citenamefont
  {Aharonov}(1990)}]{1990AAgeometry}%
  \BibitemOpen
  \bibfield  {author} {\bibinfo {author} {\bibfnamefont {J.}~\bibnamefont
  {Anandan}}\ and\ \bibinfo {author} {\bibfnamefont {Y.}~\bibnamefont
  {Aharonov}},\ }\bibfield  {title} {\bibinfo {title} {Geometry of quantum
  evolution},\ }\href {https://doi.org/10.1103/PhysRevLett.65.1697} {\bibfield
  {journal} {\bibinfo  {journal} {Phys. Rev. Lett.}\ }\textbf {\bibinfo
  {volume} {65}},\ \bibinfo {pages} {1697} (\bibinfo {year}
  {1990})}\BibitemShut {NoStop}%
\bibitem [{\citenamefont {{Resta}}(2011)}]{2011EPJB79121R}%
  \BibitemOpen
  \bibfield  {author} {\bibinfo {author} {\bibfnamefont {R.}~\bibnamefont
  {{Resta}}},\ }\bibfield  {title} {\bibinfo {title} {{The insulating state of
  matter: a geometrical theory}},\ }\href
  {https://doi.org/10.1140/epjb/e2010-10874-4} {\bibfield  {journal} {\bibinfo
  {journal} {European Physical Journal B}\ }\textbf {\bibinfo {volume} {79}},\
  \bibinfo {pages} {121} (\bibinfo {year} {2011})},\ \Eprint
  {https://arxiv.org/abs/1012.5776} {arXiv:1012.5776 [cond-mat.mtrl-sci]}
  \BibitemShut {NoStop}%
\bibitem [{\citenamefont {Marzari}\ and\ \citenamefont
  {Vanderbilt}(1997)}]{PhysRevB.56.12847}%
  \BibitemOpen
  \bibfield  {author} {\bibinfo {author} {\bibfnamefont {N.}~\bibnamefont
  {Marzari}}\ and\ \bibinfo {author} {\bibfnamefont {D.}~\bibnamefont
  {Vanderbilt}},\ }\bibfield  {title} {\bibinfo {title} {Maximally localized
  generalized wannier functions for composite energy bands},\ }\href
  {https://doi.org/10.1103/PhysRevB.56.12847} {\bibfield  {journal} {\bibinfo
  {journal} {Phys. Rev. B}\ }\textbf {\bibinfo {volume} {56}},\ \bibinfo
  {pages} {12847} (\bibinfo {year} {1997})}\BibitemShut {NoStop}%
\bibitem [{\citenamefont {Haldane}(2011)}]{PhysRevLett.107.116801}%
  \BibitemOpen
  \bibfield  {author} {\bibinfo {author} {\bibfnamefont {F.~D.~M.}\
  \bibnamefont {Haldane}},\ }\bibfield  {title} {\bibinfo {title} {Geometrical
  description of the fractional quantum hall effect},\ }\href
  {https://doi.org/10.1103/PhysRevLett.107.116801} {\bibfield  {journal}
  {\bibinfo  {journal} {Phys. Rev. Lett.}\ }\textbf {\bibinfo {volume} {107}},\
  \bibinfo {pages} {116801} (\bibinfo {year} {2011})}\BibitemShut {NoStop}%
\bibitem [{\citenamefont {Girvin}\ \emph {et~al.}(1986)\citenamefont {Girvin},
  \citenamefont {MacDonald},\ and\ \citenamefont
  {Platzman}}]{PhysRevB.33.2481}%
  \BibitemOpen
  \bibfield  {author} {\bibinfo {author} {\bibfnamefont {S.~M.}\ \bibnamefont
  {Girvin}}, \bibinfo {author} {\bibfnamefont {A.~H.}\ \bibnamefont
  {MacDonald}},\ and\ \bibinfo {author} {\bibfnamefont {P.~M.}\ \bibnamefont
  {Platzman}},\ }\bibfield  {title} {\bibinfo {title} {Magneto-roton theory of
  collective excitations in the fractional quantum hall effect},\ }\href
  {https://doi.org/10.1103/PhysRevB.33.2481} {\bibfield  {journal} {\bibinfo
  {journal} {Phys. Rev. B}\ }\textbf {\bibinfo {volume} {33}},\ \bibinfo
  {pages} {2481} (\bibinfo {year} {1986})}\BibitemShut {NoStop}%
\bibitem [{\citenamefont {{Fogler}}(2002)}]{2002LNP59598F}%
  \BibitemOpen
  \bibfield  {author} {\bibinfo {author} {\bibfnamefont {M.~M.}\ \bibnamefont
  {{Fogler}}},\ }\bibfield  {title} {\bibinfo {title} {{Stripe and Bubble
  Phases in Quantum Hall Systems}},\ }in\ \href@noop {} {\emph {\bibinfo
  {booktitle} {High Magnetic Fields}}},\ Vol.\ \bibinfo {volume} {595},\
  \bibinfo {editor} {edited by\ \bibinfo {editor} {\bibfnamefont
  {C.}~\bibnamefont {{Berthier}}}, \bibinfo {editor} {\bibfnamefont {L.~P.}\
  \bibnamefont {{L{\'e}vy}}},\ and\ \bibinfo {editor} {\bibfnamefont
  {G.}~\bibnamefont {{Martinez}}}}\ (\bibinfo {year} {2002})\ pp.\ \bibinfo
  {pages} {98--138}\BibitemShut {NoStop}%
\bibitem [{\citenamefont {{Parameswaran}}\ \emph {et~al.}(2013)\citenamefont
  {{Parameswaran}}, \citenamefont {{Roy}},\ and\ \citenamefont
  {{Sondhi}}}]{2013CRPhy14816P}%
  \BibitemOpen
  \bibfield  {author} {\bibinfo {author} {\bibfnamefont {S.~A.}\ \bibnamefont
  {{Parameswaran}}}, \bibinfo {author} {\bibfnamefont {R.}~\bibnamefont
  {{Roy}}},\ and\ \bibinfo {author} {\bibfnamefont {S.~L.}\ \bibnamefont
  {{Sondhi}}},\ }\bibfield  {title} {\bibinfo {title} {{Fractional quantum Hall
  physics in topological flat bands}},\ }\href
  {https://doi.org/10.1016/j.crhy.2013.04.003} {\bibfield  {journal} {\bibinfo
  {journal} {Comptes Rendus Physique}\ }\textbf {\bibinfo {volume} {14}},\
  \bibinfo {pages} {816} (\bibinfo {year} {2013})},\ \Eprint
  {https://arxiv.org/abs/1302.6606} {arXiv:1302.6606 [cond-mat.str-el]}
  \BibitemShut {NoStop}%
\bibitem [{\citenamefont {Dobard\ifmmode \check{z}\else
  \v{z}\fi{}i\ifmmode~\acute{c}\else \'{c}\fi{}}\ \emph
  {et~al.}(2013)\citenamefont {Dobard\ifmmode \check{z}\else
  \v{z}\fi{}i\ifmmode~\acute{c}\else \'{c}\fi{}}, \citenamefont
  {Milovanovi\ifmmode~\acute{c}\else \'{c}\fi{}},\ and\ \citenamefont
  {Regnault}}]{PhysRevB.88.115117}%
  \BibitemOpen
  \bibfield  {author} {\bibinfo {author} {\bibfnamefont {E.}~\bibnamefont
  {Dobard\ifmmode \check{z}\else \v{z}\fi{}i\ifmmode~\acute{c}\else
  \'{c}\fi{}}}, \bibinfo {author} {\bibfnamefont {M.~V.}\ \bibnamefont
  {Milovanovi\ifmmode~\acute{c}\else \'{c}\fi{}}},\ and\ \bibinfo {author}
  {\bibfnamefont {N.}~\bibnamefont {Regnault}},\ }\bibfield  {title} {\bibinfo
  {title} {Geometrical description of fractional chern insulators based on
  static structure factor calculations},\ }\href
  {https://doi.org/10.1103/PhysRevB.88.115117} {\bibfield  {journal} {\bibinfo
  {journal} {Phys. Rev. B}\ }\textbf {\bibinfo {volume} {88}},\ \bibinfo
  {pages} {115117} (\bibinfo {year} {2013})}\BibitemShut {NoStop}%
\bibitem [{\citenamefont {Roy}(2014)}]{PhysRevB.90.165139}%
  \BibitemOpen
  \bibfield  {author} {\bibinfo {author} {\bibfnamefont {R.}~\bibnamefont
  {Roy}},\ }\bibfield  {title} {\bibinfo {title} {Band geometry of fractional
  topological insulators},\ }\href {https://doi.org/10.1103/PhysRevB.90.165139}
  {\bibfield  {journal} {\bibinfo  {journal} {Phys. Rev. B}\ }\textbf {\bibinfo
  {volume} {90}},\ \bibinfo {pages} {165139} (\bibinfo {year}
  {2014})}\BibitemShut {NoStop}%
\bibitem [{\citenamefont {{Wang}}\ \emph {et~al.}(2022)\citenamefont {{Wang}},
  \citenamefont {{Klevtsov}},\ and\ \citenamefont
  {{Liu}}}]{2022arXiv221013487W}%
  \BibitemOpen
  \bibfield  {author} {\bibinfo {author} {\bibfnamefont {J.}~\bibnamefont
  {{Wang}}}, \bibinfo {author} {\bibfnamefont {S.}~\bibnamefont {{Klevtsov}}},\
  and\ \bibinfo {author} {\bibfnamefont {Z.}~\bibnamefont {{Liu}}},\ }\bibfield
   {title} {\bibinfo {title} {{Origin of Model Fractional Chern Insulators in
  All Topological Ideal Flatbands: Explicit Color-entangled Wavefunction and
  Exact Density Algebra}},\ }\href {https://doi.org/10.48550/arXiv.2210.13487}
  {\bibfield  {journal} {\bibinfo  {journal} {arXiv e-prints}\ ,\ \bibinfo
  {eid} {arXiv:2210.13487}} (\bibinfo {year} {2022})},\ \Eprint
  {https://arxiv.org/abs/2210.13487} {arXiv:2210.13487 [cond-mat.mes-hall]}
  \BibitemShut {NoStop}%
\bibitem [{\citenamefont {Neupert}\ \emph {et~al.}(2013)\citenamefont
  {Neupert}, \citenamefont {Chamon},\ and\ \citenamefont
  {Mudry}}]{PhysRevB.87.245103}%
  \BibitemOpen
  \bibfield  {author} {\bibinfo {author} {\bibfnamefont {T.}~\bibnamefont
  {Neupert}}, \bibinfo {author} {\bibfnamefont {C.}~\bibnamefont {Chamon}},\
  and\ \bibinfo {author} {\bibfnamefont {C.}~\bibnamefont {Mudry}},\ }\bibfield
   {title} {\bibinfo {title} {Measuring the quantum geometry of bloch bands
  with current noise},\ }\href {https://doi.org/10.1103/PhysRevB.87.245103}
  {\bibfield  {journal} {\bibinfo  {journal} {Phys. Rev. B}\ }\textbf {\bibinfo
  {volume} {87}},\ \bibinfo {pages} {245103} (\bibinfo {year}
  {2013})}\BibitemShut {NoStop}%
\bibitem [{\citenamefont {Srivastava}\ and\ \citenamefont
  {Imamo\ifmmode~\breve{g}\else \u{g}\fi{}lu}(2015)}]{PhysRevLett.115.166802}%
  \BibitemOpen
  \bibfield  {author} {\bibinfo {author} {\bibfnamefont {A.}~\bibnamefont
  {Srivastava}}\ and\ \bibinfo {author} {\bibfnamefont {A.~m.~c.}\ \bibnamefont
  {Imamo\ifmmode~\breve{g}\else \u{g}\fi{}lu}},\ }\bibfield  {title} {\bibinfo
  {title} {Signatures of bloch-band geometry on excitons: Nonhydrogenic spectra
  in transition-metal dichalcogenides},\ }\href
  {https://doi.org/10.1103/PhysRevLett.115.166802} {\bibfield  {journal}
  {\bibinfo  {journal} {Phys. Rev. Lett.}\ }\textbf {\bibinfo {volume} {115}},\
  \bibinfo {pages} {166802} (\bibinfo {year} {2015})}\BibitemShut {NoStop}%
\bibitem [{\citenamefont {Gao}\ \emph {et~al.}(2014)\citenamefont {Gao},
  \citenamefont {Yang},\ and\ \citenamefont {Niu}}]{PhysRevLett112166601}%
  \BibitemOpen
  \bibfield  {author} {\bibinfo {author} {\bibfnamefont {Y.}~\bibnamefont
  {Gao}}, \bibinfo {author} {\bibfnamefont {S.~A.}\ \bibnamefont {Yang}},\ and\
  \bibinfo {author} {\bibfnamefont {Q.}~\bibnamefont {Niu}},\ }\bibfield
  {title} {\bibinfo {title} {Field induced positional shift of bloch electrons
  and its dynamical implications},\ }\href
  {https://doi.org/10.1103/PhysRevLett.112.166601} {\bibfield  {journal}
  {\bibinfo  {journal} {Phys. Rev. Lett.}\ }\textbf {\bibinfo {volume} {112}},\
  \bibinfo {pages} {166601} (\bibinfo {year} {2014})}\BibitemShut {NoStop}%
\bibitem [{\citenamefont {Pi\'echon}\ \emph {et~al.}(2016)\citenamefont
  {Pi\'echon}, \citenamefont {Raoux}, \citenamefont {Fuchs},\ and\
  \citenamefont {Montambaux}}]{PhysRevB.94.134423}%
  \BibitemOpen
  \bibfield  {author} {\bibinfo {author} {\bibfnamefont {F.}~\bibnamefont
  {Pi\'echon}}, \bibinfo {author} {\bibfnamefont {A.}~\bibnamefont {Raoux}},
  \bibinfo {author} {\bibfnamefont {J.-N.}\ \bibnamefont {Fuchs}},\ and\
  \bibinfo {author} {\bibfnamefont {G.}~\bibnamefont {Montambaux}},\ }\bibfield
   {title} {\bibinfo {title} {Geometric orbital susceptibility: Quantum metric
  without berry curvature},\ }\href
  {https://doi.org/10.1103/PhysRevB.94.134423} {\bibfield  {journal} {\bibinfo
  {journal} {Phys. Rev. B}\ }\textbf {\bibinfo {volume} {94}},\ \bibinfo
  {pages} {134423} (\bibinfo {year} {2016})}\BibitemShut {NoStop}%
\bibitem [{\citenamefont {Chen}\ and\ \citenamefont
  {Huang}(2021)}]{PhysRevResearch.3.L042018}%
  \BibitemOpen
  \bibfield  {author} {\bibinfo {author} {\bibfnamefont {W.}~\bibnamefont
  {Chen}}\ and\ \bibinfo {author} {\bibfnamefont {W.}~\bibnamefont {Huang}},\
  }\bibfield  {title} {\bibinfo {title} {Quantum-geometry-induced intrinsic
  optical anomaly in multiorbital superconductors},\ }\href
  {https://doi.org/10.1103/PhysRevResearch.3.L042018} {\bibfield  {journal}
  {\bibinfo  {journal} {Phys. Rev. Research}\ }\textbf {\bibinfo {volume}
  {3}},\ \bibinfo {pages} {L042018} (\bibinfo {year} {2021})}\BibitemShut
  {NoStop}%
\bibitem [{\citenamefont {Ahn}\ and\ \citenamefont
  {Nagaosa}(2021)}]{PhysRevB.104.L100501}%
  \BibitemOpen
  \bibfield  {author} {\bibinfo {author} {\bibfnamefont {J.}~\bibnamefont
  {Ahn}}\ and\ \bibinfo {author} {\bibfnamefont {N.}~\bibnamefont {Nagaosa}},\
  }\bibfield  {title} {\bibinfo {title} {Superconductivity-induced spectral
  weight transfer due to quantum geometry},\ }\href
  {https://doi.org/10.1103/PhysRevB.104.L100501} {\bibfield  {journal}
  {\bibinfo  {journal} {Phys. Rev. B}\ }\textbf {\bibinfo {volume} {104}},\
  \bibinfo {pages} {L100501} (\bibinfo {year} {2021})}\BibitemShut {NoStop}%
\bibitem [{\citenamefont {Kozii}\ \emph {et~al.}(2021)\citenamefont {Kozii},
  \citenamefont {Avdoshkin}, \citenamefont {Zhong},\ and\ \citenamefont
  {Moore}}]{PhysRevLett.126.156602}%
  \BibitemOpen
  \bibfield  {author} {\bibinfo {author} {\bibfnamefont {V.}~\bibnamefont
  {Kozii}}, \bibinfo {author} {\bibfnamefont {A.}~\bibnamefont {Avdoshkin}},
  \bibinfo {author} {\bibfnamefont {S.}~\bibnamefont {Zhong}},\ and\ \bibinfo
  {author} {\bibfnamefont {J.~E.}\ \bibnamefont {Moore}},\ }\bibfield  {title}
  {\bibinfo {title} {Intrinsic anomalous hall conductivity in a nonuniform
  electric field},\ }\href {https://doi.org/10.1103/PhysRevLett.126.156602}
  {\bibfield  {journal} {\bibinfo  {journal} {Phys. Rev. Lett.}\ }\textbf
  {\bibinfo {volume} {126}},\ \bibinfo {pages} {156602} (\bibinfo {year}
  {2021})}\BibitemShut {NoStop}%
\bibitem [{\citenamefont {{Ahn}}\ \emph {et~al.}(2021)\citenamefont {{Ahn}},
  \citenamefont {{Guo}}, \citenamefont {{Nagaosa}},\ and\ \citenamefont
  {{Vishwanath}}}]{2021NatPh..18..290A}%
  \BibitemOpen
  \bibfield  {author} {\bibinfo {author} {\bibfnamefont {J.}~\bibnamefont
  {{Ahn}}}, \bibinfo {author} {\bibfnamefont {G.-Y.}\ \bibnamefont {{Guo}}},
  \bibinfo {author} {\bibfnamefont {N.}~\bibnamefont {{Nagaosa}}},\ and\
  \bibinfo {author} {\bibfnamefont {A.}~\bibnamefont {{Vishwanath}}},\
  }\bibfield  {title} {\bibinfo {title} {{Riemannian geometry of resonant
  optical responses}},\ }\href {https://doi.org/10.1038/s41567-021-01465-z}
  {\bibfield  {journal} {\bibinfo  {journal} {Nature Physics}\ }\textbf
  {\bibinfo {volume} {18}},\ \bibinfo {pages} {290} (\bibinfo {year} {2021})},\
  \Eprint {https://arxiv.org/abs/2103.01241} {arXiv:2103.01241
  [cond-mat.mes-hall]} \BibitemShut {NoStop}%
\bibitem [{\citenamefont {{Mitscherling}}\ and\ \citenamefont
  {{Holder}}(2022)}]{2022PhRvB.105h5154M}%
  \BibitemOpen
  \bibfield  {author} {\bibinfo {author} {\bibfnamefont {J.}~\bibnamefont
  {{Mitscherling}}}\ and\ \bibinfo {author} {\bibfnamefont {T.}~\bibnamefont
  {{Holder}}},\ }\bibfield  {title} {\bibinfo {title} {{Bound on resistivity in
  flat-band materials due to the quantum metric}},\ }\href
  {https://doi.org/10.1103/PhysRevB.105.085154} {\bibfield  {journal} {\bibinfo
   {journal} {\prb}\ }\textbf {\bibinfo {volume} {105}},\ \bibinfo {eid}
  {085154} (\bibinfo {year} {2022})},\ \Eprint
  {https://arxiv.org/abs/2110.14658} {arXiv:2110.14658 [cond-mat.mes-hall]}
  \BibitemShut {NoStop}%
\bibitem [{\citenamefont {{Peotta}}\ and\ \citenamefont
  {{T{\"o}rm{\"a}}}(2015)}]{2015NatCo68944P}%
  \BibitemOpen
  \bibfield  {author} {\bibinfo {author} {\bibfnamefont {S.}~\bibnamefont
  {{Peotta}}}\ and\ \bibinfo {author} {\bibfnamefont {P.}~\bibnamefont
  {{T{\"o}rm{\"a}}}},\ }\bibfield  {title} {\bibinfo {title} {{Superfluidity in
  topologically nontrivial flat bands}},\ }\href
  {https://doi.org/10.1038/ncomms9944} {\bibfield  {journal} {\bibinfo
  {journal} {Nature Communications}\ }\textbf {\bibinfo {volume} {6}},\
  \bibinfo {eid} {8944} (\bibinfo {year} {2015})},\ \Eprint
  {https://arxiv.org/abs/1506.02815} {arXiv:1506.02815 [cond-mat.supr-con]}
  \BibitemShut {NoStop}%
\bibitem [{\citenamefont {Liang}\ \emph {et~al.}(2017)\citenamefont {Liang},
  \citenamefont {Vanhala}, \citenamefont {Peotta}, \citenamefont {Siro},
  \citenamefont {Harju},\ and\ \citenamefont {T\"orm\"a}}]{PhysRevB.95.024515}%
  \BibitemOpen
  \bibfield  {author} {\bibinfo {author} {\bibfnamefont {L.}~\bibnamefont
  {Liang}}, \bibinfo {author} {\bibfnamefont {T.~I.}\ \bibnamefont {Vanhala}},
  \bibinfo {author} {\bibfnamefont {S.}~\bibnamefont {Peotta}}, \bibinfo
  {author} {\bibfnamefont {T.}~\bibnamefont {Siro}}, \bibinfo {author}
  {\bibfnamefont {A.}~\bibnamefont {Harju}},\ and\ \bibinfo {author}
  {\bibfnamefont {P.}~\bibnamefont {T\"orm\"a}},\ }\bibfield  {title} {\bibinfo
  {title} {Band geometry, berry curvature, and superfluid weight},\ }\href
  {https://doi.org/10.1103/PhysRevB.95.024515} {\bibfield  {journal} {\bibinfo
  {journal} {Phys. Rev. B}\ }\textbf {\bibinfo {volume} {95}},\ \bibinfo
  {pages} {024515} (\bibinfo {year} {2017})}\BibitemShut {NoStop}%
\bibitem [{\citenamefont {Iskin}(2018)}]{PhysRevA.97.033625}%
  \BibitemOpen
  \bibfield  {author} {\bibinfo {author} {\bibfnamefont {M.}~\bibnamefont
  {Iskin}},\ }\bibfield  {title} {\bibinfo {title} {Quantum-metric contribution
  to the pair mass in spin-orbit-coupled fermi superfluids},\ }\href
  {https://doi.org/10.1103/PhysRevA.97.033625} {\bibfield  {journal} {\bibinfo
  {journal} {Phys. Rev. A}\ }\textbf {\bibinfo {volume} {97}},\ \bibinfo
  {pages} {033625} (\bibinfo {year} {2018})}\BibitemShut {NoStop}%
\bibitem [{\citenamefont {T\"orm\"a}\ \emph {et~al.}(2018)\citenamefont
  {T\"orm\"a}, \citenamefont {Liang},\ and\ \citenamefont
  {Peotta}}]{PhysRevB.98.220511}%
  \BibitemOpen
  \bibfield  {author} {\bibinfo {author} {\bibfnamefont {P.}~\bibnamefont
  {T\"orm\"a}}, \bibinfo {author} {\bibfnamefont {L.}~\bibnamefont {Liang}},\
  and\ \bibinfo {author} {\bibfnamefont {S.}~\bibnamefont {Peotta}},\
  }\bibfield  {title} {\bibinfo {title} {Quantum metric and effective mass of a
  two-body bound state in a flat band},\ }\href
  {https://doi.org/10.1103/PhysRevB.98.220511} {\bibfield  {journal} {\bibinfo
  {journal} {Phys. Rev. B}\ }\textbf {\bibinfo {volume} {98}},\ \bibinfo
  {pages} {220511} (\bibinfo {year} {2018})}\BibitemShut {NoStop}%
\bibitem [{\citenamefont {{Jip Park}}\ \emph {et~al.}(2020)\citenamefont {{Jip
  Park}}, \citenamefont {{Kim}},\ and\ \citenamefont
  {{Lee}}}]{2020arXiv200716205J}%
  \BibitemOpen
  \bibfield  {author} {\bibinfo {author} {\bibfnamefont {M.}~\bibnamefont {{Jip
  Park}}}, \bibinfo {author} {\bibfnamefont {Y.~B.}\ \bibnamefont {{Kim}}},\
  and\ \bibinfo {author} {\bibfnamefont {S.}~\bibnamefont {{Lee}}},\ }\bibfield
   {title} {\bibinfo {title} {{Geometric Superconductivity in 3D Hofstadter
  Butterfly}},\ }\href@noop {} {\bibfield  {journal} {\bibinfo  {journal}
  {arXiv e-prints}\ ,\ \bibinfo {eid} {arXiv:2007.16205}} (\bibinfo {year}
  {2020})},\ \Eprint {https://arxiv.org/abs/2007.16205} {arXiv:2007.16205
  [cond-mat.supr-con]} \BibitemShut {NoStop}%
\bibitem [{\citenamefont {Hofmann}\ \emph {et~al.}(2020)\citenamefont
  {Hofmann}, \citenamefont {Berg},\ and\ \citenamefont
  {Chowdhury}}]{PhysRevB.102.201112}%
  \BibitemOpen
  \bibfield  {author} {\bibinfo {author} {\bibfnamefont {J.~S.}\ \bibnamefont
  {Hofmann}}, \bibinfo {author} {\bibfnamefont {E.}~\bibnamefont {Berg}},\ and\
  \bibinfo {author} {\bibfnamefont {D.}~\bibnamefont {Chowdhury}},\ }\bibfield
  {title} {\bibinfo {title} {Superconductivity, pseudogap, and phase separation
  in topological flat bands},\ }\href
  {https://doi.org/10.1103/PhysRevB.102.201112} {\bibfield  {journal} {\bibinfo
   {journal} {Phys. Rev. B}\ }\textbf {\bibinfo {volume} {102}},\ \bibinfo
  {pages} {201112} (\bibinfo {year} {2020})}\BibitemShut {NoStop}%
\bibitem [{\citenamefont {Iskin}(2020)}]{PhysRevA.101.053631}%
  \BibitemOpen
  \bibfield  {author} {\bibinfo {author} {\bibfnamefont {M.}~\bibnamefont
  {Iskin}},\ }\bibfield  {title} {\bibinfo {title} {Collective excitations of a
  bcs superfluid in the presence of two sublattices},\ }\href
  {https://doi.org/10.1103/PhysRevA.101.053631} {\bibfield  {journal} {\bibinfo
   {journal} {Phys. Rev. A}\ }\textbf {\bibinfo {volume} {101}},\ \bibinfo
  {pages} {053631} (\bibinfo {year} {2020})}\BibitemShut {NoStop}%
\bibitem [{\citenamefont {Julku}\ \emph {et~al.}(2016)\citenamefont {Julku},
  \citenamefont {Peotta}, \citenamefont {Vanhala}, \citenamefont {Kim},\ and\
  \citenamefont {T\"orm\"a}}]{PhysRevLett.117.045303}%
  \BibitemOpen
  \bibfield  {author} {\bibinfo {author} {\bibfnamefont {A.}~\bibnamefont
  {Julku}}, \bibinfo {author} {\bibfnamefont {S.}~\bibnamefont {Peotta}},
  \bibinfo {author} {\bibfnamefont {T.~I.}\ \bibnamefont {Vanhala}}, \bibinfo
  {author} {\bibfnamefont {D.-H.}\ \bibnamefont {Kim}},\ and\ \bibinfo {author}
  {\bibfnamefont {P.}~\bibnamefont {T\"orm\"a}},\ }\bibfield  {title} {\bibinfo
  {title} {Geometric origin of superfluidity in the lieb-lattice flat band},\
  }\href {https://doi.org/10.1103/PhysRevLett.117.045303} {\bibfield  {journal}
  {\bibinfo  {journal} {Phys. Rev. Lett.}\ }\textbf {\bibinfo {volume} {117}},\
  \bibinfo {pages} {045303} (\bibinfo {year} {2016})}\BibitemShut {NoStop}%
\bibitem [{\citenamefont {Julku}\ \emph {et~al.}(2021)\citenamefont {Julku},
  \citenamefont {Bruun},\ and\ \citenamefont
  {T\"orm\"a}}]{PhysRevLett.127.170404}%
  \BibitemOpen
  \bibfield  {author} {\bibinfo {author} {\bibfnamefont {A.}~\bibnamefont
  {Julku}}, \bibinfo {author} {\bibfnamefont {G.~M.}\ \bibnamefont {Bruun}},\
  and\ \bibinfo {author} {\bibfnamefont {P.}~\bibnamefont {T\"orm\"a}},\
  }\bibfield  {title} {\bibinfo {title} {Quantum geometry and flat band
  bose-einstein condensation},\ }\href
  {https://doi.org/10.1103/PhysRevLett.127.170404} {\bibfield  {journal}
  {\bibinfo  {journal} {Phys. Rev. Lett.}\ }\textbf {\bibinfo {volume} {127}},\
  \bibinfo {pages} {170404} (\bibinfo {year} {2021})}\BibitemShut {NoStop}%
\bibitem [{\citenamefont {Herzog-Arbeitman}\ \emph {et~al.}(2022)\citenamefont
  {Herzog-Arbeitman}, \citenamefont {Peri}, \citenamefont {Schindler},
  \citenamefont {Huber},\ and\ \citenamefont
  {Bernevig}}]{PhysRevLett.128.087002}%
  \BibitemOpen
  \bibfield  {author} {\bibinfo {author} {\bibfnamefont {J.}~\bibnamefont
  {Herzog-Arbeitman}}, \bibinfo {author} {\bibfnamefont {V.}~\bibnamefont
  {Peri}}, \bibinfo {author} {\bibfnamefont {F.}~\bibnamefont {Schindler}},
  \bibinfo {author} {\bibfnamefont {S.~D.}\ \bibnamefont {Huber}},\ and\
  \bibinfo {author} {\bibfnamefont {B.~A.}\ \bibnamefont {Bernevig}},\
  }\bibfield  {title} {\bibinfo {title} {Superfluid weight bounds from symmetry
  and quantum geometry in flat bands},\ }\href
  {https://doi.org/10.1103/PhysRevLett.128.087002} {\bibfield  {journal}
  {\bibinfo  {journal} {Phys. Rev. Lett.}\ }\textbf {\bibinfo {volume} {128}},\
  \bibinfo {pages} {087002} (\bibinfo {year} {2022})}\BibitemShut {NoStop}%
\bibitem [{\citenamefont {Huhtinen}\ \emph {et~al.}(2022)\citenamefont
  {Huhtinen}, \citenamefont {Herzog-Arbeitman}, \citenamefont {Chew},
  \citenamefont {Bernevig},\ and\ \citenamefont
  {T\"orm\"a}}]{PhysRevB.106.014518}%
  \BibitemOpen
  \bibfield  {author} {\bibinfo {author} {\bibfnamefont {K.-E.}\ \bibnamefont
  {Huhtinen}}, \bibinfo {author} {\bibfnamefont {J.}~\bibnamefont
  {Herzog-Arbeitman}}, \bibinfo {author} {\bibfnamefont {A.}~\bibnamefont
  {Chew}}, \bibinfo {author} {\bibfnamefont {B.~A.}\ \bibnamefont {Bernevig}},\
  and\ \bibinfo {author} {\bibfnamefont {P.}~\bibnamefont {T\"orm\"a}},\
  }\bibfield  {title} {\bibinfo {title} {Revisiting flat band
  superconductivity: Dependence on minimal quantum metric and band touchings},\
  }\href {https://doi.org/10.1103/PhysRevB.106.014518} {\bibfield  {journal}
  {\bibinfo  {journal} {Phys. Rev. B}\ }\textbf {\bibinfo {volume} {106}},\
  \bibinfo {pages} {014518} (\bibinfo {year} {2022})}\BibitemShut {NoStop}%
\bibitem [{\citenamefont {Hu}\ \emph {et~al.}(2022)\citenamefont {Hu},
  \citenamefont {Hyart}, \citenamefont {Pikulin},\ and\ \citenamefont
  {Rossi}}]{PhysRevB.105.L140506}%
  \BibitemOpen
  \bibfield  {author} {\bibinfo {author} {\bibfnamefont {X.}~\bibnamefont
  {Hu}}, \bibinfo {author} {\bibfnamefont {T.}~\bibnamefont {Hyart}}, \bibinfo
  {author} {\bibfnamefont {D.~I.}\ \bibnamefont {Pikulin}},\ and\ \bibinfo
  {author} {\bibfnamefont {E.}~\bibnamefont {Rossi}},\ }\bibfield  {title}
  {\bibinfo {title} {Quantum-metric-enabled exciton condensate in double
  twisted bilayer graphene},\ }\href
  {https://doi.org/10.1103/PhysRevB.105.L140506} {\bibfield  {journal}
  {\bibinfo  {journal} {Phys. Rev. B}\ }\textbf {\bibinfo {volume} {105}},\
  \bibinfo {pages} {L140506} (\bibinfo {year} {2022})}\BibitemShut {NoStop}%
\bibitem [{\citenamefont {Chan}\ \emph {et~al.}(2022)\citenamefont {Chan},
  \citenamefont {Gr\'emaud},\ and\ \citenamefont
  {Batrouni}}]{PhysRevB.106.104514}%
  \BibitemOpen
  \bibfield  {author} {\bibinfo {author} {\bibfnamefont {S.~M.}\ \bibnamefont
  {Chan}}, \bibinfo {author} {\bibfnamefont {B.}~\bibnamefont {Gr\'emaud}},\
  and\ \bibinfo {author} {\bibfnamefont {G.~G.}\ \bibnamefont {Batrouni}},\
  }\bibfield  {title} {\bibinfo {title} {Designer flat bands: Topology and
  enhancement of superconductivity},\ }\href
  {https://doi.org/10.1103/PhysRevB.106.104514} {\bibfield  {journal} {\bibinfo
   {journal} {Phys. Rev. B}\ }\textbf {\bibinfo {volume} {106}},\ \bibinfo
  {pages} {104514} (\bibinfo {year} {2022})}\BibitemShut {NoStop}%
\bibitem [{\citenamefont {{Herzog-Arbeitman}}\ \emph
  {et~al.}(2022)\citenamefont {{Herzog-Arbeitman}}, \citenamefont {{Chew}},
  \citenamefont {{Huhtinen}}, \citenamefont {{T{\"o}rm{\"a}}},\ and\
  \citenamefont {{Bernevig}}}]{2022arXiv220900007H}%
  \BibitemOpen
  \bibfield  {author} {\bibinfo {author} {\bibfnamefont {J.}~\bibnamefont
  {{Herzog-Arbeitman}}}, \bibinfo {author} {\bibfnamefont {A.}~\bibnamefont
  {{Chew}}}, \bibinfo {author} {\bibfnamefont {K.-E.}\ \bibnamefont
  {{Huhtinen}}}, \bibinfo {author} {\bibfnamefont {P.}~\bibnamefont
  {{T{\"o}rm{\"a}}}},\ and\ \bibinfo {author} {\bibfnamefont {B.~A.}\
  \bibnamefont {{Bernevig}}},\ }\bibfield  {title} {\bibinfo {title}
  {{Many-Body Superconductivity in Topological Flat Bands}},\ }\href
  {https://doi.org/10.48550/arXiv.2209.00007} {\bibfield  {journal} {\bibinfo
  {journal} {arXiv e-prints}\ ,\ \bibinfo {eid} {arXiv:2209.00007}} (\bibinfo
  {year} {2022})},\ \Eprint {https://arxiv.org/abs/2209.00007}
  {arXiv:2209.00007 [cond-mat.str-el]} \BibitemShut {NoStop}%
\bibitem [{\citenamefont {{Hofmann}}\ \emph {et~al.}(2022)\citenamefont
  {{Hofmann}}, \citenamefont {{Berg}},\ and\ \citenamefont
  {{Chowdhury}}}]{2022arXiv220402994H}%
  \BibitemOpen
  \bibfield  {author} {\bibinfo {author} {\bibfnamefont {J.~S.}\ \bibnamefont
  {{Hofmann}}}, \bibinfo {author} {\bibfnamefont {E.}~\bibnamefont {{Berg}}},\
  and\ \bibinfo {author} {\bibfnamefont {D.}~\bibnamefont {{Chowdhury}}},\
  }\bibfield  {title} {\bibinfo {title} {{Superconductivity, charge density
  wave, and supersolidity in flat bands with tunable quantum metric}},\ }\href
  {https://doi.org/10.48550/arXiv.2204.02994} {\bibfield  {journal} {\bibinfo
  {journal} {arXiv e-prints}\ ,\ \bibinfo {eid} {arXiv:2204.02994}} (\bibinfo
  {year} {2022})},\ \Eprint {https://arxiv.org/abs/2204.02994}
  {arXiv:2204.02994 [cond-mat.str-el]} \BibitemShut {NoStop}%
\bibitem [{\citenamefont {{Jiang}}\ and\ \citenamefont
  {{Barlas}}(2022)}]{2022arXiv221109846J}%
  \BibitemOpen
  \bibfield  {author} {\bibinfo {author} {\bibfnamefont {G.}~\bibnamefont
  {{Jiang}}}\ and\ \bibinfo {author} {\bibfnamefont {Y.}~\bibnamefont
  {{Barlas}}},\ }\bibfield  {title} {\bibinfo {title} {{Pair Density Waves from
  Local Band Geometry}},\ }\href {https://doi.org/10.48550/arXiv.2211.09846}
  {\bibfield  {journal} {\bibinfo  {journal} {arXiv e-prints}\ ,\ \bibinfo
  {eid} {arXiv:2211.09846}} (\bibinfo {year} {2022})},\ \Eprint
  {https://arxiv.org/abs/2211.09846} {arXiv:2211.09846 [cond-mat.supr-con]}
  \BibitemShut {NoStop}%
\bibitem [{\citenamefont {{Thonhauser}}\ and\ \citenamefont
  {{Vanderbilt}}(2006)}]{2006PhRvB..74w5111T}%
  \BibitemOpen
  \bibfield  {author} {\bibinfo {author} {\bibfnamefont {T.}~\bibnamefont
  {{Thonhauser}}}\ and\ \bibinfo {author} {\bibfnamefont {D.}~\bibnamefont
  {{Vanderbilt}}},\ }\bibfield  {title} {\bibinfo {title}
  {{Insulator/Chern-insulator transition in the Haldane model}},\ }\href
  {https://doi.org/10.1103/PhysRevB.74.235111} {\bibfield  {journal} {\bibinfo
  {journal} {\prb}\ }\textbf {\bibinfo {volume} {74}},\ \bibinfo {eid} {235111}
  (\bibinfo {year} {2006})},\ \Eprint {https://arxiv.org/abs/cond-mat/0608527}
  {arXiv:cond-mat/0608527 [cond-mat.mes-hall]} \BibitemShut {NoStop}%
\bibitem [{\citenamefont {Campos~Venuti}\ and\ \citenamefont
  {Zanardi}(2007)}]{PhysRevLett99095701}%
  \BibitemOpen
  \bibfield  {author} {\bibinfo {author} {\bibfnamefont {L.}~\bibnamefont
  {Campos~Venuti}}\ and\ \bibinfo {author} {\bibfnamefont {P.}~\bibnamefont
  {Zanardi}},\ }\bibfield  {title} {\bibinfo {title} {Quantum critical scaling
  of the geometric tensors},\ }\href
  {https://doi.org/10.1103/PhysRevLett.99.095701} {\bibfield  {journal}
  {\bibinfo  {journal} {Phys. Rev. Lett.}\ }\textbf {\bibinfo {volume} {99}},\
  \bibinfo {pages} {095701} (\bibinfo {year} {2007})}\BibitemShut {NoStop}%
\bibitem [{\citenamefont {Zanardi}\ \emph {et~al.}(2007)\citenamefont
  {Zanardi}, \citenamefont {Giorda},\ and\ \citenamefont
  {Cozzini}}]{PhysRevLett99100603}%
  \BibitemOpen
  \bibfield  {author} {\bibinfo {author} {\bibfnamefont {P.}~\bibnamefont
  {Zanardi}}, \bibinfo {author} {\bibfnamefont {P.}~\bibnamefont {Giorda}},\
  and\ \bibinfo {author} {\bibfnamefont {M.}~\bibnamefont {Cozzini}},\
  }\bibfield  {title} {\bibinfo {title} {Information-theoretic differential
  geometry of quantum phase transitions},\ }\href
  {https://doi.org/10.1103/PhysRevLett.99.100603} {\bibfield  {journal}
  {\bibinfo  {journal} {Phys. Rev. Lett.}\ }\textbf {\bibinfo {volume} {99}},\
  \bibinfo {pages} {100603} (\bibinfo {year} {2007})}\BibitemShut {NoStop}%
\bibitem [{\citenamefont {{Verma}}\ \emph {et~al.}(2021)\citenamefont
  {{Verma}}, \citenamefont {{Hazra}},\ and\ \citenamefont
  {{Randeria}}}]{2021PNAS11806744V}%
  \BibitemOpen
  \bibfield  {author} {\bibinfo {author} {\bibfnamefont {N.}~\bibnamefont
  {{Verma}}}, \bibinfo {author} {\bibfnamefont {T.}~\bibnamefont {{Hazra}}},\
  and\ \bibinfo {author} {\bibfnamefont {M.}~\bibnamefont {{Randeria}}},\
  }\bibfield  {title} {\bibinfo {title} {{Optical spectral weight, phase
  stiffness, and T$_{c}$ bounds for trivial and topological flat band
  superconductors}},\ }\href {https://doi.org/10.1073/pnas.2106744118}
  {\bibfield  {journal} {\bibinfo  {journal} {Proceedings of the National
  Academy of Science}\ }\textbf {\bibinfo {volume} {118}},\ \bibinfo {eid}
  {e2106744118} (\bibinfo {year} {2021})},\ \Eprint
  {https://arxiv.org/abs/2103.08540} {arXiv:2103.08540 [cond-mat.supr-con]}
  \BibitemShut {NoStop}%
\bibitem [{\citenamefont {{Mao}}\ and\ \citenamefont
  {{Chowdhury}}(2023)}]{2023PNAS..12017816M}%
  \BibitemOpen
  \bibfield  {author} {\bibinfo {author} {\bibfnamefont {D.}~\bibnamefont
  {{Mao}}}\ and\ \bibinfo {author} {\bibfnamefont {D.}~\bibnamefont
  {{Chowdhury}}},\ }\bibfield  {title} {\bibinfo {title} {{Diamagnetic response
  and phase stiffness for interacting isolated narrow bands}},\ }\href
  {https://doi.org/10.1073/pnas.2217816120} {\bibfield  {journal} {\bibinfo
  {journal} {Proceedings of the National Academy of Science}\ }\textbf
  {\bibinfo {volume} {120}},\ \bibinfo {eid} {e2217816120} (\bibinfo {year}
  {2023})},\ \Eprint {https://arxiv.org/abs/2209.06817} {arXiv:2209.06817
  [cond-mat.str-el]} \BibitemShut {NoStop}%
\bibitem [{\citenamefont {{Cao}}\ \emph
  {et~al.}(2018{\natexlab{a}})\citenamefont {{Cao}}, \citenamefont {{Fatemi}},
  \citenamefont {{Demir}}, \citenamefont {{Fang}}, \citenamefont {{Tomarken}},
  \citenamefont {{Luo}}, \citenamefont {{Sanchez-Yamagishi}}, \citenamefont
  {{Watanabe}}, \citenamefont {{Taniguchi}}, \citenamefont {{Kaxiras}},
  \citenamefont {{Ashoori}},\ and\ \citenamefont
  {{Jarillo-Herrero}}}]{2018Natur.556...80C}%
  \BibitemOpen
  \bibfield  {author} {\bibinfo {author} {\bibfnamefont {Y.}~\bibnamefont
  {{Cao}}}, \bibinfo {author} {\bibfnamefont {V.}~\bibnamefont {{Fatemi}}},
  \bibinfo {author} {\bibfnamefont {A.}~\bibnamefont {{Demir}}}, \bibinfo
  {author} {\bibfnamefont {S.}~\bibnamefont {{Fang}}}, \bibinfo {author}
  {\bibfnamefont {S.~L.}\ \bibnamefont {{Tomarken}}}, \bibinfo {author}
  {\bibfnamefont {J.~Y.}\ \bibnamefont {{Luo}}}, \bibinfo {author}
  {\bibfnamefont {J.~D.}\ \bibnamefont {{Sanchez-Yamagishi}}}, \bibinfo
  {author} {\bibfnamefont {K.}~\bibnamefont {{Watanabe}}}, \bibinfo {author}
  {\bibfnamefont {T.}~\bibnamefont {{Taniguchi}}}, \bibinfo {author}
  {\bibfnamefont {E.}~\bibnamefont {{Kaxiras}}}, \bibinfo {author}
  {\bibfnamefont {R.~C.}\ \bibnamefont {{Ashoori}}},\ and\ \bibinfo {author}
  {\bibfnamefont {P.}~\bibnamefont {{Jarillo-Herrero}}},\ }\bibfield  {title}
  {\bibinfo {title} {{Correlated insulator behaviour at half-filling in
  magic-angle graphene superlattices}},\ }\href
  {https://doi.org/10.1038/nature26154} {\bibfield  {journal} {\bibinfo
  {journal} {\nat}\ }\textbf {\bibinfo {volume} {556}},\ \bibinfo {pages} {80}
  (\bibinfo {year} {2018}{\natexlab{a}})},\ \Eprint
  {https://arxiv.org/abs/1802.00553} {arXiv:1802.00553 [cond-mat.mes-hall]}
  \BibitemShut {NoStop}%
\bibitem [{\citenamefont {{Cao}}\ \emph
  {et~al.}(2018{\natexlab{b}})\citenamefont {{Cao}}, \citenamefont {{Fatemi}},
  \citenamefont {{Fang}}, \citenamefont {{Watanabe}}, \citenamefont
  {{Taniguchi}}, \citenamefont {{Kaxiras}},\ and\ \citenamefont
  {{Jarillo-Herrero}}}]{2018Natur55643C}%
  \BibitemOpen
  \bibfield  {author} {\bibinfo {author} {\bibfnamefont {Y.}~\bibnamefont
  {{Cao}}}, \bibinfo {author} {\bibfnamefont {V.}~\bibnamefont {{Fatemi}}},
  \bibinfo {author} {\bibfnamefont {S.}~\bibnamefont {{Fang}}}, \bibinfo
  {author} {\bibfnamefont {K.}~\bibnamefont {{Watanabe}}}, \bibinfo {author}
  {\bibfnamefont {T.}~\bibnamefont {{Taniguchi}}}, \bibinfo {author}
  {\bibfnamefont {E.}~\bibnamefont {{Kaxiras}}},\ and\ \bibinfo {author}
  {\bibfnamefont {P.}~\bibnamefont {{Jarillo-Herrero}}},\ }\bibfield  {title}
  {\bibinfo {title} {{Unconventional superconductivity in magic-angle graphene
  superlattices}},\ }\href {https://doi.org/10.1038/nature26160} {\bibfield
  {journal} {\bibinfo  {journal} {\nat}\ }\textbf {\bibinfo {volume} {556}},\
  \bibinfo {pages} {43} (\bibinfo {year} {2018}{\natexlab{b}})},\ \Eprint
  {https://arxiv.org/abs/1803.02342} {arXiv:1803.02342 [cond-mat.mes-hall]}
  \BibitemShut {NoStop}%
\bibitem [{\citenamefont {{Lu}}\ \emph {et~al.}(2019)\citenamefont {{Lu}},
  \citenamefont {{Stepanov}}, \citenamefont {{Yang}}, \citenamefont {{Xie}},
  \citenamefont {{Aamir}}, \citenamefont {{Das}}, \citenamefont {{Urgell}},
  \citenamefont {{Watanabe}}, \citenamefont {{Taniguchi}}, \citenamefont
  {{Zhang}}, \citenamefont {{Bachtold}}, \citenamefont {{MacDonald}},\ and\
  \citenamefont {{Efetov}}}]{2019Natur.574..653L}%
  \BibitemOpen
  \bibfield  {author} {\bibinfo {author} {\bibfnamefont {X.}~\bibnamefont
  {{Lu}}}, \bibinfo {author} {\bibfnamefont {P.}~\bibnamefont {{Stepanov}}},
  \bibinfo {author} {\bibfnamefont {W.}~\bibnamefont {{Yang}}}, \bibinfo
  {author} {\bibfnamefont {M.}~\bibnamefont {{Xie}}}, \bibinfo {author}
  {\bibfnamefont {M.~A.}\ \bibnamefont {{Aamir}}}, \bibinfo {author}
  {\bibfnamefont {I.}~\bibnamefont {{Das}}}, \bibinfo {author} {\bibfnamefont
  {C.}~\bibnamefont {{Urgell}}}, \bibinfo {author} {\bibfnamefont
  {K.}~\bibnamefont {{Watanabe}}}, \bibinfo {author} {\bibfnamefont
  {T.}~\bibnamefont {{Taniguchi}}}, \bibinfo {author} {\bibfnamefont
  {G.}~\bibnamefont {{Zhang}}}, \bibinfo {author} {\bibfnamefont
  {A.}~\bibnamefont {{Bachtold}}}, \bibinfo {author} {\bibfnamefont {A.~H.}\
  \bibnamefont {{MacDonald}}},\ and\ \bibinfo {author} {\bibfnamefont {D.~K.}\
  \bibnamefont {{Efetov}}},\ }\bibfield  {title} {\bibinfo {title}
  {{Superconductors, orbital magnets and correlated states in magic-angle
  bilayer graphene}},\ }\href {https://doi.org/10.1038/s41586-019-1695-0}
  {\bibfield  {journal} {\bibinfo  {journal} {\nat}\ }\textbf {\bibinfo
  {volume} {574}},\ \bibinfo {pages} {653} (\bibinfo {year} {2019})},\ \Eprint
  {https://arxiv.org/abs/1903.06513} {arXiv:1903.06513 [cond-mat.str-el]}
  \BibitemShut {NoStop}%
\bibitem [{\citenamefont {Hazra}\ \emph {et~al.}(2019)\citenamefont {Hazra},
  \citenamefont {Verma},\ and\ \citenamefont {Randeria}}]{PhysRevX.9.031049}%
  \BibitemOpen
  \bibfield  {author} {\bibinfo {author} {\bibfnamefont {T.}~\bibnamefont
  {Hazra}}, \bibinfo {author} {\bibfnamefont {N.}~\bibnamefont {Verma}},\ and\
  \bibinfo {author} {\bibfnamefont {M.}~\bibnamefont {Randeria}},\ }\bibfield
  {title} {\bibinfo {title} {Bounds on the superconducting transition
  temperature: Applications to twisted bilayer graphene and cold atoms},\
  }\href {https://doi.org/10.1103/PhysRevX.9.031049} {\bibfield  {journal}
  {\bibinfo  {journal} {Phys. Rev. X}\ }\textbf {\bibinfo {volume} {9}},\
  \bibinfo {pages} {031049} (\bibinfo {year} {2019})}\BibitemShut {NoStop}%
\bibitem [{\citenamefont {Hu}\ \emph {et~al.}(2019)\citenamefont {Hu},
  \citenamefont {Hyart}, \citenamefont {Pikulin},\ and\ \citenamefont
  {Rossi}}]{PhysRevLett.123.237002}%
  \BibitemOpen
  \bibfield  {author} {\bibinfo {author} {\bibfnamefont {X.}~\bibnamefont
  {Hu}}, \bibinfo {author} {\bibfnamefont {T.}~\bibnamefont {Hyart}}, \bibinfo
  {author} {\bibfnamefont {D.~I.}\ \bibnamefont {Pikulin}},\ and\ \bibinfo
  {author} {\bibfnamefont {E.}~\bibnamefont {Rossi}},\ }\bibfield  {title}
  {\bibinfo {title} {Geometric and conventional contribution to the superfluid
  weight in twisted bilayer graphene},\ }\href
  {https://doi.org/10.1103/PhysRevLett.123.237002} {\bibfield  {journal}
  {\bibinfo  {journal} {Phys. Rev. Lett.}\ }\textbf {\bibinfo {volume} {123}},\
  \bibinfo {pages} {237002} (\bibinfo {year} {2019})}\BibitemShut {NoStop}%
\bibitem [{\citenamefont {Julku}\ \emph {et~al.}(2020)\citenamefont {Julku},
  \citenamefont {Peltonen}, \citenamefont {Liang}, \citenamefont {Heikkil\"a},\
  and\ \citenamefont {T\"orm\"a}}]{PhysRevB.101.060505}%
  \BibitemOpen
  \bibfield  {author} {\bibinfo {author} {\bibfnamefont {A.}~\bibnamefont
  {Julku}}, \bibinfo {author} {\bibfnamefont {T.~J.}\ \bibnamefont {Peltonen}},
  \bibinfo {author} {\bibfnamefont {L.}~\bibnamefont {Liang}}, \bibinfo
  {author} {\bibfnamefont {T.~T.}\ \bibnamefont {Heikkil\"a}},\ and\ \bibinfo
  {author} {\bibfnamefont {P.}~\bibnamefont {T\"orm\"a}},\ }\bibfield  {title}
  {\bibinfo {title} {Superfluid weight and berezinskii-kosterlitz-thouless
  transition temperature of twisted bilayer graphene},\ }\href
  {https://doi.org/10.1103/PhysRevB.101.060505} {\bibfield  {journal} {\bibinfo
   {journal} {Phys. Rev. B}\ }\textbf {\bibinfo {volume} {101}},\ \bibinfo
  {pages} {060505} (\bibinfo {year} {2020})}\BibitemShut {NoStop}%
\bibitem [{\citenamefont {Ledwith}\ \emph {et~al.}(2020)\citenamefont
  {Ledwith}, \citenamefont {Tarnopolsky}, \citenamefont {Khalaf},\ and\
  \citenamefont {Vishwanath}}]{PhysRevResearch.2.023237}%
  \BibitemOpen
  \bibfield  {author} {\bibinfo {author} {\bibfnamefont {P.~J.}\ \bibnamefont
  {Ledwith}}, \bibinfo {author} {\bibfnamefont {G.}~\bibnamefont
  {Tarnopolsky}}, \bibinfo {author} {\bibfnamefont {E.}~\bibnamefont
  {Khalaf}},\ and\ \bibinfo {author} {\bibfnamefont {A.}~\bibnamefont
  {Vishwanath}},\ }\bibfield  {title} {\bibinfo {title} {Fractional chern
  insulator states in twisted bilayer graphene: An analytical approach},\
  }\href {https://doi.org/10.1103/PhysRevResearch.2.023237} {\bibfield
  {journal} {\bibinfo  {journal} {Phys. Rev. Research}\ }\textbf {\bibinfo
  {volume} {2}},\ \bibinfo {pages} {023237} (\bibinfo {year}
  {2020})}\BibitemShut {NoStop}%
\bibitem [{\citenamefont {{Xie}}\ \emph {et~al.}(2020)\citenamefont {{Xie}},
  \citenamefont {{Song}}, \citenamefont {{Lian}},\ and\ \citenamefont
  {{Bernevig}}}]{2020PhRvL124p7002X}%
  \BibitemOpen
  \bibfield  {author} {\bibinfo {author} {\bibfnamefont {F.}~\bibnamefont
  {{Xie}}}, \bibinfo {author} {\bibfnamefont {Z.}~\bibnamefont {{Song}}},
  \bibinfo {author} {\bibfnamefont {B.}~\bibnamefont {{Lian}}},\ and\ \bibinfo
  {author} {\bibfnamefont {B.~A.}\ \bibnamefont {{Bernevig}}},\ }\bibfield
  {title} {\bibinfo {title} {{Topology-Bounded Superfluid Weight in Twisted
  Bilayer Graphene}},\ }\href {https://doi.org/10.1103/PhysRevLett.124.167002}
  {\bibfield  {journal} {\bibinfo  {journal} {\prl}\ }\textbf {\bibinfo
  {volume} {124}},\ \bibinfo {eid} {167002} (\bibinfo {year} {2020})},\ \Eprint
  {https://arxiv.org/abs/1906.02213} {arXiv:1906.02213 [cond-mat.supr-con]}
  \BibitemShut {NoStop}%
\bibitem [{\citenamefont {Chaudhary}\ \emph {et~al.}(2022)\citenamefont
  {Chaudhary}, \citenamefont {Lewandowski},\ and\ \citenamefont
  {Refael}}]{PhysRevResearch4013164}%
  \BibitemOpen
  \bibfield  {author} {\bibinfo {author} {\bibfnamefont {S.}~\bibnamefont
  {Chaudhary}}, \bibinfo {author} {\bibfnamefont {C.}~\bibnamefont
  {Lewandowski}},\ and\ \bibinfo {author} {\bibfnamefont {G.}~\bibnamefont
  {Refael}},\ }\bibfield  {title} {\bibinfo {title} {Shift-current response as
  a probe of quantum geometry and electron-electron interactions in twisted
  bilayer graphene},\ }\href {https://doi.org/10.1103/PhysRevResearch.4.013164}
  {\bibfield  {journal} {\bibinfo  {journal} {Phys. Rev. Research}\ }\textbf
  {\bibinfo {volume} {4}},\ \bibinfo {pages} {013164} (\bibinfo {year}
  {2022})}\BibitemShut {NoStop}%
\bibitem [{\citenamefont {Mera}\ and\ \citenamefont
  {Ozawa}(2021)}]{PhysRevB.104.115160}%
  \BibitemOpen
  \bibfield  {author} {\bibinfo {author} {\bibfnamefont {B.}~\bibnamefont
  {Mera}}\ and\ \bibinfo {author} {\bibfnamefont {T.}~\bibnamefont {Ozawa}},\
  }\bibfield  {title} {\bibinfo {title} {Engineering geometrically flat chern
  bands with fubini-study k\"ahler structure},\ }\href
  {https://doi.org/10.1103/PhysRevB.104.115160} {\bibfield  {journal} {\bibinfo
   {journal} {Phys. Rev. B}\ }\textbf {\bibinfo {volume} {104}},\ \bibinfo
  {pages} {115160} (\bibinfo {year} {2021})}\BibitemShut {NoStop}%
\bibitem [{\citenamefont {Kaplan}\ \emph {et~al.}(2022)\citenamefont {Kaplan},
  \citenamefont {Holder},\ and\ \citenamefont
  {Yan}}]{PhysRevResearch.4.013209}%
  \BibitemOpen
  \bibfield  {author} {\bibinfo {author} {\bibfnamefont {D.}~\bibnamefont
  {Kaplan}}, \bibinfo {author} {\bibfnamefont {T.}~\bibnamefont {Holder}},\
  and\ \bibinfo {author} {\bibfnamefont {B.}~\bibnamefont {Yan}},\ }\bibfield
  {title} {\bibinfo {title} {Twisted photovoltaics at terahertz frequencies
  from momentum shift current},\ }\href
  {https://doi.org/10.1103/PhysRevResearch.4.013209} {\bibfield  {journal}
  {\bibinfo  {journal} {Phys. Rev. Research}\ }\textbf {\bibinfo {volume}
  {4}},\ \bibinfo {pages} {013209} (\bibinfo {year} {2022})}\BibitemShut
  {NoStop}%
\bibitem [{\citenamefont {{T{\"o}rm{\"a}}}\ \emph {et~al.}(2022)\citenamefont
  {{T{\"o}rm{\"a}}}, \citenamefont {{Peotta}},\ and\ \citenamefont
  {{Bernevig}}}]{2021arXiv211100807T}%
  \BibitemOpen
  \bibfield  {author} {\bibinfo {author} {\bibfnamefont {P.}~\bibnamefont
  {{T{\"o}rm{\"a}}}}, \bibinfo {author} {\bibfnamefont {S.}~\bibnamefont
  {{Peotta}}},\ and\ \bibinfo {author} {\bibfnamefont {B.~A.}\ \bibnamefont
  {{Bernevig}}},\ }\bibfield  {title} {\bibinfo {title} {{Superfluidity and
  Quantum Geometry in Twisted Multilayer Systems}},\ }\href
  {https://doi.org/10.1038/s42254-022-00466-y} {\bibfield  {journal} {\bibinfo
  {journal} {Nat. Rev. Phys.}\ }\textbf {\bibinfo {volume} {4}},\ \bibinfo
  {pages} {528–542} (\bibinfo {year} {2022})}\BibitemShut {NoStop}%
\bibitem [{\citenamefont {Wang}\ \emph {et~al.}(2021)\citenamefont {Wang},
  \citenamefont {Cano}, \citenamefont {Millis}, \citenamefont {Liu},\ and\
  \citenamefont {Yang}}]{PhysRevLett.127.246403}%
  \BibitemOpen
  \bibfield  {author} {\bibinfo {author} {\bibfnamefont {J.}~\bibnamefont
  {Wang}}, \bibinfo {author} {\bibfnamefont {J.}~\bibnamefont {Cano}}, \bibinfo
  {author} {\bibfnamefont {A.~J.}\ \bibnamefont {Millis}}, \bibinfo {author}
  {\bibfnamefont {Z.}~\bibnamefont {Liu}},\ and\ \bibinfo {author}
  {\bibfnamefont {B.}~\bibnamefont {Yang}},\ }\bibfield  {title} {\bibinfo
  {title} {Exact landau level description of geometry and interaction in a
  flatband},\ }\href {https://doi.org/10.1103/PhysRevLett.127.246403}
  {\bibfield  {journal} {\bibinfo  {journal} {Phys. Rev. Lett.}\ }\textbf
  {\bibinfo {volume} {127}},\ \bibinfo {pages} {246403} (\bibinfo {year}
  {2021})}\BibitemShut {NoStop}%
\bibitem [{\citenamefont {Wang}\ and\ \citenamefont
  {Liu}(2022)}]{PhysRevLett.128.176403}%
  \BibitemOpen
  \bibfield  {author} {\bibinfo {author} {\bibfnamefont {J.}~\bibnamefont
  {Wang}}\ and\ \bibinfo {author} {\bibfnamefont {Z.}~\bibnamefont {Liu}},\
  }\bibfield  {title} {\bibinfo {title} {Hierarchy of ideal flatbands in chiral
  twisted multilayer graphene models},\ }\href
  {https://doi.org/10.1103/PhysRevLett.128.176403} {\bibfield  {journal}
  {\bibinfo  {journal} {Phys. Rev. Lett.}\ }\textbf {\bibinfo {volume} {128}},\
  \bibinfo {pages} {176403} (\bibinfo {year} {2022})}\BibitemShut {NoStop}%
\bibitem [{\citenamefont {{Tian}}\ \emph {et~al.}(2023)\citenamefont {{Tian}},
  \citenamefont {{Gao}}, \citenamefont {{Zhang}}, \citenamefont {{Che}},
  \citenamefont {{Xu}}, \citenamefont {{Cheung}}, \citenamefont {{Watanabe}},
  \citenamefont {{Taniguchi}}, \citenamefont {{Randeria}}, \citenamefont
  {{Zhang}}, \citenamefont {{Lau}},\ and\ \citenamefont
  {{Bockrath}}}]{2023Natur.614..440T}%
  \BibitemOpen
  \bibfield  {author} {\bibinfo {author} {\bibfnamefont {H.}~\bibnamefont
  {{Tian}}}, \bibinfo {author} {\bibfnamefont {X.}~\bibnamefont {{Gao}}},
  \bibinfo {author} {\bibfnamefont {Y.}~\bibnamefont {{Zhang}}}, \bibinfo
  {author} {\bibfnamefont {S.}~\bibnamefont {{Che}}}, \bibinfo {author}
  {\bibfnamefont {T.}~\bibnamefont {{Xu}}}, \bibinfo {author} {\bibfnamefont
  {P.}~\bibnamefont {{Cheung}}}, \bibinfo {author} {\bibfnamefont
  {K.}~\bibnamefont {{Watanabe}}}, \bibinfo {author} {\bibfnamefont
  {T.}~\bibnamefont {{Taniguchi}}}, \bibinfo {author} {\bibfnamefont
  {M.}~\bibnamefont {{Randeria}}}, \bibinfo {author} {\bibfnamefont
  {F.}~\bibnamefont {{Zhang}}}, \bibinfo {author} {\bibfnamefont {C.~N.}\
  \bibnamefont {{Lau}}},\ and\ \bibinfo {author} {\bibfnamefont {M.~W.}\
  \bibnamefont {{Bockrath}}},\ }\bibfield  {title} {\bibinfo {title} {{Evidence
  for Dirac flat band superconductivity enabled by quantum geometry}},\ }\href
  {https://doi.org/10.1038/s41586-022-05576-2} {\bibfield  {journal} {\bibinfo
  {journal} {\nat}\ }\textbf {\bibinfo {volume} {614}},\ \bibinfo {pages} {440}
  (\bibinfo {year} {2023})}\BibitemShut {NoStop}%
\bibitem [{\citenamefont {Haldane}(2004)}]{PhysRevLett.93.206602}%
  \BibitemOpen
  \bibfield  {author} {\bibinfo {author} {\bibfnamefont {F.~D.~M.}\
  \bibnamefont {Haldane}},\ }\bibfield  {title} {\bibinfo {title} {Berry
  curvature on the fermi surface: Anomalous hall effect as a topological
  fermi-liquid property},\ }\href
  {https://doi.org/10.1103/PhysRevLett.93.206602} {\bibfield  {journal}
  {\bibinfo  {journal} {Phys. Rev. Lett.}\ }\textbf {\bibinfo {volume} {93}},\
  \bibinfo {pages} {206602} (\bibinfo {year} {2004})}\BibitemShut {NoStop}%
\bibitem [{\citenamefont {{Chen}}\ and\ \citenamefont
  {{Son}}(2017)}]{2017AnPhy.377..345C}%
  \BibitemOpen
  \bibfield  {author} {\bibinfo {author} {\bibfnamefont {J.-Y.}\ \bibnamefont
  {{Chen}}}\ and\ \bibinfo {author} {\bibfnamefont {D.~T.}\ \bibnamefont
  {{Son}}},\ }\bibfield  {title} {\bibinfo {title} {{Berry Fermi liquid
  theory}},\ }\href {https://doi.org/10.1016/j.aop.2016.12.017} {\bibfield
  {journal} {\bibinfo  {journal} {Annals of Physics}\ }\textbf {\bibinfo
  {volume} {377}},\ \bibinfo {pages} {345} (\bibinfo {year} {2017})},\ \Eprint
  {https://arxiv.org/abs/1604.07857} {arXiv:1604.07857 [cond-mat.str-el]}
  \BibitemShut {NoStop}%
\bibitem [{SM()}]{SM}%
  \BibitemOpen
  \href@noop {} {\bibinfo  {journal} {Further details of results and
  calculation are available as supplementary material}\ }\BibitemShut {NoStop}%
\bibitem [{\citenamefont {{Cheng}}(2010)}]{2010arXiv1012.1337C}%
  \BibitemOpen
\bibfield  {journal} {  }\bibfield  {author} {\bibinfo {author} {\bibfnamefont
  {R.}~\bibnamefont {{Cheng}}},\ }\bibfield  {title} {\bibinfo {title}
  {{Quantum Geometric Tensor (Fubini-Study Metric) in Simple Quantum System: A
  pedagogical Introduction}},\ }\href
  {https://doi.org/10.48550/arXiv.1012.1337} {\bibfield  {journal} {\bibinfo
  {journal} {arXiv e-prints}\ ,\ \bibinfo {eid} {arXiv:1012.1337}} (\bibinfo
  {year} {2010})},\ \Eprint {https://arxiv.org/abs/1012.1337} {arXiv:1012.1337
  [quant-ph]} \BibitemShut {NoStop}%
\bibitem [{\citenamefont {Nelson}\ and\ \citenamefont
  {Kosterlitz}(1977)}]{PhysRevLett.39.1201}%
  \BibitemOpen
  \bibfield  {author} {\bibinfo {author} {\bibfnamefont {D.~R.}\ \bibnamefont
  {Nelson}}\ and\ \bibinfo {author} {\bibfnamefont {J.~M.}\ \bibnamefont
  {Kosterlitz}},\ }\bibfield  {title} {\bibinfo {title} {Universal jump in the
  superfluid density of two-dimensional superfluids},\ }\href
  {https://doi.org/10.1103/PhysRevLett.39.1201} {\bibfield  {journal} {\bibinfo
   {journal} {Phys. Rev. Lett.}\ }\textbf {\bibinfo {volume} {39}},\ \bibinfo
  {pages} {1201} (\bibinfo {year} {1977})}\BibitemShut {NoStop}%
\bibitem [{\citenamefont {Altland}\ and\ \citenamefont
  {Simons}(2010)}]{altland2010condensed}%
  \BibitemOpen
  \bibfield  {author} {\bibinfo {author} {\bibfnamefont {A.}~\bibnamefont
  {Altland}}\ and\ \bibinfo {author} {\bibfnamefont {B.~D.}\ \bibnamefont
  {Simons}},\ }\href@noop {} {\emph {\bibinfo {title} {Condensed matter field
  theory}}}\ (\bibinfo  {publisher} {Cambridge university press},\ \bibinfo
  {year} {2010})\BibitemShut {NoStop}%
\bibitem [{\citenamefont {{Bistritzer}}\ and\ \citenamefont
  {{MacDonald}}(2011)}]{2011PNAS..10812233B}%
  \BibitemOpen
  \bibfield  {author} {\bibinfo {author} {\bibfnamefont {R.}~\bibnamefont
  {{Bistritzer}}}\ and\ \bibinfo {author} {\bibfnamefont {A.~H.}\ \bibnamefont
  {{MacDonald}}},\ }\bibfield  {title} {\bibinfo {title} {{Moir{\'e} bands in
  twisted double-layer graphene}},\ }\href
  {https://doi.org/10.1073/pnas.1108174108} {\bibfield  {journal} {\bibinfo
  {journal} {Proceedings of the National Academy of Science}\ }\textbf
  {\bibinfo {volume} {108}},\ \bibinfo {pages} {12233} (\bibinfo {year}
  {2011})},\ \Eprint {https://arxiv.org/abs/1009.4203} {arXiv:1009.4203
  [cond-mat.mes-hall]} \BibitemShut {NoStop}%
\bibitem [{\citenamefont {Xie}\ and\ \citenamefont
  {MacDonald}(2020)}]{PhysRevLett.124.097601}%
  \BibitemOpen
  \bibfield  {author} {\bibinfo {author} {\bibfnamefont {M.}~\bibnamefont
  {Xie}}\ and\ \bibinfo {author} {\bibfnamefont {A.~H.}\ \bibnamefont
  {MacDonald}},\ }\bibfield  {title} {\bibinfo {title} {Nature of the
  correlated insulator states in twisted bilayer graphene},\ }\href
  {https://doi.org/10.1103/PhysRevLett.124.097601} {\bibfield  {journal}
  {\bibinfo  {journal} {Phys. Rev. Lett.}\ }\textbf {\bibinfo {volume} {124}},\
  \bibinfo {pages} {097601} (\bibinfo {year} {2020})}\BibitemShut {NoStop}%
\bibitem [{\citenamefont {{Tarnopolsky}}\ \emph {et~al.}(2019)\citenamefont
  {{Tarnopolsky}}, \citenamefont {{Kruchkov}},\ and\ \citenamefont
  {{Vishwanath}}}]{2019PhRvL.122j6405T}%
  \BibitemOpen
  \bibfield  {author} {\bibinfo {author} {\bibfnamefont {G.}~\bibnamefont
  {{Tarnopolsky}}}, \bibinfo {author} {\bibfnamefont {A.~J.}\ \bibnamefont
  {{Kruchkov}}},\ and\ \bibinfo {author} {\bibfnamefont {A.}~\bibnamefont
  {{Vishwanath}}},\ }\bibfield  {title} {\bibinfo {title} {{Origin of Magic
  Angles in Twisted Bilayer Graphene}},\ }\href
  {https://doi.org/10.1103/PhysRevLett.122.106405} {\bibfield  {journal}
  {\bibinfo  {journal} {\prl}\ }\textbf {\bibinfo {volume} {122}},\ \bibinfo
  {eid} {106405} (\bibinfo {year} {2019})},\ \Eprint
  {https://arxiv.org/abs/1808.05250} {arXiv:1808.05250 [cond-mat.str-el]}
  \BibitemShut {NoStop}%
\bibitem [{\citenamefont {Liu}\ \emph {et~al.}(2019)\citenamefont {Liu},
  \citenamefont {Liu},\ and\ \citenamefont {Dai}}]{PhysRevB.99.155415}%
  \BibitemOpen
  \bibfield  {author} {\bibinfo {author} {\bibfnamefont {J.}~\bibnamefont
  {Liu}}, \bibinfo {author} {\bibfnamefont {J.}~\bibnamefont {Liu}},\ and\
  \bibinfo {author} {\bibfnamefont {X.}~\bibnamefont {Dai}},\ }\bibfield
  {title} {\bibinfo {title} {Pseudo landau level representation of twisted
  bilayer graphene: Band topology and implications on the correlated insulating
  phase},\ }\href {https://doi.org/10.1103/PhysRevB.99.155415} {\bibfield
  {journal} {\bibinfo  {journal} {Phys. Rev. B}\ }\textbf {\bibinfo {volume}
  {99}},\ \bibinfo {pages} {155415} (\bibinfo {year} {2019})}\BibitemShut
  {NoStop}%
\bibitem [{\citenamefont {Hofstadter}(1976)}]{PhysRevB.14.2239}%
  \BibitemOpen
  \bibfield  {author} {\bibinfo {author} {\bibfnamefont {D.~R.}\ \bibnamefont
  {Hofstadter}},\ }\bibfield  {title} {\bibinfo {title} {Energy levels and wave
  functions of bloch electrons in rational and irrational magnetic fields},\
  }\href {https://doi.org/10.1103/PhysRevB.14.2239} {\bibfield  {journal}
  {\bibinfo  {journal} {Phys. Rev. B}\ }\textbf {\bibinfo {volume} {14}},\
  \bibinfo {pages} {2239} (\bibinfo {year} {1976})}\BibitemShut {NoStop}%
\bibitem [{\citenamefont {{Ozawa}}\ and\ \citenamefont
  {{Mera}}(2021)}]{2021PhRvB.104d5103O}%
  \BibitemOpen
  \bibfield  {author} {\bibinfo {author} {\bibfnamefont {T.}~\bibnamefont
  {{Ozawa}}}\ and\ \bibinfo {author} {\bibfnamefont {B.}~\bibnamefont
  {{Mera}}},\ }\bibfield  {title} {\bibinfo {title} {{Relations between
  topology and the quantum metric for Chern insulators}},\ }\href
  {https://doi.org/10.1103/PhysRevB.104.045103} {\bibfield  {journal} {\bibinfo
   {journal} {\prb}\ }\textbf {\bibinfo {volume} {104}},\ \bibinfo {eid}
  {045103} (\bibinfo {year} {2021})},\ \Eprint
  {https://arxiv.org/abs/2103.11582} {arXiv:2103.11582 [cond-mat.mes-hall]}
  \BibitemShut {NoStop}%
\bibitem [{\citenamefont {Koshino}\ \emph {et~al.}(2018)\citenamefont
  {Koshino}, \citenamefont {Yuan}, \citenamefont {Koretsune}, \citenamefont
  {Ochi}, \citenamefont {Kuroki},\ and\ \citenamefont
  {Fu}}]{PhysRevX.8.031087}%
  \BibitemOpen
  \bibfield  {author} {\bibinfo {author} {\bibfnamefont {M.}~\bibnamefont
  {Koshino}}, \bibinfo {author} {\bibfnamefont {N.~F.~Q.}\ \bibnamefont
  {Yuan}}, \bibinfo {author} {\bibfnamefont {T.}~\bibnamefont {Koretsune}},
  \bibinfo {author} {\bibfnamefont {M.}~\bibnamefont {Ochi}}, \bibinfo {author}
  {\bibfnamefont {K.}~\bibnamefont {Kuroki}},\ and\ \bibinfo {author}
  {\bibfnamefont {L.}~\bibnamefont {Fu}},\ }\bibfield  {title} {\bibinfo
  {title} {Maximally localized wannier orbitals and the extended hubbard model
  for twisted bilayer graphene},\ }\href
  {https://doi.org/10.1103/PhysRevX.8.031087} {\bibfield  {journal} {\bibinfo
  {journal} {Phys. Rev. X}\ }\textbf {\bibinfo {volume} {8}},\ \bibinfo {pages}
  {031087} (\bibinfo {year} {2018})}\BibitemShut {NoStop}%
\end{thebibliography}%



\clearpage
\newpage

\begin{widetext}

\date{\today}
\maketitle


\section*{Supplemental Material for ``Ginzburg-Landau theory of flat-band superconductors with quantum metric"}



    \renewcommand{\theequation}{S\arabic{equation}}
    \setcounter{equation}{0}
    \renewcommand{\thefigure}{S\arabic{figure}}
    \setcounter{figure}{0}

% \tableofcontents
% \hypersetup{linkcolor=blue}
\tableofcontents
\titlecontents{section}
              [0.0cm]
              {}%
              {\contentslabel{3.5em}}%
              {}%
              {\titlerule*[0.5pc]{}\contentspage}%



\titlecontents{subsection}
              [0.1cm]
              {}%
              {\contentslabel{3.0em}}%
              {}%
              {\titlerule*[0.5pc]{}\contentspage}%

\titlecontents{subsubsection}
              [1.0cm]
              {}%
              {\contentslabel{2.5em}}%
              {}%
              {\titlerule*[0.5pc]{}\contentspage}%




\subsection{Details on deriving the Free energy \label{app:free} }
We provide the derivation details of the GL theory, focusing on the isolated flatband system. Our starting point is the Hamiltonian $H=H_{0}+H_{\mathrm{int}}$, where $H_{0}$ represents the free part, and
\begin{equation}
H_{\mathrm{int}}=-g\int d\mathbf{r}a_{+}^{\dagger}(\mathbf{r})a_{-}^{\dagger}(\mathbf{r})a_{-}(\mathbf{r})a_{+}(\mathbf{r})~,
\end{equation}
represents an attractive interaction.
We define the isolated band limit for $H_{0}$ as a situation where there exists a large band gap $W$ between the band around the Fermi energy and the other bands. Moreover, we require that the interaction part does not significantly alter the band structure, which can be expressed as $W\gg\vert g\vert$. For simplicity, we assume the ideal flatband limit.
We can obtain an effective Hamiltonian,
\begin{equation}
H_{0}=0~,
\end{equation}
in the low-energy limit. To handle the interaction, we introduce the Hubbard-Stratonovich (HS) transformation,
\begin{equation}
1=\int\mathcal{D}[\Delta,\bar{\Delta}]e^{-g\int_{0}^{\beta}d\tau\int d\mathbf{r}[\Delta(\mathbf{r})-a_{-}(\mathbf{r})a_{+}(\mathbf{r})][\bar{\Delta}(\mathbf{r})-\bar{a}_{+}(\mathbf{r})\bar{a}_{-}(\mathbf{r})]}~,
\end{equation}
where $\bar{a},a$ are the Grassmann fields. By employing the HS transformation, we obtain the path integral formulation,
\begin{align}
Z & =\mathrm{Tr}e^{-\beta H_{\mathrm{int}}}=\int\mathcal{D}[\Delta,\bar{\Delta}]e^{-\int_{0}^{\beta}d\tau\int d\mathbf{r}\vert\Delta(\mathbf{r})\vert^{2}}\mathcal{Z}[\Delta,\bar{\Delta}]~,
\end{align}
where
\begin{align}
\mathcal{Z}[\Delta,\bar{\Delta}] & =\int\mathcal{D}[c,\bar{c}]e^{-\int_{0}^{\beta}d\tau\int d\mathbf{r}\mathcal{L}[a,\bar{a},\Delta,\bar{\Delta}]}~,
\end{align}
and
\begin{align}
\mathcal{L}[a,\bar{a},\Delta,\bar{\Delta}]= & (\partial_{\tau}-\mu)(\bar{a}_{+}(\mathbf{r})a_{+}(\mathbf{r})+\bar{a}_{-}(\mathbf{r})a_{-}(\mathbf{r}))-g\left[\Delta(\mathbf{r})\bar{a}_{+}(\mathbf{r})\bar a_{-}(\mathbf{r})+h.c.\right]~.
\end{align}

As highlighted in the main text, it is crucial to introduce a projection in Eq.~(2) in the main text, or alternatively,
\begin{equation}
a_{\xi}(\mathbf{r})\rightarrow\frac{1}{N}\sum_{\mathbf{q}}e^{i\mathbf{q}\cdot\mathbf{r}}g_{\mathbf{q}\xi}^{*}(\alpha)c_{\mathbf{q}\xi}~,
\end{equation}
to explicitly incorporate the Bloch wave, where $\alpha$ labels the internal degrees of freedom. Consequently, we obtain a projected Lagrangian density $\mathcal{L}[c,\bar{c},\Delta,\bar{\Delta}]$:
\begin{equation}
\mathcal{L}[a,\bar{a},\Delta,\bar{\Delta}]\rightarrow\mathcal{L}[c,\bar{c},\Delta,\bar{\Delta}]~,
\end{equation}
where 
\begin{align}
\mathcal{L}[c,\bar{c},\Delta,\bar{\Delta}]= & (\partial_{\tau}-\mu)(\bar{c}_{\mathbf{q},+}c_{\mathbf{q},+}+\bar{c}_{\mathbf{q},-}c_{\mathbf{q},-}) -\sum_{\mathbf{k}}g[\Gamma(\mathbf{q},\mathbf{k})\Delta(\mathbf{k})\bar{c}_{\mathbf{q+\frac{\mathbf{k}}{2}},+}\bar{c}_{-\mathbf{q+\frac{\mathbf{k}}{2}},-}+h.c.]~.
\end{align}
The presence of the form factor $g\Gamma(\mathbf{q},\mathbf{k})$ is crucial in describing the interaction between the bosonic field $\Delta$ and the fermion field $c$.
Hence, we obtain the GL theory $F=F[\Delta,\bar{\Delta}]$ using the formula:
\begin{align}
Z & \equiv\int\mathcal{D}[\Delta,\bar{\Delta}]e^{-\beta F[\Delta,\bar{\Delta}]}~,
\end{align}
or equivalently,
\begin{align}
F[\Delta,\bar{\Delta}] & =\sum_{\mathbf{k}}g\vert\Delta(\mathbf{k})\vert^{2}-T\ln\mathcal{Z}[\Delta,\bar{\Delta}]\nonumber \\
 & =\sum_{\mathbf{k}}g\vert\Delta(\mathbf{k})\vert^{2}-T\int\mathcal{D}[c,\bar{c}]e^{-\int_{0}^{\beta}d\tau\int d\mathbf{r}\mathcal{L}[c,\bar{c},\Delta,\bar{\Delta}]}\\
 & =\sum_{\mathbf{k}}g\vert\Delta(\mathbf{k})\vert^{2}-T\ln\det\mathrm{G}~,
\end{align}
where $\mathrm{G}$ is the kernel when we arrange $\mathcal{L}[c,\bar{c},\Delta,\bar{\Delta}]$ into a matrix form:
\begin{align}
\mathcal{L}[c,\bar{c},\Delta,\bar{\Delta}] & =\sum_{\mathbf{q}^{\prime}\mathbf{q}}\begin{bmatrix}\bar{c}_{\mathbf{q}^{\prime},+}\\
c_{-\mathbf{q}^{\prime},-}
\end{bmatrix}\mathrm{G}_{\mathbf{q}^{\prime}\mathbf{q}}\begin{bmatrix}c_{\mathbf{q},+} & \bar{c}_{-\mathbf{q},-}\end{bmatrix}~.
\end{align}
The subsequent step involves expanding the determinant $\ln\det\mathrm{G}$ using standard procedures \cite{altland2010condensed}.

Here we alternatively employ the Gor'kov's Green function approach.
We first expand the bosonic field $\Delta(\mathbf{k})$ around the extremum of the free energy as:
\begin{equation}
\Delta(\mathbf{k})=\Delta_{0}\delta_{\mathbf{k},0}+\delta\Delta(\mathbf{k})~,
\end{equation}
where $\Delta_{0}$ represents the mean field solution and $\delta\Delta(\mathbf{k})$ represents the fluctuations. We assume $\Delta_{0}$ to be real with a proper gauge choice. This decomposition splits the Lagrangian $\mathcal{L}[c,\bar{c},\Delta,\bar{\Delta}]$ into two parts:
\begin{equation}
\mathcal{L}[c,\bar{c},\Delta,\bar{\Delta}]=\mathcal{L}_{0}+\mathcal{L}_{\mathrm{int}}~,
\end{equation}
where $\mathcal{L}_{0}$ at the leading order reproduces the BCS mean field theory:
\begin{align}
\mathcal{L}_{0}= & (\partial_{\tau}-\mu)(\bar{c}_{\mathbf{q},+}c_{\mathbf{q},+}+\bar{c}_{\mathbf{q},-}c_{\mathbf{q},-}) -g[\Gamma(\mathbf{q})\Delta_{0}\bar{c}_{\mathbf{q},+}\bar{c}_{-\mathbf{q},-}+h.c.]~,\label{app:Lo}
\end{align}
with 
\begin{equation}
\Gamma(\mathbf{q})\equiv\Gamma(\mathbf{q},\mathbf{0})=\sum_{\alpha}g_{\mathbf{\mathbf{-q}},+}(\alpha)g_{\mathbf{\mathbf{q}},-}(\mathbf{\alpha})~,
\end{equation}
where $\Gamma(\mathbf{q})\equiv\Gamma(\mathbf{q},\mathbf{0})=\sum_{\alpha}g_{\mathbf{-q},+}(\alpha)g_{\mathbf{q},-}(\alpha)$ represents the interaction term. The interaction Lagrangian $\mathcal{L}_{\mathrm{int}}$
characterizes the coupling between fluctuations $\delta\Delta(\mathbf{k})$ and fermions $c$,
\begin{equation}
\mathcal{L}_{\mathrm{int}}=-g\sum_{\mathbf{q}}[\Gamma(\mathbf{q},\mathbf{k})\delta\Delta(\mathbf{k})\bar{c}_{\mathbf{q}+\mathbf{\frac{k}{2}},+}\bar{c}_{\mathbf{-q}+\frac{\mathbf{k}}{2},-}+h.c.]~.
\end{equation}
Accordingly, we can decompose the free energy $F[\Delta,\bar{\Delta}]$ in terms of the fluctuations $\delta\Delta(\mathbf{k})$,
\begin{align}
F[\Delta,\bar{\Delta}] & =F_{0}+\delta F~,
\end{align}
where $F_{0}$ represents the mean-field contribution and $\delta F$ accounts for the fluctuations. The expression for $F_{0}$ is given by:
\begin{align}
F_{0}= & \sum_{\mathbf{k}}g\vert\Delta_{0}\vert^{2}-T\int\mathcal{D}[c,\bar{c}]e^{-\int_{0}^{\beta}d\tau\sum_{\mathbf{k}}\mathcal{L}_{0}}~,\\
\delta F= & \sum_{\mathbf{k}}g\vert\delta\Delta\vert^{2}-T\int\mathcal{D}[c,\bar{c}]e^{-\int_{0}^{\beta}d\tau\sum_{\mathbf{k}}\mathcal{L}_{0}}\left(e^{-\int_{0}^{\beta}d\tau\sum_{\mathbf{k}}\mathcal{L}_{\mathrm{int}}}-1\right) \notag 
\\ \equiv &  \sum_{\mathbf{k}}g\vert\delta\Delta\vert^{2}-T \left\langle\left(e^{-\int_{0}^{\beta}d\tau\sum_{\mathbf{k}}\mathcal{L}_{\mathrm{int}}}-1\right)\right\rangle ~,
\end{align}
where $\langle\cdot \rangle$ denotes the thermal average.

The mean-field value $\Delta_{0}$ is determined by solving the self-consistent equation $\frac{\delta F}{\delta\Delta_{0}}=\frac{\delta F}{\delta\bar{\Delta}_{0}}=0$, which is given by Eq.~(6) in the main text. Once we find the value of $\Delta_{0}$, we can evaluate $F_{0}$, which represents the grand potential, as
\begin{align}
F_{0} & =gV\vert\Delta_{0}\vert^{2}-T\ln\int\mathcal{D}[c]e^{-\int_{0}^{\beta}d\tau\sum_{\mathbf{k}}\mathcal{L}_{0}}\nonumber \\
 & =gV\vert\Delta_{0}\vert^{2}-\sum_{\mathbf{k}}\left[2\ln(1+e^{-\beta\epsilon(\mathbf{k})})-\epsilon(\mathbf{k})\right]~,
\end{align}
with $V$ denoting the system volume. 

To evaluate $\delta F$ and consider fluctuations around the mean field, 
we can use Gor'kov's normal and abnormal Green functions $\mathcal{G}(q)$
and $\mathcal{F}(q)$, respectively. These functions are defined as
\begin{align}
\mathcal{G}(q) & \equiv\langle c_{\mathbf{q},+}(\omega)\bar{c}_{\mathbf{q},+}(\omega)\rangle=\langle c_{q,-}(\omega)\bar{c}_{q,-}(\omega)\rangle=\frac{i\omega+\mu}{\omega^{2}+\epsilon^{2}(\mathbf{q})},\label{eq:GorkowG-2}\\
\mathcal{F}(q) & \equiv\langle c_{-\mathbf{q},-}(-\omega)c_{\mathbf{q},+}(\omega)\rangle=\frac{g\Gamma^{*}(\mathbf{q})\Delta_{0}}{\omega^{2}+\epsilon^{2}(\mathbf{q})}~. \label{eq:GorkovF-2}
\end{align}
where $q=(\omega,\mathbf{q})$ represents the frequency and momentum.
In orders of $\delta\Delta$, we can have perturbative expansion 
\begin{equation}
\delta F=F_{2}+F_{4}+\cdots~,
\end{equation}
where terms up to second order in $\delta\Delta(\mathbf{k})$ are included. Since the mean field value is stable, there is no linear term in $\delta\Delta(\mathbf{k})$.
If we neglect temporal fluctuations, we can focus on the second-order term $F_{2}$, which corresponds to Gaussian fluctuations. Evaluating $F_{2}$ gives
\begin{align}
F_{2}= & \sum_{\mathbf{k}}g\vert\delta\Delta(\mathbf{k})\vert^{2}-T\frac{1}{2}\left\langle \left(\int_{0}^{\beta}d\tau\sum_{\mathbf{k}}\mathcal{L}_{\mathrm{int}}\right)^{2}\right\rangle \nonumber \\
= & \sum_{\mathbf{k}}g\vert\delta\Delta(\mathbf{k})\vert^{2}-T\sum_{\mathbf{k}}g^{2}\vert\Gamma(\mathbf{q},\mathbf{k})\vert^{2}|\delta\Delta(\mathbf{k})|^{2}\left[\langle\bar{c}_{\mathbf{q}+\mathbf{\frac{k}{2}},+}\bar{c}_{\mathbf{-q}+\frac{\mathbf{k}}{2},-}\rangle\langle c_{\mathbf{-q}+\frac{\mathbf{k}}{2},-}c_{\mathbf{q}+\mathbf{\frac{k}{2}},+}\rangle\right.\nonumber \\
 & +\left.\frac{1}{2}\langle c_{\mathbf{q}+\mathbf{\frac{k}{2}},+}c_{\mathbf{-q}-\frac{\mathbf{k}}{2},-}\rangle\langle c_{\mathbf{-q}+\frac{\mathbf{k}}{2},-}c_{\mathbf{q}+\mathbf{\frac{k}{2}},+}\rangle+\frac{1}{2}\langle\bar{c}_{\mathbf{q}+\mathbf{\frac{k}{2}},+}\bar{c}_{\mathbf{-q}-\frac{\mathbf{k}}{2},-}\rangle\langle\bar{c}_{\mathbf{-q}+\frac{\mathbf{k}}{2},-}\bar{c}_{\mathbf{q}+\mathbf{\frac{k}{2}},+}\rangle\right]\nonumber \\
\equiv & \sum_{\mathbf{k}}|\delta\Delta(\mathbf{k})|^{2}[g-g^{2}\chi(\mathbf{k})]~.
\end{align}
where $\chi(\mathbf{k})$ takes the form as Eq.~(9) in
the main text.

%%%%%%%%%%
\subsection{Quantum metric and Wannier functions}
In the limit of zero temperature, the minimal size of the  Cooper pair is solely determined by the quantum metric. This fundamental concept characterizes the geometric properties of the electronic band structure. Remarkably, even with stronger interactions, it is impossible to bind the electrons any closer together. To gain a better understanding, we delve into the explanation of the quantum metric and the optically localized size of Wannier wave functions. Interested readers can also refer to Refs.~\cite{PhysRevB.56.12847,2015NatCo68944P}.




We begin with the single-particle Schr{\"o}dinger equation in $d$ spatial dimensions
\begin{equation}
H\vert\psi\rangle=\left[-\frac{(\hbar\nabla)^{2}}{2m}+V(\mathbf{r})\right]\vert\psi\rangle, \label{eq:Hcont}
\end{equation}
where $V(\mathbf{r}+\mathbf{a}_{i})=V(\mathbf{r})$ represents a periodic potential, and $\mathbf{a}_{i}$ ($i=1,\cdots,d$) defines a lattice system. According to the Bloch theorem, the solutions, known as Bloch waves, for an energy band $n$ can be expressed as:
\begin{equation}
\psi_{n\mathbf{k}}(\mathbf{r})=e^{i\mathbf{k}\cdot\mathbf{r}}u_{n\mathbf{k}}(\mathbf{r}),
\end{equation}
where $u_{n\mathbf{k}}(\mathbf{r})$ is a periodic Bloch function satisfying $u_{n\mathbf{k}}(\mathbf{r})=u_{n\mathbf{k}}(\mathbf{r}+\mathbf{a}_{i})$, and $\mathbf{k}$ is the Bloch wavevector. The normalization condition for $u_{n\mathbf{k}}(\mathbf{r})$ is given by:
\begin{equation}
\int_{\text{u.c.}}d^{d}\mathbf{r} \vert u_{n\mathbf{k}}(\mathbf{r})\vert^{2}=1,
\end{equation}
where the integral is taken over one unit cell. 
Here, u.c. represents the unit cell with volume $\mathcal{A}_\mathrm{uc}$. The energy $\epsilon_n(\mathbf{k})$ satisfies periodicity with respect to the reciprocal lattice vectors $\mathbf{G}_i$, given by the condition $\mathbf{a}_i \cdot \mathbf{G}_j = 2\pi \delta_{ij}$. In other words, the energy is invariant under translations by the reciprocal lattice vectors.

We consider composite bands labeled by the band index $n$ in a specific subset $\mathcal{V}$, which is separated from other bands by sufficiently large band gaps. In this case, we can construct a set of Wannier basis states $\{\vert\mathbf{r}_i\alpha\rangle\}$ that span the same sub-Hilbert space as the Bloch waves corresponding to the bands with indices $n\in\mathcal{V}$.
The Wannier basis states can be expressed as follows:
\begin{align}
\vert\mathbf{r}_i\alpha\rangle &= \frac{\mathcal{A}_\mathrm{uc}}{(2\pi)^d}\int_\mathrm{BZ} d^d\mathbf{k}\, e^{i\mathbf{k}\cdot(\mathbf{r}-\mathbf{r}_i)} \sum_{n\in\mathcal{V}} (\mathcal{U}_\mathbf{k})_{n,\alpha} \vert u_{n\mathbf{k}}\rangle, \label{eq:wannier_Bloch1} \\
\vert u_{n\mathbf{k}}\rangle &= \sum_{\mathbf{r}_i}\sum_\alpha e^{-i\mathbf{k}\cdot(\mathbf{r}-\mathbf{r}_i)} (\mathcal{U}_\mathbf{k}^\dagger)_{\alpha,n} \vert\mathbf{r}_i\alpha\rangle. \label{eq:wannier_Bloch2}
\end{align}
Here, $\mathcal{A}_\mathrm{uc}$ is the volume of the unit cell, and $\mathbf{r}_i$ represents a lattice site spanned by the lattice vectors $\mathbf{a}_i$ $(i=1,\cdots,d)$. The integration over momentum is performed over the first Brillouin zone (BZ). The unitary matrix $\mathcal{U}_\mathbf{k}$ is chosen to optimize the localization of the Wannier functions.
The Wannier function $\langle\mathbf{r}\vert\mathbf{r}_i\alpha\rangle \equiv w_\alpha(\mathbf{r}-\mathbf{r}_i)$ is localized around the lattice site $\mathbf{r}_i$. It turns out to be the Fourier transformation of the corresponding Bloch wave, and thus inherits the orthonormality properties of the Bloch functions.


The unitary matrix $\mathcal{U}_{\mathbf{k}}$ is chosen to maximize the localization of Wannier functions by minimizing a localization functional, as introduced by Marzari and Vanderbilt in their seminal work \cite{PhysRevB.56.12847}. The localization functional is given by
\begin{align}
F & =\sum_{\alpha\in\mathcal{V}}\left[\langle\mathbf{0}\alpha\vert r^{2}\vert\mathbf{0}\alpha\rangle-\vert\langle\mathbf{0}\alpha\vert\mathbf{r}\vert\mathbf{0}\alpha\rangle\vert^{2}\right]  =F_{I}+\delta F~. 
\end{align}
Both parts, $F_I$ and $\delta F$, are non-negative, where
\begin{align}
F_{I} & =\sum_{\alpha\in\mathcal{V}}\left[\langle\mathbf{0}\alpha\vert r^{2}\vert\mathbf{0}\alpha\rangle-\sum_{\mathbf{r}_{i}}\sum_{\beta}\vert\langle\mathbf{r}_{i}\beta\vert\mathbf{r}\vert\mathbf{0}\alpha\rangle\vert^{2}\right],\\
\delta F & =\sum_{\mathbf{r}_{i}(\neq\mathbf{0})}\sum_{\beta(\neq\alpha)}\vert\langle\mathbf{r}_{i}\beta\vert\mathbf{r}\vert\mathbf{0}\alpha\rangle\vert^{2}.
\end{align}
The optimization of the unitary matrix $\mathcal{U}_{\mathbf{k}}$ aims to minimize the localization functional $F$, leading to the construction of maximally localized Wannier functions.
The term $F_I$ is independent of the unitary transformation $\mathcal{U}_{\mathbf{k}}$ and therefore gauge invariant. This allows us to choose $\mathcal{U}_{\mathbf{k}}$ as an identity matrix with components $(\mathcal{U}_{\mathbf{k}})_{\alpha,n}=\delta_{\alpha,n}$ when calculating $F_I$.
Then from the relation in Eqs.~(\ref{eq:wannier_Bloch1}) and (\ref{eq:wannier_Bloch2}), we have
\begin{align}
\langle u_{n\mathbf{k}}\vert u_{m\mathbf{k+q}}\rangle & =\sum_{\mathbf{r}_{i}}e^{-i\mathbf{k}\cdot\mathbf{r}_{i}}\langle\mathbf{r}_{i}n\vert e^{-i\mathbf{q}\cdot\mathbf{r}}\vert\mathbf{0}m\rangle ,\label{eq:uuq}
\end{align}
By taking the derivative with respect to $\mathbf{q}$ on both sides of Eq.~(\ref{eq:uuq}), we obtain a series of relations in the limit $q\rightarrow 0$. For example, taking the first and second derivatives with respect to $\mathbf{q}$ gives
\begin{align}
\langle u_{n\mathbf{k}}\vert\nabla_{\mathbf{k}}u_{m\mathbf{k}}\rangle & =-i\sum_{\mathbf{r}_{i}}e^{-i\mathbf{k}\cdot\mathbf{r}_{i}}\langle\mathbf{r}_{i}n\vert\mathbf{r}\vert\mathbf{0}m\rangle ,\\
\langle u_{n\mathbf{k}}\vert\nabla_{\mathbf{k}}^{2}u_{m\mathbf{k}}\rangle & =-\sum_{\mathbf{r}_{i}}e^{-i\mathbf{k}\cdot\mathbf{r}_{i}}\langle\mathbf{r}_{i}n\vert\mathbf{r}^{2}\vert\mathbf{0}m\rangle,
\end{align}
Similarly, we can establish the converse relations
\begin{align}
\langle\mathbf{r}_{i}n\vert\mathbf{r}\vert\mathbf{0}m\rangle & =i\frac{\mathcal{A}_{\mathrm{uc}}}{(2\pi)^{d}}\int_\mathrm{BZ} d^{d}\mathbf{k}e^{i\mathbf{k}\cdot\mathbf{r}_{i}}\langle u_{n\mathbf{k}}\vert\nabla_{\mathbf{k}}u_{m\mathbf{k}}\rangle ,\\
\langle\mathbf{r}_{i}n\vert\mathbf{r}^{2}\vert\mathbf{0}m\rangle & =\frac{\mathcal{A}_{\mathrm{uc}}}{(2\pi)^{d}}\int_\mathrm{BZ} d^{d}\mathbf{k}e^{i\mathbf{k}\cdot\mathbf{r}_{i}}\langle\nabla_{\mathbf{k}}u_{n\mathbf{k}}\vert\nabla_{\mathbf{k}}u_{m\mathbf{k}}\rangle.
\end{align}
Therefore, we can simplify $F_{I}$ as 
\begin{align}
F_{I} & =\sum_{\alpha\in\mathcal{V}}\left[\langle\mathbf{0}\alpha\vert r^{2}\vert\mathbf{0}\alpha\rangle-\sum_{\mathbf{r}_{i}}\sum_{\beta}\vert\langle\mathbf{r}_{i}\beta\vert\mathbf{r}\vert\mathbf{0}\alpha\rangle\vert^{2}\right]\\
 & =\frac{\mathcal{A}_{\mathrm{uc}}}{(2\pi)^{d}}\int_\mathrm{BZ}d^{d}\mathbf{k}\sum_{n\in\mathcal{V}}\mathrm{Re}\langle\nabla_{\mathbf{k}}u_{n\mathbf{k}}\vert(\mathbb{I}_{\mathcal{V}}-\vert u_{n\mathbf{k}}\rangle\langle u_{n\mathbf{k}}\vert)\vert\nabla_{\mathbf{k}}u_{n\mathbf{k}}\rangle~,
\end{align}
where $\mathbb{I}_{\mathcal{V}}$ is the identity operator in the sub-Hilbert space spanned by bands carrying indices in $\mathcal{V}$. This expression clearly shows that $F_{I}$ is expressed in terms of the quantum metric, as defined in Eq.~(11) of the main text.
Since $\delta F\geq 0$, we have the inequality relation,
\begin{equation}
F\geq F_{I}~.
\end{equation}
Hence, we can conclude that the quantum metric characterizes an obstruction to finding a complete set of exponentially localized Wannier functions. 
When $F_{I}$ is finite, it indicates that more bands need to be included in the composite bands in order to construct a complete set of exponentially localized Wannier functions.


Another perspective on the quantum metric arises from considering a multiband tight-binding model. Assuming we have already obtained a complete set of exponentially localized Wannier functions constructed from composite bands, 
we can approximate the continuum Hamiltonian in Eq.~(\ref{eq:Hcont}) with a tight-binding model. In the language of second quantization, the continuum model in Eq.~(\ref{eq:Hcont}) can be expressed as
\begin{equation}
H=\int d^{d}\mathbf{r}\psi^{\dagger}(\mathbf{r})\left[-\frac{(\hbar\nabla)^{2}}{2m}+V(\mathbf{r})\right]\psi(\mathbf{r})~. \label{eq:2ndH}
\end{equation}
We then expand the field operator $\psi(\mathbf{r})$ in the basis
of Wannier functions 
\begin{equation}
\psi(\mathbf{r})=\sum_{\mathbf{r}_{i}}\sum_{\alpha\in\mathcal{V}}w_{\alpha}(\mathbf{r}-\mathbf{r}_{i})a_{i\alpha}+\sum_{\mathbf{r}_{i}}\sum_{\beta\in\mathcal{V}^{\perp}}w_{\beta}^{\perp}(\mathbf{r}-\mathbf{r}_{i})b_{i\beta}~,
\end{equation}
where $w_{\beta}^{\perp}(\mathbf{r}-\mathbf{r}_{i})$ denotes Wannier
functions associated with the complementary band set $\mathcal{V}^{\perp}$. 
By substituting the expansion into the Hamiltonian in Eq.~(\ref{eq:2ndH}), we can derive a tight-binding model defined on the lattice $\{\mathbf{r}_{i}\}$
\begin{align}
H & =\sum_{\alpha\beta\in\mathcal{V}}\sum_{\mathbf{r}_{i},\mathbf{r}_{j}}\langle\mathbf{r}_{i}\alpha\vert H\vert\mathbf{r}_{j}\beta\rangle a_{i\alpha}^{\dagger}a_{j\beta}^{}+\sum_{\alpha^{\prime},\beta^{\prime}\in\mathcal{V}^{\perp}}\sum_{\mathbf{r}_{i},\mathbf{r}_{j}}\langle\mathbf{r}_{i}\alpha^{\prime}\vert H\vert\mathbf{r}_{j}\beta^{\prime}\rangle b_{i\alpha^{\prime}}^{\dagger}b_{j\beta^{\prime}}^{}\nonumber \\
 & =\sum_{\alpha\beta\in\mathcal{V}}\sum_{\mathbf{r}_{i},\mathbf{r}_{j}}t_{ij,\alpha\beta}a_{i\alpha}^{\dagger}a_{j\beta}^{}+\sum_{\alpha^{\prime},\beta^{\prime}\in\mathcal{V}^{\perp}}\sum_{\mathbf{r}_{i},\mathbf{r}_{j}}t_{ij\alpha^{\prime}\beta^{\prime}}^{\perp}b_{i\alpha^{\prime}}^{\dagger}b_{j\beta^{\prime}}^{}~, \label{eq:tbhfull}
\end{align}
where no mixing term between indices from $\mathcal{V}$ and $\mathcal{V}^{\perp}$.
Up to this point, all the derivations have been rigorous, and the expression in Eq.~(\ref{eq:tbhfull}) includes all bands. However, since our interest lies solely in the bands belonging to $\mathcal{V}$, we can utilize a complete set of exponentially localized Wannier functions to approximate the Hamiltonian in Eq.~(\ref{eq:Hcont}) with a multi-band tight-binding model $H_{\mathrm{tb}}$ by disregarding the $t^\perp$ terms
\begin{align}
H_{\mathrm{tb}} & =\sum_{\alpha\beta\in\mathcal{V}}\sum_{\mathbf{r}_{i},\mathbf{r}_{j}}t_{ij,\alpha\beta}a_{i\alpha}^{\dagger}a_{j\beta}^{{}} =\sum_{\alpha\beta\in\mathcal{V}}\sum_{\mathbf{k}}h_{\alpha\beta}(\mathbf{k})a_{\mathbf{k}\alpha}^{\dagger}a_{\mathbf{k}\beta}^{{}},\label{eq:tb}
\end{align}
where $t_{ij,\alpha\beta}$ exponentially decays with the distance $\vert\mathbf{r}_{i}-\mathbf{r}_{j}\vert$. In Eq.~(\ref{eq:tb}), we have further introduced the Fourier transformation $a_{\mathbf{k}\alpha}^{\dagger}=\frac{1}{\sqrt{N}}\sum_{\mathbf{r}_{i}}a_{i\alpha}^{\dagger}e^{i\mathbf{k}\cdot\mathbf{r}_{i}}$, where $N$ represents the total number of lattice sites.
In our specific setup, where there is a significant gap between the targeted band and the others, we can project onto the targeted band using the following expressions
\begin{equation}
a_{i\alpha}\rightarrow\frac{1}{\sqrt{N}}\sum_{\mathbf{k}}e^{i\mathbf{k}\cdot\mathbf{r}_{i}}g_{\mathbf{k}}^{*}(\alpha)c_{\mathbf{k}}~,
\end{equation}
or 
\begin{equation}
\psi({\mathbf r})\rightarrow\frac{1}{\sqrt{N}}\sum_{\mathbf{k}}e^{i\mathbf{k}\cdot\mathbf{r}_{i}}g_{\mathbf{k}}^{*}(\alpha)c_{\mathbf{k}}~,
\end{equation}
where $g_{\mathbf{k}}$ represents an eigenvector of $h_{\alpha\beta}(\mathbf{k})$, and $c_{\mathbf{k}}$ annihilates an electron in the targeted band. It is important to note that the index $\alpha$ appearing in both $a_{i\alpha}$ and $g_{\mathbf{k}}(\alpha)$ arises from the realization of a multiband tight-binding model, which accounts for the nontrivial quantum metric or Wannier obstruction. This can be inferred from the quantum metric associated with $g_{\mathbf{k}}$ as described by Eq.~(11) in the main text.
%%%%%%%%%%
\subsection{Mean field for the Dirac fermions with pseudo magnetic field\label{app:meanfield}}
\subsubsection{TBG Flatbands and Harper model}
The electronic structure of twisted bilayer graphene (TBG) at the magic angle can be described as Dirac fermions experiencing opposite pseudomagnetic fields \cite{2019PhRvL.122j6405T,PhysRevB.99.155415}. 
The flatbands in TBG can be mapped to the zeroth pseudo Landau level 
(pLL), which can be effectively simulated using a time-reversal invariant Harper lattice model \cite{PhysRevB.14.2239,2015NatCo68944P}. This mapping provides analytical convenience for modeling the superconducting phase. In this context, the Hamiltonian for the two-flavor Dirac fermions in two spatial dimensions is given by
\begin{equation}
H_{0}=\sum_{\xi}\int d^{2}\mathbf{r}\Psi_{\xi}^{\dagger}(\mathbf{r})\left[(-i\nabla+\mathbf{A}_{\xi})\cdot\sigma_{\xi}\right]\Psi_{\xi}(\mathbf{r})~,\label{eq:Ham0J}
\end{equation}
In Eq.~(\ref{eq:Ham0J}), $\Psi_{\xi}=[a_{\xi},b_{\xi}]^{T}$ is a two-component spinor representing the two sublattices $a$ and $b$, $\xi=\pm$ corresponds to the two flavor degrees of freedom, and $\sigma_{\xi}$ denotes the Pauli matrix $\sigma_{\xi}=(\xi\sigma_{x},\sigma_{y})$. The uniform pseudomagnetic field is represented by the gauge field $\mathbf{A}_{\xi}=\xi\mathbf{A}$, and the TBG flatbands are mapped to the induced zeroth pseudo Landau levels (pLL). Without loss of generality, we set $\Psi_{\xi}\propto[1,0]$ with the $a$-sublattice being occupied, allowing us to obtain the wave functions of the zeroth pLL.


Therefore, we can consider the Harper model with Hamiltonian
$H=-\sum_{\mathbf{r},\mathbf{r}^{\prime}}\sum_{\xi}t_{\mathbf{r},\mathbf{r}^{\prime}}^{\xi}a_{\xi}^{\dagger}(\mathbf{r})a_{\xi}^ {}(\mathbf{r}^{\prime})$,
where $a_{\xi}(\mathbf{r})$ and $a_{\xi}^{\dagger}(\mathbf{r})$ are the annihilation and creation operators for electrons with flavor $\xi=\pm$ at lattice site $\mathbf{r}$, respectively. The hopping matrix $t_{\mathbf{r},\mathbf{r}^{\prime}}^{\xi}$ describes the hopping processes between nearest-neighbor sites and is given by
\begin{equation}
t_{\mathbf{r}\mathbf{r^{\prime}}}^{\xi}\!=\!\omega^{\xi r_{y}}\delta_{\mathbf{r}-\mathbf{e}_{x},\mathbf{r^{\prime}}}\!+\!\omega^{-\xi r_{y}}\delta_{\mathbf{r}+\mathbf{e}_{x},\mathbf{r^{\prime}}}\!+\!\delta_{\mathbf{r}-\mathbf{e}_{y},\mathbf{r}^{\prime}}\!+\!\delta_{\mathbf{r}+\mathbf{e}_{y},\mathbf{r^{\prime}}},
\end{equation}
where $\mathbf{e}_{x}$ and $\mathbf{e}_{y}$ are the unit vectors along the $x$ and $y$ directions, respectively. The factor $\omega^{\pm\xi j_{y}}$ introduces a lattice version of the Landau gauge, and we consider a uniform commensurate flux $\Phi =\frac{2\pi}{N_{\mathrm{o}}}$, where $\omega=e^{i\Phi}$, such that $B a^{2}=\frac{2\pi}{N_{\mathrm{o}}}$, with $B$ being the pseudomagnetic field. The lattice has $N_{c}\times N_{\mathrm{o}}$ sites, with $N_{c}$ being the number of super unit cells and $N_{\mathrm{o}}$ being the number of orbitals.
In the reduced Brillouin zone (BZ) with $q_{x}\in[-\pi/(aN_{\mathrm{o}}),\pi/(aN_{\mathrm{o}})]$ and $q_{y}\in[-\pi/a,\pi/a]$, the Bloch functions $g_{\mathbf{q},\xi}(\alpha)$ for the zeroth pseudo Landau Level (pLL) can be approximated as
\begin{align}
g_{\mathbf{q},\xi}(\alpha) & \sim\sum_{s}e^{-iq_{x}(\alpha-N_{\mathrm{o}}s)a}\phi_{0}(\alpha-sN_{\mathrm{o}}-\xi\frac{N_{\mathrm{o}}q_{y}a}{2\pi}).\label{eq:Blockgk}
\end{align}
where $\phi_{0}(r)$ is a Gaussian function and $(\mathbf{r}_{c},\alpha)$ denotes the lattice site $\mathbf{r}$ with orbital index $\alpha$. 



\subsubsection{BCS mean field on the continuum model}
In the BCS mean-field theory applied to the continuum model $H_0$ in Eq.~\eqref{eq:Ham0J}, we consider the symmetry gauge with $\mathbf{A}_\xi(\mathbf{r})=\xi\frac{1}{2}B(y,-x)$, where $B$ represents the strength of the pseudomagnetic field. At the zeroth pseudo Landau Level (pLL), all electrons reside on the A-sublattice, and an attractive interaction can be described by the interaction term
\begin{equation}
H_{\mathrm{int}}=-g \int d^{2}\mathbf{r}a_{+}^{\dagger}(\mathbf{r})a_{-}^{\dagger}(\mathbf{r})a_{-}^ {}(\mathbf{r})a_{+}^ {}(\mathbf{r}).
\end{equation}
where $g$ represents the strength of the interaction. To proceed, we make a mean-field ansatz by introducing the order parameter $\Delta(\mathbf{r})$, defined as:
\begin{equation}
\Delta(\mathbf{r})= \langle a_{-}(\mathbf{r})a_{+}(\mathbf{r})\rangle=\sum_{n}\phi_{n+}(\mathbf{r})\mathbf{\phi}_{n-}(\mathbf{r})\langle c_{n-}c_{n+}\rangle,\label{eq:meanfield-1}
\end{equation}
where $\phi_{n\xi}(\mathbf{r})$ represents the wave function of an electron with angular momentum $L_z=\xi n$. The mean-field ansatz allows for a coordinate-dependent order parameter $\Delta(\mathbf{r})$. The time-reversal symmetry invariance guarantees that the localization centers of two electrons in a Cooper pair coincide.
The BdG equation takes the form as 
\begin{equation}
H_{\mathrm{BdG}}=-g\int d^{2}\mathbf{r}\Delta(\mathbf{r})[a_{+}^{\dagger}(\mathbf{r})a_{-}^{\dagger}(\mathbf{r})+a_{-}(\mathbf{r})a_{+}(\mathbf{r})]~,
\end{equation}
where we choose a gauge such that $\Delta$ is real.


In the large pseudomagnetic field ($B$) limit, we can project electrons onto the zeroth pseudo Landau Levels (pLLs) through a truncated expansion. We express the electron operators $a_{+}(\mathbf{r})$ and $a_{-}(\mathbf{r})$ in terms of the zeroth pLL wave functions
\begin{align}
a_{+}(\mathbf{r}) & =\sum_{n}\phi_{0n+}(\mathbf{r})c_{n+},\\
a_{-}(\mathbf{r}) & =\sum_{n}\phi^*_{0n-}(\mathbf{r})c_{n-},
\end{align}
where $c_{n\xi}^{\dagger}$ creates an electron with angular momentum $L_z=\xi n$ at the valley $\xi$, and $\phi_{0n\xi}(\mathbf{r})$ represents the wave function of the electron in the zeroth pLL, characterized by a localization center $\sqrt{2n}\ell_{0}$. The operators $c_{n\pm}$ are associated with the electrons in the zeroth pLLs. By substituting these expressions into the Hamiltonian, we can reformulate the BdG Hamiltonian as
\begin{align}
H_{\mathrm{BdG}}= & -\sum_{n}\mu(c_{n+}^{\dagger}c_{n+}^ {}+c_{n-}^{\dagger}c_{n-}^ {}) -\sum_{n}g\Delta_{n}(c_{n+}^{\dagger}c_{n-}^{\dagger}+c_{n-}c_{n+}),\label{eq:BdG}
\end{align}
where $\mu$ is the chemical potential and $\Delta_{n}$ is the pairing potential given by
\begin{equation}
\Delta_{n}=\int d^{2}\mathbf{r}\phi_{0n+}\Delta(\mathbf{r})\phi^*_{0n-},\quad n=0,1,2,\cdots.
\end{equation}
We can expect a uniform order parameter for any angular momentum. We can diagonalize the BdG Hamiltonian in Eq.~(\ref{eq:BdG}) using the Bogoliubov transformation,
\begin{equation}
c_{m+}=u_{m}\gamma_{m+}-v_{m}\gamma_{m-}^{\dagger} ~,\quad
c_{m-}^{\dagger}=v_{m}\gamma_{m+}+u_{m}\gamma_{m-}^{\dagger}~,
\end{equation}
with 
\begin{equation}
u_{n}=\frac{1}{\sqrt{2}}\sqrt{1-\frac{\mu}{\epsilon_{n}}},v_{n}=\frac{1}{\sqrt{2}}\sqrt{1+\frac{\mu}{\epsilon_{n}}}.
\end{equation}
The dispersion for the quasiparticles is 
\begin{equation}\epsilon_{n}=\sqrt{\mu^{2}+|g\Delta_{n}|^{2}}~.\end{equation}
Meanwhile, we can determine the order parameter by the self-consistent
gap equation, 
\begin{equation}
\Delta(\mathbf{r})=\sum_{n=0}\phi_{0n+}^{*}(\mathbf{r})\phi_{0n-}(\mathbf{r})u_{n}v_{n}\tanh\frac{\beta\epsilon_{n}}{2}.\label{eq:self-eq1}
\end{equation}
Equivalently for $\Delta_{n}$, we have the gap equations as
\begin{equation}
\Delta_{m}=\sum_{n}\mathsf{K}_{nm}u_{n}v_{n}\tanh\frac{\beta\epsilon_{n}}{2},\label{eq:self-eq2-1}
\end{equation}
with 
\begin{equation} 
\mathsf{K}_{nm}=\int d^{2}\mathbf{r}\phi_{0n+}\phi_{0n-}^{*}\phi_{0m+}\phi_{0m-}^{*}.
\end{equation}
The chemical potential gets shifted according to the number equation
\begin{equation}
2\nu=\lim_{N\rightarrow\infty}\frac{1}{N}\left(\sum_{m=0}^{N}\langle c_{m+}^{\dagger}c_{m+}^ {}\rangle+\sum_{m=0}^{N}\langle c_{m-}^{\dagger}c_{m-}^ {}\rangle\right) = 1+\frac{\mu}{\epsilon}\tanh\frac{\beta\epsilon}{2}.\label{eq:numerEq-1}
\end{equation}
Here $2\nu N=2\nu BS/2\pi$ is the total numbers of electrons
occupying the zeroth pLLs with $S$ the sample area. 
By coupling the equations \eqref{eq:self-eq1} and \eqref{eq:numerEq-1}, we can obtain the pairing order parameter and renormalized chemical potential.

The simplified self-consistent equation for the mean field $\Delta(\mathbf{r})$ at zero temperature ($T=0$) allows for an analytical solution. Due to the translational symmetry, the order parameter is constant throughout the system, $\Delta(\mathbf{r}) \equiv \Delta_{0}(T=0)$. This yields an analytical expression for the mean field at $T=0$, given by
\begin{equation}
\Delta_{0}(T=0) = \frac{\sqrt{\nu(1-\nu)}}{2\pi\ell_0^{2}},
\end{equation}
where $\nu$ is the filling factor and $\ell_0$ is the magnetic length.
In general cases, numerical methods are required to solve the self-consistent gap equation and find the mean field $\Delta(\mathbf{r})$ or $\Delta_{n}$. The critical temperature $T_{\mathrm{MF}}$ and its relation to the magnetic field $B_{\mathrm{r}}$ are of interest. Near $T_{\mathrm{MF}}$, the gap equations can be linearized, resulting in an eigenvalue problem for the matrix $\mathsf{K}_{nm}$:
\begin{equation}
\Delta_{m} = \frac{g\bar{\nu}\beta}{4}\sum_{n}\mathsf{K}_{nm}\Delta_{n},
\end{equation}
where $\bar{\nu} = \frac{2(1-2\nu)}{\ln(\nu^{-1}-1)}$ and $\beta$ is the inverse temperature.
The critical temperature $T_{\mathrm{MF}}$ is determined by the condition that the maximal eigenvalue of $\mathsf{K}_{nm}$ satisfies the linearized gap equation. The matrix $\mathsf{K}_{nm}$ has an eigenvector $\Delta_{n} = 1$ (for $n=0,1,2,\dots$) with the eigenvalue $\frac{\ell_0^{2}}{2}$, which is the dominant eigenvalue according to the Perron-Frobenius theorem.
From this analysis, we identify the critical temperature $T_{\mathrm{BCS}}$ that is linearly dependent on the magnetic field $B$
\begin{equation}
T_{\mathrm{MF}} = \bar{\nu}\tau_{c},
\end{equation}
where $\tau_{c} = \frac{g}{8\pi\ell_{0}^{2}}$ is the critical temperature and $\nu = \frac{1}{2}$.
Remarkably, the maximum $T_{\mathrm{MF}}$ for superconductivity occurs at half-filling, i.e., $\mu=0$, $\nu=1/2$. In this case, the quasiparticle spectrum remains flat. However, it is worth noting that a dispersive quasiparticle band can appear when an external magnetic field is applied.
Additionally, the characteristic quantity $g\Delta_0(T=0)/T_{\mathrm{MF}} = 4\bar\nu^{-1}\sqrt{\nu(1-\nu)}$, and around half-filling $\nu=1/2$, $g\Delta_0(T=0)/T_{\mathrm{MF}} = 2$. This value is larger than that of a conventional BCS superconductor with a large Fermi velocity.






\subsubsection{The mean field and BKT transition}
In the effective theory for the Goldstone mode, we consider a mean-field value $\Delta_0(T)$ at temperature $T$ that satisfies the self-consistent equations given in Eq.~\eqref{eq:self-eq1} or Eq.~(6) in the main text. For an isotropic system with a flat quasiparticle band, we collect the self-consistent equations necessary to determine the BKT transition temperature
\begin{align}
\epsilon  & =  2 \tau_c \tanh \frac{\beta_\mathrm{BKT} \epsilon}{2}~,\\
2\nu  & = 1+\frac{\mu}{\epsilon}\tanh\frac{\beta_\mathrm{BKT}\epsilon}{2}~,\\
T_{\mathrm{BKT}} &=\frac{\pi}{8}g\Delta_{0}(T_{\mathrm{BKT}})\frac{\sqrt{\mathrm{det}\bar{\gamma}_{2}^{ab}}}{\mathcal{A}_{\mathrm{uc}}}
\end{align}
where $\beta_\mathrm{BKT} = T_\mathrm{BKT}^{-1}$ and $\epsilon = \sqrt{|\Delta_0(T_\mathrm{BKT})|^2 + \mu^2}$. The quantum metric $\sqrt{\mathrm{det}\bar{\gamma}_{2}^{ab}}$ appears in the expression for the BKT transition temperature.
In general, these equations need to be solved numerically. However, when the quantum metric $\sqrt{\mathrm{det}\bar{\gamma}_{2}^{ab}}$ is small, such that $T\mathrm{BKT}$ is much smaller than $T_\mathrm{MF}$, we can approximate $\Delta_0(T_\mathrm{BKT})$ with the pairing gap e.g. $\Delta_0$ at $T =0 $ to approximate $\Delta_0(T_\mathrm{BKT})$. In this case, a good approximation for the BKT transition temperature is given by
\begin{equation}
T_\mathrm{BKT} = \frac{T_\mathrm{MF}}{2} \frac{\sqrt{\mathrm{det}\bar{\gamma}_{2}^{ab}}}{\mathcal{A}{\mathrm{uc}}},\quad \text{for small } \sqrt{\mathrm{det}\bar \gamma_2^{ab}}  .
\end{equation}
It is important to note that the quantum metric $\sqrt{\mathrm{det}\bar{\gamma}_{2}^{ab}}$ possesses its own independent degree of freedom and can be tuned to be sufficiently large. As a result, the BKT transition temperature can become comparable to $T_\mathrm{MF}$.
To further illustrate this point, we can linearize the self-consistent equations at half-filling ($\mu = 0$) for simplicity. Using the expansion $\tanh x = x - \frac{1}{3}x^3 + \mathcal{O}(x^5)$, we obtain the equation
\begin{align}
g\Delta_0(T_\mathrm{BKT}) & = 2\tau_0\left[ \frac{\beta_{BKT}g\Delta_0(T_\mathrm{BKT})}{2}- \frac{1}{3}\left(\frac{\beta_{BKT}g\Delta_0(T_\mathrm{BKT})}{2}\right)^3\right] ~,\\
T_{\mathrm{BKT}} &=\frac{\pi}{8}g\Delta_{0}(T_{\mathrm{BKT}})\frac{\sqrt{\mathrm{det}\bar{\gamma}_{2}^{ab}}}{\mathcal{A}_{\mathrm{uc}}}.
\end{align}
and one can solve them with a solution 
\begin{equation}
T_{\mathrm{BKT}} =\frac{12\alpha^{2}-1}{12\alpha^{2}}T_\mathrm{MF},\quad \Delta_{0}(T_{\mathrm{BKT}})=\frac{12\alpha^{2}-1}{12\alpha^{3}}T_\mathrm{MF},
\end{equation}
with $\alpha \equiv \frac{\pi}{8}\frac{\sqrt{\mathrm{det}\bar{\gamma}_{2}^{ab}}}{\mathcal{A}_{\mathrm{uc}}}$. 
It is evident that in the limit $\alpha\rightarrow \infty$, the BKT transition temperature $T_\mathrm{BKT}$ approaches the mean-field transition temperature $T_\mathrm{MF}$
\begin{equation}
T_\mathrm{BKT} \rightarrow T_\mathrm{MF}, \quad \text{for large }  \sqrt{\mathrm{det}\bar \gamma_2^{ab}} 
\end{equation}
 This observation holds true for arbitrary fillings as well.

\subsection{Harper model and the effective GL theory \label{sec:Harper-model-and}}
We can employ a time-reversal invariant (TRI) Harper lattice model to regularize the zeroth Landau level (pLL) in a system with $N = N_{c} \times N_{\mathrm{o}}$ lattice sites, where $N_{c}$ is the number of super unit cells and $N_{\mathrm{o}}$ is the number of orbitals per super unit cell.

We can regularize the zeroth pLL using a TRI Harper lattice model  on
a system $N=N_{c}\times N_{\mathrm{o}}$ with $N_{c}$ super unit
cells and $N_{\mathrm{o}}$ orbitals.
To facilitate this regularization, we relabel the original lattice site as $\mathbf{r} = (\mathbf{r}_{c},\alpha)$, where $\mathbf{r}_{c}$ represents the super unit cell and $\alpha$ denotes the orbital index. By introducing multi-band fermion operators $a_{\xi}(\mathbf{r}_{c},\alpha)$, which annihilate fermions at the specified super unit cell $\mathbf{r}_{c}$ and orbital $\alpha$, we can effectively describe the fermionic degrees of freedom on the TRI Harper lattice.
In this framework, we can reformulate the attractive interaction given in Eq.~(1) of the main text as
\begin{equation}
H_{\mathrm{int}}=-g\sum_{\mathbf{r}_{c},\alpha}a_{+}^{\dagger}(\mathbf{r}_{c},\alpha)a_{-}^{\dagger}(\mathbf{r}_{c},\alpha)a_{-}^ {}(\mathbf{r}_{c},\alpha)a_{+}^ {}(\mathbf{r}_{c},\alpha)~.
\end{equation}
In this context, the auxiliary field $\Delta_{\alpha}(\mathbf{r}_{c})$ now acquires a dependence on the orbital index, given by
\begin{equation}
\Delta_{\alpha}(\mathbf{r}_{c})=a_{-}(\mathbf{r},\alpha)a_{+}(\mathbf{r},\alpha)~.
\end{equation}
By applying the Hubbard-Stratonovich transformation, we can express the interaction part of the Lagrangian within the path integral framework
\begin{align}
L_{\mathrm{int}}[a,\bar{a},\Delta,\bar{\Delta}] &= -\sum_{\mathbf{r}_c,\alpha}g\left[\Delta_{\alpha}(\mathbf{r}_c)\bar{a}_{+}(\mathbf{r}_c,\alpha)\bar{a}_{-}(\mathbf{r}_c,\alpha)+\text{h.c.}\right] \nonumber \\
&= -\sum_{\mathbf{k},\mathbf{q},\alpha}g\left[\Gamma_{\alpha}(\mathbf{q},\mathbf{k})\Delta_{\alpha}(\mathbf{k})\bar{c}_{\mathbf{k}+\frac{\mathbf{q}}{2},+}\bar{c}_{-\mathbf{k}+\frac{\mathbf{q}}{2},-}+\text{h.c.}\right]~.
\end{align}
In the second line, we perform a projection onto the zeroth pLL
\begin{equation}
a_{\xi}(\mathbf{r}_c,\alpha) \rightarrow \sum_{\mathbf{k}}e^{-i\mathbf k\cdot\mathbf r_c}g_{\mathbf{k},\xi}^{*}(\alpha)c_{\mathbf{k}\xi}~,
\end{equation}
where
\begin{equation}
\Gamma_{\alpha}(\mathbf{q},\mathbf{k}) = g_{\mathbf{k}+\frac{\mathbf{q}}{2},+}(\alpha)g_{-\mathbf{k}+\frac{\mathbf{q}}{2},-}(\alpha)~.
\end{equation}
The Bloch wave $g_{\mathbf{k},\xi}(\alpha)$ captures the orbital dependence, and the normalization condition $\frac{1}{N_c}\sum_{\mathbf{k}}\vert g_{\mathbf{k}}(\alpha)\vert^2 = \frac{1}{N_{\mathrm{orb}}}$ is satisfied, taking into account the enlargement of the unit cell. 
Combining the interaction Lagrangian $L_{\mathrm{int}}$ with the free fermion part $L_{0} = \sum_{\mathbf{k}\xi}(-i\omega-\mu)\bar{c}_{\mathbf{k}\xi}c_{\mathbf{k}\xi}$, we obtain the total Lagrangian $L = L_{0} + L_{\mathrm{int}}$.



We can expand the bosonic field around the mean field configuration:
\begin{equation}
\Delta_{\alpha}(\mathbf{k}) = \Delta_{\alpha0}\delta_{\mathbf{k}0} + \delta\Delta_{\alpha}(\mathbf{k})~,
\end{equation}
which decomposes the Lagrangian $L$ into two parts. The first part corresponds to the BCS mean field and can be expressed as:
\begin{align}
\mathcal{L}_0[c,\bar{c}] &= (-i\omega-\mu)(\bar{c}_{\mathbf{q},+}c_{\mathbf{q},+}+\bar{c}_{\mathbf{q},-}c_{\mathbf{q},-}) -\sum_{\alpha}\left[\Gamma_{\alpha}(\mathbf{q})\Delta_{\alpha}\bar{c}_{\mathbf{q},+}\bar{c}_{-\mathbf{q},-}+\text{h.c.}\right]~,
\end{align}
where $\Gamma_{\alpha}(\mathbf{q})\equiv\Gamma_{\alpha}(\mathbf{q},0)$.
From this, we can derive the Bogoliubov quasiparticle dispersion relation $\epsilon(\mathbf{q})=\sqrt{\left|g\sum_{\alpha}\Gamma_{\alpha}(\mathbf{q})\Delta_{\alpha0}\right|^{2}+\mu^{2}}$ as well as the self-consistent equation for $\Delta_{\alpha0}$ given by:
\begin{equation}
\Delta_{\alpha0}=\frac{1}{N_{c}}\sum_{\mathbf{q}}\sum_{\gamma}\Delta_{\gamma0}\frac{\Gamma_{\alpha}(\mathbf{q})\Gamma_{\gamma}^{*}(\mathbf{q})}{2\epsilon_{\mathbf{q}}}\tanh\frac{\beta\epsilon_{\mathbf{q}}}{2}~.
\end{equation}
We can neglect the fluctuations in the orbital indices and make the approximation:
\begin{equation}
\Delta_{\alpha}(\mathbf{k})\equiv\Delta(\mathbf{k}),\quad \Delta_{\alpha0}\equiv\Delta_{0}\quad\forall\alpha~,
\end{equation}
which simplifies the calculations while capturing the main physics. By integrating out the fermion fields and considering the phase fluctuations $\delta\Delta(\mathbf{r})=\Delta_{0}e^{2i\theta(\mathbf{r})}-\Delta_{0}\simeq2i\theta(\mathbf{r})\Delta_{0}$ (ignoring amplitude fluctuations), we obtain the effective action for the phase fluctuations:
\begin{equation}
\mathcal{L}[\theta]=\frac{1}{2}\mathsf{D}_s(\nabla \theta)^2~,
\end{equation}
where $\mathsf{D}_s$ is called superfluid weight that characterizes the stiffness of the phase fluctuations.
For the zeroth pLL, the factors are given by $\gamma_{0}(\mathbf{q})=1$ and $\gamma_{2}^{ab}(\mathbf{q})=\frac{N_{\mathrm{o}}a^{2}}{4\pi}\delta_{ab}$ \cite{2021PhRvB.104d5103O,2015NatCo68944P}.
At temperature $T=0$, according to Eq.~(13) in the main text, the superfluid weight is defined as $\mathsf{D}_{s}\equiv\frac{(g\Delta_{0})^{2}}{2\pi\epsilon}$. For filling around $\mu=0$, the ratio $\mathsf{D}_{s}/(g\Delta_{0})\sim0.16$, which is in good agreement with experimental results \cite{2023Natur.614..440T}.



\subsection{Model for twisted bilayer graphene\label{sec:Model-for-twisted}}
For the twisted bilayer graphene (TBG) system \cite{2011PNAS..10812233B}, we consider the AA-stacking configuration where the two layers $\ell=1$ and $\ell=2$ are rotated around a pair of registered B sites by angles $-\theta/2$ and $+\theta/2$, respectively. 
Before rotation, the lattice vectors are denoted as $\mathbf{a}_{1}=a(1,0)$ and $\mathbf{a}_{2}=a(1/2,\sqrt{3}/2)$ for the AA-stacked bilayer, where $a=0.246$ nm is the lattice constant of graphene. The corresponding reciprocal lattice vectors are $\mathbf{g}_{1}=\frac{2\pi}{a}(1,-1/\sqrt{3})$ and $\mathbf{g}_{2}=\frac{2\pi}{a}(0,2/\sqrt{3})$.
After rotation, the lattice vectors $\mathbf{a}^{(\ell)}$ of the two layers $\ell=1,2$ can be constructed as $\mathbf{a}_{i}^{(1)}=R(-\theta/2)\mathbf{a}_{i}$ and $\mathbf{a}_{i}^{(2)}=R(+\theta/2)\mathbf{a}_{i}$, where $R(\theta)$ is the rotation matrix with an angle $\theta$. Similarly, the reciprocal lattice vectors for the two layers after rotation are given by $\mathbf{g}_{i}^{(1)}=R(-\theta/2)\mathbf{g}_{i}$ and $\mathbf{g}_{i}^{(2)}=R(+\theta/2)\mathbf{g}_{i}$.
When the rotation angle $\theta$ is small, the TBG exhibits a moir\'e pattern with a very long period. The reciprocal lattice vectors for the moiré pattern are given by
\begin{equation}
\mathbf{G}_{i}^{\mathrm{m}}=\mathbf{g}_{i}^{(1)}-\mathbf{g}_{i}^{(2)}~.
\end{equation}
The real-space lattice vectors $\mathbf{L}_{j}^{\mathrm{m}}$ satisfy the condition $\mathbf{G}_{i}^{\mathrm{m}}\cdot\mathbf{L}_{j}^{\mathrm{m}}=2\pi\delta_{ij}$, and the moiré lattice constant $L_{\mathrm{m}}$ is given by $L_{\mathrm{m}}=\vert\mathbf{L}_{1}^{\mathrm{m}}\vert=\vert\mathbf{L}_{2}^{\mathrm{m}}\vert=\frac{a}{2\sin\left(\theta/2\right)}$.


At small angles, such as $\theta=1.08^\circ$ as considered in the main text, an effective continuum model can be applied. In this regime, intervalley mixing can be neglected, allowing for the block-diagonalization of the Hamiltonian. For a given valley $\rho$ and spin $\sigma$, the effective Hamiltonian $H^{(\rho)}$ takes the form
\begin{equation}
H^{(\sigma\rho)}=H_{\mathrm{kin}}^{1}+H_{\mathrm{kin}}^{2}+H_{\mathrm{kin}}^{\perp}-\mu N+H_{\mathrm{int}}~. \label{app:Hrho}
\end{equation}
The first two terms, $H_{\mathrm{kin}}^{\ell}$ ($\ell=1,2$), describe the intralayer hopping and can be approximated by the two-dimensional Weyl equation
\begin{align}
H_{\mathrm{kin}}^{\ell} & =-\int d^{2}\mathbf{r} v_{F}\psi_{\sigma\rho,\ell}^{\dagger}(\mathbf{r})\left[R(\mp\theta/2)(-i\nabla-\mathbf{K}_{\rho}^{(\ell)})\right]\cdot\boldsymbol{\sigma}^{\rho}\psi_{\sigma\rho,\ell}(\mathbf{r})\nonumber \\
 & =-\int_{\mathrm{mBZ}}\frac{d^{2}\mathbf{k}}{(2\pi)^{2}} v_{F}\psi_{\sigma\rho,\ell}^{\dagger}(\mathbf{k})\left[R(\mp\theta/2)(\mathbf{k}-\mathbf{K}_{\rho}^{(\ell)})\right]\cdot\boldsymbol{\sigma}^{\rho}\psi_{\sigma\rho,\ell}(\mathbf{k})~.
\end{align}
where $\mathbf{K}_{\rho}^{(\ell)}=-\rho[2\mathbf{g}{1}^{(\ell)}+\mathbf{g}{2}^{(\ell)}]/3$ for layer $\ell$ and valley $\rho$, and $(\mp)$ corresponds to $\rho=1(2)$ in $R(\mp\theta/2)$. Here, mBZ denotes the Brillouin zone for the moir\'e lattice, and the Fermi velocity is $v_{F}=7.98\times10^{5}$ m/s. The vector $\psi_{\sigma\rho,\ell}(\mathbf{r})$ represents a sublattice space vector, given by $\psi_{\sigma\rho,\ell}(\mathbf{r})=\begin{bmatrix}\psi_{\sigma\rho,\ell A}(\mathbf{r}) & \psi_{\sigma\rho,\ell B}(\mathbf{r})\end{bmatrix}^{T}$, and the fermion operator $\psi_{\sigma\rho,\ell\xi}(\mathbf{r})$ annihilates an electron with spin index $\sigma$, layer $\ell$, and sublattice $\xi$ at the valley $\rho$. The vector $\boldsymbol{\sigma}^{\rho}=(\rho\sigma_{x},\sigma_{y})$ contains the Pauli matrices in the sublattice system $(\rho=\pm)$.
The third part $H_{\mathrm{kin}}^{\perp}$ describes the effective interlayer hopping. It can be written as
\begin{align}
H_{\mathrm{kin}}^{\perp} & =\int d^{2}\mathbf{r}\begin{bmatrix}\psi_{\sigma\rho,1A}^{\dagger}(\mathbf{r}) & \psi_{\sigma\rho,1B}^{\dagger}(\mathbf{r})\end{bmatrix}T_{12}(\mathbf{r})\begin{bmatrix}\psi_{\sigma\rho,2A}(\mathbf{r})\\
\psi_{\sigma\rho,2B}(\mathbf{r})
\end{bmatrix}+h.c.\\
 & =\int d^{2}\mathbf{r}\sum_{\ell,\xi\xi^{\prime}}\psi_{\sigma\rho,\ell\xi}^{\dagger}(\mathbf{r})T_{\ell\bar{\ell},\xi\xi^{\prime}}(\mathbf{r})\psi_{\sigma\rho,\bar{\ell}\xi^{\prime}}(\mathbf{r})~,
\end{align}
where $\bar{\ell}$ denotes the opposite layer to $\ell$. The elements of the $2\times2$ matrix $[T_{12}(\mathbf{r})]_{\xi\xi^{\prime}}=T_{12,\xi\xi^{\prime}}(\mathbf{r})$ have the following form
\begin{align}
T_{12}(\mathbf{r})= & \begin{bmatrix}t_{AA} & t_{AB}\\
t_{BA} & t_{BB}
\end{bmatrix}+\begin{bmatrix}t_{AA} & t_{AB}\omega^{-\xi}\\
t_{BA}\omega^{\xi} & t_{BB}
\end{bmatrix}e^{i\xi\mathbf{G}_{1}^{\mathrm m}\cdot\mathbf{r}} +\begin{bmatrix}t_{AA} & t_{AB}\omega^{\xi}\\
t_{BA}\omega^{-\xi} & t_{BB}
\end{bmatrix}e^{i\xi(\mathbf{G}_{1}^{\mathrm m}+\mathbf{G}_{2}^{\mathrm m})\cdot\mathbf{r}}~.
\end{align}
where $\omega=e^{2\pi i/3}$. 
The TBG system exhibits a corrugation effect, causing variations in the interlayer spacing. Consequently, the model parameters deviate from the condition of a flat TBG, where $t_{AA}=t_{AB}=t_{BA}=t_{BB}$. In the optimized lattice structure of TBG, the corrugation occurs in the out-of-plane direction, enabling the use of different parameter values: $t_{AA}=t_{BB}=79.7$~meV and $t_{AB}=t_{BA}=97.5$~meV, within the effective model. This discrepancy between $t_{AA}(t_{BB})$ and $t_{AB}(t_{BA})$ introduces an energy gap between the lowest bands and excited bands, which is consistent with experimental observations \cite{PhysRevX.8.031087}.
The third term in Eq.~(\ref{app:Hrho}) corresponds to the chemical potential, denoted as
\begin{equation}
-\mu N = -\mu \int d^{2}\mathbf{r} \sum_{\sigma\rho\ell\xi} \psi_{\sigma\rho,\ell\xi}^{\dagger}(\mathbf{r}) \psi_{\sigma\rho,\ell\xi}(\mathbf{r}),
\end{equation}
where $N$ represents the total number operator, and $\mu$ is the chemical potential that controls the filling condition of the system.
The last term in the equation represents an attractive interaction as previously studied in \cite{PhysRevB.101.060505,PhysRevLett.123.237002}
\begin{equation}
H_{\mathrm{int}} = -g\int d^{2}\mathbf{r} \sum_{\sigma_1\sigma_2\rho\ell\xi} \psi_{\sigma_1\rho,\ell\xi}^{\dagger}(\mathbf{r}) \psi_{\sigma_2\rho,\ell\xi}^{\dagger}(\mathbf{r}) \psi_{\sigma_2\rho,\ell\xi}(\mathbf{r}) \psi_{\sigma_1\rho,\ell\xi}(\mathbf{r}),
\end{equation}
where $g$ represents the effective attractive interaction potential. The operators $\psi_{\sigma_1\rho,\ell\xi}(\mathbf{r})$ and $\psi_{\sigma_2\rho,\ell\xi}(\mathbf{r})$ annihilate electrons with spin indices $\sigma_1$ and $\sigma_2$, respectively, within the same layer $\ell$, sublattice $\xi$, and valley $\rho$. 


The mean field theory with an order parameter $\Delta_{\rho,\ell\xi}(\mathbf{r})=\langle\psi_{\downarrow\bar{\rho},\ell\xi}(\mathbf{r})\psi_{\uparrow\rho,\ell\xi}(\mathbf{r})\rangle$ has been extensively investigated in Refs.~\cite{PhysRevB.101.060505,PhysRevLett.123.237002}. However, our current focus is on the critical region. In particular, we are interested in the energy level depicted in Fig.~1(a), which exhibits two nearly flat bands associated with a specific valley $\rho$ and spin $\sigma$.
To study these flat bands, we introduce a projection onto these states, labeled as $n=1,2$. This projection is achieved by transforming the electron field operator:
\begin{equation}
\psi_{\sigma\rho,\ell\xi}(\mathbf{r})\rightarrow\int_{\mathrm{mBZ}}\frac{d^{2}\mathbf{k}}{(2\pi)^{2}}e^{-i\mathbf{k}\cdot\mathbf{r}}\sum_{n=1,2}g^*_{\sigma\rho,n\mathbf{k}}(\alpha)c_{\sigma\rho,n}(\mathbf{k}),
\end{equation}
where $c_{\sigma\rho,n}(\mathbf{k})$ is the annihilation operator for the $n$th band with energy $h_{\rho,n}(\mathbf{k})$, and $\alpha$ represents an additional internal degree of freedom beyond spin and valley. It is worth noting that the unit cell area of the twisted lattice, denoted as $\mathcal A_\mathrm{uc}$, is included in this transformation.
By employing this projection, we can focus specifically on the physics associated with the nearly flat bands of interest.





In accordance with the GL theory framework discussed in the main text, we introduce the bosonic auxiliary field $\Delta_{\rho,\ell\xi}(\mathbf{r})=\psi_{\downarrow\bar{\rho},\ell\xi}(\mathbf{r})\psi_{\uparrow\rho,\ell\xi}(\mathbf{r})$ and make the approximation of neglecting fluctuations in the layer $\ell$ and sublattice $\xi$ by assuming $\Delta_{\rho,\ell\xi}(\mathbf{r})=\Delta_{\rho}(\mathbf{r})$. For notation clarifications, in the critical region where the mean field values vanish, we continue to use $\Delta_{\rho,\ell\xi}(\mathbf{r})$ to denote the fluctuations. By employing the Hubbard-Stratonovich transformation, we obtain the action for $\psi_{\sigma\rho,\ell\xi}(\mathbf{r})$ as,
\begin{align}
Z & =\int\mathcal{D}[\Delta,\bar{\Delta}]e^{-g\int_{0}^{\beta}d\tau\int d^{2}\mathbf{r}\sum_{\rho\ell\xi}\vert\Delta_{\rho,\ell\xi}\vert^{2}}\mathcal{Z}[\Delta,\bar{\Delta}]~,
\end{align}
with 
\begin{align}
\mathcal{Z}[\Delta,\bar{\Delta}] & =\int\mathcal{D}[\psi,\bar{\psi}]e^{-\int_{0}^{\beta}d\tau\int d^{2}\mathbf{r}\mathcal{L}[\psi,\bar{\psi},\Delta,\bar{\Delta}]} ~,\\
\mathcal{L}[\psi,\bar{\psi},\Delta,\bar{\Delta}] & =\sum_{\sigma\rho\ell\xi}\bar \psi_{\sigma\rho,\ell\xi}(\mathbf{r})[\partial_\tau+h(-i\nabla)-\mu]\psi_{\sigma\rho,\ell\xi}(\mathbf{r})-g\sum_{\rho\ell\xi}\left[\bar{\Delta}_{\rho}(\mathbf{r})\psi_{\downarrow\bar{\rho},\ell\xi}(\mathbf{r})\psi_{\uparrow\rho,\ell\xi}(\mathbf{r})+h.c.\right]  ,
\end{align}
Here, $h(-i\nabla)$ represents the kinetic operator.
After projecting onto the two nearly flat bands, as previously emphasized, it is necessary to incorporate the nontrivial Bloch waves. This entails modifying the Lagrangian
\begin{equation}
\mathcal{L}[\psi,\bar{\psi},\Delta,\bar{\Delta}]\rightarrow\mathcal{L}[c,\bar{c},\Delta,\bar{\Delta}]~.
\end{equation}
The projected Lagrangian $\mathcal{L}[c,\bar{c},\Delta,\bar{\Delta}]$
is 
\begin{equation}
\mathcal{L}[c,\bar{c},\Delta,\bar{\Delta}] =\mathcal{L}_{0}[c,\bar{c}]+\mathcal{L}_{\mathrm{int}}[c,\bar{c},\Delta,\bar{\Delta}]\nonumber ~,
\end{equation}
with the free part for the projected fermions $c$,
\begin{equation}
\mathcal{L}_{0}[c,\bar{c}] =\sum_{\sigma\rho}\bar c_{\sigma\rho,n}(\mathbf{k})[\partial_\tau+\mathcal{E}_{\rho,n}(\mathbf{k})]c_{\sigma\rho,n}(\mathbf{k})~,
\end{equation}
and the interaction between bosonic field $\Delta$ and projected ferimions $c$,
\begin{equation}
\mathcal{L}_{\mathrm{int}}[c,\bar{c},\Delta,\bar{\Delta}]  =-\frac{g}{\mathcal A_\mathrm{uc}}\sum_{\rho}\sum_{n,m}\left[\Gamma_{\rho,nm}(\mathbf{q},\mathbf{k})\bar{\Delta}_{\rho}(\mathbf{k})c_{\downarrow\bar{\rho},n}(-\mathbf{q}+\frac{\mathbf{k}}{2})c_{\uparrow\rho,m}(\mathbf{\mathbf{q}+\frac{\mathbf{k}}{2}})+h.c.\right]~,
\end{equation}
where $\mathcal{E}_{\rho,n}(\mathbf{k})=h_{\rho,n}(\mathbf{k})-\mu$
, $h_{\rho,n}(\mathbf{k})$ is the dispersion of the targeted band 
$n$ and the prefactor $\Gamma_{\rho,nm}$ is
\begin{equation}
\Gamma_{\rho,nm}(\mathbf{q},\mathbf{k})=\sum_{\alpha}g_{\mathbf{\downarrow\rho},n\mathbf{-q}+\frac{\mathbf{k}}{2}}(\alpha)g_{\mathbf{\uparrow\rho},n\mathbf{\mathbf{q}+\frac{\mathbf{k}}{2}}}(\alpha) ~.
\end{equation}
Around the critical region, we have the Gor'kov's Green function
\begin{align}
\mathcal{G}_{\rho,n}(q) &=\langle c_{\sigma\rho,n}(\mathbf{q})\bar c_{\sigma\rho,n}(\mathbf{q})\rangle =\frac{1}{-i\omega+\mathcal{E}_{\rho,n}(\mathbf{k})}~,\\
\mathcal{F}_{\rho,n}(q)& =\langle c_{\sigma\rho,n}(\mathbf{q})c_{\sigma\rho,n}(\mathbf{-q})\rangle =0~.
\end{align}
with $q=(\mathbf q,\omega)$.
Integrate out the fermion field and we then have 
\begin{align}
F[\Delta,\bar{\Delta}] & =\sum_{\rho}g\vert\Delta_{\rho}\vert^{2}-\frac{1}{2}\left\langle \left(\int_{0}^{\beta}d\tau\int\frac{d^{2}\mathbf{k}}{(2\pi)^{2}}\mathcal{L}_{\mathrm{int}}[c,\bar{c},\Delta,\bar{\Delta}]\right)^{2}\right\rangle \nonumber \\
 & =\sum_{\rho}(g-g^{2}\chi_{\rho}(\mathbf{k}))\vert\Delta_{\rho}(\mathbf{k})\vert^{2} ~,
\end{align}
with 
\begin{align}
\chi_{\rho}(\mathbf{k})&=\frac{T}{2\pi}\sum_{\omega}\int\frac{d^{2}\mathbf{q}}{(2\pi)^{2}}\vert\Gamma_{\rho,nm}(\mathbf{q},\mathbf{k})\vert^{2}\mathcal{G}_{\rho,n}(q)\mathcal{G}_{\rho,m}(-q)\\
&=\frac{T}{2\pi}\sum_{n,m}\sum_{\omega}\int\frac{d^{2}\mathbf{q}}{(2\pi)^{2}}\vert\Gamma_{\rho,nm}(\mathbf{q},\mathbf{k})\vert^{2}\cdot\frac{1}{-i\omega+\mathcal{E}_{\rho,n}(\mathbf{q}+\frac{\mathbf{k}}{2})}\frac{1}{i\omega+\mathcal{E}_{\rho,m}(\mathbf{-\mathbf{q}+\frac{\mathbf{k}}{2}})}\\
&=\frac{1}{2\pi}\sum_{n,m}\int\frac{d^{2}\mathbf{q}}{(2\pi)^{2}}\vert\Gamma_{\rho,nm}(\mathbf{q},\mathbf{k})\vert^{2}\cdot\frac{\tanh\left(\frac{\beta}{2}\mathcal{E}_{\rho,n}(\mathbf{q}+\frac{\mathbf{k}}{2})\right)+\tanh\left(\frac{\beta}{2}\mathcal{E}_{\rho,m}(\mathbf{q}-\frac{\mathbf{k}}{2})\right)}{2(\mathcal{E}_{\rho,n}(\mathbf{q}+\frac{\mathbf{k}}{2})+\mathcal{E}_{\rho,m}(\mathbf{q}-\frac{\mathbf{k}}{2}))}~,
\end{align}
where we use $\mathcal{E}_{\rho,n}(\mathbf{\mathbf{q}-\frac{\mathbf{k}}{2}})=\mathcal{E}_{\rho,n}(-\mathbf{\mathbf{q}+\frac{\mathbf{k}}{2}})$
due to the time reversal symmetry. 
We can numerically calculate the quantity $\chi_{\rho}(\mathbf{k})$ and then expand it around small momentum $\mathbf{k}$ using Taylor series. This expansion leads to the coherence length $\xi$.
\end{widetext}




\end{document}
