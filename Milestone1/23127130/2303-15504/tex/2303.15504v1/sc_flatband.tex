\documentclass[aps,prx,notitlepage,superscriptaddress,showpacs,twocolumn]{revtex4-2}
\usepackage{graphicx,subfigure,epsfig}
 \usepackage{array,multirow}
\usepackage{dcolumn}
\usepackage{amssymb,amsmath,amsfonts,mathrsfs}
\usepackage{array}
\usepackage{times,setspace}
\usepackage{latexsym}
\usepackage{float,flafter,bm,bbm}
\usepackage{epstopdf,color,multirow}
\usepackage[colorlinks,linkcolor=blue,anchorcolor=blue,urlcolor=blue,citecolor=blue]{hyperref}
\usepackage{footnote}

\hypersetup{
    colorlinks=true,
    linkcolor=blue,
    filecolor=magenta,
    urlcolor=blue,
}

\begin{document}
\title{Towards a Ginzburg-Landau theory of the quantum geometric effect in
superconductors}
\author{Shuai A. Chen}
\email{chsh@ust.hk}

\affiliation{Department of Physics, Hong Kong University of Science and Technology,
Clear Water Bay, Hong Kong, China}
\author{K. T. Law}
\affiliation{Department of Physics, Hong Kong University of Science and Technology,
Clear Water Bay, Hong Kong, China}
\date{\today}
\begin{abstract}
Superconductivity in flat-band systems, such as twisted bilayer graphene, exhibits unconventional behaviours which cannot be described by the BCS theory. In this work, we derive an effective Lagrangian from a microscopic flat-band Hamiltonian which allows us to establish a Ginzburg-Landau (GL) theory includes the quantum geometric effects in flat-band superconductors. First of all, we deduce the  the critical temperature of a flat-band superconductor within the mean field description which is determined by the quantum metric. Secondly, going beyond the mean-field approximation by taking into account the fluctuations of the order parameter, we determine the superfluid weight, superconducting coherent length and the upper critical field and their dependence on the quantum metric of the flat bands. Thirdly, we apply the GL to twisted bilayer graphene (TBG). By calculating the quantum metric of flat bands of TBG,  the GL theory allows the estimation of the superconducting coherence length and the upper critical field. The results of the superconducting coherence length and upper critical field match the experimental results incredibly well without the fine tuning of parameters. The GL theory with quantum metric provides a new and general theoretical understanding of unconventional behaviors of flat-band superconductors.
\end{abstract}


\maketitle

\emph{\color{blue}Introduction.---} Our deeper understanding of
the geometric properties of Bloch electrons has greatly facilitated
the comprehension of quantum states of matter. Overall, a Berry curvature
and a quantum metric respectively constitute the real and imaginary
parts of the quantum geometric tensor \cite{1980CMaPh76289P,berry1989quantum}.
The former originates from the phase difference between two neighboring
Bloch states, and is essential for characterizing the band topology,
such as in quantum Hall(QH) and Chern insulators(CI) \cite{prlIQH,1982Thoulessprl,1984Berry,1994JMP355373B,TIRMP,TISCRMP,berry1989quantum}.
The latter, attributed as quantum geometry, measures the distance
of two neighboring Bloch states \cite{1990AAgeometry,2011EPJB79121R},
portraying the wave function extension and quantifying an obstruction
towards a complete localized Wannier basis \citep{PhysRevB.56.12847,PhysRevB.87.245103}.
Also, it has been argued to be relevant to the stability of fractional
QH \citep{PhysRevLett.107.116801,PhysRevB.33.2481,2002LNP59598F,2013CRPhy14816P}
and fractional CI states \citep{PhysRevB.88.115117,PhysRevB.90.165139}.
Unlike the Berry curvature, quantum geometry may be operative even
in a topologically trivial band. Recent studies have revealed the
fundamental role in various physical phenomena such as quantum transport
and electromagnetic responses \citep{PhysRevLett112166601,PhysRevB.94.134423,PhysRevResearch.3.L042018,PhysRevB.104.L100501,PhysRevLett.126.156602,2021NatPh..18..290A,2022PhRvB.105h5154M},
superfluidity/superconductivity in flat bands \citep{2015NatCo68944P,PhysRevB.95.024515,PhysRevB.98.220511,2020arXiv200716205J,PhysRevLett.117.045303,PhysRevLett.127.170404,PhysRevLett.128.087002,PhysRevB.106.014518,PhysRevB.105.L140506,PhysRevB.106.104514,2022arXiv220900007H},
quantum phase transition \citep{2006PhRvB..74w5111T,PhysRevLett99095701,PhysRevLett99100603,2021PNAS11806744V},
\emph{etc}. Much of the discussions have involved twisted graphene
and the like \citep{2018Natur.556...80C,2018Natur55643C,2019Natur.574..653L,PhysRevX.9.031049,PhysRevLett.123.237002,PhysRevB.101.060505,PhysRevResearch.2.023237,2020PhRvL124p7002X,PhysRevResearch4013164,PhysRevB.104.115160,PhysRevResearch.4.013209,2021arXiv211100807T,PhysRevLett.127.246403,PhysRevLett.128.176403}.
In particular, it is experimentally demonstrated \cite{2021arXiv211213401T}
that a flat band Dirac superconductor originates from quantum geometry
in a twisted bilayer gaphene. Nevertheless, a complete understanding
of the quantum geometric effect in superconductors is yet to come.

As a macroscopic quantum phenomenon, superconductivity since its discovery,
has exerted tremendous impacts on quantum matter physics. It features
the Meissner effect with stable supercurrents to phase out bulk magnetic
fields and harbour dissipation-free current flows. The Bardeen-Cooper-Schrieffer
(BCS) theory predicts the critical temperature $T_{c}\propto\exp(-1/g\rho_{0})$,
with an attractive interaction strength $g$ and density of states
at the Fermi surface $\rho_{0}$ \citep{altland2010condensed}. An
interesting situation occurs in the flatband limit, where the divergent
density of states triggers an exponentially enhanced $T_{c}\propto g$.
This, however, does not promise a coherent superconducting ground
state. The special limit, concomitant with an infinity effective mass
$m^{*}\!\!\rightarrow\!\!\infty$, seemingly implies the absence of
a finite superfluid weight $\mathsf{D}_{s}\!\propto\!1/m^{*}\!\!\rightarrow0$,
and thus the absence of Meissner effect \citep{PhysRevLett.68.2830,PhysRevB.47.7995}.
It is confirmed by recent studies that a SC phase is realizable via
a mean-field theory when the flatbands possess nontrivial quantum
geometry, i.e., finite Bloch function interference \citep{2015NatCo68944P,PhysRevB.95.024515,2020arXiv200716205J,PhysRevLett.117.045303,PhysRevB.98.220511,PhysRevLett.128.087002,PhysRevLett.127.170404,PhysRevB.106.014518}.
Nevertheless, the understanding of quantum geometry in SC is in its
early stages, such as its role in upper critical field and coherent
length. Therefore, it is desired for a GL theory encompassing the
quantum geometric correction. 

In this paper, we develop the GL theory to incorporate the quantum
geometrical properties of the paired electrons, in particular for
isolated flat- or narrow-band systems. We aim to uncover the fundamental
role of the quantum geometry in various aspects of a superconducting
phase such as superfluid weight and the upper critical field. We find
that the superfluid weight depends on quantum metric and is proportional
to attractive interaction strength, which is consistent with previous
result \cite{2015NatCo68944P,PhysRevB.106.014518}. For a vortex excitation,
we apply the Gor'kov's semiclassical path approximation \citep{gor1959microscopic}
to reveal that quantum metric sets a scale for superconducting coherent
length $\xi$ and controls the upper critical field $H_{c2}$, which
manifests even for a topologically trivial band. For a topologically
nontrivial band, $H_{c2}$ can have upper bound related to the Chern
number. We apply the generalized GL theory to the twisted bilayer
graphene (TBG). A SC phase has been reported with originality from quantum
geometry. Within the framework of the developed GL theory, we analyze
several quantities like superfluid weight, coherent length and the
upper critical field which are comparable to experimental measurements.

We organize the paper as follows. We first sketch a general framework
of a GL theory that incorporates quantum geometry of Bloch electrons.
We then systematically formulate the superfluid weight, and put forwards
a theory on the upper critical field and superconducting coherent
length. Then, we apply on SC phase in the twisted bilayer graphene
to provide an explanation on some experimental results. Lastly,
we talk about other promising further work.

%%%%%%%%%%%%%%%%%%%%%%%%%%%%%%%%%%%%%%%%
\emph{\color{blue}Framework of a Ginzburg-Landau theory.---} To
begin with, we consider a model Hamiltonian $H=H_{0}+H_{\mathrm{int}}$
on a lattice with $N$ sites. For simplicity, we assume $H_{0}$ possesses
an isolated flat/quasi-flat band around the Fermi surface. To go beyond
the atomic band limit, we may require an indispensable multi-band
lattice realization for the specified flatband in $H_{0}$ with a
Bloch wave forming $e^{-i\mathbf{k}\cdot\mathbf{r}}g_{\mathbf{k}\xi}(\alpha)$
with $\alpha$ indexing the orbital degrees of freedom. In other words,
the targeted band has Wannier obstruction that can be bounded by the
quantum geometry \cite{PhysRevB.56.12847,PhysRevB.90.165139,2015NatCo68944P}.
One typical example is a topologically non-trivial band for which
one fails to find a complete localized Wannier basis \citep{PhysRevLett.68.2830}.
A superconducting phase comes into play in the presence of an attractive
interaction 
\begin{equation}
H_{\mathrm{int}}=-g\int d\mathbf{r}a_{+}^{\dagger}(\mathbf{r})a_{-}^{\dagger}(\mathbf{r})a_{-}(\mathbf{r})a_{+}(\mathbf{r})~,\label{eq:Hint}
\end{equation}
with $a_{\xi}(\mathbf{r})$ as the electron operator carrying two
flavors $\xi=\pm$. To resolve the role of quantum geometry in interactions,
we emphasize that it is essential to introduce a projection that explicitly
brings in the Bloch functions of the targeted flat band 
\begin{equation}
a_{\xi}(\mathbf{r})\rightarrow\frac{1}{N}\sum_{\mathbf{q}}e^{i\mathbf{q}\cdot\mathbf{r}}g_{\mathbf{q}\xi}^{*}(\alpha)c_{\mathbf{q}\xi}~,\label{eq:projectiongk}
\end{equation}
where $c_{\mathbf{q}\xi}$ annihilates an electron on the targeted
band with $-\pi\leq q_{x,y}<\pi$ and $\alpha$ represents internal
degree of Bloch wave. The expansion in Eq.~(\ref{eq:projectiongk})
projects out other bands, which obeys the spirit of Landau' Fermi
liquid while $g_{\mathbf{k}\xi}$ encodes the quantum geometry \cite{PhysRevLett.93.206602,2017AnPhy.377..345C}.
We proceed with the Hubbard-Stratonovich transformation by introducing
a bosonic field $\Delta(\mathbf{r})$, 
\begin{align}
\Delta(\mathbf{r}) & =a_{-}(\mathbf{r})a_{+}(\mathbf{r})~,
\end{align}
Then we can obtain the Lagrangian density $\mathcal{L}$ under the
framework of path integral with 
\begin{align}
\mathcal{L}= & (i\omega-\mu)(\bar{c}_{\mathbf{k},+}c_{\mathbf{k},+}+\bar{c}_{\mathbf{k},-}c_{\mathbf{k},-})\nonumber \\
 & -g\sum_{\mathbf{q}}[\Gamma(\mathbf{q},\mathbf{k})\Delta(\mathbf{k})\bar{c}_{\mathbf{q}+\mathbf{\frac{k}{2}},+}\bar{c}_{\mathbf{-q}+\frac{\mathbf{k}}{2},-}+h.c.],
\end{align}
where $c_{\mathbf{q},\xi}$ denotes the Grassmann fields, $\mu$ is
the chemical potential and bosonic field $\Delta(\mathbf{k})\equiv1/N\sum_{\mathbf{r}}\Delta(\mathbf{r})e^{i\mathbf{k}\cdot\mathbf{r}}$
in the momentum space. The projection in Eq.~\eqref{eq:projectiongk}
decorates the coupling constant $g$ by an extra factor $\Gamma(\mathbf{q},\mathbf{k})\equiv\sum_{\alpha}g_{\mathbf{\mathbf{-q}+\mathbf{k}/2},+}(\alpha)g_{\mathbf{\mathbf{q}+\mathbf{k}/2},-}(\mathbf{\alpha})$
that will play a vital role in a superconductor. Formally, the GL
theory $F[\Delta]$ is manifested by integrating out the fermion fields
at finite temperature $T$, 
\begin{equation}
F[\Delta]=\sum_{\mathbf{k}}\!g\bar{\Delta}(\mathbf{k})\Delta(\mathbf{k})-T\ln\int\mathcal{D}[c,\bar{c}]e^{-\int_{0}^{\beta}d\tau\sum_{\mathbf{q}}\mathcal{L}}~.\label{eq:GL_S}
\end{equation}
To facilitate the evaluation on GL theory in Eq. (\ref{eq:GL_S}),
we shall make an expansion $\Delta(\mathbf{k})=\Delta_{0}\delta_{\mathbf{k},\mathbf{0}}+\delta\Delta(\mathbf{k})$
around the extremum of $F[\Delta]$. Here $\Delta_{0}$ corresponds
to the mean-field value and $\delta\Delta(\mathbf{k})$ represents
the order parameter fluctuations. By minimizing the GL action $\frac{\partial F[\Delta]}{\partial\Delta_{0}}=\frac{\partial F[\Delta]}{\partial\bar{\Delta}_{0}}=0$,
we can determine the parameter $\Delta_{0}$ from a self-consistent
gap equation 
\begin{equation}
1=\frac{1}{N}\sum_{\mathbf{q}}\Gamma(\mathbf{q})\frac{g\Gamma^{*}(\mathbf{q})}{2\epsilon(\mathbf{q})}\tanh\frac{\beta\epsilon(\mathbf{q})}{2}.\label{eq:self-eq}
\end{equation}
where $\epsilon(\mathbf{q})=\sqrt{\left|g\Gamma(\mathbf{q})\Delta_{0}\right|^{2}+\mu^{2}}$
denotes the dispersion of Bogoliubov quasiparticles. In the presence
of time-reversal symmetry $g_{-\mathbf{k},-}=g_{\mathbf{k},+}^{*}$,
$\Gamma(\mathbf{q})=1$ and under a uniform ansatz $\Delta(\mathbf{r})=\Delta_{0}$,
from Eq.~\eqref{eq:self-eq} one may extract a relation $g\Delta_{0}/T_{\mathrm{BCS}}=2$
with $T_{\mathrm{BCS}}$ as the critical temperature, which is larger
than the ratio ($\sim1.7$) from a conventional BCS theory with a
large Fermi velocity.

We are ready to expand $F[\Delta]=F_{0}+F_{2}+\mathcal{O}(|\delta\Delta|^{4})$
up to the second order of $\delta\Delta(\mathbf{k})$. In particular,
$F_{0}$ recovers the grand potential 
\begin{equation}
F_{0}=\sum_{\mathbf{k}}\left[g\vert\Delta_{0}\vert^{2}-2\ln(1+e^{-\beta\epsilon(\mathbf{k})})+\epsilon(\mathbf{k})\right].\label{eq:grandpot}
\end{equation}
The second order $F_{2}\equiv\sum_{\mathbf{k}}\mathcal{L}[\delta\Delta]$
describes the Gaussian fluctuations with 
\begin{equation}
\mathcal{L}[\delta\Delta]=\left[g-g^{2}\chi(\mathbf{k})\right]\delta\bar{\Delta}(\mathbf{k})\delta\Delta(\mathbf{k})~,\label{eq:L_Delta}
\end{equation}
where $\chi(\mathbf{k})$ is the four-point correlation function,
\begin{align}
\chi(\mathbf{k})\equiv & \frac{1}{N}\sum_{q}\vert\Gamma(\mathbf{q},\mathbf{k})\vert^{2}[\mathcal{G}(q+k/2)\mathcal{G}(-q+k/2)\nonumber \\
 & +\mathcal{F}(q+k/2)\mathcal{F}(-q+k/2)]~,\label{eq:chi}
\end{align}
with Gor'kov's normal and abnormal Green functions $\mathcal{G}(q)$
and $\mathcal{F}(q)$ ($q=(\mathbf{q},\omega)$). Further details
regarding derivation details and higher order terms can be found in
SM. By contrast to a conventional SC, the Bloch waves matter in both the
effective interaction and the quasiparticle dispersion. Meanwhile,
a prefactor $|\Gamma(\mathbf{q},\mathbf{k})|^{2}$ is highlighted
in Eq.~(\ref{eq:chi}) to include interference between Bloch waves
at different momentum. Such a feature will encode the quantum geometry
as we will see. The effective theory in Eq.~(\ref{eq:L_Delta}) along
Eq.~(\ref{eq:chi}) is the main result of the paper. With $\mathcal{L}[\delta\Delta]$
in Eq.~(\ref{eq:L_Delta}), we can analyze various aspects of a SC
phase. Fundamentally, $\chi(\mathbf{k})$ varies for $\mathbf{k}^{2}$
since the pairing strength of a finite momentum Cooper pair is weighed
by $\Gamma(\mathbf{q},\mathbf{k})$, which will generate a finite
superfluid weight to account for the Meissner effect even though the
effective mass of electrons diverges. By contrast, for a trivial band, $\chi(\mathbf k)$ is independent on $\mathbf k$, which indicates absence of Goldtone mode and thus no SC occurs.
Crucially, it will be more natural
to introduce minimal coupling with electromagnetic field in Eq.~(\ref{eq:L_Delta}).
We now reveal the role of quantum geometry.

%%%%%%%%%%%%%
\emph{\color{blue}Superfluid weight and quantum metric.---} In general,
we can expect an expansion 
\begin{equation}
\vert\Gamma(\mathbf{q},\mathbf{k})\vert^{2}=\gamma_{0}(\mathbf{q})-\sum_{ab}\gamma_{2}^{ab}(\mathbf{q})k_{a}k_{b},\label{eq:chi2expan}
\end{equation}
where the absence of a linear term is simply due to the stability
of mean-field ansatz. For the time-reversal invariant system $g_{\mathbf{q},+}=g_{\mathbf{-q},-}^{*}\equiv g_{\mathbf{q}}$,
$\gamma_{0}(\mathbf{q})$ relates to an inner product $\gamma_{0}(\mathbf{q})\equiv\vert\langle g_{\mathbf{q}}\vert g_{\mathbf{q}}\rangle\vert^{2}=1$
and $\gamma_{2}(\mathbf{q})$ in the continuum limit recovers a Fubini-Study
metric \citep{1980CMaPh76289P,2010arXiv1012.1337C} with components
\begin{equation}
\gamma_{2}^{ab}(\mathbf{q})\equiv\mathrm{Re}\langle\partial_{\mathbf{q}_{a}}g_{\mathbf{q}}\vert(1-|g_{\mathbf{q}}\rangle\langle g_{\mathbf{q}}|)\vert\partial_{\mathbf{q}_{b}}g_{\mathbf{q}}\rangle,\label{eq:gab}
\end{equation}
which measures a Bures distance between two quantum states and characterizes
how the Bloch functions interfere. The appearance of quantum geometry
in Eq.~\eqref{eq:gab} straightforwardly demonstrates the critical
role of a quantum geometry effect in determining superconductivity
fluctuations. By contrast, it does not manifest in the mean-field
level as shown in Eq.~\eqref{eq:grandpot}. After integrating out
the Matsubara frequency along with the expansion in Eq.~(\ref{eq:chi2expan}),
we have $\chi(\mathbf{k})=\chi_{0}-\frac{1}{8}\sum_{ab}\mathsf{D}_{s}^{ab}k_{a}k_{b}$,
with the explicit form as 
\begin{align}
\chi_{0} & =\frac{1}{N}\sum_{\mathbf{q}}\frac{\gamma_{0}(\mathbf{q})}{2}\frac{1}{\epsilon(\mathbf{q})},\label{eq:chi0}\\
\mathsf{D}_{s}^{ab} & =\frac{2g^{2}\Delta_{0}^{2}}{N}\sum_{\mathbf{q}}\frac{\tanh\left(\frac{\beta\epsilon(\mathbf{q})}{2}\right)}{\epsilon(\mathbf{q})}\gamma_{2}^{ab}(\mathbf{q}).\label{eq:chiab}
\end{align}
Thus we reach an effective theory $\mathcal{L}[\delta\Delta]$ 
\begin{equation}
\mathcal{L}[\delta\Delta]=\frac{1}{8}\sum_{ab}\mathsf{D}_{s}^{ab}\partial_{a}\delta\bar{\Delta}(\mathbf{k})\partial_{b}\delta\Delta(\mathbf{k}).\label{eq:LdeltaD}
\end{equation}
The factor $\mathsf{D}_{s}^{ab}$ containing $\gamma_{2}^{ab}(\mathbf{q})$
indicates that the dynamics in Eq.~(\ref{eq:L_Delta}) originates from
the quantum geometry. The factor $\gamma_{2}^{ab}(\mathbf{q})$ measures
a superfluid weight. By contrast, there will be no fluctuations under
the circumstances of vanishing quantum geometry $\gamma_{2}^{ab}$,
that is, $\mathsf{D}_{s}^{ab}=0$ in Eq.~\eqref{eq:LdeltaD}. The
Meissner effects can be accessed by considering the phase fluctuations
$\delta\Delta(\mathbf{r})=\Delta_{0}e^{2i\theta(\mathbf{r})}-\Delta_{0}\simeq2i\theta(\mathbf{r})\Delta_{0}$
and ignoring the massive amplitude fluctuations, yielding 
\begin{equation}
\mathcal{L}[\theta]=\frac{1}{2}\text{\ensuremath{\sum_{ab}}}\mathsf{D}_{s}^{ab}\partial_{a}\theta\partial_{b}\theta~,
\end{equation}
and the supercurrent $j_{b}=\sum_{a}\mathsf{D}_{s}^{ab}\partial_{a}\theta$.
Therefore, we can recognize the factor $\mathsf{D}_{s}^{ab}$ in Eq.
(\ref{eq:chiab}) intrinsically to be the superfluid weight, which
is consistent with previous results by a different method \cite{2015NatCo68944P,PhysRevB.106.014518}.
Straightforwardly, we can deduce that the SC phase is dominated by
a BKT physics with $T_{\mathrm{BKT}}=\pi\sqrt{\mathrm{det}\mathsf{D}_{s}^{ab}}/8$
\citep{PhysRevLett.39.1201}. For an isotropic SC with a flat Bogoliubov
quasiparticle band, we have a simple relation $T_{\mathrm{BKT}}/g\Delta_{0}=\frac{\pi}{8}\sqrt{\mathrm{det}\bar{\gamma}_{2}^{ab}}$
around filling $\mu=0$ with $\bar{\gamma}_{2}^{ab}=\frac{1}{N}\sum_{\mathbf{q}}\gamma_{2}^{ab}(\mathbf{q})$.
In case of time reversal symmetry, the quantity $\bar{\gamma}_{2}^{ab}$
as in Eq.~\eqref{eq:gab} is essentially averaged quantum metric
over the BZ. The superfluidity $\mathsf{D}_{s}^{ab}$ here represents
a non-linear response to a momentum shift of a Cooper pair or magnetic
field. 
Basically, the quantum geometric contribution exists for both flat
and non-flat bands, which will be clarified in the application TBG.
Its effect may be ignored due to the large bandwidth or the featureless
Wannier basis. The flatband limit provides a suitable platform to
highlight such a geometry effect that is independent of Berry curvature.

%%%%%%%%%%%%%%%%%%%%%%%%%%%%%%%%
\emph{\color{blue}Upper critical field $H_{c2}$.---} Another fundamental
problem is the role of the quantum geometry in the vortex excitations
and the upper critical field $H_{c2}$. Around $H_{c2}$, the BCS
order parameters get destroyed by vortex excitations, and thus we
expect no shift on the chemical potential from the interaction. 
The GL theory follows the above procedures 
along with a vanishing mean field $\Delta_{0}=0$ in Eq.~\eqref{eq:GL_S}.
The Green function takes the form as $\mathcal{G}(q)=\frac{1}{i\omega}$
and $\mathcal{F}=0$ in Eq.~\eqref{eq:chi}, which yields 
\begin{align}
\chi(\mathbf{k}) & =(\bar{\gamma}_{0}-\sum_{ab}\bar{\gamma}_{2}^{ab}k_{a}k_{b})\frac{\beta}{4},\label{eq:chi_hc2}
\end{align}
where $\bar{\gamma}_{0}=\frac{1}{N}\sum_{\mathbf{q}}\gamma_{0}(\mathbf{q})$. 
The quantity $\bar{\gamma}_{2}^{ab}=\frac{1}{N}\sum_{\mathbf{q}}\gamma_{2}^{ab}(\mathbf{q})$
is the averaged quantum metric over the BZ. In Eq.~(\ref{eq:chi_hc2}),
we apply an identity $\sum_{n\in\mathbb{Z}}\frac{1}{(2n+1)^{2}}=\frac{\pi^{2}}{4}$.
For an isotropic system, the Lagrangian then can be arranged in a
familiar from 
\begin{equation}
\mathcal{L}[\delta\Delta]\!=\!\frac{1}{2m^{*}}\vert\nabla\delta\Delta\vert^{2}+a(T)\vert\delta\Delta\vert^{2}+\mathcal{O}(\vert\delta\Delta\vert^{4}),\label{eq:Lhc2}
\end{equation}
with 
\begin{align}
\frac{1}{2m^{*}} & =\frac{\beta g^{2}}{4}\sqrt{\mathrm{det}(\bar{\gamma}_{2}^{ab})},\\
a(T) & =g-g^{2}\frac{\beta\bar{\gamma}_{0}}{4}.
\end{align}
We find that the quantum metric $\sqrt{\mathrm{det}(\bar{\gamma}_{2}^{ab})}$
enters the effective mass of Cooper pairs while the sign change of
$a(T)$ before $\vert\delta\Delta\vert^{2}$ marks the BCS critical
temperature $T_{\mathrm{BCS}}=g\frac{\bar{\gamma}_{0}}{4}$. A routine
towards the vortex involves the Gor'kov's semiclassical integral approximation
\citep{gor1959microscopic}. We attribute the magnetic field effect
$\mathbf{A}$ to a phase shift on the Green function $\mathcal{G}(\mathbf{r}-\mathbf{r}^{\prime})\rightarrow\mathcal{G}(\mathbf{r}-\mathbf{r}^{\prime})e^{i\int_{\mathbf{r}^{\prime}}^{\mathbf{r}}d\mathbf{s}\cdot\mathbf{A}(\mathbf{s})}$
with a line integral along a straight path linking $\mathbf{r}^{\prime}$
and $\mathbf{r}$. It is equivalent to minimally coupling the magnetic
field $-i\nabla\rightarrow-i\nabla+2e\mathbf{A}$ in Eq.~(\ref{eq:Lhc2})
while the prefactor factor $\Gamma(\mathbf{q},\mathbf{k})$ stays unaffected.
Hence, we can determine the upper critical field $H_{c2}$ by 
\begin{equation}
H_{c2}=\frac{2\pi}{\sqrt{\det(\bar{\gamma}_{2}^{ab})}}\frac{\vert4-g\beta\bar{\gamma}_{0}\vert}{\beta g}, \label{eq:hc2}
\end{equation}
which works also for anisotropic cases. For a time-reversal invariant
system, the factor $\bar{\gamma}_{2}^{ab}$ contains the quantum metric.
From a local inequality, $\sqrt{\det\gamma_{2}^{ab}(\mathbf{k})}\geq \frac{1}{2}\vert\mathrm{Tr}\mathcal{B}(\mathbf{k})\vert$
for two dimensions \citep{1980CMaPh76289P,2015NatCo68944P}, we have
$\sqrt{\det(\bar{\gamma}_{2}^{ab})}\geq \int \frac{d^2\mathbf k}{(2\pi)^2}\sqrt{\mathrm{det}(\gamma_2^{ab})}\geq\int\frac{d^{2}\mathbf{k}}{(2\pi)^{2}}\frac{1}{2}\vert\mathrm{Tr}\mathcal{B}(\mathbf{k})\vert\geq\frac{|C|}{2\pi},$
where $\mathcal{B}(\mathbf{k})$ is the Berry curvature and $C$ is the Chern number. We can have
two indications. Firstly, the upper critical field $H_{c2}$ is bound
by $H_{c2}\leq\frac{8\pi}{|C|}\frac{\vert1-\beta\tau_{c}\vert}{\beta g}$
for a topological band. Second, even for
a topological trivial band, as long as the Berry curvature is locally
finite, the quantum geometry matters to be bound by the averaged $\vert\mathrm{Tr}\mathcal{B}(\mathbf{k})\vert$.
On the other hand, from the condition $H_{c2}\xi^{2}=2\pi$, we can
deduce in Eq.~\eqref{eq:hc2} the superconducting coherent
length 
\begin{equation}
\xi=\sqrt{\frac{\beta g}{\vert4-g\beta\bar{\gamma}_{0}\vert}}[\mathrm{det}(\bar{\gamma}_{2}^{ab})]^{1/4}.\label{eq:cohlen}
\end{equation}
At very low temperature $\beta\rightarrow\infty$, we have $\xi=[\mathrm{det}(\bar{\gamma}_{2}^{ab})]^{1/4}$
with a time-reversal symmetry, which can be alternatively expected
from a simple dimensional analysis in the continuum limit. The coherent
length $\xi$ is experimentally feasible and one may use it to identity
such a non-BCS SC phase. 

%%%%%%%%%%%%%%%%%%%%%%%%%%%%%%%%
\begin{figure}[th]
\centering \includegraphics[scale=0.35]{metric} \caption{(a) Band structure and the distribution of the quantum metric $\sqrt{\mathrm{det}(\gamma_{2}^{ab})}$ in Eq.~\eqref{eq:gab}
over the moir\'e BZ in a TBG. The quantum metric diverges over two
Dirac points at $\mathrm{K}$ and $\mathrm{K}^{\prime}$ while it
contributes a lot at $\Gamma$. (b) The superconducting coherent length
$\xi$ \emph{vs.} chemical potential $\mu$ in the low temperature
limit. The $\xi$ decreases first when doping goes away from $\mu=-0.1\mathrm{meV}$,
and then increases around $\mu=-0.3$meV. In calculations, 
we set the temperature $\beta=200\mathrm{meV}^{-1}$ and only
show one spin-valley species for simplification.}
\label{Fig:TBG} 
\end{figure}

\emph{\color{blue}Application on TBG---} We consider the twisted
bilayer graphene. It is demonstrated experimentally \cite{2021arXiv211213401T}
that the quantum geometry contributes to a flat band Dirac superconductor
at filling between $-1/2$ and $-3/4$ in TBG
with highly enhanced upper critical field $H_{c2}$ and coherent length
$\xi$. Previous theoretical studies have provided an explanation
on the superconducting gap and the superfluid weight \cite{PhysRevB.101.060505,PhysRevLett.123.237002}.
For a complete understanding on the SC in TBG, here we apply the GL
theory to deal with the fluctuations. 
First of all, the TBG electronic
structure can be well captured as the Dirac fermions subjected to
opposite pseudomagnetic field at the magic angle \cite{2019PhRvL.122j6405T,PhysRevB.99.155415},
and the TBG flatbands are mapped to the induced zeroth pseudo Landau Level(pLL), which can be simulated by a time-reversal invariant Harper lattice model  \citep{PhysRevB.14.2239,2015NatCo68944P} [See SM]. 
The Harper model $H=-\sum_{\mathbf{r},\mathbf{r}^{\prime}}\sum_{\xi}t_{\mathbf{r},\mathbf{r}^{\prime}}^{\xi}a_{\xi}^{\dagger}(\mathbf{r})a_{\xi}^ {}(\mathbf{r}^{\prime})$
is a tight-binding system in which electrons at two flavors $\xi=\pm$
are subject to opposite magnetic fields on a square lattice ($N=N_{c}\times N_{\mathrm{o}}$)
with $N_{c}$ super unit cells and $N_{\mathrm{o}}$ orbitals. The
hopping matrix $t_{\mathbf{r},\mathbf{r}^{\prime}}^{\xi}$ describes
the hopping processes between nearest-neighbor sites, and it reads
\begin{equation}
\!\!\!t_{\mathbf{r}\mathbf{r^{\prime}}}^{\xi}\!=\!\omega^{\xi r_{y}}\delta_{\mathbf{r}-\mathbf{e}_{x},\mathbf{r^{\prime}}}\!+\!\omega^{-\xi r_{y}}\delta_{\mathbf{r}+\mathbf{e}_{x},\mathbf{r^{\prime}}}\!+\!\delta_{\mathbf{r}-\mathbf{e}_{y},\mathbf{r}^{\prime}}\!+\!\delta_{\mathbf{r}+\mathbf{e}_{y},\mathbf{r^{\prime}}},
\end{equation}
with $\mathbf{e}_{x}$($\mathbf{e}_{y}$) the unit vector along $x$($y$)-direction.
Here a factor $\omega^{\pm\xi j_{y}}$ represents a lattice version
of Landau gauge and we consider a uniform commensurate flux 
$\Phi_{\mathrm{p}}=\frac{2\pi}{N_{\mathrm{o}}}$
($\omega=e^{i\Phi_{\mathrm{p}}}$) such that $B_{\mathrm{p}}a^{2}=\frac{2\pi}{N_{\mathrm{o}}}$.
The zeroth pLLs can be recognized as the lowest bands under the condition $N_{\mathrm{o}}\gg1$, which gives us the
Bloch waves $g_{\mathbf{q},\xi}(\alpha)$ \citep{PhysRevB.90.075104} with $(\mathbf{r}_{c},\alpha)$ relabeling the site $\mathbf{r}$.
Then we have a projection onto the zeroth pLL $a_{\xi}(\mathbf{r}_{c},\alpha)\rightarrow\sum_{\mathbf{k}}g_{\mathbf{k},\xi}^{*}(\alpha)c_{\mathbf{k},\xi}$.
Generally, we assume an s-wave pairing with an attractive interaction in Eq.~\eqref{eq:Hint}. For the auxiliary field $\Delta_{\alpha}(\mathbf{r}_{c})=a_{-}(\mathbf{r}_{c},\alpha)a_{+}(\mathbf{r}_{c},\alpha)$, we ignore orbital fluctuations $\Delta_{\alpha}(\mathbf{r}_{c})\equiv\Delta(\mathbf{r}_{c})$
[See SM]. For the zeroth pLL,
we have the factors $\gamma_{0}(\mathbf{q})=1$ and $\gamma_{2}^{ab}(\mathbf{q})=\frac{N_{\mathrm{o}}a^{2}}{4\pi}\delta_{ab}$ \cite{2021PhRvB.104d5103O,2015NatCo68944P}.
Due to the translation symmetry, we can apply the uniform mean field
ansatz $\Delta_{0}$, which can be determined by the self-consistent
equation in Eq.~\eqref{eq:self-eq}. At temperature $T=0$, according
to Eq.~(\ref{eq:chiab}), we have a superfluid weight $\mathsf{D}_{s}\equiv\frac{(g\Delta_{0})^{2}}{2\pi\epsilon}$.
At filling around $\mu=0$, we have the ratio $\mathsf{D}_{s}/(g\Delta_{0})\sim0.16$,
which is comparable to the experimental result \cite{2021arXiv211213401T}.



In fact, we can have a more accuracy prediction on the superconducting
coherent length $\xi$ in the low temperature as indicated in
Eq.~(\ref{eq:cohlen}). We start with the Bistrizer-Macdonald model
\cite{2011PNAS..10812233B} to simulate the moir\'e band structure of
TBG at a small angle. Furthermore, in account of the corrugation effect
induced by lattice relaxation \cite{PhysRevX.8.031087}, we set AA(BB)
and AB interlayer tunneling respectively as $t_{AA}=t_{BB}=79.7$meV, $t_{AB}=97.5$meV
while the Fermi velocity is $v_{F}=7.98\times10^{5}$m/s for the graphene
layer. In Fig.~\ref{Fig:TBG}(a) we depict the band structure at
twisted angle $\theta=1.08^{\circ}$ with one spin-vallay species
for simplicity while the dash line crossing the two Dirac points ($\mathsf{K}$
and $\mathsf{K}^{\prime}$) marks the chemical potential $\mu=0$,
that is, the filling $\nu=0$. Although the type of interaction is
unknown, we assume a simple singlet pairing potential \cite{PhysRevB.101.060505,PhysRevLett.123.237002}
$H_{\mathrm{int}}=-g\int d^2\mathbf r\sum_{\ell\xi}\psi_{\uparrow\rho,\ell\xi}^{\dagger}(\mathbf{r})\psi_{\downarrow\bar{\rho},\ell\xi}^{\dagger}(\mathbf{r})\psi_{\downarrow\bar{\rho},\ell\xi}(\mathbf{r})\psi_{\uparrow\rho,\ell\xi}(\mathbf{r})$
with the valley indices $\rho=\pm$, sublattices $\xi=A,B$, and layer
$\ell$. Here $\psi_{\sigma\rho,\ell\xi}$ denotes the Fermion operators
in the continuum limit. The mean-field treatment have been thoughtfully
studied \cite{PhysRevB.101.060505,PhysRevLett.123.237002} whose results
are consist with the model of two-flavor Dirac fermions subject to
pseudo magnetic field.

We can acquire the coherent length within the GL theory. From Eq.~\eqref{eq:Lhc2},
one can infer that at the low temperature limit $\beta\rightarrow\infty$,
the superconducting coherent length that originates from quantum geometry
can be intrinsically only determined by the quantum metric $\xi=[\mathrm{(}\mathrm{det}\bar{\gamma}_{2}^{ab})]^{1/4}$
that is independent on the pairing coupling $g$. In Fig.~\ref{Fig:TBG}(a),
we show that the quantum metric diverges around two Dirac point, and
takes a relatively large value at $\Gamma$. In evaluating the coherent
length $\xi$, we take into account the finite bandwidth in Eq.~\eqref{eq:chi}.
Fig.~\ref{Fig:TBG}(b) shows the tendency of $\xi$ \emph{vs.} various
chemical potential $\mu$. At filling $-1/2$ with $\mu=-0.12\mathrm{meV}$,
the $\xi$ reaches about $40\mathrm{nm}$ that is much larger than
the one from a conventional SC theory. When the filling goes away
from the $-1/2$, due to the lower density of states $\xi$ decreases
that is discovered in the experiment \cite{2021arXiv211213401T}.
When we are close to $-3/4$, $\xi$ increases again which may arise
from the quantum geometry around $\Gamma$. Since $\xi\sim35\mathrm{nm}$,
we can determine the upper critical field in the order of $H_{c2}\sim0.26\mathrm{T}$,
which is also comparable to the experimental results. The calculation
details can be found in SM.

\emph{\color{blue}Summary.---} We develop the framework of the GL
theory to advance the understanding of the quantum geometry effect
in superconductivity. By incorporating the quantum geometry of Bloch
functions, we find that the quantum metric plays a fundamental role
in various aspects. We derive a formula of a superfluid weight $\mathsf{D}_{s}$
that indicates the quantum geometrical originality of the Meissner
effect. We further determine the coherent length and the upper critical
field $H_{c2}$. Essentially, the quantum geometry reflects one of
intrinsic aspects of a multiband system, and thus the developed GL
theory has a broad scope of application. We illustrate the GL theory
on the TBG. In particular, the developed GL well explains some aspects
of experimental measurements \cite{2021arXiv211213401T}. Many future
directions are appealing due to the fundamental role of the quantum
geometry in various systems. For example, it remains to be answered
whether a quantum geometry can stabilize quantum order like anti-ferromagnetic
and charge density wave. It also is appealing to realize a novel quantum
phases by designing a multi-orbital system with nontrivial quantum
geometry \citep{2022arXiv220613682K,2022arXiv220802285C,2022PhRvB.106r4507K}
such as pair density wave order. Since the develop GL theory is flexible
in the isolated band system, it inspires us to consider the role of
quantum geometry of a broaden fields, such as electron-phonon interaction,
magnetism and etc.

\emph{Acknowledgments.---} S. Chen acknowledges valuable discussions
with Tai-Kai Ng, Wen Huang and Yan-bin Yang. K.T. Law acknowledges
the support of Ministry of Science and Technology, China and HKRGC
through MOST20SC04, RFS2021-6S03, C602519G, AoE/P-701/20, 16310520,
16310219 and 16309718.


% \bibliography{ref}

%apsrev4-2.bst 2019-01-14 (MD) hand-edited version of apsrev4-1.bst
%Control: key (0)
%Control: author (8) initials jnrlst
%Control: editor formatted (1) identically to author
%Control: production of article title (0) allowed
%Control: page (0) single
%Control: year (1) truncated
%Control: production of eprint (0) enabled
\begin{thebibliography}{77}%
\makeatletter
\providecommand \@ifxundefined [1]{%
 \@ifx{#1\undefined}
}%
\providecommand \@ifnum [1]{%
 \ifnum #1\expandafter \@firstoftwo
 \else \expandafter \@secondoftwo
 \fi
}%
\providecommand \@ifx [1]{%
 \ifx #1\expandafter \@firstoftwo
 \else \expandafter \@secondoftwo
 \fi
}%
\providecommand \natexlab [1]{#1}%
\providecommand \enquote  [1]{``#1''}%
\providecommand \bibnamefont  [1]{#1}%
\providecommand \bibfnamefont [1]{#1}%
\providecommand \citenamefont [1]{#1}%
\providecommand \href@noop [0]{\@secondoftwo}%
\providecommand \href [0]{\begingroup \@sanitize@url \@href}%
\providecommand \@href[1]{\@@startlink{#1}\@@href}%
\providecommand \@@href[1]{\endgroup#1\@@endlink}%
\providecommand \@sanitize@url [0]{\catcode `\\12\catcode `\$12\catcode
  `\&12\catcode `\#12\catcode `\^12\catcode `\_12\catcode `\%12\relax}%
\providecommand \@@startlink[1]{}%
\providecommand \@@endlink[0]{}%
\providecommand \url  [0]{\begingroup\@sanitize@url \@url }%
\providecommand \@url [1]{\endgroup\@href {#1}{\urlprefix }}%
\providecommand \urlprefix  [0]{URL }%
\providecommand \Eprint [0]{\href }%
\providecommand \doibase [0]{https://doi.org/}%
\providecommand \selectlanguage [0]{\@gobble}%
\providecommand \bibinfo  [0]{\@secondoftwo}%
\providecommand \bibfield  [0]{\@secondoftwo}%
\providecommand \translation [1]{[#1]}%
\providecommand \BibitemOpen [0]{}%
\providecommand \bibitemStop [0]{}%
\providecommand \bibitemNoStop [0]{.\EOS\space}%
\providecommand \EOS [0]{\spacefactor3000\relax}%
\providecommand \BibitemShut  [1]{\csname bibitem#1\endcsname}%
\let\auto@bib@innerbib\@empty
%</preamble>
\bibitem [{\citenamefont {{Provost}}\ and\ \citenamefont
  {{Vallee}}(1980)}]{1980CMaPh76289P}%
  \BibitemOpen
  \bibfield  {author} {\bibinfo {author} {\bibfnamefont {J.~P.}\ \bibnamefont
  {{Provost}}}\ and\ \bibinfo {author} {\bibfnamefont {G.}~\bibnamefont
  {{Vallee}}},\ }\bibfield  {title} {\bibinfo {title} {{Riemannian structure on
  manifolds of quantum states}},\ }\href {https://doi.org/10.1007/BF02193559}
  {\bibfield  {journal} {\bibinfo  {journal} {Communications in Mathematical
  Physics}\ }\textbf {\bibinfo {volume} {76}},\ \bibinfo {pages} {289}
  (\bibinfo {year} {1980})}\BibitemShut {NoStop}%
\bibitem [{\citenamefont {Berry}(1989)}]{berry1989quantum}%
  \BibitemOpen
  \bibfield  {author} {\bibinfo {author} {\bibfnamefont {M.~V.}\ \bibnamefont
  {Berry}},\ }\bibfield  {title} {\bibinfo {title} {The quantum phase, five
  years after},\ }in\ \href@noop {} {\emph {\bibinfo {booktitle} {Geometric
  phases in physics}}},\ \bibinfo {editor} {edited by\ \bibinfo {editor}
  {\bibfnamefont {A.}~\bibnamefont {Shapere}}\ and\ \bibinfo {editor}
  {\bibfnamefont {F.}~\bibnamefont {Wilczek}}}\ (\bibinfo  {publisher} {World
  scientific},\ \bibinfo {address} {Singapore},\ \bibinfo {year} {1989})\
  Chap.~\bibinfo {chapter} {1}, pp.\ \bibinfo {pages} {1--28}\BibitemShut
  {NoStop}%
\bibitem [{\citenamefont {Klitzing}\ \emph {et~al.}(1980)\citenamefont
  {Klitzing}, \citenamefont {Dorda},\ and\ \citenamefont {Pepper}}]{prlIQH}%
  \BibitemOpen
  \bibfield  {author} {\bibinfo {author} {\bibfnamefont {K.~v.}\ \bibnamefont
  {Klitzing}}, \bibinfo {author} {\bibfnamefont {G.}~\bibnamefont {Dorda}},\
  and\ \bibinfo {author} {\bibfnamefont {M.}~\bibnamefont {Pepper}},\
  }\bibfield  {title} {\bibinfo {title} {New method for high-accuracy
  determination of the fine-structure constant based on quantized hall
  resistance},\ }\href {https://doi.org/10.1103/PhysRevLett.45.494} {\bibfield
  {journal} {\bibinfo  {journal} {Phys. Rev. Lett.}\ }\textbf {\bibinfo
  {volume} {45}},\ \bibinfo {pages} {494} (\bibinfo {year} {1980})}\BibitemShut
  {NoStop}%
\bibitem [{\citenamefont {Thouless}\ \emph {et~al.}(1982)\citenamefont
  {Thouless}, \citenamefont {Kohmoto}, \citenamefont {Nightingale},\ and\
  \citenamefont {den Nijs}}]{1982Thoulessprl}%
  \BibitemOpen
  \bibfield  {author} {\bibinfo {author} {\bibfnamefont {D.~J.}\ \bibnamefont
  {Thouless}}, \bibinfo {author} {\bibfnamefont {M.}~\bibnamefont {Kohmoto}},
  \bibinfo {author} {\bibfnamefont {M.~P.}\ \bibnamefont {Nightingale}},\ and\
  \bibinfo {author} {\bibfnamefont {M.}~\bibnamefont {den Nijs}},\ }\bibfield
  {title} {\bibinfo {title} {Quantized hall conductance in a two-dimensional
  periodic potential},\ }\href {https://doi.org/10.1103/PhysRevLett.49.405}
  {\bibfield  {journal} {\bibinfo  {journal} {Phys. Rev. Lett.}\ }\textbf
  {\bibinfo {volume} {49}},\ \bibinfo {pages} {405} (\bibinfo {year}
  {1982})}\BibitemShut {NoStop}%
\bibitem [{\citenamefont {{Berry}}(1984)}]{1984Berry}%
  \BibitemOpen
  \bibfield  {author} {\bibinfo {author} {\bibfnamefont {M.~V.}\ \bibnamefont
  {{Berry}}},\ }\bibfield  {title} {\bibinfo {title} {{Quantal Phase Factors
  Accompanying Adiabatic Changes}},\ }\href
  {https://doi.org/10.1098/rspa.1984.0023} {\bibfield  {journal} {\bibinfo
  {journal} {Proceedings of the Royal Society of London Series A}\ }\textbf
  {\bibinfo {volume} {392}},\ \bibinfo {pages} {45} (\bibinfo {year}
  {1984})}\BibitemShut {NoStop}%
\bibitem [{\citenamefont {{Bellissard}}\ \emph {et~al.}(1994)\citenamefont
  {{Bellissard}}, \citenamefont {{van Elst}},\ and\ \citenamefont
  {{Schulz-Baldes}}}]{1994JMP355373B}%
  \BibitemOpen
  \bibfield  {author} {\bibinfo {author} {\bibfnamefont {J.}~\bibnamefont
  {{Bellissard}}}, \bibinfo {author} {\bibfnamefont {A.}~\bibnamefont {{van
  Elst}}},\ and\ \bibinfo {author} {\bibfnamefont {H.}~\bibnamefont
  {{Schulz-Baldes}}},\ }\bibfield  {title} {\bibinfo {title} {{The
  noncommutative geometry of the quantum Hall effect}},\ }\href
  {https://doi.org/10.1063/1.530758} {\bibfield  {journal} {\bibinfo  {journal}
  {Journal of Mathematical Physics}\ }\textbf {\bibinfo {volume} {35}},\
  \bibinfo {pages} {5373} (\bibinfo {year} {1994})},\ \Eprint
  {https://arxiv.org/abs/cond-mat/9411052} {arXiv:cond-mat/9411052 [cond-mat]}
  \BibitemShut {NoStop}%
\bibitem [{\citenamefont {Hasan}\ and\ \citenamefont {Kane}(2010)}]{TIRMP}%
  \BibitemOpen
  \bibfield  {author} {\bibinfo {author} {\bibfnamefont {M.~Z.}\ \bibnamefont
  {Hasan}}\ and\ \bibinfo {author} {\bibfnamefont {C.~L.}\ \bibnamefont
  {Kane}},\ }\bibfield  {title} {\bibinfo {title} {Colloquium: Topological
  insulators},\ }\href {https://doi.org/10.1103/RevModPhys.82.3045} {\bibfield
  {journal} {\bibinfo  {journal} {Rev. Mod. Phys.}\ }\textbf {\bibinfo {volume}
  {82}},\ \bibinfo {pages} {3045} (\bibinfo {year} {2010})}\BibitemShut
  {NoStop}%
\bibitem [{\citenamefont {Qi}\ and\ \citenamefont {Zhang}(2011)}]{TISCRMP}%
  \BibitemOpen
  \bibfield  {author} {\bibinfo {author} {\bibfnamefont {X.-L.}\ \bibnamefont
  {Qi}}\ and\ \bibinfo {author} {\bibfnamefont {S.-C.}\ \bibnamefont {Zhang}},\
  }\bibfield  {title} {\bibinfo {title} {Topological insulators and
  superconductors},\ }\href {https://doi.org/10.1103/RevModPhys.83.1057}
  {\bibfield  {journal} {\bibinfo  {journal} {Rev. Mod. Phys.}\ }\textbf
  {\bibinfo {volume} {83}},\ \bibinfo {pages} {1057} (\bibinfo {year}
  {2011})}\BibitemShut {NoStop}%
\bibitem [{\citenamefont {Anandan}\ and\ \citenamefont
  {Aharonov}(1990)}]{1990AAgeometry}%
  \BibitemOpen
  \bibfield  {author} {\bibinfo {author} {\bibfnamefont {J.}~\bibnamefont
  {Anandan}}\ and\ \bibinfo {author} {\bibfnamefont {Y.}~\bibnamefont
  {Aharonov}},\ }\bibfield  {title} {\bibinfo {title} {Geometry of quantum
  evolution},\ }\href {https://doi.org/10.1103/PhysRevLett.65.1697} {\bibfield
  {journal} {\bibinfo  {journal} {Phys. Rev. Lett.}\ }\textbf {\bibinfo
  {volume} {65}},\ \bibinfo {pages} {1697} (\bibinfo {year}
  {1990})}\BibitemShut {NoStop}%
\bibitem [{\citenamefont {{Resta}}(2011)}]{2011EPJB79121R}%
  \BibitemOpen
  \bibfield  {author} {\bibinfo {author} {\bibfnamefont {R.}~\bibnamefont
  {{Resta}}},\ }\bibfield  {title} {\bibinfo {title} {{The insulating state of
  matter: a geometrical theory}},\ }\href
  {https://doi.org/10.1140/epjb/e2010-10874-4} {\bibfield  {journal} {\bibinfo
  {journal} {European Physical Journal B}\ }\textbf {\bibinfo {volume} {79}},\
  \bibinfo {pages} {121} (\bibinfo {year} {2011})},\ \Eprint
  {https://arxiv.org/abs/1012.5776} {arXiv:1012.5776 [cond-mat.mtrl-sci]}
  \BibitemShut {NoStop}%
\bibitem [{\citenamefont {Marzari}\ and\ \citenamefont
  {Vanderbilt}(1997)}]{PhysRevB.56.12847}%
  \BibitemOpen
  \bibfield  {author} {\bibinfo {author} {\bibfnamefont {N.}~\bibnamefont
  {Marzari}}\ and\ \bibinfo {author} {\bibfnamefont {D.}~\bibnamefont
  {Vanderbilt}},\ }\bibfield  {title} {\bibinfo {title} {Maximally localized
  generalized wannier functions for composite energy bands},\ }\href
  {https://doi.org/10.1103/PhysRevB.56.12847} {\bibfield  {journal} {\bibinfo
  {journal} {Phys. Rev. B}\ }\textbf {\bibinfo {volume} {56}},\ \bibinfo
  {pages} {12847} (\bibinfo {year} {1997})}\BibitemShut {NoStop}%
\bibitem [{\citenamefont {Neupert}\ \emph {et~al.}(2013)\citenamefont
  {Neupert}, \citenamefont {Chamon},\ and\ \citenamefont
  {Mudry}}]{PhysRevB.87.245103}%
  \BibitemOpen
  \bibfield  {author} {\bibinfo {author} {\bibfnamefont {T.}~\bibnamefont
  {Neupert}}, \bibinfo {author} {\bibfnamefont {C.}~\bibnamefont {Chamon}},\
  and\ \bibinfo {author} {\bibfnamefont {C.}~\bibnamefont {Mudry}},\ }\bibfield
   {title} {\bibinfo {title} {Measuring the quantum geometry of bloch bands
  with current noise},\ }\href {https://doi.org/10.1103/PhysRevB.87.245103}
  {\bibfield  {journal} {\bibinfo  {journal} {Phys. Rev. B}\ }\textbf {\bibinfo
  {volume} {87}},\ \bibinfo {pages} {245103} (\bibinfo {year}
  {2013})}\BibitemShut {NoStop}%
\bibitem [{\citenamefont {Haldane}(2011)}]{PhysRevLett.107.116801}%
  \BibitemOpen
  \bibfield  {author} {\bibinfo {author} {\bibfnamefont {F.~D.~M.}\
  \bibnamefont {Haldane}},\ }\bibfield  {title} {\bibinfo {title} {Geometrical
  description of the fractional quantum hall effect},\ }\href
  {https://doi.org/10.1103/PhysRevLett.107.116801} {\bibfield  {journal}
  {\bibinfo  {journal} {Phys. Rev. Lett.}\ }\textbf {\bibinfo {volume} {107}},\
  \bibinfo {pages} {116801} (\bibinfo {year} {2011})}\BibitemShut {NoStop}%
\bibitem [{\citenamefont {Girvin}\ \emph {et~al.}(1986)\citenamefont {Girvin},
  \citenamefont {MacDonald},\ and\ \citenamefont
  {Platzman}}]{PhysRevB.33.2481}%
  \BibitemOpen
  \bibfield  {author} {\bibinfo {author} {\bibfnamefont {S.~M.}\ \bibnamefont
  {Girvin}}, \bibinfo {author} {\bibfnamefont {A.~H.}\ \bibnamefont
  {MacDonald}},\ and\ \bibinfo {author} {\bibfnamefont {P.~M.}\ \bibnamefont
  {Platzman}},\ }\bibfield  {title} {\bibinfo {title} {Magneto-roton theory of
  collective excitations in the fractional quantum hall effect},\ }\href
  {https://doi.org/10.1103/PhysRevB.33.2481} {\bibfield  {journal} {\bibinfo
  {journal} {Phys. Rev. B}\ }\textbf {\bibinfo {volume} {33}},\ \bibinfo
  {pages} {2481} (\bibinfo {year} {1986})}\BibitemShut {NoStop}%
\bibitem [{\citenamefont {{Fogler}}(2002)}]{2002LNP59598F}%
  \BibitemOpen
  \bibfield  {author} {\bibinfo {author} {\bibfnamefont {M.~M.}\ \bibnamefont
  {{Fogler}}},\ }\bibfield  {title} {\bibinfo {title} {{Stripe and Bubble
  Phases in Quantum Hall Systems}},\ }in\ \href@noop {} {\emph {\bibinfo
  {booktitle} {High Magnetic Fields}}},\ Vol.\ \bibinfo {volume} {595},\
  \bibinfo {editor} {edited by\ \bibinfo {editor} {\bibfnamefont
  {C.}~\bibnamefont {{Berthier}}}, \bibinfo {editor} {\bibfnamefont {L.~P.}\
  \bibnamefont {{L{\'e}vy}}},\ and\ \bibinfo {editor} {\bibfnamefont
  {G.}~\bibnamefont {{Martinez}}}}\ (\bibinfo {year} {2002})\ pp.\ \bibinfo
  {pages} {98--138}\BibitemShut {NoStop}%
\bibitem [{\citenamefont {{Parameswaran}}\ \emph {et~al.}(2013)\citenamefont
  {{Parameswaran}}, \citenamefont {{Roy}},\ and\ \citenamefont
  {{Sondhi}}}]{2013CRPhy14816P}%
  \BibitemOpen
  \bibfield  {author} {\bibinfo {author} {\bibfnamefont {S.~A.}\ \bibnamefont
  {{Parameswaran}}}, \bibinfo {author} {\bibfnamefont {R.}~\bibnamefont
  {{Roy}}},\ and\ \bibinfo {author} {\bibfnamefont {S.~L.}\ \bibnamefont
  {{Sondhi}}},\ }\bibfield  {title} {\bibinfo {title} {{Fractional quantum Hall
  physics in topological flat bands}},\ }\href
  {https://doi.org/10.1016/j.crhy.2013.04.003} {\bibfield  {journal} {\bibinfo
  {journal} {Comptes Rendus Physique}\ }\textbf {\bibinfo {volume} {14}},\
  \bibinfo {pages} {816} (\bibinfo {year} {2013})},\ \Eprint
  {https://arxiv.org/abs/1302.6606} {arXiv:1302.6606 [cond-mat.str-el]}
  \BibitemShut {NoStop}%
\bibitem [{\citenamefont {Dobard\ifmmode \check{z}\else
  \v{z}\fi{}i\ifmmode~\acute{c}\else \'{c}\fi{}}\ \emph
  {et~al.}(2013)\citenamefont {Dobard\ifmmode \check{z}\else
  \v{z}\fi{}i\ifmmode~\acute{c}\else \'{c}\fi{}}, \citenamefont
  {Milovanovi\ifmmode~\acute{c}\else \'{c}\fi{}},\ and\ \citenamefont
  {Regnault}}]{PhysRevB.88.115117}%
  \BibitemOpen
  \bibfield  {author} {\bibinfo {author} {\bibfnamefont {E.}~\bibnamefont
  {Dobard\ifmmode \check{z}\else \v{z}\fi{}i\ifmmode~\acute{c}\else
  \'{c}\fi{}}}, \bibinfo {author} {\bibfnamefont {M.~V.}\ \bibnamefont
  {Milovanovi\ifmmode~\acute{c}\else \'{c}\fi{}}},\ and\ \bibinfo {author}
  {\bibfnamefont {N.}~\bibnamefont {Regnault}},\ }\bibfield  {title} {\bibinfo
  {title} {Geometrical description of fractional chern insulators based on
  static structure factor calculations},\ }\href
  {https://doi.org/10.1103/PhysRevB.88.115117} {\bibfield  {journal} {\bibinfo
  {journal} {Phys. Rev. B}\ }\textbf {\bibinfo {volume} {88}},\ \bibinfo
  {pages} {115117} (\bibinfo {year} {2013})}\BibitemShut {NoStop}%
\bibitem [{\citenamefont {Roy}(2014)}]{PhysRevB.90.165139}%
  \BibitemOpen
  \bibfield  {author} {\bibinfo {author} {\bibfnamefont {R.}~\bibnamefont
  {Roy}},\ }\bibfield  {title} {\bibinfo {title} {Band geometry of fractional
  topological insulators},\ }\href {https://doi.org/10.1103/PhysRevB.90.165139}
  {\bibfield  {journal} {\bibinfo  {journal} {Phys. Rev. B}\ }\textbf {\bibinfo
  {volume} {90}},\ \bibinfo {pages} {165139} (\bibinfo {year}
  {2014})}\BibitemShut {NoStop}%
\bibitem [{\citenamefont {Gao}\ \emph {et~al.}(2014)\citenamefont {Gao},
  \citenamefont {Yang},\ and\ \citenamefont {Niu}}]{PhysRevLett112166601}%
  \BibitemOpen
  \bibfield  {author} {\bibinfo {author} {\bibfnamefont {Y.}~\bibnamefont
  {Gao}}, \bibinfo {author} {\bibfnamefont {S.~A.}\ \bibnamefont {Yang}},\ and\
  \bibinfo {author} {\bibfnamefont {Q.}~\bibnamefont {Niu}},\ }\bibfield
  {title} {\bibinfo {title} {Field induced positional shift of bloch electrons
  and its dynamical implications},\ }\href
  {https://doi.org/10.1103/PhysRevLett.112.166601} {\bibfield  {journal}
  {\bibinfo  {journal} {Phys. Rev. Lett.}\ }\textbf {\bibinfo {volume} {112}},\
  \bibinfo {pages} {166601} (\bibinfo {year} {2014})}\BibitemShut {NoStop}%
\bibitem [{\citenamefont {Pi\'echon}\ \emph {et~al.}(2016)\citenamefont
  {Pi\'echon}, \citenamefont {Raoux}, \citenamefont {Fuchs},\ and\
  \citenamefont {Montambaux}}]{PhysRevB.94.134423}%
  \BibitemOpen
  \bibfield  {author} {\bibinfo {author} {\bibfnamefont {F.}~\bibnamefont
  {Pi\'echon}}, \bibinfo {author} {\bibfnamefont {A.}~\bibnamefont {Raoux}},
  \bibinfo {author} {\bibfnamefont {J.-N.}\ \bibnamefont {Fuchs}},\ and\
  \bibinfo {author} {\bibfnamefont {G.}~\bibnamefont {Montambaux}},\ }\bibfield
   {title} {\bibinfo {title} {Geometric orbital susceptibility: Quantum metric
  without berry curvature},\ }\href
  {https://doi.org/10.1103/PhysRevB.94.134423} {\bibfield  {journal} {\bibinfo
  {journal} {Phys. Rev. B}\ }\textbf {\bibinfo {volume} {94}},\ \bibinfo
  {pages} {134423} (\bibinfo {year} {2016})}\BibitemShut {NoStop}%
\bibitem [{\citenamefont {Chen}\ and\ \citenamefont
  {Huang}(2021)}]{PhysRevResearch.3.L042018}%
  \BibitemOpen
  \bibfield  {author} {\bibinfo {author} {\bibfnamefont {W.}~\bibnamefont
  {Chen}}\ and\ \bibinfo {author} {\bibfnamefont {W.}~\bibnamefont {Huang}},\
  }\bibfield  {title} {\bibinfo {title} {Quantum-geometry-induced intrinsic
  optical anomaly in multiorbital superconductors},\ }\href
  {https://doi.org/10.1103/PhysRevResearch.3.L042018} {\bibfield  {journal}
  {\bibinfo  {journal} {Phys. Rev. Research}\ }\textbf {\bibinfo {volume}
  {3}},\ \bibinfo {pages} {L042018} (\bibinfo {year} {2021})}\BibitemShut
  {NoStop}%
\bibitem [{\citenamefont {Ahn}\ and\ \citenamefont
  {Nagaosa}(2021)}]{PhysRevB.104.L100501}%
  \BibitemOpen
  \bibfield  {author} {\bibinfo {author} {\bibfnamefont {J.}~\bibnamefont
  {Ahn}}\ and\ \bibinfo {author} {\bibfnamefont {N.}~\bibnamefont {Nagaosa}},\
  }\bibfield  {title} {\bibinfo {title} {Superconductivity-induced spectral
  weight transfer due to quantum geometry},\ }\href
  {https://doi.org/10.1103/PhysRevB.104.L100501} {\bibfield  {journal}
  {\bibinfo  {journal} {Phys. Rev. B}\ }\textbf {\bibinfo {volume} {104}},\
  \bibinfo {pages} {L100501} (\bibinfo {year} {2021})}\BibitemShut {NoStop}%
\bibitem [{\citenamefont {Kozii}\ \emph {et~al.}(2021)\citenamefont {Kozii},
  \citenamefont {Avdoshkin}, \citenamefont {Zhong},\ and\ \citenamefont
  {Moore}}]{PhysRevLett.126.156602}%
  \BibitemOpen
  \bibfield  {author} {\bibinfo {author} {\bibfnamefont {V.}~\bibnamefont
  {Kozii}}, \bibinfo {author} {\bibfnamefont {A.}~\bibnamefont {Avdoshkin}},
  \bibinfo {author} {\bibfnamefont {S.}~\bibnamefont {Zhong}},\ and\ \bibinfo
  {author} {\bibfnamefont {J.~E.}\ \bibnamefont {Moore}},\ }\bibfield  {title}
  {\bibinfo {title} {Intrinsic anomalous hall conductivity in a nonuniform
  electric field},\ }\href {https://doi.org/10.1103/PhysRevLett.126.156602}
  {\bibfield  {journal} {\bibinfo  {journal} {Phys. Rev. Lett.}\ }\textbf
  {\bibinfo {volume} {126}},\ \bibinfo {pages} {156602} (\bibinfo {year}
  {2021})}\BibitemShut {NoStop}%
\bibitem [{\citenamefont {{Ahn}}\ \emph {et~al.}(2021)\citenamefont {{Ahn}},
  \citenamefont {{Guo}}, \citenamefont {{Nagaosa}},\ and\ \citenamefont
  {{Vishwanath}}}]{2021NatPh..18..290A}%
  \BibitemOpen
  \bibfield  {author} {\bibinfo {author} {\bibfnamefont {J.}~\bibnamefont
  {{Ahn}}}, \bibinfo {author} {\bibfnamefont {G.-Y.}\ \bibnamefont {{Guo}}},
  \bibinfo {author} {\bibfnamefont {N.}~\bibnamefont {{Nagaosa}}},\ and\
  \bibinfo {author} {\bibfnamefont {A.}~\bibnamefont {{Vishwanath}}},\
  }\bibfield  {title} {\bibinfo {title} {{Riemannian geometry of resonant
  optical responses}},\ }\href {https://doi.org/10.1038/s41567-021-01465-z}
  {\bibfield  {journal} {\bibinfo  {journal} {Nature Physics}\ }\textbf
  {\bibinfo {volume} {18}},\ \bibinfo {pages} {290} (\bibinfo {year} {2021})},\
  \Eprint {https://arxiv.org/abs/2103.01241} {arXiv:2103.01241
  [cond-mat.mes-hall]} \BibitemShut {NoStop}%
\bibitem [{\citenamefont {{Mitscherling}}\ and\ \citenamefont
  {{Holder}}(2022)}]{2022PhRvB.105h5154M}%
  \BibitemOpen
  \bibfield  {author} {\bibinfo {author} {\bibfnamefont {J.}~\bibnamefont
  {{Mitscherling}}}\ and\ \bibinfo {author} {\bibfnamefont {T.}~\bibnamefont
  {{Holder}}},\ }\bibfield  {title} {\bibinfo {title} {{Bound on resistivity in
  flat-band materials due to the quantum metric}},\ }\href
  {https://doi.org/10.1103/PhysRevB.105.085154} {\bibfield  {journal} {\bibinfo
   {journal} {\prb}\ }\textbf {\bibinfo {volume} {105}},\ \bibinfo {eid}
  {085154} (\bibinfo {year} {2022})},\ \Eprint
  {https://arxiv.org/abs/2110.14658} {arXiv:2110.14658 [cond-mat.mes-hall]}
  \BibitemShut {NoStop}%
\bibitem [{\citenamefont {{Peotta}}\ and\ \citenamefont
  {{T{\"o}rm{\"a}}}(2015)}]{2015NatCo68944P}%
  \BibitemOpen
  \bibfield  {author} {\bibinfo {author} {\bibfnamefont {S.}~\bibnamefont
  {{Peotta}}}\ and\ \bibinfo {author} {\bibfnamefont {P.}~\bibnamefont
  {{T{\"o}rm{\"a}}}},\ }\bibfield  {title} {\bibinfo {title} {{Superfluidity in
  topologically nontrivial flat bands}},\ }\href
  {https://doi.org/10.1038/ncomms9944} {\bibfield  {journal} {\bibinfo
  {journal} {Nature Communications}\ }\textbf {\bibinfo {volume} {6}},\
  \bibinfo {eid} {8944} (\bibinfo {year} {2015})},\ \Eprint
  {https://arxiv.org/abs/1506.02815} {arXiv:1506.02815 [cond-mat.supr-con]}
  \BibitemShut {NoStop}%
\bibitem [{\citenamefont {Liang}\ \emph {et~al.}(2017)\citenamefont {Liang},
  \citenamefont {Vanhala}, \citenamefont {Peotta}, \citenamefont {Siro},
  \citenamefont {Harju},\ and\ \citenamefont {T\"orm\"a}}]{PhysRevB.95.024515}%
  \BibitemOpen
  \bibfield  {author} {\bibinfo {author} {\bibfnamefont {L.}~\bibnamefont
  {Liang}}, \bibinfo {author} {\bibfnamefont {T.~I.}\ \bibnamefont {Vanhala}},
  \bibinfo {author} {\bibfnamefont {S.}~\bibnamefont {Peotta}}, \bibinfo
  {author} {\bibfnamefont {T.}~\bibnamefont {Siro}}, \bibinfo {author}
  {\bibfnamefont {A.}~\bibnamefont {Harju}},\ and\ \bibinfo {author}
  {\bibfnamefont {P.}~\bibnamefont {T\"orm\"a}},\ }\bibfield  {title} {\bibinfo
  {title} {Band geometry, berry curvature, and superfluid weight},\ }\href
  {https://doi.org/10.1103/PhysRevB.95.024515} {\bibfield  {journal} {\bibinfo
  {journal} {Phys. Rev. B}\ }\textbf {\bibinfo {volume} {95}},\ \bibinfo
  {pages} {024515} (\bibinfo {year} {2017})}\BibitemShut {NoStop}%
\bibitem [{\citenamefont {T\"orm\"a}\ \emph {et~al.}(2018)\citenamefont
  {T\"orm\"a}, \citenamefont {Liang},\ and\ \citenamefont
  {Peotta}}]{PhysRevB.98.220511}%
  \BibitemOpen
  \bibfield  {author} {\bibinfo {author} {\bibfnamefont {P.}~\bibnamefont
  {T\"orm\"a}}, \bibinfo {author} {\bibfnamefont {L.}~\bibnamefont {Liang}},\
  and\ \bibinfo {author} {\bibfnamefont {S.}~\bibnamefont {Peotta}},\
  }\bibfield  {title} {\bibinfo {title} {Quantum metric and effective mass of a
  two-body bound state in a flat band},\ }\href
  {https://doi.org/10.1103/PhysRevB.98.220511} {\bibfield  {journal} {\bibinfo
  {journal} {Phys. Rev. B}\ }\textbf {\bibinfo {volume} {98}},\ \bibinfo
  {pages} {220511} (\bibinfo {year} {2018})}\BibitemShut {NoStop}%
\bibitem [{\citenamefont {{Jip Park}}\ \emph {et~al.}(2020)\citenamefont {{Jip
  Park}}, \citenamefont {{Kim}},\ and\ \citenamefont
  {{Lee}}}]{2020arXiv200716205J}%
  \BibitemOpen
  \bibfield  {author} {\bibinfo {author} {\bibfnamefont {M.}~\bibnamefont {{Jip
  Park}}}, \bibinfo {author} {\bibfnamefont {Y.~B.}\ \bibnamefont {{Kim}}},\
  and\ \bibinfo {author} {\bibfnamefont {S.}~\bibnamefont {{Lee}}},\ }\bibfield
   {title} {\bibinfo {title} {{Geometric Superconductivity in 3D Hofstadter
  Butterfly}},\ }\href@noop {} {\bibfield  {journal} {\bibinfo  {journal}
  {arXiv e-prints}\ ,\ \bibinfo {eid} {arXiv:2007.16205}} (\bibinfo {year}
  {2020})},\ \Eprint {https://arxiv.org/abs/2007.16205} {arXiv:2007.16205
  [cond-mat.supr-con]} \BibitemShut {NoStop}%
\bibitem [{\citenamefont {Julku}\ \emph {et~al.}(2016)\citenamefont {Julku},
  \citenamefont {Peotta}, \citenamefont {Vanhala}, \citenamefont {Kim},\ and\
  \citenamefont {T\"orm\"a}}]{PhysRevLett.117.045303}%
  \BibitemOpen
  \bibfield  {author} {\bibinfo {author} {\bibfnamefont {A.}~\bibnamefont
  {Julku}}, \bibinfo {author} {\bibfnamefont {S.}~\bibnamefont {Peotta}},
  \bibinfo {author} {\bibfnamefont {T.~I.}\ \bibnamefont {Vanhala}}, \bibinfo
  {author} {\bibfnamefont {D.-H.}\ \bibnamefont {Kim}},\ and\ \bibinfo {author}
  {\bibfnamefont {P.}~\bibnamefont {T\"orm\"a}},\ }\bibfield  {title} {\bibinfo
  {title} {Geometric origin of superfluidity in the lieb-lattice flat band},\
  }\href {https://doi.org/10.1103/PhysRevLett.117.045303} {\bibfield  {journal}
  {\bibinfo  {journal} {Phys. Rev. Lett.}\ }\textbf {\bibinfo {volume} {117}},\
  \bibinfo {pages} {045303} (\bibinfo {year} {2016})}\BibitemShut {NoStop}%
\bibitem [{\citenamefont {Julku}\ \emph {et~al.}(2021)\citenamefont {Julku},
  \citenamefont {Bruun},\ and\ \citenamefont
  {T\"orm\"a}}]{PhysRevLett.127.170404}%
  \BibitemOpen
  \bibfield  {author} {\bibinfo {author} {\bibfnamefont {A.}~\bibnamefont
  {Julku}}, \bibinfo {author} {\bibfnamefont {G.~M.}\ \bibnamefont {Bruun}},\
  and\ \bibinfo {author} {\bibfnamefont {P.}~\bibnamefont {T\"orm\"a}},\
  }\bibfield  {title} {\bibinfo {title} {Quantum geometry and flat band
  bose-einstein condensation},\ }\href
  {https://doi.org/10.1103/PhysRevLett.127.170404} {\bibfield  {journal}
  {\bibinfo  {journal} {Phys. Rev. Lett.}\ }\textbf {\bibinfo {volume} {127}},\
  \bibinfo {pages} {170404} (\bibinfo {year} {2021})}\BibitemShut {NoStop}%
\bibitem [{\citenamefont {Herzog-Arbeitman}\ \emph {et~al.}(2022)\citenamefont
  {Herzog-Arbeitman}, \citenamefont {Peri}, \citenamefont {Schindler},
  \citenamefont {Huber},\ and\ \citenamefont
  {Bernevig}}]{PhysRevLett.128.087002}%
  \BibitemOpen
  \bibfield  {author} {\bibinfo {author} {\bibfnamefont {J.}~\bibnamefont
  {Herzog-Arbeitman}}, \bibinfo {author} {\bibfnamefont {V.}~\bibnamefont
  {Peri}}, \bibinfo {author} {\bibfnamefont {F.}~\bibnamefont {Schindler}},
  \bibinfo {author} {\bibfnamefont {S.~D.}\ \bibnamefont {Huber}},\ and\
  \bibinfo {author} {\bibfnamefont {B.~A.}\ \bibnamefont {Bernevig}},\
  }\bibfield  {title} {\bibinfo {title} {Superfluid weight bounds from symmetry
  and quantum geometry in flat bands},\ }\href
  {https://doi.org/10.1103/PhysRevLett.128.087002} {\bibfield  {journal}
  {\bibinfo  {journal} {Phys. Rev. Lett.}\ }\textbf {\bibinfo {volume} {128}},\
  \bibinfo {pages} {087002} (\bibinfo {year} {2022})}\BibitemShut {NoStop}%
\bibitem [{\citenamefont {Huhtinen}\ \emph {et~al.}(2022)\citenamefont
  {Huhtinen}, \citenamefont {Herzog-Arbeitman}, \citenamefont {Chew},
  \citenamefont {Bernevig},\ and\ \citenamefont
  {T\"orm\"a}}]{PhysRevB.106.014518}%
  \BibitemOpen
  \bibfield  {author} {\bibinfo {author} {\bibfnamefont {K.-E.}\ \bibnamefont
  {Huhtinen}}, \bibinfo {author} {\bibfnamefont {J.}~\bibnamefont
  {Herzog-Arbeitman}}, \bibinfo {author} {\bibfnamefont {A.}~\bibnamefont
  {Chew}}, \bibinfo {author} {\bibfnamefont {B.~A.}\ \bibnamefont {Bernevig}},\
  and\ \bibinfo {author} {\bibfnamefont {P.}~\bibnamefont {T\"orm\"a}},\
  }\bibfield  {title} {\bibinfo {title} {Revisiting flat band
  superconductivity: Dependence on minimal quantum metric and band touchings},\
  }\href {https://doi.org/10.1103/PhysRevB.106.014518} {\bibfield  {journal}
  {\bibinfo  {journal} {Phys. Rev. B}\ }\textbf {\bibinfo {volume} {106}},\
  \bibinfo {pages} {014518} (\bibinfo {year} {2022})}\BibitemShut {NoStop}%
\bibitem [{\citenamefont {Hu}\ \emph {et~al.}(2022)\citenamefont {Hu},
  \citenamefont {Hyart}, \citenamefont {Pikulin},\ and\ \citenamefont
  {Rossi}}]{PhysRevB.105.L140506}%
  \BibitemOpen
  \bibfield  {author} {\bibinfo {author} {\bibfnamefont {X.}~\bibnamefont
  {Hu}}, \bibinfo {author} {\bibfnamefont {T.}~\bibnamefont {Hyart}}, \bibinfo
  {author} {\bibfnamefont {D.~I.}\ \bibnamefont {Pikulin}},\ and\ \bibinfo
  {author} {\bibfnamefont {E.}~\bibnamefont {Rossi}},\ }\bibfield  {title}
  {\bibinfo {title} {Quantum-metric-enabled exciton condensate in double
  twisted bilayer graphene},\ }\href
  {https://doi.org/10.1103/PhysRevB.105.L140506} {\bibfield  {journal}
  {\bibinfo  {journal} {Phys. Rev. B}\ }\textbf {\bibinfo {volume} {105}},\
  \bibinfo {pages} {L140506} (\bibinfo {year} {2022})}\BibitemShut {NoStop}%
\bibitem [{\citenamefont {Chan}\ \emph {et~al.}(2022)\citenamefont {Chan},
  \citenamefont {Gr\'emaud},\ and\ \citenamefont
  {Batrouni}}]{PhysRevB.106.104514}%
  \BibitemOpen
  \bibfield  {author} {\bibinfo {author} {\bibfnamefont {S.~M.}\ \bibnamefont
  {Chan}}, \bibinfo {author} {\bibfnamefont {B.}~\bibnamefont {Gr\'emaud}},\
  and\ \bibinfo {author} {\bibfnamefont {G.~G.}\ \bibnamefont {Batrouni}},\
  }\bibfield  {title} {\bibinfo {title} {Designer flat bands: Topology and
  enhancement of superconductivity},\ }\href
  {https://doi.org/10.1103/PhysRevB.106.104514} {\bibfield  {journal} {\bibinfo
   {journal} {Phys. Rev. B}\ }\textbf {\bibinfo {volume} {106}},\ \bibinfo
  {pages} {104514} (\bibinfo {year} {2022})}\BibitemShut {NoStop}%
\bibitem [{\citenamefont {{Herzog-Arbeitman}}\ \emph
  {et~al.}(2022)\citenamefont {{Herzog-Arbeitman}}, \citenamefont {{Chew}},
  \citenamefont {{Huhtinen}}, \citenamefont {{T{\"o}rm{\"a}}},\ and\
  \citenamefont {{Bernevig}}}]{2022arXiv220900007H}%
  \BibitemOpen
  \bibfield  {author} {\bibinfo {author} {\bibfnamefont {J.}~\bibnamefont
  {{Herzog-Arbeitman}}}, \bibinfo {author} {\bibfnamefont {A.}~\bibnamefont
  {{Chew}}}, \bibinfo {author} {\bibfnamefont {K.-E.}\ \bibnamefont
  {{Huhtinen}}}, \bibinfo {author} {\bibfnamefont {P.}~\bibnamefont
  {{T{\"o}rm{\"a}}}},\ and\ \bibinfo {author} {\bibfnamefont {B.~A.}\
  \bibnamefont {{Bernevig}}},\ }\bibfield  {title} {\bibinfo {title}
  {{Many-Body Superconductivity in Topological Flat Bands}},\ }\href
  {https://doi.org/10.48550/arXiv.2209.00007} {\bibfield  {journal} {\bibinfo
  {journal} {arXiv e-prints}\ ,\ \bibinfo {eid} {arXiv:2209.00007}} (\bibinfo
  {year} {2022})},\ \Eprint {https://arxiv.org/abs/2209.00007}
  {arXiv:2209.00007 [cond-mat.str-el]} \BibitemShut {NoStop}%
\bibitem [{\citenamefont {{Thonhauser}}\ and\ \citenamefont
  {{Vanderbilt}}(2006)}]{2006PhRvB..74w5111T}%
  \BibitemOpen
  \bibfield  {author} {\bibinfo {author} {\bibfnamefont {T.}~\bibnamefont
  {{Thonhauser}}}\ and\ \bibinfo {author} {\bibfnamefont {D.}~\bibnamefont
  {{Vanderbilt}}},\ }\bibfield  {title} {\bibinfo {title}
  {{Insulator/Chern-insulator transition in the Haldane model}},\ }\href
  {https://doi.org/10.1103/PhysRevB.74.235111} {\bibfield  {journal} {\bibinfo
  {journal} {\prb}\ }\textbf {\bibinfo {volume} {74}},\ \bibinfo {eid} {235111}
  (\bibinfo {year} {2006})},\ \Eprint {https://arxiv.org/abs/cond-mat/0608527}
  {arXiv:cond-mat/0608527 [cond-mat.mes-hall]} \BibitemShut {NoStop}%
\bibitem [{\citenamefont {Campos~Venuti}\ and\ \citenamefont
  {Zanardi}(2007)}]{PhysRevLett99095701}%
  \BibitemOpen
  \bibfield  {author} {\bibinfo {author} {\bibfnamefont {L.}~\bibnamefont
  {Campos~Venuti}}\ and\ \bibinfo {author} {\bibfnamefont {P.}~\bibnamefont
  {Zanardi}},\ }\bibfield  {title} {\bibinfo {title} {Quantum critical scaling
  of the geometric tensors},\ }\href
  {https://doi.org/10.1103/PhysRevLett.99.095701} {\bibfield  {journal}
  {\bibinfo  {journal} {Phys. Rev. Lett.}\ }\textbf {\bibinfo {volume} {99}},\
  \bibinfo {pages} {095701} (\bibinfo {year} {2007})}\BibitemShut {NoStop}%
\bibitem [{\citenamefont {Zanardi}\ \emph {et~al.}(2007)\citenamefont
  {Zanardi}, \citenamefont {Giorda},\ and\ \citenamefont
  {Cozzini}}]{PhysRevLett99100603}%
  \BibitemOpen
  \bibfield  {author} {\bibinfo {author} {\bibfnamefont {P.}~\bibnamefont
  {Zanardi}}, \bibinfo {author} {\bibfnamefont {P.}~\bibnamefont {Giorda}},\
  and\ \bibinfo {author} {\bibfnamefont {M.}~\bibnamefont {Cozzini}},\
  }\bibfield  {title} {\bibinfo {title} {Information-theoretic differential
  geometry of quantum phase transitions},\ }\href
  {https://doi.org/10.1103/PhysRevLett.99.100603} {\bibfield  {journal}
  {\bibinfo  {journal} {Phys. Rev. Lett.}\ }\textbf {\bibinfo {volume} {99}},\
  \bibinfo {pages} {100603} (\bibinfo {year} {2007})}\BibitemShut {NoStop}%
\bibitem [{\citenamefont {{Verma}}\ \emph {et~al.}(2021)\citenamefont
  {{Verma}}, \citenamefont {{Hazra}},\ and\ \citenamefont
  {{Randeria}}}]{2021PNAS11806744V}%
  \BibitemOpen
  \bibfield  {author} {\bibinfo {author} {\bibfnamefont {N.}~\bibnamefont
  {{Verma}}}, \bibinfo {author} {\bibfnamefont {T.}~\bibnamefont {{Hazra}}},\
  and\ \bibinfo {author} {\bibfnamefont {M.}~\bibnamefont {{Randeria}}},\
  }\bibfield  {title} {\bibinfo {title} {{Optical spectral weight, phase
  stiffness, and T$_{c}$ bounds for trivial and topological flat band
  superconductors}},\ }\href {https://doi.org/10.1073/pnas.2106744118}
  {\bibfield  {journal} {\bibinfo  {journal} {Proceedings of the National
  Academy of Science}\ }\textbf {\bibinfo {volume} {118}},\ \bibinfo {eid}
  {e2106744118} (\bibinfo {year} {2021})},\ \Eprint
  {https://arxiv.org/abs/2103.08540} {arXiv:2103.08540 [cond-mat.supr-con]}
  \BibitemShut {NoStop}%
\bibitem [{\citenamefont {{Cao}}\ \emph
  {et~al.}(2018{\natexlab{a}})\citenamefont {{Cao}}, \citenamefont {{Fatemi}},
  \citenamefont {{Demir}}, \citenamefont {{Fang}}, \citenamefont {{Tomarken}},
  \citenamefont {{Luo}}, \citenamefont {{Sanchez-Yamagishi}}, \citenamefont
  {{Watanabe}}, \citenamefont {{Taniguchi}}, \citenamefont {{Kaxiras}},
  \citenamefont {{Ashoori}},\ and\ \citenamefont
  {{Jarillo-Herrero}}}]{2018Natur.556...80C}%
  \BibitemOpen
  \bibfield  {author} {\bibinfo {author} {\bibfnamefont {Y.}~\bibnamefont
  {{Cao}}}, \bibinfo {author} {\bibfnamefont {V.}~\bibnamefont {{Fatemi}}},
  \bibinfo {author} {\bibfnamefont {A.}~\bibnamefont {{Demir}}}, \bibinfo
  {author} {\bibfnamefont {S.}~\bibnamefont {{Fang}}}, \bibinfo {author}
  {\bibfnamefont {S.~L.}\ \bibnamefont {{Tomarken}}}, \bibinfo {author}
  {\bibfnamefont {J.~Y.}\ \bibnamefont {{Luo}}}, \bibinfo {author}
  {\bibfnamefont {J.~D.}\ \bibnamefont {{Sanchez-Yamagishi}}}, \bibinfo
  {author} {\bibfnamefont {K.}~\bibnamefont {{Watanabe}}}, \bibinfo {author}
  {\bibfnamefont {T.}~\bibnamefont {{Taniguchi}}}, \bibinfo {author}
  {\bibfnamefont {E.}~\bibnamefont {{Kaxiras}}}, \bibinfo {author}
  {\bibfnamefont {R.~C.}\ \bibnamefont {{Ashoori}}},\ and\ \bibinfo {author}
  {\bibfnamefont {P.}~\bibnamefont {{Jarillo-Herrero}}},\ }\bibfield  {title}
  {\bibinfo {title} {{Correlated insulator behaviour at half-filling in
  magic-angle graphene superlattices}},\ }\href
  {https://doi.org/10.1038/nature26154} {\bibfield  {journal} {\bibinfo
  {journal} {\nat}\ }\textbf {\bibinfo {volume} {556}},\ \bibinfo {pages} {80}
  (\bibinfo {year} {2018}{\natexlab{a}})},\ \Eprint
  {https://arxiv.org/abs/1802.00553} {arXiv:1802.00553 [cond-mat.mes-hall]}
  \BibitemShut {NoStop}%
\bibitem [{\citenamefont {{Cao}}\ \emph
  {et~al.}(2018{\natexlab{b}})\citenamefont {{Cao}}, \citenamefont {{Fatemi}},
  \citenamefont {{Fang}}, \citenamefont {{Watanabe}}, \citenamefont
  {{Taniguchi}}, \citenamefont {{Kaxiras}},\ and\ \citenamefont
  {{Jarillo-Herrero}}}]{2018Natur55643C}%
  \BibitemOpen
  \bibfield  {author} {\bibinfo {author} {\bibfnamefont {Y.}~\bibnamefont
  {{Cao}}}, \bibinfo {author} {\bibfnamefont {V.}~\bibnamefont {{Fatemi}}},
  \bibinfo {author} {\bibfnamefont {S.}~\bibnamefont {{Fang}}}, \bibinfo
  {author} {\bibfnamefont {K.}~\bibnamefont {{Watanabe}}}, \bibinfo {author}
  {\bibfnamefont {T.}~\bibnamefont {{Taniguchi}}}, \bibinfo {author}
  {\bibfnamefont {E.}~\bibnamefont {{Kaxiras}}},\ and\ \bibinfo {author}
  {\bibfnamefont {P.}~\bibnamefont {{Jarillo-Herrero}}},\ }\bibfield  {title}
  {\bibinfo {title} {{Unconventional superconductivity in magic-angle graphene
  superlattices}},\ }\href {https://doi.org/10.1038/nature26160} {\bibfield
  {journal} {\bibinfo  {journal} {\nat}\ }\textbf {\bibinfo {volume} {556}},\
  \bibinfo {pages} {43} (\bibinfo {year} {2018}{\natexlab{b}})},\ \Eprint
  {https://arxiv.org/abs/1803.02342} {arXiv:1803.02342 [cond-mat.mes-hall]}
  \BibitemShut {NoStop}%
\bibitem [{\citenamefont {{Lu}}\ \emph {et~al.}(2019)\citenamefont {{Lu}},
  \citenamefont {{Stepanov}}, \citenamefont {{Yang}}, \citenamefont {{Xie}},
  \citenamefont {{Aamir}}, \citenamefont {{Das}}, \citenamefont {{Urgell}},
  \citenamefont {{Watanabe}}, \citenamefont {{Taniguchi}}, \citenamefont
  {{Zhang}}, \citenamefont {{Bachtold}}, \citenamefont {{MacDonald}},\ and\
  \citenamefont {{Efetov}}}]{2019Natur.574..653L}%
  \BibitemOpen
  \bibfield  {author} {\bibinfo {author} {\bibfnamefont {X.}~\bibnamefont
  {{Lu}}}, \bibinfo {author} {\bibfnamefont {P.}~\bibnamefont {{Stepanov}}},
  \bibinfo {author} {\bibfnamefont {W.}~\bibnamefont {{Yang}}}, \bibinfo
  {author} {\bibfnamefont {M.}~\bibnamefont {{Xie}}}, \bibinfo {author}
  {\bibfnamefont {M.~A.}\ \bibnamefont {{Aamir}}}, \bibinfo {author}
  {\bibfnamefont {I.}~\bibnamefont {{Das}}}, \bibinfo {author} {\bibfnamefont
  {C.}~\bibnamefont {{Urgell}}}, \bibinfo {author} {\bibfnamefont
  {K.}~\bibnamefont {{Watanabe}}}, \bibinfo {author} {\bibfnamefont
  {T.}~\bibnamefont {{Taniguchi}}}, \bibinfo {author} {\bibfnamefont
  {G.}~\bibnamefont {{Zhang}}}, \bibinfo {author} {\bibfnamefont
  {A.}~\bibnamefont {{Bachtold}}}, \bibinfo {author} {\bibfnamefont {A.~H.}\
  \bibnamefont {{MacDonald}}},\ and\ \bibinfo {author} {\bibfnamefont {D.~K.}\
  \bibnamefont {{Efetov}}},\ }\bibfield  {title} {\bibinfo {title}
  {{Superconductors, orbital magnets and correlated states in magic-angle
  bilayer graphene}},\ }\href {https://doi.org/10.1038/s41586-019-1695-0}
  {\bibfield  {journal} {\bibinfo  {journal} {\nat}\ }\textbf {\bibinfo
  {volume} {574}},\ \bibinfo {pages} {653} (\bibinfo {year} {2019})},\ \Eprint
  {https://arxiv.org/abs/1903.06513} {arXiv:1903.06513 [cond-mat.str-el]}
  \BibitemShut {NoStop}%
\bibitem [{\citenamefont {Hazra}\ \emph {et~al.}(2019)\citenamefont {Hazra},
  \citenamefont {Verma},\ and\ \citenamefont {Randeria}}]{PhysRevX.9.031049}%
  \BibitemOpen
  \bibfield  {author} {\bibinfo {author} {\bibfnamefont {T.}~\bibnamefont
  {Hazra}}, \bibinfo {author} {\bibfnamefont {N.}~\bibnamefont {Verma}},\ and\
  \bibinfo {author} {\bibfnamefont {M.}~\bibnamefont {Randeria}},\ }\bibfield
  {title} {\bibinfo {title} {Bounds on the superconducting transition
  temperature: Applications to twisted bilayer graphene and cold atoms},\
  }\href {https://doi.org/10.1103/PhysRevX.9.031049} {\bibfield  {journal}
  {\bibinfo  {journal} {Phys. Rev. X}\ }\textbf {\bibinfo {volume} {9}},\
  \bibinfo {pages} {031049} (\bibinfo {year} {2019})}\BibitemShut {NoStop}%
\bibitem [{\citenamefont {Hu}\ \emph {et~al.}(2019)\citenamefont {Hu},
  \citenamefont {Hyart}, \citenamefont {Pikulin},\ and\ \citenamefont
  {Rossi}}]{PhysRevLett.123.237002}%
  \BibitemOpen
  \bibfield  {author} {\bibinfo {author} {\bibfnamefont {X.}~\bibnamefont
  {Hu}}, \bibinfo {author} {\bibfnamefont {T.}~\bibnamefont {Hyart}}, \bibinfo
  {author} {\bibfnamefont {D.~I.}\ \bibnamefont {Pikulin}},\ and\ \bibinfo
  {author} {\bibfnamefont {E.}~\bibnamefont {Rossi}},\ }\bibfield  {title}
  {\bibinfo {title} {Geometric and conventional contribution to the superfluid
  weight in twisted bilayer graphene},\ }\href
  {https://doi.org/10.1103/PhysRevLett.123.237002} {\bibfield  {journal}
  {\bibinfo  {journal} {Phys. Rev. Lett.}\ }\textbf {\bibinfo {volume} {123}},\
  \bibinfo {pages} {237002} (\bibinfo {year} {2019})}\BibitemShut {NoStop}%
\bibitem [{\citenamefont {Julku}\ \emph {et~al.}(2020)\citenamefont {Julku},
  \citenamefont {Peltonen}, \citenamefont {Liang}, \citenamefont {Heikkil\"a},\
  and\ \citenamefont {T\"orm\"a}}]{PhysRevB.101.060505}%
  \BibitemOpen
  \bibfield  {author} {\bibinfo {author} {\bibfnamefont {A.}~\bibnamefont
  {Julku}}, \bibinfo {author} {\bibfnamefont {T.~J.}\ \bibnamefont {Peltonen}},
  \bibinfo {author} {\bibfnamefont {L.}~\bibnamefont {Liang}}, \bibinfo
  {author} {\bibfnamefont {T.~T.}\ \bibnamefont {Heikkil\"a}},\ and\ \bibinfo
  {author} {\bibfnamefont {P.}~\bibnamefont {T\"orm\"a}},\ }\bibfield  {title}
  {\bibinfo {title} {Superfluid weight and berezinskii-kosterlitz-thouless
  transition temperature of twisted bilayer graphene},\ }\href
  {https://doi.org/10.1103/PhysRevB.101.060505} {\bibfield  {journal} {\bibinfo
   {journal} {Phys. Rev. B}\ }\textbf {\bibinfo {volume} {101}},\ \bibinfo
  {pages} {060505} (\bibinfo {year} {2020})}\BibitemShut {NoStop}%
\bibitem [{\citenamefont {Ledwith}\ \emph {et~al.}(2020)\citenamefont
  {Ledwith}, \citenamefont {Tarnopolsky}, \citenamefont {Khalaf},\ and\
  \citenamefont {Vishwanath}}]{PhysRevResearch.2.023237}%
  \BibitemOpen
  \bibfield  {author} {\bibinfo {author} {\bibfnamefont {P.~J.}\ \bibnamefont
  {Ledwith}}, \bibinfo {author} {\bibfnamefont {G.}~\bibnamefont
  {Tarnopolsky}}, \bibinfo {author} {\bibfnamefont {E.}~\bibnamefont
  {Khalaf}},\ and\ \bibinfo {author} {\bibfnamefont {A.}~\bibnamefont
  {Vishwanath}},\ }\bibfield  {title} {\bibinfo {title} {Fractional chern
  insulator states in twisted bilayer graphene: An analytical approach},\
  }\href {https://doi.org/10.1103/PhysRevResearch.2.023237} {\bibfield
  {journal} {\bibinfo  {journal} {Phys. Rev. Research}\ }\textbf {\bibinfo
  {volume} {2}},\ \bibinfo {pages} {023237} (\bibinfo {year}
  {2020})}\BibitemShut {NoStop}%
\bibitem [{\citenamefont {{Xie}}\ \emph {et~al.}(2020)\citenamefont {{Xie}},
  \citenamefont {{Song}}, \citenamefont {{Lian}},\ and\ \citenamefont
  {{Bernevig}}}]{2020PhRvL124p7002X}%
  \BibitemOpen
  \bibfield  {author} {\bibinfo {author} {\bibfnamefont {F.}~\bibnamefont
  {{Xie}}}, \bibinfo {author} {\bibfnamefont {Z.}~\bibnamefont {{Song}}},
  \bibinfo {author} {\bibfnamefont {B.}~\bibnamefont {{Lian}}},\ and\ \bibinfo
  {author} {\bibfnamefont {B.~A.}\ \bibnamefont {{Bernevig}}},\ }\bibfield
  {title} {\bibinfo {title} {{Topology-Bounded Superfluid Weight in Twisted
  Bilayer Graphene}},\ }\href {https://doi.org/10.1103/PhysRevLett.124.167002}
  {\bibfield  {journal} {\bibinfo  {journal} {\prl}\ }\textbf {\bibinfo
  {volume} {124}},\ \bibinfo {eid} {167002} (\bibinfo {year} {2020})},\ \Eprint
  {https://arxiv.org/abs/1906.02213} {arXiv:1906.02213 [cond-mat.supr-con]}
  \BibitemShut {NoStop}%
\bibitem [{\citenamefont {Chaudhary}\ \emph {et~al.}(2022)\citenamefont
  {Chaudhary}, \citenamefont {Lewandowski},\ and\ \citenamefont
  {Refael}}]{PhysRevResearch4013164}%
  \BibitemOpen
  \bibfield  {author} {\bibinfo {author} {\bibfnamefont {S.}~\bibnamefont
  {Chaudhary}}, \bibinfo {author} {\bibfnamefont {C.}~\bibnamefont
  {Lewandowski}},\ and\ \bibinfo {author} {\bibfnamefont {G.}~\bibnamefont
  {Refael}},\ }\bibfield  {title} {\bibinfo {title} {Shift-current response as
  a probe of quantum geometry and electron-electron interactions in twisted
  bilayer graphene},\ }\href {https://doi.org/10.1103/PhysRevResearch.4.013164}
  {\bibfield  {journal} {\bibinfo  {journal} {Phys. Rev. Research}\ }\textbf
  {\bibinfo {volume} {4}},\ \bibinfo {pages} {013164} (\bibinfo {year}
  {2022})}\BibitemShut {NoStop}%
\bibitem [{\citenamefont {Mera}\ and\ \citenamefont
  {Ozawa}(2021)}]{PhysRevB.104.115160}%
  \BibitemOpen
  \bibfield  {author} {\bibinfo {author} {\bibfnamefont {B.}~\bibnamefont
  {Mera}}\ and\ \bibinfo {author} {\bibfnamefont {T.}~\bibnamefont {Ozawa}},\
  }\bibfield  {title} {\bibinfo {title} {Engineering geometrically flat chern
  bands with fubini-study k\"ahler structure},\ }\href
  {https://doi.org/10.1103/PhysRevB.104.115160} {\bibfield  {journal} {\bibinfo
   {journal} {Phys. Rev. B}\ }\textbf {\bibinfo {volume} {104}},\ \bibinfo
  {pages} {115160} (\bibinfo {year} {2021})}\BibitemShut {NoStop}%
\bibitem [{\citenamefont {Kaplan}\ \emph {et~al.}(2022)\citenamefont {Kaplan},
  \citenamefont {Holder},\ and\ \citenamefont
  {Yan}}]{PhysRevResearch.4.013209}%
  \BibitemOpen
  \bibfield  {author} {\bibinfo {author} {\bibfnamefont {D.}~\bibnamefont
  {Kaplan}}, \bibinfo {author} {\bibfnamefont {T.}~\bibnamefont {Holder}},\
  and\ \bibinfo {author} {\bibfnamefont {B.}~\bibnamefont {Yan}},\ }\bibfield
  {title} {\bibinfo {title} {Twisted photovoltaics at terahertz frequencies
  from momentum shift current},\ }\href
  {https://doi.org/10.1103/PhysRevResearch.4.013209} {\bibfield  {journal}
  {\bibinfo  {journal} {Phys. Rev. Research}\ }\textbf {\bibinfo {volume}
  {4}},\ \bibinfo {pages} {013209} (\bibinfo {year} {2022})}\BibitemShut
  {NoStop}%
\bibitem [{\citenamefont {{T{\"o}rm{\"a}}}\ \emph {et~al.}(2022)\citenamefont
  {{T{\"o}rm{\"a}}}, \citenamefont {{Peotta}},\ and\ \citenamefont
  {{Bernevig}}}]{2021arXiv211100807T}%
  \BibitemOpen
  \bibfield  {author} {\bibinfo {author} {\bibfnamefont {P.}~\bibnamefont
  {{T{\"o}rm{\"a}}}}, \bibinfo {author} {\bibfnamefont {S.}~\bibnamefont
  {{Peotta}}},\ and\ \bibinfo {author} {\bibfnamefont {B.~A.}\ \bibnamefont
  {{Bernevig}}},\ }\bibfield  {title} {\bibinfo {title} {{Superfluidity and
  Quantum Geometry in Twisted Multilayer Systems}},\ }\href
  {https://doi.org/10.1038/s42254-022-00466-y} {\bibfield  {journal} {\bibinfo
  {journal} {Nat. Rev. Phys.}\ }\textbf {\bibinfo {volume} {4}},\ \bibinfo
  {pages} {528–542} (\bibinfo {year} {2022})}\BibitemShut {NoStop}%
\bibitem [{\citenamefont {Wang}\ \emph {et~al.}(2021)\citenamefont {Wang},
  \citenamefont {Cano}, \citenamefont {Millis}, \citenamefont {Liu},\ and\
  \citenamefont {Yang}}]{PhysRevLett.127.246403}%
  \BibitemOpen
  \bibfield  {author} {\bibinfo {author} {\bibfnamefont {J.}~\bibnamefont
  {Wang}}, \bibinfo {author} {\bibfnamefont {J.}~\bibnamefont {Cano}}, \bibinfo
  {author} {\bibfnamefont {A.~J.}\ \bibnamefont {Millis}}, \bibinfo {author}
  {\bibfnamefont {Z.}~\bibnamefont {Liu}},\ and\ \bibinfo {author}
  {\bibfnamefont {B.}~\bibnamefont {Yang}},\ }\bibfield  {title} {\bibinfo
  {title} {Exact landau level description of geometry and interaction in a
  flatband},\ }\href {https://doi.org/10.1103/PhysRevLett.127.246403}
  {\bibfield  {journal} {\bibinfo  {journal} {Phys. Rev. Lett.}\ }\textbf
  {\bibinfo {volume} {127}},\ \bibinfo {pages} {246403} (\bibinfo {year}
  {2021})}\BibitemShut {NoStop}%
\bibitem [{\citenamefont {Wang}\ and\ \citenamefont
  {Liu}(2022)}]{PhysRevLett.128.176403}%
  \BibitemOpen
  \bibfield  {author} {\bibinfo {author} {\bibfnamefont {J.}~\bibnamefont
  {Wang}}\ and\ \bibinfo {author} {\bibfnamefont {Z.}~\bibnamefont {Liu}},\
  }\bibfield  {title} {\bibinfo {title} {Hierarchy of ideal flatbands in chiral
  twisted multilayer graphene models},\ }\href
  {https://doi.org/10.1103/PhysRevLett.128.176403} {\bibfield  {journal}
  {\bibinfo  {journal} {Phys. Rev. Lett.}\ }\textbf {\bibinfo {volume} {128}},\
  \bibinfo {pages} {176403} (\bibinfo {year} {2022})}\BibitemShut {NoStop}%
\bibitem [{\citenamefont {{Tian}}\ \emph {et~al.}(2021)\citenamefont {{Tian}},
  \citenamefont {{Che}}, \citenamefont {{Xu}}, \citenamefont {{Cheung}},
  \citenamefont {{Watanabe}}, \citenamefont {{Taniguchi}}, \citenamefont
  {{Randeria}}, \citenamefont {{Zhang}}, \citenamefont {{Lau}},\ and\
  \citenamefont {{Bockrath}}}]{2021arXiv211213401T}%
  \BibitemOpen
  \bibfield  {author} {\bibinfo {author} {\bibfnamefont {H.}~\bibnamefont
  {{Tian}}}, \bibinfo {author} {\bibfnamefont {S.}~\bibnamefont {{Che}}},
  \bibinfo {author} {\bibfnamefont {T.}~\bibnamefont {{Xu}}}, \bibinfo {author}
  {\bibfnamefont {P.}~\bibnamefont {{Cheung}}}, \bibinfo {author}
  {\bibfnamefont {K.}~\bibnamefont {{Watanabe}}}, \bibinfo {author}
  {\bibfnamefont {T.}~\bibnamefont {{Taniguchi}}}, \bibinfo {author}
  {\bibfnamefont {M.}~\bibnamefont {{Randeria}}}, \bibinfo {author}
  {\bibfnamefont {F.}~\bibnamefont {{Zhang}}}, \bibinfo {author} {\bibfnamefont
  {C.~N.}\ \bibnamefont {{Lau}}},\ and\ \bibinfo {author} {\bibfnamefont
  {M.~W.}\ \bibnamefont {{Bockrath}}},\ }\bibfield  {title} {\bibinfo {title}
  {{Evidence for Flat Band Dirac Superconductor Originating from Quantum
  Geometry}},\ }\href {https://doi.org/10.48550/arXiv.2112.13401} {\bibfield
  {journal} {\bibinfo  {journal} {arXiv e-prints}\ ,\ \bibinfo {eid}
  {arXiv:2112.13401}} (\bibinfo {year} {2021})},\ \Eprint
  {https://arxiv.org/abs/2112.13401} {arXiv:2112.13401 [cond-mat.supr-con]}
  \BibitemShut {NoStop}%
\bibitem [{\citenamefont {Altland}\ and\ \citenamefont
  {Simons}(2010)}]{altland2010condensed}%
  \BibitemOpen
  \bibfield  {author} {\bibinfo {author} {\bibfnamefont {A.}~\bibnamefont
  {Altland}}\ and\ \bibinfo {author} {\bibfnamefont {B.~D.}\ \bibnamefont
  {Simons}},\ }\href@noop {} {\emph {\bibinfo {title} {Condensed matter field
  theory}}}\ (\bibinfo  {publisher} {Cambridge university press},\ \bibinfo
  {year} {2010})\BibitemShut {NoStop}%
\bibitem [{\citenamefont {Scalapino}\ \emph {et~al.}(1992)\citenamefont
  {Scalapino}, \citenamefont {White},\ and\ \citenamefont
  {Zhang}}]{PhysRevLett.68.2830}%
  \BibitemOpen
  \bibfield  {author} {\bibinfo {author} {\bibfnamefont {D.~J.}\ \bibnamefont
  {Scalapino}}, \bibinfo {author} {\bibfnamefont {S.~R.}\ \bibnamefont
  {White}},\ and\ \bibinfo {author} {\bibfnamefont {S.~C.}\ \bibnamefont
  {Zhang}},\ }\bibfield  {title} {\bibinfo {title} {Superfluid density and the
  drude weight of the hubbard model},\ }\href
  {https://doi.org/10.1103/PhysRevLett.68.2830} {\bibfield  {journal} {\bibinfo
   {journal} {Phys. Rev. Lett.}\ }\textbf {\bibinfo {volume} {68}},\ \bibinfo
  {pages} {2830} (\bibinfo {year} {1992})}\BibitemShut {NoStop}%
\bibitem [{\citenamefont {Scalapino}\ \emph {et~al.}(1993)\citenamefont
  {Scalapino}, \citenamefont {White},\ and\ \citenamefont
  {Zhang}}]{PhysRevB.47.7995}%
  \BibitemOpen
  \bibfield  {author} {\bibinfo {author} {\bibfnamefont {D.~J.}\ \bibnamefont
  {Scalapino}}, \bibinfo {author} {\bibfnamefont {S.~R.}\ \bibnamefont
  {White}},\ and\ \bibinfo {author} {\bibfnamefont {S.}~\bibnamefont {Zhang}},\
  }\bibfield  {title} {\bibinfo {title} {Insulator, metal, or superconductor:
  The criteria},\ }\href {https://doi.org/10.1103/PhysRevB.47.7995} {\bibfield
  {journal} {\bibinfo  {journal} {Phys. Rev. B}\ }\textbf {\bibinfo {volume}
  {47}},\ \bibinfo {pages} {7995} (\bibinfo {year} {1993})}\BibitemShut
  {NoStop}%
\bibitem [{\citenamefont {Gor’kov}(1959)}]{gor1959microscopic}%
  \BibitemOpen
  \bibfield  {author} {\bibinfo {author} {\bibfnamefont {L.~P.}\ \bibnamefont
  {Gor’kov}},\ }\bibfield  {title} {\bibinfo {title} {{Microscopic derivation
  of the Ginzburg-Landau equations in the theory of superconductivity}},\
  }\href@noop {} {\bibfield  {journal} {\bibinfo  {journal} {Sov. Phys. JETP}\
  }\textbf {\bibinfo {volume} {9}},\ \bibinfo {pages} {1364} (\bibinfo {year}
  {1959})}\BibitemShut {NoStop}%
\bibitem [{\citenamefont {Haldane}(2004)}]{PhysRevLett.93.206602}%
  \BibitemOpen
  \bibfield  {author} {\bibinfo {author} {\bibfnamefont {F.~D.~M.}\
  \bibnamefont {Haldane}},\ }\bibfield  {title} {\bibinfo {title} {Berry
  curvature on the fermi surface: Anomalous hall effect as a topological
  fermi-liquid property},\ }\href
  {https://doi.org/10.1103/PhysRevLett.93.206602} {\bibfield  {journal}
  {\bibinfo  {journal} {Phys. Rev. Lett.}\ }\textbf {\bibinfo {volume} {93}},\
  \bibinfo {pages} {206602} (\bibinfo {year} {2004})}\BibitemShut {NoStop}%
\bibitem [{\citenamefont {{Chen}}\ and\ \citenamefont
  {{Son}}(2017)}]{2017AnPhy.377..345C}%
  \BibitemOpen
  \bibfield  {author} {\bibinfo {author} {\bibfnamefont {J.-Y.}\ \bibnamefont
  {{Chen}}}\ and\ \bibinfo {author} {\bibfnamefont {D.~T.}\ \bibnamefont
  {{Son}}},\ }\bibfield  {title} {\bibinfo {title} {{Berry Fermi liquid
  theory}},\ }\href {https://doi.org/10.1016/j.aop.2016.12.017} {\bibfield
  {journal} {\bibinfo  {journal} {Annals of Physics}\ }\textbf {\bibinfo
  {volume} {377}},\ \bibinfo {pages} {345} (\bibinfo {year} {2017})},\ \Eprint
  {https://arxiv.org/abs/1604.07857} {arXiv:1604.07857 [cond-mat.str-el]}
  \BibitemShut {NoStop}%
\bibitem [{\citenamefont {{Cheng}}(2010)}]{2010arXiv1012.1337C}%
  \BibitemOpen
  \bibfield  {author} {\bibinfo {author} {\bibfnamefont {R.}~\bibnamefont
  {{Cheng}}},\ }\bibfield  {title} {\bibinfo {title} {{Quantum Geometric Tensor
  (Fubini-Study Metric) in Simple Quantum System: A pedagogical
  Introduction}},\ }\href {https://doi.org/10.48550/arXiv.1012.1337} {\bibfield
   {journal} {\bibinfo  {journal} {arXiv e-prints}\ ,\ \bibinfo {eid}
  {arXiv:1012.1337}} (\bibinfo {year} {2010})},\ \Eprint
  {https://arxiv.org/abs/1012.1337} {arXiv:1012.1337 [quant-ph]} \BibitemShut
  {NoStop}%
\bibitem [{\citenamefont {Nelson}\ and\ \citenamefont
  {Kosterlitz}(1977)}]{PhysRevLett.39.1201}%
  \BibitemOpen
  \bibfield  {author} {\bibinfo {author} {\bibfnamefont {D.~R.}\ \bibnamefont
  {Nelson}}\ and\ \bibinfo {author} {\bibfnamefont {J.~M.}\ \bibnamefont
  {Kosterlitz}},\ }\bibfield  {title} {\bibinfo {title} {Universal jump in the
  superfluid density of two-dimensional superfluids},\ }\href
  {https://doi.org/10.1103/PhysRevLett.39.1201} {\bibfield  {journal} {\bibinfo
   {journal} {Phys. Rev. Lett.}\ }\textbf {\bibinfo {volume} {39}},\ \bibinfo
  {pages} {1201} (\bibinfo {year} {1977})}\BibitemShut {NoStop}%
\bibitem [{\citenamefont {{Tarnopolsky}}\ \emph {et~al.}(2019)\citenamefont
  {{Tarnopolsky}}, \citenamefont {{Kruchkov}},\ and\ \citenamefont
  {{Vishwanath}}}]{2019PhRvL.122j6405T}%
  \BibitemOpen
  \bibfield  {author} {\bibinfo {author} {\bibfnamefont {G.}~\bibnamefont
  {{Tarnopolsky}}}, \bibinfo {author} {\bibfnamefont {A.~J.}\ \bibnamefont
  {{Kruchkov}}},\ and\ \bibinfo {author} {\bibfnamefont {A.}~\bibnamefont
  {{Vishwanath}}},\ }\bibfield  {title} {\bibinfo {title} {{Origin of Magic
  Angles in Twisted Bilayer Graphene}},\ }\href
  {https://doi.org/10.1103/PhysRevLett.122.106405} {\bibfield  {journal}
  {\bibinfo  {journal} {\prl}\ }\textbf {\bibinfo {volume} {122}},\ \bibinfo
  {eid} {106405} (\bibinfo {year} {2019})},\ \Eprint
  {https://arxiv.org/abs/1808.05250} {arXiv:1808.05250 [cond-mat.str-el]}
  \BibitemShut {NoStop}%
\bibitem [{\citenamefont {Liu}\ \emph {et~al.}(2019)\citenamefont {Liu},
  \citenamefont {Liu},\ and\ \citenamefont {Dai}}]{PhysRevB.99.155415}%
  \BibitemOpen
  \bibfield  {author} {\bibinfo {author} {\bibfnamefont {J.}~\bibnamefont
  {Liu}}, \bibinfo {author} {\bibfnamefont {J.}~\bibnamefont {Liu}},\ and\
  \bibinfo {author} {\bibfnamefont {X.}~\bibnamefont {Dai}},\ }\bibfield
  {title} {\bibinfo {title} {Pseudo landau level representation of twisted
  bilayer graphene: Band topology and implications on the correlated insulating
  phase},\ }\href {https://doi.org/10.1103/PhysRevB.99.155415} {\bibfield
  {journal} {\bibinfo  {journal} {Phys. Rev. B}\ }\textbf {\bibinfo {volume}
  {99}},\ \bibinfo {pages} {155415} (\bibinfo {year} {2019})}\BibitemShut
  {NoStop}%
\bibitem [{\citenamefont {Hofstadter}(1976)}]{PhysRevB.14.2239}%
  \BibitemOpen
  \bibfield  {author} {\bibinfo {author} {\bibfnamefont {D.~R.}\ \bibnamefont
  {Hofstadter}},\ }\bibfield  {title} {\bibinfo {title} {Energy levels and wave
  functions of bloch electrons in rational and irrational magnetic fields},\
  }\href {https://doi.org/10.1103/PhysRevB.14.2239} {\bibfield  {journal}
  {\bibinfo  {journal} {Phys. Rev. B}\ }\textbf {\bibinfo {volume} {14}},\
  \bibinfo {pages} {2239} (\bibinfo {year} {1976})}\BibitemShut {NoStop}%
\bibitem [{\citenamefont {Harper}\ \emph {et~al.}(2014)\citenamefont {Harper},
  \citenamefont {Simon},\ and\ \citenamefont {Roy}}]{PhysRevB.90.075104}%
  \BibitemOpen
  \bibfield  {author} {\bibinfo {author} {\bibfnamefont {F.}~\bibnamefont
  {Harper}}, \bibinfo {author} {\bibfnamefont {S.~H.}\ \bibnamefont {Simon}},\
  and\ \bibinfo {author} {\bibfnamefont {R.}~\bibnamefont {Roy}},\ }\bibfield
  {title} {\bibinfo {title} {Perturbative approach to flat chern bands in the
  hofstadter model},\ }\href {https://doi.org/10.1103/PhysRevB.90.075104}
  {\bibfield  {journal} {\bibinfo  {journal} {Phys. Rev. B}\ }\textbf {\bibinfo
  {volume} {90}},\ \bibinfo {pages} {075104} (\bibinfo {year}
  {2014})}\BibitemShut {NoStop}%
\bibitem [{\citenamefont {{Ozawa}}\ and\ \citenamefont
  {{Mera}}(2021)}]{2021PhRvB.104d5103O}%
  \BibitemOpen
  \bibfield  {author} {\bibinfo {author} {\bibfnamefont {T.}~\bibnamefont
  {{Ozawa}}}\ and\ \bibinfo {author} {\bibfnamefont {B.}~\bibnamefont
  {{Mera}}},\ }\bibfield  {title} {\bibinfo {title} {{Relations between
  topology and the quantum metric for Chern insulators}},\ }\href
  {https://doi.org/10.1103/PhysRevB.104.045103} {\bibfield  {journal} {\bibinfo
   {journal} {\prb}\ }\textbf {\bibinfo {volume} {104}},\ \bibinfo {eid}
  {045103} (\bibinfo {year} {2021})},\ \Eprint
  {https://arxiv.org/abs/2103.11582} {arXiv:2103.11582 [cond-mat.mes-hall]}
  \BibitemShut {NoStop}%
\bibitem [{\citenamefont {{Bistritzer}}\ and\ \citenamefont
  {{MacDonald}}(2011)}]{2011PNAS..10812233B}%
  \BibitemOpen
  \bibfield  {author} {\bibinfo {author} {\bibfnamefont {R.}~\bibnamefont
  {{Bistritzer}}}\ and\ \bibinfo {author} {\bibfnamefont {A.~H.}\ \bibnamefont
  {{MacDonald}}},\ }\bibfield  {title} {\bibinfo {title} {{Moir{\'e} bands in
  twisted double-layer graphene}},\ }\href
  {https://doi.org/10.1073/pnas.1108174108} {\bibfield  {journal} {\bibinfo
  {journal} {Proceedings of the National Academy of Science}\ }\textbf
  {\bibinfo {volume} {108}},\ \bibinfo {pages} {12233} (\bibinfo {year}
  {2011})},\ \Eprint {https://arxiv.org/abs/1009.4203} {arXiv:1009.4203
  [cond-mat.mes-hall]} \BibitemShut {NoStop}%
\bibitem [{\citenamefont {Koshino}\ \emph {et~al.}(2018)\citenamefont
  {Koshino}, \citenamefont {Yuan}, \citenamefont {Koretsune}, \citenamefont
  {Ochi}, \citenamefont {Kuroki},\ and\ \citenamefont
  {Fu}}]{PhysRevX.8.031087}%
  \BibitemOpen
  \bibfield  {author} {\bibinfo {author} {\bibfnamefont {M.}~\bibnamefont
  {Koshino}}, \bibinfo {author} {\bibfnamefont {N.~F.~Q.}\ \bibnamefont
  {Yuan}}, \bibinfo {author} {\bibfnamefont {T.}~\bibnamefont {Koretsune}},
  \bibinfo {author} {\bibfnamefont {M.}~\bibnamefont {Ochi}}, \bibinfo {author}
  {\bibfnamefont {K.}~\bibnamefont {Kuroki}},\ and\ \bibinfo {author}
  {\bibfnamefont {L.}~\bibnamefont {Fu}},\ }\bibfield  {title} {\bibinfo
  {title} {Maximally localized wannier orbitals and the extended hubbard model
  for twisted bilayer graphene},\ }\href
  {https://doi.org/10.1103/PhysRevX.8.031087} {\bibfield  {journal} {\bibinfo
  {journal} {Phys. Rev. X}\ }\textbf {\bibinfo {volume} {8}},\ \bibinfo {pages}
  {031087} (\bibinfo {year} {2018})}\BibitemShut {NoStop}%
\bibitem [{\citenamefont {{Kitamura}}\ \emph
  {et~al.}(2022{\natexlab{a}})\citenamefont {{Kitamura}}, \citenamefont
  {{Daido}},\ and\ \citenamefont {{Yanase}}}]{2022arXiv220613682K}%
  \BibitemOpen
  \bibfield  {author} {\bibinfo {author} {\bibfnamefont {T.}~\bibnamefont
  {{Kitamura}}}, \bibinfo {author} {\bibfnamefont {A.}~\bibnamefont
  {{Daido}}},\ and\ \bibinfo {author} {\bibfnamefont {Y.}~\bibnamefont
  {{Yanase}}},\ }\bibfield  {title} {\bibinfo {title} {{Quantum geometric
  effect on Fulde-Ferrell-Larkin-Ovchinnikov superconductivity}},\ }\href@noop
  {} {\bibfield  {journal} {\bibinfo  {journal} {arXiv e-prints}\ ,\ \bibinfo
  {eid} {arXiv:2206.13682}} (\bibinfo {year} {2022}{\natexlab{a}})},\ \Eprint
  {https://arxiv.org/abs/2206.13682} {arXiv:2206.13682 [cond-mat.supr-con]}
  \BibitemShut {NoStop}%
\bibitem [{\citenamefont {{Chen}}\ and\ \citenamefont
  {{Huang}}(2022)}]{2022arXiv220802285C}%
  \BibitemOpen
  \bibfield  {author} {\bibinfo {author} {\bibfnamefont {W.}~\bibnamefont
  {{Chen}}}\ and\ \bibinfo {author} {\bibfnamefont {W.}~\bibnamefont
  {{Huang}}},\ }\bibfield  {title} {\bibinfo {title} {{Pair density wave
  facilitated by Bloch quantum geometry in nearly flat band multiorbital
  superconductors}},\ }\href@noop {} {\bibfield  {journal} {\bibinfo  {journal}
  {arXiv e-prints}\ ,\ \bibinfo {eid} {arXiv:2208.02285}} (\bibinfo {year}
  {2022})},\ \Eprint {https://arxiv.org/abs/2208.02285} {arXiv:2208.02285
  [cond-mat.supr-con]} \BibitemShut {NoStop}%
\bibitem [{\citenamefont {{Kitamura}}\ \emph
  {et~al.}(2022{\natexlab{b}})\citenamefont {{Kitamura}}, \citenamefont
  {{Daido}},\ and\ \citenamefont {{Yanase}}}]{2022PhRvB.106r4507K}%
  \BibitemOpen
  \bibfield  {author} {\bibinfo {author} {\bibfnamefont {T.}~\bibnamefont
  {{Kitamura}}}, \bibinfo {author} {\bibfnamefont {A.}~\bibnamefont
  {{Daido}}},\ and\ \bibinfo {author} {\bibfnamefont {Y.}~\bibnamefont
  {{Yanase}}},\ }\bibfield  {title} {\bibinfo {title} {{Quantum geometric
  effect on Fulde-Ferrell-Larkin-Ovchinnikov superconductivity}},\ }\href
  {https://doi.org/10.1103/PhysRevB.106.184507} {\bibfield  {journal} {\bibinfo
   {journal} {\prb}\ }\textbf {\bibinfo {volume} {106}},\ \bibinfo {eid}
  {184507} (\bibinfo {year} {2022}{\natexlab{b}})},\ \Eprint
  {https://arxiv.org/abs/2206.13682} {arXiv:2206.13682 [cond-mat.supr-con]}
  \BibitemShut {NoStop}%
\bibitem [{\citenamefont {{Kopnin}}\ and\ \citenamefont
  {{Sonin}}(2008)}]{2008PhRvL.100x6808K}%
  \BibitemOpen
  \bibfield  {author} {\bibinfo {author} {\bibfnamefont {N.~B.}\ \bibnamefont
  {{Kopnin}}}\ and\ \bibinfo {author} {\bibfnamefont {E.~B.}\ \bibnamefont
  {{Sonin}}},\ }\bibfield  {title} {\bibinfo {title} {{BCS Superconductivity of
  Dirac Electrons in Graphene Layers}},\ }\href
  {https://doi.org/10.1103/PhysRevLett.100.246808} {\bibfield  {journal}
  {\bibinfo  {journal} {\prl}\ }\textbf {\bibinfo {volume} {100}},\ \bibinfo
  {eid} {246808} (\bibinfo {year} {2008})},\ \Eprint
  {https://arxiv.org/abs/0803.3772} {arXiv:0803.3772 [cond-mat.supr-con]}
  \BibitemShut {NoStop}%
\bibitem [{\citenamefont {Kauppila}\ \emph {et~al.}(2016)\citenamefont
  {Kauppila}, \citenamefont {Aikebaier},\ and\ \citenamefont
  {Heikkil\"a}}]{PhysRevB.93.214505}%
  \BibitemOpen
  \bibfield  {author} {\bibinfo {author} {\bibfnamefont {V.~J.}\ \bibnamefont
  {Kauppila}}, \bibinfo {author} {\bibfnamefont {F.}~\bibnamefont
  {Aikebaier}},\ and\ \bibinfo {author} {\bibfnamefont {T.~T.}\ \bibnamefont
  {Heikkil\"a}},\ }\bibfield  {title} {\bibinfo {title} {Flat-band
  superconductivity in strained dirac materials},\ }\href
  {https://doi.org/10.1103/PhysRevB.93.214505} {\bibfield  {journal} {\bibinfo
  {journal} {Phys. Rev. B}\ }\textbf {\bibinfo {volume} {93}},\ \bibinfo
  {pages} {214505} (\bibinfo {year} {2016})}\BibitemShut {NoStop}%
\bibitem [{\citenamefont {{Peltonen}}\ and\ \citenamefont
  {{Heikkil{\"a}}}(2020)}]{2020JPCM...32J5603P}%
  \BibitemOpen
  \bibfield  {author} {\bibinfo {author} {\bibfnamefont {T.~J.}\ \bibnamefont
  {{Peltonen}}}\ and\ \bibinfo {author} {\bibfnamefont {T.~T.}\ \bibnamefont
  {{Heikkil{\"a}}}},\ }\bibfield  {title} {\bibinfo {title} {{Flat-band
  superconductivity in periodically strained graphene: mean-field and
  Berezinskii-Kosterlitz-Thouless transition}},\ }\href
  {https://doi.org/10.1088/1361-648X/ab8b9d} {\bibfield  {journal} {\bibinfo
  {journal} {Journal of Physics Condensed Matter}\ }\textbf {\bibinfo {volume}
  {32}},\ \bibinfo {eid} {365603} (\bibinfo {year} {2020})},\ \Eprint
  {https://arxiv.org/abs/1910.06671} {arXiv:1910.06671 [cond-mat.supr-con]}
  \BibitemShut {NoStop}%
\bibitem [{\citenamefont {Horn}\ and\ \citenamefont
  {Johnson}(2012)}]{horn2012matrix}%
  \BibitemOpen
  \bibfield  {author} {\bibinfo {author} {\bibfnamefont {R.~A.}\ \bibnamefont
  {Horn}}\ and\ \bibinfo {author} {\bibfnamefont {C.~R.}\ \bibnamefont
  {Johnson}},\ }\href@noop {} {\emph {\bibinfo {title} {Matrix analysis}}}\
  (\bibinfo  {publisher} {Cambridge university press},\ \bibinfo {year}
  {2012})\BibitemShut {NoStop}%
\end{thebibliography}%


\newpage
\clearpage

\begin{appendix}
\onecolumngrid
\begin{center}
\textbf{\centering \large Supplemental Material for ``Towards a Ginzburg-Landau theory of the quantum geometric effect in
superconductors"}\\
        \medskip
        \text{Shuai A. Chen, K. T. Law}
\end{center}
       

% \maketitle
%\appendix
\section{Details on deriving the Free energy \label{app:free} }

We present the derivation details on the GL theory. We illustrate
the isolated flatband system. We begin with a Hamiltonian $H=H_{0}+H_{\mathrm{int}}$, with
$H_{0}$ the free part and 
\begin{equation}
H_{\mathrm{int}}=-g\int d\mathbf{r}a_{+}^{\dagger}(\mathbf{r})a_{-}^{\dagger}(\mathbf{r})a_{-}(\mathbf{r})a_{+}(\mathbf{r})
\end{equation}
as the an attractive interaction. We say that an isolated band limit
for $H_{0}$ is referred to as a large band gap $W$ between the band
around the Fermi energy and others while the interaction part should
not significantly change the band structure $W\gg\vert g\vert$. For
simplicity, consider the ideal flatband limit, and then we can have
an effective Hamiltonian 
\begin{equation}
H_{0}=0~,
\end{equation}
in the low-energy limit. To deal with the interaction we introduce
the Hubbard-Stratonovich (HS) transformation 
\begin{equation}
1=\int\mathcal{D}[\Delta,\bar{\Delta}]e^{-g\int_{0}^{\beta}d\tau\int d\mathbf{r}[\Delta(\mathbf{r})-a_{-}(\mathbf{r})a_{+}(\mathbf{r})][\bar{\Delta}(\mathbf{r})-\bar{a}_{+}(\mathbf{r})\bar{a}_{+}(\mathbf{r})]}~,
\end{equation}
with $\bar{a},a$ being the Grassmann fields. With the HS transformation,
we have the path integral formula 
\begin{align}
Z & =\mathrm{Tr}e^{-\beta H_{\mathrm{int}}}=\int\mathcal{D}[\Delta,\bar{\Delta}]e^{-\int_{0}^{\beta}d\tau\int d\mathbf{r}\vert\Delta(\mathbf{r})\vert^{2}}\mathcal{Z}[\Delta,\bar{\Delta}]~,
\end{align}
with 
\begin{align}
\mathcal{Z}[\Delta,\bar{\Delta}] & =\int\mathcal{D}[c,\bar{c}]e^{-\int_{0}^{\beta}d\tau\int d\mathbf{r}\mathcal{L}[a,\bar{a},\Delta,\bar{\Delta}]}~,
\end{align}
and 
\begin{align}
\mathcal{L}[a,\bar{a},\Delta,\bar{\Delta}]= & (\partial_{\tau}-\mu)(\bar{a}_{+}(\mathbf{r})a_{+}(\mathbf{r})+\bar{a}_{-}(\mathbf{r})a_{-}(\mathbf{r}))-g\left[\Delta(\mathbf{r})\bar{a}_{+}(\mathbf{r})a_{+}(\mathbf{r})+h.c.\right]~.
\end{align}
As emphasized in the main text, it is essential to introduce a projection
in Eq.~(2) in the main text or 
\begin{equation}
a_{\xi}(\mathbf{r})\rightarrow\frac{1}{N}\sum_{\mathbf{q}}e^{i\mathbf{q}\cdot\mathbf{r}}g_{\mathbf{q}\xi}^{*}(\alpha)c_{\mathbf{q}\xi}~,
\end{equation}
to explicitly encode the Bloch wave with $\alpha$ labeling the internal degrees of freedom.
Consequently, we have a projected Lagrangian density $\mathcal{L}[c,\bar{c},\Delta,\bar{\Delta}]$
\begin{equation}
\mathcal{L}[a,\bar{a},\Delta,\bar{\Delta}]\rightarrow\mathcal{L}[c,\bar{c},\Delta,\bar{\Delta}]~,
\end{equation}
with 
\begin{align}
\mathcal{L}[c,\bar{c},\Delta,\bar{\Delta}]= & (\partial_{\tau}-\mu)(\bar{c}_{\mathbf{q},+}c_{\mathbf{q},+}+\bar{c}_{\mathbf{q},-}c_{\mathbf{q},-}) -\sum_{\mathbf{k}}g[\Gamma(\mathbf{q},\mathbf{k})\Delta(\mathbf{k})\bar{c}_{\mathbf{q+\frac{\mathbf{k}}{2}},+}\bar{c}_{-\mathbf{q+\frac{\mathbf{k}}{2}},-}+h.c.]~.
\end{align}
It is crucial that the interaction between the bosonic field $\Delta$
and the fermion field $c$ is decorated by a form factor $g\Gamma(\mathbf{q},\mathbf{k})$.
Therefore, we arrive at the GL theory $S=S[\Delta,\bar{\Delta}]$
by the formula 
\begin{align}
Z & \equiv\int\mathcal{D}[\Delta,\bar{\Delta}]e^{-\beta F[\Delta,\bar{\Delta}]}~,
\end{align}
or 
\begin{align}
F[\Delta,\bar{\Delta}] & =\sum_{\mathbf{k}}g\vert\Delta(\mathbf{k})\vert^{2}-T\ln\mathcal{Z}[\Delta,\bar{\Delta}]\nonumber \\
 & =\sum_{\mathbf{k}}g\vert\Delta(\mathbf{k})\vert^{2}-T\int\mathcal{D}[c,\bar{c}]e^{-\int_{0}^{\beta}d\tau\int d\mathbf{r}\mathcal{L}[c,\bar{c},\Delta,\bar{\Delta}]}\\
 & =\sum_{\mathbf{k}}g\vert\Delta(\mathbf{k})\vert^{2}-T\ln\det\mathrm{G}~,
\end{align}
where $\mathrm{G}$ is the kernel that we arrange $\mathcal{L}[c,\bar{c},\Delta,\bar{\Delta}]$
into a matrix form 
\begin{align}
\mathcal{L}[c,\bar{c},\Delta,\bar{\Delta}] & =\sum_{\mathbf{q}^{\prime}\mathbf{q}}\begin{bmatrix}\bar{c}_{\mathbf{q}^{\prime},+}\\
c_{-\mathbf{q}^{\prime},-}
\end{bmatrix}\mathrm{G}_{\mathbf{q}^{\prime}\mathbf{q}}\begin{bmatrix}c_{\mathbf{q},+} & \bar{c}_{-\mathbf{q},-}\end{bmatrix}~.
\end{align}
The remaining step is to expand the determinant $\ln\det\mathrm{G}$
by following the standard procedures \cite{altland2010condensed}.
Here instead, we can apply the Gor'kov's Green functions. We shall
first make the expansion on the bosonic field $\Delta(\mathbf{k})$
around the extremum free energy 
\begin{equation}
\Delta(\mathbf{k})=\Delta_{0}\delta_{\mathbf{k},0}+\delta\Delta(\mathbf{k})~,
\end{equation}
where $\Delta_{0}$ and $\delta\Delta(\mathbf{k})$ respectively denote
the mean field solution and the fluctuations. We also assume $\Delta_{0}$
real with a proper gauge choice. The decomposition splits the $\mathcal{L}[c,\bar{c},\Delta,\bar{\Delta}]$
into two parts 
\begin{equation}
\mathcal{L}[c,\bar{c},\Delta,\bar{\Delta}]=\mathcal{L}_{0}+\mathcal{L}_{\mathrm{int}}~,
\end{equation}
where $\mathcal{L}_{0}$ as the leading order reproduces the BCS mean
field theory 
\begin{align}
\mathcal{L}_{0}= & (\partial_{\tau}-\mu)(\bar{c}_{\mathbf{q},+}c_{\mathbf{q},+}+\bar{c}_{\mathbf{q},-}c_{\mathbf{q},-}) -g[\Gamma(\mathbf{q})\Delta_{0}\bar{c}_{\mathbf{q},+}\bar{c}_{-\mathbf{q},-}+h.c.]~,\label{app:Lo}
\end{align}
with $\Gamma(\mathbf{q})=\Gamma(\mathbf{q},\mathbf{0})\equiv\sum_{\alpha}g_{\mathbf{\mathbf{-q}},+}(\alpha)g_{\mathbf{\mathbf{q}},-}(\mathbf{\alpha})$
and the interaction $\mathcal{L}_{\mathrm{int}}$ characterizes the
coupling between fluctuations $\delta\Delta(\mathbf{k})$ and fermions
$c$, 
\begin{equation}
\mathcal{L}_{\mathrm{int}}=-g\sum_{\mathbf{q}}[\Gamma(\mathbf{q},\mathbf{k})\delta\Delta(\mathbf{k})\bar{c}_{\mathbf{q}+\mathbf{\frac{k}{2}},+}\bar{c}_{\mathbf{-q}+\frac{\mathbf{k}}{2},-}+h.c.]~.
\end{equation}
Accordingly, the free energy $F[\Delta,\bar{\Delta}]$ can be evaluated
in the orders of $\delta\Delta(\mathbf{k})$ 
\begin{align}
F[\Delta,\bar{\Delta}] & =F_{0}+\delta F~,
\end{align}
with 
\begin{align}
F_{0}= & \sum_{\mathbf{k}}g\vert\Delta_{0}\vert^{2}-T\int\mathcal{D}[c,\bar{c}]e^{-\int_{0}^{\beta}d\tau\sum_{\mathbf{k}}\mathcal{L}_{0}}~,\\
\delta F= & \sum_{\mathbf{k}}g\vert\delta\Delta\vert^{2}-T\int\mathcal{D}[c,\bar{c}]e^{-\int_{0}^{\beta}d\tau\sum_{\mathbf{k}}\mathcal{L}_{0}}\left(e^{-\int_{0}^{\beta}d\tau\sum_{\mathbf{k}}\mathcal{L}_{\mathrm{int}}}-1\right) \notag 
\\ \equiv &  \sum_{\mathbf{k}}g\vert\delta\Delta\vert^{2}-T \left\langle\left(e^{-\int_{0}^{\beta}d\tau\sum_{\mathbf{k}}\mathcal{L}_{\mathrm{int}}}-1\right)\right\rangle ~.
\end{align}
The mean-field value $\Delta_{0}$ that meets the condition $\frac{\delta F}{\delta\Delta_{0}}=\frac{\delta F}{\delta\bar{\Delta}_{0}}=0$,
can be determined by the self-consistent equation in Eq.~(6) in the main text.
Then we obtain $F_{0}$ that is just the grand potential 
\begin{align}
F_{0} & =gV\vert\Delta_{0}\vert^{2}-T\ln\int\mathcal{D}[c]e^{-\int_{0}^{\beta}d\tau\sum_{\mathbf{k}}\mathcal{L}_{0}}\nonumber \\
 & =gV\vert\Delta_{0}\vert^{2}-\sum_{\mathbf{k}}\left[2\ln(1+e^{-\beta\epsilon(\mathbf{k})})-\epsilon(\mathbf{k})\right]~,
\end{align}
with $V$ denoting the system volume. To facilitate the evaluation
on $\delta F$, from Eq.~(\ref{app:Lo}) we can deduce the Gor'kov's
normal and abnormal Green functions $\mathcal{G}(q)$ and $\mathcal{F}(q)$
{[}$q=(\mathbf{q},\omega)${]} 
\begin{align}
\mathcal{G}(q) & \equiv\langle c_{\mathbf{q},+}(\omega)\bar{c}_{\mathbf{q},+}(\omega)\rangle=\langle c_{q,-}(\omega)\bar{c}_{q,-}(\omega)\rangle=\frac{i\omega+\mu}{\omega^{2}+\epsilon^{2}(\mathbf{q})},\label{eq:GorkowG-2}\\
\mathcal{F}(q) & \equiv\langle c_{-\mathbf{q},-}(-\omega)c_{\mathbf{q},+}(\omega)\rangle=\frac{g\Gamma^{*}(\mathbf{q})\Delta_{0}}{\omega^{2}+\epsilon^{2}(\mathbf{q})}~. \label{eq:GorkovF-2}
\end{align}
In orders of $\delta\Delta$, we can have perturbative expansion 
\begin{equation}
\delta F=F_{2}+F_{4}+\cdots~,
\end{equation}
where due to the stability of the mean field value, there is no linear
term to $\delta\Delta(\mathbf{k})$. For example, ignore temporal
fluctuations and we evaluate the the second order $F_{2}$, that is,
the Gaussian fluctuations by 
\begin{align}
F_{2}= & \sum_{\mathbf{k}}g\vert\delta\Delta(\mathbf{k})\vert^{2}-T\frac{1}{2}\left\langle \left(\int_{0}^{\beta}d\tau\sum_{\mathbf{k}}\mathcal{L}_{\mathrm{int}}\right)^{2}\right\rangle \nonumber \\
= & \sum_{\mathbf{k}}g\vert\delta\Delta(\mathbf{k})\vert^{2}-T\sum_{\mathbf{k}}g^{2}\vert\Gamma(\mathbf{q},\mathbf{k})\vert^{2}|\delta\Delta(\mathbf{k})|^{2}\left[\langle\bar{c}_{\mathbf{q}+\mathbf{\frac{k}{2}},+}\bar{c}_{\mathbf{-q}+\frac{\mathbf{k}}{2},-}\rangle\langle c_{\mathbf{-q}+\frac{\mathbf{k}}{2},-}c_{\mathbf{q}+\mathbf{\frac{k}{2}},+}\rangle\right.\nonumber \\
 & +\left.\frac{1}{2}\langle c_{\mathbf{q}+\mathbf{\frac{k}{2}},+}c_{\mathbf{-q}-\frac{\mathbf{k}}{2},-}\rangle\langle c_{\mathbf{-q}+\frac{\mathbf{k}}{2},-}c_{\mathbf{q}+\mathbf{\frac{k}{2}},+}\rangle+\frac{1}{2}\langle\bar{c}_{\mathbf{q}+\mathbf{\frac{k}{2}},+}\bar{c}_{\mathbf{-q}-\frac{\mathbf{k}}{2},-}\rangle\langle\bar{c}_{\mathbf{-q}+\frac{\mathbf{k}}{2},-}\bar{c}_{\mathbf{q}+\mathbf{\frac{k}{2}},+}\rangle\right]\nonumber \\
\equiv & \sum_{\mathbf{k}}|\delta\Delta(\mathbf{k})|^{2}[g-g^{2}\chi(\mathbf{k})]~.
\end{align}
where $\chi(\mathbf{k})$ takes the form as Eq.~(9) in
the main text.


%%%%%%%%%%
\section{Mean field for the Dirac fermions with pseudo magnetic field\label{app:meanfield}}
\subsection{TBG Flat bands and Harper model}
The TBG electronic structure can be well captured by the Dirac fermions which is subjected to opposite pseudomagnetic field at the magic angle \cite{2019PhRvL.122j6405T,PhysRevB.99.155415}.
The TBG flatbands are then mapped to the induced zeroth pseudo Landau Level(pLL). which can be simulated by a time-reversal invariant Harper lattice model  \citep{PhysRevB.14.2239,2015NatCo68944P}, which enlightens us for analytical convenience to model the SC phase
by starting with by two-flavor Dirac fermions at two spatial dimensions
\begin{equation}
H_{0}=\sum_{\xi}\int d^{2}\mathbf{r}\Psi_{\xi}^{\dagger}(\mathbf{r})\left[(-i\nabla+\mathbf{A}_{\xi})\cdot\sigma_{\xi}\right]\Psi_{\xi}(\mathbf{r})~,\label{eq:Ham0J}
\end{equation}
by ignoring the bandwidth. Here $\Psi_{\xi}=[a_{\xi},b_{\xi}]^{T}$
is a spinor at two sublattices $a$ and $b$, indices $\xi=\pm$ represents
the two flavor degrees of freedom, and $\sigma_{\xi}$ denotes the
Pauli matrix $\sigma_{\xi}=($$\xi\sigma_{x},\sigma_{y})$. A gauge
field $\mathbf{A}_{\xi}=\xi\mathbf{A}$ is the uniform pseudomagnetic
field and the TBG flatbands are mapped to the induced zeroth pseudo
Landau Levels (pLL). 
Without loss of generality, we set $\Psi_{\xi}\propto[1,0]$ with $a$-sublattice being occupied and we can then we can simulate
zeroth pLL wave functions.
Therefore, we can consider the Harper model with Hamiltonian
$H=-\sum_{\mathbf{r},\mathbf{r}^{\prime}}\sum_{\xi}t_{\mathbf{r},\mathbf{r}^{\prime}}^{\xi}a_{\xi}^{\dagger}(\mathbf{r})a_{\xi}^ {}(\mathbf{r}^{\prime})$,
in which electrons at two flavors $\xi=\pm$
are subject to opposite magnetic fields on a square lattice ($N=N_{c}\times N_{\mathrm{o}}$)
with $N_{c}$ super unit cells and $N_{\mathrm{o}}$ orbitals. The
hopping matrix $t_{\mathbf{r},\mathbf{r}^{\prime}}^{\xi}$ describes
the hopping processes between nearest-neighbor sites, and it reads
\begin{equation}
\!\!\!t_{\mathbf{r}\mathbf{r^{\prime}}}^{\xi}\!=\!\omega^{\xi r_{y}}\delta_{\mathbf{r}-\mathbf{e}_{x},\mathbf{r^{\prime}}}\!+\!\omega^{-\xi r_{y}}\delta_{\mathbf{r}+\mathbf{e}_{x},\mathbf{r^{\prime}}}\!+\!\delta_{\mathbf{r}-\mathbf{e}_{y},\mathbf{r}^{\prime}}\!+\!\delta_{\mathbf{r}+\mathbf{e}_{y},\mathbf{r^{\prime}}},
\end{equation}
with $\mathbf{e}_{x}$($\mathbf{e}_{y}$) the unit vector along $x$($y$)-direction.
Here a factor $\omega^{\pm\xi j_{y}}$ represents a lattice version
of Landau gauge and we consider a uniform commensurate flux 
$\Phi_{\mathrm{p}}=\frac{2\pi}{N_{\mathrm{o}}}$
($\omega=e^{i\Phi_{\mathrm{p}}}$) such that $B_{\mathrm{p}}a^{2}=\frac{2\pi}{N_{\mathrm{o}}}$.
In the reduced Brillouin
zone (BZ) $q_{x}\in[-\pi/(aN_{\mathrm{o}}),\pi/(aN_{\mathrm{o}})]$,
$q_{y}\in[-\pi/a,\pi/a]$, the Bloch functions $g_{\mathbf{q},\xi}(\alpha)$
set in \citep{PhysRevB.90.075104} for the zeroth pLL, approximately in the form, 
\begin{align}
\!\!\!g_{\mathbf{q},\xi}(\alpha) & \sim\sum_{s}e^{-iq_{x}(\alpha-N_{\mathrm{o}}s)a}\phi_{0}(\alpha-sN_{\mathrm{o}}-\xi\frac{N_{\mathrm{o}}q_{y}a}{2\pi}).\label{eq:Blockgk}
\end{align}
Here $\phi_{0}(r)$ is of a Gaussian form and $(\mathbf{r}_{c},\alpha)$ denotes the $\mathbf{r}$ site.


\subsection{BCS mean field on the continuum model}
Here we give details on the BCS mean field theory on the continuum
model $H_{0}$ in Eq.~\eqref{eq:Ham0J} \citep{2008PhRvL.100x6808K,PhysRevB.93.214505,2020JPCM...32J5603P}.
For convenience, we use the symmetry gauge with $\mathbf{A}_{\xi}(\mathbf{r})=\xi\frac{1}{2}B_{\mathrm{p}}(y,-x)$.
At the zeroth pLL, all electrons reside A-sublattice, and an attractive
interaction can set in with 
\begin{equation}
H_{\mathrm{int}}=-g \int d^{2}\mathbf{r}a_{+}^{\dagger}(\mathbf{r})a_{-}^{\dagger}(\mathbf{r})a_{-}^ {}(\mathbf{r})a_{+}^ {}(\mathbf{r}).
\end{equation}
We make the mean-field ansatz by defining the s-wave order parameter
$\Delta(\mathbf{r})$, 
\begin{equation}
\!\!\Delta(\mathbf{r})\!=\!\langle a_{-}(\mathbf{r})a_{+}(\mathbf{r})\rangle=\sum_{n}\phi_{n+}(\mathbf{r})\mathbf{\phi}_{n-}(\mathbf{r})\langle c_{n-}c_{n+}\rangle,\label{eq:meanfield-1}
\end{equation}
where the fermion operator $c_{n\xi}^{\dagger}$ creates an electron
with angular momentum $L_{z}=\xi n$. In general, we allow the mean
field $\Delta(\mathbf{r})$ to have coordinate dependence. The time-reversal
symmetry invariance ensures the coincidence of the localization centers
of two electrons in a Cooper pair. The BdG equation takes the form
as 
\begin{equation}
H_{\mathrm{BdG}}=-g\int d^{2}\mathbf{r}\Delta(\mathbf{r})[a_{+}^{\dagger}(\mathbf{r})a_{-}^{\dagger}(\mathbf{r})+a_{+}(\mathbf{r})a_{-}(\mathbf{r})]~,
\end{equation}
where we choose a gauge such that $\Delta$ is real.
In general, one may consider the effective magnetic fields as $B_{\pm}=B_{\mathrm{p}}\pm B_{\mathrm{r}}$.
The Cooper pairs in Eq.~(\ref{eq:meanfield-1}) will be suppressed
due to the misalignment of the localization centers from the unequal
magnetic lengths $\ell_{\pm}=1/\sqrt{(B_{\mathrm{p}}\pm B_{\mathrm{r}})}$.
In the large $B_{\mathrm{p}}$ limit, we can safely project electrons
onto the zeroth pLLs via a truncated expansion, 
\begin{align}
a_{+}(\mathbf{r}) & =\sum_{n}\phi_{0n+}(\mathbf{r})c_{n+},\\
a_{-}(\mathbf{r}) & =\sum_{n}\bar{\phi}_{0n-}(\mathbf{r})c_{n-},
\end{align}
where $c_{n\xi}^{\dagger}$ creates an electron with $L_{z}=\xi n$
at the valley $\xi$ and $\phi_{0n\xi}(\mathbf{r})=\mathcal{N}_{n\xi}r^{n}e^{in\theta}e^{-r^{2}/4\ell_{\xi}^{2}}$
with a localization center $\sqrt{2n}\ell_{\xi}$. We can reformulate
the BdG Hamiltonian via $c_{n\pm}$, 
\begin{align}
H_{\mathrm{BdG}}= & -\sum_{n}\mu(c_{n+}^{\dagger}c_{n+}^ {}+c_{n-}^{\dagger}c_{n-}^ {}) -\sum_{n}g\Delta_{n}(c_{n+}^{\dagger}c_{n-}^{\dagger}+c_{n-}c_{n+}),\label{eq:BdG}
\end{align}
with 
\begin{equation}
\Delta_{n}=\int d^{2}\mathbf{r}\phi_{0n+}\Delta(\mathbf{r})\bar{\phi}_{0n-},\quad n=0,1,2,\cdots.
\end{equation}
A chemical potential $\mu$ is explicitly shown.
For the zeroth pLL without interaction terms, the chemical potential
stays at zero $\mu=0$ regardless of the filling factor. However,
the attractive pairing interactions obviously dominates over the vanishing
bandwidth and may produce an effective band dispersion. We can expect
that an interaction term can exert the significant renormalization
of $\mu$. The separation of the magnetic lengths $\ell_{\pm}$ leads
to diminishing pairing abilities as angular moment index $n$ increases.
In other words, it is special for $B_{\mathrm{r}}=0$ case, where
we can expect a uniform order parameter for any angular momentum.
We can diagonalize the BdG Hamiltonian in Eq.~(\ref{eq:BdG}) by
the Bogoliubov transformation 
\begin{align}
c_{m+}&=u_{m}\gamma_{m+}-v_{m}\gamma_{m-}^{\dagger} ~, \notag \\
c_{m-}^{\dagger}&=v_{m}\gamma_{m+}+u_{m}\gamma_{m-}^{\dagger}~,
\end{align}
with 
\begin{equation}
u_{n}=\frac{1}{\sqrt{2}}\sqrt{1-\frac{\mu}{\epsilon_{n}}},v_{n}=\frac{1}{\sqrt{2}}\sqrt{1+\frac{\mu}{\epsilon_{n}}}.
\end{equation}
The dispersion for the quasiparticles is 
\begin{equation}\epsilon_{n}=\sqrt{\mu^{2}+(g\Delta_{n})^{2}}~.\end{equation}
Meanwhile, we can determine the order parameter by the self-consistent
gap equation, 
\begin{equation}
\Delta(\mathbf{r})=\sum_{n=0}\phi_{0n+}^{*}(\mathbf{r})\phi_{0n-}(\mathbf{r})u_{n}v_{n}\tanh\frac{\beta\epsilon_{n}}{2}.\label{eq:self-eq1}
\end{equation}
Or, equivalently for $\Delta_{n}$, we have the gap equations as
\begin{equation}
\Delta_{m}=\sum_{n}\mathsf{K}_{nm}u_{n}v_{n}\tanh\frac{\beta\epsilon_{n}}{2},\label{eq:self-eq2-1}
\end{equation}
with \begin{equation} 
\mathsf{K}_{nm}=\int d^{2}\mathbf{r}\phi_{0n+}\phi_{0n-}^{*}\phi_{0m+}\phi_{0m-}^{*}.
\end{equation}
The chemical potential gets shifted according to the number equation,
\begin{equation}
2\nu=\lim_{N\rightarrow\infty}\frac{1}{N}\left(\sum_{m=0}^{N_{+}}\langle c_{m+}^{\dagger}c_{m+}^ {}\rangle+\sum_{m=0}^{N_{-}}\langle c_{m-}^{\dagger}c_{m-}^ {}\rangle\right).\label{eq:numerEq-1}
\end{equation}
Here $2\nu N=2\nu B_{\mathrm{p}}S/2\pi$ is the total numbers of electrons
occupying the zeroth pLLs with $S$ the sample area and $N_{\pm}=(B_{\mathrm{p}}\pm B_{\mathrm{r}})S/2\pi$
are the degeneracy of the zeroth pLL. In the thermodynamic limit $S\rightarrow\infty$,
the number equation in Eq.~(\ref{eq:numerEq-1}) reduces to a relation
$\nu=\tanh\frac{\beta\mu}{2}$. It tells that the filling factors
for the two valleys stay invariant $\nu_{+}=\nu_{-}=\nu$ due to the
redistribution of electrons.

The self-consistent equation in \eqref{eq:self-eq1} is simplified
when $T=0$ and $B_{\mathrm{r}}=0$. Due to the translational symmetry,
we have $\Delta(\mathbf{r})\equiv\Delta_{0}$, which yields analytically
a mean-field solution $\Delta_{0}=\frac{1}{4\pi\ell^{2}}$ at $T=0,B_{\mathrm{r}}=0$.
In general case, we need to numerically solve the self-consistent
gap equation to find the mean field $\Delta(\mathbf{r})$ or $\Delta_{n}$.
We are interested in the critical temperature $T_{\mathrm{BCS}}$
and its relation to $B_{\mathrm{r}}$. Around $T_{\mathrm{BCS}}$,
we can linearize the gap equations in Eq.~(\ref{eq:self-eq2-1}),
\begin{equation}
\Delta_{m}=\frac{g\beta}{4}\frac{1-2\nu}{\mathrm{arctanh(1-}2\nu)}\sum_{n}\mathsf{K}_{nm}\Delta_{n}.\label{eq:lineargapeq-1}
\end{equation}
It reduces to an eigenvalue problem of the matrix $\mathsf{K}_{nm}$.
The critical temperature $T_{\mathrm{BCS}}$ obeys the criteria that
the maximal eigenvalue of $\mathsf{K}_{nm}$ meet Eq.~(\ref{eq:lineargapeq-1}).
From the form of $\mathsf{K}_{nm}$, we find one eigenvector 
\begin{equation}
\Delta_{n}=\left(\frac{1-r}{1+r}\right)^{n/2}~,
\end{equation}
with the eigenvalue $\frac{\ell^{2}(1-r)}{2}$ and $r=B_{\mathrm{r}}/B_{\mathrm{p}}$.
It indeed is the dominant one from the Perron-Frobenius theorem
\cite{horn2012matrix}. Therefore, we exactly identify the critical
temperature $T_{\mathrm{BCS}}$ that is linear to the real magnetic
field $B_{\mathrm{r}}$ 
\begin{equation}
T_{\mathrm{BCS}}=\tau_{c}\frac{1-2\nu}{\mathrm{arctanh(}1-2\nu)}(1-r),\label{eq:Tc1mu-1}
\end{equation}
where $\text{\ensuremath{\tau_{c}}}=\frac{g}{8\pi\ell_{0}^{2}}$ is
the critical temperature for $B_{\mathrm{r}}=0$ and $\nu=1/2$ with
$\ell_{0}=1/\sqrt{B_{\mathrm{p}}}$. The factor $\frac{1-2\nu}{\mathrm{arctanh}(1-2\nu)}$
in Eq.~(\ref{eq:Tc1mu-1}) controls the influence of the filling.
The factor $1-r$ in Eq.~(\ref{eq:Tc1mu-1}) indicates the pseudomagnetic
field $B_{\mathrm{p}}$ sets an upper bound for critical magnetic
field $B_{\mathrm{r}}$, above which Cooper pairs are shattered. It
can be reasonably imaged since the zeroth pLL density of states at
one valley is suppressed by $B_{\mathrm{r}}$. In particular, SC has
the maximal $T_{\mathrm{BCS}}$ at half-filling $\mu=0$, $\nu=1/2$.
The quasiparticle spectrum stays flat when $B_{\mathrm{r}}=0$ while
it is dispersive for finite $B_{\mathrm{r}}$ and approaches zero
at large $L_{z}$. Moreover, one can also extract a characteristic
quantity $g\Delta/T_{\mathrm{BCS}}=2$, which is larger than a conventional
BCS SC phase with a large fermi velocity.

\section{Harper model and the effective GL theory \label{sec:Harper-model-and}}

The zeroth pLL can be regularized by a TRI Harper lattice model on
a system $N=N_{c}\times N_{\mathrm{o}}$ with $N_{c}$ super unit
cells and $N_{\mathrm{o}}$ orbitals. That is, we relabel original
lattice site $\mathbf{r}=(\mathbf{r}_{c},\alpha)$. We accordingly
introduce multi-band fermion operators $a_{\xi}(\mathbf{r}_{c},\alpha)$
that annihilates a fermion at super unit cell $\mathbf{r}_{c}$ and
$\alpha$ orbital. We can reformulate the attractive interaction in
Eq.~(1)
\begin{equation}
H_{\mathrm{int}}=-g\sum_{\mathbf{r}_{c},\alpha}a_{+}^{\dagger}(\mathbf{r}_{c},\alpha)a_{-}^{\dagger}(\mathbf{r}_{c},\alpha)a_{-}^ {}(\mathbf{r}_{c},\alpha)a_{+}^ {}(\mathbf{r}_{c},\alpha)~.
\end{equation}
Accordingly, the auxiliary field $\Delta_{\alpha}(\mathbf{r}_{c})$
acquires dependence on the orbital indices 
\begin{equation}
\Delta_{\alpha}(\mathbf{r}_{c})=a_{-}(\mathbf{r},\alpha)a_{+}(\mathbf{r},\alpha)~,
\end{equation}
Then by means of the Hubbard-Stratonovich transformation, we have
the interaction Lagrangian in the framework of the path integral with
\begin{align}
\!\!\!L_{\mathrm{int}}[a,\bar a,\Delta,\bar \Delta]=\! & -\sum_{\mathbf r_c,\alpha}g\left[\Delta_{\alpha}(\mathbf{r}_{c})\bar{a}_{+}(\mathbf{r}_{c},\alpha)\bar{a}_{-}(\mathbf{r}_{c},\alpha)+h.c.\right]\nonumber \\
=\! & -\sum_{\mathbf{k}\mathbf{q}\alpha}g\!\!\left[\Gamma_{\alpha}(\mathbf{q},\mathbf{k})\Delta_{\alpha}(\mathbf{k})\bar{c}_{\mathbf{k}+\frac{\mathbf{q}}{2},+}\bar{c}_{-\mathbf{k}+\frac{\mathbf{q}}{2},-}\!+\!h.c.\right]~.
\end{align}
Here the momentum $\mathbf{q}$ is quantized in the reduced BZ. In
the second line, we conduct a projection onto the zeroth pLL via 
\begin{equation}
a_{\xi}(\mathbf{r}_{c},\alpha)\rightarrow\sum_{\mathbf{k}}g_{\mathbf{k},\xi}^{*}(\alpha)c_{\mathbf{k}\xi}~,
\end{equation}
with $\Gamma_{\alpha}(\mathbf{q},\mathbf{k})=g_{\mathbf{k+\frac{\mathbf{q}}{2}},+}(\alpha)g_{\mathbf{-k}+\frac{\mathbf{q}}{2},-}(\alpha)$.
Provided the unit cell enlargement, $g_{\mathbf{k}}(\alpha)$ should
meet the normalization,$\frac{1}{N_{c}}\sum_{\mathbf{k}}\vert g_{\mathbf{k}}(\alpha)\vert^{2}=\frac{1}{N_{\mathrm{orb}}}$
on the reduced BZ. The orbital dependence manifests in the Bloch wave
$g_{\mathbf{k},\xi}(\alpha)$. Along with $L_{0}=\sum_{\mathbf{k}\xi}(i\omega-\mu)\bar{c}_{\mathbf{k}\xi}c_{\mathbf{k}\xi}$,
we have the total Lagrangian $L=L_{0}+L_{\mathrm{int}}$.

We expand the bosonic field around the mean field configuration 
\begin{equation}
\Delta_{\alpha}(\mathbf{k})=\Delta_{\alpha0}\delta_{\mathbf{k}0}+\delta\Delta_{\alpha}(\mathbf{k})~,
\end{equation}
which decomposes the Lagrangian $L$ into two parts. The first reduces
to the BCS mean field with 
\begin{align}
\mathcal L_0[c,\bar c]= & (i\omega-\mu)(\bar{c}_{\mathbf{q},+}c_{\mathbf{q},+}+\bar{c}_{\mathbf{q},-}c_{\mathbf{q},-}) -\sum_{\alpha}[\Gamma_{\alpha}(\mathbf{q})\Delta_{\alpha}\bar{c}_{\mathbf{q},+}\bar{c}_{-\mathbf{q},-}+h.c.]~,
\end{align}
where $\Gamma_{\alpha}(\mathbf{q})\equiv\Gamma_{\alpha}(\mathbf{q},0)$.
We can then derive the Bogoliubov quasiparticle dispersion $\epsilon(\mathbf{q})=\sqrt{\left|g\sum_{\alpha}\Gamma_{\alpha}(\mathbf{q})\Delta_{\alpha0}\right|^{2}+\mu^{2}}$
as well as the self-consistent equation for $\Delta_{\alpha0}$ 
\begin{equation}
\Delta_{\alpha0}=\frac{1}{N_{c}}\sum_{\mathbf{q}}\sum_{\gamma}\Delta_{\gamma0}\frac{\Gamma_{\alpha}(\mathbf{q})\Gamma_{\gamma}^{*}(\mathbf{q})}{2\epsilon_{\mathbf{q}}}\tanh\frac{\beta\epsilon_{\mathbf{q}}}{2}~.
\end{equation}
As mentioned in the main body, we can ignore the fluctuations of the
orbital fluctuations, by approximating 
\begin{equation}
\Delta_{\alpha}(\mathbf{k})\equiv\Delta(\mathbf{k}),\Delta_{\alpha0}\equiv\Delta_{0}\quad\forall\alpha~.
\end{equation}
which will simplify our calculations while capturing the main physics. After integrating out the fermion
fields, we then arrive at the effective action for the fluctuations.


%%%%%%%%%%%%

\section{Model for twisted bilayer graphene\label{sec:Model-for-twisted}}

For the twisted bilayer graphene system \cite{2011PNAS..10812233B}, we consider the AA-stacking
while rotating the two layers $\ell=1$ and $2$ around a pair of
registered B site respectively by an angle $-\theta/2$ and $+\theta/2$.
Before rotation we denote the lattice vectors as $\mathbf{a}_{1}=a(1,0)$
and $\mathbf{a}_{2}=a(1/2,\sqrt{3}/2)$ for the AA stacking bilayer
with $a=0.246$nm being the lattice constant of graphene. Correspondingly,
we have the reciprocal lattice vector $\mathbf{g}_{1}=\frac{2\pi}{a}(1,-1/\sqrt{3})$
and $\mathbf{g}_{2}=\frac{2\pi}{a}(0,2/\sqrt{3})$ before rotation.After
rotation, the lattice vectors $\mathbf{a}^{(\ell)}$ of the two layers
$\ell=1,2$ can be constructed to be $\mathbf{a}_{i}^{(1)}=R(-\theta/2)\mathbf{a}_{i}$
and $\mathbf{a}_{i}^{(2)}=R(+\theta/2)\mathbf{a}_{i}$, where $R(\theta)$
is the rotation matrix of a rotation angle $\theta$. Similarly, we
can have the reciprocal lattice vectors $\mathbf g_{i}^{(1)}=R(-\theta/2)\mathbf{g}_{i}$
and $\mathbf{g}_{i}^{(2)}=R(+\theta/2)\mathbf{g}_{i}$ for the two
layers $\ell=1,2$ after rotation. When the rotation angle $\theta$
is small, the TBG will possess a very long period and the reciprocal
lattice vectors for the moir\'e pattern appear as
\begin{equation}
\mathbf{G}_{i}^{\mathrm m}=\mathbf{g}_{i}^{(1)}-\mathbf{g}_{i}^{(2)}~,
\end{equation}
The real-space lattice vectors $\mathbf{L}_{j}^{\mathrm{m}}$ satisfy
the condition $\mathbf{G}_{i}^{\mathrm{m}}\cdot\mathbf{L}_{j}^{\mathrm{m}}=2\pi\delta_{ij}$
and the moir\'e lattice constant $L_{\mathrm m}=\vert\mathbf{L}_{1}^{\mathrm{m}}\vert=\vert\mathbf{L}_{2}^{\mathrm{m}}\vert$
is $L_{\mathrm m}=\frac{a}{2\sin\left(\theta/2\right)}$.

At the small angle, such as $\theta=1.08^{\circ}$ considered in the
main text, we can apply an effective continuum model. We can safely
neglect the intervalley mixing, which allows block-diagnoalization
of the Hamiltonian. Given a valley $\rho$ and spin $\sigma$, the effective Hamiltonian
$H^{(\rho)}$ takes the form as 
\begin{equation}
H^{(\sigma\rho)}=H_{\mathrm{kin}}^{1}+H_{\mathrm{kin}}^{2}+H_{\mathrm{kin}}^{\perp}-\mu N+H_{\mathrm{int}}~. \label{app:Hrho}
\end{equation}
The first two terms $H_{\mathrm{kin}}^{\ell}$($\ell=1,2$) describes
the intralayer hopping and can be approximated by the two-dimensional
Weyl equation
\begin{align}
H_{\mathrm{kin}}^{\ell} & =-\int d^{2}\mathbf{r} v_{F}\psi_{\sigma\rho,\ell}^{\dagger}(\mathbf{r})\left[R(\mp\theta/2)(\mathbf{-i}\nabla-\mathbf{K}_{\rho}^{(\ell)})\right]\cdot\boldsymbol{\sigma}^{\rho}\psi_{\sigma\rho,\ell}(\mathbf{r})\nonumber \\
 & =-\int_{\mathrm{mBZ}}\frac{d^{2}\mathbf{k}}{(2\pi)^{2}} v_{F}\psi_{\sigma\rho,\ell}^{\dagger}(\mathbf{k})\left[R(\mp\theta/2)(\mathbf{k}-\mathbf{K}_{\rho}^{(\ell)})\right]\cdot\boldsymbol{\sigma}^{\rho}\psi_{\sigma\rho,\ell}(\mathbf{k})~.
\end{align}
where $\mathbf{K}_{\rho}^{(\ell)}=-\rho[2\mathbf{g}_{1}^{(\ell)}+\mathbf{g}_{2}^{(\ell)}]/3$
for the layer $\ell$ and valley $\rho$, $-(+)$ for $\rho=1(2)$
in $R(\mp\theta/2)$, mBZ denotes the BZ for the moir\'e lattice and the fermi velocity is $v_{F}=7.98\times10^{5}$m/s.
The $\psi_{\sigma\rho,\ell}(\mathbf{r})$ is a vector in the sublattice
space $\psi_{\sigma\rho,\ell}(\mathbf{r})=\begin{bmatrix}\psi_{\sigma\rho,\ell A}(\mathbf{r}) & \psi_{\sigma\rho,\ell B}(\mathbf{r})\end{bmatrix}^{T}$
and the fermion operator $\psi_{\sigma\rho,\ell\xi}(\mathbf{r})$
annihilates an electron with spin index $\sigma$, layer $\ell$,
and sublattice $\xi$ at the valley $\rho$. The $\boldsymbol{\sigma}^{\rho}=(\rho\sigma_{x},\sigma_{y})$
is a vector of Pauli matrix in the sublattice system $(\rho=\pm)$.
The third part $H_{\mathrm{Kin}}^{\perp}$ describes the effective
interlayer hopping in the form as
\begin{align}
H_{\mathrm{kin}}^{\perp} & =\int d^{2}\mathbf{r}\begin{bmatrix}\psi_{\sigma\rho,1A}^{\dagger}(\mathbf{r}) & \psi_{\sigma\rho,1B}^{\dagger}(\mathbf{r})\end{bmatrix}T_{12}(\mathbf{r})\begin{bmatrix}\psi_{\sigma\rho,2A}(\mathbf{r})\\
\psi_{\sigma\rho,2B}(\mathbf{r})
\end{bmatrix}+h.c.\\
 & =\int d^{2}\mathbf{r}\sum_{\ell,\xi\xi^{\prime}}\psi_{\sigma\rho,\ell\xi}^{\dagger}(\mathbf{r})T_{\ell\bar{\ell},\xi\xi^{\prime}}(\mathbf{r})\psi_{\sigma\rho,\bar{\ell}\xi^{\prime}}(\mathbf{r})~,
\end{align}
with $\bar{\ell}$ denoting the opposite layer to $\ell$. The elements
of the $2\times2$ matrix $[T_{12}(\mathbf{r})]_{\xi\xi^{\prime}}=T_{12,\xi\xi^{\prime}}(\mathbf{r})$ in general assume the form as
\begin{align}
T_{12}(\mathbf{r})= & \begin{bmatrix}t_{AA} & t_{AB}\\
t_{BA} & t_{BB}
\end{bmatrix}+\begin{bmatrix}t_{AA} & t_{AB}\omega^{-\xi}\\
t_{BA}\omega^{\xi} & t_{BB}
\end{bmatrix}e^{i\xi\mathbf{G}_{1}^{\mathrm m}\cdot\mathbf{r}} +\begin{bmatrix}t_{AA} & t_{AB}\omega^{\xi}\\
t_{BA}\omega^{-\xi} & t_{BB}
\end{bmatrix}e^{i\xi(\mathbf{G}_{1}^{\mathrm m}+\mathbf{G}_{2}^{\mathrm m})\cdot\mathbf{r}}~.
\end{align}
where $\omega=e^{2\pi i/3}$. In general due to the corrugation effect \cite{PhysRevX.8.031087},
the model parameters no longer satisfy $t_{AA}=t_{AB}=t_{BA}=t_{BB}$
that is referred to as a flat TBG whose interlayer spacing is uniform.
In fact, the optimized lattice structure of TBG will be corrugated
in the out-of-plane, which enables a choice on parameters $t_{AA}=t_{BB}=79.7$meV
and $t_{AB}=t_{BA}=97.5$meV in the effective model. Owing to a mismatch
between $t_{AA}(t_{BB})$ and $t_{AB}(t_{BA})$, an energy gap arises
between the lowest bands and excited bands, which is consistent with
experiments \cite{PhysRevX.8.031087}.

The third term in Eq. (\ref{app:Hrho}) is the chemical potential
with $N=\int d^{2}\mathbf{r}\sum_{\sigma\rho\ell\xi}\psi_{\sigma\rho,\ell\xi}^{\dagger}(\mathbf{r})\psi_{\sigma\rho,\ell\xi}(\mathbf{r})$
that tunes the filling condition. The last term is an attractive interaction.
Although it is still elusive what type of interaction contributes
to superconductivity, we assume a simple form  \cite{PhysRevB.101.060505,PhysRevLett.123.237002} with
\begin{equation}
H_{\mathrm{int}}=-g\int d^{2}\mathbf{r}\sum_{\rho\ell\xi}\psi_{\uparrow\rho,\ell\xi}^{\dagger}(\mathbf{r})\psi_{\downarrow\bar{\rho},\ell\xi}^{\dagger}(\mathbf{r})\psi_{\downarrow\bar{\rho},\ell\xi}(\mathbf{r})\psi_{\uparrow\rho,\ell\xi}(\mathbf{r}).
\end{equation}
The mean-field theory $\Delta_{\rho,\ell\xi}(\mathbf{r})=\langle\psi_{\downarrow\bar{\rho},\ell\xi}(\mathbf{r})\psi_{\uparrow\rho,\ell\xi}(\mathbf{r})\rangle$
has been thoughtfully investigated in Ref.~\cite{PhysRevB.101.060505,PhysRevLett.123.237002}. Instead,
we concentrate on the critical region. From the band structure shown
in Fig.~1(a), there are 2 nearly flat bands around the
interested energy given the valley $\rho$ and spin $\sigma$. Thus,
we should introduce the projection onto the two nearly flat bands
that we label as $n=1,2$,
\begin{equation}
\psi_{\sigma\rho,\ell\xi}(\mathbf{r})\rightarrow\int_{\mathrm{mBZ}}\frac{d^{2}\mathbf{k}}{(2\pi)^{2}}e^{-i\mathbf{k}\cdot\mathbf{r}}\sum_{n=1,2}g_{\sigma\rho,n\mathbf{k}}(\alpha)c_{\sigma\rho,n}(\mathbf{k})~,
\end{equation}
with the an eigenenergy $h_{\rho,n}(\mathbf{k})$ for $c_{\sigma\rho,n}(\mathbf{k})$
and $\alpha$ labelling the internal degree of freedom apart from
the spin and valley. Following the framework of GL theory in the main
text, we introduce the bosonic field $\Delta_{\rho,\ell\xi}(\mathbf{r})=\psi_{\downarrow\bar{\rho},\ell\xi}(\mathbf{r})\psi_{\uparrow\rho,\ell\xi}(\mathbf{r})$
and further ignore the fluctuations in the layer $\ell$ and sublattice
$\xi$ by further assuming $\Delta_{\rho,\ell\xi}(\mathbf{r})=\Delta_{\rho}(\mathbf{r})$.
In the critical region with vanishing mean field values, we still
use $\Delta_{\rho,\ell\xi}(\mathbf{r})$ to denote the fluctuations.
By means of the Hubbard-Stratonovich transformation, we have the action
for $\psi_{\sigma\rho,\ell\xi}(\mathbf{r})$
\begin{align}
Z & =\int\mathcal{D}[\Delta,\bar{\Delta}]e^{-g\int_{0}^{\beta}d\tau\int d^{2}\mathbf{r}\sum_{\rho\ell\xi}\vert\Delta_{\rho,\ell\xi}\vert^{2}}\mathcal{Z}[\Delta,\bar{\Delta}]~,
\end{align}
with 
\begin{align}
\mathcal{Z}[\Delta,\bar{\Delta}] & =\int\mathcal{D}[\psi,\bar{\psi}]e^{-\int_{0}^{\beta}d\tau\int d^{2}\mathbf{r}\mathcal{L}[\psi,\bar{\psi},\Delta,\bar{\Delta}]} ~,\\
\mathcal{L}[\psi,\bar{\psi},\Delta,\bar{\Delta}] & =\sum_{\sigma\rho\ell\xi}\psi_{\sigma\rho,\ell\xi}^{\dagger}(\mathbf{r})[h(-i\nabla)-\mu]\psi_{\sigma\rho,\ell\xi}(\mathbf{r})-g\sum_{\rho\ell\xi}\left[\bar{\Delta}_{\rho}(\mathbf{r})\psi_{\downarrow\bar{\rho},\ell\xi}(\mathbf{r})\psi_{\uparrow\rho,\ell\xi}(\mathbf{r})+h.c.\right]  ,
\end{align}
where $h(-i\nabla)$ is the kinetic operator. After projection onto
the two nearly flat bands, as emphasized before, we should incorporate
the nontrivial Bloch waves with 
\begin{equation}
\mathcal{L}[\psi,\bar{\psi},\Delta,\bar{\Delta}]\rightarrow\mathcal{L}[c,\bar{c},\Delta,\bar{\Delta}]~.
\end{equation}
The projected Lagrangian $\mathcal{L}[c,\bar{c},\Delta,\bar{\Delta}]$
is 
\begin{equation}
\mathcal{L}[c,\bar{c},\Delta,\bar{\Delta}] =\mathcal{L}_{0}[c,\bar{c}]+\mathcal{L}_{\mathrm{int}}[c,\bar{c},\Delta,\bar{\Delta}]\nonumber ~,
\end{equation}
with the free part for the projected fermions $c$,
\begin{equation}
\mathcal{L}_{0}[c,\bar{c}] =\sum_{\sigma\rho}c_{\sigma\rho,n}^{\dagger}(\mathbf{k})\mathcal{E}_{\rho,n}(\mathbf{k})c_{\sigma\rho,n}(\mathbf{k})~,
\end{equation}
and the interaction between bosonic field $\Delta$ and projected ferimions $c$,
\begin{equation}
\mathcal{L}_{\mathrm{int}}[c,\bar{c},\Delta,\bar{\Delta}]  =-g\sum_{\rho}\sum_{n,m}\left[\Gamma_{\rho,nm}(\mathbf{q},\mathbf{k})\bar{\Delta}_{\rho}(\mathbf{k})c_{\downarrow\bar{\rho},n}(-\mathbf{q}+\frac{\mathbf{k}}{2})c_{\uparrow\rho,m}(\mathbf{\mathbf{q}+\frac{\mathbf{k}}{2}})+h.c.\right]~,
\end{equation}
where $\mathcal{E}_{\rho,n}(\mathbf{k})=h_{\rho,n}(\mathbf{k})-\mu$
, $h_{\rho,n}(\mathbf{k})$ is the dispersion of the targeted band 
$n$ and the prefactor $\Gamma_{\rho,nm}$ is
\begin{equation}
\Gamma_{\rho,nm}(\mathbf{q},\mathbf{k})=\sum_{\alpha}g_{\mathbf{\downarrow\rho},n\mathbf{-q}+\frac{\mathbf{k}}{2}}(\alpha)g_{\mathbf{\uparrow\rho},n\mathbf{\mathbf{q}+\frac{\mathbf{k}}{2}}}(\alpha) ~.
\end{equation}
Around the critical region, we have the Gor'kov's Green function
\begin{align}
\mathcal{G}_{\rho,n}(q) &=\langle c_{\sigma\rho,n}(\mathbf{q})c_{\sigma\rho,n}^{\dagger}(\mathbf{q})\rangle =\frac{1}{i\omega-\mathcal{E}_{\rho,n}(\mathbf{k})}~,\\
\mathcal{F}_{\rho,n}(q)& =\langle c_{\sigma\rho,n}(\mathbf{q})c_{\sigma\rho,n}(\mathbf{-q})\rangle =0~.
\end{align}
Integrate out the fermion field and we then have 
\begin{align}
F[\Delta,\bar{\Delta}] & =\sum_{\rho}g\vert\Delta_{\rho}\vert^{2}-\frac{1}{2}\left\langle \left(\int_{0}^{\beta}\tau\int\frac{d^{2}\mathbf{k}}{(2\pi)^{2}}\mathcal{L}_{\mathrm{int}}[c,\bar{c},\Delta,\bar{\Delta}]\right)^{2}\right\rangle \nonumber \\
 & =\sum_{\rho}(g-g^{2}\chi_{\rho}(\mathbf{k}))\vert\Delta_{\rho}(\mathbf{k})\vert^{2} ~,
\end{align}
with 
\begin{align}
\chi_{\rho}(\mathbf{k})&=\frac{T}{2\pi}\sum_{\omega}\int\frac{d^{2}\mathbf{q}}{(2\pi)^{2}}\vert\Gamma_{\rho,nm}(\mathbf{q},\mathbf{k})\vert^{2}\mathcal{G}_{\rho,n}(q)\mathcal{G}_{\rho,m}(-q)\\
&=\frac{T}{2\pi}\sum_{n,m}\sum_{\omega}\int\frac{d^{2}\mathbf{q}}{(2\pi)^{2}}\vert\Gamma_{\rho,nm}(\mathbf{q},\mathbf{k})\vert^{2}\cdot\frac{1}{i\omega-\mathcal{E}_{\rho,n}(\mathbf{q}+\frac{\mathbf{k}}{2})}\frac{1}{-i\omega-\mathcal{E}_{\rho,m}(\mathbf{-\mathbf{q}+\frac{\mathbf{k}}{2}})}\\
&=\frac{1}{2\pi}\sum_{n,m}\int\frac{d^{2}\mathbf{q}}{(2\pi)^{2}}\vert\Gamma_{\rho,nm}(\mathbf{q},\mathbf{k})\vert^{2}\cdot\frac{\tanh\left(\frac{\beta}{2}\mathcal{E}_{\rho,n}(\mathbf{q}+\frac{\mathbf{k}}{2})\right)+\tanh\left(\frac{\beta}{2}\mathcal{E}_{\rho,m}(\mathbf{q}-\frac{\mathbf{k}}{2})\right)}{2(\mathcal{E}_{\rho,n}(\mathbf{q}+\frac{\mathbf{k}}{2})+\mathcal{E}_{\rho,m}(\mathbf{q}-\frac{\mathbf{k}}{2}))}~,
\end{align}
where we use $\mathcal{E}_{\rho,n}(\mathbf{\mathbf{q}+\frac{\mathbf{k}}{2}})=\mathcal{E}_{\rho,n}(-\mathbf{\mathbf{q}+\frac{\mathbf{k}}{2}})$
due to the time reversal symmetry. We can numerically simulate the
quantity $\chi_{\rho}(\mathbf{k})$ and expand it around small momentum
$\mathbf{k}$, which then gives rise to the coherent length $\xi$.

\end{appendix}


\end{document}
