%%%%%%%%%%%%%%%%%%%%%%%%%%%%%%%%%%%%%%%%%%%%%%%%%%%%%%%%%%%%%%%%%%%%%%%%%%%%%%%%
%2345678901234567890123456789012345678901234567890123456789012345678901234567890
%        1         2         3         4         5         6         7         8

\documentclass[letterpaper, 10 pt, conference]{ieeeconf}  % Comment this line out if you need a4paper

%\documentclass[a4paper, 10pt, conference]{ieeeconf}      % Use this line for a4 paper

\IEEEoverridecommandlockouts                              % This command is only needed if 
                                                          % you want to use the \thanks command

\overrideIEEEmargins                                      % Needed to meet printer requirements.

%In case you encounter the following error:
%Error 1010 The PDF file may be corrupt (unable to open PDF file) OR
%Error 1000 An error occurred while parsing a contents stream. Unable to analyze the PDF file.
%This is a known problem with pdfLaTeX conversion filter. The file cannot be opened with acrobat reader
%Please use one of the alternatives below to circumvent this error by uncommenting one or the other
%\pdfobjcompresslevel=0
%\pdfminorversion=4

% See the \addtolength command later in the file to balance the column lengths
% on the last page of the document

% The following packages can be found on http:\\www.ctan.org
%\usepackage{graphics} % for pdf, bitmapped graphics files
%\usepackage{epsfig} % for postscript graphics files
%\usepackage{mathptmx} % assumes new font selection scheme installed
%\usepackage{times} % assumes new font selection scheme installed
%\usepackage{amsmath} % assumes amsmath package installed
%\usepackage{amssymb}  % assumes amsmath package installed

% ########## OUR PACKAGE ADDITIONS ##############
\usepackage[utf8]{inputenc}
\usepackage{soul,color}
\usepackage{amsmath,amssymb,amsfonts}
\usepackage{mathtools}
\usepackage{graphicx}
\usepackage{array}\newcolumntype{M}[1]{>{\centering\arraybackslash}m{#1}}
\usepackage{float} 
\usepackage{cite} % for grouping citations
\usepackage{subfigure}
\usepackage{tikz} % for tikz figures
\usepackage{enumerate}
\usepackage[algo2e,norelsize,linesnumbered,ruled]{algorithm2e}
\usepackage{balance} % equal column lengths on last page
\usepackage{comment}
\usepackage{accents}
\usepackage{makecell}
\usepackage{multirow}
\usepackage{graphicx}
\newcommand{\ubar}[1]{\underaccent{\bar}{#1}}
\DeclareMathAlphabet{\mathcal}{OMS}{cmsy}{m}{n} % Used to change the \mathcal font to default
\usepackage{flushend} % to balance columns on last page
\usepackage{textcomp, gensymb}
% ##############################################
\bibliographystyle{unsrt}
\usepackage{amsmath}

\title{\LARGE \bf
Switched Moving Boundary Modeling of Phase Change Thermal Energy Storage Systems*
}
\title{Switched Moving Boundary Modeling of Phase Change Thermal Energy Storage Systems}
\author{Trent J. Sakakini and Justin P. Koeln}
\date{January 2023}

\author{Trent J. Sakakini and Justin P. Koeln% <-this % stops a space
\thanks{*This work was not supported by any organization}% <-this % stops a space
\thanks{The authors are with the Department of Mechanical Engineering, The University of Texas at Dallas, 800 W. Campbell Rd, Richardson, TX, USA. Email: {\tt\small(trent.sakakini, justin.koeln)@utdallas.edu}.}}%


\begin{document}



\maketitle
\thispagestyle{empty}
\pagestyle{empty}


%%%%%%%%%%%%%%%%%%%%%%%%%%%%%%%%%%%%%%%%%%%%%%%%%%%%%%%%%%%%%%%%%%%%%%%%%%%%%%%%
\begin{abstract}

Thermal Energy Storage (TES) devices, which leverage the constant-temperature thermal capacity of the latent heat of a Phase Change Material (PCM), provide benefits to a variety of thermal management systems by decoupling the absorption and rejection of thermal energy. While performing a role similar to a battery in an electrical system, it is critical to know when to charge (freeze) and discharge (melt) the TES to maximize the capabilities and efficiency of the overall system. Therefore, control-oriented models of TES are needed to predict the behavior of the TES and make informed control decisions. While existing modeling approaches divide the TES in to multiple sections using a Fixed Grid (FG) approach, this paper proposes a switched Moving Boundary (MB) model that captures the key dynamics of the TES with significantly fewer dynamic states. Specifically, a graph-based modeling approach is used to model the heat flow through the TES and a MB approach is used to model the time-varying liquid and solid regions of the TES. Additionally, a Finite State Machine (FSM) is used to switch between four different modes of operation based on the State-of-Charge (SOC) of the TES. Numerical simulations comparing the proposed approach with a more traditional FG approach show that the MB model is capable of accurately modeling the behavior of the FG model while using far fewer states, leading to five times faster simulations. 

\end{abstract}

\section{Introduction}
The need for higher performance and more efficient thermal management systems has driven the design of systems with integrated Thermal Energy Storage (TES) devices that leverage the latent heat of a Phase Change Material (PCM). The design and performance of PCM-based TES has been well-studied \cite{Nazir2019,Sharma2009,Zalba2003}, resulting in a wide range of applications including building \cite{Lee2020,Ma2012} and aircraft \cite{Laird2021,Laird2019} thermal management, power electronics cooling \cite{Pangborn2020}, and combined heating and cooling \cite{Bird2020}.%,Jain2014}.

The utility of a TES is heavily dependent on the dynamics associated with charging (where the PCM solidifies from liquid to solid), discharging (where the PCM melts from solid to liquid), and strategic switching between these two modes of operation. Therefore, accurate control-oriented models of PCM-based TES are needed that capture their hybrid, nonlinear dynamics to be used in predictive controllers like Model Predictive Control (MPC), which have been developed for single-phase \cite{Ma2012,Lee2020} and phase change TES \cite{Pangborn2020}. 
% Thermal management solutions for devices are on the rise as heat dissipation is becoming a crucial factor in system operation. Thermal energy storage (TES) has seen a peak in interest for its capability to take a material that can store energy as a change in internal energy from a combination of sensible and latent heat. Phase change materials (PCM) are notorious as a means for latent heat energy storage, where thermal energy is stored when the material is changing phase \cite{Sharma2009}. Heat transfer can be improved with PCMs as the temperature at phase change is constant and can be selected in a way to enhance system performance \cite{Zalba2003}.

% TES devices work into system architectures much like a battery, with the distinction being the storage of thermal rather than electrical energy. The TES system has two main functions: charging, where the PCM freezes from liquid to solid, and discharging, where the PCM melts from solid to liquid. The activation of these main functions play an important role in the system capabilities and overall performance. Control-oriented modeling captures the thermal dynamics of the system and can capture nonlinearities to predict TES behavior. Improved estimations of TES behavior is critical to be incorporated in control, such as MPC, as it can optimize actual system performance 

Traditional TES modeling approaches rely on dividing the PCM into multiple sections, where each section is modeled using a lumped-parameter approach.  This Fixed Grid (FG) approach, also referred to as Finite Volume, is widely used in the literature \cite{Shanks2022,Pangborn2015,Fasl2013} and is similar to a finite difference scheme \cite{Fortunato2012}. While this approach has proven to accurately model the complex dynamics of a TES device using relatively simple dynamics for each individual grid section, a large number of grid sections is needed to achieve this accuracy, resulting in a large number of dynamic states that is no longer practical for many control designs.

% This simplified theoretical model can be useful but requires a large amount of grid sections in order to be accurate. When incorporating into experiments, small step sizes can lead to higher accuracy for systems, which could be a disadvantage when trying to incorporate a large number of states and implement control \cite{Fasl2013} \cite{Shanks2022}. This can prove to be difficult when trying to model ice melting around a cylinder, as it melts in an egg-like shape \cite{Bathelt1979}.

This paper aims to develop accurate control-oriented models of PCM-based TES devices using a graph-based switched Moving Boundary (MB) approach. Graph-based modeling \cite{Wang2020,Laird2021,Laird2019,Pangborn2020,Shanks2022} is used to develop both FG and MB models, where a graph is used to clearly identify the underlying structure of thermal energy storage and transfer throughout the TES device.  While the FG model divides the PCM into $ n $ sections, each with its own dynamic enthalpy state, the proposed MB approach only requires three states corresponding to the enthalpies of the solid and liquid regions of the PCM and the overall State-of-Charge (SOC), defined as the mass of the solid portion compared to the total mass of the PCM.

% The main dynamics of the TES are modeled using a graph-based approach, which divides components of the system and portrays the interactions between each part of the system \cite{Wang2020}. Simplifications to TES modeling try to minimize the required number of states by grouping the entire TES as a singular component. This can be advantageous as everything is grouped up as one component, but this is only successful for PCMs with a high thermal conductivity \cite{Laird2021}.
\begin{figure*}[t]
    \centering
    \includegraphics[width=\textwidth] {Figures/Fixed_Grid_Modeling.pdf}
    \vspace{-22pt}
    \caption{Fixed Grid modeling framework. 
    \textbf{LEFT:} Cylindrical TES with inner and outer walls and the PCM divided into $n$ grid sections. 
    \textbf{TOP RIGHT:} Identification of key radii used to model the 1-dimensional radial heat transfer.
    \textbf{BOTTOM RIGHT:} Graph-based FG model with $ n $ PCM vertices.}
    \label{fig:Fixed_Grid_Modeling}
    \end{figure*}
    
Several MB approaches to TES modeling have recently been proposed \cite{Fasl2013,Laird2019,Pangborn2020} but each has limitations. Specifically, the TES devices modeled in both \cite{Laird2019} and \cite{Fasl2013} are limited to operation where heat flows in only one direction through the PCM, i.e., heat always enters on one side and exits on the other. However, many TES devices operate by exchanging heat with a single working fluid flowing through the center of the TES. The TES model developed in \cite{Pangborn2020} captures how a heat transfer effectiveness coefficient for the PCM is a function of SOC and the mode of operation (charging or discharging) but modeling the completely melted or solidified modes of operation is left for future work.
% TES operation in the latent heat region has been done previously \cite{Pangborn2020}, but this approach cannot handle the cases of completely solid or liquid.

% \hl{should add some discussion of Hierarchical Hybrid MPC for Management of Distributed Phase Change Thermal Energy Storage by Herschel Pangborn, for example that work only assumes operation in the latent heat region and does not handle fully solid or fully liquid operation.}

% incorporate modeling of the PCM with a MB, which causes the solid-liquid boundary to move based on whether latent heat is being absorbed or released \cite{Zalba2003}. This splits up the TES into three regions: solid, liquid, and the interface between the solid and liquid regions \cite{Fasl2013}. Modeling with this approach can create limitations on how the TES can be used in system models. These limitations can result in only being able to account for phase change in one direction \cite{Laird2019}. This problem can be addressed with a finite state machine (FSM) \cite{Dede2016}, which can incorporate the dynamics of the TES in different modes of operation, such as charging and discharging.

The proposed switched MB approach overcomes these limitations using a Finite State Machine (FSM) to model the mode-dependent dynamics associated with freezing, melting, completely solid, and completely liquid operation. A similar FSM is used to model the operation of ultracapacitors \cite{Dede2016}. When compared to the FG model in a simulated example, the proposed switched MB model achieves a maximum error of $5\%$ for the SOC of the PCM with a $80\%$ reduction in computational time. Therefore, the primary contribution of this paper is the specific formulation of a graph-based switched MB model that is able to accurately predicted the key dynamics of PCM-based TES devices with far fewer states when compared to a more traditional FG model.

The remainder of the paper is organized as follows. Section II introduces and motivates the graph-based modeling framework. The traditional FG approach for TES modeling is presented in Section III while the proposed switched MB approach is presented in Section IV. The simulation accuracy and computational efficiency of the proposed modeling framework is compared to the FG approach in Section V. Finally, conclusions and future work are summarized in Section VI.


\section{Graph-Based Modeling Framework}

This paper employs graph-based modeling to capture the storage and transfer of energy in PCM-based TES devices. Specifically, Fig. \ref{fig:Fixed_Grid_Modeling} shows the graph-based model using the FG approach for a cylindrical TES device.
% graph-based modeling of the candidate system shown in Fig. \ref{fig:Fixed_Grid_Modeling}, which captures the design of the thermal energy storage system based on the FG framework.
This TES device consists of two concentric cylindrical pipes, where the PCM is encapsulated between the inner and outer pipes. A working fluid flows through the inner pipe with an inlet temperature $T_{in}$ and is the main mechanism in which heat is transferred between the TES device and the remainder of the overall thermal management system (which is not modeled in this work). The outer wall of the outer pipe is assumed to exchange heat with ambient air at temperature $ T_{air} $. In the FG approach, the PCM is divided to $ n $ grid sections, where the $i^{th}$ grid section is assumed to have a uniform enthalpy $ h_i $.  This paper only considers the radial heat transfer of the TES device and assumes uniform behavior along the length of the  device $ L $. While the proposed approach is intended to extend to TES of different geometries, only the one-dimensional radial case is considered in this paper.
% the center pipe flow is the working fluid and between the pipes encapsulates the PCM, which is broken up into a total of $n$ sections. The working fluid is used to be the main mechanism to transfer heat to and from the phase change TES system. Additionally pictured is the labeling of distances in one dimension, with the corresponding graph-based model that will be elaborated on in this section.

When capturing the structured dynamics of a system, a graph consists of a set of $ N_v $ dynamic vertices $V = \{v_{i}:i\in[1,N_{v}]\}$, representing energy stored by capacitative sections of a system, and a set of $ N_e $ edges $E=\{e_{j}:j\in[1,N_{e}]\}$, representing power flows among these capacitative sections. Note that $ [1,N_{v}] $ is used to denote the set of integers between $ 1 $ and $ N_v $. Each edge $ e_j $ has an orientation denoting the direction of positive power flow $ P_j $ from the tail vertex $ v_{j}^{tail} $ to the head vertex $ v_{j}^{head} $. Based on conservation of energy, the energy stored by $i^{th}$ vertex $v_{i}$ (quantified by the dynamic state $x_{i}$) can be expressed as
\begin{equation} \label{eq:conservationOFEnergy}
	C_{i}{\dot x}_{i}=\sum\limits_{e_{j} \in {E_{i}}^{in}} P_{j}-\sum\limits_{e_{j} \in {E_{i}}^{out}} P_{j},
\end{equation}
where $ C_i $ is the energy storage capacitance while ${E_{i}}^{in}$ and ${E_{i}}^{out}$ are the set of edges directed into and out of vertex $v_{i}$. Generally, in a graph-based modeling framework, the power flow $P_{j}$ is constrained to be a function of an associated input $\tilde{u}_{j}$ and the state of the tail and head vertices, $x_{j}^{tail}$ and $x_{j}^{head}$, such that 
\begin{equation} \label{eq:powerFlow}
	P_{j}= f_{j}(x_{j}^{tail},x_{j}^{head},\tilde{u}_{j}).
\end{equation}

In general, the graph-based modeling framework allows for power to enter the system along source edges as discussed in \cite{Wang2020}.  For the TES device shown in Fig. \ref{fig:Fixed_Grid_Modeling}, there are two sources into the system: heat transfer with the main working fluid ($wf$) and with the surrounding air. 
% The thermal insulation for the system is assumed to be high, which will neglect heat transfer from outside the system.  
For heat that is being transferred out of the system, a sink vertex denoted $V^{out} = \{v_{i}^{out}:i\in[1,N_{v}^{out}]\}$ is included in the graph. This vertex has an associated state $ x_i^{out} $ that serves as the outlet of the working fluid.

The structure of the graph, including both the dynamic vertices and sink vertices, is captured by the incidence matrix $ M = [m_{ij}] \in \mathbb{R}^{(N_v + N_v^{out}) \times N_e } $ defined as
\begin{equation} \label{eq:Incidence_Matrix_Creation}
m_{ij}=  \begin{cases}
+1 & \text{if } v_i \text{ is the tail of } e_j, \\
-1 & \text{if } v_i \text{ is the head of } e_j, \\
0 & \text{else}.
\end{cases}
\end{equation}
The incidence matrix is partitioned based on dynamic and sink vertices such that
\begin{equation}
	M = \begin{bmatrix} \bar{M} \\ \ubar{M} \end{bmatrix} \text{ with } \bar{M} \in \mathbb{R}^{N_v \times N_e },
\end{equation}
where the indexing of vertices is assumed to be ordered such that $ \bar{M} $ is a structural mapping from power flows 
\begin{equation} \label{eq:powerFlows}
P = F(x,x^{out},\tilde{u}) = [f_j(x_j^{tail},x_j^{head},\tilde{u}_j)],
\end{equation}
to states $ x = [x_i] $, $ i \in [1,N_v] $, and $ \ubar{M} $ is a structural mapping from $ P $ to sink states $ x^{out} = [x_i^{out}] $, $ i \in [1,N_v^{out}] $. 
Combining the individual conservation equations from \eqref{eq:conservationOFEnergy} using the structure of the graph captured by $ \bar{M} $, the overall system dynamics are
\begin{equation} \label{eq:graphDynamics}
C \dot{x} = -\bar{M} P =  -\bar{M} F(x,x^{out},\tilde{u}),
\end{equation}
where $ C = diag([C_i]) $, $ i \in [1,N_v] $ is a diagonal matrix of capacitances. Since some edges do not have a control input and a single input can affect multiple edges, it is often advantageous to let $ \tilde{u} \in \mathbb{R}^{N_e} $ be a virtual input vector, corresponding to the $ N_e $ edges, and define $ u \in \mathbb{R}^{N_u} $ as a system input vector, corresponding to the subset of $ N_u $ unique inputs that affect the system.  As such, the matrix $ \Phi \in \mathbb{R}^{N_e \times N_u} $ can be used to map the system inputs to the virtual inputs such that $ \tilde{u} = \Phi u $.

One benefit of a graph-based modeling framework is that the linear structure of the graph is captured by \eqref{eq:graphDynamics} and the majority of the modeling effort focuses on defining the potentially nonlinear power flow relationships in \eqref{eq:powerFlows}. The following section presents the graph capturing the structure of the system shown in Fig. \ref{fig:Fixed_Grid_Modeling} and the vertex and edge properties used to model the dynamics. 
% The graph structure is shown in Fig. \ref{fig:Fixed_Grid_Modeling}.



\section{Fixed Grid TES Modeling Framework}
\subsection{Modeling Assumptions}
% While the proposed modeling approach is applicable to TES with either cylindrical or rectangular geometries, a cylindrical geometry is assumed for the remainder of the paper. 
% The main components of TES are the working fluid, the inside wall, the PCM, and the outer wall. The TES will be incorporated into the graph-based framework, with the following assumptions:
The dynamics of the TES, comprised of the working fluid, inner wall, PCM, and outer wall, are modeled using a graph-based framework with the following assumptions.
\begin{itemize}
    \item Heat transfer within the TES is radially symmetric and uniform along the length of the device.
    \item Heat transfer is assumed to be purely conductive. Natural convection in the liquid is not taken into account, similar to \cite{Laird2019}. Future experimental work similar to \cite{jalil2006,Kahraman1998} will focus on quantifying and incorporating the effects of natural convection into the graph-based modeling framework.
    % This would speed up the melting process, but is difficult to calculate numerically without experimentation \cite{Jalil2006,Kahraman1998}.
    \item The mass in the PCM is assumed to be constant with time-varying volume based on the density changes associated with phase change.
    \item Heat transfer between the working fluid and the inner pipe is governed by the outlet temperature of the working fluid.
    \item All material properties are phase dependent but constant within each phase.
    \item The pressure of the PCM is assumed to be constant over time, space, and phase and does not influence the TES dynamics.
\end{itemize}

The following graph-based models use enthalpies as system states, since temperature cannot be used to quantify thermal energy during phase change. The PCM is generically assumed to have a saturated solid state enthalpy of $ h = 0 \, kJ/kg $, a latent heat of fusion of $ h_f $, and a saturated temperature of $ T_{sat} $.
% The temperature of the $j^{th}$ section, $T_j$, is used in the heat transfer analysis and 
Temperature $T $ for the PCM is defined as
\begin{equation}
    T =  \begin{cases}
    \frac{h}{C_{p,\sigma}} + T_{sat} & \text{if } h < 0, \\
    T_{sat} & \text{if } 0 \leq h \leq h_f, \\
    \frac{h-h_f}{C_{p,\sigma}} + T_{sat} & \text{if } h > h_f, \\
    \end{cases}
    \end{equation}
where $C_{p,\sigma}$ is the phase-dependent specific heat capacity of the PCM and the phase
$\sigma$, either solid (S) or liquid (L), is 
\begin{equation}
    \sigma =  \begin{cases}
    S & \text{if } T < T_{sat}, \\
    L & \text{if } T \geq T_{sat}.
    \end{cases}
\end{equation}
For single-phase materials, such as the working fluid and the pipe walls, temperature is defined as $T = \frac{h}{C_p}$,
% \begin{equation}
%     T = \frac{h}{C_p},
% \end{equation}
where $C_p$ is the specific heat capacity of the material.


\subsection{Fixed Grid Approach}
The traditional FG approach to modeling PCM-based TES devices divides the volume into $n$ grid sections \cite{Shanks2022,Fasl2013}. The FG modeling framework is used as a reference in this paper, representing the true dynamic behavior of the TES to be approximated by the proposed switched MB approach. 
% This model works by splitting up the PCM region into $n$ cross sections. 
% The distance between sections, $r_{n+1} - r_n$ $\forall n=4,5,...,28$ remains the same, which will result in varying amounts of mass in each region, with $m_{n+1} > m_n$ $\forall n$.
%The following graph-based framework defines the state of the system as the enthalpy of each region ($h_j$) because it can capture the energy across phase change. This approach requires a total of $h_j \forall j \leq n+3$ states. With heat transfer properties varying with phase, $\sigma$ is defined to account for properties that will be either solid (S) or liquid (L),
%This model is transferred into the graph-based framework, where all energy storage components of the system are represented as states through vertices on the graph. Power flows between vertices represent the heat transfer between states that are in contact with each other. This approach requires a total of $h_j \forall j <= n+3$ states. The following modeling framework will take the state of the system as the enthalpy ($h$) of the region. This state can capture the energy of a two-phase system better than the temperature because the temperature across the phase change does not change.
% defines the state of the system as the enthalpy of each region ($h_j$) because it can capture the energy across phase change. 
As shown in Fig. \ref{fig:Fixed_Grid_Modeling}, the FG approach requires a total of $ n+3 $ states such that $ x \in \mathbb{R}^{n+3} $, where $ x = [h_{wf}, \, h_{inn.}, \, h_1, \, \dots, \, h_n, h_{out.} ]^\top $ are the enthalpies of the working fluid, the inner wall, the $ n $ sections of PCM, and the outer wall.
% , $ h_j,  j \in [1,n+3]$, corresponding to the working fluid, the inner and outer walls, and the $ n $ sections of PCM 

% The enthalpy for the saturated solid state of the PCM is generically assumed to be $ h = 0 \, kJ/kg $ and have a latent heat of fusion of $ h_f \, kJ/kg $. 
% Zero is defined as the point where the solid PCM begins to melt, where $h_j \leq 0$ represents the section in solid state. 

% \subsection{FG Approach JK}
The following graph-based FG model is derived from the approach presented in \cite{Fasl2013} and the radial heat transfer equations from \cite{Bergman2011}. Modeling each vertex in Fig. \ref{fig:Fixed_Grid_Modeling} using conservation of energy, with state $ h_i $ for the $ i^{th} $ vertex, the energy storage capacitance $ C_i $ from \eqref{eq:conservationOFEnergy} is the mass of the vertex such that $ C_i = \rho_i  V_i $ for the single-phase material vertices $ i \in \{1, 2, n+3\} $ and $ C_i = \rho_{i,\sigma} V_i $ for the PCM vertices $ i \in \{3,n+2\} $. The density $ \rho_i $ is assumed constant for single-phase materials while $ \rho_{i,\sigma} $ denotes the fact that the PCM density is phase-dependent. The volumes $ V_i $ for the three single-phase vertices are defined as $ V_1 = \pi L r_1^2 $, $ V_2 = \pi L (r_3^2 - r_1^2) $, and $ V_{n+3} = \pi L (r_{n+6}^2 - r_{n+4}^2) $, based on the radii labelled in Fig. \ref{fig:Fixed_Grid_Modeling}, where $ L $ is the length of the TES device. The PCM is divided into $ n $ sections of equal width $ \Delta r = \frac{r_{n+4} - r_3}{n} $ such that the volumes $ V_i, \, i \in [3,n+2] $, are defined as $ V_i = \pi L [(r_{i+1} + \frac{\Delta r}{2})^2 - (r_{i+1} - \frac{\Delta r}{2})^2 ]$.

Each power flow $ P_j $ can be expressed in the form of \eqref{eq:powerFlow} assuming positive power flow in the direction of the arrows shown in Fig. \ref{fig:Fixed_Grid_Modeling}. The advective power flows associated with the working fluid are $ P^{in}_1 = \dot{m}_{wf} C_{p,wf} T_{in} $ and $ P_1 = \dot{m}_{wf} C_{p,wf} T_1 $,
% \begin{gather}
%     P^{in}_1 = \dot{m}_{wf} C_{p,wf} T_{in}, \\
%     P_1 = \dot{m}_{wf} C_{p,wf} T_1, 
%     \end{gather}
where $\dot{m}_{wf}$ is the mass flow rate and $C_{p,wf}$ is the specific heat capacity of the working fluid. For heat transfer from the surrounding air into the TES, the outer wall is a combination of the pipe material and insulation,
\begin{equation}
     \begin{gathered}
     P^{in}_2 = \frac{1}{R_{out.} + R_{air}} (T_{air} - T_{n+3}),\\
     R_{out.} = \frac{ln(\frac{r_{n+6}}{r_{n+5}})}{2 \pi L k_{out.}} + R_{ins.},
     R_{air} = \frac{1}{2 \pi r_{n+6} L h_{air}},
    \end{gathered}
\end{equation}
where $ k_{out.} $ is the thermal conductivity of the outer pipe, $ h_{air} $ is the convective heat transfer coefficient for the air, and $R_{ins}$ is the insulation resistance. For power flows $ P_j, \, j \in [2,n+3] $,
\begin{equation} \label{eq:Power_Flow}
    P_j = \frac{1}{R_j} (T_{j} - T_{j-1}).
\end{equation}
Since each power flow $ P_j, \, j \in [2,n+3] $, goes through two different materials, the total thermal resistance is defined as $ R_j = R_{j,A} + R_{j,B} $, where $ R_{2,A} = \frac{1}{2 \pi r_{1} L h_{wf}} $, $ R_{2,B} = \frac{ln(\frac{r_2}{r_1})}{2 \pi L k_{inn.}}$, $ R_{3,A} = \frac{ln(\frac{r_3}{r_2})}{2 \pi L k_{inn.}}$, $ R_{n+3,B} = \frac{ln(\frac{r_{n+5}}{r_{n+4}})}{2 \pi L k_{out.}}$, and, $ \forall j \in [3,n+2] $,
\begin{equation}
    R_{j,B} = \frac{ln(\frac{r_{j+1}}{r_{j+1}-\frac{\Delta r}{2}})}{2 \pi L k_{j,\sigma}}, \; R_{j+1,A} = \frac{ln(\frac{r_{j+1}+\frac{\Delta r}{2}}{r_{j+1}})}{2 \pi L k_{j,\sigma}},
\end{equation}
where $ k_{inn.} $ is the thermal conductivity of the inner pipe and $ h_{wf} $ is the convective heat transfer coefficient for the working fluid.


%\subsection{FG Approach OLD}
%Following the FG approach from \cite{Fasl2013} and radial heat transfer equations from \cite{Bergman2011}, conservation of energy applied to the $ j^{th} $ grid section results in the enthalpy dynamics,
%\begin{equation} 
    %\begin{gathered} \label{eq:FG_ConservationOfEnergy}
    %\rho_{\sigma,j} V_j \frac{dh_j}{dt} = k_{\sigma,j-1} A_{s,j-1} \left(\frac{dT}{dr}\right)_{j-1} + k_{\sigma,j} A_{s,j} \left(\frac{dT}{dr}\right)_{j},\\
    %V_j = \pi L [(r_j+\Delta r_j)^2 - (r_j-\Delta r_j)^2], \\
    %\left(\frac{dT}{dr}\right)_{j} = \frac{T_j - T_{j-1}}{\Delta r_j},\\
    %A_{s,j} = 2 \pi r_j L, \quad
    %\Delta r_j = (r_{v_j}-r_{v_{j-1}}),
    %\end{gathered}
    %\end{equation}
%where $\rho_j$ is the density, $\Delta r_j$ is the difference in radii between sections $v_j$ and $v_{j-1}$, $V_j$ is the volume, $A_{s,j}$ is the surface area, and $L$ is the length of TES device. By assuming that heat transfer is constant in the radial direction, \eqref{eq:FG_ConservationOfEnergy} can be expressed as \eqref{eq:conservationOFEnergy},

%\begin{equation} \label{eq:FG_COE_Final}
%\begin{gathered}
%    \rho_j V_j \frac{dh_j}{dt} = k_{\sigma} 2 \pi L ln(\frac{r_{j-1} + \Delta r_{j-1} /2}{r_{j-1}}) (T_{j+1}-T_j) + \\
%    k_{\sigma} 2 \pi L ln(\frac{r_{j}}{r{j} + \Delta r_j /2}) (T_{j-1}-T_{j}),
%    \end{gathered}
%    \end{equation}

%Now taking equations \eqref{eq:FG_ConservationOfEnergy} and \eqref{eq:FG_VA} to rearrange them into the form of \eqref{eq:conservationOFEnergy},

%\begin{equation}
    %\rho_j V_j \frac{dh_j}{dt} = k_{\sigma,j-1} A_{s,j-1} \frac{T_{j+1}-T_j}{\Delta r_{j}} + k_{\sigma,j} A_{s,j} \frac{T_{j-1}-T_{j}}{\Delta r_{j-1}}.
%\end{equation}

%To express \eqref{eq:FG_ConservationOfEnergy} in the form of \eqref{eq:conservationOFEnergy}, the capacitance of the $ j^{th} $ section is
%\begin{equation}
%        C_j = \rho_j V_j  \quad \quad j \in [1,n+3].
%    \end{equation} 

%The power flows are pulled out and are taken to where each section has its own thermal resistance,
%For sections within the TES, power flow takes all sections between regions and treats each with a separate resistance. To incorporate each region, 
%\begin{subequations}
%    \begin{gather}
%        P_j = \frac{1}{R_{jA} + R_{jB}} (T_{j-1} - T_{j}),  \quad j \in [2,n+3],\\
%         R_{jA} = k_{\sigma} 2 \pi L ln(\frac{r_{j-1} + \Delta r_{j-1} /2}{r_{j-1}}),\\
%         R_{jB} = k_{\sigma} 2 \pi L ln(\frac{r_{j}}{r_{j} + \Delta r_j /2}),
%    \end{gather}
%    \end{subequations}
%where both $R_{jA}$ and $R_{jB}$ are the conductive thermal resistances for the first and second sections, respectively, and both $k_j$ and $k_{j-1}$ are the thermal conductivity of the specified section.

% The power flow is taken as one-dimensional conductive heat transfer in a radial system \cite{Bergman2011}. 
%Heat transfer for a fluid based on internal flow \cite{Bergman2011} is used for the working fluid,
%\begin{equation} \label{eq:convective_heat_flow}
%    P_{conv} = \dot{m} C_{p,\sigma} (T_{j+1} - T_j),
%\end{equation}
%where $P_{conv}$ is the power flow for convective heat transfer. The power flows associated with the working fluid can be defined from \eqref{eq:convective_heat_flow},
%\begin{gather}
%    P^{in}_1 = \dot{m}_{wf} C_{p,wf} T_{in}, \\
%    P_1 = \dot{m}_{wf} C_{p,wf} T_{wf}, 
%    \end{gather}
%where $\dot{m}_{wf}$ is the mass flow rate and $C_{p,wf}$ is the specific heat of the working fluid.

%Heat transfer between both the working fluid to the inner wall and the outer wall to the air incorporates conductive and convective heat transfer,
%    \begin{gather}
%        P_2 = \frac{1}{R_{wf} + R_{inn.}} (T_{inner} - T_{wf}),\\
%         R_{wf} = \frac{1}{2 \pi r_{1} L h_{wf}},
%         R_{inn.} = \frac{ln(\frac{r_1}{r_1/2})}{2 \pi L k_{inner}},
%         \end{gather}
%         \begin{gather}
%        P^{in}_2 = \frac{1}{R_{out.} + R_{air}} (T_{air} - T_{outer}),\\
%         R_{out.} = \frac{ln(\frac{r_{n+2}}{r_{n+1} + \Delta r})}{2 \pi L k_{outer}},
%         R_{air} = \frac{1}{2 \pi r_{n+2} L h_{air}},
%    \end{gather}
% where $R_{inn.}$ and $R_{out.}$ are the conductive thermal resistance of the inner and outer wall, respectively, $k_{inner}$ and $k_{outer}$ are the thermal conductivity of the inner and outer wall, respectively, $R_{wg}$ and $R_{air}$ are the convective thermal resistance of the working fluid and air, respectively, and $h_{wf}$ and $h_{air}$ are the convective heat transfer coefficient of the working fluid and air, respectively.



%\subsection{FG Approach NEW}
%Following the FG approach from \cite{Fasl2013} and the radial heat transfer equations from \cite{Bergman2011}, the rate of which energy is transferred through conduction ($P_{cond}$) or convection ($P_{conv}$) across a cylindrical surface are defined as,
%\begin{equation} 
%    \begin{gathered} \label{eq:FG_Power_Resistance_Definition}
%    P_{cond} = \frac{1}{R_{cond}} (T_x - T_y),\quad
%    R_{cond} = \frac{ln(r_y/r_x)}{2 \pi L k}\\
%    \end{gathered}
%    \end{equation}
%    \begin{equation}
%    \begin{gathered}
%    P_{conv} = \frac{1}{R_{conv}} (T_x - T_y),\quad
%    R_{conv} = 2 \pi r_y L h, 
%    \end{gathered}
%    \end{equation}
%where $r_x$ and $r_y$ are defined radii of vertices x and y, respectively,  by the relation $r_y > r_x$, $R_{cond}$ and $R_{conv}$ are the thermal resistances for conductive and convective heat transfer, respectively, $T_x$ and $T_y$ are the temperatures of locations x and y, respectively, $k$ is the thermal conductivity, $h$ is the convective heat transfer coefficient, and $L$ is the length of TES device.

%With neighboring vertices in Fig. \ref{fig:Fixed_Grid_Modeling} all having different properties based on the material and phase, heat transfer between two vertices accounts for each vertex having its own thermal resistance. The resistances are connected in series, stating the total resistance between two vertices as,
%\begin{equation} \label{eq:Total_Resistance}
%    R_{tot} = \sum R_i,
%\end{equation}
%where $R_i$ is the thermal resistance of section $ i $ between two vertices. This can be applied in the heat transfer rate equation and with \ref{eq:Incidence_Matrix_Creation}, in which we assume the power flow goes from vertex $x$ to vertex $y$,
%\begin{equation} \label{eq:Total_Power_Resistance}
%    P_{tot} = \frac{1}{R_{tot}} (T_x - T_y).
%\end{equation}


%Heat transfer between both the inner wall to the working fluid ($P_2$) and the air to the outer wall ($P^{in}_2$) incorporates conductive and convective heat transfer, which uses \eqref{eq:FG_Power_Resistance_Definition} and \eqref{eq:Total_Power_Resistance},
%\begin{equation}
%\begin{gathered}
%    P_2 = \frac{1}{R_{wf} + R_{inn.,A}} (T_{inner} - T_{wf}),\\
%     R_{wf} = \frac{1}{2 \pi r_{1} L h_{wf}},
%     R_{inn.,A} = \frac{ln(\frac{r_2}{r_1})}{2 \pi L k_{inner}},
%     \end{gathered}
%     \end{equation}
%\begin{equation}
%     \begin{gathered}
%    P^{in}_2 = \frac{1}{R_{out.,B} + R_{air}} (T_{air} - T_{outer}),\\
%     R_{out.,B} = \frac{ln(\frac{r_{n+6}}{r_{n+5}})}{2 \pi L k_{outer}},
%     R_{air} = \frac{1}{2 \pi r_{n+6} L h_{air}},
%    \end{gathered}
%    \end{equation}
% where $k_{inner}$ and $k_{outer}$ are the thermal conductivity of the inner and outer wall, respectively, $R_{wg}$ and $R_{air}$ are the convective thermal resistance of the working fluid and air, respectively, and $h_{wf}$ and $h_{air}$ are the convective heat transfer coefficient of the working fluid and air, respectively. $R_{inn.,A}$ and $R_{out.,B}$ are the conductive thermal resistance of the section of the inner or outer wall between the working fluid or air, respectively, with $ A $ referencing the section to the left of the vertex and $ B $ referencing the section to the right of the vertex.

% Heat transfer with the PCM to the inner wall ($P_3$) and the outer wall to the PCM ($P_{n+3}$) include sections that have different thermal resistances,
%\begin{equation}
%\begin{gathered}
%%    P_3 = \frac{1}{R_{inn.,B} + R_{h_1,A}} (T_{h_1} - T_{inner}),\\
%     R_{inn.,B} = \frac{ln(\frac{r_3}{r_2})}{2 \pi L k_{inner}},
%     R_{h_1,A} = \frac{ln(\frac{r_4}{r_3})}{2 \pi L k_{\sigma}},
%     \end{gathered}
%     \end{equation}
%\begin{equation}
%    \begin{gathered}
%    P_{n+3} = \frac{1}{R_{h_n,B} + R_{out._A}} (T_{outer} - T_{h_n}),\\
%     R_{h_n,B} = \frac{ln(\frac{r_{n+4}}{r_{n+3}})}{2 \pi L k_{\sigma}},
%     R_{out.,A} = \frac{ln(\frac{r_{n+5}}{r_{n+4}})}{2 \pi L k_{outer}},
%    \end{gathered}
%    \end{equation}
% where $k_{\sigma}$ is the thermal conductivity of the PCM, depending on $\sigma$, and both $R_{h_1,A}$ and $R_{h_n,B}$ are thermal resistance of sections 1 and n of the PCM, respectively.

%Heat transfer between vertices in the PCM are dependent on the phase of each vertex,
%\begin{equation}
%\begin{gathered} \label{eq:PCM_Power_Flow}
%    P_j = \frac{1}{R_{h_{v_{j-1}},B} + R_{h_{v_j},A}} (T_{v_j} - T_{v_{j-1}}), \quad j \in [4,n+2],\\
%     R_{h_{v_{j-1}},B} = \frac{ln(\frac{r_{v_{j-1}}+\Delta r}{r_{v_{j-1}}})}{2 \pi L %k_{\sigma}},
%     R_{h_{v_j},A} = \frac{ln(\frac{r_{v_j}}{r_{v_{j-1}+\Delta r}})}{2 \pi L k_{\sigma}},
%     \end{gathered}
%     \end{equation}
%where subscript $v_j$ is used to specify a radius, power flow, resistance, or temperature at a specific vertex, and $\Delta r$ is the distance between two vertices, which is a constant.

%Heat transfer for a fluid based on internal flow \cite{Bergman2011} is used for the working fluid,
%\begin{equation} \label{eq:convective_heat_flow2}
%    P_{conv} = \dot{m} C_{p,\sigma} (T_{j+1} - T_j),
%\end{equation}
%where $P_{conv}$ is the power flow for convective heat transfer. The power flows associated with the working fluid can be defined from \eqref{eq:convective_heat_flow2},
%\begin{gather}
%    P^{in}_1 = \dot{m}_{wf} C_{p,wf} T_{in}, \\
%    P_1 = \dot{m}_{wf} C_{p,wf} T_{wf}, 
%    \end{gather}
%where $\dot{m}_{wf}$ is the mass flow rate and $C_{p,wf}$ is the specific heat of the working fluid.

%The change in energy at a vertex uses \eqref{eq:conservationOFEnergy}, with the state of the vertex being the change in enthalpy ($\dot{h}$). The capacitance associated with the $ j^{th} $ vertex is the total mass at the vertex,
%\begin{equation}
%\begin{gathered}
%        C_j = \rho_j  V_j \quad \quad j \in \{1, 2, n+3\},\\
%        V_j = \pi L [(r_{v_j}+\Delta r_{j+1}/2)^2 - (r_{v_j}-\Delta r_j/2)^2],\\
%        \Delta r_j = r_{v_j} - r_{v_{j-1}}
%        \end{gathered}
%    \end{equation} 
%    \begin{equation}
%        \begin{gathered}
%        C_j = \rho_{\sigma} V_j  \quad \quad j \in \{3,n+2\},\\
%        V_j = \pi L [(r_j+\Delta r)^2 - (r_j-\Delta r)^2],
%        \end{gathered}
%    \end{equation} 
%where at the $ j^{th} $ vertex is the density ($\rho_j$) and volume ($V_j$), and $\Delta r_j$ is the difference in radii between two vertices, which is defined to account for varying thicknesses of both the inner and outer wall.

%\hl{Everything above this up until FG Approach NEW is the new proposed section}


For the numerical example presented in Section \ref{Results}, Fig. \ref{fig:Freeze_Comp_Steps.pdf} shows the results of a series of tests to determine the behavior of the FG model as a function of $ n $. The top plot shows the simulated time required to completely freeze the TES, $t_{freeze}$, for different values of $ n $. While $t_{freeze}$ converges for increasing $ n $, the second plot shows the associated increase in computation time for the simulation, $t_{comp}$. These simulations were conducted in MATLAB Simulink using the variable step solver ode23tb. The third plot shows that the increase in computation time is due to an increasing number of states and simulation time steps. Based on the results of Fig. \ref{fig:Freeze_Comp_Steps.pdf}, $ n = 35 $ sections was chosen for comparison with the proposed switched MB approach presented in the following section.

% Implementation of the method included testing for what $n$ of grid sections would be best representative of TES as a whole. To find the convergence of the FG model, Fig. \ref{fig:Freeze_Comp_Steps.pdf} shows the time to completely freeze, $t_{freeze}$ for a specified number of grid sections. This determined that $n = 35$ nodes is the minimum number of nodes required to lower computational efforts while maintaining high accuracy.

\begin{figure}[t]
        %\centering
        \includegraphics[width=\columnwidth] {Figures/Freeze_Computation_Steps.pdf}
        \vspace{-15pt}
        \caption{Computational comparisons of the FG and MB approaches. 
        \textbf{TOP:} Time the model estimates for the PCM to completely freeze, $t_{freeze}$. 
        \textbf{MIDDLE:} Computational time, $t_{comp}$. 
        \textbf{BOTTOM:} Number of time steps taken with the ode23tb variable step solver, $n_{steps}$. 
        All results are taken as an average over 50 simulations.}
        \label{fig:Freeze_Comp_Steps.pdf}
        \end{figure}

% For this to be used in practice, each state would have to be monitored in each iteration of the usage. Fig. \ref{fig:Freeze_Comp_Steps.pdf} shows the computational time, $t_{comp}$, for varying $n$. The simulations use a variable step solver, ode23tb, to try and optimize necessary number of steps, which are compared to the MB Line as a reference. As shown, having a large number of states can make control strategy implementation difficult and impractical for use in TES. The following section will show how the proposed solution can solve issues with the FG method.

\begin{figure*}[h]
    \centering
    \includegraphics[width=\textwidth] {Figures/Moving_Boundary_Modeling.pdf}
    \vspace{-22pt}
    \caption{Proposed MB modeling framework. 
    \textbf{LEFT:} Cylindrical TES with inner and outer walls and the PCM divided into solid and liquid regions, with states $h_S$ and $h_L$, respectively. 
    \textbf{TOP RIGHT:} Identification of key radii used to model the 1-dimensional radial heat transfer.
    \textbf{BOTTOM RIGHT:} Graph-based MB model with three vertices for the PCM.}
    \label{fig:Moving_Boundary_Modeling}
    \end{figure*}

\section{Proposed Switched Moving Boundary Modeling Framework}
\subsection{Moving Boundary Model}
The MB approach aims to capture the primary dynamics of the TES using a reduced number of states to generate a model that can be directly used for control design.
% looks to lower the states of the model while maintaining the complexity of the system. 
As shown in Fig.~\ref{fig:Moving_Boundary_Modeling}, the proposed MB approach requires a total of six states such that $ x \in \mathbb{R}^{6} $, where $ x = [h_{wf}, \, h_{inn.}, \, h_S, \, SOC, \, h_L, \, h_{out.} ]^\top $ has only three PCM states corresponding to the enthalpies of the solid ($h_S$) and liquid ($h_L$) regions of the PCM as well as the SOC.
% Fig. \ref{fig:Moving_Boundary_Modeling} shows the proposed graph-based approach to model the PCM of the TES with only three states corresponding to the enthalpies of the solid ($h_S$) and liquid ($h_L$) regions of the PCM as well as the SOC. When combined with the additional three states for the enthalpies of the working fluid, inner wall, and outer wall, the moving bounding graph-based model has a total of six states. 
This can be significantly fewer states than the FG approach which requires $ n+3 $ states, where $ n = 35 $ was determined to be a practical balance between model accuracy and computational cost. The results of the MB mode are also presented in Fig. \ref{fig:Freeze_Comp_Steps.pdf} as the horizontal red lines, which show that the MB model accurately predicts $t_{freeze}$ with significantly less computation time and simulation steps.

% This comparison between FG and MB models in regards to freezing time, computational cost, and number of steps is shown in Fig. \ref{fig:Freeze_Comp_Steps.pdf}.
% \subsection{MB Approach OLD}
% For the MB model, conservation of energy and mass for a closed system is applied to the solid region ($\sigma = S$) such that
% \begin{gather} \label{eq:Solid_MB_COE}
%     \frac{d}{dt}(M_{\sigma} h_{\sigma}) = P_{\sigma,in} - P_{\sigma,out} + \dot{m}_{SOC} h_{\sigma}, \\
%     M_{\sigma} = M_{tot} (1-SOC), \label{eq:Solid_MB_COE2}
% \end{gather}
% where $M_{\sigma}$ is the mass of the region, $P_{\sigma,in}$ and $P_{\sigma,out}$ are the power flows going into and out of the region, respectively, $\dot{m}_{SOC}$ is the rate at which PCM mass is being solidified, and $M_{tot}$ is the total mass of the PCM. Note that $\dot{m}_{SOC} = \dot{M}_S = -\dot{M}_L$ based on conservation of mass. Similarly, the liquid region ($\sigma = L$) is modeled as
% \begin{gather}\label{eq:Liquid_MB_COE}
%     \frac{d}{dt}(M_{\sigma} h_{\sigma}) = P_{\sigma,in} - P_{\sigma,out} - \dot{m}_{SOC} h_{\sigma}, \\
%     M_{\sigma} = M_{tot} SOC.\label{eq:Liquid_MB_COE2}
% \end{gather}

% % The phase change interface derivation assumes that the amount of energy stored is negligible (equal to zero), meaning all heat and mass transfer will go directly between solid and liquid regions. The conservation of energy is applied at the interface, with heat flowing from the liquid to the solid region, 
% Assuming zero volume for the interface between the solid and liquid regions and that mass enters the interface with enthalpy $ h_L $ and exits with enthalpy $ h_S $, the algebraic conservation of energy equation for the interface is
% \begin{equation} \label{eq:algebraicInterface}
%     0 = (P_{L,out} + \dot{m}_{SOC} h_L) - (P_{S,in} + \dot{m}_{SOC} h_S).
% \end{equation}

% Taking the time derivative of \eqref{eq:Liquid_MB_COE2} and noting that $\dot{m}_{SOC} = -\dot{M}_L$, \eqref{eq:algebraicInterface} results in the following differential equation for SOC,
% \begin{equation}
%     (h_L - h_S) M_{tot} \dot{SOC} = P_{L,out} - P_{S,in}.
% \end{equation}
% % where $\dot{SOC}$ is the change in SOC of the PCM, based on the movement of the interface.

% % The graph-based model for the proposed approach is shown in Fig. \ref{fig:Moving_Boundary_Modeling}, which has six total states as opposed to $n+3$ states previously. For this modeling approach, we had assumed that both freezing and melting start from the center, but as pictured in Fig. \ref{fig:Moving_Boundary_Modeling}, it only shows the freezing case. The proposed graph model has edges connecting all states of the PCM ($v_3, v_4, \text{ and } v_5$) to both the inner and outer walls. This allows for the model to turn edges on and off depending on what mode of operation that the TES is configured to.

% Each vertex of the graph model has a, potentially time-varying, capacitance defined as
% \begin{gather}
%     C_j = \rho_j  V_j \quad \quad j \in \{1, 2, 6\},\\
%     C_3 = M_{tot}  SOC,\\
%     C_4 = M_{tot}  (h_S - h_L),\\
%     C_5 = M_{tot}  (1-SOC).
%     \end{gather}

% Since $h_S < h_L $, $C_4 < 0 $. While capacitances are typically positive in a graph-based modeling framework, $C_4 < 0 $ comes directly from the fact that the state, $ x_4 = SOC $, is defined as 
% \begin{equation}
%     SOC = \frac{M_S}{M_{tot}},
%     \end{equation}
% $M_S$ is the mass of the solid portion of the PCM. As such, $ SOC = 0 $ and $ SOC = 1 $ correspond to the PCM being completely liquid and completely solid, respectively and a larger $ SOC $ corresponds to a lower amount of energy stored in the PCM.

% The graph-based model incorporates instantaneous changes in the operation modes that is dependent on if the TES is charging or discharging. The graph captures these changes through a FSM to implement turning on and off edges for each mode of operation, as shown in Fig. \ref{fig:Finite_State_Machine}.

    
% \subsection{Finite State Machine}
% The TES has four major modes of operation: completely liquid, completely solid, freezing, and melting modes. The FSM accounts for all possible operation modes and uses switching criteria based on the inner pipe surface temperature ($T_{s_i}$) and the SOC. Fig. \ref{fig:Finite_State_Machine} shows the different modes of operation with their respective mode numbers. %A similar FSM model is incorporated in the operation of ultracapacitors \cite{Dede2016}.

% Power flows are adjusted to be able to operate in all four possible operation conditions. Each power flow is as follows,

% \begin{gather} \label{eq:FSM_Power_Flows}
%     P_j = \frac{1}{R_{jA} + R_{jB}} (T^{Tail}_j - T^{Head}_j), \\
%     R_{jA} = \frac{ln(\frac{r^{Tail}_{j-1}}{r^{Head}_{j}})}{2 \pi L k_{\sigma}},
%     R_{jB} = \frac{ln(\frac{r^{Tail}_{j}}{r^{Tail}_{j-1}})}{2 \pi L k_{\sigma}},\quad j \in \{3,8\},\\
%     R_{jA} = \frac{ln(\frac{r^{Tail}_{j+1}}{r^{Tail}_{j}})}{2 \pi L k_{\sigma}},
%     R_{jB} = \frac{ln(\frac{r^{Head}_{j}}{r^{Tail}_{j+1}})}{2 \pi L k_{\sigma}},\quad j \in \{4,7\},\\
%     R_{j} = \frac{ln(\frac{r^{Tail}_{j}}{r^{Head}_{j}})}{2 \pi L k_{\sigma}},\quad j \in \{5,6\}.
%     \end{gather}
% The thermal resistances shown in \eqref{eq:FSM_Power_Flows} all are very similar, but all account for different operation conditions dependent on the location of different sections. 


% \subsection{Moving Boundary Approach}
Since the MB model only changes the configuration of the PCM, power flows $ P_j, \, j \in \{1,2\} $, power inputs $ P^{in}_j, \, j \in \{1,2\} $, capacitances $ C_i, \, i \in \{1,2,6\} $, and resistances $ R_{j,A}, \, j \in \{1,2,3\} $ and $ R_{j,B}, \, j \in \{1,2\} $ are all the same for the working fluid, and the inner and outer wall as defined in the FG model, by replacing $v_{n+3}$ in the FG with $v_6$ in the MB. 

 Modeling the three new vertices in Fig. \ref{fig:Moving_Boundary_Modeling} using conservation of energy, the energy storage capacitance $ C_i $ from \eqref{eq:conservationOFEnergy} is the mass of the vertex such that $ C_3 = M_{tot} SOC $, $C_4 = M_{tot} (h_S-h_L)$, and $C_5 = M_{tot} (1-SOC)$, where $M_{tot}$ is the total mass of the PCM, $SOC = \frac{M_S}{M_{tot}}$, and $M_S$ is the mass of the solid PCM. 
 % Since $h_S < h_L $, $C_4 < 0 $. 
 While capacitances are typically positive in a graph-based modeling framework, $C_4 < 0 $ since $h_S < h_L $, which comes directly from the fact that the state, $ x_4 = SOC $, increases with a decrease in energy storted in the PCM such that $ SOC = 0 $ and $ SOC = 1 $ correspond to the PCM being completely liquid and completely solid, respectively. 
 
The power flow $ P_j, \, j \in [3,8] $, are defined similarly to \eqref{eq:Power_Flow}, such that power flow is driven by the temperature difference between the tail and head vertex temperatures for each edge. Note that $ T_{sat} $ is used as the vertex temperature for $ v_4 $ with state corresponding to SOC. The total total thermal resistance is also still defined as $ R_j = R_{j,A} + R_{j,B} $ but now the radii associated with the solid and liquid regions are time varying. For example, as shown in Fig. \ref{fig:Moving_Boundary_Modeling}, $r_5 = \sqrt{r_3^2 + \frac{M_S}{\rho_S \pi L}}$.
%\hl{$r_5 = \sqrt{r_3^2 + \frac{M_\sigma}{\rho_\sigma \pi L}}$}, where $\sigma = S$ if $Mode \in \{1,2\}$ and $\sigma = L$ if $Mode \in \{3,4\}$ 
%\begin{equation}
%    r_5 =  \begin{cases}
%    \sqrt{r_3^2 + \frac{M_S}{\rho_S \pi L}} & \text{if } Mode = \{1,2\}, \\
%    \sqrt{r_3^2 + \frac{M_L}{\rho_L \pi L}} & \text{if } Mode = \{3,4\}. \\
%    \end{cases}
%    \end{equation}

The following section shows how a FSM is used to turn on and off power flows in Fig. \ref{fig:Moving_Boundary_Modeling} to accurately model the dynamics of the TES device under four distinct modes of operation.

 % is defined as $SOC = \frac{M_S}{M_{tot}}$, where $M_S$ is the mass of the solid PCM. As such, $ SOC = 0 $ and $ SOC = 1 $ correspond to the PCM being completely liquid and completely solid, respectively and a larger $ SOC $ corresponds to a lower amount of energy stored in the PCM. 

% The graph-based model incorporates instantaneous changes in the operation modes that is dependent on if the TES is charging or discharging. The graph captures these changes through a FSM to implement turning on and off edges for each mode of operation, as shown in Fig. \ref{fig:Finite_State_Machine}.

\begin{figure}[t]
    \centering
    \includegraphics[width=\columnwidth] {Figures/Finite_State_Machine.pdf}
    \vspace{-30pt}
    \caption{FSM with switching criteria for the four modes of the MB model.}
    \label{fig:Finite_State_Machine}
\end{figure}
    
\subsection{Finite State Machine}
The TES device has four major modes of operation: completely liquid, completely solid, freezing, and melting modes, as shown in Fig. \ref{fig:Finite_State_Machine} with their respective mode numbers. Mode switching is based on the SOC and the surface temperature $T_{s_i}$ between the inner pipe wall and the PCM defined as  
%\begin{equation}
%    T_{s_i} = T_{inn.} + R_{3,A} P_3.
%\end{equation}
\begin{equation} \label{eq:Surface_Temperature}
    T_{s_i} =  \begin{cases}
    T_{inn.} + R_{3,A} P_3& \text{if } Mode \in \{2,3\}, \\
    T_{inn.} + R_{4,A} P_4& \text{if } Mode \in \{1,4\}.
    \end{cases}
    \end{equation}
    
Assuming the PCM starts in a completely liquid state (Mode~1), when the inlet working fluid temperature $ T_{in} < T_{sat} $ eventually $T_{s_i} < T_{sat}$ and the freezing process begins, switching the model into Mode 2 of the FSM. During the freezing process, the SOC will increase until the PCM is completely solid where $SOC = 1$ and the model switches to Mode 3. If the inlet working fluid temperature increases such that $ T_{in} > T_{sat} $, then eventually $T_{s_i} > T_{sat}$, and the melting process begins by switching to Mode 4. Once the PCM is complete liquid where $ SOC = 0 $, the model switches back into Mode 1. If the inlet working fluid temperature changes when the system is in Modes 2 or 4 before completely freezing or melting, the model can switch directly between Modes 2 and 4 with the PCM in a partially frozen state. During such transitions, note that the model makes a non-physical assumption that locations of the solid and liquid regions instantaneously switch such that solid is surrounded by liquid in Mode 2 and vice versa in Mode 4, as shown in Fig. \ref{fig:Finite_State_Machine}. While the SOC state still evolves continuously, the radii associated with the solid and liquid regions will change instantaneously.

While the graph in Fig. \ref{fig:Moving_Boundary_Modeling} shows all of the potential power flows through the PCM, power flows $ P_3 $ through $ P_8 $ are turned on and off based on the mode of operation as summarized in Table \ref{tab:Power_Flow_Modes}. For example, when the PCM is completely liquid (Mode 1), power flows $P_3$, $P_5$, $P_6$, and $P_7$ are all turned off to completely disconnect vertices $ v_3 $ and $ v_4 $ and allow both the inner and outer walls to exchange heat with only the liquid, vertex $ v_5 $. 
% The proposed graph model has edges connecting all states of the PCM ($v_3, v_4, \text{ and } v_5$) to both the inner and outer walls. This allows for the model to turn edges on and off depending on what mode of operation that the TES is configured to. Table \ref{tab:Power_Flow_Modes} shows which power flows are turned on and off for each mode of operation. 

\begin{table}[t]
    \scriptsize
    \caption{\uppercase{Power Flows for each FSM Mode}}
    \centering
    \label{tab:Power_Flow_Modes}
    \begin{tabular}{p{1.5cm}<{\centering} p{0.8cm}<{\centering} p{0.8cm}<{\centering} p{0.8cm}<{\centering} p{0.8cm}<{\centering}}
     \hline
     Power Flow & Mode 1 & Mode 2 & Mode 3 & Mode 4 \\ [0.5ex] 
     \hline
     $P_3$ & off & on & on & off \\ 
     $P_4$ & on & off & off & on \\ 
     $P_5$ & off & on & off & on \\ 
     $P_6$ & off & on & off & on \\ 
     $P_7$ & off & off & on & on \\ 
     $P_8$ & on & on & off & off \\ [1ex] 
     \hline
    \end{tabular}
    \end{table}
Finally, the radii labeled in Fig. \ref{fig:Moving_Boundary_Modeling} only correspond to Mode 2 of the FSM and are used in computing the thermal resistances $R_{j,A}$ and $R_{j,B}$. For the other three modes, the equations for these thermal resistances must be modified to reflect the geometry and corresponding radii for each mode.  

% Fig. \ref{fig:Moving_Boundary_Modeling} only shows the freezing case, but can represent all cases shown in Fig. \ref{fig:Finite_State_Machine} by updating the key radii. 
% Since each power flow$ P_j, \, j \in [3,8] $, is defined similarly to \eqref{eq:Power_Flow}, the total thermal resistance is defined as $ R_j = R_{j,A} + R_{j,B} $, where $ R_{3,B} = \frac{ln(\frac{r_4}{r_3})}{2 \pi L k_{j,\sigma}}$, $ R_{8,A} = \frac{ln(\frac{r_7}{r_6})}{2 \pi L k_{j,\sigma}}$, $ R_{7,B} = R_{8,B} = \frac{ln(\frac{r_{8}}{r_{7}})}{2 \pi L k_{out.}}$, and mode dependent resistances,
% \begin{equation}
%     \begin{gathered}
%     R_{4,B} = \frac{ln(\frac{r_6}{r_3})}{2 \pi L k_{j,\sigma}}, R_{7,A} = R_{8,A} \quad \text{if } Mode = 1, \\
%     R_{4,B} = R_{3,B}, R_{7,A} = \frac{ln(\frac{r_7}{r_4})}{2 \pi L k_{j,\sigma}} \quad \text{if } Mode = 3, \\
%     \end{gathered}
%     \end{equation}

% \begin{equation}
%     \begin{gathered}
%     R_5 = \frac{ln(\frac{r_6}{r_5})}{2 \pi L k_{j,\sigma}}, R_6 = \frac{ln(\frac{r_5}{r_4})}{2 \pi L k_{j,\sigma}} \quad \text{if } Mode = 2, \\
%     R_5 = \frac{ln(\frac{r_5}{r_4})}{2 \pi L k_{j,\sigma}}, R_6 = \frac{ln(\frac{r_6}{r_5})}{2 \pi L k_{j,\sigma}} \quad \text{if } Mode = 4.
%     \end{gathered}
%     \end{equation}

% This is possible by switching the locations of $h_S$ and $h_L$ for $Mode \in \{1,4\}$, and keeping the radii the same as in Fig. \ref{fig:Moving_Boundary_Modeling} for $Mode \in \{2,3\}$.


%Fig. \ref{fig:Moving_Boundary_Modeling} only shows the freezing case, but can represent all cases shown in Fig. \ref{fig:Finite_State_Machine} by updating the key radii. 
%The proposed graph model has edges connecting all states of the PCM ($v_3, v_4, \text{ and } v_5$) to both the inner and outer walls. This allows for the model to turn edges on and off depending on what mode of operation that the TES is configured to. Table \ref{tab:Power_Flow_Modes} shows which power flows are turned on and off for each mode of operation. 

% \hl{This is the end of what I have written to change.}

% Mode switching is based on the SOC and the surface temperature $T_{s_i}$ between the inner pipe wall and the PCM defined as
% \begin{equation}
%     T_{s_i} = T_{inner} + R_{A,3} P_3.
% \end{equation}

% Assuming the PCM starts in a completely liquid state (Mode 1), when the inlet working fluid temperature $ T_{in} < T_{sat} $ eventually $T_{s_i} < T_{sat}$ and the freezing process begins, switching the model into Mode 2 of the FSM. Duting the freezing process, the SOC will increase until the PCM is completely solid where $SOC = 1$ and the model switches to Mode 3. If the inlet working fluid temperature increases such that $ T_{in} > T_{sat} $, then eventually $T_{s_i} > T_{sat}$, and the melting process begins by switching to Mode 4. Once the PCM is complete liquid where $ SOC = 0 $, the model switches back into Mode 1. If the inlet working fluid temperature changes when the system is in Modes 2 or 4 before completely freezing or melting, the model can switch directly between Modes 2 and 4 with the PCM in a partially frozen state. During such transitions, note that the model makes a non-physical assumption that locations of the solid and liquid regions instantaneously switch such that solid is surrounded by liquid in Mode 2 and vice versa in Mode 4, as shown in Fig. \ref{fig:Finite_State_Machine}. While the SOC state still evolves continuously, the radii associated with the solid and liquid regions will change instantaneously.
% Mode 3 begins and will stay in this mode until $T_{s_i} > T_{sat}$. Mode 4 is characteristic of the melting process, and will continue until $SOC = 0$, in which will trigger the change back to Mode 1. Connections between Modes 2 and 4 incorporates the idea of partial freezing and thawing, which is not the main objective of this work.

%Mode switching based on $T_{s_i}$ is due to the melting point of the PCM, which for water is $0 \degree C$. This factor analyzes what is capable in the current configuration, meaning if $T_{s_i}$ is less than the melting point of the PCM, then freezing can occur, and vice versa for melting. SOC is another switching parameter which will finalize the operating mode. When the SOC equals 1 or 0, the mode is completely solid or liquid, respectively. The case where the SOC is not at one of the extremes will allow for freezing or melting, depending on $T_{s_i}$. These coupled together configure modes of operation for the FSM.

% These approaches can be used in different shaped frameworks, as the modeling shown is cylindrical, Fig. \ref{fig:Moving_Boundary_Modeling} shows how the framework can also apply to the 1-Dimensional case. This highlights the length dimension across the heat transfer direction for the TES.
    \begin{figure*}[t]
            \centering
            \includegraphics[width=\textwidth]{Figures/Full_Freeze_Melt_Plot.pdf}
            \caption{Differences between FG (with $n = 35$) and MB models for two complete freezing and melting cycles.}
            \label{fig:Freeze_Melt_Plot.pdf}
            \end{figure*}    

\section{Results} \label{Results}
\subsection{Simulation Setup}
Table \ref{tab:SimParams} shows the simulation parameters used to compare the proposed MB model with the more traditional FG model. The material properties for the simulated TES device assume that the working fluid is a 50/50 water-glycol mixture ($wg$), the inside pipe is copper ($Cu$), the PCM is water, and the outer pipe ($PVC$) is assumed to be well insulated.% with a large resistance.

\begin{table}[t]
    \scriptsize
    \caption{\uppercase{Simulation Parameters}}
    \centering
    \label{tab:SimParams}
    \begin{tabular}{p{0.9cm}<{\centering} p{3.9cm}<{\centering} p{1cm}<{\centering} p{1.1cm}<{\centering}}
     \hline
     Variable & Description & Value & Units \\ [0.5ex] 
     \hline
     $T_{in}$ & Working Fluid Inlet Temperature  & \{-18,18\} & $\degree C$ \\ 
     $T_{air}$ & Air Temperature & 18 & $\degree C$ \\
     $T_{sat}$ & Saturation Temperature & 0 & $\degree C$ \\
     $\dot{m}_{wf}$ & Mass Flow Rate & 0.10 & $kg/s$ \\
     $C_{p,wg}$ & Specific Heat ($wg$) & 3.4 & $kJ/(kg \degree C)$ \\
     $h_{wg}$ & Convective Heat Transfer Coeff. ($wg$) & $10^{4}$ & $ W/( m^2\degree C) $ \\
     $\rho_{wg}$ & Density ($wg$) & 1090 & $kg/m^3$ \\
     $C_{p,inn.}$ & Specific Heat ($Cu$) & 0.39 & $kJ/(kg \degree C)$ \\
     $k_{inn.}$ & Thermal Conductivity ($Cu$) & 401 & $ W/( m \degree C) $ \\
     $\rho_{inn.}$ & Density ($Cu$) & 8960 & $kg/m^3$ \\
     $h_f$ & Heat of Fusion & 334 & $kJ/kg$ \\
     $C_{p,S}, C_{p,L}$ & Specific Heat (Solid, Liquid) & 2.11, 4.18 & $kJ/(kg \degree C)$ \\
     $\rho_S, \rho_L$ & Density (Solid, Liquid) & 916, 1000 & $kg/m^3$ \\
     $k_S, k_L$ & Thermal Conductivity (Solid, Liquid) & 2.3, 0.58 & $W/(m \degree C)$ \\
     $h_{air}$ & Convective Heat Transfer Coeff. ($air$) & $5$ & $ W/( m^2\degree C) $ \\
     $L$ & Length of pipe & 1.00 & $m$ \\
     $R_{ins.}$ & Insulation Resistance & $10^{14}$ & $ \degree C / W $ \\
     $C_{p,out.}$ & Specific Heat ($PVC$) & 0.88 & $kJ/(kg \degree C)$ \\
     $k_{out.}$ & Thermal Conductivity ($PVC$) & 0.20 & $ W/( m \degree C) $ \\
     $\rho_{out.}$ & Density ($PVC$) & 1350 & $kg/m^3$ \\
     $r_1$ & Inner Wall Radius & $6.0$ & $ mm $ \\
     $r_5$ & Outer Wall Radius & $28.8$ & $ mm $ \\
     $\Delta r_{inn.}$ & Inner Wall Thickness & $0.8$ & $ mm $ \\
     $\Delta r_{out.}$ & Outer Wall Thickness & $6.4$ & $ mm $ \\
     $\Delta r_{PCM}$ & PCM Thickness & $19.1$ & $ mm $ \\
     $\Delta r$ & Difference in PCM radii & $0.55$ & $ mm $ \\
     $M_{tot}$ & Total mass of PCM & 1.90 & $kg$ \\ [1ex]
     \hline
    \end{tabular}
    \end{table}
% To compute SOC for the FG model, the summation of mass of the solid regions is taken as an estimate,
While $ SOC $ is a state of the MB model, for the FG model, $SOC$ is computed as
\begin{equation}
    SOC_{FG} = \frac{1}{M_{tot}}\sum_{j=3}^{n+2} C_j \left(1-\frac{max(0,min(h_j,h_f))}{h_f}\right).
    % \quad \text{if} \quad h_j \leq h_f.
\end{equation}

To compare the SOC for the FG and MB approach, the absolute difference is computed as
\begin{equation}
    \Delta_{SOC} = |SOC_{MB} - SOC_{FG}|.
\end{equation}


\subsection{Complete Freezing and Melting}
% The main cycle of operation incorporates all four modes of operation from the FSM. The cycle will consist of the freezing process until the TES is completely solid, then it will move into melting mode back to its start point of being all liquid water. This will allow for a direct comparison between the FG and the proposed MB solutions. 

Fig. \ref{fig:Freeze_Melt_Plot.pdf} shows the simulation results for the FG and MB models for two complete freezing and melting cycles. The top left plot shows trajectories for the simulated SOC using the FG and MB models, denoted $ SOC_{FG} $ and $ SOC_{MB} $, while the top right plot shows $\Delta_{SOC}$. Notably, the maximum and average values of $ \Delta_{SOC} $ are $ 0.05 $ and $ 0.03 $, with the total computation of $ 1.2 $ seconds for the MB model and $ 6.1 $ seconds for the FG model. The lower left plot in Fig. \ref{fig:Freeze_Melt_Plot.pdf} shows the mode switching for the MB model and the lower right plot shows the solid and liquid region enthalpies for the MB model and all $ n = 35 $ enthalpies of for the PCM of the FG model. While simulating roughly five times faster than the FG model, the MB model is remarkably accurate when simulating complete freezing and melting cycles.
% Two cycles of the process are conducted, where the $T_{in}$ temperature is held constant for freezing and then is flipped to $18 \degree C$ for melting mode. The results of the SOC between both models are shown in Fig. \ref{fig:Freeze_Melt_Plot.pdf}. Fig. \ref{fig:Freeze_Melt_Plot.pdf} shows the absolute difference between both models at each time step, as shown by, 
% \begin{equation}
%     \Delta_{SOC} = SOC_{MB} - SOC_{FG},
%     \end{equation}

% The accuracy of the proposed solution is within $0.06$ of the FG solution, with accuracy within a value of $0.02$ after the completion of the first freezing process. The proposed solution modes of operation in the simulation are shown in Fig. \ref{fig:Freeze_Melt_Plot.pdf}. The enthalpy for each state of the MB and FG models are shown in the final plot.

\begin{figure*}[t]
        \centering
        \includegraphics[width=\textwidth]{Figures/Partial_Freeze_Melt_Plot.pdf}
        \caption{Differences between FG (with $n = 35$) and MB models for six partial freezing and melting cycles.}
        \label{fig:Partial_Freeze_Melt_Plot.pdf}
        \end{figure*}



\subsection{Partial Freezing and Melting}
% Partial freezing and melting for the model accounts for the case that the operation mode switches while the PCM in the TES is two-phase. The following simulation incorporates a switching criteria that has the temperature of the water glycol inside of the inner wall fluctuate between the negative and positive inlet temperatures while the PCM is in the two-phase region.

Fig. \ref{fig:Partial_Freeze_Melt_Plot.pdf} shows how the MB approach loses accuracy when simulating partial freezing and melting of the PCM. When switching between Modes 2 and 4, the maximum value of $ \Delta_{SOC} $ increases significantly up to $ 0.25 $. This is due to the fact that partial freezing can create complex geometries with multiple regions of solid and liquid.  This complex geometry, which results in additional heat transfer surface area between the solid and liquid regions, cannot be captured by the proposed MB model. This is why the $ SOC $ decreases significantly faster using the FG model during the first period of operation in Mode 4.  Note that the accuracy of the MB increases on average when the PCM freezes completely in the latter half of the simulation. In summary, the proposed MB modeling approach is only recommended when complete freezing and melting of the PCM is expected and future work will focus on modifying the MB formulation to more accurately capture behavior associated with partial freezing and melting.
% we start to incorporate the case where partial freezing and melting can occur in the TES. Fig. \ref{fig:Partial_Freeze_Melt_Plot.pdf} shows the absolute difference between both models, $\Delta_{SOC}$. The FSM, in Fig. \ref{fig:Partial_Freeze_Melt_Plot.pdf}, shows how the model can account for these quick mode switches and still operate as desired. The enthalpy for each state of the MB and FG models are shown in the final plot.

        
        

\section{Conclusions}
This paper presented a switched moving boundary approach as a control-oriented method for modeling thermal energy storage devices with phase change material. Graph-based modeling was used to identify the structure of the dynamics when using a fixed grid and the proposed moving bounding modeling approaches. A finite state machine allowed the moving boundary model to switch modes to capture the dynamics associated with freezing, melting, completely solid, and completely liquid Phase change material. A numerical example demonstrated the accuracy and computational efficiency of the switched moving boundary model. 

% addresses the capabilities of a switched MB approach for modeling PCM-based TES devices. Utilizing a FSM allows for freezing and melting to occur both inward and outwards radially. This can incorporate a multitude of applications to the model, including cooling hot air flowing on the outside. The model is not applicable for partial freezing and melting sequences, as this leads to higher inaccuracies in the model.

Future work will focus on experimental validation, capturing the effects of natural convection heat transfer in the liquid regions, and accurately modeling partial freezing and melting operation. Additionally, the switched moving boundary approach will also be extended to model thermal energy storage devices in three dimensions where heat transfer is no longer uniform along the length of the device.
% the model to work in three-dimensions will be incorporated as the current model can match radially, but freezing along the length of the pipe can look cone-shaped, depending on the flow rate of the working fluid. After completion of the model, model-based control strategies to maximize the efficiency of the TES.



\bibliography{./main_formatted}



\end{document}
