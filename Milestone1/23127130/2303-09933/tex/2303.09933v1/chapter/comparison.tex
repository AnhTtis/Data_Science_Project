\section{Evaluation}
\label{sec:comparison}

\ac{MPEG} is currently working on its documents using the workflow described in \autoref{sec:standards-workflow}.
We propose that \ac{MPEG} and similar bodies adopt the process proposed in \autoref{sec:process}.
This section compares the existing \ac{ISO} collaboration process of \autoref{sec:standards-workflow} based on the Microsoft Word desktop application, with the proposed collaboration workflow from \autoref{sec:process}.
Additionally, we compare both approaches with \wftw as the state-of-the-art approach for collaborative document creation.
In the following, we use "Word" to refer to both local Microsoft Word as well as \wftw.
For referring to one of the solutions, we explicitly name it.
We assess the aspects of accessibility, efficiency, traceability, interoperability, rendering, and error probability as well as the human factor.
An overview of the comparison is given in \autoref{tab:tldr}.

\begin{table*}[ht!]
    \centering
    \resizebox {\textwidth} {!} {
        \begin{tabular}{ L{0.15\textwidth} | C{0.3\textwidth} | C{0.3\textwidth} | C{0.3\textwidth}}
            \toprule
                                   & \textbf{Microsoft Word}                                                              & \textbf{\wftw}                                                                                 & \textbf{\magit}                                                                                 \\
            \midrule\midrule
            Primary output format  & Paper                                                                                & Paper                                                                                          & Web                                                                                             \\
            \midrule
            Editing mode           & Responsive \ac{WYSIWYG}                                                              & Responsive \ac{WYSIWYG}                                                                        & Syntax highlighting, live preview, any editor                                                   \\
            \midrule
            Sharing                & \red{Monolithic, local file}                                                         & \orange{Monolithic, shared file, available for all team members}                               & \green{Multiple files, \acp{URL} can link to anchors of documents}                              \\
            \midrule
            Scalable collaboration & \red{Limited, interfering reviews, single simultaneous edit}                         & \orange{Same reviews as Word, synchronous simultaneous edits, sufficient for small teams only} & \green{Proven, scalable review tools, (a-)synchronous simultaneous edits, separate discussions} \\
            \midrule
            Traceability           & \red{limited file history, multiple sources of truth}                                & \orange{file history, only automatic time-based versioning}                                    & \green{all advantages of \git, full traceability}                                               \\
            \midrule
            Interoperability       & \red{lock-in to proprietary files and software}                                      & \red{lock-in to proprietary files and software}                                                & \green{open source, extensible software, openness to other tools}                               \\
            \midrule
            Styling                & \red{Plethora of styling options leads to accidental issues/misuse}                  & \red{Similar to Word, but reduced set of options}                                              & \green{Few accidents through constrained, explicit styling. Rendering pipeline controls layout} \\
            \midrule
            Errors                 & \green{Powerful spelling and grammar checker}                                        & \green{Powerful spelling and grammar checker}                                                  & \green{Proven review process, option of additional verification tools}                          \\
            \midrule
            Setup                  & \green{Single tool}                                                                  & \green{Single tool}                                                                            & \red{Huge variety of tools might overwhelm novices}                                             \\
            \midrule
            Learning               & \green{Widely used among amateurs and professionals, GUI facilitates disoverability} & \green{Similar to Word}                                                                        & \orange{Known to professionals, but requires training for novices}                              \\
            \bottomrule
        \end{tabular}
    }
    \caption{Relevant features of using Microsoft Word, \wftw, and the proposed \magit process.}
    \label{tab:tldr}
\end{table*}

\subsection{Accessibility}

The way a document is displayed and shared defines its accessibility.
We discuss both accessibility aspects in this section.

\subsubsection{Document Display}
\label{sssec:document-consumption}

Over the past decades, the display of documents has changed fundamentally.
Several decades ago, the majority of documents still consisted of paper.
Since the invention of computers, digital documents have gained increasing importance for document exchange and processing.
Besides editing, sharing and processing digital documents on desktop computers, documents are nowadays also often used on mobile devices.
This change of medium also changes the requirements for displaying and processing the document.
When paper documents were prevalent, a document processing tool chain had to focus on creating documents laid out for printing to actual paper.
For many modern documents, paper is not the main medium anymore.
Instead, documents are viewed on screens of varying size and aspect ratio.
Consequently, modern document tool chains need to focus on these commonly web-based, diverse consumption scenarios.
A prominent example is Martin Flower's \textit{Refactoring: Improving the Design of Existing Code}~\cite{flower1999refactoring}, which is designed as a \textit{web-first} book.
Here, the online version of the book is the primary version, which contains more content than the physical book and the content is maintained over time.

Word is conventional in this respect.
Word documents mimic documents printed to physical paper, with a white background and defined document and font sizes.
These constraints limit accessibility of Word documents on web-based and mobile devices.

Markup languages, on the other hand, provide great support for modern, web-based document types.
Additionally, many formats such as AsciiDoc support publishing to classical paper as well as to digital formats.
Finally, plain text based files are future proof thanks to the simple file format.
The file itself can be used as an example of the format definition.
A prominent example for this is David Thomas' and Andrew Hunt's book \textit{The Pragmatic Programmer: your journey to mastery}~\cite{thomas2019pragmatic}, which is written in plain text and even recommends the benefits of using such.

\subsubsection{Document Sharing}

In an interconnected society in which shared resources are available to everyone, sharing is most efficiently done by not exchanging a file itself, but a pointer to the file.
In the dominant web-based environment, these file pointers are \acp{URL}.
Accessing an \ac{URL} in a web browser opens the desired document.
Word files can be shared and accessed this way, as can markup files.

Markup files have three significant advantages, however.
First, large documents can be split into multiple files, see \autoref{ssec:scalability}.
This allows straightforward, highly specific sharing.
If a user would like to share the entire document, this is simply done by sharing the \ac{URL} of the top-level file collecting all sub-files.
If sharing a specific section only, the user shares the \ac{URL} of that file only.

Second, markup languages such as HTML, Markdown, and AsciiDoc take that concept even further, allowing to create \acp{URL} to sections within one document.
This enables efficient sharing, as collaborators and readers can point to highly specific positions in their document with little effort.
Furthermore, the \acp{URL} are typically human readable which aids in communication.
To give one example, section \acp{URL} are widely used when sharing a specific style guide section of the extensive Google \Cpp Style Guide such as the style guide on static and global variables\footnote{https://google.github.io/styleguide/cppguide.html\#Static\_and\_Global\_Variables, Last accessed 09/26/2022}.
In markup source code, it is even possible to link to any line.

Third, direct accessibility is the core of markup languages.
A document consumer opens an \ac{URL}, and the web browser directly renders the document.
Word files take a different approach.
They require a dedicated application to render a Word file, requiring to use a \wftw instance or to download a document and open it in the dedicated application.
This contradicts modern internet usage, however, and often requires more time than a quick document lookup itself, disqualifying it in many applications.

\subsection{Efficiency \& Scalability}
\label{ssec:scalability}

This section focuses on the specifics of large documents and edits by large groups of collaborators.

\subsubsection{Text Editing}

First, a document processing system needs to provide an efficient text editing process.
WYSIWYG applications such as Word enable editing of the rendered output view.
This immediate feedback loop facilitates quick visual adjustments, but tends to become slow for large documents.

With plain text, one edits the source code of the document.
Widespread languages and text editors offer syntax highlighting for easy orientation inside the documents.
Given that plain text is simple to render, even long documents are rendered without delay.
Furthermore, many of the plain text editors allow advanced edits such as search and replace all, search with regular expressions, multiple cursors in the document, and more.
To merge the best from both worlds, many plain text formats offer live previews\footnote{https://docs.asciidoctor.org/asciidoctor/latest/tooling/\#visual-studio-code, Last accessed 09/26/2022}, updating the output e.g. each second given the set of source files.

\subsubsection{Collaboration}
\label{sssec:collaboration}

In the proposed process, a large document is structured into multiple files, which allows collaborators to effortlessly edit distinct files in parallel, without causing merge conflicts.
With Word's single \docx file, only one person at a time can edit the document.
With growing team sizes, this would quickly bring the editing process to a halt.
For comparison, imagine that only one of Google's engineers may work on its codebase at a time.
With \wftw, multiple collaborators can edit the same document simultaneously with live updates.

However, \wftw is still restricted to a single monolithic file.
A document split into multiple files simplifies navigation and allows to open only a subset of the document.
For large documents, both options become significant advantages compared to being forced to operate within one huge file.

Finally, the collaborative review process is facilitated by \gh.
Isolated discussions on different changesets (\acp{PR}) in the proposed process allow documents and collaborator teams to scale efficiently.
For example, at the time of this writing, there are 126 open \acp{PR} on the \Cpp standard~\cite{cpp_draft}, which is written in \LaTeX\xspace and hosted on \gh.
Each \ac{PR} has a distinct topic and discussion, without unnecessary interference between the discussions.
In contrast, there is no way of organizing all discussions taking place in the same Word document.
In Word, comments and suggestions provide a way of proposing changes without immediately changing the document.
While a comment always refers to a single place in the document, multiple comments across the document cannot be grouped into a single changeset, equivalent to a \ac{PR}.
This quickly leads to an unmanageable chaos as the number of simultaneous changes grows.

\subsection{Traceability}

A detailed, meaningful document history can be as important as the document itself.
Clear traceability of the document development timeline is important for collaborators~\cite{alwis_why_2009} and in a legal context~\cite{herkert2020boeing}.
It helps answering questions such as reasons for changes and responsibility.

First, we discuss versioning which is central to \acp{VCS} such as \git.
Significant effort has gone into making commits fast to create and apply alongside informative commit messages.
\git is the versioning system that dominates the software industry, hence it is expected to satisfy the vast majority of use cases.
The Word desktop application, on the other hand, just offers manual file saving.
In \wftw, the document is auto-saved regularly, without users being able to trigger saving.
Both options severely limit traceability compared to professional \acp{VCS}.

Second, the history of changeset discussions needs to be accessible in a persistent, structured way, which is the case with \gh \acp{PR}.
PRs are permanent recordings of an isolated discussion on a set of changes.
PRs can be labeled and searched for.
All requirements for discussion traceability are fulfilled.
Word is insufficient in this respect.
One can only search for discussions in the pool of all resolved and unresolved comments, while deleted comments are not visible anymore.
There is no option to search within the pool of comments.

\subsection{Interoperability \& Flexibility}
\label{ssec:interop}

Word locks projects into its ecosystem.
Transitioning to another program and process for collaborative document editing is difficult because of Word's proprietary, compressed file formats.
For many platforms such as Chrome \ac{OS} and Linux based devices, there is no native Word application.
Thus, users of these platforms have to resort to the web application, which has limited functionality as discussed in \autoref{ssec:word-online}.
With Word, a project depends on a commercial company's closed source software.
Users have heavily limited options for adapting the proprietary software to their needs.
Finally, users need to continuously pay software license fees.
All the above limitations do not exist for the \magit approach.
Plain text files can be easily parsed and modified with automated tools and can be edited with any editor. Given that the entire toolchain is open source, even significant case adaptions are feasible if sufficient resources can be invested.

Finally, plain text files parsed by automated tools enable new use cases.
Standard documents that contain syntax excerpts can have these syntax elements parsed by code generation.
The resulting code can subsequently be passed to compilers, which automatically uncover correctness or consistency issues.
This can be done continuously in automated tasks that are triggered in the \gh or \gl repository using \ac{CI} tools.

As a more practical example, Hofbauer~et~al.~\cite{hofbauer_software_lab} used this approach for creating a lecture in which the lecture slides are written in Markdown.
The lecture slides contain descriptions of the homework tasks for the students.
Using comment-based delimiters not visible on the lecture slides, the authors separate a task from the remaining slide content.
This allowed them to build a simple toolchain that can create issues for all student groups with the task title and description from the slides with a single command.
Such an approach scales well and enables a single definition of each homework task.
This would not be possible with Microsoft PowerPoint-based lecture slides.
These are just two examples for what is possible when document content is easily accessible from other tools.
If plain text formats become the norm, we expect many more such processes to be invented.

\subsection{Output Styling}

In the \ac{ISO} standardization process, collaborators frequently change style by accident, e.g. to a slightly different font or font size.
This happens because style is implicit in Word and because an abundance of styling options is offered.
Such changes leads to inconsistent documents, requiring manual fixing effort.
Usually a single \ac{ISO} contributor spends several hours or even days in creating a consistent layout when a submission is due.

In markup toolchains, styling options are constrained and a deviation from the consistent style is explicitly instructed.
These restrictions lead to fewer accidents and hence higher quality documents and less manual effort.
Style selection such as the rendering font is a question for the rendering tool, not for the author while writing the text.
Markup also allows for consistent integration into a particularly styled environment without the document imposing hard requirements on e.g. font, font size, and page layout.

\subsection{Errors}

Word supports authors with sophisticated spell and grammar checkers, preventing many language issues.
However, technical and higher level errors still can only be prevented through human review.
While Word provides a side by side comparison of changes and inline suggestions, this lacks behind the \git review process, as presented in \autoref{sssec:collaboration}.
Further, the plain text files of markup languages can be automatically validated for technical errors.
Such checks can be executed locally during the creation process and in \ac{CI} as an additional check before the changeset is integrated.
As mentioned in \autoref{ssec:interop}, the sky is the limit for what people invent.

\subsection{Generalizability}

The benefits outlined in this section, summarized in \autoref{tab:tldr}, are not restricted to the \ac{ISO} standardization process, but applicable to all processes in which a group of contributors has to work on a common document.

While the benefits of \git and \gh are not new, many processes such as \ac{ISO} standardization are still done working on a single Word document and can be optimized.
The proposed approach offers a unified, flexible, and modern way for creating such documents relying on established and well know workflows of the software development domain.

\subsection{Human Factor}

To maximize the benefits of the proposed \magit workflow, users require more education to handle the higher technical complexity compared to the basic Word workflow.
Since we often referred to the \ac{ISO} standardization process as an example, we can expect contributors with a solid technical background where most of them already use tools related to \magit in their daily workflow.
For this user group, using the same process for document creation should be seamless.
Even novice users with a less technical background can learn these processes.
The available platforms such as \gh or \gl already provide web IDEs to make the first steps for novice users as easy as possible.
