\section{Conclusion}
\label{sec:conclusion}

In this paper, we presented a flexible and scalable approach for collaborative creation of documentation or specifications in large teams.
While the majority of specifications, such as for \ac{ISO} standardization, are still created in a single, large Word document, we propose to use well-established processes from software development together with a lightweight markup language.
We performed a thorough comparison to the most widespread workflow using Microsoft Word and discussed the benefits and drawbacks of the proposed approach.
The proposed approach comes with several benefits such as plain text based document sources that allow for higher levels of automated validation processes.
Another major benefit is the collaboration and review support provided by the well-established tool \git and collaboration platforms such as \gh or \gl, which are specifically designed for collaborative creation.
The higher technical complexity of the proposed \magit approach requires users with more education compared to the basic Word workflow.

We provide a template repository publicly available on \gh that can be used as a one-click solution to set up a collaborative document creation process such as required for the \ac{ISO} standardization.

For future work, we plan to further assess the proposed approach using objective metrics.
Conducting an empirical study of a large-scale projects such as the \Cpp \ac{ISO} draft~\cite{cpp_draft} can yield valuable quantitative insights into the benefits of the proposed \magit approach for efficient collaborative writing.
