\section{Proposed Process}
\label{sec:process}

In this section, we present the proposed tooling and workflow for efficient collaborative writing.
The proposed documentation workflow is similar to workflows existing for collaboration on source code documents and incorporates many processes described in \autoref{sec:github}.
Each collaborator has a copy of the repository with all relevant documents.
When a collaborator wants to edit the document, a new branch for the changes is created.
Other collaborators can simultaneously work on the documents, as long as they have agreed to work on different sections/topics, for example through issues.
The entire process is summarized in \autoref{fig:workflow_proposed}.

\begin{figure}[ht!]
    \centering
    \resizebox {.8\columnwidth} {!}{\scalefont{0.9}
\begin{tikzpicture}

    \tikzset{block/.style= {draw, rectangle, align=center,minimum width=1cm,minimum height=0.5cm}}

    \node[](anchor){};

    \node [block, text width = .75\columnwidth,fill=gray,fill opacity = 0.1, text opacity = 1, below = 0cm of anchor]  (get_repository) {
            \textbf{1. Get repository}
            \vspace{-.8em}
            \begin{itemize}
                \setlength\itemsep{-.4em}
                \item One repository with all documents
                \item Based on established git structures
            \end{itemize}};

    \node [block, text width = .75\columnwidth,fill=gray,fill opacity = 0.1, text opacity = 1, below = 0.5cm of get_repository]  (implement_changes) {
            \textbf{2. Implement changes}
            \vspace{-.8em}
            \begin{itemize}
                \setlength\itemsep{-.4em}
                \item Create new branch
                \item Perform changes and commit
            \end{itemize}};

    \node [block, text width = .75\columnwidth,fill=gray,fill opacity = 0.1, text opacity = 1, below = 0.5cm of implement_changes]  (finalize) {
            \textbf{3. Perform update}
            \vspace{-.8em}
            \begin{itemize}
                \setlength\itemsep{-.4em}
                \item Create pull request
                \item Merge after reviews of collaborators
            \end{itemize}};

    \path[draw, -{Latex[length=2.5mm,width=1.5mm]}]
    (get_repository.south) edge (implement_changes.north)
    (implement_changes.south) to (finalize.north);
\end{tikzpicture}
}
    \caption{Overview of the proposed workflow for efficient large-scale collaborative writing.}
    \label{fig:workflow_proposed}
\end{figure}

While editing, the collaborator creates commits when atomic work-packages are completed.
When the work is done, the author creates a \ac{PR} in the shared repository, other collaborators review, and finally merge the changes.

We select the widely used \git~\cite{chacon_pro_2014} for a prototype of the proposed system.
\git enables the usage of collaboration platforms such as \gh or \gl.
These platforms are rather similar, in particular in terms of collaboration features.
We selected \gh because it has the larger market share of about \SI{88}{\percent}~\cite{gh_market_share_2020}.
The results of \cite{longo_use_2015} and \cite{marquardson_learning_2019} show that the proposed process contains significantly fewer manual, error-prone steps compared to \autoref{fig:workflow_sota}.
Note that the proposed approach is very straightforward since it combines established techniques and well known workflows from software development.
We argue that this is an important quality of the proposed workflow.

To further simplify the usability of the proposed approach, we provide a \gh template repository which allows to use the proposed workflow as a single-click solution.
The template repository can also be used with other platforms such as \gl.
It contains an AsciiDoc template and \ac{CI} configurations running automated checks and validation processes.
This template is publicly available on \gh\footnote{https://github.com/plain-docs/asciidoc-starter}.
