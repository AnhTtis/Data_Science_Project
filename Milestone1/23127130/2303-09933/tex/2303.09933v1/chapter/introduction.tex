\section{Introduction}

Documentation of a project is vital for successfully sharing knowledge in both academia and industry~\cite{Halsey2019}.
In collaborative writing, multiple individuals need to access the same content, often at the same time, which makes documentation a complex task.
Currently, many individuals and organizations simply exchange files such as Microsoft Word documents for collaborative editing of documents.
A reason for this workflow can be that the first draft of the document was created as such a file due to simplicity, or that this process is the traditional way at a workplace.
As soon as more people are getting involved, the single document is then shared with them.
This practice is not only the case for small student teams working on a short-lived essay, but widespread throughout all levels of industry and academia for various document types.
To give a prominent example, the \ac{ISO} requires committees to submit standard drafts as Word files~\cite{iso_word_template}.
These committees, which can consist of dozens of experts, therefore need to work on shared Word documents with potentially hundreds of pages.
Microsoft Word has not been designed for distributed, large-scale collaboration on a set of files.
Consequently, the processes required to collaborate this way are inefficient, not scalable, and often lack structure.

In contrast, \acp{VCS} such as SVN, Mercurial, and \git~\cite{chacon_pro_2014} have been created for version control and collaboration on plain text source code in software projects.
Plain text markup languages such as Markdown and AsciiDoc are therefore suitable for \acp{VCS} such as \git and enable collaboration on proven platforms such as \gh or \gl.

In this paper, we propose a new workflow for collaborative writing that uses the benefits from systems used for software development.
We propose a combination of a plain text based markup language such as AsciiDoc, a \ac{VCS} such as \git, and a collaboration platform such as \gh or \gl.
For large scale collaborative writing, we argue that this approach is a superior alternative to Word or comparable office suites based on binary document types as well as shared online platforms such as \wftw.

In summary, we make the following contributions:

\begin{itemize}
    \item We discuss existing tools for collaborative writing.
    \item We propose a framework for efficient collaborative writing that makes use of the benefits of modern \acp{VCS}, markup languages, and collaboration platforms.
    \item We compare the proposed approach to Word and \wftw as representatives for established tools.
\end{itemize}

The rest of this paper is organized as follows.
In \autoref{sec:collaboration}, we summarize existing methods in the field of version control and collaborative writing.
Next, we introduce the proposed framework for efficiently managing a large-scale collaboration such as an \ac{ISO} standardization process in \autoref{sec:process}.
In \autoref{sec:comparison}, we present a detailed comparison of Microsoft Word and \wftw to the proposed approach.
\autoref{sec:conclusion} concludes the paper.
