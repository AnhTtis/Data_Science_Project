
\section{The Roulette formalism and its computation} 



\begin{figure}%[t]
   \begin{center}
\includegraphics[width=0.9\columnwidth]{Images/model_2Db.jpeg}
   \end{center}
    \caption{The figure shows the set-up for the flat-sky approximation, with the source plane (the lens plane) a distance $D_{\mathrm{S}}$ ($D_{\mathrm{L}}$) from the observer. Compare with Figure~\ref{fig1} for more details.}
 \label{fig:model_2D}
\end{figure}
   
\begin{figure*}%[t]
   \begin{center}
\includegraphics[width=0.9\textwidth]{Images/model_3D.jpeg}
   \end{center}
    \caption{The figure shows the set-up for the model used. In particular, the local coordinate systems used in the source plane and lens plane are shown. Compare with Figure~\ref{fig:model_2D}.}
    \label{fig1}
\end{figure*}


The Roulette formalism was introduced by 
\citet{clarkson16a}, and
\citet{Clarkson_2016_II} provide complete details. 
Our presentation below will differ a little from conventional
presentations in physics, in order to emphasise computable
functions which can be used to calculate the distorted image.
Readers who want a fuller understanding of the algebraic model
should consult Clarkson's original work.

We study two distant objects in the universe, namely the
(gravitational) lens $L$ at distance $D_\textrm{L}$ from Earth
and the (light) source $S$ at distance $D_\textrm{S}$.
Adopting the \textit{thin-lens approximation}, we assume that
the lens mass is concentrated in a plane orthogonal on the line
of sight through its centre.
The source image is considered only as the 2D projection (image) 
of its emitted light.
With astronomical distances and a relatively small angle of view,
we can assume planar projections; this is known as the \textit{flat-sky
approximation}.
We consider two different images of the source.
The source image is the ideal projection,
as it would have been observed absent any obstructions.
The distorted image is the image as it can be observed 
when light is deflected by the lens.
\begin{table*}
   \begin{minipage}{\textwidth}
\begin{align}
   \alpha_s^m&= - \frac1{2^{\delta_{0s}}} D_\textrm{L}^{m+1}
   \sum_{k=0}^m\binom{m}{k}\left({\mathcal{C}}_s^{m(k)}\partial_{\xi_1}
                                +{\mathcal{C}}_s^{m(k+1)}\partial_{\xi_2}\right)
                      \partial_{\xi_1}^{m-k}\partial_{\xi_2}^k\psi,
                      \label{Alpha}\\
   {\mathcal{C}}_s^{m(k)}&=\frac{1}{\pi}\int_{-\pi}^{\pi}{\rm d}\phi\sin^k\phi\cos^{m-k+1}\phi\cos s\phi\label{C},\\
   \beta_s^m&=-D_\textrm{L}^{m+1}\sum_{k=0}^m\binom{m}{k}\left({\mathcal{S}}_s^{m(k)}\partial_{\xi_1}+{\mathcal{S}}_s^{m(k+1)}\partial_{\xi_2}\right)\partial_{\xi_1}^{m-k}\partial_{\xi_2}^k\psi\label{Beta} \\
   {\mathcal{S}}_s^{m(k)}&=\frac{1}{\pi}\int_{-\pi}^{\pi}{\rm d}\phi\sin^k\phi\cos^{m-k+1}\phi\sin s\phi.\label{S}
\end{align}
   \end{minipage}
   \caption{Constitiuent definitions for the distortion function.}
   \label{tab:eq}
\end{table*}

The observed lensing is decomposed into two steps, as shown in Figure~\ref{fig1}.
The first step is a translation (deflection), corresponding to the difference 
$\boldsymbol{\Delta\eta}$ between actual ($\boldsymbol{\eta}_\textrm{act}$)
and apparent ($\boldsymbol{\eta}_\textrm{app}$) source-plane position. In the roulette formalism, this translational part of the lensing is given as
\begin{align}
\label{Deta}
   \boldsymbol{\Delta\eta}
   & =\boldsymbol{\eta}_\textrm{app}-\boldsymbol{\eta}_\textrm{act}
   =-D_\textrm{S}\cdot(\alpha^0_1,\beta^0_1),
\end{align}
where $(\alpha^0_1,\beta^0_1)$ is a vector of roulette amplitudes, as
defined in Table~\ref{tab:eq}.
The second step is the actual, non-linear distortion.
The distorted image is drawn in a local co-ordinate system in the lens
plane, centred at $\boldsymbol{\xi}=(\xi_1,\xi_2)$, which corresponds to 
$\boldsymbol{\eta}_\textrm{app}$ in the source plane.
We write $\xi=|\boldsymbol{\xi}|$ for the distance between the distorted
image and the lens in the lens plane.
Since $\boldsymbol{\xi}$ and $\boldsymbol{\eta}_{\mathrm{app}}$ lie on the same
line through the viewpoint (cf.\ Figure~\ref{fig:model_2D}), we have
\[ \xi = |\boldsymbol{\xi}| = \frac{D_\textrm{L}}{D_\textrm{S}}\cdot|\boldsymbol{\eta}_{\mathrm{app}}|.\]
Following Clarkson, we use polar co-ordinates $(r,\phi)$ for the
distorted image.
The source image is described in Cartesian co-ordinates $(x^\prime,y^\prime)$ centered
at $\boldsymbol{\eta}_\textrm{act}$ in the source plane.
Thus the light observed at a position (pixel) $(r,\phi)$ is drawn from
a different position (pixel) $(x',y')=\mathcal{D}$$(r,\phi)$ in the source image.
From~Eq.~48 in \citet{Clarkson_2016_II} it is possible to show that 
% \textbf{TODO} CHECK 
the mapping $\mathcal{D}$ is given as
\begin{align}
   % \begin{split}
      \frac{D_{\mathrm{L}}}{D_{\mathrm{S}}}\cdot
   \begin{bmatrix} x' \\ y' \end{bmatrix} &=
   r\cdot\begin{bmatrix} \cos\phi \\ \sin\phi \end{bmatrix} 
    \label{eqn:general mapping}
      + \sum_{m=1}^{\infty} \frac{r^m}{m!\cdot D_{\mathrm{L}}^{m-1}}
      F_s^m 
\end{align}
          {where}
\begin{align}
      F_s^m &=
      \sum_{s=0}^{m+1} c_{m+s}
       \left(\alpha_s^m \boldsymbol{A}_{s} + \beta_s^m \boldsymbol{B}_{s} \right) 
       \begin{bmatrix} C^+ \\ C^- \end{bmatrix}
          \\
   % \end{split}
   C^\pm &= \pm \frac{s}{m+1},\\
   c_{m+s} &= 
      \frac{1 - (-1)^{m+s}}{4} =
   \begin{cases}
      0, \quad m+s \text{ is even},\\
      \frac12, \quad m+s \text{ is odd},
   \end{cases}\\
    \boldsymbol{A}_{s} &= \begin{bmatrix}
    \cos{(s-1)\phi} & \cos{(s+1)\phi} \\ 
    -\sin{(s-1)\phi} &  \sin{(s+1)\phi} \end{bmatrix},
    \\
    \boldsymbol{B}_{s} &=
    \begin{bmatrix} 
        \sin{(s-1)\phi} & \sin{(s+1)\phi} \\
        \cos{(s-1)\phi} & -\cos{(s+1)\phi} 
    \end{bmatrix}.
\end{align}
The coefficients $\alpha_m^s$ and $\beta_m^s$ depend on the lens potential
$\psi(\xi_1,\xi_2)$, from which one may derive the physical properties of the lens.
The general formulae are shown in Table~\ref{tab:eq}.
In practice the sum in \eqref{eqn:general mapping} has to be
truncated by limiting $m\le m_0$ for some $m_0$.

A general implementation for arbitrary $\psi$ would be intractible,
but for many common lens models, it is possible to derive computationally
tractible forms.
The two simplest, but yet very popular, lens models are the 
point mass and singular isothermal sphere (SIS). Confer e.g. with~\cite{bok:schneider92_SEF}, Sections 8.1.2 and 8.1.4 for more on the point-mass and SIS profiles, respectively.
For the point mass, an exact model exists, and we have implemented
both this, and its Roulette approximation.
For SIS, there is no exact model, and we have implemented it in 
the Roulette formalism.

\begin{comment}
In the above, $\partial_{\xi_i}\equiv\frac{\partial}{\partial_{\xi_i}}$. The function $\psi$ is called the lens potential, and contains information about the lens. In particular, the dimensionless surface-mass density may be calculated therof. But also the deflection angle, and higher-order distortions are all derived from the information contained in $\psi$.

Implementing this model in its full generality is intractable at present, and we concentrate here on a couple of lens profiles which allow for extensive simplification. The first of these, is the point-mass lens, which is chiefly interesting because of its analytically tractable nature. As such, it is a useful toy model that provides important grounds for debugging and proof of concept. Its potential is given by
\begin{equation}
   \psi(\xi)=\psi_0\ln(\xi), CHECK \textbf{TODO}
\end{equation}
The second profile is the singular isothermal sphere profile, which is a more realistic model for a galaxy lens. For instance, it is capable of reproducing flat rotation curves, which seems to be a signature of spiral galaxies. Its potential is given by
\begin{align}
   \psi(\xi) &= \frac{R_\textrm{E} \xi}{D_\textrm{L}^2},
   \label{eqn:psi_gamma}
\end{align}
where $R_\textrm{E}$ is the so-called Einstein radius and $\xi=\sqrt{\xi_1^2+\xi_2^2}$ is the distance from the centre of mass of the lens.
\end{comment}


% \section{Lens Models and Gritty Details}
% \label{section:formulas}

\subsection{Point-mass lens}

Without loss of generality, one may assume that the centre of mass of the source is located on the positive $x$-axis.
Using the general equations of \citet{Clarkson_2016_II}, it is
straight forward to find the following formula for point-mass lenses
as a special case:
\begin{align}
   \begin{split}
      \frac{D_\textrm{L}}{D_\textrm{S}} \begin{bmatrix} x' \\ y' \end{bmatrix} &=
       r \begin{bmatrix} \cos{\phi} \\ \sin{\phi} \end{bmatrix} 
          \\&
     - \frac{R_\textrm{E}^2}{\xi} \sum_{m=1}^\infty(-1)^m
     \left(\frac{r}{\xi}\right)^m 
     \begin{bmatrix} \cos(m\phi) \\ -\sin(m\phi) \end{bmatrix}.
   \end{split}
\label{eqn:point sum}
\end{align}
In the above, $R_\textrm{E}$ is the Einstein radius, which is determined
by the gravity (or mass) of the point-mass lens, and thus determines the strength
of the lensing effect.
An approximation of the mapping can be calculated using the sum from $m = 1$
to some finite number $m_0$ with increasing accuracy as $m_0 \rightarrow \infty$.
% This will produce a distorted image with $(m_0+1)$ spurious images and will only
% be valid within $r<\xi$.
This model will be referred to as the 
\textit{finite point-mass model}. Using analytic continuation,
the infinite sum can be calculated and extended outside this region.
Using  geometric series, it can be written in closed form as follows 
\citep{Clarkson_2016_II}:
\begin{align}
   \begin{split}
    \frac{D_\textrm{L}}{D_\textrm{S}} \begin{bmatrix} x' \\ y' \end{bmatrix} 
       &= r \begin{bmatrix} \cos{\phi} \\ \sin{\phi} \end{bmatrix} 
          \\&
       + \frac{R_\textrm{E}^2r}{r^2 + \xi^2 + 2r\xi\cos(\phi)} \begin{bmatrix} 
       \frac{r}{\xi} + \cos(\phi) \\ -\sin(\phi) \end{bmatrix}
   \end{split}.
\label{eqn:point closed}
\end{align}

This is a standard result in the case of a point mass, and is not a result unique to 
the Roulette formalism.
This model will be referred to as the \textit{exact point-mass model}.
The apparent position by the following well-known formula,
\begin{align}
    %% \frac{D_\textrm{S} R_\textrm{E}^2}{D_\textrm{L} \xi}
   |\boldsymbol{\eta}_{\mathrm{app}}|
   =
   \frac{|\boldsymbol{\eta}_{\mathrm{act}}|}{2}
   + \sqrt{ \frac{|\boldsymbol{\eta}_{\mathrm{act}}|^2}{4} + 
            \big(\frac{D_\textrm{S}R_\textrm{E}}{D_\textrm{L}}\big)^2 }.
\label{eqn:shift}
\end{align}

\subsection{General recursive formulae}

A key element of the Roulettes formalism is recursive expressions
for the amplitudes $\alpha_s^m$ and $\beta_s^m$.
Proofs are given by \citet{normann2020}.
The base case is given as,
\begin{align}
    \label{eqn:alpha01}
        \alpha_1^0 = -D_\textrm{L} \frac{\partial \psi}{\partial \xi_1}
    \\
    \label{eqn:beta01}
        \beta_1^0 = -D_\textrm{L} \frac{\partial \psi}{\partial \xi_2}
\end{align}
The recursive relations are given as
\begin{align}
   \alpha_{s+1}^{m+1} &= (C_+^+)_{s+1}^{m+1} (\frac{\partial \alpha_s^m}{\partial \xi_1} 
    - \frac{\partial \beta_s^m}{\partial \xi_2})
    \label{eqn:recursion1}
    \\
     \beta_{s+1}^{m+1} &= (C_+^+)_{s+1}^{m+1} (\frac{\partial \beta_s^m}{\partial \xi_1} + \frac{\partial \alpha_s^m}{\partial \xi_2})
    \label{eqn:recursion2}
    \\
       \alpha_{s-1}^{m+1} &= (C_-^+)_{s-1}^{m+1} (\frac{\partial \alpha_s^m}{\partial \xi_1} + \frac{\partial \beta_s^m}{\partial \xi_2})
    \label{eqn:recursion3}
    \\
     \beta_{s-1}^{m+1} &= (C_-^+)_{s-1}^{m+1} (\frac{\partial \beta_s^m}{\partial \xi_1} - \frac{\partial \alpha_s^m}{\partial \xi_2})
    \label{eqn:recursion4}
\end{align}
with
\begin{align}
	(C_+^+)_s^m &= 2^{\delta_{0(s-1)}} \frac{m + 1}{m + 1 + s} D_\textrm{L}
            \label{eqn:c++}
	    \\
	    (C_-^+)_s^m &= 2^{-\delta_{0s}} \frac{m + 1}{m + 1 - s} D_\textrm{L}
            \label{eqn:c+-}
\end{align}
The astute reader may notice that amplitudes for even sums $s+m$
cannot be found through these relations.
However, the contribution from these terms are equal to zero,
because of the factor $c_{m+s}$ in Equation \eqref{eqn:general mapping}.
In other words, one can calculate all the amplitudes needed from the aforementioned relations.

\subsection{The Singular Isothermal Sphere (SIS)}

The SIS model is somewhat similar to the point-mass model as they both 
have circular symmetry.
The SIS-model however, assumes that the mass of the GL is distributed
in a spherical shape rather than concentrated at a single point.
This means that the final simplifications that was used to get the
simple equations \eqref{eqn:point sum} and \eqref{eqn:point closed}
cannot be used for the SIS model.
However, we can use the recursive formulae from the previous subsection,
with the lens potential given as
%% \footnote{It should be noted here that it is not clear to us how to arrive safely at equation 134 in~\cite{Clarkson_2016_II}, as there seems to be a factor of 1/2 missing. However, the results are consistent with e.g.~\cite{bok:schneider92_SEF}, so we trust it to be correct. Given that it is correct,~\eqref{psiSIS} follows, and $R_{\mathrm{E}}$ corresponds to $\xi_0$ defined in equation~8.34a in~\cite{bok:schneider92_SEF}.}
\begin{align}
\label{psiSIS}
\psi_\textrm{SIS}(\xi)=\frac{R_\textrm{E}}{D_\textrm{L}^2}\xi.
\end{align}
In this case, the Einstein radius $R_\textrm{E}$ depends not only on the total mass,
but also on the size of the SIS lens.  In general it is determined by the
mass distribution of the lens.

\begin{remark}\label{rem1}
   Readers who inspect the source code will note that we use
   have omitted the factors $D_L$ in 
   $\psi$, $\alpha_s^m$ and $\beta_s^m$ ($C_\pm^+$),
   and the right hand side of \eqref{eqn:general mapping}.
   The reason for this is that they all cancel out.
   Verifying this is tedious but straight forward.
\end{remark}

The formula for the apparent position is also different. From Eq.~\eqref{Deta} it follows that
\begin{align}
   |\boldsymbol{\eta}_{\mathrm{app}}| =
   |\boldsymbol{\eta}_{\mathrm{act}}| + \frac{D_\textrm{S}R_\textrm{E}}{D_\textrm{L}},
\label{eqn:shift:sis}
\end{align}
and consequently
\begin{align*}
   \xi = 
   \frac{D_{\mathrm{L}}}{D_{\mathrm{S}}}\cdot|\boldsymbol{\eta}_{\mathrm{app}}| =
   \frac{D_{\mathrm{L}}}{D_{\mathrm{S}}}\cdot|\boldsymbol{\eta}_{\mathrm{act}}|
   + R_{\mathrm{E}}.
\end{align*}
