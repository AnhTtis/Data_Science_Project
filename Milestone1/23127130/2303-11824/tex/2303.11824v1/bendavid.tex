
\section{Background on Gravitational Lensing}

All matter, ordinary or dark alike, acts as a lens, distorting the images of distant galaxies. In 1919, the deflection of light by the sun was measured by Eddington during a solar eclipse, and shown to agree with Einstein's theory of general relativity. Since then, theoretical work on lensing has been extensively developed. The scarcity and low resolution of observations, have however caused pessimism concerning the applicability of this tool for actual observation.
But one step at a time, the cosmological community has found ways to observe the phenomenon,
and at present, it is booming both with applications and observation.
Indeed, GL has become one of the major tools for mining information from the night sky.

One of the clear applications of GL is in understanding the nature of dark matter, which according to the present paradigm of cosmology is one of the main constituents in the universe. The other two are ordinary luminous matter ($\sim\,5\%$) and the so-called dark energy ($\sim 68\%$). While the former is the stuff that make up stars, planets and all the rest, dark energy is what causes the accelerated expansion of the late universe. Finally, dark matter ($\sim 27\%$) is the name given to mass indirectly observed in galaxies, but yet not seen. Its elusive nature has haunted cosmology since the 1930s.
Although dark matter does not emit light, it must have mass, and thus it bends light like ordinary (so-called luminous) matter. This means that a study of lensing by a distant galaxy is implicitly a study of the dark matter in the lens. By studying lenses at different locations in the sky, one may thus create maps of the distribution of dark matter in the universe. This is important in order to understand the nature of dark matter, since the distribution of a substance says much about the nature of it.


Traditionally, the algebraic framework for GL is divided into two regimes;
\textit{weak} and \textit{strong} GL, depending on the level of distortion.
These are two sides of the same physical phenomenon, and both types may be used to reconstruct DM distribution in cosmological surveys. It is therefore unfortunate, and artificial, that both lensing effects are not studied within the same algebraic framework. Especially when it comes to cluster-lenses, where effects from both regimes appear.

\citet{clarkson16a,Clarkson_2016_II}
found a way of solving this problem by developing an algebraic framework that can model both weak and strong GL,
henceforth referred to as the  \emph{Roulette formalism}. 
In theory, it should be possible to use the Roulette formalism to reconstruct the lens mass from images of distant galaxies subject to GL.
The usefulness of this formalism is also seen in its ability to go beyond shear 
measurements, by incorporating higher-order effects (to arbitrary order).
The ability to do so could prove useful as data received from the night 
sky drastically increases in amount and accuracy.

In order to make use of the Roulette formalism to such ends,
a number of questions should be answered. 
Firstly, since the Roulette formalism builds on a series expansion
which has to be truncated, it is important to know if the region
of convergence is large enough to give satisfactory images in 
practice.
Secondly, is it possible to generate images at a reasonable speed? 
In this work we answer these questions by implementing numerical computation
of the Roulettes formalism.

