
%%%%%%%%%%%%%%%%%%%%%%%%%%%%%%%%%%%%%%%%%%%%%%%%%%%%%%%%%%%%%%%%%%%%%%%%
%%%%% PACKAGES %%%%%%%%%%%%%%%%%%%%%%%%%%%%%%%%%%%%%%%%%%%%%%%%%%%%%%%%%
%%%%%%%%%%%%%%%%%%%%%%%%%%%%%%%%%%%%%%%%%%%%%%%%%%%%%%%%%%%%%%%%%%%%%%%%
%\usepackage[sectionbib,numbers,sort&compress]{natbib}
% \usepackage[htt]{hyphenat}
\usepackage[pdftex]{graphicx}
\usepackage[percent]{overpic}
\usepackage{multirow}
\usepackage{verbatim}
\usepackage{colortbl}
\usepackage{wrapfig}
\usepackage{soul}
\usepackage{url}
%\usepackage[font={small,it}]{caption}
\usepackage{algorithm}
\newcommand{\algorithmautorefname}{Algorithm}
\newcommand{\algorithmiccomment}[1]{\hskip1em$/*$ #1 $*/$}
% \usepackage{algorithmic}
%\usepackage[linesnumbered,ruled,vlined]{algorithm2e}
\usepackage[noend]{algpseudocode}
\usepackage{amsmath}
%\usepackage{amsthm}
\usepackage{amssymb}
\usepackage{mathtools}
\usepackage{overpic}
\usepackage{longtable}
%\usepackage{minibox}
\usepackage{float}
\usepackage{multicol}
%\usepackage{enumitem}
\usepackage{hyperref}
\usepackage{textcomp}

\usepackage{pdfpages}
%\usepackage{subfigure}
%\renewcommand{\subfigtopskip}{0pt}
%\renewcommand{\subfigbottomskip}{-6pt}
%\renewcommand{\subfigcapskip}{0pt}
%\renewcommand{\subfigcapmargin}{0pt}

\usepackage[normalem]{ulem}
\def\ul#1{\uline{#1}}

%\usepackage[style=geschichtsfrkl,sorting=none]{biblatex}
%\addbibresource{pubs.bib}


\usepackage[sc]{mathpazo}
%\linespread{0.9} %using this command the font is just as big as the times (the standard).

%%%%%%%%%%%%%%%%%%%%%%%%%%%%%%%%%%%%%%%%%%%%%%%%%%%%%%%%%%%%%%%%%%%%%%%%
%%%%% GENERAL COMMANDS - SIZE OF PAGE %%%%%%%%%%%%%%%%%%%%%%%%%%%%%%%%%%
%%%%%%%%%%%%%%%%%%%%%%%%%%%%%%%%%%%%%%%%%%%%%%%%%%%%%%%%%%%%%%%%%%%%%%%%

%\pagestyle{empty}

%%%%%%%%%%%%%%%%%%%%%%%%%%% Start - margin note related %%%%%%%%%%%%%%%%

%\usepackage[margin=1in]{geometry}
%\setlength{\marginparwidth}{2cm}

% To remove margin note format, un-comment the following 7 lines
% \setlength{\oddsidemargin}{0in}
% \setlength{\textwidth}{6.5in}
% \setlength{\headheight}{0pt}
% \setlength{\headsep}{0pt}
% \setlength{\topmargin}{0in}
% \setlength{\textheight}{9in}
% \def\mnote#1#2{}

% To remove margin note format, comment out the following
%\usepackage[letterpaper,left=0.25in,right=1.75in,top=1in,bottom=1in,
%			footskip=.25in,heightrounded,marginparwidth=1.55in,
%			marginparsep=0.1in]{geometry}
\usepackage{xcolor}
%\usepackage{xargs}
% \usepackage[textsize=footnotesize]{todonotes}
% \newcommandx{\mnote}[3][1=]{\todo[linecolor=green,backgroundcolor=green!10,bordercolor=green,#1]{#2: #3}}

%%%%%%%%%%%%%%%%%%%%%%%%%%% End - margin note related %%%%%%%%%%%%%%%%%%

% get rid of the \vspace
%\usepackage[compact]{titlesec}
%\titlespacing{\section} {0pt} {-2pt} {-2pt}
%\titlespacing{\subsection} {0pt} {-2pt} {-2pt}
%\titlespacing{\subsubsection} {0pt} {-2pt} {-2pt}
%\setlength{\parindent}{0.0in}
%\setlength{\parskip}{0.086in}

% \makeatletter
% \long\def\@makecaption#1#2{%
%   \vskip\abovecaptionskip
%   \sbox\@tempboxa{\small #1: #2}%
%   \ifdim \wd\@tempboxa >\hsize
%     \small #1: #2\par
%   \else
%     \global \@minipagefalse
%     \hb@xt@\hsize{\hfil\box\@tempboxa\hfil}%
%   \fi
%   \vskip\belowcaptionskip}
% \makeatother
%%%%%%%%%%%%%%%%%%%%%%%%%%%%%%%%%%%%%%%%%%%%%%%%%%%%%%%%%%%%%%%%%%%%%%%%
%%%%% NEW COMMANDS %%%%%%%%%%%%%%%%%%%%%%%%%%%%%%%%%%%%%%%%%%%%%%%%%%%%%
%%%%%%%%%%%%%%%%%%%%%%%%%%%%%%%%%%%%%%%%%%%%%%%%%%%%%%%%%%%%%%%%%%%%%%%%

\newtheorem{thm}{Theorem}[section]
\newtheorem{defn}[thm]{Definition}
\newtheorem{lemma}[thm]{Lemma}
\newtheorem{corollary}[thm]{Corollary}
\newtheorem{conjecture}[thm]{Conjecture}
\newtheorem{fact}[thm]{Fact}
\newtheorem{proposition}[thm]{Proposition}
\newtheorem{prop}[thm]{Proposition}
\newtheorem{lem}[thm]{Lemma}
% \newtheorem{procedure}[thm]{Procedure}
\newtheorem{definition}[thm]{Definition}
\newtheorem{claim}[thm]{Claim}
\newtheorem{remark}[thm]{Remark}
\newtheorem{invariant}[thm]{Invariant}
\newtheorem{problem}[thm]{Problem}
\newtheorem{property}[thm]{Property}
\newtheorem{example}[thm]{Example}

\newcommand{\propref}[1]{Property \ref{prop:#1}}
\newcommand{\defnref}[1]{Definition \ref{defn:#1}}
\newcommand{\lemref}[1]{Lemma \ref{lem:#1}}
\newcommand{\exref}[1]{Example \ref{ex:#1}}
\newenvironment{myitem}{\begin{list}{$\bullet$}
{\setlength{\itemsep}{-0pt}
\setlength{\topsep}{0pt}
%\setlength{\labelwidth}{0pt}
%\setlength{\labelsep}{0pt}
\setlength{\leftmargin}{12pt}
\setlength{\parsep}{0pt}
\setlength{\itemsep}{0pt}
\setlength{\partopsep}{0pt}}}%
{\end{list}}

\newif\ifremark
\long\def\remark#1{
\ifremark%
   \begingroup%
   \dimen0=\textwidth
   \advance\dimen0 by -1in%
   \setbox0=\hbox{\parbox[b]{\dimen0}{\protect\em #1}}
   \dimen1=\ht0\advance\dimen1 by 2pt%
   \dimen2=\dp0\advance\dimen2 by 2pt%
   \vskip 0.25pt%
   \hbox to \textwidth{%
      \vrule height\dimen1 width 3pt depth\dimen2%
      \hss\copy0\hss%
      \vrule height\dimen1 width 3pt depth\dimen2%
   }%
   \endgroup%
\fi}

\renewcommand{\labelitemi}{$\circ$}

\newenvironment{calendar}
{\begin{list}{}{\raggedright\itemsep=1pt plus 1pt\parsep=0pt
\leftmargin=1in\labelwidth=0pt\labelsep=0pt\itemindent=0pt
\def\makelabel##1{\llap{\hbox to 1in{\textsl{##1}\hss}}}}}
{\end{list}}

\definecolor{Red}{rgb}{0.5,0.1,0.1}
