\newcommand{\nil}{{\tt null }}
\newcommand{\seE}{\mathbf{SE(3)}}
\newcommand{\obdb}{\mathbb{O}} % object data base
\newcommand{\obe}{o} % object element from data base
\newcommand{\obc}{s} % object current configuration
\newcommand{\obt}{d} % threshold of object discovered needed for reconstruction
\newcommand{\os}{O} % occlusion space
\newcommand{\dos}{\tilde{O}} % direct occlusion space
\newcommand{\obis}{\mathbb{S}} % objects in the scene
\newcommand{\obd}{\mathbb{D}} % vector of discovered objects
\newcommand{\mplan}[1]{{\tt MotionPlanner(#1)}}
\newcommand{\pick}[1]{{\tt MPPick(#1)}}
\newcommand{\place}[1]{{\tt MPPlace(#1)}}

% $$\seE \obdb \obe \obc_{\obe_i} \obd $$
% $$\mplan{test,test}$$

Consider an environment with a set $\obis = \{\obe_1, ..., \obe_n\} \subset \obdb$ of $n$ objects for which there are available 3D models. The objects are stably resting on a support surface. Objects are allowed to be initially stacked and occlude each other from the camera view.

The robot has one fixed RGB-D sensor at its disposal. 
%\edited{remove the "which gives rise to object occlusions?"}, which gives rise to object occlusions.
%The set of discovered objects in the scene at time $t_i$ is denoted $\obd_{t_i} \subset \obis$.
Discovered objects are those recognized given the observation history. An objects is assumed to be recognized once an image segmentation process recognizes it as an individual object in the observed image. 
Similarly, a perception method for detecting the target object once observed is assumed. The region of the workspace occluded by object $\obe_i$ at pose $\obc_i$ is denoted as $\os_i(\obc_i)$. Similarly the space uniquely occluded by object $\obe_i$ at pose $\obc_i$, called the direct occlusion space, is denoted as $\dos_i(\obc_i)$. 
%\edited{the following sentence is unnecessary and confusing?}
The proposed algorithms gradually removes occlusions and recognizes objects. A motion planner is used to plan pick-and-place actions.
%The proposed algorithms decide when to sense the scene to update belief and find pose of the detected objects. 

%if more than some fraction $\obt \in (0,1]$ of the object's minimum cross-sectional area has been observed in total from past images.

%Practically these RGB-D images can be collected continuously (or periodically at high frequency) to constantly update the robot's knowledge of the environment.

%The proposed task planning approach also has access to a motion planner for the robot arm that exposes primitives \edited{for picking and placing objects (returning failure if motion planning fails).}
%\begin{myitem}
%    \item $\pick{\obe_i}$, where $\obe_i =$ is the object to be picked, which returns a trajectory to grasp object $\obe_i$ or \nil upon motion planning failure
%    \item $\place{\obe_i,c}$, where $\obe_i =$ is the object to be placed and $c \in \seE =$ is the desired goal pose (i.e., both position and orientation) for the object; it returns a trajectory to place the object $\obe_i$ at $c$ or \nil upon motion planning failure.
%\end{myitem}

While objects can start out stacked, they are not re-stacked and are only placed on the ground surface during actions.
While objects can start out stacked, no reasoning about stability and ability to re-stack objects is considered. Thus, once an action to pick up a stacked object is taken, that object will only be placed on the ground surface.
For further assumptions and required properties of the motion planner see \autoref{sec:completeness}.

The objective for the object retrieval task is to determine a sequence of pick and place actions in order to discover and subsequently retrieve the target object; the target need not be directly visible or pickable from the robot's sensors. The corresponding solution should provide desirable guarantees: (a) safety, by avoiding collisions with sensed obstacles and objects as well as occluded regions, and (b) resolution completeness ({\tt  RC}) - or alternatively probabilistic completeness, depending on the implementation of the underlying motion planner, grasping process and object placement sampling.
%Resolution completeness indicates that if a solution exists, it will be eventually found as the resolution of algorithmic parameters increase.
The optimization objective is to minimize the number of performed actions until the target object is retrieved.

%The proposed method can also detect and return failure in common cases where solutions don't exist. See \autoref{sec:completeness} for more details. 