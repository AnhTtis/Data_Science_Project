%%%%%%%%%%%%%%%%%%%%%%%%%%%%%%%%%%%%%%%%%%%%%%%%%%%%%%%%%%%%%%%%%%%%%%%%%%%%%%%%
%2345678901234567890123456789012345678901234567890123456789012345678901234567890
%        1         2         3         4         5         6         7         8

\documentclass[letterpaper, 10 pt, conference]{ieeeconf}  % Comment this line out if you need a4paper
%\usepackage{subcaption}
%\captionsetup[sub]{font={small,sf}}
%\documentclass[a4paper, 10pt, conference]{ieeeconf}      % Use this line for a4 paper

\IEEEoverridecommandlockouts                              % This command is only needed if 
                                                          % you want to use the \thanks command

\overrideIEEEmargins                                      % Needed to meet printer requirements.

%In case you encounter the following error:
%Error 1010 The PDF file may be corrupt (unable to open PDF file) OR
%Error 1000 An error occurred while parsing a contents stream. Unable to analyze the PDF file.
%This is a known problem with pdfLaTeX conversion filter. The file cannot be opened with acrobat reader
%Please use one of the alternatives below to circumvent this error by uncommenting one or the other
%\pdfobjcompresslevel=0
%\pdfminorversion=4

% See the \addtolength command later in the file to balance the column lengths
% on the last page of the document

% The following packages can be found on http:\\www.ctan.org
%\usepackage{graphics} % for pdf, bitmapped graphics files
%\usepackage{epsfig} % for postscript graphics files
%\usepackage{mathptmx} % assumes new font selection scheme installed
%\usepackage{times} % assumes new font selection scheme installed
%\usepackage{amsmath} % assumes amsmath package installed
%\usepackage{amssymb}  % assumes amsmath package installed

%%%%%%%%%%%%%%%%%%%%%%%%%%%%%%%%%%%%%%%%%%%%%%%%%%%%%%%%%%%%%%%%%%%%%%%%
%%%%% PACKAGES %%%%%%%%%%%%%%%%%%%%%%%%%%%%%%%%%%%%%%%%%%%%%%%%%%%%%%%%%
%%%%%%%%%%%%%%%%%%%%%%%%%%%%%%%%%%%%%%%%%%%%%%%%%%%%%%%%%%%%%%%%%%%%%%%%
%\usepackage[sectionbib,numbers,sort&compress]{natbib}
% \usepackage[htt]{hyphenat}
\usepackage[pdftex]{graphicx}
\usepackage[percent]{overpic}
\usepackage{multirow}
\usepackage{verbatim}
\usepackage{colortbl}
\usepackage{wrapfig}
\usepackage{soul}
\usepackage{url}
%\usepackage[font={small,it}]{caption}
\usepackage{algorithm}
\newcommand{\algorithmautorefname}{Algorithm}
\newcommand{\algorithmiccomment}[1]{\hskip1em$/*$ #1 $*/$}
% \usepackage{algorithmic}
%\usepackage[linesnumbered,ruled,vlined]{algorithm2e}
\usepackage[noend]{algpseudocode}
\usepackage{amsmath}
%\usepackage{amsthm}
\usepackage{amssymb}
\usepackage{mathtools}
\usepackage{overpic}
\usepackage{longtable}
%\usepackage{minibox}
\usepackage{float}
\usepackage{multicol}
%\usepackage{enumitem}
\usepackage{hyperref}
\usepackage{textcomp}

\usepackage{pdfpages}
%\usepackage{subfigure}
%\renewcommand{\subfigtopskip}{0pt}
%\renewcommand{\subfigbottomskip}{-6pt}
%\renewcommand{\subfigcapskip}{0pt}
%\renewcommand{\subfigcapmargin}{0pt}

\usepackage[normalem]{ulem}
\def\ul#1{\uline{#1}}

%\usepackage[style=geschichtsfrkl,sorting=none]{biblatex}
%\addbibresource{pubs.bib}


\usepackage[sc]{mathpazo}
%\linespread{0.9} %using this command the font is just as big as the times (the standard).

%%%%%%%%%%%%%%%%%%%%%%%%%%%%%%%%%%%%%%%%%%%%%%%%%%%%%%%%%%%%%%%%%%%%%%%%
%%%%% GENERAL COMMANDS - SIZE OF PAGE %%%%%%%%%%%%%%%%%%%%%%%%%%%%%%%%%%
%%%%%%%%%%%%%%%%%%%%%%%%%%%%%%%%%%%%%%%%%%%%%%%%%%%%%%%%%%%%%%%%%%%%%%%%

%\pagestyle{empty}

%%%%%%%%%%%%%%%%%%%%%%%%%%% Start - margin note related %%%%%%%%%%%%%%%%

%\usepackage[margin=1in]{geometry}
%\setlength{\marginparwidth}{2cm}

% To remove margin note format, un-comment the following 7 lines
% \setlength{\oddsidemargin}{0in}
% \setlength{\textwidth}{6.5in}
% \setlength{\headheight}{0pt}
% \setlength{\headsep}{0pt}
% \setlength{\topmargin}{0in}
% \setlength{\textheight}{9in}
% \def\mnote#1#2{}

% To remove margin note format, comment out the following
%\usepackage[letterpaper,left=0.25in,right=1.75in,top=1in,bottom=1in,
%			footskip=.25in,heightrounded,marginparwidth=1.55in,
%			marginparsep=0.1in]{geometry}
\usepackage{xcolor}
%\usepackage{xargs}
% \usepackage[textsize=footnotesize]{todonotes}
% \newcommandx{\mnote}[3][1=]{\todo[linecolor=green,backgroundcolor=green!10,bordercolor=green,#1]{#2: #3}}

%%%%%%%%%%%%%%%%%%%%%%%%%%% End - margin note related %%%%%%%%%%%%%%%%%%

% get rid of the \vspace
%\usepackage[compact]{titlesec}
%\titlespacing{\section} {0pt} {-2pt} {-2pt}
%\titlespacing{\subsection} {0pt} {-2pt} {-2pt}
%\titlespacing{\subsubsection} {0pt} {-2pt} {-2pt}
%\setlength{\parindent}{0.0in}
%\setlength{\parskip}{0.086in}

% \makeatletter
% \long\def\@makecaption#1#2{%
%   \vskip\abovecaptionskip
%   \sbox\@tempboxa{\small #1: #2}%
%   \ifdim \wd\@tempboxa >\hsize
%     \small #1: #2\par
%   \else
%     \global \@minipagefalse
%     \hb@xt@\hsize{\hfil\box\@tempboxa\hfil}%
%   \fi
%   \vskip\belowcaptionskip}
% \makeatother
%%%%%%%%%%%%%%%%%%%%%%%%%%%%%%%%%%%%%%%%%%%%%%%%%%%%%%%%%%%%%%%%%%%%%%%%
%%%%% NEW COMMANDS %%%%%%%%%%%%%%%%%%%%%%%%%%%%%%%%%%%%%%%%%%%%%%%%%%%%%
%%%%%%%%%%%%%%%%%%%%%%%%%%%%%%%%%%%%%%%%%%%%%%%%%%%%%%%%%%%%%%%%%%%%%%%%

\newtheorem{thm}{Theorem}[section]
\newtheorem{defn}[thm]{Definition}
\newtheorem{lemma}[thm]{Lemma}
\newtheorem{corollary}[thm]{Corollary}
\newtheorem{conjecture}[thm]{Conjecture}
\newtheorem{fact}[thm]{Fact}
\newtheorem{proposition}[thm]{Proposition}
\newtheorem{prop}[thm]{Proposition}
\newtheorem{lem}[thm]{Lemma}
% \newtheorem{procedure}[thm]{Procedure}
\newtheorem{definition}[thm]{Definition}
\newtheorem{claim}[thm]{Claim}
\newtheorem{remark}[thm]{Remark}
\newtheorem{invariant}[thm]{Invariant}
\newtheorem{problem}[thm]{Problem}
\newtheorem{property}[thm]{Property}
\newtheorem{example}[thm]{Example}

\newcommand{\propref}[1]{Property \ref{prop:#1}}
\newcommand{\defnref}[1]{Definition \ref{defn:#1}}
\newcommand{\lemref}[1]{Lemma \ref{lem:#1}}
\newcommand{\exref}[1]{Example \ref{ex:#1}}
\newenvironment{myitem}{\begin{list}{$\bullet$}
{\setlength{\itemsep}{-0pt}
\setlength{\topsep}{0pt}
%\setlength{\labelwidth}{0pt}
%\setlength{\labelsep}{0pt}
\setlength{\leftmargin}{12pt}
\setlength{\parsep}{0pt}
\setlength{\itemsep}{0pt}
\setlength{\partopsep}{0pt}}}%
{\end{list}}

\newif\ifremark
\long\def\remark#1{
\ifremark%
   \begingroup%
   \dimen0=\textwidth
   \advance\dimen0 by -1in%
   \setbox0=\hbox{\parbox[b]{\dimen0}{\protect\em #1}}
   \dimen1=\ht0\advance\dimen1 by 2pt%
   \dimen2=\dp0\advance\dimen2 by 2pt%
   \vskip 0.25pt%
   \hbox to \textwidth{%
      \vrule height\dimen1 width 3pt depth\dimen2%
      \hss\copy0\hss%
      \vrule height\dimen1 width 3pt depth\dimen2%
   }%
   \endgroup%
\fi}

\renewcommand{\labelitemi}{$\circ$}

\newenvironment{calendar}
{\begin{list}{}{\raggedright\itemsep=1pt plus 1pt\parsep=0pt
\leftmargin=1in\labelwidth=0pt\labelsep=0pt\itemindent=0pt
\def\makelabel##1{\llap{\hbox to 1in{\textsl{##1}\hss}}}}}
{\end{list}}

\definecolor{Red}{rgb}{0.5,0.1,0.1}

% \documentclass[a4paper]{amsart}%[a4paper]
% %%%%% GENERAL MATH COMMANDS
% Reals
\newcommand{\R}{{\mathbb R}}
% Integers
\newcommand{\Z}{{\mathbb Z}}
% Naturals
\newcommand{\N}{{\mathbb N}}
% Expectation
\DeclareMathOperator*{\E}{\mathbb{E}}
% ^th notation
\newcommand{\tth}{^{\text{th}}}
% Small dots for integer range [a .. b]
\newcommand{\sdots}{\,..\,}
% Vectorized version of matrix
\newcommand{\matvec}{\mbox{vec}}

% := sign
\newcommand{\defeq}{\vcentcolon=}
% Zero function
\newcommand{\zf}{\mathbf{0}}
% Vector of ones
\newcommand{\ones}{\mathbf{1}}

% Argmin and argmax definitions
\DeclareMathOperator*{\argmax}{arg\,max}
\DeclareMathOperator*{\argmin}{arg\,min}


%%%%% PROBLEM STATEMENT NOTATION 
% \newcommandtwoopt{\St}[2][t][]{{S_{#1}^{#2}}} % State
\newcommand{\task}[1][i]{{\mathcal{T}_{#1}}} % Task, optionally takes index
\newcommand{\tasks}{\{ \task \}_{i=1}^N}
\newcommand{\losst}[1][i]{{l_{#1}}}
\newcommand{\lossv}[1][i]{{l_{#1}^{\textrm{val}}}}
\newcommand{\tasktarget}{{\mathcal{T}_{\textrm{target}}}}
\newcommand{\lossttarget}{l_{\textrm{target}}}
\newcommand{\lossvtarget}{l_{\textrm{target}}^{\textrm{val}}}
\newcommand{\lossttargetit}{l_{\textrm{target}}^{(k)}}
\newcommand{\losstotal}{l^{\textrm{total}}}
\newcommand{\lossopt}{l^*}

\newcommand{\thetait}[2]{\theta_{#1}^{(#2)}}
\newcommand{\phit}[1]{\phi^{(#1)}}
\newcommand{\hist}[2]{S_{#1}^{(#2)}}
\newcommand{\grad}[2]{G_{#1}^{(#2)}}

\newcommand{\Alg}{\textup{\textbf{Opt}}}
\newcommand{\MetaAlg}{\textup{\textbf{MetaOpt}}}

%%%%% Theorems
\newtheoremstyle{mytheoremstyle} % name
    {\topsep}                    % Space above
    {\topsep}                    % Space below
    {\itshape}                   % Body font
    {}                           % Indent amount
    {\scshape}                   % Theorem head font
    {.}                          % Punctuation after theorem head
    {.5em}                       % Space after theorem head
    {}  % Theorem head spec (can be left empty, meaning ‘normal’)
\theoremstyle{mytheoremstyle}
\theoremstyle{plain}
\newtheorem{theorem}{Theorem}
\newtheorem{proposition}{Proposition}
\newtheorem{assumption}{Assumption}
\newtheorem{definition}{Definition}
\newtheorem{lemma}{Lemma}
\theoremstyle{remark}
\newtheorem{remark}{Remark}

%
% \begin{document}
% \section{notation}\label{sec:notation}
For a positive integer $d$, we define $[d]:=\{1,2,\ldots,d\}$. 
The set of non-negative integers is denoted by $\NN:=\{0,1,2,\ldots\}$.
The cardinality of a set $S$ is denoted by $|S|$.
%Operations on $[d]$ cyclically.

Our \emph{graphs} are finite and undirected. We allow multiple edges and loops. A \emph{simple graph} is a graph without multiple edges or loops. 


A \emph{plane map} is a connected planar graph drawn in the plane without edge crossing, considered up to continuous deformation. 
The \emph{faces} of a plane map are the connected components of the complement of the graph. The infinite face is called \emph{outer face}, and the finite faces are called \emph{inner faces}. The vertices and edges incident to the outer face are called \emph{outer} while the other are called \emph{inner}. 
The numbers $\vv$, $\ee$ and $\ff$ of vertices, edges and faces of a plane map are related by the \emph{Euler relation}  $\vv+\ff=\ee+2$. 


We now define the class of plane maps which will be relevant for this article.
\begin{definition}\label{def:d-adapted}
A \emph{$d$-map} is a plane map such that the inner faces have degree at most $d$, and the outer face has degree $d$ and is incident to $d$ distinct vertices (in other words, the contour of the outer face is a simple cycle). 
We will assume that the outer vertices of a $d$-map are labeled $v_1,v_2,\ldots, v_d$ in clockwise order along the boundary of the outer face. %, as in Figure \ref{???}.\\
A \emph{$d$-adapted map} is a $d$-map such that any simple cycle which is not the contour of a face has length at least $d$.\\
\end{definition}
We point out that $d$-adapted maps are necessarily 2-connected (because a cut point in a $d$-map $G$ implies the existence of a simple cycle of length strictly less than the degree of an inner face of $G$, which shows that $G$ is not $d$-adapted).


In a plane map, a \emph{corner} is the sector delimited by two consecutive (half-)edges around a vertex. It is called an \emph{inner corner} if it lies in an inner face, and an \emph{outer corner} otherwise.
The \emph{degree} of a vertex or face is its number of incident corners. A  \emph{$d$-angulation} is a plane map with all faces of degree $d$. A \emph{$d$-angulation of the $k$-gon} is a plane map with inner faces of degree $d$, and outer face of degree $k$. 
A graph is \emph{bipartite} if it admits a bicoloring of its vertices such that adjacent vertices have different colors. It is known that a plane map is bipartite if and only if all its faces have even degree. For $k\geq 2$, a graph is called \emph{$k$-connected} if it is connected and the deletion of any subset of $(k-1)$ vertices does not disconnect it (loops are forbidden for $k\geq 2$, multiple edges are forbidden for $k\geq 3$). 




Let $G$ be an undirected graph. An \emph{arc} of $G$ is an edge $e$ of $G$ together with a chosen orientation of $e$ (so each edge of $G$ correspond to two arcs). The arc \emph{opposite} to an arc $a$, denoted by $-a$, is the arc corresponding to the same edge as $a$ but with the opposite direction. 
The endpoints of an arc $a$ are called the \emph{initial} and \emph{terminal} vertices of $a$ (with $a$ oriented from the initial vertex to the terminal vertex).  If $v$ is the initial (resp. terminal) vertex of the arc $a$, then we say that $a$ is an \emph{outgoing arc} (resp. \emph{ingoing arc}) at $v$. 
\\

%In a graph, a \emph{walk} (of length $k$) is a sequence $v_1,e_1,v_2,\ldots,e_k,v_{k+1}$ that alternates vertices and edges, such that $e_i$ connects $v_i$ to $v_{i+1}$ for $i\in[k]$. It is called a \emph{closed walk} if $v_1=v_{k+1}$. 
%\OB{Made a change in the def of walk (talking about arcs instead). Should we call them ``paths'' rather than ``walks''?}
A \emph{path} in an undirected graph $G$ is a sequence of arcs $a_1,a_2,\ldots,a_k$ such that the terminal vertex of $a_i$ is the initial vertex of $a_{i+1}$ for all $i\in[k-1]$. It is called a \emph{closed path} if the terminal vertex of $a_k$ is the initial vertex of $a_1$. A \emph{cycle} is a (cyclically ordered) closed path. A path or cycle is called \emph{simple} if it does not pass twice by the same vertex. The \emph{girth} of a graph is the minimum length of its simple cycles.   In a plane map, a closed path formed by the arcs around a face is called \emph{contour} of that face. It is known that face contours are simple cycles if the plane map is 2-connected. 
A simple cycle on a plane map is called \emph{counterclockwise} (resp. \emph{clockwise}) if the direction of arcs is counterclockwise (resp. clockwise) around the cycle.

Let $G$ be a graph.  Given an orientation of $G$, a \emph{directed path} (resp. \emph{directed cycle}) is a path (resp. cycle) $a_1,a_2,\ldots,a_k$ such that every arc $a_i$ is oriented according to the orientation of $G$.
A \emph{weighted orientation} of $G$ is an assignment of a non-negative integer to each arc of $G$. Given a weighted orientation $\cW$ of $G$, we call \emph{weight} of an edge the sum of the weights of the two corresponding arcs. 
Weighted orientations are a generalization of the classical (unweighted) orientations of $G$. Indeed the ``unweighted'' orientations of $G$ can be identified to the weighted orientations of $G$ such that the weight of every edge is 1 (for each edge, the arc of weight 1 is taken as the orientation of the edge). The \emph{outgoing weight} (shortly, the \emph{weight}) of a vertex $v$ is the sum of the weights of the arcs going out of $v$. Given a weighted orientation, we call \emph{positive path} (resp. \emph{positive cycle}) a path (resp. cycle) $a_1,a_2,\ldots,a_k$ such that the weight of every arc is positive (this generalizes the notion of \emph{directed path} and \emph{directed cycle}).  




A \emph{tree} is a connected, acyclic graph. For a tree $T$ with a vertex $v$ distinguished as its \emph{root}, we apply the usual ``genealogy'' vocabulary about trees, where $v$ is an \emph{ancestor} of all the other vertices, and every non-root vertex incident to $T$ has a \emph{parent} in $T$, etc. 
We say that we \emph{orient the tree $T$ toward its root} by orienting every edge from child to parent. With this orientation, every non-root vertex of $T$ is incident to one outgoing edge in $T$ (the edge leading to its parent).
%\OB{changed: calling ``subtree'' instead of ``tree''}
A \emph{subtree} of a graph $G$ is a subset of edges of $G$ such that this set of edges together with the incident vertices forms a tree. A \emph{spanning tree} of $G$ is a subtree of $G$ incident to every vertex of $G$. 





%\end{document}

\newcommand{\kostas}[1]{{\textcolor{red}{#1}}}

\title{\LARGE \bf
Resolution Complete In-Place Object Retrieval\\given Known Object Models
}


\author{Daniel Nakhimovich, Yinglong Miao, and Kostas E. Bekris% stops a space
\thanks{ The authors are with the Dept. of Computer Science, Rutgers, New Brunswick, NJ. Email: {\tt \{d.nak, ym420,kb572\}@rutgers.edu}. The work is partially supported by NSF awards 1934924 and 2021628. The opinions expressed here are those of the authors and do not reflect the positions of the sponsor.}%
}



\begin{document}



\maketitle
\thispagestyle{empty}
\pagestyle{empty}


%%%%%%%%%%%%%%%%%%%%%%%%%%%%%%%%%%%%%%%%%%%%%%%%%%%%%%%%%%%%%%%%%%%%%%%%%%%%%%%%
\begin{abstract}
This work proposes a robot task planning framework for retrieving a target object in a confined workspace among multiple stacked objects that obstruct the target. The robot can use prehensile picking and in-workspace placing actions.
%, i.e., it cannot access external buffers for the placement of obstructing objects.
The method assumes access to 3D models for the visible objects in the scene. The key contribution is in achieving desirable properties, i.e., to provide (a) safety, by avoiding collisions with sensed obstacles, objects, and occluded regions, and (b) resolution completeness ({$\tt RC$}) - or probabilistic completeness ({$\tt PC$}) depending on implementation - which indicates a solution will be eventually found (if it exists) as the resolution of algorithmic parameters increases. A heuristic variant of the basic {$\tt RC$} algorithm is also proposed to solve the task more efficiently while retaining the desirable properties. Simulation results compare using random picking and placing operations against the basic {$\tt RC$} algorithm that reasons about object dependency as well as its heuristic variant. The success rate is higher for the {$\tt RC$} approaches given the same amount of time. The heuristic variant is able to solve the problem even more efficiently than the basic approach. The integration of the $\tt RC$ algorithm with perception, where an RGB-D sensor detects the objects as they are being moved, enables real robot demonstrations of safely retrieving target objects from a cluttered shelf.
\end{abstract}

\section{INTRODUCTION}

% what is the problem
Robotic manipulation has the potential of being integrated into daily lives of people, such as in household service areas \cite{wong2013manipulation, garrett2020online}. A useful skill for such household settings involves the retrieval of a target object from a confined and cluttered workspace, such as a fridge or a shelf, which may also require the rearrangement of other objects in the process. In this context, it is important to consider how to safely retrieve objects while minimizing the time spent or the amount of pick and place operations, so as to assist humans efficiently. 

\begin{figure}[thpb]
  %\vspace{-0.3in}
  \centering
  \includegraphics[width=\linewidth,
  trim={0 3cm 0 1cm},clip]{figures/real_setup.pdf}
  \includegraphics[width=0.465\linewidth,trim={13.9cm 0 12cm 12.9cm},clip]{figures/real_scene_voxel_grid.jpg}
  \includegraphics[width=0.52\linewidth,trim={4cm 7cm 5cm 2cm},clip]{figures/voxel_grid.png}
  \vspace{-0.3in}
  \caption{(Top) Setup for the real demonstration using an RGB-D sensor, robotiq gripper, and Yaskawa Motoman robot to retrieve the target bottle. (Bottom Left) The camera view in which objects are occluded. (Bottom Right) The corresponding voxel map.}
  \label{fig:real_setup}
  \vspace{-0.32in}
\end{figure}

One of the challenging aspects of these problems that requires explicit reasoning relates to heavy occlusions in the scene, as the sensor is often mounted on the robot and has limited visibility. These visibility constraints complicate the task planning process, as rearranging one object can limit placements for others and can introduce new occlusions. Moreover, real-world scenes in household setups are often unstructured and involve objects with complex spatial relationships, such as objects stacked on each other.%These relationships constrain which objects are reachable to the robot.
%action sequence that can be executed by the robot so as to ensure objects remain physically stable. 

Many previous efforts on object retrieval have focused on cases where blocking objects are extracted from the workspace \cite{dogar2014object, nam2021fast}, which simplifies the challenge as it does not require identifying temporary placement locations for the objects within the confined space. In-place rearrangement has been considered in some prior efforts \cite{ahn2021integrated}. While this prior method is efficient, it is not complete as it limits the reasoning on the largest object in the scene to analyze object traversability \cite{nam2021fast}. Alternatives use machine learning to guide the decision making \cite{bejjani2021occlusion, huang2022mechanical, price2019inferring}, which is an exciting direction but does not easily allow for performance guarantees, such as resolution completeness. Setups where object stacking arise have received less attention and most solutions that do consider stacking are dependent on machine learning for reasoning \cite{huang2022mechanicalstack, zhu2021hierarchical, kumar2022graph}. Some works have proposed testbeds \cite{liu2021ocrtoc} that can help evaluate solutions in this domain.

This work focuses on object retrieval in clutter where occlusions arise and objects may be  initially stacked under the assumption of known object models (e.g. \autoref{fig:dep_graph}). It aims at a theoretical understanding to show the algorithm has safety and $\tt RC$ guarantees. A heuristic variant improves the practical efficiency. Key features of the proposed $\tt RC$ framework are the following:

    %\item discretizing and identifying likely occlusion regions based on volume%through a variety of heuristics

\begin{myitem}
    \item it employs an adaptive dependency graph data structure inspired by solutions in object rearrangement with performance guarantees \cite{wang_uniform_2021} that express a larger variety of object relationships than previously considered (namely occlusion dependencies);
    \item it computes the occlusion volume of detected objects as a heuristic to inform the planning process;
    \item it reasons about the collision-free placement of objects in the confined workspace efficiently by utilizing a voxelized representation of the space;% in a way that minimizes future obstructions;
    \item it achieves $\tt RC$ (or $\tt PC$) depending on the implementation of the underlying sampling subroutines;
    \item it provides an early termination criterion when a solution cannot be found for the given resolution. 
\end{myitem}

Simulation results, using a model of a Yaskawa Motoman manipulator for rearranging objects on a tabletop as shown in Fig. \ref{fig:real_setup}, evaluate the proposed {$\tt RC$} framework against a baseline using random picking and placing operations. Both variants of the  {$\tt RC$} reason about object dependencies. One doesn't use heuristics and one is heuristically guided while retaining {$\tt RC$}. Both of the {$\tt RC$} approaches outperform the baseline. The heuristically guided solution is able to solve the problem more efficiently than the basic {$\tt RC$} solution. The integration of the proposed approach with perception, where an RGB-D sensor detects the objects as they are being moved, provides real robot demonstrations of safe object retrieval from a cluttered shelf.

\section{RELATED WORK}

%Object retrieval (sometimes called mechanical search) has had many recent works focus on cluttered and occluded scenes. \cite{bejjani2021occlusion,huang2020mechanical,kurenkov2020visuomotor,danielczuk2019mechanical} that use various models
% TODO: state explicitly which models
%to reason over the scene to determine manipulations that will make progress towards the discovery of a target object. This work uses a voxelization of the workspace to reason about object occupancy and occlusion.

% {
% \color{red}
% \textbf{General Related Work: Object manipulation under occlusion, Mechanical Search, and Object retrieval problem.}
% Related work about different settings, completeness analysis, and heuristic methods, and their limitations.
% }


%Past work has focused on cases where objects are extracted from the workspace \cite{dogar2014object, nam2021fast}, or uses machine learning methods to guide the decision making \cite{bejjani2021occlusion, huang2022mechanical}. In-place rearrangement is considered in \cite{ahn2021integrated}. Although the method is efficient, it is not complete as it uses the largest object in the scene to analyze the traversability of objects \cite{nam2021fast}. Object stacking has limited visibility and most work applies machine learning methods for reasoning \cite{huang2022mechanicalstack, zhu2021hierarchical}.

Some works on object retrieval rely on geometric analysis of object occlusion \cite{dogar2014object,nam2021fast}. They provide theoretical insights but frequently do not limit actions to in-place rearrangement of blocking objects. Specifically, one method constructs a dependency graph taking into account objects that jointly occlude a region and objects that block others \cite{dogar2014object}. The occlusion volume is used to estimate belief regarding the target object position and helps to construct an optimal A* algorithm. An alternative constructs a Traversability graph (T-graph) \cite{nam2021fast}, where the edges encode if the largest object in the scene can be moved between two poses. It then constructs an  algorithm to extract the target object, but is limited as the traversability edges are too constraining. The POMDP formulation is popular for the task \cite{zhao2021hierarchical, xiao2019online}, which allows the application of general POMDP solvers. The POMDP formulation was also adopted by the work that formalizes object retrieval in unstructured scenes as "mechanical search" \cite{danielczuk2019mechanical}.

Alternatives rely on learning-based methods to solve such challenges, such as reinforcement learning \cite{bejjani2021occlusion} or target belief prediction \cite{huang2022mechanical, huang2022mechanicalstack, huang2021mechanical}. They report good performance but do not provide theoretical guarantees given the black-box nature of the solutions. In particular, a reinforcement learning solution \cite{bejjani2021occlusion} uses the rendered top-down projections of the scene to predict the target poses. A recent follow-up effort \cite{huang2022mechanical} on previous work \cite{huang2021mechanical} estimates the 1D position belief of the target object on the shelf via machine learning. It then constructs a policy based on the distribution change after applying pushing and suction actions. It incorporates stacking and unstacking actions, where object stacking is represented by a tree structure. Other works such as \cite{zeng2022robotic} utilize learning for planning grasps to greedily empty bins of complex and novel objects.

A related work \cite{miao2022safe} proposes a complete framework to safely reconstruct all objects in the scene amidst object occlusions. Nevertheless, object retrieval may not require reconstructing all objects and requires a search procedure that is more task-driven for efficiency. There are also previous works \cite{gupta2013interactive, miao2022safe} that construct a voxelization of the environment to model object occlusions, similar to the current work. This representation is used to compute an object's occlusion volume, which provides heuristic guidance. Object spatial relationships are often represented by scene graphs \cite{zhu2021hierarchical, kumar2022graph}, or implicitly in machine learning solutions \cite{danielczuk2019mechanical, poon2019probabilistic, novkovic2020object}.

What stands out in this work is that it proposes a general template for a $\tt RC$ or $\tt PC$ approach to task retrieval in occluded environments that only relies on basic motion and perception primitives. This modular nature allows for quick sim-to-real transfer and passive performance improvement as the primitives are improved over time.
Furthermore, an efficient implementation is demonstrated utilizing a
voxelized representation of the environment for quick collision filtering of object placements as well as providing an effective heuristic to rank object manipulations. Thus, the framework enables effective in-workspace manipulation.
%The combination of theoretical guarantees and implementation enable effective in-workspace manipulation without removing objects from the table or shelf.

%However, \cite{dogar2014object,nam2021fast} assume extra placements are available outside the workspace, hence do not apply to in-place rearrangement actions.

%Beyond that, most cases with completeness guarantees assume extra placements outside the confined space are available, which may not be the case in general. Meanwhile, past works do not consider complicated spatial relations of objects, such as object stacking on each other.

% {\color{red}
% \textbf{Specific methods for handling different complexities, such as occlusion modeling, and action modeling (rearrangement). Keep or Remove?}}










%Regarding occlusion reasoning, \cite{huang2022mechanical} utilizes object footprint area to reason about possible positions of the hidden object during training. This work uses similar modeling but using the volume of objects to consider possible object locations during runtime.




%{
%\color{red}
%\textbf{How NLP has developed and helped with similar domains, and more related lines.}
%}

%Initial approaches integrating natural language descriptions would would use sentence parsers to ``translate'' sentences in natural language into a more formal language, such as LTL \cite{raman2013sorry} or semspec \cite{pomarlan2018robot}, and then apply cleverly constructed grammars to parse the intermediate languages into an executable program for a given robot system. Such approaches have the benefit of handling multiple failure modes in task specification and can be used to produce explanations of why a given task was found to be invalid. Drawbacks of such approaches are their complexity in adapting to new robot systems and ability to handle ambiguity in user descriptions.
%The focus of this work is similar in terms of reasoning about feasibility of language instructions but also integrates a conversation agent that could resolve contradictory information or ambiguity.

%Other related efforts have used natural language to learn new task primitives \cite{suddrey_teaching_2017} or as input into a hierarchical search process \cite{kurenkov_semantic_2021}. One of the motivations for exploring how a human instruction can complement the partial scene observation is to speed up scene reconstruction and object discovery. To that end, Human-in-the-loop systems \cite{papallas2020non} have been implemented but rely on more direct user interface elements rather than natural language, and don't deal with object occlusion. The proposed work differs by using the language-based task description to inform the planner about the unobserved space.
%A recent work \cite{zheng2021spatial} has a similar motivation to integrate language reasoning to speed up task planning for a city-scale navigation environment.
%Work on Hierarchical Mechanical Search \cite{kurenkov_semantic_2021} does appear similar to the proposed setup, but handles scene representation and language processing separately, while the current work aims to integrate these two sources of information. 


%At least one work does integrate language and task reasoning \cite{nguyen_robot_2020} in order to retrieve objects based on their utility. This work focus not on the object selection but on object discovery within an occluded scene.

%Camera on gripper:
%\url{https://arxiv.org/pdf/2011.03334.pdf}

\section{PROBLEM STATEMENT}

\newcommand{\nil}{{\tt null }}
\newcommand{\seE}{\mathbf{SE(3)}}
\newcommand{\obdb}{\mathbb{O}} % object data base
\newcommand{\obe}{o} % object element from data base
\newcommand{\obc}{s} % object current configuration
\newcommand{\obt}{d} % threshold of object discovered needed for reconstruction
\newcommand{\os}{O} % occlusion space
\newcommand{\dos}{\tilde{O}} % direct occlusion space
\newcommand{\obis}{\mathbb{S}} % objects in the scene
\newcommand{\obd}{\mathbb{D}} % vector of discovered objects
\newcommand{\mplan}[1]{{\tt MotionPlanner(#1)}}
\newcommand{\pick}[1]{{\tt MPPick(#1)}}
\newcommand{\place}[1]{{\tt MPPlace(#1)}}

% $$\seE \obdb \obe \obc_{\obe_i} \obd $$
% $$\mplan{test,test}$$

Consider an environment with a set $\obis = \{\obe_1, ..., \obe_n\} \subset \obdb$ of $n$ objects for which there are available 3D models. The objects are stably resting on a support surface. Objects are allowed to be initially stacked and occlude each other from the camera view.

The robot has one fixed RGB-D sensor at its disposal. 
%\edited{remove the "which gives rise to object occlusions?"}, which gives rise to object occlusions.
%The set of discovered objects in the scene at time $t_i$ is denoted $\obd_{t_i} \subset \obis$.
Discovered objects are those recognized given the observation history. An objects is assumed to be recognized once an image segmentation process recognizes it as an individual object in the observed image. 
Similarly, a perception method for detecting the target object once observed is assumed. The region of the workspace occluded by object $\obe_i$ at pose $\obc_i$ is denoted as $\os_i(\obc_i)$. Similarly the space uniquely occluded by object $\obe_i$ at pose $\obc_i$, called the direct occlusion space, is denoted as $\dos_i(\obc_i)$. 
%\edited{the following sentence is unnecessary and confusing?}
The proposed algorithms gradually removes occlusions and recognizes objects. A motion planner is used to plan pick-and-place actions.
%The proposed algorithms decide when to sense the scene to update belief and find pose of the detected objects. 

%if more than some fraction $\obt \in (0,1]$ of the object's minimum cross-sectional area has been observed in total from past images.

%Practically these RGB-D images can be collected continuously (or periodically at high frequency) to constantly update the robot's knowledge of the environment.

%The proposed task planning approach also has access to a motion planner for the robot arm that exposes primitives \edited{for picking and placing objects (returning failure if motion planning fails).}
%\begin{myitem}
%    \item $\pick{\obe_i}$, where $\obe_i =$ is the object to be picked, which returns a trajectory to grasp object $\obe_i$ or \nil upon motion planning failure
%    \item $\place{\obe_i,c}$, where $\obe_i =$ is the object to be placed and $c \in \seE =$ is the desired goal pose (i.e., both position and orientation) for the object; it returns a trajectory to place the object $\obe_i$ at $c$ or \nil upon motion planning failure.
%\end{myitem}

While objects can start out stacked, they are not re-stacked and are only placed on the ground surface during actions.
While objects can start out stacked, no reasoning about stability and ability to re-stack objects is considered. Thus, once an action to pick up a stacked object is taken, that object will only be placed on the ground surface.
For further assumptions and required properties of the motion planner see \autoref{sec:completeness}.

The objective for the object retrieval task is to determine a sequence of pick and place actions in order to discover and subsequently retrieve the target object; the target need not be directly visible or pickable from the robot's sensors. The corresponding solution should provide desirable guarantees: (a) safety, by avoiding collisions with sensed obstacles and objects as well as occluded regions, and (b) resolution completeness ({\tt  RC}) - or alternatively probabilistic completeness, depending on the implementation of the underlying motion planner, grasping process and object placement sampling.
%Resolution completeness indicates that if a solution exists, it will be eventually found as the resolution of algorithmic parameters increase.
The optimization objective is to minimize the number of performed actions until the target object is retrieved.

%The proposed method can also detect and return failure in common cases where solutions don't exist. See \autoref{sec:completeness} for more details. 

\section{METHOD}

\newcommand{\depgraph}[1]{{\tt DepGraph(#1)}}
\newcommand{\placement}[1]{{\tt SamplePlacement(#1)}}
\newcommand{\addHiddenEdges}[1]{{\tt AddHiddenEdges(#1)}}

%\subsection{Pipeline}

The proposed pipeline is detailed in
%\autoref{alg:subtasks} and 
\autoref{alg:pipeline}. First a voxelized representation of the scene and a dependency graph are computed (lines 3,4,5), which are detailed in \autoref{subsec:voxel_dep}. The dependency graph contains a belief state of the current scene based on visibility and reachability constraints. All the \textit{sinks} of this directed graph represent likely pickable objects.
The \textit{ranks} are described later (see \autoref{subsec:target-prediction}).
If the target object is pickable then it is retrieved and the pipeline is terminated line 6.
Otherwise, a placement is planned (see \autoref{subsec:placement_sampling}) one at a time for each object in a shuffled order biased by the \textit{ranks} (\textit{TryMoveOne} line 7). The first successful plan is executed.

If no placement is found for any of the pickable objects, then a fallback procedure is called
%\edited{(redundant to say "on the same set of objects?")} on the same set of objects
to try and pick one object and move it temporarily out of view of the camera; the first successful plan is executed, the scene is re-sensed, and a new placement is sampled for the object or the object is simply put back at the same spot (\textit{MoveOrPlaceback} line 10). At this stage, the pipeline could restart if a new object was discovered (lines 12-15). Otherwise the same set of pickable objects 
%\edited{(redundent to say "the same set of pickable objects"?)} 
are tested for new placements in the scene (line 16). If these two operations of ``moving an object to look behind it''\textit{(MoveOrPlaceback)} followed by retrying to ``move one of the pickable objects to a new spot''\textit{(TryMoveOne)} fail for all objects, then the pipeline can return and report failure for the current resolution (line 22). If the sequence of operations succeeds, then the pipeline can restart (line $19\rightarrow2$).

%\newcommand{\exec}[1]{{\tt Execute(#1)}}
%\newcommand{\revx}[1]{{\tt Reverse(#1)}}
\newcommand{\trymoveone}[1]{{\tt TryMoveOne(#1)}}
\newcommand{\move}[1]{{\tt MoveOrPlaceback(#1)}}
%\begin{algorithm}
%\caption{SubTask Routines}\label{alg:subtasks}
%\begin{algorithmic}[thpb]
%
%\Function{$\tt TryMoveOne$}{$\tt objects$}
%    \For{$\tt obj \in WeightedShuffle(objects)$}
%        \Comment{a.k.a pick without replacement with probabilities adjusted by rank}
%        \State $\tt pickTraj \gets \pick{obj}$
%        \If{$\tt pickTraj \neq \nil$} 
%            \State $\tt p \gets \placement{obj}$
%            \State $\tt placeTraj \gets \place{obj,p}$
%            \If{$\tt placeTraj \neq \nil$}
%                \State $\tt \exec{pickTraj}$
%                \State $\tt SenseEnvironment()$
%                \State $\tt \exec{placeTraj}$
%                \State \Return $\tt true$
%            \EndIf
%        \EndIf
%    \EndFor
%    \State \Return $\tt false$
%\EndFunction
%
%\Function{$\tt MoveOrPlaceback$}{$\tt object$}
%    \State $\tt pickTraj \gets \pick{object}$
%    \If{$\tt pickTraj \neq \nil$} 
%        \State $\tt \exec{pickTraj}$
%        \State $\tt SenseEnvironment()$
%        \State $\tt p \gets \placement{object}$
%        \State $\tt placeTraj \gets \place{object,p}$
%        \If{$\tt placeTraj \neq \nil$}
%            \State $\tt \exec{placeTraj}$
%        \Else
%            \State $\tt \revx{pickTraj}$
%        \EndIf
%        \State \Return $\tt true$
%    \EndIf
%    \State \Return $\tt false$
%\EndFunction
%
%\end{algorithmic}
%\end{algorithm}

\begin{algorithm}[thpb]
\caption{$\tt RC\_Pipeline$($\tt target$)}\label{alg:pipeline}
\begin{algorithmic}[1]
    \State $\tt failure \gets false$
    \While{$\tt failure = false$}
        \State $\tt space \gets UpdateVoxelsFromImage()$
        \State $\tt dg \gets \depgraph{space}$
        \State $\tt sinks,ranks \gets RankSinks(target,dg)$ 
        \If{$\tt target \in sinks$} $\tt break$\EndIf
        \If{$\tt \trymoveone{sinks,ranks} = false$}
            \State $\tt failure \gets true$
            \For{$\tt sink \in sinks$}
                \If{$\tt \move{sink} = false$}
                    \State $\tt continue$
                \EndIf
                \State $\tt space \gets UpdateVoxelsFromImage()$
                \If{$\tt DidDiscoverObject(space)$}
                    \State $\tt failure \gets false$
                    \State $\tt break$
                \EndIf
                \If{$\tt \trymoveone{sinks,\emptyset} = false$}
                    \State $\tt continue$
                \EndIf
                \State $\tt failure \gets false$
                \State $\tt break$
            \EndFor
        \EndIf
    \EndWhile
    \If{$\tt not~failure$} 
        \State $\tt Retrieve(target)$
    \EndIf
    \State $\tt return~failure$
    
\end{algorithmic}
\end{algorithm}

\begin{figure*}[thpb]
%digraph dep_graph {
%	rankdir=BT
%	node [
%			shape=circle,
%			fixedsize=true,
%			width=0.34,
%           color=black,
%			fillcolor="#eeeeee",
%			style="filled,solid",
%			fontsize=22
%		]
%	T [shape = doublecircle, fillcolor=red]
%	2 [fillcolor=purple]
%	3 [fillcolor=green3]
%	4 [fillcolor=cyan3]
%	5 [fillcolor=greenyellow]
%	6 [fillcolor=pink1]
%	3 -> 4 [ label = "below\n100%" ]
%    2 -> 6 [ label = "grasp blocked by\n100%" ]
%    T -> 2 [ label = "hidden_by\n75.13%" ]
%    T -> 3 [ label = "hidden_by\n8.84%" ]
%    T -> 5 [ label = "hidden_by\n8.69%" ]
%    T -> 6 [ label = "hidden_by\n7.34%" ]
%}

    %\includegraphics[width=.33\linewidth,trim={0 140 40 22},clip]{figures/sim_exp/step1_ws.png}
    \begin{overpic}[width=0.336\linewidth,trim={0 140 40 22},clip]
    {figures/sim_exp/step1_ws.png}
        \put (3,10) {\huge$\displaystyle 1)$}
    \end{overpic}
    %\includegraphics[width=.33\linewidth,trim={0 140 40 22},clip]
    \begin{overpic}[width=0.336\linewidth,trim={0 140 40 22},clip]
    {figures/sim_exp/step2_ws.png}
        \put (3,10) {\huge$\displaystyle 2)$}
    \end{overpic}
    %\includegraphics[width=.33\linewidth,trim={0 140 40 22},clip]
    \begin{overpic}[width=0.336\linewidth,trim={0 140 40 22},clip]
    {figures/sim_exp/step3_ws.png}
        \put (3,10) {\huge$\displaystyle 3)$}
    \end{overpic}
    %\includegraphics[width=.33\linewidth,trim={0 140 40 22},clip]
    \begin{overpic}[width=0.336\linewidth,trim={0 140 40 22},clip]
    {figures/sim_exp/step4_ws.png}
        \put (3,10) {\huge$\displaystyle 4)$}
    \end{overpic}
    %\includegraphics[width=.33\linewidth,trim={0 140 40 22},clip]
    \begin{overpic}[width=0.336\linewidth,trim={0 140 40 22},clip]
    {figures/sim_exp/step5_ws.png}
        \put (3,10) {\huge$\displaystyle 5)$}
    \end{overpic}
    %\includegraphics[width=.33\linewidth,trim={0 140 40 22},clip]
    \begin{overpic}[width=0.336\linewidth,trim={0 140 40 22},clip]
    {figures/sim_exp/step6_ws.png}
        \put (3,10) {\huge$\displaystyle 6)$}
    \end{overpic}
    %\includegraphics[width=0.245\linewidth]
    \begin{overpic}[width=0.245\linewidth]
    {figures/sim_exp/step1_dg_c.png}
        \put (-1,5) {\huge a)}
    \end{overpic}
    %\includegraphics[width=0.245\linewidth]
    \begin{overpic}[width=0.245\linewidth]
    {figures/sim_exp/step2_dg.png}
        \put (-1,5) {\huge b)}
    \end{overpic}
    %\includegraphics[width=0.245\linewidth]
    \begin{overpic}[width=0.245\linewidth]
    {figures/sim_exp/step3_dg.png}
        \put (-1,5) {\huge c)}
    \end{overpic}
    %\includegraphics[width=0.245\linewidth]
    \begin{overpic}[width=0.245\linewidth]
    {figures/sim_exp/step4_dg.png}
        \put (-1,5) {\huge d)}
    \end{overpic}
    %\includegraphics[width=0.2\linewidth]{figures/sim_exp/step5_dg.png}
  %\framebox{\includegraphics[width=\linewidth]{figures/example_scene_labeled.png}}
  %\framebox{\includegraphics[width=\linewidth]{figures/example_scene_both_views.png}}
  %\framebox{\includegraphics[width=\linewidth]{figures/dep_graph.png}}
  %\framebox{\includegraphics[width=\linewidth]{figures/example_scene_camera_view.png}}
  %\vspace{0.1in}
  %\includegraphics[width=.196\linewidth]{figures/sim_exp/step1_ws_I.png}
  %\includegraphics[width=.196\linewidth]{figures/sim_exp/step2_ws_I.png}
  %\includegraphics[width=.196\linewidth]{figures/sim_exp/step3_ws_I.png}
  %\includegraphics[width=.196\linewidth]{figures/sim_exp/step4_ws_I.png}
  %\includegraphics[width=.196\linewidth]{figures/sim_exp/step5_ws_I.png}
  \caption{
  Images (1)-(6) show a simulated experiment from initial configuration to final one action at a time with corresponding camera views in the top left. The corresponding generated dependency graphs transitioning between the images (1)-(2), (2)-(3), (3)-(4), and (4)-(5) are shown in images (a)-(d). 
  The colors of nodes corresponds with the objects in the scene and the red object (labeled `T') is the target object for the trial.
  %The final pick is shown in \autoref{fig:object_pick}.
  The last graph between images 5-6 is not shown since it is trivial having no dependencies between any objects.
  }
  \label{fig:dep_graph}
  \vspace{-0.28in}
\end{figure*}


\subsection{Voxel Map and Dependency Graph}
\label{subsec:voxel_dep}

A 3D occlusion voxel grid within the workspace is constructed from RGB-D images.
First the point cloud (in world frame) is generated using the RGB-D image and the inverse of the camera projection. These points are down-sampled into a voxelization of the scene.
Given segmentation image information, each object is associated with a portion of the voxel grid. Object geometry is used to label voxels as occupied and remaining associated voxels as occluded.
The occluded regions of objects may intersect when jointly occluded.

The dependency graph is a directed graph where each node represents a visible (or target) object and a labeled edge $(\obe_i, \obe_j, r)$ represents a relation $r$ from object $\obe_i$ to $\obe_j$ that necessitates $\obe_j$ to be picked and placed before $\obe_i$ could be picked.
%, where $p \in (0,1]$ denotes a heuristic estimate of the probability for that relation $\obe_i \overset{r}{\longrightarrow} \obe_j$ to be true. For the non-heuristic variant of the pipeline, $p$ does not need to be included.
Valid relations in this work include ``below'', ``grasp blocked by'', and -- for the prediction of the target object -- ``hidden by''. See \autoref{fig:dep_graph} for a sequence of such dependency graphs generated during an example experiment. 

Object x is defined to be ``below'' object y ($x \xrightarrow{below} y$) if object x touches object y and the z-coordinate of the center of mass of x is less than that of y. Note that this isn't guaranteed to capture all intuitive cases of one object being below another for non-convex objects. This relation is computed using object models and poses given by the perception system.
%\edited{Question: what if multiple objects are stacked. Does the edge go from the first object to the third object? Or is it only for objects that are immediately stacked?}

Object x has its ``grasp blocked by'' y ($x \xrightarrow{blocked} y$) if there are no collision free grasp poses for object x and the arm is in collision with y for one or more valid grasp poses.
(Grasp poses are sampled and tested by inverse kinematics (IK) for discovered objects)
Grasp poses are sampled using inverse kinematics (IK) to discovered objects. 
(or, if the grasping pose collides with objects, an edge to each collided object is added)
If there exists a collision free grasp for an object, no blocking edges are added; otherwise, an edge to each object that has a collision with the arm is added. Note that although this relation guarantees that the source object blocks the target, it doesn't capture all such reachability dependencies. This however is not an issue for completeness as \autoref{alg:pipeline} will eventually try to grasp all objects in the case of motion planning failure.
%with a heuristic weight equal to the fraction of total sampled grasps for which that object is in collision with the arm. Note, the weights of grasp blocking edges coming out of any object need not add to one since the arm could be colliding with multiple objects for any single grasp.

The target object t is possibly ``hidden by'' x ($t \xrightarrow{hidden} x$) if the target isn't sensed in the scene and object x is touching the table. This relation is used to keep track of the belief state of where the target is. Each edge is assigned a probability based on the volume of the occluded space behind object x (see \autoref{subsec:target-prediction}).

% WRAPFIGURE IF SPACE
% \begin{figure}[thpb]
%     \centering
%     \includegraphics[width=\linewidth,trim={0 140 40 22},clip]{figures/sim_exp/step6_ws.png}
%     \caption{The final pick action for the experiment shown in \autoref{fig:dep_graph}.}
%     \label{fig:object_pick}
% \end{figure}

%%%%%%%%%%%%%%%
%\begin{algorithm}
%\caption{$\tt \depgraph{}$}\label{alg:dep_graph}
%\begin{algorithmic}[thpb]
%    \State $todo$
%\end{algorithmic}
%\end{algorithm}

\vspace{-0.11in}
\subsection{Placement Sampling}
\label{subsec:placement_sampling}
A valid placement is one that doesn't collide with another object or any undiscovered area of the workspace.
%\edited{(remove the following two sentences if needed space?)}
%This is computed exhaustively up to a resolution.
%The voxelized representation of the space naturally lends itself to sampling objects placements in a grid of the same resolotion.
Instead of randomly sampling x,y coordinates for an object and checking for collision at that point, we create a grid matching the horizontal extents of the workspace and add a collision mask which is the shadow of the object occupancy and occlusion voxel grid looking from a birds eye view. This mask is then convolved with the shadow of the object that is to be placed. 
The occupied pixel indices indicate collision-free placements and are converted to world coordinates.
%The indices of the `1' pixels of this last image are converted to world coordinates and thus used to enumerate all object placements that don't collide with discovered objects or hidden areas of the workspace. 
Object orientations can be enumerated by rotating the object shadow.
%To sample multiple object orientations, the process can be repeated, rotating the object shadow first, as desired.

%heuristic kernels:
%    1) definite occlusion (cone with object as base)
%    2) likely occlusion (cone with object as point)
%        - normalize weight to generate 2D heat map

%\begin{algorithm}
%\caption{$\placement{object}$}\label{alg:placement}
%\begin{algorithmic}[thpb]
%    \State $todo$
%\end{algorithmic}
%\end{algorithm}

\subsection{Target Object Prediction}
\label{subsec:target-prediction}
%\begin{algorithm}
%\caption{$\tt \addHiddenEdges{target,G}$}\label{alg:obj_predict}
%\begin{algorithmic}[thpb]
%    \State $\tt H \gets \emptyset$
%    \State $\tt total = 0$
%    \For{$\tt o \in Nodes(G) \setminus \{target\}$}
%        \State $\Tilde{O_{o}} = {\tt DirectOcclusionSpace(o)}$
%        \State ${\tt h = Heuristic}(\Tilde{O_{o}})$
%        \Comment{See \autoref{subsec:heuristics}}
%        \If{$\tt h \neq 0$}
%            \State $\tt H \gets H \cup \{(o,h)\}$
%            \State $\tt total \gets total + h$
%        \EndIf
%    \EndFor
%    \For{$\tt o,h \in H$}
%        \State $\tt AddEdge((target,o,\text{``hidden by''},\frac{h}{total}), G)$
%    \EndFor
%    %\While{$\tt target \notin \obd$}
%\end{algorithmic}
%\end{algorithm}
Intelligent object location prediction is achieved by applying a heuristic which ranks the pickable objects determined by the dependency graph (line 4 in \autoref{alg:pipeline}).
This is done by augmenting the dependency graph edges with a weight $p \in (0,1]$ estimating the probability for the relation $\obe_i \overset{r}{\longrightarrow} \obe_j$ to be true.
%This is done by augmenting the dependency graph edges to include a weight $p \in (0,1]$ which denotes an estimate of the probability for the given relation $\obe_i \overset{r}{\longrightarrow} \obe_j$ to be true.
%\edited{(What does the following sentence mean? What do you do product on?)}
To rank a pickable object, the sum of products of edge weights of all simple paths between the target and the object is computed. When sampling from the list of pickable objects, this rank is used as a probability weight.

For the ``below'' relation $p=1$ since object segmentation is assumed to be reliable. For the ``grasp blocked by'' relation p is equal to the fraction of total sampled grasps for which that object is in collision with the arm. Note, the weights of grasp blocking edges coming out of any object need not add to one (hence don't truly represent probabilities) since the arm could be colliding with multiple objects for any single grasp.
%A few heuristics are proposed next, but the general intuition is to bias towards larger direct occlusion regions so that the robot gains the most knowledge of the scene after every pick and place action.
For the ``hidden by'' relation, the goal is to encourage knowledge gain of the environment. This is done by normalizing the volume of the direct occlusion region of each stack of objects and assigning the inverse as the probability estimate that the target is hidden behind each stack. This heuristic biases the pipeline towards discovering large volumes of occluded workspace.

From an algorithmic point of view, there is technically no reason to normalize the output of the heuristic, however, representing the heuristics as probabilities is insightful since - without prior knowledge - the probability the hidden object to be in a larger volume is larger than the probability of it being in a smaller volume. Furthermore, modeling the dependency graph edges with probabilities as opposed to non-normalized weights is conducive for exploring future work which might seek to combine the probabilities based on the proposed volumetric-heuristics with priors based on the semantics of the objects involved or from an additional human instruction (see \autoref{sec:conclusion} work).

%\setcounter{subsecnumdepth}{0}
%\subsubsection*{\textbf{"Hidden by" Heuristics:}}
%\label{subsec:heuristics}
%\begin{enumerate}
%    \item The simplest uniform heuristic function is to return the volume of the given object's direct occlusion space. This seems like a reasonable heuristic, however, if a direct occlusion space happens to have large volume but is very narrow then it might not even be feasible for the target object to exist in that occlusion space.
%    \item A heuristic that solves the above problem is to sample target object placements within each occlusion region. That is, for $n$ trials, select a random stable object orientation and coordinate interior to the total occlusion space for all objects; check if the object placement puts it completely inside the total occlusion volume and increment a counter success counter $c$. If the object placement also intersect with the direct occlusion space for the object of interest then increment another counter $d$. The heuristic would then return $d/c$ the fraction of occluded samples that are at least partially occluded by the object of interest. This is also a simple heuristic but its non-deterministic.

%   {\color{red}
%   I assume we don't know what the target object is, and thus need to recognize it from the scene. If this is true, maybe this method isn't a good one?
%   Is the heuristics used in the code?
%   }
    
%   \item A deterministic heuristic function which is also robust to volume ambiguity relies on a similar technique to placement sampling described in the next section, \autoref{subsec:placement_sampling}. In brief, iterate through stable object orientations up to some resolution. For each stable orientation, generate a 2D kernel of the object as if looking from a birds eye view and perform a convolution with it on the negated, flattened occlusion space (projecting the voxels down to pixels and inverting the boolean values). The resulting array has 0 values corresponding to valid placements of the object interior to the occlusion space for the given orientation. Counting the zeros effectively returns the area of the 2D regions where the object would be completely hidden within the occlusion region for a given orientation. This heuristic would then return the total area for all stable object orientations.
%\end{enumerate}




\section{RESOLUTION COMPLETENESS}

\label{sec:completeness}
\newcommand{\algrand}{{\tt RAND-ACT}}
\newcommand{\algdep}{{\tt DEP-ACT}}

%A complete task planner should be able to find a sequence of actions to retrieve the object if it exists or return failure if no such sequence exists. Ultimately, task and motion planners cannot be formally complete since a continuous space of configurations can't be exhaustively explored. However, a planner could be probabilistically complete, meaning the probability of failing to converge to a solution approaches zero as time goes to infinity, or it could be resolution complete, meaning it is guaranteed to find a solution for a fine enough discretizing of the configuration space.


%The proposed planner will inherit one of these two forms of completeness depending on the underlying motion planner, grasp sampler, and object placement sampler. Namely, 
Given a (formally) complete motion planner (finds solution in finite time if exists),
%[Why do we need to include "a continuous space"? It seems redundent. Maybe we can say complete grasp sampler? Do we need to say both prob. complete and resol. complete? can we simply say resol. complete, and just say similarly the prob. complete holds?] 
a continuous space grasp sampler, and continuous placement sampler, the proposed pipeline would be $\tt PC$
%(converges in the limit to solution if exists). 
Given a $\tt RC$ motion planner, 
%(guaranteed to find solution if it exists for some $\epsilon$-discretization)
a discrete space grasp sampler, and discrete placement sampler, the proposed pipeline would be $\tt RC$.
%Since, truly complete motion planners don't exist, the proposed implementation in this work is resolution complete.

To show the $\tt PC$ or $\tt RC$ of the algorithm proposed in this work, a simpler version of the algorithm is analysed first. Without loss of generality, the actual proposed algorithm will be likewise proven complete.

Consider a much simpler algorithm that, at every iteration, tries to pick an object at random and, if is not the target object, subsequently place it randomly in the explored region of the workspace; call this \algrand.
%The termination condition for \algrand~is to pick the target object. 

\begin{blemma}
\algrand~is $\tt PC$ or $\tt RC$ (depending on the planning and sampling subroutines).
\end{blemma}
\begin{proof}
At every iteration, \algrand~attempts to perform a random action. Consequently, \algrand~executes a random walk on the space of all actions. Since pick and place actions are reversible, if a sequence of such valid actions exists, this algorithm will eventually perform it (or an augmented version of the sequence - i.e. placing an object back to where it was picked from) in the limit (or finite time for {\tt RC} subroutines).
\end{proof}

\begin{coro}
\autoref{alg:pipeline} is $\tt PC$ or $\tt RC$ (depending on the planning and sampling subroutines)).
\end{coro}
\begin{proof}
Indeed \autoref{alg:pipeline} is really a fancy implementation of \algrand. At every iteration, the dependency graph is used to identify (and heuristically rank) the currently pickable objects. One of these is chosen randomly for a pick and place action via the $\tt TryMoveOne$ subroutine. 
%\edited{(Not sure if I understand the following sentence...)} 
The $\tt MoveOrPlaceback$ subroutine acts as a fallback in case there are no new discovered placements found; it delays placement sampling till after the environment is re-sensed with the picked object moved out of the way.
Notice further, that even if the pickable objects are sampled weighted according to their ranking rather than uniformly, the action space is still explored entirely because each action has a positive probability of being sampled.
Thus, {\bf w.l.o.g. \autoref{alg:pipeline}} is $\tt PC$ or $\tt RC$ as well.
\end{proof} 

{\bf Failure Detection:} 
The implementation in this work uses the $\tt RC$ approach.
In addition to $\tt RC$, \autoref{alg:pipeline} takes a step closer towards achieving general completeness by actually detecting certain unsolvable cases within the completeness constraints of the motion and sampling subroutines.
The detectable unsolvable instances for which the algorithm will return failure in finite time are as follows.
\begin{myitem}
    \item No object can be grasped. This could happen if two objects each block the grasp of the other.
    \item No objects can be placed anywhere (except its current spot). This could happen in a highly cluttered scene where the only valid placement for each object is to just put it back where it was.
\end{myitem}
Thus, \autoref{alg:pipeline} has stronger guarantees than $\tt RC$ but is not formally complete since it may run forever by juggling two objects between two placements.

{\bf A Caveat:} A fundamental assumption in the argument presented is that actions are reversible. This is not always true in practice depending on the implementation of the sampling subroutines. And indeed, the implementation of the placement sampling process proposed in \autoref{subsec:placement_sampling} implies irreversible actions for scenes with stacked objects because it does not consider the possibility of re-stacking objects. Thus, the proposed pipeline (as implemented) is only resolution complete on the any sub-task where the objects are no longer stacked. Implementation of stacking actions is planned for future work.
%{
%\color{red}
%\subsection{Time Analysis}
%Since \autoref{alg:pipeline} is probabilistic complete, it doesn't have an absolute bound on its running time. However, it is possible to bound the expected running time. For resolution based sampling routines (as implemented in this paper) assume the maximum number of possible placements for all objects is $m$, with $n$ objects in total. 

%How to convert next part to expected runtime?

%Then by restricting the algorithm to only visit unexplored arrangement configurations, there are in total of $m^n$ configurations. Assuming that no repeated actions are required (e.g. to reveal occlusion or removing blocks), then the worst case scenario requires $O(m^n)$ pick and place actions, and $O((m^n)!)$ if the order of configurations matter.
%}

%Look more into:
%\url{https://arxiv.org/pdf/2003.11420.pdf}
%\url{https://arxiv.org/pdf/2003.10863.pdf}

\section{EXPERIMENTS}

%\newcommand{\algheur}{{\tt DEP-HACT}}
\begin{figure*}[thpb]
\vspace{0.1in}
\includegraphics[width=0.162\linewidth,trim={0 0 0 0},clip]{figures/real_exp/seq1.pdf}
\includegraphics[width=0.162\linewidth,trim={0 0 0 0},clip]{figures/real_exp/seq2.pdf}
\includegraphics[width=0.162\linewidth,trim={0 0 0 0},clip]{figures/real_exp/seq3.pdf}
\includegraphics[width=0.162\linewidth,trim={0 0 0 0},clip]{figures/real_exp/seq4.pdf}
\includegraphics[width=0.162\linewidth,trim={0 0 0 0},clip]{figures/real_exp/seq5.pdf}
\includegraphics[width=0.162\linewidth,trim={0 0 0 0},clip]{figures/real_exp/seq6.pdf}
\caption{Execution on the real robot %and integration with perception
: (a) Initial scene where the red bottle is hidden. (b) The robot moves the yellow bottle, which occludes the most space. (c) The robot moves the second yellow bottle, revealing the red bottle.
%, which is blocked by the green and blue bottles and not graspable,
(d and e) The robot moves the green and the blue bottles to reach the red bottle. (f) Target is now reachable.}
\label{fig:exp-real-data}
\vspace{-0.28in}
\end{figure*}

Simulated experiments and the real demonstrations are performed with the Yaskawa Motoman sda10f, with a robotiq 85 gripper attached on the right arm.

The simulated trials are randomly generated by picking random objects, dimensions, and collision-free placements within the specified workspace. 20 scenes with each of 6, 8, 10, 12, and 14 objects were used, all of which contain objects occluded from the camera. Each of the 100 trials is a unique scene. The target object in each scene was selected to be the hidden object with the most objects above it, if any.

All tested algorithms were given 20 minutes to run before being terminated. A trial run is considered successful only if the target object was retrieved within the time limit. Discovering but failing to pick up the object was still considered a failure. 

Comparisons of the success rate, number of actions for solved trials,
total run-time for solved trials,
and number of timed out trials
%and time till termination for unsolved trials
are shown in \autoref{fig:exp-data} for 3 algorithms.
The algorithms compared are the baseline random action approach (blue in figure), the proposed resolution complete pipeline without the object ranking heuristic (orange) and with it (green).
For the resolution complete approaches timing-out is not the only failure mode since they could detect certain infeasible problems (\autoref{sec:completeness}: Failure Detection)
%The system is robust to tackle hard instances as shown in Figure \ref{fig:exp-real-data}

The success rate of the resolution complete approaches is higher than that of the random baseline. 
Although its not directly apparent from the plotted results in \autoref{fig:exp-data}, the resolution complete approaches (as expected) always found a solution when the random baseline found a solution; however in one such trial the resolution complete approach without heuristic exceeded the 20 min time-limit.
It is also clear that the heuristic approach has better success than the non heuristic approach even though they are both complete. Looking at the data for timed-out experiments, it becomes clear that the increased success of the heuristic approach is due to timing out less frequently. This also coincides with the data showing that the heuristic approach overwhelmingly finds solutions faster and with fewer object manipulations. In fact, while the non-heuristic {\tt RC} approach started timing out linearly with increase in the number of objects, the heuristic approach had virtually no issue until the scenes got very cluttered with 14 objects.

It is clear that for all methods, success rate starts dropping off significantly at around 14 objects.
This marks the difficulty level for the given industrial Motoman robot and the workspace.
%This may not seem like too heavy a clutter but, if once observes the the size of the arm and the gripper in \autoref{fig:dep_graph}, it becomes clear that the industrial Motoman robot used for the simulated experiments and practical demonstration faced a more challenging task than the pristine, spacious, and structured environments it was designed for. 
A more compact robot with a streamlined end-effector (such as the ``bluction'' tool \cite{huang2022mechanical}) could scale to more cluttered scenes.%with the same pipeline.

\begin{figure}[thpb]
    \centering
    %\includegraphics[width=.49\linewidth]{figures/success_rate3.png}
    %\includegraphics[width=.49\linewidth]{figures/timed_out3.png}
    %\includegraphics[width=.49\linewidth]{figures/num_actions3.png}
    %\includegraphics[width=.49\linewidth]{figures/times3.png}
    \includegraphics[width=.99\linewidth]{figures/trial_results_succ.png}
    \caption{On top are shown the graphs of the: success rate (left) and number of timed out trials (right). On the bottom are the number of actions and the total runtime for the subset of trials in which all algorithms were successful.}
    \label{fig:exp-data}
    \vspace{-0.25in}
\end{figure}

\subsection{Integration with Perception \& Real Robot Demonstration}

The pipeline is directly transferable to scenarios on the real robot to retrieve a target red bottle from a cluttered shelf. Due to time constraints, a simple implementation of a perception system is used which only segments and detects colored cylinders without stacking. Despite the simplifications, a scene with significant object occlusion is still demonstrated with a successful retrieval.
The proposed pipeline (with heuristic) is run online and communicates with the robot controller and the RGBD camera for execution and sensing.
The camera extrinsic matrix is estimated by a classical robot-camera calibration procedure using ArUco markers \cite{tsai1989new}.
For object recognition, the perception component is implemented via plane fitting, DBSCAN segmentation \cite{ester1996density}, and cylinder fitting using Open3D \cite{zhou2018open3d}. 
The plane fitting algorithm extracts the boundaries of the workspace, which is used to construct the collision geometries in MoveIt \cite{chitta2016moveit}.
The inliers of each segmented cylinder are used to produce the segmentation mask for the RGBD image, which is used to label occlusion correspondence for each object.
To ensure safety, additional cubic collision geometries are added to the planning scene to avoid collisions between the robot and the camera.
\footnote{Videos can be found at \url{https://sites.google.com/scarletmail.rutgers.edu/occluded-obj-retrieval}}.
Extensive experiments of the proposed pipeline were not performed on the real robot but the demonstration presented was performed a few times and the pipeline was observed to have qualitatively similar performance as in simulated experiments; however, calibration and perception issues were observed to lead to pipeline failure.

\section{DISCUSSION}

\label{sec:conclusion}
%\subsection{Data Analysis}


%\subsection{Future Work}
It's worth mentioning that the physical execution accounts for over 60\% of time used for the trials. This shows that there could be room for performance improvement by performing scene perception asynchronously, since a lot can still be sensed while the robot is moving. Further performance improvement can be found by parallelizing the planning of picks and placements for multiple objects as well. 

While this work applies heuristics for selecting objects based on the occlusion volume, additional information regarding effective placements can also improve practical performance. In order to solve a larger variety of problems it would be useful to adapt the placement primitive to allow placing objects on top of others when there is limited space on the workspace surface. %%%KB: doesn't this affect completeness??

Another direction is to integrate the task planner with human instructions. For instance, it would be helpful to use human language to identify the target as well as influence the search at some regions over others. Additional heuristics can also be obtained from semantic reasoning of the scene when objects of the same category tend to be placed closer \cite{li2016act}.
Since current experiments only include simple geometries, such as cylinders and rectangular prisms, future work can investigate more complex objects where state-of-the-art perception algorithms are necessary. This would also be necessary for realistic human-robot integration.

%A conclusion section is not required. Although a conclusion may review the main points of the paper, do not replicate the abstract as the conclusion. A conclusion might elaborate on the importance of the work or suggest applications and extensions. 

%\addtolength{\textheight}{-12cm}   % This command serves to balance the column lengths
                                  % on the last page of the document manually. It shortens
                                  % the textheight of the last page by a suitable amount.
                                  % This command does not take effect until the next page
                                  % so it should come on the page before the last. Make
                                  % sure that you do not shorten the textheight too much.

%%%%%%%%%%%%%%%%%%%%%%%%%%%%%%%%%%%%%%%%%%%%%%%%%%%%%%%%%%%%%%%%%%%%%%%%%%%%%%%%



%%%%%%%%%%%%%%%%%%%%%%%%%%%%%%%%%%%%%%%%%%%%%%%%%%%%%%%%%%%%%%%%%%%%%%%%%%%%%%%%



%%%%%%%%%%%%%%%%%%%%%%%%%%%%%%%%%%%%%%%%%%%%%%%%%%%%%%%%%%%%%%%%%%%%%%%%%%%%%%%%
%\section*{APPENDIX}


%\section*{ACKNOWLEDGMENT}


%\bibliographystyle{format/myunsrtnat}
\bibliographystyle{ieeetr}
\bibliography{bib/HRI,bib/kostas,bib/stone,bib/occlusion}

\end{document}
