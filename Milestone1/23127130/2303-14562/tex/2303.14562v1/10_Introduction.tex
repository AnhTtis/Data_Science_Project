% what is the problem
Robotic manipulation has the potential of being integrated into daily lives of people, such as in household service areas \cite{wong2013manipulation, garrett2020online}. A useful skill for such household settings involves the retrieval of a target object from a confined and cluttered workspace, such as a fridge or a shelf, which may also require the rearrangement of other objects in the process. In this context, it is important to consider how to safely retrieve objects while minimizing the time spent or the amount of pick and place operations, so as to assist humans efficiently. 

\begin{figure}[thpb]
  %\vspace{-0.3in}
  \centering
  \includegraphics[width=\linewidth,
  trim={0 3cm 0 1cm},clip]{figures/real_setup.pdf}
  \includegraphics[width=0.465\linewidth,trim={13.9cm 0 12cm 12.9cm},clip]{figures/real_scene_voxel_grid.jpg}
  \includegraphics[width=0.52\linewidth,trim={4cm 7cm 5cm 2cm},clip]{figures/voxel_grid.png}
  \vspace{-0.3in}
  \caption{(Top) Setup for the real demonstration using an RGB-D sensor, robotiq gripper, and Yaskawa Motoman robot to retrieve the target bottle. (Bottom Left) The camera view in which objects are occluded. (Bottom Right) The corresponding voxel map.}
  \label{fig:real_setup}
  \vspace{-0.32in}
\end{figure}

One of the challenging aspects of these problems that requires explicit reasoning relates to heavy occlusions in the scene, as the sensor is often mounted on the robot and has limited visibility. These visibility constraints complicate the task planning process, as rearranging one object can limit placements for others and can introduce new occlusions. Moreover, real-world scenes in household setups are often unstructured and involve objects with complex spatial relationships, such as objects stacked on each other.%These relationships constrain which objects are reachable to the robot.
%action sequence that can be executed by the robot so as to ensure objects remain physically stable. 

Many previous efforts on object retrieval have focused on cases where blocking objects are extracted from the workspace \cite{dogar2014object, nam2021fast}, which simplifies the challenge as it does not require identifying temporary placement locations for the objects within the confined space. In-place rearrangement has been considered in some prior efforts \cite{ahn2021integrated}. While this prior method is efficient, it is not complete as it limits the reasoning on the largest object in the scene to analyze object traversability \cite{nam2021fast}. Alternatives use machine learning to guide the decision making \cite{bejjani2021occlusion, huang2022mechanical, price2019inferring}, which is an exciting direction but does not easily allow for performance guarantees, such as resolution completeness. Setups where object stacking arise have received less attention and most solutions that do consider stacking are dependent on machine learning for reasoning \cite{huang2022mechanicalstack, zhu2021hierarchical, kumar2022graph}. Some works have proposed testbeds \cite{liu2021ocrtoc} that can help evaluate solutions in this domain.

This work focuses on object retrieval in clutter where occlusions arise and objects may be  initially stacked under the assumption of known object models (e.g. \autoref{fig:dep_graph}). It aims at a theoretical understanding to show the algorithm has safety and $\tt RC$ guarantees. A heuristic variant improves the practical efficiency. Key features of the proposed $\tt RC$ framework are the following:

    %\item discretizing and identifying likely occlusion regions based on volume%through a variety of heuristics

\begin{myitem}
    \item it employs an adaptive dependency graph data structure inspired by solutions in object rearrangement with performance guarantees \cite{wang_uniform_2021} that express a larger variety of object relationships than previously considered (namely occlusion dependencies);
    \item it computes the occlusion volume of detected objects as a heuristic to inform the planning process;
    \item it reasons about the collision-free placement of objects in the confined workspace efficiently by utilizing a voxelized representation of the space;% in a way that minimizes future obstructions;
    \item it achieves $\tt RC$ (or $\tt PC$) depending on the implementation of the underlying sampling subroutines;
    \item it provides an early termination criterion when a solution cannot be found for the given resolution. 
\end{myitem}

Simulation results, using a model of a Yaskawa Motoman manipulator for rearranging objects on a tabletop as shown in Fig. \ref{fig:real_setup}, evaluate the proposed {$\tt RC$} framework against a baseline using random picking and placing operations. Both variants of the  {$\tt RC$} reason about object dependencies. One doesn't use heuristics and one is heuristically guided while retaining {$\tt RC$}. Both of the {$\tt RC$} approaches outperform the baseline. The heuristically guided solution is able to solve the problem more efficiently than the basic {$\tt RC$} solution. The integration of the proposed approach with perception, where an RGB-D sensor detects the objects as they are being moved, provides real robot demonstrations of safe object retrieval from a cluttered shelf.