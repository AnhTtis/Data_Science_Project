\label{sec:conclusion}
%\subsection{Data Analysis}


%\subsection{Future Work}
It's worth mentioning that the physical execution accounts for over 60\% of time used for the trials. This shows that there could be room for performance improvement by performing scene perception asynchronously, since a lot can still be sensed while the robot is moving. Further performance improvement can be found by parallelizing the planning of picks and placements for multiple objects as well. 

While this work applies heuristics for selecting objects based on the occlusion volume, additional information regarding effective placements can also improve practical performance. In order to solve a larger variety of problems it would be useful to adapt the placement primitive to allow placing objects on top of others when there is limited space on the workspace surface. %%%KB: doesn't this affect completeness??

Another direction is to integrate the task planner with human instructions. For instance, it would be helpful to use human language to identify the target as well as influence the search at some regions over others. Additional heuristics can also be obtained from semantic reasoning of the scene when objects of the same category tend to be placed closer \cite{li2016act}.
Since current experiments only include simple geometries, such as cylinders and rectangular prisms, future work can investigate more complex objects where state-of-the-art perception algorithms are necessary. This would also be necessary for realistic human-robot integration.