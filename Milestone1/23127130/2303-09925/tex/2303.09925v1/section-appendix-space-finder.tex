% % !TEX options=--shell-escape
% \documentclass[12pt]{iopart}
% \pdfoutput=1 % go for pdflatex

% % == Standard packages ==
% \usepackage{iopams}
% 
\ifdefined\siggraph
\usepackage{times}
\fi

%\usepackage{parskip}
\usepackage{color}
\usepackage{ifthen}
\usepackage{float}
\usepackage{alltt}
\usepackage{newlfont} % for Box
\usepackage{array}
\usepackage{wrapfig}
\usepackage{booktabs}
\usepackage{multirow}
\usepackage{amsfonts}
\usepackage{dsfont}
\usepackage[linesnumbered,ruled,vlined]{algorithm2e}
 
%%% Coloring the comment as blue
\newcommand\mycommfont[1]{\footnotesize\ttfamily\textcolor{blue}{#1}}
\SetCommentSty{mycommfont}
 



% % == Theorem environments ==
% \theoremstyle{definition}
% % A definition introduces a new concept rigorously:
% \newtheorem{definition}{Definition}[section]
% % A remark introduces tangential considerations: 
% \newtheorem{remark}{Remark}[section]
% % A theorem is a key result:
% \newtheorem{theorem}{Theorem}[section]
% % A proposition is a result requiring explicit proof:
% \newtheorem{proposition}[theorem]{Proposition}
% % A lemma is an intermediate result for a theorem or a proposition, requiring explicit proof:
% \newtheorem{lemma}[theorem]{Lemma}
% % An observation is a result not requiring explicit proof:
% \newtheorem{observation}[theorem]{Observation}
% % A corollary is a result directly following from a previous result:
% \newtheorem{corollary}[theorem]{Corollary}

% % == Macros ==
% \definecolor{purple}{rgb}{1, 0, 1}

\newcommand{\ie}{\emph{i.e.,}\xspace}
\newcommand{\eg}{\emph{e.g.,}\xspace}
\newcommand{\abr}{\emph{abbr.}\xspace}
\newcommand{\ea}{\emph{et al.}\xspace}
\newcommand{\gensync}{\emph{GenSync}\xspace}
\newcommand{\colosseum}{\emph{Colosseum}\xspace}
\newcommand{\srep}{\emph{SREP}\xspace} % Set Reconciliation Enhances
\newcommand{\srepsim}{\emph{SREPSim}\xspace}
% Propagation
\newcommand{\esrep}{\emph{E-SREP}\xspace}
\newcommand{\epsrep}{\emph{EP-SREP}\xspace}
\newcommand{\mesrep}{\emph{ME-SREP}\xspace}
\newcommand{\mempoolsync}{\emph{MempoolSync}}

\newcommand{\fref}[1]{Fig.~\ref{#1}}
\newcommand{\tref}[1]{Table~\ref{#1}}
\newcommand{\aref}[1]{Algorithm~\ref{#1}}
\newcommand{\procref}[1]{Procedure~\ref{#1}}
\newcommand{\sref}[1]{Section~\ref{#1}}
\newcommand{\lineref}[1]{line~\ref{#1}}
\newcommand{\appref}[1]{Appendix~\ref{#1}}

% Change \eqref
\LetLtxMacro{\originaleqref}{\eqref}
\renewcommand{\eqref}{Eq.~\originaleqref}

% Theorems and corollaries
\newcounter{theoremcount}
\setcounter{theoremcount}{0}
\DeclareRobustCommand{\theorem}[1]{%
  \refstepcounter{theoremcount}%
  \noindent\textit{\textbf{Theorem \thetheoremcount\label{theorem:#1}: }}%
}
\DeclareRobustCommand{\theoremref}[1]{Theorem~\ref{theorem:#1}}

\DeclareRobustCommand{\proof}{\emph{Proof:}\xspace}
\DeclareRobustCommand{\qqed}{\hfill$\blacksquare$}

\newcounter{corollcount}
\setcounter{corollcount}{0}
\DeclareRobustCommand{\coroll}[1]{%
  \refstepcounter{corollcount}%
  \noindent\textit{\textbf{Corollary \thecorollcount\label{coroll:#1}: }}%
}
\DeclareRobustCommand{\corollref}[1]{Corollary~\ref{coroll:#1}}

\newcounter{lemmacount}
\setcounter{lemmacount}{0}
\DeclareRobustCommand{\lemma}[1]{%
  \refstepcounter{lemmacount}%
  \noindent\textit{\textbf{Lemma \thelemmacount\label{lemma:#1}: }}%
}
\DeclareRobustCommand{\lemmaref}[1]{Lemma~\ref{lemma:#1}}

\newcounter{definitioncount}
\setcounter{definitioncount}{0}
\DeclareRobustCommand{\definition}[1]{%
  \refstepcounter{definitioncount}%
  \noindent\textit{\textbf{Definition \thedefinitioncount\label{definition:#1}: }}%
}
\DeclareRobustCommand{\defref}[1]{Definition~\ref{definition:#1}}

%notes of different authors
\newif\ifnotes
\notestrue
\notesfalse

\newif\ifdiff
\difftrue
\difffalse

\newcommand{\anote}[1]{\ifnotes $\ll$\textsf{\textcolor{purple}{Ari: {#1}}}$\gg$ \fi}
\newcommand{\nnote}[1]{\ifnotes $\ll$\textsf{\textcolor{orange}{Novak: {#1}}}$\gg$ \fi}
\newcommand{\diff}[1]{\ifdiff\textcolor{orange}{#1}\else#1\fi}

%%% Local Variables:
%%% mode: latex
%%% TeX-master: "main"
%%% End:

% \newcommand{\TODO}[1]{{\marginpar{\color{gray}\textsf{[TODO]}} \color{gray}\textsf{[#1]}}}
% \newcommand{\NOTE}[1]{{\marginpar{\color{gray}\textsf{[NOTE]}} \color{gray}\textsf{[#1]}}}

% \begin{document}

% % == Dummy Sections ==

% \section{Introduction}
% \label{section: introduction}

% \section{Causal orders}
% \label{section:causal-orders}

% \section{Spaces of input histories}
% \label{section:spaces}

% \section{The topology of causality}
% \label{section:topology-causality}

% \section{The geometry of causality}
% \label{section:geometry-causality}

% \section{Conclusions}
% \label{section:conclusions}

% % == Dummy Biblio ==

% % \bibliographystyle{iopart-num}
% % \bibliography{biblio}
% % \nocite{*}

% \appendix

% \section{Causally Complete Spaces on 3 Events with Binary Inputs}
% \label{appendix:all-spaces-ABC}

% \newpage

% === COMMENT ABOVE BEFORE COMPILING MAIN FILE ===


\section{Algorithm to Find Causally Complete Spaces}
\label{appendix:space-finder-algo}

In the sub-sections below, we break down the algorithm that was used to enumerate the hierarchies of causally complete spaces on binary inputs that were presented in this work.
Each sub-section start by presenting a full listing for a block of related code, typically a single function or class method, and then proceeds to explain how the various parts work.

The algorithm is limited to the binary case, for added efficiency, but can be easily extended to arbitrary input combinations.
The algorithm can also be extended to find causally complete sub-spaces of a given space, but this requires non-trivial modification (because of the layered approach, which presumes that all histories in a layer of recursion have same-sized domains).

All code is written for Python 3.10, using PEP484 type hints.
It has been checked with the Mypy static type-checker and it has been linted with with Pylint.
% The source code is part of the \href{https://github.com/hashberg-io/qcaus}{qcaus} library.

\subsection{Imports and utilities}

Imports for the whole file, all from standard libraries.

\begin{minted}[firstnumber=1]{python}
from collections import deque
from collections.abc import Collection, Iterable, Iterator, Mapping, Set
from functools import cache
from itertools import chain, combinations, islice, permutations, product
from logging import Logger
from math import ceil
from numbers import Number
import sys
from time import perf_counter
from typing import (Any, BinaryIO, cast, Literal, Optional,
                    Sequence, TypeVar, Union)
\end{minted}

\noindent Below is a snippet for recursive memory size calculation of Python objects, courtesy of a StackOverflow answer.

\begin{minted}[firstnumber=last]{python}
def getsize(obj_0: Any) -> int:
    """
    Recursively iterate to sum size of object & members.
    Courtesy: https://stackoverflow.com/questions/449560/
              how-do-i-determine-the-size-of-an-object-in-python#answer-30316760
    """
    _seen_ids = set()
    def inner(obj: Any) -> int:
        obj_id = id(obj)
        if obj_id in _seen_ids:
            return 0
        _seen_ids.add(obj_id)
        size = sys.getsizeof(obj)
        if isinstance(obj, (str, bytes, Number, range, bytearray)):
            pass # bypass remaining control flow and return
        elif isinstance(obj, (tuple, list, Set, deque)):
            size += sum(inner(i) for i in obj)
        elif isinstance(obj, Mapping) or hasattr(obj, 'items'):
            size += sum(inner(k) + inner(v) for k, v in getattr(obj, 'items')())
        # Check for custom object instances - may subclass above too
        if hasattr(obj, '__dict__'):
            size += inner(vars(obj))
        if hasattr(obj, '__slots__'): # can have __slots__ with __dict__
            size += sum(inner(getattr(obj, s))
                        for s in obj.__slots__ if hasattr(obj, s))
        return size
    return inner(obj_0)
\end{minted}

\noindent The \mintinline{python}{powerset} function takes an iterable of elements some generic type \mintinline{python}{_T_co} and iterates through all possible subsets of those elements, yielded as \mintinline{python}{tuple}s of elements (rather than sets).

\begin{minted}[firstnumber=last]{python}
_T_co = TypeVar("_T_co")
def powerset(iterable: Iterable[_T_co]) -> Iterable[tuple[_T_co, ...]]:
    """
    Courtesy: https://docs.python.org/3/library/itertools.html#itertools-recipes
    powerset([1,2,3]) --> () (1,) (2,) (3,) (1,2) (1,3) (2,3) (1,2,3)
    """
    s = list(iterable)
    return chain.from_iterable(combinations(s, r) for r in range(len(s)+1))
\end{minted}


\subsection{Bitvectors}

We encode sets of non-negative in a memory-efficient way, as bitvectors.
Python natively supports bitwise operations on \mintinline{python}{int} objects, so we use those for our bitvector implementation.
The \mintinline{python}{bitvec} function takes an iterable of non-negative integers and returns the bitvector corresponding to the set of its elements, where $x_1,...,x_n \geq 0$ are encoded as $\sum_{j=1}^n 2^{x_j} \geq 0$.

\begin{minted}[firstnumber=last]{python}
Bitvec = int
""" Type alias for a bitvector, as a non-negative integer. """

def bitvec(elements: Iterable[int]) -> Bitvec:
    """
    Creates a bitvector representing the set of given
    non-negative integers.
    """
    v = 0
    for el in elements:
        assert el >= 0, f"Found negative integer {el}."
        v |= 2**el
    return v
\end{minted}

\noindent As an example, below is the encoding of the set $\{1, 3, 5, 8\}$ as a bitvector, coinciding with the number $128 = 2^8+2^5+2^3+2^1$ (bits read right-to-left).

\begin{minted}[linenos=false]{python}
bv = bitvec({1, 3, 5, 8})
print(f"{bv = } = {bv:b}")
# bv = 298 = 100101010
\end{minted}

\noindent The \mintinline{python}{sub} function takes the bitvector representations of two sets and returns the bitvector corresponding to their set difference.

\begin{minted}[firstnumber=60]{python}
def sub(u: Bitvec, v: Bitvec) -> Bitvec:
    """
    Returns the bitvector containing the elements that are
    in ``u`` but not in ``v``.
    """
    assert u >= 0, "Invalid bitvector."
    assert v >= 0, "Invalid bitvector."
    return u^(u&v)
\end{minted}

\noindent The \mintinline{python}{is_subset} function takes the bitvector representations of two sets are returns a \mintinline{python}{bool} value (\mintinline{python}{True} or \mintinline{python}{False}) stating whether the first set is a subset of the second set.

\begin{minted}[firstnumber=last]{python}
def is_subset(u: Bitvec, v: Bitvec) -> bool:
    """
    Returns whether the bitvector ``u`` is a subset of
    the bitvector ``v``.
    """
    return u == u&v
\end{minted}

\noindent The \mintinline{python}{iter_bitvec} function takes the bitvector representation of a set and iterates through the elements in the set, in strictly increasing order.

\begin{minted}[firstnumber=last]{python}
def iter_bitvec(u: Bitvec) -> Iterator[int]:
    """
    Iterates over the elements in a bitvector.
    """
    assert u >= 0, "Invalid bitvector."
    el = 0
    while u > 0:
        u, b = divmod(u, 2)
        if b == 1:
            yield el
        el += 1
\end{minted}

\noindent The \mintinline{python}{bitvec2set} function takes the bitvector representation of a set and returns the corresponding \mintinline{python}{set} instance.

\begin{minted}[firstnumber=last]{python}
def bitvec2set(u: Bitvec) -> set[int]:
    """
    Turns a bitvector into the corresponding set of non-negative integers.
    """
    return set(iter_bitvec(u))
\end{minted}


\subsection{Histories}

The code below supports histories up to 26 events, labelled by uppercase Latin alphabet letters (\mintinline{python}{"A"} to \mintinline{python}{"Z"}), with binary input values (\mintinline{python}{0} or \mintinline{python}{1}).
A ``history item'' is an event-input pair and a history is a set of such pairs, encoded as a bitvector.

\begin{minted}[firstnumber=last]{python}
Event = Literal["A", "B", "C", "D", "E", "F", "G", "H", "I", "J", "K", "L", "M",
                "N", "O", "P", "Q", "R", "S", "T", "U", "V", "W", "X", "Y", "Z"]
""" Type alias for events (single char "A" to "Z"). """

InputValue = Literal[0, 1]
""" Type alias for input/output values (binary 0 or 1). """

HistoryItem = tuple[Event, InputValue]
""" Type alias for an item (event-value pair) in an (ext) input history. """

HistoryItems = tuple[HistoryItem, ...]
""" Type alias for a variable length tuple of history items. """

History = Bitvec
"""
    Type alias for an history, a bitvector of integers indicating its items.
"""
\end{minted}

\noindent The functions \mintinline{python}{event2idx} and \mintinline{python}{idx2event} convert between events and non-negative integers in the range $0-25$:
\begin{itemize}
    \item \mintinline{python}{0} $\leftrightarrow$ \mintinline{python}{"A"}
    \item \mintinline{python}{1} $\leftrightarrow$ \mintinline{python}{"B"}
    \item ...
    \item \mintinline{python}{25} $\leftrightarrow$ \mintinline{python}{"Z"}
\end{itemize}

\begin{minted}[firstnumber=last]{python}
@cache
def event2idx(e: Event) -> int:
    """ Turns an event into the corresponding index. """
    idx = ord(e)-ord("A")
    return idx

@cache
def idx2event(idx: int) -> Event:
    """ Turns a index into the corresponding event. """
    assert 0 <= idx <= 25, "Events indexes be 0-25."
    return cast(Event, chr(ord("A")+idx))
\end{minted}

\noindent The functions \mintinline{python}{item2idx} and \mintinline{python}{idx2item} convert between history items (event-input pairs) and non-negative integers in the range $0-51$:
\begin{itemize}
    \item \mintinline{python}{0} $\leftrightarrow$ \mintinline{python}{("A", 0)}
    \item \mintinline{python}{1} $\leftrightarrow$ \mintinline{python}{("A", 1)}
    \item \mintinline{python}{2} $\leftrightarrow$ \mintinline{python}{("B", 0)}
    \item \mintinline{python}{3} $\leftrightarrow$ \mintinline{python}{("B", 1)}
    \item ...
    \item \mintinline{python}{51} $\leftrightarrow$ \mintinline{python}{("Z", 1)}
\end{itemize}

\begin{minted}[firstnumber=last]{python}
@cache
def item2idx(item: HistoryItem) -> int:
    """ Turns an history item into the corresponding index. """
    e, v = item
    return 2*event2idx(e)+v

@cache
def idx2item(idx: int) -> HistoryItem:
    """ Turns a index into the corresponding history item. """
    assert 0 <= idx <= 51, "Item indexes must be 0-51."
    return (idx2event(idx//2), cast(InputValue, idx%2))
\end{minted}

\noindent The \mintinline{python}{history} function creates a history bitvector from an event-input mapping or an iterable of history items, by first encoding the event-input pairs to non-negative integers using \mintinline{python}{item2idx}.

\begin{minted}[firstnumber=last]{python}
def history(h_items: Union[Mapping[Event, InputValue],
                           Iterable[HistoryItem]]) -> History:
    """
    Creates a history from an event-value mapping
    or an iterable of event-value pairs.
    """
    if isinstance(h_items, Mapping):
        h_items = h_items.items()
    return bitvec(item2idx(item) for item in h_items)
\end{minted}

\noindent For example, the history $\hist{A/0, B/1, D/1}$ on 4 events $\ev{A}, \ev{B}, \ev{C}, \ev{D}$ is first turned into the set \mintinline{python}{{("A", 0), ("B", 1), ("D", 1)}} of event-input, then turned into the set \mintinline{python}{{0, 3, 7}} of corresponding indices, and finally encoded into a bitvector, as the number $137 = 2^7+2^3+2^0$.

\begin{minted}[linenos=false]{python}
h = history({"A": 0, "B": 1, "D": 1})
print(f"{h = } = {h:b}")
# h = 137 = 10001001
\end{minted}

\noindent Given a history bitvector, the \mintinline{python}{history2items} function returns the corresponding immutable sequence (a \mintinline{python}{tuple}) of history items, in increasing order.

\begin{minted}[firstnumber=138]{python}
def history2items(h: History) -> HistoryItems:
    """
    Turns a history into the tuple of its items, in order of increasing index.
    """
    assert h >= 0, "Invalid history."
    return tuple(idx2item(idx) for idx in iter_bitvec(h))
\end{minted}

\noindent Given a history bitvector, the \mintinline{python}{history2items} function returns the corresponding event-input dictionary.

\begin{minted}[firstnumber=last]{python}
def history2dict(h: History) -> dict[Event, InputValue]:
    """
    Turns a history into the corresponding event-value mapping,
    with events in order of increasing index.
    """
    d: dict[Event, InputValue] = {}
    for idx in iter_bitvec(h):
        e, v = idx2item(idx)
        assert e not in d, f"History has multiple values for event {e}."
        d[e] = v
    return d
\end{minted}

\noindent Given a history bitvector, the \mintinline{python}{dom} function returns the \mintinline{python}{frozenset} of events in the history's domain.

\begin{minted}[firstnumber=last]{python}
def dom(h: History) -> frozenset[Event]:
    """
    Returns the domain of a history, as a frozenset of events.
    """
    assert h >= 0, "Invalid history."
    d = set()
    idx = 0
    while h > 0:
        h, b = divmod(h, 2)
        if b != 0:
            d.add(idx2item(idx)[0])
        idx += 1
    return frozenset(d)
\end{minted}

\noindent Given a history bitvector, the \mintinline{python}{domsize} function returns the size of the history's domain.

\begin{minted}[firstnumber=last]{python}
def domsize(h: History) -> int:
    """
    Returns the size of the domain of a history.
    """
    assert h >= 0, "Invalid history."
    size = 0
    idx = 0
    while h > 0:
        h, b = divmod(h, 2)
        if b != 0:
            size += 1
        idx += 1
    return size
\end{minted}

\noindent Given a history bitvector, the \mintinline{python}{history_sort_key} function returns a corresponding sorting key, for use with the builtin \mintinline{python}{sorted} function.
Using this key, histories are first sorted by length, and then by the items they contain (according to the sorting rules for \mintinline{python}{tuple}).

\begin{minted}[firstnumber=last]{python}
def history_sort_key(h: History) -> tuple[int, HistoryItems]:
    """
    Returns a sorting key for histories,
    sorted first by length and then by content.
    """
    h_items = history2items(h)
    return (len(h_items), tuple(sorted(h_items)))
\end{minted}

\noindent The \mintinline{python}{max_histories} function returns the sequence of maximal extended input histories on a given number of events.
The events are selected starting from \mintinline{python}{"A"}, in order.

\begin{minted}[firstnumber=last]{python}
def max_histories(num_events: int) -> tuple[History, ...]:
    """
    Returns the sequence of maximal extended input histories on a given number
    of events. Events are selected in order, starting from ``"A"``.
    """
    dom_events = tuple(idx2event(idx) for idx in range(num_events))
    return tuple(
        history(zip(dom_events, input_choices))
        for input_choices in product(cast(Sequence[InputValue], [0,1]),
                                     repeat=num_events)
    )
\end{minted}

\noindent Given a history bitvector, the \mintinline{python}{child_histories} function returns the immutable sequence of bitvectors for the history's children, the histories obtained by removing each individual event in the history's domain in turn.

\begin{minted}[firstnumber=last]{python}
def child_histories(h: History) -> tuple[History, ...]:
    """
    Given a history, returns the sequence of child histories,
    obtained by removing a single event (in turn).
    """
    h_dict = history2dict(h)
    if len(h_dict) <= 1:
        return tuple()
    return tuple(
        history((e, v) for e, v in h_dict.items() if e != removed_e)
        for removed_e in reversed(h_dict)
    )
\end{minted}

\noindent Given a sequence of history bitvectors, the \mintinline{python}{sub_histories} function returns the immutable sequence of bitvectors for all sub-histories for the given histories, in the order discovered by breadth-first search of history children.

\begin{minted}[firstnumber=last]{python}
def sub_histories(hs: Sequence[History]) -> tuple[History, ...]:
    """
    Given an ordered set of histories, returns the sequence of sub-histories,
    in order of breadth-first discovery and including the initial histories.
    """
    visited = set()
    q = deque(hs)
    sub_hs = []
    while q:
        h = q.popleft()
        if h not in visited:
            sub_hs.append(h)
            visited.add(h)
        for k in child_histories(h):
            if k not in visited:
                q.append(k)
    return tuple(sub_hs)
\end{minted}


\subsection{Event-input Permutation}

Given a sequence of events $\underline{e} = (e_1,...,e_n)$ without repetition, the elements
\[
\underline{g} = (g_E,g_{I_{e_1}},...,g_{I_{e_n}}) \in S(E) \times \prod_{j=1}^n S(\{0,1\})
\]
of the event-input permutation group on $E = \{e_1,...,e_n\}$ with binary inputs are uniquely represented by the element-wise action of $g_E$ on the sequence of events $\underline{e}$ and the action of each $g_{I_{e_j}}$ on the input value $0$:
\[
\Bigg(
    \Big(g_E(e_1),...,g_E(e_n)\Big),\,
    \Big(g_{I_{e_1}}(0),...,g_{I_{e_n}}(0)\Big)
\Bigg)
\]

\begin{minted}[firstnumber=last]{python}
PermGroupEl = tuple[Sequence[Event], Sequence[InputValue]]
"""
Type alias for elements of the event-value permutation group on a given ordered
set of events (as a unique sequence, or sequence without repetitions).
Group elements are represented by pairs:

1. an ordered set of permuted events (represented by a unique sequence),
   defining the permutation action on events
2. a sequence of values, one for each event,
   defining the permutation action on values (by bitwise XOR)
"""
\end{minted}

\noindent The \mintinline{python}{iter_perm_group} function iterates through the permutation group elements for a given sequence of unique events.

\begin{minted}[firstnumber=last]{python}
def iter_perm_group(events_list: Sequence[Event]) -> Iterator[PermGroupEl]:
    """
    Iterates through the elements of the event-value permutation group on the
    given events to permute.
    """
    num_events = len(events_list)
    assert len(set(events_list)) == num_events, "Events must not be repeated."
    for events_perm in permutations(events_list):
        for value_perm in product(cast(Sequence[InputValue], [0,1]),
                                  repeat=num_events):
            yield (events_perm, value_perm)
\end{minted}

\noindent Given a history bitvector and an event-input permutation group element, the \mintinline{python}{permute_history} function returns the bitvector for the permuted history.

\begin{minted}[firstnumber=last]{python}
def permute_history(h: History, g: PermGroupEl) -> History:
    """
    Given a history, returns the history obtained by applying the given
    element of the event-value permutation group.
    """
    h_dict = history2dict(h)
    events_perm, value_perm = g
    return history({
        e_perm: cast(InputValue, h_dict[e]^value_perm[e_idx])
        for e_idx, (e, e_perm) in enumerate(zip(sorted(events_perm),
                                                events_perm))
        if e in h_dict
    })
\end{minted}

\noindent Given a history bitvector and an iterable of event-input permutation group elements, the \mintinline{python}{history_perms} function efficiently iterates through all group elements and bitvectors of the corresponding permuted histories.

\begin{minted}[firstnumber=last]{python}
def history_perms(h: History,
                  perm_group: Iterable[PermGroupEl]
                  ) -> Iterator[tuple[PermGroupEl, History]]:
    """
    Given a history and an iterable of event-value permutation group elements,
    iterates through all pairs of permutation group element and corresponding
    permuted history.
    """
    h_dict = history2dict(h)
    for events_perm, value_perm in perm_group:
        yield (events_perm, value_perm), history({
            e_perm: cast(InputValue, h_dict[e]^value_perm[e_idx])
            for e_idx, (e, e_perm) in enumerate(zip(sorted(events_perm),
                                                    events_perm))
            if e in h_dict
        })
\end{minted}

\noindent Given a history bitvector and an iterable of event-input permutation group elements, the \mintinline{python}{history_perms} function returns the sequence of group elements stabilizing the history, in the order they are encountered and without repetition.

\begin{minted}[firstnumber=last]{python}
def stabiliser(h: History,
               perm_group: Iterable[PermGroupEl]) -> Sequence[PermGroupEl]:
    """
    Given a history and an iterable of event-value permutation group elements,
    returns the stabiliser subgroup for the history,
    as sequence of group elements.
    """
    orbit = list(history_perms(h, perm_group))
    return tuple(perm for perm, h_perm in orbit if h_perm == h)
\end{minted}


\subsection{Utilities for the \mintinline{python}{SpaceFinder} Class}
\label{appendix:space-finder-algo:section:utilities}


The \mintinline{python}{SpaceFinder} class encapsulates the data and code necessary to search for causally complete spaces on a given number of events.
Some of the required functionality is independent of the search state, and is implemented by the utilities functions below.
Given a collection of history bitvectors (as a sequence, without repetition), the \mintinline{python}{parents} function computes the mapping of histories to their sets of parents in the collection.

\begin{minted}[firstnumber=288]{python}
def parents(hs: Sequence[History]) -> dict[History, frozenset[History]]:
    """
    Given an ordered set of histories, returns the mapping of
    each history to the set of its parent histories (those of
    which they are a child history) in the set.
    """
    ps: dict[History, set[History]] = {h: set() for h in hs}
    for h in hs:
        for k in sorted(child_histories(h), key=history_sort_key):
            if k not in ps:
                ps[k] = set()
            ps[k].add(h)
    return {h: frozenset(ks) for h, ks in ps.items()}
\end{minted}

\noindent The \mintinline{python}{time_str} function provides a friendly representation of a time quantity, for use in search status updates.

\begin{minted}[firstnumber=last]{python}
def time_str(time: Union[int, float]) -> str:
    """ Formats a given time value in seconds, for printing. """
    # pylint: disable = too-many-return-statements
    assert time >= 0
    if time == 0:
        return "0s"
    if time < 1e-6:
        return f"{time*1e9:.2f}ns"
    if time < 1e-3:
        return f"{time*1e6:.2f}us"
    if time < 1:
        return f"{time*1e3:.2f}ms"
    if time < 60:
        return f"{time:.2f}s"
    time = int(time)
    if time < 60*60:
        return f"{time//60}m{time%60}s"
    if time < 24*3600:
        return f"{time//3600}h{(time%3600)//60}m"
    return f"{time//86400}d{(time%86400)//3600}h{(time%3600)//60}m"
\end{minted}

\noindent The \mintinline{python}{memory_str} function provides a friendly representation of a memory quantity, for use in search status updates.

\begin{minted}[firstnumber=last]{python}
def memory_str(mem: int) -> str:
    """ Formats a given memory value in bytes, for printing. """
    assert mem >= 0
    if mem < 1024:
        return f"{mem}B"
    if mem < 1024**2:
        return f"{mem/1024:.2f}KiB"
    if mem < 1024**3:
        return f"{mem/(1024**2):.2f}MiB"
    return f"{mem/(1024**3):.2f}GiB"
\end{minted}


\subsection{The \mintinline{python}{SpaceFinder} Class}

The \mintinline{python}{SpaceFinder} class encapsulates the data and code necessary to search for causally complete spaces on a given number of events.
This Appendix presents a simplified version of the class, with no state serialisation and no symmetry optimisaton of the top-level children subsets: for the advanced version, used in the search for causally complete spaces on 4 events, see \ref{appendix:space-finder-algo-adv} (p.\pageref{appendix:space-finder-algo-adv}).
To keep line numberings consistent with the advanced version, code omitted in the simplified version has been replaced by blank comment lines.

An object of the class is instantiated by passing a number \mintinline{python}{num_events} of events---exposed by the homonymous property---and the constructor pre-computes certain data for use during the search.
The optional \mintinline{python}{verbose} keyword argument (kwarg) determines whether status updates will be printed/logged;
the optional \mintinline{python}{update_period} kwarg determines the frequency of the updates (as a minimum number of equivalence classes discovered in between updates);

\begin{minted}[linenos=false]{python}
class SpaceFinder:
    def __init__(self, num_events: int, *,
                 verbose: bool = True,
                 update_period: Optional[int] = None) -> None:
        ...
    @property
    def num_events(self) -> int:
        ...
    ...
\end{minted}

\noindent Calling the \mintinline{python}{blank_state} method initialises the finder for a new search.

\begin{minted}[linenos=false]{python}
class SpaceFinder:
    ...
    def blank_state(self) -> None:
        ...
    ...
\end{minted}

\noindent After the search state has been set---to a blank state or to the state loaded from a file---the \mintinline{python}{find_eq_classes} method can be called to start the search, which will run until completion.

\begin{minted}[linenos=false]{python}
class SpaceFinder:
    ...
    def find_eq_classes(self) -> None:
        ...
    ...
\end{minted}

\noindent Below is a minimal example searching for causally complete spaces on 2 events: 

\begin{minted}[linenos=false]{python}
finder2 = SpaceFinder(2)
finder2.blank_state()
finder2.find_eq_classes()
\end{minted}

\noindent Basic benchmarking quantities are made available through public properties:
the \mintinline{python}{time_elapsed} property exposes the time elapsed since \mintinline{python}{find_eq_classes} was called;
the \mintinline{python}{memsize} property exposes an estimate of the amount of memory currently occupied by temporary data structures and discovered spaces;
the \mintinline{python}{perc_completed} property exposes an estimate of the percentage of search space explored so far.

\begin{minted}[linenos=false]{python}
class SpaceFinder:
    ...
    @property
    def time_elapsed(self) -> float:
        ...
    @property
    def memsize(self) -> int:
        ...
    @property
    def perc_completed(self) -> float:
        ...
    ...
\end{minted}

\noindent Below is the final portion of the output for the \mintinline{python}{finder2} 2-event example presented above, showing the time, memory and completion metrics as the algorithm progressed.

\begin{minted}{text}
[...]
      time       spaces    eq. cls     memory  completed
    2.75ms            4          1   14.20KiB   16.6667%
    2.87ms            5          2   14.22KiB   33.3333%
    2.96ms            5          2   14.22KiB   66.6667%
    3.04ms            5          2   14.22KiB   83.3333%
    3.13ms            7          3   14.25KiB  100.0000%
Found 7 spaces in 3 equivalence classes.
\end{minted}

\noindent The search process progressively uncovers equivalence classes of causally complete spaces under event-input permutation symmetry.
For reasons of memory efficiency, a single representative---the bitvector for the set of input histories in the space---is stored for each equivalence class: these representatives can be iterated using the \mintinline{python}{iter_eq_classes} property (with no guarantees on order), while their number is efficiently exposed by the \mintinline{python}{num_eq_classes} property.
The search process also efficiently computes the total number of causally complete spaces in the equivalence classes discovered so far, which is exposed by the \mintinline{python}{num_spaces} property.
The \mintinline{python}{iter_spaces} property, finally, iterates through all causally complete spaces discovered so far, by iterating through all equivalence classes and iterating over all spaces in each class by applying the event-input permutation group elements to the representative (without repetition of spaces); for each space, a pair is yielded, with the equivalence class representative as the first element and the permuted space as the second element.

\begin{minted}[linenos=false]{python}
class SpaceFinder:
    ...
    @property
    def num_eq_classes(self) -> int:
        ...
    @property
    def iter_eq_classes(self) -> Iterator[HistorySet]:
        ...
    @property
    def num_spaces(self) -> int:
        ...
    @property
    def iter_spaces(self) -> Iterator[tuple[HistorySet, HistorySet]]:
        ...
    ...
\end{minted}

\noindent In the \mintinline{python}{finder2} 2-event example presented above, there are 7 causally complete spaces in 3 equivalence classes.
Below we list both the equivalence class representatives and all spaces, using colours to highlight spaces in the same equivalence class.

\begin{minted}[linenos=false]{python}
print(f"Eq Classes = {list(finder2.iter_eq_classes)}")
print(f"All Spaces = {[space for _, space in finder2.iter_spaces]}")
\end{minted}

\begin{minted}[escapeinside=||]{text}
Eq Classes = [|\colorbox{evred}{1362}|, |\colorbox{evgreen}{1638}|, |\colorbox{evblue}{278}|]
All Spaces = [|\colorbox{evred}{1362, 820, 1558, 358}|, |\colorbox{evgreen}{1638, 1904}|, |\colorbox{evblue}{278}|]
\end{minted}

\noindent To understand which spaces were discovered, we look at the bitvector for each space (from \mintinline{python}{finder2.iter_spaces}), we extract the \mintinline{python}{int} representation of the bitvectors for the histories within (using \mintinline{python}{iter_bitvec}), and we then extract the event-input pairs for each history from the corresponding bitvector (using \mintinline{python}{history2dict}).

\begin{minted}[linenos=false]{python}
curr_eq_class_repr = None
for eq_class_repr, space in finder2.iter_spaces:
    if eq_class_repr != curr_eq_class_repr:
        curr_eq_class_repr = eq_class_repr
        print(f"Eq. class of space {eq_class_repr}:")
    print(f"  Input histories for space {space}:")
    for h in iter_bitvec(space):
        print(f"    {history2dict(h)}")
\end{minted}

\noindent The first equivalence class discovered contains the 4 spaces in the middle layer of the hierarchy of causally complete spaces on 2 events with binary inputs.

\begin{minted}[escapeinside=||]{text}
Eq. class of space |\colorbox{evred}{1362}|=10101010010:
  Input histories for space |\colorbox{evred}{1362}|=10101010010:
     1=0001 -> {'A': 0}
     4=0100 -> {'B': 0}
     6=0110 -> {'A': 1, 'B': 0}
     8=1000 -> {'B': 1}
    10=1010 -> {'A': 1, 'B': 1}
  Input histories for space |\colorbox{evred}{820}|=01100110100:
     2=0010 -> {'A': 1}
     4=0100 -> {'B': 0}
     5=0101 -> {'A': 0, 'B': 0}
     8=1000 -> {'B': 1}
     9=1001 -> {'A': 0, 'B': 1}
  Input histories for space |\colorbox{evred}{1558}|=11000010110:
     1=0001 -> {'A': 0}
     2=0010 -> {'A': 1}
     4=0100 -> {'B': 0}
     9=1001 -> {'A': 0, 'B': 1}
    10=1010 -> {'A': 1, 'B': 1}
  Input histories for space |\colorbox{evred}{358}|=00101100110:
     1=0001 -> {'A': 0}
     2=0010 -> {'A': 1}
     5=0101 -> {'A': 0, 'B': 0}
     6=0110 -> {'A': 1, 'B': 0}
     8=1000 -> {'B': 1}
\end{minted}

\noindent The second equivalence class discovered contains the 2 spaces in the top layer of the hierarchy of causally complete spaces on 2 events with binary inputs, i.e. the spaces for the 2 total orders.

\begin{minted}[escapeinside=||]{text}
Eq. class of space |\colorbox{evgreen}{1638}|=11001100110:
  Input histories for space |\colorbox{evgreen}{1638}|=11001100110:
     1=0001 -> {'A': 0}
     2=0010 -> {'A': 1}
     5=0101 -> {'A': 0, 'B': 0}
     6=0110 -> {'A': 1, 'B': 0}
     9=1001 -> {'A': 0, 'B': 1}
    10=1010 -> {'A': 1, 'B': 1}
  Input histories for space |\colorbox{evgreen}{1904}|=11101110000:
     4=0100 -> {'B': 0}
     5=0101 -> {'A': 0, 'B': 0}
     6=0110 -> {'A': 1, 'B': 0}
     8=1000 -> {'B': 1}
     9=1001 -> {'A': 0, 'B': 1}
    10=1010 -> {'A': 1, 'B': 1}
\end{minted}

\noindent The third equivalence class discovered contains the space in the bottom layer of the hierarchy of causally complete spaces on 2 events with binary inputs, i.e. the discrete space.

\begin{minted}[escapeinside=||]{text}
Eq. class of space |\colorbox{evblue}{278}|=00100010110:
  Input histories for space |\colorbox{evblue}{278}|=00100010110:
     1=0001 -> {'A': 0}
     2=0010 -> {'A': 1}
     4=0100 -> {'B': 0}
     8=1000 -> {'B': 1}
\end{minted}

\noindent In the coming sub-subsections, we will review the entire code for the \mintinline{python}{SpaceFinder} class.


\subsection{\mintinline{python}{SpaceFinder} Class -- Constructor}

The constructor accepts some search options (already discussed above) and pre-computes some data, to improve the time performance of the search process.
Data pre-computed includes:
the sequence \mintinline{python}{self._events} of events;
the sequence \mintinline{python}{self._max_histories} of maximal extended input histories;
the sequence \mintinline{python}{self._perm_group} of event-input permutation group elements;
a dictionary \mintinline{python}{self._histories_perm_dict} mapping permutation group elements to their action on histories (as a history-history mapping);
a dictionary \mintinline{python}{self._children_set} mapping histories to the \mintinline{python}{frozenset} of their children;
a dictionary \mintinline{python}{self._children} mapping histories to the list of their children;
a dictionary \mintinline{python}{self._parents} mapping histories to the \mintinline{python}{frozenset} of their parents;
a dictionary \mintinline{python}{self._domsize} mapping histories to the size of their domain;
a boolean \mintinline{python}{self._initialised} indicating whether the search state has been initialised (by calling the \mintinline{python}{blank_state} or \mintinline{python}{load_state} method);
the maximum number \mintinline{python}{self._max_space_size} of bytes occupied by the bitvector for a space of input histories;
the number \mintinline{python}{self._fixed_memsize} of bytes occupied by the data structures computed above.

\begin{minted}[firstnumber=371]{python}

class SpaceFinder:
    # pylint: disable = too-many-instance-attributes
    """
    An instance of this class encapsulates all the data used in the
    search for causally complete spaces on a given number of events.
    """

    # Options:
    _num_events: int
    _verbose: bool
    _update_period: Optional[int]
#
#
#

    # Pre-computed data:
    _events: tuple[Event, ...]
    _max_histories: Sequence[History]
    _perm_group: tuple[PermGroupEl, ...]
    _histories_perm_dict: dict[PermGroupEl, dict[History, History]]
    _children_set: dict[History, frozenset[History]]
    _children: dict[History, tuple[History, ...]]
    _parents: dict[History, frozenset[History]]
    _domsize: dict[History, int]
    _max_space_size: int
    _initialised: bool
    _fixed_memsize: int

    def __init__(self, num_events: int,
                 *, verbose: bool = True,
                 update_period: Optional[int] = None) -> None:
#
#
#
        """
        Creates a new space finder instance, which will search for causally
        complete spaces on the given number of events.

        :param num_events: the number of events
        :param verbose: whether to print status updates, defaults to True
        :param update_period: how frequently to print status updates
                              (min number of equivalence classes discovered
                              between updates), defaults to None (print update
                              after every each top-level cycle)
        :param filename: internal state is regularly saved to a binary file with
                         this name, defaults to None (don't save internal state)
        :param save_period: how frequently to save internal state (min number of
                            equivalence classes discovered between saves),
                            defaults to None (save only once, at the end)
        """
        assert update_period is None or update_period > 0
        self._num_events = num_events
        self._verbose = verbose
        self._update_period = update_period
#
#
#
#
#
#
#
        self._max_histories = max_histories(num_events)
        self._events = tuple(idx2event(idx) for idx in range(num_events))
        self._perm_group = tuple(iter_perm_group(self._events))
        hs = sorted(sub_histories(self._max_histories),
                    key=history_sort_key)
        _histories_perm_dict = {
            h: dict(history_perms(h, self._perm_group))
            for h in hs
        }
        self._histories_perm_dict = {
            p: {
                h: _histories_perm_dict[h][p]
                for h in hs
            }
            for p in self._perm_group
        }
        self._children_set = {
            h: frozenset(child_histories(h))
            for h in hs
        }
        self._children = {
            h: tuple(sorted(self._children_set[h]))
            for h in hs
        }
        self._parents = parents(hs)
        self._domsize = {h: domsize(h) for h in hs}
        self._max_space_size = sys.getsizeof(2**(2**(2*len(self._events)))-1)
        self._initialised = False
        self._fixed_memsize = getsize(self)
\end{minted}


\subsection{\mintinline{python}{SpaceFinder} Class -- Spaces}

As new equivalence classes of causally complete spaces are discovered during the search, a single representative for the equivalence class is added to \mintinline{python}{self._eq_classes}, a mutable \mintinline{python}{set}, and the number of spaces in the equivalence class is added to \mintinline{python}{self._num_spaces}.
The \mintinline{python}{num_events}, \mintinline{python}{num_eq_classes} and \mintinline{python}{num_spaces} properties expose read-only access to the number of events, of equivalence classes discovered so far, and of spaces discovered so far.
The \mintinline{python}{iter_eq_classes} property exposes an iterator over the set of representatives for equivalence classes discovered so far, while the \mintinline{python}{iter_spaces} property exposes an iterator over all causally complete spaces discovered so far (computed on the fly by permuting each equivalence class representative in turn).

\begin{minted}[firstnumber=last]{python}
    _eq_classes: set[HistorySet]
    _num_spaces: int

    @property
    def num_events(self) -> int:
        """
        Number of events.
        """
        return self._num_events

    @property
    def num_eq_classes(self) -> int:
        """
        Number of equivalence classes of causally complete spaces,
        discovered so far.
        """
        return len(self._eq_classes)

    @property
    def iter_eq_classes(self) -> Iterator[HistorySet]:
        """
        Iterate over the equivalence classes of causally complete spaces
        discovered so far, yielding a representative space for
        each equivalence class.
        """
        return iter(self._eq_classes)

    @property
    def num_spaces(self) -> int:
        """
        Number of causally complete spaces discovered so far.
        """
        return self._num_spaces

    @property
    def iter_spaces(self) -> Iterator[tuple[HistorySet, HistorySet]]:
        """
        Iterate over the causally complete spaces discovered so far.
        """
        for eq_class_repr in self.iter_eq_classes:
            space_set = bitvec2set(eq_class_repr)
            eq_class = set()
            for g in self._perm_group:
                g_action = self._histories_perm_dict[g]
                space_set_perm = {g_action[h] for h in space_set}
                space_perm = bitvec(space_set_perm)
                if space_perm not in eq_class:
                    eq_class.add(space_perm)
                    yield eq_class_repr, space_perm
\end{minted}


\subsection{\mintinline{python}{SpaceFinder} Class -- Search Metrics}

The search algorithm is recursive and operates by levels, corresponding to histories with progressively decreasing number of events in their domain.
At the top level, the algorithm iterates through sets of children for the maximal extended input histories.
After an initial step, the number of sets of children to be iterated over is stored in \mintinline{python}{self._num_todo}, while \mintinline{python}{self._num_done} is increased by one after each top-level search step is completed: the ratio \mintinline{python}{self._num_done/self._num_todo} is used as a rough estimate of completion, exposed by the \mintinline{python}{perc_completed} property.
Note that this estimate is very rough: most spaces are discovered at the beginning, so that later iterations are likely to proceed significantly faster than early ones.
We'll explore precisely how this subset selection is done later on, when talking about the search logic.

\begin{minted}[firstnumber=last]{python}
    _num_done: int
    _num_todo: int

    @property
    def perc_completed(self) -> float:
        """
        Percent estimate of search completion.
        """
        return self._num_done/self._num_todo
\end{minted}

\noindent The search status printouts also include time and memory columns.
The memory size of fixed data structures is computed exactly by the class constructor, stored in \mintinline{python}{self._fixed_memsize}, while the size of variable data structures is given as an upper-bound, computed from their length and the maximum size of a history space bitvector for the given number of events.

\begin{minted}[firstnumber=542]{python}
    _start_time: float
    _partial_spaces_visited: set[HistorySet]

    @property
    def time_elapsed(self) -> float:
        """
        Time elapsed since search started.
        """
        return perf_counter()-self._start_time

    @property
    def memsize(self) -> int:
        """
        An estimate of memory currently occupied by the auxiliary data (bytes).
        """
        partial_visited = self._partial_spaces_visited
        full_visited = self._eq_classes
        child_choices_list = self._child_choices_list
        remaining_children_list = self._remaining_children_list
        max_space_size = self._max_space_size
        memsize = self._fixed_memsize
        memsize += (sys.getsizeof(partial_visited)
                    +len(partial_visited)*max_space_size)
        memsize += (sys.getsizeof(full_visited)
                    +len(full_visited)*max_space_size)
        memsize += (sys.getsizeof(child_choices_list)
                    +len(child_choices_list)*max_space_size)
        memsize += (sys.getsizeof(remaining_children_list)
                    +len(remaining_children_list)*max_space_size)
        return memsize
\end{minted}

\noindent Finally, some utility functions are implemented to print and log all status data.

\begin{minted}[firstnumber=last]{python}
    def _print(self, *args: str, **kwargs: Any) -> None:
        """ Prints to console. """
        print(*args, **kwargs)
#
#

    def _print_status_header(self) -> None:
        if self._verbose:
            header = (f"{'time': >10} {'spaces': >12} {'eq. cls': >10}"
                      f" {'memory': >10} {'completed': >10}")
#
            self._print(header, flush=True)

    def _print_status_line(self) -> None:
        if self._verbose:
            time = time_str(self.time_elapsed)
            spaces = str(self.num_spaces)
            eq_classes = str(self.num_eq_classes)
            memory = memory_str(self.memsize)
            completed = f"{self.perc_completed:.4%}"
#
#
            line = (f"{time: >10} {spaces: >12} {eq_classes: >10}"
                    f" {memory: >10} {completed: >10}")
#
            self._print(line, flush=True)

    def _describe(self) -> None:
        """
            If in verbose mode,
            prints a summary description of the search results (so far).
        """
        if self._verbose:
            self._print(f"Found {self.num_spaces} spaces in "
                       f"{self.num_eq_classes} equivalence classes.")
\end{minted}


\subsection{\mintinline{python}{SpaceFinder._find_eq_classes}}

The private \mintinline{python}{_find_eq_classes} method contains the main space-finding logic.
It operates recursively by levels, each level corresponding to a set of histories \mintinline{python}{set(new)_hs} with a fixed number of events: 
\begin{enumerate}
    \item The \mintinline{python}{new_hs} sequence contains the candidate input histories for this level: in the search on $n$ events, level $l$ corresponds to input histories with $n-l$ events (where $l=0, ..., n-2$).
    \item The \mintinline{python}{hs} sequence contains all candidate input histories thus far, excluding those in \mintinline{python}{new_hs}.
    \item The \mintinline{python}{hs_rest} sequence has the same length as \mintinline{python}{hs} and it contains the difference between each history in \mintinline{python}{h in hs} and all the sub-histories \mintinline{python}{k in sub_histories([h])} which have been discovered so far. If this difference becomes empty, the history \mintinline{python}{h} is guaranteed to not be $\vee$-prime, and hence is removed from the candidate input histories (later on, we refer to this process as ``winnowing'').
\end{enumerate}
In the top-level (\mintinline{python}{level=0}) call to \mintinline{python}{_find_eq_classes} that starts the search, \mintinline{python}{new_hs} is set to the maximal extended input histories on the given number of events, while \mintinline{python}{hs} and \mintinline{python}{hs_rest} are set to empty.

\begin{minted}[firstnumber=691]{python}
    def _find_eq_classes(self, new_hs: Sequence[History],
                         hs: Sequence[History] = tuple(),
                         hs_rest: Sequence[History] = tuple(),
                         level: int = 0) -> Iterator[HistorySet]:
        # pylint: disable = too-many-locals
        hs_so_far = tuple(chain(new_hs, hs))
        hs_perm_dict = self._histories_perm_dict
        partial_spaces_visited = self._partial_spaces_visited
        spaces_visited = self._eq_classes
#
#
#
#
        for child_subset in self._iter_child_subsets(new_hs):
            child_subset_sorted = sorted(child_subset, key=history_sort_key)
            hs_so_far_rest = list(chain(new_hs, hs_rest))
            for k in child_subset_sorted:
                for j, h in enumerate(hs_so_far):
                    if is_subset(k, h):
                        hs_so_far_rest[j] = sub(hs_so_far_rest[j], k)
            winnowed_hs = tuple(h for j, h in enumerate(hs_so_far)
                                       if hs_so_far_rest[j])
            winnowed_hs_rest = tuple(
                h for h in hs_so_far_rest if h)
            partial_space = set(chain(child_subset_sorted, winnowed_hs))
            partial_space_bitvec = bitvec(partial_space)
            already_seen = (partial_space_bitvec in partial_spaces_visited
                            or partial_space_bitvec in spaces_visited)
            if not already_seen:
                eq_class = set()
                for p in self._perm_group:
                    p_action = hs_perm_dict[p]
                    perm_partial_space_bitvec = bitvec(p_action[h]
                                                       for h in partial_space)
                    eq_class.add(perm_partial_space_bitvec)
                    if (perm_partial_space_bitvec in partial_spaces_visited
                        or perm_partial_space_bitvec in spaces_visited):
                        already_seen = True
                        break
            if not already_seen:
                if all(self._domsize[h] == 1 for h in child_subset_sorted):
                    self._num_spaces += len(eq_class)
                    yield partial_space_bitvec
                else:
                    partial_spaces_visited.add(partial_space_bitvec)
                    yield from self._find_eq_classes(child_subset_sorted,
                                                     winnowed_hs,
                                                     winnowed_hs_rest,
                                                     level+1)
\end{minted}

\noindent At the start of the method, the sequence \mintinline{python}{hs_so_far} of all current candidate input histories is computed, and some persistent data structures are accessed and given shorter names.

\begin{minted}[firstnumber=696]{python}
        hs_so_far = tuple(chain(new_hs, hs))
        hs_perm_dict = self._histories_perm_dict
        partial_spaces_visited = self._partial_spaces_visited
        spaces_visited = self._eq_classes
\end{minted}

\noindent The rest of the method iterates over all subsets of child histories for the histories in \mintinline{python}{new_hs}.
Two different methods are used: the symmetry-optimised \mintinline{python}{self._iter_child_subsets_toplevel} is used at top-level (\mintinline{python}{level=0}), while the general purpose \mintinline{python}{self._iter_child_subsets} is used for all other levels.
In principle, the \mintinline{python}{self._iter_child_subsets} method could be used to iterate over child subsets at all levels, but the added efficiency of the \mintinline{python}{self._iter_child_subsets_toplevel} is helpful in the search on 4 events (plus, the symmetry argument is interesting in itself).
Each child subset yielded is sorted to make the search deterministic.

\begin{minted}[firstnumber=700]{python}
#
#
#
#
        for child_subset in self._iter_child_subsets(new_hs):
            child_subset_sorted = sorted(child_subset, key=history_sort_key)
\end{minted}

\noindent For each chosen subset \mintinline{python}{child_subset} of child histories, we wish to compute the part of each candidate input history in \mintinline{python}{hs_so_far} that is not covered by any sub-history discovered so far.
Excluding any sub-histories in \mintinline{python}{child_subset}, the entirety of each history in \mintinline{python}{new_hs} is uncovered---because we haven't seen any of their sub-histories yet---while the ``uncovered'' part for the histories in \mintinline{python}{hs} is given by \mintinline{python}{hs_rest}.
Hence, we initialise the sequence of ``uncovered'' parts for \mintinline{python}{hs_so_far} to \mintinline{python}{hs_so_far_rest}.
For each child history \mintinline{python}{k in child_subset}, we then proceed to subtract \mintinline{python}{k} from all its super-histories in \mintinline{python}{hs_so_far}, updating the corresponding ``uncovered'' part in \mintinline{python}{hs_so_far_rest}.
At the end, \mintinline{python}{winnowed_hs} contains the histories that have non-empty ``uncovered'' part (i.e. those which might still be $\vee$-prime), and \mintinline{python}{winnowed_hs_rest} contains the corresponding ``uncovered'' parts, to be used by recursive calls to \mintinline{python}{_find_eq_classes}.

\begin{minted}[firstnumber=706]{python}
            hs_so_far_rest = list(chain(new_hs, hs_rest))
            for k in child_subset_sorted:
                for j, h in enumerate(hs_so_far):
                    if is_subset(k, h):
                        hs_so_far_rest[j] = sub(hs_so_far_rest[j], k)
            winnowed_hs = tuple(h for j, h in enumerate(hs_so_far)
                                       if hs_so_far_rest[j])
            winnowed_hs_rest = tuple(
                h for h in hs_so_far_rest if h)
\end{minted}

\noindent The new set of candidate input histories contains the currently selected subset of child histories together with the histories that survived the winnowing step above: taken together, they form \mintinline{python}{partial_space}, a ``partial'' space of input histories so far (which is not causally complete unless we are at the bottom level $l=n-2$).
Before proceeding, we check whether we have already encountered this ``partial'' space, or any of its permutations under event-input permutation symmetry; in the process, we also compute its equivalence class.

\begin{minted}[firstnumber=715]{python}
            partial_space = set(chain(child_subset_sorted, winnowed_hs))
            partial_space_bitvec = bitvec(partial_space)
            already_seen = (partial_space_bitvec in partial_spaces_visited
                            or partial_space_bitvec in spaces_visited)
            if not already_seen:
                eq_class = set()
                for p in self._perm_group:
                    p_action = hs_perm_dict[p]
                    perm_partial_space_bitvec = bitvec(p_action[h]
                                                       for h in partial_space)
                    eq_class.add(perm_partial_space_bitvec)
                    if (perm_partial_space_bitvec in partial_spaces_visited
                        or perm_partial_space_bitvec in spaces_visited):
                        already_seen = True
                        break
\end{minted}

\noindent If \mintinline{python}{partial_space} was not encountered before, we proceed.
If we reached the bottom level---i.e. if all child histories have exactly 1 event---then \mintinline{python}{partial_space} is a representative in a newly discovered equivalence class of causally complete spaces, so we update the total number of spaces and we yield the representative.
If we instead haven't reached the bottom level, then we register the ``partial'' space \mintinline{python}{partial_space} as visited and recurse to one level below.

\begin{minted}[firstnumber=730]{python}
            if not already_seen:
                if all(self._domsize[h] == 1 for h in child_subset_sorted):
                    self._num_spaces += len(eq_class)
                    yield partial_space_bitvec
                else:
                    partial_spaces_visited.add(partial_space_bitvec)
                    yield from self._find_eq_classes(child_subset_sorted,
                                                     winnowed_hs,
                                                     winnowed_hs_rest,
                                                     level+1)
\end{minted}


\subsection{\mintinline{python}{SpaceFinder.find_eq_classes}}

The \mintinline{python}{find_eq_classes} method checks that the finder instance has been initialised, sets start time and the \mintinline{python}{self._num_eq_classes_since_last_save} counter, and then starts the search by calling \mintinline{python}{_find_eq_classes} on the maximal extended input histories for the chosen number of events.
For each equivalence class discovered, the representative yielded by \mintinline{python}{_find_eq_classes} is added to the set of equivalence classes, the \mintinline{python}{self._num_eq_classes_since_last_save} counter is increased, and a new line in the search status table is printed (if required).
When all equivalence classes have been discovered, a final line in the search status table is printed (if required), the state is saved (if required) and the number of discovered spaces and equivalence classes is printed (if required).

\begin{minted}[firstnumber=740]{python}
    def find_eq_classes(self) -> None:
        """
            Start the search for equivalence classes.
        """
        if not self._initialised:
            raise Exception("Must initialise using blank_state() "
                            "or load_state(file).")
        update_period = self._update_period
        self._start_time = perf_counter()
        self._num_eq_classes_since_last_save = 0
        init_histories = max_histories(self.num_events)
        for eq_class_repr in self._find_eq_classes(init_histories):
            self._eq_classes.add(eq_class_repr)
            self._num_eq_classes_since_last_save += 1
            if (update_period is not None
                and self.num_eq_classes % update_period == 0):
                self._print_status_line()
        if update_period is not None:
            self._print_status_line()
        self._partial_spaces_visited.clear()
#
        self._describe()
\end{minted}


\subsection{\mintinline{python}{SpaceFinder._iter_child_subsets}}

Given a sequence \mintinline{python}{hs} of histories, the private \mintinline{python}{_iter_child_subsets} method iterates through all subsets \mintinline{python}{child_subset} of \mintinline{python}{child_histories(hs)} such that each history in \mintinline{python}{h in hs} has at least one of its child histories in \mintinline{python}{child_subset}.
This is done by sorting all child histories and then iterating through all bitvectors \mintinline{python}{child_subset_bitvec} for their non-empty subsets: for each bitvector, the call to the \mintinline{python}{_child_subset} returns the corresponding subset if it satisfies the condition above, or \mintinline{python}{None} if it doesn't.

\begin{minted}[firstnumber=last]{python}
    def _iter_child_subsets(self,
                            hs: Sequence[History]
                            ) -> Iterator[set[History]]:
        child_hists: list[History] = sorted(
            {k for h in hs for k in self._children[h]},
            key=history_sort_key)
        num_child_subsets = 2**len(child_hists)
        for child_subset_bitvec in range(1, num_child_subsets):
            child_subset = self._child_subset(hs, child_hists,
                                              child_subset_bitvec)
            if child_subset is not None:
                yield child_subset
\end{minted}

\noindent In principle, this method could be used to iterate over child subsets at all levels.
However, the simplicity of its logic is paid for by the unnecessary inefficiency at the top level, where event-input permutation symmetry can be used to reduce the number of subsets that need to be explored.
For 2 events, symmetry reduces the number of subsets by about 3 times, from 16 to 6;
for 3 events, it reduces the number by about 4 times, from 4096 to 922;
for 4 events, it reduces the number by about 13 times, from 4294967296 to 315981136.
For 5 events, symmetry reduces the number of subsets by 58 times or more; however, this doesn't really matter, since the number of top-level subsets to iterate over has in excess of 20 decimal digits, making a search on 5 events impossible with our algorithm anyway.


\subsection{\mintinline{python}{SpaceFinder._child_subset}}

The private \mintinline{python}{_child_subsets} method takes as its arguments a sequence \mintinline{python}{hs} of histories, a sequence \mintinline{python}{child_hists} of their children and the bitvector for a non-empty subset of children.

\begin{minted}[firstnumber=last]{python}
    def _child_subset(self,
                      hs: Sequence[History],
                      child_hists: Sequence[History],
                      child_subset_bitvec: int,
#
#
                      ) -> Optional[set[History]]:
        # pylint: disable = too-many-arguments
        hs_still_to_cover = set(hs)
        child_subset = set()
        idx = 0
        while child_subset_bitvec > 0:
            child_subset_bitvec, b = divmod(child_subset_bitvec, 2)
            if b:
                k = child_hists[idx]
                child_subset.add(k)
                if hs_still_to_cover:
                    hs_still_to_cover -= self._parents[k]
            idx += 1
        if not hs_still_to_cover:
            return child_subset
        return None
\end{minted}

\noindent The method starts by creating a set \mintinline{python}{hs_to_cover} of histories in \mintinline{python}{hs} still to be ``covered''---i.e. where at least one child has to yet appear in the subets---and a set \mintinline{python}{child_subset} of children extracted from the bitvector.

\begin{minted}[firstnumber=782]{python}
        hs_still_to_cover = set(hs)
        child_subset = set()
\end{minted}

\noindent The method then proceeds to iterate through the bits in the subset bitvector, including into \mintinline{python}{child_subset} any child history \mintinline{python}{k} whose bit is set to 1.
The parents of \mintinline{python}{k} are then subtracted from \mintinline{python}{hs_still_to_cover} (unnecessary if it's already empty).

\begin{minted}[firstnumber=784]{python}
        idx = 0
        while child_subset_bitvec > 0:
            child_subset_bitvec, b = divmod(child_subset_bitvec, 2)
            if b:
                k = child_hists[idx]
                child_subset.add(k)
                if hs_still_to_cover:
                    hs_still_to_cover -= self._parents[k]
            idx += 1
\end{minted}

\noindent After all children indicated by the bitvector have been added to \mintinline{python}{child_subset}, we check whether there are any histories in \mintinline{python}{hs_still_to_cover}: if not, the subset is yielded; if so, \mintinline{python}{None} is yielded instead, to indicate that the subsets should not be explored.

\begin{minted}[firstnumber=793]{python}
        if not hs_still_to_cover:
            return child_subset
        return None
\end{minted}



% === COMMENT BELOW BEFORE COMPILING MAIN FILE ===

% \newpage

% \section{Advanced Algorithm to Find Causally Complete Spaces}
% \label{appendix:space-finder-algo-adv}

% \end{document}
