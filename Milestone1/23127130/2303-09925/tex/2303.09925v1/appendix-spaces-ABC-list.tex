\newpage
\subsection*{Space 0}

Space 0 is induced by the definite causal order $\discrete{\ev{A},\ev{B},\ev{C}}$.
Its equivalence class under event-input permutation symmetry contains 1 space.

\noindent Below are the histories and extended histories for space 0: 
\begin{center}
    \begin{tabular}{cc}
    \includegraphics[height=3.5cm]{svg-inkscape/space-ABC-unique-tight-0-highlighted_svg-tex.pdf}
    &
    \includegraphics[height=3.5cm]{svg-inkscape/space-ABC-unique-tight-0-ext-highlighted_svg-tex.pdf}
    \\
    $\Theta_{0}$
    &
    $\Ext{\Theta_{0}}$
    \end{tabular}
\end{center}

\noindent The standard causaltope for Space 0 has dimension 26.
Below is a plot of the homogeneous linear system of causality and quasi-normalisation equations for the standard causaltope, put in reduced row echelon form:

\begin{center}
    \includegraphics[width=11cm]{svg-inkscape/space0-rref-eqs_svg-tex.pdf}
\end{center}

\noindent Rows correspond to the 37 independent linear equations.
Columns in the plot correspond to entries of empirical models, indexed as $i_A i_B i_C$ $o_A o_B o_C$.
Coefficients in the equations are color-coded as white=0, red=+1 and blue=-1.

Space 0 is the global minimum of the hierarchy, with no refinements.
It has closest coarsenings in equivalence class 1; 
it is the meet of its (closest) coarsenings.
It has 64 causal functions.
It is a tight space.

The standard causaltope for Space 0 is the meet of the standard causaltopes for its closest coarsenings.
For completeness, below is a plot of the full homogeneous linear system of causality and quasi-normalisation equations for the standard causaltope:

\begin{center}
    \includegraphics[width=12cm]{svg-inkscape/space0-eqs_svg-tex.pdf}
\end{center}

\noindent Rows correspond to the 91 linear equations, of which 37 are independent.


\newpage
\subsection*{Space 1}

Space 1 is not induced by a causal order, but it is a refinement of the space in equivalence class 33 induced by the definite causal order $\total{\ev{B},\ev{A}}\vee\discrete{\ev{C}}$ (note that the space induced by the order is not the same as space 33).
Its equivalence class under event-input permutation symmetry contains 6 spaces.
Space 1 differs as follows from the space induced by causal order $\total{\ev{B},\ev{A}}\vee\discrete{\ev{C}}$:
\begin{itemize}
  \item The outputs at events \evset{\ev{A}, \ev{C}} are independent of the input at event \ev{B} when the inputs at events \evset{A, C} are given by \hist{A/0,C/1}, \hist{A/0,C/0}, \hist{A/1,C/0} and \hist{A/1,C/1}.
  \item The output at event \ev{A} is independent of the input at event \ev{B} when the input at event A is given by \hist{A/0}.
\end{itemize}

\noindent Below are the histories and extended histories for space 1: 
\begin{center}
    \begin{tabular}{cc}
    \includegraphics[height=3.5cm]{svg-inkscape/space-ABC-unique-untight-1-highlighted_svg-tex.pdf}
    &
    \includegraphics[height=3.5cm]{svg-inkscape/space-ABC-unique-untight-1-ext-highlighted_svg-tex.pdf}
    \\
    $\Theta_{1}$
    &
    $\Ext{\Theta_{1}}$
    \end{tabular}
\end{center}

\noindent The standard causaltope for Space 1 has dimension 26.
Below is a plot of the homogeneous linear system of causality and quasi-normalisation equations for the standard causaltope, put in reduced row echelon form:

\begin{center}
    \includegraphics[width=11cm]{svg-inkscape/space1-rref-eqs_svg-tex.pdf}
\end{center}

\noindent Rows correspond to the 37 independent linear equations.
Columns in the plot correspond to entries of empirical models, indexed as $i_A i_B i_C$ $o_A o_B o_C$.
Coefficients in the equations are color-coded as white=0, red=+1 and blue=-1.

Space 1 has closest refinements in equivalence class 0; 
it does not arise as a nontrivial join in the hierarchy.
It has closest coarsenings in equivalence classes 2 and 3; 
it is the meet of its (closest) coarsenings.
It has 64 causal functions, all of which are causal for at least one of its refinements.
It is not a tight space: for event \ev{A}, a causal function must yield identical output values on input histories \hist{A/1,B/0}, \hist{A/1,B/1}, \hist{A/1,C/0} and \hist{A/1,C/1}.

The standard causaltope for Space 1 coincides with that of its subspace in equivalence class 0.
The standard causaltope for Space 1 is the meet of the standard causaltopes for its closest coarsenings.
For completeness, below is a plot of the full homogeneous linear system of causality and quasi-normalisation equations for the standard causaltope:

\begin{center}
    \includegraphics[width=12cm]{svg-inkscape/space1-eqs_svg-tex.pdf}
\end{center}

\noindent Rows correspond to the 85 linear equations, of which 37 are independent.


\newpage
\subsection*{Space 2}

Space 2 is not induced by a causal order, but it is a refinement of the space in equivalence class 33 induced by the definite causal order $\total{\ev{C},\ev{B}}\vee\discrete{\ev{A}}$ (note that the space induced by the order is not the same as space 33).
Its equivalence class under event-input permutation symmetry contains 24 spaces.
Space 2 differs as follows from the space induced by causal order $\total{\ev{C},\ev{B}}\vee\discrete{\ev{A}}$:
\begin{itemize}
  \item The outputs at events \evset{\ev{A}, \ev{B}} are independent of the input at event \ev{C} when the inputs at events \evset{A, B} are given by \hist{A/0,B/0}, \hist{A/0,B/1} and \hist{A/1,B/0}.
  \item The output at event \ev{B} is independent of the input at event \ev{C} when the input at event B is given by \hist{B/0}.
\end{itemize}

\noindent Below are the histories and extended histories for space 2: 
\begin{center}
    \begin{tabular}{cc}
    \includegraphics[height=3.5cm]{svg-inkscape/space-ABC-unique-untight-2-highlighted_svg-tex.pdf}
    &
    \includegraphics[height=3.5cm]{svg-inkscape/space-ABC-unique-untight-2-ext-highlighted_svg-tex.pdf}
    \\
    $\Theta_{2}$
    &
    $\Ext{\Theta_{2}}$
    \end{tabular}
\end{center}

\noindent The standard causaltope for Space 2 has dimension 27.
Below is a plot of the homogeneous linear system of causality and quasi-normalisation equations for the standard causaltope, put in reduced row echelon form:

\begin{center}
    \includegraphics[width=11cm]{svg-inkscape/space2-rref-eqs_svg-tex.pdf}
\end{center}

\noindent Rows correspond to the 36 independent linear equations.
Columns in the plot correspond to entries of empirical models, indexed as $i_A i_B i_C$ $o_A o_B o_C$.
Coefficients in the equations are color-coded as white=0, red=+1 and blue=-1.

Space 2 has closest refinements in equivalence class 1; 
it does not arise as a nontrivial join in the hierarchy.
It has closest coarsenings in equivalence classes 4, 5, 6 and 7; 
it is the meet of its (closest) coarsenings.
It has 64 causal functions, all of which are causal for at least one of its refinements.
It is not a tight space: for event \ev{B}, a causal function must yield identical output values on input histories \hist{A/0,B/1}, \hist{B/1,C/0} and \hist{B/1,C/1}.

The standard causaltope for Space 2 has 1 more dimension than that of its subspace in equivalence class 1.
The standard causaltope for Space 2 is the meet of the standard causaltopes for its closest coarsenings.
For completeness, below is a plot of the full homogeneous linear system of causality and quasi-normalisation equations for the standard causaltope:

\begin{center}
    \includegraphics[width=12cm]{svg-inkscape/space2-eqs_svg-tex.pdf}
\end{center}

\noindent Rows correspond to the 81 linear equations, of which 36 are independent.


\newpage
\subsection*{Space 3}

Space 3 is not induced by a causal order, but it is a refinement of the space 33 induced by the definite causal order $\total{\ev{A},\ev{B}}\vee\discrete{\ev{C}}$.
Its equivalence class under event-input permutation symmetry contains 3 spaces.
Space 3 differs as follows from the space induced by causal order $\total{\ev{A},\ev{B}}\vee\discrete{\ev{C}}$:
\begin{itemize}
  \item The outputs at events \evset{\ev{B}, \ev{C}} are independent of the input at event \ev{A} when the inputs at events \evset{B, C} are given by \hist{B/1,C/0}, \hist{B/1,C/1}, \hist{B/0,C/0} and \hist{B/0,C/1}.
\end{itemize}

\noindent Below are the histories and extended histories for space 3: 
\begin{center}
    \begin{tabular}{cc}
    \includegraphics[height=3.5cm]{svg-inkscape/space-ABC-unique-untight-3-highlighted_svg-tex.pdf}
    &
    \includegraphics[height=3.5cm]{svg-inkscape/space-ABC-unique-untight-3-ext-highlighted_svg-tex.pdf}
    \\
    $\Theta_{3}$
    &
    $\Ext{\Theta_{3}}$
    \end{tabular}
\end{center}

\noindent The standard causaltope for Space 3 has dimension 26.
Below is a plot of the homogeneous linear system of causality and quasi-normalisation equations for the standard causaltope, put in reduced row echelon form:

\begin{center}
    \includegraphics[width=11cm]{svg-inkscape/space3-rref-eqs_svg-tex.pdf}
\end{center}

\noindent Rows correspond to the 37 independent linear equations.
Columns in the plot correspond to entries of empirical models, indexed as $i_A i_B i_C$ $o_A o_B o_C$.
Coefficients in the equations are color-coded as white=0, red=+1 and blue=-1.

Space 3 has closest refinements in equivalence class 1; 
it is the join of its (closest) refinements.
It has closest coarsenings in equivalence class 6; 
it is the meet of its (closest) coarsenings.
It has 64 causal functions, all of which are causal for at least one of its refinements.
It is not a tight space: for event \ev{B}, a causal function must yield identical output values on input histories \hist{A/0,B/0}, \hist{A/1,B/0}, \hist{B/0,C/0} and \hist{B/0,C/1}, and it must also yield identical output values on input histories \hist{A/0,B/1}, \hist{A/1,B/1}, \hist{B/1,C/0} and \hist{B/1,C/1}.

The standard causaltope for Space 3 coincides with that of its 2 subspaces in equivalence class 1.
The standard causaltope for Space 3 is the meet of the standard causaltopes for its closest coarsenings.
For completeness, below is a plot of the full homogeneous linear system of causality and quasi-normalisation equations for the standard causaltope:

\begin{center}
    \includegraphics[width=12cm]{svg-inkscape/space3-eqs_svg-tex.pdf}
\end{center}

\noindent Rows correspond to the 79 linear equations, of which 37 are independent.


\newpage
\subsection*{Space 4}

Space 4 is not induced by a causal order, but it is a refinement of the space in equivalence class 92 induced by the definite causal order $\total{\ev{A},\ev{B}}\vee\total{\ev{C},\ev{B}}$ (note that the space induced by the order is not the same as space 92).
Its equivalence class under event-input permutation symmetry contains 24 spaces.
Space 4 differs as follows from the space induced by causal order $\total{\ev{A},\ev{B}}\vee\total{\ev{C},\ev{B}}$:
\begin{itemize}
  \item The outputs at events \evset{\ev{A}, \ev{B}} are independent of the input at event \ev{C} when the inputs at events \evset{A, B} are given by \hist{A/0,B/0}, \hist{A/0,B/1} and \hist{A/1,B/0}.
  \item The outputs at events \evset{\ev{B}, \ev{C}} are independent of the input at event \ev{A} when the inputs at events \evset{B, C} are given by \hist{B/1,C/1}, \hist{B/0,C/0} and \hist{B/0,C/1}.
  \item The output at event \ev{B} is independent of the inputs at events \evset{\ev{A}, \ev{C}} when the input at event B is given by \hist{B/0}.
\end{itemize}

\noindent Below are the histories and extended histories for space 4: 
\begin{center}
    \begin{tabular}{cc}
    \includegraphics[height=3.5cm]{svg-inkscape/space-ABC-unique-untight-4-highlighted_svg-tex.pdf}
    &
    \includegraphics[height=3.5cm]{svg-inkscape/space-ABC-unique-untight-4-ext-highlighted_svg-tex.pdf}
    \\
    $\Theta_{4}$
    &
    $\Ext{\Theta_{4}}$
    \end{tabular}
\end{center}

\noindent The standard causaltope for Space 4 has dimension 29.
Below is a plot of the homogeneous linear system of causality and quasi-normalisation equations for the standard causaltope, put in reduced row echelon form:

\begin{center}
    \includegraphics[width=11cm]{svg-inkscape/space4-rref-eqs_svg-tex.pdf}
\end{center}

\noindent Rows correspond to the 34 independent linear equations.
Columns in the plot correspond to entries of empirical models, indexed as $i_A i_B i_C$ $o_A o_B o_C$.
Coefficients in the equations are color-coded as white=0, red=+1 and blue=-1.

Space 4 has closest refinements in equivalence class 2; 
it is the join of its (closest) refinements.
It has closest coarsenings in equivalence classes 8, 11 and 15; 
it is the meet of its (closest) coarsenings.
It has 128 causal functions, 64 of which are not causal for any of its refinements.
It is not a tight space: for event \ev{B}, a causal function must yield identical output values on input histories \hist{A/0,B/1} and \hist{B/1,C/1}.

The standard causaltope for Space 4 has 2 more dimensions than those of its 2 subspaces in equivalence class 2.
The standard causaltope for Space 4 is the meet of the standard causaltopes for its closest coarsenings.
For completeness, below is a plot of the full homogeneous linear system of causality and quasi-normalisation equations for the standard causaltope:

\begin{center}
    \includegraphics[width=12cm]{svg-inkscape/space4-eqs_svg-tex.pdf}
\end{center}

\noindent Rows correspond to the 77 linear equations, of which 34 are independent.


\newpage
\subsection*{Space 5}

Space 5 is not induced by a causal order, but it is a refinement of the space 33 induced by the definite causal order $\total{\ev{A},\ev{B}}\vee\discrete{\ev{C}}$.
Its equivalence class under event-input permutation symmetry contains 12 spaces.
Space 5 differs as follows from the space induced by causal order $\total{\ev{A},\ev{B}}\vee\discrete{\ev{C}}$:
\begin{itemize}
  \item The outputs at events \evset{\ev{B}, \ev{C}} are independent of the input at event \ev{A} when the inputs at events \evset{B, C} are given by \hist{B/1,C/0} and \hist{B/1,C/1}.
  \item The output at event \ev{B} is independent of the input at event \ev{A} when the input at event B is given by \hist{B/1}.
\end{itemize}

\noindent Below are the histories and extended histories for space 5: 
\begin{center}
    \begin{tabular}{cc}
    \includegraphics[height=3.5cm]{svg-inkscape/space-ABC-unique-tight-5-highlighted_svg-tex.pdf}
    &
    \includegraphics[height=3.5cm]{svg-inkscape/space-ABC-unique-tight-5-ext-highlighted_svg-tex.pdf}
    \\
    $\Theta_{5}$
    &
    $\Ext{\Theta_{5}}$
    \end{tabular}
\end{center}

\noindent The standard causaltope for Space 5 has dimension 29.
Below is a plot of the homogeneous linear system of causality and quasi-normalisation equations for the standard causaltope, put in reduced row echelon form:

\begin{center}
    \includegraphics[width=11cm]{svg-inkscape/space5-rref-eqs_svg-tex.pdf}
\end{center}

\noindent Rows correspond to the 34 independent linear equations.
Columns in the plot correspond to entries of empirical models, indexed as $i_A i_B i_C$ $o_A o_B o_C$.
Coefficients in the equations are color-coded as white=0, red=+1 and blue=-1.

Space 5 has closest refinements in equivalence class 2; 
it is the join of its (closest) refinements.
It has closest coarsenings in equivalence classes 8, 12 and 14; 
it is the meet of its (closest) coarsenings.
It has 128 causal functions, 64 of which are not causal for any of its refinements.
It is a tight space.

The standard causaltope for Space 5 has 2 more dimensions than those of its 2 subspaces in equivalence class 2.
The standard causaltope for Space 5 is the meet of the standard causaltopes for its closest coarsenings.
For completeness, below is a plot of the full homogeneous linear system of causality and quasi-normalisation equations for the standard causaltope:

\begin{center}
    \includegraphics[width=12cm]{svg-inkscape/space5-eqs_svg-tex.pdf}
\end{center}

\noindent Rows correspond to the 77 linear equations, of which 34 are independent.


\newpage
\subsection*{Space 6}

Space 6 is not induced by a causal order, but it is a refinement of the space 33 induced by the definite causal order $\total{\ev{A},\ev{B}}\vee\discrete{\ev{C}}$.
Its equivalence class under event-input permutation symmetry contains 24 spaces.
Space 6 differs as follows from the space induced by causal order $\total{\ev{A},\ev{B}}\vee\discrete{\ev{C}}$:
\begin{itemize}
  \item The outputs at events \evset{\ev{B}, \ev{C}} are independent of the input at event \ev{A} when the inputs at events \evset{B, C} are given by \hist{B/1,C/0}, \hist{B/1,C/1} and \hist{B/0,C/1}.
\end{itemize}

\noindent Below are the histories and extended histories for space 6: 
\begin{center}
    \begin{tabular}{cc}
    \includegraphics[height=3.5cm]{svg-inkscape/space-ABC-unique-untight-6-highlighted_svg-tex.pdf}
    &
    \includegraphics[height=3.5cm]{svg-inkscape/space-ABC-unique-untight-6-ext-highlighted_svg-tex.pdf}
    \\
    $\Theta_{6}$
    &
    $\Ext{\Theta_{6}}$
    \end{tabular}
\end{center}

\noindent The standard causaltope for Space 6 has dimension 27.
Below is a plot of the homogeneous linear system of causality and quasi-normalisation equations for the standard causaltope, put in reduced row echelon form:

\begin{center}
    \includegraphics[width=11cm]{svg-inkscape/space6-rref-eqs_svg-tex.pdf}
\end{center}

\noindent Rows correspond to the 36 independent linear equations.
Columns in the plot correspond to entries of empirical models, indexed as $i_A i_B i_C$ $o_A o_B o_C$.
Coefficients in the equations are color-coded as white=0, red=+1 and blue=-1.

Space 6 has closest refinements in equivalence classes 2 and 3; 
it is the join of its (closest) refinements.
It has closest coarsenings in equivalence classes 9, 10, 13, 14 and 15; 
it is the meet of its (closest) coarsenings.
It has 64 causal functions, all of which are causal for at least one of its refinements.
It is not a tight space: for event \ev{B}, a causal function must yield identical output values on input histories \hist{A/0,B/0}, \hist{A/1,B/0} and \hist{B/0,C/1}, and it must also yield identical output values on input histories \hist{A/0,B/1}, \hist{A/1,B/1}, \hist{B/1,C/0} and \hist{B/1,C/1}.

The standard causaltope for Space 6 coincides with that of its subspace in equivalence class 2.
The standard causaltope for Space 6 is the meet of the standard causaltopes for its closest coarsenings.
For completeness, below is a plot of the full homogeneous linear system of causality and quasi-normalisation equations for the standard causaltope:

\begin{center}
    \includegraphics[width=12cm]{svg-inkscape/space6-eqs_svg-tex.pdf}
\end{center}

\noindent Rows correspond to the 75 linear equations, of which 36 are independent.


\newpage
\subsection*{Space 7}

Space 7 is not induced by a causal order, but it is a refinement of the space in equivalence class 77 induced by the definite causal order $\total{\ev{C},\ev{A}}\vee\total{\ev{C},\ev{B}}$ (note that the space induced by the order is not the same as space 77).
Its equivalence class under event-input permutation symmetry contains 12 spaces.
Space 7 differs as follows from the space induced by causal order $\total{\ev{C},\ev{A}}\vee\total{\ev{C},\ev{B}}$:
\begin{itemize}
  \item The outputs at events \evset{\ev{A}, \ev{B}} are independent of the input at event \ev{C} when the inputs at events \evset{A, B} are given by \hist{A/0,B/0}, \hist{A/0,B/1} and \hist{A/1,B/0}.
  \item The output at event \ev{B} is independent of the input at event \ev{C} when the input at event B is given by \hist{B/0}.
  \item The output at event \ev{A} is independent of the input at event \ev{C} when the input at event A is given by \hist{A/0}.
\end{itemize}

\noindent Below are the histories and extended histories for space 7: 
\begin{center}
    \begin{tabular}{cc}
    \includegraphics[height=3.5cm]{svg-inkscape/space-ABC-unique-untight-7-highlighted_svg-tex.pdf}
    &
    \includegraphics[height=3.5cm]{svg-inkscape/space-ABC-unique-untight-7-ext-highlighted_svg-tex.pdf}
    \\
    $\Theta_{7}$
    &
    $\Ext{\Theta_{7}}$
    \end{tabular}
\end{center}

\noindent The standard causaltope for Space 7 has dimension 27.
Below is a plot of the homogeneous linear system of causality and quasi-normalisation equations for the standard causaltope, put in reduced row echelon form:

\begin{center}
    \includegraphics[width=11cm]{svg-inkscape/space7-rref-eqs_svg-tex.pdf}
\end{center}

\noindent Rows correspond to the 36 independent linear equations.
Columns in the plot correspond to entries of empirical models, indexed as $i_A i_B i_C$ $o_A o_B o_C$.
Coefficients in the equations are color-coded as white=0, red=+1 and blue=-1.

Space 7 has closest refinements in equivalence class 2; 
it is the join of its (closest) refinements.
It has closest coarsenings in equivalence classes 11 and 12; 
it is the meet of its (closest) coarsenings.
It has 64 causal functions, all of which are causal for at least one of its refinements.
It is not a tight space: for event \ev{A}, a causal function must yield identical output values on input histories \hist{A/1,B/0}, \hist{A/1,C/0} and \hist{A/1,C/1}; for event \ev{B}, a causal function must yield identical output values on input histories \hist{A/0,B/1}, \hist{B/1,C/0} and \hist{B/1,C/1}.

The standard causaltope for Space 7 coincides with that of its 2 subspaces in equivalence class 2.
The standard causaltope for Space 7 is the meet of the standard causaltopes for its closest coarsenings.
For completeness, below is a plot of the full homogeneous linear system of causality and quasi-normalisation equations for the standard causaltope:

\begin{center}
    \includegraphics[width=12cm]{svg-inkscape/space7-eqs_svg-tex.pdf}
\end{center}

\noindent Rows correspond to the 75 linear equations, of which 36 are independent.


\newpage
\subsection*{Space 8}

Space 8 is not induced by a causal order, but it is a refinement of the space in equivalence class 92 induced by the definite causal order $\total{\ev{A},\ev{B}}\vee\total{\ev{C},\ev{B}}$ (note that the space induced by the order is not the same as space 92).
Its equivalence class under event-input permutation symmetry contains 24 spaces.
Space 8 differs as follows from the space induced by causal order $\total{\ev{A},\ev{B}}\vee\total{\ev{C},\ev{B}}$:
\begin{itemize}
  \item The outputs at events \evset{\ev{A}, \ev{B}} are independent of the input at event \ev{C} when the inputs at events \evset{A, B} are given by \hist{A/0,B/0}, \hist{A/0,B/1} and \hist{A/1,B/0}.
  \item The outputs at events \evset{\ev{B}, \ev{C}} are independent of the input at event \ev{A} when the inputs at events \evset{B, C} are given by \hist{B/0,C/0} and \hist{B/0,C/1}.
  \item The output at event \ev{B} is independent of the inputs at events \evset{\ev{A}, \ev{C}} when the input at event B is given by \hist{B/0}.
\end{itemize}

\noindent Below are the histories and extended histories for space 8: 
\begin{center}
    \begin{tabular}{cc}
    \includegraphics[height=3.5cm]{svg-inkscape/space-ABC-unique-tight-8-highlighted_svg-tex.pdf}
    &
    \includegraphics[height=3.5cm]{svg-inkscape/space-ABC-unique-tight-8-ext-highlighted_svg-tex.pdf}
    \\
    $\Theta_{8}$
    &
    $\Ext{\Theta_{8}}$
    \end{tabular}
\end{center}

\noindent The standard causaltope for Space 8 has dimension 31.
Below is a plot of the homogeneous linear system of causality and quasi-normalisation equations for the standard causaltope, put in reduced row echelon form:

\begin{center}
    \includegraphics[width=11cm]{svg-inkscape/space8-rref-eqs_svg-tex.pdf}
\end{center}

\noindent Rows correspond to the 32 independent linear equations.
Columns in the plot correspond to entries of empirical models, indexed as $i_A i_B i_C$ $o_A o_B o_C$.
Coefficients in the equations are color-coded as white=0, red=+1 and blue=-1.

Space 8 has closest refinements in equivalence classes 4 and 5; 
it is the join of its (closest) refinements.
It has closest coarsenings in equivalence classes 18, 20, 22 and 26; 
it is the meet of its (closest) coarsenings.
It has 256 causal functions, 128 of which are not causal for any of its refinements.
It is a tight space.

The standard causaltope for Space 8 has 2 more dimensions than those of its 3 subspaces in equivalence classes 4 and 5.
The standard causaltope for Space 8 is the meet of the standard causaltopes for its closest coarsenings.
For completeness, below is a plot of the full homogeneous linear system of causality and quasi-normalisation equations for the standard causaltope:

\begin{center}
    \includegraphics[width=12cm]{svg-inkscape/space8-eqs_svg-tex.pdf}
\end{center}

\noindent Rows correspond to the 73 linear equations, of which 32 are independent.


\newpage
\subsection*{Space 9}

Space 9 is not induced by a causal order, but it is a refinement of the space 33 induced by the definite causal order $\total{\ev{A},\ev{B}}\vee\discrete{\ev{C}}$.
Its equivalence class under event-input permutation symmetry contains 12 spaces.
Space 9 differs as follows from the space induced by causal order $\total{\ev{A},\ev{B}}\vee\discrete{\ev{C}}$:
\begin{itemize}
  \item The outputs at events \evset{\ev{B}, \ev{C}} are independent of the input at event \ev{A} when the inputs at events \evset{B, C} are given by \hist{B/1,C/0} and \hist{B/0,C/1}.
\end{itemize}

\noindent Below are the histories and extended histories for space 9: 
\begin{center}
    \begin{tabular}{cc}
    \includegraphics[height=3.5cm]{svg-inkscape/space-ABC-unique-untight-9-highlighted_svg-tex.pdf}
    &
    \includegraphics[height=3.5cm]{svg-inkscape/space-ABC-unique-untight-9-ext-highlighted_svg-tex.pdf}
    \\
    $\Theta_{9}$
    &
    $\Ext{\Theta_{9}}$
    \end{tabular}
\end{center}

\noindent The standard causaltope for Space 9 has dimension 28.
Below is a plot of the homogeneous linear system of causality and quasi-normalisation equations for the standard causaltope, put in reduced row echelon form:

\begin{center}
    \includegraphics[width=11cm]{svg-inkscape/space9-rref-eqs_svg-tex.pdf}
\end{center}

\noindent Rows correspond to the 35 independent linear equations.
Columns in the plot correspond to entries of empirical models, indexed as $i_A i_B i_C$ $o_A o_B o_C$.
Coefficients in the equations are color-coded as white=0, red=+1 and blue=-1.

Space 9 has closest refinements in equivalence class 6; 
it is the join of its (closest) refinements.
It has closest coarsenings in equivalence classes 19 and 24; 
it is the meet of its (closest) coarsenings.
It has 64 causal functions, all of which are causal for at least one of its refinements.
It is not a tight space: for event \ev{B}, a causal function must yield identical output values on input histories \hist{A/0,B/0}, \hist{A/1,B/0} and \hist{B/0,C/1}, and it must also yield identical output values on input histories \hist{A/0,B/1}, \hist{A/1,B/1} and \hist{B/1,C/0}.

The standard causaltope for Space 9 has 1 more dimension than those of its 2 subspaces in equivalence class 6.
The standard causaltope for Space 9 is the meet of the standard causaltopes for its closest coarsenings.
For completeness, below is a plot of the full homogeneous linear system of causality and quasi-normalisation equations for the standard causaltope:

\begin{center}
    \includegraphics[width=12cm]{svg-inkscape/space9-eqs_svg-tex.pdf}
\end{center}

\noindent Rows correspond to the 71 linear equations, of which 35 are independent.


\newpage
\subsection*{Space 10}

Space 10 is not induced by a causal order, but it is a refinement of the space in equivalence class 92 induced by the definite causal order $\total{\ev{A},\ev{B}}\vee\total{\ev{C},\ev{B}}$ (note that the space induced by the order is not the same as space 92).
Its equivalence class under event-input permutation symmetry contains 24 spaces.
Space 10 differs as follows from the space induced by causal order $\total{\ev{A},\ev{B}}\vee\total{\ev{C},\ev{B}}$:
\begin{itemize}
  \item The outputs at events \evset{\ev{A}, \ev{B}} are independent of the input at event \ev{C} when the inputs at events \evset{A, B} are given by \hist{A/0,B/0}, \hist{A/0,B/1} and \hist{A/1,B/0}.
  \item The outputs at events \evset{\ev{B}, \ev{C}} are independent of the input at event \ev{A} when the inputs at events \evset{B, C} are given by \hist{B/1,C/0}, \hist{B/1,C/1} and \hist{B/0,C/1}.
\end{itemize}

\noindent Below are the histories and extended histories for space 10: 
\begin{center}
    \begin{tabular}{cc}
    \includegraphics[height=3.5cm]{svg-inkscape/space-ABC-unique-untight-10-highlighted_svg-tex.pdf}
    &
    \includegraphics[height=3.5cm]{svg-inkscape/space-ABC-unique-untight-10-ext-highlighted_svg-tex.pdf}
    \\
    $\Theta_{10}$
    &
    $\Ext{\Theta_{10}}$
    \end{tabular}
\end{center}

\noindent The standard causaltope for Space 10 has dimension 28.
Below is a plot of the homogeneous linear system of causality and quasi-normalisation equations for the standard causaltope, put in reduced row echelon form:

\begin{center}
    \includegraphics[width=11cm]{svg-inkscape/space10-rref-eqs_svg-tex.pdf}
\end{center}

\noindent Rows correspond to the 35 independent linear equations.
Columns in the plot correspond to entries of empirical models, indexed as $i_A i_B i_C$ $o_A o_B o_C$.
Coefficients in the equations are color-coded as white=0, red=+1 and blue=-1.

Space 10 has closest refinements in equivalence class 6; 
it is the join of its (closest) refinements.
It has closest coarsenings in equivalence classes 16, 23 and 24; 
it is the meet of its (closest) coarsenings.
It has 64 causal functions, all of which are causal for at least one of its refinements.
It is not a tight space: for event \ev{B}, a causal function must yield identical output values on input histories \hist{A/0,B/0}, \hist{A/1,B/0} and \hist{B/0,C/1}, and it must also yield identical output values on input histories \hist{A/0,B/1}, \hist{B/1,C/0} and \hist{B/1,C/1}.

The standard causaltope for Space 10 has 1 more dimension than those of its 2 subspaces in equivalence class 6.
The standard causaltope for Space 10 is the meet of the standard causaltopes for its closest coarsenings.
For completeness, below is a plot of the full homogeneous linear system of causality and quasi-normalisation equations for the standard causaltope:

\begin{center}
    \includegraphics[width=12cm]{svg-inkscape/space10-eqs_svg-tex.pdf}
\end{center}

\noindent Rows correspond to the 71 linear equations, of which 35 are independent.


\newpage
\subsection*{Space 11}

Space 11 is not induced by a causal order, but it is a refinement of the space in equivalence class 100 induced by the definite causal order $\total{\ev{C},\ev{A},\ev{B}}$ (note that the space induced by the order is not the same as space 100).
Its equivalence class under event-input permutation symmetry contains 48 spaces.
Space 11 differs as follows from the space induced by causal order $\total{\ev{C},\ev{A},\ev{B}}$:
\begin{itemize}
  \item The outputs at events \evset{\ev{A}, \ev{B}} are independent of the input at event \ev{C} when the inputs at events \evset{A, B} are given by \hist{A/0,B/0}, \hist{A/0,B/1} and \hist{A/1,B/0}.
  \item The outputs at events \evset{\ev{B}, \ev{C}} are independent of the input at event \ev{A} when the inputs at events \evset{B, C} are given by \hist{B/1,C/1}, \hist{B/0,C/0} and \hist{B/0,C/1}.
  \item The output at event \ev{B} is independent of the inputs at events \evset{\ev{A}, \ev{C}} when the input at event B is given by \hist{B/0}.
  \item The output at event \ev{A} is independent of the input at event \ev{C} when the input at event A is given by \hist{A/0}.
\end{itemize}

\noindent Below are the histories and extended histories for space 11: 
\begin{center}
    \begin{tabular}{cc}
    \includegraphics[height=3.5cm]{svg-inkscape/space-ABC-unique-untight-11-highlighted_svg-tex.pdf}
    &
    \includegraphics[height=3.5cm]{svg-inkscape/space-ABC-unique-untight-11-ext-highlighted_svg-tex.pdf}
    \\
    $\Theta_{11}$
    &
    $\Ext{\Theta_{11}}$
    \end{tabular}
\end{center}

\noindent The standard causaltope for Space 11 has dimension 29.
Below is a plot of the homogeneous linear system of causality and quasi-normalisation equations for the standard causaltope, put in reduced row echelon form:

\begin{center}
    \includegraphics[width=11cm]{svg-inkscape/space11-rref-eqs_svg-tex.pdf}
\end{center}

\noindent Rows correspond to the 34 independent linear equations.
Columns in the plot correspond to entries of empirical models, indexed as $i_A i_B i_C$ $o_A o_B o_C$.
Coefficients in the equations are color-coded as white=0, red=+1 and blue=-1.

Space 11 has closest refinements in equivalence classes 4 and 7; 
it is the join of its (closest) refinements.
It has closest coarsenings in equivalence classes 17, 20, 21 and 22; 
it is the meet of its (closest) coarsenings.
It has 128 causal functions, all of which are causal for at least one of its refinements.
It is not a tight space: for event \ev{A}, a causal function must yield identical output values on input histories \hist{A/1,B/0}, \hist{A/1,C/0} and \hist{A/1,C/1}; for event \ev{B}, a causal function must yield identical output values on input histories \hist{A/0,B/1} and \hist{B/1,C/1}.

The standard causaltope for Space 11 coincides with that of its subspace in equivalence class 4.
The standard causaltope for Space 11 is the meet of the standard causaltopes for its closest coarsenings.
For completeness, below is a plot of the full homogeneous linear system of causality and quasi-normalisation equations for the standard causaltope:

\begin{center}
    \includegraphics[width=12cm]{svg-inkscape/space11-eqs_svg-tex.pdf}
\end{center}

\noindent Rows correspond to the 71 linear equations, of which 34 are independent.


\newpage
\subsection*{Space 12}

Space 12 is not induced by a causal order, but it is a refinement of the space 77 induced by the definite causal order $\total{\ev{A},\ev{B}}\vee\total{\ev{A},\ev{C}}$.
Its equivalence class under event-input permutation symmetry contains 24 spaces.
Space 12 differs as follows from the space induced by causal order $\total{\ev{A},\ev{B}}\vee\total{\ev{A},\ev{C}}$:
\begin{itemize}
  \item The outputs at events \evset{\ev{B}, \ev{C}} are independent of the input at event \ev{A} when the inputs at events \evset{B, C} are given by \hist{B/1,C/0} and \hist{B/1,C/1}.
  \item The output at event \ev{C} is independent of the input at event \ev{A} when the input at event C is given by \hist{C/0}.
  \item The output at event \ev{B} is independent of the input at event \ev{A} when the input at event B is given by \hist{B/1}.
\end{itemize}

\noindent Below are the histories and extended histories for space 12: 
\begin{center}
    \begin{tabular}{cc}
    \includegraphics[height=3.5cm]{svg-inkscape/space-ABC-unique-untight-12-highlighted_svg-tex.pdf}
    &
    \includegraphics[height=3.5cm]{svg-inkscape/space-ABC-unique-untight-12-ext-highlighted_svg-tex.pdf}
    \\
    $\Theta_{12}$
    &
    $\Ext{\Theta_{12}}$
    \end{tabular}
\end{center}

\noindent The standard causaltope for Space 12 has dimension 29.
Below is a plot of the homogeneous linear system of causality and quasi-normalisation equations for the standard causaltope, put in reduced row echelon form:

\begin{center}
    \includegraphics[width=11cm]{svg-inkscape/space12-rref-eqs_svg-tex.pdf}
\end{center}

\noindent Rows correspond to the 34 independent linear equations.
Columns in the plot correspond to entries of empirical models, indexed as $i_A i_B i_C$ $o_A o_B o_C$.
Coefficients in the equations are color-coded as white=0, red=+1 and blue=-1.

Space 12 has closest refinements in equivalence classes 5 and 7; 
it is the join of its (closest) refinements.
It has closest coarsenings in equivalence classes 17, 22, 25 and 27; 
it is the meet of its (closest) coarsenings.
It has 128 causal functions, all of which are causal for at least one of its refinements.
It is not a tight space: for event \ev{C}, a causal function must yield identical output values on input histories \hist{A/0,C/1}, \hist{A/1,C/1} and \hist{B/1,C/1}.

The standard causaltope for Space 12 coincides with that of its subspace in equivalence class 5.
The standard causaltope for Space 12 is the meet of the standard causaltopes for its closest coarsenings.
For completeness, below is a plot of the full homogeneous linear system of causality and quasi-normalisation equations for the standard causaltope:

\begin{center}
    \includegraphics[width=12cm]{svg-inkscape/space12-eqs_svg-tex.pdf}
\end{center}

\noindent Rows correspond to the 71 linear equations, of which 34 are independent.


\newpage
\subsection*{Space 13}

Space 13 is not induced by a causal order, but it is a refinement of the space 33 induced by the definite causal order $\total{\ev{A},\ev{B}}\vee\discrete{\ev{C}}$.
Its equivalence class under event-input permutation symmetry contains 12 spaces.
Space 13 differs as follows from the space induced by causal order $\total{\ev{A},\ev{B}}\vee\discrete{\ev{C}}$:
\begin{itemize}
  \item The outputs at events \evset{\ev{B}, \ev{C}} are independent of the input at event \ev{A} when the inputs at events \evset{B, C} are given by \hist{B/1,C/0} and \hist{B/0,C/0}.
\end{itemize}

\noindent Below are the histories and extended histories for space 13: 
\begin{center}
    \begin{tabular}{cc}
    \includegraphics[height=3.5cm]{svg-inkscape/space-ABC-unique-untight-13-highlighted_svg-tex.pdf}
    &
    \includegraphics[height=3.5cm]{svg-inkscape/space-ABC-unique-untight-13-ext-highlighted_svg-tex.pdf}
    \\
    $\Theta_{13}$
    &
    $\Ext{\Theta_{13}}$
    \end{tabular}
\end{center}

\noindent The standard causaltope for Space 13 has dimension 28.
Below is a plot of the homogeneous linear system of causality and quasi-normalisation equations for the standard causaltope, put in reduced row echelon form:

\begin{center}
    \includegraphics[width=11cm]{svg-inkscape/space13-rref-eqs_svg-tex.pdf}
\end{center}

\noindent Rows correspond to the 35 independent linear equations.
Columns in the plot correspond to entries of empirical models, indexed as $i_A i_B i_C$ $o_A o_B o_C$.
Coefficients in the equations are color-coded as white=0, red=+1 and blue=-1.

Space 13 has closest refinements in equivalence class 6; 
it is the join of its (closest) refinements.
It has closest coarsenings in equivalence classes 19, 23 and 27; 
it is the meet of its (closest) coarsenings.
It has 64 causal functions, all of which are causal for at least one of its refinements.
It is not a tight space: for event \ev{B}, a causal function must yield identical output values on input histories \hist{A/0,B/0}, \hist{A/1,B/0} and \hist{B/0,C/0}, and it must also yield identical output values on input histories \hist{A/0,B/1}, \hist{A/1,B/1} and \hist{B/1,C/0}.

The standard causaltope for Space 13 has 1 more dimension than those of its 2 subspaces in equivalence class 6.
The standard causaltope for Space 13 is the meet of the standard causaltopes for its closest coarsenings.
For completeness, below is a plot of the full homogeneous linear system of causality and quasi-normalisation equations for the standard causaltope:

\begin{center}
    \includegraphics[width=12cm]{svg-inkscape/space13-eqs_svg-tex.pdf}
\end{center}

\noindent Rows correspond to the 71 linear equations, of which 35 are independent.


\newpage
\subsection*{Space 14}

Space 14 is not induced by a causal order, but it is a refinement of the space 33 induced by the definite causal order $\total{\ev{A},\ev{B}}\vee\discrete{\ev{C}}$.
Its equivalence class under event-input permutation symmetry contains 12 spaces.
Space 14 differs as follows from the space induced by causal order $\total{\ev{A},\ev{B}}\vee\discrete{\ev{C}}$:
\begin{itemize}
  \item The outputs at events \evset{\ev{B}, \ev{C}} are independent of the input at event \ev{A} when the inputs at events \evset{B, C} are given by \hist{B/1,C/0} and \hist{B/1,C/1}.
\end{itemize}

\noindent Below are the histories and extended histories for space 14: 
\begin{center}
    \begin{tabular}{cc}
    \includegraphics[height=3.5cm]{svg-inkscape/space-ABC-unique-untight-14-highlighted_svg-tex.pdf}
    &
    \includegraphics[height=3.5cm]{svg-inkscape/space-ABC-unique-untight-14-ext-highlighted_svg-tex.pdf}
    \\
    $\Theta_{14}$
    &
    $\Ext{\Theta_{14}}$
    \end{tabular}
\end{center}

\noindent The standard causaltope for Space 14 has dimension 29.
Below is a plot of the homogeneous linear system of causality and quasi-normalisation equations for the standard causaltope, put in reduced row echelon form:

\begin{center}
    \includegraphics[width=11cm]{svg-inkscape/space14-rref-eqs_svg-tex.pdf}
\end{center}

\noindent Rows correspond to the 34 independent linear equations.
Columns in the plot correspond to entries of empirical models, indexed as $i_A i_B i_C$ $o_A o_B o_C$.
Coefficients in the equations are color-coded as white=0, red=+1 and blue=-1.

Space 14 has closest refinements in equivalence classes 5 and 6; 
it is the join of its (closest) refinements.
It has closest coarsenings in equivalence classes 16, 19 and 26; 
it is the meet of its (closest) coarsenings.
It has 128 causal functions, all of which are causal for at least one of its refinements.
It is not a tight space: for event \ev{B}, a causal function must yield identical output values on input histories \hist{A/0,B/1}, \hist{A/1,B/1}, \hist{B/1,C/0} and \hist{B/1,C/1}.

The standard causaltope for Space 14 coincides with that of its subspace in equivalence class 5.
The standard causaltope for Space 14 is the meet of the standard causaltopes for its closest coarsenings.
For completeness, below is a plot of the full homogeneous linear system of causality and quasi-normalisation equations for the standard causaltope:

\begin{center}
    \includegraphics[width=12cm]{svg-inkscape/space14-eqs_svg-tex.pdf}
\end{center}

\noindent Rows correspond to the 71 linear equations, of which 34 are independent.


\newpage
\subsection*{Space 15}

Space 15 is not induced by a causal order, but it is a refinement of the space in equivalence class 92 induced by the definite causal order $\total{\ev{A},\ev{B}}\vee\total{\ev{C},\ev{B}}$ (note that the space induced by the order is not the same as space 92).
Its equivalence class under event-input permutation symmetry contains 24 spaces.
Space 15 differs as follows from the space induced by causal order $\total{\ev{A},\ev{B}}\vee\total{\ev{C},\ev{B}}$:
\begin{itemize}
  \item The outputs at events \evset{\ev{A}, \ev{B}} are independent of the input at event \ev{C} when the inputs at events \evset{A, B} are given by \hist{A/0,B/0}, \hist{A/0,B/1} and \hist{A/1,B/1}.
  \item The outputs at events \evset{\ev{B}, \ev{C}} are independent of the input at event \ev{A} when the inputs at events \evset{B, C} are given by \hist{B/1,C/0}, \hist{B/1,C/1} and \hist{B/0,C/1}.
\end{itemize}

\noindent Below are the histories and extended histories for space 15: 
\begin{center}
    \begin{tabular}{cc}
    \includegraphics[height=3.5cm]{svg-inkscape/space-ABC-unique-untight-15-highlighted_svg-tex.pdf}
    &
    \includegraphics[height=3.5cm]{svg-inkscape/space-ABC-unique-untight-15-ext-highlighted_svg-tex.pdf}
    \\
    $\Theta_{15}$
    &
    $\Ext{\Theta_{15}}$
    \end{tabular}
\end{center}

\noindent The standard causaltope for Space 15 has dimension 29.
Below is a plot of the homogeneous linear system of causality and quasi-normalisation equations for the standard causaltope, put in reduced row echelon form:

\begin{center}
    \includegraphics[width=11cm]{svg-inkscape/space15-rref-eqs_svg-tex.pdf}
\end{center}

\noindent Rows correspond to the 34 independent linear equations.
Columns in the plot correspond to entries of empirical models, indexed as $i_A i_B i_C$ $o_A o_B o_C$.
Coefficients in the equations are color-coded as white=0, red=+1 and blue=-1.

Space 15 has closest refinements in equivalence classes 4 and 6; 
it is the join of its (closest) refinements.
It has closest coarsenings in equivalence classes 23, 24 and 26; 
it is the meet of its (closest) coarsenings.
It has 128 causal functions, 64 of which are not causal for any of its refinements.
It is not a tight space: for event \ev{B}, a causal function must yield identical output values on input histories \hist{A/0,B/0} and \hist{B/0,C/1}, and it must also yield identical output values on input histories \hist{A/0,B/1}, \hist{A/1,B/1}, \hist{B/1,C/0} and \hist{B/1,C/1}.

The standard causaltope for Space 15 coincides with that of its subspace in equivalence class 4.
The standard causaltope for Space 15 is the meet of the standard causaltopes for its closest coarsenings.
For completeness, below is a plot of the full homogeneous linear system of causality and quasi-normalisation equations for the standard causaltope:

\begin{center}
    \includegraphics[width=12cm]{svg-inkscape/space15-eqs_svg-tex.pdf}
\end{center}

\noindent Rows correspond to the 71 linear equations, of which 34 are independent.


\newpage
\subsection*{Space 16}

Space 16 is not induced by a causal order, but it is a refinement of the space in equivalence class 92 induced by the definite causal order $\total{\ev{A},\ev{B}}\vee\total{\ev{C},\ev{B}}$ (note that the space induced by the order is not the same as space 92).
Its equivalence class under event-input permutation symmetry contains 24 spaces.
Space 16 differs as follows from the space induced by causal order $\total{\ev{A},\ev{B}}\vee\total{\ev{C},\ev{B}}$:
\begin{itemize}
  \item The outputs at events \evset{\ev{A}, \ev{B}} are independent of the input at event \ev{C} when the inputs at events \evset{A, B} are given by \hist{A/0,B/0}, \hist{A/0,B/1} and \hist{A/1,B/0}.
  \item The outputs at events \evset{\ev{B}, \ev{C}} are independent of the input at event \ev{A} when the inputs at events \evset{B, C} are given by \hist{B/1,C/0} and \hist{B/1,C/1}.
\end{itemize}

\noindent Below are the histories and extended histories for space 16: 
\begin{center}
    \begin{tabular}{cc}
    \includegraphics[height=3.5cm]{svg-inkscape/space-ABC-unique-untight-16-highlighted_svg-tex.pdf}
    &
    \includegraphics[height=3.5cm]{svg-inkscape/space-ABC-unique-untight-16-ext-highlighted_svg-tex.pdf}
    \\
    $\Theta_{16}$
    &
    $\Ext{\Theta_{16}}$
    \end{tabular}
\end{center}

\noindent The standard causaltope for Space 16 has dimension 30.
Below is a plot of the homogeneous linear system of causality and quasi-normalisation equations for the standard causaltope, put in reduced row echelon form:

\begin{center}
    \includegraphics[width=11cm]{svg-inkscape/space16-rref-eqs_svg-tex.pdf}
\end{center}

\noindent Rows correspond to the 33 independent linear equations.
Columns in the plot correspond to entries of empirical models, indexed as $i_A i_B i_C$ $o_A o_B o_C$.
Coefficients in the equations are color-coded as white=0, red=+1 and blue=-1.

Space 16 has closest refinements in equivalence classes 10 and 14; 
it is the join of its (closest) refinements.
It has closest coarsenings in equivalence classes 29, 35, 39 and 41; 
it is the meet of its (closest) coarsenings.
It has 128 causal functions, all of which are causal for at least one of its refinements.
It is not a tight space: for event \ev{B}, a causal function must yield identical output values on input histories \hist{A/0,B/1}, \hist{B/1,C/0} and \hist{B/1,C/1}.

The standard causaltope for Space 16 has 1 more dimension than that of its subspace in equivalence class 14.
The standard causaltope for Space 16 is the meet of the standard causaltopes for its closest coarsenings.
For completeness, below is a plot of the full homogeneous linear system of causality and quasi-normalisation equations for the standard causaltope:

\begin{center}
    \includegraphics[width=12cm]{svg-inkscape/space16-eqs_svg-tex.pdf}
\end{center}

\noindent Rows correspond to the 67 linear equations, of which 33 are independent.


\newpage
\subsection*{Space 17}

Space 17 is not induced by a causal order, but it is a refinement of the space 100 induced by the definite causal order $\total{\ev{A},\ev{B},\ev{C}}$.
Its equivalence class under event-input permutation symmetry contains 48 spaces.
Space 17 differs as follows from the space induced by causal order $\total{\ev{A},\ev{B},\ev{C}}$:
\begin{itemize}
  \item The outputs at events \evset{\ev{B}, \ev{C}} are independent of the input at event \ev{A} when the inputs at events \evset{B, C} are given by \hist{B/1,C/0} and \hist{B/1,C/1}.
  \item The outputs at events \evset{\ev{A}, \ev{C}} are independent of the input at event \ev{B} when the inputs at events \evset{A, C} are given by \hist{A/0,C/0}, \hist{A/1,C/0} and \hist{A/1,C/1}.
  \item The output at event \ev{C} is independent of the inputs at events \evset{\ev{A}, \ev{B}} when the input at event C is given by \hist{C/0}.
  \item The output at event \ev{B} is independent of the input at event \ev{A} when the input at event B is given by \hist{B/1}.
\end{itemize}

\noindent Below are the histories and extended histories for space 17: 
\begin{center}
    \begin{tabular}{cc}
    \includegraphics[height=3.5cm]{svg-inkscape/space-ABC-unique-untight-17-highlighted_svg-tex.pdf}
    &
    \includegraphics[height=3.5cm]{svg-inkscape/space-ABC-unique-untight-17-ext-highlighted_svg-tex.pdf}
    \\
    $\Theta_{17}$
    &
    $\Ext{\Theta_{17}}$
    \end{tabular}
\end{center}

\noindent The standard causaltope for Space 17 has dimension 31.
Below is a plot of the homogeneous linear system of causality and quasi-normalisation equations for the standard causaltope, put in reduced row echelon form:

\begin{center}
    \includegraphics[width=11cm]{svg-inkscape/space17-rref-eqs_svg-tex.pdf}
\end{center}

\noindent Rows correspond to the 32 independent linear equations.
Columns in the plot correspond to entries of empirical models, indexed as $i_A i_B i_C$ $o_A o_B o_C$.
Coefficients in the equations are color-coded as white=0, red=+1 and blue=-1.

Space 17 has closest refinements in equivalence classes 11 and 12; 
it is the join of its (closest) refinements.
It has closest coarsenings in equivalence classes 28, 30, 32 and 43; 
it is the meet of its (closest) coarsenings.
It has 256 causal functions, 128 of which are not causal for any of its refinements.
It is not a tight space: for event \ev{C}, a causal function must yield identical output values on input histories \hist{A/1,C/1} and \hist{B/1,C/1}.

The standard causaltope for Space 17 has 2 more dimensions than those of its 2 subspaces in equivalence classes 11 and 12.
The standard causaltope for Space 17 is the meet of the standard causaltopes for its closest coarsenings.
For completeness, below is a plot of the full homogeneous linear system of causality and quasi-normalisation equations for the standard causaltope:

\begin{center}
    \includegraphics[width=12cm]{svg-inkscape/space17-eqs_svg-tex.pdf}
\end{center}

\noindent Rows correspond to the 67 linear equations, of which 32 are independent.


\newpage
\subsection*{Space 18}

Space 18 is not induced by a causal order, but it is a refinement of the space 92 induced by the definite causal order $\total{\ev{A},\ev{C}}\vee\total{\ev{B},\ev{C}}$.
Its equivalence class under event-input permutation symmetry contains 6 spaces.
Space 18 differs as follows from the space induced by causal order $\total{\ev{A},\ev{C}}\vee\total{\ev{B},\ev{C}}$:
\begin{itemize}
  \item The outputs at events \evset{\ev{A}, \ev{C}} are independent of the input at event \ev{B} when the inputs at events \evset{A, C} are given by \hist{A/0,C/1} and \hist{A/1,C/1}.
  \item The outputs at events \evset{\ev{B}, \ev{C}} are independent of the input at event \ev{A} when the inputs at events \evset{B, C} are given by \hist{B/1,C/1} and \hist{B/0,C/1}.
  \item The output at event \ev{C} is independent of the inputs at events \evset{\ev{A}, \ev{B}} when the input at event C is given by \hist{C/1}.
\end{itemize}

\noindent Below are the histories and extended histories for space 18: 
\begin{center}
    \begin{tabular}{cc}
    \includegraphics[height=3.5cm]{svg-inkscape/space-ABC-unique-tight-18-highlighted_svg-tex.pdf}
    &
    \includegraphics[height=3.5cm]{svg-inkscape/space-ABC-unique-tight-18-ext-highlighted_svg-tex.pdf}
    \\
    $\Theta_{18}$
    &
    $\Ext{\Theta_{18}}$
    \end{tabular}
\end{center}

\noindent The standard causaltope for Space 18 has dimension 33.
Below is a plot of the homogeneous linear system of causality and quasi-normalisation equations for the standard causaltope, put in reduced row echelon form:

\begin{center}
    \includegraphics[width=11cm]{svg-inkscape/space18-rref-eqs_svg-tex.pdf}
\end{center}

\noindent Rows correspond to the 30 independent linear equations.
Columns in the plot correspond to entries of empirical models, indexed as $i_A i_B i_C$ $o_A o_B o_C$.
Coefficients in the equations are color-coded as white=0, red=+1 and blue=-1.

Space 18 has closest refinements in equivalence class 8; 
it is the join of its (closest) refinements.
It has closest coarsenings in equivalence classes 31 and 44; 
it is the meet of its (closest) coarsenings.
It has 512 causal functions, 448 of which are not causal for any of its refinements.
It is a tight space.

The standard causaltope for Space 18 has 2 more dimensions than those of its 4 subspaces in equivalence class 8.
The standard causaltope for Space 18 is the meet of the standard causaltopes for its closest coarsenings.
For completeness, below is a plot of the full homogeneous linear system of causality and quasi-normalisation equations for the standard causaltope:

\begin{center}
    \includegraphics[width=12cm]{svg-inkscape/space18-eqs_svg-tex.pdf}
\end{center}

\noindent Rows correspond to the 69 linear equations, of which 30 are independent.


\newpage
\subsection*{Space 19}

Space 19 is not induced by a causal order, but it is a refinement of the space 33 induced by the definite causal order $\total{\ev{A},\ev{B}}\vee\discrete{\ev{C}}$.
Its equivalence class under event-input permutation symmetry contains 24 spaces.
Space 19 differs as follows from the space induced by causal order $\total{\ev{A},\ev{B}}\vee\discrete{\ev{C}}$:
\begin{itemize}
  \item The outputs at events \evset{\ev{B}, \ev{C}} are independent of the input at event \ev{A} when the inputs at events \evset{B, C} are given by \hist{B/1,C/0}.
\end{itemize}

\noindent Below are the histories and extended histories for space 19: 
\begin{center}
    \begin{tabular}{cc}
    \includegraphics[height=3.5cm]{svg-inkscape/space-ABC-unique-untight-19-highlighted_svg-tex.pdf}
    &
    \includegraphics[height=3.5cm]{svg-inkscape/space-ABC-unique-untight-19-ext-highlighted_svg-tex.pdf}
    \\
    $\Theta_{19}$
    &
    $\Ext{\Theta_{19}}$
    \end{tabular}
\end{center}

\noindent The standard causaltope for Space 19 has dimension 30.
Below is a plot of the homogeneous linear system of causality and quasi-normalisation equations for the standard causaltope, put in reduced row echelon form:

\begin{center}
    \includegraphics[width=11cm]{svg-inkscape/space19-rref-eqs_svg-tex.pdf}
\end{center}

\noindent Rows correspond to the 33 independent linear equations.
Columns in the plot correspond to entries of empirical models, indexed as $i_A i_B i_C$ $o_A o_B o_C$.
Coefficients in the equations are color-coded as white=0, red=+1 and blue=-1.

Space 19 has closest refinements in equivalence classes 9, 13 and 14; 
it is the join of its (closest) refinements.
It has closest coarsenings in equivalence classes 29, 33, 34 and 38; 
it is the meet of its (closest) coarsenings.
It has 128 causal functions, all of which are causal for at least one of its refinements.
It is not a tight space: for event \ev{B}, a causal function must yield identical output values on input histories \hist{A/0,B/1}, \hist{A/1,B/1} and \hist{B/1,C/0}.

The standard causaltope for Space 19 has 1 more dimension than that of its subspace in equivalence class 14.
The standard causaltope for Space 19 is the meet of the standard causaltopes for its closest coarsenings.
For completeness, below is a plot of the full homogeneous linear system of causality and quasi-normalisation equations for the standard causaltope:

\begin{center}
    \includegraphics[width=12cm]{svg-inkscape/space19-eqs_svg-tex.pdf}
\end{center}

\noindent Rows correspond to the 67 linear equations, of which 33 are independent.


\newpage
\subsection*{Space 20}

Space 20 is not induced by a causal order, but it is a refinement of the space 100 induced by the definite causal order $\total{\ev{A},\ev{B},\ev{C}}$.
Its equivalence class under event-input permutation symmetry contains 24 spaces.
Space 20 differs as follows from the space induced by causal order $\total{\ev{A},\ev{B},\ev{C}}$:
\begin{itemize}
  \item The outputs at events \evset{\ev{B}, \ev{C}} are independent of the input at event \ev{A} when the inputs at events \evset{B, C} are given by \hist{B/1,C/0}, \hist{B/0,C/0} and \hist{B/0,C/1}.
  \item The output at event \ev{B} is independent of the input at event \ev{A} when the input at event B is given by \hist{B/0}.
  \item The outputs at events \evset{\ev{A}, \ev{C}} are independent of the input at event \ev{B} when the inputs at events \evset{A, C} are given by \hist{A/0,C/0} and \hist{A/1,C/0}.
  \item The output at event \ev{C} is independent of the inputs at events \evset{\ev{A}, \ev{B}} when the input at event C is given by \hist{C/0}.
\end{itemize}

\noindent Below are the histories and extended histories for space 20: 
\begin{center}
    \begin{tabular}{cc}
    \includegraphics[height=3.5cm]{svg-inkscape/space-ABC-unique-untight-20-highlighted_svg-tex.pdf}
    &
    \includegraphics[height=3.5cm]{svg-inkscape/space-ABC-unique-untight-20-ext-highlighted_svg-tex.pdf}
    \\
    $\Theta_{20}$
    &
    $\Ext{\Theta_{20}}$
    \end{tabular}
\end{center}

\noindent The standard causaltope for Space 20 has dimension 31.
Below is a plot of the homogeneous linear system of causality and quasi-normalisation equations for the standard causaltope, put in reduced row echelon form:

\begin{center}
    \includegraphics[width=11cm]{svg-inkscape/space20-rref-eqs_svg-tex.pdf}
\end{center}

\noindent Rows correspond to the 32 independent linear equations.
Columns in the plot correspond to entries of empirical models, indexed as $i_A i_B i_C$ $o_A o_B o_C$.
Coefficients in the equations are color-coded as white=0, red=+1 and blue=-1.

Space 20 has closest refinements in equivalence classes 8 and 11; 
it is the join of its (closest) refinements.
It has closest coarsenings in equivalence classes 28 and 31; 
it is the meet of its (closest) coarsenings.
It has 256 causal functions, 192 of which are not causal for any of its refinements.
It is not a tight space: for event \ev{B}, a causal function must yield identical output values on input histories \hist{A/0,B/1}, \hist{A/1,B/1} and \hist{B/1,C/0}.

The standard causaltope for Space 20 coincides with that of its subspace in equivalence class 8.
The standard causaltope for Space 20 is the meet of the standard causaltopes for its closest coarsenings.
For completeness, below is a plot of the full homogeneous linear system of causality and quasi-normalisation equations for the standard causaltope:

\begin{center}
    \includegraphics[width=12cm]{svg-inkscape/space20-eqs_svg-tex.pdf}
\end{center}

\noindent Rows correspond to the 67 linear equations, of which 32 are independent.


\newpage
\subsection*{Space 21}

Space 21 is not induced by a causal order, but it is a refinement of the space induced by the indefinite causal order $\total{\ev{A},\{\ev{B},\ev{C}\}}$.
Its equivalence class under event-input permutation symmetry contains 24 spaces.
Space 21 differs as follows from the space induced by causal order $\total{\ev{A},\{\ev{B},\ev{C}\}}$:
\begin{itemize}
  \item The outputs at events \evset{\ev{A}, \ev{B}} are independent of the input at event \ev{C} when the inputs at events \evset{A, B} are given by \hist{A/0,B/0}, \hist{A/0,B/1} and \hist{A/1,B/0}.
  \item The outputs at events \evset{\ev{B}, \ev{C}} are independent of the input at event \ev{A} when the inputs at events \evset{B, C} are given by \hist{B/1,C/0}, \hist{B/0,C/0} and \hist{B/0,C/1}.
  \item The outputs at events \evset{\ev{A}, \ev{C}} are independent of the input at event \ev{B} when the inputs at events \evset{A, C} are given by \hist{A/0,C/0}, \hist{A/1,C/0} and \hist{A/1,C/1}.
  \item The output at event \ev{B} is independent of the inputs at events \evset{\ev{A}, \ev{C}} when the input at event B is given by \hist{B/0}.
  \item The output at event \ev{C} is independent of the inputs at events \evset{\ev{A}, \ev{B}} when the input at event C is given by \hist{C/0}.
\end{itemize}

\noindent Below are the histories and extended histories for space 21: 
\begin{center}
    \begin{tabular}{cc}
    \includegraphics[height=3.5cm]{svg-inkscape/space-ABC-unique-untight-21-highlighted_svg-tex.pdf}
    &
    \includegraphics[height=3.5cm]{svg-inkscape/space-ABC-unique-untight-21-ext-highlighted_svg-tex.pdf}
    \\
    $\Theta_{21}$
    &
    $\Ext{\Theta_{21}}$
    \end{tabular}
\end{center}

\noindent The standard causaltope for Space 21 has dimension 31.
Below is a plot of the homogeneous linear system of causality and quasi-normalisation equations for the standard causaltope, put in reduced row echelon form:

\begin{center}
    \includegraphics[width=11cm]{svg-inkscape/space21-rref-eqs_svg-tex.pdf}
\end{center}

\noindent Rows correspond to the 32 independent linear equations.
Columns in the plot correspond to entries of empirical models, indexed as $i_A i_B i_C$ $o_A o_B o_C$.
Coefficients in the equations are color-coded as white=0, red=+1 and blue=-1.

Space 21 has closest refinements in equivalence class 11; 
it is the join of its (closest) refinements.
It has closest coarsenings in equivalence class 32; 
it is the meet of its (closest) coarsenings.
It has 256 causal functions, 64 of which are not causal for any of its refinements.
It is not a tight space: for event \ev{B}, a causal function must yield identical output values on input histories \hist{A/0,B/1} and \hist{B/1,C/0}; for event \ev{C}, a causal function must yield identical output values on input histories \hist{A/1,C/1} and \hist{B/0,C/1}.

The standard causaltope for Space 21 has 2 more dimensions than those of its 2 subspaces in equivalence class 11.
The standard causaltope for Space 21 is the meet of the standard causaltopes for its closest coarsenings.
For completeness, below is a plot of the full homogeneous linear system of causality and quasi-normalisation equations for the standard causaltope:

\begin{center}
    \includegraphics[width=12cm]{svg-inkscape/space21-eqs_svg-tex.pdf}
\end{center}

\noindent Rows correspond to the 67 linear equations, of which 32 are independent.


\newpage
\subsection*{Space 22}

Space 22 is not induced by a causal order, but it is a refinement of the space 100 induced by the definite causal order $\total{\ev{A},\ev{B},\ev{C}}$.
Its equivalence class under event-input permutation symmetry contains 48 spaces.
Space 22 differs as follows from the space induced by causal order $\total{\ev{A},\ev{B},\ev{C}}$:
\begin{itemize}
  \item The outputs at events \evset{\ev{A}, \ev{C}} are independent of the input at event \ev{B} when the inputs at events \evset{A, C} are given by \hist{A/0,C/1}, \hist{A/1,C/0} and \hist{A/1,C/1}.
  \item The outputs at events \evset{\ev{B}, \ev{C}} are independent of the input at event \ev{A} when the inputs at events \evset{B, C} are given by \hist{B/1,C/1} and \hist{B/0,C/1}.
  \item The output at event \ev{C} is independent of the inputs at events \evset{\ev{A}, \ev{B}} when the input at event C is given by \hist{C/1}.
  \item The output at event \ev{B} is independent of the input at event \ev{A} when the input at event B is given by \hist{B/1}.
\end{itemize}

\noindent Below are the histories and extended histories for space 22: 
\begin{center}
    \begin{tabular}{cc}
    \includegraphics[height=3.5cm]{svg-inkscape/space-ABC-unique-untight-22-highlighted_svg-tex.pdf}
    &
    \includegraphics[height=3.5cm]{svg-inkscape/space-ABC-unique-untight-22-ext-highlighted_svg-tex.pdf}
    \\
    $\Theta_{22}$
    &
    $\Ext{\Theta_{22}}$
    \end{tabular}
\end{center}

\noindent The standard causaltope for Space 22 has dimension 31.
Below is a plot of the homogeneous linear system of causality and quasi-normalisation equations for the standard causaltope, put in reduced row echelon form:

\begin{center}
    \includegraphics[width=11cm]{svg-inkscape/space22-rref-eqs_svg-tex.pdf}
\end{center}

\noindent Rows correspond to the 32 independent linear equations.
Columns in the plot correspond to entries of empirical models, indexed as $i_A i_B i_C$ $o_A o_B o_C$.
Coefficients in the equations are color-coded as white=0, red=+1 and blue=-1.

Space 22 has closest refinements in equivalence classes 8, 11 and 12; 
it is the join of its (closest) refinements.
It has closest coarsenings in equivalence classes 30, 31, 32 and 42; 
it is the meet of its (closest) coarsenings.
It has 256 causal functions, all of which are causal for at least one of its refinements.
It is not a tight space: for event \ev{B}, a causal function must yield identical output values on input histories \hist{A/0,B/0}, \hist{A/1,B/0} and \hist{B/0,C/1}.

The standard causaltope for Space 22 coincides with that of its subspace in equivalence class 8.
The standard causaltope for Space 22 is the meet of the standard causaltopes for its closest coarsenings.
For completeness, below is a plot of the full homogeneous linear system of causality and quasi-normalisation equations for the standard causaltope:

\begin{center}
    \includegraphics[width=12cm]{svg-inkscape/space22-eqs_svg-tex.pdf}
\end{center}

\noindent Rows correspond to the 67 linear equations, of which 32 are independent.


\newpage
\subsection*{Space 23}

Space 23 is not induced by a causal order, but it is a refinement of the space in equivalence class 92 induced by the definite causal order $\total{\ev{A},\ev{B}}\vee\total{\ev{C},\ev{B}}$ (note that the space induced by the order is not the same as space 92).
Its equivalence class under event-input permutation symmetry contains 48 spaces.
Space 23 differs as follows from the space induced by causal order $\total{\ev{A},\ev{B}}\vee\total{\ev{C},\ev{B}}$:
\begin{itemize}
  \item The outputs at events \evset{\ev{A}, \ev{B}} are independent of the input at event \ev{C} when the inputs at events \evset{A, B} are given by \hist{A/0,B/0}, \hist{A/0,B/1} and \hist{A/1,B/0}.
  \item The outputs at events \evset{\ev{B}, \ev{C}} are independent of the input at event \ev{A} when the inputs at events \evset{B, C} are given by \hist{B/1,C/1} and \hist{B/0,C/1}.
\end{itemize}

\noindent Below are the histories and extended histories for space 23: 
\begin{center}
    \begin{tabular}{cc}
    \includegraphics[height=3.5cm]{svg-inkscape/space-ABC-unique-untight-23-highlighted_svg-tex.pdf}
    &
    \includegraphics[height=3.5cm]{svg-inkscape/space-ABC-unique-untight-23-ext-highlighted_svg-tex.pdf}
    \\
    $\Theta_{23}$
    &
    $\Ext{\Theta_{23}}$
    \end{tabular}
\end{center}

\noindent The standard causaltope for Space 23 has dimension 30.
Below is a plot of the homogeneous linear system of causality and quasi-normalisation equations for the standard causaltope, put in reduced row echelon form:

\begin{center}
    \includegraphics[width=11cm]{svg-inkscape/space23-rref-eqs_svg-tex.pdf}
\end{center}

\noindent Rows correspond to the 33 independent linear equations.
Columns in the plot correspond to entries of empirical models, indexed as $i_A i_B i_C$ $o_A o_B o_C$.
Coefficients in the equations are color-coded as white=0, red=+1 and blue=-1.

Space 23 has closest refinements in equivalence classes 10, 13 and 15; 
it is the join of its (closest) refinements.
It has closest coarsenings in equivalence classes 29, 34, 35, 36, 37 and 43; 
it is the meet of its (closest) coarsenings.
It has 128 causal functions, all of which are causal for at least one of its refinements.
It is not a tight space: for event \ev{B}, a causal function must yield identical output values on input histories \hist{A/0,B/1} and \hist{B/1,C/1}, and it must also yield identical output values on input histories \hist{A/0,B/0}, \hist{A/1,B/0} and \hist{B/0,C/1}.

The standard causaltope for Space 23 has 1 more dimension than that of its subspace in equivalence class 15.
The standard causaltope for Space 23 is the meet of the standard causaltopes for its closest coarsenings.
For completeness, below is a plot of the full homogeneous linear system of causality and quasi-normalisation equations for the standard causaltope:

\begin{center}
    \includegraphics[width=12cm]{svg-inkscape/space23-eqs_svg-tex.pdf}
\end{center}

\noindent Rows correspond to the 67 linear equations, of which 33 are independent.


\newpage
\subsection*{Space 24}

Space 24 is not induced by a causal order, but it is a refinement of the space in equivalence class 92 induced by the definite causal order $\total{\ev{A},\ev{B}}\vee\total{\ev{C},\ev{B}}$ (note that the space induced by the order is not the same as space 92).
Its equivalence class under event-input permutation symmetry contains 48 spaces.
Space 24 differs as follows from the space induced by causal order $\total{\ev{A},\ev{B}}\vee\total{\ev{C},\ev{B}}$:
\begin{itemize}
  \item The outputs at events \evset{\ev{A}, \ev{B}} are independent of the input at event \ev{C} when the inputs at events \evset{A, B} are given by \hist{A/0,B/0}, \hist{A/0,B/1} and \hist{A/1,B/0}.
  \item The outputs at events \evset{\ev{B}, \ev{C}} are independent of the input at event \ev{A} when the inputs at events \evset{B, C} are given by \hist{B/1,C/0} and \hist{B/0,C/1}.
\end{itemize}

\noindent Below are the histories and extended histories for space 24: 
\begin{center}
    \begin{tabular}{cc}
    \includegraphics[height=3.5cm]{svg-inkscape/space-ABC-unique-untight-24-highlighted_svg-tex.pdf}
    &
    \includegraphics[height=3.5cm]{svg-inkscape/space-ABC-unique-untight-24-ext-highlighted_svg-tex.pdf}
    \\
    $\Theta_{24}$
    &
    $\Ext{\Theta_{24}}$
    \end{tabular}
\end{center}

\noindent The standard causaltope for Space 24 has dimension 30.
Below is a plot of the homogeneous linear system of causality and quasi-normalisation equations for the standard causaltope, put in reduced row echelon form:

\begin{center}
    \includegraphics[width=11cm]{svg-inkscape/space24-rref-eqs_svg-tex.pdf}
\end{center}

\noindent Rows correspond to the 33 independent linear equations.
Columns in the plot correspond to entries of empirical models, indexed as $i_A i_B i_C$ $o_A o_B o_C$.
Coefficients in the equations are color-coded as white=0, red=+1 and blue=-1.

Space 24 has closest refinements in equivalence classes 9, 10 and 15; 
it is the join of its (closest) refinements.
It has closest coarsenings in equivalence classes 29, 34, 36, 39 and 40; 
it is the meet of its (closest) coarsenings.
It has 128 causal functions, all of which are causal for at least one of its refinements.
It is not a tight space: for event \ev{B}, a causal function must yield identical output values on input histories \hist{A/0,B/1} and \hist{B/1,C/0}, and it must also yield identical output values on input histories \hist{A/0,B/0}, \hist{A/1,B/0} and \hist{B/0,C/1}.

The standard causaltope for Space 24 has 1 more dimension than that of its subspace in equivalence class 15.
The standard causaltope for Space 24 is the meet of the standard causaltopes for its closest coarsenings.
For completeness, below is a plot of the full homogeneous linear system of causality and quasi-normalisation equations for the standard causaltope:

\begin{center}
    \includegraphics[width=12cm]{svg-inkscape/space24-eqs_svg-tex.pdf}
\end{center}

\noindent Rows correspond to the 67 linear equations, of which 33 are independent.


\newpage
\subsection*{Space 25}

Space 25 is not induced by a causal order, but it is a refinement of the space 77 induced by the definite causal order $\total{\ev{A},\ev{B}}\vee\total{\ev{A},\ev{C}}$.
Its equivalence class under event-input permutation symmetry contains 12 spaces.
Space 25 differs as follows from the space induced by causal order $\total{\ev{A},\ev{B}}\vee\total{\ev{A},\ev{C}}$:
\begin{itemize}
  \item The outputs at events \evset{\ev{B}, \ev{C}} are independent of the input at event \ev{A} when the inputs at events \evset{B, C} are given by \hist{B/1,C/0}.
  \item The output at event \ev{C} is independent of the input at event \ev{A} when the input at event C is given by \hist{C/0}.
  \item The output at event \ev{B} is independent of the input at event \ev{A} when the input at event B is given by \hist{B/1}.
\end{itemize}

\noindent Below are the histories and extended histories for space 25: 
\begin{center}
    \begin{tabular}{cc}
    \includegraphics[height=3.5cm]{svg-inkscape/space-ABC-unique-tight-25-highlighted_svg-tex.pdf}
    &
    \includegraphics[height=3.5cm]{svg-inkscape/space-ABC-unique-tight-25-ext-highlighted_svg-tex.pdf}
    \\
    $\Theta_{25}$
    &
    $\Ext{\Theta_{25}}$
    \end{tabular}
\end{center}

\noindent The standard causaltope for Space 25 has dimension 31.
Below is a plot of the homogeneous linear system of causality and quasi-normalisation equations for the standard causaltope, put in reduced row echelon form:

\begin{center}
    \includegraphics[width=11cm]{svg-inkscape/space25-rref-eqs_svg-tex.pdf}
\end{center}

\noindent Rows correspond to the 32 independent linear equations.
Columns in the plot correspond to entries of empirical models, indexed as $i_A i_B i_C$ $o_A o_B o_C$.
Coefficients in the equations are color-coded as white=0, red=+1 and blue=-1.

Space 25 has closest refinements in equivalence class 12; 
it is the join of its (closest) refinements.
It has closest coarsenings in equivalence classes 30 and 38; 
it is the meet of its (closest) coarsenings.
It has 256 causal functions, 128 of which are not causal for any of its refinements.
It is a tight space.

The standard causaltope for Space 25 has 2 more dimensions than those of its 2 subspaces in equivalence class 12.
The standard causaltope for Space 25 is the meet of the standard causaltopes for its closest coarsenings.
For completeness, below is a plot of the full homogeneous linear system of causality and quasi-normalisation equations for the standard causaltope:

\begin{center}
    \includegraphics[width=12cm]{svg-inkscape/space25-eqs_svg-tex.pdf}
\end{center}

\noindent Rows correspond to the 67 linear equations, of which 32 are independent.


\newpage
\subsection*{Space 26}

Space 26 is not induced by a causal order, but it is a refinement of the space in equivalence class 92 induced by the definite causal order $\total{\ev{A},\ev{B}}\vee\total{\ev{C},\ev{B}}$ (note that the space induced by the order is not the same as space 92).
Its equivalence class under event-input permutation symmetry contains 24 spaces.
Space 26 differs as follows from the space induced by causal order $\total{\ev{A},\ev{B}}\vee\total{\ev{C},\ev{B}}$:
\begin{itemize}
  \item The outputs at events \evset{\ev{A}, \ev{B}} are independent of the input at event \ev{C} when the inputs at events \evset{A, B} are given by \hist{A/0,B/0}, \hist{A/0,B/1} and \hist{A/1,B/1}.
  \item The outputs at events \evset{\ev{B}, \ev{C}} are independent of the input at event \ev{A} when the inputs at events \evset{B, C} are given by \hist{B/1,C/0} and \hist{B/1,C/1}.
\end{itemize}

\noindent Below are the histories and extended histories for space 26: 
\begin{center}
    \begin{tabular}{cc}
    \includegraphics[height=3.5cm]{svg-inkscape/space-ABC-unique-untight-26-highlighted_svg-tex.pdf}
    &
    \includegraphics[height=3.5cm]{svg-inkscape/space-ABC-unique-untight-26-ext-highlighted_svg-tex.pdf}
    \\
    $\Theta_{26}$
    &
    $\Ext{\Theta_{26}}$
    \end{tabular}
\end{center}

\noindent The standard causaltope for Space 26 has dimension 31.
Below is a plot of the homogeneous linear system of causality and quasi-normalisation equations for the standard causaltope, put in reduced row echelon form:

\begin{center}
    \includegraphics[width=11cm]{svg-inkscape/space26-rref-eqs_svg-tex.pdf}
\end{center}

\noindent Rows correspond to the 32 independent linear equations.
Columns in the plot correspond to entries of empirical models, indexed as $i_A i_B i_C$ $o_A o_B o_C$.
Coefficients in the equations are color-coded as white=0, red=+1 and blue=-1.

Space 26 has closest refinements in equivalence classes 8, 14 and 15; 
it is the join of its (closest) refinements.
It has closest coarsenings in equivalence classes 34, 35, 39 and 44; 
it is the meet of its (closest) coarsenings.
It has 256 causal functions, 64 of which are not causal for any of its refinements.
It is not a tight space: for event \ev{B}, a causal function must yield identical output values on input histories \hist{A/0,B/1}, \hist{A/1,B/1}, \hist{B/1,C/0} and \hist{B/1,C/1}.

The standard causaltope for Space 26 coincides with that of its subspace in equivalence class 8.
The standard causaltope for Space 26 is the meet of the standard causaltopes for its closest coarsenings.
For completeness, below is a plot of the full homogeneous linear system of causality and quasi-normalisation equations for the standard causaltope:

\begin{center}
    \includegraphics[width=12cm]{svg-inkscape/space26-eqs_svg-tex.pdf}
\end{center}

\noindent Rows correspond to the 67 linear equations, of which 32 are independent.


\newpage
\subsection*{Space 27}

Space 27 is not induced by a causal order, but it is a refinement of the space 77 induced by the definite causal order $\total{\ev{A},\ev{B}}\vee\total{\ev{A},\ev{C}}$.
Its equivalence class under event-input permutation symmetry contains 12 spaces.
Space 27 differs as follows from the space induced by causal order $\total{\ev{A},\ev{B}}\vee\total{\ev{A},\ev{C}}$:
\begin{itemize}
  \item The outputs at events \evset{\ev{B}, \ev{C}} are independent of the input at event \ev{A} when the inputs at events \evset{B, C} are given by \hist{B/1,C/0} and \hist{B/1,C/1}.
  \item The output at event \ev{B} is independent of the input at event \ev{A} when the input at event B is given by \hist{B/1}.
\end{itemize}

\noindent Below are the histories and extended histories for space 27: 
\begin{center}
    \begin{tabular}{cc}
    \includegraphics[height=3.5cm]{svg-inkscape/space-ABC-unique-untight-27-highlighted_svg-tex.pdf}
    &
    \includegraphics[height=3.5cm]{svg-inkscape/space-ABC-unique-untight-27-ext-highlighted_svg-tex.pdf}
    \\
    $\Theta_{27}$
    &
    $\Ext{\Theta_{27}}$
    \end{tabular}
\end{center}

\noindent The standard causaltope for Space 27 has dimension 29.
Below is a plot of the homogeneous linear system of causality and quasi-normalisation equations for the standard causaltope, put in reduced row echelon form:

\begin{center}
    \includegraphics[width=11cm]{svg-inkscape/space27-rref-eqs_svg-tex.pdf}
\end{center}

\noindent Rows correspond to the 34 independent linear equations.
Columns in the plot correspond to entries of empirical models, indexed as $i_A i_B i_C$ $o_A o_B o_C$.
Coefficients in the equations are color-coded as white=0, red=+1 and blue=-1.

Space 27 has closest refinements in equivalence classes 12 and 13; 
it is the join of its (closest) refinements.
It has closest coarsenings in equivalence classes 38, 42 and 43; 
it is the meet of its (closest) coarsenings.
It has 128 causal functions, all of which are causal for at least one of its refinements.
It is not a tight space: for event \ev{C}, a causal function must yield identical output values on input histories \hist{A/0,C/0}, \hist{A/1,C/0} and \hist{B/1,C/0}, and it must also yield identical output values on input histories \hist{A/0,C/1}, \hist{A/1,C/1} and \hist{B/1,C/1}.

The standard causaltope for Space 27 coincides with that of its 2 subspaces in equivalence class 12.
The standard causaltope for Space 27 is the meet of the standard causaltopes for its closest coarsenings.
For completeness, below is a plot of the full homogeneous linear system of causality and quasi-normalisation equations for the standard causaltope:

\begin{center}
    \includegraphics[width=12cm]{svg-inkscape/space27-eqs_svg-tex.pdf}
\end{center}

\noindent Rows correspond to the 65 linear equations, of which 34 are independent.


\newpage
\subsection*{Space 28}

Space 28 is not induced by a causal order, but it is a refinement of the space 100 induced by the definite causal order $\total{\ev{A},\ev{B},\ev{C}}$.
Its equivalence class under event-input permutation symmetry contains 24 spaces.
Space 28 differs as follows from the space induced by causal order $\total{\ev{A},\ev{B},\ev{C}}$:
\begin{itemize}
  \item The outputs at events \evset{\ev{B}, \ev{C}} are independent of the input at event \ev{A} when the inputs at events \evset{B, C} are given by \hist{B/1,C/0} and \hist{B/1,C/1}.
  \item The outputs at events \evset{\ev{A}, \ev{C}} are independent of the input at event \ev{B} when the inputs at events \evset{A, C} are given by \hist{A/0,C/0} and \hist{A/1,C/0}.
  \item The output at event \ev{C} is independent of the inputs at events \evset{\ev{A}, \ev{B}} when the input at event C is given by \hist{C/0}.
  \item The output at event \ev{B} is independent of the input at event \ev{A} when the input at event B is given by \hist{B/1}.
\end{itemize}

\noindent Below are the histories and extended histories for space 28: 
\begin{center}
    \begin{tabular}{cc}
    \includegraphics[height=3.5cm]{svg-inkscape/space-ABC-unique-tight-28-highlighted_svg-tex.pdf}
    &
    \includegraphics[height=3.5cm]{svg-inkscape/space-ABC-unique-tight-28-ext-highlighted_svg-tex.pdf}
    \\
    $\Theta_{28}$
    &
    $\Ext{\Theta_{28}}$
    \end{tabular}
\end{center}

\noindent The standard causaltope for Space 28 has dimension 33.
Below is a plot of the homogeneous linear system of causality and quasi-normalisation equations for the standard causaltope, put in reduced row echelon form:

\begin{center}
    \includegraphics[width=11cm]{svg-inkscape/space28-rref-eqs_svg-tex.pdf}
\end{center}

\noindent Rows correspond to the 30 independent linear equations.
Columns in the plot correspond to entries of empirical models, indexed as $i_A i_B i_C$ $o_A o_B o_C$.
Coefficients in the equations are color-coded as white=0, red=+1 and blue=-1.

Space 28 has closest refinements in equivalence classes 17 and 20; 
it is the join of its (closest) refinements.
It has closest coarsenings in equivalence classes 45 and 54; 
it is the meet of its (closest) coarsenings.
It has 512 causal functions, 192 of which are not causal for any of its refinements.
It is a tight space.

The standard causaltope for Space 28 has 2 more dimensions than those of its 3 subspaces in equivalence classes 17 and 20.
The standard causaltope for Space 28 is the meet of the standard causaltopes for its closest coarsenings.
For completeness, below is a plot of the full homogeneous linear system of causality and quasi-normalisation equations for the standard causaltope:

\begin{center}
    \includegraphics[width=12cm]{svg-inkscape/space28-eqs_svg-tex.pdf}
\end{center}

\noindent Rows correspond to the 63 linear equations, of which 30 are independent.


\newpage
\subsection*{Space 29}

Space 29 is not induced by a causal order, but it is a refinement of the space in equivalence class 92 induced by the definite causal order $\total{\ev{A},\ev{B}}\vee\total{\ev{C},\ev{B}}$ (note that the space induced by the order is not the same as space 92).
Its equivalence class under event-input permutation symmetry contains 48 spaces.
Space 29 differs as follows from the space induced by causal order $\total{\ev{A},\ev{B}}\vee\total{\ev{C},\ev{B}}$:
\begin{itemize}
  \item The outputs at events \evset{\ev{A}, \ev{B}} are independent of the input at event \ev{C} when the inputs at events \evset{A, B} are given by \hist{A/0,B/0}, \hist{A/0,B/1} and \hist{A/1,B/0}.
  \item The outputs at events \evset{\ev{B}, \ev{C}} are independent of the input at event \ev{A} when the inputs at events \evset{B, C} are given by \hist{B/1,C/1}.
\end{itemize}

\noindent Below are the histories and extended histories for space 29: 
\begin{center}
    \begin{tabular}{cc}
    \includegraphics[height=3.5cm]{svg-inkscape/space-ABC-unique-untight-29-highlighted_svg-tex.pdf}
    &
    \includegraphics[height=3.5cm]{svg-inkscape/space-ABC-unique-untight-29-ext-highlighted_svg-tex.pdf}
    \\
    $\Theta_{29}$
    &
    $\Ext{\Theta_{29}}$
    \end{tabular}
\end{center}

\noindent The standard causaltope for Space 29 has dimension 32.
Below is a plot of the homogeneous linear system of causality and quasi-normalisation equations for the standard causaltope, put in reduced row echelon form:

\begin{center}
    \includegraphics[width=11cm]{svg-inkscape/space29-rref-eqs_svg-tex.pdf}
\end{center}

\noindent Rows correspond to the 31 independent linear equations.
Columns in the plot correspond to entries of empirical models, indexed as $i_A i_B i_C$ $o_A o_B o_C$.
Coefficients in the equations are color-coded as white=0, red=+1 and blue=-1.

Space 29 has closest refinements in equivalence classes 16, 19, 23 and 24; 
it is the join of its (closest) refinements.
It has closest coarsenings in equivalence classes 46, 49, 50, 53 and 56; 
it is the meet of its (closest) coarsenings.
It has 256 causal functions, 64 of which are not causal for any of its refinements.
It is not a tight space: for event \ev{B}, a causal function must yield identical output values on input histories \hist{A/0,B/1} and \hist{B/1,C/1}.

The standard causaltope for Space 29 has 2 more dimensions than those of its 4 subspaces in equivalence classes 16, 19, 23 and 24.
The standard causaltope for Space 29 is the meet of the standard causaltopes for its closest coarsenings.
For completeness, below is a plot of the full homogeneous linear system of causality and quasi-normalisation equations for the standard causaltope:

\begin{center}
    \includegraphics[width=12cm]{svg-inkscape/space29-eqs_svg-tex.pdf}
\end{center}

\noindent Rows correspond to the 63 linear equations, of which 31 are independent.


\newpage
\subsection*{Space 30}

Space 30 is not induced by a causal order, but it is a refinement of the space 100 induced by the definite causal order $\total{\ev{A},\ev{B},\ev{C}}$.
Its equivalence class under event-input permutation symmetry contains 48 spaces.
Space 30 differs as follows from the space induced by causal order $\total{\ev{A},\ev{B},\ev{C}}$:
\begin{itemize}
  \item The outputs at events \evset{\ev{B}, \ev{C}} are independent of the input at event \ev{A} when the inputs at events \evset{B, C} are given by \hist{B/1,C/0}.
  \item The outputs at events \evset{\ev{A}, \ev{C}} are independent of the input at event \ev{B} when the inputs at events \evset{A, C} are given by \hist{A/0,C/0}, \hist{A/1,C/0} and \hist{A/1,C/1}.
  \item The output at event \ev{C} is independent of the inputs at events \evset{\ev{A}, \ev{B}} when the input at event C is given by \hist{C/0}.
  \item The output at event \ev{B} is independent of the input at event \ev{A} when the input at event B is given by \hist{B/1}.
\end{itemize}

\noindent Below are the histories and extended histories for space 30: 
\begin{center}
    \begin{tabular}{cc}
    \includegraphics[height=3.5cm]{svg-inkscape/space-ABC-unique-tight-30-highlighted_svg-tex.pdf}
    &
    \includegraphics[height=3.5cm]{svg-inkscape/space-ABC-unique-tight-30-ext-highlighted_svg-tex.pdf}
    \\
    $\Theta_{30}$
    &
    $\Ext{\Theta_{30}}$
    \end{tabular}
\end{center}

\noindent The standard causaltope for Space 30 has dimension 33.
Below is a plot of the homogeneous linear system of causality and quasi-normalisation equations for the standard causaltope, put in reduced row echelon form:

\begin{center}
    \includegraphics[width=11cm]{svg-inkscape/space30-rref-eqs_svg-tex.pdf}
\end{center}

\noindent Rows correspond to the 30 independent linear equations.
Columns in the plot correspond to entries of empirical models, indexed as $i_A i_B i_C$ $o_A o_B o_C$.
Coefficients in the equations are color-coded as white=0, red=+1 and blue=-1.

Space 30 has closest refinements in equivalence classes 17, 22 and 25; 
it is the join of its (closest) refinements.
It has closest coarsenings in equivalence classes 45, 47, 48 and 57; 
it is the meet of its (closest) coarsenings.
It has 512 causal functions, 64 of which are not causal for any of its refinements.
It is a tight space.

The standard causaltope for Space 30 has 2 more dimensions than those of its 3 subspaces in equivalence classes 17, 22 and 25.
The standard causaltope for Space 30 is the meet of the standard causaltopes for its closest coarsenings.
For completeness, below is a plot of the full homogeneous linear system of causality and quasi-normalisation equations for the standard causaltope:

\begin{center}
    \includegraphics[width=12cm]{svg-inkscape/space30-eqs_svg-tex.pdf}
\end{center}

\noindent Rows correspond to the 63 linear equations, of which 30 are independent.


\newpage
\subsection*{Space 31}

Space 31 is not induced by a causal order, but it is a refinement of the space 100 induced by the definite causal order $\total{\ev{A},\ev{B},\ev{C}}$.
Its equivalence class under event-input permutation symmetry contains 24 spaces.
Space 31 differs as follows from the space induced by causal order $\total{\ev{A},\ev{B},\ev{C}}$:
\begin{itemize}
  \item The outputs at events \evset{\ev{A}, \ev{C}} are independent of the input at event \ev{B} when the inputs at events \evset{A, C} are given by \hist{A/0,C/1} and \hist{A/1,C/1}.
  \item The outputs at events \evset{\ev{B}, \ev{C}} are independent of the input at event \ev{A} when the inputs at events \evset{B, C} are given by \hist{B/1,C/1} and \hist{B/0,C/1}.
  \item The output at event \ev{B} is independent of the input at event \ev{A} when the input at event B is given by \hist{B/0}.
  \item The output at event \ev{C} is independent of the inputs at events \evset{\ev{A}, \ev{B}} when the input at event C is given by \hist{C/1}.
\end{itemize}

\noindent Below are the histories and extended histories for space 31: 
\begin{center}
    \begin{tabular}{cc}
    \includegraphics[height=3.5cm]{svg-inkscape/space-ABC-unique-untight-31-highlighted_svg-tex.pdf}
    &
    \includegraphics[height=3.5cm]{svg-inkscape/space-ABC-unique-untight-31-ext-highlighted_svg-tex.pdf}
    \\
    $\Theta_{31}$
    &
    $\Ext{\Theta_{31}}$
    \end{tabular}
\end{center}

\noindent The standard causaltope for Space 31 has dimension 33.
Below is a plot of the homogeneous linear system of causality and quasi-normalisation equations for the standard causaltope, put in reduced row echelon form:

\begin{center}
    \includegraphics[width=11cm]{svg-inkscape/space31-rref-eqs_svg-tex.pdf}
\end{center}

\noindent Rows correspond to the 30 independent linear equations.
Columns in the plot correspond to entries of empirical models, indexed as $i_A i_B i_C$ $o_A o_B o_C$.
Coefficients in the equations are color-coded as white=0, red=+1 and blue=-1.

Space 31 has closest refinements in equivalence classes 18, 20 and 22; 
it is the join of its (closest) refinements.
It has closest coarsenings in equivalence classes 45 and 51; 
it is the meet of its (closest) coarsenings.
It has 512 causal functions, all of which are causal for at least one of its refinements.
It is not a tight space: for event \ev{B}, a causal function must yield identical output values on input histories \hist{A/0,B/1}, \hist{A/1,B/1} and \hist{B/1,C/1}.

The standard causaltope for Space 31 coincides with that of its subspace in equivalence class 18.
The standard causaltope for Space 31 is the meet of the standard causaltopes for its closest coarsenings.
For completeness, below is a plot of the full homogeneous linear system of causality and quasi-normalisation equations for the standard causaltope:

\begin{center}
    \includegraphics[width=12cm]{svg-inkscape/space31-eqs_svg-tex.pdf}
\end{center}

\noindent Rows correspond to the 63 linear equations, of which 30 are independent.


\newpage
\subsection*{Space 32}

Space 32 is not induced by a causal order, but it is a refinement of the space induced by the indefinite causal order $\total{\ev{A},\{\ev{B},\ev{C}\}}$.
Its equivalence class under event-input permutation symmetry contains 48 spaces.
Space 32 differs as follows from the space induced by causal order $\total{\ev{A},\{\ev{B},\ev{C}\}}$:
\begin{itemize}
  \item The outputs at events \evset{\ev{A}, \ev{B}} are independent of the input at event \ev{C} when the inputs at events \evset{A, B} are given by \hist{A/0,B/0}, \hist{A/0,B/1} and \hist{A/1,B/0}.
  \item The outputs at events \evset{\ev{A}, \ev{C}} are independent of the input at event \ev{B} when the inputs at events \evset{A, C} are given by \hist{A/0,C/1}, \hist{A/1,C/0} and \hist{A/1,C/1}.
  \item The outputs at events \evset{\ev{B}, \ev{C}} are independent of the input at event \ev{A} when the inputs at events \evset{B, C} are given by \hist{B/1,C/1} and \hist{B/0,C/1}.
  \item The output at event \ev{B} is independent of the inputs at events \evset{\ev{A}, \ev{C}} when the input at event B is given by \hist{B/0}.
  \item The output at event \ev{C} is independent of the inputs at events \evset{\ev{A}, \ev{B}} when the input at event C is given by \hist{C/1}.
\end{itemize}

\noindent Below are the histories and extended histories for space 32: 
\begin{center}
    \begin{tabular}{cc}
    \includegraphics[height=3.5cm]{svg-inkscape/space-ABC-unique-untight-32-highlighted_svg-tex.pdf}
    &
    \includegraphics[height=3.5cm]{svg-inkscape/space-ABC-unique-untight-32-ext-highlighted_svg-tex.pdf}
    \\
    $\Theta_{32}$
    &
    $\Ext{\Theta_{32}}$
    \end{tabular}
\end{center}

\noindent The standard causaltope for Space 32 has dimension 33.
Below is a plot of the homogeneous linear system of causality and quasi-normalisation equations for the standard causaltope, put in reduced row echelon form:

\begin{center}
    \includegraphics[width=11cm]{svg-inkscape/space32-rref-eqs_svg-tex.pdf}
\end{center}

\noindent Rows correspond to the 30 independent linear equations.
Columns in the plot correspond to entries of empirical models, indexed as $i_A i_B i_C$ $o_A o_B o_C$.
Coefficients in the equations are color-coded as white=0, red=+1 and blue=-1.

Space 32 has closest refinements in equivalence classes 17, 21 and 22; 
it is the join of its (closest) refinements.
It has closest coarsenings in equivalence classes 48 and 52; 
it is the meet of its (closest) coarsenings.
It has 512 causal functions, 192 of which are not causal for any of its refinements.
It is not a tight space: for event \ev{B}, a causal function must yield identical output values on input histories \hist{A/0,B/1} and \hist{B/1,C/1}.

The standard causaltope for Space 32 has 2 more dimensions than those of its 3 subspaces in equivalence classes 17, 21 and 22.
The standard causaltope for Space 32 is the meet of the standard causaltopes for its closest coarsenings.
For completeness, below is a plot of the full homogeneous linear system of causality and quasi-normalisation equations for the standard causaltope:

\begin{center}
    \includegraphics[width=12cm]{svg-inkscape/space32-eqs_svg-tex.pdf}
\end{center}

\noindent Rows correspond to the 63 linear equations, of which 30 are independent.


\newpage
\subsection*{Space 33}

Space 33 is induced by the definite causal order $\total{\ev{A},\ev{B}}\vee\discrete{\ev{C}}$.
Its equivalence class under event-input permutation symmetry contains 6 spaces.

\noindent Below are the histories and extended histories for space 33: 
\begin{center}
    \begin{tabular}{cc}
    \includegraphics[height=3.5cm]{svg-inkscape/space-ABC-unique-tight-33-highlighted_svg-tex.pdf}
    &
    \includegraphics[height=3.5cm]{svg-inkscape/space-ABC-unique-tight-33-ext-highlighted_svg-tex.pdf}
    \\
    $\Theta_{33}$
    &
    $\Ext{\Theta_{33}}$
    \end{tabular}
\end{center}

\noindent The standard causaltope for Space 33 has dimension 32.
Below is a plot of the homogeneous linear system of causality and quasi-normalisation equations for the standard causaltope, put in reduced row echelon form:

\begin{center}
    \includegraphics[width=11cm]{svg-inkscape/space33-rref-eqs_svg-tex.pdf}
\end{center}

\noindent Rows correspond to the 31 independent linear equations.
Columns in the plot correspond to entries of empirical models, indexed as $i_A i_B i_C$ $o_A o_B o_C$.
Coefficients in the equations are color-coded as white=0, red=+1 and blue=-1.

Space 33 has closest refinements in equivalence class 19; 
it is the join of its (closest) refinements.
It has closest coarsenings in equivalence classes 46 and 58; 
it is the meet of its (closest) coarsenings.
It has 256 causal functions, 64 of which are not causal for any of its refinements.
It is a tight space.

The standard causaltope for Space 33 has 2 more dimensions than those of its 4 subspaces in equivalence class 19.
The standard causaltope for Space 33 is the meet of the standard causaltopes for its closest coarsenings.
For completeness, below is a plot of the full homogeneous linear system of causality and quasi-normalisation equations for the standard causaltope:

\begin{center}
    \includegraphics[width=12cm]{svg-inkscape/space33-eqs_svg-tex.pdf}
\end{center}

\noindent Rows correspond to the 63 linear equations, of which 31 are independent.


\newpage
\subsection*{Space 34}

Space 34 is not induced by a causal order, but it is a refinement of the space in equivalence class 92 induced by the definite causal order $\total{\ev{A},\ev{B}}\vee\total{\ev{C},\ev{B}}$ (note that the space induced by the order is not the same as space 92).
Its equivalence class under event-input permutation symmetry contains 48 spaces.
Space 34 differs as follows from the space induced by causal order $\total{\ev{A},\ev{B}}\vee\total{\ev{C},\ev{B}}$:
\begin{itemize}
  \item The outputs at events \evset{\ev{A}, \ev{B}} are independent of the input at event \ev{C} when the inputs at events \evset{A, B} are given by \hist{A/0,B/0}, \hist{A/0,B/1} and \hist{A/1,B/0}.
  \item The outputs at events \evset{\ev{B}, \ev{C}} are independent of the input at event \ev{A} when the inputs at events \evset{B, C} are given by \hist{B/0,C/1}.
\end{itemize}

\noindent Below are the histories and extended histories for space 34: 
\begin{center}
    \begin{tabular}{cc}
    \includegraphics[height=3.5cm]{svg-inkscape/space-ABC-unique-untight-34-highlighted_svg-tex.pdf}
    &
    \includegraphics[height=3.5cm]{svg-inkscape/space-ABC-unique-untight-34-ext-highlighted_svg-tex.pdf}
    \\
    $\Theta_{34}$
    &
    $\Ext{\Theta_{34}}$
    \end{tabular}
\end{center}

\noindent The standard causaltope for Space 34 has dimension 32.
Below is a plot of the homogeneous linear system of causality and quasi-normalisation equations for the standard causaltope, put in reduced row echelon form:

\begin{center}
    \includegraphics[width=11cm]{svg-inkscape/space34-rref-eqs_svg-tex.pdf}
\end{center}

\noindent Rows correspond to the 31 independent linear equations.
Columns in the plot correspond to entries of empirical models, indexed as $i_A i_B i_C$ $o_A o_B o_C$.
Coefficients in the equations are color-coded as white=0, red=+1 and blue=-1.

Space 34 has closest refinements in equivalence classes 19, 23, 24 and 26; 
it is the join of its (closest) refinements.
It has closest coarsenings in equivalence classes 46, 50, 53, 57 and 59; 
it is the meet of its (closest) coarsenings.
It has 256 causal functions, all of which are causal for at least one of its refinements.
It is not a tight space: for event \ev{B}, a causal function must yield identical output values on input histories \hist{A/0,B/0}, \hist{A/1,B/0} and \hist{B/0,C/1}.

The standard causaltope for Space 34 has 1 more dimension than that of its subspace in equivalence class 26.
The standard causaltope for Space 34 is the meet of the standard causaltopes for its closest coarsenings.
For completeness, below is a plot of the full homogeneous linear system of causality and quasi-normalisation equations for the standard causaltope:

\begin{center}
    \includegraphics[width=12cm]{svg-inkscape/space34-eqs_svg-tex.pdf}
\end{center}

\noindent Rows correspond to the 63 linear equations, of which 31 are independent.


\newpage
\subsection*{Space 35}

Space 35 is not induced by a causal order, but it is a refinement of the space in equivalence class 92 induced by the definite causal order $\total{\ev{A},\ev{B}}\vee\total{\ev{C},\ev{B}}$ (note that the space induced by the order is not the same as space 92).
Its equivalence class under event-input permutation symmetry contains 24 spaces.
Space 35 differs as follows from the space induced by causal order $\total{\ev{A},\ev{B}}\vee\total{\ev{C},\ev{B}}$:
\begin{itemize}
  \item The outputs at events \evset{\ev{A}, \ev{B}} are independent of the input at event \ev{C} when the inputs at events \evset{A, B} are given by \hist{A/0,B/0} and \hist{A/0,B/1}.
  \item The outputs at events \evset{\ev{B}, \ev{C}} are independent of the input at event \ev{A} when the inputs at events \evset{B, C} are given by \hist{B/1,C/0} and \hist{B/1,C/1}.
\end{itemize}

\noindent Below are the histories and extended histories for space 35: 
\begin{center}
    \begin{tabular}{cc}
    \includegraphics[height=3.5cm]{svg-inkscape/space-ABC-unique-untight-35-highlighted_svg-tex.pdf}
    &
    \includegraphics[height=3.5cm]{svg-inkscape/space-ABC-unique-untight-35-ext-highlighted_svg-tex.pdf}
    \\
    $\Theta_{35}$
    &
    $\Ext{\Theta_{35}}$
    \end{tabular}
\end{center}

\noindent The standard causaltope for Space 35 has dimension 32.
Below is a plot of the homogeneous linear system of causality and quasi-normalisation equations for the standard causaltope, put in reduced row echelon form:

\begin{center}
    \includegraphics[width=11cm]{svg-inkscape/space35-rref-eqs_svg-tex.pdf}
\end{center}

\noindent Rows correspond to the 31 independent linear equations.
Columns in the plot correspond to entries of empirical models, indexed as $i_A i_B i_C$ $o_A o_B o_C$.
Coefficients in the equations are color-coded as white=0, red=+1 and blue=-1.

Space 35 has closest refinements in equivalence classes 16, 23 and 26; 
it is the join of its (closest) refinements.
It has closest coarsenings in equivalence classes 49, 53, 54 and 59; 
it is the meet of its (closest) coarsenings.
It has 256 causal functions, all of which are causal for at least one of its refinements.
It is not a tight space: for event \ev{B}, a causal function must yield identical output values on input histories \hist{A/0,B/1}, \hist{B/1,C/0} and \hist{B/1,C/1}.

The standard causaltope for Space 35 has 1 more dimension than that of its subspace in equivalence class 26.
The standard causaltope for Space 35 is the meet of the standard causaltopes for its closest coarsenings.
For completeness, below is a plot of the full homogeneous linear system of causality and quasi-normalisation equations for the standard causaltope:

\begin{center}
    \includegraphics[width=12cm]{svg-inkscape/space35-eqs_svg-tex.pdf}
\end{center}

\noindent Rows correspond to the 63 linear equations, of which 31 are independent.


\newpage
\subsection*{Space 36}

Space 36 is not induced by a causal order, but it is a refinement of the space 92 induced by the definite causal order $\total{\ev{A},\ev{C}}\vee\total{\ev{B},\ev{C}}$.
Its equivalence class under event-input permutation symmetry contains 24 spaces.
Space 36 differs as follows from the space induced by causal order $\total{\ev{A},\ev{C}}\vee\total{\ev{B},\ev{C}}$:
\begin{itemize}
  \item The outputs at events \evset{\ev{B}, \ev{C}} are independent of the input at event \ev{A} when the inputs at events \evset{B, C} are given by \hist{B/1,C/0} and \hist{B/0,C/1}.
  \item The outputs at events \evset{\ev{A}, \ev{C}} are independent of the input at event \ev{B} when the inputs at events \evset{A, C} are given by \hist{A/1,C/0} and \hist{A/1,C/1}.
\end{itemize}

\noindent Below are the histories and extended histories for space 36: 
\begin{center}
    \begin{tabular}{cc}
    \includegraphics[height=3.5cm]{svg-inkscape/space-ABC-unique-untight-36-highlighted_svg-tex.pdf}
    &
    \includegraphics[height=3.5cm]{svg-inkscape/space-ABC-unique-untight-36-ext-highlighted_svg-tex.pdf}
    \\
    $\Theta_{36}$
    &
    $\Ext{\Theta_{36}}$
    \end{tabular}
\end{center}

\noindent The standard causaltope for Space 36 has dimension 32.
Below is a plot of the homogeneous linear system of causality and quasi-normalisation equations for the standard causaltope, put in reduced row echelon form:

\begin{center}
    \includegraphics[width=11cm]{svg-inkscape/space36-rref-eqs_svg-tex.pdf}
\end{center}

\noindent Rows correspond to the 31 independent linear equations.
Columns in the plot correspond to entries of empirical models, indexed as $i_A i_B i_C$ $o_A o_B o_C$.
Coefficients in the equations are color-coded as white=0, red=+1 and blue=-1.

Space 36 has closest refinements in equivalence classes 23 and 24; 
it is the join of its (closest) refinements.
It has closest coarsenings in equivalence classes 50, 53 and 55; 
it is the meet of its (closest) coarsenings.
It has 256 causal functions, 64 of which are not causal for any of its refinements.
It is not a tight space: for event \ev{C}, a causal function must yield identical output values on input histories \hist{A/1,C/0} and \hist{B/1,C/0}, and it must also yield identical output values on input histories \hist{A/1,C/1} and \hist{B/0,C/1}.

The standard causaltope for Space 36 has 2 more dimensions than those of its 4 subspaces in equivalence classes 23 and 24.
The standard causaltope for Space 36 is the meet of the standard causaltopes for its closest coarsenings.
For completeness, below is a plot of the full homogeneous linear system of causality and quasi-normalisation equations for the standard causaltope:

\begin{center}
    \includegraphics[width=12cm]{svg-inkscape/space36-eqs_svg-tex.pdf}
\end{center}

\noindent Rows correspond to the 63 linear equations, of which 31 are independent.


\newpage
\subsection*{Space 37}

Space 37 is not induced by a causal order, but it is a refinement of the space in equivalence class 92 induced by the definite causal order $\total{\ev{A},\ev{B}}\vee\total{\ev{C},\ev{B}}$ (note that the space induced by the order is not the same as space 92).
Its equivalence class under event-input permutation symmetry contains 12 spaces.
Space 37 differs as follows from the space induced by causal order $\total{\ev{A},\ev{B}}\vee\total{\ev{C},\ev{B}}$:
\begin{itemize}
  \item The outputs at events \evset{\ev{A}, \ev{B}} are independent of the input at event \ev{C} when the inputs at events \evset{A, B} are given by \hist{A/0,B/0} and \hist{A/0,B/1}.
  \item The outputs at events \evset{\ev{B}, \ev{C}} are independent of the input at event \ev{A} when the inputs at events \evset{B, C} are given by \hist{B/1,C/1} and \hist{B/0,C/1}.
\end{itemize}

\noindent Below are the histories and extended histories for space 37: 
\begin{center}
    \begin{tabular}{cc}
    \includegraphics[height=3.5cm]{svg-inkscape/space-ABC-unique-untight-37-highlighted_svg-tex.pdf}
    &
    \includegraphics[height=3.5cm]{svg-inkscape/space-ABC-unique-untight-37-ext-highlighted_svg-tex.pdf}
    \\
    $\Theta_{37}$
    &
    $\Ext{\Theta_{37}}$
    \end{tabular}
\end{center}

\noindent The standard causaltope for Space 37 has dimension 32.
Below is a plot of the homogeneous linear system of causality and quasi-normalisation equations for the standard causaltope, put in reduced row echelon form:

\begin{center}
    \includegraphics[width=11cm]{svg-inkscape/space37-rref-eqs_svg-tex.pdf}
\end{center}

\noindent Rows correspond to the 31 independent linear equations.
Columns in the plot correspond to entries of empirical models, indexed as $i_A i_B i_C$ $o_A o_B o_C$.
Coefficients in the equations are color-coded as white=0, red=+1 and blue=-1.

Space 37 has closest refinements in equivalence class 23; 
it is the join of its (closest) refinements.
It has closest coarsenings in equivalence classes 53 and 60; 
it is the meet of its (closest) coarsenings.
It has 256 causal functions, 64 of which are not causal for any of its refinements.
It is not a tight space: for event \ev{B}, a causal function must yield identical output values on input histories \hist{A/0,B/0} and \hist{B/0,C/1}, and it must also yield identical output values on input histories \hist{A/0,B/1} and \hist{B/1,C/1}.

The standard causaltope for Space 37 has 2 more dimensions than those of its 4 subspaces in equivalence class 23.
The standard causaltope for Space 37 is the meet of the standard causaltopes for its closest coarsenings.
For completeness, below is a plot of the full homogeneous linear system of causality and quasi-normalisation equations for the standard causaltope:

\begin{center}
    \includegraphics[width=12cm]{svg-inkscape/space37-eqs_svg-tex.pdf}
\end{center}

\noindent Rows correspond to the 63 linear equations, of which 31 are independent.


\newpage
\subsection*{Space 38}

Space 38 is not induced by a causal order, but it is a refinement of the space 77 induced by the definite causal order $\total{\ev{A},\ev{B}}\vee\total{\ev{A},\ev{C}}$.
Its equivalence class under event-input permutation symmetry contains 24 spaces.
Space 38 differs as follows from the space induced by causal order $\total{\ev{A},\ev{B}}\vee\total{\ev{A},\ev{C}}$:
\begin{itemize}
  \item The outputs at events \evset{\ev{B}, \ev{C}} are independent of the input at event \ev{A} when the inputs at events \evset{B, C} are given by \hist{B/1,C/0}.
  \item The output at event \ev{C} is independent of the input at event \ev{A} when the input at event C is given by \hist{C/0}.
\end{itemize}

\noindent Below are the histories and extended histories for space 38: 
\begin{center}
    \begin{tabular}{cc}
    \includegraphics[height=3.5cm]{svg-inkscape/space-ABC-unique-untight-38-highlighted_svg-tex.pdf}
    &
    \includegraphics[height=3.5cm]{svg-inkscape/space-ABC-unique-untight-38-ext-highlighted_svg-tex.pdf}
    \\
    $\Theta_{38}$
    &
    $\Ext{\Theta_{38}}$
    \end{tabular}
\end{center}

\noindent The standard causaltope for Space 38 has dimension 31.
Below is a plot of the homogeneous linear system of causality and quasi-normalisation equations for the standard causaltope, put in reduced row echelon form:

\begin{center}
    \includegraphics[width=11cm]{svg-inkscape/space38-rref-eqs_svg-tex.pdf}
\end{center}

\noindent Rows correspond to the 32 independent linear equations.
Columns in the plot correspond to entries of empirical models, indexed as $i_A i_B i_C$ $o_A o_B o_C$.
Coefficients in the equations are color-coded as white=0, red=+1 and blue=-1.

Space 38 has closest refinements in equivalence classes 19, 25 and 27; 
it is the join of its (closest) refinements.
It has closest coarsenings in equivalence classes 47, 56, 57 and 58; 
it is the meet of its (closest) coarsenings.
It has 256 causal functions, all of which are causal for at least one of its refinements.
It is not a tight space: for event \ev{B}, a causal function must yield identical output values on input histories \hist{A/0,B/1}, \hist{A/1,B/1} and \hist{B/1,C/0}.

The standard causaltope for Space 38 coincides with that of its subspace in equivalence class 25.
The standard causaltope for Space 38 is the meet of the standard causaltopes for its closest coarsenings.
For completeness, below is a plot of the full homogeneous linear system of causality and quasi-normalisation equations for the standard causaltope:

\begin{center}
    \includegraphics[width=12cm]{svg-inkscape/space38-eqs_svg-tex.pdf}
\end{center}

\noindent Rows correspond to the 61 linear equations, of which 32 are independent.


\newpage
\subsection*{Space 39}

Space 39 is not induced by a causal order, but it is a refinement of the space 92 induced by the definite causal order $\total{\ev{A},\ev{C}}\vee\total{\ev{B},\ev{C}}$.
Its equivalence class under event-input permutation symmetry contains 24 spaces.
Space 39 differs as follows from the space induced by causal order $\total{\ev{A},\ev{C}}\vee\total{\ev{B},\ev{C}}$:
\begin{itemize}
  \item The outputs at events \evset{\ev{A}, \ev{C}} are independent of the input at event \ev{B} when the inputs at events \evset{A, C} are given by \hist{A/0,C/1} and \hist{A/1,C/0}.
  \item The outputs at events \evset{\ev{B}, \ev{C}} are independent of the input at event \ev{A} when the inputs at events \evset{B, C} are given by \hist{B/1,C/1} and \hist{B/0,C/1}.
\end{itemize}

\noindent Below are the histories and extended histories for space 39: 
\begin{center}
    \begin{tabular}{cc}
    \includegraphics[height=3.5cm]{svg-inkscape/space-ABC-unique-untight-39-highlighted_svg-tex.pdf}
    &
    \includegraphics[height=3.5cm]{svg-inkscape/space-ABC-unique-untight-39-ext-highlighted_svg-tex.pdf}
    \\
    $\Theta_{39}$
    &
    $\Ext{\Theta_{39}}$
    \end{tabular}
\end{center}

\noindent The standard causaltope for Space 39 has dimension 32.
Below is a plot of the homogeneous linear system of causality and quasi-normalisation equations for the standard causaltope, put in reduced row echelon form:

\begin{center}
    \includegraphics[width=11cm]{svg-inkscape/space39-rref-eqs_svg-tex.pdf}
\end{center}

\noindent Rows correspond to the 31 independent linear equations.
Columns in the plot correspond to entries of empirical models, indexed as $i_A i_B i_C$ $o_A o_B o_C$.
Coefficients in the equations are color-coded as white=0, red=+1 and blue=-1.

Space 39 has closest refinements in equivalence classes 16, 24 and 26; 
it is the join of its (closest) refinements.
It has closest coarsenings in equivalence classes 49, 50 and 59; 
it is the meet of its (closest) coarsenings.
It has 256 causal functions, all of which are causal for at least one of its refinements.
It is not a tight space: for event \ev{C}, a causal function must yield identical output values on input histories \hist{A/0,C/1}, \hist{B/0,C/1} and \hist{B/1,C/1}.

The standard causaltope for Space 39 has 1 more dimension than that of its subspace in equivalence class 26.
The standard causaltope for Space 39 is the meet of the standard causaltopes for its closest coarsenings.
For completeness, below is a plot of the full homogeneous linear system of causality and quasi-normalisation equations for the standard causaltope:

\begin{center}
    \includegraphics[width=12cm]{svg-inkscape/space39-eqs_svg-tex.pdf}
\end{center}

\noindent Rows correspond to the 63 linear equations, of which 31 are independent.


\newpage
\subsection*{Space 40}

Space 40 is not induced by a causal order, but it is a refinement of the space 92 induced by the definite causal order $\total{\ev{A},\ev{C}}\vee\total{\ev{B},\ev{C}}$.
Its equivalence class under event-input permutation symmetry contains 12 spaces.
Space 40 differs as follows from the space induced by causal order $\total{\ev{A},\ev{C}}\vee\total{\ev{B},\ev{C}}$:
\begin{itemize}
  \item The outputs at events \evset{\ev{A}, \ev{C}} are independent of the input at event \ev{B} when the inputs at events \evset{A, C} are given by \hist{A/0,C/1} and \hist{A/1,C/0}.
  \item The outputs at events \evset{\ev{B}, \ev{C}} are independent of the input at event \ev{A} when the inputs at events \evset{B, C} are given by \hist{B/1,C/0} and \hist{B/0,C/1}.
\end{itemize}

\noindent Below are the histories and extended histories for space 40: 
\begin{center}
    \begin{tabular}{cc}
    \includegraphics[height=3.5cm]{svg-inkscape/space-ABC-unique-untight-40-highlighted_svg-tex.pdf}
    &
    \includegraphics[height=3.5cm]{svg-inkscape/space-ABC-unique-untight-40-ext-highlighted_svg-tex.pdf}
    \\
    $\Theta_{40}$
    &
    $\Ext{\Theta_{40}}$
    \end{tabular}
\end{center}

\noindent The standard causaltope for Space 40 has dimension 32.
Below is a plot of the homogeneous linear system of causality and quasi-normalisation equations for the standard causaltope, put in reduced row echelon form:

\begin{center}
    \includegraphics[width=11cm]{svg-inkscape/space40-rref-eqs_svg-tex.pdf}
\end{center}

\noindent Rows correspond to the 31 independent linear equations.
Columns in the plot correspond to entries of empirical models, indexed as $i_A i_B i_C$ $o_A o_B o_C$.
Coefficients in the equations are color-coded as white=0, red=+1 and blue=-1.

Space 40 has closest refinements in equivalence class 24; 
it is the join of its (closest) refinements.
It has closest coarsenings in equivalence class 50; 
it is the meet of its (closest) coarsenings.
It has 256 causal functions, 64 of which are not causal for any of its refinements.
It is not a tight space: for event \ev{C}, a causal function must yield identical output values on input histories \hist{A/0,C/1} and \hist{B/0,C/1}, and it must also yield identical output values on input histories \hist{A/1,C/0} and \hist{B/1,C/0}.

The standard causaltope for Space 40 has 2 more dimensions than those of its 4 subspaces in equivalence class 24.
The standard causaltope for Space 40 is the meet of the standard causaltopes for its closest coarsenings.
For completeness, below is a plot of the full homogeneous linear system of causality and quasi-normalisation equations for the standard causaltope:

\begin{center}
    \includegraphics[width=12cm]{svg-inkscape/space40-eqs_svg-tex.pdf}
\end{center}

\noindent Rows correspond to the 63 linear equations, of which 31 are independent.


\newpage
\subsection*{Space 41}

Space 41 is not induced by a causal order, but it is a refinement of the space 92 induced by the definite causal order $\total{\ev{A},\ev{C}}\vee\total{\ev{B},\ev{C}}$.
Its equivalence class under event-input permutation symmetry contains 6 spaces.
Space 41 differs as follows from the space induced by causal order $\total{\ev{A},\ev{C}}\vee\total{\ev{B},\ev{C}}$:
\begin{itemize}
  \item The outputs at events \evset{\ev{A}, \ev{C}} are independent of the input at event \ev{B} when the inputs at events \evset{A, C} are given by \hist{A/0,C/1} and \hist{A/1,C/1}.
  \item The outputs at events \evset{\ev{B}, \ev{C}} are independent of the input at event \ev{A} when the inputs at events \evset{B, C} are given by \hist{B/1,C/0} and \hist{B/0,C/0}.
\end{itemize}

\noindent Below are the histories and extended histories for space 41: 
\begin{center}
    \begin{tabular}{cc}
    \includegraphics[height=3.5cm]{svg-inkscape/space-ABC-unique-tight-41-highlighted_svg-tex.pdf}
    &
    \includegraphics[height=3.5cm]{svg-inkscape/space-ABC-unique-tight-41-ext-highlighted_svg-tex.pdf}
    \\
    $\Theta_{41}$
    &
    $\Ext{\Theta_{41}}$
    \end{tabular}
\end{center}

\noindent The standard causaltope for Space 41 has dimension 32.
Below is a plot of the homogeneous linear system of causality and quasi-normalisation equations for the standard causaltope, put in reduced row echelon form:

\begin{center}
    \includegraphics[width=11cm]{svg-inkscape/space41-rref-eqs_svg-tex.pdf}
\end{center}

\noindent Rows correspond to the 31 independent linear equations.
Columns in the plot correspond to entries of empirical models, indexed as $i_A i_B i_C$ $o_A o_B o_C$.
Coefficients in the equations are color-coded as white=0, red=+1 and blue=-1.

Space 41 has closest refinements in equivalence class 16; 
it is the join of its (closest) refinements.
It has closest coarsenings in equivalence class 49; 
it is the meet of its (closest) coarsenings.
It has 256 causal functions, 64 of which are not causal for any of its refinements.
It is a tight space.

The standard causaltope for Space 41 has 2 more dimensions than those of its 4 subspaces in equivalence class 16.
The standard causaltope for Space 41 is the meet of the standard causaltopes for its closest coarsenings.
For completeness, below is a plot of the full homogeneous linear system of causality and quasi-normalisation equations for the standard causaltope:

\begin{center}
    \includegraphics[width=12cm]{svg-inkscape/space41-eqs_svg-tex.pdf}
\end{center}

\noindent Rows correspond to the 63 linear equations, of which 31 are independent.


\newpage
\subsection*{Space 42}

Space 42 is not induced by a causal order, but it is a refinement of the space 100 induced by the definite causal order $\total{\ev{A},\ev{B},\ev{C}}$.
Its equivalence class under event-input permutation symmetry contains 24 spaces.
Space 42 differs as follows from the space induced by causal order $\total{\ev{A},\ev{B},\ev{C}}$:
\begin{itemize}
  \item The outputs at events \evset{\ev{A}, \ev{C}} are independent of the input at event \ev{B} when the inputs at events \evset{A, C} are given by \hist{A/0,C/1}, \hist{A/0,C/0} and \hist{A/1,C/0}.
  \item The outputs at events \evset{\ev{B}, \ev{C}} are independent of the input at event \ev{A} when the inputs at events \evset{B, C} are given by \hist{B/1,C/0} and \hist{B/0,C/0}.
  \item The output at event \ev{C} is independent of the inputs at events \evset{\ev{A}, \ev{B}} when the input at event C is given by \hist{C/0}.
\end{itemize}

\noindent Below are the histories and extended histories for space 42: 
\begin{center}
    \begin{tabular}{cc}
    \includegraphics[height=3.5cm]{svg-inkscape/space-ABC-unique-untight-42-highlighted_svg-tex.pdf}
    &
    \includegraphics[height=3.5cm]{svg-inkscape/space-ABC-unique-untight-42-ext-highlighted_svg-tex.pdf}
    \\
    $\Theta_{42}$
    &
    $\Ext{\Theta_{42}}$
    \end{tabular}
\end{center}

\noindent The standard causaltope for Space 42 has dimension 31.
Below is a plot of the homogeneous linear system of causality and quasi-normalisation equations for the standard causaltope, put in reduced row echelon form:

\begin{center}
    \includegraphics[width=11cm]{svg-inkscape/space42-rref-eqs_svg-tex.pdf}
\end{center}

\noindent Rows correspond to the 32 independent linear equations.
Columns in the plot correspond to entries of empirical models, indexed as $i_A i_B i_C$ $o_A o_B o_C$.
Coefficients in the equations are color-coded as white=0, red=+1 and blue=-1.

Space 42 has closest refinements in equivalence classes 22 and 27; 
it is the join of its (closest) refinements.
It has closest coarsenings in equivalence classes 47, 51 and 52; 
it is the meet of its (closest) coarsenings.
It has 256 causal functions, all of which are causal for at least one of its refinements.
It is not a tight space: for event \ev{B}, a causal function must yield identical output values on input histories \hist{A/0,B/0}, \hist{A/1,B/0} and \hist{B/0,C/0}, and it must also yield identical output values on input histories \hist{A/0,B/1}, \hist{A/1,B/1} and \hist{B/1,C/0}.

The standard causaltope for Space 42 coincides with that of its 2 subspaces in equivalence class 22.
The standard causaltope for Space 42 is the meet of the standard causaltopes for its closest coarsenings.
For completeness, below is a plot of the full homogeneous linear system of causality and quasi-normalisation equations for the standard causaltope:

\begin{center}
    \includegraphics[width=12cm]{svg-inkscape/space42-eqs_svg-tex.pdf}
\end{center}

\noindent Rows correspond to the 61 linear equations, of which 32 are independent.


\newpage
\subsection*{Space 43}

Space 43 is not induced by a causal order, but it is a refinement of the space in equivalence class 100 induced by the definite causal order $\total{\ev{A},\ev{C},\ev{B}}$ (note that the space induced by the order is not the same as space 100).
Its equivalence class under event-input permutation symmetry contains 48 spaces.
Space 43 differs as follows from the space induced by causal order $\total{\ev{A},\ev{C},\ev{B}}$:
\begin{itemize}
  \item The outputs at events \evset{\ev{A}, \ev{B}} are independent of the input at event \ev{C} when the inputs at events \evset{A, B} are given by \hist{A/0,B/0}, \hist{A/0,B/1} and \hist{A/1,B/0}.
  \item The outputs at events \evset{\ev{B}, \ev{C}} are independent of the input at event \ev{A} when the inputs at events \evset{B, C} are given by \hist{B/1,C/0} and \hist{B/0,C/0}.
  \item The output at event \ev{C} is independent of the input at event \ev{A} when the input at event C is given by \hist{C/0}.
\end{itemize}

\noindent Below are the histories and extended histories for space 43: 
\begin{center}
    \begin{tabular}{cc}
    \includegraphics[height=3.5cm]{svg-inkscape/space-ABC-unique-untight-43-highlighted_svg-tex.pdf}
    &
    \includegraphics[height=3.5cm]{svg-inkscape/space-ABC-unique-untight-43-ext-highlighted_svg-tex.pdf}
    \\
    $\Theta_{43}$
    &
    $\Ext{\Theta_{43}}$
    \end{tabular}
\end{center}

\noindent The standard causaltope for Space 43 has dimension 31.
Below is a plot of the homogeneous linear system of causality and quasi-normalisation equations for the standard causaltope, put in reduced row echelon form:

\begin{center}
    \includegraphics[width=11cm]{svg-inkscape/space43-rref-eqs_svg-tex.pdf}
\end{center}

\noindent Rows correspond to the 32 independent linear equations.
Columns in the plot correspond to entries of empirical models, indexed as $i_A i_B i_C$ $o_A o_B o_C$.
Coefficients in the equations are color-coded as white=0, red=+1 and blue=-1.

Space 43 has closest refinements in equivalence classes 17, 23 and 27; 
it is the join of its (closest) refinements.
It has closest coarsenings in equivalence classes 52, 54, 55, 56, 57 and 60; 
it is the meet of its (closest) coarsenings.
It has 256 causal functions, 64 of which are not causal for any of its refinements.
It is not a tight space: for event \ev{B}, a causal function must yield identical output values on input histories \hist{A/0,B/1} and \hist{B/1,C/0}, and it must also yield identical output values on input histories \hist{A/0,B/0}, \hist{A/1,B/0} and \hist{B/0,C/0}.

The standard causaltope for Space 43 coincides with that of its subspace in equivalence class 17.
The standard causaltope for Space 43 is the meet of the standard causaltopes for its closest coarsenings.
For completeness, below is a plot of the full homogeneous linear system of causality and quasi-normalisation equations for the standard causaltope:

\begin{center}
    \includegraphics[width=12cm]{svg-inkscape/space43-eqs_svg-tex.pdf}
\end{center}

\noindent Rows correspond to the 61 linear equations, of which 32 are independent.


\newpage
\subsection*{Space 44}

Space 44 is not induced by a causal order, but it is a refinement of the space 92 induced by the definite causal order $\total{\ev{A},\ev{C}}\vee\total{\ev{B},\ev{C}}$.
Its equivalence class under event-input permutation symmetry contains 6 spaces.
Space 44 differs as follows from the space induced by causal order $\total{\ev{A},\ev{C}}\vee\total{\ev{B},\ev{C}}$:
\begin{itemize}
  \item The outputs at events \evset{\ev{A}, \ev{C}} are independent of the input at event \ev{B} when the inputs at events \evset{A, C} are given by \hist{A/0,C/1} and \hist{A/1,C/1}.
  \item The outputs at events \evset{\ev{B}, \ev{C}} are independent of the input at event \ev{A} when the inputs at events \evset{B, C} are given by \hist{B/1,C/1} and \hist{B/0,C/1}.
\end{itemize}

\noindent Below are the histories and extended histories for space 44: 
\begin{center}
    \begin{tabular}{cc}
    \includegraphics[height=3.5cm]{svg-inkscape/space-ABC-unique-untight-44-highlighted_svg-tex.pdf}
    &
    \includegraphics[height=3.5cm]{svg-inkscape/space-ABC-unique-untight-44-ext-highlighted_svg-tex.pdf}
    \\
    $\Theta_{44}$
    &
    $\Ext{\Theta_{44}}$
    \end{tabular}
\end{center}

\noindent The standard causaltope for Space 44 has dimension 33.
Below is a plot of the homogeneous linear system of causality and quasi-normalisation equations for the standard causaltope, put in reduced row echelon form:

\begin{center}
    \includegraphics[width=11cm]{svg-inkscape/space44-rref-eqs_svg-tex.pdf}
\end{center}

\noindent Rows correspond to the 30 independent linear equations.
Columns in the plot correspond to entries of empirical models, indexed as $i_A i_B i_C$ $o_A o_B o_C$.
Coefficients in the equations are color-coded as white=0, red=+1 and blue=-1.

Space 44 has closest refinements in equivalence classes 18 and 26; 
it is the join of its (closest) refinements.
It has closest coarsenings in equivalence class 59; 
it is the meet of its (closest) coarsenings.
It has 512 causal functions, all of which are causal for at least one of its refinements.
It is not a tight space: for event \ev{C}, a causal function must yield identical output values on input histories \hist{A/0,C/1}, \hist{A/1,C/1}, \hist{B/0,C/1} and \hist{B/1,C/1}.

The standard causaltope for Space 44 coincides with that of its subspace in equivalence class 18.
The standard causaltope for Space 44 is the meet of the standard causaltopes for its closest coarsenings.
For completeness, below is a plot of the full homogeneous linear system of causality and quasi-normalisation equations for the standard causaltope:

\begin{center}
    \includegraphics[width=12cm]{svg-inkscape/space44-eqs_svg-tex.pdf}
\end{center}

\noindent Rows correspond to the 63 linear equations, of which 30 are independent.


\newpage
\subsection*{Space 45}

Space 45 is not induced by a causal order, but it is a refinement of the space 100 induced by the definite causal order $\total{\ev{A},\ev{B},\ev{C}}$.
Its equivalence class under event-input permutation symmetry contains 24 spaces.
Space 45 differs as follows from the space induced by causal order $\total{\ev{A},\ev{B},\ev{C}}$:
\begin{itemize}
  \item The outputs at events \evset{\ev{A}, \ev{C}} are independent of the input at event \ev{B} when the inputs at events \evset{A, C} are given by \hist{A/0,C/1} and \hist{A/1,C/1}.
  \item The outputs at events \evset{\ev{B}, \ev{C}} are independent of the input at event \ev{A} when the inputs at events \evset{B, C} are given by \hist{B/1,C/1}.
  \item The output at event \ev{C} is independent of the inputs at events \evset{\ev{A}, \ev{B}} when the input at event C is given by \hist{C/1}.
  \item The output at event \ev{B} is independent of the input at event \ev{A} when the input at event B is given by \hist{B/1}.
\end{itemize}

\noindent Below are the histories and extended histories for space 45: 
\begin{center}
    \begin{tabular}{cc}
    \includegraphics[height=3.5cm]{svg-inkscape/space-ABC-unique-tight-45-highlighted_svg-tex.pdf}
    &
    \includegraphics[height=3.5cm]{svg-inkscape/space-ABC-unique-tight-45-ext-highlighted_svg-tex.pdf}
    \\
    $\Theta_{45}$
    &
    $\Ext{\Theta_{45}}$
    \end{tabular}
\end{center}

\noindent The standard causaltope for Space 45 has dimension 35.
Below is a plot of the homogeneous linear system of causality and quasi-normalisation equations for the standard causaltope, put in reduced row echelon form:

\begin{center}
    \includegraphics[width=11cm]{svg-inkscape/space45-rref-eqs_svg-tex.pdf}
\end{center}

\noindent Rows correspond to the 28 independent linear equations.
Columns in the plot correspond to entries of empirical models, indexed as $i_A i_B i_C$ $o_A o_B o_C$.
Coefficients in the equations are color-coded as white=0, red=+1 and blue=-1.

Space 45 has closest refinements in equivalence classes 28, 30 and 31; 
it is the join of its (closest) refinements.
It has closest coarsenings in equivalence classes 61 and 76; 
it is the meet of its (closest) coarsenings.
It has 1024 causal functions, 448 of which are not causal for any of its refinements.
It is a tight space.

The standard causaltope for Space 45 has 2 more dimensions than those of its 4 subspaces in equivalence classes 28, 30 and 31.
The standard causaltope for Space 45 is the meet of the standard causaltopes for its closest coarsenings.
For completeness, below is a plot of the full homogeneous linear system of causality and quasi-normalisation equations for the standard causaltope:

\begin{center}
    \includegraphics[width=12cm]{svg-inkscape/space45-eqs_svg-tex.pdf}
\end{center}

\noindent Rows correspond to the 59 linear equations, of which 28 are independent.


\newpage
\subsection*{Space 46}

Space 46 is not induced by a causal order, but it is a refinement of the space in equivalence class 92 induced by the definite causal order $\total{\ev{A},\ev{B}}\vee\total{\ev{C},\ev{B}}$ (note that the space induced by the order is not the same as space 92).
Its equivalence class under event-input permutation symmetry contains 24 spaces.
Space 46 differs as follows from the space induced by causal order $\total{\ev{A},\ev{B}}\vee\total{\ev{C},\ev{B}}$:
\begin{itemize}
  \item The outputs at events \evset{\ev{A}, \ev{B}} are independent of the input at event \ev{C} when the inputs at events \evset{A, B} are given by \hist{A/0,B/0}, \hist{A/0,B/1} and \hist{A/1,B/0}.
\end{itemize}

\noindent Below are the histories and extended histories for space 46: 
\begin{center}
    \begin{tabular}{cc}
    \includegraphics[height=3.5cm]{svg-inkscape/space-ABC-unique-tight-46-highlighted_svg-tex.pdf}
    &
    \includegraphics[height=3.5cm]{svg-inkscape/space-ABC-unique-tight-46-ext-highlighted_svg-tex.pdf}
    \\
    $\Theta_{46}$
    &
    $\Ext{\Theta_{46}}$
    \end{tabular}
\end{center}

\noindent The standard causaltope for Space 46 has dimension 34.
Below is a plot of the homogeneous linear system of causality and quasi-normalisation equations for the standard causaltope, put in reduced row echelon form:

\begin{center}
    \includegraphics[width=11cm]{svg-inkscape/space46-rref-eqs_svg-tex.pdf}
\end{center}

\noindent Rows correspond to the 29 independent linear equations.
Columns in the plot correspond to entries of empirical models, indexed as $i_A i_B i_C$ $o_A o_B o_C$.
Coefficients in the equations are color-coded as white=0, red=+1 and blue=-1.

Space 46 has closest refinements in equivalence classes 29, 33 and 34; 
it is the join of its (closest) refinements.
It has closest coarsenings in equivalence classes 67, 68, 70 and 74; 
it is the meet of its (closest) coarsenings.
It has 512 causal functions, 64 of which are not causal for any of its refinements.
It is a tight space.

The standard causaltope for Space 46 has 2 more dimensions than those of its 5 subspaces in equivalence classes 29, 33 and 34.
The standard causaltope for Space 46 is the meet of the standard causaltopes for its closest coarsenings.
For completeness, below is a plot of the full homogeneous linear system of causality and quasi-normalisation equations for the standard causaltope:

\begin{center}
    \includegraphics[width=12cm]{svg-inkscape/space46-eqs_svg-tex.pdf}
\end{center}

\noindent Rows correspond to the 59 linear equations, of which 29 are independent.


\newpage
\subsection*{Space 47}

Space 47 is not induced by a causal order, but it is a refinement of the space 100 induced by the definite causal order $\total{\ev{A},\ev{B},\ev{C}}$.
Its equivalence class under event-input permutation symmetry contains 48 spaces.
Space 47 differs as follows from the space induced by causal order $\total{\ev{A},\ev{B},\ev{C}}$:
\begin{itemize}
  \item The outputs at events \evset{\ev{B}, \ev{C}} are independent of the input at event \ev{A} when the inputs at events \evset{B, C} are given by \hist{B/1,C/0}.
  \item The outputs at events \evset{\ev{A}, \ev{C}} are independent of the input at event \ev{B} when the inputs at events \evset{A, C} are given by \hist{A/0,C/0}, \hist{A/1,C/0} and \hist{A/1,C/1}.
  \item The output at event \ev{C} is independent of the inputs at events \evset{\ev{A}, \ev{B}} when the input at event C is given by \hist{C/0}.
\end{itemize}

\noindent Below are the histories and extended histories for space 47: 
\begin{center}
    \begin{tabular}{cc}
    \includegraphics[height=3.5cm]{svg-inkscape/space-ABC-unique-untight-47-highlighted_svg-tex.pdf}
    &
    \includegraphics[height=3.5cm]{svg-inkscape/space-ABC-unique-untight-47-ext-highlighted_svg-tex.pdf}
    \\
    $\Theta_{47}$
    &
    $\Ext{\Theta_{47}}$
    \end{tabular}
\end{center}

\noindent The standard causaltope for Space 47 has dimension 33.
Below is a plot of the homogeneous linear system of causality and quasi-normalisation equations for the standard causaltope, put in reduced row echelon form:

\begin{center}
    \includegraphics[width=11cm]{svg-inkscape/space47-rref-eqs_svg-tex.pdf}
\end{center}

\noindent Rows correspond to the 30 independent linear equations.
Columns in the plot correspond to entries of empirical models, indexed as $i_A i_B i_C$ $o_A o_B o_C$.
Coefficients in the equations are color-coded as white=0, red=+1 and blue=-1.

Space 47 has closest refinements in equivalence classes 30, 38 and 42; 
it is the join of its (closest) refinements.
It has closest coarsenings in equivalence classes 61, 62, 64 and 71; 
it is the meet of its (closest) coarsenings.
It has 512 causal functions, all of which are causal for at least one of its refinements.
It is not a tight space: for event \ev{B}, a causal function must yield identical output values on input histories \hist{A/0,B/1}, \hist{A/1,B/1} and \hist{B/1,C/0}.

The standard causaltope for Space 47 coincides with that of its subspace in equivalence class 30.
The standard causaltope for Space 47 is the meet of the standard causaltopes for its closest coarsenings.
For completeness, below is a plot of the full homogeneous linear system of causality and quasi-normalisation equations for the standard causaltope:

\begin{center}
    \includegraphics[width=12cm]{svg-inkscape/space47-eqs_svg-tex.pdf}
\end{center}

\noindent Rows correspond to the 57 linear equations, of which 30 are independent.


\newpage
\subsection*{Space 48}

Space 48 is not induced by a causal order, but it is a refinement of the space induced by the indefinite causal order $\total{\ev{A},\{\ev{B},\ev{C}\}}$.
Its equivalence class under event-input permutation symmetry contains 24 spaces.
Space 48 differs as follows from the space induced by causal order $\total{\ev{A},\{\ev{B},\ev{C}\}}$:
\begin{itemize}
  \item The outputs at events \evset{\ev{A}, \ev{B}} are independent of the input at event \ev{C} when the inputs at events \evset{A, B} are given by \hist{A/0,B/0}, \hist{A/0,B/1} and \hist{A/1,B/0}.
  \item The outputs at events \evset{\ev{A}, \ev{C}} are independent of the input at event \ev{B} when the inputs at events \evset{A, C} are given by \hist{A/0,C/1}, \hist{A/1,C/0} and \hist{A/1,C/1}.
  \item The output at event \ev{B} is independent of the inputs at events \evset{\ev{A}, \ev{C}} when the input at event B is given by \hist{B/0}.
  \item The output at event \ev{C} is independent of the inputs at events \evset{\ev{A}, \ev{B}} when the input at event C is given by \hist{C/1}.
  \item The outputs at events \evset{\ev{B}, \ev{C}} are independent of the input at event \ev{A} when the inputs at events \evset{B, C} are given by \hist{B/0,C/1}.
\end{itemize}

\noindent Below are the histories and extended histories for space 48: 
\begin{center}
    \begin{tabular}{cc}
    \includegraphics[height=3.5cm]{svg-inkscape/space-ABC-unique-tight-48-highlighted_svg-tex.pdf}
    &
    \includegraphics[height=3.5cm]{svg-inkscape/space-ABC-unique-tight-48-ext-highlighted_svg-tex.pdf}
    \\
    $\Theta_{48}$
    &
    $\Ext{\Theta_{48}}$
    \end{tabular}
\end{center}

\noindent The standard causaltope for Space 48 has dimension 35.
Below is a plot of the homogeneous linear system of causality and quasi-normalisation equations for the standard causaltope, put in reduced row echelon form:

\begin{center}
    \includegraphics[width=11cm]{svg-inkscape/space48-rref-eqs_svg-tex.pdf}
\end{center}

\noindent Rows correspond to the 28 independent linear equations.
Columns in the plot correspond to entries of empirical models, indexed as $i_A i_B i_C$ $o_A o_B o_C$.
Coefficients in the equations are color-coded as white=0, red=+1 and blue=-1.

Space 48 has closest refinements in equivalence classes 30 and 32; 
it is the join of its (closest) refinements.
It has closest coarsenings in equivalence class 71; 
it is the meet of its (closest) coarsenings.
It has 1024 causal functions, 384 of which are not causal for any of its refinements.
It is a tight space.

The standard causaltope for Space 48 has 2 more dimensions than those of its 4 subspaces in equivalence classes 30 and 32.
The standard causaltope for Space 48 is the meet of the standard causaltopes for its closest coarsenings.
For completeness, below is a plot of the full homogeneous linear system of causality and quasi-normalisation equations for the standard causaltope:

\begin{center}
    \includegraphics[width=12cm]{svg-inkscape/space48-eqs_svg-tex.pdf}
\end{center}

\noindent Rows correspond to the 59 linear equations, of which 28 are independent.


\newpage
\subsection*{Space 49}

Space 49 is not induced by a causal order, but it is a refinement of the space 92 induced by the definite causal order $\total{\ev{A},\ev{C}}\vee\total{\ev{B},\ev{C}}$.
Its equivalence class under event-input permutation symmetry contains 24 spaces.
Space 49 differs as follows from the space induced by causal order $\total{\ev{A},\ev{C}}\vee\total{\ev{B},\ev{C}}$:
\begin{itemize}
  \item The outputs at events \evset{\ev{B}, \ev{C}} are independent of the input at event \ev{A} when the inputs at events \evset{B, C} are given by \hist{B/1,C/1} and \hist{B/0,C/1}.
  \item The outputs at events \evset{\ev{A}, \ev{C}} are independent of the input at event \ev{B} when the inputs at events \evset{A, C} are given by \hist{A/1,C/0}.
\end{itemize}

\noindent Below are the histories and extended histories for space 49: 
\begin{center}
    \begin{tabular}{cc}
    \includegraphics[height=3.5cm]{svg-inkscape/space-ABC-unique-tight-49-highlighted_svg-tex.pdf}
    &
    \includegraphics[height=3.5cm]{svg-inkscape/space-ABC-unique-tight-49-ext-highlighted_svg-tex.pdf}
    \\
    $\Theta_{49}$
    &
    $\Ext{\Theta_{49}}$
    \end{tabular}
\end{center}

\noindent The standard causaltope for Space 49 has dimension 34.
Below is a plot of the homogeneous linear system of causality and quasi-normalisation equations for the standard causaltope, put in reduced row echelon form:

\begin{center}
    \includegraphics[width=11cm]{svg-inkscape/space49-rref-eqs_svg-tex.pdf}
\end{center}

\noindent Rows correspond to the 29 independent linear equations.
Columns in the plot correspond to entries of empirical models, indexed as $i_A i_B i_C$ $o_A o_B o_C$.
Coefficients in the equations are color-coded as white=0, red=+1 and blue=-1.

Space 49 has closest refinements in equivalence classes 29, 35, 39 and 41; 
it is the join of its (closest) refinements.
It has closest coarsenings in equivalence classes 66, 68 and 75; 
it is the meet of its (closest) coarsenings.
It has 512 causal functions, 192 of which are not causal for any of its refinements.
It is a tight space.

The standard causaltope for Space 49 has 2 more dimensions than those of its 5 subspaces in equivalence classes 29, 35, 39 and 41.
The standard causaltope for Space 49 is the meet of the standard causaltopes for its closest coarsenings.
For completeness, below is a plot of the full homogeneous linear system of causality and quasi-normalisation equations for the standard causaltope:

\begin{center}
    \includegraphics[width=12cm]{svg-inkscape/space49-eqs_svg-tex.pdf}
\end{center}

\noindent Rows correspond to the 59 linear equations, of which 29 are independent.


\newpage
\subsection*{Space 50}

Space 50 is not induced by a causal order, but it is a refinement of the space 92 induced by the definite causal order $\total{\ev{A},\ev{C}}\vee\total{\ev{B},\ev{C}}$.
Its equivalence class under event-input permutation symmetry contains 48 spaces.
Space 50 differs as follows from the space induced by causal order $\total{\ev{A},\ev{C}}\vee\total{\ev{B},\ev{C}}$:
\begin{itemize}
  \item The outputs at events \evset{\ev{B}, \ev{C}} are independent of the input at event \ev{A} when the inputs at events \evset{B, C} are given by \hist{B/1,C/0} and \hist{B/0,C/1}.
  \item The outputs at events \evset{\ev{A}, \ev{C}} are independent of the input at event \ev{B} when the inputs at events \evset{A, C} are given by \hist{A/1,C/1}.
\end{itemize}

\noindent Below are the histories and extended histories for space 50: 
\begin{center}
    \begin{tabular}{cc}
    \includegraphics[height=3.5cm]{svg-inkscape/space-ABC-unique-untight-50-highlighted_svg-tex.pdf}
    &
    \includegraphics[height=3.5cm]{svg-inkscape/space-ABC-unique-untight-50-ext-highlighted_svg-tex.pdf}
    \\
    $\Theta_{50}$
    &
    $\Ext{\Theta_{50}}$
    \end{tabular}
\end{center}

\noindent The standard causaltope for Space 50 has dimension 34.
Below is a plot of the homogeneous linear system of causality and quasi-normalisation equations for the standard causaltope, put in reduced row echelon form:

\begin{center}
    \includegraphics[width=11cm]{svg-inkscape/space50-rref-eqs_svg-tex.pdf}
\end{center}

\noindent Rows correspond to the 29 independent linear equations.
Columns in the plot correspond to entries of empirical models, indexed as $i_A i_B i_C$ $o_A o_B o_C$.
Coefficients in the equations are color-coded as white=0, red=+1 and blue=-1.

Space 50 has closest refinements in equivalence classes 29, 34, 36, 39 and 40; 
it is the join of its (closest) refinements.
It has closest coarsenings in equivalence classes 66, 69, 70 and 73; 
it is the meet of its (closest) coarsenings.
It has 512 causal functions, 128 of which are not causal for any of its refinements.
It is not a tight space: for event \ev{C}, a causal function must yield identical output values on input histories \hist{A/1,C/1} and \hist{B/0,C/1}.

The standard causaltope for Space 50 has 2 more dimensions than those of its 5 subspaces in equivalence classes 29, 34, 36, 39 and 40.
The standard causaltope for Space 50 is the meet of the standard causaltopes for its closest coarsenings.
For completeness, below is a plot of the full homogeneous linear system of causality and quasi-normalisation equations for the standard causaltope:

\begin{center}
    \includegraphics[width=12cm]{svg-inkscape/space50-eqs_svg-tex.pdf}
\end{center}

\noindent Rows correspond to the 59 linear equations, of which 29 are independent.


\newpage
\subsection*{Space 51}

Space 51 is not induced by a causal order, but it is a refinement of the space 100 induced by the definite causal order $\total{\ev{A},\ev{B},\ev{C}}$.
Its equivalence class under event-input permutation symmetry contains 12 spaces.
Space 51 differs as follows from the space induced by causal order $\total{\ev{A},\ev{B},\ev{C}}$:
\begin{itemize}
  \item The outputs at events \evset{\ev{A}, \ev{C}} are independent of the input at event \ev{B} when the inputs at events \evset{A, C} are given by \hist{A/0,C/1} and \hist{A/1,C/1}.
  \item The outputs at events \evset{\ev{B}, \ev{C}} are independent of the input at event \ev{A} when the inputs at events \evset{B, C} are given by \hist{B/1,C/1} and \hist{B/0,C/1}.
  \item The output at event \ev{C} is independent of the inputs at events \evset{\ev{A}, \ev{B}} when the input at event C is given by \hist{C/1}.
\end{itemize}

\noindent Below are the histories and extended histories for space 51: 
\begin{center}
    \begin{tabular}{cc}
    \includegraphics[height=3.5cm]{svg-inkscape/space-ABC-unique-untight-51-highlighted_svg-tex.pdf}
    &
    \includegraphics[height=3.5cm]{svg-inkscape/space-ABC-unique-untight-51-ext-highlighted_svg-tex.pdf}
    \\
    $\Theta_{51}$
    &
    $\Ext{\Theta_{51}}$
    \end{tabular}
\end{center}

\noindent The standard causaltope for Space 51 has dimension 33.
Below is a plot of the homogeneous linear system of causality and quasi-normalisation equations for the standard causaltope, put in reduced row echelon form:

\begin{center}
    \includegraphics[width=11cm]{svg-inkscape/space51-rref-eqs_svg-tex.pdf}
\end{center}

\noindent Rows correspond to the 30 independent linear equations.
Columns in the plot correspond to entries of empirical models, indexed as $i_A i_B i_C$ $o_A o_B o_C$.
Coefficients in the equations are color-coded as white=0, red=+1 and blue=-1.

Space 51 has closest refinements in equivalence classes 31 and 42; 
it is the join of its (closest) refinements.
It has closest coarsenings in equivalence class 61; 
it is the meet of its (closest) coarsenings.
It has 512 causal functions, all of which are causal for at least one of its refinements.
It is not a tight space: for event \ev{B}, a causal function must yield identical output values on input histories \hist{A/0,B/0}, \hist{A/1,B/0} and \hist{B/0,C/1}, and it must also yield identical output values on input histories \hist{A/0,B/1}, \hist{A/1,B/1} and \hist{B/1,C/1}.

The standard causaltope for Space 51 coincides with that of its 2 subspaces in equivalence class 31.
The standard causaltope for Space 51 is the meet of the standard causaltopes for its closest coarsenings.
For completeness, below is a plot of the full homogeneous linear system of causality and quasi-normalisation equations for the standard causaltope:

\begin{center}
    \includegraphics[width=12cm]{svg-inkscape/space51-eqs_svg-tex.pdf}
\end{center}

\noindent Rows correspond to the 57 linear equations, of which 30 are independent.


\newpage
\subsection*{Space 52}

Space 52 is not induced by a causal order, but it is a refinement of the space induced by the indefinite causal order $\total{\ev{A},\{\ev{B},\ev{C}\}}$.
Its equivalence class under event-input permutation symmetry contains 48 spaces.
Space 52 differs as follows from the space induced by causal order $\total{\ev{A},\{\ev{B},\ev{C}\}}$:
\begin{itemize}
  \item The outputs at events \evset{\ev{A}, \ev{B}} are independent of the input at event \ev{C} when the inputs at events \evset{A, B} are given by \hist{A/0,B/0}, \hist{A/0,B/1} and \hist{A/1,B/0}.
  \item The outputs at events \evset{\ev{A}, \ev{C}} are independent of the input at event \ev{B} when the inputs at events \evset{A, C} are given by \hist{A/0,C/1}, \hist{A/1,C/0} and \hist{A/1,C/1}.
  \item The outputs at events \evset{\ev{B}, \ev{C}} are independent of the input at event \ev{A} when the inputs at events \evset{B, C} are given by \hist{B/1,C/1} and \hist{B/0,C/1}.
  \item The output at event \ev{C} is independent of the inputs at events \evset{\ev{A}, \ev{B}} when the input at event C is given by \hist{C/1}.
\end{itemize}

\noindent Below are the histories and extended histories for space 52: 
\begin{center}
    \begin{tabular}{cc}
    \includegraphics[height=3.5cm]{svg-inkscape/space-ABC-unique-untight-52-highlighted_svg-tex.pdf}
    &
    \includegraphics[height=3.5cm]{svg-inkscape/space-ABC-unique-untight-52-ext-highlighted_svg-tex.pdf}
    \\
    $\Theta_{52}$
    &
    $\Ext{\Theta_{52}}$
    \end{tabular}
\end{center}

\noindent The standard causaltope for Space 52 has dimension 33.
Below is a plot of the homogeneous linear system of causality and quasi-normalisation equations for the standard causaltope, put in reduced row echelon form:

\begin{center}
    \includegraphics[width=11cm]{svg-inkscape/space52-rref-eqs_svg-tex.pdf}
\end{center}

\noindent Rows correspond to the 30 independent linear equations.
Columns in the plot correspond to entries of empirical models, indexed as $i_A i_B i_C$ $o_A o_B o_C$.
Coefficients in the equations are color-coded as white=0, red=+1 and blue=-1.

Space 52 has closest refinements in equivalence classes 32, 42 and 43; 
it is the join of its (closest) refinements.
It has closest coarsenings in equivalence classes 62, 71 and 72; 
it is the meet of its (closest) coarsenings.
It has 512 causal functions, all of which are causal for at least one of its refinements.
It is not a tight space: for event \ev{B}, a causal function must yield identical output values on input histories \hist{A/0,B/1} and \hist{B/1,C/1}, and it must also yield identical output values on input histories \hist{A/0,B/0}, \hist{A/1,B/0} and \hist{B/0,C/1}.

The standard causaltope for Space 52 coincides with that of its subspace in equivalence class 32.
The standard causaltope for Space 52 is the meet of the standard causaltopes for its closest coarsenings.
For completeness, below is a plot of the full homogeneous linear system of causality and quasi-normalisation equations for the standard causaltope:

\begin{center}
    \includegraphics[width=12cm]{svg-inkscape/space52-eqs_svg-tex.pdf}
\end{center}

\noindent Rows correspond to the 57 linear equations, of which 30 are independent.


\newpage
\subsection*{Space 53}

Space 53 is not induced by a causal order, but it is a refinement of the space in equivalence class 92 induced by the definite causal order $\total{\ev{A},\ev{B}}\vee\total{\ev{C},\ev{B}}$ (note that the space induced by the order is not the same as space 92).
Its equivalence class under event-input permutation symmetry contains 48 spaces.
Space 53 differs as follows from the space induced by causal order $\total{\ev{A},\ev{B}}\vee\total{\ev{C},\ev{B}}$:
\begin{itemize}
  \item The outputs at events \evset{\ev{A}, \ev{B}} are independent of the input at event \ev{C} when the inputs at events \evset{A, B} are given by \hist{A/0,B/0} and \hist{A/0,B/1}.
  \item The outputs at events \evset{\ev{B}, \ev{C}} are independent of the input at event \ev{A} when the inputs at events \evset{B, C} are given by \hist{B/1,C/1}.
\end{itemize}

\noindent Below are the histories and extended histories for space 53: 
\begin{center}
    \begin{tabular}{cc}
    \includegraphics[height=3.5cm]{svg-inkscape/space-ABC-unique-untight-53-highlighted_svg-tex.pdf}
    &
    \includegraphics[height=3.5cm]{svg-inkscape/space-ABC-unique-untight-53-ext-highlighted_svg-tex.pdf}
    \\
    $\Theta_{53}$
    &
    $\Ext{\Theta_{53}}$
    \end{tabular}
\end{center}

\noindent The standard causaltope for Space 53 has dimension 34.
Below is a plot of the homogeneous linear system of causality and quasi-normalisation equations for the standard causaltope, put in reduced row echelon form:

\begin{center}
    \includegraphics[width=11cm]{svg-inkscape/space53-rref-eqs_svg-tex.pdf}
\end{center}

\noindent Rows correspond to the 29 independent linear equations.
Columns in the plot correspond to entries of empirical models, indexed as $i_A i_B i_C$ $o_A o_B o_C$.
Coefficients in the equations are color-coded as white=0, red=+1 and blue=-1.

Space 53 has closest refinements in equivalence classes 29, 34, 35, 36 and 37; 
it is the join of its (closest) refinements.
It has closest coarsenings in equivalence classes 63, 65, 66, 67 and 69; 
it is the meet of its (closest) coarsenings.
It has 512 causal functions, 64 of which are not causal for any of its refinements.
It is not a tight space: for event \ev{B}, a causal function must yield identical output values on input histories \hist{A/0,B/1} and \hist{B/1,C/1}.

The standard causaltope for Space 53 has 2 more dimensions than those of its 5 subspaces in equivalence classes 29, 34, 35, 36 and 37.
The standard causaltope for Space 53 is the meet of the standard causaltopes for its closest coarsenings.
For completeness, below is a plot of the full homogeneous linear system of causality and quasi-normalisation equations for the standard causaltope:

\begin{center}
    \includegraphics[width=12cm]{svg-inkscape/space53-eqs_svg-tex.pdf}
\end{center}

\noindent Rows correspond to the 59 linear equations, of which 29 are independent.


\newpage
\subsection*{Space 54}

Space 54 is not induced by a causal order, but it is a refinement of the space in equivalence class 100 induced by the definite causal order $\total{\ev{C},\ev{A},\ev{B}}$ (note that the space induced by the order is not the same as space 100).
Its equivalence class under event-input permutation symmetry contains 24 spaces.
Space 54 differs as follows from the space induced by causal order $\total{\ev{C},\ev{A},\ev{B}}$:
\begin{itemize}
  \item The outputs at events \evset{\ev{A}, \ev{B}} are independent of the input at event \ev{C} when the inputs at events \evset{A, B} are given by \hist{A/0,B/0} and \hist{A/0,B/1}.
  \item The outputs at events \evset{\ev{B}, \ev{C}} are independent of the input at event \ev{A} when the inputs at events \evset{B, C} are given by \hist{B/1,C/0} and \hist{B/1,C/1}.
  \item The output at event \ev{A} is independent of the input at event \ev{C} when the input at event A is given by \hist{A/0}.
\end{itemize}

\noindent Below are the histories and extended histories for space 54: 
\begin{center}
    \begin{tabular}{cc}
    \includegraphics[height=3.5cm]{svg-inkscape/space-ABC-unique-untight-54-highlighted_svg-tex.pdf}
    &
    \includegraphics[height=3.5cm]{svg-inkscape/space-ABC-unique-untight-54-ext-highlighted_svg-tex.pdf}
    \\
    $\Theta_{54}$
    &
    $\Ext{\Theta_{54}}$
    \end{tabular}
\end{center}

\noindent The standard causaltope for Space 54 has dimension 33.
Below is a plot of the homogeneous linear system of causality and quasi-normalisation equations for the standard causaltope, put in reduced row echelon form:

\begin{center}
    \includegraphics[width=11cm]{svg-inkscape/space54-rref-eqs_svg-tex.pdf}
\end{center}

\noindent Rows correspond to the 30 independent linear equations.
Columns in the plot correspond to entries of empirical models, indexed as $i_A i_B i_C$ $o_A o_B o_C$.
Coefficients in the equations are color-coded as white=0, red=+1 and blue=-1.

Space 54 has closest refinements in equivalence classes 28, 35 and 43; 
it is the join of its (closest) refinements.
It has closest coarsenings in equivalence classes 65, 75 and 76; 
it is the meet of its (closest) coarsenings.
It has 512 causal functions, 256 of which are not causal for any of its refinements.
It is not a tight space: for event \ev{B}, a causal function must yield identical output values on input histories \hist{A/0,B/1}, \hist{B/1,C/0} and \hist{B/1,C/1}.

The standard causaltope for Space 54 coincides with that of its subspace in equivalence class 28.
The standard causaltope for Space 54 is the meet of the standard causaltopes for its closest coarsenings.
For completeness, below is a plot of the full homogeneous linear system of causality and quasi-normalisation equations for the standard causaltope:

\begin{center}
    \includegraphics[width=12cm]{svg-inkscape/space54-eqs_svg-tex.pdf}
\end{center}

\noindent Rows correspond to the 57 linear equations, of which 30 are independent.


\newpage
\subsection*{Space 55}

Space 55 is not induced by a causal order, but it is a refinement of the space in equivalence class 100 induced by the definite causal order $\total{\ev{C},\ev{A},\ev{B}}$ (note that the space induced by the order is not the same as space 100).
Its equivalence class under event-input permutation symmetry contains 24 spaces.
Space 55 differs as follows from the space induced by causal order $\total{\ev{C},\ev{A},\ev{B}}$:
\begin{itemize}
  \item The outputs at events \evset{\ev{A}, \ev{B}} are independent of the input at event \ev{C} when the inputs at events \evset{A, B} are given by \hist{A/0,B/0} and \hist{A/0,B/1}.
  \item The outputs at events \evset{\ev{B}, \ev{C}} are independent of the input at event \ev{A} when the inputs at events \evset{B, C} are given by \hist{B/1,C/0} and \hist{B/0,C/1}.
  \item The output at event \ev{A} is independent of the input at event \ev{C} when the input at event A is given by \hist{A/0}.
\end{itemize}

\noindent Below are the histories and extended histories for space 55: 
\begin{center}
    \begin{tabular}{cc}
    \includegraphics[height=3.5cm]{svg-inkscape/space-ABC-unique-untight-55-highlighted_svg-tex.pdf}
    &
    \includegraphics[height=3.5cm]{svg-inkscape/space-ABC-unique-untight-55-ext-highlighted_svg-tex.pdf}
    \\
    $\Theta_{55}$
    &
    $\Ext{\Theta_{55}}$
    \end{tabular}
\end{center}

\noindent The standard causaltope for Space 55 has dimension 33.
Below is a plot of the homogeneous linear system of causality and quasi-normalisation equations for the standard causaltope, put in reduced row echelon form:

\begin{center}
    \includegraphics[width=11cm]{svg-inkscape/space55-rref-eqs_svg-tex.pdf}
\end{center}

\noindent Rows correspond to the 30 independent linear equations.
Columns in the plot correspond to entries of empirical models, indexed as $i_A i_B i_C$ $o_A o_B o_C$.
Coefficients in the equations are color-coded as white=0, red=+1 and blue=-1.

Space 55 has closest refinements in equivalence classes 36 and 43; 
it is the join of its (closest) refinements.
It has closest coarsenings in equivalence classes 65 and 73; 
it is the meet of its (closest) coarsenings.
It has 512 causal functions, 192 of which are not causal for any of its refinements.
It is not a tight space: for event \ev{B}, a causal function must yield identical output values on input histories \hist{A/0,B/0} and \hist{B/0,C/1}, and it must also yield identical output values on input histories \hist{A/0,B/1} and \hist{B/1,C/0}.

The standard causaltope for Space 55 has 1 more dimension than that of its subspace in equivalence class 36.
The standard causaltope for Space 55 is the meet of the standard causaltopes for its closest coarsenings.
For completeness, below is a plot of the full homogeneous linear system of causality and quasi-normalisation equations for the standard causaltope:

\begin{center}
    \includegraphics[width=12cm]{svg-inkscape/space55-eqs_svg-tex.pdf}
\end{center}

\noindent Rows correspond to the 57 linear equations, of which 30 are independent.


\newpage
\subsection*{Space 56}

Space 56 is not induced by a causal order, but it is a refinement of the space in equivalence class 100 induced by the definite causal order $\total{\ev{A},\ev{C},\ev{B}}$ (note that the space induced by the order is not the same as space 100).
Its equivalence class under event-input permutation symmetry contains 48 spaces.
Space 56 differs as follows from the space induced by causal order $\total{\ev{A},\ev{C},\ev{B}}$:
\begin{itemize}
  \item The outputs at events \evset{\ev{A}, \ev{B}} are independent of the input at event \ev{C} when the inputs at events \evset{A, B} are given by \hist{A/0,B/0}, \hist{A/0,B/1} and \hist{A/1,B/0}.
  \item The outputs at events \evset{\ev{B}, \ev{C}} are independent of the input at event \ev{A} when the inputs at events \evset{B, C} are given by \hist{B/1,C/0}.
  \item The output at event \ev{C} is independent of the input at event \ev{A} when the input at event C is given by \hist{C/0}.
\end{itemize}

\noindent Below are the histories and extended histories for space 56: 
\begin{center}
    \begin{tabular}{cc}
    \includegraphics[height=3.5cm]{svg-inkscape/space-ABC-unique-untight-56-highlighted_svg-tex.pdf}
    &
    \includegraphics[height=3.5cm]{svg-inkscape/space-ABC-unique-untight-56-ext-highlighted_svg-tex.pdf}
    \\
    $\Theta_{56}$
    &
    $\Ext{\Theta_{56}}$
    \end{tabular}
\end{center}

\noindent The standard causaltope for Space 56 has dimension 33.
Below is a plot of the homogeneous linear system of causality and quasi-normalisation equations for the standard causaltope, put in reduced row echelon form:

\begin{center}
    \includegraphics[width=11cm]{svg-inkscape/space56-rref-eqs_svg-tex.pdf}
\end{center}

\noindent Rows correspond to the 30 independent linear equations.
Columns in the plot correspond to entries of empirical models, indexed as $i_A i_B i_C$ $o_A o_B o_C$.
Coefficients in the equations are color-coded as white=0, red=+1 and blue=-1.

Space 56 has closest refinements in equivalence classes 29, 38 and 43; 
it is the join of its (closest) refinements.
It has closest coarsenings in equivalence classes 62, 63, 73, 74 and 75; 
it is the meet of its (closest) coarsenings.
It has 512 causal functions, 64 of which are not causal for any of its refinements.
It is not a tight space: for event \ev{B}, a causal function must yield identical output values on input histories \hist{A/0,B/1} and \hist{B/1,C/0}.

The standard causaltope for Space 56 has 1 more dimension than that of its subspace in equivalence class 29.
The standard causaltope for Space 56 is the meet of the standard causaltopes for its closest coarsenings.
For completeness, below is a plot of the full homogeneous linear system of causality and quasi-normalisation equations for the standard causaltope:

\begin{center}
    \includegraphics[width=12cm]{svg-inkscape/space56-eqs_svg-tex.pdf}
\end{center}

\noindent Rows correspond to the 57 linear equations, of which 30 are independent.


\newpage
\subsection*{Space 57}

Space 57 is not induced by a causal order, but it is a refinement of the space 100 induced by the definite causal order $\total{\ev{A},\ev{B},\ev{C}}$.
Its equivalence class under event-input permutation symmetry contains 48 spaces.
Space 57 differs as follows from the space induced by causal order $\total{\ev{A},\ev{B},\ev{C}}$:
\begin{itemize}
  \item The outputs at events \evset{\ev{A}, \ev{C}} are independent of the input at event \ev{B} when the inputs at events \evset{A, C} are given by \hist{A/0,C/1}, \hist{A/1,C/0} and \hist{A/1,C/1}.
  \item The outputs at events \evset{\ev{B}, \ev{C}} are independent of the input at event \ev{A} when the inputs at events \evset{B, C} are given by \hist{B/1,C/1}.
  \item The output at event \ev{B} is independent of the input at event \ev{A} when the input at event B is given by \hist{B/1}.
\end{itemize}

\noindent Below are the histories and extended histories for space 57: 
\begin{center}
    \begin{tabular}{cc}
    \includegraphics[height=3.5cm]{svg-inkscape/space-ABC-unique-untight-57-highlighted_svg-tex.pdf}
    &
    \includegraphics[height=3.5cm]{svg-inkscape/space-ABC-unique-untight-57-ext-highlighted_svg-tex.pdf}
    \\
    $\Theta_{57}$
    &
    $\Ext{\Theta_{57}}$
    \end{tabular}
\end{center}

\noindent The standard causaltope for Space 57 has dimension 33.
Below is a plot of the homogeneous linear system of causality and quasi-normalisation equations for the standard causaltope, put in reduced row echelon form:

\begin{center}
    \includegraphics[width=11cm]{svg-inkscape/space57-rref-eqs_svg-tex.pdf}
\end{center}

\noindent Rows correspond to the 30 independent linear equations.
Columns in the plot correspond to entries of empirical models, indexed as $i_A i_B i_C$ $o_A o_B o_C$.
Coefficients in the equations are color-coded as white=0, red=+1 and blue=-1.

Space 57 has closest refinements in equivalence classes 30, 34, 38 and 43; 
it is the join of its (closest) refinements.
It has closest coarsenings in equivalence classes 63, 71, 73, 74 and 76; 
it is the meet of its (closest) coarsenings.
It has 512 causal functions, all of which are causal for at least one of its refinements.
It is not a tight space: for event \ev{C}, a causal function must yield identical output values on input histories \hist{A/0,C/1}, \hist{A/1,C/1} and \hist{B/1,C/1}.

The standard causaltope for Space 57 coincides with that of its subspace in equivalence class 30.
The standard causaltope for Space 57 is the meet of the standard causaltopes for its closest coarsenings.
For completeness, below is a plot of the full homogeneous linear system of causality and quasi-normalisation equations for the standard causaltope:

\begin{center}
    \includegraphics[width=12cm]{svg-inkscape/space57-eqs_svg-tex.pdf}
\end{center}

\noindent Rows correspond to the 57 linear equations, of which 30 are independent.


\newpage
\subsection*{Space 58}

Space 58 is not induced by a causal order, but it is a refinement of the space 77 induced by the definite causal order $\total{\ev{A},\ev{B}}\vee\total{\ev{A},\ev{C}}$.
Its equivalence class under event-input permutation symmetry contains 12 spaces.
Space 58 differs as follows from the space induced by causal order $\total{\ev{A},\ev{B}}\vee\total{\ev{A},\ev{C}}$:
\begin{itemize}
  \item The output at event \ev{C} is independent of the input at event \ev{A} when the input at event C is given by \hist{C/0}.
\end{itemize}

\noindent Below are the histories and extended histories for space 58: 
\begin{center}
    \begin{tabular}{cc}
    \includegraphics[height=3.5cm]{svg-inkscape/space-ABC-unique-tight-58-highlighted_svg-tex.pdf}
    &
    \includegraphics[height=3.5cm]{svg-inkscape/space-ABC-unique-tight-58-ext-highlighted_svg-tex.pdf}
    \\
    $\Theta_{58}$
    &
    $\Ext{\Theta_{58}}$
    \end{tabular}
\end{center}

\noindent The standard causaltope for Space 58 has dimension 33.
Below is a plot of the homogeneous linear system of causality and quasi-normalisation equations for the standard causaltope, put in reduced row echelon form:

\begin{center}
    \includegraphics[width=11cm]{svg-inkscape/space58-rref-eqs_svg-tex.pdf}
\end{center}

\noindent Rows correspond to the 30 independent linear equations.
Columns in the plot correspond to entries of empirical models, indexed as $i_A i_B i_C$ $o_A o_B o_C$.
Coefficients in the equations are color-coded as white=0, red=+1 and blue=-1.

Space 58 has closest refinements in equivalence classes 33 and 38; 
it is the join of its (closest) refinements.
It has closest coarsenings in equivalence classes 64, 74 and 77; 
it is the meet of its (closest) coarsenings.
It has 512 causal functions, 128 of which are not causal for any of its refinements.
It is a tight space.

The standard causaltope for Space 58 has 1 more dimension than that of its subspace in equivalence class 33.
The standard causaltope for Space 58 is the meet of the standard causaltopes for its closest coarsenings.
For completeness, below is a plot of the full homogeneous linear system of causality and quasi-normalisation equations for the standard causaltope:

\begin{center}
    \includegraphics[width=12cm]{svg-inkscape/space58-eqs_svg-tex.pdf}
\end{center}

\noindent Rows correspond to the 57 linear equations, of which 30 are independent.


\newpage
\subsection*{Space 59}

Space 59 is not induced by a causal order, but it is a refinement of the space 92 induced by the definite causal order $\total{\ev{A},\ev{C}}\vee\total{\ev{B},\ev{C}}$.
Its equivalence class under event-input permutation symmetry contains 24 spaces.
Space 59 differs as follows from the space induced by causal order $\total{\ev{A},\ev{C}}\vee\total{\ev{B},\ev{C}}$:
\begin{itemize}
  \item The outputs at events \evset{\ev{A}, \ev{C}} are independent of the input at event \ev{B} when the inputs at events \evset{A, C} are given by \hist{A/0,C/1}.
  \item The outputs at events \evset{\ev{B}, \ev{C}} are independent of the input at event \ev{A} when the inputs at events \evset{B, C} are given by \hist{B/1,C/1} and \hist{B/0,C/1}.
\end{itemize}

\noindent Below are the histories and extended histories for space 59: 
\begin{center}
    \begin{tabular}{cc}
    \includegraphics[height=3.5cm]{svg-inkscape/space-ABC-unique-untight-59-highlighted_svg-tex.pdf}
    &
    \includegraphics[height=3.5cm]{svg-inkscape/space-ABC-unique-untight-59-ext-highlighted_svg-tex.pdf}
    \\
    $\Theta_{59}$
    &
    $\Ext{\Theta_{59}}$
    \end{tabular}
\end{center}

\noindent The standard causaltope for Space 59 has dimension 34.
Below is a plot of the homogeneous linear system of causality and quasi-normalisation equations for the standard causaltope, put in reduced row echelon form:

\begin{center}
    \includegraphics[width=11cm]{svg-inkscape/space59-rref-eqs_svg-tex.pdf}
\end{center}

\noindent Rows correspond to the 29 independent linear equations.
Columns in the plot correspond to entries of empirical models, indexed as $i_A i_B i_C$ $o_A o_B o_C$.
Coefficients in the equations are color-coded as white=0, red=+1 and blue=-1.

Space 59 has closest refinements in equivalence classes 34, 35, 39 and 44; 
it is the join of its (closest) refinements.
It has closest coarsenings in equivalence classes 68, 69 and 76; 
it is the meet of its (closest) coarsenings.
It has 512 causal functions, all of which are causal for at least one of its refinements.
It is not a tight space: for event \ev{C}, a causal function must yield identical output values on input histories \hist{A/0,C/1}, \hist{B/0,C/1} and \hist{B/1,C/1}.

The standard causaltope for Space 59 has 1 more dimension than that of its subspace in equivalence class 44.
The standard causaltope for Space 59 is the meet of the standard causaltopes for its closest coarsenings.
For completeness, below is a plot of the full homogeneous linear system of causality and quasi-normalisation equations for the standard causaltope:

\begin{center}
    \includegraphics[width=12cm]{svg-inkscape/space59-eqs_svg-tex.pdf}
\end{center}

\noindent Rows correspond to the 59 linear equations, of which 29 are independent.


\newpage
\subsection*{Space 60}

Space 60 is not induced by a causal order, but it is a refinement of the space in equivalence class 100 induced by the definite causal order $\total{\ev{A},\ev{C},\ev{B}}$ (note that the space induced by the order is not the same as space 100).
Its equivalence class under event-input permutation symmetry contains 24 spaces.
Space 60 differs as follows from the space induced by causal order $\total{\ev{A},\ev{C},\ev{B}}$:
\begin{itemize}
  \item The outputs at events \evset{\ev{A}, \ev{B}} are independent of the input at event \ev{C} when the inputs at events \evset{A, B} are given by \hist{A/0,B/0} and \hist{A/0,B/1}.
  \item The outputs at events \evset{\ev{B}, \ev{C}} are independent of the input at event \ev{A} when the inputs at events \evset{B, C} are given by \hist{B/1,C/0} and \hist{B/0,C/0}.
  \item The output at event \ev{C} is independent of the input at event \ev{A} when the input at event C is given by \hist{C/0}.
\end{itemize}

\noindent Below are the histories and extended histories for space 60: 
\begin{center}
    \begin{tabular}{cc}
    \includegraphics[height=3.5cm]{svg-inkscape/space-ABC-unique-untight-60-highlighted_svg-tex.pdf}
    &
    \includegraphics[height=3.5cm]{svg-inkscape/space-ABC-unique-untight-60-ext-highlighted_svg-tex.pdf}
    \\
    $\Theta_{60}$
    &
    $\Ext{\Theta_{60}}$
    \end{tabular}
\end{center}

\noindent The standard causaltope for Space 60 has dimension 33.
Below is a plot of the homogeneous linear system of causality and quasi-normalisation equations for the standard causaltope, put in reduced row echelon form:

\begin{center}
    \includegraphics[width=11cm]{svg-inkscape/space60-rref-eqs_svg-tex.pdf}
\end{center}

\noindent Rows correspond to the 30 independent linear equations.
Columns in the plot correspond to entries of empirical models, indexed as $i_A i_B i_C$ $o_A o_B o_C$.
Coefficients in the equations are color-coded as white=0, red=+1 and blue=-1.

Space 60 has closest refinements in equivalence classes 37 and 43; 
it is the join of its (closest) refinements.
It has closest coarsenings in equivalence classes 63, 65 and 72; 
it is the meet of its (closest) coarsenings.
It has 512 causal functions, 192 of which are not causal for any of its refinements.
It is not a tight space: for event \ev{B}, a causal function must yield identical output values on input histories \hist{A/0,B/0} and \hist{B/0,C/0}, and it must also yield identical output values on input histories \hist{A/0,B/1} and \hist{B/1,C/0}.

The standard causaltope for Space 60 has 1 more dimension than that of its subspace in equivalence class 37.
The standard causaltope for Space 60 is the meet of the standard causaltopes for its closest coarsenings.
For completeness, below is a plot of the full homogeneous linear system of causality and quasi-normalisation equations for the standard causaltope:

\begin{center}
    \includegraphics[width=12cm]{svg-inkscape/space60-eqs_svg-tex.pdf}
\end{center}

\noindent Rows correspond to the 57 linear equations, of which 30 are independent.


\newpage
\subsection*{Space 61}

Space 61 is not induced by a causal order, but it is a refinement of the space 100 induced by the definite causal order $\total{\ev{A},\ev{B},\ev{C}}$.
Its equivalence class under event-input permutation symmetry contains 24 spaces.
Space 61 differs as follows from the space induced by causal order $\total{\ev{A},\ev{B},\ev{C}}$:
\begin{itemize}
  \item The outputs at events \evset{\ev{A}, \ev{C}} are independent of the input at event \ev{B} when the inputs at events \evset{A, C} are given by \hist{A/0,C/1} and \hist{A/1,C/1}.
  \item The outputs at events \evset{\ev{B}, \ev{C}} are independent of the input at event \ev{A} when the inputs at events \evset{B, C} are given by \hist{B/1,C/1}.
  \item The output at event \ev{C} is independent of the inputs at events \evset{\ev{A}, \ev{B}} when the input at event C is given by \hist{C/1}.
\end{itemize}

\noindent Below are the histories and extended histories for space 61: 
\begin{center}
    \begin{tabular}{cc}
    \includegraphics[height=3.5cm]{svg-inkscape/space-ABC-unique-untight-61-highlighted_svg-tex.pdf}
    &
    \includegraphics[height=3.5cm]{svg-inkscape/space-ABC-unique-untight-61-ext-highlighted_svg-tex.pdf}
    \\
    $\Theta_{61}$
    &
    $\Ext{\Theta_{61}}$
    \end{tabular}
\end{center}

\noindent The standard causaltope for Space 61 has dimension 35.
Below is a plot of the homogeneous linear system of causality and quasi-normalisation equations for the standard causaltope, put in reduced row echelon form:

\begin{center}
    \includegraphics[width=11cm]{svg-inkscape/space61-rref-eqs_svg-tex.pdf}
\end{center}

\noindent Rows correspond to the 28 independent linear equations.
Columns in the plot correspond to entries of empirical models, indexed as $i_A i_B i_C$ $o_A o_B o_C$.
Coefficients in the equations are color-coded as white=0, red=+1 and blue=-1.

Space 61 has closest refinements in equivalence classes 45, 47 and 51; 
it is the join of its (closest) refinements.
It has closest coarsenings in equivalence class 78; 
it does not arise as a nontrivial meet in the hierarchy.
It has 1024 causal functions, all of which are causal for at least one of its refinements.
It is not a tight space: for event \ev{B}, a causal function must yield identical output values on input histories \hist{A/0,B/1}, \hist{A/1,B/1} and \hist{B/1,C/1}.

The standard causaltope for Space 61 coincides with that of its subspace in equivalence class 45.
The standard causaltope for Space 61 has 2 dimensions fewer than the meet of the standard causaltopes for its closest coarsenings.
For completeness, below is a plot of the full homogeneous linear system of causality and quasi-normalisation equations for the standard causaltope:

\begin{center}
    \includegraphics[width=12cm]{svg-inkscape/space61-eqs_svg-tex.pdf}
\end{center}

\noindent Rows correspond to the 53 linear equations, of which 28 are independent.


\newpage
\subsection*{Space 62}

Space 62 is not induced by a causal order, but it is a refinement of the space induced by the indefinite causal order $\total{\ev{A},\{\ev{B},\ev{C}\}}$.
Its equivalence class under event-input permutation symmetry contains 48 spaces.
Space 62 differs as follows from the space induced by causal order $\total{\ev{A},\{\ev{B},\ev{C}\}}$:
\begin{itemize}
  \item The outputs at events \evset{\ev{A}, \ev{B}} are independent of the input at event \ev{C} when the inputs at events \evset{A, B} are given by \hist{A/0,B/0}, \hist{A/0,B/1} and \hist{A/1,B/0}.
  \item The outputs at events \evset{\ev{A}, \ev{C}} are independent of the input at event \ev{B} when the inputs at events \evset{A, C} are given by \hist{A/0,C/1}, \hist{A/1,C/0} and \hist{A/1,C/1}.
  \item The outputs at events \evset{\ev{B}, \ev{C}} are independent of the input at event \ev{A} when the inputs at events \evset{B, C} are given by \hist{B/1,C/1}.
  \item The output at event \ev{C} is independent of the inputs at events \evset{\ev{A}, \ev{B}} when the input at event C is given by \hist{C/1}.
\end{itemize}

\noindent Below are the histories and extended histories for space 62: 
\begin{center}
    \begin{tabular}{cc}
    \includegraphics[height=3.5cm]{svg-inkscape/space-ABC-unique-untight-62-highlighted_svg-tex.pdf}
    &
    \includegraphics[height=3.5cm]{svg-inkscape/space-ABC-unique-untight-62-ext-highlighted_svg-tex.pdf}
    \\
    $\Theta_{62}$
    &
    $\Ext{\Theta_{62}}$
    \end{tabular}
\end{center}

\noindent The standard causaltope for Space 62 has dimension 35.
Below is a plot of the homogeneous linear system of causality and quasi-normalisation equations for the standard causaltope, put in reduced row echelon form:

\begin{center}
    \includegraphics[width=11cm]{svg-inkscape/space62-rref-eqs_svg-tex.pdf}
\end{center}

\noindent Rows correspond to the 28 independent linear equations.
Columns in the plot correspond to entries of empirical models, indexed as $i_A i_B i_C$ $o_A o_B o_C$.
Coefficients in the equations are color-coded as white=0, red=+1 and blue=-1.

Space 62 has closest refinements in equivalence classes 47, 52 and 56; 
it is the join of its (closest) refinements.
It has closest coarsenings in equivalence classes 79 and 85; 
it is the meet of its (closest) coarsenings.
It has 1024 causal functions, 192 of which are not causal for any of its refinements.
It is not a tight space: for event \ev{B}, a causal function must yield identical output values on input histories \hist{A/0,B/1} and \hist{B/1,C/1}.

The standard causaltope for Space 62 has 2 more dimensions than those of its 3 subspaces in equivalence classes 47, 52 and 56.
The standard causaltope for Space 62 is the meet of the standard causaltopes for its closest coarsenings.
For completeness, below is a plot of the full homogeneous linear system of causality and quasi-normalisation equations for the standard causaltope:

\begin{center}
    \includegraphics[width=12cm]{svg-inkscape/space62-eqs_svg-tex.pdf}
\end{center}

\noindent Rows correspond to the 53 linear equations, of which 28 are independent.


\newpage
\subsection*{Space 63}

Space 63 is not induced by a causal order, but it is a refinement of the space 100 induced by the definite causal order $\total{\ev{A},\ev{B},\ev{C}}$.
Its equivalence class under event-input permutation symmetry contains 48 spaces.
Space 63 differs as follows from the space induced by causal order $\total{\ev{A},\ev{B},\ev{C}}$:
\begin{itemize}
  \item The outputs at events \evset{\ev{B}, \ev{C}} are independent of the input at event \ev{A} when the inputs at events \evset{B, C} are given by \hist{B/1,C/1}.
  \item The outputs at events \evset{\ev{A}, \ev{C}} are independent of the input at event \ev{B} when the inputs at events \evset{A, C} are given by \hist{A/1,C/0} and \hist{A/1,C/1}.
  \item The output at event \ev{B} is independent of the input at event \ev{A} when the input at event B is given by \hist{B/1}.
\end{itemize}

\noindent Below are the histories and extended histories for space 63: 
\begin{center}
    \begin{tabular}{cc}
    \includegraphics[height=3.5cm]{svg-inkscape/space-ABC-unique-untight-63-highlighted_svg-tex.pdf}
    &
    \includegraphics[height=3.5cm]{svg-inkscape/space-ABC-unique-untight-63-ext-highlighted_svg-tex.pdf}
    \\
    $\Theta_{63}$
    &
    $\Ext{\Theta_{63}}$
    \end{tabular}
\end{center}

\noindent The standard causaltope for Space 63 has dimension 35.
Below is a plot of the homogeneous linear system of causality and quasi-normalisation equations for the standard causaltope, put in reduced row echelon form:

\begin{center}
    \includegraphics[width=11cm]{svg-inkscape/space63-rref-eqs_svg-tex.pdf}
\end{center}

\noindent Rows correspond to the 28 independent linear equations.
Columns in the plot correspond to entries of empirical models, indexed as $i_A i_B i_C$ $o_A o_B o_C$.
Coefficients in the equations are color-coded as white=0, red=+1 and blue=-1.

Space 63 has closest refinements in equivalence classes 53, 56, 57 and 60; 
it is the join of its (closest) refinements.
It has closest coarsenings in equivalence classes 80, 82, 83 and 85; 
it is the meet of its (closest) coarsenings.
It has 1024 causal functions, 384 of which are not causal for any of its refinements.
It is not a tight space: for event \ev{C}, a causal function must yield identical output values on input histories \hist{A/1,C/1} and \hist{B/1,C/1}.

The standard causaltope for Space 63 has 1 more dimension than that of its subspace in equivalence class 53.
The standard causaltope for Space 63 is the meet of the standard causaltopes for its closest coarsenings.
For completeness, below is a plot of the full homogeneous linear system of causality and quasi-normalisation equations for the standard causaltope:

\begin{center}
    \includegraphics[width=12cm]{svg-inkscape/space63-eqs_svg-tex.pdf}
\end{center}

\noindent Rows correspond to the 53 linear equations, of which 28 are independent.


\newpage
\subsection*{Space 64}

Space 64 is not induced by a causal order, but it is a refinement of the space 100 induced by the definite causal order $\total{\ev{A},\ev{B},\ev{C}}$.
Its equivalence class under event-input permutation symmetry contains 24 spaces.
Space 64 differs as follows from the space induced by causal order $\total{\ev{A},\ev{B},\ev{C}}$:
\begin{itemize}
  \item The outputs at events \evset{\ev{A}, \ev{C}} are independent of the input at event \ev{B} when the inputs at events \evset{A, C} are given by \hist{A/0,C/0}, \hist{A/1,C/0} and \hist{A/1,C/1}.
  \item The output at event \ev{C} is independent of the inputs at events \evset{\ev{A}, \ev{B}} when the input at event C is given by \hist{C/0}.
\end{itemize}

\noindent Below are the histories and extended histories for space 64: 
\begin{center}
    \begin{tabular}{cc}
    \includegraphics[height=3.5cm]{svg-inkscape/space-ABC-unique-tight-64-highlighted_svg-tex.pdf}
    &
    \includegraphics[height=3.5cm]{svg-inkscape/space-ABC-unique-tight-64-ext-highlighted_svg-tex.pdf}
    \\
    $\Theta_{64}$
    &
    $\Ext{\Theta_{64}}$
    \end{tabular}
\end{center}

\noindent The standard causaltope for Space 64 has dimension 35.
Below is a plot of the homogeneous linear system of causality and quasi-normalisation equations for the standard causaltope, put in reduced row echelon form:

\begin{center}
    \includegraphics[width=11cm]{svg-inkscape/space64-rref-eqs_svg-tex.pdf}
\end{center}

\noindent Rows correspond to the 28 independent linear equations.
Columns in the plot correspond to entries of empirical models, indexed as $i_A i_B i_C$ $o_A o_B o_C$.
Coefficients in the equations are color-coded as white=0, red=+1 and blue=-1.

Space 64 has closest refinements in equivalence classes 47 and 58; 
it is the join of its (closest) refinements.
It has closest coarsenings in equivalence classes 78, 79 and 88; 
it is the meet of its (closest) coarsenings.
It has 1024 causal functions, 256 of which are not causal for any of its refinements.
It is a tight space.

The standard causaltope for Space 64 has 2 more dimensions than those of its 3 subspaces in equivalence classes 47 and 58.
The standard causaltope for Space 64 is the meet of the standard causaltopes for its closest coarsenings.
For completeness, below is a plot of the full homogeneous linear system of causality and quasi-normalisation equations for the standard causaltope:

\begin{center}
    \includegraphics[width=12cm]{svg-inkscape/space64-eqs_svg-tex.pdf}
\end{center}

\noindent Rows correspond to the 53 linear equations, of which 28 are independent.


\newpage
\subsection*{Space 65}

Space 65 is not induced by a causal order, but it is a refinement of the space 100 induced by the definite causal order $\total{\ev{A},\ev{B},\ev{C}}$.
Its equivalence class under event-input permutation symmetry contains 48 spaces.
Space 65 differs as follows from the space induced by causal order $\total{\ev{A},\ev{B},\ev{C}}$:
\begin{itemize}
  \item The outputs at events \evset{\ev{B}, \ev{C}} are independent of the input at event \ev{A} when the inputs at events \evset{B, C} are given by \hist{B/1,C/0} and \hist{B/1,C/1}.
  \item The output at event \ev{B} is independent of the input at event \ev{A} when the input at event B is given by \hist{B/1}.
  \item The outputs at events \evset{\ev{A}, \ev{C}} are independent of the input at event \ev{B} when the inputs at events \evset{A, C} are given by \hist{A/1,C/1}.
\end{itemize}

\noindent Below are the histories and extended histories for space 65: 
\begin{center}
    \begin{tabular}{cc}
    \includegraphics[height=3.5cm]{svg-inkscape/space-ABC-unique-untight-65-highlighted_svg-tex.pdf}
    &
    \includegraphics[height=3.5cm]{svg-inkscape/space-ABC-unique-untight-65-ext-highlighted_svg-tex.pdf}
    \\
    $\Theta_{65}$
    &
    $\Ext{\Theta_{65}}$
    \end{tabular}
\end{center}

\noindent The standard causaltope for Space 65 has dimension 35.
Below is a plot of the homogeneous linear system of causality and quasi-normalisation equations for the standard causaltope, put in reduced row echelon form:

\begin{center}
    \includegraphics[width=11cm]{svg-inkscape/space65-rref-eqs_svg-tex.pdf}
\end{center}

\noindent Rows correspond to the 28 independent linear equations.
Columns in the plot correspond to entries of empirical models, indexed as $i_A i_B i_C$ $o_A o_B o_C$.
Coefficients in the equations are color-coded as white=0, red=+1 and blue=-1.

Space 65 has closest refinements in equivalence classes 53, 54, 55 and 60; 
it is the join of its (closest) refinements.
It has closest coarsenings in equivalence classes 80, 81 and 82; 
it is the meet of its (closest) coarsenings.
It has 1024 causal functions, 256 of which are not causal for any of its refinements.
It is not a tight space: for event \ev{C}, a causal function must yield identical output values on input histories \hist{A/1,C/1} and \hist{B/1,C/1}.

The standard causaltope for Space 65 has 1 more dimension than that of its subspace in equivalence class 53.
The standard causaltope for Space 65 is the meet of the standard causaltopes for its closest coarsenings.
For completeness, below is a plot of the full homogeneous linear system of causality and quasi-normalisation equations for the standard causaltope:

\begin{center}
    \includegraphics[width=12cm]{svg-inkscape/space65-eqs_svg-tex.pdf}
\end{center}

\noindent Rows correspond to the 53 linear equations, of which 28 are independent.


\newpage
\subsection*{Space 66}

Space 66 is not induced by a causal order, but it is a refinement of the space 92 induced by the definite causal order $\total{\ev{A},\ev{C}}\vee\total{\ev{B},\ev{C}}$.
Its equivalence class under event-input permutation symmetry contains 24 spaces.
Space 66 differs as follows from the space induced by causal order $\total{\ev{A},\ev{C}}\vee\total{\ev{B},\ev{C}}$:
\begin{itemize}
  \item The outputs at events \evset{\ev{B}, \ev{C}} are independent of the input at event \ev{A} when the inputs at events \evset{B, C} are given by \hist{B/1,C/1}.
  \item The outputs at events \evset{\ev{A}, \ev{C}} are independent of the input at event \ev{B} when the inputs at events \evset{A, C} are given by \hist{A/1,C/0}.
\end{itemize}

\noindent Below are the histories and extended histories for space 66: 
\begin{center}
    \begin{tabular}{cc}
    \includegraphics[height=3.5cm]{svg-inkscape/space-ABC-unique-tight-66-highlighted_svg-tex.pdf}
    &
    \includegraphics[height=3.5cm]{svg-inkscape/space-ABC-unique-tight-66-ext-highlighted_svg-tex.pdf}
    \\
    $\Theta_{66}$
    &
    $\Ext{\Theta_{66}}$
    \end{tabular}
\end{center}

\noindent The standard causaltope for Space 66 has dimension 36.
Below is a plot of the homogeneous linear system of causality and quasi-normalisation equations for the standard causaltope, put in reduced row echelon form:

\begin{center}
    \includegraphics[width=11cm]{svg-inkscape/space66-rref-eqs_svg-tex.pdf}
\end{center}

\noindent Rows correspond to the 27 independent linear equations.
Columns in the plot correspond to entries of empirical models, indexed as $i_A i_B i_C$ $o_A o_B o_C$.
Coefficients in the equations are color-coded as white=0, red=+1 and blue=-1.

Space 66 has closest refinements in equivalence classes 49, 50 and 53; 
it is the join of its (closest) refinements.
It has closest coarsenings in equivalence classes 80 and 84; 
it is the meet of its (closest) coarsenings.
It has 1024 causal functions, 192 of which are not causal for any of its refinements.
It is a tight space.

The standard causaltope for Space 66 has 2 more dimensions than those of its 6 subspaces in equivalence classes 49, 50 and 53.
The standard causaltope for Space 66 is the meet of the standard causaltopes for its closest coarsenings.
For completeness, below is a plot of the full homogeneous linear system of causality and quasi-normalisation equations for the standard causaltope:

\begin{center}
    \includegraphics[width=12cm]{svg-inkscape/space66-eqs_svg-tex.pdf}
\end{center}

\noindent Rows correspond to the 55 linear equations, of which 27 are independent.


\newpage
\subsection*{Space 67}

Space 67 is not induced by a causal order, but it is a refinement of the space 92 induced by the definite causal order $\total{\ev{A},\ev{C}}\vee\total{\ev{B},\ev{C}}$.
Its equivalence class under event-input permutation symmetry contains 12 spaces.
Space 67 differs as follows from the space induced by causal order $\total{\ev{A},\ev{C}}\vee\total{\ev{B},\ev{C}}$:
\begin{itemize}
  \item The outputs at events \evset{\ev{B}, \ev{C}} are independent of the input at event \ev{A} when the inputs at events \evset{B, C} are given by \hist{B/1,C/0} and \hist{B/1,C/1}.
\end{itemize}

\noindent Below are the histories and extended histories for space 67: 
\begin{center}
    \begin{tabular}{cc}
    \includegraphics[height=3.5cm]{svg-inkscape/space-ABC-unique-tight-67-highlighted_svg-tex.pdf}
    &
    \includegraphics[height=3.5cm]{svg-inkscape/space-ABC-unique-tight-67-ext-highlighted_svg-tex.pdf}
    \\
    $\Theta_{67}$
    &
    $\Ext{\Theta_{67}}$
    \end{tabular}
\end{center}

\noindent The standard causaltope for Space 67 has dimension 36.
Below is a plot of the homogeneous linear system of causality and quasi-normalisation equations for the standard causaltope, put in reduced row echelon form:

\begin{center}
    \includegraphics[width=11cm]{svg-inkscape/space67-rref-eqs_svg-tex.pdf}
\end{center}

\noindent Rows correspond to the 27 independent linear equations.
Columns in the plot correspond to entries of empirical models, indexed as $i_A i_B i_C$ $o_A o_B o_C$.
Coefficients in the equations are color-coded as white=0, red=+1 and blue=-1.

Space 67 has closest refinements in equivalence classes 46 and 53; 
it is the join of its (closest) refinements.
It has closest coarsenings in equivalence classes 81, 83 and 84; 
it is the meet of its (closest) coarsenings.
It has 1024 causal functions, 256 of which are not causal for any of its refinements.
It is a tight space.

The standard causaltope for Space 67 has 2 more dimensions than those of its 6 subspaces in equivalence classes 46 and 53.
The standard causaltope for Space 67 is the meet of the standard causaltopes for its closest coarsenings.
For completeness, below is a plot of the full homogeneous linear system of causality and quasi-normalisation equations for the standard causaltope:

\begin{center}
    \includegraphics[width=12cm]{svg-inkscape/space67-eqs_svg-tex.pdf}
\end{center}

\noindent Rows correspond to the 55 linear equations, of which 27 are independent.


\newpage
\subsection*{Space 68}

Space 68 is not induced by a causal order, but it is a refinement of the space 92 induced by the definite causal order $\total{\ev{A},\ev{C}}\vee\total{\ev{B},\ev{C}}$.
Its equivalence class under event-input permutation symmetry contains 12 spaces.
Space 68 differs as follows from the space induced by causal order $\total{\ev{A},\ev{C}}\vee\total{\ev{B},\ev{C}}$:
\begin{itemize}
  \item The outputs at events \evset{\ev{B}, \ev{C}} are independent of the input at event \ev{A} when the inputs at events \evset{B, C} are given by \hist{B/1,C/1} and \hist{B/0,C/1}.
\end{itemize}

\noindent Below are the histories and extended histories for space 68: 
\begin{center}
    \begin{tabular}{cc}
    \includegraphics[height=3.5cm]{svg-inkscape/space-ABC-unique-tight-68-highlighted_svg-tex.pdf}
    &
    \includegraphics[height=3.5cm]{svg-inkscape/space-ABC-unique-tight-68-ext-highlighted_svg-tex.pdf}
    \\
    $\Theta_{68}$
    &
    $\Ext{\Theta_{68}}$
    \end{tabular}
\end{center}

\noindent The standard causaltope for Space 68 has dimension 36.
Below is a plot of the homogeneous linear system of causality and quasi-normalisation equations for the standard causaltope, put in reduced row echelon form:

\begin{center}
    \includegraphics[width=11cm]{svg-inkscape/space68-rref-eqs_svg-tex.pdf}
\end{center}

\noindent Rows correspond to the 27 independent linear equations.
Columns in the plot correspond to entries of empirical models, indexed as $i_A i_B i_C$ $o_A o_B o_C$.
Coefficients in the equations are color-coded as white=0, red=+1 and blue=-1.

Space 68 has closest refinements in equivalence classes 46, 49 and 59; 
it is the join of its (closest) refinements.
It has closest coarsenings in equivalence classes 84 and 86; 
it is the meet of its (closest) coarsenings.
It has 1024 causal functions, 128 of which are not causal for any of its refinements.
It is a tight space.

The standard causaltope for Space 68 has 2 more dimensions than those of its 6 subspaces in equivalence classes 46, 49 and 59.
The standard causaltope for Space 68 is the meet of the standard causaltopes for its closest coarsenings.
For completeness, below is a plot of the full homogeneous linear system of causality and quasi-normalisation equations for the standard causaltope:

\begin{center}
    \includegraphics[width=12cm]{svg-inkscape/space68-eqs_svg-tex.pdf}
\end{center}

\noindent Rows correspond to the 55 linear equations, of which 27 are independent.


\newpage
\subsection*{Space 69}

Space 69 is not induced by a causal order, but it is a refinement of the space 92 induced by the definite causal order $\total{\ev{A},\ev{C}}\vee\total{\ev{B},\ev{C}}$.
Its equivalence class under event-input permutation symmetry contains 24 spaces.
Space 69 differs as follows from the space induced by causal order $\total{\ev{A},\ev{C}}\vee\total{\ev{B},\ev{C}}$:
\begin{itemize}
  \item The outputs at events \evset{\ev{B}, \ev{C}} are independent of the input at event \ev{A} when the inputs at events \evset{B, C} are given by \hist{B/0,C/1}.
  \item The outputs at events \evset{\ev{A}, \ev{C}} are independent of the input at event \ev{B} when the inputs at events \evset{A, C} are given by \hist{A/1,C/1}.
\end{itemize}

\noindent Below are the histories and extended histories for space 69: 
\begin{center}
    \begin{tabular}{cc}
    \includegraphics[height=3.5cm]{svg-inkscape/space-ABC-unique-untight-69-highlighted_svg-tex.pdf}
    &
    \includegraphics[height=3.5cm]{svg-inkscape/space-ABC-unique-untight-69-ext-highlighted_svg-tex.pdf}
    \\
    $\Theta_{69}$
    &
    $\Ext{\Theta_{69}}$
    \end{tabular}
\end{center}

\noindent The standard causaltope for Space 69 has dimension 36.
Below is a plot of the homogeneous linear system of causality and quasi-normalisation equations for the standard causaltope, put in reduced row echelon form:

\begin{center}
    \includegraphics[width=11cm]{svg-inkscape/space69-rref-eqs_svg-tex.pdf}
\end{center}

\noindent Rows correspond to the 27 independent linear equations.
Columns in the plot correspond to entries of empirical models, indexed as $i_A i_B i_C$ $o_A o_B o_C$.
Coefficients in the equations are color-coded as white=0, red=+1 and blue=-1.

Space 69 has closest refinements in equivalence classes 50, 53 and 59; 
it is the join of its (closest) refinements.
It has closest coarsenings in equivalence classes 82 and 84; 
it is the meet of its (closest) coarsenings.
It has 1024 causal functions, 256 of which are not causal for any of its refinements.
It is not a tight space: for event \ev{C}, a causal function must yield identical output values on input histories \hist{A/1,C/1} and \hist{B/0,C/1}.

The standard causaltope for Space 69 has 2 more dimensions than those of its 6 subspaces in equivalence classes 50, 53 and 59.
The standard causaltope for Space 69 is the meet of the standard causaltopes for its closest coarsenings.
For completeness, below is a plot of the full homogeneous linear system of causality and quasi-normalisation equations for the standard causaltope:

\begin{center}
    \includegraphics[width=12cm]{svg-inkscape/space69-eqs_svg-tex.pdf}
\end{center}

\noindent Rows correspond to the 55 linear equations, of which 27 are independent.


\newpage
\subsection*{Space 70}

Space 70 is not induced by a causal order, but it is a refinement of the space 92 induced by the definite causal order $\total{\ev{A},\ev{C}}\vee\total{\ev{B},\ev{C}}$.
Its equivalence class under event-input permutation symmetry contains 12 spaces.
Space 70 differs as follows from the space induced by causal order $\total{\ev{A},\ev{C}}\vee\total{\ev{B},\ev{C}}$:
\begin{itemize}
  \item The outputs at events \evset{\ev{B}, \ev{C}} are independent of the input at event \ev{A} when the inputs at events \evset{B, C} are given by \hist{B/1,C/0} and \hist{B/0,C/1}.
\end{itemize}

\noindent Below are the histories and extended histories for space 70: 
\begin{center}
    \begin{tabular}{cc}
    \includegraphics[height=3.5cm]{svg-inkscape/space-ABC-unique-tight-70-highlighted_svg-tex.pdf}
    &
    \includegraphics[height=3.5cm]{svg-inkscape/space-ABC-unique-tight-70-ext-highlighted_svg-tex.pdf}
    \\
    $\Theta_{70}$
    &
    $\Ext{\Theta_{70}}$
    \end{tabular}
\end{center}

\noindent The standard causaltope for Space 70 has dimension 36.
Below is a plot of the homogeneous linear system of causality and quasi-normalisation equations for the standard causaltope, put in reduced row echelon form:

\begin{center}
    \includegraphics[width=11cm]{svg-inkscape/space70-rref-eqs_svg-tex.pdf}
\end{center}

\noindent Rows correspond to the 27 independent linear equations.
Columns in the plot correspond to entries of empirical models, indexed as $i_A i_B i_C$ $o_A o_B o_C$.
Coefficients in the equations are color-coded as white=0, red=+1 and blue=-1.

Space 70 has closest refinements in equivalence classes 46 and 50; 
it is the join of its (closest) refinements.
It has closest coarsenings in equivalence classes 84 and 87; 
it is the meet of its (closest) coarsenings.
It has 1024 causal functions, 128 of which are not causal for any of its refinements.
It is a tight space.

The standard causaltope for Space 70 has 2 more dimensions than those of its 6 subspaces in equivalence classes 46 and 50.
The standard causaltope for Space 70 is the meet of the standard causaltopes for its closest coarsenings.
For completeness, below is a plot of the full homogeneous linear system of causality and quasi-normalisation equations for the standard causaltope:

\begin{center}
    \includegraphics[width=12cm]{svg-inkscape/space70-eqs_svg-tex.pdf}
\end{center}

\noindent Rows correspond to the 55 linear equations, of which 27 are independent.


\newpage
\subsection*{Space 71}

Space 71 is not induced by a causal order, but it is a refinement of the space induced by the indefinite causal order $\total{\ev{A},\{\ev{B},\ev{C}\}}$.
Its equivalence class under event-input permutation symmetry contains 48 spaces.
Space 71 differs as follows from the space induced by causal order $\total{\ev{A},\{\ev{B},\ev{C}\}}$:
\begin{itemize}
  \item The outputs at events \evset{\ev{A}, \ev{B}} are independent of the input at event \ev{C} when the inputs at events \evset{A, B} are given by \hist{A/0,B/0}, \hist{A/0,B/1} and \hist{A/1,B/0}.
  \item The outputs at events \evset{\ev{A}, \ev{C}} are independent of the input at event \ev{B} when the inputs at events \evset{A, C} are given by \hist{A/0,C/1}, \hist{A/1,C/0} and \hist{A/1,C/1}.
  \item The output at event \ev{C} is independent of the inputs at events \evset{\ev{A}, \ev{B}} when the input at event C is given by \hist{C/1}.
  \item The outputs at events \evset{\ev{B}, \ev{C}} are independent of the input at event \ev{A} when the inputs at events \evset{B, C} are given by \hist{B/0,C/1}.
\end{itemize}

\noindent Below are the histories and extended histories for space 71: 
\begin{center}
    \begin{tabular}{cc}
    \includegraphics[height=3.5cm]{svg-inkscape/space-ABC-unique-untight-71-highlighted_svg-tex.pdf}
    &
    \includegraphics[height=3.5cm]{svg-inkscape/space-ABC-unique-untight-71-ext-highlighted_svg-tex.pdf}
    \\
    $\Theta_{71}$
    &
    $\Ext{\Theta_{71}}$
    \end{tabular}
\end{center}

\noindent The standard causaltope for Space 71 has dimension 35.
Below is a plot of the homogeneous linear system of causality and quasi-normalisation equations for the standard causaltope, put in reduced row echelon form:

\begin{center}
    \includegraphics[width=11cm]{svg-inkscape/space71-rref-eqs_svg-tex.pdf}
\end{center}

\noindent Rows correspond to the 28 independent linear equations.
Columns in the plot correspond to entries of empirical models, indexed as $i_A i_B i_C$ $o_A o_B o_C$.
Coefficients in the equations are color-coded as white=0, red=+1 and blue=-1.

Space 71 has closest refinements in equivalence classes 47, 48, 52 and 57; 
it is the join of its (closest) refinements.
It has closest coarsenings in equivalence classes 79 and 85; 
it is the meet of its (closest) coarsenings.
It has 1024 causal functions, all of which are causal for at least one of its refinements.
It is not a tight space: for event \ev{B}, a causal function must yield identical output values on input histories \hist{A/0,B/0}, \hist{A/1,B/0} and \hist{B/0,C/1}.

The standard causaltope for Space 71 coincides with that of its subspace in equivalence class 48.
The standard causaltope for Space 71 is the meet of the standard causaltopes for its closest coarsenings.
For completeness, below is a plot of the full homogeneous linear system of causality and quasi-normalisation equations for the standard causaltope:

\begin{center}
    \includegraphics[width=12cm]{svg-inkscape/space71-eqs_svg-tex.pdf}
\end{center}

\noindent Rows correspond to the 53 linear equations, of which 28 are independent.


\newpage
\subsection*{Space 72}

Space 72 is not induced by a causal order, but it is a refinement of the space induced by the indefinite causal order $\total{\ev{A},\{\ev{B},\ev{C}\}}$.
Its equivalence class under event-input permutation symmetry contains 24 spaces.
Space 72 differs as follows from the space induced by causal order $\total{\ev{A},\{\ev{B},\ev{C}\}}$:
\begin{itemize}
  \item The outputs at events \evset{\ev{A}, \ev{B}} are independent of the input at event \ev{C} when the inputs at events \evset{A, B} are given by \hist{A/0,B/0}, \hist{A/0,B/1} and \hist{A/1,B/1}.
  \item The outputs at events \evset{\ev{B}, \ev{C}} are independent of the input at event \ev{A} when the inputs at events \evset{B, C} are given by \hist{B/1,C/0} and \hist{B/1,C/1}.
  \item The outputs at events \evset{\ev{A}, \ev{C}} are independent of the input at event \ev{B} when the inputs at events \evset{A, C} are given by \hist{A/1,C/0} and \hist{A/1,C/1}.
  \item The output at event \ev{B} is independent of the inputs at events \evset{\ev{A}, \ev{C}} when the input at event B is given by \hist{B/1}.
\end{itemize}

\noindent Below are the histories and extended histories for space 72: 
\begin{center}
    \begin{tabular}{cc}
    \includegraphics[height=3.5cm]{svg-inkscape/space-ABC-unique-untight-72-highlighted_svg-tex.pdf}
    &
    \includegraphics[height=3.5cm]{svg-inkscape/space-ABC-unique-untight-72-ext-highlighted_svg-tex.pdf}
    \\
    $\Theta_{72}$
    &
    $\Ext{\Theta_{72}}$
    \end{tabular}
\end{center}

\noindent The standard causaltope for Space 72 has dimension 35.
Below is a plot of the homogeneous linear system of causality and quasi-normalisation equations for the standard causaltope, put in reduced row echelon form:

\begin{center}
    \includegraphics[width=11cm]{svg-inkscape/space72-rref-eqs_svg-tex.pdf}
\end{center}

\noindent Rows correspond to the 28 independent linear equations.
Columns in the plot correspond to entries of empirical models, indexed as $i_A i_B i_C$ $o_A o_B o_C$.
Coefficients in the equations are color-coded as white=0, red=+1 and blue=-1.

Space 72 has closest refinements in equivalence classes 52 and 60; 
it is the join of its (closest) refinements.
It has closest coarsenings in equivalence class 85; 
it is the meet of its (closest) coarsenings.
It has 1024 causal functions, 384 of which are not causal for any of its refinements.
It is not a tight space: for event \ev{C}, a causal function must yield identical output values on input histories \hist{A/1,C/0} and \hist{B/1,C/0}, and it must also yield identical output values on input histories \hist{A/1,C/1} and \hist{B/1,C/1}.

The standard causaltope for Space 72 has 2 more dimensions than those of its 3 subspaces in equivalence classes 52 and 60.
The standard causaltope for Space 72 is the meet of the standard causaltopes for its closest coarsenings.
For completeness, below is a plot of the full homogeneous linear system of causality and quasi-normalisation equations for the standard causaltope:

\begin{center}
    \includegraphics[width=12cm]{svg-inkscape/space72-eqs_svg-tex.pdf}
\end{center}

\noindent Rows correspond to the 53 linear equations, of which 28 are independent.


\newpage
\subsection*{Space 73}

Space 73 is not induced by a causal order, but it is a refinement of the space 100 induced by the definite causal order $\total{\ev{A},\ev{B},\ev{C}}$.
Its equivalence class under event-input permutation symmetry contains 48 spaces.
Space 73 differs as follows from the space induced by causal order $\total{\ev{A},\ev{B},\ev{C}}$:
\begin{itemize}
  \item The outputs at events \evset{\ev{A}, \ev{C}} are independent of the input at event \ev{B} when the inputs at events \evset{A, C} are given by \hist{A/0,C/1} and \hist{A/1,C/0}.
  \item The outputs at events \evset{\ev{B}, \ev{C}} are independent of the input at event \ev{A} when the inputs at events \evset{B, C} are given by \hist{B/1,C/1}.
  \item The output at event \ev{B} is independent of the input at event \ev{A} when the input at event B is given by \hist{B/1}.
\end{itemize}

\noindent Below are the histories and extended histories for space 73: 
\begin{center}
    \begin{tabular}{cc}
    \includegraphics[height=3.5cm]{svg-inkscape/space-ABC-unique-untight-73-highlighted_svg-tex.pdf}
    &
    \includegraphics[height=3.5cm]{svg-inkscape/space-ABC-unique-untight-73-ext-highlighted_svg-tex.pdf}
    \\
    $\Theta_{73}$
    &
    $\Ext{\Theta_{73}}$
    \end{tabular}
\end{center}

\noindent The standard causaltope for Space 73 has dimension 35.
Below is a plot of the homogeneous linear system of causality and quasi-normalisation equations for the standard causaltope, put in reduced row echelon form:

\begin{center}
    \includegraphics[width=11cm]{svg-inkscape/space73-rref-eqs_svg-tex.pdf}
\end{center}

\noindent Rows correspond to the 28 independent linear equations.
Columns in the plot correspond to entries of empirical models, indexed as $i_A i_B i_C$ $o_A o_B o_C$.
Coefficients in the equations are color-coded as white=0, red=+1 and blue=-1.

Space 73 has closest refinements in equivalence classes 50, 55, 56 and 57; 
it is the join of its (closest) refinements.
It has closest coarsenings in equivalence classes 80, 82 and 87; 
it is the meet of its (closest) coarsenings.
It has 1024 causal functions, 192 of which are not causal for any of its refinements.
It is not a tight space: for event \ev{C}, a causal function must yield identical output values on input histories \hist{A/0,C/1} and \hist{B/1,C/1}.

The standard causaltope for Space 73 has 1 more dimension than that of its subspace in equivalence class 50.
The standard causaltope for Space 73 is the meet of the standard causaltopes for its closest coarsenings.
For completeness, below is a plot of the full homogeneous linear system of causality and quasi-normalisation equations for the standard causaltope:

\begin{center}
    \includegraphics[width=12cm]{svg-inkscape/space73-eqs_svg-tex.pdf}
\end{center}

\noindent Rows correspond to the 53 linear equations, of which 28 are independent.


\newpage
\subsection*{Space 74}

Space 74 is not induced by a causal order, but it is a refinement of the space in equivalence class 100 induced by the definite causal order $\total{\ev{A},\ev{C},\ev{B}}$ (note that the space induced by the order is not the same as space 100).
Its equivalence class under event-input permutation symmetry contains 48 spaces.
Space 74 differs as follows from the space induced by causal order $\total{\ev{A},\ev{C},\ev{B}}$:
\begin{itemize}
  \item The outputs at events \evset{\ev{A}, \ev{B}} are independent of the input at event \ev{C} when the inputs at events \evset{A, B} are given by \hist{A/0,B/0}, \hist{A/0,B/1} and \hist{A/1,B/0}.
  \item The output at event \ev{C} is independent of the input at event \ev{A} when the input at event C is given by \hist{C/0}.
\end{itemize}

\noindent Below are the histories and extended histories for space 74: 
\begin{center}
    \begin{tabular}{cc}
    \includegraphics[height=3.5cm]{svg-inkscape/space-ABC-unique-tight-74-highlighted_svg-tex.pdf}
    &
    \includegraphics[height=3.5cm]{svg-inkscape/space-ABC-unique-tight-74-ext-highlighted_svg-tex.pdf}
    \\
    $\Theta_{74}$
    &
    $\Ext{\Theta_{74}}$
    \end{tabular}
\end{center}

\noindent The standard causaltope for Space 74 has dimension 35.
Below is a plot of the homogeneous linear system of causality and quasi-normalisation equations for the standard causaltope, put in reduced row echelon form:

\begin{center}
    \includegraphics[width=11cm]{svg-inkscape/space74-rref-eqs_svg-tex.pdf}
\end{center}

\noindent Rows correspond to the 28 independent linear equations.
Columns in the plot correspond to entries of empirical models, indexed as $i_A i_B i_C$ $o_A o_B o_C$.
Coefficients in the equations are color-coded as white=0, red=+1 and blue=-1.

Space 74 has closest refinements in equivalence classes 46, 56, 57 and 58; 
it is the join of its (closest) refinements.
It has closest coarsenings in equivalence classes 79, 83, 86, 87 and 88; 
it is the meet of its (closest) coarsenings.
It has 1024 causal functions, 128 of which are not causal for any of its refinements.
It is a tight space.

The standard causaltope for Space 74 has 1 more dimension than that of its subspace in equivalence class 46.
The standard causaltope for Space 74 is the meet of the standard causaltopes for its closest coarsenings.
For completeness, below is a plot of the full homogeneous linear system of causality and quasi-normalisation equations for the standard causaltope:

\begin{center}
    \includegraphics[width=12cm]{svg-inkscape/space74-eqs_svg-tex.pdf}
\end{center}

\noindent Rows correspond to the 53 linear equations, of which 28 are independent.


\newpage
\subsection*{Space 75}

Space 75 is not induced by a causal order, but it is a refinement of the space 100 induced by the definite causal order $\total{\ev{A},\ev{B},\ev{C}}$.
Its equivalence class under event-input permutation symmetry contains 24 spaces.
Space 75 differs as follows from the space induced by causal order $\total{\ev{A},\ev{B},\ev{C}}$:
\begin{itemize}
  \item The outputs at events \evset{\ev{A}, \ev{C}} are independent of the input at event \ev{B} when the inputs at events \evset{A, C} are given by \hist{A/0,C/1} and \hist{A/1,C/1}.
  \item The outputs at events \evset{\ev{B}, \ev{C}} are independent of the input at event \ev{A} when the inputs at events \evset{B, C} are given by \hist{B/1,C/0}.
  \item The output at event \ev{B} is independent of the input at event \ev{A} when the input at event B is given by \hist{B/1}.
\end{itemize}

\noindent Below are the histories and extended histories for space 75: 
\begin{center}
    \begin{tabular}{cc}
    \includegraphics[height=3.5cm]{svg-inkscape/space-ABC-unique-tight-75-highlighted_svg-tex.pdf}
    &
    \includegraphics[height=3.5cm]{svg-inkscape/space-ABC-unique-tight-75-ext-highlighted_svg-tex.pdf}
    \\
    $\Theta_{75}$
    &
    $\Ext{\Theta_{75}}$
    \end{tabular}
\end{center}

\noindent The standard causaltope for Space 75 has dimension 35.
Below is a plot of the homogeneous linear system of causality and quasi-normalisation equations for the standard causaltope, put in reduced row echelon form:

\begin{center}
    \includegraphics[width=11cm]{svg-inkscape/space75-rref-eqs_svg-tex.pdf}
\end{center}

\noindent Rows correspond to the 28 independent linear equations.
Columns in the plot correspond to entries of empirical models, indexed as $i_A i_B i_C$ $o_A o_B o_C$.
Coefficients in the equations are color-coded as white=0, red=+1 and blue=-1.

Space 75 has closest refinements in equivalence classes 49, 54 and 56; 
it is the join of its (closest) refinements.
It has closest coarsenings in equivalence classes 80 and 86; 
it is the meet of its (closest) coarsenings.
It has 1024 causal functions, 320 of which are not causal for any of its refinements.
It is a tight space.

The standard causaltope for Space 75 has 1 more dimension than that of its subspace in equivalence class 49.
The standard causaltope for Space 75 is the meet of the standard causaltopes for its closest coarsenings.
For completeness, below is a plot of the full homogeneous linear system of causality and quasi-normalisation equations for the standard causaltope:

\begin{center}
    \includegraphics[width=12cm]{svg-inkscape/space75-eqs_svg-tex.pdf}
\end{center}

\noindent Rows correspond to the 53 linear equations, of which 28 are independent.


\newpage
\subsection*{Space 76}

Space 76 is not induced by a causal order, but it is a refinement of the space 100 induced by the definite causal order $\total{\ev{A},\ev{B},\ev{C}}$.
Its equivalence class under event-input permutation symmetry contains 24 spaces.
Space 76 differs as follows from the space induced by causal order $\total{\ev{A},\ev{B},\ev{C}}$:
\begin{itemize}
  \item The outputs at events \evset{\ev{A}, \ev{C}} are independent of the input at event \ev{B} when the inputs at events \evset{A, C} are given by \hist{A/0,C/1} and \hist{A/1,C/1}.
  \item The outputs at events \evset{\ev{B}, \ev{C}} are independent of the input at event \ev{A} when the inputs at events \evset{B, C} are given by \hist{B/1,C/1}.
  \item The output at event \ev{B} is independent of the input at event \ev{A} when the input at event B is given by \hist{B/1}.
\end{itemize}

\noindent Below are the histories and extended histories for space 76: 
\begin{center}
    \begin{tabular}{cc}
    \includegraphics[height=3.5cm]{svg-inkscape/space-ABC-unique-untight-76-highlighted_svg-tex.pdf}
    &
    \includegraphics[height=3.5cm]{svg-inkscape/space-ABC-unique-untight-76-ext-highlighted_svg-tex.pdf}
    \\
    $\Theta_{76}$
    &
    $\Ext{\Theta_{76}}$
    \end{tabular}
\end{center}

\noindent The standard causaltope for Space 76 has dimension 35.
Below is a plot of the homogeneous linear system of causality and quasi-normalisation equations for the standard causaltope, put in reduced row echelon form:

\begin{center}
    \includegraphics[width=11cm]{svg-inkscape/space76-rref-eqs_svg-tex.pdf}
\end{center}

\noindent Rows correspond to the 28 independent linear equations.
Columns in the plot correspond to entries of empirical models, indexed as $i_A i_B i_C$ $o_A o_B o_C$.
Coefficients in the equations are color-coded as white=0, red=+1 and blue=-1.

Space 76 has closest refinements in equivalence classes 45, 54, 57 and 59; 
it is the join of its (closest) refinements.
It has closest coarsenings in equivalence classes 82 and 86; 
it is the meet of its (closest) coarsenings.
It has 1024 causal functions, all of which are causal for at least one of its refinements.
It is not a tight space: for event \ev{C}, a causal function must yield identical output values on input histories \hist{A/0,C/1}, \hist{A/1,C/1} and \hist{B/1,C/1}.

The standard causaltope for Space 76 coincides with that of its subspace in equivalence class 45.
The standard causaltope for Space 76 is the meet of the standard causaltopes for its closest coarsenings.
For completeness, below is a plot of the full homogeneous linear system of causality and quasi-normalisation equations for the standard causaltope:

\begin{center}
    \includegraphics[width=12cm]{svg-inkscape/space76-eqs_svg-tex.pdf}
\end{center}

\noindent Rows correspond to the 53 linear equations, of which 28 are independent.


\newpage
\subsection*{Space 77}

Space 77 is induced by the definite causal order $\total{\ev{A},\ev{B}}\vee\total{\ev{A},\ev{C}}$.
Its equivalence class under event-input permutation symmetry contains 3 spaces.

\noindent Below are the histories and extended histories for space 77: 
\begin{center}
    \begin{tabular}{cc}
    \includegraphics[height=3.5cm]{svg-inkscape/space-ABC-unique-tight-77-highlighted_svg-tex.pdf}
    &
    \includegraphics[height=3.5cm]{svg-inkscape/space-ABC-unique-tight-77-ext-highlighted_svg-tex.pdf}
    \\
    $\Theta_{77}$
    &
    $\Ext{\Theta_{77}}$
    \end{tabular}
\end{center}

\noindent The standard causaltope for Space 77 has dimension 34.
Below is a plot of the homogeneous linear system of causality and quasi-normalisation equations for the standard causaltope, put in reduced row echelon form:

\begin{center}
    \includegraphics[width=11cm]{svg-inkscape/space77-rref-eqs_svg-tex.pdf}
\end{center}

\noindent Rows correspond to the 29 independent linear equations.
Columns in the plot correspond to entries of empirical models, indexed as $i_A i_B i_C$ $o_A o_B o_C$.
Coefficients in the equations are color-coded as white=0, red=+1 and blue=-1.

Space 77 has closest refinements in equivalence class 58; 
it is the join of its (closest) refinements.
It has closest coarsenings in equivalence class 88; 
it is the meet of its (closest) coarsenings.
It has 1024 causal functions, 320 of which are not causal for any of its refinements.
It is a tight space.

The standard causaltope for Space 77 has 1 more dimension than those of its 4 subspaces in equivalence class 58.
The standard causaltope for Space 77 is the meet of the standard causaltopes for its closest coarsenings.
For completeness, below is a plot of the full homogeneous linear system of causality and quasi-normalisation equations for the standard causaltope:

\begin{center}
    \includegraphics[width=12cm]{svg-inkscape/space77-eqs_svg-tex.pdf}
\end{center}

\noindent Rows correspond to the 51 linear equations, of which 29 are independent.


\newpage
\subsection*{Space 78}

Space 78 is not induced by a causal order, but it is a refinement of the space 100 induced by the definite causal order $\total{\ev{A},\ev{B},\ev{C}}$.
Its equivalence class under event-input permutation symmetry contains 12 spaces.
Space 78 differs as follows from the space induced by causal order $\total{\ev{A},\ev{B},\ev{C}}$:
\begin{itemize}
  \item The outputs at events \evset{\ev{A}, \ev{C}} are independent of the input at event \ev{B} when the inputs at events \evset{A, C} are given by \hist{A/0,C/1} and \hist{A/1,C/1}.
  \item The output at event \ev{C} is independent of the inputs at events \evset{\ev{A}, \ev{B}} when the input at event C is given by \hist{C/1}.
\end{itemize}

\noindent Below are the histories and extended histories for space 78: 
\begin{center}
    \begin{tabular}{cc}
    \includegraphics[height=3.5cm]{svg-inkscape/space-ABC-unique-tight-78-highlighted_svg-tex.pdf}
    &
    \includegraphics[height=3.5cm]{svg-inkscape/space-ABC-unique-tight-78-ext-highlighted_svg-tex.pdf}
    \\
    $\Theta_{78}$
    &
    $\Ext{\Theta_{78}}$
    \end{tabular}
\end{center}

\noindent The standard causaltope for Space 78 has dimension 37.
Below is a plot of the homogeneous linear system of causality and quasi-normalisation equations for the standard causaltope, put in reduced row echelon form:

\begin{center}
    \includegraphics[width=11cm]{svg-inkscape/space78-rref-eqs_svg-tex.pdf}
\end{center}

\noindent Rows correspond to the 26 independent linear equations.
Columns in the plot correspond to entries of empirical models, indexed as $i_A i_B i_C$ $o_A o_B o_C$.
Coefficients in the equations are color-coded as white=0, red=+1 and blue=-1.

Space 78 has closest refinements in equivalence classes 61 and 64; 
it is the join of its (closest) refinements.
It has closest coarsenings in equivalence class 94; 
it does not arise as a nontrivial meet in the hierarchy.
It has 2048 causal functions, 640 of which are not causal for any of its refinements.
It is a tight space.

The standard causaltope for Space 78 has 2 more dimensions than those of its 4 subspaces in equivalence classes 61 and 64.
The standard causaltope for Space 78 has 1 dimension fewer than the meet of the standard causaltopes for its closest coarsenings.
For completeness, below is a plot of the full homogeneous linear system of causality and quasi-normalisation equations for the standard causaltope:

\begin{center}
    \includegraphics[width=12cm]{svg-inkscape/space78-eqs_svg-tex.pdf}
\end{center}

\noindent Rows correspond to the 49 linear equations, of which 26 are independent.


\newpage
\subsection*{Space 79}

Space 79 is not induced by a causal order, but it is a refinement of the space induced by the indefinite causal order $\total{\ev{A},\{\ev{B},\ev{C}\}}$.
Its equivalence class under event-input permutation symmetry contains 48 spaces.
Space 79 differs as follows from the space induced by causal order $\total{\ev{A},\{\ev{B},\ev{C}\}}$:
\begin{itemize}
  \item The outputs at events \evset{\ev{A}, \ev{B}} are independent of the input at event \ev{C} when the inputs at events \evset{A, B} are given by \hist{A/0,B/0}, \hist{A/0,B/1} and \hist{A/1,B/0}.
  \item The outputs at events \evset{\ev{A}, \ev{C}} are independent of the input at event \ev{B} when the inputs at events \evset{A, C} are given by \hist{A/0,C/1}, \hist{A/1,C/0} and \hist{A/1,C/1}.
  \item The output at event \ev{C} is independent of the inputs at events \evset{\ev{A}, \ev{B}} when the input at event C is given by \hist{C/1}.
\end{itemize}

\noindent Below are the histories and extended histories for space 79: 
\begin{center}
    \begin{tabular}{cc}
    \includegraphics[height=3.5cm]{svg-inkscape/space-ABC-unique-tight-79-highlighted_svg-tex.pdf}
    &
    \includegraphics[height=3.5cm]{svg-inkscape/space-ABC-unique-tight-79-ext-highlighted_svg-tex.pdf}
    \\
    $\Theta_{79}$
    &
    $\Ext{\Theta_{79}}$
    \end{tabular}
\end{center}

\noindent The standard causaltope for Space 79 has dimension 37.
Below is a plot of the homogeneous linear system of causality and quasi-normalisation equations for the standard causaltope, put in reduced row echelon form:

\begin{center}
    \includegraphics[width=11cm]{svg-inkscape/space79-rref-eqs_svg-tex.pdf}
\end{center}

\noindent Rows correspond to the 26 independent linear equations.
Columns in the plot correspond to entries of empirical models, indexed as $i_A i_B i_C$ $o_A o_B o_C$.
Coefficients in the equations are color-coded as white=0, red=+1 and blue=-1.

Space 79 has closest refinements in equivalence classes 62, 64, 71 and 74; 
it is the join of its (closest) refinements.
It has closest coarsenings in equivalence classes 93 and 96; 
it is the meet of its (closest) coarsenings.
It has 2048 causal functions, 192 of which are not causal for any of its refinements.
It is a tight space.

The standard causaltope for Space 79 has 2 more dimensions than those of its 4 subspaces in equivalence classes 62, 64, 71 and 74.
The standard causaltope for Space 79 is the meet of the standard causaltopes for its closest coarsenings.
For completeness, below is a plot of the full homogeneous linear system of causality and quasi-normalisation equations for the standard causaltope:

\begin{center}
    \includegraphics[width=12cm]{svg-inkscape/space79-eqs_svg-tex.pdf}
\end{center}

\noindent Rows correspond to the 49 linear equations, of which 26 are independent.


\newpage
\subsection*{Space 80}

Space 80 is not induced by a causal order, but it is a refinement of the space 100 induced by the definite causal order $\total{\ev{A},\ev{B},\ev{C}}$.
Its equivalence class under event-input permutation symmetry contains 48 spaces.
Space 80 differs as follows from the space induced by causal order $\total{\ev{A},\ev{B},\ev{C}}$:
\begin{itemize}
  \item The outputs at events \evset{\ev{B}, \ev{C}} are independent of the input at event \ev{A} when the inputs at events \evset{B, C} are given by \hist{B/1,C/1}.
  \item The outputs at events \evset{\ev{A}, \ev{C}} are independent of the input at event \ev{B} when the inputs at events \evset{A, C} are given by \hist{A/1,C/0}.
  \item The output at event \ev{B} is independent of the input at event \ev{A} when the input at event B is given by \hist{B/1}.
\end{itemize}

\noindent Below are the histories and extended histories for space 80: 
\begin{center}
    \begin{tabular}{cc}
    \includegraphics[height=3.5cm]{svg-inkscape/space-ABC-unique-tight-80-highlighted_svg-tex.pdf}
    &
    \includegraphics[height=3.5cm]{svg-inkscape/space-ABC-unique-tight-80-ext-highlighted_svg-tex.pdf}
    \\
    $\Theta_{80}$
    &
    $\Ext{\Theta_{80}}$
    \end{tabular}
\end{center}

\noindent The standard causaltope for Space 80 has dimension 37.
Below is a plot of the homogeneous linear system of causality and quasi-normalisation equations for the standard causaltope, put in reduced row echelon form:

\begin{center}
    \includegraphics[width=11cm]{svg-inkscape/space80-rref-eqs_svg-tex.pdf}
\end{center}

\noindent Rows correspond to the 26 independent linear equations.
Columns in the plot correspond to entries of empirical models, indexed as $i_A i_B i_C$ $o_A o_B o_C$.
Coefficients in the equations are color-coded as white=0, red=+1 and blue=-1.

Space 80 has closest refinements in equivalence classes 63, 65, 66, 73 and 75; 
it is the join of its (closest) refinements.
It has closest coarsenings in equivalence classes 89 and 90; 
it is the meet of its (closest) coarsenings.
It has 2048 causal functions, 192 of which are not causal for any of its refinements.
It is a tight space.

The standard causaltope for Space 80 has 1 more dimension than that of its subspace in equivalence class 66.
The standard causaltope for Space 80 is the meet of the standard causaltopes for its closest coarsenings.
For completeness, below is a plot of the full homogeneous linear system of causality and quasi-normalisation equations for the standard causaltope:

\begin{center}
    \includegraphics[width=12cm]{svg-inkscape/space80-eqs_svg-tex.pdf}
\end{center}

\noindent Rows correspond to the 49 linear equations, of which 26 are independent.


\newpage
\subsection*{Space 81}

Space 81 is not induced by a causal order, but it is a refinement of the space 100 induced by the definite causal order $\total{\ev{A},\ev{B},\ev{C}}$.
Its equivalence class under event-input permutation symmetry contains 12 spaces.
Space 81 differs as follows from the space induced by causal order $\total{\ev{A},\ev{B},\ev{C}}$:
\begin{itemize}
  \item The outputs at events \evset{\ev{B}, \ev{C}} are independent of the input at event \ev{A} when the inputs at events \evset{B, C} are given by \hist{B/1,C/0} and \hist{B/1,C/1}.
  \item The output at event \ev{B} is independent of the input at event \ev{A} when the input at event B is given by \hist{B/1}.
\end{itemize}

\noindent Below are the histories and extended histories for space 81: 
\begin{center}
    \begin{tabular}{cc}
    \includegraphics[height=3.5cm]{svg-inkscape/space-ABC-unique-tight-81-highlighted_svg-tex.pdf}
    &
    \includegraphics[height=3.5cm]{svg-inkscape/space-ABC-unique-tight-81-ext-highlighted_svg-tex.pdf}
    \\
    $\Theta_{81}$
    &
    $\Ext{\Theta_{81}}$
    \end{tabular}
\end{center}

\noindent The standard causaltope for Space 81 has dimension 37.
Below is a plot of the homogeneous linear system of causality and quasi-normalisation equations for the standard causaltope, put in reduced row echelon form:

\begin{center}
    \includegraphics[width=11cm]{svg-inkscape/space81-rref-eqs_svg-tex.pdf}
\end{center}

\noindent Rows correspond to the 26 independent linear equations.
Columns in the plot correspond to entries of empirical models, indexed as $i_A i_B i_C$ $o_A o_B o_C$.
Coefficients in the equations are color-coded as white=0, red=+1 and blue=-1.

Space 81 has closest refinements in equivalence classes 65 and 67; 
it is the join of its (closest) refinements.
It has closest coarsenings in equivalence class 89; 
it is the meet of its (closest) coarsenings.
It has 2048 causal functions, 256 of which are not causal for any of its refinements.
It is a tight space.

The standard causaltope for Space 81 has 1 more dimension than that of its subspace in equivalence class 67.
The standard causaltope for Space 81 is the meet of the standard causaltopes for its closest coarsenings.
For completeness, below is a plot of the full homogeneous linear system of causality and quasi-normalisation equations for the standard causaltope:

\begin{center}
    \includegraphics[width=12cm]{svg-inkscape/space81-eqs_svg-tex.pdf}
\end{center}

\noindent Rows correspond to the 49 linear equations, of which 26 are independent.


\newpage
\subsection*{Space 82}

Space 82 is not induced by a causal order, but it is a refinement of the space 100 induced by the definite causal order $\total{\ev{A},\ev{B},\ev{C}}$.
Its equivalence class under event-input permutation symmetry contains 48 spaces.
Space 82 differs as follows from the space induced by causal order $\total{\ev{A},\ev{B},\ev{C}}$:
\begin{itemize}
  \item The outputs at events \evset{\ev{B}, \ev{C}} are independent of the input at event \ev{A} when the inputs at events \evset{B, C} are given by \hist{B/1,C/1}.
  \item The output at event \ev{B} is independent of the input at event \ev{A} when the input at event B is given by \hist{B/1}.
  \item The outputs at events \evset{\ev{A}, \ev{C}} are independent of the input at event \ev{B} when the inputs at events \evset{A, C} are given by \hist{A/1,C/1}.
\end{itemize}

\noindent Below are the histories and extended histories for space 82: 
\begin{center}
    \begin{tabular}{cc}
    \includegraphics[height=3.5cm]{svg-inkscape/space-ABC-unique-untight-82-highlighted_svg-tex.pdf}
    &
    \includegraphics[height=3.5cm]{svg-inkscape/space-ABC-unique-untight-82-ext-highlighted_svg-tex.pdf}
    \\
    $\Theta_{82}$
    &
    $\Ext{\Theta_{82}}$
    \end{tabular}
\end{center}

\noindent The standard causaltope for Space 82 has dimension 37.
Below is a plot of the homogeneous linear system of causality and quasi-normalisation equations for the standard causaltope, put in reduced row echelon form:

\begin{center}
    \includegraphics[width=11cm]{svg-inkscape/space82-rref-eqs_svg-tex.pdf}
\end{center}

\noindent Rows correspond to the 26 independent linear equations.
Columns in the plot correspond to entries of empirical models, indexed as $i_A i_B i_C$ $o_A o_B o_C$.
Coefficients in the equations are color-coded as white=0, red=+1 and blue=-1.

Space 82 has closest refinements in equivalence classes 63, 65, 69, 73 and 76; 
it is the join of its (closest) refinements.
It has closest coarsenings in equivalence classes 89 and 90; 
it is the meet of its (closest) coarsenings.
It has 2048 causal functions, 256 of which are not causal for any of its refinements.
It is not a tight space: for event \ev{C}, a causal function must yield identical output values on input histories \hist{A/1,C/1} and \hist{B/1,C/1}.

The standard causaltope for Space 82 has 1 more dimension than that of its subspace in equivalence class 69.
The standard causaltope for Space 82 is the meet of the standard causaltopes for its closest coarsenings.
For completeness, below is a plot of the full homogeneous linear system of causality and quasi-normalisation equations for the standard causaltope:

\begin{center}
    \includegraphics[width=12cm]{svg-inkscape/space82-eqs_svg-tex.pdf}
\end{center}

\noindent Rows correspond to the 49 linear equations, of which 26 are independent.


\newpage
\subsection*{Space 83}

Space 83 is not induced by a causal order, but it is a refinement of the space 100 induced by the definite causal order $\total{\ev{A},\ev{B},\ev{C}}$.
Its equivalence class under event-input permutation symmetry contains 24 spaces.
Space 83 differs as follows from the space induced by causal order $\total{\ev{A},\ev{B},\ev{C}}$:
\begin{itemize}
  \item The outputs at events \evset{\ev{A}, \ev{C}} are independent of the input at event \ev{B} when the inputs at events \evset{A, C} are given by \hist{A/1,C/0} and \hist{A/1,C/1}.
  \item The output at event \ev{B} is independent of the input at event \ev{A} when the input at event B is given by \hist{B/1}.
\end{itemize}

\noindent Below are the histories and extended histories for space 83: 
\begin{center}
    \begin{tabular}{cc}
    \includegraphics[height=3.5cm]{svg-inkscape/space-ABC-unique-tight-83-highlighted_svg-tex.pdf}
    &
    \includegraphics[height=3.5cm]{svg-inkscape/space-ABC-unique-tight-83-ext-highlighted_svg-tex.pdf}
    \\
    $\Theta_{83}$
    &
    $\Ext{\Theta_{83}}$
    \end{tabular}
\end{center}

\noindent The standard causaltope for Space 83 has dimension 37.
Below is a plot of the homogeneous linear system of causality and quasi-normalisation equations for the standard causaltope, put in reduced row echelon form:

\begin{center}
    \includegraphics[width=11cm]{svg-inkscape/space83-rref-eqs_svg-tex.pdf}
\end{center}

\noindent Rows correspond to the 26 independent linear equations.
Columns in the plot correspond to entries of empirical models, indexed as $i_A i_B i_C$ $o_A o_B o_C$.
Coefficients in the equations are color-coded as white=0, red=+1 and blue=-1.

Space 83 has closest refinements in equivalence classes 63, 67 and 74; 
it is the join of its (closest) refinements.
It has closest coarsenings in equivalence classes 90, 91 and 93; 
it is the meet of its (closest) coarsenings.
It has 2048 causal functions, 320 of which are not causal for any of its refinements.
It is a tight space.

The standard causaltope for Space 83 has 1 more dimension than that of its subspace in equivalence class 67.
The standard causaltope for Space 83 is the meet of the standard causaltopes for its closest coarsenings.
For completeness, below is a plot of the full homogeneous linear system of causality and quasi-normalisation equations for the standard causaltope:

\begin{center}
    \includegraphics[width=12cm]{svg-inkscape/space83-eqs_svg-tex.pdf}
\end{center}

\noindent Rows correspond to the 49 linear equations, of which 26 are independent.


\newpage
\subsection*{Space 84}

Space 84 is not induced by a causal order, but it is a refinement of the space 92 induced by the definite causal order $\total{\ev{A},\ev{C}}\vee\total{\ev{B},\ev{C}}$.
Its equivalence class under event-input permutation symmetry contains 24 spaces.
Space 84 differs as follows from the space induced by causal order $\total{\ev{A},\ev{C}}\vee\total{\ev{B},\ev{C}}$:
\begin{itemize}
  \item The outputs at events \evset{\ev{B}, \ev{C}} are independent of the input at event \ev{A} when the inputs at events \evset{B, C} are given by \hist{B/1,C/1}.
\end{itemize}

\noindent Below are the histories and extended histories for space 84: 
\begin{center}
    \begin{tabular}{cc}
    \includegraphics[height=3.5cm]{svg-inkscape/space-ABC-unique-tight-84-highlighted_svg-tex.pdf}
    &
    \includegraphics[height=3.5cm]{svg-inkscape/space-ABC-unique-tight-84-ext-highlighted_svg-tex.pdf}
    \\
    $\Theta_{84}$
    &
    $\Ext{\Theta_{84}}$
    \end{tabular}
\end{center}

\noindent The standard causaltope for Space 84 has dimension 38.
Below is a plot of the homogeneous linear system of causality and quasi-normalisation equations for the standard causaltope, put in reduced row echelon form:

\begin{center}
    \includegraphics[width=11cm]{svg-inkscape/space84-rref-eqs_svg-tex.pdf}
\end{center}

\noindent Rows correspond to the 25 independent linear equations.
Columns in the plot correspond to entries of empirical models, indexed as $i_A i_B i_C$ $o_A o_B o_C$.
Coefficients in the equations are color-coded as white=0, red=+1 and blue=-1.

Space 84 has closest refinements in equivalence classes 66, 67, 68, 69 and 70; 
it is the join of its (closest) refinements.
It has closest coarsenings in equivalence classes 89, 90 and 92; 
it is the meet of its (closest) coarsenings.
It has 2048 causal functions, 128 of which are not causal for any of its refinements.
It is a tight space.

The standard causaltope for Space 84 has 2 more dimensions than those of its 7 subspaces in equivalence classes 66, 67, 68, 69 and 70.
The standard causaltope for Space 84 is the meet of the standard causaltopes for its closest coarsenings.
For completeness, below is a plot of the full homogeneous linear system of causality and quasi-normalisation equations for the standard causaltope:

\begin{center}
    \includegraphics[width=12cm]{svg-inkscape/space84-eqs_svg-tex.pdf}
\end{center}

\noindent Rows correspond to the 51 linear equations, of which 25 are independent.


\newpage
\subsection*{Space 85}

Space 85 is not induced by a causal order, but it is a refinement of the space induced by the indefinite causal order $\total{\ev{A},\{\ev{B},\ev{C}\}}$.
Its equivalence class under event-input permutation symmetry contains 48 spaces.
Space 85 differs as follows from the space induced by causal order $\total{\ev{A},\{\ev{B},\ev{C}\}}$:
\begin{itemize}
  \item The outputs at events \evset{\ev{A}, \ev{B}} are independent of the input at event \ev{C} when the inputs at events \evset{A, B} are given by \hist{A/0,B/0}, \hist{A/0,B/1} and \hist{A/1,B/0}.
  \item The output at event \ev{B} is independent of the inputs at events \evset{\ev{A}, \ev{C}} when the input at event B is given by \hist{B/0}.
  \item The outputs at events \evset{\ev{A}, \ev{C}} are independent of the input at event \ev{B} when the inputs at events \evset{A, C} are given by \hist{A/1,C/0} and \hist{A/1,C/1}.
  \item The outputs at events \evset{\ev{B}, \ev{C}} are independent of the input at event \ev{A} when the inputs at events \evset{B, C} are given by \hist{B/0,C/1}.
\end{itemize}

\noindent Below are the histories and extended histories for space 85: 
\begin{center}
    \begin{tabular}{cc}
    \includegraphics[height=3.5cm]{svg-inkscape/space-ABC-unique-untight-85-highlighted_svg-tex.pdf}
    &
    \includegraphics[height=3.5cm]{svg-inkscape/space-ABC-unique-untight-85-ext-highlighted_svg-tex.pdf}
    \\
    $\Theta_{85}$
    &
    $\Ext{\Theta_{85}}$
    \end{tabular}
\end{center}

\noindent The standard causaltope for Space 85 has dimension 37.
Below is a plot of the homogeneous linear system of causality and quasi-normalisation equations for the standard causaltope, put in reduced row echelon form:

\begin{center}
    \includegraphics[width=11cm]{svg-inkscape/space85-rref-eqs_svg-tex.pdf}
\end{center}

\noindent Rows correspond to the 26 independent linear equations.
Columns in the plot correspond to entries of empirical models, indexed as $i_A i_B i_C$ $o_A o_B o_C$.
Coefficients in the equations are color-coded as white=0, red=+1 and blue=-1.

Space 85 has closest refinements in equivalence classes 62, 63, 71 and 72; 
it is the join of its (closest) refinements.
It has closest coarsenings in equivalence class 93; 
it does not arise as a nontrivial meet in the hierarchy.
It has 2048 causal functions, 640 of which are not causal for any of its refinements.
It is not a tight space: for event \ev{C}, a causal function must yield identical output values on input histories \hist{A/1,C/1} and \hist{B/0,C/1}.

The standard causaltope for Space 85 has 2 more dimensions than those of its 4 subspaces in equivalence classes 62, 63, 71 and 72.
The standard causaltope for Space 85 has 2 dimensions fewer than the meet of the standard causaltopes for its closest coarsenings.
For completeness, below is a plot of the full homogeneous linear system of causality and quasi-normalisation equations for the standard causaltope:

\begin{center}
    \includegraphics[width=12cm]{svg-inkscape/space85-eqs_svg-tex.pdf}
\end{center}

\noindent Rows correspond to the 49 linear equations, of which 26 are independent.


\newpage
\subsection*{Space 86}

Space 86 is not induced by a causal order, but it is a refinement of the space in equivalence class 100 induced by the definite causal order $\total{\ev{B},\ev{A},\ev{B}}$ (note that the space induced by the order is not the same as space 100).
Its equivalence class under event-input permutation symmetry contains 24 spaces.
Space 86 differs as follows from the space induced by causal order $\total{\ev{B},\ev{A},\ev{B}}$:
\begin{itemize}
  \item The outputs at events \evset{\ev{B}, \ev{C}} are independent of the input at event \ev{A} when the inputs at events \evset{B, C} are given by \hist{B/1,C/1} and \hist{B/0,C/1}.
  \item The output at event \ev{A} is independent of the input at event \ev{B} when the input at event A is given by \hist{A/0}.
\end{itemize}

\noindent Below are the histories and extended histories for space 86: 
\begin{center}
    \begin{tabular}{cc}
    \includegraphics[height=3.5cm]{svg-inkscape/space-ABC-unique-tight-86-highlighted_svg-tex.pdf}
    &
    \includegraphics[height=3.5cm]{svg-inkscape/space-ABC-unique-tight-86-ext-highlighted_svg-tex.pdf}
    \\
    $\Theta_{86}$
    &
    $\Ext{\Theta_{86}}$
    \end{tabular}
\end{center}

\noindent The standard causaltope for Space 86 has dimension 37.
Below is a plot of the homogeneous linear system of causality and quasi-normalisation equations for the standard causaltope, put in reduced row echelon form:

\begin{center}
    \includegraphics[width=11cm]{svg-inkscape/space86-rref-eqs_svg-tex.pdf}
\end{center}

\noindent Rows correspond to the 26 independent linear equations.
Columns in the plot correspond to entries of empirical models, indexed as $i_A i_B i_C$ $o_A o_B o_C$.
Coefficients in the equations are color-coded as white=0, red=+1 and blue=-1.

Space 86 has closest refinements in equivalence classes 68, 74, 75 and 76; 
it is the join of its (closest) refinements.
It has closest coarsenings in equivalence classes 90 and 94; 
it is the meet of its (closest) coarsenings.
It has 2048 causal functions, 1024 of which are not causal for any of its refinements.
It is a tight space.

The standard causaltope for Space 86 has 1 more dimension than that of its subspace in equivalence class 68.
The standard causaltope for Space 86 is the meet of the standard causaltopes for its closest coarsenings.
For completeness, below is a plot of the full homogeneous linear system of causality and quasi-normalisation equations for the standard causaltope:

\begin{center}
    \includegraphics[width=12cm]{svg-inkscape/space86-eqs_svg-tex.pdf}
\end{center}

\noindent Rows correspond to the 49 linear equations, of which 26 are independent.


\newpage
\subsection*{Space 87}

Space 87 is not induced by a causal order, but it is a refinement of the space in equivalence class 100 induced by the definite causal order $\total{\ev{B},\ev{A},\ev{B}}$ (note that the space induced by the order is not the same as space 100).
Its equivalence class under event-input permutation symmetry contains 24 spaces.
Space 87 differs as follows from the space induced by causal order $\total{\ev{B},\ev{A},\ev{B}}$:
\begin{itemize}
  \item The outputs at events \evset{\ev{B}, \ev{C}} are independent of the input at event \ev{A} when the inputs at events \evset{B, C} are given by \hist{B/1,C/0} and \hist{B/0,C/1}.
  \item The output at event \ev{A} is independent of the input at event \ev{B} when the input at event A is given by \hist{A/0}.
\end{itemize}

\noindent Below are the histories and extended histories for space 87: 
\begin{center}
    \begin{tabular}{cc}
    \includegraphics[height=3.5cm]{svg-inkscape/space-ABC-unique-tight-87-highlighted_svg-tex.pdf}
    &
    \includegraphics[height=3.5cm]{svg-inkscape/space-ABC-unique-tight-87-ext-highlighted_svg-tex.pdf}
    \\
    $\Theta_{87}$
    &
    $\Ext{\Theta_{87}}$
    \end{tabular}
\end{center}

\noindent The standard causaltope for Space 87 has dimension 37.
Below is a plot of the homogeneous linear system of causality and quasi-normalisation equations for the standard causaltope, put in reduced row echelon form:

\begin{center}
    \includegraphics[width=11cm]{svg-inkscape/space87-rref-eqs_svg-tex.pdf}
\end{center}

\noindent Rows correspond to the 26 independent linear equations.
Columns in the plot correspond to entries of empirical models, indexed as $i_A i_B i_C$ $o_A o_B o_C$.
Coefficients in the equations are color-coded as white=0, red=+1 and blue=-1.

Space 87 has closest refinements in equivalence classes 70, 73 and 74; 
it is the join of its (closest) refinements.
It has closest coarsenings in equivalence classes 90 and 95; 
it is the meet of its (closest) coarsenings.
It has 2048 causal functions, 512 of which are not causal for any of its refinements.
It is a tight space.

The standard causaltope for Space 87 has 1 more dimension than that of its subspace in equivalence class 70.
The standard causaltope for Space 87 is the meet of the standard causaltopes for its closest coarsenings.
For completeness, below is a plot of the full homogeneous linear system of causality and quasi-normalisation equations for the standard causaltope:

\begin{center}
    \includegraphics[width=12cm]{svg-inkscape/space87-eqs_svg-tex.pdf}
\end{center}

\noindent Rows correspond to the 49 linear equations, of which 26 are independent.


\newpage
\subsection*{Space 88}

Space 88 is not induced by a causal order, but it is a refinement of the space 100 induced by the definite causal order $\total{\ev{A},\ev{B},\ev{C}}$.
Its equivalence class under event-input permutation symmetry contains 24 spaces.
Space 88 differs as follows from the space induced by causal order $\total{\ev{A},\ev{B},\ev{C}}$:
\begin{itemize}
  \item The outputs at events \evset{\ev{A}, \ev{C}} are independent of the input at event \ev{B} when the inputs at events \evset{A, C} are given by \hist{A/0,C/1}, \hist{A/1,C/0} and \hist{A/1,C/1}.
\end{itemize}

\noindent Below are the histories and extended histories for space 88: 
\begin{center}
    \begin{tabular}{cc}
    \includegraphics[height=3.5cm]{svg-inkscape/space-ABC-unique-tight-88-highlighted_svg-tex.pdf}
    &
    \includegraphics[height=3.5cm]{svg-inkscape/space-ABC-unique-tight-88-ext-highlighted_svg-tex.pdf}
    \\
    $\Theta_{88}$
    &
    $\Ext{\Theta_{88}}$
    \end{tabular}
\end{center}

\noindent The standard causaltope for Space 88 has dimension 36.
Below is a plot of the homogeneous linear system of causality and quasi-normalisation equations for the standard causaltope, put in reduced row echelon form:

\begin{center}
    \includegraphics[width=11cm]{svg-inkscape/space88-rref-eqs_svg-tex.pdf}
\end{center}

\noindent Rows correspond to the 27 independent linear equations.
Columns in the plot correspond to entries of empirical models, indexed as $i_A i_B i_C$ $o_A o_B o_C$.
Coefficients in the equations are color-coded as white=0, red=+1 and blue=-1.

Space 88 has closest refinements in equivalence classes 64, 74 and 77; 
it is the join of its (closest) refinements.
It has closest coarsenings in equivalence classes 91, 94, 95 and 96; 
it is the meet of its (closest) coarsenings.
It has 2048 causal functions, 256 of which are not causal for any of its refinements.
It is a tight space.

The standard causaltope for Space 88 has 1 more dimension than those of its 3 subspaces in equivalence classes 64 and 74.
The standard causaltope for Space 88 is the meet of the standard causaltopes for its closest coarsenings.
For completeness, below is a plot of the full homogeneous linear system of causality and quasi-normalisation equations for the standard causaltope:

\begin{center}
    \includegraphics[width=12cm]{svg-inkscape/space88-eqs_svg-tex.pdf}
\end{center}

\noindent Rows correspond to the 47 linear equations, of which 27 are independent.


\newpage
\subsection*{Space 89}

Space 89 is not induced by a causal order, but it is a refinement of the space 100 induced by the definite causal order $\total{\ev{A},\ev{B},\ev{C}}$.
Its equivalence class under event-input permutation symmetry contains 24 spaces.
Space 89 differs as follows from the space induced by causal order $\total{\ev{A},\ev{B},\ev{C}}$:
\begin{itemize}
  \item The outputs at events \evset{\ev{B}, \ev{C}} are independent of the input at event \ev{A} when the inputs at events \evset{B, C} are given by \hist{B/1,C/1}.
  \item The output at event \ev{B} is independent of the input at event \ev{A} when the input at event B is given by \hist{B/1}.
\end{itemize}

\noindent Below are the histories and extended histories for space 89: 
\begin{center}
    \begin{tabular}{cc}
    \includegraphics[height=3.5cm]{svg-inkscape/space-ABC-unique-tight-89-highlighted_svg-tex.pdf}
    &
    \includegraphics[height=3.5cm]{svg-inkscape/space-ABC-unique-tight-89-ext-highlighted_svg-tex.pdf}
    \\
    $\Theta_{89}$
    &
    $\Ext{\Theta_{89}}$
    \end{tabular}
\end{center}

\noindent The standard causaltope for Space 89 has dimension 39.
Below is a plot of the homogeneous linear system of causality and quasi-normalisation equations for the standard causaltope, put in reduced row echelon form:

\begin{center}
    \includegraphics[width=11cm]{svg-inkscape/space89-rref-eqs_svg-tex.pdf}
\end{center}

\noindent Rows correspond to the 24 independent linear equations.
Columns in the plot correspond to entries of empirical models, indexed as $i_A i_B i_C$ $o_A o_B o_C$.
Coefficients in the equations are color-coded as white=0, red=+1 and blue=-1.

Space 89 has closest refinements in equivalence classes 80, 81, 82 and 84; 
it is the join of its (closest) refinements.
It has closest coarsenings in equivalence class 98; 
it does not arise as a nontrivial meet in the hierarchy.
It has 4096 causal functions, 256 of which are not causal for any of its refinements.
It is a tight space.

The standard causaltope for Space 89 has 1 more dimension than that of its subspace in equivalence class 84.
The standard causaltope for Space 89 has 2 dimensions fewer than the meet of the standard causaltopes for its closest coarsenings.
For completeness, below is a plot of the full homogeneous linear system of causality and quasi-normalisation equations for the standard causaltope:

\begin{center}
    \includegraphics[width=12cm]{svg-inkscape/space89-eqs_svg-tex.pdf}
\end{center}

\noindent Rows correspond to the 45 linear equations, of which 24 are independent.


\newpage
\subsection*{Space 90}

Space 90 is not induced by a causal order, but it is a refinement of the space 100 induced by the definite causal order $\total{\ev{A},\ev{B},\ev{C}}$.
Its equivalence class under event-input permutation symmetry contains 48 spaces.
Space 90 differs as follows from the space induced by causal order $\total{\ev{A},\ev{B},\ev{C}}$:
\begin{itemize}
  \item The outputs at events \evset{\ev{A}, \ev{C}} are independent of the input at event \ev{B} when the inputs at events \evset{A, C} are given by \hist{A/1,C/0}.
  \item The output at event \ev{B} is independent of the input at event \ev{A} when the input at event B is given by \hist{B/1}.
\end{itemize}

\noindent Below are the histories and extended histories for space 90: 
\begin{center}
    \begin{tabular}{cc}
    \includegraphics[height=3.5cm]{svg-inkscape/space-ABC-unique-tight-90-highlighted_svg-tex.pdf}
    &
    \includegraphics[height=3.5cm]{svg-inkscape/space-ABC-unique-tight-90-ext-highlighted_svg-tex.pdf}
    \\
    $\Theta_{90}$
    &
    $\Ext{\Theta_{90}}$
    \end{tabular}
\end{center}

\noindent The standard causaltope for Space 90 has dimension 39.
Below is a plot of the homogeneous linear system of causality and quasi-normalisation equations for the standard causaltope, put in reduced row echelon form:

\begin{center}
    \includegraphics[width=11cm]{svg-inkscape/space90-rref-eqs_svg-tex.pdf}
\end{center}

\noindent Rows correspond to the 24 independent linear equations.
Columns in the plot correspond to entries of empirical models, indexed as $i_A i_B i_C$ $o_A o_B o_C$.
Coefficients in the equations are color-coded as white=0, red=+1 and blue=-1.

Space 90 has closest refinements in equivalence classes 80, 82, 83, 84, 86 and 87; 
it is the join of its (closest) refinements.
It has closest coarsenings in equivalence classes 97 and 98; 
it is the meet of its (closest) coarsenings.
It has 4096 causal functions, 896 of which are not causal for any of its refinements.
It is a tight space.

The standard causaltope for Space 90 has 1 more dimension than that of its subspace in equivalence class 84.
The standard causaltope for Space 90 is the meet of the standard causaltopes for its closest coarsenings.
For completeness, below is a plot of the full homogeneous linear system of causality and quasi-normalisation equations for the standard causaltope:

\begin{center}
    \includegraphics[width=12cm]{svg-inkscape/space90-eqs_svg-tex.pdf}
\end{center}

\noindent Rows correspond to the 45 linear equations, of which 24 are independent.


\newpage
\subsection*{Space 91}

Space 91 is not induced by a causal order, but it is a refinement of the space 100 induced by the definite causal order $\total{\ev{A},\ev{B},\ev{C}}$.
Its equivalence class under event-input permutation symmetry contains 12 spaces.
Space 91 differs as follows from the space induced by causal order $\total{\ev{A},\ev{B},\ev{C}}$:
\begin{itemize}
  \item The outputs at events \evset{\ev{A}, \ev{C}} are independent of the input at event \ev{B} when the inputs at events \evset{A, C} are given by \hist{A/1,C/0} and \hist{A/1,C/1}.
\end{itemize}

\noindent Below are the histories and extended histories for space 91: 
\begin{center}
    \begin{tabular}{cc}
    \includegraphics[height=3.5cm]{svg-inkscape/space-ABC-unique-tight-91-highlighted_svg-tex.pdf}
    &
    \includegraphics[height=3.5cm]{svg-inkscape/space-ABC-unique-tight-91-ext-highlighted_svg-tex.pdf}
    \\
    $\Theta_{91}$
    &
    $\Ext{\Theta_{91}}$
    \end{tabular}
\end{center}

\noindent The standard causaltope for Space 91 has dimension 38.
Below is a plot of the homogeneous linear system of causality and quasi-normalisation equations for the standard causaltope, put in reduced row echelon form:

\begin{center}
    \includegraphics[width=11cm]{svg-inkscape/space91-rref-eqs_svg-tex.pdf}
\end{center}

\noindent Rows correspond to the 25 independent linear equations.
Columns in the plot correspond to entries of empirical models, indexed as $i_A i_B i_C$ $o_A o_B o_C$.
Coefficients in the equations are color-coded as white=0, red=+1 and blue=-1.

Space 91 has closest refinements in equivalence classes 83 and 88; 
it is the join of its (closest) refinements.
It has closest coarsenings in equivalence classes 97 and 99; 
it is the meet of its (closest) coarsenings.
It has 4096 causal functions, 576 of which are not causal for any of its refinements.
It is a tight space.

The standard causaltope for Space 91 has 1 more dimension than those of its 2 subspaces in equivalence class 83.
The standard causaltope for Space 91 is the meet of the standard causaltopes for its closest coarsenings.
For completeness, below is a plot of the full homogeneous linear system of causality and quasi-normalisation equations for the standard causaltope:

\begin{center}
    \includegraphics[width=12cm]{svg-inkscape/space91-eqs_svg-tex.pdf}
\end{center}

\noindent Rows correspond to the 43 linear equations, of which 25 are independent.


\newpage
\subsection*{Space 92}

Space 92 is induced by the definite causal order $\total{\ev{A},\ev{C}}\vee\total{\ev{B},\ev{C}}$.
Its equivalence class under event-input permutation symmetry contains 3 spaces.

\noindent Below are the histories and extended histories for space 92: 
\begin{center}
    \begin{tabular}{cc}
    \includegraphics[height=3.5cm]{svg-inkscape/space-ABC-unique-tight-92-highlighted_svg-tex.pdf}
    &
    \includegraphics[height=3.5cm]{svg-inkscape/space-ABC-unique-tight-92-ext-highlighted_svg-tex.pdf}
    \\
    $\Theta_{92}$
    &
    $\Ext{\Theta_{92}}$
    \end{tabular}
\end{center}

\noindent The standard causaltope for Space 92 has dimension 40.
Below is a plot of the homogeneous linear system of causality and quasi-normalisation equations for the standard causaltope, put in reduced row echelon form:

\begin{center}
    \includegraphics[width=11cm]{svg-inkscape/space92-rref-eqs_svg-tex.pdf}
\end{center}

\noindent Rows correspond to the 23 independent linear equations.
Columns in the plot correspond to entries of empirical models, indexed as $i_A i_B i_C$ $o_A o_B o_C$.
Coefficients in the equations are color-coded as white=0, red=+1 and blue=-1.

Space 92 has closest refinements in equivalence class 84; 
it is the join of its (closest) refinements.
It has closest coarsenings in equivalence class 98; 
it is the meet of its (closest) coarsenings.
It has 4096 causal functions, 512 of which are not causal for any of its refinements.
It is a tight space.

The standard causaltope for Space 92 has 2 more dimensions than those of its 8 subspaces in equivalence class 84.
The standard causaltope for Space 92 is the meet of the standard causaltopes for its closest coarsenings.
For completeness, below is a plot of the full homogeneous linear system of causality and quasi-normalisation equations for the standard causaltope:

\begin{center}
    \includegraphics[width=12cm]{svg-inkscape/space92-eqs_svg-tex.pdf}
\end{center}

\noindent Rows correspond to the 47 linear equations, of which 23 are independent.


\newpage
\subsection*{Space 93}

Space 93 is not induced by a causal order, but it is a refinement of the space induced by the indefinite causal order $\total{\ev{A},\{\ev{B},\ev{C}\}}$.
Its equivalence class under event-input permutation symmetry contains 24 spaces.
Space 93 differs as follows from the space induced by causal order $\total{\ev{A},\{\ev{B},\ev{C}\}}$:
\begin{itemize}
  \item The outputs at events \evset{\ev{A}, \ev{B}} are independent of the input at event \ev{C} when the inputs at events \evset{A, B} are given by \hist{A/0,B/0}, \hist{A/0,B/1} and \hist{A/1,B/0}.
  \item The output at event \ev{B} is independent of the inputs at events \evset{\ev{A}, \ev{C}} when the input at event B is given by \hist{B/0}.
  \item The outputs at events \evset{\ev{A}, \ev{C}} are independent of the input at event \ev{B} when the inputs at events \evset{A, C} are given by \hist{A/1,C/0} and \hist{A/1,C/1}.
\end{itemize}

\noindent Below are the histories and extended histories for space 93: 
\begin{center}
    \begin{tabular}{cc}
    \includegraphics[height=3.5cm]{svg-inkscape/space-ABC-unique-tight-93-highlighted_svg-tex.pdf}
    &
    \includegraphics[height=3.5cm]{svg-inkscape/space-ABC-unique-tight-93-ext-highlighted_svg-tex.pdf}
    \\
    $\Theta_{93}$
    &
    $\Ext{\Theta_{93}}$
    \end{tabular}
\end{center}

\noindent The standard causaltope for Space 93 has dimension 39.
Below is a plot of the homogeneous linear system of causality and quasi-normalisation equations for the standard causaltope, put in reduced row echelon form:

\begin{center}
    \includegraphics[width=11cm]{svg-inkscape/space93-rref-eqs_svg-tex.pdf}
\end{center}

\noindent Rows correspond to the 24 independent linear equations.
Columns in the plot correspond to entries of empirical models, indexed as $i_A i_B i_C$ $o_A o_B o_C$.
Coefficients in the equations are color-coded as white=0, red=+1 and blue=-1.

Space 93 has closest refinements in equivalence classes 79, 83 and 85; 
it is the join of its (closest) refinements.
It has closest coarsenings in equivalence class 99; 
it does not arise as a nontrivial meet in the hierarchy.
It has 4096 causal functions, 896 of which are not causal for any of its refinements.
It is a tight space.

The standard causaltope for Space 93 has 2 more dimensions than those of its 5 subspaces in equivalence classes 79, 83 and 85.
The standard causaltope for Space 93 has 1 dimension fewer than the meet of the standard causaltopes for its closest coarsenings.
For completeness, below is a plot of the full homogeneous linear system of causality and quasi-normalisation equations for the standard causaltope:

\begin{center}
    \includegraphics[width=12cm]{svg-inkscape/space93-eqs_svg-tex.pdf}
\end{center}

\noindent Rows correspond to the 45 linear equations, of which 24 are independent.


\newpage
\subsection*{Space 94}

Space 94 is not induced by a causal order, but it is a refinement of the space 100 induced by the definite causal order $\total{\ev{A},\ev{B},\ev{C}}$.
Its equivalence class under event-input permutation symmetry contains 12 spaces.
Space 94 differs as follows from the space induced by causal order $\total{\ev{A},\ev{B},\ev{C}}$:
\begin{itemize}
  \item The outputs at events \evset{\ev{A}, \ev{C}} are independent of the input at event \ev{B} when the inputs at events \evset{A, C} are given by \hist{A/0,C/1} and \hist{A/1,C/1}.
\end{itemize}

\noindent Below are the histories and extended histories for space 94: 
\begin{center}
    \begin{tabular}{cc}
    \includegraphics[height=3.5cm]{svg-inkscape/space-ABC-unique-tight-94-highlighted_svg-tex.pdf}
    &
    \includegraphics[height=3.5cm]{svg-inkscape/space-ABC-unique-tight-94-ext-highlighted_svg-tex.pdf}
    \\
    $\Theta_{94}$
    &
    $\Ext{\Theta_{94}}$
    \end{tabular}
\end{center}

\noindent The standard causaltope for Space 94 has dimension 38.
Below is a plot of the homogeneous linear system of causality and quasi-normalisation equations for the standard causaltope, put in reduced row echelon form:

\begin{center}
    \includegraphics[width=11cm]{svg-inkscape/space94-rref-eqs_svg-tex.pdf}
\end{center}

\noindent Rows correspond to the 25 independent linear equations.
Columns in the plot correspond to entries of empirical models, indexed as $i_A i_B i_C$ $o_A o_B o_C$.
Coefficients in the equations are color-coded as white=0, red=+1 and blue=-1.

Space 94 has closest refinements in equivalence classes 78, 86 and 88; 
it is the join of its (closest) refinements.
It has closest coarsenings in equivalence class 97; 
it is the meet of its (closest) coarsenings.
It has 4096 causal functions, 640 of which are not causal for any of its refinements.
It is a tight space.

The standard causaltope for Space 94 has 1 more dimension than those of its 3 subspaces in equivalence classes 78 and 86.
The standard causaltope for Space 94 is the meet of the standard causaltopes for its closest coarsenings.
For completeness, below is a plot of the full homogeneous linear system of causality and quasi-normalisation equations for the standard causaltope:

\begin{center}
    \includegraphics[width=12cm]{svg-inkscape/space94-eqs_svg-tex.pdf}
\end{center}

\noindent Rows correspond to the 43 linear equations, of which 25 are independent.


\newpage
\subsection*{Space 95}

Space 95 is not induced by a causal order, but it is a refinement of the space 100 induced by the definite causal order $\total{\ev{A},\ev{B},\ev{C}}$.
Its equivalence class under event-input permutation symmetry contains 12 spaces.
Space 95 differs as follows from the space induced by causal order $\total{\ev{A},\ev{B},\ev{C}}$:
\begin{itemize}
  \item The outputs at events \evset{\ev{A}, \ev{C}} are independent of the input at event \ev{B} when the inputs at events \evset{A, C} are given by \hist{A/0,C/1} and \hist{A/1,C/0}.
\end{itemize}

\noindent Below are the histories and extended histories for space 95: 
\begin{center}
    \begin{tabular}{cc}
    \includegraphics[height=3.5cm]{svg-inkscape/space-ABC-unique-tight-95-highlighted_svg-tex.pdf}
    &
    \includegraphics[height=3.5cm]{svg-inkscape/space-ABC-unique-tight-95-ext-highlighted_svg-tex.pdf}
    \\
    $\Theta_{95}$
    &
    $\Ext{\Theta_{95}}$
    \end{tabular}
\end{center}

\noindent The standard causaltope for Space 95 has dimension 38.
Below is a plot of the homogeneous linear system of causality and quasi-normalisation equations for the standard causaltope, put in reduced row echelon form:

\begin{center}
    \includegraphics[width=11cm]{svg-inkscape/space95-rref-eqs_svg-tex.pdf}
\end{center}

\noindent Rows correspond to the 25 independent linear equations.
Columns in the plot correspond to entries of empirical models, indexed as $i_A i_B i_C$ $o_A o_B o_C$.
Coefficients in the equations are color-coded as white=0, red=+1 and blue=-1.

Space 95 has closest refinements in equivalence classes 87 and 88; 
it is the join of its (closest) refinements.
It has closest coarsenings in equivalence class 97; 
it is the meet of its (closest) coarsenings.
It has 4096 causal functions, 1024 of which are not causal for any of its refinements.
It is a tight space.

The standard causaltope for Space 95 has 1 more dimension than those of its 2 subspaces in equivalence class 87.
The standard causaltope for Space 95 is the meet of the standard causaltopes for its closest coarsenings.
For completeness, below is a plot of the full homogeneous linear system of causality and quasi-normalisation equations for the standard causaltope:

\begin{center}
    \includegraphics[width=12cm]{svg-inkscape/space95-eqs_svg-tex.pdf}
\end{center}

\noindent Rows correspond to the 43 linear equations, of which 25 are independent.


\newpage
\subsection*{Space 96}

Space 96 is not induced by a causal order, but it is a refinement of the space induced by the indefinite causal order $\total{\ev{A},\{\ev{B},\ev{C}\}}$.
Its equivalence class under event-input permutation symmetry contains 24 spaces.
Space 96 differs as follows from the space induced by causal order $\total{\ev{A},\{\ev{B},\ev{C}\}}$:
\begin{itemize}
  \item The outputs at events \evset{\ev{A}, \ev{B}} are independent of the input at event \ev{C} when the inputs at events \evset{A, B} are given by \hist{A/0,B/0}, \hist{A/0,B/1} and \hist{A/1,B/0}.
  \item The outputs at events \evset{\ev{A}, \ev{C}} are independent of the input at event \ev{B} when the inputs at events \evset{A, C} are given by \hist{A/0,C/1}, \hist{A/1,C/0} and \hist{A/1,C/1}.
\end{itemize}

\noindent Below are the histories and extended histories for space 96: 
\begin{center}
    \begin{tabular}{cc}
    \includegraphics[height=3.5cm]{svg-inkscape/space-ABC-unique-tight-96-highlighted_svg-tex.pdf}
    &
    \includegraphics[height=3.5cm]{svg-inkscape/space-ABC-unique-tight-96-ext-highlighted_svg-tex.pdf}
    \\
    $\Theta_{96}$
    &
    $\Ext{\Theta_{96}}$
    \end{tabular}
\end{center}

\noindent The standard causaltope for Space 96 has dimension 38.
Below is a plot of the homogeneous linear system of causality and quasi-normalisation equations for the standard causaltope, put in reduced row echelon form:

\begin{center}
    \includegraphics[width=11cm]{svg-inkscape/space96-rref-eqs_svg-tex.pdf}
\end{center}

\noindent Rows correspond to the 25 independent linear equations.
Columns in the plot correspond to entries of empirical models, indexed as $i_A i_B i_C$ $o_A o_B o_C$.
Coefficients in the equations are color-coded as white=0, red=+1 and blue=-1.

Space 96 has closest refinements in equivalence classes 79 and 88; 
it is the join of its (closest) refinements.
It has closest coarsenings in equivalence class 99; 
it is the meet of its (closest) coarsenings.
It has 4096 causal functions, 384 of which are not causal for any of its refinements.
It is a tight space.

The standard causaltope for Space 96 has 1 more dimension than those of its 2 subspaces in equivalence class 79.
The standard causaltope for Space 96 is the meet of the standard causaltopes for its closest coarsenings.
For completeness, below is a plot of the full homogeneous linear system of causality and quasi-normalisation equations for the standard causaltope:

\begin{center}
    \includegraphics[width=12cm]{svg-inkscape/space96-eqs_svg-tex.pdf}
\end{center}

\noindent Rows correspond to the 43 linear equations, of which 25 are independent.


\newpage
\subsection*{Space 97}

Space 97 is not induced by a causal order, but it is a refinement of the space 100 induced by the definite causal order $\total{\ev{A},\ev{B},\ev{C}}$.
Its equivalence class under event-input permutation symmetry contains 24 spaces.
Space 97 differs as follows from the space induced by causal order $\total{\ev{A},\ev{B},\ev{C}}$:
\begin{itemize}
  \item The outputs at events \evset{\ev{A}, \ev{C}} are independent of the input at event \ev{B} when the inputs at events \evset{A, C} are given by \hist{A/1,C/1}.
\end{itemize}

\noindent Below are the histories and extended histories for space 97: 
\begin{center}
    \begin{tabular}{cc}
    \includegraphics[height=3.5cm]{svg-inkscape/space-ABC-unique-tight-97-highlighted_svg-tex.pdf}
    &
    \includegraphics[height=3.5cm]{svg-inkscape/space-ABC-unique-tight-97-ext-highlighted_svg-tex.pdf}
    \\
    $\Theta_{97}$
    &
    $\Ext{\Theta_{97}}$
    \end{tabular}
\end{center}

\noindent The standard causaltope for Space 97 has dimension 40.
Below is a plot of the homogeneous linear system of causality and quasi-normalisation equations for the standard causaltope, put in reduced row echelon form:

\begin{center}
    \includegraphics[width=11cm]{svg-inkscape/space97-rref-eqs_svg-tex.pdf}
\end{center}

\noindent Rows correspond to the 23 independent linear equations.
Columns in the plot correspond to entries of empirical models, indexed as $i_A i_B i_C$ $o_A o_B o_C$.
Coefficients in the equations are color-coded as white=0, red=+1 and blue=-1.

Space 97 has closest refinements in equivalence classes 90, 91, 94 and 95; 
it is the join of its (closest) refinements.
It has closest coarsenings in equivalence class 100; 
it does not arise as a nontrivial meet in the hierarchy.
It has 8192 causal functions, 1280 of which are not causal for any of its refinements.
It is a tight space.

The standard causaltope for Space 97 has 1 more dimension than those of its 2 subspaces in equivalence class 90.
The standard causaltope for Space 97 has 2 dimensions fewer than the meet of the standard causaltopes for its closest coarsenings.
For completeness, below is a plot of the full homogeneous linear system of causality and quasi-normalisation equations for the standard causaltope:

\begin{center}
    \includegraphics[width=12cm]{svg-inkscape/space97-eqs_svg-tex.pdf}
\end{center}

\noindent Rows correspond to the 39 linear equations, of which 23 are independent.


\newpage
\subsection*{Space 98}

Space 98 is not induced by a causal order, but it is a refinement of the space 100 induced by the definite causal order $\total{\ev{A},\ev{B},\ev{C}}$.
Its equivalence class under event-input permutation symmetry contains 12 spaces.
Space 98 differs as follows from the space induced by causal order $\total{\ev{A},\ev{B},\ev{C}}$:
\begin{itemize}
  \item The output at event \ev{B} is independent of the input at event \ev{A} when the input at event B is given by \hist{B/1}.
\end{itemize}

\noindent Below are the histories and extended histories for space 98: 
\begin{center}
    \begin{tabular}{cc}
    \includegraphics[height=3.5cm]{svg-inkscape/space-ABC-unique-tight-98-highlighted_svg-tex.pdf}
    &
    \includegraphics[height=3.5cm]{svg-inkscape/space-ABC-unique-tight-98-ext-highlighted_svg-tex.pdf}
    \\
    $\Theta_{98}$
    &
    $\Ext{\Theta_{98}}$
    \end{tabular}
\end{center}

\noindent The standard causaltope for Space 98 has dimension 41.
Below is a plot of the homogeneous linear system of causality and quasi-normalisation equations for the standard causaltope, put in reduced row echelon form:

\begin{center}
    \includegraphics[width=11cm]{svg-inkscape/space98-rref-eqs_svg-tex.pdf}
\end{center}

\noindent Rows correspond to the 22 independent linear equations.
Columns in the plot correspond to entries of empirical models, indexed as $i_A i_B i_C$ $o_A o_B o_C$.
Coefficients in the equations are color-coded as white=0, red=+1 and blue=-1.

Space 98 has closest refinements in equivalence classes 89, 90 and 92; 
it is the join of its (closest) refinements.
It has closest coarsenings in equivalence class 100; 
it does not arise as a nontrivial meet in the hierarchy.
It has 8192 causal functions, 512 of which are not causal for any of its refinements.
It is a tight space.

The standard causaltope for Space 98 has 1 more dimension than that of its subspace in equivalence class 92.
The standard causaltope for Space 98 has 1 dimension fewer than the meet of the standard causaltopes for its closest coarsenings.
For completeness, below is a plot of the full homogeneous linear system of causality and quasi-normalisation equations for the standard causaltope:

\begin{center}
    \includegraphics[width=12cm]{svg-inkscape/space98-eqs_svg-tex.pdf}
\end{center}

\noindent Rows correspond to the 41 linear equations, of which 22 are independent.


\newpage
\subsection*{Space 99}

Space 99 is not induced by a causal order, but it is a refinement of the space induced by the indefinite causal order $\total{\ev{A},\{\ev{B},\ev{C}\}}$.
Its equivalence class under event-input permutation symmetry contains 24 spaces.
Space 99 differs as follows from the space induced by causal order $\total{\ev{A},\{\ev{B},\ev{C}\}}$:
\begin{itemize}
  \item The outputs at events \evset{\ev{A}, \ev{B}} are independent of the input at event \ev{C} when the inputs at events \evset{A, B} are given by \hist{A/0,B/0}, \hist{A/0,B/1} and \hist{A/1,B/0}.
  \item The outputs at events \evset{\ev{A}, \ev{C}} are independent of the input at event \ev{B} when the inputs at events \evset{A, C} are given by \hist{A/1,C/0} and \hist{A/1,C/1}.
\end{itemize}

\noindent Below are the histories and extended histories for space 99: 
\begin{center}
    \begin{tabular}{cc}
    \includegraphics[height=3.5cm]{svg-inkscape/space-ABC-unique-tight-99-highlighted_svg-tex.pdf}
    &
    \includegraphics[height=3.5cm]{svg-inkscape/space-ABC-unique-tight-99-ext-highlighted_svg-tex.pdf}
    \\
    $\Theta_{99}$
    &
    $\Ext{\Theta_{99}}$
    \end{tabular}
\end{center}

\noindent The standard causaltope for Space 99 has dimension 40.
Below is a plot of the homogeneous linear system of causality and quasi-normalisation equations for the standard causaltope, put in reduced row echelon form:

\begin{center}
    \includegraphics[width=11cm]{svg-inkscape/space99-rref-eqs_svg-tex.pdf}
\end{center}

\noindent Rows correspond to the 23 independent linear equations.
Columns in the plot correspond to entries of empirical models, indexed as $i_A i_B i_C$ $o_A o_B o_C$.
Coefficients in the equations are color-coded as white=0, red=+1 and blue=-1.

Space 99 has closest refinements in equivalence classes 91, 93 and 96; 
it is the join of its (closest) refinements.
It has closest coarsenings in equivalence class 101; 
it does not arise as a nontrivial meet in the hierarchy.
It has 8192 causal functions, 1024 of which are not causal for any of its refinements.
It is a tight space.

The standard causaltope for Space 99 has 1 more dimension than that of its subspace in equivalence class 93.
The standard causaltope for Space 99 has 2 dimensions fewer than the meet of the standard causaltopes for its closest coarsenings.
For completeness, below is a plot of the full homogeneous linear system of causality and quasi-normalisation equations for the standard causaltope:

\begin{center}
    \includegraphics[width=12cm]{svg-inkscape/space99-eqs_svg-tex.pdf}
\end{center}

\noindent Rows correspond to the 39 linear equations, of which 23 are independent.


\newpage
\subsection*{Space 100}

Space 100 is induced by the definite causal order $\total{\ev{A},\ev{B},\ev{C}}$.
Its equivalence class under event-input permutation symmetry contains 6 spaces.

\noindent Below are the histories and extended histories for space 100: 
\begin{center}
    \begin{tabular}{cc}
    \includegraphics[height=3.5cm]{svg-inkscape/space-ABC-unique-tight-100-highlighted_svg-tex.pdf}
    &
    \includegraphics[height=3.5cm]{svg-inkscape/space-ABC-unique-tight-100-ext-highlighted_svg-tex.pdf}
    \\
    $\Theta_{100}$
    &
    $\Ext{\Theta_{100}}$
    \end{tabular}
\end{center}

\noindent The standard causaltope for Space 100 has dimension 42.
Below is a plot of the homogeneous linear system of causality and quasi-normalisation equations for the standard causaltope, put in reduced row echelon form:

\begin{center}
    \includegraphics[width=11cm]{svg-inkscape/space100-rref-eqs_svg-tex.pdf}
\end{center}

\noindent Rows correspond to the 21 independent linear equations.
Columns in the plot correspond to entries of empirical models, indexed as $i_A i_B i_C$ $o_A o_B o_C$.
Coefficients in the equations are color-coded as white=0, red=+1 and blue=-1.

Space 100 has closest refinements in equivalence classes 97 and 98; 
it is the join of its (closest) refinements.
It is a global maximum of the hierarchy, with no coarsenings.
It has 16384 causal functions, 3072 of which are not causal for any of its refinements.
It is a tight space.

The standard causaltope for Space 100 has 1 more dimension than those of its 2 subspaces in equivalence class 98.
For completeness, below is a plot of the full homogeneous linear system of causality and quasi-normalisation equations for the standard causaltope:

\begin{center}
    \includegraphics[width=12cm]{svg-inkscape/space100-eqs_svg-tex.pdf}
\end{center}

\noindent Rows correspond to the 35 linear equations, of which 21 are independent.


\newpage
\subsection*{Space 101}

Space 101 is not induced by a causal order, but it is a refinement of the space induced by the indefinite causal order $\total{\ev{A},\{\ev{B},\ev{C}\}}$.
Its equivalence class under event-input permutation symmetry contains 6 spaces.
Space 101 differs as follows from the space induced by causal order $\total{\ev{A},\{\ev{B},\ev{C}\}}$:
\begin{itemize}
  \item The outputs at events \evset{\ev{A}, \ev{B}} are independent of the input at event \ev{C} when the inputs at events \evset{A, B} are given by \hist{A/0,B/0} and \hist{A/0,B/1}.
  \item The outputs at events \evset{\ev{A}, \ev{C}} are independent of the input at event \ev{B} when the inputs at events \evset{A, C} are given by \hist{A/1,C/0} and \hist{A/1,C/1}.
\end{itemize}

\noindent Below are the histories and extended histories for space 101: 
\begin{center}
    \begin{tabular}{cc}
    \includegraphics[height=3.5cm]{svg-inkscape/space-ABC-unique-tight-101-highlighted_svg-tex.pdf}
    &
    \includegraphics[height=3.5cm]{svg-inkscape/space-ABC-unique-tight-101-ext-highlighted_svg-tex.pdf}
    \\
    $\Theta_{101}$
    &
    $\Ext{\Theta_{101}}$
    \end{tabular}
\end{center}

\noindent The standard causaltope for Space 101 has dimension 42.
Below is a plot of the homogeneous linear system of causality and quasi-normalisation equations for the standard causaltope, put in reduced row echelon form:

\begin{center}
    \includegraphics[width=11cm]{svg-inkscape/space101-rref-eqs_svg-tex.pdf}
\end{center}

\noindent Rows correspond to the 21 independent linear equations.
Columns in the plot correspond to entries of empirical models, indexed as $i_A i_B i_C$ $o_A o_B o_C$.
Coefficients in the equations are color-coded as white=0, red=+1 and blue=-1.

Space 101 has closest refinements in equivalence class 99; 
it is the join of its (closest) refinements.
It is a global maximum of the hierarchy, with no coarsenings.
It has 16384 causal functions, 7296 of which are not causal for any of its refinements.
It is a tight space.

The standard causaltope for Space 101 has 2 more dimensions than those of its 4 subspaces in equivalence class 99.
For completeness, below is a plot of the full homogeneous linear system of causality and quasi-normalisation equations for the standard causaltope:

\begin{center}
    \includegraphics[width=12cm]{svg-inkscape/space101-eqs_svg-tex.pdf}
\end{center}

\noindent Rows correspond to the 35 linear equations, of which 21 are independent.


