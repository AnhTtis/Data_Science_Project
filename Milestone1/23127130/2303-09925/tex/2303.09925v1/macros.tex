\usepackage{xifthen}

\newcommand\pig[1]{\scalerel*[6pt]{\big#1}{%
  \ensurestackMath{\addstackgap[1.5pt]{\big#1}}}}
\newcommand\pigl[1]{\mathopen{\pig{#1}}}
\newcommand\pigr[1]{\mathclose{\pig{#1}}}


% == Macro to apply operators to something ==
\newcommand{\opapp}[2]{\ensuremath{#1\left(#2\right)}} % math mode operators
\newcommand{\opapptxt}[2]{\ensuremath{\text{#1}\left(#2\right)}} % text operators

% == Named mathematical sets ==
\newcommand{\nats}{\ensuremath{\mathbb{N}}}
\newcommand{\ints}{\ensuremath{\mathbb{Z}}}
\newcommand{\reals}{\ensuremath{\mathbb{R}}}
\newcommand{\cplxs}{\ensuremath{\mathbb{C}}}
\newcommand{\intsmod}[1]{\ensuremath{\integers_{#1}}}
\newcommand{\preals}{\ensuremath{\reals^+}}

% == General-purpose macros ==
\newcommand{\tsuchthat}[2]{\ensuremath{\left\{#1\middle|#2\right\}}} % tight \suchthat
\newcommand{\suchthat}[2]{\tsuchthat{\,#1\,}{\,#2\,}} % set definition
\newcommand{\mod}[1]{\ensuremath{\text{ mod } #1}}
\newcommand{\bra}[1]{\ensuremath{\left\langle#1\right|}}
\newcommand{\ket}[1]{\ensuremath{\left|#1\right\rangle}}
\newcommand{\braket}[2]{\ensuremath{\left\langle#1\middle|#2\right\rangle}}
\newcommand{\downset}[1]{\ensuremath{#1\!\downarrow}}
\newcommand{\upset}[1]{\ensuremath{#1\!\uparrow}}
\newcommand{\domSym}{\text{dom}}
\newcommand{\dom}[1]{\opapp{\domSym}{#1}}
\newcommand{\suppSym}{\text{supp}}
\newcommand{\supp}[1]{\opapp{\suppSym}{#1}}
\newcommand{\prob}[1]{\opapp{\mathbb{P}}{#1}}
\newcommand{\cprob}[2]{\opapp{\mathbb{P}}{#1\middle|#2}}
\newcommand{\cdist}[3]{\opapp{#1}{#2\middle|#3}}
\newcommand{\restrict}[2]{#1|_{#2}}
\newcommand{\Subsets}[1]{\opapp{\mathcal{P}\!}{#1}} % powerset
\newcommand{\PFun}[1]{\opapptxt{PFun}{#1}} % partial functions
\newcommand{\id}[1]{\text{id}_{#1}}

% == Macros specific to this work ==
% \newcommand{\ev}[1]{\textsf{#1}} % sans-serif font for events (similar to Jupyter)
\newcommand{\ev}[1]{\text{#1}} % plain-text for events
\newcommand{\discrete}[1]{\opapptxt{discrete}{#1}} % discrete causal order
\newcommand{\indiscrete}[1]{\opapptxt{indiscrete}{#1}} % indiscrete causal order
\newcommand{\total}[1]{\opapptxt{total}{#1}} % total causal order
\newcommand{\seqcomposeSym}{\rightsquigarrow}
\newcommand{\causeqcls}[1]{\ensuremath{\left[#1\right]_{\simeq}}}
\newcommand{\LsetsSym}{\Lambda} % lowersets
\newcommand{\Lsets}[1]{\opapp{\LsetsSym}{#1}} % lowersets
\newcommand{\Hist}[1]{\opapptxt{Hist}{#1}} % space of histories
\newcommand{\ExtHist}[1]{\opapptxt{ExtHist}{#1}} % space of extended histories
\newcommand{\ExtSym}{\text{Ext}} % Ext function
\newcommand{\PrimeSym}{\text{Prime}} % Prime function
\newcommand{\Ext}[1]{\opapptxt{Ext}{#1}} % Ext function
\newcommand{\Prime}[1]{\opapptxt{Prime}{#1}} % Prime function
\newcommand{\allJoinsSym}{\;\dot{\vee}\;}
\newcommand{\tips}[2]{\opapp{\text{tips}_{#1}}{#2}} % tip events
\newcommand{\tip}[2]{\opapp{\text{tip}_{#1}}{#2}} % tip events
\newcommand{\CausCompl}[1]{\opapptxt{CausCompl}{#1}} % causal completions
\newcommand{\Events}[1]{{E}^{#1}} % events of a space of input histories
\newcommand{\Inputs}[1]{{I}^{#1}} % events of a space of input histories
\newcommand{\Tight}[1]{\opapptxt{Tight}{#1}} % Tightening function
\newcommand{\AllSpaces}{\ensuremath{\text{Spaces}}} % Hierarchy of spaces of input histories
\newcommand{\Spaces}[1]{\opapptxt{Spaces}{#1}} % Hierarchy of spaces of input histories for given events/inputs
\newcommand{\SpacesFC}[1]{\opapp{\text{Spaces}_{\text{FC}}}{#1}} % Hierarchy of spaces of input histories satisfying the free-choice condition for I
\newcommand{\CCSpaces}[1]{\opapptxt{CCSpaces}{#1}} % Hierarchy of causally complete spaces
\newcommand{\CSwitchSpaces}[1]{\opapptxt{CSwitchSpaces}{#1}} % Causal switch spaces
\newcommand{\CausFun}[1]{\opapptxt{CausFun}{#1}} % causal functions
\newcommand{\ExtFun}[1]{\opapptxt{ExtFun}{#1}} % extended function
\newcommand{\ExtCausFun}[1]{\opapptxt{ExtCausFun}{#1}} % extended causal function
\newcommand{\CausFunInj}[4]{i_{#1, #2; #3, #4}}
\newcommand{\TipHists}[2]{\opapp{\text{TipHists}_{#1}}{#2}} % Input histories with given event as a tip event
\newcommand{\histconstrSym}[1]{\sim_{#1}}
\newcommand{\histconstr}[3]{#2\!\histconstrSym{#1}\!\!#3}
\newcommand{\nothistconstr}[3]{#2\!\not\histconstrSym{#1}\!\!#3}
\newcommand{\histconstreqcls}[2]{\ensuremath{\left[#1\right]_{\histconstrSym{#2}}}}
\newcommand{\TipEqCls}[2]{\opapp{\text{TipEq}_{#1}}{#2}}
% \newcommand{\DistSymBare}{\mathcal{D}}
% \newcommand{\DistSymR}[1]{\DistSymBare_{#1}}
% \newcommand{\DistR}[2]{\opapp{\DistSymR{#1}}{#2}}
% \newcommand{\DistSym}{\DistSymR{\mathbb{R}^+\!}}
% \newcommand{\Dist}[1]{\DistR{\mathbb{R}^+\!}{#1}}
\newcommand{\DistSym}{\mathcal{D}}
\newcommand{\Dist}[1]{\opapp{\DistSym}{#1}}
% \newcommand{\DistSym}{\text{Dist}}
% \newcommand{\Dist}[1]{\opapptxt{Dist}{#1}}
\newcommand{\CausDist}[1]{\opapptxt{CausDist}{#1}} % causal functions
\newcommand{\ExtCausDist}[1]{\opapptxt{ExtCausDist}{#1}} % extended causal function
\newcommand{\topdist}[1]{\left\lceil #1 \right\rceil} % top element distribution
\newcommand{\extdist}[2]{\left\lfloor #1 \right\rfloor_{#2}} % extended distribution
\newcommand{\StdCov}[1]{\opapptxt{StdCov}{#1}} % standard cover
\newcommand{\SolCov}[1]{\opapptxt{SolCov}{#1}} % solipsistic cover
\newcommand{\GlobCov}[1]{\opapptxt{GlobCov}{#1}} % global cover (DEPRECATED)
\newcommand{\ClsCov}[1]{\opapptxt{ClsCov}{#1}} % classical cover
\newcommand{\Covers}[1]{\opapptxt{Covers}{#1}} % all covers
\newcommand{\EmpModels}[1]{\opapptxt{EmpMod}{#1}} % empirical models
\newcommand{\Vertices}[1]{\opapp{\mathcal{V}}{#1}} % vertices of a polytope
\newcommand{\Faces}[1]{\opapp{\mathcal{F}}{#1}} % faces of a polytope
\newcommand{\FacePoset}[1]{\opapp{\overline{\mathcal{F}}}{#1}} % face poset of a polytope
\newcommand{\AffSubsp}[1]{\opapp{\mathbb{A}}{#1}} % minimal affine subspace containing a polytope
\newcommand{\boundary}[1]{\opapp{\partial}{#1}} % topological boundary
\newcommand{\interior}[1]{#1\backslash\boundary{#1}} % topological boundary
\newcommand{\Slice}[2]{\opapp{\text{Slice}_{#1}}{#2}}
\newcommand{\NormEqs}[1]{\opapptxt{NormEqs}{#1}} % Affine subspace defined by the normalisation equations
\newcommand{\QNormEqs}[1]{\opapptxt{QNormEqs}{#1}} % Linear subspace defined by the quasi-normalisation equations
\newcommand{\mass}[1]{\opapptxt{mass}{#1}} % mass of a sub-normalised conditional distribution
\newcommand{\CCPD}[1]{\opapptxt{CCPD}{#1}} % constrained conditional probability distributions
\newcommand{\QNCCPD}[1]{\opapp{\text{CCPD}_{\text{QNorm}}}{#1}} % quasi-normalised constrained conditional probability distributions
\newcommand{\PsEmpModels}[1]{\opapptxt{PEmpMods}{#1}} % empirical models
\newcommand{\PsEmpModelsVec}[1]{\left\langle \PsEmpModels{#1} \right\rangle} % empirical models vec space
\newcommand{\outhistinj}[1]{i^{(oh)}_{#1}}
\newcommand{\CausEqs}[1]{\opapptxt{CausEqs}{#1}} % Causality equations
\newcommand{\StdCausEqs}[1]{\opapp{\text{CausEqs}_{std}}{#1}} % Causality equations on standard cover
\newcommand{\SolCausEqs}[1]{\opapp{\text{CausEqs}_{sol}}{#1}} % Causality equations on solipsistic cover
\newcommand{\Causaltope}[1]{\opapp{\text{Caus}}{#1}} % causaltope on standard cover
\newcommand{\StdCausaltope}[1]{\opapp{\text{Caus}_{std}}{#1}} % causaltope on standard cover
\newcommand{\SolCausaltope}[1]{\opapp{\text{Caus}_{sol}}{#1}} % causaltope on solipsistic cover

% https://www.dickimaw-books.com/latex/admin/html/docsvlist.shtml#gls:foreach1
% https://tex.stackexchange.com/questions/16198/pgffor-special-treatment-for-last-item-in-foreach-list/59808#answer-16352
% https://www.overleaf.com/learn/latex/Spacing_in_math_mode#Reference_guide
\newcommand{\hist}[1]{% macro for a history, e.g. \hist{A/0,B/0,C/1}
    \ensuremath{
        \left\{
            \foreach \i\j [count=\idx] in {#1}{%
                \ifnum\idx=1%
                    \ev{\i}\!:\!\j%
                \else%
                    ,\,\ev{\i}\!:\!\j%
                \fi%
            }
        \right\}
    }
}
\newcommand{\evset}[1]{% macro for a set of events, e.g. \evset{A,B,C}
    \ensuremath{
        \left\{
            \foreach \i [count=\idx] in {#1}{%
                \ifnum\idx=1%
                    \ev{\i}%
                \else%
                    ,\ev{\i}%
                \fi%
            }
        \right\}
    }
}
