\documentclass[a4paper,prl,superscriptaddress,aps,notitlepage,floatfix, twocolumn]{revtex4-2}
\usepackage[utf8]{inputenc}
\usepackage[T1]{fontenc}
\usepackage{amsmath, amssymb, array}
\usepackage[normalem]{ulem}
\usepackage[usenames]{color}
\usepackage{graphicx}
\usepackage{hhline,layout}
\usepackage{siunitx}
\usepackage{float}
\usepackage[pdftex,unicode,colorlinks,citecolor=blue,linkcolor=blue,
bookmarks=true,bookmarksopen=true,bookmarksopenlevel=3,bookmarksnumbered=true]{hyperref}

%\topmargin=-1in
%\headsep=0.1in

\newcommand{\I}[2]{\int\limits_{#1}^{#2}\,}

\newcommand{\abs}[1]{\left|\,#1\right|}
\definecolor{MKcolor}{rgb}{0.9, 0.15, 0}
\newcommand{\clr}[1]{\textcolor{red}{#1}}
\newcommand{\MK}[1]{\textcolor{MKcolor}{#1}}
\newcommand{\HRule}{\rule{\linewidth}{0.5mm}}
\renewcommand{\Re}{\mathop{\rm Re}\nolimits}
\newcommand{\hence}{\ \Rightarrow\ }
\newcommand{\figref}[1]{Fig.~\ref{fig:#1}}

\newcommand{\todo}[1]{\textcolor[RGB]{214,39,40}%
  {$\blacktriangleright$ #1 $\blacktriangleleft$}}
\newcommand{\needsref}{\textcolor[RGB]{255,127,14}{[REF!]}}
\newcommand{\maybe}[1]{\textcolor[RGB]{255,127,14}{#1}}

\newcommand{\tcb}[1]{\textcolor{black}{#1}}
\newcommand{\tcm}[1]{\textcolor{magenta}{#1}}

%\bibliographystyle{phaip}

\begin{document}

\title{Fundamental quantum limit achieved by internal squeezing in cavity-enhanced interferometric measurements}
\author{M. Korobko}
\email{mkorobko@physnet.uni-hamburg.de}
\affiliation{Institut f\"ur Laserphysik und Zentrum f\"ur Optische Quantentechnologien der Universit\"at Hamburg,\\%
Luruper Chaussee 149, 22761 Hamburg, Germany}
\author{J. S\"udbeck}
\affiliation{Institut f\"ur Laserphysik und Zentrum f\"ur Optische Quantentechnologien der Universit\"at Hamburg,\\%
Luruper Chaussee 149, 22761 Hamburg, Germany}
\author{S. Steinlechner}
\affiliation{ Faculty of Science and Engineering, Maastricht University, Duboisdomein 30, 6229 GT Maastricht, The Netherlands}
\affiliation{ Nikhef, Science Park 105, 1098 XG Amsterdam, The Netherlands }
\author{R. Schnabel}
\affiliation{Institut f\"ur Laserphysik und Zentrum f\"ur Optische Quantentechnologien der Universit\"at Hamburg,\\%
Luruper Chaussee 149, 22761 Hamburg, Germany}
\date{\today}
\begin{abstract}
  The quantum noise limited sensitivity of cavity-enhanced laser-interferometric displacement measurements depends on the photon flux in the interferometer arms integrated over the measuring time, the kind of quantum states used, and the decoherence rate. 
  Here, we complement the injection of externally generated squeezed states with an optimally tuned quantum squeeze operation inside the interferometer's cavity to reduce the impact of decoherence on sensitivity.
  We analytically derive the new, generalized fundamental sensitivity limit.
  Furthermore, we report on the experimental test of the new fundamental bound by combining for the first time externally generated squeezed light with an additional internal squeeze operation.
  We experimentally demonstrate the enhanced sensitivity independent on the detection loss.
  Our results apply to a broad range of quantum sensors, from table-top to large-scale, such as gravitational-wave detectors.
\end{abstract}

\maketitle
\textit{Introduction} Cavities increase the precision in various kinds of optical sensing devices: from biological\,\cite{Taylor2016} and medical\,\cite{Nolte2012} sensors, to accelerometers\,\cite{Krause2012}, ultra-precise magnetometers\,\cite{forstner2012} and gravitational-wave detectors\,\cite{GW150914}.
The purpose of the cavities is to resonantly enhance the power of the carrier light as well as the signal due to constructive interference over the many cavity round trips.
In quantum noise limited devices, the power of the carrier light is ideally maximized up to a value at which either the measured sample gets disturbed or the optical hardware of the sensor loses quality.
Further increase in sensitivity beyond this limit requires implementation of  squeezed states of light~\cite{Caves1981, Schnabel2017, Yuen1976}, which suppress the fluctuations in the observable to which the signal is coupled.
Squeezed states of light have been successfully implemented in various sensors: gravitational-wave detectors~\cite{LSC2011, Acernese2019, Tse2019, Yu2020, Lough2021}, optomechanical devices~\cite{Yap2020}, dark matter sensors~\cite{backesQuantumEnhancedSearch2021,Carney2021}, biological~\cite{Taylor2013, Taylor2016} and magnetometers~\cite{Li2018}.

Usually, squeezed states are injected into the detector from an external source. 
However, recently an \textit{internal} squeezing approach has also been investigated both theoretically\,\cite{Rehbein2005, V.Peano2015, korobko2018engineering, korobko2019quantum,adyaQuantumEnhancedKHz2020} and experimentally\,\cite{korobko2017beating}.
In this approach quantum noise squeezing is generated directly inside the detector cavity.
For this, a second-order nonlinear crystal is placed inside the cavity.
When pumped with second harmonic light, it amplifies a quadrature of the optical field selected by the pump phase.
The orthogonal quadrature is deamplified. 
If the signal corresponds to a displacement of the quadrature being squeezed, it gets deamplified.
Despite this, the signal-to-noise ratio (SNR) is still improved\,\cite{Rehbein2005, korobko2017beating}. 
Internal squeezing can be used in combination with externally injected squeezing.

The well-known Quantum Cramer-Rao Bound (QCRB) defines the best possible sensitivity of a device at every angular signal frequency $\Omega$ in the absence of decoherence and for a given optical power\,\cite{Braginsky2000, Tsang2011, Miao2017}. 
In reality, decoherence always prevents the QCRB from being reached.
It also affects the optimal strategy for interferometric sensing.
For instance, in the absence of loss, the best phase sensitivity of a Michelson interferometer at a given energy per measuring time would be achieved with a \textit{N00N state}, where $N$ is the deterministic total photon number\,\cite{Dowling2008}.
In practice, however, the fragility of this state towards decoherence quickly makes it suboptimal.
Instead, almost always the best strategy for reaching high sensitivity is the use of monochromatic light with the relevant signal sideband spectrum in a squeezed vacuum state~\cite{Demkowicz-Dobrzanski2013}.
The effect of decoherence on sensitivity has been investigated in the general metrological context of open quantum systems\,\cite{tsang2013quantum,alipour2014quantum,fiderer2019maximal}, but every specific system requires its own consideration.
For optical states, photon loss is arguably the most fundamental kind of decoherence.
Its effect on the sensitivity limit was explored in~\cite{Dorner2009,Demkowicz-Dobrzanski2013, Demkowicz-Dobrzanski2015, Miao2019}.
Ref.~\cite{Miao2019} analysed cavity-enhanced sensors with external and internal squeezing as well as intra-cavity loss and computed the fundamental sensitivity limit for sufficiently small losses as the sum of the QCRB and the noises added due to decoherence.

Here we show that the ultimate value of the fundamental limit surpasses the one presented in \cite{Miao2019}.
This is achieved by compensating a significant part of decoherence by optimizing the internal squeezing.
Our limit ultimately is defined solely by the losses inside the detector cavity.
We realize an experiment that supports the new fundamental limit by optimizing the internal squeeze factor for a given finite external squeeze factor.
We achieve an SNR enhancement of 4\,dB independent on the detection loss. 
This serves as the first demonstration of decoherence compensation for continuous signals, opening the path towards implementation of the approach in various metrological experiments.
We claim that the new fundamental sensitivity limit is the most general one for cavity-enhanced sensing devices such as gravitational-wave detectors.

\begin{figure}
  \includegraphics*[width=0.9\columnwidth]{setup_alt_3.png}
  \caption{
  Conceptual representation of a cavity-enhanced sensor with internal and external squeezing.
  The detector cavity is used to measure the displacement $x$ of the movable end mirror. 
  This displacement imprints a phase modulation on the light field, which is amplified by the cavity.
  Internal squeezing is generated by a nonlinear crystal inside the cavity by pumping it with second harmonic pump. 
  Depending on the phase of the pump, the signal is either amplified or deamplified. 
  External squeezing is generated independently and is injected into the detector cavity through a Faraday isolator (FI).
  The output signal with external and internal squeezing $S_x$ is detected on a phase-sensitive detector.
  The impact of the detection loss $\epsilon_{\rm det}$ can be reduced by optimal choice of the internal gain $q$, defined by the pump strength.
  Optimal internal squeeze operation allows to reach the fundamental limit, defined by internal loss $\epsilon_{\rm int}$.
  }\label{fig:1}
\end{figure}

\textit{Internal squeezing for enhanced sensitivity}
The simplest cavity-enhanced sensor is an optical cavity with a movable end mirror.
More complex devices can often be mapped into this model~\cite{Buonanno2003}.
The external force displaces the movable mirror, which creates a phase modulation signal on the reflected light field, which is detected with a balanced homodyne detector.
Our analysis considers both injected (quadrature) squeezed light as well as a squeeze operation inside the sensor cavity (Fig.\,\ref{fig:1}).
We focus on the shot-noise limited sensitivity, ignoring quantum back-action effects such as photon radiation pressure, since they can be circumvented by quantum back-action evasion techniques \cite{Kimble2001}.

Optical loss in the system influences the sensitivity in different ways, depending on whether the loss occurs (i) {\it before} the measurement interaction, i.e.~on the meter, (ii) {\it after} the measurement, i.e.~on the meter that carries the full information, or (iii) {\it during} the time when the meter is accumulating the signal.
In the first case, the loss occurs between the squeeze operation and injection in the cavity.
This loss sets an upper bound on the external squeeze factor, which of course should be as high as possible while maintaining the high purity of the state. 
In the second case, the loss occurs after the information has been imprinted on the meter, e.g.~due to an imperfect quantum efficiency of the photo-electric detection.
In this case the signal-to-noise ratio is reduced due to the additional vacuum being mixed in. 
If the signal can be parametrically amplified before the loss occurs, the signal-to-noise-ratio is maintained, and made more robust against optical loss that occurs after the amplification\,\cite{Caves1981}. 
The parametric gain should be as high as possible, which results in anti-squeezed quantum noise.
This was recently explored in other contexts\,\cite{colomboTimereversalbasedQuantumMetrology2022, manceauDetectionLossTolerant2017, ouEnhancementPhasemeasurementSensitivity2012, knyazevOvercomingInefficientDetection2019,frascellaOvercomingDetectionLoss2021}.
Case (iii) is central to this work and more complex.
The loss inside the cavity affects the measurement process itself, and the resulting signal-to-noise ratio is ultimately limited by this internal loss.
No consequent operation could improve that, thus the internal loss defines the fundamental limit.
Depending on the combination of the cavity internal loss, the detection loss on the out-coupled light and the injected squeeze factor, either internal noiseless parametric amplification or deamplification is beneficial.

We use the input-output formalism\,\cite{Caves1985a, Danilishin2012, Schumaker1985a} to derive the noise spectrum $S_{sn}(\Omega)$ and the power of signal transfer function $\mathcal{T}_x(\Omega)$, as well as the noise-to-signal ratio $S_x(\Omega)$.
They take the form:
\begin{align}
& S_{\rm sn}(\Omega) = 1 - \frac{1-\epsilon_{\rm det}}{(T_c + \epsilon_{\rm int} + q)^2 + \Omega^2}\times \nonumber\\ &\times \left[4 T_c q + \left(1-\beta^{-1}\right) \left( (T_c -\epsilon_{\rm int} - q)^2 + \Omega^2\right)\right],\\
&|\mathcal{T}_x(\Omega)|^2 = \frac{8 \pi P_c}{\hbar \lambda c}\times\frac{T_c (1-\epsilon_{\rm det})}{(T_c + \epsilon_{\rm int} + q)^2 + \Omega^2},\\
&S_x(\Omega) = \frac{S_{\rm sn}(\Omega)}{|\mathcal{T}_x^{\rm}(\Omega)|^2}\label{eq:snr},
\end{align}
where $T_c$ is the coupling mirror power transmission, $\epsilon_{\rm int}$ is the internal power loss, and $q$ is the roundtrip parametric power gain, $\epsilon_{\rm det}$ is the detection power loss, see Fig.\,\ref{fig:1}, $\beta^{-1} = e^{-2r_{\rm ext}}$ is the external squeezing, with $r_{\rm ext}$ the corresponding squeeze parameter\,\cite{Stoler1970},  $c$ is the speed of light, $\lambda$ is the central wavelength and $P_c$ is intra-cavity light power. 
In deriving these equations we used the single-mode approximation, where only one optical mode acquires the signal, and $\{T_c, \epsilon_{\rm int}, q\}\ll1$\,\cite{korobko2023long}. 
We fixed the average intra-cavity power $P_c$ and assumed no loss or depletion on the second harmonic pump power.

In order to simplify the conceptual explanation, we focus on the case of the peak sensitivity, which occurs at $\Omega=0$, and leave out the frequency dependence due to the cavity linewidth.
From the input-output relation in the lossless case we obtain the QCRB:
\begin{equation}
  S_{\rm QCRB}(0) = \frac{\hbar \lambda c}{8 \pi P_c} \times\frac{(T_c-q)^2}{\beta T_c},
\end{equation}
which turns to zero in the limit of infinite input squeezing $\beta\rightarrow\infty$ or at the (lossless) parametric threshold for internal squeezing $q = T_c$, resulting in the well-known theoretical limit of infinite SNR in lossless systems.

In \cite{Miao2019}, it was proposed that the loss-induced sensitivity limit is defined  by the sum of the QCRB for a lossless system and additional noise added due to the loss.
There it was suggested that it can be achieved when the internal gain reaches its threshold (assuming that internal loss is small, i.e. $\epsilon_{\rm int}\ll T_c$):
\begin{align}\label{eq:sth}
  &S_{\rm th}(0) \approx \frac{\hbar \lambda c}{8 \pi P_c} \times\frac{4 \left(T_c \epsilon_{\rm det} + \epsilon_{\rm int} \right)}{1-\epsilon_{\rm det}},\\
  &q = q^{\rm th} := T_c + \epsilon_{\rm int}.
\end{align}
However, this expression does not take into account the possibility to optimize internal squeezing.



Internal squeezing enhances the sensitivity in one of two ways.
(a) When the detection loss or the external squeezing are small, it deamplifies and squeezes the signal quadrature.
The deamplification factor of the signal is limited to 6\,dB, while the squeeze factor of the quantum noise can in principle approach infinity\,\cite{Milburn1981,Collett1984,korobko2017beating}, increasing the overall SNR.
The numbers hold for zero internal loss and at a pump power that corresponds to the optical oscillation threshold of the $\chi^{(2)}$ process.
The factors are different because the quantum noise enters the cavity through the coupling mirror while the signal is exclusively generated inside the cavity.
In practice, there is an optimal parametric pump power value below threshold, which depends on the nonzero optical loss value.\\
(b) When the detection loss or the external squeeze factor are high, the internal squeeze parameter has the opposite sign in order to amplify and anti-squeeze the signal quadrature. 
In this case, the impact of the detection loss is mitigated, as proposed in \cite{Caves1981}.
Here this amplification is realised inside the cavity, i.e.~already during the time when the signal is accumulated.


The optimal internal gain $q$ depends on the quantities $\epsilon_{\rm int}$, $T_c$, $\epsilon_{\rm det}$, and $\beta$. 
It is computed by minimizing the value of the sensitivity $S_x(0)$ in Eq.\,\eqref{eq:snr}, which result in the optimal sensitivity:
\begin{align}
  S^{\rm opt}_{1/\beta}(0) &=\frac{\hbar \lambda c}{8 \pi P_c} \times 4 \left(\epsilon_{\rm int} +  \frac{T_c \epsilon_{\rm det}}{\epsilon_{\rm det}\beta + (1-\epsilon_{\rm det})}\right),\label{eq:optimal}\\
  q^{\rm opt} &= T_c \left(1 - \frac{2 \epsilon_{\rm det}}{\beta (1-\epsilon_{\rm det}) - \epsilon_{\rm det}}\right) - \epsilon_{\rm int}.\label{eq:optimal_gain}
\end{align}
The expression for $S^{\rm opt}_{1/\beta}$ is strictly lower than the limit proposed in \cite{Miao2019} as given by Eq.\,\eqref{eq:sth} for non-zero detection loss.
Highly squeezed beam has significant photon power, therefore the  measurement bandwidth is assumed to be sufficiently small, such that the power over this bandwidth remains small.
The fundamental sensitivity limit of our work is achieved for infinite external squeezing, $\beta\rightarrow\infty$.
In this case the optimal internal gain maximally amplifies the signal quadrature, $q = -q_{\rm th}$, and the limit becomes defined solely by the internal loss:
\begin{equation}\label{eq:limit}
  S^{\rm lim}_0(0) = 4\epsilon_{\rm int} \times \frac{\hbar \lambda c}{8 \pi P_c}.
\end{equation}
The equation formalizes the main statement of our paper: for a fixed power in the carrier field, if the external squeeze factor approaches infinity (over a measurement bandwidth approaching zero), the noise-to-signal ratio becomes \textit{independent on the detection loss} and approaches zero when the cavity internal loss approaches zero.

As we discussed before, injection loss would cause the impurity of the external squeezing, and thus limit external squeeze parameter.
In this case the fundamental limit in Eq.\,\eqref{eq:limit} cannot be achieved, but for small values of loss the optimal sensitivity in Eq.\,\eqref{eq:optimal} is still attainable.
We derive the full model including the injection loss in Ref.\,\cite{korobko2023long}.

\begin{figure}[t]
  \includegraphics[width=\linewidth]{setup21.png}
  \caption{Schematic representation of the experimental setup.
  Phase modulation signal from the electro-optic modulator (EOM) was injected from the back of the detector cavity and was observed on the balanced homodyne detector by overlapping with local oscillator (LO). 
  The phase of the LO allowed to choose the quadrature of the light to observe.
  Pump light was used to create internal amplification in the detector cavity.
  The internal gain was adjusted by tuning the pump power with a variable optical attenuator (VOA).
  External squeezing was injected into the ISC through a Faraday Isolator (FI).
  The detection loss was adjusted with a VOA.
  }
  \label{fig:setup_exp}
\end{figure}

\textit{Experimental validation}. We demonstrated the compensation of the detection loss and the existence of optimal internal squeezing in a table-top experimental setup, see \figref{setup_exp}.
Our internal squeezing cavity (ISC) was a Fabry-Perot cavity with a nonlinear periodically poled KTP crystal inside acting as an optical-parametric amplifier.
Depending on the power of the second harmonic pump the gain of the internal squeezing $q$ was varied.
The phase of the pump was actively controlled to keep the amplification phase stable.
\begin{figure}[ht!]
  \includegraphics[width=\linewidth]{SNR_gain_mod3.png}
  \caption{Experimental demonstration of optimal sensitivity with internal and external squeezing approaching the optimal limit in Eq.\,\eqref{eq:optimal}.
  Each plot shows the improvement in the SNR with respect to the case without internal and with external squeezing.
  This improvement was observed as a function of internal gain in the detector cavity: negative gain means deamplification (squeezing), positive -- amplification.
  Solid curves are the theoretical predictions based on the independently measured parameters.
  Plots (a-c) demonstrate the different regimes of internal squeezing for different level of external squeezing: 6.5\,dB, 12\,dB and 20\,dB (values inferred at production).
  For low values of external squeezing, peak SNR is achieved when squeezing is generated also internally.
  For high values of external squeezing it becomes optimal to amplify the signal quadrature.
  Plots (c-e) demonstrate the independence of the peak SNR enhancement on the detection loss.
  The enhancement of $\sim$4\,dB is achieved for 10\%, 20\% and 30\% of detection loss, for different levels of internal gain. 
  In all data sets, for the case of high squeezing (close to gain equal to $-1$), the effect of phase noise played a significant role: due to the jitter in the phase of injected squeezing, some part of anti-squeezed noise coupled into the readout quadrature, which further degraded the sensitivity (seen as significant nonlinearity of the curve close to gain equal to $-1$).
  The error bars on the experimental data are not shown, see the discussion in the main text.
  }
  \label{fig:full_exp}
\end{figure}
The cavity output field was analyzed with a balanced-homodyne detector (BHD).
The phase of the BHD's local oscillator was actively controlled to keep the readout quadrature stable.
We injected a weak field carrying a 5 MHz phase modulation signal from the back of the ISC, which emulated the measurement signal.
Depending on the phase of the pump, we could observe amplification or deamplification of the signal, as well as anti-squeezing or squeezing of the noise.
By taking the spectrum of the signal and the noise, we could compute the change in the SNR compared to the case when the pump was off.
Further, we injected external squeezing from a second squeeze laser\,\cite{Schnabel2017}.
We kept the external squeeze field without any bright carrier field and periodically varied its phase.
On the output of the homodyne detector, we recorded the spectrogram (spectrum as a function of time) of the observed signal.
As a result, we consecutively measured two orthogonal quadratures of injected squeeze field, and used it to extract optical parameters of the setup:
the transmission of the incoupling mirror $T_c = 11$\%, the internal loss $\epsilon_{\rm int} = 1.2$\%, the injection loss of 8\% and the detection loss $\epsilon_{\rm det} = 10$\%\,\cite{Vahlbruch2016}.
Initial injected external squeezing was inferred to be 20\,dB (before injection and detection loss). 
Phase noise was inferred from the measurements with different pump strength to be around 50 mrad.
We observed less phase noise for smaller values of external squeezing: 40 mrad for 12\,dB and 15 mrad for 6.5\,dB. 
We conclude that the main contribution came from the phase noise of the external squeezing interacting with the internal squeezing process.

We changed the internal squeezing gain and recorded squeezing spectra together with amplified or de-amplified signal.
In the first stage of the experiment, we gradually increased the injected squeezing value from 6.5\,dB to 20\,dB.
When both the detection loss and the injected squeezing were small, the optimal internal gain was a squeezing process, see Fig.\ref{fig:full_exp}(a).
As we increased injected squeezing, the optimal internal gain approached zero, see Fig.\ref{fig:full_exp}(b), and then it became optimal to amplify inside the detector cavity, Fig.\ref{fig:full_exp}(c).

We further artificially increased the detection loss from 10\% to 30\%, by dumping part of the squeezed light on a polarizing beam splitter.
By taking the full range from maximal deamplification to maximal amplification we could find the optimal point where the SNR was the highest.
We saw maximal SNR improvement of ~4\,dB, independent of the detection loss, see Fig.\,\ref{fig:full_exp}(c-e).
This way, we were able to demonstrate for the first time the compensation of quantum decoherence with optimal choice of internal gain.
In the case of 10\% of detection loss, the optimal internal gain was close to zero. 
For higher loss, as expected from our theory in Eq.\,\eqref{eq:optimal_gain}, it became optimal to \textit{amplify} the signal inside the cavity.
Our results show good match with theory and demonstrate the significant enhancement over the case of maximal intra-cavity squeezing, which was considered in\,\cite{Miao2019}.

We could not deduce meaningful error bars for Fig.\ref{fig:full_exp}. 
Most of the source data were averaged directly on the spectrum analyser, which did not allow to extract variances.
However, we used the theoretical description together with independently measured parameters of our setup to calculate the theoretical curves in the graphs.
All data points are statistically independent. 
Therefore, the good match between the data and the theory allows to be confident in the significance of our observed results in Fig.\ref{fig:full_exp}.

\textit{Discussion and outlook}.
Our result can be placed in more general context of computing the impact of the purity on the QCRB\,\cite{tsang2013quantum,alipour2014quantum,fiderer2019maximal}, which has not been done for cavity-enhanced sensors.
While we do not derive the QCRB for our setup from first principles, our argument follows the same spirit as Refs.\,\cite{tsang2013quantum,alipour2014quantum,fiderer2019maximal}.
There are two general conditions for achieving QCRB: (i) the detector should be in a pure state, and (ii) the back-action of the meter should be evaded\,\cite{Tsang2011,Miao2017}.
If the state is not pure, it not only prevents achieving arbitrary low uncertainty in one quadrature, but also prohibits efficient back-action evasion, which relies on quantum correlations between the two quadratures.
Therefore the main limitation on the achievable sensitivity is defined by the purity of the state upon interaction between the meter and the object.
This is directly manifested in our fundamental limit in Eq.\,\eqref{eq:limit}, via its proportionality to cavity internal loss.

We also note that the output amplification proposed in Ref.\,\cite{Caves1981} also evades the detection loss. 
Compared to the internal squeezing, it amplifies the signal \textit{after} it has exited the detector cavity.
Such an approach could also yield high sensitivity approaching the fundamental limit.
However, in some applications, like gravitational-wave detection, the main source of the detection loss would occur between the interferometer and the output amplifier.
In other cases, like chip-based sensors, propagation and coupling losses could be dominating.
In these cases implementing output amplification might be in fact not beneficial or challenging, while internal squeezing provides a natural way to use the least lossy part of the system -- the detector cavity itself.
In the supplementary paper\,\cite{korobko2023long} we provide more details on the comparison of the two approaches.

Our results are readily applicable to quantum-metrological devices that are limited by quantum shot noise, and whose principle schemes can be mapped to a single cavity. 
It is especially promising for the cavities that naturally have nonlinear materials in them, such as on-chip devices\,\cite{Ramelow2019, Strekalov2016}, or whispering-gallery-mode sensors\,\cite{93a1BrGoIl, 94a1GoIl, Foreman2015, Schliesser2010}.
For these devices squeezing injection might be challenging, and the readout is often subject to significant losses.
Then internal squeezing can become a useful tool for compensating these losses and achieving further quantum improvement to the sensitivity. 
Even in the systems with several cavities internal squeezing allows to enhance the sensitivity, by quantum-expanding the linewidth\,\cite{korobko2019quantum,adyaQuantumEnhancedKHz2020} or creating PT-symmetric configurations\,\cite{dmitrievEnhancingSensitivityInterferometers2022, gardner2022nondegenerate}. 
Our work contributes to the detailed understanding of the limits on quantum metrological experiments in such devices, and enables a wider range of acceptable losses in the system.

\textit{Acknowledgements} This work was supported by the Deutsche Forschungsgemeinschaft (DFG) under Germany's Excellence Strategy EXC 2121 ``Quantum Universe''-390833306.

\bibliography{bibliographyCompensating}

\end{document}
