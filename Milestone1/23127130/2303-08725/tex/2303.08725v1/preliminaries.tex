\section{Preliminaries}\label{sec:preliminares}


\subsection{Sequences, alignments, and scoring functions}

% alfabeto, sequencias, etc.
An \emph{alphabet} is a finite non-empty set of symbols. A \emph{sequence} over an alphabet $\Sigma$ is a finite ordered list of symbols from $\Sigma$ and the set of all sequences over $\Sigma$ is denoted by $\Sigma^*$.
We denote a sequence $s \in \Sigma^*$ by $s = s(1)s(2) \cdots s(n)$, where $s(i) \in \Sigma$ is the $i$-th symbol of $s$ and $n = \abs{s}$ is the
\emph{length} of $s$. We denote by $\seqVazia$ the sequence with zero symbols, also called the \emph{empty sequence}. If $s$ and $t$ are sequences over $\Sigma$, then $st$ is a sequence that represents the \emph{concatenation} of $s$ and $t$. The concatenation of $n$ copies of $s$ has length $n|s|$ and is denoted by $s^{n}$. 

% o s\'imbolo $\espaco$ and alignment
Let $\alphabet_ {\espaco} = \alphabet \cup \{\espaco\}$, with $\espaco \not\in \alphabet$. The symbol $\espaco$ is a \emph{space} and represents an insertion or a deletion. An \emph{alignment} of sequences $s, t \in \Sigma^*$ is a pair of sequences $\alinhamento{s', t'}$ obtained by inserting spaces into $s$ and $t$ such that $\tamanho{s'} = \tamanho{t'}$ and there is no $j$ such that $s'(j) = t'(j) = \espaco$. An alignment can be seen as a sequence of edit operations that transforms $s$ into $t$ when each symbol in $s$ must be edited precisely once. We say that the pair of symbols $[s'(j), t'(j)]$ is \emph{aligned} in the $j$-th column of $ \alinhamento {s', t'}$. We also say that $\tamanho{\alinhamento{s', t'}} = \tamanho{s'} = \tamanho{t'}$ is the \emph{length} of alignment $\alinhamento{s', t'}$. We denote the set of all alignments of $s, t$ by $ \cjtoAlinha{s, t}$.
% exemplo de alinhamento
An alignment $\alinhamento{s', t'}$ can be seen by placing $s'$ above $t'$ as follows 
\[
%\small
\alinhamentoA{
\begin{array}{cccccccccc}
\A & \C & \espaco & \C & \B & \espaco & \B & \B & \B & \espaco\\
\C & \espaco & \A & \espaco & \A & \C & \espaco & \espaco & \C & \B
\end{array}
} \text{and} 
\alinhamentoA{
\begin{array}{ccccccccccc}
\espaco & \espaco & \espaco & \espaco & 
\A & \C & \C & \B & \B & \B & \B \\
\C & \A & \A & \C & \C & \B & 
\espaco & \espaco & \espaco & \espaco & \espaco
\end{array}
}
\]
where two different alignments of sequences $\A \C \C \B \B \B \B, \C \A \A \C \C \B$ are represented: the alignment $\alinhamento{\A \C \espaco \C \B \espaco \B \B \B \espaco, \C \espaco \A \espaco \A \C \espaco \espaco \C \B}$ at the left side and the alignment $\alinhamento{\espaco \espaco \espaco \espaco \A \C \C \B \B \B \B, \C \A \A \C \C \B \espaco \espaco \espaco \espaco \espaco}$ at the right.

% custo convencional e normalizado de um alinhamento

Given a scoring matrix $\matriz$, we define the functions $\custoA{\matriz}$ and $\custoN{\matriz}$ for each alignment $\alinhamento{s', t'}$:  if $s = t = \seqVazia$, $\custoA{\matriz} \alinhamento{s', t'} = \custoN{\matriz} \alinhamento{s', t'} = 0$; otherwise,
\[
\custoA{\matriz} \alinhamento{s', t'} = 
\sum_{j = 1}^{\tamanho{\alinhamento{s', t'}}} 
\pont{\matriz}{s'(j)}{t'(j)} 
\quad \text{and} \quad
\custoN{\matriz} \alinhamento{s', t'} = 
\frac{\sum_{j = 1}^{\tamanho{\alinhamento{s', t'}}} 
\pont{\matriz}{s'(j)}{t'(j)}}{\tamanho{\alinhamento{s', t'}}}\,.
\]
We say that $\custoA{\matriz}[A]$ is the \emph{score} and $\custoN{\matriz}[A]$ is the \emph{normalized score} of alignment $A$.


An \emph{extended alignment} of the sequences $s, t$ is shown as $\alinhamento{c_{1}, \ldots, c_{n}}$, where each $c_{j}$ is a finite sequence with $m_{j} > 0$ symbols in $\alphabet_{\espaco}$, at least one symbol is different from ``$\espaco$'' but no two consecutive symbols are equal to ``$\espaco$'', and $\alinhamento{c_{1}(1) \ldots c_{n}(1), c_{1}(m_{1}) \ldots c_{n}(m_{n})}$ is an alignment of $s, t$. For the sake of simplicity, we say ``alignment'' instead of ``extended alignment'' when it is clear from the context. We say that $c_{j}$ is the $j$-th \emph{column} and $n = \tamanho{\alinhamento{c_{1}, \ldots, c_{n}}}$ is the \emph{length} of the alignment $\alinhamento{c_{1}, \ldots, c_{n}}$. 

Thus, each sequence $c_j$ is represented by a column and it is written from the top to the bottom. For example, $A = \alinhamento{\A, \A\B\C\D, \B\espaco, \espaco\D, \C\A, \espaco\D\A\B\C, \espaco\A\B\C, \D\A\B}$ is an extended alignment of $\A\A\B\C\D, \A\D\D\A\C\C\B$ and it can be represented as
\[
%\small
A = \alinhamentoA{
\begin{array}{c}
\A
\end{array} 
\begin{array}{c}
\A \\ \B \\ \C \\ \D
\end{array} 
\begin{array}{c}
\B\\ \espaco
\end{array} 
\begin{array}{c}
\espaco\\ \D
\end{array} 
\begin{array}{c}
\C \\ \A
\end{array} 
\begin{array}{c}
\espaco\\ \D\\ \A\\ \B\\ \C
\end{array} 
\begin{array}{c}
\espaco\\ \A\\ \B\\ \C
\end{array} 
\begin{array}{c}
\D\\ \A \\ \B
\end{array}}.
\]

Notice that, since $\alinhamento{c_{1}(1) \ldots c_{n}(1), c_{1}(m_{1}) \ldots c_{n}(m_{n})}$ is an alignment of $s, t$, we have $c_j(1) \not= \espaco$ or $c_j(m_j) \not= \espaco$ for each $j$.

% custo de um alinhamento estendido
The \emph{weight} of $c_{j}$ is given by $\custoE{\matriz} [c_{j}] = \sum_{i=1}^{m_{j}-1} \pont{\matriz}{c_{j}(i)}{c_{j}(i+1)}$ and the \emph{score} of the extended alignment $A = \alinhamento{c_{1}, \ldots, c_{n}}$ is
\[
\custoE{\matriz}[A] = \sum_{j} \custoE{\matriz} [c_{j}]\,.
\]

Considering the alignment $A = \alinhamento{\A, \A\B\C\D, \B\espaco, \espaco\D, \C\A, \espaco\D\A\B\C, \espaco\A\B\C, \D\A\B}$ above and that $\pont{\matriz}{\A}{\B} = 0$ if $\A = \B$ and $\pont{\matriz}{\A}{\B} = 1$ if $\A \neq \B$, its score is
\begin{align*}
\custoE{\matriz}[A] &=  \sum_{j} \custoE{\matriz} [c_{j}]\\
&= \custoE{\matriz}[\A] + \custoE{\matriz}[\A\B\C\D]  + \custoE{\matriz}[\B\espaco] + \custoE{\matriz}[\espaco\D]  + \\ 
& \phantom{~=} \custoE{\matriz}[\C\A] + \custoE{\matriz}[\espaco\D\A\B\C]  + \custoE{\matriz}[\espaco\A\B\C]  + \custoE{\matriz}[\D\A\B]\\
&= (0) + (1+1+1) + (1) + (1) + \\
&\phantom{~=} (1) + (1+ 1+ 1 + 1) + (1 + 1 + 1) + (1 + 1) = 15\,.\\
\end{align*}

We denote the set of all extended alignments of $s, t$ by $\cjtoAlinhaExt{s, t}$.

\medskip

% funcoes otimas
We also define the functions $\distanciaA{\matriz},
\distanciaN{\matriz}$ and $\distanciaE{\matriz}$ as follows:
\begin{align*}
\distanciaA{\matriz}(s, t) &= \min_{A \in \cjtoAlinha{s, t}}
\custoA{\matriz}[A]\,, \\
\distanciaN{\matriz}(s, t) &= \min_{A \in \cjtoAlinha{s, t}}
\custoN{\matriz}[A]\,, \\
\distanciaE{\matriz}(s, t) &= \min_{A \in \cjtoAlinhaExt{s, t}}
\custoE{\matriz}[A]\,.
\end{align*}

If $A$ is an alignment and $\custoA{\matriz}[A] =
\distanciaA{\matriz}(s, t)$ or $\custoN{\matriz}[A] =
\distanciaN{\matriz}(s, t)$, we say that $A$ is an \emph{A-optimal} or
\emph{N-optimal} alignment for $\matriz$, respectively. Similarly, if
$A$ is an extended alignment and $\custoE{\matriz}[A] =
\distanciaE{\matriz}(s, t)$, we say that $A$ is an \emph{E-optimal}
alignment for $\matriz$.


\subsection{Weighted digraphs of scoring matrices}

% digrafos de matrizes de pontua\c{c}\~ao
We can represent a scoring matrix as a weighted digraph. Thus, the
weighted digraph for the scoring matrix $\matriz$ is
\[
D(\matriz) = (\alphabet_{\espaco}, ( \alphabet_{\espaco} \times
\alphabet_{\espaco}) \setminus \{ (\espaco, \espaco) \}, \matriz)\,,
\]
where the weight of arc $\A \rightarrow \B$ is $\pont{\matriz}{\A}{\B}$.
A \emph{walk} from vertex $\A_0$ to $\A_n$ in
$D(\gamma)$ is a sequence $W = \A_0, \A_1, \ldots, \A_n$, $\A_i \in \Sigma_{\espaco}$
such that 
$\A_i = \espaco$ implies $\A_{i-1} \not= \espaco$ for each $i = 1, \ldots, n$, and its \emph{weight} is
$\cost(W) = \sum_{i=1}^n \pont{\gamma}{\A_{i-1}}{\A_i}$.
The walk $W$ is also called
\emph{cycle} if $\A_0 = \A_n$ and in this case we can assume that 
$\A_0 = \A_n \not= \espaco$.
If $\cost(W)$ is the minimum weight of any walk from $\A_0$ to $\A_n$, we say that $W$ is an
\emph{optimal walk} from $\A_0$ to $\A_n$.
The weight of an optimal walk from $\A_0$ to $\A_n$ is denoted by $d_{\gamma}(\A_0, \A_n)$
or simply $d (\A_0, \A_n)$ when $\gamma$ is clear in the context.


\subsection{Properties of metric functions}


% definicao de metrica e de espaco metrico
For a given set $S$, we say that a function $f: S \times S \rightarrow
\mathbb{R}$ is a \emph{metric} on $S$ if $f$ satisfies the following
properties for each $x, y, z \in S$:
\begin{enumerate}
\item[(\textit{i})] $f(x, x) = 0$ (\emph{reflexivity})\,,
\item[(\textit{ii})] $f(x, y) \ge 0$ (\emph{non-negativity})\,,
\item[(\textit{iii})] $f(x, y) > 0$ if $x \not= y$ (\emph{positivity})\,,
\item[(\textit{iv})] $f(x, y) = f(y, x)$ (\emph{symmetry})\,, and
\item[(\textit{v})] $f(x, z) \le f(x, y) + f(y, z)$ (\emph{triangle
  inequality})\,.
\end{enumerate}
Observe that (\textit{ii}) is a direct consequence of (\textit{i}) and (\textit{iii}). Additionally, when the set $S$ is clear from the context, we simply say that $f$ is a metric.

% definicao de espacos mais gerais
In order to obtain more general functions than metrics, we can relax
the aforementioned properties. For example, for a given set
$S$, we say that $f$ is a \emph{premetric} if $f(x, x) = 0$ and $f(x,
y) \ge 0$ for each ${x, y \in S}$~\citep{Topologia90}.

The table below shows the properties of the following functions:
\emph{premetric} ($\prametrica$), \emph{semimetric}
($\semimetrica$)~\citep{Wilson31-semi}, \emph{hemimetric} or
\emph{pseudoquasimetric} ($\hemimetrica$)~\citep{pseudoQuasi1968},
\emph{pseudometric} ($\pseudometrica$)~\citep{pseudo-1970}, \emph{quasimetric}
($\quasimetrica$)~\citep{Wilson31-quasi} and \emph{metric}
($\metrica$)~\citep{pseudo-1970}. Figure~\ref{figura-espacos} shows
the relationships between these functions.

\begin{center}
\begin{tabular}{lccccccc}
property $\backslash$ function & $\prametrica$ & $\semimetrica$ &
$\hemimetrica$ & $\pseudometrica$ & $\quasimetrica$ & $\metrica$
\\ \hline $f(x, x) = 0, f(x, y) \ge 0$ & yes & yes & yes & yes & yes &
yes \\ $f(x, y) > 0$ for $x \not= y$ & & yes & & & yes & yes \\ $f(x,
y) = f(y, x)$ & & yes & & yes & & yes \\ $f(x, y) \le f(x, z) + f(z,
y)$ & & & yes & yes & yes & yes
\end{tabular}
\end{center}

\begin{figure}[hbt]
  \begin{center}
    \includegraphics[scale=.5]{conjuntos}
    \caption{Relationships between functions. Shaded area represents
      the set of metric functions, which is the intersection of all
      the sets.}
    \label{figura-espacos}
  \end{center}
\end{figure}

% induz 
If a function $\distanciaA{\matriz}, \distanciaN{\matriz}$ or
$\distanciaE{\matriz}$ has a property $p$, we say that the \emph{scoring matrix $\matriz$
induces $\distanciaA{\matriz}\text{-}p, \distanciaN{\matriz}\text{-}p$, or
$\distanciaE{\matriz}\text{-}p$ on sequences}.

% classes de matrizes que induzem metrica em sequencias
A standard class of scoring matrices is $\metricaC$, which has the following properties.
For $\A, \B, \C, \in \alphabet_ {\espaco}$,
\begin{enumerate}
\item[(\textit{i})] $\pont{\matriz}{\A}{\B} > 0$ if $\A \not= \B$,
  and $\pont{\matriz}{\A}{\B} = 0$ if $\A = \B$\,,
\item[(\textit{ii})] $\pont{\matriz}{\A}{\B} =
  \pont{\matriz}{\B}{\A}$\,, and
\item[(\textit{iii})] $\pont{\matriz}{\A}{\C} \le
  \pont{\matriz}{\A}{\B} + \pont{\matriz}{\A}{\C}$\,.
\end{enumerate}
\citet{Sellers1974} showed that scoring matrices in $\metricaC$ induce
an $\distanciaA{\matriz}$-metric on sequences.
The class of scoring matrices $\metricaA$ is such that, for each $\A, \B, \C
\in \alphabet$,
\begin{enumerate}
\item[(\textit{i})] $\pont{\matriz}{\A}{\espaco} =
  \pont{\matriz}{\espaco}{\A} > 0$\,,
\item[(\textit{ii})] $\pont{\matriz}{\A}{\B} > 0$ if $\A \not= \B$,
  and $\pont{\matriz}{\A}{\B} = 0$ if $\A = \B$\,,
\item[(\textit{iii})] if $\pont{\matriz}{\A}{\B} <
  \pont{\matriz}{\A}{\espaco} + \pont {\matriz}{\espaco}{\B}$, then
  $\pont{\matriz}{\A}{\B} = \pont{\matriz}{\B}{\A}$\,,
\item[(\textit{iv})] $\pont{\matriz}{\A}{\espaco} \le
  \pont{\matriz}{\A}{\B} + \pont{\matriz}{\B}{\espaco}$\,, and
\item[(\textit{v})] $\min \{ \pont{\matriz}{\A}{\C},
  \pont{\matriz}{\A}{\espaco} + \pont {\matriz}{\espaco}{\C} \} \le
  \pont{\matriz}{\A}{\B} + \pont{\matriz}{\B}{\C}$\,,
\end{enumerate}
and we show that they contain precisely all
the matrices that induce $\distanciaA{\matriz}$-metric on sequences.
Moreover, \citet{MarzalV1993} mentioned that not every matrix in
$\metricaC$ induces $\distanciaN{\matriz}$-metric on sequences. They showed,
for example, the matrix
\[
\begin{array}{c|ccc}
\gamma & \A & \B & \espaco \\ \hline
\A & 0 & 5 & 5 \\
\B & 5 & 0 & 1 \\
\espaco & 5 & 1
\end{array}
\]
that belongs to $\metricaC$ but 
does not induce $\distanciaN{\matriz}$-triangle inequality since
\begin{align*}
\distanciaN{\matriz} (\A, \B) &= 
\min \left\{ \custoN{\matriz} \alinhamentoB{
\begin{array}{cc}
\A & \espaco\\
\espaco & \B
\end{array}
} ,\; 
\custoN{\matriz} \alinhamentoB{ 
\begin{array}{cc}
\espaco & \A \\
\B & \espaco 
\end{array}
} ,\; 
\custoN{\matriz} \alinhamentoB{ 
\begin{array}{c}
\A\\
\B
\end{array}
}  
\right\} = 3 = \frac{18}{6}\\
&> \frac{17}{6} = \frac{1}{2} + \frac{7}{3} = \custoN{\matriz}  \alinhamentoB{ 
\begin{array}{cc}
\A & \espaco\\
\A & \B
\end{array}
}  + 
\custoN{\matriz} \alinhamentoB{ 
\begin{array}{ccc}
\A & \B & \espaco\\
\espaco & \espaco & \B
\end{array}
} \\
&\ge 
\distanciaN{\matriz}(\A, \A \B) + \distanciaN{\matriz}(\A\B, \B)\,.
\end{align*}
\citet{YujianB2007} pointed out that it was an
open question whether a scoring matrix $\matriz$ induces $\distanciaN{\matriz}$-metric. Nevertheless, we show that matrices in $\metricaN$ induce
$\distanciaN{\matriz}$-metric on sequences. The class of matrices $\metricaN$ is such that
\begin{enumerate}
\item[(\textit{i})] $\metricaN \subseteq \metricaA$\,, and
\item[(\textit{ii})] $\pont{\matriz}{\A}{\espaco} \le
  2\,\pont{\matriz}{\B}{\espaco}$ for each $\A, \B \in \alphabet$\,.
\end{enumerate}  
Discussion above shows that $\metricaC \not\subseteq \metricaN$ and 
we can easily check that scoring matrices $\gamma_1 \in \metricaN \setminus \metricaC$
and $\gamma_2 \in \metricaN \cap \metricaC$, where
\[
\begin{array}{c|ccc}
\gamma_1 & \A & \B & \espaco \\ \hline
\A & 0 & 4 & 1 \\
\B & 3 & 0 & 1 \\
\espaco & 1 & 1
\end{array}
\hspace{1cm}
\mbox{and}
\hspace{1cm}
\begin{array}{c|ccc}
\gamma_2 & \A & \B & \espaco \\ \hline
\A & 0 & 1 & 1 \\
\B & 1 & 0 & 1 \\
\espaco & 1 & 1
\end{array}
\]
which implies that $\metricaN \not\subseteq \metricaC$
and $\metricaC \cap \metricaN \not= \emptyset$.


A third class of scoring matrices $\gamma$ we study is $\metricaE$:
\begin{enumerate}
\item[(\textit{i})] $\pont{\matriz}{\A}{\A} \ge 0$ for each $\A \in
  \alphabet$\,,
\item[(\textit{ii})] $\pont{\matriz}{\A}{\B},
  \pont{\matriz}{\A}{\espaco}, \pont{\matriz}{\espaco}{\A} > 0$ for
  each $\A \not= \B$, $\A, \B \in \alphabet$\,, and
\item[(\textit{iii})] $d_{\matriz}(\A, \B) = d_{\matriz}(\B, \A)$ for each $\A,
  \B \in \Sigma_{\espaco}$\,.
\end{enumerate}

\begin{fact}\label{fato-preliminares}\rm
$\metricaA \subseteq \metricaE$.
\end{fact}

\begin{proof}
  Let $\matriz \in \metricaA$.
  It follows that $\pont{\matriz}{\A}{\espaco} =
  \pont{\matriz}{\espaco}{\A} > 0$, $\pont{\matriz}{\A}{\B} > 0$ if $\A \not= \B$,
  $\pont{\matriz}{\A}{\A} = 0$, 
  $\pont{\matriz}{\A}{\B} = \pont{\matriz}{\B}{\A}$ if $\pont{\matriz}{\A}{\B} <
  \pont{\matriz}{\A}{\espaco} + \pont {\matriz}{\espaco}{\B}$,
  $\pont{\matriz}{\A}{\espaco} \le
  \pont{\matriz}{\A}{\B} + \pont{\matriz}{\B}{\espaco}$, and 
  $\min \{ \pont{\matriz}{\A}{\C},
  \pont{\matriz}{\A}{\espaco} + \pont {\matriz}{\espaco}{\C} \} \le
  \pont{\matriz}{\A}{\B} + \pont{\matriz}{\B}{\C}$ for $\A, \B, \C \in \Sigma$.

  Since $\pont{\matriz}{\A}{\A} = 0$, $\pont{\matriz}{\A}{\espaco} =
  \pont{\matriz}{\espaco}{\A} > 0$, and $\pont{\matriz}{\A}{\B} > 0$ if $\A \not= \B$,
  we have that $\gamma$ satisfies properties
  (\textit{i}) and (\textit{ii}) of $\metricaE$.
  


Let $W = u_{0}, u_{1}, \ldots, u_{n}$ be an optimal walk
from $\A = u_0$ to $\B = u_n$ in
$D(\matriz)$, and $W_i = u_{i-1}, u_i$. We construct a walk $\overline{W}_{i}$ from $u_{i}$ to
$u_{i-1}$ for each $i = 1, \ldots, n$, in accordance with the cases
below.
\begin{description}
\item[Case 1:] $u_{i-1} = \espaco$ or $u_{i} = \espaco$. Put
  $\overline{W}_{i} = u_{i}, u_{i-1}$.
  Since
  $\pont{\gamma}{u_i}{\espaco} = \pont{\gamma}{\espaco}{u_i}$ and
  $\pont{\gamma}{u_{i-1}}{\espaco} = \pont{\gamma}{\espaco}{u_{i-1}}$,
  it follows that
  $\cost(\overline{W}_{i}) = \pont{\matriz}{u_{i}}{u_{i-1}} =
  \pont{\matriz}{u_{i-1}}{u_{i}} = \cost(W_{i})$\,;
\item[Case 2:] $u_{i-1} \not= \espaco$, $u_{i} \not= \espaco$ and
  $\pont{\matriz}{u_{i-1}}{u_{i}} < \pont{\matriz}{u_{i-1}}{\espaco} +
  \pont{\matriz}{\espaco}{u_{i}}$. Put $\overline{W}_{i} =
  u_{i}, u_{i-1}$.
  Since $\pont{\matriz}{u_{i-1}}{u_{i}} < \pont{\matriz}{u_{i-1}}{\espaco} +
  \pont{\matriz}{\espaco}{u_{i}}$, 
  it follows that $\cost(\overline{W}_{i}) =
  \pont{\matriz}{u_{i}}{u_{i-1}} = \pont{\matriz}{u_{i-1}}{u_{i}} = \cost(W_{i})$\,;
\item[Case 3:] $u_{i-1} \not= \espaco$, $u_{i} \not= \espaco$ and
  $\pont{\matriz}{u_{i-1}}{u_{i}} = \pont{\matriz}{u_{i-1}}{\espaco} +
  \pont{\matriz}{\espaco}{u_{i}}$. Put $\overline{W}_{i} = u_{i}, \espaco, u_{i-1}$.
  Since
  $\pont{\gamma}{u_i}{\espaco} = \pont{\gamma}{\espaco}{u_i}$ and
  $\pont{\gamma}{u_{i-1}}{\espaco} = \pont{\gamma}{\espaco}{u_{i-1}}$,
  it follows that
  $\cost(\overline{W}_i) = \pont{\matriz}{u_{i}}{\espaco} +
  \pont{\matriz}{\espaco}{u_{i-1}} = \pont{\matriz}{u_{i-1}}{\espaco}
  + \pont{\matriz}{\espaco}{u_{i}} =
  \pont{\matriz}{u_{i-1}}{u_{i}} = \cost(W_{i})$\,.
\end{description}

Let $\overline{W}$ be a walk formed by concatenating 
$\overline{W}_{n}, \overline{W}_{n-1}, \ldots, \overline{W}_{1}$.
From the above observations, it follows that
\begin{eqnarray*}
d (u_{0}, u_{n}) = \cost(W) & = &
\sum_{i=1}^{n} \cost(W_{i})\\ & = & 
\sum_{i=1}^{n} \cost(\overline{W}_{i}) =
\cost(\overline{W}) \ge d(u_{n},u_{0})\,.
\end{eqnarray*}
Using similar reasoning, we have $d(u_{n},u_{0}) \ge 
d(u_{0}, u_{n})$.
It follows that 
$d(u_{n},u_{0}) = d(u_{0}, u_{n})$, which implies that $\gamma$ satisfies 
  property (\textit{iii}) of $\metricaE$.

Since $\gamma$ satisfies properties (\textit{i}), (\textit{ii}), and (\textit{iii}),
we have that $\gamma \in \metricaC$.
\end{proof}

It follows from definitions and Fact~\ref{fato-preliminares} that
$\metricaC \subseteq \metricaA \subseteq \metricaE$, $\metricaN
\subseteq \metricaA$, $\metricaC \not\subseteq \metricaN$,
$\metricaN \not\subseteq \metricaC$, and $\metricaC \cap \metricaN \not= \emptyset$. See
Figure~\ref{figura-conjuntos}.

\begin{figure}[hbt]
  \begin{center}
    \includegraphics[scale=.6]{conjuntos2}
    \caption{Relationships between classes $\metricaE$, $\metricaA$,
      $\metricaN$, and $\metricaC$.}
    \label{figura-conjuntos}
  \end{center}
\end{figure}
