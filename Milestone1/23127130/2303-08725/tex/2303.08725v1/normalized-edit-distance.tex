\section{Normalized edit distance}\label{sec:normalizado}

In this section, we describe the classes of scoring matrices that
induce $\distanciaN{\matriz}$-$p$ on sequences for each axiom $p$ of a metric
when $\distanciaN{\matriz} \in \prametrica$, as can be seen in Lemmas~\ref{lemaPrametricaN}, \ref{lema-maiorQueZero}, \ref{distanciaNSimetrica}, and \ref{distanciaNTriangular}, and summarized in Table~\ref{tabela3}. They allow us to characterize matrices that induces each of the more general metric functions described in Section~\ref{sec:preliminares}. As a consequence, we present an important result previously stated in Section~\ref{sec:preliminares} as following:

\begin{theorem} \label{theo:norm} \rm
$\distanciaN{\matriz} \in \metrica$ if and only if $\matriz \in \metricaN$.
\end{theorem}

As in the previous section, we present below a sequence of auxiliary results to finally obtain a proof of Theorem~\ref{theo:norm}. 

\begin{lemma}\label{lemaPrametricaN}\rm
Let $\matriz$ be a scoring matrix.  Then $\distanciaA{\matriz} \in
\prametrica$ if and only if $\distanciaN{\matriz} \in \prametrica$.
\end{lemma}
\begin{proof}
Let $s, t$ be sequences in $\alphabet^{*}$. In order to prove this
result, we show that $\distanciaA{\matriz}(s, s) = 0$ if and only if
$\distanciaN{\matriz}(s, s) = 0$, and that $\distanciaA{\matriz}(s, t) \ge
0$ if and only if $\distanciaN{\matriz}(s, t) \ge~0$
for each $s, t \in \Sigma^*$.

Consider first that $\distanciaA{\matriz}(s, s) = 0$. If $s =
\seqVazia$, we have $\distanciaA{\matriz}(s, s) =
\distanciaN{\matriz}(s, s) = 0$ and the proof is done. Suppose then
that $s \not= \seqVazia$. Let $A$ be an A-optimal alignment of $s,
s$. Since $\distanciaA{\matriz}(s, s) = 0$, we have that
$\custoA{\matriz}[A] = 0$ and, since $s \not= \seqVazia$, we have that
$\tamanho{A} > 0$. It follows that
\begin{equation}
\distanciaN{\matriz}(s, s) \le \custoN{\matriz}[A] =
\frac{\custoA{\matriz}[A]}{\tamanho{A}} = 0\,. \label{pra3}
\end{equation}
Let $B$ be an N-optimal alignment of $s, s$. Since $s \not= \seqVazia$,
we have $\tamanho{B} > 0$, which implies, since
$\distanciaA{\matriz}(s, s) = 0$ and $\custoA{\matriz}[B] \ge
\distanciaA{\matriz}(s, s)$, that
\[
\distanciaN{\matriz}(s, s) = \frac{\custoA{\matriz}[B]}{\tamanho{B}}
\ge \frac{\distanciaA{\matriz}(s, s)}{\tamanho{B}} = 0\,.
\]
It follows from Equation~(\ref{pra3}) that $\distanciaN{\matriz}(s, s)
= 0$ and, therefore, the proof is also done when $s \not= \seqVazia$.

Similar arguments can be used to prove that $\distanciaN{\matriz}(s,
s) = 0$ implies that $\distanciaA{\matriz}(s, s) = 0$,
and 
$\distanciaA{\matriz}(s, t) \ge 0$ if and only if
$\distanciaN{\matriz}(s, t) \ge 0$.
\end{proof} 

\begin{corolary}\label{corolaryPrametrica}\rm
Let $\matriz$ be a scoring matrix and $\A, \B \in \alphabet$. Then
$\distanciaN{\matriz} \in \prametrica$ if and only if the following
conditions are true:
\begin{enumerate}
\item[(\textit{i})] $\pont{\matriz}{\A}{\A} = 0$ or
  $\pont{\matriz}{\A}{\espaco} + \pont{\matriz}{\espaco}{\A} = 0$\,, and
\item[(\textit{ii})] $\pont{\matriz}{\A}{\espaco},
  \pont{\matriz}{\espaco}{\A}, \pont{\matriz}{\A}{\B} \ge 0$\,,
\end{enumerate}
for each $\A, \B \in \Sigma$.
\end{corolary}
\begin{proof}
Suppose that $\distanciaN{\matriz} \in \prametrica$. It follows from
Lemma~\ref{lemaPrametricaN} that $\distanciaA{\matriz} \in
\prametrica$, which implies from Table~\ref{tabela2} lines (b) and
(c), that the conditions (\textit{i}) and (\textit{ii}) are true.

Conversely, suppose that the conditions (\textit{i}) and (\textit{ii})
are true.  
It follows that (b) and (c) of Table~\ref{tabela2} are true.
Since (\textit{ii}) is true, we have that $D(\matriz)$ has no negative cycle, which implies that (a) of Table~\ref{tabela2} is also true. 
It follows that $\distanciaA{\matriz} \in \prametrica$, implying from
Lemma~\ref{lemaPrametricaN} that $\distanciaN{\matriz} \in
\prametrica$.
\end{proof}

\begin{lemma}\label{lema-maiorQueZero}\rm
Let $\distanciaN{\matriz} \in \prametrica$. Then,
$\distanciaN{\matriz}(s, t) > 0$ for any $s \not= t \in \alphabet^{*}$
if and only if $\pont{\matriz}{\A}{\espaco},
\pont{\matriz}{\espaco}{\A}, \pont{\matriz}{\A}{\B} > 0$ for any $\A
\not= \B \in \alphabet$.
\end{lemma}
\begin{proof}
Let $\distanciaN{\matriz} \in \prametrica$. Suppose that
$\distanciaN{\matriz}(s, t) > 0$ for any $s \not= t \in
\alphabet^{*}$. Then,
\begin{align*}
\pont{\matriz}{\A}{\espaco} &= \custoN{\matriz}\alinhamento{\A,
  \espaco} = \distanciaN{\matriz}[\A, \seqVazia] > 0\,,\\
\pont{\matriz}{\espaco}{\A} &= \custoN{\matriz}\alinhamento{\espaco,
  \A} = \distanciaN{\matriz}(\seqVazia, \A) > 0\,,\\
\pont{\matriz}{\A}{\B} &= \custoN{\matriz}\alinhamento{\A, \B} \ge
\distanciaN{\matriz}[\A, \B] > 0\,.
\end{align*}

Conversely, suppose that $\pont{\matriz}{\A}{\espaco},
\pont{\matriz}{\espaco}{\A}, \pont{\matriz}{\A}{\B} > 0$ for any $\A
\not= \B \in \alphabet$. Since $\distanciaN{\matriz} \in \prametrica$,
from Corollary~\ref{corolaryPrametrica} we have that
$\pont{\matriz}{\A}{\A} \ge 0$. It follows from
Lemma~\ref{distanciaMaiorQue0} that $\distanciaA{\matriz}(s, t) > 0$
for $s \not= t \in \alphabet^{*}$. Let $A$ be a N-optimal alignment
of $s, t$. It follows that
\[
\distanciaN{\matriz}(s, t) = \custoN{\matriz}[A] =
\frac{\custoA{\matriz}[A]}{\tamanho{A}} \ge
\frac{\distanciaA{\matriz}(s, t)}{\tamanho{A}} > 0\,.
\] 
\end{proof}

We denote by $\Maior$ the value of $\max_{\A \in \alphabet}
\{\pont{\matriz}{\A}{\espaco}, \pont{\matriz}{\espaco}{\A}\}$ and by
$\maior \in \alphabet$ the symbol in $\alphabet$ such that $\Maior =
\max \{\pont{\matriz}{\maior}{\espaco},
\pont{\matriz}{\espaco}{\maior}\}$.

\begin{proposition}\label{propAux2}\rm
Let $s, t \in \alphabet^{*}$. Then, $\distanciaN{\matriz}(s, t) \le
\Maior$.
\end{proposition}
\begin{proof}
Notice that $\alinhamento{s \espaco^{\tamanho{t}},
  \espaco^{\tamanho{s}} t}$ is an alignment of $s, t$. Thus,
\begin{align*}
\distanciaN{\matriz}(s, t) &\le \custoN{\matriz}  \alinhamentoB{
  \begin{array}{cc}
    s & \espaco^{\tamanho{t}}\\
    \espaco^{\tamanho{s}} & t
  \end{array}
}  \le \frac{\tamanho{s} \Maior + \tamanho{t} \Maior}{\tamanho{s}
  + \tamanho{t}} = \Maior\,.
\end{align*}
\end{proof}

\begin{proposition}\label{Q=0}\rm
Let $\distanciaN{\matriz} \in \prametrica$, $\A \in \alphabet$, and
$s, t \in \alphabet^{*}$. Then,
\begin{enumerate}
\item[(\textit{i})] If $\Maior = 0$, then $\distanciaN{\matriz}(s, t)
  = \pont{\matriz}{\A}{\espaco} = \pont{\matriz}{\espaco}{\A} = 0$\,, and
\item[(\textit{ii})] If $\Maior \not= 0$, then
  $\pont{\matriz}{\maior}{\espaco} + \pont{\matriz}{\espaco}{\maior} >
  0$ and $\pont{\matriz}{\maior}{\maior} = 0$\,.
\end{enumerate}
\end{proposition}
\begin{proof}
Suppose that $\Maior = 0$.
Since $\distanciaN{\matriz} \in \prametrica$, we have that
$\distanciaN{\matriz}(s, t) \ge 0$ and, from
Proposition~\ref{propAux2}, we have that $\distanciaN{\matriz}(s, t) \le
\Maior = 0$. It follows that $\distanciaN{\matriz}(s, t) = 0$. Since
this is true for each $s, t \in \alphabet^{*}$ and $\alinhamento{\A,
  \espaco}$, $\alinhamento{\espaco, \A}$ are the only alignments of
$\A, \seqVazia$ and $\seqVazia, \A$, respectively, we have that
\begin{align*}
  \pont{\matriz}{\A}{\espaco} &= \custoA{\matriz}\alinhamento{\A,
    \espaco} = \distanciaN{\matriz}(\A,\seqVazia) = 0 \quad
  \text{and} \\
  \pont{\matriz}{\espaco}{\A} &=
  \custoA{\matriz}\alinhamento{\espaco, \A} =
  \distanciaN{\matriz}(\seqVazia, \A) = 0\,.
\end{align*}

Suppose that $\Maior \not= 0$. Since $\distanciaN{\matriz} \in
\prametrica$, we have $\min\{ \pont{\matriz}{\maior}{\espaco},
\pont{\matriz}{\espaco}{\maior} \} \ge 0$.
%it follows from Corollary~\ref{corolaryPrametrica} that
It follows that 
$\Maior = \max \{ \pont{\matriz}{\maior}{\espaco},
\pont{\matriz}{\espaco}{\maior} \} \ge 
\min\{ \pont{\matriz}{\maior}{\espaco},
\pont{\matriz}{\espaco}{\maior} \} \ge 0$ and since
by hypothesis $\Maior \not= 0$, we have that 
$\max \{ \pont{\matriz}{\maior}{\espaco},
\pont{\matriz}{\espaco}{\maior} \} = \Maior > 0$.
It follows that
\[
\pont{\matriz}{\maior}{\espaco} + \pont{\matriz}{\espaco}{\maior} =
\max \{ \pont{\matriz}{\maior}{\espaco},
\pont{\matriz}{\espaco}{\maior} \} + \min \{
\pont{\matriz}{\maior}{\espaco}, \pont{\matriz}{\espaco}{\maior} \} >
0\,,
\]
which also implies, from Corollary~\ref{corolaryPrametrica}, that
$\pont{\matriz}{\maior}{\maior} = 0$.
\end{proof}

\begin{fact}\label{fato1}\rm
Let $x, z, k, w$ be real numbers. If $k \ge 0$ and $w > 0$, then
\[
\frac{kx + z}{k + w} \ge \min \left\{ x, \frac{z}{w} \right\}.
\]
\end{fact}


\begin{proposition}\label{propAux}\rm
Let $\distanciaN{\matriz} \in \prametrica$ and $\Maior \not= 0$. If
\[
\pont{\matriz}{q}{\espaco} = \pont{\matriz}{\espaco}{q} \quad
\text{or} \quad (\pont{\matriz}{\A}{\espaco} \le
\pont{\matriz}{\A}{\maior} +
\pont{\matriz}{\maior}{\espaco}~\text{and}~\pont{\matriz}{\espaco}{\B}
\le \pont{\matriz}{\espaco}{\maior} + \pont{\matriz} {\maior}{\B})\,,
\]
for each $\A, \B \in \alphabet$, then there is $n_{0}$ such that, for
each integer $n \ge n_{0}$, we have that
\[
\distanciaN{\matriz} (q^{n} \A, q^{n} \B) = \min \left\{
\begin{array}{l}
\custoN{\matriz} \alinhamento{q^{n} \A, q^{n} \B} =
\frac{\pont{\matriz}{\A}{\B}}{n + 1}\,,\\
\custoN{\matriz} \alinhamento{q^{n} \A\espaco, q^{n} \espaco\B} =
\frac{\pont{\matriz}{\A}{\espaco} + \pont{\matriz}{\espaco}{\B}}{n +
  2}
\end{array} \right\}.
\]
\end{proposition}
\begin{proof}
Let $\distanciaN{\matriz} \in \prametrica$ and $\Maior \not= 0$ such
that $\pont{\matriz}{q}{\espaco} = \pont{\matriz}{\espaco}{q}$ or
$\pont{\matriz}{\A}{\espaco} \le \pont{\matriz}{\A}{\maior} +
\pont{\matriz}{\maior}{\espaco}$ and $\pont{\matriz}{\espaco}{\B} \le
\pont{\matriz}{\espaco}{\maior} + \pont{\matriz} {\maior}{\B}$, for
each $\A, \B \in \alphabet$. Define, for each $\A, \B \in \alphabet$,
\[
n > \frac{\max \{ 
  \pont{\matriz}{\A}{\B}, 
  \pont{\matriz}{\A}{\espaco}+\pont{\matriz}{\espaco}{\B},
  \pont{\matriz}{\A}{\maior}+\pont{\matriz}{\espaco}{\B} + 
  \pont{\matriz}{\maior}{\espaco},
  \pont{\matriz}{\maior}{\B}+\pont{\matriz}{\A}{\espaco} + 
  \pont{\matriz}{\espaco}{\maior}
  \} } 
{\pont{\matriz}{\maior}{\espaco} + \pont{\matriz}{\espaco}{\maior}}\,.
\]
Since $\distanciaN{\matriz} \in \prametrica$ and $\Maior \not= 0$, we have that
$\pont{\matriz}{\maior}{\espaco} + \pont{\matriz}{\espaco}{\maior} >
0$ and $\pont{\matriz}{\maior}{\maior} = 0$
from Proposition~\ref{Q=0}, and since $\pont{\matriz}{\maior}{\maior} = 0$, we have that
\[
\custoN{\matriz} \alinhamentoB{\begin{array}{cc}
    \maior^{n} & \A\\
    \maior^{n} & \B
  \end{array}
}  = \frac{\pont{\matriz}{\A}{\B}}{n + 1}
\quad \text{and} \quad
\custoN{\matriz} \alinhamentoB{ \begin{array}{ccc}
    \maior^{n} & \A & \espaco\\
    \maior^{n} & \espaco & \B\\
  \end{array}
}  = \frac{\pont{\matriz}{\A}{\espaco} +
  \pont{\matriz}{\espaco}{\B}}{n + 2}\,.
\]
Hence, in order to prove this proposition, we have to show that 
\[
\custoN{\matriz}\alinhamento{s', t'} \ge \min \left\{
\frac{\pont{\matriz}{\A}{\B}}{n + 1},
\frac{\pont{\matriz}{\A}{\espaco} + \pont{\matriz}{\espaco}{\B}}{n +
  2} \right\},
\]
for each alignment $\alinhamento{s', t'}$ of $\maior^{n} \A,
\maior^{n} \B$.

Let $k$ be the number of symbols~$\espaco$ in $s'$. It follows that the number of symbols $\espaco$ in $t'$ is also $k$ and
$\tamanho{\alinhamento{s',t'}} = k + n + 1$. We examine four cases,
covering all possible alignments of $\maior^{n}\A ,\maior^{n}\B$.

\begin{description}
\item[Case 1:] $[\A, \B]$ is aligned in $\alinhamento{s',t'}$.

In this case, $k \ge 0$. 
Since $\pont{\matriz}{\maior}{\maior} = 0$,
we have that
\begin{align}
  \custoN{\matriz}\alinhamento{s',t'} &=
  \frac{k(\pont{\matriz}{\maior}{\espaco} +
    \pont{\matriz}{\espaco}{\maior}) + \pont{\matriz}{\A}{\B}}{k +
    n+1} \notag \\ &\ge \min \Big\{ \pont{\matriz}{\maior}{\espaco} +
  \pont{\matriz}{\espaco}{\maior}, \frac{\pont{\matriz}{\A}{\B}}{n+1}
  \Big\} = \frac{\pont{\gamma}{\A}{\B}}{n+1}\,. \label{p2}
\end{align}
Since $k \ge 0$ and $n + 1 > 0$, the inequality~(\ref{p2}) follows
from Fact~\ref{fato1} and the equality follows by the choice of $n$.

\begin{comment}
Now, since $n > \pont{\matriz}{\A}{\B} /
(\pont{\matriz}{\maior}{\espaco} + \pont{\matriz}{\espaco}{\maior})$
and $\pont{\matriz}{\maior}{\espaco} + \pont{\matriz}{\espaco}{\maior}
> 0$, we have that
\[
n > \frac{\pont{\matriz}{\A}{\B}}{\pont{\matriz}{\maior}{\espaco} + 
\pont{\matriz}{\espaco}{\maior}} > 
\frac{\pont{\matriz}{\A}{\B}- 
(\pont{\matriz}{\maior}{\espaco} + \pont{\matriz}{\espaco}{\maior})}
{\pont{\matriz}{\maior}{\espaco} + 
\pont{\matriz}{\espaco}{\maior}}\, ,
\]
which implies, since $n > 0$, that $\pont{\matriz}{\maior}{\espaco} +
\pont{\matriz}{\espaco}{\maior} > \pont{\matriz}{\A}{\B}/(n+1)$. It
follows from inequality~(\ref{p2}) that
\[
\custoN{\matriz}\alinhamento{s',t'} \ge \min \Big\{
\pont{\matriz}{\maior}{\espaco} + \pont{\matriz}{\espaco}{\maior},
\frac{\pont{\matriz}{\A}{\B}}{n+1} \Big\} =
\frac{\pont{\matriz}{\A}{\B}}{n+1}\,.
\]
\end{comment}

\item[Case 2:] $[\A,\espaco]$ and $[\espaco, \B]$ are aligned in
  $\alinhamento{s',t'}$.

In this case, $k \ge 1$. Since $\pont{\matriz}{\maior}{\maior} = 0$,
we have that
\begin{align*}
  \custoN{\matriz}\alinhamento{s',t'} &= \frac{(k - 1)
    (\pont{\matriz}{\maior}{\espaco} +
    \pont{\matriz}{\espaco}{\maior}) + \pont{\matriz}{\A}{\espaco} +
    \pont{\matriz}{\espaco}{\B}} {(k - 1) + n + 2}\\
  &\ge \min \Big\{ \pont{\matriz}{\maior}{\espaco} +
  \pont{\matriz}{\espaco}{\maior}, \frac{\pont{\matriz}{\A}{\espaco} +
    \pont{\matriz}{\espaco}{\B}} {n + 2} \Big\} =
\frac{\pont{\matriz}{\A}{\espaco} +
  \pont{\matriz}{\espaco}{\B}}{n+2}\,.
\end{align*}
The inequality, since $k - 1 \ge 0$ and $n + 2 > 0$, follows from
Fact~\ref{fato1} and the last equality follows by the choice of $n$.

\begin{comment}
Since $n + 2 > 0$, $n > (\pont{\matriz}{\A}{\espaco}
+ \pont{\matriz}{\espaco}{\B}) / (\pont{\matriz}{\maior}{\espaco} +
\pont{\matriz}{\espaco}{\maior})$ and $\pont{\matriz}{\maior}{\espaco}
+ \pont{\matriz}{\espaco}{\maior} > 0$, using similar arguments as in
Case 1, it follows that
\[
\custoN{\matriz}\alinhamento{s',t'} \ge
\frac{\pont{\matriz}{\A}{\espaco} +
  \pont{\matriz}{\espaco}{\B}}{n+2}\,.
\]
\end{comment}

\item[Case 3:] $[\A,\maior]$ is aligned in $\alinhamento{s',t'}$. 

In this case, $k \ge 1$. Also, since $\pont{\matriz}{\maior}{\maior} =
0$, we have that
\begin{align}
  \custoN{\matriz}\alinhamento{s',t'} &= \frac{(k - 1)
    (\pont{\matriz}{\maior}{\espaco} +
    \pont{\matriz}{\espaco}{\maior}) +
    \pont{\matriz}{\A}{\maior}+\pont{\matriz}{\espaco}{\B} +
    \pont{\matriz}{\maior}{\espaco}}{(k - 1)+ n+2} \nonumber \\
  &\ge \min \Big\{ \pont{\matriz}{\maior}{\espaco} +
  \pont{\matriz}{\espaco}{\maior},
  \frac{\pont{\matriz}{\A}{\maior}+\pont{\matriz}{\espaco}{\B}+
    \pont{\matriz}{\maior}{\espaco}}{n+2} \Big\} \label{p3}\\
  &= \frac{\pont{\matriz}{\A}{\maior}+\pont{\matriz}{\espaco}{\B}+
    \pont{\matriz}{\maior}{\espaco}}{n+2}\,. \label{p4}
\end{align}
Since $k - 1 \ge 0$ and $n + 2 > 0$, inequality~(\ref{p3}) follows
from Fact~\ref{fato1} and equality~(\ref{p4}) follows 
by the choice of $n$.

\begin{comment}
Using similar arguments as in Case 1, since $n
+ 2 > 0$, $n > (\pont{\matriz}{\A}{\maior} +
\pont{\matriz}{\espaco}{\B} + \pont{\matriz}{\maior}{\espaco}) /
(\pont{\matriz}{\maior}{\espaco} + \pont{\matriz}{\espaco}{\maior})$
and $\pont{\matriz}{\maior}{\espaco} + \pont{\matriz}{\espaco}{\maior}
> 0$, it follows the equality~(\ref{p4}).
\end{comment}

Suppose that $\pont{\matriz}{\maior}{\espaco} =
\pont{\matriz}{\espaco}{\maior}$. Then,
$\pont{\matriz}{\maior}{\espaco} = \min
\{\pont{\matriz}{\maior}{\espaco},\pont{\matriz}{\espaco}{\maior} \} =
\Maior$. Since $\distanciaN{\matriz} \in \prametrica$, we have from
Corollary~\ref{corolaryPrametrica} that $\pont{\matriz}{\A}{\maior}
\ge 0$. It follows from equality~(\ref{p4}) and the definition of
$\Maior$ that
\begin{align*}
  \custoN{\matriz}\alinhamento{s',t'} &\ge
  \frac{\pont{\matriz}{\A}{\maior}+\pont{\matriz}{\espaco}{\B}+
    \pont{\matriz}{\maior}{\espaco}}{n+2} \ge 
    \frac{0 + \pont{\matriz}{\espaco}{\B}+ \Maior}{n+2} \\
    &\ge \frac{\pont{\matriz}{\espaco}{\B} +
    \pont{\matriz}{\A}{\espaco}}{n+2}
    = \frac{\pont{\matriz}{\A}{\espaco} + \pont{\matriz}{\espaco}{\B}
    }{n+2}\,,
\end{align*}
and the proof is complete. Suppose then that $\pont{\matriz}{\maior}{\espaco} \not=
\pont{\matriz}{\espaco}{\maior}$. By hypothesis, this implies that
$\pont{\matriz}{\A}{\espaco} \le \pont{\matriz}{\A}{q} +
\pont{\matriz}{q}{\espaco}$. It follows from equality~(\ref{p4}) that
\[
\custoN{\matriz}\alinhamento{s',t'} \ge
\frac{\pont{\matriz}{\A}{\maior}+\pont{\matriz}{\espaco}{\B}+
  \pont{\matriz}{\maior}{\espaco}}{n+2} \ge
\frac{\pont{\matriz}{\A}{\espaco}+\pont{\matriz}{\espaco}{\B}}{n+2}\,.
\]

\item[Case 4:] $[\maior, \B]$ is aligned in $\alinhamento{s',t'}$. Similar to Case 3.
\begin{comment}
, since $n + 2 > 0$, $k - 1 \ge 0$, $n >
\pont{\matriz}{\maior}{\B}+\pont{\matriz}{\A}{\espaco} +
\pont{\matriz}{\espaco}{\maior}$, and $\pont{\matriz}{\espaco}{\B} \le
\pont{\matriz}{\espaco}{\maior} + \pont{\matriz}{\maior}{\B}$.
\end{comment}
\end{description}
\end{proof}

\begin{proposition}\label{propAux-3}\rm
Suppose that $\distanciaN{\matriz} \in \prametrica$. Let $\A, \B \in
\alphabet$ and $s, t \in \alphabet^{*}$. If $[\A, \B]$ is aligned in
an N-optimal alignment of $s, t$ of maximum length, then
\[
\pont{\matriz}{\A}{\B} < \pont{\matriz}{\A}{\espaco} +
\pont{\matriz}{\espaco}{\B}\,.
\] 
\end{proposition}
\begin{proof}
Let $\alinhamento{s', t'}$ be a N-optimal alignment of maximum length
of $s, t$ and $j$ be an integer such that 
$s'(j) = \A$ and $t'(j) = \B$. Consider the following alignment of $s, t$
\[
A =  
\alinhamentoB{ \begin{array}{cccccccc}
s'(1) & \ldots & s'(j - 1) & \A & \espaco & s'(j+1) & \ldots & 
s'(\tamanho{s'})\\
t'(1) & \ldots & t'(j - 1) & \espaco & \B & t'(j+1) & \ldots & 
t'(\tamanho{t'})
\end{array}
}.
\]

Since $\alinhamento{s',t'}$ is an N-optimal alignment
of maximum length and $\abs{[s',t']} < \abs{A}$, we have that
\begin{align*}
\frac{\custoA{\matriz}\alinhamento{s',t'}}{\tamanho{\alinhamento{s',t'}}}
= \custoN{\matriz}\alinhamento{s', t'} &< \custoN{\matriz}[A] =
\frac{\custoA{\matriz}[A]}{\tamanho{A}}\\
&= \frac{\custoA{\matriz}\alinhamento{s',t'} +
  \pont{\matriz}{\A}{\espaco} + \pont{\matriz}{\espaco}{\B} -
  \pont{\matriz}{\A}{\B}}{\tamanho{\alinhamento{s',t'}}+1}\,,
\end{align*}
which implies that 
\[
\frac{\custoA{\matriz}\alinhamento{s',t'}}{\tamanho{\alinhamento{s',t'}}}
< \pont{\matriz}{\A}{\espaco} + \pont{\matriz}{\espaco}{\B} -
\pont{\matriz}{\A}{\B}\,. %\label{eq-propAux-3}
\]
Since $\distanciaN{\matriz} \in \prametrica$, we have that
$\distanciaN{\matriz}(s, t) \ge 0$.
%% This implies, since, by hypothesis,
%% $\distanciaN{\matriz}(s, t) \not= 0$, that $\distanciaN{\matriz}(s, t) > 0$. 
%% It follows by~(\ref{eq-propAux-3}), since $(s', t')$ is a N-optimal
%% alignment of $(s, t)$, that
It follows that
\[
0 \le \distanciaN{\matriz}(s, t) =
\frac{\custoA{\matriz}\alinhamento{s',t'}}{\tamanho{\alinhamento{s',t'}}}
< \pont{\matriz}{\A}{\espaco} + \pont{\matriz}{\espaco}{\B} -
\pont{\matriz}{\A}{\B}\,,
\]
which implies that $\pont{\matriz}{\A}{\B} <
\pont{\matriz}{\A}{\espaco} + \pont{\matriz}{\espaco}{\B}$.
\end{proof}

\begin{fact}\label{fato2}\rm
Let $x, y \in \mathbb{R}$.  If $x < y$ then there exists an
integer $n_{0}$ such that
\[
\frac{x}{n + 1} < \frac{y}{n + 2}\,,
\]
for each $n > n_{0}$.
\end{fact}
 
\begin{lemma}\label{distanciaNSimetrica}\rm
Let $\distanciaN{\matriz} \in \prametrica$. Then,
$\distanciaN{\matriz}(s, t) = \distanciaN{\matriz}(t, s)$ for each $s,
t \in \alphabet^{*}$ if and only if
\begin{enumerate}
\item [(\textit{i})] $\pont{\matriz}{\A}{\espaco} =
  \pont{\matriz}{\espaco}{\A}$\,, and
\item [(\textit{ii})] if $\pont{\matriz}{\A}{\B} <
  \pont{\matriz}{\A}{\espaco} + \pont{\matriz}{\espaco}{\B}$, then
  $\pont{\matriz}{\A}{\B} = \pont{\matriz}{\B}{\A}$\,,
\end{enumerate}
for each $\A, \B \in \alphabet$.
\end{lemma}
\begin{proof}
By Corollary~\ref{corolaryPrametrica}, we have that
$\pont{\matriz}{\A}{\B}, \pont{\matriz}{\maior}{\espaco},
\pont{\matriz}{\espaco}{\maior} \ge 0$.

Suppose that $\distanciaN{\matriz}(s, t) =
\distanciaN{\matriz}(t, s)$ for each $s, t \in \alphabet^{*}$.
Since $\alinhamento{\A, \espaco}$ and $\alinhamento{\espaco, \A}$ are
the only alignments of $\A, \seqVazia$ and~$\seqVazia, \A$,
respectively, and by hypothesis, $\distanciaN{\matriz}(\A, \seqVazia) =
\distanciaN{\matriz}(\seqVazia, \A)$, we have that
\[
\pont{\matriz}{\A}{\espaco} =
\custoN{\matriz}\alinhamento{\A, \espaco} = 
\distanciaN{\matriz}(\A, \seqVazia) =  
\distanciaN{\matriz}(\seqVazia, \A) = 
\custoN{\matriz}\alinhamento{\espaco, \A} =  
\pont{\matriz}{\espaco}{\A}\,,
\]
which implies that (\textit{i}) is true.
In order to show (\textit{ii}), we consider two possibilities: $\Maior
= 0$ and $\Maior \not= 0$. If $\Maior = 0$, since
$\pont{\matriz}{\A}{\B} \ge 0$ and $\Maior \ge
\pont{\matriz}{\A}{\espaco}, \pont{\matriz}{\espaco}{\B}$, we have
that
$
\pont{\matriz}{\A}{\B} \ge 0 = 0 + 0 = \Maior + \Maior \ge 
\pont{\matriz}{\A}{\espaco} + \pont{\matriz}{\espaco}{\B}
$ and in this case 
(\textit{ii}) is vacuously satisfied. Thus, assume that $\Maior \not= 0$ and suppose that $\pont{\matriz}{\A}{\B} <
\pont{\matriz}{\A}{\espaco} + \pont{\matriz}{\espaco}{\B}$. Choose $n$
big enough satisfying both Proposition~\ref{propAux}, since $\pont{\matriz}{\maior}{\espaco} = \pont{\matriz}{\espaco}{\maior}$, and 
Fact~\ref{fato2}. It follows that
\begin{align*}
\distanciaN{\matriz}(\maior^{n} \A, \maior^{n} \B) &= 
\min \left\{ \frac{\pont{\matriz}{\A}{\B}}{n + 1}, 
\frac{\pont{\matriz}{\A}{\espaco} + \pont{\matriz}{\espaco}{\B}}{n + 2}
\right\}= \frac{\pont{\matriz}{\A}{\B}}{n + 1}
< \frac{\pont{\matriz}{\A}{\espaco} +  \pont{\matriz}{\espaco}{\B}}{n +
 2} \hspace{0.2cm} \mbox{and}\\
\distanciaN{\matriz}(\maior^{n} \B, \maior^{n} \A) &=  
\min \left\{ \frac{\pont{\matriz}{\B}{\A}}{n + 1}, 
\frac{\pont{\matriz}{\B}{\espaco} + \pont{\matriz}{\espaco}{\A}}{n + 2}
\right\}.
\end{align*}
Since $\pont{\matriz}{\A}{\espaco} =
\pont{\matriz}{\espaco}{\A}$, $\pont{\matriz}{\B}{\espaco} =
\pont{\matriz}{\espaco}{\B}$, and $\distanciaN{\matriz}(\maior^{n} \A,
\maior^{n} \B) = \distanciaN{\matriz}(\maior^{n} \B, \maior^{n}
\A)$, we have that $(\pont{\matriz}{\B}{\espaco} +
\pont{\matriz}{\espaco}{\A})/(n + 2)$ is not a value of
$\distanciaN{\matriz}(\maior^{n} \B, \maior^{n} \A)$, which implies
that $\distanciaN{\matriz}(\maior^{n} \B, \maior^{n} \A) =
\pont{\matriz}{\B}{\A}/(n + 1)$. Therefore,
\[
\frac{\pont{\matriz}{\A}{\B}}{n + 1} =
\distanciaN{\matriz}(\maior^{n} \A, \maior^{n} \B) = 
\distanciaN{\matriz}(\maior^{n} \B, \maior^{n} \A) = 
\frac{\pont{\matriz}{\B}{\A}}{n + 1}\,,
\]
and thus $\pont{\matriz}{\A}{\B} = \pont{\matriz}{\B}{\A}$.

Conversely, consider that (\textit{i}) and (\textit{ii}) are true.
Let $\alinhamento{s', t'}$ be an N-optimal alignment of maximum length
of $s, t$. If $s'(j) = \espaco$ or $t'(j) = \espaco$ then from
(\textit{i}) we have that $\pont{\matriz}{s'(j)}{t'(j)} =
\pont{\matriz}{t'(j)}{s'(j)}$. If $s'(j) \not= \espaco$ and $t'(j)
\not= \espaco$ then, since 
$\distanciaN{\matriz} \in \prametrica$, we have from
Proposition~\ref{propAux-3} that $\pont{\matriz}{s'(j)}{t'(j)} <
\pont{\matriz}{s'(j)}{\espaco} + \pont{\matriz}{\espaco}{t'(j)}$,
which implies by (\textit{ii}) that $\pont{\matriz}{s'(j)}{t'(j)} =
\pont{\matriz}{t'(j)}{s'(j)}$. Using these observations, we have that
\begin{align*}
\distanciaN{\matriz}(s, t) &= \custoN{\matriz}\alinhamento{s', t'} =
\frac{\sum_{j} \pont{\matriz}{s'(j)}{t'(j)}}{\tamanho{(s', t')}} \\
&= \frac{\sum_{j} \pont{\matriz}{t'(j)}{s'(j)}}{\tamanho{(t', s')}} =
\custoN{\matriz}\alinhamento{t', s'} \ge \distanciaN{\matriz}(t,
s)\,.
\end{align*}
Similarly, we have $\distanciaN{\matriz}(s, t) \le
\distanciaN{\matriz}(t, s)$, which allows us to conclude that
$\distanciaN{\matriz}(s, t) = \distanciaN{\matriz}(t, s)$.
\end{proof}

\begin{proposition}\label{AuxdistanciaNTriangular-1}\rm
Let $\distanciaN{\matriz} \in \prametrica$.  If $\distanciaN{\matriz}(s, t) \le \distanciaN{\matriz}
(s, u) + \distanciaN{\matriz} (u, t)$ for all $s, t, u \in
\alphabet^{*}$ then
\[
\pont{\matriz}{\A}{\espaco} \le 
\pont{\matriz}{\A}{\B} + \pont{\matriz}{\B}{\espaco}
\qquad \text{and} \qquad
\pont{\matriz}{\espaco}{\A} \le
\pont{\matriz}{\espaco}{\B} + \pont{\matriz}{\B}{\A}\,,
\]
for all $\A, \B \in
\alphabet$.
\end{proposition}
\begin{proof}
Suppose that $\distanciaN{\matriz}(s, t) \le \distanciaN{\matriz}(s, u) +
\distanciaN{\matriz}(u, t)$ for all $s, t, u \in \alphabet^{*}$.
Since $\alinhamento{\A, \espaco}$ is the only alignment of $\A,
\seqVazia$, we have that
$
\distanciaN{\matriz}(\A, \seqVazia) = 
%\custoN{\matriz}\alinhamento{\A, \espaco} =
\pont{\matriz}{\A}{\espaco}$, and since $\alinhamento{\A, \B}$ and $\alinhamento{\B, \espaco}$ are
alignments of $\A, \B$ and $\B, \seqVazia$, it follows
that
\begin{align*}
\pont{\matriz}{\A}{\espaco} = \distanciaN{\matriz}(\A, \seqVazia) &\le
\distanciaN{\matriz}(\A, \B) + \distanciaN{\matriz}(\B, \seqVazia) \\
&\le \custoN{\matriz}\alinhamento{\A, \B} +
\custoN{\matriz}\alinhamento{\B, \espaco} \\
&= \pont{\matriz}{\A}{\B} + \pont{\matriz}{\B}{\espaco}\,.
\end{align*}

Using similar arguments we also prove that
$\pont{\matriz}{\espaco}{\A} \le \pont{\matriz}{\espaco}{\B} +
\pont{\matriz}{\B}{\A}$.
\end{proof}

\begin{proposition}\label{AuxdistanciaNTriangular}\rm
Let $\distanciaN{\matriz} \in \prametrica$. If
$\distanciaN{\matriz}(s, t) \le \distanciaN{\matriz}(s, u) +
\distanciaN{\matriz}(u, t)$ for each $s, t, u \in \alphabet^{*}$
then
\[
\max\{\pont{\matriz}{\A}{\espaco}, \pont{\matriz}{\espaco}{\A}\}
\le \pont{\matriz}{\B}{\espaco} + \pont{\matriz}{\espaco}{\B}\,,
\]
for each and $\A, \B \in \alphabet$.
\end{proposition}
\begin{proof}
Suppose that
$\distanciaN{\matriz}(s, t) \le \distanciaN{\matriz}(s, u) +
\distanciaN{\matriz}(u, t)$ for all $s, t, u \in \alphabet^{*}$.

Since $\distanciaN{\matriz} \in \prametrica$,
if $\Maior = 0$, then
$\pont{\matriz}{\A}{\espaco} = \pont{\matriz}{\espaco}{\A} =
\pont{\matriz}{\B}{\espaco} = 
\pont{\matriz}{\espaco}{\B} = 
0$ as a consequence of Corollary~\ref{corolaryPrametrica}
and, therefore, the propositions is proved. 
Thus, assume that $\Maior \not= 0$. That is,
$\Maior > 0$ from Corollary~\ref{corolaryPrametrica}.
W.l.o.g., assume that 
$\pont{\matriz}{\maior}{\espaco} = \Maior$.

Suppose that there exists $\A, \B \in \alphabet$
such that $\max\{\pont{\matriz}{\A}{\espaco}, \pont{\matriz}{\espaco}{\A}\}
> \pont{\matriz}{\B}{\espaco} +
\pont{\matriz}{\espaco}{\B}$ by contradiction. This implies that
$\Maior > \pont{\matriz}{\B}{\espaco} +
\pont{\matriz}{\espaco}{\B}$. Let $k$ be an positive integer such that
\[
k > \frac{\pont{\matriz}{\B}{\espaco}+\pont{\matriz}{\espaco}{\B}}
{\Maior - (\pont{\matriz}{\B}{\espaco} +
  \pont{\matriz}{\espaco}{\B})}\,.
\]


Since $\distanciaN{\matriz}
\in \prametrica$, we have that
$\pont{\matriz}{\maior}{\espaco}, \pont{\matriz}{\espaco}{\maior} \ge 0$,
which implies, since 
$\pont{\matriz}{\maior}{\espaco} = \Maior >0$, that $\pont{\matriz}{\maior}{\espaco} + \pont{\matriz}{\espaco}{\maior} > 0$,
which in turn implies that  $\pont{\matriz}{\maior}{\maior} = 0$
as a consequence of
Corollary~\ref{corolaryPrametrica}. Then,
\begin{align}
\distanciaN{\matriz}(\maior^{k} ,\maior^{k} \B) &\le \custoN{\matriz} 
\alinhamentoB{
\begin{array}{cc}
  \maior^{k} & \espaco \\
  \maior^{k} & \B
\end{array}
} =
\frac{\pont{\matriz}{\espaco}{\B}}{k+1}\,, \label{AuxdistanciaNTriangular2}\\
\distanciaN{\matriz}(\maior^{k} \B, \B) &\le \custoN{\matriz} 
\alinhamentoB{
\begin{array}{ccc}
  \maior^{k} & \B & \espaco \\
\espaco^{k} & \espaco & \B
\end{array}
} = 
\frac{k \pont{\matriz}{\maior}{\espaco} + (\pont{\matriz}{\B}{\espaco} +
  \pont{\matriz}{\espaco}{\B})}{k+2}\,.\label{AuxdistanciaNTriangular3}
\end{align} 

In any alignment of $\maior^{k}, \B$ either $[\maior, \B]$ or $[\espaco, \B]$ is aligned. Consequently,
\begin{align*}
\distanciaN{\matriz}(\maior^{k}, \B) &= 
\min \left\{ \custoN{\matriz} 
\alinhamentoB{
  \begin{array}{cc}
    \maior^{k-1} & \maior \\
    \espaco^{k-1} & \B
  \end{array}
},
\custoN{\matriz}
\alinhamentoB{
  \begin{array}{cc}
    \maior^{k} & \espaco \\
    \espaco^{k} & \B
  \end{array}
}
\right\} \\
&= 
\min \left\{ \frac{(k-1) \pont{\matriz}{\maior}{\espaco} +
  \pont{\matriz}{\maior}{\B}}{k}, \frac{k \pont{\matriz}{\maior}{\espaco} +
  \pont{\matriz}{\espaco}{\B}}{k+1} \right\}\,.
\end{align*}
Suppose that $\distanciaN{\gamma}(\maior^k,\B) = 
{((k-1) \pont{\matriz}{\maior}{\espaco} +
  \pont{\matriz}{\maior}{\B})}/{k}$. 
It follows that 
\[
\frac{(k-1) \pont{\matriz}{\maior}{\espaco} + \pont{\matriz}{\maior}{\B}}{k}
\le \frac{k \pont{\matriz}{\maior}{\espaco} +
  \pont{\matriz}{\espaco}{\B}}{k+1}\,.
\]
We
have that $\pont{\matriz}{\maior}{\espaco} \le \pont{\matriz}{\maior}{\B} +
\pont{\matriz}{\B}{\espaco}$ by Proposition~\ref{AuxdistanciaNTriangular-1}, which implies, since $k > 0$,  that
\[
\frac{k \pont{\matriz}{\maior}{\espaco} - \pont{\matriz}{\B}{\espaco}}{k}
\le \frac{(k-1) \pont{\matriz}{\maior}{\espaco} +
  \pont{\matriz}{\maior}{\B}}{k}\,.
\]
It follows that
\[
\frac{k \pont{\matriz}{\maior}{\espaco} - \pont{\matriz}{\B}{\espaco}}{k}
\le \frac{k \pont{\matriz}{\maior}{\espaco} +
  \pont{\matriz}{\espaco}{\B}}{k+1}\,,
\]
and, since $k>0$, that
$k(\pont{\gamma}{\maior}{\espaco} -(\pont{\matriz}{\B}{\espaco} +
\pont{\matriz}{\espaco}{\B})) \le  \pont{\matriz}{\B}{\espaco}$
which implies, since $\Maior =
\pont{\matriz}{\maior}{\espaco} > \pont{\matriz}{\B}{\espaco} +
\pont{\matriz}{\espaco}{\B}$, that $k \le \pont{\matriz}{\B}{\espaco}/
(\Maior - (\pont{\matriz}{\B}{\espaco} +
\pont{\matriz}{\espaco}{\B}))$, contradicting the choice of
$k$ since $\pont{\matriz}{\espaco}{\B} \ge 0$. Thus,
we assume that
\begin{equation}
\distanciaN{\matriz}(\maior^{k}, \B) = \frac{k \pont{\matriz}{\maior}{\espaco}
  + \pont{\matriz}{\espaco}{\B}}{k+1}\,.
\label{AuxdistanciaNTriangular4}
\end{equation}

Since $\distanciaN{\matriz}(\maior^{k}, \B) \le
\distanciaN{\matriz}(\maior^{k} ,\maior^{k} \B) + \distanciaN{\matriz}(\maior^{k}
\B, \B)$, we have from Equations~(\ref{AuxdistanciaNTriangular2}),
(\ref{AuxdistanciaNTriangular3}), and~(\ref{AuxdistanciaNTriangular4})
that
\begin{align*}
\frac{k \pont{\matriz}{\maior}{\espaco} +
  \pont{\matriz}{\espaco}{\B}}{k+1} = \distanciaN{\matriz}(\maior^{k}, \B)
&\le \distanciaN{\matriz}(\maior^{k} ,\maior^{k} \B) +
\distanciaN{\matriz}(\maior^{k} \B, \B) \\
&\le \frac{\pont{\matriz}{\espaco}{\B}}{k+1} + \frac{k
  \pont{\matriz}{\maior}{\espaco} + (\pont{\matriz}{\B}{\espaco} +
  \pont{\matriz}{\espaco}{\B})}{k+2}\,,
\end{align*}
which implies, since $k > 0$ and $\Maior = \pont{\matriz}{\maior}{\espaco} >
\pont{\matriz}{\B}{\espaco} + \pont{\matriz}{\espaco}{\B}$, that
\[
k \le \frac{\pont{\matriz}{\B}{\espaco} + \pont{\matriz}{\espaco}{\B}}
{\Maior - (\pont{\matriz}{\B}{\espaco} +
  \pont{\matriz}{\espaco}{\B})}\,,
\]
contradicting again the choice of $k$. Thus, 
$\max\{\pont{\matriz}{\A}{\espaco}, \pont{\matriz}{\espaco}{\A}\}
\le \pont{\matriz}{\B}{\espaco} + \pont{\matriz}{\espaco}{\B}$
for each $\A,\B~\in~\Sigma$. 
\end{proof}

\begin{comment}

\begin{proposition}\label{AuxdistanciaNTriangular}\rm
Let $\distanciaN{\matriz} \in \prametrica$. If
$\distanciaN{\matriz}(s, t) \le \distanciaN{\matriz}(s, u) +
\distanciaN{\matriz}(u, t)$ for each $s, t, u \in \alphabet^{*}$,
then
\[
\max\{\pont{\matriz}{\A}{\espaco}, \pont{\matriz}{\espaco}{\A}\}
\le \pont{\matriz}{\B}{\espaco} + \pont{\matriz}{\espaco}{\B}\,.
\]
for each and $\B \in \alphabet$.
\end{proposition}
\begin{proof}
Suppose that 
$\distanciaN{\matriz}(s, t) \le \distanciaN{\matriz}(s, u) +
\distanciaN{\matriz}(u, t)$ for all $s, t, u \in \alphabet^{*}$ and let
$\A \in \alphabet$ such that $\mathcal{G} = \max_{\A \in \Sigma}\{
\pont{\matriz}{\A}{\espaco}, \pont{\matriz}{\espaco}{\A}\}$.

Since $\distanciaN{\matriz} \in \prametrica$,
if $\mathcal{G} = 0$, then
$\pont{\matriz}{\A}{\espaco} = \pont{\matriz}{\espaco}{\A} =
\pont{\matriz}{\B}{\espaco} = 
\pont{\matriz}{\espaco}{\B} = 
0$ as a consequence of Corollary~\ref{corolaryPrametrica}
and, therefore, the propositions is proved. 
Thus, assume that $\mathcal{G} \not= 0$. That is,
$\mathcal{G} > 0$ from Corollary~\ref{corolaryPrametrica}.

Suppose there exists $\B \in \alphabet$
such that $\mathcal{G} > \pont{\matriz}{\B}{\espaco} +
\pont{\matriz}{\espaco}{\B}$ by contradiction. Let $k$ be an positive integer such that
\[
k > \frac{\pont{\matriz}{\B}{\espaco}+\pont{\matriz}{\espaco}{\B}}
{\mathcal{G} - (\pont{\matriz}{\B}{\espaco} +
  \pont{\matriz}{\espaco}{\B})}\,.
\]
Since $\distanciaN{\matriz}
\in \prametrica$ and 
$\pont{\matriz}{\A}{\espaco} >
0$, we have $\pont{\matriz}{\A}{\A} = 0$
as a consequence of
Corollary~\ref{corolaryPrametrica}. Then,
\begin{align}
\distanciaN{\matriz}(\A^{k} ,\A^{k} \B) &\le \custoA{\matriz} 
\alinhamentoB{
\begin{array}{cc}
  \A^{k} & \espaco \\
  \A^{k} & \B
\end{array}
} =
\frac{\pont{\matriz}{\espaco}{\B}}{k+1}\,, \label{AuxdistanciaNTriangular2}\\
\distanciaN{\matriz}(\A^{k} \B, \B) &\le \custoN{\matriz} 
\alinhamentoB{
\begin{array}{ccc}
  \A^{k} & \B & \espaco \\
\espaco^{k} & \espaco & \B
\end{array}
} = 
\frac{k \pont{\matriz}{\A}{\espaco} + (\pont{\matriz}{\B}{\espaco} +
  \pont{\matriz}{\espaco}{\B})}{k+2}\,.\label{AuxdistanciaNTriangular3}
\end{align} 

In any alignment of $\A^{k}, \B$, either $[\A, \B]$ or $[\espaco, \B]$ is aligned. Consequently,
\begin{align*}
\distanciaN{\matriz}(\A^{k}, \B) &= 
\min \left\{ \custoN{\matriz} 
\alinhamentoB{
  \begin{array}{cc}
    \A^{k-1} & \A \\
    \espaco^{k-1} & \B
  \end{array}
},
\custoN{\matriz}
\alinhamentoB{
  \begin{array}{cc}
    \A^{k} & \espaco \\
    \espaco^{k} & \B
  \end{array}
}
\right\} \\
&= 
\min \left\{ \frac{(k-1) \pont{\matriz}{\A}{\espaco} +
  \pont{\matriz}{\A}{\B}}{k}, \frac{k \pont{\matriz}{\A}{\espaco} +
  \pont{\matriz}{\espaco}{\B}}{k+1} \right\}\,.
\end{align*}
Suppose that $\distanciaN{\gamma}(\A^k,\B) = 
{((k-1) \pont{\matriz}{\A}{\espaco} +
  \pont{\matriz}{\A}{\B})}/{k}$. 
It follows that 
\[
\frac{(k-1) \pont{\matriz}{\A}{\espaco} + \pont{\matriz}{\A}{\B}}{k}
\le \frac{k \pont{\matriz}{\A}{\espaco} +
  \pont{\matriz}{\espaco}{\B}}{k+1}\,.
\]
We
have that $\pont{\matriz}{\A}{\espaco} \le \pont{\matriz}{\A}{\B} +
\pont{\matriz}{\B}{\espaco}$ by Proposition~\ref{AuxdistanciaNTriangular-1}, which implies, since $k > 0$,  that
\[
\frac{k \pont{\matriz}{\A}{\espaco} - \pont{\matriz}{\B}{\espaco}}{k}
\le \frac{(k-1) \pont{\matriz}{\A}{\espaco} +
  \pont{\matriz}{\A}{\B}}{k}\,.
\]
It follows that
\[
\frac{k \pont{\matriz}{\A}{\espaco} - \pont{\matriz}{\B}{\espaco}}{k}
\le \frac{k \pont{\matriz}{\A}{\espaco} +
  \pont{\matriz}{\espaco}{\B}}{k+1}\,,
\]
and, since $k>0$, that
$k(\pont{\gamma}{\A}{\espaco} -(\pont{\matriz}{\B}{\espaco} +
\pont{\matriz}{\espaco}{\B})) \le  \pont{\matriz}{\B}{\espaco}$
which implies, since $\mathcal{G} =
\pont{\matriz}{\A}{\espaco} > \pont{\matriz}{\B}{\espaco} +
\pont{\matriz}{\espaco}{\B}$, that $k \le \pont{\matriz}{\B}{\espaco}/
(\mathcal{G} - (\pont{\matriz}{\B}{\espaco} +
\pont{\matriz}{\espaco}{\B}))$ contradicting the choice of
$k$ since $\pont{\matriz}{\espaco}{\B} \ge 0$. Thus,
we assume that
\begin{equation}
\distanciaN{\matriz}(\A^{k}, \B) = \frac{k \pont{\matriz}{\A}{\espaco}
  + \pont{\matriz}{\espaco}{\B}}{k+1}\,.
\label{AuxdistanciaNTriangular4}
\end{equation}

Since $\distanciaN{\matriz}(\A^{k}, \B) \le
\distanciaN{\matriz}(\A^{k} ,\A^{k} \B) + \distanciaN{\matriz}(\A^{k}
\B, \B)$, we have from Equations~(\ref{AuxdistanciaNTriangular2}),
(\ref{AuxdistanciaNTriangular3}), and~(\ref{AuxdistanciaNTriangular4})
that
\begin{align*}
\frac{k \pont{\matriz}{\A}{\espaco} +
  \pont{\matriz}{\espaco}{\B}}{k+1} = \distanciaN{\matriz}(\A^{k}, \B)
&\le \distanciaN{\matriz}(\A^{k} ,\A^{k} \B) +
\distanciaN{\matriz}(\A^{k} \B, \B) \\
&\le \frac{\pont{\matriz}{\espaco}{\B}}{k+1} + \frac{k
  \pont{\matriz}{\A}{\espaco} + (\pont{\matriz}{\B}{\espaco} +
  \pont{\matriz}{\espaco}{\B})}{k+2}\,,
\end{align*}
which implies, since $k > 0$ and $\mathcal{G} = \pont{\matriz}{\A}{\espaco} >
\pont{\matriz}{\B}{\espaco} + \pont{\matriz}{\espaco}{\B}$, that
\[
k \le \frac{\pont{\matriz}{\B}{\espaco} + \pont{\matriz}{\espaco}{\B}}
{\mathcal{G} - (\pont{\matriz}{\B}{\espaco} +
  \pont{\matriz}{\espaco}{\B})}\,,
\]
contradicting again the choice of $k$. Thus, 
$\mathcal{G} \le \pont{\matriz}{\B}{\espaco} + \pont{\matriz}{\espaco}{\B}$ for each $\B \in \Sigma$. 
\end{proof}





\begin{fact}\label{fato3}\rm
For each pair of numbers $x, y$, there exists a number $n_{0}$ such
that
\[
x \le y \quad \text{if and only if} \quad \frac{x}{n+2} <
\frac{y}{n+1}\,,
\]
for each $n \ge n_{0}$.
\end{fact}
\end{comment}


\begin{proposition}\label{AuxdistanciaNTriangular-2}\rm
Let $\distanciaN{\matriz} \in \prametrica$. If $\distanciaN{\matriz}(s, t) \le
\distanciaN{\matriz}(s, u) + \distanciaN{\matriz}(u, t)$ for each $s,
t, u \in \alphabet^{*}$ then
\[
\min \{ \pont{\matriz}{\A}{\C}, \pont{\matriz}{\A}{\espaco} +
\pont{\matriz}{\espaco}{\C} \} \le \pont{\matriz}{\A}{\B} +
\pont{\matriz}{\B}{\C}\,,
\]
for each $\A, \B, \C \in \alphabet$.
\end{proposition}
\begin{proof}
Since $\distanciaN{\gamma} \in \prametrica$, from  Corollary~\ref{corolaryPrametrica} we have that
$\pont{\gamma}{\A}{\B}, \pont{\gamma}{\A}{\C}, \pont{\gamma}{\B}{\C} \ge 0$ for each $\A, \B, \C \in \Sigma$.

If $\mathcal{Q} = 0$, from Proposition~\ref{Q=0}, we have 
$\pont{\gamma}{\A}{\espaco} = \pont{\gamma}{\espaco}{\A} = 0$
for each $\A \in \Sigma$ since $\distanciaN{\gamma} \in \prametrica$.
Since $\pont{\gamma}{\A}{\B}, \pont{\gamma}{\A}{\C}, \pont{\gamma}{\B}{\C} \ge 0$ for each $\A, \B, \C \in \Sigma$,
it follows that 
$\min \{ \pont{\matriz}{\A}{\C}, \pont{\matriz}{\A}{\espaco} +
\pont{\matriz}{\espaco}{\C} \} = 0 \le \pont{\matriz}{\A}{\B} +
\pont{\matriz}{\B}{\C}$ and the proposition is proved. 
Thus, assume that $\mathcal{Q} \not= 0$.

Suppose that
$\distanciaN{\matriz}(s, t) \le \distanciaN{\matriz}(s, u) +
\distanciaN{\matriz}(u, t)$ for each $s, t, u \in\alphabet^{*}$. 
We have that
$\pont{\matriz}{\A}{\espaco} \le \pont{\matriz}{\A}{\maior} +
\pont{\matriz}{\maior}{\espaco}$ 
and 
$\pont{\matriz}{\espaco}{\C} \le \pont{\matriz}{\espaco}{\maior} +
\pont{\matriz}{\maior}{\C}$ 
for each $\A, \C \in \alphabet$,
from
Proposition~\ref{AuxdistanciaNTriangular-1}. Let
$n_{0}$ be an integer satisfying Proposition~\ref{propAux}
and Fact~\ref{fato2}. Then, for $n \ge n_{0}$, it follows that
\begin{align*}
\min \left\{ 
\frac{\pont{\matriz}{\A}{\C}}{n+1},
\frac{\pont{\matriz}{\A}{\espaco} + \pont{\matriz}{\espaco}{\C}}{n+2}
\right\} &= \distanciaN{\matriz}(\maior^{n}\A, \maior^{n}\C) \\
 &\le \distanciaN{\matriz}(\maior^{n}\A, \maior^{n}\B) 
+ \distanciaN{\matriz}(\maior^{n}\B, \maior^{n}\C) \\
&\le \frac{\pont{\matriz}{\A}{\B}}{n+1} +
\frac{\pont{\matriz}{\B}{\C}}{n+1}\,.
\end{align*}
This implies that 
if $\distanciaN{\gamma}(q^n \A,q^n \C) = 
\pont{\gamma}{\A}{\C}/(n+1)$, then 
$\pont{\gamma}{\A}{\C} \le \pont{\gamma}{\A}{\B} + \pont{\gamma}{\B}{\C}$ and the proposition is proved. 
\begin{comment}


Consider now that $n_{0}$ also satisfies Fact~\ref{fato3}, i.e., that
\[
\pont{\matriz}{\A}{\espaco} + \pont{\matriz}{\espaco}{\C} \le
\pont{\matriz}{\A}{\B} + \pont{\matriz}{\B}{\C} \quad \text{if and
  only if} \quad \frac{\pont{\matriz}{\A}{\espaco} +
  \pont{\matriz}{\espaco}{\C}}{n+2} < \frac{\pont{\matriz}{\A}{\B} +
  \pont{\matriz}{\B}{\C}}{n+1}\,,
\]
for each $n \ge n_{0}$.

Suppose that
\[
\frac{\pont{\matriz}{\A}{\C}}{n+1} = \min \left\{
\frac{\pont{\matriz}{\A}{\C}}{n+1}, \frac{\pont{\matriz}{\A}{\espaco}
  + \pont{\matriz}{\espaco}{\C}}{n+2} \right\} \le
\frac{\pont{\matriz}{\A}{\B}}{n+1} +
\frac{\pont{\matriz}{\B}{\C}}{n+1}\,.
\]
In this case, since $n > 0$, we have that $\pont{\matriz}{\A}{\C} \le
\pont{\matriz}{\A}{\B} + \pont{\matriz}{\B}{\C}$ and the proof is
done.
\end{comment}
Then, assume that
\[
\frac{\pont{\matriz}{\A}{\espaco} + \pont{\matriz}{\espaco}{\C}}{n+2}
= \min \left\{ \frac{\pont{\matriz}{\A}{\C}}{n+1},
\frac{\pont{\matriz}{\A}{\espaco} + \pont{\matriz}{\espaco}{\C}}{n+2}
\right\} \le \frac{\pont{\matriz}{\A}{\B}}{n+1} +
\frac{\pont{\matriz}{\B}{\C}}{n+1}\,.
\]
It follows from contrapositive of Fact~\ref{fato2} that 
$\pont{\gamma}{\A}{\espaco} + \pont{\gamma}{\espaco}{\A} \le \pont{\gamma}{\A}{\B}+\pont{\gamma}{\B}{\C}$ since $n$ is big enough and the proposition is proved.


\begin{comment}
From Fact~\ref{fato3}, we have that $\pont{\matriz}{\A}{\espaco} +
\pont{\matriz}{\espaco}{\C} \le \pont{\matriz}{\A}{\B} +
\pont{\matriz}{\B}{\C}$ and the proof is also done.

Finally, suppose that 
\[
\frac{\pont{\matriz}{\A}{\C}}{n+1} \not\le
\frac{\pont{\matriz}{\A}{\B}}{n+1} +
\frac{\pont{\matriz}{\B}{\C}}{n+1} \quad \text{and} \quad
\frac{\pont{\matriz}{\A}{\espaco} + \pont{\matriz}{\espaco}{\C}}{n+2}
\not< \frac{\pont{\matriz}{\A}{\B}}{n+1} +
\frac{\pont{\matriz}{\B}{\C}}{n+1}\,,
\]
for each $n \ge n_{0}$. In this case we have that
\[
\frac{\pont{\matriz}{\A}{\espaco} + \pont{\matriz}{\espaco}{\C}}{n+2}
= \min \left\{ \frac{\pont{\matriz}{\A}{\C}}{n+1},
\frac{\pont{\matriz}{\A}{\espaco} + \pont{\matriz}{\espaco}{\C}}{n+2}
\right\} = \frac{\pont{\matriz}{\A}{\B}}{n+1} +
\frac{\pont{\matriz}{\B}{\C}}{n+1}\,,
\]
for each $n \ge n_{0}$. It implies that $\pont{\matriz}{\A}{\espaco} =
\pont{\matriz}{\espaco}{\C} = \pont{\matriz}{\A}{\B} =
\pont{\matriz}{\B}{\C}=0$ and, thus, it is also true that
$\pont{\matriz}{\A}{\espaco} + \pont{\matriz}{\espaco}{\C} =
\pont{\matriz}{\A}{\B}+ \pont{\matriz}{\B}{\C}$.
\end{comment}
\end{proof}

\begin{lemma}\label{distanciaNTriangular}\rm
Let $\distanciaN{\matriz} \in \prametrica$. Then,
$\distanciaN{\matriz}(s, t) \le \distanciaN{\matriz} (s, u) +
\distanciaN{\matriz} (u, t)$ for each $s, t, u \in \alphabet^{*}$ if
and only if
\begin{enumerate}
\item [(\textit{i})] $\pont{\matriz}{\A}{\espaco} \le
  \pont{\matriz}{\A}{\B} + \pont{\matriz}{\B}{\espaco}$\,,
\item [(\textit{ii})] $\pont{\matriz}{\espaco}{\A} \le
  \pont{\matriz}{\espaco}{\B} + \pont{\matriz}{\B}{\A}$\,,
\item [(\textit{iii})] $\min \{ \pont{\matriz}{\A}{\C},
  \pont{\matriz}{\A}{\espaco} + \pont{\matriz}{\espaco}{\C} \} \le
  \pont{\matriz}{\A}{\B} + \pont{\matriz}{\B}{\C}$\,, and
\item [(\textit{iv})] $\max \{ \pont{\matriz}{\A}{\espaco},
  \pont{\matriz}{\espaco}{\A} \} \le \pont{\matriz}{\B}{\espaco} +
  \pont{\matriz}{\espaco}{\B}$\,,
\end{enumerate}
for each $\A, \B, \C \in \alphabet$.
\end{lemma}
\begin{proof}
Suppose that $\distanciaN{\matriz}(s, t) \le
\distanciaN{\matriz}(s, u) + \distanciaN{\matriz}(u, t)$ for all $s,
t, u \in \alphabet^{*}$. It follows from
Propositions~\ref{AuxdistanciaNTriangular-1},
\ref{AuxdistanciaNTriangular}, and~\ref{AuxdistanciaNTriangular-2}
that conditions (\textit{i}) to (\textit{iv})
are true.

Conversely, suppose that conditions (\textit{i}) to
(\textit{iv}) are true. Let $s, t, u \in \alphabet^{*}$ and $A, B$ be
N-optimal alignments of $s, u$ and $u, t$, respectively. It follows
from Proposition~\ref{desigTriangularGeral} that there exists an
alignment $C$ of $s, t$ and an integer $k \ge 0$ such that
\[
\custoA{\matriz}[A] + \custoA{\matriz}[B] \ge \custoA{\matriz}[C] + k
\Maior\,, \quad \tamanho{A} \le \tamanho{C} + k\,, \quad \text{and} \quad
\tamanho{B} \le \tamanho{C} + k\,.
\]
As a consequence of $\distanciaN{\matriz} \in \prametrica$, we have
$\custoA{\matriz}[A], \custoA{\matriz}[B] \ge 0$. It follows that
\[
\distanciaN{\matriz} (s, u) + \distanciaN{\matriz}(u,t) =
\frac{\custoA{\matriz}[A]}{\tamanho{A}} +
\frac{\custoA{\matriz}[B]}{\tamanho{B}} \ge \frac{\custoA{\matriz}[C]
  + k \Maior}{\tamanho{C} + k}\,.
\]
Since $\abs{C} >0$ and $k \ge 0$, we have from Fact~\ref{fato1} that
%\begin{comment}
\begin{align*}
    \frac{\custoA{\matriz}[C]
  + k \Maior}{\tamanho{C} + k}  & \ge \min \left\{ \mathcal{Q}, \frac{\custoA{\matriz}[C]}{\tamanho{C}} = \custoN{\gamma}[C]
  \right\}\,,
\end{align*}
%\end{comment}
$\mathcal{Q} \ge \distanciaN{\gamma}(s, t)$ by Proposition~\ref{propAux2},
and $\custoN{\gamma}[C] \ge \distanciaN{\gamma}(s, t)$. Hence, 
\begin{comment}
Consequently, to prove the lemma, it is enough to prove that
\[
\frac{\custoA{\matriz}[C] + k \Maior}{\tamanho{C} + k} \ge
\distanciaN{\matriz}(s, u)\,.
\]

We have now two possibilities: 
\begin{align*}
(\custoA{\matriz}[C] + k \Maior)/(\tamanho{C} + k) &\ge 
\custoA{\matriz}[C] / \tamanho{C}\,, \text{or} \\
(\custoA{\matriz}[C] + k \Maior)/(\tamanho{C} + k) &< 
\custoA{\matriz}[C] / \tamanho{C}\,.
\end{align*}

If $(\custoA{\matriz}[C] + k \Maior)/(\tamanho{C} + k) \ge
\custoA{\matriz}[C] / \tamanho{C}$ then, since $C$ is an alignment of
$s, u$, we have that
\[
\frac{\custoA{\matriz}[C] + k \Maior}{\tamanho{C} + k} \ge
\frac{\custoA{\matriz}[C]}{\tamanho{C}} \ge \distanciaN{\matriz}(s, u)\,,
\]
and the proof is done.

On the other hand, if $(\custoA{\matriz}[C] + k \Maior)/(\tamanho{C} +
k) < \custoA{\matriz}[C] / \tamanho{C}$, since $k \ge 0$ and
$\tamanho{C} \ge 0$, we have that $\custoA{\matriz}[C] > \tamanho{C}
\Maior$. Since, $\distanciaN{\matriz}(s, u) \le \Maior$ from
Proposition~\ref{propAux2}, it follows that
\[
\frac{\custoA{\matriz}[C] + k \Maior}{\tamanho{C} + k} >
\frac{ \tamanho{C} \Maior + k \Maior}{\tamanho{C} + k} =
\Maior \ge \distanciaN{\matriz}(s, u)\,.
\]
\end{comment}
\[
\distanciaN{\matriz} (s, t) \le \distanciaN{\matriz}(s,u) +
\distanciaN{\gamma}(u, t)\,.
\]
\end{proof}

Finally, using the results presented in this section, we can establish the following proof of the highlighted theorem.

\begin{proof} (of Theorem~\ref{theo:norm})

Suppose that $\distanciaN{\matriz} \in \metrica$. Consequently, for each $s, t, u \in \alphabet^{*}$, we have that $\distanciaN{\matriz}(s, s) = 0$,
$\distanciaN{\matriz}(s, t) > 0$ if $s \not= t$, $\distanciaN{\matriz}(s, t) =
\distanciaN{\matriz}(t, s)$, and $\distanciaN{\matriz}(s, u) \le
\distanciaN{\matriz}(s, t) + \distanciaN{\matriz}(t, u)$.

Let $\A, \B, \C \in \alphabet$, $\A \neq \B \neq \C$. Since
$\distanciaN{\matriz}(s, s) = 0$ and $\distanciaN{\matriz}(s, t) > 0$,
we have that $\distanciaN{\matriz} \in \prametrica$. Since
$\distanciaN{\matriz}(s, t) > 0$
if $s \not= t$ and $\distanciaN{\matriz} \in
\prametrica$, from Lemma~\ref{lema-maiorQueZero}, we have that
$\pont{\matriz}{\A}{\espaco}, \pont{\matriz}{\espaco}{\A},
\pont{\matriz}{\A}{\B} > 0$. Since $\distanciaN{\matriz} \in
\prametrica$, it follows from Corollary~\ref{corolaryPrametrica} that
$\pont{\matriz}{\A}{\A} = 0$ or $\pont{\matriz}{\A}{\espaco} +
\pont{\matriz}{\espaco}{\A} = 0$ and, since
$\pont{\matriz}{\A}{\espaco}, \pont{\matriz}{\espaco}{\A} > 0$, we
have that $\pont{\matriz}{\A}{\A} = 0$. Since $\distanciaN{\matriz}(s,
t) = \distanciaN{\matriz}(t, s)$ for each $s, t \in \alphabet^{*}$ and
$\distanciaN{\matriz} \in \prametrica$, from
Lemma~\ref{distanciaNSimetrica} we have that
$\pont{\matriz}{\A}{\espaco} = \pont{\matriz}{\espaco}{\A}$, and if
$\pont{\matriz}{\A}{\B} < \pont{\matriz}{\A}{\espaco}+
\pont{\matriz}{\espaco}{\B}$, then $\pont{\matriz}{\A}{\B} =
\pont{\matriz}{\B}{\A}$. Since $\distanciaN{\matriz}(s, u) \le
\distanciaN{\matriz}(s, t) + \distanciaN{\matriz}(t, u)$ and
$\distanciaN{\matriz} \in \prametrica$, from
Lemma~\ref{distanciaNTriangular} we have that
$\pont{\matriz}{\A}{\espaco} \le \pont{\matriz}{\A}{\B} +
\pont{\matriz}{\B}{\espaco}$ and $\min \{\pont{\matriz}{\A}{\C},
\pont{\matriz}{\A}{\espaco} + \pont{\matriz}{\espaco}{\C} \} \le
\pont{\matriz}{\A}{\B} + \pont{\matriz}{\B}{\C}$.

From the observations above, we have that if $\matriz \in \metricaN$,
then $\matriz \in \metricaA$. Besides that, since
$\pont{\matriz}{\A}{\espaco} = \pont{\matriz}{\espaco}{\A}$,
$\pont{\matriz}{\B}{\espaco} = \pont{\matriz}{\espaco}{\B}$ and $\max
\{ \pont{\matriz}{\A}{\espaco}, \pont{\matriz}{\espaco}{\A} \} \le
\pont{\matriz}{\B}{\espaco} + \pont{\matriz}{\espaco}{\B}$ from
Lemma~\ref{distanciaNTriangular}, we have that
\[
\pont{\matriz}{\A}{\espaco} = \max \{ \pont{\matriz}{\A}{\espaco},
\pont{\matriz}{\espaco}{\A} \} \le \pont{\matriz}{\B}{\espaco} +
\pont{\matriz}{\espaco}{\B} = \pont{\matriz}{\B}{\espaco}
+\pont{\matriz}{\B}{\espaco} = 2\,\pont{\matriz}{\B}{\espaco}\,.
\]

Conversely, suppose that $\matriz \in \metricaN$. By the definition of
$\metricaN$, we have that $\matriz \in \metricaA$ and
$\pont{\matriz}{\A}{\espaco} \le 2 \, \pont{\matriz}{\B}{\espaco}$
for each $\A, \B \in \alphabet$. Since $\matriz \in \metricaA$, we
have that $\distanciaA{\matriz} \in \metrica \subseteq \prametrica$
and $\pont{\matriz}{\A}{\espaco} = \pont{\matriz}{\espaco}{\A} > 0$;
$\pont{\matriz}{\A}{\B} > 0$ if $\A \not= \B$, and
$\pont{\matriz}{\A}{\B} = 0$ if $\A = \B$; if $\pont{\matriz}{\A}{\B}
< \pont{\matriz}{\A}{\espaco} + \pont {\matriz}{\espaco}{\B}$, then
$\pont{\matriz}{\A}{\B} = \pont{\matriz}{\B}{\A}$;
$\pont{\matriz}{\A}{\espaco} \le \pont{\matriz}{\A}{\B} +
\pont{\matriz}{\B}{\espaco}$; $\min \{ \pont{\matriz}{\A}{\C},
\pont{\matriz}{\A}{\espaco} + \pont {\matriz}{\espaco}{\C} \} \le
\pont{\matriz}{\A}{\B} + \pont{\matriz}{\B}{\C}$ for each $\A, \B, \C
\in \alphabet$.  In order to prove that $\distanciaN{\matriz} \in
\metrica$, we show, for each $s, t, u \in \alphabet^{*}$, that
$\distanciaN{\matriz}(s, s) = 0$, $\distanciaN{\matriz}(s, t) > 0$ for
$s \not= t$, $\distanciaN{\matriz}(s, t) =\distanciaN{\matriz}(t, s)$,
and $\distanciaN{\matriz}(s, u) \le \distanciaN{\matriz}(s, t) +
\distanciaN{\matriz}(t, u)$.

Since $\distanciaA{\matriz} \in \prametrica$, we have from
Lemma~\ref{lemaPrametricaN} that $\distanciaN{\matriz} \in
\prametrica$ and, thus, $\distanciaN{\matriz}(s, s) = 0$. Since
$\distanciaN{\matriz} \in \prametrica$, $\pont{\matriz}{\A}{\espaco} =
\pont{\matriz}{\espaco}{\A} > 0$, and $\pont{\matriz}{\A}{\B} > 0$ if
$\A \not= \B$, consequently from Lemma~\ref{lema-maiorQueZero} we have
that $\distanciaN{\matriz}(s, t) > 0$ for $s \not= t$. Since
$\distanciaN{\matriz} \in \prametrica$, $\pont{\matriz}{\A}{\espaco} =
\pont{\matriz}{\espaco}{\A}$, and 
$\pont{\matriz}{\A}{\B} = \pont{\matriz}{\B}{\A}$ if $\pont{\matriz}{\A}{\B} <
\pont{\matriz}{\A}{\espaco} + \pont {\matriz}{\espaco}{\B}$, we have from
Lemma~\ref{distanciaNSimetrica} that $\distanciaN{\matriz}(s,
t) =\distanciaN{\matriz}(t, s)$.

Since $\pont{\matriz}{\B}{\A} < \pont{\matriz}{\B}{\espaco} +
\pont{\matriz}{\espaco}{\A}$, we have that
$\pont{\matriz}{\B}{\A} = \pont{\matriz}{\A}{\B}$ whereas $\matriz \in \metricaA$.
Since $\pont{\matriz}{\A}{\espaco} = \pont{\matriz}{\espaco}{\A}$,
$\pont{\matriz}{\B}{\espaco} = \pont{\matriz}{\espaco}{\B}$, and
$\pont{\matriz}{\A}{\espaco} \le \pont{\matriz}{\A}{\B} +
\pont{\matriz}{\B}{\espaco}$, we have that
\[
\pont{\matriz}{\espaco}{\A} = \pont{\matriz}{\A}{\espaco} \le 
\pont{\matriz}{\A}{\B} +
\pont{\matriz}{\B}{\espaco} = \pont{\matriz}{\espaco}{\B} + 
\pont{\matriz}{\B}{\A}\,.
\]
Since $\pont{\matriz}{\A}{\espaco} = \pont{\matriz}{\espaco}{\A}$ and
$\pont{\matriz}{\B}{\espaco} = \pont{\matriz}{\espaco}{\B}$, it follows
from hypothesis that
\[
\max \{ \pont{\matriz}{\A}{\espaco}, \pont{\matriz}{\espaco}{\A} \} =
\pont{\matriz}{\A}{\espaco} \le 2\,\pont{\matriz}{\B}{\espaco} =
\pont{\matriz}{\B}{\espaco} + \pont{\matriz}{\espaco}{\B}\,.
\]

Since $\pont{\matriz}{\A}{\espaco} \le \pont{\matriz}{\A}{\B} +
\pont{\matriz}{\B}{\espaco}$, $\pont{\matriz}{\espaco}{\A} \le
\pont{\matriz}{\espaco}{\B} + \pont{\matriz}{\B}{\A}$, $\min \{
\pont{\matriz}{\A}{\C}, \pont{\matriz}{\A}{\espaco} +
\pont{\matriz}{\espaco}{\C} \} \le \pont{\matriz}{\A}{\B} +
\pont{\matriz}{\B}{\C},$ and $\max \{ \pont{\matriz}{\A}{\espaco},
\pont{\matriz}{\espaco}{\A} \} \le \pont{\matriz}{\B}{\espaco} +
\pont{\matriz}{\espaco}{\B}$, we have that $\distanciaN{\matriz}(s, u)
\le \distanciaN{\matriz}(s, t) + \distanciaN{\matriz}(t, u)$ from
Lemma~\ref{distanciaNTriangular}.
\end{proof}

Table~\ref{tabela3} summarizes properties of scoring matrices $\gamma$ for metric space and some other generalized metric space. 

\begin{table}[htpb]
\begin{minipage}{\textwidth}
\begin{center}
\begin{tabular}{clcccccc}
  & & $\prametrica$ & $\semimetrica$ & $\hemimetrica$ &
  $\pseudometrica$ & $\quasimetrica$ & $\metrica$ \\ \hline & & \\
  (a) & $\distanciaA{\gamma}$ is a premetric & \yes & \yes & \yes & \yes
  & \yes & \yes \\ & & \\
  (b) & $\pont{\matriz}{\A}{\espaco}, \pont{\matriz}{\espaco}{\A} > 0$
  and $\pont{\matriz}{\A}{\B} > 0$ if $\A \not= \B$ & & \yes & & &\yes
  & \yes \\& & \\
  (c) & $\pont{\matriz}{\A}{\espaco} = \pont{\matriz}{\espaco}{\A}$ &
  & \yes & & \yes & & \yes \\ & & \\
  (d) &
  \begin{tabular}{l}
    if $\pont{\matriz}{\A}{\B} < \pont{\matriz}{\A}{\espaco}
    + \pont{\matriz}{\espaco}{\B}$ then \\
    $\pont{\matriz}{\A}{\B} = \pont{\matriz}{\B}{\A}$
  \end{tabular}  
  & & \yes & & \yes & & \yes \\ & & \\
  (e) & $\pont{\matriz}{\A}{\espaco} \le \pont{\matriz}{\A}{\B} +
  \pont{\matriz}{\B}{\espaco}$ & & & \yes & \yes &\yes & \yes \\ & &
  \\
  (f) & $\pont{\matriz}{\espaco}{\A} \le \pont{\matriz}{\espaco}{\B} +
  \pont{\matriz}{\B}{\A}$ & & & \yes & \yes &\yes & \yes\\ & & \\
  (g) & 
  $\min \Big\{ \begin{array}{l}
    \pont{\matriz}{\A}{\C}, \\
    \pont{\matriz}{\A}{\espaco} + \pont{\matriz}{\espaco}{\C} 
  \end{array} \Big\} 
  \le \pont{\matriz}{\A}{\B} + \pont{\matriz}{\B}{\C}$
  &  &  & \yes & \yes &\yes & \yes\\ & & \\
  (h) &   $\max \{ \pont{\gamma}{\A}{\espaco}, \pont{\gamma}{\espaco}{\A} \}
  \le \pont{\matriz}{\B}{\espaco} + \pont{\matriz}{\espaco}{\B}$ & & & \yes & \yes &\yes & \yes
\end{tabular}
\end{center}
\end{minipage}
\caption{Necessary and sufficient conditions for scoring matrix
  $\matriz$ to induce $\distanciaN{\matriz}\text{-}p$ on sequences when $\distanciaN{\matriz} \in 
  \prametrica$.
  As in Table~\ref{tabela2},  
  the properties are also used to define metric ($\metrica$) and 
  generalized metric spaces
  such as \emph{premetric} ($\prametrica$),
  \emph{semimetric} ($\semimetrica$), \emph{hemimetric}
  ($\hemimetrica$), \emph{pseudometric} ($\pseudometrica$) and 
  \emph{quasimetric} ($\quasimetrica$).
  Results are obtained using
  definitions presented in Section~\ref{sec:preliminares} and
  lemmas in Section~\ref{sec:normalizado}.} \label{tabela3}
\end{table}

