%% 
%% Copyright 2007-2020 Elsevier Ltd
%% 
%% This file is part of the 'Elsarticle Bundle'.
%% ---------------------------------------------
%% 
%% It may be distributed under the conditions of the LaTeX Project Public
%% License, either version 1.2 of this license or (at your option) any
%% later version.  The latest version of this license is in
%%    http://www.latex-project.org/lppl.txt
%% and version 1.2 or later is part of all distributions of LaTeX
%% version 1999/12/01 or later.
%% 
%% The list of all files belonging to the 'Elsarticle Bundle' is
%% given in the file `manifest.txt'.
%% 
%% Template article for Elsevier's document class `elsarticle'
%% with harvard style bibliographic references

\documentclass[preprint,12pt,authoryear]{elsarticle}

%% Use the option review to obtain double line spacing
%% \documentclass[authoryear,preprint,review,12pt]{elsarticle}

%% Use the options 1p,twocolumn; 3p; 3p,twocolumn; 5p; or 5p,twocolumn
%% for a journal layout:
%% \documentclass[final,1p,times,authoryear]{elsarticle}
%% \documentclass[final,1p,times,twocolumn,authoryear]{elsarticle}
%% \documentclass[final,3p,times,authoryear]{elsarticle}
%% \documentclass[final,3p,times,twocolumn,authoryear]{elsarticle}
%% \documentclass[final,5p,times,authoryear]{elsarticle}
%% \documentclass[final,5p,times,twocolumn,authoryear]{elsarticle}

%% For including figures, graphicx.sty has been loaded in
%% elsarticle.cls. If you prefer to use the old commands
%% please give \usepackage{epsfig}

%% The amssymb package provides various useful mathematical symbols
\usepackage{amssymb}
%% The amsthm package provides extended theorem environments
\usepackage{amsthm}
\usepackage{amsmath}

%% Colors
\usepackage{color}

%% The lineno packages adds line numbers. Start line numbering with
%% \begin{linenumbers}, end it with \end{linenumbers}. Or switch it on
%% for the whole article with \linenumbers.
%% \usepackage{lineno}

\usepackage{comment}

\journal{Discrete Applied Mathematics}

\usepackage{hyperref}
\hypersetup{colorlinks=true,linkcolor=blue,filecolor=magenta,urlcolor=cyan}
\urlstyle{same}

% Definitions

\newtheorem{fact}{Fact}
\newtheorem{lemma}[fact]{Lemma}
\newtheorem{proposition}[fact]{Proposition}
\newtheorem{theorem}[fact]{Theorem}
\newtheorem{corolary}[fact]{Corollary}

% \def\proof{\noindent \textbf{Proof}. \ignorespaces}
% \def\endproof{\vspace*{-.2cm}\hfill $_{\Box}$ \vspace*{5mm}}
% \spnewtheorem{lema}[theorem]{Lemma}{\bfseries}{\rmfamily}
% \spnewtheorem{lema}[theorem]{Problem}{\bfseries}{\rmfamily}

% sets
\newcommand{\alphabet}{\Sigma}
\newcommand{\cjtoAlinha}[1]{\mathcal{A}_{{#1}}}
\newcommand{\cjtoAlinhaExt}[1]{\mathcal{E}_{{#1}}}

\newcommand{\alinhamento}[1]{[{#1}]}
\newcommand{\alinhamentoA}[1]{\left[ {#1} \right]}
\newcommand{\alinhamentoB}[1]{\Big[ {#1} \Big]}
\newcommand{\Par}[2]{({#1},{#2})}
\newcommand{\mi}{\mu}

\DeclareMathAlphabet{\mathpzc}{OT1}{pzc}{m}{it}

\newcommand{\fonte}{\mathpzc}
\newcommand{\fonteDois}{\mathbb}
\newcommand{\metricaC}{\fonteDois{M}^{\text{C}}}
\newcommand{\metricaE}{\fonteDois{M}^{\text{E}}}
\newcommand{\metricaA}{\fonteDois{M}^{\text{A}}}
\newcommand{\metricaN}{\fonteDois{M}^{\text{N}}}

\newcommand{\metrica}{\fonte{M}}
\newcommand{\prametrica}{\fonte{Pr}}
\newcommand{\prametricaE}{\fonte{R}^{\mathtt{E}}}
\newcommand{\prametricaA}{\fonte{R}^{\mathtt{A}}}
\newcommand{\prametricaN}{\fonte{R}^{\mathtt{N}}}

\newcommand{\hemimetrica}{\fonte{H}}
\newcommand{\hemimetricaE}{\fonte{H}^{\mathtt{E}}}
\newcommand{\hemimetricaA}{\fonte{H}^{\mathtt{A}}}
\newcommand{\hemimetricaN}{\fonte{H}^{\mathtt{N}}}

\newcommand{\pseudometrica}{\fonte{P}}
\newcommand{\pseudometricaE}{\fonte{P}^{\mathtt{E}}}
\newcommand{\pseudometricaA}{\fonte{P}^{\mathtt{A}}}
\newcommand{\pseudometricaN}{\fonte{P}^{\mathtt{N}}}

\newcommand{\semimetrica}{\fonte{S}}
\newcommand{\semimetricaE}{\fonte{S}^{\mathtt{E}}}
\newcommand{\semimetricaA}{\fonte{S}^{\mathtt{A}}}
\newcommand{\semimetricaN}{\fonte{S}^{\mathtt{N}}}

\newcommand{\quasimetrica}{\fonte{Q}}
\newcommand{\quasimetricaE}{\fonte{Q}^{\mathtt{E}}}
\newcommand{\quasimetricaA}{\fonte{Q}^{\mathtt{A}}}
\newcommand{\quasimetricaN}{\fonte{Q}^{\mathtt{N}}}

% functions
\newcommand{\matriz}{\gamma}
\newcommand{\matrizDois}{\delta}
\newcommand{\matrizTres}{\sigma}
\newcommand{\pont}[3]{{#1}_{{#2} \rightarrow {#3}}}
\newcommand{\tamanho}[1]{|{#1}|}
\newcommand{\grauE}[1]{\delta^{-}({#1})}
\newcommand{\grauS}[1]{\delta^{+}({#1})}

\newcommand{\opt}{\ensuremath\protect\mathrm{opt}}
\newcommand{\cost}{\ensuremath\protect{w}}

\newcommand{\distanciaA}[1]{\opt{\rm A}_{{#1}}}
\newcommand{\distanciaN}[1]{\opt{\rm N}_{{#1}}}
\newcommand{\distanciaE}[1]{\opt{\rm E}_{{#1}}}

\newcommand{\custoA}[1]{v{\rm A}_{{#1}}}
\newcommand{\custoN}[1]{v{\rm N}_{{#1}}}
\newcommand{\custoE}[1]{v{\rm E}_{{#1}}}

% symbols
\newcommand{\seqVazia}{\varepsilon}
\newcommand{\A}{\texttt{a}}
\newcommand{\B}{\texttt{b}}
\newcommand{\C}{\texttt{c}}
\newcommand{\D}{\texttt{d}}
\newcommand{\E}{\texttt{e}}
\newcommand{\F}{\texttt{f}}
% \newcommand{\espaco}{\mbox{\tt\char`\ }}
\newcommand{\espaco}{\mbox{-}}

\newcommand{\yes}{$\star$}
\newcommand{\Maior}{\mathcal{Q}}
\newcommand{\maior}{q}

\newcommand{\abs}[1]{\left\vert#1\right\vert}

\newcommand{\portugues}[1]{}

\newcommand\EA[1]{{\color{blue}(EA: #1)}}
\newcommand\FM[1]{{\color{red}(FM: #1)}}
\newcommand\CH[1]{{\color{magenta}(CH: #1)}}

\begin{document}

\begin{frontmatter}

%% Title, authors and addresses

%% use the tnoteref command within \title for footnotes;
%% use the tnotetext command for theassociated footnote;
%% use the fnref command within \author or \affiliation for footnotes;
%% use the fntext command for theassociated footnote;
%% use the corref command within \author for corresponding author footnotes;
%% use the cortext command for theassociated footnote;
%% use the ead command for the email address,
%% and the form \ead[url] for the home page:
%% \title{Title\tnoteref{label1}}
%% \tnotetext[label1]{}
%% \author{Name\corref{cor1}\fnref{label2}}
%% \ead{email address}
%% \ead[url]{home page}
%% \fntext[label2]{}
%% \cortext[cor1]{}
%% \affiliation{organization={},
%%            addressline={}, 
%%            city={},
%%            postcode={}, 
%%            state={},
%%            country={}}
%% \fntext[label3]{}

\title{Matrices inducing generalized metric on sequences}

%% use optional labels to link authors explicitly to addresses:
%% \author[label1,label2]{}
%% \affiliation[label1]{organization={},
%%             addressline={},
%%             city={},
%%             postcode={},
%%             state={},
%%             country={}}
%%
%% \affiliation[label2]{organization={},
%%             addressline={},
%%             city={},
%%             postcode={},
%%             state={},
%%             country={}}

\author[UFMS]{Eloi Araujo}
\ead{francisco.araujo@ufms.br}

\author[UFMS]{Fábio V.~Martinez}\corref{corr}
\ead{fabio.martinez@ufms.br}

\author[UFMS]{Carlos H.~A.~Higa}
\ead{carlos.aguena@ufms.br}

\author[USP]{José Soares}
\ead{jose@ime.usp.br}

\cortext[corr]{Corresponding author.}

\address[UFMS]{Faculdade de Computação, Universidade Federal de Mato Grosso do Sul, Brasil}

\address[USP]{Departamento de Ciência da Computação, Instituto de Matemática e Estatística, Universidade de São Paulo, Brasil}

\begin{abstract}
Sequence comparison is a basic task to capture similarities and 
differences between two or more sequences of symbols, with countless 
applications such as in computational biology. An alignment is a way to compare sequences, where a giving scoring function determines the degree of similarity between them. Many scoring functions are obtained from scoring matrices. However, not all scoring matrices induce scoring functions which are distances, since the scoring function is not necessarily a metric. In this work we establish necessary and sufficient conditions for scoring matrices to induce each one of the properties of a metric in weighted edit distances. For a subset of scoring matrices that induce normalized edit distances, we also characterize each class of scoring matrices inducing normalized edit distances. Furthermore, we define an extended edit distance, which takes into account a set of editing operations that transforms one sequence into another regardless of the existence of a usual corresponding alignment to represent them, describing a criterion to find a sequence of edit operations whose weight is minimum. Similarly, we determine the class of scoring matrices that induces extended edit
distances for each of the properties of a metric.
\end{abstract}


\begin{keyword}
  Scoring matrices \sep Scoring functions \sep Metrics \sep Alignments 
%% keywords here, in the form: keyword \sep keyword

%% PACS codes here, in the form: \PACS code \sep code

%% MSC codes here, in the form: \MSC code \sep code
%% or \MSC[2008] code \sep code (2000 is the default)

\end{keyword}

\end{frontmatter}

%% \linenumbers

%% main text
\section{Introduction}
\label{sec:introduction}
% \begin{itemize}
%     % Diffusion of FL
%     \item {\st{Diffusion of FL}}
%     % Security threats to FL
%     \item {\st{Security threats to FL with particular focus on model poisoning}}
%     % Limitations of existing countermeasures
%     \item {\st{Current countermeasures (e.g., KRUM) and their limitations}}
%     % Proposed method and its advantages
%     \item {\st{Intuitive description of the proposed method and its difference (i.e., advantages) w.r.t. state of the art}}
%     % Main contributions
%     \item {\st{Summary of the main contributions of this work}}
%     % Paper's structure and organization
%     \item {\st{Paper's structure and organization}}
% \end{itemize}

% Diffusion of FL
Recently, {\em federated learning} (FL) has emerged as the leading paradigm for training distributed, large-scale, and privacy-preserving machine learning (ML) systems~\cite{mcmahan2017googleai,mcmahan2017aistats}. 
The core idea of FL is to allow multiple edge clients to collaboratively train a shared, global model without disclosing their local private training data.
%Specifically, an FL system consists of a central server and many edge clients; 
A typical FL round involves the following steps: {\em(i)} the server randomly picks some clients and sends them the current, global model; {\em(ii)} each selected client locally trains its model with its own private data; then, it sends the resulting local model to the server;\footnote{Whenever we refer to global/local model, we mean global/local model {\em parameters}.} {\em(iii)} the server updates the global model by computing an \emph{aggregation function}, usually the average (FedAvg), on the local models received from clients.
% \begin{enumerate}
%     \item[{\em(i)}] the server sends the current, global model to the clients and appoints some of them for training;
%     \item[{\em(ii)}] each selected client locally trains its copy of the global model with its own private data; then, it sends the resulting local model back to the server;\footnote{Whenever we refer to global/local model, we mean global/local model {\em parameters}.}
%     \item[{\em(iii)}] the server updates the global model by computing an \emph{aggregation function} on the local models received from clients (by default, the average, also referred to as FedAvg~\cite{mcmahan2017aistats}).
% \end{enumerate}
This process goes on until the global model converges. %(e.g., after a certain number of rounds or other similar stopping criteria).
%\\
% The advantages of FL over the traditional, centralized learning paradigm are undoubtedly clear in terms of flexibility/scalability (clients can join/disconnect from the FL network dynamically), network communications (only model weights\footnote{We will use \textit{parameters} and \textit{weights} interchangeably.} are exchanged between clients and server), and privacy (each client's private training data is kept local at the client's end and not uploaded to the server).
\\
% Security threats to FL
%However, the growing adoption of FL also raises security concerns~\cite{costa2022covert}, particularly about its confidentiality, integrity, and availability.
Although its advantages over standard ML, FL also raises security concerns~\cite{costa2022covert}. %, particularly about its confidentiality, integrity, and availability~\cite{costa2022covert}.
% OLD, LONG VERSION
% Indeed, some work deals with privacy leakage that may expose the local data of some clients~\cite{melis2019sp}. 
% A large body of work, instead, investigates attacks that usually aim to detriment the predictive accuracy of the learned global model. For instance, \emph{data poisoning} attacks achieve this goal by letting an adversary pollute the training set of some corrupt FL clients with maliciously crafted examples~\cite{jagielski2018sp}.
% Similarly, in \emph{model poisoning} the attacker attempts to tweak the global model weights~\cite{bhagoji2019pmlr} by directly perturbing the local model's weights of some infected FL clients before these are sent to the central server for aggregation, usually via so-called Byzantine attacks. 
% It turns out that Byzantine model poisoning attacks severely impact standard FedAvg; therefore, more robust aggregation functions must be designed to make FL systems secure.
Here, we focus on \emph{untargeted model poisoning} attacks~\cite{bhagoji2019pmlr}, where an adversary attempts to tweak the global model weights %\footnote{We will use the terms \textit{parameters} and \textit{weights} interchangeably.} 
by directly perturbing the local model's parameters of some infected clients before these are sent to the central server for aggregation.
In doing so, the adversary aims to jeopardize the global model \textit{indiscriminately} at inference time.
Such model poisoning attacks severely impact standard FedAvg; therefore, more robust aggregation functions must be designed to secure FL systems.
\\
% In this paper, we focus on designing a novel robust aggregation scheme at the server's end to contrast the effect of Byzantine model poisoning attacks.
%
% Current countermeasures and their limitations
%Several countermeasures have been proposed in the literature to combat model poisoning attacks on FL systems.
% Some methods use simple statistics more robust than plain average to smooth the impact of malicious updates (e.g., Trimmed Mean and FedMedian~\cite{yin2018icml}). 
% Other defenses implement outlier detection techniques to discard malicious updates from the aggregation performed at the server's end. Those are either based on heuristics (e.g., Krum/Multi-Krum~\cite{blanchard2017nips} and Bulyan~\cite{mhamdi2018pmlr}) or data-driven approaches (e.g., K-means clustering~\cite{shen2016acm} or DnC via spectral analysis~\cite{shejwalkar2021ndss}). 
% Finally, some strategies rely on a centralized ``source of trust'' to spot potential malicious updates (e.g., FLTrust~\cite{cao2020fltrust}).
% Several countermeasures have been proposed in the literature to combat model poisoning attacks on FL systems, i.e., to discard possible malicious local updates from the aggregation performed at the server's end. 
% These techniques range from simple statistics more robust than plain average (e.g., Trimmed Mean and FedMedian~\cite{yin2018icml}) to outlier detection heuristics (e.g., Krum/Multi-Krum~\cite{blanchard2017nips} and Bulyan~\cite{mhamdi2018pmlr}) or data-driven approaches (e.g., spectral analysis via K-means clustering~\cite{shen2016acm} or spectral analysis), or methods based on ``source of trust'' (e.g., FLTrust~\cite{cao2020fltrust}).
% OLD, LONG VERSION
%Several countermeasures have been proposed in the literature to combat Byzantine model poisoning attacks on FL systems.
% Descriptive statistics
% For example, Trimmed Mean and FedMedian aggregate local model updates using more robust statistics than standard average~\cite{yin2018icml}.
%
% % Heuristics for outlier detection
% Many existing Byzantine-resilient strategies implement some outlier detection heuristics to discard the model updates sent by potentially malicious clients from the input of the aggregation function.
% One of the most popular heuristics is Krum~\cite{blanchard2017nips}.
% This strategy tries to mitigate the impact of Byzantine attacks by selecting as a global model the local model with the smallest sum of Euclidean distances to {\em all} the other local models.
% Although powerful, Krum requires the server to know (or, at least, estimate) the number of malicious FL clients upfront, which is generally impossible in a realistic attack scenario. %
% Moreover, Krum may become ineffective for complex, high-dimensional model parameter spaces due to the curse of dimensionality.
% Bulyan~\cite{mhamdi2018pmlr} tries to overcome this issue by combining Krum with a variant of Trimmed Mean.
% % Data-driven outlier detection
% Other strategies use data-driven outlier detection techniques -- e.g., via K-means clustering~\cite{shen2016acm} -- to spot potential malicious local model updates. 
% %For instance, Shen et al. propose to cluster local model updates with K-means and thus identify outliers.
%
% % Other techniques
% As far as the server is concerned, any local model received can be from a potential malicious client. 
% FLTrust~\cite{cao2020fltrust} assumes the server acts as a client, i.e., trains a local model on an additional {\em trustworthy} dataset at the server's end and compares it against all the local models from other clients. 
% This way, the server can rely on some ``source of trust'' when discarding potentially malicious clients.
%\\
% Limitations of existing Byzantine-resilient strategies
Unfortunately, existing defense mechanisms either rely on simple heuristics (e.g., Trimmed Mean and FedMedian by~\cite{yin2018icml}) or need strong and unrealistic assumptions to work effectively (e.g., foreknowledge or estimation of the number of malicious clients in the FL system, as for Krum/Multi-Krum~\cite{blanchard2017nips} and Bulyan~\cite{mhamdi2018pmlr}, which, however, cannot exceed a fixed threshold).
Furthermore, outlier detection methods using K-means clustering~\cite{shen2016acm} or spectral analysis like DnC~\cite{shejwalkar2021ndss} do not directly consider the temporal evolution of local model updates received.
Finally, strategies like FLTrust~\cite{cao2020fltrust} require the server to collect its own dataset and act as a proper client, thereby altering the standard FL protocol.
\\
% OLD, LONG VERSION
% Overall, existing Byzantine-resilient strategies are either simple heuristics (e.g., FedMedian) or, if they are more complex, they rely on strong and unrealistic assumptions to work effectively (e.g., knowing the number of malicious clients in the FL system in advance, as for Krum and alike).
% Furthermore, data-driven outlier detection methods do not consider the temporary evolution of local model updates received (e.g., K-means clustering). 
% Finally, strategies like FLTrust requires the server to collect its own dataset and act as a proper client, thereby altering the standard FL protocol.
%
% Description of the proposed method
This work introduces a novel pre-aggregation \textit{filter} robust to untargeted model poisoning attacks. Notably, this filter $(i)$ operates without requiring prior knowledge or constraints on the number of malicious clients and $(ii)$ inherently integrates temporal dependencies. 
The FL server can employ this filter as a preprocessing step before applying \textit{any} aggregation function, be it standard like FedAvg or robust like Krum or Bulyan.
Specifically, we formulate the problem of identifying corrupted updates as a multidimensional (i.e., matrix-valued) time series anomaly detection task. 
The key idea is that legitimate local updates, resulting from well-calibrated iterative procedures like stochastic gradient descent (SGD) with an appropriate learning rate, show \textit{higher predictability} compared to malicious updates. This hypothesis stems from the fact that the sequence of gradients (thus, model parameters) observed during legitimate training exhibit regular patterns, as validated in Section~\ref{subsec:intuition}. %until convergence. 
%This regularity may be more pronounced for smooth convex loss functions, but it can still be captured within an appropriate time window, even for more complex and convoluted loss surfaces. 
%We provide evidence of this claim in Appendix~B, where we show that the average mutual information (i.e., ``predictability''), calculated over pairs of legitimate model updates sent at different FL rounds, is significantly higher than the corresponding computation for a malicious client.
\\
Inspired by the matrix autoregressive (MAR) framework for multidimensional time series forecasting~\cite{chen2021je}, we propose the FLANDERS ({\em \textbf{F}ederated \textbf{L}earning meets \textbf{AN}omaly \textbf{DE}tection for a \textbf{R}obust and \textbf{S}ecure}) filter.
The main advantages of FLANDERS over existing strategies like FLDetector~\cite{zhao2020multivariate} are its resilience to large-scale attacks, where $50\%$ or more FL participants are hostile, and the capability of working under realistic non-iid scenarios.
We attribute such a capability to two key factors: $(i)$ FLANDERS works without knowing a priori the ratio of corrupted clients, and $(ii)$ it embodies temporal dependencies between intra- and inter-client updates, quickly recognizing local model drifts caused by evil players. Below, we summarize our main contributions:

\begin{itemize}
\item[{\em(i)}]
We provide empirical evidence that the sequence of models sent by legitimate clients is more predictable than those of malicious participants performing untargeted model poisoning attacks.
\\
\item[{\em(ii)}] 
We introduce FLANDERS, the first pre-aggregation filter for FL robust to untargeted model poisoning based on multidimensional time series anomaly detection.
\\
\item[{\em(iii)}] 
We integrate FLANDERS into Flower,\footnote{\scriptsize{\url{https://flower.dev/}}} a popular FL simulation framework for reproducibility.
\\
\item[{\em(iv)}] 
We show that FLANDERS improves the robustness of the existing aggregation methods under multiple settings: different datasets, client's data distribution (non-iid), models, and attack scenarios.
\\
\item[{\em(v)}] 
We publicly release all the implementation code of FLANDERS along with our experiments.\footnote{\scriptsize{\url{https://anonymous.4open.science/r/flanders_exp-7EEB}}}
\end{itemize}

% Paper's structure and organization
The remainder of the paper is structured as follows. %some related work and the current state-of-the-art solutions to security issues that FL entails. 
Section~\ref{sec:background} covers background and preliminaries. 
In Section~\ref{sec:related}, we discuss related work.
Section~\ref{sec:problem} and Section~\ref{sec:method} describe the problem formulation and the method proposed. % to tackle it. 
Section~\ref{sec:experiments} gathers experimental results. %, and Section~\ref{sec:limitations} discusses some limitations of this work.
Finally, we conclude in Section~\ref{sec:conclusion}.
 %discusses the limitations of this work and draws future research directions.
%reports conclusions and draws perspectives for future research directions.

%%%%%%% OLD %%%%%%%
%to overcome the resilience of Byzantine failures in distributed Stochastic Gradient Descent computations. 
% The strength of Krum is its time complexity, which is linear in the gradient dimension. 
% However, the robustness of the approach is guaranteed for gradient-based learning applications only when the majority of the clients are not compromised. 
% Besides, the aggregation mechanism of Krum, as well as that of similar methods, is robust from a coarse-grained perspective and does not provide solutions to errors and perturbations that may occur at inference time.
%A related approach to~\cite{blanchard2017nips} is the work of Su et al.~\cite{su2016dc}. Here, the authors propose an iterated approximate agreement to tackle a multi-layer scenario attacked by Byzantine agents. 
%However, the method works efficiently on the sole discrete context and it is inapplicable to continuous state environments.
%\gabri{Maybe, we should just talk about the main limitations of existing countermeasures without digging into their details (or, we can just mention Krum as this is the most popular one). I will move the description of all these methods to the Related Work section.}

\section{Notation and Preliminaries}\label{sec_prel}
Let $\mathbb{Z}_{>0}$ denote the set of positive integers and let $\mathbb{Z}_{[a,b]}$ denote the set of integers in the interval $[a,b]$. The $m\times m$ identity matrix is denoted by $I_m$ and its columns by $e_i$ for $i\in\mathbb{Z}_{[1,m]}$. We use $\mathbf{0}$ to denote a vector or a matrix of zeros of appropriate dimensions. For a sequence $\{z_k\}_{k=0}^{N-1}$ with $z_k\in\mathbb{R}^\eta$, we denote its stacked vector as $z = \begin{bmatrix}z_0^\top &z_1^\top & \dots & z_{N-1}^\top\end{bmatrix}^\top$ and a stacked window of it as $z_{[l,j]} = \begin{bmatrix}z_l^\top &z_{l+1}^\top & \dots & z_{j}^\top\end{bmatrix}^\top$ with $0\leq l<j$.\par
Persistence of excitation of a sequence and its extension to multiple sequences \cite{vanWaarde20} are defined as follows.
\begin{definition} The sequence \(\{z_k\}_{k=0}^{N-1}\), $z_k\in\mathbb{R}^{\eta}$, is said to be persistently exciting of order \(L\) if \(\textup{rank}(\mathscr{H}_{L}(z))=\eta L\), where $\mathscr{H}_L(z) = \begin{bmatrix}
		z_{[0,L-1]} & z_{[1,L]} & \cdots & z_{[N-L,N-1]}
	\end{bmatrix}$.
	\label{def_PE}
\end{definition}
\begin{definition}[\cite{vanWaarde20}]\label{def_cPE}
	The sequences $\{z_k^{(j)}\}_{k=0}^{N_j-1}$, with $z_k^{(j)}\in\mathbb{R}^\eta$ and $j\in\mathbb{Z}_{[1,r]}$, are said to be \textit{collectively persistently exciting} of order $L$ if rank$(\mathcal{H}_L(\mathscr{Z}))=\eta L$, where $\mathscr{Z} = \begin{bmatrix}
		(z^{(1)})^\top & \cdots & (z^{(r)})^\top
	\end{bmatrix}^\top,$ and
	\begin{equation*}
		\mathcal{H}_L(\mathscr{Z}) = \begin{bmatrix}
			\mathscr{H}_L(z^{(1)}) & \cdots & \mathscr{H}_L(z^{(r)})
		\end{bmatrix}.
	\end{equation*}
\end{definition}

\section{Alignment distance}\label{sec:regular}


In this section, we describe the classes of scoring matrices that
induce $\distanciaA{\matriz}$-$p$ on sequences for each axiom $p$ of a metric. We do it through a sequence of results, such as Lemmas~\ref{distancia=0}, \ref{distanciaMaiorOuIgualAZero}, \ref{distanciaMaiorQue0}, \ref{simetriaNormal}, and   \ref{desigualdadeTriangularNormal}, which are summarized in Table~\ref{tabela2} and allow us to characterize matrices that induces each of the more general metric functions described in Section~\ref{sec:preliminares}. Moreover, as a consequence of that, we conclude this section with the important result previously stated in Section~\ref{sec:preliminares} and depicted by the following:

\begin{theorem} \label{theo:align} \rm
Let $\alphabet$ be an alphabet and $\matriz$ be a scoring matrix.
Then $\distanciaA{\matriz}$ is a metric on $\alphabet^{*}$ if and only
if 
\begin{enumerate}
\item[(\textit{i})] $\pont{\matriz}{\A}{\espaco} =
  \pont{\matriz}{\espaco}{\A} > 0$\,,
\item[(\textit{ii})] $\pont{\matriz}{\A}{\B} > 0$ if $\A \not= \B$ and
  $\pont{\matriz}{\A}{\B} = 0$ if $\A = \B$\,,
\item[(\textit{iii})] if $\pont{\matriz}{\A}{\B} <
  \pont{\matriz}{\A}{\espaco} + \pont{\matriz}{\espaco}{\B}$, then
  $\pont{\matriz}{\A}{\B} = \pont{\matriz}{\B}{\A}$\,,
\item[(\textit{iv})] $\pont{\matriz}{\A}{\espaco} \le
  \pont{\matriz}{\A}{\B} + \pont{\matriz}{\B}{\espaco}$\,, and
\item[$(v)$] $\min\{ \pont{\matriz}{\A}{\C},
  \pont{\matriz}{\A}{\espaco} + \pont{\matriz}{\espaco}{\C} \} \le
  \pont{\matriz}{\A}{\B} + \pont{\matriz}{\B}{\C}$\,,
\end{enumerate}
for each $\A, \B, \C \in \alphabet$.
\end{theorem}

We need a sequence of auxiliary results, as seen below, to eventually prove this theorem at the end of the section. 

\begin{fact}\label{fciclonegativo}\rm
Let $\matriz$ be a scoring matrix. If $D(\matriz)$ has a negative
cycle, then there exists $s \in \alphabet^{*}$ such that
$\distanciaA{\matriz}(s, s) < 0$.
\end{fact}

\begin{proof}
Suppose that $D (\matriz)$ has a cycle $C = \A_{0}, \ldots,
\A_{m}$, $\A_i \in \Sigma_{\espaco}$ such that
$\A_0 = \A_m \not= \espaco$,
$\A_{i} \not= \A_{j}$ for each
$i, j > 0$ and $\cost(C) = -X < 0$.
Let $n$ be a nonnegative integer such that $n >
(\pont{\matriz}{\A_{1}}{\espaco} +
\pont{\matriz}{\espaco}{\A_{1}})/X$ and 
$s = (\A_0 \A_1 \ldots \A_m)^n \A_0$.
Therefore,
\[
\distanciaA{\matriz}(s, s) \le \custoA{\matriz} \alinhamentoB{
  \begin{array}{ccc}
    \begin{array}{c}
      \espaco\\ \A_{0}
    \end{array} &
    \Big(
    \begin{array}{ccccc}
      \A_{0} & \A_{1} & \ldots & \A_{m - 1} & \A_{m} \\
      \A_{1} & \A_{2} & \ldots & \A_{m} & \A_{0} 
    \end{array} 
    \Big)^{n} &
    \begin{array}{c}
      \A_{0} \\ \espaco
    \end{array}\\
  \end{array}
}  <  0\,.
\]
\end{proof}

Observe that, for each $\A \in \alphabet$, we have $\custoA{\matriz}\alinhamento{\A,
  \espaco} = \pont{\matriz}{\A}{\espaco}$,
$\custoA{\matriz}\alinhamento{\espaco, \A} =
\pont{\matriz}{\espaco}{\A}$. Besides that, the only alignments of $\A, \seqVazia$
and $\seqVazia, \A$ are $\alinhamento{\A, \espaco}$ and
$\alinhamento{\espaco, \A}$, respectively. Thus, we have that
\begin{align}
  \distanciaA{\matriz}(\A, \seqVazia) &=
  \custoA{\matriz}\alinhamento{\A, \espaco} =
  \pont{\matriz}{\A}{\espaco}\,, \label{basica1}\\
  \distanciaA{\matriz}(\seqVazia, \A) &=
  \custoA{\matriz}\alinhamento{\espaco, \A} =
  \pont{\matriz}{\espaco}{\A}\,. \label{basica2}
\end{align}

Now, for each $\A, \B \in \alphabet$, since
\[
\cjtoAlinha{(\A, \B)} = \Big\{ \alinhamentoB{
  \begin{array}{c}
    \A\\
    \B
  \end{array}},
  \alinhamentoB{ 
    \begin{array}{cc}
      \A & \espaco\\
      \espaco & \B\\
    \end{array}}, 
  \alinhamentoB{
    \begin{array}{cc}
      \espaco & \A\\
      \B & \espaco\\
    \end{array}}
  \Big\}\,,
\]
$\custoA{\matriz}\alinhamento{\A,\B} = \pont{\matriz}{\A}{\B}$, and
$\custoA{\matriz}\alinhamento{\A\espaco,\espaco\B} =
\custoA{\matriz}\alinhamento{\espaco\A, \B\espaco} =
\pont{\matriz}{\A}{\espaco} + \pont{\matriz}{\espaco}{\B}$, we have
that
\begin{align}
  \distanciaA{\matriz}(\A,\B) &=  \min \bigg\{
  \begin{array}{rcl} 
    \custoA{\matriz}\alinhamento{\A, \B} & \!\!\! = \!\!\! &
    \pont{\matriz}{\A}{\B}\,, \\
    \custoA{\matriz}\alinhamento{\A\espaco, \espaco\B} & \!\!\! =
    \!\!\! & \pont{\matriz}{\A}{\espaco} +
    \pont{\matriz}{\espaco}{\B}\,.
  \end{array}
  \label{basica3}
\end{align}

\begin{lemma}\label{distancia=0}\rm
$\distanciaA{\matriz}(s, s) = 0$ for each $s \in \alphabet^{*}$
if and only if  
\begin{enumerate}
\item[(\textit{i})] $D(\matriz)$ has no negative cycle\,, and
\item[(\textit{ii})] $\pont{\matriz}{\A}{\A} = 0$ or
  $\pont{\matriz}{\A}{\espaco} + \pont{\matriz}{\espaco}{\A} = 0$, for
  each $\A \in \alphabet$.
\end{enumerate}
\end{lemma}
\begin{proof}
  Suppose that $\distanciaA{\matriz}(s, s) = 0$ for each $s \in \alphabet^{*}$.
  It follows from Fact~\ref{fciclonegativo} that (\textit{i}) is true;
  and from equation~(\ref{basica3}) that (\textit{ii}) is also true.

  Conversely, suppose that (\textit{i}) and (\textit{ii}) are true.
  Let $s = s(1) \cdots s(n) \in \alphabet^{*}$ and let $A$ be an
  A-optimal alignment of $s, s$. We define the weighted direct multigraph $H$ from
  alignment $A$ such as
  \begin{align*}
    V(H) &= \Sigma_{\,\espaco}\,, \\
    E(H) &=   \{k : [\A, \B]~\text{is aligned in $k$-th column of $A$} \}\,,\\
    \cost(\A, \B) &= \pont{\matriz}{\A}{\B}~\text{for each arc in $H$}.
\end{align*}

  By construction, $\custoA{\matriz}[A] = \cost(H)$.
  Notice that the indegree and outdegree of each vertex of $H$ are the same, which implies that 
  $H$ is an Eulerian graph
  % \citep{BT2006},
  and then $E$ can be decomposed into arc-disjoint cycles.
  For each cycle of this decomposition, there
  exists a cycle in $D(\matriz)$ with the same weight. Since, by
  hypothesis, $D(\matriz)$ has no negative cycle, it follows that
  $\cost(H) \ge 0$.

Therefore,
\begin{equation}
\distanciaA{\matriz}(s, s) = \custoA{\matriz} [A] = \cost(H) \ge 0\,.
\label{distancia=0eq1}
\end{equation}

Moreover, we are able to construct the following alignment $B$. Align
symbols $s(i), s(i)$ if $\pont{\matriz}{s(i)}{s(i)} = 0$, and align
$s(i), \espaco$ and $\espaco, s(i)$ otherwise. Since (\textit{ii})
is true, it follows that $\custoA{\matriz}[B] = 0$. Thus,
\begin{equation}
\distanciaA{\matriz}(s, s) \le \custoA{\matriz}[B] =
0\,. \label{distancia=0eq2}
\end{equation}

Using expressions~(\ref{distancia=0eq1}) and~(\ref{distancia=0eq2}),
and since the argument is used for any $s \in \alphabet^{*}$, we
conclude that $\distanciaA{\matriz}(s, s) = 0$ for each $s \in
\alphabet^{*}$.
\end{proof}

\begin{lemma}\label{distanciaMaiorOuIgualAZero}\rm
  $\distanciaA{\matriz}(s, t) \ge 0$ for each $s, t \in
  \alphabet^{*}$ if and only if for each $\A, \B \in \alphabet$,  $\pont{\matriz}{\A}{\espaco}, \pont{\matriz}{\espaco}{\B},
  \pont{\matriz}{\A}{\B} \ge 0$.
\end{lemma}
\begin{proof}
Consider $\distanciaA{\matriz}(s, t) \ge 0$ for each $s, t \in
\alphabet^{*}$. It follows from
equations~(\ref{basica1}),~(\ref{basica2}), and~(\ref{basica3}) that
$\pont{\matriz}{\A}{\espaco}  = \distanciaA{\matriz}(\A, \seqVazia) \ge 0,
\pont{\matriz}{\espaco}{\B} = \distanciaA{\matriz}(\seqVazia, \B) \ge 0$, and $\pont{\matriz}{\A}{\B} \ge \distanciaA{\matriz}(\A, \B) \ge 0$.

Conversely, consider $\pont{\matriz}{\A}{\espaco},
\pont{\matriz}{\espaco}{\B}, \pont{\matriz}{\A}{\B} \ge 0$, for each
$\A, \B \in \alphabet$. Let $A$ be an A-optimal alignment of $s,
t$. Since $\custoA{\matriz}[A]$ is the sum of entries of $\matriz$
that, by definition, are nonnegative, we have $\distanciaA{\matriz}(s,
t) = \custoA{\matriz}[A] \ge 0$.
\end{proof}

\begin{lemma}\label{distanciaMaiorQue0}\rm
  $\distanciaA{\matriz}(s, t) > 0$ for each $s \not= t \in
  \alphabet^{*}$ if and only if
  \begin{enumerate}
  \item[(\textit{i})] $\pont{\matriz}{\A}{\A} \ge 0$\,, and
  \item[(\textit{ii})] $\pont{\matriz}{\A}{\B},
    \pont{\matriz}{\A}{\espaco}, \pont{\matriz}{\espaco}{\B} > 0$\,,
  \end{enumerate}
  for each $\A \not= \B \in \alphabet$.
\end{lemma}
\begin{proof}
Consider first that $\distanciaA{\matriz}(s, t) > 0$ for each $s \not=
t \in \alphabet^{*}$. Since, for any $n \ge 0$, $\alinhamento{\A^{n +
    1}, \A^{n} \espaco}$ is an alignment of $(\A^{n + 1}, \A^{n})$
and, by hypothesis, $\distanciaA{\matriz}(\A^{n + 1}, \A^{n}) > 0$, we
have
\[
n \pont{\matriz}{\A}{\A} + \pont{\matriz}{\A}{\espaco} = 
\custoA{\matriz}  \alinhamentoB{
  \begin{array}{cc}
    \A^{n} & \A \\
    \A^{n} & \espaco
  \end{array} }
 \ge \distanciaA{\matriz}(\A^{n + 1}, \A^{n}) > 0\,.
\]
Since the expression above is valid for any $n$, this implies that
$\pont{\matriz}{\A}{\A} \ge 0$. Thus, (\textit{i}) is true. A similar
argument used in the first paragraph of proof of
Lemma~\ref{distanciaMaiorOuIgualAZero} shows that (\textit{ii}) is
also true.

Conversely, consider that (\textit{i}) and (\textit{ii}) are true.
Let $\alinhamento{s', t'}$ be an A-optimal alignment of $s, t$, $s
\not= t$. Since $s \not= t$, there exists $h$ such that $s'(h) \not=
t'(h)$, which implies by (\textit{ii}) that
$\pont{\matriz}{s'(h)}{t'(h)} > 0$. By (\textit{i}) and (\textit{ii}),
$\pont{\matriz}{s'(j)}{t'(j)} \ge 0$ for each $j \not= h$ and, by
hypothesis, $\distanciaA{\matriz}(s, t) =
\custoA{\matriz}\alinhamento{s', t'}$. Then, it follows that
\[
\distanciaA{\matriz}(s, t) = \custoA{\matriz}\alinhamento{s', t'} =
\pont{\matriz}{s'(h)}{t'(h)} + \sum_{j \not= h}
\pont{\matriz}{s'(j)}{t'(j)} 
\ge \pont{\matriz}{s'(h)}{t'(h)}
> 0\,.
\]
\end{proof}

\begin{lemma}\label{simetriaNormal}\rm
$\distanciaA{\matriz}(s, t) = \distanciaA{\matriz}(t, s)$ 
for each $s, t \in \alphabet^{*}$
if and only if
\begin{enumerate}
\item[(\textit{i})] $\pont{\matriz}{\A}{\espaco} =
  \pont{\matriz}{\espaco}{\A}$\,, and
\item[(\textit{ii})] if $\pont{\matriz}{\A}{\B} <
  \pont{\matriz}{\A}{\espaco} + \pont{\matriz}{\espaco}{\B}$, then
  $\pont{\matriz}{\A}{\B} = \pont{\matriz}{\B}{\A}$\,,
\end{enumerate}
for each $\A, \B \in \alphabet$.
\end{lemma}
\begin{proof}
Consider that $\distanciaA{\matriz}(s, t) = \distanciaA{\matriz}(t,
s)$ for each $s, t \in \alphabet^{*}$. It follows as a consequence
of equations~(\ref{basica1}) and~(\ref{basica2}) that
\[
\pont{\matriz}{\A}{\espaco} = \distanciaA{\matriz}(\A, \seqVazia) =
\distanciaA{\matriz}(\seqVazia, \A) = \pont{\matriz}{\espaco}{\A}\,.
\]
Thus, (\textit{i}) is true.

In order to check (\textit{ii}), suppose that $\pont{\matriz}{\A}{\B}
< \pont{\matriz}{\A}{\espaco} + \pont{\matriz}{\espaco}{\B}$. This
implies that
$
\custoA{\matriz}\alinhamento{\A, \B} = \pont{\matriz}{\A}{\B} < 
\pont{\matriz}{\A}{\espaco} + \pont{\matriz}{\espaco}{\B} =
\custoA{\matriz}\alinhamento{\A \espaco, \espaco \B} =
\custoA{\matriz}\alinhamento{\espaco \A, \B \espaco}\,.
$
Thus, from~(\ref{basica3}), it follows that $\distanciaA{\gamma}(\A,\B) =
\custoA{\gamma}\alinhamento{\A, \B}$. Furthermore, since
$\distanciaA{\matriz}(\B, \A) = \distanciaA{\matriz}(\A, \B)$ and,
by~$(i)$, $\pont{\matriz}{\A}{\espaco} = \pont{\matriz}{\espaco}{\A}$
and $\pont{\matriz}{\B}{\espaco} = \pont{\matriz}{\espaco}{\B}$, we
have that
\begin{align}
\distanciaA{\matriz}(\B, \A) &= \distanciaA{\matriz}(\A, \B)
= \custoA{\matriz}\alinhamento{\A,\B} = 
\pont{\matriz}{\A}{\B}\label{simetriaNormal1}
\\
&< \pont{\matriz}{\A}{\espaco} + \pont{\matriz}{\espaco}{\B} =
\pont{\matriz}{\B}{\espaco} + \pont{\matriz}{\espaco}{\A} =
\custoA{\matriz}\alinhamento{\B \espaco,
  \espaco \A} = \custoA{\matriz}\alinhamento{\espaco \B, \A
  \espaco}\,,
\nonumber \end{align}
which implies that neither $\alinhamento{\B \espaco,
  \espaco \A}$ nor $\alinhamento{\espaco \B, \A \espaco}$ are
A-optimal alignments of $\B, \A$. It follows from~(\ref{basica3}) that
\begin{equation}
\distanciaA{\matriz}(\B, \A) = \custoA{\matriz}\alinhamento{\B, \A}
= \pont{\matriz}{\B}{\A}\,.\label{simetriaNormal2}
\end{equation}
Using equations~(\ref{simetriaNormal1})
and~(\ref{simetriaNormal2}), it follows that (\textit{ii}) is also true.

Conversely, consider that (\textit{i}) and (\textit{ii}) are true and assume
that $\alinhamento{s', t'}$ is an A-optimal alignment of maximum length
of $s, t$. If $s'(h)$ or $t'(h)$ is $\espaco$, we have by~(\textit{i})
that $\pont{\matriz}{s'(h)}{t'(h)} = \pont{\matriz}{t'(h)}{s'(h)}$. If
$s'(h), t'(h) \in \alphabet$ then, by the chosen alignment $[s',t']$,
we have
\begin{align*}
\pont{\matriz}{s'(h)}{t'(h)} &+ \sum_{j \not= h}
\pont{\matriz}{s'(h)}{t'(h)} = \custoA{\matriz}[s', t'] \\
&< \custoA{\matriz} \alinhamentoB{
\begin{array}{cccc}
s'(1 \ldots j -1) & s'(j) & \espaco & s'(j + 1 \ldots \tamanho{(s',
  t')})\\
t'(1 \ldots j -1) & \espaco & t'(j) & t'(j + 1 \ldots \tamanho{(s',
  t')})
\end{array}}  \\
&= \pont{\matriz}{s'(h)}{\espaco} + \pont{\matriz}{\espaco}{t'(h)} +
\sum_{j \not= h} \pont {\matriz}{s'(h)}{t'(h)}\,.
\end{align*}
It follows that $\pont{\matriz}{s'(j)}{t'(j)} <
\pont{\matriz}{s'(j)}{\espaco} + \pont{\matriz}{\espaco}{t'(j)}$, which implies from~(\textit{ii}) that $\pont{\matriz}{s'(i)}{t'(i)} =
\pont{\matriz}{t'(i)}{s'(i)}$. Thus, regardless of the symbols $s'(i)$
and $t'(i)$, we have $\pont{\matriz}{s'(i)}{t'(i)} =
\pont{\matriz}{t'(i)}{s'(i)}$ for each $i$, which implies that
$\custoA{\matriz}\alinhamento{s', t'} =
\custoA{\matriz}\alinhamento{t', s'}$. Since $\alinhamento{s', t'}$
is an A-optimal alignment of $s, t$ and $\alinhamento{t', s'}$ is an
alignment of $t, s$, it follows that
\[
\distanciaA{\matriz}(s, t) = \custoA{\matriz}\alinhamento{s',t'} =
\custoA{\matriz}\alinhamento{t', s'} \ge \distanciaA{\matriz}(t,
s)\,.
\]
Using the same arguments we can prove that $\distanciaA{\matriz}(t, s)
\ge \distanciaA{\matriz}(s, t)$, which allows us to conclude that
$\distanciaA{\matriz}(s, t) = \distanciaA{\matriz}(t, s)$.
\end{proof}

\begin{proposition}\label{desigTriangularGeral}\rm
Let $\matriz$ be a scoring matrix, $Q$ an integer, and $s, t, u \in
\alphabet^{*}$. If
\begin{enumerate}
\item[(\textit{i})] $\pont{\matriz}{\A}{\espaco} \le
  \pont{\matriz}{\A}{\B} + \pont{\matriz}{\B}{\espaco}$\,,
\item[(\textit{ii})] $\pont{\matriz}{\espaco}{\A} \le
  \pont{\matriz}{\espaco}{\B} + \pont{\matriz}{\B}{\A}$\,,
\item[(\textit{iii})] $\min \{\pont{\matriz}{\A}{\C},
  \pont{\matriz}{\A}{\espaco} + \pont{\matriz}{\espaco}{\C}\} \le
  \pont{\matriz}{\A}{\B} + \pont{\matriz}{\B}{\C}$\,, and
\item[(\textit{iv})] $\pont{\matriz}{\B}{\espaco} +
  \pont{\matriz}{\espaco}{\B} \ge Q$\,,
\end{enumerate}
for each $\A, \B, \C \in \alphabet$, then, for each alignment $A$ of
$s, t$ and each alignment $B$ of $t, u$, there exists an alignment
$C$ of $s, u$ and an integer $k \ge 0$ such that
\[
\custoA{\matriz}[A] + \custoA{\matriz}[B] \ge \custoA{\matriz}[C] +
kQ\,, \quad \tamanho{A} \le \tamanho{C} + k\,, \quad \text{and} \quad
\tamanho{B} \le \tamanho{C} + k\,.
\]  
\end{proposition}
\begin{proof}
Suppose that (\textit{i}), (\textit{ii}), (\textit{iii}),
(\textit{iv}) are true. Let $A, B$ be alignments of $s, t$ and $t,
u$, respectively. We define
\begin{align*}
  \mathcal{C}_{1} &= \left\{h : [s(h), \espaco]~\text{is aligned in $A$}\right\}\,, \\
  \mathcal{C}_{2} &= \left\{k : [\espaco, u(k)]~\text{is aligned in $B$}\right\}\,, \\
  \mathcal{C}_{3} &= \left\{j : [\espaco, t (j)]~\text{is aligned in $A$ and $[t(j), \espaco]$
    is aligned in $B$}\right\}\,, \\
  \mathcal{C}_{4} &= \left\{(h, j) :
  ~\text{$[s(h), t(j)]$ is aligned in $A$ and 
     $[t(j), \espaco]$ is aligned in $B$}
  \right\}\,,
  \\
  \mathcal{C}_{5} &= \left\{(j, k) :~\text{$[\espaco, t (j)]$ is aligned in $A$ and $[t(j), u(k)]$
    is aligned in $B$}\right\}\,, \\
  \mathcal{C}_{6} &= \Big\{\!\! \begin{array}{l}
    (h, j, k) :~\text{$[s(h), t(j)]$ is aligned in $A$, $[t(j), u(k)]$ is aligned in $B$} \!\! \\
    \text{and}~\pont{\matriz}{s(h)}{u(k)} \le
    \pont{\matriz}{s(h)}{t(j)} + \pont{\matriz}{t(j)}{u(k)}
  \end{array} \Big\}\,, \\
  \mathcal{C}_{7} &= \Big\{\!\! \begin{array}{l}
    (h, j, k) :~\text{$[s(h), t(j)]$ is aligned in $A$, $[t(j), u(k)]$ is aligned in $B$} \!\!\\
    \text{and}~\pont{\matriz}{s(h)}{u(k)} > \pont{\matriz}{s(h)}{t(j)}
    + \pont{\matriz}{t(j)}{u(k)}
  \end{array} \Big\}\,.
\end{align*}

Thus,
\begin{align*}
\custoA{\matriz}[A] = &\sum_{h \in \mathcal{C}_{1}}
\pont{\matriz}{s(h)}{\espaco} + \sum_{j \in \mathcal{C}_{3}}
\pont{\matriz}{\espaco}{t(j)} + \sum_{(h, j) \in \mathcal{C}_{4}}
\pont{\matriz}{s(h)}{t(j)} + 
\sum_{(j, k) \in \mathcal{C}_{5}} \pont{\matriz}{\espaco}{t(j)}
\\ & + \sum_{(h, j, k) \in \mathcal{C}_{6}} \pont{\matriz}{s(h)}{t(j)} +
\sum_{(h, j, k) \in \mathcal{C}_{7}}
\pont{\matriz}{s(h)}{t(j)}
\end{align*}
and
\begin{align*}
\custoA{\matriz}[B] = &\sum_{k \in \mathcal{C}_{2}}
\pont{\matriz}{\espaco}{u(k)} + \sum_{j \in \mathcal{C}_{3}}
\pont{\matriz}{t (j)}{\espaco} + \sum_{(h, j) \in \mathcal{C}_{4}}
\pont{\matriz}{t (j)}{\espaco} + 
\sum_{(j, k) \in \mathcal{C}_{5}} \pont{\matriz}{t(j)}{u(k)}\\
& + 
\sum_{(h, j, k) \in \mathcal{C}_{6}} \pont{\matriz}{t(j)}{u(k)} + \sum_{(h, j, k) \in \mathcal{C}_{7}}
\pont{\matriz}{t(j)}{u(k)}\,,
\end{align*}
which implies that 
\begin{align*}
  \custoA{\matriz}[A] &+ \custoA{\matriz}[B] =\\
  & 
  \sum_{h \in \mathcal{C}_{1}}
  \pont{\matriz}{s(h)}{\espaco} + 
  \sum_{k \in \mathcal{C}_{2}}
  \pont{\matriz}{\espaco}{u(k)} +
  \sum_{j \in \mathcal{C}_{3}} \left(\pont{\matriz}{\espaco}{t (j)} + \pont{\matriz}{t (j)}{\espaco}\right) + \\
  & \sum_{(h, j) \in \mathcal{C}_{4}} \left(
  \pont{\matriz}{s(h)}{t (j)} + \pont{\matriz}{t (j)}{\espaco} \right) + 
  \sum_{(j, k) \in \mathcal{C}_{5}} \left( \pont{\matriz}{\espaco}{t(j)}+\pont{\matriz}{t(j)}{u(k)} \right)+ \\
  &  
  \sum_{(h, j, k) \in \mathcal{C}_{6}} \left( \pont{\matriz}{s(h)}{t(j)}+\pont{\matriz}{t(j)}{u(k)}\right) + \sum_{(h, j, k) \in \mathcal{C}_{7}}
  \left(\pont{\matriz}{s(h)}{t(j)}+ \pont{\matriz}{t(j)}{u(k)} \right)\,.
\end{align*}

Define the alignment $C$ of $s, u$ such that if $(h, j, k) \in \mathcal{C}_{6}$ then it aligns 
$[s(h), u(k)]$ and, for each remaining $s(h)$ and $u(k)$, it aligns $[s(h),
\espaco]$ and $[\espaco, u(k)]$. Thus,
\begin{align*}
\custoA{\matriz}[C] = &\sum_{h \in \mathcal{C}_{1}}
\pont{\matriz}{s(h)}{\espaco} + \sum_{k \in \mathcal{C}_{2}}
\pont{\matriz}{\espaco}{u (k)} + \sum_{(h, j) \in \mathcal{C}_{4}}
\pont{\matriz}{s(h)}{\espaco} + \sum_{(j, k) \in \mathcal{C}_{5}} \pont{\matriz}{\espaco}{u
  (k)}\\& + \sum_{(h, j, k) \in \mathcal{C}_{6}} \pont{\matriz}{s(h)}{u
  (k)} + \sum_{(h, j, k) \in \mathcal{C}_{7}} \big(
\pont{\matriz}{s(h)}{\espaco} + \pont{\matriz}{\espaco}{u(k)} \big)\,.
\end{align*}

If $j \in \mathcal{C}_{3}$, then, since (\textit{iv}) is true, we have
\begin{align}
\sum_{j \in
  \mathcal{C}_{3}} \big( \pont{\matriz}{t(j)}{\espaco} +
\pont{\matriz}{\espaco}{t(j)} \big) \ge \sum_{j \in \mathcal{C}_{3}} Q =
\tamanho{\mathcal{C}_{3}}\,{Q}\,. \label{C3}
\end{align}

From (\textit{i}),~(\textit{ii}), and the definition of
$\mathcal{C}_{6}$, we have, respectively,
\begin{align}
  \sum_{(h, j) \in \mathcal{C}_{4}} \big( \pont{\matriz}{s(h)}{t(j)} +
  \pont{\matriz}{t(j)}{\espaco} \big) &\ge \sum_{(h, j) \in \mathcal{C}_{4}}
  \pont{\matriz}{s(h)}{\espaco}\,. \label{C4} \\
  \sum_{(j, k) \in \mathcal{C}_{5}} \big(
  \pont{\matriz}{\espaco}{t(j)} + \pont{\matriz}{t(j)}{u(k)} \big)
  &\ge \sum_{(j, k) \in \mathcal{C}_{5}}
  \pont{\matriz}{\espaco}{u(k)}\,. \label{C5} \\
  \sum_{(h, j, k) \in \mathcal{C}_{6}} \big(
  \pont{\matriz}{s(h)}{t(j)} + \pont{\matriz}{t(j)}{u(k)} \big)
  &\ge \sum_{(h, j, k) \in \mathcal{C}_{6}}
  \pont{\matriz}{s(h)}{u(k)}\,. \label{C6}
\end{align}

Suppose that $(h, j, k) \in \mathcal{C}_{7}$. Then, by definition of
$\mathcal{C}_{7}$, we know that $\pont{\matriz}{s(h)}{u(k)} >
\pont{\matriz}{s(h)}{t(j)} + \pont{\matriz}{t(j)}{u(k)}$. It follows
from (\textit{iii}) that $\pont{\matriz}{s(h)}{t(j)} +
\pont{\matriz}{t(j)}{u(k)} \ge \pont{\matriz}{s(h)}{\espaco} +
\pont{\matriz}{\espaco}{u(k)}$. Thus,
\begin{align}
  \sum_{(h, j, k) \in \mathcal{C}_{7}} \big(
  \pont{\matriz}{s(h)}{t(j)} + \pont{\matriz}{t(j)}{u(k)} \big)
  \ge \sum_{(h, j, k) \in \mathcal{C}_{7}} \big(
  \pont{\matriz}{s(h)}{\espaco} + \pont{\matriz}{\espaco}{u(k)}
  \big)\,. \label{C7}
\end{align}

From equations~(\ref{C3}),~(\ref{C4}),~(\ref{C5}),~(\ref{C6}),
and~(\ref{C7}), we have
\[
\custoA{\matriz}[A] + \custoA{\matriz}[B] \ge \custoA{\matriz}[C] +
\tamanho{\mathcal{C}_{3}} \, Q\,.
\]
Hence, to finish the proof, it is enough to show that $\tamanho{A} \le
\tamanho{C} + \tamanho{\mathcal{C}_{3}}$ and $\tamanho{B} \le
\tamanho{C} + \tamanho{\mathcal{C}_{3}}$.
Since 
\begin{align*}
\tamanho{A} &= \sum_{i} \tamanho{\mathcal{C}_{i}} -
\tamanho{\mathcal{C}_{2}} \le \sum_{i} \tamanho{\mathcal{C}_{i}}\,,
\\
\tamanho{B} &= \sum_{i} \tamanho{\mathcal{C}_{i}} -
\tamanho{\mathcal{C}_{1}} \le \sum_{i} \tamanho{\mathcal{C}_{i}}\,,
\\
\tamanho{C} &= \sum_{i} \tamanho{\mathcal{C}_{i}} -
\tamanho{\mathcal{C}_{3}} + \tamanho{\mathcal{C}_{7}} \ge \sum_{i}
\tamanho{\mathcal{C}_{i}} - \tamanho{\mathcal{C}_{3}}\,,
\end{align*}
we have $\tamanho{A} \le \sum_{i} \tamanho{\mathcal{C}_{i}} \le
\tamanho{C} + \tamanho{\mathcal{C}_{3}}$ and $\tamanho{B} \le \sum_{i}
\tamanho{\mathcal{C}_{i}} \le \tamanho{C} +
\tamanho{\mathcal{C}_{3}}$.
\end{proof}

\begin{lemma}\label{desigualdadeTriangularNormal}\rm
$\distanciaA{\matriz}(s, u) \le \distanciaA{\matriz}(s, t) +
  \distanciaA{\matriz}(t, u)$ for each $s, t, u \in \alphabet^{*}$
  if and only if
\begin{enumerate}
\item[(\textit{i})] $\pont{\matriz}{\A}{\espaco} \le
  \pont{\matriz}{\A}{\B} + \pont{\matriz}{\B}{\espaco}$\,,
\item[(\textit{ii})] $\pont{\matriz}{\espaco}{\A} \le
  \pont{\matriz}{\espaco}{\B} + \pont{\matriz}{\B}{\A}$\,,
\item[(\textit{iii})] $\min \{ \pont{\matriz}{\A}{\C},
  \pont{\matriz}{\A}{\espaco} + \pont{\matriz}{\espaco}{\C} \} \le
  \pont{\matriz}{\A}{\B} + \pont{\matriz}{\B}{\C}$\,, and
\item[(\textit{iv})] $\pont{\matriz}{\B}{\espaco} +
  \pont{\matriz}{\espaco}{\B} \ge 0$\,,
\end{enumerate}
for each $\A, \B, \C \in \alphabet$.
\end{lemma}
\begin{proof}
Suppose that $\distanciaA{\matriz}(s, u) \le
\distanciaA{\matriz}(s, t) + \distanciaA{\matriz}(t, u)$ for each $s,
t, u \in \alphabet^{*}$. 
It follows from equations~(\ref{basica1}),~(\ref{basica2}),
and~(\ref{basica3}) that
\begin{align*}
\pont{\matriz}{\A}{\espaco} =  \distanciaA{\matriz}(\A, \seqVazia) &\le
\distanciaA{\matriz}(\A, \B) + \distanciaA{\matriz}(\B, \espaco) \\
&\le \custoA{\matriz}\alinhamento{\A, \B} +
\custoA{\matriz}\alinhamento{\espaco, \B} = \pont{\matriz}{\A}{\B} + \pont{\matriz}{\espaco}{\B}\,,
& \\
\pont{\matriz}{\espaco}{\A} =
\distanciaA{\matriz}(\seqVazia, \A) &\le 
\distanciaA{\matriz}(\espaco, \B) + \distanciaA{\matriz}(\B, \A) \\
&\le \custoA{\matriz}\alinhamento{\espaco, \B} +
\custoA{\matriz}\alinhamento{\B, \A} = \pont{\matriz}{\espaco}{\B} + \pont{\matriz}{\B}{\A}\,,\\
  \min \{ \pont{\matriz}{\A}{\C}, \pont{\matriz}{\A}{\espaco} +
  \pont{\matriz}{\espaco}{\C} \} &= \distanciaA{\matriz}(\A, \C) \\
  &\le \distanciaA{\matriz}(\A, \B) + \distanciaA{\matriz}(\B, \C) \le \pont{\matriz}{\A}{\B} + \pont{\matriz}{\B}{\C}\,.
\end{align*}
Therefore, (\textit{i}), (\textit{ii}), (\textit{iii}) are true.

If $\pont{\matriz}{\B}{\espaco} + \pont{\matriz}{\espaco}{\B} = 0$,
(\textit{iv}) is true. Assume then that
$\pont{\matriz}{\B}{\espaco} + \pont{\matriz}{\espaco}{\B} \not= 0$.
Let 
\[
n > \frac{\distanciaA{\matriz}(\A, \C) - (\pont{\matriz}{\A}{\espaco}
  + \pont{\matriz}{\espaco}{\C})} {\pont{\matriz}{\B}{\espaco} +
  \pont{\matriz}{\espaco}{\B}}
\] 
be a positive integer. Since $\alinhamento{\A \espaco^{n}, \espaco
  \B^{n}}$ and $\alinhamento{\B^{n} \espaco, \espaco^{n} \C}$ are
alignments of $\A, \B^{n}$ and $\B^{n}, \C$, respectively, it follows
that
\begin{align*}
\distanciaA{\matriz}(\A, \C) &\le \distanciaA{\matriz}(\A, \B^{n}) +
\distanciaA{\matriz} (\B^{n}, \C) \\
&\le \custoA{\matriz}\alinhamento{\A \espaco^{n}, \espaco \B^{n}} +
\custoA {\matriz} \alinhamento{\B^{n} \espaco, \espaco^{n} \C} \\
&= \pont{\matriz}{\A}{\espaco} + n \pont{\matriz}{\espaco}{\B} + n
\pont{\matriz}{\B}{\espaco} + \pont{\matriz}{\espaco}{\C}\,,
\end{align*}
and thus, by the choice of $n$, we have that $\pont{\matriz}{\B}{\espaco} 
+ \pont{\matriz}{\espaco}{\B} > 0$, which implies that $(\textit{iv})$ is 
also true. %We conclude that if $\distanciaA{\matriz}(s, u) \le \distanciaA{\matriz}(s, t) + \distanciaA{\matriz}(t, u)$ for each $s, t, u \in \alphabet^{*}$, then $(\textit{i})$, $(\textit{ii})$, $(\textit{iii})$, and $(\textit{iv})$ are true.

Conversely, suppose that (\textit{i}), (\textit{ii}), (\textit{iii}), and
(\textit{iv}) are true.
Let $A$ and $B$ be A-optimal alignments of $s, t$ and $t, u$,
respectively. It follows from Proposition~\ref{desigTriangularGeral}
that there exist an integer $k$ and an alignment $C$ of $s, u$ such
that
\[
\custoA{\matriz}[C] \le \custoA{\matriz}[A] + \custoA{\matriz}[B] +
0\,k = \custoA{\matriz}[A] + \custoA{\matriz}[B]\,.
\]
Consequently, since $A$ and $B$ are A-optimal alignments of $s, t$
and $t, u$, and $C$ is an alignment of $s, u$, it follows that
\[
\distanciaA{\matriz}(s, u) \le \custoA{\matriz}[C] \le
\custoA{\matriz}[A] + \custoA{\matriz}[B] = \distanciaA{\matriz}(s, t)
+ \distanciaA{\matriz}(t, u)\,.
\]
%In this case, we also conclude that if $(\textit{i})$, $(\textit{ii})$, $(\textit{iii})$, and $(\textit{iv})$ are true, and finally we have that $\distanciaA{\matriz}(s, u) \le \distanciaA{\matriz}(s, t) +\distanciaA{\matriz}(t, u)$ for each $s, t, u \in \alphabet^{*}$.
\end{proof}

\begin{table}[htpb]
\begin{minipage}{\textwidth}
\begin{center}
\begin{tabular}{clcccccc}
  & & $\prametrica$ & $\semimetrica$ & $\hemimetrica$ &
  $\pseudometrica$ & $\quasimetrica$ & $\metrica$ \\ \hline & & \\
  (a) & $D(\matriz)$ has no negative cycle & \yes & \yes & \yes & \yes
  &\yes & \yes \\ & & \\
  (b) & $\pont{\matriz}{\A}{\A} = 0$ or $\pont{\matriz}{\A}{\espaco} +
  \pont{\matriz}{\espaco}{\A} = 0$ & \yes & \yes & \yes & \yes &\yes &
  \yes \\ & & \\
  (c) & $\pont{\matriz}{\A}{\espaco}, \pont{\matriz}{\espaco}{\B},
  \pont{\matriz}{\A}{\B} \ge 0$ & \yes & \yes & \yes & \yes &\yes &
  \yes \\ & & \\
  (d) & $\pont{\matriz}{\A}{\A} \ge 0$& & \yes & & &\yes & \yes \\ & &
  \\
  (e) & $\pont{\matriz}{\A}{\espaco}, \pont{\matriz}{\espaco}{\A} > 0$
  and $\pont{\matriz}{\A}{\B} > 0$ if $\A \not= \B$ & & \yes & & &\yes
  & \yes \\& & \\
  (f) & $\pont{\matriz}{\A}{\espaco} = \pont{\matriz}{\espaco}{\A}$ &
  & \yes & & \yes & & \yes \\ & & \\
  (g) &
  \begin{tabular}{l}
    if $\pont{\matriz}{\A}{\B} < \pont{\matriz}{\A}{\espaco}
    + \pont{\matriz}{\espaco}{\B}$ then \\
    $\pont{\matriz}{\A}{\B} = \pont{\matriz}{\B}{\A}$
  \end{tabular}  
  & & \yes & & \yes & & \yes \\ & & \\
  (h) & $\pont{\matriz}{\A}{\espaco} \le \pont{\matriz}{\A}{\B} +
  \pont{\matriz}{\B}{\espaco}$ & & & \yes & \yes &\yes & \yes \\ & &
  \\
  (i) & $\pont{\matriz}{\espaco}{\A} \le \pont{\matriz}{\espaco}{\B} +
  \pont{\matriz}{\B}{\A}$ & & & \yes & \yes &\yes & \yes\\ & & \\
  (j) & 
  $\min \Big\{ \begin{array}{l}
    \pont{\matriz}{\A}{\C}, \\
    \pont{\matriz}{\A}{\espaco} + \pont{\matriz}{\espaco}{\C} 
  \end{array} \Big\} 
  \le \pont{\matriz}{\A}{\B} + \pont{\matriz}{\B}{\C}$
  &  &  & \yes & \yes &\yes & \yes\\ & & \\
  (k) & $\pont{\matriz}{\B}{\espaco} + \pont{\matriz}{\espaco}{\B} \ge
  0$ & & & \yes & \yes &\yes & \yes
\end{tabular}
\end{center}
\end{minipage}
\caption{Necessary and sufficient conditions for scoring matrix
  $\matriz$ induce $\distanciaA{\gamma}\text{-}p$ on sequences. 
  Besides \emph{metric} ($\metrica$), these properties are also used to define 
  generalized metric spaces such as \emph{premetric} ($\prametrica$),
  \emph{semimetric} ($\semimetrica$), \emph{hemimetric}
  ($\hemimetrica$), \emph{pseudometric} ($\pseudometrica$), and 
  \emph{quasimetric} ($\quasimetrica$). Results are obtained using
  definitions presented in Section~\ref{sec:preliminares}, and
  Lemmas~\ref{distancia=0}, \ref{distanciaMaiorOuIgualAZero},
  \ref{distanciaMaiorQue0}, \ref{simetriaNormal}, and
  \ref{desigualdadeTriangularNormal}, for each $\A, \B, \C \in
  \alphabet$.} \label{tabela2}
\end{table}

Table~\ref{tabela2} summarizes the results of scoring matrices $\matriz$ that induce $\distanciaA{\matriz}\text{-}p$, where $p$ is a property that allows to characterize each axiom of the metric on sequences. Finally, we can prove the preeminent result of this section.

\begin{proof} (of Theorem~\ref{theo:align})

Suppose that $\distanciaA{\matriz}$ is a metric. 
Thus, all conditions in Table~\ref{tabela2} are satisfied.
From~(e) and~(f), we have that $\pont{\matriz}{\A}{\espaco} =
\pont{\matriz}{\espaco}{\A} > 0$ and therefore (\textit{i}) is true.
Since (\textit{i}) is true, we have that
$\pont{\matriz}{\A}{\espaco} + \pont{\matriz}{\espaco}{\A} \not= 0$,
which implies by~(b) that $\pont{\matriz}{\A}{\A} = 0$;
moreover, by~(e), we have that $\pont{\matriz}{\A}{\B} > 0$ if $\A \not= \B$;
it follows that (\textit{ii}) is true.
From~(g), (h), and (j), we have that (\textit{iii}), (\textit{iv}), and
(\textit{v}) are also true.
Therefore, if $\distanciaA{\matriz}$ is a
metric, conditions (\textit{i}) to (\textit{v})
are satisfied.

Conversely, suppose that the conditions (\textit{i}) to (\textit{v})
are true and, in order to prove that these are sufficient conditions
for $\distanciaA{\matriz}$ to be a metric, we check whether all 
conditions in Table~\ref{tabela2} are satisfied.
We have, by (\textit{i}) and (\textit{ii}), that the conditions (a) to
(f), and the condition (k), are satisfied.
Since 
(\textit{iii}), (\textit{iv}) and (\textit{v}) are true,
it follows that 
(g), (h), and (j) are also true. 
Then, it is enough to show that the condition 
(i) in Table~\ref{tabela2} is true.

Suppose that $\pont{\matriz}{\B}{\A} \ge 
\pont{\matriz}{\B}{\espaco} + \pont{\matriz}{\espaco}{\A}
$. 
Since (k) is true, it follows that 
\[\pont{\matriz}{\espaco}{\A}\le 
\pont{\matriz}{\espaco}{\B} + 
\pont{\matriz}{\B}{\espaco} +
\pont{\matriz}{\espaco}{\A} 
\le 
\pont{\matriz}{\espaco}{\B} + 
\pont{\matriz}{\B}{\A}\]
and the proof is done. 
Then, assume that $\pont{\matriz}{\B}{\A} <
\pont{\matriz}{\B}{\espaco} + \pont{\matriz}{\espaco}{\A}$.
It follows from (\textit{iii}) 
that 
$\pont{\matriz}{\B}{\A} =
\pont{\matriz}{\A}{\B}$,
which implies from (\textit{i}) and (\textit{iv}) that 
\[
\pont{\matriz}{\espaco}{\A}
= \pont{\matriz}{\A}{\espaco}
\le 
\pont{\matriz}{\A}{\B} + 
\pont{\matriz}{\B}{\espaco} 
=
\pont{\matriz}{\espaco}{\B} + 
\pont{\matriz}{\B}{\A}\,.\]
Therefore, if conditions (\textit{i}) to (\textit{v})
are true, then all conditions in Table~\ref{tabela2}
are satisfied, which implies that $\distanciaA{\matriz}$ is a
metric on sequences.
\end{proof}



\section{Normalized edit distance}\label{sec:normalizado}

In this section, we describe the classes of scoring matrices that
induce $\distanciaN{\matriz}$-$p$ on sequences for each axiom $p$ of a metric
when $\distanciaN{\matriz} \in \prametrica$, as can be seen in Lemmas~\ref{lemaPrametricaN}, \ref{lema-maiorQueZero}, \ref{distanciaNSimetrica}, and \ref{distanciaNTriangular}, and summarized in Table~\ref{tabela3}. They allow us to characterize matrices that induces each of the more general metric functions described in Section~\ref{sec:preliminares}. As a consequence, we present an important result previously stated in Section~\ref{sec:preliminares} as following:

\begin{theorem} \label{theo:norm} \rm
$\distanciaN{\matriz} \in \metrica$ if and only if $\matriz \in \metricaN$.
\end{theorem}

As in the previous section, we present below a sequence of auxiliary results to finally obtain a proof of Theorem~\ref{theo:norm}. 

\begin{lemma}\label{lemaPrametricaN}\rm
Let $\matriz$ be a scoring matrix.  Then $\distanciaA{\matriz} \in
\prametrica$ if and only if $\distanciaN{\matriz} \in \prametrica$.
\end{lemma}
\begin{proof}
Let $s, t$ be sequences in $\alphabet^{*}$. In order to prove this
result, we show that $\distanciaA{\matriz}(s, s) = 0$ if and only if
$\distanciaN{\matriz}(s, s) = 0$, and that $\distanciaA{\matriz}(s, t) \ge
0$ if and only if $\distanciaN{\matriz}(s, t) \ge~0$
for each $s, t \in \Sigma^*$.

Consider first that $\distanciaA{\matriz}(s, s) = 0$. If $s =
\seqVazia$, we have $\distanciaA{\matriz}(s, s) =
\distanciaN{\matriz}(s, s) = 0$ and the proof is done. Suppose then
that $s \not= \seqVazia$. Let $A$ be an A-optimal alignment of $s,
s$. Since $\distanciaA{\matriz}(s, s) = 0$, we have that
$\custoA{\matriz}[A] = 0$ and, since $s \not= \seqVazia$, we have that
$\tamanho{A} > 0$. It follows that
\begin{equation}
\distanciaN{\matriz}(s, s) \le \custoN{\matriz}[A] =
\frac{\custoA{\matriz}[A]}{\tamanho{A}} = 0\,. \label{pra3}
\end{equation}
Let $B$ be an N-optimal alignment of $s, s$. Since $s \not= \seqVazia$,
we have $\tamanho{B} > 0$, which implies, since
$\distanciaA{\matriz}(s, s) = 0$ and $\custoA{\matriz}[B] \ge
\distanciaA{\matriz}(s, s)$, that
\[
\distanciaN{\matriz}(s, s) = \frac{\custoA{\matriz}[B]}{\tamanho{B}}
\ge \frac{\distanciaA{\matriz}(s, s)}{\tamanho{B}} = 0\,.
\]
It follows from Equation~(\ref{pra3}) that $\distanciaN{\matriz}(s, s)
= 0$ and, therefore, the proof is also done when $s \not= \seqVazia$.

Similar arguments can be used to prove that $\distanciaN{\matriz}(s,
s) = 0$ implies that $\distanciaA{\matriz}(s, s) = 0$,
and 
$\distanciaA{\matriz}(s, t) \ge 0$ if and only if
$\distanciaN{\matriz}(s, t) \ge 0$.
\end{proof} 

\begin{corolary}\label{corolaryPrametrica}\rm
Let $\matriz$ be a scoring matrix and $\A, \B \in \alphabet$. Then
$\distanciaN{\matriz} \in \prametrica$ if and only if the following
conditions are true:
\begin{enumerate}
\item[(\textit{i})] $\pont{\matriz}{\A}{\A} = 0$ or
  $\pont{\matriz}{\A}{\espaco} + \pont{\matriz}{\espaco}{\A} = 0$\,, and
\item[(\textit{ii})] $\pont{\matriz}{\A}{\espaco},
  \pont{\matriz}{\espaco}{\A}, \pont{\matriz}{\A}{\B} \ge 0$\,,
\end{enumerate}
for each $\A, \B \in \Sigma$.
\end{corolary}
\begin{proof}
Suppose that $\distanciaN{\matriz} \in \prametrica$. It follows from
Lemma~\ref{lemaPrametricaN} that $\distanciaA{\matriz} \in
\prametrica$, which implies from Table~\ref{tabela2} lines (b) and
(c), that the conditions (\textit{i}) and (\textit{ii}) are true.

Conversely, suppose that the conditions (\textit{i}) and (\textit{ii})
are true.  
It follows that (b) and (c) of Table~\ref{tabela2} are true.
Since (\textit{ii}) is true, we have that $D(\matriz)$ has no negative cycle, which implies that (a) of Table~\ref{tabela2} is also true. 
It follows that $\distanciaA{\matriz} \in \prametrica$, implying from
Lemma~\ref{lemaPrametricaN} that $\distanciaN{\matriz} \in
\prametrica$.
\end{proof}

\begin{lemma}\label{lema-maiorQueZero}\rm
Let $\distanciaN{\matriz} \in \prametrica$. Then,
$\distanciaN{\matriz}(s, t) > 0$ for any $s \not= t \in \alphabet^{*}$
if and only if $\pont{\matriz}{\A}{\espaco},
\pont{\matriz}{\espaco}{\A}, \pont{\matriz}{\A}{\B} > 0$ for any $\A
\not= \B \in \alphabet$.
\end{lemma}
\begin{proof}
Let $\distanciaN{\matriz} \in \prametrica$. Suppose that
$\distanciaN{\matriz}(s, t) > 0$ for any $s \not= t \in
\alphabet^{*}$. Then,
\begin{align*}
\pont{\matriz}{\A}{\espaco} &= \custoN{\matriz}\alinhamento{\A,
  \espaco} = \distanciaN{\matriz}[\A, \seqVazia] > 0\,,\\
\pont{\matriz}{\espaco}{\A} &= \custoN{\matriz}\alinhamento{\espaco,
  \A} = \distanciaN{\matriz}(\seqVazia, \A) > 0\,,\\
\pont{\matriz}{\A}{\B} &= \custoN{\matriz}\alinhamento{\A, \B} \ge
\distanciaN{\matriz}[\A, \B] > 0\,.
\end{align*}

Conversely, suppose that $\pont{\matriz}{\A}{\espaco},
\pont{\matriz}{\espaco}{\A}, \pont{\matriz}{\A}{\B} > 0$ for any $\A
\not= \B \in \alphabet$. Since $\distanciaN{\matriz} \in \prametrica$,
from Corollary~\ref{corolaryPrametrica} we have that
$\pont{\matriz}{\A}{\A} \ge 0$. It follows from
Lemma~\ref{distanciaMaiorQue0} that $\distanciaA{\matriz}(s, t) > 0$
for $s \not= t \in \alphabet^{*}$. Let $A$ be a N-optimal alignment
of $s, t$. It follows that
\[
\distanciaN{\matriz}(s, t) = \custoN{\matriz}[A] =
\frac{\custoA{\matriz}[A]}{\tamanho{A}} \ge
\frac{\distanciaA{\matriz}(s, t)}{\tamanho{A}} > 0\,.
\] 
\end{proof}

We denote by $\Maior$ the value of $\max_{\A \in \alphabet}
\{\pont{\matriz}{\A}{\espaco}, \pont{\matriz}{\espaco}{\A}\}$ and by
$\maior \in \alphabet$ the symbol in $\alphabet$ such that $\Maior =
\max \{\pont{\matriz}{\maior}{\espaco},
\pont{\matriz}{\espaco}{\maior}\}$.

\begin{proposition}\label{propAux2}\rm
Let $s, t \in \alphabet^{*}$. Then, $\distanciaN{\matriz}(s, t) \le
\Maior$.
\end{proposition}
\begin{proof}
Notice that $\alinhamento{s \espaco^{\tamanho{t}},
  \espaco^{\tamanho{s}} t}$ is an alignment of $s, t$. Thus,
\begin{align*}
\distanciaN{\matriz}(s, t) &\le \custoN{\matriz}  \alinhamentoB{
  \begin{array}{cc}
    s & \espaco^{\tamanho{t}}\\
    \espaco^{\tamanho{s}} & t
  \end{array}
}  \le \frac{\tamanho{s} \Maior + \tamanho{t} \Maior}{\tamanho{s}
  + \tamanho{t}} = \Maior\,.
\end{align*}
\end{proof}

\begin{proposition}\label{Q=0}\rm
Let $\distanciaN{\matriz} \in \prametrica$, $\A \in \alphabet$, and
$s, t \in \alphabet^{*}$. Then,
\begin{enumerate}
\item[(\textit{i})] If $\Maior = 0$, then $\distanciaN{\matriz}(s, t)
  = \pont{\matriz}{\A}{\espaco} = \pont{\matriz}{\espaco}{\A} = 0$\,, and
\item[(\textit{ii})] If $\Maior \not= 0$, then
  $\pont{\matriz}{\maior}{\espaco} + \pont{\matriz}{\espaco}{\maior} >
  0$ and $\pont{\matriz}{\maior}{\maior} = 0$\,.
\end{enumerate}
\end{proposition}
\begin{proof}
Suppose that $\Maior = 0$.
Since $\distanciaN{\matriz} \in \prametrica$, we have that
$\distanciaN{\matriz}(s, t) \ge 0$ and, from
Proposition~\ref{propAux2}, we have that $\distanciaN{\matriz}(s, t) \le
\Maior = 0$. It follows that $\distanciaN{\matriz}(s, t) = 0$. Since
this is true for each $s, t \in \alphabet^{*}$ and $\alinhamento{\A,
  \espaco}$, $\alinhamento{\espaco, \A}$ are the only alignments of
$\A, \seqVazia$ and $\seqVazia, \A$, respectively, we have that
\begin{align*}
  \pont{\matriz}{\A}{\espaco} &= \custoA{\matriz}\alinhamento{\A,
    \espaco} = \distanciaN{\matriz}(\A,\seqVazia) = 0 \quad
  \text{and} \\
  \pont{\matriz}{\espaco}{\A} &=
  \custoA{\matriz}\alinhamento{\espaco, \A} =
  \distanciaN{\matriz}(\seqVazia, \A) = 0\,.
\end{align*}

Suppose that $\Maior \not= 0$. Since $\distanciaN{\matriz} \in
\prametrica$, we have $\min\{ \pont{\matriz}{\maior}{\espaco},
\pont{\matriz}{\espaco}{\maior} \} \ge 0$.
%it follows from Corollary~\ref{corolaryPrametrica} that
It follows that 
$\Maior = \max \{ \pont{\matriz}{\maior}{\espaco},
\pont{\matriz}{\espaco}{\maior} \} \ge 
\min\{ \pont{\matriz}{\maior}{\espaco},
\pont{\matriz}{\espaco}{\maior} \} \ge 0$ and since
by hypothesis $\Maior \not= 0$, we have that 
$\max \{ \pont{\matriz}{\maior}{\espaco},
\pont{\matriz}{\espaco}{\maior} \} = \Maior > 0$.
It follows that
\[
\pont{\matriz}{\maior}{\espaco} + \pont{\matriz}{\espaco}{\maior} =
\max \{ \pont{\matriz}{\maior}{\espaco},
\pont{\matriz}{\espaco}{\maior} \} + \min \{
\pont{\matriz}{\maior}{\espaco}, \pont{\matriz}{\espaco}{\maior} \} >
0\,,
\]
which also implies, from Corollary~\ref{corolaryPrametrica}, that
$\pont{\matriz}{\maior}{\maior} = 0$.
\end{proof}

\begin{fact}\label{fato1}\rm
Let $x, z, k, w$ be real numbers. If $k \ge 0$ and $w > 0$, then
\[
\frac{kx + z}{k + w} \ge \min \left\{ x, \frac{z}{w} \right\}.
\]
\end{fact}


\begin{proposition}\label{propAux}\rm
Let $\distanciaN{\matriz} \in \prametrica$ and $\Maior \not= 0$. If
\[
\pont{\matriz}{q}{\espaco} = \pont{\matriz}{\espaco}{q} \quad
\text{or} \quad (\pont{\matriz}{\A}{\espaco} \le
\pont{\matriz}{\A}{\maior} +
\pont{\matriz}{\maior}{\espaco}~\text{and}~\pont{\matriz}{\espaco}{\B}
\le \pont{\matriz}{\espaco}{\maior} + \pont{\matriz} {\maior}{\B})\,,
\]
for each $\A, \B \in \alphabet$, then there is $n_{0}$ such that, for
each integer $n \ge n_{0}$, we have that
\[
\distanciaN{\matriz} (q^{n} \A, q^{n} \B) = \min \left\{
\begin{array}{l}
\custoN{\matriz} \alinhamento{q^{n} \A, q^{n} \B} =
\frac{\pont{\matriz}{\A}{\B}}{n + 1}\,,\\
\custoN{\matriz} \alinhamento{q^{n} \A\espaco, q^{n} \espaco\B} =
\frac{\pont{\matriz}{\A}{\espaco} + \pont{\matriz}{\espaco}{\B}}{n +
  2}
\end{array} \right\}.
\]
\end{proposition}
\begin{proof}
Let $\distanciaN{\matriz} \in \prametrica$ and $\Maior \not= 0$ such
that $\pont{\matriz}{q}{\espaco} = \pont{\matriz}{\espaco}{q}$ or
$\pont{\matriz}{\A}{\espaco} \le \pont{\matriz}{\A}{\maior} +
\pont{\matriz}{\maior}{\espaco}$ and $\pont{\matriz}{\espaco}{\B} \le
\pont{\matriz}{\espaco}{\maior} + \pont{\matriz} {\maior}{\B}$, for
each $\A, \B \in \alphabet$. Define, for each $\A, \B \in \alphabet$,
\[
n > \frac{\max \{ 
  \pont{\matriz}{\A}{\B}, 
  \pont{\matriz}{\A}{\espaco}+\pont{\matriz}{\espaco}{\B},
  \pont{\matriz}{\A}{\maior}+\pont{\matriz}{\espaco}{\B} + 
  \pont{\matriz}{\maior}{\espaco},
  \pont{\matriz}{\maior}{\B}+\pont{\matriz}{\A}{\espaco} + 
  \pont{\matriz}{\espaco}{\maior}
  \} } 
{\pont{\matriz}{\maior}{\espaco} + \pont{\matriz}{\espaco}{\maior}}\,.
\]
Since $\distanciaN{\matriz} \in \prametrica$ and $\Maior \not= 0$, we have that
$\pont{\matriz}{\maior}{\espaco} + \pont{\matriz}{\espaco}{\maior} >
0$ and $\pont{\matriz}{\maior}{\maior} = 0$
from Proposition~\ref{Q=0}, and since $\pont{\matriz}{\maior}{\maior} = 0$, we have that
\[
\custoN{\matriz} \alinhamentoB{\begin{array}{cc}
    \maior^{n} & \A\\
    \maior^{n} & \B
  \end{array}
}  = \frac{\pont{\matriz}{\A}{\B}}{n + 1}
\quad \text{and} \quad
\custoN{\matriz} \alinhamentoB{ \begin{array}{ccc}
    \maior^{n} & \A & \espaco\\
    \maior^{n} & \espaco & \B\\
  \end{array}
}  = \frac{\pont{\matriz}{\A}{\espaco} +
  \pont{\matriz}{\espaco}{\B}}{n + 2}\,.
\]
Hence, in order to prove this proposition, we have to show that 
\[
\custoN{\matriz}\alinhamento{s', t'} \ge \min \left\{
\frac{\pont{\matriz}{\A}{\B}}{n + 1},
\frac{\pont{\matriz}{\A}{\espaco} + \pont{\matriz}{\espaco}{\B}}{n +
  2} \right\},
\]
for each alignment $\alinhamento{s', t'}$ of $\maior^{n} \A,
\maior^{n} \B$.

Let $k$ be the number of symbols~$\espaco$ in $s'$. It follows that the number of symbols $\espaco$ in $t'$ is also $k$ and
$\tamanho{\alinhamento{s',t'}} = k + n + 1$. We examine four cases,
covering all possible alignments of $\maior^{n}\A ,\maior^{n}\B$.

\begin{description}
\item[Case 1:] $[\A, \B]$ is aligned in $\alinhamento{s',t'}$.

In this case, $k \ge 0$. 
Since $\pont{\matriz}{\maior}{\maior} = 0$,
we have that
\begin{align}
  \custoN{\matriz}\alinhamento{s',t'} &=
  \frac{k(\pont{\matriz}{\maior}{\espaco} +
    \pont{\matriz}{\espaco}{\maior}) + \pont{\matriz}{\A}{\B}}{k +
    n+1} \notag \\ &\ge \min \Big\{ \pont{\matriz}{\maior}{\espaco} +
  \pont{\matriz}{\espaco}{\maior}, \frac{\pont{\matriz}{\A}{\B}}{n+1}
  \Big\} = \frac{\pont{\gamma}{\A}{\B}}{n+1}\,. \label{p2}
\end{align}
Since $k \ge 0$ and $n + 1 > 0$, the inequality~(\ref{p2}) follows
from Fact~\ref{fato1} and the equality follows by the choice of $n$.

\begin{comment}
Now, since $n > \pont{\matriz}{\A}{\B} /
(\pont{\matriz}{\maior}{\espaco} + \pont{\matriz}{\espaco}{\maior})$
and $\pont{\matriz}{\maior}{\espaco} + \pont{\matriz}{\espaco}{\maior}
> 0$, we have that
\[
n > \frac{\pont{\matriz}{\A}{\B}}{\pont{\matriz}{\maior}{\espaco} + 
\pont{\matriz}{\espaco}{\maior}} > 
\frac{\pont{\matriz}{\A}{\B}- 
(\pont{\matriz}{\maior}{\espaco} + \pont{\matriz}{\espaco}{\maior})}
{\pont{\matriz}{\maior}{\espaco} + 
\pont{\matriz}{\espaco}{\maior}}\, ,
\]
which implies, since $n > 0$, that $\pont{\matriz}{\maior}{\espaco} +
\pont{\matriz}{\espaco}{\maior} > \pont{\matriz}{\A}{\B}/(n+1)$. It
follows from inequality~(\ref{p2}) that
\[
\custoN{\matriz}\alinhamento{s',t'} \ge \min \Big\{
\pont{\matriz}{\maior}{\espaco} + \pont{\matriz}{\espaco}{\maior},
\frac{\pont{\matriz}{\A}{\B}}{n+1} \Big\} =
\frac{\pont{\matriz}{\A}{\B}}{n+1}\,.
\]
\end{comment}

\item[Case 2:] $[\A,\espaco]$ and $[\espaco, \B]$ are aligned in
  $\alinhamento{s',t'}$.

In this case, $k \ge 1$. Since $\pont{\matriz}{\maior}{\maior} = 0$,
we have that
\begin{align*}
  \custoN{\matriz}\alinhamento{s',t'} &= \frac{(k - 1)
    (\pont{\matriz}{\maior}{\espaco} +
    \pont{\matriz}{\espaco}{\maior}) + \pont{\matriz}{\A}{\espaco} +
    \pont{\matriz}{\espaco}{\B}} {(k - 1) + n + 2}\\
  &\ge \min \Big\{ \pont{\matriz}{\maior}{\espaco} +
  \pont{\matriz}{\espaco}{\maior}, \frac{\pont{\matriz}{\A}{\espaco} +
    \pont{\matriz}{\espaco}{\B}} {n + 2} \Big\} =
\frac{\pont{\matriz}{\A}{\espaco} +
  \pont{\matriz}{\espaco}{\B}}{n+2}\,.
\end{align*}
The inequality, since $k - 1 \ge 0$ and $n + 2 > 0$, follows from
Fact~\ref{fato1} and the last equality follows by the choice of $n$.

\begin{comment}
Since $n + 2 > 0$, $n > (\pont{\matriz}{\A}{\espaco}
+ \pont{\matriz}{\espaco}{\B}) / (\pont{\matriz}{\maior}{\espaco} +
\pont{\matriz}{\espaco}{\maior})$ and $\pont{\matriz}{\maior}{\espaco}
+ \pont{\matriz}{\espaco}{\maior} > 0$, using similar arguments as in
Case 1, it follows that
\[
\custoN{\matriz}\alinhamento{s',t'} \ge
\frac{\pont{\matriz}{\A}{\espaco} +
  \pont{\matriz}{\espaco}{\B}}{n+2}\,.
\]
\end{comment}

\item[Case 3:] $[\A,\maior]$ is aligned in $\alinhamento{s',t'}$. 

In this case, $k \ge 1$. Also, since $\pont{\matriz}{\maior}{\maior} =
0$, we have that
\begin{align}
  \custoN{\matriz}\alinhamento{s',t'} &= \frac{(k - 1)
    (\pont{\matriz}{\maior}{\espaco} +
    \pont{\matriz}{\espaco}{\maior}) +
    \pont{\matriz}{\A}{\maior}+\pont{\matriz}{\espaco}{\B} +
    \pont{\matriz}{\maior}{\espaco}}{(k - 1)+ n+2} \nonumber \\
  &\ge \min \Big\{ \pont{\matriz}{\maior}{\espaco} +
  \pont{\matriz}{\espaco}{\maior},
  \frac{\pont{\matriz}{\A}{\maior}+\pont{\matriz}{\espaco}{\B}+
    \pont{\matriz}{\maior}{\espaco}}{n+2} \Big\} \label{p3}\\
  &= \frac{\pont{\matriz}{\A}{\maior}+\pont{\matriz}{\espaco}{\B}+
    \pont{\matriz}{\maior}{\espaco}}{n+2}\,. \label{p4}
\end{align}
Since $k - 1 \ge 0$ and $n + 2 > 0$, inequality~(\ref{p3}) follows
from Fact~\ref{fato1} and equality~(\ref{p4}) follows 
by the choice of $n$.

\begin{comment}
Using similar arguments as in Case 1, since $n
+ 2 > 0$, $n > (\pont{\matriz}{\A}{\maior} +
\pont{\matriz}{\espaco}{\B} + \pont{\matriz}{\maior}{\espaco}) /
(\pont{\matriz}{\maior}{\espaco} + \pont{\matriz}{\espaco}{\maior})$
and $\pont{\matriz}{\maior}{\espaco} + \pont{\matriz}{\espaco}{\maior}
> 0$, it follows the equality~(\ref{p4}).
\end{comment}

Suppose that $\pont{\matriz}{\maior}{\espaco} =
\pont{\matriz}{\espaco}{\maior}$. Then,
$\pont{\matriz}{\maior}{\espaco} = \min
\{\pont{\matriz}{\maior}{\espaco},\pont{\matriz}{\espaco}{\maior} \} =
\Maior$. Since $\distanciaN{\matriz} \in \prametrica$, we have from
Corollary~\ref{corolaryPrametrica} that $\pont{\matriz}{\A}{\maior}
\ge 0$. It follows from equality~(\ref{p4}) and the definition of
$\Maior$ that
\begin{align*}
  \custoN{\matriz}\alinhamento{s',t'} &\ge
  \frac{\pont{\matriz}{\A}{\maior}+\pont{\matriz}{\espaco}{\B}+
    \pont{\matriz}{\maior}{\espaco}}{n+2} \ge 
    \frac{0 + \pont{\matriz}{\espaco}{\B}+ \Maior}{n+2} \\
    &\ge \frac{\pont{\matriz}{\espaco}{\B} +
    \pont{\matriz}{\A}{\espaco}}{n+2}
    = \frac{\pont{\matriz}{\A}{\espaco} + \pont{\matriz}{\espaco}{\B}
    }{n+2}\,,
\end{align*}
and the proof is complete. Suppose then that $\pont{\matriz}{\maior}{\espaco} \not=
\pont{\matriz}{\espaco}{\maior}$. By hypothesis, this implies that
$\pont{\matriz}{\A}{\espaco} \le \pont{\matriz}{\A}{q} +
\pont{\matriz}{q}{\espaco}$. It follows from equality~(\ref{p4}) that
\[
\custoN{\matriz}\alinhamento{s',t'} \ge
\frac{\pont{\matriz}{\A}{\maior}+\pont{\matriz}{\espaco}{\B}+
  \pont{\matriz}{\maior}{\espaco}}{n+2} \ge
\frac{\pont{\matriz}{\A}{\espaco}+\pont{\matriz}{\espaco}{\B}}{n+2}\,.
\]

\item[Case 4:] $[\maior, \B]$ is aligned in $\alinhamento{s',t'}$. Similar to Case 3.
\begin{comment}
, since $n + 2 > 0$, $k - 1 \ge 0$, $n >
\pont{\matriz}{\maior}{\B}+\pont{\matriz}{\A}{\espaco} +
\pont{\matriz}{\espaco}{\maior}$, and $\pont{\matriz}{\espaco}{\B} \le
\pont{\matriz}{\espaco}{\maior} + \pont{\matriz}{\maior}{\B}$.
\end{comment}
\end{description}
\end{proof}

\begin{proposition}\label{propAux-3}\rm
Suppose that $\distanciaN{\matriz} \in \prametrica$. Let $\A, \B \in
\alphabet$ and $s, t \in \alphabet^{*}$. If $[\A, \B]$ is aligned in
an N-optimal alignment of $s, t$ of maximum length, then
\[
\pont{\matriz}{\A}{\B} < \pont{\matriz}{\A}{\espaco} +
\pont{\matriz}{\espaco}{\B}\,.
\] 
\end{proposition}
\begin{proof}
Let $\alinhamento{s', t'}$ be a N-optimal alignment of maximum length
of $s, t$ and $j$ be an integer such that 
$s'(j) = \A$ and $t'(j) = \B$. Consider the following alignment of $s, t$
\[
A =  
\alinhamentoB{ \begin{array}{cccccccc}
s'(1) & \ldots & s'(j - 1) & \A & \espaco & s'(j+1) & \ldots & 
s'(\tamanho{s'})\\
t'(1) & \ldots & t'(j - 1) & \espaco & \B & t'(j+1) & \ldots & 
t'(\tamanho{t'})
\end{array}
}.
\]

Since $\alinhamento{s',t'}$ is an N-optimal alignment
of maximum length and $\abs{[s',t']} < \abs{A}$, we have that
\begin{align*}
\frac{\custoA{\matriz}\alinhamento{s',t'}}{\tamanho{\alinhamento{s',t'}}}
= \custoN{\matriz}\alinhamento{s', t'} &< \custoN{\matriz}[A] =
\frac{\custoA{\matriz}[A]}{\tamanho{A}}\\
&= \frac{\custoA{\matriz}\alinhamento{s',t'} +
  \pont{\matriz}{\A}{\espaco} + \pont{\matriz}{\espaco}{\B} -
  \pont{\matriz}{\A}{\B}}{\tamanho{\alinhamento{s',t'}}+1}\,,
\end{align*}
which implies that 
\[
\frac{\custoA{\matriz}\alinhamento{s',t'}}{\tamanho{\alinhamento{s',t'}}}
< \pont{\matriz}{\A}{\espaco} + \pont{\matriz}{\espaco}{\B} -
\pont{\matriz}{\A}{\B}\,. %\label{eq-propAux-3}
\]
Since $\distanciaN{\matriz} \in \prametrica$, we have that
$\distanciaN{\matriz}(s, t) \ge 0$.
%% This implies, since, by hypothesis,
%% $\distanciaN{\matriz}(s, t) \not= 0$, that $\distanciaN{\matriz}(s, t) > 0$. 
%% It follows by~(\ref{eq-propAux-3}), since $(s', t')$ is a N-optimal
%% alignment of $(s, t)$, that
It follows that
\[
0 \le \distanciaN{\matriz}(s, t) =
\frac{\custoA{\matriz}\alinhamento{s',t'}}{\tamanho{\alinhamento{s',t'}}}
< \pont{\matriz}{\A}{\espaco} + \pont{\matriz}{\espaco}{\B} -
\pont{\matriz}{\A}{\B}\,,
\]
which implies that $\pont{\matriz}{\A}{\B} <
\pont{\matriz}{\A}{\espaco} + \pont{\matriz}{\espaco}{\B}$.
\end{proof}

\begin{fact}\label{fato2}\rm
Let $x, y \in \mathbb{R}$.  If $x < y$ then there exists an
integer $n_{0}$ such that
\[
\frac{x}{n + 1} < \frac{y}{n + 2}\,,
\]
for each $n > n_{0}$.
\end{fact}
 
\begin{lemma}\label{distanciaNSimetrica}\rm
Let $\distanciaN{\matriz} \in \prametrica$. Then,
$\distanciaN{\matriz}(s, t) = \distanciaN{\matriz}(t, s)$ for each $s,
t \in \alphabet^{*}$ if and only if
\begin{enumerate}
\item [(\textit{i})] $\pont{\matriz}{\A}{\espaco} =
  \pont{\matriz}{\espaco}{\A}$\,, and
\item [(\textit{ii})] if $\pont{\matriz}{\A}{\B} <
  \pont{\matriz}{\A}{\espaco} + \pont{\matriz}{\espaco}{\B}$, then
  $\pont{\matriz}{\A}{\B} = \pont{\matriz}{\B}{\A}$\,,
\end{enumerate}
for each $\A, \B \in \alphabet$.
\end{lemma}
\begin{proof}
By Corollary~\ref{corolaryPrametrica}, we have that
$\pont{\matriz}{\A}{\B}, \pont{\matriz}{\maior}{\espaco},
\pont{\matriz}{\espaco}{\maior} \ge 0$.

Suppose that $\distanciaN{\matriz}(s, t) =
\distanciaN{\matriz}(t, s)$ for each $s, t \in \alphabet^{*}$.
Since $\alinhamento{\A, \espaco}$ and $\alinhamento{\espaco, \A}$ are
the only alignments of $\A, \seqVazia$ and~$\seqVazia, \A$,
respectively, and by hypothesis, $\distanciaN{\matriz}(\A, \seqVazia) =
\distanciaN{\matriz}(\seqVazia, \A)$, we have that
\[
\pont{\matriz}{\A}{\espaco} =
\custoN{\matriz}\alinhamento{\A, \espaco} = 
\distanciaN{\matriz}(\A, \seqVazia) =  
\distanciaN{\matriz}(\seqVazia, \A) = 
\custoN{\matriz}\alinhamento{\espaco, \A} =  
\pont{\matriz}{\espaco}{\A}\,,
\]
which implies that (\textit{i}) is true.
In order to show (\textit{ii}), we consider two possibilities: $\Maior
= 0$ and $\Maior \not= 0$. If $\Maior = 0$, since
$\pont{\matriz}{\A}{\B} \ge 0$ and $\Maior \ge
\pont{\matriz}{\A}{\espaco}, \pont{\matriz}{\espaco}{\B}$, we have
that
$
\pont{\matriz}{\A}{\B} \ge 0 = 0 + 0 = \Maior + \Maior \ge 
\pont{\matriz}{\A}{\espaco} + \pont{\matriz}{\espaco}{\B}
$ and in this case 
(\textit{ii}) is vacuously satisfied. Thus, assume that $\Maior \not= 0$ and suppose that $\pont{\matriz}{\A}{\B} <
\pont{\matriz}{\A}{\espaco} + \pont{\matriz}{\espaco}{\B}$. Choose $n$
big enough satisfying both Proposition~\ref{propAux}, since $\pont{\matriz}{\maior}{\espaco} = \pont{\matriz}{\espaco}{\maior}$, and 
Fact~\ref{fato2}. It follows that
\begin{align*}
\distanciaN{\matriz}(\maior^{n} \A, \maior^{n} \B) &= 
\min \left\{ \frac{\pont{\matriz}{\A}{\B}}{n + 1}, 
\frac{\pont{\matriz}{\A}{\espaco} + \pont{\matriz}{\espaco}{\B}}{n + 2}
\right\}= \frac{\pont{\matriz}{\A}{\B}}{n + 1}
< \frac{\pont{\matriz}{\A}{\espaco} +  \pont{\matriz}{\espaco}{\B}}{n +
 2} \hspace{0.2cm} \mbox{and}\\
\distanciaN{\matriz}(\maior^{n} \B, \maior^{n} \A) &=  
\min \left\{ \frac{\pont{\matriz}{\B}{\A}}{n + 1}, 
\frac{\pont{\matriz}{\B}{\espaco} + \pont{\matriz}{\espaco}{\A}}{n + 2}
\right\}.
\end{align*}
Since $\pont{\matriz}{\A}{\espaco} =
\pont{\matriz}{\espaco}{\A}$, $\pont{\matriz}{\B}{\espaco} =
\pont{\matriz}{\espaco}{\B}$, and $\distanciaN{\matriz}(\maior^{n} \A,
\maior^{n} \B) = \distanciaN{\matriz}(\maior^{n} \B, \maior^{n}
\A)$, we have that $(\pont{\matriz}{\B}{\espaco} +
\pont{\matriz}{\espaco}{\A})/(n + 2)$ is not a value of
$\distanciaN{\matriz}(\maior^{n} \B, \maior^{n} \A)$, which implies
that $\distanciaN{\matriz}(\maior^{n} \B, \maior^{n} \A) =
\pont{\matriz}{\B}{\A}/(n + 1)$. Therefore,
\[
\frac{\pont{\matriz}{\A}{\B}}{n + 1} =
\distanciaN{\matriz}(\maior^{n} \A, \maior^{n} \B) = 
\distanciaN{\matriz}(\maior^{n} \B, \maior^{n} \A) = 
\frac{\pont{\matriz}{\B}{\A}}{n + 1}\,,
\]
and thus $\pont{\matriz}{\A}{\B} = \pont{\matriz}{\B}{\A}$.

Conversely, consider that (\textit{i}) and (\textit{ii}) are true.
Let $\alinhamento{s', t'}$ be an N-optimal alignment of maximum length
of $s, t$. If $s'(j) = \espaco$ or $t'(j) = \espaco$ then from
(\textit{i}) we have that $\pont{\matriz}{s'(j)}{t'(j)} =
\pont{\matriz}{t'(j)}{s'(j)}$. If $s'(j) \not= \espaco$ and $t'(j)
\not= \espaco$ then, since 
$\distanciaN{\matriz} \in \prametrica$, we have from
Proposition~\ref{propAux-3} that $\pont{\matriz}{s'(j)}{t'(j)} <
\pont{\matriz}{s'(j)}{\espaco} + \pont{\matriz}{\espaco}{t'(j)}$,
which implies by (\textit{ii}) that $\pont{\matriz}{s'(j)}{t'(j)} =
\pont{\matriz}{t'(j)}{s'(j)}$. Using these observations, we have that
\begin{align*}
\distanciaN{\matriz}(s, t) &= \custoN{\matriz}\alinhamento{s', t'} =
\frac{\sum_{j} \pont{\matriz}{s'(j)}{t'(j)}}{\tamanho{(s', t')}} \\
&= \frac{\sum_{j} \pont{\matriz}{t'(j)}{s'(j)}}{\tamanho{(t', s')}} =
\custoN{\matriz}\alinhamento{t', s'} \ge \distanciaN{\matriz}(t,
s)\,.
\end{align*}
Similarly, we have $\distanciaN{\matriz}(s, t) \le
\distanciaN{\matriz}(t, s)$, which allows us to conclude that
$\distanciaN{\matriz}(s, t) = \distanciaN{\matriz}(t, s)$.
\end{proof}

\begin{proposition}\label{AuxdistanciaNTriangular-1}\rm
Let $\distanciaN{\matriz} \in \prametrica$.  If $\distanciaN{\matriz}(s, t) \le \distanciaN{\matriz}
(s, u) + \distanciaN{\matriz} (u, t)$ for all $s, t, u \in
\alphabet^{*}$ then
\[
\pont{\matriz}{\A}{\espaco} \le 
\pont{\matriz}{\A}{\B} + \pont{\matriz}{\B}{\espaco}
\qquad \text{and} \qquad
\pont{\matriz}{\espaco}{\A} \le
\pont{\matriz}{\espaco}{\B} + \pont{\matriz}{\B}{\A}\,,
\]
for all $\A, \B \in
\alphabet$.
\end{proposition}
\begin{proof}
Suppose that $\distanciaN{\matriz}(s, t) \le \distanciaN{\matriz}(s, u) +
\distanciaN{\matriz}(u, t)$ for all $s, t, u \in \alphabet^{*}$.
Since $\alinhamento{\A, \espaco}$ is the only alignment of $\A,
\seqVazia$, we have that
$
\distanciaN{\matriz}(\A, \seqVazia) = 
%\custoN{\matriz}\alinhamento{\A, \espaco} =
\pont{\matriz}{\A}{\espaco}$, and since $\alinhamento{\A, \B}$ and $\alinhamento{\B, \espaco}$ are
alignments of $\A, \B$ and $\B, \seqVazia$, it follows
that
\begin{align*}
\pont{\matriz}{\A}{\espaco} = \distanciaN{\matriz}(\A, \seqVazia) &\le
\distanciaN{\matriz}(\A, \B) + \distanciaN{\matriz}(\B, \seqVazia) \\
&\le \custoN{\matriz}\alinhamento{\A, \B} +
\custoN{\matriz}\alinhamento{\B, \espaco} \\
&= \pont{\matriz}{\A}{\B} + \pont{\matriz}{\B}{\espaco}\,.
\end{align*}

Using similar arguments we also prove that
$\pont{\matriz}{\espaco}{\A} \le \pont{\matriz}{\espaco}{\B} +
\pont{\matriz}{\B}{\A}$.
\end{proof}

\begin{proposition}\label{AuxdistanciaNTriangular}\rm
Let $\distanciaN{\matriz} \in \prametrica$. If
$\distanciaN{\matriz}(s, t) \le \distanciaN{\matriz}(s, u) +
\distanciaN{\matriz}(u, t)$ for each $s, t, u \in \alphabet^{*}$
then
\[
\max\{\pont{\matriz}{\A}{\espaco}, \pont{\matriz}{\espaco}{\A}\}
\le \pont{\matriz}{\B}{\espaco} + \pont{\matriz}{\espaco}{\B}\,,
\]
for each and $\A, \B \in \alphabet$.
\end{proposition}
\begin{proof}
Suppose that
$\distanciaN{\matriz}(s, t) \le \distanciaN{\matriz}(s, u) +
\distanciaN{\matriz}(u, t)$ for all $s, t, u \in \alphabet^{*}$.

Since $\distanciaN{\matriz} \in \prametrica$,
if $\Maior = 0$, then
$\pont{\matriz}{\A}{\espaco} = \pont{\matriz}{\espaco}{\A} =
\pont{\matriz}{\B}{\espaco} = 
\pont{\matriz}{\espaco}{\B} = 
0$ as a consequence of Corollary~\ref{corolaryPrametrica}
and, therefore, the propositions is proved. 
Thus, assume that $\Maior \not= 0$. That is,
$\Maior > 0$ from Corollary~\ref{corolaryPrametrica}.
W.l.o.g., assume that 
$\pont{\matriz}{\maior}{\espaco} = \Maior$.

Suppose that there exists $\A, \B \in \alphabet$
such that $\max\{\pont{\matriz}{\A}{\espaco}, \pont{\matriz}{\espaco}{\A}\}
> \pont{\matriz}{\B}{\espaco} +
\pont{\matriz}{\espaco}{\B}$ by contradiction. This implies that
$\Maior > \pont{\matriz}{\B}{\espaco} +
\pont{\matriz}{\espaco}{\B}$. Let $k$ be an positive integer such that
\[
k > \frac{\pont{\matriz}{\B}{\espaco}+\pont{\matriz}{\espaco}{\B}}
{\Maior - (\pont{\matriz}{\B}{\espaco} +
  \pont{\matriz}{\espaco}{\B})}\,.
\]


Since $\distanciaN{\matriz}
\in \prametrica$, we have that
$\pont{\matriz}{\maior}{\espaco}, \pont{\matriz}{\espaco}{\maior} \ge 0$,
which implies, since 
$\pont{\matriz}{\maior}{\espaco} = \Maior >0$, that $\pont{\matriz}{\maior}{\espaco} + \pont{\matriz}{\espaco}{\maior} > 0$,
which in turn implies that  $\pont{\matriz}{\maior}{\maior} = 0$
as a consequence of
Corollary~\ref{corolaryPrametrica}. Then,
\begin{align}
\distanciaN{\matriz}(\maior^{k} ,\maior^{k} \B) &\le \custoN{\matriz} 
\alinhamentoB{
\begin{array}{cc}
  \maior^{k} & \espaco \\
  \maior^{k} & \B
\end{array}
} =
\frac{\pont{\matriz}{\espaco}{\B}}{k+1}\,, \label{AuxdistanciaNTriangular2}\\
\distanciaN{\matriz}(\maior^{k} \B, \B) &\le \custoN{\matriz} 
\alinhamentoB{
\begin{array}{ccc}
  \maior^{k} & \B & \espaco \\
\espaco^{k} & \espaco & \B
\end{array}
} = 
\frac{k \pont{\matriz}{\maior}{\espaco} + (\pont{\matriz}{\B}{\espaco} +
  \pont{\matriz}{\espaco}{\B})}{k+2}\,.\label{AuxdistanciaNTriangular3}
\end{align} 

In any alignment of $\maior^{k}, \B$ either $[\maior, \B]$ or $[\espaco, \B]$ is aligned. Consequently,
\begin{align*}
\distanciaN{\matriz}(\maior^{k}, \B) &= 
\min \left\{ \custoN{\matriz} 
\alinhamentoB{
  \begin{array}{cc}
    \maior^{k-1} & \maior \\
    \espaco^{k-1} & \B
  \end{array}
},
\custoN{\matriz}
\alinhamentoB{
  \begin{array}{cc}
    \maior^{k} & \espaco \\
    \espaco^{k} & \B
  \end{array}
}
\right\} \\
&= 
\min \left\{ \frac{(k-1) \pont{\matriz}{\maior}{\espaco} +
  \pont{\matriz}{\maior}{\B}}{k}, \frac{k \pont{\matriz}{\maior}{\espaco} +
  \pont{\matriz}{\espaco}{\B}}{k+1} \right\}\,.
\end{align*}
Suppose that $\distanciaN{\gamma}(\maior^k,\B) = 
{((k-1) \pont{\matriz}{\maior}{\espaco} +
  \pont{\matriz}{\maior}{\B})}/{k}$. 
It follows that 
\[
\frac{(k-1) \pont{\matriz}{\maior}{\espaco} + \pont{\matriz}{\maior}{\B}}{k}
\le \frac{k \pont{\matriz}{\maior}{\espaco} +
  \pont{\matriz}{\espaco}{\B}}{k+1}\,.
\]
We
have that $\pont{\matriz}{\maior}{\espaco} \le \pont{\matriz}{\maior}{\B} +
\pont{\matriz}{\B}{\espaco}$ by Proposition~\ref{AuxdistanciaNTriangular-1}, which implies, since $k > 0$,  that
\[
\frac{k \pont{\matriz}{\maior}{\espaco} - \pont{\matriz}{\B}{\espaco}}{k}
\le \frac{(k-1) \pont{\matriz}{\maior}{\espaco} +
  \pont{\matriz}{\maior}{\B}}{k}\,.
\]
It follows that
\[
\frac{k \pont{\matriz}{\maior}{\espaco} - \pont{\matriz}{\B}{\espaco}}{k}
\le \frac{k \pont{\matriz}{\maior}{\espaco} +
  \pont{\matriz}{\espaco}{\B}}{k+1}\,,
\]
and, since $k>0$, that
$k(\pont{\gamma}{\maior}{\espaco} -(\pont{\matriz}{\B}{\espaco} +
\pont{\matriz}{\espaco}{\B})) \le  \pont{\matriz}{\B}{\espaco}$
which implies, since $\Maior =
\pont{\matriz}{\maior}{\espaco} > \pont{\matriz}{\B}{\espaco} +
\pont{\matriz}{\espaco}{\B}$, that $k \le \pont{\matriz}{\B}{\espaco}/
(\Maior - (\pont{\matriz}{\B}{\espaco} +
\pont{\matriz}{\espaco}{\B}))$, contradicting the choice of
$k$ since $\pont{\matriz}{\espaco}{\B} \ge 0$. Thus,
we assume that
\begin{equation}
\distanciaN{\matriz}(\maior^{k}, \B) = \frac{k \pont{\matriz}{\maior}{\espaco}
  + \pont{\matriz}{\espaco}{\B}}{k+1}\,.
\label{AuxdistanciaNTriangular4}
\end{equation}

Since $\distanciaN{\matriz}(\maior^{k}, \B) \le
\distanciaN{\matriz}(\maior^{k} ,\maior^{k} \B) + \distanciaN{\matriz}(\maior^{k}
\B, \B)$, we have from Equations~(\ref{AuxdistanciaNTriangular2}),
(\ref{AuxdistanciaNTriangular3}), and~(\ref{AuxdistanciaNTriangular4})
that
\begin{align*}
\frac{k \pont{\matriz}{\maior}{\espaco} +
  \pont{\matriz}{\espaco}{\B}}{k+1} = \distanciaN{\matriz}(\maior^{k}, \B)
&\le \distanciaN{\matriz}(\maior^{k} ,\maior^{k} \B) +
\distanciaN{\matriz}(\maior^{k} \B, \B) \\
&\le \frac{\pont{\matriz}{\espaco}{\B}}{k+1} + \frac{k
  \pont{\matriz}{\maior}{\espaco} + (\pont{\matriz}{\B}{\espaco} +
  \pont{\matriz}{\espaco}{\B})}{k+2}\,,
\end{align*}
which implies, since $k > 0$ and $\Maior = \pont{\matriz}{\maior}{\espaco} >
\pont{\matriz}{\B}{\espaco} + \pont{\matriz}{\espaco}{\B}$, that
\[
k \le \frac{\pont{\matriz}{\B}{\espaco} + \pont{\matriz}{\espaco}{\B}}
{\Maior - (\pont{\matriz}{\B}{\espaco} +
  \pont{\matriz}{\espaco}{\B})}\,,
\]
contradicting again the choice of $k$. Thus, 
$\max\{\pont{\matriz}{\A}{\espaco}, \pont{\matriz}{\espaco}{\A}\}
\le \pont{\matriz}{\B}{\espaco} + \pont{\matriz}{\espaco}{\B}$
for each $\A,\B~\in~\Sigma$. 
\end{proof}

\begin{comment}

\begin{proposition}\label{AuxdistanciaNTriangular}\rm
Let $\distanciaN{\matriz} \in \prametrica$. If
$\distanciaN{\matriz}(s, t) \le \distanciaN{\matriz}(s, u) +
\distanciaN{\matriz}(u, t)$ for each $s, t, u \in \alphabet^{*}$,
then
\[
\max\{\pont{\matriz}{\A}{\espaco}, \pont{\matriz}{\espaco}{\A}\}
\le \pont{\matriz}{\B}{\espaco} + \pont{\matriz}{\espaco}{\B}\,.
\]
for each and $\B \in \alphabet$.
\end{proposition}
\begin{proof}
Suppose that 
$\distanciaN{\matriz}(s, t) \le \distanciaN{\matriz}(s, u) +
\distanciaN{\matriz}(u, t)$ for all $s, t, u \in \alphabet^{*}$ and let
$\A \in \alphabet$ such that $\mathcal{G} = \max_{\A \in \Sigma}\{
\pont{\matriz}{\A}{\espaco}, \pont{\matriz}{\espaco}{\A}\}$.

Since $\distanciaN{\matriz} \in \prametrica$,
if $\mathcal{G} = 0$, then
$\pont{\matriz}{\A}{\espaco} = \pont{\matriz}{\espaco}{\A} =
\pont{\matriz}{\B}{\espaco} = 
\pont{\matriz}{\espaco}{\B} = 
0$ as a consequence of Corollary~\ref{corolaryPrametrica}
and, therefore, the propositions is proved. 
Thus, assume that $\mathcal{G} \not= 0$. That is,
$\mathcal{G} > 0$ from Corollary~\ref{corolaryPrametrica}.

Suppose there exists $\B \in \alphabet$
such that $\mathcal{G} > \pont{\matriz}{\B}{\espaco} +
\pont{\matriz}{\espaco}{\B}$ by contradiction. Let $k$ be an positive integer such that
\[
k > \frac{\pont{\matriz}{\B}{\espaco}+\pont{\matriz}{\espaco}{\B}}
{\mathcal{G} - (\pont{\matriz}{\B}{\espaco} +
  \pont{\matriz}{\espaco}{\B})}\,.
\]
Since $\distanciaN{\matriz}
\in \prametrica$ and 
$\pont{\matriz}{\A}{\espaco} >
0$, we have $\pont{\matriz}{\A}{\A} = 0$
as a consequence of
Corollary~\ref{corolaryPrametrica}. Then,
\begin{align}
\distanciaN{\matriz}(\A^{k} ,\A^{k} \B) &\le \custoA{\matriz} 
\alinhamentoB{
\begin{array}{cc}
  \A^{k} & \espaco \\
  \A^{k} & \B
\end{array}
} =
\frac{\pont{\matriz}{\espaco}{\B}}{k+1}\,, \label{AuxdistanciaNTriangular2}\\
\distanciaN{\matriz}(\A^{k} \B, \B) &\le \custoN{\matriz} 
\alinhamentoB{
\begin{array}{ccc}
  \A^{k} & \B & \espaco \\
\espaco^{k} & \espaco & \B
\end{array}
} = 
\frac{k \pont{\matriz}{\A}{\espaco} + (\pont{\matriz}{\B}{\espaco} +
  \pont{\matriz}{\espaco}{\B})}{k+2}\,.\label{AuxdistanciaNTriangular3}
\end{align} 

In any alignment of $\A^{k}, \B$, either $[\A, \B]$ or $[\espaco, \B]$ is aligned. Consequently,
\begin{align*}
\distanciaN{\matriz}(\A^{k}, \B) &= 
\min \left\{ \custoN{\matriz} 
\alinhamentoB{
  \begin{array}{cc}
    \A^{k-1} & \A \\
    \espaco^{k-1} & \B
  \end{array}
},
\custoN{\matriz}
\alinhamentoB{
  \begin{array}{cc}
    \A^{k} & \espaco \\
    \espaco^{k} & \B
  \end{array}
}
\right\} \\
&= 
\min \left\{ \frac{(k-1) \pont{\matriz}{\A}{\espaco} +
  \pont{\matriz}{\A}{\B}}{k}, \frac{k \pont{\matriz}{\A}{\espaco} +
  \pont{\matriz}{\espaco}{\B}}{k+1} \right\}\,.
\end{align*}
Suppose that $\distanciaN{\gamma}(\A^k,\B) = 
{((k-1) \pont{\matriz}{\A}{\espaco} +
  \pont{\matriz}{\A}{\B})}/{k}$. 
It follows that 
\[
\frac{(k-1) \pont{\matriz}{\A}{\espaco} + \pont{\matriz}{\A}{\B}}{k}
\le \frac{k \pont{\matriz}{\A}{\espaco} +
  \pont{\matriz}{\espaco}{\B}}{k+1}\,.
\]
We
have that $\pont{\matriz}{\A}{\espaco} \le \pont{\matriz}{\A}{\B} +
\pont{\matriz}{\B}{\espaco}$ by Proposition~\ref{AuxdistanciaNTriangular-1}, which implies, since $k > 0$,  that
\[
\frac{k \pont{\matriz}{\A}{\espaco} - \pont{\matriz}{\B}{\espaco}}{k}
\le \frac{(k-1) \pont{\matriz}{\A}{\espaco} +
  \pont{\matriz}{\A}{\B}}{k}\,.
\]
It follows that
\[
\frac{k \pont{\matriz}{\A}{\espaco} - \pont{\matriz}{\B}{\espaco}}{k}
\le \frac{k \pont{\matriz}{\A}{\espaco} +
  \pont{\matriz}{\espaco}{\B}}{k+1}\,,
\]
and, since $k>0$, that
$k(\pont{\gamma}{\A}{\espaco} -(\pont{\matriz}{\B}{\espaco} +
\pont{\matriz}{\espaco}{\B})) \le  \pont{\matriz}{\B}{\espaco}$
which implies, since $\mathcal{G} =
\pont{\matriz}{\A}{\espaco} > \pont{\matriz}{\B}{\espaco} +
\pont{\matriz}{\espaco}{\B}$, that $k \le \pont{\matriz}{\B}{\espaco}/
(\mathcal{G} - (\pont{\matriz}{\B}{\espaco} +
\pont{\matriz}{\espaco}{\B}))$ contradicting the choice of
$k$ since $\pont{\matriz}{\espaco}{\B} \ge 0$. Thus,
we assume that
\begin{equation}
\distanciaN{\matriz}(\A^{k}, \B) = \frac{k \pont{\matriz}{\A}{\espaco}
  + \pont{\matriz}{\espaco}{\B}}{k+1}\,.
\label{AuxdistanciaNTriangular4}
\end{equation}

Since $\distanciaN{\matriz}(\A^{k}, \B) \le
\distanciaN{\matriz}(\A^{k} ,\A^{k} \B) + \distanciaN{\matriz}(\A^{k}
\B, \B)$, we have from Equations~(\ref{AuxdistanciaNTriangular2}),
(\ref{AuxdistanciaNTriangular3}), and~(\ref{AuxdistanciaNTriangular4})
that
\begin{align*}
\frac{k \pont{\matriz}{\A}{\espaco} +
  \pont{\matriz}{\espaco}{\B}}{k+1} = \distanciaN{\matriz}(\A^{k}, \B)
&\le \distanciaN{\matriz}(\A^{k} ,\A^{k} \B) +
\distanciaN{\matriz}(\A^{k} \B, \B) \\
&\le \frac{\pont{\matriz}{\espaco}{\B}}{k+1} + \frac{k
  \pont{\matriz}{\A}{\espaco} + (\pont{\matriz}{\B}{\espaco} +
  \pont{\matriz}{\espaco}{\B})}{k+2}\,,
\end{align*}
which implies, since $k > 0$ and $\mathcal{G} = \pont{\matriz}{\A}{\espaco} >
\pont{\matriz}{\B}{\espaco} + \pont{\matriz}{\espaco}{\B}$, that
\[
k \le \frac{\pont{\matriz}{\B}{\espaco} + \pont{\matriz}{\espaco}{\B}}
{\mathcal{G} - (\pont{\matriz}{\B}{\espaco} +
  \pont{\matriz}{\espaco}{\B})}\,,
\]
contradicting again the choice of $k$. Thus, 
$\mathcal{G} \le \pont{\matriz}{\B}{\espaco} + \pont{\matriz}{\espaco}{\B}$ for each $\B \in \Sigma$. 
\end{proof}





\begin{fact}\label{fato3}\rm
For each pair of numbers $x, y$, there exists a number $n_{0}$ such
that
\[
x \le y \quad \text{if and only if} \quad \frac{x}{n+2} <
\frac{y}{n+1}\,,
\]
for each $n \ge n_{0}$.
\end{fact}
\end{comment}


\begin{proposition}\label{AuxdistanciaNTriangular-2}\rm
Let $\distanciaN{\matriz} \in \prametrica$. If $\distanciaN{\matriz}(s, t) \le
\distanciaN{\matriz}(s, u) + \distanciaN{\matriz}(u, t)$ for each $s,
t, u \in \alphabet^{*}$ then
\[
\min \{ \pont{\matriz}{\A}{\C}, \pont{\matriz}{\A}{\espaco} +
\pont{\matriz}{\espaco}{\C} \} \le \pont{\matriz}{\A}{\B} +
\pont{\matriz}{\B}{\C}\,,
\]
for each $\A, \B, \C \in \alphabet$.
\end{proposition}
\begin{proof}
Since $\distanciaN{\gamma} \in \prametrica$, from  Corollary~\ref{corolaryPrametrica} we have that
$\pont{\gamma}{\A}{\B}, \pont{\gamma}{\A}{\C}, \pont{\gamma}{\B}{\C} \ge 0$ for each $\A, \B, \C \in \Sigma$.

If $\mathcal{Q} = 0$, from Proposition~\ref{Q=0}, we have 
$\pont{\gamma}{\A}{\espaco} = \pont{\gamma}{\espaco}{\A} = 0$
for each $\A \in \Sigma$ since $\distanciaN{\gamma} \in \prametrica$.
Since $\pont{\gamma}{\A}{\B}, \pont{\gamma}{\A}{\C}, \pont{\gamma}{\B}{\C} \ge 0$ for each $\A, \B, \C \in \Sigma$,
it follows that 
$\min \{ \pont{\matriz}{\A}{\C}, \pont{\matriz}{\A}{\espaco} +
\pont{\matriz}{\espaco}{\C} \} = 0 \le \pont{\matriz}{\A}{\B} +
\pont{\matriz}{\B}{\C}$ and the proposition is proved. 
Thus, assume that $\mathcal{Q} \not= 0$.

Suppose that
$\distanciaN{\matriz}(s, t) \le \distanciaN{\matriz}(s, u) +
\distanciaN{\matriz}(u, t)$ for each $s, t, u \in\alphabet^{*}$. 
We have that
$\pont{\matriz}{\A}{\espaco} \le \pont{\matriz}{\A}{\maior} +
\pont{\matriz}{\maior}{\espaco}$ 
and 
$\pont{\matriz}{\espaco}{\C} \le \pont{\matriz}{\espaco}{\maior} +
\pont{\matriz}{\maior}{\C}$ 
for each $\A, \C \in \alphabet$,
from
Proposition~\ref{AuxdistanciaNTriangular-1}. Let
$n_{0}$ be an integer satisfying Proposition~\ref{propAux}
and Fact~\ref{fato2}. Then, for $n \ge n_{0}$, it follows that
\begin{align*}
\min \left\{ 
\frac{\pont{\matriz}{\A}{\C}}{n+1},
\frac{\pont{\matriz}{\A}{\espaco} + \pont{\matriz}{\espaco}{\C}}{n+2}
\right\} &= \distanciaN{\matriz}(\maior^{n}\A, \maior^{n}\C) \\
 &\le \distanciaN{\matriz}(\maior^{n}\A, \maior^{n}\B) 
+ \distanciaN{\matriz}(\maior^{n}\B, \maior^{n}\C) \\
&\le \frac{\pont{\matriz}{\A}{\B}}{n+1} +
\frac{\pont{\matriz}{\B}{\C}}{n+1}\,.
\end{align*}
This implies that 
if $\distanciaN{\gamma}(q^n \A,q^n \C) = 
\pont{\gamma}{\A}{\C}/(n+1)$, then 
$\pont{\gamma}{\A}{\C} \le \pont{\gamma}{\A}{\B} + \pont{\gamma}{\B}{\C}$ and the proposition is proved. 
\begin{comment}


Consider now that $n_{0}$ also satisfies Fact~\ref{fato3}, i.e., that
\[
\pont{\matriz}{\A}{\espaco} + \pont{\matriz}{\espaco}{\C} \le
\pont{\matriz}{\A}{\B} + \pont{\matriz}{\B}{\C} \quad \text{if and
  only if} \quad \frac{\pont{\matriz}{\A}{\espaco} +
  \pont{\matriz}{\espaco}{\C}}{n+2} < \frac{\pont{\matriz}{\A}{\B} +
  \pont{\matriz}{\B}{\C}}{n+1}\,,
\]
for each $n \ge n_{0}$.

Suppose that
\[
\frac{\pont{\matriz}{\A}{\C}}{n+1} = \min \left\{
\frac{\pont{\matriz}{\A}{\C}}{n+1}, \frac{\pont{\matriz}{\A}{\espaco}
  + \pont{\matriz}{\espaco}{\C}}{n+2} \right\} \le
\frac{\pont{\matriz}{\A}{\B}}{n+1} +
\frac{\pont{\matriz}{\B}{\C}}{n+1}\,.
\]
In this case, since $n > 0$, we have that $\pont{\matriz}{\A}{\C} \le
\pont{\matriz}{\A}{\B} + \pont{\matriz}{\B}{\C}$ and the proof is
done.
\end{comment}
Then, assume that
\[
\frac{\pont{\matriz}{\A}{\espaco} + \pont{\matriz}{\espaco}{\C}}{n+2}
= \min \left\{ \frac{\pont{\matriz}{\A}{\C}}{n+1},
\frac{\pont{\matriz}{\A}{\espaco} + \pont{\matriz}{\espaco}{\C}}{n+2}
\right\} \le \frac{\pont{\matriz}{\A}{\B}}{n+1} +
\frac{\pont{\matriz}{\B}{\C}}{n+1}\,.
\]
It follows from contrapositive of Fact~\ref{fato2} that 
$\pont{\gamma}{\A}{\espaco} + \pont{\gamma}{\espaco}{\A} \le \pont{\gamma}{\A}{\B}+\pont{\gamma}{\B}{\C}$ since $n$ is big enough and the proposition is proved.


\begin{comment}
From Fact~\ref{fato3}, we have that $\pont{\matriz}{\A}{\espaco} +
\pont{\matriz}{\espaco}{\C} \le \pont{\matriz}{\A}{\B} +
\pont{\matriz}{\B}{\C}$ and the proof is also done.

Finally, suppose that 
\[
\frac{\pont{\matriz}{\A}{\C}}{n+1} \not\le
\frac{\pont{\matriz}{\A}{\B}}{n+1} +
\frac{\pont{\matriz}{\B}{\C}}{n+1} \quad \text{and} \quad
\frac{\pont{\matriz}{\A}{\espaco} + \pont{\matriz}{\espaco}{\C}}{n+2}
\not< \frac{\pont{\matriz}{\A}{\B}}{n+1} +
\frac{\pont{\matriz}{\B}{\C}}{n+1}\,,
\]
for each $n \ge n_{0}$. In this case we have that
\[
\frac{\pont{\matriz}{\A}{\espaco} + \pont{\matriz}{\espaco}{\C}}{n+2}
= \min \left\{ \frac{\pont{\matriz}{\A}{\C}}{n+1},
\frac{\pont{\matriz}{\A}{\espaco} + \pont{\matriz}{\espaco}{\C}}{n+2}
\right\} = \frac{\pont{\matriz}{\A}{\B}}{n+1} +
\frac{\pont{\matriz}{\B}{\C}}{n+1}\,,
\]
for each $n \ge n_{0}$. It implies that $\pont{\matriz}{\A}{\espaco} =
\pont{\matriz}{\espaco}{\C} = \pont{\matriz}{\A}{\B} =
\pont{\matriz}{\B}{\C}=0$ and, thus, it is also true that
$\pont{\matriz}{\A}{\espaco} + \pont{\matriz}{\espaco}{\C} =
\pont{\matriz}{\A}{\B}+ \pont{\matriz}{\B}{\C}$.
\end{comment}
\end{proof}

\begin{lemma}\label{distanciaNTriangular}\rm
Let $\distanciaN{\matriz} \in \prametrica$. Then,
$\distanciaN{\matriz}(s, t) \le \distanciaN{\matriz} (s, u) +
\distanciaN{\matriz} (u, t)$ for each $s, t, u \in \alphabet^{*}$ if
and only if
\begin{enumerate}
\item [(\textit{i})] $\pont{\matriz}{\A}{\espaco} \le
  \pont{\matriz}{\A}{\B} + \pont{\matriz}{\B}{\espaco}$\,,
\item [(\textit{ii})] $\pont{\matriz}{\espaco}{\A} \le
  \pont{\matriz}{\espaco}{\B} + \pont{\matriz}{\B}{\A}$\,,
\item [(\textit{iii})] $\min \{ \pont{\matriz}{\A}{\C},
  \pont{\matriz}{\A}{\espaco} + \pont{\matriz}{\espaco}{\C} \} \le
  \pont{\matriz}{\A}{\B} + \pont{\matriz}{\B}{\C}$\,, and
\item [(\textit{iv})] $\max \{ \pont{\matriz}{\A}{\espaco},
  \pont{\matriz}{\espaco}{\A} \} \le \pont{\matriz}{\B}{\espaco} +
  \pont{\matriz}{\espaco}{\B}$\,,
\end{enumerate}
for each $\A, \B, \C \in \alphabet$.
\end{lemma}
\begin{proof}
Suppose that $\distanciaN{\matriz}(s, t) \le
\distanciaN{\matriz}(s, u) + \distanciaN{\matriz}(u, t)$ for all $s,
t, u \in \alphabet^{*}$. It follows from
Propositions~\ref{AuxdistanciaNTriangular-1},
\ref{AuxdistanciaNTriangular}, and~\ref{AuxdistanciaNTriangular-2}
that conditions (\textit{i}) to (\textit{iv})
are true.

Conversely, suppose that conditions (\textit{i}) to
(\textit{iv}) are true. Let $s, t, u \in \alphabet^{*}$ and $A, B$ be
N-optimal alignments of $s, u$ and $u, t$, respectively. It follows
from Proposition~\ref{desigTriangularGeral} that there exists an
alignment $C$ of $s, t$ and an integer $k \ge 0$ such that
\[
\custoA{\matriz}[A] + \custoA{\matriz}[B] \ge \custoA{\matriz}[C] + k
\Maior\,, \quad \tamanho{A} \le \tamanho{C} + k\,, \quad \text{and} \quad
\tamanho{B} \le \tamanho{C} + k\,.
\]
As a consequence of $\distanciaN{\matriz} \in \prametrica$, we have
$\custoA{\matriz}[A], \custoA{\matriz}[B] \ge 0$. It follows that
\[
\distanciaN{\matriz} (s, u) + \distanciaN{\matriz}(u,t) =
\frac{\custoA{\matriz}[A]}{\tamanho{A}} +
\frac{\custoA{\matriz}[B]}{\tamanho{B}} \ge \frac{\custoA{\matriz}[C]
  + k \Maior}{\tamanho{C} + k}\,.
\]
Since $\abs{C} >0$ and $k \ge 0$, we have from Fact~\ref{fato1} that
%\begin{comment}
\begin{align*}
    \frac{\custoA{\matriz}[C]
  + k \Maior}{\tamanho{C} + k}  & \ge \min \left\{ \mathcal{Q}, \frac{\custoA{\matriz}[C]}{\tamanho{C}} = \custoN{\gamma}[C]
  \right\}\,,
\end{align*}
%\end{comment}
$\mathcal{Q} \ge \distanciaN{\gamma}(s, t)$ by Proposition~\ref{propAux2},
and $\custoN{\gamma}[C] \ge \distanciaN{\gamma}(s, t)$. Hence, 
\begin{comment}
Consequently, to prove the lemma, it is enough to prove that
\[
\frac{\custoA{\matriz}[C] + k \Maior}{\tamanho{C} + k} \ge
\distanciaN{\matriz}(s, u)\,.
\]

We have now two possibilities: 
\begin{align*}
(\custoA{\matriz}[C] + k \Maior)/(\tamanho{C} + k) &\ge 
\custoA{\matriz}[C] / \tamanho{C}\,, \text{or} \\
(\custoA{\matriz}[C] + k \Maior)/(\tamanho{C} + k) &< 
\custoA{\matriz}[C] / \tamanho{C}\,.
\end{align*}

If $(\custoA{\matriz}[C] + k \Maior)/(\tamanho{C} + k) \ge
\custoA{\matriz}[C] / \tamanho{C}$ then, since $C$ is an alignment of
$s, u$, we have that
\[
\frac{\custoA{\matriz}[C] + k \Maior}{\tamanho{C} + k} \ge
\frac{\custoA{\matriz}[C]}{\tamanho{C}} \ge \distanciaN{\matriz}(s, u)\,,
\]
and the proof is done.

On the other hand, if $(\custoA{\matriz}[C] + k \Maior)/(\tamanho{C} +
k) < \custoA{\matriz}[C] / \tamanho{C}$, since $k \ge 0$ and
$\tamanho{C} \ge 0$, we have that $\custoA{\matriz}[C] > \tamanho{C}
\Maior$. Since, $\distanciaN{\matriz}(s, u) \le \Maior$ from
Proposition~\ref{propAux2}, it follows that
\[
\frac{\custoA{\matriz}[C] + k \Maior}{\tamanho{C} + k} >
\frac{ \tamanho{C} \Maior + k \Maior}{\tamanho{C} + k} =
\Maior \ge \distanciaN{\matriz}(s, u)\,.
\]
\end{comment}
\[
\distanciaN{\matriz} (s, t) \le \distanciaN{\matriz}(s,u) +
\distanciaN{\gamma}(u, t)\,.
\]
\end{proof}

Finally, using the results presented in this section, we can establish the following proof of the highlighted theorem.

\begin{proof} (of Theorem~\ref{theo:norm})

Suppose that $\distanciaN{\matriz} \in \metrica$. Consequently, for each $s, t, u \in \alphabet^{*}$, we have that $\distanciaN{\matriz}(s, s) = 0$,
$\distanciaN{\matriz}(s, t) > 0$ if $s \not= t$, $\distanciaN{\matriz}(s, t) =
\distanciaN{\matriz}(t, s)$, and $\distanciaN{\matriz}(s, u) \le
\distanciaN{\matriz}(s, t) + \distanciaN{\matriz}(t, u)$.

Let $\A, \B, \C \in \alphabet$, $\A \neq \B \neq \C$. Since
$\distanciaN{\matriz}(s, s) = 0$ and $\distanciaN{\matriz}(s, t) > 0$,
we have that $\distanciaN{\matriz} \in \prametrica$. Since
$\distanciaN{\matriz}(s, t) > 0$
if $s \not= t$ and $\distanciaN{\matriz} \in
\prametrica$, from Lemma~\ref{lema-maiorQueZero}, we have that
$\pont{\matriz}{\A}{\espaco}, \pont{\matriz}{\espaco}{\A},
\pont{\matriz}{\A}{\B} > 0$. Since $\distanciaN{\matriz} \in
\prametrica$, it follows from Corollary~\ref{corolaryPrametrica} that
$\pont{\matriz}{\A}{\A} = 0$ or $\pont{\matriz}{\A}{\espaco} +
\pont{\matriz}{\espaco}{\A} = 0$ and, since
$\pont{\matriz}{\A}{\espaco}, \pont{\matriz}{\espaco}{\A} > 0$, we
have that $\pont{\matriz}{\A}{\A} = 0$. Since $\distanciaN{\matriz}(s,
t) = \distanciaN{\matriz}(t, s)$ for each $s, t \in \alphabet^{*}$ and
$\distanciaN{\matriz} \in \prametrica$, from
Lemma~\ref{distanciaNSimetrica} we have that
$\pont{\matriz}{\A}{\espaco} = \pont{\matriz}{\espaco}{\A}$, and if
$\pont{\matriz}{\A}{\B} < \pont{\matriz}{\A}{\espaco}+
\pont{\matriz}{\espaco}{\B}$, then $\pont{\matriz}{\A}{\B} =
\pont{\matriz}{\B}{\A}$. Since $\distanciaN{\matriz}(s, u) \le
\distanciaN{\matriz}(s, t) + \distanciaN{\matriz}(t, u)$ and
$\distanciaN{\matriz} \in \prametrica$, from
Lemma~\ref{distanciaNTriangular} we have that
$\pont{\matriz}{\A}{\espaco} \le \pont{\matriz}{\A}{\B} +
\pont{\matriz}{\B}{\espaco}$ and $\min \{\pont{\matriz}{\A}{\C},
\pont{\matriz}{\A}{\espaco} + \pont{\matriz}{\espaco}{\C} \} \le
\pont{\matriz}{\A}{\B} + \pont{\matriz}{\B}{\C}$.

From the observations above, we have that if $\matriz \in \metricaN$,
then $\matriz \in \metricaA$. Besides that, since
$\pont{\matriz}{\A}{\espaco} = \pont{\matriz}{\espaco}{\A}$,
$\pont{\matriz}{\B}{\espaco} = \pont{\matriz}{\espaco}{\B}$ and $\max
\{ \pont{\matriz}{\A}{\espaco}, \pont{\matriz}{\espaco}{\A} \} \le
\pont{\matriz}{\B}{\espaco} + \pont{\matriz}{\espaco}{\B}$ from
Lemma~\ref{distanciaNTriangular}, we have that
\[
\pont{\matriz}{\A}{\espaco} = \max \{ \pont{\matriz}{\A}{\espaco},
\pont{\matriz}{\espaco}{\A} \} \le \pont{\matriz}{\B}{\espaco} +
\pont{\matriz}{\espaco}{\B} = \pont{\matriz}{\B}{\espaco}
+\pont{\matriz}{\B}{\espaco} = 2\,\pont{\matriz}{\B}{\espaco}\,.
\]

Conversely, suppose that $\matriz \in \metricaN$. By the definition of
$\metricaN$, we have that $\matriz \in \metricaA$ and
$\pont{\matriz}{\A}{\espaco} \le 2 \, \pont{\matriz}{\B}{\espaco}$
for each $\A, \B \in \alphabet$. Since $\matriz \in \metricaA$, we
have that $\distanciaA{\matriz} \in \metrica \subseteq \prametrica$
and $\pont{\matriz}{\A}{\espaco} = \pont{\matriz}{\espaco}{\A} > 0$;
$\pont{\matriz}{\A}{\B} > 0$ if $\A \not= \B$, and
$\pont{\matriz}{\A}{\B} = 0$ if $\A = \B$; if $\pont{\matriz}{\A}{\B}
< \pont{\matriz}{\A}{\espaco} + \pont {\matriz}{\espaco}{\B}$, then
$\pont{\matriz}{\A}{\B} = \pont{\matriz}{\B}{\A}$;
$\pont{\matriz}{\A}{\espaco} \le \pont{\matriz}{\A}{\B} +
\pont{\matriz}{\B}{\espaco}$; $\min \{ \pont{\matriz}{\A}{\C},
\pont{\matriz}{\A}{\espaco} + \pont {\matriz}{\espaco}{\C} \} \le
\pont{\matriz}{\A}{\B} + \pont{\matriz}{\B}{\C}$ for each $\A, \B, \C
\in \alphabet$.  In order to prove that $\distanciaN{\matriz} \in
\metrica$, we show, for each $s, t, u \in \alphabet^{*}$, that
$\distanciaN{\matriz}(s, s) = 0$, $\distanciaN{\matriz}(s, t) > 0$ for
$s \not= t$, $\distanciaN{\matriz}(s, t) =\distanciaN{\matriz}(t, s)$,
and $\distanciaN{\matriz}(s, u) \le \distanciaN{\matriz}(s, t) +
\distanciaN{\matriz}(t, u)$.

Since $\distanciaA{\matriz} \in \prametrica$, we have from
Lemma~\ref{lemaPrametricaN} that $\distanciaN{\matriz} \in
\prametrica$ and, thus, $\distanciaN{\matriz}(s, s) = 0$. Since
$\distanciaN{\matriz} \in \prametrica$, $\pont{\matriz}{\A}{\espaco} =
\pont{\matriz}{\espaco}{\A} > 0$, and $\pont{\matriz}{\A}{\B} > 0$ if
$\A \not= \B$, consequently from Lemma~\ref{lema-maiorQueZero} we have
that $\distanciaN{\matriz}(s, t) > 0$ for $s \not= t$. Since
$\distanciaN{\matriz} \in \prametrica$, $\pont{\matriz}{\A}{\espaco} =
\pont{\matriz}{\espaco}{\A}$, and 
$\pont{\matriz}{\A}{\B} = \pont{\matriz}{\B}{\A}$ if $\pont{\matriz}{\A}{\B} <
\pont{\matriz}{\A}{\espaco} + \pont {\matriz}{\espaco}{\B}$, we have from
Lemma~\ref{distanciaNSimetrica} that $\distanciaN{\matriz}(s,
t) =\distanciaN{\matriz}(t, s)$.

Since $\pont{\matriz}{\B}{\A} < \pont{\matriz}{\B}{\espaco} +
\pont{\matriz}{\espaco}{\A}$, we have that
$\pont{\matriz}{\B}{\A} = \pont{\matriz}{\A}{\B}$ whereas $\matriz \in \metricaA$.
Since $\pont{\matriz}{\A}{\espaco} = \pont{\matriz}{\espaco}{\A}$,
$\pont{\matriz}{\B}{\espaco} = \pont{\matriz}{\espaco}{\B}$, and
$\pont{\matriz}{\A}{\espaco} \le \pont{\matriz}{\A}{\B} +
\pont{\matriz}{\B}{\espaco}$, we have that
\[
\pont{\matriz}{\espaco}{\A} = \pont{\matriz}{\A}{\espaco} \le 
\pont{\matriz}{\A}{\B} +
\pont{\matriz}{\B}{\espaco} = \pont{\matriz}{\espaco}{\B} + 
\pont{\matriz}{\B}{\A}\,.
\]
Since $\pont{\matriz}{\A}{\espaco} = \pont{\matriz}{\espaco}{\A}$ and
$\pont{\matriz}{\B}{\espaco} = \pont{\matriz}{\espaco}{\B}$, it follows
from hypothesis that
\[
\max \{ \pont{\matriz}{\A}{\espaco}, \pont{\matriz}{\espaco}{\A} \} =
\pont{\matriz}{\A}{\espaco} \le 2\,\pont{\matriz}{\B}{\espaco} =
\pont{\matriz}{\B}{\espaco} + \pont{\matriz}{\espaco}{\B}\,.
\]

Since $\pont{\matriz}{\A}{\espaco} \le \pont{\matriz}{\A}{\B} +
\pont{\matriz}{\B}{\espaco}$, $\pont{\matriz}{\espaco}{\A} \le
\pont{\matriz}{\espaco}{\B} + \pont{\matriz}{\B}{\A}$, $\min \{
\pont{\matriz}{\A}{\C}, \pont{\matriz}{\A}{\espaco} +
\pont{\matriz}{\espaco}{\C} \} \le \pont{\matriz}{\A}{\B} +
\pont{\matriz}{\B}{\C},$ and $\max \{ \pont{\matriz}{\A}{\espaco},
\pont{\matriz}{\espaco}{\A} \} \le \pont{\matriz}{\B}{\espaco} +
\pont{\matriz}{\espaco}{\B}$, we have that $\distanciaN{\matriz}(s, u)
\le \distanciaN{\matriz}(s, t) + \distanciaN{\matriz}(t, u)$ from
Lemma~\ref{distanciaNTriangular}.
\end{proof}

Table~\ref{tabela3} summarizes properties of scoring matrices $\gamma$ for metric space and some other generalized metric space. 

\begin{table}[htpb]
\begin{minipage}{\textwidth}
\begin{center}
\begin{tabular}{clcccccc}
  & & $\prametrica$ & $\semimetrica$ & $\hemimetrica$ &
  $\pseudometrica$ & $\quasimetrica$ & $\metrica$ \\ \hline & & \\
  (a) & $\distanciaA{\gamma}$ is a premetric & \yes & \yes & \yes & \yes
  & \yes & \yes \\ & & \\
  (b) & $\pont{\matriz}{\A}{\espaco}, \pont{\matriz}{\espaco}{\A} > 0$
  and $\pont{\matriz}{\A}{\B} > 0$ if $\A \not= \B$ & & \yes & & &\yes
  & \yes \\& & \\
  (c) & $\pont{\matriz}{\A}{\espaco} = \pont{\matriz}{\espaco}{\A}$ &
  & \yes & & \yes & & \yes \\ & & \\
  (d) &
  \begin{tabular}{l}
    if $\pont{\matriz}{\A}{\B} < \pont{\matriz}{\A}{\espaco}
    + \pont{\matriz}{\espaco}{\B}$ then \\
    $\pont{\matriz}{\A}{\B} = \pont{\matriz}{\B}{\A}$
  \end{tabular}  
  & & \yes & & \yes & & \yes \\ & & \\
  (e) & $\pont{\matriz}{\A}{\espaco} \le \pont{\matriz}{\A}{\B} +
  \pont{\matriz}{\B}{\espaco}$ & & & \yes & \yes &\yes & \yes \\ & &
  \\
  (f) & $\pont{\matriz}{\espaco}{\A} \le \pont{\matriz}{\espaco}{\B} +
  \pont{\matriz}{\B}{\A}$ & & & \yes & \yes &\yes & \yes\\ & & \\
  (g) & 
  $\min \Big\{ \begin{array}{l}
    \pont{\matriz}{\A}{\C}, \\
    \pont{\matriz}{\A}{\espaco} + \pont{\matriz}{\espaco}{\C} 
  \end{array} \Big\} 
  \le \pont{\matriz}{\A}{\B} + \pont{\matriz}{\B}{\C}$
  &  &  & \yes & \yes &\yes & \yes\\ & & \\
  (h) &   $\max \{ \pont{\gamma}{\A}{\espaco}, \pont{\gamma}{\espaco}{\A} \}
  \le \pont{\matriz}{\B}{\espaco} + \pont{\matriz}{\espaco}{\B}$ & & & \yes & \yes &\yes & \yes
\end{tabular}
\end{center}
\end{minipage}
\caption{Necessary and sufficient conditions for scoring matrix
  $\matriz$ to induce $\distanciaN{\matriz}\text{-}p$ on sequences when $\distanciaN{\matriz} \in 
  \prametrica$.
  As in Table~\ref{tabela2},  
  the properties are also used to define metric ($\metrica$) and 
  generalized metric spaces
  such as \emph{premetric} ($\prametrica$),
  \emph{semimetric} ($\semimetrica$), \emph{hemimetric}
  ($\hemimetrica$), \emph{pseudometric} ($\pseudometrica$) and 
  \emph{quasimetric} ($\quasimetrica$).
  Results are obtained using
  definitions presented in Section~\ref{sec:preliminares} and
  lemmas in Section~\ref{sec:normalizado}.} \label{tabela3}
\end{table}



\section{Extended alignment of two sequences}\label{sec:estendido}


We describe in this section the classes of scoring matrices that induce $\distanciaE{\matriz}$-$p$ on sequences for each axiom $p$ of a metric. Lemmas~\ref{lema-ZeroE}--\ref{desigualdadeE} establish these properties, which are summarized in Table~\ref{tabela4}, and allow us to characterize matrices that induces each of the more general metric functions described in Section~\ref{sec:preliminares}. Lastly, we present the following important result previously stated in Section~\ref{sec:preliminares}:

\begin{theorem} \label{theo:extend} \rm
$\distanciaE{\matriz} \in \metrica$ if and only if $\matriz \in \metricaE$.
\end{theorem}

To prove this result, we proceed as in the previous sections and present some intermediary results as follows.

\begin{fact}\label{fato-permuta}\rm
Each weighted directed multigraph obtained by arcs that represent edit operations that transform a sequence into itself is Eulerian.
\begin{comment}
Let $s \in \alphabet^{*}$ and $\alinhamento{c_{1}, \ldots, c_{n}}$ be
an extended alignment of $s, s$. There exists a permutation $(j_{1},
\ldots, j_{n})$ of $(1, \ldots, n)$ such that $c_{j_{1}}(m_{j_{1}}) =
c_{j_{2}}(1)$, \ldots, $c_{j_{n-1}}(m_{j_{n-1}})= c_{j_{n}}(1)$,
$c_{j_{n}}(m_{j_{n}})= c_{j_{1}}(1)$, which implies that if $C$ is the
concatenation of $c_{j_{1}}, \ldots, c_{j_{n}}$, then $C$ is a cycle
in $D(\matriz)$ and $\cost(C) = \custoE{\matriz}[A]$.
\end{comment}
\end{fact}

\begin{lemma}\label{lema-ZeroE}\rm
Let $s \in \alphabet^{*}$ and $\matriz$ a scoring matrix. Then,
$\distanciaE{\matriz}(s, s) = 0$ if and only if $D(\matriz)$ has no
negative cycle.
\end{lemma}


\begin{proof}
Suppose that $\distanciaE{\matriz}(s, s) = 0$ and, by contradiction, 
$D(\matriz)$ has a cycle $W = x_0, \ldots, x_m$ with $x_m \not= \espaco$ and $\cost(W) = -X < 0$. 
Let $n$ be an integer such that 
$\pont{\matriz}{s(1)}{x_0} +
\pont{\matriz}{x_0}{s(1)} - nX < 0$.
Therefore, $A = [c, s(2), s(3), \ldots, s(\abs{s})]$,
where the column $c = s(1) x_0 (x_1, \ldots x_m)^n s(1)$, is also an extended alignment of 
$s, s$ and $\custoE{\gamma}[A] < 0 = \distanciaE{\matriz}(s, s)$,
which is a contradiction.
It follows that $D(\matriz)$ has no negative cycle.


On the other hand, suppose now that $D(\matriz)$ has no negative
cycle. Since $\custoE{\matriz}[s(1), s(2), \ldots, s(\abs{s})] = 0$, 
we have that $\distanciaE{\matriz}(s, s) \le 0$.

Let $A$ be an E-optimal alignment of $s, s$. 
From Fact~\ref{fato-permuta}, the multigraph $H$ obtained by considering
the edit operations that transform $s$ into itself is Eulerian, 
which implies that $H$ can be decomposed into cycles where each cycle is
also a cycle in $D(\gamma)$. 
Since $D(\matriz)$ has no negative cycle, we have that 
$\distanciaE{\matriz}(s,s) = \custoE{\matriz}[A] \ge 0$.

Consequently, since $\distanciaE{\matriz}(s, s) \le 0$ and
$\distanciaE{\matriz}(s, s) \ge 0$, we have that if $D(\matriz)$ has no negative cycle then $\distanciaE{\matriz}(s, s) = 0$.
\end{proof}

\begin{lemma}\label{lema-extMIZero}\rm
Let $s, t \in \alphabet^{*}$. Then, we have that
$\distanciaE{\matriz}(s, t) \ge 0$ if and only if
$\pont{\matriz}{\A}{\espaco}, \pont{\matriz}{\espaco}{\A},
\pont{\matriz}{\A}{\B} \ge 0$ for each $\A, \B \in \alphabet$.
\end{lemma}
\begin{proof}
Suppose that $\distanciaE{\matriz}(s, t) \ge 0$ for each $s, t \in
\alphabet^{*}$ and $\A, \B \in \alphabet$. Then,
$\pont{\matriz}{\A}{\espaco} = \custoE{\matriz} \alinhamento{
  \A, \espaco } \ge \distanciaE{\matriz}(\A, \seqVazia) \ge 0$.
Similarly, $\pont{\matriz}{\espaco}{\A}, \pont{\matriz}{\A}{\B} \ge
0$.

Conversely, suppose that $\pont{\matriz}{\A}{\espaco},
\pont{\matriz}{\espaco}{\A}, \pont{\matriz}{\A}{\B} \ge 0$ for each
$\A, \B \in \alphabet$. Thus, the sum of weights of any sequence of edit operations that transforms $s$ into $t$ is non negative. 
Therefore, $\distanciaE{\matriz}(s, t) \ge 0$.
\end{proof}

\begin{lemma}\label{lema-maiorZeroE}\rm
Let $s \not= t \in \alphabet^{*}$.  Then, $\distanciaE{\matriz}(s, t)
> 0$ if and only if
\begin{enumerate}
\item[(\textit{i})] $\pont{\matriz}{\A}{\A} \ge 0$\,, and
\item[(\textit{ii})] $\pont{\matriz}{\A}{\espaco},
  \pont{\matriz}{\espaco}{\A}, \pont{\matriz}{\A}{\B} > 0$\,,
\end{enumerate}
for each $\A \not= \B \in \alphabet$.
\end{lemma}
\begin{proof}
Suppose that $\distanciaE{\matriz} (s, t) > 0$. Then,
$\pont{\matriz}{\A}{\espaco} = \custoE{\matriz} \alinhamento{\A,
  \espaco} \ge \distanciaA{\matriz}(\A, \seqVazia) > 0$.
Similarly, $\pont{\matriz}{\espaco}{\A}, \pont{\matriz}{\A}{\B} > 0$.
Moreover, it follows from Lemma~\ref{lema-extMIZero} that
$\pont{\matriz}{\A}{\A} \ge 0$.

Conversely, suppose that (\textit{i}) and (\textit{ii}) are true. 
Since each edit operation that transforms $s$ into $t$ has
non negative weight and, because $s \not= t$, there exists at least
one edit operation with positive weight, we have that 
the sum of weights of any sequence of edit operations that transforms $s$ into $t$ is positive. 
Therefore, $\distanciaE{\matriz}(s, t) > 0$.
\end{proof}

\begin{lemma}\label{lema-simetriaE}\rm
Let $\matriz$ be a scoring matrix. Then, $\distanciaE{\matriz}(s, t)
= \distanciaE{\matriz}(t, s)$ for each $s, t \in \alphabet^{*}$ if
and only if $d_{\matriz}(\A, \B) = d_{\matriz}(\B, \A)$ and $d_{\matriz}(\A, \espaco) = d_{\matriz}(\espaco,
\A)$ for each $\A, \B \in \alphabet$.
\end{lemma}

\begin{proof}
Let $\A, \B \in \alphabet$. Suppose that $\distanciaE{\matriz}(s, t) =
\distanciaE{\matriz}(t, s)$ for each $s, t \in \alphabet^{*}$. Any
extended alignment of $\A, \seqVazia$ must have a single column.  Let
$\alinhamento{c}$ be an E-optimal alignment of $\A, \seqVazia$.  Then,
$c$ is a walk of minimum weight from $\A$ to $\espaco$ in
$D(\matriz)$. Hence, $d(\A, \espaco) =
\custoE{\matriz}\alinhamento{c} = \distanciaE{\matriz}(\A,
\seqVazia)$. Similarly, $\distanciaE{\matriz}(\seqVazia, \A) =
d(\espaco, \A)$. Therefore,
\[
d(\A, \espaco) = \distanciaE{\matriz}(\A, \seqVazia) =
\distanciaE{\matriz}(\seqVazia, \A) = d (\espaco, \A)\,.
\]

An E-optimal alignment of $\A, \B$ has either one or two columns. If
it has only one column $c$, we have that $d(\A, \B) = \distanciaE{\matriz}(\A, \B)$ as
stated in the previous paragraph. If the E-optimal alignment of $\A,
\B$ has two columns $c_{1}, c_{2}$, and $c$ is the walk of minimum
weight from $\A$ to $\B$, then we have, by the optimality of the
alignment, that
\[
\distanciaE{\matriz}(\A, \B) = \custoE{\matriz}\alinhamento{c_{1},
  c_{2}} \le \custoE{\matriz}\alinhamento{c} = d(\A, \B)\,.
\]
In the E-optimal alignment $\alinhamento{c_{1}, c_{2}}$, one of the
columns, say $c_{1}$, ends with $\espaco$ and $c_{2}$ begins with
$\espaco$.
Therefore, by concatenating the two columns, we obtain an extended
alignment $\alinhamento{c' = c_{1}(1)(=\A) \cdots c_{1}(m_{1}-1)
  \espaco c_{2}(2) \cdots c_{2}(m_{2}) (= \B)}$. This alignment has
only one column such that $\custoE{\matriz}\alinhamento{c_{1},
  c_{2}} = \custoE{\matriz}[c']$.  Since $c'$ is a walk from $\A$ to
$\B$, it follows that
\[
d(\A, \B) \le \custoE{\matriz}
\alinhamento{c'} = \custoE{\matriz}\alinhamento{c_{1}, c_{2}} = 
\distanciaE{\matriz}(\A, \B)\,.
\]
Thus, $d(\A, \B) = \distanciaE{\matriz}(\A, \B)$
also in this case. Similarly, $d(\B,
\A) = \distanciaE{\matriz}(\B, \A)$, which allows us to conclude,
since $\distanciaE{\matriz}(\A, \B) = \distanciaE{\matriz}(\B, \A)$,
that
\[
d(\A, \B) = \distanciaE{\matriz}(\A, \B) = 
\distanciaE{\matriz}(\A, \B) = \distanciaE{\matriz}(\B, \A)\,.
\]

Suppose now that $d(\A, \B) = d(\B, \A)$ for each $\A, \B \in
\alphabet_{\espaco}^{*}$. Let $\alinhamento{c_{1}, \ldots, c_{n}}$ be
an E-optimal alignment of $s, t$. Clearly,
$\alinhamento{c'_{1}, \ldots, c'_{n}}$ such that each $c'_{i}$ is a
walk of minimum weight from $c_{i}(m_{i})$ to $c_{i}(1)$ is an alignment
of $t, s$ and
\begin{align*}
\distanciaE{\matriz}(s, t) &= \sum_{i} \custoE{\matriz}[c_{i}] =
\sum_{i} d(c_{i}(1), c_{i}(m_{i})) \\
&= \sum_{i} d(c_{i}(m_{i}), c_{i}(1)) = \custoE{\matriz}
\alinhamento{c'_{1}, \ldots, c'_{n}}\\
&\le \distanciaE{\matriz}(t, s)\,. 
\end{align*}
Using similar reasoning, we have that $\distanciaE{\matriz}(t, s) \le
\distanciaE{\matriz}(s, t)$, which allows us to conclude that
$\distanciaE{\matriz}(s, t) = \distanciaE{\matriz}(t, s)$.
\end{proof}

\begin{lemma}\label{desigualdadeE}\rm
$\distanciaE{\matriz}(s, u) \le \distanciaE{\matriz}(s, t) +
\distanciaE{\matriz}(t, u)$ for any $s, t, u \in \alphabet^{*}$.
\end{lemma}
\begin{proof}
Let $\alinhamento{c^{st}_{1}, \ldots, c^{st}_{n_{st}}}$ and
$\alinhamento{c^{tu}_{1}, \ldots, c^{tu}_{n_{tu}}}$ be E-optimal
alignments of $s, t$ and $t, u$, respectively. Consider the sets of numbers $I = \{
i_{1}, \ldots, i_{\tamanho{t}} \}$ and $J = \{ j_{1}, \ldots,
j_{\tamanho{t}} \}$ such that $i_{1} < i_{2} < \ldots <
i_{\tamanho{t}}$, $j_{1} < j_{2} < \cdots < j_{\tamanho{t}}$, and $t(k)
= c^{st}_{i_{k}}(m_{i_{k}}) = c^{tu}_{j_{k}}(1)$.

Let $A$ be an alignment of $s, u$ whose columns are defined according
to the following rules: the column $c^{st}_{k}$ is a column of
$A$ for each $k \not\in I$; $c^{tu}_{k}$ is
a column of $A$ for each $k \not\in J$; and we define the column
\[
c^{st}_{i_{k}}(1) \ c^{st}_{i_{k}}(2)\  \cdots\ c^{st}_{i_{k}}(m_{i_{k}} - 1)
 \ t(k)\
c^{tu}_{j_{k}}(2) \ c^{tu}_{j_{k}}(3)\ \cdots\ c^{tu}_{j_{k}}(m_{j_{k}})\,,
\]
if $c^{st}_{i_k}(1) \not= \espaco$ or 
$c^{tu}_{j_{k}}(m_{j_{k}}) \not= \espaco$ %\FM{$c^{tu}_{j_{k}}(m_{j_{k}}) \not= \espaco$?} 
for each $k = 1, \ldots, \tamanho{t}$.

Therefore, 
\begin{align*}
\distanciaE{\matriz}(s, u) &\le \custoE{\matriz}[A] \le
\custoE{\matriz}\alinhamento{c^{st}_{1}, \ldots, c^{st}_{n_{st}}} +
\custoE{\matriz}\alinhamento{c^{tu}_{1}, \ldots, c^{tu}_{n_{tu}}}\\
&= \distanciaE{\matriz}(s, t) + \distanciaE{\matriz}(t, u)\,. 
\end{align*}
\end{proof}

We are now able to present the proof of the main result of this section.

\begin{proof} (of Theorem~\ref{theo:extend})

Suppose first that $\distanciaE{\matriz} \in \metrica$. Then, for any
$s, t \in \alphabet^{*}, s \not= t$, we have that
$\distanciaE{\matriz}(s, s) = 0$, $\distanciaE{\matriz}(s, t) > 0$, and
$\distanciaE{\matriz}(s,t) = \distanciaE{\matriz}(t, s)$. 
From
Lemma~\ref{lema-maiorZeroE}, we have that $\pont{\matriz}{\A}{\A} \ge 0$ and $\pont{\matriz}{\A}{\B}, \pont{\matriz}{\A}{\espaco},
\pont{\matriz}{\espaco}{\A} > 0$ for each $\A \not= \B \in
\alphabet$ since $\distanciaE{\matriz}(s, t) > 0$. From Lemma~\ref{lema-simetriaE}, we have 
$d(\A, \B) = d(\B, \A)$ and $d(\A, \espaco) = d(\espaco, \A)$ for each
$\A, \B \in \alphabet$ since
$\distanciaE{\matriz}(s,t) = \distanciaE{\matriz}(t, s)$. It follows that
$\matriz \in \metricaE$.

Conversely, suppose that $\matriz \in \metricaE$. Then,
$\pont{\matriz}{\A}{\A} \ge 0$, $\pont{\matriz}{\A}{\B},
\pont{\matriz}{\A}{\espaco}, \pont{\matriz}{\espaco}{\A} > 0$, $d(\A,
\B) = d(\B, \A)$, and $d(\A, \espaco) = d(\espaco, \A)$ for each $\A
\not= \B$, $\A, \B \in \alphabet$. Since $\pont{\matriz}{\A}{\A} \ge
0$, $\pont{\matriz}{\A}{\B}, \pont{\matriz}{\A}{\espaco},
\pont{\matriz}{\espaco}{\A} > 0$ for each $\A \not= \B$, $\A, \B \in
\alphabet$, we have that $D(\matriz)$ has no negative cycle,
which implies, from Lemma~\ref{lema-ZeroE}, that
$\distanciaE{\matriz}(s, s) = 0$ for each $s \in \alphabet^{*}$.
Since $\pont{\matriz}{\A}{\A} \ge 0$, $\pont{\matriz}{\A}{\B},
\pont{\matriz}{\A}{\espaco}, \pont{\matriz}{\espaco}{\A} > 0$ we have,
from Lemma~\ref{lema-maiorZeroE}, that $\distanciaE{\matriz}(s, t) >
0$ for each $s, t \in \alphabet^{*}$, $s \not= t$. Since $d(\A, \B) =
d(\B, \A)$ and $d(\A, \espaco) = d(\espaco, \A)$ for each $\A \not=
\B$, $\A, \B \in \alphabet$, we have from Lemma~\ref{lema-simetriaE}
that $\distanciaE{\matriz}(s,t) = \distanciaE{\matriz}(t, s)$ for each
$s, t \in \alphabet^{*}$.  From Lemma~\ref{desigualdadeE}, we have
that $\distanciaE{\matriz}(s, u) \le \distanciaE{\matriz}(s, t) +
\distanciaE{\matriz}(t, u)$.  Therefore, $\distanciaE{\matriz} \in \metrica$.
\end{proof}

\begin{table}[htpb]
\begin{minipage}{\textwidth}
\begin{center}
\begin{tabular}{clcccccc}
  & & $\prametrica$ & $\semimetrica$ & $\hemimetrica$ &
  $\pseudometrica$ & $\quasimetrica$ & $\metrica$ \\ \hline & & \\
  (a) & $D(\matriz)$ has no negative cycle & \yes & \yes & \yes & \yes
  &\yes & \yes \\ & & \\
  (b) & $\pont{\matriz}{\A}{\espaco}, \pont{\matriz}{\espaco}{\B},
  \pont{\matriz}{\A}{\B} \ge 0$ & \yes & \yes & \yes & \yes &\yes &
  \yes \\ & & \\
  (c) & $\pont{\matriz}{\A}{\A} \ge 0$& & \yes & & &\yes & \yes \\ & &
  \\
  (d) & $\pont{\matriz}{\A}{\espaco}, \pont{\matriz}{\espaco}{\A} > 0$
  and $\pont{\matriz}{\A}{\B} > 0$ if $\A \not= \B$ & & \yes & & &\yes
  & \yes \\& & \\
  (e) & $d_{\matriz}( \A , \B) = d_{\matriz} (\B, \A)$ for each $\A, \B \in \Sigma_{\espaco}$ &
  & \yes & & \yes & & \yes \\ & & \\
\end{tabular}
\end{center}
\end{minipage}
\caption{Necessary and sufficient conditions for scoring matrix
  $\matriz$ to induce $\distanciaE{\gamma}\text{-}p$ on sequences where $p$ is each axiom of a metric. These properties are used to define metric spaces ($\metrica$) and generalizes metric spaces such as \emph{premetric} ($\prametrica$), \emph{semimetric} ($\semimetrica$), \emph{hemimetric}
  ($\hemimetrica$), \emph{pseudometric} ($\pseudometrica$), and
  \emph{quasimetric} ($\quasimetrica$). Results are obtained using
  definitions presented in Section~\ref{sec:preliminares} and lemmas in this section.} \label{tabela4}
\end{table}



\section{Conclusion}\label{sec:conclusion}
In this work, we focus on addressing the fundamental challenge of OOD detection tasks, which is how to fully understand the semantic discrepancy between the ID/OOD samples. We reveal that the key to success in the realistic SCOOD task is to allocate as many ID samples in the unlabeled set correctly as possible. To this end, we propose a novel uncertainty-aware optimal transport scheme that introduces class-specific energy scores as guidance for effective label assignment. Experimental results show that our method achieves better performance than previous state-of-the-art methods on SCOOD benchmarks.

\textbf{Limitations.} In addition to temperature scaling, other techniques such as feature clipping applied in ReAct~\cite{sun2021react} also enhance the performance of energy score, so how to obtain an OOD score that best fits the SCOOD task can be further explored. Moreover, a setting highly related to SCOOD has been proposed in \cite{katz2022training} and formulated as a constrained optimization problem. We will also theoretically analyze these practical OOD settings in our feature work.

% \section*{Acknowledgments}
\textbf{Acknowledgments.} 
This work is supported by National Key R\&D Program of China under Grant 2020AAA0105701, National Natural Science Foundation of China (NSFC) under Grants 61872327, Major Special Science and Technology Project of Anhui, National Natural Science Foundation of China (62033012) and Ant Group through Ant Research Intern Program.


%% The Appendices part is started with the command \appendix;
%% appendix sections are then done as normal sections
%% \appendix

%% \section{}
%% \label{}

%% If you have bibdatabase file and want bibtex to generate the
%% bibitems, please use
%%
\bibliographystyle{elsarticle-harv} 
\bibliography{references}

%% else use the following coding to input the bibitems directly in the
%% TeX file.

%% \begin{thebibliography}{00}

%% \bibitem[Author(year)]{label}
%% Text of bibliographic item

%% \bibitem[ ()]{}

% \input{newtables}

%% \end{thebibliography}
\end{document}

\endinput
%%
%% End of file `elsarticle-template-harv.tex'.
