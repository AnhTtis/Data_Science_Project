\section{Alignment distance}\label{sec:regular}


In this section, we describe the classes of scoring matrices that
induce $\distanciaA{\matriz}$-$p$ on sequences for each axiom $p$ of a metric. We do it through a sequence of results, such as Lemmas~\ref{distancia=0}, \ref{distanciaMaiorOuIgualAZero}, \ref{distanciaMaiorQue0}, \ref{simetriaNormal}, and   \ref{desigualdadeTriangularNormal}, which are summarized in Table~\ref{tabela2} and allow us to characterize matrices that induces each of the more general metric functions described in Section~\ref{sec:preliminares}. Moreover, as a consequence of that, we conclude this section with the important result previously stated in Section~\ref{sec:preliminares} and depicted by the following:

\begin{theorem} \label{theo:align} \rm
Let $\alphabet$ be an alphabet and $\matriz$ be a scoring matrix.
Then $\distanciaA{\matriz}$ is a metric on $\alphabet^{*}$ if and only
if 
\begin{enumerate}
\item[(\textit{i})] $\pont{\matriz}{\A}{\espaco} =
  \pont{\matriz}{\espaco}{\A} > 0$\,,
\item[(\textit{ii})] $\pont{\matriz}{\A}{\B} > 0$ if $\A \not= \B$ and
  $\pont{\matriz}{\A}{\B} = 0$ if $\A = \B$\,,
\item[(\textit{iii})] if $\pont{\matriz}{\A}{\B} <
  \pont{\matriz}{\A}{\espaco} + \pont{\matriz}{\espaco}{\B}$, then
  $\pont{\matriz}{\A}{\B} = \pont{\matriz}{\B}{\A}$\,,
\item[(\textit{iv})] $\pont{\matriz}{\A}{\espaco} \le
  \pont{\matriz}{\A}{\B} + \pont{\matriz}{\B}{\espaco}$\,, and
\item[$(v)$] $\min\{ \pont{\matriz}{\A}{\C},
  \pont{\matriz}{\A}{\espaco} + \pont{\matriz}{\espaco}{\C} \} \le
  \pont{\matriz}{\A}{\B} + \pont{\matriz}{\B}{\C}$\,,
\end{enumerate}
for each $\A, \B, \C \in \alphabet$.
\end{theorem}

We need a sequence of auxiliary results, as seen below, to eventually prove this theorem at the end of the section. 

\begin{fact}\label{fciclonegativo}\rm
Let $\matriz$ be a scoring matrix. If $D(\matriz)$ has a negative
cycle, then there exists $s \in \alphabet^{*}$ such that
$\distanciaA{\matriz}(s, s) < 0$.
\end{fact}

\begin{proof}
Suppose that $D (\matriz)$ has a cycle $C = \A_{0}, \ldots,
\A_{m}$, $\A_i \in \Sigma_{\espaco}$ such that
$\A_0 = \A_m \not= \espaco$,
$\A_{i} \not= \A_{j}$ for each
$i, j > 0$ and $\cost(C) = -X < 0$.
Let $n$ be a nonnegative integer such that $n >
(\pont{\matriz}{\A_{1}}{\espaco} +
\pont{\matriz}{\espaco}{\A_{1}})/X$ and 
$s = (\A_0 \A_1 \ldots \A_m)^n \A_0$.
Therefore,
\[
\distanciaA{\matriz}(s, s) \le \custoA{\matriz} \alinhamentoB{
  \begin{array}{ccc}
    \begin{array}{c}
      \espaco\\ \A_{0}
    \end{array} &
    \Big(
    \begin{array}{ccccc}
      \A_{0} & \A_{1} & \ldots & \A_{m - 1} & \A_{m} \\
      \A_{1} & \A_{2} & \ldots & \A_{m} & \A_{0} 
    \end{array} 
    \Big)^{n} &
    \begin{array}{c}
      \A_{0} \\ \espaco
    \end{array}\\
  \end{array}
}  <  0\,.
\]
\end{proof}

Observe that, for each $\A \in \alphabet$, we have $\custoA{\matriz}\alinhamento{\A,
  \espaco} = \pont{\matriz}{\A}{\espaco}$,
$\custoA{\matriz}\alinhamento{\espaco, \A} =
\pont{\matriz}{\espaco}{\A}$. Besides that, the only alignments of $\A, \seqVazia$
and $\seqVazia, \A$ are $\alinhamento{\A, \espaco}$ and
$\alinhamento{\espaco, \A}$, respectively. Thus, we have that
\begin{align}
  \distanciaA{\matriz}(\A, \seqVazia) &=
  \custoA{\matriz}\alinhamento{\A, \espaco} =
  \pont{\matriz}{\A}{\espaco}\,, \label{basica1}\\
  \distanciaA{\matriz}(\seqVazia, \A) &=
  \custoA{\matriz}\alinhamento{\espaco, \A} =
  \pont{\matriz}{\espaco}{\A}\,. \label{basica2}
\end{align}

Now, for each $\A, \B \in \alphabet$, since
\[
\cjtoAlinha{(\A, \B)} = \Big\{ \alinhamentoB{
  \begin{array}{c}
    \A\\
    \B
  \end{array}},
  \alinhamentoB{ 
    \begin{array}{cc}
      \A & \espaco\\
      \espaco & \B\\
    \end{array}}, 
  \alinhamentoB{
    \begin{array}{cc}
      \espaco & \A\\
      \B & \espaco\\
    \end{array}}
  \Big\}\,,
\]
$\custoA{\matriz}\alinhamento{\A,\B} = \pont{\matriz}{\A}{\B}$, and
$\custoA{\matriz}\alinhamento{\A\espaco,\espaco\B} =
\custoA{\matriz}\alinhamento{\espaco\A, \B\espaco} =
\pont{\matriz}{\A}{\espaco} + \pont{\matriz}{\espaco}{\B}$, we have
that
\begin{align}
  \distanciaA{\matriz}(\A,\B) &=  \min \bigg\{
  \begin{array}{rcl} 
    \custoA{\matriz}\alinhamento{\A, \B} & \!\!\! = \!\!\! &
    \pont{\matriz}{\A}{\B}\,, \\
    \custoA{\matriz}\alinhamento{\A\espaco, \espaco\B} & \!\!\! =
    \!\!\! & \pont{\matriz}{\A}{\espaco} +
    \pont{\matriz}{\espaco}{\B}\,.
  \end{array}
  \label{basica3}
\end{align}

\begin{lemma}\label{distancia=0}\rm
$\distanciaA{\matriz}(s, s) = 0$ for each $s \in \alphabet^{*}$
if and only if  
\begin{enumerate}
\item[(\textit{i})] $D(\matriz)$ has no negative cycle\,, and
\item[(\textit{ii})] $\pont{\matriz}{\A}{\A} = 0$ or
  $\pont{\matriz}{\A}{\espaco} + \pont{\matriz}{\espaco}{\A} = 0$, for
  each $\A \in \alphabet$.
\end{enumerate}
\end{lemma}
\begin{proof}
  Suppose that $\distanciaA{\matriz}(s, s) = 0$ for each $s \in \alphabet^{*}$.
  It follows from Fact~\ref{fciclonegativo} that (\textit{i}) is true;
  and from equation~(\ref{basica3}) that (\textit{ii}) is also true.

  Conversely, suppose that (\textit{i}) and (\textit{ii}) are true.
  Let $s = s(1) \cdots s(n) \in \alphabet^{*}$ and let $A$ be an
  A-optimal alignment of $s, s$. We define the weighted direct multigraph $H$ from
  alignment $A$ such as
  \begin{align*}
    V(H) &= \Sigma_{\,\espaco}\,, \\
    E(H) &=   \{k : [\A, \B]~\text{is aligned in $k$-th column of $A$} \}\,,\\
    \cost(\A, \B) &= \pont{\matriz}{\A}{\B}~\text{for each arc in $H$}.
\end{align*}

  By construction, $\custoA{\matriz}[A] = \cost(H)$.
  Notice that the indegree and outdegree of each vertex of $H$ are the same, which implies that 
  $H$ is an Eulerian graph
  % \citep{BT2006},
  and then $E$ can be decomposed into arc-disjoint cycles.
  For each cycle of this decomposition, there
  exists a cycle in $D(\matriz)$ with the same weight. Since, by
  hypothesis, $D(\matriz)$ has no negative cycle, it follows that
  $\cost(H) \ge 0$.

Therefore,
\begin{equation}
\distanciaA{\matriz}(s, s) = \custoA{\matriz} [A] = \cost(H) \ge 0\,.
\label{distancia=0eq1}
\end{equation}

Moreover, we are able to construct the following alignment $B$. Align
symbols $s(i), s(i)$ if $\pont{\matriz}{s(i)}{s(i)} = 0$, and align
$s(i), \espaco$ and $\espaco, s(i)$ otherwise. Since (\textit{ii})
is true, it follows that $\custoA{\matriz}[B] = 0$. Thus,
\begin{equation}
\distanciaA{\matriz}(s, s) \le \custoA{\matriz}[B] =
0\,. \label{distancia=0eq2}
\end{equation}

Using expressions~(\ref{distancia=0eq1}) and~(\ref{distancia=0eq2}),
and since the argument is used for any $s \in \alphabet^{*}$, we
conclude that $\distanciaA{\matriz}(s, s) = 0$ for each $s \in
\alphabet^{*}$.
\end{proof}

\begin{lemma}\label{distanciaMaiorOuIgualAZero}\rm
  $\distanciaA{\matriz}(s, t) \ge 0$ for each $s, t \in
  \alphabet^{*}$ if and only if for each $\A, \B \in \alphabet$,  $\pont{\matriz}{\A}{\espaco}, \pont{\matriz}{\espaco}{\B},
  \pont{\matriz}{\A}{\B} \ge 0$.
\end{lemma}
\begin{proof}
Consider $\distanciaA{\matriz}(s, t) \ge 0$ for each $s, t \in
\alphabet^{*}$. It follows from
equations~(\ref{basica1}),~(\ref{basica2}), and~(\ref{basica3}) that
$\pont{\matriz}{\A}{\espaco}  = \distanciaA{\matriz}(\A, \seqVazia) \ge 0,
\pont{\matriz}{\espaco}{\B} = \distanciaA{\matriz}(\seqVazia, \B) \ge 0$, and $\pont{\matriz}{\A}{\B} \ge \distanciaA{\matriz}(\A, \B) \ge 0$.

Conversely, consider $\pont{\matriz}{\A}{\espaco},
\pont{\matriz}{\espaco}{\B}, \pont{\matriz}{\A}{\B} \ge 0$, for each
$\A, \B \in \alphabet$. Let $A$ be an A-optimal alignment of $s,
t$. Since $\custoA{\matriz}[A]$ is the sum of entries of $\matriz$
that, by definition, are nonnegative, we have $\distanciaA{\matriz}(s,
t) = \custoA{\matriz}[A] \ge 0$.
\end{proof}

\begin{lemma}\label{distanciaMaiorQue0}\rm
  $\distanciaA{\matriz}(s, t) > 0$ for each $s \not= t \in
  \alphabet^{*}$ if and only if
  \begin{enumerate}
  \item[(\textit{i})] $\pont{\matriz}{\A}{\A} \ge 0$\,, and
  \item[(\textit{ii})] $\pont{\matriz}{\A}{\B},
    \pont{\matriz}{\A}{\espaco}, \pont{\matriz}{\espaco}{\B} > 0$\,,
  \end{enumerate}
  for each $\A \not= \B \in \alphabet$.
\end{lemma}
\begin{proof}
Consider first that $\distanciaA{\matriz}(s, t) > 0$ for each $s \not=
t \in \alphabet^{*}$. Since, for any $n \ge 0$, $\alinhamento{\A^{n +
    1}, \A^{n} \espaco}$ is an alignment of $(\A^{n + 1}, \A^{n})$
and, by hypothesis, $\distanciaA{\matriz}(\A^{n + 1}, \A^{n}) > 0$, we
have
\[
n \pont{\matriz}{\A}{\A} + \pont{\matriz}{\A}{\espaco} = 
\custoA{\matriz}  \alinhamentoB{
  \begin{array}{cc}
    \A^{n} & \A \\
    \A^{n} & \espaco
  \end{array} }
 \ge \distanciaA{\matriz}(\A^{n + 1}, \A^{n}) > 0\,.
\]
Since the expression above is valid for any $n$, this implies that
$\pont{\matriz}{\A}{\A} \ge 0$. Thus, (\textit{i}) is true. A similar
argument used in the first paragraph of proof of
Lemma~\ref{distanciaMaiorOuIgualAZero} shows that (\textit{ii}) is
also true.

Conversely, consider that (\textit{i}) and (\textit{ii}) are true.
Let $\alinhamento{s', t'}$ be an A-optimal alignment of $s, t$, $s
\not= t$. Since $s \not= t$, there exists $h$ such that $s'(h) \not=
t'(h)$, which implies by (\textit{ii}) that
$\pont{\matriz}{s'(h)}{t'(h)} > 0$. By (\textit{i}) and (\textit{ii}),
$\pont{\matriz}{s'(j)}{t'(j)} \ge 0$ for each $j \not= h$ and, by
hypothesis, $\distanciaA{\matriz}(s, t) =
\custoA{\matriz}\alinhamento{s', t'}$. Then, it follows that
\[
\distanciaA{\matriz}(s, t) = \custoA{\matriz}\alinhamento{s', t'} =
\pont{\matriz}{s'(h)}{t'(h)} + \sum_{j \not= h}
\pont{\matriz}{s'(j)}{t'(j)} 
\ge \pont{\matriz}{s'(h)}{t'(h)}
> 0\,.
\]
\end{proof}

\begin{lemma}\label{simetriaNormal}\rm
$\distanciaA{\matriz}(s, t) = \distanciaA{\matriz}(t, s)$ 
for each $s, t \in \alphabet^{*}$
if and only if
\begin{enumerate}
\item[(\textit{i})] $\pont{\matriz}{\A}{\espaco} =
  \pont{\matriz}{\espaco}{\A}$\,, and
\item[(\textit{ii})] if $\pont{\matriz}{\A}{\B} <
  \pont{\matriz}{\A}{\espaco} + \pont{\matriz}{\espaco}{\B}$, then
  $\pont{\matriz}{\A}{\B} = \pont{\matriz}{\B}{\A}$\,,
\end{enumerate}
for each $\A, \B \in \alphabet$.
\end{lemma}
\begin{proof}
Consider that $\distanciaA{\matriz}(s, t) = \distanciaA{\matriz}(t,
s)$ for each $s, t \in \alphabet^{*}$. It follows as a consequence
of equations~(\ref{basica1}) and~(\ref{basica2}) that
\[
\pont{\matriz}{\A}{\espaco} = \distanciaA{\matriz}(\A, \seqVazia) =
\distanciaA{\matriz}(\seqVazia, \A) = \pont{\matriz}{\espaco}{\A}\,.
\]
Thus, (\textit{i}) is true.

In order to check (\textit{ii}), suppose that $\pont{\matriz}{\A}{\B}
< \pont{\matriz}{\A}{\espaco} + \pont{\matriz}{\espaco}{\B}$. This
implies that
$
\custoA{\matriz}\alinhamento{\A, \B} = \pont{\matriz}{\A}{\B} < 
\pont{\matriz}{\A}{\espaco} + \pont{\matriz}{\espaco}{\B} =
\custoA{\matriz}\alinhamento{\A \espaco, \espaco \B} =
\custoA{\matriz}\alinhamento{\espaco \A, \B \espaco}\,.
$
Thus, from~(\ref{basica3}), it follows that $\distanciaA{\gamma}(\A,\B) =
\custoA{\gamma}\alinhamento{\A, \B}$. Furthermore, since
$\distanciaA{\matriz}(\B, \A) = \distanciaA{\matriz}(\A, \B)$ and,
by~$(i)$, $\pont{\matriz}{\A}{\espaco} = \pont{\matriz}{\espaco}{\A}$
and $\pont{\matriz}{\B}{\espaco} = \pont{\matriz}{\espaco}{\B}$, we
have that
\begin{align}
\distanciaA{\matriz}(\B, \A) &= \distanciaA{\matriz}(\A, \B)
= \custoA{\matriz}\alinhamento{\A,\B} = 
\pont{\matriz}{\A}{\B}\label{simetriaNormal1}
\\
&< \pont{\matriz}{\A}{\espaco} + \pont{\matriz}{\espaco}{\B} =
\pont{\matriz}{\B}{\espaco} + \pont{\matriz}{\espaco}{\A} =
\custoA{\matriz}\alinhamento{\B \espaco,
  \espaco \A} = \custoA{\matriz}\alinhamento{\espaco \B, \A
  \espaco}\,,
\nonumber \end{align}
which implies that neither $\alinhamento{\B \espaco,
  \espaco \A}$ nor $\alinhamento{\espaco \B, \A \espaco}$ are
A-optimal alignments of $\B, \A$. It follows from~(\ref{basica3}) that
\begin{equation}
\distanciaA{\matriz}(\B, \A) = \custoA{\matriz}\alinhamento{\B, \A}
= \pont{\matriz}{\B}{\A}\,.\label{simetriaNormal2}
\end{equation}
Using equations~(\ref{simetriaNormal1})
and~(\ref{simetriaNormal2}), it follows that (\textit{ii}) is also true.

Conversely, consider that (\textit{i}) and (\textit{ii}) are true and assume
that $\alinhamento{s', t'}$ is an A-optimal alignment of maximum length
of $s, t$. If $s'(h)$ or $t'(h)$ is $\espaco$, we have by~(\textit{i})
that $\pont{\matriz}{s'(h)}{t'(h)} = \pont{\matriz}{t'(h)}{s'(h)}$. If
$s'(h), t'(h) \in \alphabet$ then, by the chosen alignment $[s',t']$,
we have
\begin{align*}
\pont{\matriz}{s'(h)}{t'(h)} &+ \sum_{j \not= h}
\pont{\matriz}{s'(h)}{t'(h)} = \custoA{\matriz}[s', t'] \\
&< \custoA{\matriz} \alinhamentoB{
\begin{array}{cccc}
s'(1 \ldots j -1) & s'(j) & \espaco & s'(j + 1 \ldots \tamanho{(s',
  t')})\\
t'(1 \ldots j -1) & \espaco & t'(j) & t'(j + 1 \ldots \tamanho{(s',
  t')})
\end{array}}  \\
&= \pont{\matriz}{s'(h)}{\espaco} + \pont{\matriz}{\espaco}{t'(h)} +
\sum_{j \not= h} \pont {\matriz}{s'(h)}{t'(h)}\,.
\end{align*}
It follows that $\pont{\matriz}{s'(j)}{t'(j)} <
\pont{\matriz}{s'(j)}{\espaco} + \pont{\matriz}{\espaco}{t'(j)}$, which implies from~(\textit{ii}) that $\pont{\matriz}{s'(i)}{t'(i)} =
\pont{\matriz}{t'(i)}{s'(i)}$. Thus, regardless of the symbols $s'(i)$
and $t'(i)$, we have $\pont{\matriz}{s'(i)}{t'(i)} =
\pont{\matriz}{t'(i)}{s'(i)}$ for each $i$, which implies that
$\custoA{\matriz}\alinhamento{s', t'} =
\custoA{\matriz}\alinhamento{t', s'}$. Since $\alinhamento{s', t'}$
is an A-optimal alignment of $s, t$ and $\alinhamento{t', s'}$ is an
alignment of $t, s$, it follows that
\[
\distanciaA{\matriz}(s, t) = \custoA{\matriz}\alinhamento{s',t'} =
\custoA{\matriz}\alinhamento{t', s'} \ge \distanciaA{\matriz}(t,
s)\,.
\]
Using the same arguments we can prove that $\distanciaA{\matriz}(t, s)
\ge \distanciaA{\matriz}(s, t)$, which allows us to conclude that
$\distanciaA{\matriz}(s, t) = \distanciaA{\matriz}(t, s)$.
\end{proof}

\begin{proposition}\label{desigTriangularGeral}\rm
Let $\matriz$ be a scoring matrix, $Q$ an integer, and $s, t, u \in
\alphabet^{*}$. If
\begin{enumerate}
\item[(\textit{i})] $\pont{\matriz}{\A}{\espaco} \le
  \pont{\matriz}{\A}{\B} + \pont{\matriz}{\B}{\espaco}$\,,
\item[(\textit{ii})] $\pont{\matriz}{\espaco}{\A} \le
  \pont{\matriz}{\espaco}{\B} + \pont{\matriz}{\B}{\A}$\,,
\item[(\textit{iii})] $\min \{\pont{\matriz}{\A}{\C},
  \pont{\matriz}{\A}{\espaco} + \pont{\matriz}{\espaco}{\C}\} \le
  \pont{\matriz}{\A}{\B} + \pont{\matriz}{\B}{\C}$\,, and
\item[(\textit{iv})] $\pont{\matriz}{\B}{\espaco} +
  \pont{\matriz}{\espaco}{\B} \ge Q$\,,
\end{enumerate}
for each $\A, \B, \C \in \alphabet$, then, for each alignment $A$ of
$s, t$ and each alignment $B$ of $t, u$, there exists an alignment
$C$ of $s, u$ and an integer $k \ge 0$ such that
\[
\custoA{\matriz}[A] + \custoA{\matriz}[B] \ge \custoA{\matriz}[C] +
kQ\,, \quad \tamanho{A} \le \tamanho{C} + k\,, \quad \text{and} \quad
\tamanho{B} \le \tamanho{C} + k\,.
\]  
\end{proposition}
\begin{proof}
Suppose that (\textit{i}), (\textit{ii}), (\textit{iii}),
(\textit{iv}) are true. Let $A, B$ be alignments of $s, t$ and $t,
u$, respectively. We define
\begin{align*}
  \mathcal{C}_{1} &= \left\{h : [s(h), \espaco]~\text{is aligned in $A$}\right\}\,, \\
  \mathcal{C}_{2} &= \left\{k : [\espaco, u(k)]~\text{is aligned in $B$}\right\}\,, \\
  \mathcal{C}_{3} &= \left\{j : [\espaco, t (j)]~\text{is aligned in $A$ and $[t(j), \espaco]$
    is aligned in $B$}\right\}\,, \\
  \mathcal{C}_{4} &= \left\{(h, j) :
  ~\text{$[s(h), t(j)]$ is aligned in $A$ and 
     $[t(j), \espaco]$ is aligned in $B$}
  \right\}\,,
  \\
  \mathcal{C}_{5} &= \left\{(j, k) :~\text{$[\espaco, t (j)]$ is aligned in $A$ and $[t(j), u(k)]$
    is aligned in $B$}\right\}\,, \\
  \mathcal{C}_{6} &= \Big\{\!\! \begin{array}{l}
    (h, j, k) :~\text{$[s(h), t(j)]$ is aligned in $A$, $[t(j), u(k)]$ is aligned in $B$} \!\! \\
    \text{and}~\pont{\matriz}{s(h)}{u(k)} \le
    \pont{\matriz}{s(h)}{t(j)} + \pont{\matriz}{t(j)}{u(k)}
  \end{array} \Big\}\,, \\
  \mathcal{C}_{7} &= \Big\{\!\! \begin{array}{l}
    (h, j, k) :~\text{$[s(h), t(j)]$ is aligned in $A$, $[t(j), u(k)]$ is aligned in $B$} \!\!\\
    \text{and}~\pont{\matriz}{s(h)}{u(k)} > \pont{\matriz}{s(h)}{t(j)}
    + \pont{\matriz}{t(j)}{u(k)}
  \end{array} \Big\}\,.
\end{align*}

Thus,
\begin{align*}
\custoA{\matriz}[A] = &\sum_{h \in \mathcal{C}_{1}}
\pont{\matriz}{s(h)}{\espaco} + \sum_{j \in \mathcal{C}_{3}}
\pont{\matriz}{\espaco}{t(j)} + \sum_{(h, j) \in \mathcal{C}_{4}}
\pont{\matriz}{s(h)}{t(j)} + 
\sum_{(j, k) \in \mathcal{C}_{5}} \pont{\matriz}{\espaco}{t(j)}
\\ & + \sum_{(h, j, k) \in \mathcal{C}_{6}} \pont{\matriz}{s(h)}{t(j)} +
\sum_{(h, j, k) \in \mathcal{C}_{7}}
\pont{\matriz}{s(h)}{t(j)}
\end{align*}
and
\begin{align*}
\custoA{\matriz}[B] = &\sum_{k \in \mathcal{C}_{2}}
\pont{\matriz}{\espaco}{u(k)} + \sum_{j \in \mathcal{C}_{3}}
\pont{\matriz}{t (j)}{\espaco} + \sum_{(h, j) \in \mathcal{C}_{4}}
\pont{\matriz}{t (j)}{\espaco} + 
\sum_{(j, k) \in \mathcal{C}_{5}} \pont{\matriz}{t(j)}{u(k)}\\
& + 
\sum_{(h, j, k) \in \mathcal{C}_{6}} \pont{\matriz}{t(j)}{u(k)} + \sum_{(h, j, k) \in \mathcal{C}_{7}}
\pont{\matriz}{t(j)}{u(k)}\,,
\end{align*}
which implies that 
\begin{align*}
  \custoA{\matriz}[A] &+ \custoA{\matriz}[B] =\\
  & 
  \sum_{h \in \mathcal{C}_{1}}
  \pont{\matriz}{s(h)}{\espaco} + 
  \sum_{k \in \mathcal{C}_{2}}
  \pont{\matriz}{\espaco}{u(k)} +
  \sum_{j \in \mathcal{C}_{3}} \left(\pont{\matriz}{\espaco}{t (j)} + \pont{\matriz}{t (j)}{\espaco}\right) + \\
  & \sum_{(h, j) \in \mathcal{C}_{4}} \left(
  \pont{\matriz}{s(h)}{t (j)} + \pont{\matriz}{t (j)}{\espaco} \right) + 
  \sum_{(j, k) \in \mathcal{C}_{5}} \left( \pont{\matriz}{\espaco}{t(j)}+\pont{\matriz}{t(j)}{u(k)} \right)+ \\
  &  
  \sum_{(h, j, k) \in \mathcal{C}_{6}} \left( \pont{\matriz}{s(h)}{t(j)}+\pont{\matriz}{t(j)}{u(k)}\right) + \sum_{(h, j, k) \in \mathcal{C}_{7}}
  \left(\pont{\matriz}{s(h)}{t(j)}+ \pont{\matriz}{t(j)}{u(k)} \right)\,.
\end{align*}

Define the alignment $C$ of $s, u$ such that if $(h, j, k) \in \mathcal{C}_{6}$ then it aligns 
$[s(h), u(k)]$ and, for each remaining $s(h)$ and $u(k)$, it aligns $[s(h),
\espaco]$ and $[\espaco, u(k)]$. Thus,
\begin{align*}
\custoA{\matriz}[C] = &\sum_{h \in \mathcal{C}_{1}}
\pont{\matriz}{s(h)}{\espaco} + \sum_{k \in \mathcal{C}_{2}}
\pont{\matriz}{\espaco}{u (k)} + \sum_{(h, j) \in \mathcal{C}_{4}}
\pont{\matriz}{s(h)}{\espaco} + \sum_{(j, k) \in \mathcal{C}_{5}} \pont{\matriz}{\espaco}{u
  (k)}\\& + \sum_{(h, j, k) \in \mathcal{C}_{6}} \pont{\matriz}{s(h)}{u
  (k)} + \sum_{(h, j, k) \in \mathcal{C}_{7}} \big(
\pont{\matriz}{s(h)}{\espaco} + \pont{\matriz}{\espaco}{u(k)} \big)\,.
\end{align*}

If $j \in \mathcal{C}_{3}$, then, since (\textit{iv}) is true, we have
\begin{align}
\sum_{j \in
  \mathcal{C}_{3}} \big( \pont{\matriz}{t(j)}{\espaco} +
\pont{\matriz}{\espaco}{t(j)} \big) \ge \sum_{j \in \mathcal{C}_{3}} Q =
\tamanho{\mathcal{C}_{3}}\,{Q}\,. \label{C3}
\end{align}

From (\textit{i}),~(\textit{ii}), and the definition of
$\mathcal{C}_{6}$, we have, respectively,
\begin{align}
  \sum_{(h, j) \in \mathcal{C}_{4}} \big( \pont{\matriz}{s(h)}{t(j)} +
  \pont{\matriz}{t(j)}{\espaco} \big) &\ge \sum_{(h, j) \in \mathcal{C}_{4}}
  \pont{\matriz}{s(h)}{\espaco}\,. \label{C4} \\
  \sum_{(j, k) \in \mathcal{C}_{5}} \big(
  \pont{\matriz}{\espaco}{t(j)} + \pont{\matriz}{t(j)}{u(k)} \big)
  &\ge \sum_{(j, k) \in \mathcal{C}_{5}}
  \pont{\matriz}{\espaco}{u(k)}\,. \label{C5} \\
  \sum_{(h, j, k) \in \mathcal{C}_{6}} \big(
  \pont{\matriz}{s(h)}{t(j)} + \pont{\matriz}{t(j)}{u(k)} \big)
  &\ge \sum_{(h, j, k) \in \mathcal{C}_{6}}
  \pont{\matriz}{s(h)}{u(k)}\,. \label{C6}
\end{align}

Suppose that $(h, j, k) \in \mathcal{C}_{7}$. Then, by definition of
$\mathcal{C}_{7}$, we know that $\pont{\matriz}{s(h)}{u(k)} >
\pont{\matriz}{s(h)}{t(j)} + \pont{\matriz}{t(j)}{u(k)}$. It follows
from (\textit{iii}) that $\pont{\matriz}{s(h)}{t(j)} +
\pont{\matriz}{t(j)}{u(k)} \ge \pont{\matriz}{s(h)}{\espaco} +
\pont{\matriz}{\espaco}{u(k)}$. Thus,
\begin{align}
  \sum_{(h, j, k) \in \mathcal{C}_{7}} \big(
  \pont{\matriz}{s(h)}{t(j)} + \pont{\matriz}{t(j)}{u(k)} \big)
  \ge \sum_{(h, j, k) \in \mathcal{C}_{7}} \big(
  \pont{\matriz}{s(h)}{\espaco} + \pont{\matriz}{\espaco}{u(k)}
  \big)\,. \label{C7}
\end{align}

From equations~(\ref{C3}),~(\ref{C4}),~(\ref{C5}),~(\ref{C6}),
and~(\ref{C7}), we have
\[
\custoA{\matriz}[A] + \custoA{\matriz}[B] \ge \custoA{\matriz}[C] +
\tamanho{\mathcal{C}_{3}} \, Q\,.
\]
Hence, to finish the proof, it is enough to show that $\tamanho{A} \le
\tamanho{C} + \tamanho{\mathcal{C}_{3}}$ and $\tamanho{B} \le
\tamanho{C} + \tamanho{\mathcal{C}_{3}}$.
Since 
\begin{align*}
\tamanho{A} &= \sum_{i} \tamanho{\mathcal{C}_{i}} -
\tamanho{\mathcal{C}_{2}} \le \sum_{i} \tamanho{\mathcal{C}_{i}}\,,
\\
\tamanho{B} &= \sum_{i} \tamanho{\mathcal{C}_{i}} -
\tamanho{\mathcal{C}_{1}} \le \sum_{i} \tamanho{\mathcal{C}_{i}}\,,
\\
\tamanho{C} &= \sum_{i} \tamanho{\mathcal{C}_{i}} -
\tamanho{\mathcal{C}_{3}} + \tamanho{\mathcal{C}_{7}} \ge \sum_{i}
\tamanho{\mathcal{C}_{i}} - \tamanho{\mathcal{C}_{3}}\,,
\end{align*}
we have $\tamanho{A} \le \sum_{i} \tamanho{\mathcal{C}_{i}} \le
\tamanho{C} + \tamanho{\mathcal{C}_{3}}$ and $\tamanho{B} \le \sum_{i}
\tamanho{\mathcal{C}_{i}} \le \tamanho{C} +
\tamanho{\mathcal{C}_{3}}$.
\end{proof}

\begin{lemma}\label{desigualdadeTriangularNormal}\rm
$\distanciaA{\matriz}(s, u) \le \distanciaA{\matriz}(s, t) +
  \distanciaA{\matriz}(t, u)$ for each $s, t, u \in \alphabet^{*}$
  if and only if
\begin{enumerate}
\item[(\textit{i})] $\pont{\matriz}{\A}{\espaco} \le
  \pont{\matriz}{\A}{\B} + \pont{\matriz}{\B}{\espaco}$\,,
\item[(\textit{ii})] $\pont{\matriz}{\espaco}{\A} \le
  \pont{\matriz}{\espaco}{\B} + \pont{\matriz}{\B}{\A}$\,,
\item[(\textit{iii})] $\min \{ \pont{\matriz}{\A}{\C},
  \pont{\matriz}{\A}{\espaco} + \pont{\matriz}{\espaco}{\C} \} \le
  \pont{\matriz}{\A}{\B} + \pont{\matriz}{\B}{\C}$\,, and
\item[(\textit{iv})] $\pont{\matriz}{\B}{\espaco} +
  \pont{\matriz}{\espaco}{\B} \ge 0$\,,
\end{enumerate}
for each $\A, \B, \C \in \alphabet$.
\end{lemma}
\begin{proof}
Suppose that $\distanciaA{\matriz}(s, u) \le
\distanciaA{\matriz}(s, t) + \distanciaA{\matriz}(t, u)$ for each $s,
t, u \in \alphabet^{*}$. 
It follows from equations~(\ref{basica1}),~(\ref{basica2}),
and~(\ref{basica3}) that
\begin{align*}
\pont{\matriz}{\A}{\espaco} =  \distanciaA{\matriz}(\A, \seqVazia) &\le
\distanciaA{\matriz}(\A, \B) + \distanciaA{\matriz}(\B, \espaco) \\
&\le \custoA{\matriz}\alinhamento{\A, \B} +
\custoA{\matriz}\alinhamento{\espaco, \B} = \pont{\matriz}{\A}{\B} + \pont{\matriz}{\espaco}{\B}\,,
& \\
\pont{\matriz}{\espaco}{\A} =
\distanciaA{\matriz}(\seqVazia, \A) &\le 
\distanciaA{\matriz}(\espaco, \B) + \distanciaA{\matriz}(\B, \A) \\
&\le \custoA{\matriz}\alinhamento{\espaco, \B} +
\custoA{\matriz}\alinhamento{\B, \A} = \pont{\matriz}{\espaco}{\B} + \pont{\matriz}{\B}{\A}\,,\\
  \min \{ \pont{\matriz}{\A}{\C}, \pont{\matriz}{\A}{\espaco} +
  \pont{\matriz}{\espaco}{\C} \} &= \distanciaA{\matriz}(\A, \C) \\
  &\le \distanciaA{\matriz}(\A, \B) + \distanciaA{\matriz}(\B, \C) \le \pont{\matriz}{\A}{\B} + \pont{\matriz}{\B}{\C}\,.
\end{align*}
Therefore, (\textit{i}), (\textit{ii}), (\textit{iii}) are true.

If $\pont{\matriz}{\B}{\espaco} + \pont{\matriz}{\espaco}{\B} = 0$,
(\textit{iv}) is true. Assume then that
$\pont{\matriz}{\B}{\espaco} + \pont{\matriz}{\espaco}{\B} \not= 0$.
Let 
\[
n > \frac{\distanciaA{\matriz}(\A, \C) - (\pont{\matriz}{\A}{\espaco}
  + \pont{\matriz}{\espaco}{\C})} {\pont{\matriz}{\B}{\espaco} +
  \pont{\matriz}{\espaco}{\B}}
\] 
be a positive integer. Since $\alinhamento{\A \espaco^{n}, \espaco
  \B^{n}}$ and $\alinhamento{\B^{n} \espaco, \espaco^{n} \C}$ are
alignments of $\A, \B^{n}$ and $\B^{n}, \C$, respectively, it follows
that
\begin{align*}
\distanciaA{\matriz}(\A, \C) &\le \distanciaA{\matriz}(\A, \B^{n}) +
\distanciaA{\matriz} (\B^{n}, \C) \\
&\le \custoA{\matriz}\alinhamento{\A \espaco^{n}, \espaco \B^{n}} +
\custoA {\matriz} \alinhamento{\B^{n} \espaco, \espaco^{n} \C} \\
&= \pont{\matriz}{\A}{\espaco} + n \pont{\matriz}{\espaco}{\B} + n
\pont{\matriz}{\B}{\espaco} + \pont{\matriz}{\espaco}{\C}\,,
\end{align*}
and thus, by the choice of $n$, we have that $\pont{\matriz}{\B}{\espaco} 
+ \pont{\matriz}{\espaco}{\B} > 0$, which implies that $(\textit{iv})$ is 
also true. %We conclude that if $\distanciaA{\matriz}(s, u) \le \distanciaA{\matriz}(s, t) + \distanciaA{\matriz}(t, u)$ for each $s, t, u \in \alphabet^{*}$, then $(\textit{i})$, $(\textit{ii})$, $(\textit{iii})$, and $(\textit{iv})$ are true.

Conversely, suppose that (\textit{i}), (\textit{ii}), (\textit{iii}), and
(\textit{iv}) are true.
Let $A$ and $B$ be A-optimal alignments of $s, t$ and $t, u$,
respectively. It follows from Proposition~\ref{desigTriangularGeral}
that there exist an integer $k$ and an alignment $C$ of $s, u$ such
that
\[
\custoA{\matriz}[C] \le \custoA{\matriz}[A] + \custoA{\matriz}[B] +
0\,k = \custoA{\matriz}[A] + \custoA{\matriz}[B]\,.
\]
Consequently, since $A$ and $B$ are A-optimal alignments of $s, t$
and $t, u$, and $C$ is an alignment of $s, u$, it follows that
\[
\distanciaA{\matriz}(s, u) \le \custoA{\matriz}[C] \le
\custoA{\matriz}[A] + \custoA{\matriz}[B] = \distanciaA{\matriz}(s, t)
+ \distanciaA{\matriz}(t, u)\,.
\]
%In this case, we also conclude that if $(\textit{i})$, $(\textit{ii})$, $(\textit{iii})$, and $(\textit{iv})$ are true, and finally we have that $\distanciaA{\matriz}(s, u) \le \distanciaA{\matriz}(s, t) +\distanciaA{\matriz}(t, u)$ for each $s, t, u \in \alphabet^{*}$.
\end{proof}

\begin{table}[htpb]
\begin{minipage}{\textwidth}
\begin{center}
\begin{tabular}{clcccccc}
  & & $\prametrica$ & $\semimetrica$ & $\hemimetrica$ &
  $\pseudometrica$ & $\quasimetrica$ & $\metrica$ \\ \hline & & \\
  (a) & $D(\matriz)$ has no negative cycle & \yes & \yes & \yes & \yes
  &\yes & \yes \\ & & \\
  (b) & $\pont{\matriz}{\A}{\A} = 0$ or $\pont{\matriz}{\A}{\espaco} +
  \pont{\matriz}{\espaco}{\A} = 0$ & \yes & \yes & \yes & \yes &\yes &
  \yes \\ & & \\
  (c) & $\pont{\matriz}{\A}{\espaco}, \pont{\matriz}{\espaco}{\B},
  \pont{\matriz}{\A}{\B} \ge 0$ & \yes & \yes & \yes & \yes &\yes &
  \yes \\ & & \\
  (d) & $\pont{\matriz}{\A}{\A} \ge 0$& & \yes & & &\yes & \yes \\ & &
  \\
  (e) & $\pont{\matriz}{\A}{\espaco}, \pont{\matriz}{\espaco}{\A} > 0$
  and $\pont{\matriz}{\A}{\B} > 0$ if $\A \not= \B$ & & \yes & & &\yes
  & \yes \\& & \\
  (f) & $\pont{\matriz}{\A}{\espaco} = \pont{\matriz}{\espaco}{\A}$ &
  & \yes & & \yes & & \yes \\ & & \\
  (g) &
  \begin{tabular}{l}
    if $\pont{\matriz}{\A}{\B} < \pont{\matriz}{\A}{\espaco}
    + \pont{\matriz}{\espaco}{\B}$ then \\
    $\pont{\matriz}{\A}{\B} = \pont{\matriz}{\B}{\A}$
  \end{tabular}  
  & & \yes & & \yes & & \yes \\ & & \\
  (h) & $\pont{\matriz}{\A}{\espaco} \le \pont{\matriz}{\A}{\B} +
  \pont{\matriz}{\B}{\espaco}$ & & & \yes & \yes &\yes & \yes \\ & &
  \\
  (i) & $\pont{\matriz}{\espaco}{\A} \le \pont{\matriz}{\espaco}{\B} +
  \pont{\matriz}{\B}{\A}$ & & & \yes & \yes &\yes & \yes\\ & & \\
  (j) & 
  $\min \Big\{ \begin{array}{l}
    \pont{\matriz}{\A}{\C}, \\
    \pont{\matriz}{\A}{\espaco} + \pont{\matriz}{\espaco}{\C} 
  \end{array} \Big\} 
  \le \pont{\matriz}{\A}{\B} + \pont{\matriz}{\B}{\C}$
  &  &  & \yes & \yes &\yes & \yes\\ & & \\
  (k) & $\pont{\matriz}{\B}{\espaco} + \pont{\matriz}{\espaco}{\B} \ge
  0$ & & & \yes & \yes &\yes & \yes
\end{tabular}
\end{center}
\end{minipage}
\caption{Necessary and sufficient conditions for scoring matrix
  $\matriz$ induce $\distanciaA{\gamma}\text{-}p$ on sequences. 
  Besides \emph{metric} ($\metrica$), these properties are also used to define 
  generalized metric spaces such as \emph{premetric} ($\prametrica$),
  \emph{semimetric} ($\semimetrica$), \emph{hemimetric}
  ($\hemimetrica$), \emph{pseudometric} ($\pseudometrica$), and 
  \emph{quasimetric} ($\quasimetrica$). Results are obtained using
  definitions presented in Section~\ref{sec:preliminares}, and
  Lemmas~\ref{distancia=0}, \ref{distanciaMaiorOuIgualAZero},
  \ref{distanciaMaiorQue0}, \ref{simetriaNormal}, and
  \ref{desigualdadeTriangularNormal}, for each $\A, \B, \C \in
  \alphabet$.} \label{tabela2}
\end{table}

Table~\ref{tabela2} summarizes the results of scoring matrices $\matriz$ that induce $\distanciaA{\matriz}\text{-}p$, where $p$ is a property that allows to characterize each axiom of the metric on sequences. Finally, we can prove the preeminent result of this section.

\begin{proof} (of Theorem~\ref{theo:align})

Suppose that $\distanciaA{\matriz}$ is a metric. 
Thus, all conditions in Table~\ref{tabela2} are satisfied.
From~(e) and~(f), we have that $\pont{\matriz}{\A}{\espaco} =
\pont{\matriz}{\espaco}{\A} > 0$ and therefore (\textit{i}) is true.
Since (\textit{i}) is true, we have that
$\pont{\matriz}{\A}{\espaco} + \pont{\matriz}{\espaco}{\A} \not= 0$,
which implies by~(b) that $\pont{\matriz}{\A}{\A} = 0$;
moreover, by~(e), we have that $\pont{\matriz}{\A}{\B} > 0$ if $\A \not= \B$;
it follows that (\textit{ii}) is true.
From~(g), (h), and (j), we have that (\textit{iii}), (\textit{iv}), and
(\textit{v}) are also true.
Therefore, if $\distanciaA{\matriz}$ is a
metric, conditions (\textit{i}) to (\textit{v})
are satisfied.

Conversely, suppose that the conditions (\textit{i}) to (\textit{v})
are true and, in order to prove that these are sufficient conditions
for $\distanciaA{\matriz}$ to be a metric, we check whether all 
conditions in Table~\ref{tabela2} are satisfied.
We have, by (\textit{i}) and (\textit{ii}), that the conditions (a) to
(f), and the condition (k), are satisfied.
Since 
(\textit{iii}), (\textit{iv}) and (\textit{v}) are true,
it follows that 
(g), (h), and (j) are also true. 
Then, it is enough to show that the condition 
(i) in Table~\ref{tabela2} is true.

Suppose that $\pont{\matriz}{\B}{\A} \ge 
\pont{\matriz}{\B}{\espaco} + \pont{\matriz}{\espaco}{\A}
$. 
Since (k) is true, it follows that 
\[\pont{\matriz}{\espaco}{\A}\le 
\pont{\matriz}{\espaco}{\B} + 
\pont{\matriz}{\B}{\espaco} +
\pont{\matriz}{\espaco}{\A} 
\le 
\pont{\matriz}{\espaco}{\B} + 
\pont{\matriz}{\B}{\A}\]
and the proof is done. 
Then, assume that $\pont{\matriz}{\B}{\A} <
\pont{\matriz}{\B}{\espaco} + \pont{\matriz}{\espaco}{\A}$.
It follows from (\textit{iii}) 
that 
$\pont{\matriz}{\B}{\A} =
\pont{\matriz}{\A}{\B}$,
which implies from (\textit{i}) and (\textit{iv}) that 
\[
\pont{\matriz}{\espaco}{\A}
= \pont{\matriz}{\A}{\espaco}
\le 
\pont{\matriz}{\A}{\B} + 
\pont{\matriz}{\B}{\espaco} 
=
\pont{\matriz}{\espaco}{\B} + 
\pont{\matriz}{\B}{\A}\,.\]
Therefore, if conditions (\textit{i}) to (\textit{v})
are true, then all conditions in Table~\ref{tabela2}
are satisfied, which implies that $\distanciaA{\matriz}$ is a
metric on sequences.
\end{proof}

