\section{Introduction} \label{sec:intro}

Sequence comparison is a classical problem in computer science and
has applications in several areas such as computational biology, text
processing, and pattern recognition. A scoring function is a measure to determine the degree of similarity between two sequences and can be defined by edit operations that transform one sequence into another. Typical edit operations are insertion, deletion, and substitution of symbols.

A simple scoring function to compare two given sequences $s$ and $t$ is the Levenshtein distance~\citep{Levenshtein1965}, or \emph{edit distance}, defined as the minimum number of edit operations that transform $s$ into $t$. Given that, the edit distance is a way of quantifying the  dissimilarity of two strings and, as a consequence, it is used in several applications where the data can be represented by strings. For instance, in~\citep{BARTON2015} the edit distance was used in an application related to genome assembly where the authors showed that adding the flexibility of bounding the number of gaps inserted in an alignment strengthens the classical sequence alignment scheme of scoring matrices.

There are some variations of the edit distance that worth mentioning. When each symbol in $s$ must be edited exactly once, one can usually compare $s$ and $t$ by computing the \emph{weighted edit distance}, that is, the minimum weight to transform $s$ into $t$ through a sequence of weighted edit operations. It is used, for instance, in spelling correction systems~\citep{GOSH2016}. Another function is the \emph{normalized edit distance}~\citep{MarzalV1993}, where the length of the two strings is taken into account when computing the distance between them. The result of this function is the minimum average weight of a set of edit operations required to transform $s$ into $t$. Examples of applications of the normalized edit distance are the text reading from street view images~\citep{SUN2019} and software verification. In the latter, runs of a system are represented using words and it is customary to analyze the relationship between the set of words that satisfy a given specification and the set  of words that the system under examination produces~\citep{FISMAN2022}. The \emph{generalized edit distance} proposed by \cite {YujianB2007} is a metric and it is a simple function of the strings lengths; it was applied in a handwritten digit recognition study and showed that it can generally provide similar results to some other normalized edit distances. Another variation is the \emph{contextual normalized edit distance}, where the cost of each edit operation is divided by the length of the string on which the edit operation takes place. It was introduced by \cite{MICO2008} and they showed that this variation is useful for classification purposes.


Considering the weighted edit distance, any set of edit operations that transforms $s$ into $t$, restricted so that each symbol of $s$ must be edited once and only once, can be represented by an alignment. Consider the following alignment:
\[
\left[
  \begin{array}{cccccc}
    \A & \A & \B & \espaco &\A  & \espaco\\
    \A & \espaco & \A & \B & \B & \A
    \end{array}
  \right]\enspace .
\]
It represents the transformation of $s = \A \A \B \A$ into $t = \A \A \B \B \A$ by replacing the first and third symbol with $\A$, the last one with $\B$, deleting the second symbol, and inserting a $\B$ and an $\A$ before and after the last symbol in $s$. Alignments have been a standard way to compare two or more sequences with a myriad of applications in computational biology~\citep{NeedlemanW1970,SW1981,LP1985,Altschul.et.al.1990,LAK1989,Chenna.et.al.2003,NHH2000,KBH1998,ARRM2021}. 

Let $\alphabet$ be an alphabet and $\alphabet_{\espaco} = \alphabet \cup \{ \espaco \}$, where $\espaco$ is a symbol not belonging to $\alphabet$ which represents insertion and deletion operations. A \emph{scoring matrix} $\matriz$ for $\alphabet$ has its rows and columns indexed by elements of~$\alphabet_{\espaco}$ and represents the weighted edit operations. We denote the entry of $\matriz$ in row $\A$ and column $\B$ by $\pont{\matriz}{\A}{\B}$. If $\A, \B \in \alphabet$, then $\pont{\matriz}{\A}{\B}$ is the weight of replacing the symbol $\A$ with $\B$, $\pont{\matriz}{\A}{\espaco}$ is the weight of deleting the symbol $\A$, $\pont{\matriz}{\espaco}{\B}$ is the weight of inserting the symbol $\B$, and $\pont{\gamma}{\espaco}{\espaco}$ is not defined. Notice that the Levenshtein distance is a weighted edit distance when $\pont{\gamma}{\A}{\B} = 0$ for $\A = \B$, and $\pont{\gamma}{\A}{\B} = 1$ otherwise, for each pair of symbols $\A$ and $\B$.

Denote by $\alphabet^{*}$ the set of all finite sequences on $\alphabet$. Many different scoring functions based on edit operations
are induced by scoring matrices. \citet{Pea2013} presented ways to select scoring matrices with practical significance in bioinformatics, to estimate distance and similarity.

For a matrix $\gamma$, we have that $\opt_\gamma: \alphabet^{*} \times \alphabet^{*} \rightarrow \mathbb{R}$ represents a general scoring function \emph{induced} by $\gamma$ such that $\opt_\gamma(s, t)$ is the minimum score of the edit operations transforming sequence $s$ into sequence $t$. The scoring function $\opt_\gamma$ is a metric on sequences if $\opt_\gamma$ satisfies the axioms of reflexivity, non-negativity, positivity, symmetry, and triangle inequality; and we say that $\gamma$ \emph{induces} an $\opt_\gamma$-metric on sequences.
If $\opt_{\gamma}$ satisfies the property $p$, we say that $\gamma$ \emph{induces} $\opt_{\gamma}\text{-}p$ on sequences.
  
The weighted edit distance between sequences $s$ and $t$ is denoted by
$\distanciaA{\gamma}$. Observe that not all scoring matrices induce 
scoring functions that are distances, since the scoring function is not necessarily a metric. \citet{Sellers1974} described a sufficient condition for a weighted edit distance be a metric on $\alphabet^{*}$, 
\textit{i.e.}, proved that a scoring matrix $\matriz$ induces an 
$\distanciaA{\gamma}$-metric on sequences. \citet{AraujoS2006} presented necessary and sufficient conditions for a scoring matrix $\matriz$ induce a weighted edit distance as a metric on $\alphabet^{*}$. For example, the scoring matrix
\[
\begin{array}{c|cccc}
  \gamma & \A & \B & \C & \espaco\\
  \hline
  \A & 0 & 1 & 3 & 1\\
  \B & 1 & 0 & 1 & 1\\
  \C & 4 & 1 & 0 & 1\\
  \espaco & 1 & 1 & 1 & \espaco
\end{array}
\]
induces a weighted edit distance, although $\pont{\matriz}{\A}{\C} \not= \pont{\matriz}{\C}{\A}$, $\pont{\matriz}{\A}{\C} \not\le \pont{\matriz}{\A}{\espaco} + \pont{\matriz}{\espaco}{\C}$, and $\pont{\matriz}{\C}{\A} \not\le \pont{\matriz}{\C}{\espaco} + \pont{\matriz}{\espaco}{\A}$. Moreover, there are many ways to generalize the concept of metric by relaxing the axioms that define the metric space, which gives rise to different generalized metric spaces such as quasimetric, semimetric, and others. In this work we extended investigations from~\citet{AraujoS2006} establishing necessary and
sufficient conditions for scoring matrices $\gamma$ to induce $\distanciaA{\gamma}\text{-}p$ for each axiom $p$ of a metric on sequences. 

\citet{MarzalV1993} set another scoring function based on edit operations and induced by a scoring matrix $\gamma$, called normalized edit distance and denoted by $\distanciaN{\gamma}$. Given sequences $s$ and $t$, considering that each symbol in $s$ must be edited exactly once, then $\distanciaN{\gamma}(s, t)$ is the minimum average weight
of a set of edit operations required to transform $s$ into $t$.
Similarly, not all scoring matrices $\gamma$ induce an $\distanciaN{\gamma}$-metric and \cite{AraujoS2006} characterize the
class of scoring matrices inducing normalized edit distances. In this work we also characterize each class of scoring matrices inducing 
$\distanciaN{\gamma}\text{-}p$ on sequences for some generalized metric spaces $p$. Recently,~\citet{FISMAN2022} proved that the normalized edit distance proposed in \citep{MarzalV1993} is a metric when cost of all edit operations are the same. Given the contributions of our work aforementioned, we highlight that this article is more general in the sense that it covers the case treated in \citet{FISMAN2022}.

Additionally to the contributions mentioned above, we define a third scoring function based on edit operations induced by scoring matrices $\gamma$, called extended edit distance and denoted by $\distanciaE{\gamma}$. This scoring function takes into account a set of edit operations that transforms one sequence $s$ into another sequence $t$, with no restriction to the number of edit operations in each symbol of $s$, \textit{i.e.}, a symbol in $s$ can be edited an arbitrary number of times. We characterize each class of scoring matrices inducing $\distanciaE{\gamma}\text{-}p$, for each axiom $p$ of a metric.

Summarizing, the main contributions of this paper are:
\begin{itemize}
\item We extended investigations from~\citet{AraujoS2006} establishing necessary and sufficient conditions for scoring matrices $\gamma$ to induce $\distanciaA{\gamma}\text{-}p$ for each axiom $p$ of a metric on sequences;
\item We characterize each class of scoring matrices inducing 
$\distanciaN{\gamma}\text{-}p$ on sequences for some generalized metric spaces $p$; and 
\item We define a scoring function based on edit operations induced by scoring matrices $\gamma$, called extended edit distance, denoted by $\distanciaE{\gamma}$; we characterize each class of scoring matrices inducing $\distanciaE{\gamma}\text{-}p$ for each axiom $p$ of a metric.
\end{itemize}

This paper is organized as follows. Section~\ref{sec:preliminares}
provides a brief description of basic concepts and characterizes the classes of matrices that induce metric properties for the aimed scoring functions: weighted edit distance, normalized edit distance and extended edit distance. Sections \ref{sec:regular}, \ref{sec:normalizado}, and \ref{sec:estendido} present the main results for these distances. Section~\ref{sec:conclusion} concludes.


