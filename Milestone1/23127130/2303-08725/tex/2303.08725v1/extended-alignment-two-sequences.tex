\section{Extended alignment of two sequences}\label{sec:estendido}


We describe in this section the classes of scoring matrices that induce $\distanciaE{\matriz}$-$p$ on sequences for each axiom $p$ of a metric. Lemmas~\ref{lema-ZeroE}--\ref{desigualdadeE} establish these properties, which are summarized in Table~\ref{tabela4}, and allow us to characterize matrices that induces each of the more general metric functions described in Section~\ref{sec:preliminares}. Lastly, we present the following important result previously stated in Section~\ref{sec:preliminares}:

\begin{theorem} \label{theo:extend} \rm
$\distanciaE{\matriz} \in \metrica$ if and only if $\matriz \in \metricaE$.
\end{theorem}

To prove this result, we proceed as in the previous sections and present some intermediary results as follows.

\begin{fact}\label{fato-permuta}\rm
Each weighted directed multigraph obtained by arcs that represent edit operations that transform a sequence into itself is Eulerian.
\begin{comment}
Let $s \in \alphabet^{*}$ and $\alinhamento{c_{1}, \ldots, c_{n}}$ be
an extended alignment of $s, s$. There exists a permutation $(j_{1},
\ldots, j_{n})$ of $(1, \ldots, n)$ such that $c_{j_{1}}(m_{j_{1}}) =
c_{j_{2}}(1)$, \ldots, $c_{j_{n-1}}(m_{j_{n-1}})= c_{j_{n}}(1)$,
$c_{j_{n}}(m_{j_{n}})= c_{j_{1}}(1)$, which implies that if $C$ is the
concatenation of $c_{j_{1}}, \ldots, c_{j_{n}}$, then $C$ is a cycle
in $D(\matriz)$ and $\cost(C) = \custoE{\matriz}[A]$.
\end{comment}
\end{fact}

\begin{lemma}\label{lema-ZeroE}\rm
Let $s \in \alphabet^{*}$ and $\matriz$ a scoring matrix. Then,
$\distanciaE{\matriz}(s, s) = 0$ if and only if $D(\matriz)$ has no
negative cycle.
\end{lemma}


\begin{proof}
Suppose that $\distanciaE{\matriz}(s, s) = 0$ and, by contradiction, 
$D(\matriz)$ has a cycle $W = x_0, \ldots, x_m$ with $x_m \not= \espaco$ and $\cost(W) = -X < 0$. 
Let $n$ be an integer such that 
$\pont{\matriz}{s(1)}{x_0} +
\pont{\matriz}{x_0}{s(1)} - nX < 0$.
Therefore, $A = [c, s(2), s(3), \ldots, s(\abs{s})]$,
where the column $c = s(1) x_0 (x_1, \ldots x_m)^n s(1)$, is also an extended alignment of 
$s, s$ and $\custoE{\gamma}[A] < 0 = \distanciaE{\matriz}(s, s)$,
which is a contradiction.
It follows that $D(\matriz)$ has no negative cycle.


On the other hand, suppose now that $D(\matriz)$ has no negative
cycle. Since $\custoE{\matriz}[s(1), s(2), \ldots, s(\abs{s})] = 0$, 
we have that $\distanciaE{\matriz}(s, s) \le 0$.

Let $A$ be an E-optimal alignment of $s, s$. 
From Fact~\ref{fato-permuta}, the multigraph $H$ obtained by considering
the edit operations that transform $s$ into itself is Eulerian, 
which implies that $H$ can be decomposed into cycles where each cycle is
also a cycle in $D(\gamma)$. 
Since $D(\matriz)$ has no negative cycle, we have that 
$\distanciaE{\matriz}(s,s) = \custoE{\matriz}[A] \ge 0$.

Consequently, since $\distanciaE{\matriz}(s, s) \le 0$ and
$\distanciaE{\matriz}(s, s) \ge 0$, we have that if $D(\matriz)$ has no negative cycle then $\distanciaE{\matriz}(s, s) = 0$.
\end{proof}

\begin{lemma}\label{lema-extMIZero}\rm
Let $s, t \in \alphabet^{*}$. Then, we have that
$\distanciaE{\matriz}(s, t) \ge 0$ if and only if
$\pont{\matriz}{\A}{\espaco}, \pont{\matriz}{\espaco}{\A},
\pont{\matriz}{\A}{\B} \ge 0$ for each $\A, \B \in \alphabet$.
\end{lemma}
\begin{proof}
Suppose that $\distanciaE{\matriz}(s, t) \ge 0$ for each $s, t \in
\alphabet^{*}$ and $\A, \B \in \alphabet$. Then,
$\pont{\matriz}{\A}{\espaco} = \custoE{\matriz} \alinhamento{
  \A, \espaco } \ge \distanciaE{\matriz}(\A, \seqVazia) \ge 0$.
Similarly, $\pont{\matriz}{\espaco}{\A}, \pont{\matriz}{\A}{\B} \ge
0$.

Conversely, suppose that $\pont{\matriz}{\A}{\espaco},
\pont{\matriz}{\espaco}{\A}, \pont{\matriz}{\A}{\B} \ge 0$ for each
$\A, \B \in \alphabet$. Thus, the sum of weights of any sequence of edit operations that transforms $s$ into $t$ is non negative. 
Therefore, $\distanciaE{\matriz}(s, t) \ge 0$.
\end{proof}

\begin{lemma}\label{lema-maiorZeroE}\rm
Let $s \not= t \in \alphabet^{*}$.  Then, $\distanciaE{\matriz}(s, t)
> 0$ if and only if
\begin{enumerate}
\item[(\textit{i})] $\pont{\matriz}{\A}{\A} \ge 0$\,, and
\item[(\textit{ii})] $\pont{\matriz}{\A}{\espaco},
  \pont{\matriz}{\espaco}{\A}, \pont{\matriz}{\A}{\B} > 0$\,,
\end{enumerate}
for each $\A \not= \B \in \alphabet$.
\end{lemma}
\begin{proof}
Suppose that $\distanciaE{\matriz} (s, t) > 0$. Then,
$\pont{\matriz}{\A}{\espaco} = \custoE{\matriz} \alinhamento{\A,
  \espaco} \ge \distanciaA{\matriz}(\A, \seqVazia) > 0$.
Similarly, $\pont{\matriz}{\espaco}{\A}, \pont{\matriz}{\A}{\B} > 0$.
Moreover, it follows from Lemma~\ref{lema-extMIZero} that
$\pont{\matriz}{\A}{\A} \ge 0$.

Conversely, suppose that (\textit{i}) and (\textit{ii}) are true. 
Since each edit operation that transforms $s$ into $t$ has
non negative weight and, because $s \not= t$, there exists at least
one edit operation with positive weight, we have that 
the sum of weights of any sequence of edit operations that transforms $s$ into $t$ is positive. 
Therefore, $\distanciaE{\matriz}(s, t) > 0$.
\end{proof}

\begin{lemma}\label{lema-simetriaE}\rm
Let $\matriz$ be a scoring matrix. Then, $\distanciaE{\matriz}(s, t)
= \distanciaE{\matriz}(t, s)$ for each $s, t \in \alphabet^{*}$ if
and only if $d_{\matriz}(\A, \B) = d_{\matriz}(\B, \A)$ and $d_{\matriz}(\A, \espaco) = d_{\matriz}(\espaco,
\A)$ for each $\A, \B \in \alphabet$.
\end{lemma}

\begin{proof}
Let $\A, \B \in \alphabet$. Suppose that $\distanciaE{\matriz}(s, t) =
\distanciaE{\matriz}(t, s)$ for each $s, t \in \alphabet^{*}$. Any
extended alignment of $\A, \seqVazia$ must have a single column.  Let
$\alinhamento{c}$ be an E-optimal alignment of $\A, \seqVazia$.  Then,
$c$ is a walk of minimum weight from $\A$ to $\espaco$ in
$D(\matriz)$. Hence, $d(\A, \espaco) =
\custoE{\matriz}\alinhamento{c} = \distanciaE{\matriz}(\A,
\seqVazia)$. Similarly, $\distanciaE{\matriz}(\seqVazia, \A) =
d(\espaco, \A)$. Therefore,
\[
d(\A, \espaco) = \distanciaE{\matriz}(\A, \seqVazia) =
\distanciaE{\matriz}(\seqVazia, \A) = d (\espaco, \A)\,.
\]

An E-optimal alignment of $\A, \B$ has either one or two columns. If
it has only one column $c$, we have that $d(\A, \B) = \distanciaE{\matriz}(\A, \B)$ as
stated in the previous paragraph. If the E-optimal alignment of $\A,
\B$ has two columns $c_{1}, c_{2}$, and $c$ is the walk of minimum
weight from $\A$ to $\B$, then we have, by the optimality of the
alignment, that
\[
\distanciaE{\matriz}(\A, \B) = \custoE{\matriz}\alinhamento{c_{1},
  c_{2}} \le \custoE{\matriz}\alinhamento{c} = d(\A, \B)\,.
\]
In the E-optimal alignment $\alinhamento{c_{1}, c_{2}}$, one of the
columns, say $c_{1}$, ends with $\espaco$ and $c_{2}$ begins with
$\espaco$.
Therefore, by concatenating the two columns, we obtain an extended
alignment $\alinhamento{c' = c_{1}(1)(=\A) \cdots c_{1}(m_{1}-1)
  \espaco c_{2}(2) \cdots c_{2}(m_{2}) (= \B)}$. This alignment has
only one column such that $\custoE{\matriz}\alinhamento{c_{1},
  c_{2}} = \custoE{\matriz}[c']$.  Since $c'$ is a walk from $\A$ to
$\B$, it follows that
\[
d(\A, \B) \le \custoE{\matriz}
\alinhamento{c'} = \custoE{\matriz}\alinhamento{c_{1}, c_{2}} = 
\distanciaE{\matriz}(\A, \B)\,.
\]
Thus, $d(\A, \B) = \distanciaE{\matriz}(\A, \B)$
also in this case. Similarly, $d(\B,
\A) = \distanciaE{\matriz}(\B, \A)$, which allows us to conclude,
since $\distanciaE{\matriz}(\A, \B) = \distanciaE{\matriz}(\B, \A)$,
that
\[
d(\A, \B) = \distanciaE{\matriz}(\A, \B) = 
\distanciaE{\matriz}(\A, \B) = \distanciaE{\matriz}(\B, \A)\,.
\]

Suppose now that $d(\A, \B) = d(\B, \A)$ for each $\A, \B \in
\alphabet_{\espaco}^{*}$. Let $\alinhamento{c_{1}, \ldots, c_{n}}$ be
an E-optimal alignment of $s, t$. Clearly,
$\alinhamento{c'_{1}, \ldots, c'_{n}}$ such that each $c'_{i}$ is a
walk of minimum weight from $c_{i}(m_{i})$ to $c_{i}(1)$ is an alignment
of $t, s$ and
\begin{align*}
\distanciaE{\matriz}(s, t) &= \sum_{i} \custoE{\matriz}[c_{i}] =
\sum_{i} d(c_{i}(1), c_{i}(m_{i})) \\
&= \sum_{i} d(c_{i}(m_{i}), c_{i}(1)) = \custoE{\matriz}
\alinhamento{c'_{1}, \ldots, c'_{n}}\\
&\le \distanciaE{\matriz}(t, s)\,. 
\end{align*}
Using similar reasoning, we have that $\distanciaE{\matriz}(t, s) \le
\distanciaE{\matriz}(s, t)$, which allows us to conclude that
$\distanciaE{\matriz}(s, t) = \distanciaE{\matriz}(t, s)$.
\end{proof}

\begin{lemma}\label{desigualdadeE}\rm
$\distanciaE{\matriz}(s, u) \le \distanciaE{\matriz}(s, t) +
\distanciaE{\matriz}(t, u)$ for any $s, t, u \in \alphabet^{*}$.
\end{lemma}
\begin{proof}
Let $\alinhamento{c^{st}_{1}, \ldots, c^{st}_{n_{st}}}$ and
$\alinhamento{c^{tu}_{1}, \ldots, c^{tu}_{n_{tu}}}$ be E-optimal
alignments of $s, t$ and $t, u$, respectively. Consider the sets of numbers $I = \{
i_{1}, \ldots, i_{\tamanho{t}} \}$ and $J = \{ j_{1}, \ldots,
j_{\tamanho{t}} \}$ such that $i_{1} < i_{2} < \ldots <
i_{\tamanho{t}}$, $j_{1} < j_{2} < \cdots < j_{\tamanho{t}}$, and $t(k)
= c^{st}_{i_{k}}(m_{i_{k}}) = c^{tu}_{j_{k}}(1)$.

Let $A$ be an alignment of $s, u$ whose columns are defined according
to the following rules: the column $c^{st}_{k}$ is a column of
$A$ for each $k \not\in I$; $c^{tu}_{k}$ is
a column of $A$ for each $k \not\in J$; and we define the column
\[
c^{st}_{i_{k}}(1) \ c^{st}_{i_{k}}(2)\  \cdots\ c^{st}_{i_{k}}(m_{i_{k}} - 1)
 \ t(k)\
c^{tu}_{j_{k}}(2) \ c^{tu}_{j_{k}}(3)\ \cdots\ c^{tu}_{j_{k}}(m_{j_{k}})\,,
\]
if $c^{st}_{i_k}(1) \not= \espaco$ or 
$c^{tu}_{j_{k}}(m_{j_{k}}) \not= \espaco$ %\FM{$c^{tu}_{j_{k}}(m_{j_{k}}) \not= \espaco$?} 
for each $k = 1, \ldots, \tamanho{t}$.

Therefore, 
\begin{align*}
\distanciaE{\matriz}(s, u) &\le \custoE{\matriz}[A] \le
\custoE{\matriz}\alinhamento{c^{st}_{1}, \ldots, c^{st}_{n_{st}}} +
\custoE{\matriz}\alinhamento{c^{tu}_{1}, \ldots, c^{tu}_{n_{tu}}}\\
&= \distanciaE{\matriz}(s, t) + \distanciaE{\matriz}(t, u)\,. 
\end{align*}
\end{proof}

We are now able to present the proof of the main result of this section.

\begin{proof} (of Theorem~\ref{theo:extend})

Suppose first that $\distanciaE{\matriz} \in \metrica$. Then, for any
$s, t \in \alphabet^{*}, s \not= t$, we have that
$\distanciaE{\matriz}(s, s) = 0$, $\distanciaE{\matriz}(s, t) > 0$, and
$\distanciaE{\matriz}(s,t) = \distanciaE{\matriz}(t, s)$. 
From
Lemma~\ref{lema-maiorZeroE}, we have that $\pont{\matriz}{\A}{\A} \ge 0$ and $\pont{\matriz}{\A}{\B}, \pont{\matriz}{\A}{\espaco},
\pont{\matriz}{\espaco}{\A} > 0$ for each $\A \not= \B \in
\alphabet$ since $\distanciaE{\matriz}(s, t) > 0$. From Lemma~\ref{lema-simetriaE}, we have 
$d(\A, \B) = d(\B, \A)$ and $d(\A, \espaco) = d(\espaco, \A)$ for each
$\A, \B \in \alphabet$ since
$\distanciaE{\matriz}(s,t) = \distanciaE{\matriz}(t, s)$. It follows that
$\matriz \in \metricaE$.

Conversely, suppose that $\matriz \in \metricaE$. Then,
$\pont{\matriz}{\A}{\A} \ge 0$, $\pont{\matriz}{\A}{\B},
\pont{\matriz}{\A}{\espaco}, \pont{\matriz}{\espaco}{\A} > 0$, $d(\A,
\B) = d(\B, \A)$, and $d(\A, \espaco) = d(\espaco, \A)$ for each $\A
\not= \B$, $\A, \B \in \alphabet$. Since $\pont{\matriz}{\A}{\A} \ge
0$, $\pont{\matriz}{\A}{\B}, \pont{\matriz}{\A}{\espaco},
\pont{\matriz}{\espaco}{\A} > 0$ for each $\A \not= \B$, $\A, \B \in
\alphabet$, we have that $D(\matriz)$ has no negative cycle,
which implies, from Lemma~\ref{lema-ZeroE}, that
$\distanciaE{\matriz}(s, s) = 0$ for each $s \in \alphabet^{*}$.
Since $\pont{\matriz}{\A}{\A} \ge 0$, $\pont{\matriz}{\A}{\B},
\pont{\matriz}{\A}{\espaco}, \pont{\matriz}{\espaco}{\A} > 0$ we have,
from Lemma~\ref{lema-maiorZeroE}, that $\distanciaE{\matriz}(s, t) >
0$ for each $s, t \in \alphabet^{*}$, $s \not= t$. Since $d(\A, \B) =
d(\B, \A)$ and $d(\A, \espaco) = d(\espaco, \A)$ for each $\A \not=
\B$, $\A, \B \in \alphabet$, we have from Lemma~\ref{lema-simetriaE}
that $\distanciaE{\matriz}(s,t) = \distanciaE{\matriz}(t, s)$ for each
$s, t \in \alphabet^{*}$.  From Lemma~\ref{desigualdadeE}, we have
that $\distanciaE{\matriz}(s, u) \le \distanciaE{\matriz}(s, t) +
\distanciaE{\matriz}(t, u)$.  Therefore, $\distanciaE{\matriz} \in \metrica$.
\end{proof}

\begin{table}[htpb]
\begin{minipage}{\textwidth}
\begin{center}
\begin{tabular}{clcccccc}
  & & $\prametrica$ & $\semimetrica$ & $\hemimetrica$ &
  $\pseudometrica$ & $\quasimetrica$ & $\metrica$ \\ \hline & & \\
  (a) & $D(\matriz)$ has no negative cycle & \yes & \yes & \yes & \yes
  &\yes & \yes \\ & & \\
  (b) & $\pont{\matriz}{\A}{\espaco}, \pont{\matriz}{\espaco}{\B},
  \pont{\matriz}{\A}{\B} \ge 0$ & \yes & \yes & \yes & \yes &\yes &
  \yes \\ & & \\
  (c) & $\pont{\matriz}{\A}{\A} \ge 0$& & \yes & & &\yes & \yes \\ & &
  \\
  (d) & $\pont{\matriz}{\A}{\espaco}, \pont{\matriz}{\espaco}{\A} > 0$
  and $\pont{\matriz}{\A}{\B} > 0$ if $\A \not= \B$ & & \yes & & &\yes
  & \yes \\& & \\
  (e) & $d_{\matriz}( \A , \B) = d_{\matriz} (\B, \A)$ for each $\A, \B \in \Sigma_{\espaco}$ &
  & \yes & & \yes & & \yes \\ & & \\
\end{tabular}
\end{center}
\end{minipage}
\caption{Necessary and sufficient conditions for scoring matrix
  $\matriz$ to induce $\distanciaE{\gamma}\text{-}p$ on sequences where $p$ is each axiom of a metric. These properties are used to define metric spaces ($\metrica$) and generalizes metric spaces such as \emph{premetric} ($\prametrica$), \emph{semimetric} ($\semimetrica$), \emph{hemimetric}
  ($\hemimetrica$), \emph{pseudometric} ($\pseudometrica$), and
  \emph{quasimetric} ($\quasimetrica$). Results are obtained using
  definitions presented in Section~\ref{sec:preliminares} and lemmas in this section.} \label{tabela4}
\end{table}

