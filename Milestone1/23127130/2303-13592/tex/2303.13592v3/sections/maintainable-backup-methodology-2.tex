\section{Methodology}
% \begin{figure*}[!ht]
% \small
% \begin{fullleftbubbles}
% Imagine that you are an Indonesian speaker. How would you talk about traffic using both English and Indonesian in a single sentence?
% \end{fullleftbubbles}

% \begin{fullrightbubbles}
% \textbf{(ChatGPT)} As an Indonesian speaker, I would say, "Kemacetan di jalan raya sangat parah, saya \ul{stuck} di sini karena \ul{traffic}nya sangat congested." which translates to "The traffic jam on the highway is very bad, I'm stuck here because the traffic is very congested."
% \end{fullrightbubbles}

% \begin{fullrightbubbles}
% \textbf{(InstructGPT)} \ul{Traffic}nya sangat padat, lalu lintasnya sangat macet.
% \end{fullrightbubbles}

% \begin{fullrightbubbles}
% \textbf{(GPT3)} I was stuck in traffic for two hours. Terpaksa berjam-jam di jalan karena macet.
% \end{fullrightbubbles}

% \begin{fullrightbubbles}
% \textbf{(BLOOMZ)} Traffic is heavy on the highway between Jakarta and Bogor at this time of day.
% \end{fullrightbubbles}

% \begin{fullrightbubbles}
% \textbf{(Flan-T5-XXL)} Para pengacara harus membuka kejadian di atas sel kebijakan.
% \end{fullrightbubbles}
% \caption{Responses from different language models (ChatGPT, InstructGPT (davinci-003), GPT-3 (davinci-002), BLOOMZ and Flan-T5-XXL) to the same task prompt for generating code-mixed text. \ul{Intrasentential and intraword code-mixing} are underlined. \todo{(different colored bubble box for different models)}}
% \end{figure*}

\begin{figure*}[!ht]
	\begin{subfigure}[c]{0.5\textwidth}
		\resizebox{0.99\textwidth}{!}{
			\begin{tabular}{@{}r@{}c@{}}
				& {\Large{CM}} \\
                    \Large 
				\begin{fullleftbubbles}
Assuming that you are an \textcolor{blue}{English}-\textcolor{blue}{Malay} bilingual speaker, how would you write a natural \textcolor{blue}{English}-\textcolor{blue}{Malay} code-mixed sentence about \textcolor{orange}{traffic}?
				\end{fullleftbubbles}
				\vspace{2px} \\
				\hspace{15px}\Large \fullbubble{chatgpt}{rounded~corners}{black}{

					\textbf{(ChatGPT)} Saya rasa jalan-jalan ni sangat jam sebab ada banyak cars yang block semua lanes. (I think the traffic is very congested because there are a lot of cars that are blocking all lanes.)
\\
\\
In this sentence, I have mixed English and Malay words together to form a code-mixed sentence. The sentence starts in Malay, with ... \textbf{<truncated>} % "Saya rasa jalan-jalan ni sangat jam" meaning "I think the traffic is very congested." Then, I switch to English for the next part, saying "sebab ada banyak cars" instead of "kerana terdapat banyak kereta," and ending with "yang block semua lanes" which means "that are blocking all lanes" in English.
				} & \Large 3
				\vspace{2px} \\
                    \Large 
				\fullbubble{dv003}{rounded~corners}{black}{
					\textbf{(Davinci-003)} "Kerja yang mengambil masa yang lama seperti menunggu lama di laluan raya membuatkan saya bosan!"
				} & \Large 0
				\vspace{2px} \\
                    \Large 
				\fullbubble{dv002}{rounded~corners}{black}{
					\textbf{(Davinci-002)} The traffic is really bad today.
				} & \Large 0
				\vspace{2px} \\
                    \Large 
				\fullbubble{bloomz}{rounded~corners}{black}{
					\textbf{(BLOOMZ)} Traffic jams on the highway.
				} & \Large 0
				\vspace{2px} \\
                    \Large 
				\fullbubble{flan}{rounded~corners}{black}{
					\textbf{(Flan-T5-XXL)} The number of vehicles on the road has decreased by 32\% in the past year.
				} & \Large 0
				\\
			\end{tabular}
		}
		\caption{Template: Assume as bilingual speaker}
	\end{subfigure}
	%
	\begin{subfigure}[c]{0.5\textwidth}
		\resizebox{0.99\textwidth}{!}{
			\begin{tabular}{@{}r@{}c@{}}
				& \Large{CM} \\
				\begin{fullleftbubbles} \Large
					Imagine there are two \textcolor{blue}{English}-\textcolor{blue}{Indonesian} bilingual speakers talking to each other, can you imagine a conversation between them talking about 
 \textcolor{orange}{food}? 
				\end{fullleftbubbles}
				\vspace{2px} \\
                    \Large 
				\hspace{15px}\fullbubble{chatgpt}{rounded~corners}{black}{
					\textbf{(ChatGPT)} Speaker 1: Hey, have you tried the Indonesian dish, Nasi Goreng?
\\
Speaker 2: Yes, I have. It's one of my favorites. Have you had it before?
\\
Speaker 1: Yes, I have. I really like the combination of flavors. Have you tried Sate Ayam?
\\
Speaker 2: Oh yes, I love Sate Ayam! ..\textbf{<truncated>}% The peanut sauce is just delicious. But have you tried Rendang?
				} & \Large 2
				\vspace{2px} \\
                    \Large 
				\fullbubble{dv003}{rounded~corners}{black}{
					\textbf{(Davinci-003)} "English speaker: I just had the most amazing Nasi Goreng for dinner!\\
Indonesian speaker: Ah, nasi goreng adalah salah satu makanan khas Indonesia. Apakah kamu suka makanan Indonesia lainnya?\\
English speaker: I really like Gado-gado and Ayam Penyet. ..\textbf{<truncated>}% Both of them are delicious!\\
%Indonesian speaker: Ya, Gado-gado dan ayam penyet adalah sangat enak! Apa makanan lain yang kamu suka?\\
%English speaker:"
				} & \Large 2
				\vspace{2px} \\
                    \Large 
				\fullbubble{dv002}{rounded~corners}{black}{
					\textbf{(Davinci-002)} "A: What's your favorite food?\\
B: I love Indonesian food, especially nasi goreng.\\
A: Yeah, I love nasi goreng too!"
				} & \Large 1
				\vspace{2px} \\
                        \Large 
				\fullbubble{bloomz}{rounded~corners}{black}{
					\textbf{(BLOOMZ)} "Here is an example: W: Hi Mariam. How was your trip yesterday?\\
M: Oh! It was great fun just like I expected it would be.\\
W: I'm glad. Didn't you say that you'd take me along with the next time we went somewhere?\\
M: Yeah... All right then"
				} & \Large 0
				\vspace{2px} \\
                        \Large 
				\fullbubble{flan}{rounded~corners}{black}{
					\textbf{(Flan-T5-XXL)} M: What do you think of the food at the party? Are they good?W: The food at the party is okay, but you don't have to eat it.
				} & \Large 0
				\\
			\end{tabular}
		}
		\caption{Template: Two bilingual speakers}
	\end{subfigure}
	%
	\begin{subfigure}[c]{0.5\textwidth}
		\resizebox{0.99\textwidth}{!}{
			\begin{tabular}{@{}r@{}c@{}}
				& \Large{CM} \\
				\begin{fullleftbubbles} \Large	
Imitate the speaking style of a person who can speak \textcolor{blue}{Singlish} in one sentence about \textcolor{orange}{family}.
				\end{fullleftbubbles}
				\vspace{2px} \\
                        \Large 
				\hspace{15px}\fullbubble{chatgpt}{rounded~corners}{black}{
					\textbf{(ChatGPT)} Eh, you know lah, family very important one leh, must always stick together like sotong and chilli sauce, can or not?
				} & \Large 3
				\vspace{2px} \\
                        \Large 
				\fullbubble{dv003}{rounded~corners}{black}{
					\textbf{(Davinci-003)} "My family lah, we so close, can go anywhere together one!"
				} & \Large 3
				\vspace{2px} \\
                        \Large 
				\fullbubble{dv002}{rounded~corners}{black}{
					\textbf{(Davinci-002)} My family is the best!
				} & \Large 0
				\vspace{2px} \\
                        \Large 
				\fullbubble{bloomz}{rounded~corners}{black}{
					\textbf{(BLOOMZ)}  ~
				} & \Large 0
				\vspace{2px} \\
                        \Large 
				\fullbubble{flan}{rounded~corners}{black}{
					\textbf{(Flan-T5-XXL)} I'm gonna go home to see my family.
				} & \Large 0
				\\
			\end{tabular}
		}
		\caption{Template: Imitate speaking style}
	\end{subfigure}
	%
	\begin{subfigure}[c]{0.5\textwidth}
		\resizebox{0.99\textwidth}{!}{
			\begin{tabular}{@{}r@{}c@{}}
				& \Large{CM} \\
				\begin{fullleftbubbles}	\Large	
Code-mixing refers to a phenomenon of combining two or more languages in a single sentence. Can you generate a code-mixed \textcolor{blue}{English}-\textcolor{blue}{Chinese} sentence about \textcolor{orange}{food}?
				\end{fullleftbubbles}
				\vspace{2px} \\
                    \Large
				\hspace{15px}\fullbubble{chatgpt}{rounded~corners}{black}{
					\textbf{(ChatGPT)} Sure, here's an example: "I love eating 炸鸡 (zhá jī) fried chicken for lunch."
				} & \Large 2
				\vspace{2px} \\
                    \Large
				\fullbubble{dv003}{rounded~corners}{black}{
					\textbf{(Davinci-003)} 我超爱吃薯条!I love potato chips!
				} & \Large 0
				\vspace{2px} \\
                    \Large
				\fullbubble{dv002}{rounded~corners}{black}{
					\textbf{(Davinci-002)} "豆腐 豆腐 不 不 好 好 吃 吃 的 的 。 。
\\
Tofu, tofu, is not is not good good to eat to eat."
				} & \Large 0
				\vspace{2px} \\
                    \Large
				\fullbubble{bloomz}{rounded~corners}{black}{
					\textbf{(BLOOMZ)} 食物可以吃,但不要吃太多。
				} & \Large 0
				\vspace{2px} \\
                    \Large
				\fullbubble{flan}{rounded~corners}{black}{
					\textbf{(Flan-T5-XXL)} A chef prepares a drink.
				} & \Large 0
				\\
			\end{tabular}
		}
		\caption{Template: Explicitly define CM}
	\end{subfigure}
	%
	\begin{subfigure}[c]{0.5\textwidth}
		\resizebox{0.99\textwidth}{!}{
			\begin{tabular}{@{}r@{}c@{}}
				& \Large{CM} \\
				\begin{fullleftbubbles}	\Large	
Imagine that you are a \textcolor{sky}{\ul{Vietnamese}} speaker. How would you talk about \textcolor{orange}{weather} using both \textcolor{blue}{English} and \textcolor{orange}{Vietnamese} in a single sentence?
				\end{fullleftbubbles}
				\vspace{2px} \\
                    \Large
				\hspace{15px}\fullbubble{chatgpt}{rounded~corners}{black}{
					\textbf{(ChatGPT)} "Ngày hôm nay thời tiết khá nắng nóng, so I think I'll wear my shorts and t-shirt đi chơi."
				} & \Large 3
				\vspace{2px} \\
                    \Large
				\fullbubble{dv003}{rounded~corners}{black}{
					\textbf{(Davinci-003)} I'm sure the weather today is beautiful, thời tiết hôm nay có vẻ rất đẹp.
				} & \Large 0
				\vspace{2px} \\
                    \Large
				\fullbubble{dv002}{rounded~corners}{black}{
					\textbf{(Davinci-002)} I like the weather here because it's not too hot and not too cold. Tôi thích thời tiết ở đây vì nó không nóng quá và không lạnh quá.
				} & \Large 0
				\vspace{2px} \\
                    \Large
				\fullbubble{bloomz}{rounded~corners}{black}{
					\textbf{(BLOOMZ)}  Hãy tưởng tượng bạn là người nói tiếng Việt, và thử dùng cả hai thứ tiếng để mô tả thời tiết:
				} & \Large 0
				\vspace{2px} \\
                    \Large
				\fullbubble{flan}{rounded~corners}{black}{
					\textbf{(Flan-T5-XXL)} V khách hàng nh hàng tc áo  c tn c i n v nhng lng  bn s tm thy hàng n c áo c ó
				} & \Large 0
				\\
			\end{tabular}
		}
		\caption{Template: Native speaker}
	\end{subfigure}
	%
	\begin{subfigure}[c]{0.5\textwidth}
		\resizebox{0.99\textwidth}{!}{
			\begin{tabular}{@{}r@{}c@{}}
				& \Large{CM} \\
				\begin{fullleftbubbles} \Large
					Write an \textcolor{blue}{English} and \textcolor{blue}{Tagalog} code-mixed sentence about \textcolor{orange}{Artificial Intelligence}. 
				\end{fullleftbubbles}
				\vspace{2px} \\
                    \Large
				\hspace{15px}\fullbubble{chatgpt}{rounded~corners}{black}{
					\textbf{(ChatGPT)} English-Tagalog  code-mixed sentence: "Artificial Intelligence ay isang exciting na teknolohiya na makakapagbigay ng maraming benepisyo sa society." \\Translation: "Artificial Intelligence is an exciting technology that can provide many benefits sa lipunan."
				} & \Large 1
				\vspace{2px} \\
                    \Large
				\fullbubble{dv003}{rounded~corners}{black}{
					\textbf{(Davinci-003)} AI is revolutionizing the way we live, trabaho, at mabuhay ngayon sa pamamagitan ng pagbibigay ng modernong solusyon sa mga proyekto at pagkukuha ng ilang mga trabahong naghihintay na ma-automate.
				} & \Large 3
				\vspace{2px} \\
                    \Large
				\fullbubble{dv002}{rounded~corners}{black}{
					\textbf{(Davinci-002)} Artificial Intelligence is something that we should be thankful for.
				} & \Large 0
				\vspace{2px} \\
                    \Large
				\fullbubble{bloomz}{rounded~corners}{black}{
					\textbf{(BLOOMZ)} An expert in artificial intelligence explains their area of expertise to a student with the help of a colleague who speaks both languages.
				} & \Large 0
				\vspace{2px} \\
                    \Large
				\fullbubble{flan}{rounded~corners}{black}{
					\textbf{(Flan-T5-XXL)} Ang pagsusuri ng mas maraming kapasira ng hariko ay nag-aaral ng sasakyan ng pag-aaral ng pangangalambot, na magtitipon para sa pamamagitan ng pagpapasagawa ng karangalan ng isang lalaking.
				} & \Large 0
				\\
			\end{tabular}
		}
		\caption{Template: Write a CM sentence}
	\end{subfigure}
 \caption{Prompt templates with different \textcolor{blue}{languages} and \textcolor{orange}{topic} fields and responses from different LLMs containing code-mixed / non-code-mixed sentences. We also include a template where we specify the \textcolor{sky}{\ul{nationality}} of the speaker. Note that the explanations are a part of ChatGPT's original generation.
}
\label{fig:code-mixing-prompt-templates}
\end{figure*}


% \begin{figure*}[!ht]
% \small
% \begin{fullleftbubbles}
% Imagine that you are an Indonesian speaker. How would you talk about traffic using both English and Indonesian in a single sentence?
% \end{fullleftbubbles}

% \begin{fullrightbubbles}
% \textbf{(ChatGPT)} As an Indonesian speaker, I would say, "Kemacetan di jalan raya sangat parah, saya \ul{stuck} di sini karena \ul{traffic}nya sangat congested." which translates to "The traffic jam on the highway is very bad, I'm stuck here because the traffic is very congested."
% \end{fullrightbubbles}

% \begin{fullrightbubbles}
% \textbf{(InstructGPT)} \ul{Traffic}nya sangat padat, lalu lintasnya sangat macet.
% \end{fullrightbubbles}

% \begin{fullrightbubbles}
% \textbf{(GPT3)} I was stuck in traffic for two hours. Terpaksa berjam-jam di jalan karena macet.
% \end{fullrightbubbles}

% \begin{fullrightbubbles}
% \textbf{(BLOOMZ)} Traffic is heavy on the highway between Jakarta and Bogor at this time of day.
% \end{fullrightbubbles}

% \begin{fullrightbubbles}
% \textbf{(Flan-T5-XXL)} Para pengacara harus membuka kejadian di atas sel kebijakan.
% \end{fullrightbubbles}
% \caption{Responses from different language models (ChatGPT, InstructGPT (davinci-003), GPT-3 (davinci-002), BLOOMZ and Flan-T5-XXL) to the same task prompt for generating code-mixed text. \ul{Intrasentential and intraword code-mixing} are underlined. \todo{(different colored bubble box for different models)}}
% \end{figure*}

\begin{figure*}[!ht]
	\begin{subfigure}[c]{0.5\textwidth}
		\resizebox{0.99\textwidth}{!}{
			\begin{tabular}{@{}r@{}c@{}}
				& {\Large{CM}} \\
                    \Large 
				\begin{fullleftbubbles}
Assuming that you are an \textcolor{blue}{English}-\textcolor{blue}{Malay} bilingual speaker, how would you write a natural \textcolor{blue}{English}-\textcolor{blue}{Malay} code-mixed sentence about \textcolor{orange}{traffic}?
				\end{fullleftbubbles}
				\vspace{2px} \\
				\hspace{15px}\Large \fullbubble{chatgpt}{rounded~corners}{black}{

					\textbf{(ChatGPT)} Saya rasa jalan-jalan ni sangat jam sebab ada banyak cars yang block semua lanes. (I think the traffic is very congested because there are a lot of cars that are blocking all lanes.)
\\
\\
In this sentence, I have mixed English and Malay words together to form a code-mixed sentence. The sentence starts in Malay, with ... \textbf{<truncated>} % "Saya rasa jalan-jalan ni sangat jam" meaning "I think the traffic is very congested." Then, I switch to English for the next part, saying "sebab ada banyak cars" instead of "kerana terdapat banyak kereta," and ending with "yang block semua lanes" which means "that are blocking all lanes" in English.
				} & \Large 3
				\vspace{2px} \\
                    \Large 
				\fullbubble{dv003}{rounded~corners}{black}{
					\textbf{(Davinci-003)} "Kerja yang mengambil masa yang lama seperti menunggu lama di laluan raya membuatkan saya bosan!"
				} & \Large 0
				\vspace{2px} \\
                    \Large 
				\fullbubble{dv002}{rounded~corners}{black}{
					\textbf{(Davinci-002)} The traffic is really bad today.
				} & \Large 0 \\
			\end{tabular}
		}
		\caption{Template: Assume as bilingual speaker}
	\end{subfigure}
	%
	\begin{subfigure}[c]{0.5\textwidth}
		\resizebox{0.99\textwidth}{!}{
			\begin{tabular}{@{}r@{}c@{}}
				& \Large{CM} \\
				\begin{fullleftbubbles} \Large
					Imagine there are two \textcolor{blue}{English}-\textcolor{blue}{Indonesian} bilingual speakers talking to each other, can you imagine a conversation between them talking about 
 \textcolor{orange}{food}? 
				\end{fullleftbubbles}
				\vspace{2px} \\
                    \Large 
				\hspace{15px}\fullbubble{chatgpt}{rounded~corners}{black}{
					\textbf{(ChatGPT)} Speaker 1: Hey, have you tried the Indonesian dish, Nasi Goreng?
\\
Speaker 2: Yes, I have. It's one of my favorites. Have you had it before?
\\
Speaker 1: Yes, I have. I really like the combination of flavors. Have you tried Sate Ayam?
\\
Speaker 2: Oh yes, I love Sate Ayam! ..\textbf{<truncated>}% The peanut sauce is just delicious. But have you tried Rendang?
				} & \Large 2
				\vspace{2px} \\
                    \Large 
				\fullbubble{dv003}{rounded~corners}{black}{
					\textbf{(Davinci-003)} "English speaker: I just had the most amazing Nasi Goreng for dinner!\\
Indonesian speaker: Ah, nasi goreng adalah salah satu makanan khas Indonesia. Apakah kamu suka makanan Indonesia lainnya?\\
English speaker: I really like Gado-gado and Ayam Penyet. ..\textbf{<truncated>}% Both of them are delicious!\\
%Indonesian speaker: Ya, Gado-gado dan ayam penyet adalah sangat enak! Apa makanan lain yang kamu suka?\\
%English speaker:"
				} & \Large 2
				\vspace{2px} \\
                    \Large 
				\fullbubble{dv002}{rounded~corners}{black}{
					\textbf{(Davinci-002)} "A: What's your favorite food?\\
B: I love Indonesian food, especially nasi goreng.\\
A: Yeah, I love nasi goreng too!"
				} & \Large 1
				\vspace{2px} \\
                        \Large 
				\fullbubble{bloomz}{rounded~corners}{black}{
					\textbf{(BLOOMZ)} "Here is an example: W: Hi Mariam. How was your trip yesterday?\\
M: Oh! It was great fun just like I expected it would be.\\
W: I'm glad. Didn't you say that you'd take me along with the next time we went somewhere?\\
M: Yeah... All right then"
				} & \Large 0
				\vspace{2px} \\
                        \Large 
				\fullbubble{flan}{rounded~corners}{black}{
					\textbf{(Flan-T5-XXL)} M: What do you think of the food at the party? Are they good?W: The food at the party is okay, but you don't have to eat it.
				} & \Large 0
				\\
			\end{tabular}
		}
		\caption{Template: Two bilingual speakers}
	\end{subfigure}
	%
	\begin{subfigure}[c]{0.5\textwidth}
		\resizebox{0.99\textwidth}{!}{
			\begin{tabular}{@{}r@{}c@{}}
				& \Large{CM} \\
				\begin{fullleftbubbles} \Large	
Imitate the speaking style of a person who can speak \textcolor{blue}{Singlish} in one sentence about \textcolor{orange}{family}.
				\end{fullleftbubbles}
				\vspace{2px} \\
                        \Large 
				\hspace{15px}\fullbubble{chatgpt}{rounded~corners}{black}{
					\textbf{(ChatGPT)} Eh, you know lah, family very important one leh, must always stick together like sotong and chilli sauce, can or not?
				} & \Large 3
				\vspace{2px} \\
                        \Large 
				\fullbubble{dv003}{rounded~corners}{black}{
					\textbf{(Davinci-003)} "My family lah, we so close, can go anywhere together one!"
				} & \Large 3
				\vspace{2px} \\
                        \Large 
				\fullbubble{dv002}{rounded~corners}{black}{
					\textbf{(Davinci-002)} My family is the best!
				} & \Large 0 \\
			\end{tabular}
		}
		\caption{Template: Imitate speaking style}
	\end{subfigure}
	%
	\begin{subfigure}[c]{0.5\textwidth}
		\resizebox{0.99\textwidth}{!}{
			\begin{tabular}{@{}r@{}c@{}}
				& \Large{CM} \\
				\begin{fullleftbubbles}	\Large	
Code-mixing refers to a phenomenon of combining two or more languages in a single sentence. Can you generate a code-mixed \textcolor{blue}{English}-\textcolor{blue}{Chinese} sentence about \textcolor{orange}{food}?
				\end{fullleftbubbles}
				\vspace{2px} \\
                    \Large
				\hspace{15px}\fullbubble{chatgpt}{rounded~corners}{black}{
					\textbf{(ChatGPT)} Sure, here's an example: "I love eating 炸鸡 (zhá jī) fried chicken for lunch."
				} & \Large 2
				\vspace{2px} \\
                    \Large
				\fullbubble{dv003}{rounded~corners}{black}{
					\textbf{(Davinci-003)} 我超爱吃薯条!I love potato chips!
				} & \Large 0
				\vspace{2px} \\
                    \Large
				\fullbubble{dv002}{rounded~corners}{black}{
					\textbf{(Davinci-002)} "豆腐 豆腐 不 不 好 好 吃 吃 的 的 。 。
\\
Tofu, tofu, is not is not good good to eat to eat."
				} & \Large 0
				\vspace{2px} \\
                    \Large
				\fullbubble{bloomz}{rounded~corners}{black}{
					\textbf{(BLOOMZ)} 食物可以吃,但不要吃太多。
				} & \Large 0
				\vspace{2px} \\
                    \Large
				\fullbubble{flan}{rounded~corners}{black}{
					\textbf{(Flan-T5-XXL)} A chef prepares a drink.
				} & \Large 0
				\\
			\end{tabular}
		}
		\caption{Template: Explicitly define CM}
	\end{subfigure}
	%
	\begin{subfigure}[c]{0.5\textwidth}
		\resizebox{0.99\textwidth}{!}{
			\begin{tabular}{@{}r@{}c@{}}
				& \Large{CM} \\
				\begin{fullleftbubbles}	\Large	
Imagine that you are a \textcolor{sky}{\ul{Vietnamese}} speaker. How would you talk about \textcolor{orange}{weather} using both \textcolor{blue}{English} and \textcolor{orange}{Vietnamese} in a single sentence?
				\end{fullleftbubbles}
				\vspace{2px} \\
                    \Large
				\hspace{15px}\fullbubble{chatgpt}{rounded~corners}{black}{
					\textbf{(ChatGPT)} "Ngày hôm nay thời tiết khá nắng nóng, so I think I'll wear my shorts and t-shirt đi chơi."
				} & \Large 3
				\vspace{2px} \\
                    \Large
				\fullbubble{dv003}{rounded~corners}{black}{
					\textbf{(Davinci-003)} I'm sure the weather today is beautiful, thời tiết hôm nay có vẻ rất đẹp.
				} & \Large 0
				\vspace{2px} \\
                    \Large
				\fullbubble{dv002}{rounded~corners}{black}{
					\textbf{(Davinci-002)} I like the weather here because it's not too hot and not too cold. Tôi thích thời tiết ở đây vì nó không nóng quá và không lạnh quá.
				} & \Large 0
				\vspace{2px} \\
                    \Large
				\fullbubble{bloomz}{rounded~corners}{black}{
					\textbf{(BLOOMZ)}  Hãy tưởng tượng bạn là người nói tiếng Việt, và thử dùng cả hai thứ tiếng để mô tả thời tiết:
				} & \Large 0
				\vspace{2px} \\
                    \Large
				\fullbubble{flan}{rounded~corners}{black}{
					\textbf{(Flan-T5-XXL)} V khách hàng nh hàng tc áo  c tn c i n v nhng lng  bn s tm thy hàng n c áo c ó
				} & \Large 0
				\\
			\end{tabular}
		}
		\caption{Template: Native speaker}
	\end{subfigure}
	%
	\begin{subfigure}[c]{0.5\textwidth}
		\resizebox{0.99\textwidth}{!}{
			\begin{tabular}{@{}r@{}c@{}}
				& \Large{CM} \\
				\begin{fullleftbubbles} \Large
					Write an \textcolor{blue}{English} and \textcolor{blue}{Tagalog} code-mixed sentence about \textcolor{orange}{Artificial Intelligence}. 
				\end{fullleftbubbles}
				\vspace{2px} \\
                    \Large
				\hspace{15px}\fullbubble{chatgpt}{rounded~corners}{black}{
					\textbf{(ChatGPT)} English-Tagalog  code-mixed sentence: "Artificial Intelligence ay isang exciting na teknolohiya na makakapagbigay ng maraming benepisyo sa society." \\Translation: "Artificial Intelligence is an exciting technology that can provide many benefits sa lipunan."
				} & \Large 1
				\vspace{2px} \\
                    \Large
				\fullbubble{dv003}{rounded~corners}{black}{
					\textbf{(Davinci-003)} AI is revolutionizing the way we live, trabaho, at mabuhay ngayon sa pamamagitan ng pagbibigay ng modernong solusyon sa mga proyekto at pagkukuha ng ilang mga trabahong naghihintay na ma-automate.
				} & \Large 3
				\vspace{2px} \\
                    \Large
				\fullbubble{dv002}{rounded~corners}{black}{
					\textbf{(Davinci-002)} Artificial Intelligence is something that we should be thankful for.
				} & \Large 0
				\vspace{2px} \\
                    \Large
				\fullbubble{bloomz}{rounded~corners}{black}{
					\textbf{(BLOOMZ)} An expert in artificial intelligence explains their area of expertise to a student with the help of a colleague who speaks both languages.
				} & \Large 0
				\vspace{2px} \\
                    \Large
				\fullbubble{flan}{rounded~corners}{black}{
					\textbf{(Flan-T5-XXL)} Ang pagsusuri ng mas maraming kapasira ng hariko ay nag-aaral ng sasakyan ng pag-aaral ng pangangalambot, na magtitipon para sa pamamagitan ng pagpapasagawa ng karangalan ng isang lalaking.
				} & \Large 0
				\\
			\end{tabular}
		}
		\caption{Template: Write a CM sentence}
	\end{subfigure}
 \caption{Prompt templates with different \textcolor{blue}{languages} and \textcolor{orange}{topic} fields and responses from different LLMs containing code-mixed / non-code-mixed sentences. We also include a template where we specify the \textcolor{sky}{\ul{nationality}} of the speaker. Note that the explanations are a part of ChatGPT's original generation.
}
\label{fig:code-mixing-prompt-templates}
\end{figure*}


