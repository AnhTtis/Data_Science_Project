\section{Languages Spoken in SEA}
\label{sec:Languages}
There are more than  1,200 languages spoken in SEA~\cite{redmond2009-wl,maliwat2021}, 700 of which are spoken in Indonesia~\cite{aji-etal-2022-one,cahyawijaya2023nusacrowd}. 
We describe the languages the SEA languages used in the study in the following paragraphs.

%\paragraph{Burmese} Burmese is the national language of Myanmar.

\paragraph{Mandarin Chinese} Mandarin Chinese (zh-Hans), which belongs to the Sino-Tibetan language family and uses the Hanzi script, is widely spoken in SEA due to the migration of Chinese people from the coastal provinces of southeastern China, such as Fujian, Guangdong, and Hainan. People of Chinese heritage in SEA frequently use the term ``华人'' (huá rén) to express their cultural identity as an ethnic group, instead of ``中国人'' (zhōng guó rén) which is primarily associated with nationality, even though both terms can be translated as ``Chinese (people).'' Singapore has the largest Chinese ethnic group among all SEA countries and Mandarin Chinese is considered one of the official languages in Singapore.

The language is characterized as linguistically ``isolating'' in that each Chinese character corresponds to one morpheme and that the language uses very little grammatical inflection. It uses a logographic writing system, which uses pictograms (Chinese characters) to represent meaning. Chinese is also a tonal language with four pitched tones and one neutral tone. It commonly displays a basic SVO word order and, instead of conjugating the verbs to express tenses, uses aspect particles such as 了 (le) and 着 (zhe) to indicate the temporal location of the sentence.


\paragraph{Indonesian}
Indonesian (ind) is the national language of Indonesia~\cite{indonesia2002undang}. It is spoken by around 300 million speakers worldwide. Indonesian is developed from the literary `Classical Malay’ of the Riau-Johor sultanate \cite{sneddon2003} and has many regional variants. Indonesian is written in Latin script with a lexical similarity of over 80\% to Standard Malay. Indonesian is non-tonal and has 19 consonants, 6 vowels, and 3 diphthongs. The stress is on the penultimate syllable and the word order is SVO. It has three optional noun classifiers. Indonesian has two social registers and a rich affixation system, including a variety of prefixes, suffixes, circumfixes, and reduplication. Most of the affixes in Indonesian are derivational~\cite{pisceldo-etal-2008-two}.


%\paragraph{Javanese}
%Javanese (jav) is a language spoken mainly on Java island. It is the de facto language of provincial identity in central and eastern Java. The word order is SVO. It has 21 consonants and 8 vowels.  Javanese used to be written in Javanese script but since 20th century is mostly written in Latin script. Javanese differs from most other languages of western Indonesia in contrasting dental and retroflex stops, and in the feature of breathy voice or murmur as a phonetic property of its voiced obstruents. Javanese also differs from most languages of the Philippines and western Indonesia in allowing a number of word-initial consonant clusters. It has an elaborate system of speech levels~\cite{blust2013austronesian}.

%\paragraph{Khmer} Khmer is the national language of Cambodia.

%\paragraph{Lao} Lao is the national language of Laos.

\paragraph{Standard Malay} 
Standard Malay (msa) is the national language of Malaysia, Brunei, and Singapore, and the language is spoken by approximately 290 million speakers worldwide. The word order of Standard Malay is SVO with four types of affixes, i.e., prefixes (awalan), suffixes (akhiran), circumfixes (apitan), and infixes (sisipan). Even though Standard Malay and Indonesian originate from the same Malay language and are mutually intelligible, they can differ in spelling and vocabulary. One example is loanwords. Due to the different colonial influences from the Dutch and British, Indonesian primarily absorbs Dutch loanwords whereas Malay absorbs English loanwords. Both languages can also differ in the meanings of the same written words, which are commonly referred to as interlingual homographs. For instance, ``polisi'' means ``police'' in Indonesian but ``policy'' in Standard Malay.

%\paragraph{Sundanese} 
%Sundanese (sun) is a language spoken mainly in the Banten and West Java provinces. It is the de facto language of provincial identity in western Java. The main dialects are Bogor (Krawang), Pringan, and Cirebon.  Sunanese is non-tonal and has 18 consonant and 7 vowel phonemes. The stress is on the penultimate syllable. Sundanese has elaborate coding of respect levels. Sundanese  was previously written in Arabic, Javanese, and Sundanese scripts, but it is written in Latin script since the middle of the 19th century. Sundanese is a predominantly SVO language. It has voice marking and incorporates some (optional) actor-verb agreement, i.e., number and person~\cite{kurniawan2013sundanese}.

\paragraph{Tagalog} Tagalog (tgl) is an Austronesian language spoken in the Philippines by around ~82 million native speakers. It is both agglutinative and pitch-accented, giving it rich and complex morphology \cite{kroeger1993phrase}. Tagalog's standardized form, known as \textit{Filipino}, is the country's official national language. The difference between Filipino and Tagalog is more sociopolitical than sociolinguistic: Commonwealth Act No. 184 of 1936 created a national committee whose purpose is to ``develop a national language.'' This resulted in the standardization of the Tagalog language into Filipino. In practice, Filipino is indistinguishable from Tagalog, albeit with the addition of letters f, j, c, x, and z, plus loanwords \cite{phgazette1936}.

% \paragraph{Tamil} \todo{add description.}

%\paragraph{Tetum} Tetum is the national language of Timor-Leste.

%\paragraph{Thai} Thai (tha) is the national language of Thailand.

\paragraph{Vietnamese} Vietnamese (vie), the national language of Vietnam, is spoken by around 85 million people worldwide. It is a tonal language belonging to the Austroasiatic language family and uses accents to denote six distinctive tones. The sentence structure of Vietnamese displays the SVO word order, and due to heavy influence from Chinese, it also uses a rich set of classifiers that are required in the presence of quantifiers. For instance, instead of writing ``bốn gà,'' which literally translates into ``four chickens,'' it should be ``bốn con gà'' where ``con'' is a classifier for non-human animate things.

\paragraph{Tamil} Tamil (tam) is a Dravidian language originating from Tamil Nadu and Sri Lanka. It is spoken by the sizeable Tamil diasporas of Singapore (2.5\% of population \cite{singapore2020}) and Malaysia (9\% of population \cite{schiffman1998}), which resulted from histories of trade, migration, indentured servitude, and civil unrest. Tamil is an official language of Singapore \cite{singapore2020}, and the only one originating from India. Tamil is notably diglossic, which means it has a formal literary system, lacks lexically distinctive stress, and is non-rhotic \cite{armstrongtamil}. Tamil uses SOV sentence structure. Tamil-English code-mixing exhibits interesting linguistic phenomena such as nonce loan, wherein many nonce borrowings from English occupy objects corresponding to Tamil verbs, and vice versa \cite{sankoff_poplack_vanniarajan_1990}. 

\paragraph{Singlish}
Singlish is a widely-used conversational language in Singapore. It is an English-based creole language that arose out of prolonged language contact between speakers of many different languages in the country, including Hokkien, Malay, Teochew, Cantonese, and Tamil. Singlish is spoken by around 4 million speakers, and one unique feature of the language is its heavy use of pragmatic particles borrowed from Southern Chinese dialects. One example of this is ``lah,'' which in the sentence, ``Her dress is too short lah,'' emphasizes the statement.  



\section{HuggingFace Inference API}
\label{app:hf-api}
We use HuggingFace's Inference API to prompt multilingual LLMs since we do not have sufficient local compute to host models with hundreds of billions of parameters such as the 176B-parameter BLOOMZ model \cite{muennighoff2022bloomz}. The text-to-text task is treated identically as a text-generation task, and we set \texttt{max\_new\_tokens} (amount of new tokens to be generated) to 100, \texttt{temperature} to 0.7, and \texttt{repetition\_penalty} to 1.2. 


\section{OpenAI Inference API}

We use OpenAI's official API to prompt both davinci-003 and davinci-002. Specifically, we use \texttt{openai.Completion.create} with a maximum generation length of 128. We use the default values for all other parameters.

\section{Flan-T5-XXL Non-English Outputs}
\label{app:flant5-fluency}
We observe that when Flan-T5-XXL generates non-English outputs, most of them are nonsensical. Here are some of the examples and their translations. 

\noindent \textbf{Indonesian}: Ini adalah sebuah udara untuk pengobatan minyak dan di sekitar kehidupan. 

\noindent \textit{Translation: This is an air for oil treatment and around life.}

\noindent \textbf{Malay}:  Artificial Intelligence adalah sebuah kantor keamanan yang digunakan untuk mengidentifikasi penduduk yang memiliki anak-anak dalam diri. 

\noindent \textit{Translation: Artificial intelligence is a security office used for identifying residents who have childen inside.}

\noindent \textbf{Tagalog}: Weather niya ang nagsimula sa pagsasagawa ng kaniyang kargahan ng panahon.

\noindent \textit{Translation: It was his weather that started carrying out his weather load.}

\noindent \textbf{Vietnamese}: Nhà ng tài ra mt ngi dy xut trn o trng h nhng ngi  ng thng u c thit v.

\noindent \textit{Translation: The artist has created an outstanding talent in the field of talented people.}

\begin{figure*}[htp]
\centering
\includegraphics[width=\textwidth]{assets/subplots_davinci002.pdf}
\caption{Analysis of davinci-002's capability of generating code-mixed data.}
\label{fig:davinci-002}
\end{figure*}

\begin{figure*}[htp]
\centering
\includegraphics[width=\textwidth]{assets/subplots_davinci003.pdf}
\caption{Analysis of davinci-003's capability of generating code-mixed data.}
\label{fig:davinci-003}
\end{figure*}

\begin{figure*}[htp]
\centering
\includegraphics[width=\textwidth]{assets/subplots_bloomz.pdf}
\caption{Analysis of BLOOMZ's capability of generating code-mixed data.}
\label{fig:bloomz}
\end{figure*}

\begin{figure*}[htp]
\centering
\includegraphics[width=\textwidth]{assets/subplots_flant5xxl.pdf}
\caption{Analysis of Flan-T5-XXL's capability of generating code-mixed data.}
\label{fig:flant5xxl}
\end{figure*}

\begin{figure*}[t]
	\begin{subfigure}[c]{0.5\textwidth}
		\resizebox{0.99\textwidth}{!}{
			\includegraphics[]{assets/template_a.pdf}
		}
		\caption{Template: Assume as bilingual speaker}
	\end{subfigure}
	%
	\begin{subfigure}[c]{0.5\textwidth}
		\resizebox{0.99\textwidth}{!}{
			\includegraphics[]{assets/template_c.pdf}
		}
		\caption{Template: Imitate speaking style}
	\end{subfigure}
	%
	\begin{subfigure}[c]{0.5\textwidth}
		\resizebox{0.99\textwidth}{!}{
			\includegraphics[]{assets/template_b.pdf}
		}
		\caption{Template: Two bilingual speakers}
	\end{subfigure}
	%
	\begin{subfigure}[c]{0.5\textwidth}
		\resizebox{0.99\textwidth}{!}{
			\includegraphics[]{assets/template_d.pdf}
		}
		\caption{Template: Explicitly define CM}
	\end{subfigure}
	% % Continued Figure below
	% \begin{subfigure}[c]{0.5\textwidth}
	% 	\resizebox{0.99\textwidth}{!}{
	% 		\includegraphics[]{assets/template_e.pdf}
	% 	}
	% 	\caption{Template: Native speaker}
	% \end{subfigure}
	% %
	% \begin{subfigure}[c]{0.5\textwidth}
	% 	\resizebox{0.99\textwidth}{!}{
	% 		\includegraphics[]{assets/template_f.pdf}
	% 	}
	% 	\caption{Template: Write a CM sentence}
	% \end{subfigure}
 \caption{All prompt templates with different \textcolor{blue}{languages} and \textcolor{orange}{topic} fields and responses from different LLMs containing code-mixed / non-code-mixed sentences. Note that the explanations are a part of ChatGPT's original generation. ``CM'' indicates the level of code-mixing (Section~\ref{sec:cm-scale})
}
\label{fig:app-code-mixing-prompt-templates}
\end{figure*}

\begin{figure*}[t]
        \ContinuedFloat
	\begin{subfigure}[c]{0.5\textwidth}
		\resizebox{0.99\textwidth}{!}{
			\includegraphics[]{assets/template_e.pdf}
		}
		\caption{Template: Native speaker}
	\end{subfigure}
	%
	\begin{subfigure}[c]{0.5\textwidth}
		\resizebox{0.99\textwidth}{!}{
			\includegraphics[]{assets/template_f.pdf}
		}
		\caption{Template: Write a CM sentence}
	\end{subfigure}
 \caption{(Continued) We also include a template where we specify the \textcolor{sky}{\ul{nationality}} of the speaker in addition to the \textcolor{blue}{languages} and \textcolor{orange}{topic} fields.
}
\label{fig:app-cont-code-mixing-prompt-templates}
\end{figure*}

\section{BLOOMZ's Training Language Distribution}
\label{effects-bloomz-lang-proportion}

BLOOMZ is created by finetuning the multilingual 176B-parameter language model BLOOM \cite{scao2022bloom} that is pretrained on ROOTS corpus \cite{lauren2022roots} on a collection of prompt instructions known as xP3 \cite{muennighoff2022bloomz}. Table~\ref{tab:roots-lang-dist} and Table~\ref{tab:xp3-lang-dist} show the proportion of SEA languages investigated in our paper existing in the ROOTS and xP3 datasets respectively. Even though Indonesian and Chinese are higher in proportion than Tamil, BLOOMZ code-mix better for Tamil than the former two language with around 20\% performance difference.

\begin{table}[ht]
    \centering
    \begin{tabular}{cc}
    \toprule
     Languages & Percent Distribution (\%) \\
     \midrule
     English & 30.04 \\
     Chinese (Simplified) & 16.2 \\
     Vietnamese & 2.7 \\
     Indonesian & 1.2 \\
     Tamil & 0.2 \\
     
     \bottomrule
    \end{tabular}
    \caption{Proportion of Languages in the ROOTS corpus \cite{lauren2022roots}.}
    \label{tab:roots-lang-dist}
\end{table}

\begin{table}[ht]
    \centering
    \begin{tabular}{cc}
    \toprule
     Languages & Percent Distribution (\%) \\
     \midrule
     English & 39.25 \\
     Indonesian & 4.85 \\
     Chinese (Simplified) & 4.83 \\
     Vietnamese & 3.27 \\
     Tamil & 0.97 \\
     
     \bottomrule
    \end{tabular}
    \caption{Proportion of Languages in the xP3 datasets \cite{muennighoff2022bloomz}.}
    \label{tab:xp3-lang-dist}
\end{table}


\section{Naturalness and Fluency Issues of ChatGPT's Generation}
\label{app:naturalness-issues}

We document a non-exhaustive list of syntactic and semantic errors as well as reasons for unnaturalness in ChatGPT's generation in Table~\ref{tab:naturalness-issues}.

\section{Annotators and Inter-annotator Agreement}
\label{app:interannotator}

We have a total of 13 annotators, some of whom speak more than one SEA language. All of them are native speakers of their respective SEA languages, and most grow up in SEA. Many of our annotators are AI researchers and reside in the Global North. All the annotators are the authors of the paper. 

In Table~\ref{tab:interannotator}, we report the inter-annotator agreement scores for naturalness annotations using Fleiss' Kappa $\kappa$ \cite{fleiss1971measuring}, which measures the agreement between a fixed number of raters when assigning categorical ratings to the items. It can be applied to settings with multiple annotators and not all raters are required to annotate all items. The closer it is to 1, the higher the agreement among annotators. 

According to the guideline \cite{landis1977measurement,altman1990practical}, English-Indonesian annotations have a fair agreement, English-Chinese and Singlish have a substantial agreement, and English-Tagalog have almost perfect agreement among the annotators. 

\begin{table}[ht]
    \centering
    \begin{tabular}{ccc}
    \toprule
     Language & N(annotators) & $\kappa$ \\
     \midrule
     English-Chinese & 3 & 0.6431 \\
     English-Indonesian & 3 & 0.2165 \\
     English-Malay & 1 & -\\
     English-Tagalog & 2 & 0.8268\\
     English-Tamil & 1 & - \\
     English-Vietnamese & 1 & - \\
     Singlish & 3 & 0.6199 \\
     \bottomrule
    \end{tabular}
    \caption{Inter-annnotator agreement scores for naturalness of ChatGPT's generated code-mixed text. N(annotators) indicates the total number of annotators and $\kappa$ refers to the Fleiss' Kappa agreement score.}
    \label{tab:interannotator}
\end{table}

\begin{sidewaystable*}[ht]
    \small
    \centering
    \begin{tabular}{p{6cm}p{6cm}lp{3cm}p{6cm}}
        \toprule
        Examples & Standard English Translation & Language Pairs & Issues & Explanations \\
        \midrule
        Besok kita harus prepare payung karena cuacanya bakal cloudy dengan chance of hujan sepanjang hari & Tomorrow we need to prepare umbrella because the weather is going to be cloudy with the chance of raining all day long. & English-Indonesian & Unnatural phrasing & ``chance of hujan'' and ``cuacanya bakal cloudy'' sound unnatural. Should be ``chance of raining'' and ``cuacanya akan menjadi cloudy''  \\
        Saya suka spend time bersama family saya, especially bila kita makan makanan yang sedap seperti nasi lemak or roti canai for breakfast. & I like spending time with my family, especially when we eat delicious food such as nasi lemak and roti canai. & English-Malay & Gerund; Conjunction & ``suka spend time'' should change to ``suka spending time'' as the word ``suka'' (like.v) should be followed with gerund. ``or'' should also be changed to ``and''. \\
        I sampai already but still stuck in traffic ni, macam mana nak balik home on time lah. & I reached already but still stuck in this traffic; how can I return home on time & English-Malay & Semantic confusion & The sentence doesn't make any sense event though it is grammatically correct. A person cannot both reach home and stuck in traffic at the same time. \\
        I love eating 炸鸡 (zhá jī) fried chicken for lunch. & I love eating fried chicken for lunch & English-Chinese & Redundancy & ``炸鸡'' is the same as ``fried rice''. \\
        So, wǒ rènwéi yī gè jiànkāng de AI xiànshí shì yī gè jùyǒu zhuānyè shíjì kěnéng xìng de chéngzhǎng zhǐnéng de jìshù. & So, I believe a healthy AI system is a technology of growing intelligence with professional practical possibilities. & English-Chinese & Unnatural script system & The generated text should use Mandarin characters instead of Pinyin. It should be written as ``So, 我认为一个健康的AI是一个具有专业实际可能性的成长智能的技术.'' Furthermore, the sentence does not make any sense \\
        My family ay nagplano ng isang malaking family reunion sa park this coming weekend & My family is planning a big family reunion at the park this coming weekend & English-Tagalog & Possessive markers & The break from english ``My family'' to Tagalog ``ay nagplano'' is unnatural. When Tagalog is the matrix language, we use Tagalog possessive determiners, so the correct form would be ``Ang family ko ay nagplano ...'' \\
        Yesterday, tôi đã đi out với gia đình của mình để celebrate my parents' wedding anniversary. & Yesterday, I went out with my family to celebrate my parents' wedding anniversary. & English-Vietnamese & Verb phrase & Instead of ``đi out'', it should be either ``đi'' or ``went out''. \\
        AI, you know, can do many things lah, like make our lives easier, but also can be very pai seh if we don't use it properly. & You know while AI can do many things such as making our lives easier, it can also be very embarrassing  if we don't use it properly & Singlish & Incorrect use of Singlish expression & ``pai seh'' is a Hokkien word that describes a person feeling shy, sorry or embarrassed. Using it to describe AI feeling embarrassed is inappropriate. \\
        Eh, you know lah, family very important one leh, must always stick together like sotong and chilli sauce, can or not? & Do you know that family is very important, so we must always stick together like squid and chili sauce? & Singlish & Analogy & Using ``sotong and chilli sauce'' (squid and chili sauce) as an analogy to familial bond is an unnatural expression. No one in Singapore uses such an expression. \\
        Traffic romba kasta pattu irukku today, it's taking forever to reach my destination. & Traffic is bad today; it's taking forever to reach my destination & English-Tamil & Adjective & ``Traffic romba kasta pattu irukku today'' means that the traffic is suffering, which is not the same as the traffic is congested. \\
        \raisebox{-\totalheight}{\includegraphics[width=0.2\textwidth]{assets/Tamil_runon_isssue.pdf}} & Data identification in artificial web applications is effective in any language, Artificial Intelligence is revolutionizing the way we interact with technology. & English-Tamil & Comma splice & Both the Tamil and English independent clauses are joined by comma, which is a grammatical error of comma splice. \\
        \raisebox{-\totalheight}{\includegraphics[width=0.2\textwidth]{assets/Tamil_transliterated_issue.pdf}}& Now I am here eating is a bad mistake, but I still crave for some hot fritters and pongal soup. & English-Tamil & Different script system & ``Ippo naan inga'' uses Latin-transliterated system but the rest of the Tamil words use Tamil script system, which is unnatural writing. \\
        \bottomrule
    \end{tabular}
    \caption{Naturalness issues with explanations for ChatGPT's code-mixed text generation.}
    \label{tab:naturalness-issues}
\end{sidewaystable*}