\section{Languages Spoken in SEA}
\label{sec:Languages}
There are more than  1,200 languages spoken in SEA~\cite{redmond2009-wl,maliwat2021}, 700 of which are spoken in Indonesia~\cite{aji-etal-2022-one,cahyawijaya2023nusacrowd}. 
We describe the languages the SEA languages used in the study in the following paragraphs.

%\paragraph{Burmese} Burmese is the national language of Myanmar.

\paragraph{Mandarin Chinese} Mandarin Chinese language (zh-Hans), which belongs to the Sino-Tibetan language family and uses the Hanzi script, is widely spoken in SEA due to the migration of Chinese people from the coastal provinces of southeastern China, such as Fujian, Guangdong, and Hainan. People of Chinese heritage in SEA frequently use the term "华人" (huá rén) to express their cultural identity as an ethnic group, instead of "中国人" (zhōng guó rén) which is primarily associated with nationality, even though both terms can be translated as "Chinese (people)." Singapore has the largest Chinese ethnic group among all SEA countries and Mandarin Chinese is considered one of the official languages in Singapore.

The language is characterized as linguistically "isolating" in that each Chinese character corresponds to one morpheme and that the language uses very little grammatical inflection. It uses a logographic writing system, which uses pictograms (Chinese characters) to represent meaning. Chinese is also a tonal language with four pitched tones and one neutral tone. It commonly displays a basic SVO word order and, instead of conjugating the verbs to express tenses, uses aspect particles such as 了 (le) and 着 (zhe) to indicate the temporal location of the sentence.


\paragraph{Indonesian}
Indonesian (ind) is the national language of Indonesia~\cite{indonesia2002undang}. It is spoken by around 300 million speakers worldwide. Indonesian is developed from the literary `Classical Malay’ of the Riau-Johor sultanate \cite{sneddon2003} and has many regional variants. Indonesian is written in Latin script with a lexical similarity of over 80\% to Standard Malay. Indonesian is non-tonal and has 19 consonants, 6 vowels, and 3 diphthongs. The stress is on the penultimate syllable and the word order is SVO. It has three optional noun classifiers. Indonesian has two social registers and a rich affixation system, including a variety of prefixes, suffixes, circumfixes, and reduplication. Most of the affixes in Indonesian are derivational~\cite{pisceldo-etal-2008-two}.


%\paragraph{Javanese}
%Javanese (jav) is a language spoken mainly on Java island. It is the de facto language of provincial identity in central and eastern Java. The word order is SVO. It has 21 consonants and 8 vowels.  Javanese used to be written in Javanese script but since 20th century is mostly written in Latin script. Javanese differs from most other languages of western Indonesia in contrasting dental and retroflex stops, and in the feature of breathy voice or murmur as a phonetic property of its voiced obstruents. Javanese also differs from most languages of the Philippines and western Indonesia in allowing a number of word-initial consonant clusters. It has an elaborate system of speech levels~\cite{blust2013austronesian}.

%\paragraph{Khmer} Khmer is the national language of Cambodia.

%\paragraph{Lao} Lao is the national language of Laos.

\paragraph{Standard Malay} 
Standard Malay (msa) is the national language of Malaysia, Brunei, and Singapore, and the language is spoken by approximately 290 million speakers worldwide. The word order of Standard Malay is SVO with four types of affixes, i.e., prefixes (awalan), suffixes (akhiran), circumfixes (apitan), and infixes (sisipan). Even though Standard Malay and Indonesian originate from the same Malay language and are mutually intelligible, they can differ in spelling and vocabulary. One example is loanwords. Due to the different colonial influences from the Dutch and British, Indonesian primarily absorbs Dutch loanwords whereas Malay absorbs English loanwords. Both languages can also differ in the meanings of the same written words, which are commonly referred to as interlingual homographs. For instance, "polisi" means "police" in Indonesian but "policy" in Standard Malay.

%\paragraph{Sundanese} 
%Sundanese (sun) is a language spoken mainly in the Banten and West Java provinces. It is the de facto language of provincial identity in western Java. The main dialects are Bogor (Krawang), Pringan, and Cirebon.  Sunanese is non-tonal and has 18 consonant and 7 vowel phonemes. The stress is on the penultimate syllable. Sundanese has elaborate coding of respect levels. Sundanese  was previously written in Arabic, Javanese, and Sundanese scripts, but it is written in Latin script since the middle of the 19th century. Sundanese is a predominantly SVO language. It has voice marking and incorporates some (optional) actor-verb agreement, i.e., number and person~\cite{kurniawan2013sundanese}.

\paragraph{Tagalog} Tagalog (tgl) is an Austronesian language spoken in the Philippines by around ~82 million native speakers. It is both agglutinative and pitch-accented, giving it rich and complex morphology \cite{kroeger1993phrase}. Tagalog's standardized form, known as \textit{Filipino}, is the country's official national language. The difference between Filipino and Tagalog is more sociopolitical than sociolinguistic: Commonwealth Act No. 184 of 1936 created a national committee whose purpose is to ``develop a national language.'' This resulted in the standardization of the Tagalog language into Filipino. In practice, Filipino is indistinguishable from Tagalog, albeit with the addition of letters f, j, c, x, and z, plus loanwords \cite{phgazette1936}.

%\paragraph{Tetum} Tetum is the national language of Timor-Leste.

%\paragraph{Thai} Thai (tha) is the national language of Thailand.

\paragraph{Vietnamese} Vietnamese (vie), the national language of Vietnam, is spoken by around 85 million people worldwide. It is a tonal language belonging to the Austroasiatic language family and uses accents to denote six distinctive tones. The sentence structure of Vietnamese displays the SVO word order, and due to heavy influence from Chinese, it also uses a rich set of classifiers that are required in the presence of quantifiers. For instance, instead of writing "bốn gà," which literally translates into "four chickens," it should be "bốn con gà" where "con" is a classifier for non-human animate things.

\paragraph{Singlish}
Singlish is a widely-used conversational language in Singapore. It is an English-based creole language that arose out of prolonged language contact between speakers of many different languages in the country, including Hokkien, Malay, Teochew, Cantonese, and Tamil. Singlish is spoken by around 4 million speakers, and one unique feature of the language is its heavy use of pragmatic particles borrowed from Southern Chinese dialects. One example of this is "lah," which in the sentence, "Her dress is too short lah," emphasizes the statement.  


\section{HuggingFace Inference API}
\label{app:hf-api}
We use HuggingFace's Inference API to prompt multilingual LLMs since we do not have sufficient local compute to host models with hundreds of billions of parameters. The text-to-text task is treated identically as a text-generation task, and we set \texttt{max\_new\_tokens} (amount of new tokens to be generated) to 100, \texttt{temperature} to 0.7, and \texttt{repetition\_penalty} to 1.2. 

\section{OpenAI Inference API}

We use OpenAI's official API to prompt both davinci-003 and davinci-002. Specifically, we use \texttt{openai.Completion.create} with a maximum generation length of 128. We use the default values for all other parameters.

\section{Flan-T5-XXL Non-English Outputs}
\label{app:flant5-fluency}
We observe that when Flan-T5-XXL generates non-English outputs, most of them are nonsensical. Here are some of the examples and their translations. 

\noindent \textbf{Indonesian}: Ini adalah sebuah udara untuk pengobatan minyak dan di sekitar kehidupan. 

\noindent \textit{Translation: This is an air for oil treatment and around life.}

\noindent \textbf{Malay}:  Artificial Intelligence adalah sebuah kantor keamanan yang digunakan untuk mengidentifikasi penduduk yang memiliki anak-anak dalam diri. 

\noindent \textit{Translation: Artificial intelligence is a security office used for identifying residents who have childen inside.}

\noindent \textbf{Tagalog}: Weather niya ang nagsimula sa pagsasagawa ng kaniyang kargahan ng panahon.

\noindent \textit{Translation: It was his weather that started carrying out his weather load.}

\noindent \textbf{Vietnamese}: Nhà ng tài ra mt ngi dy xut trn o trng h nhng ngi  ng thng u c thit v.

\noindent \textit{Translation: The artist has created an outstanding talent in the field of talented people.}

\begin{figure*}[htp]
\centering
\includegraphics[width=\textwidth]{assets/subplots_davinci002.pdf}
\caption{Analysis of davinci-002's capability of generating code-mixed data.}
\label{fig:davinci-002}
\end{figure*}

\begin{figure*}[htp]
\centering
\includegraphics[width=\textwidth]{assets/subplots_bloomz.pdf}
\caption{Analysis of BLOOMZ's capability of generating code-mixed data.}
\label{fig:bloomz}
\end{figure*}

\begin{figure*}[htp]
\centering
\includegraphics[width=\textwidth]{assets/subplots_flant5xxl.pdf}
\caption{Analysis of Flan-T5-XXL's capability of generating code-mixed data.}
\label{fig:flant5xxl}
\end{figure*}