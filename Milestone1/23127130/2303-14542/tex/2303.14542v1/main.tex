
%% bare_jrnl_compsoc.tex
%% V1.4b
%% 2015/08/26
%% by Michael Shell
%% See:
%% http://www.michaelshell.org/
%% for current contact information.
%%
%% This is a skeleton file demonstrating the use of IEEEtran.cls
%% (requires IEEEtran.cls version 1.8b or later) with an IEEE
%% Computer Society journal paper.
%%
%% Support sites:
%% http://www.michaelshell.org/tex/ieeetran/
%% http://www.ctan.org/pkg/ieeetran
%% and
%% http://www.ieee.org/

%%*************************************************************************
%% Legal Notice:
%% This code is offered as-is without any warranty either expressed or
%% implied; without even the implied warranty of MERCHANTABILITY or
%% FITNESS FOR A PARTICULAR PURPOSE!
%% User assumes all risk.
%% In no event shall the IEEE or any contributor to this code be liable for
%% any damages or losses, including, but not limited to, incidental,
%% consequential, or any other damages, resulting from the use or misuse
%% of any information contained here.
%%
%% All comments are the opinions of their respective authors and are not
%% necessarily endorsed by the IEEE.
%%
%% This work is distributed under the LaTeX Project Public License (LPPL)
%% ( http://www.latex-project.org/ ) version 1.3, and may be freely used,
%% distributed and modified. A copy of the LPPL, version 1.3, is included
%% in the base LaTeX documentation of all distributions of LaTeX released
%% 2003/12/01 or later.
%% Retain all contribution notices and credits.
%% ** Modified files should be clearly indicated as such, including  **
%% ** renaming them and changing author support contact information. **
%%*************************************************************************


% *** Authors should verify (and, if needed, correct) their LaTeX system  ***
% *** with the testflow diagnostic prior to trusting their LaTeX platform ***
% *** with production work. The IEEE's font choices and paper sizes can   ***
% *** trigger bugs that do not appear when using other class files.       ***                          ***
% The testflow support page is at:
% http://www.michaelshell.org/tex/testflow/


\documentclass[10pt,conference]{IEEEtran}
%
% If IEEEtran.cls has not been installed into the LaTeX system files,
% manually specify the path to it like:
% \documentclass[10pt,journal,compsoc]{../sty/IEEEtran}





% Some very useful LaTeX packages include:
% (uncomment the ones you want to load)


% *** MISC UTILITY PACKAGES ***
%
%\usepackage{ifpdf}
% Heiko Oberdiek's ifpdf.sty is very useful if you need conditional
% compilation based on whether the output is pdf or dvi.
% usage:
% \ifpdf
%   % pdf code
% \else
%   % dvi code
% \fi
% The latest version of ifpdf.sty can be obtained from:
% http://www.ctan.org/pkg/ifpdf
% Also, note that IEEEtran.cls V1.7 and later provides a builtin
% \ifCLASSINFOpdf conditional that works the same way.
% When switching from latex to pdflatex and vice-versa, the compiler may
% have to be run twice to clear warning/error messages.






% *** CITATION PACKAGES ***
%
\ifCLASSOPTIONcompsoc
  % IEEE Computer Society needs nocompress option
  % requires cite.sty v4.0 or later (November 2003)
  \usepackage[nocompress]{cite}
\else
  % normal IEEE
  \usepackage{cite}
\fi
% cite.sty was written by Donald Arseneau
% V1.6 and later of IEEEtran pre-defines the format of the cite.sty package
% \cite{} output to follow that of the IEEE. Loading the cite package will
% result in citation numbers being automatically sorted and properly
% "compressed/ranged". e.g., [1], [9], [2], [7], [5], [6] without using
% cite.sty will become [1], [2], [5]--[7], [9] using cite.sty. cite.sty's
% \cite will automatically add leading space, if needed. Use cite.sty's
% noadjust option (cite.sty V3.8 and later) if you want to turn this off
% such as if a citation ever needs to be enclosed in parenthesis.
% cite.sty is already installed on most LaTeX systems. Be sure and use
% version 5.0 (2009-03-20) and later if using hyperref.sty.
% The latest version can be obtained at:
% http://www.ctan.org/pkg/cite
% The documentation is contained in the cite.sty file itself.
%
% Note that some packages require special options to format as the Computer
% Society requires. In particular, Computer Society  papers do not use
% compressed citation ranges as is done in typical IEEE papers
% (e.g., [1]-[4]). Instead, they list every citation separately in order
% (e.g., [1], [2], [3], [4]). To get the latter we need to load the cite
% package with the nocompress option which is supported by cite.sty v4.0
% and later. Note also the use of a CLASSOPTION conditional provided by
% IEEEtran.cls V1.7 and later.





% *** GRAPHICS RELATED PACKAGES ***
%
\ifCLASSINFOpdf
  % \usepackage[pdftex]{graphicx}
  % declare the path(s) where your graphic files are
  % \graphicspath{{../pdf/}{../jpeg/}}
  % and their extensions so you won't have to specify these with
  % every instance of \includegraphics
  % \DeclareGraphicsExtensions{.pdf,.jpeg,.png}
\else
  % or other class option (dvipsone, dvipdf, if not using dvips). graphicx
  % will default to the driver specified in the system graphics.cfg if no
  % driver is specified.
  % \usepackage[dvips]{graphicx}
  % declare the path(s) where your graphic files are
  % \graphicspath{{../eps/}}
  % and their extensions so you won't have to specify these with
  % every instance of \includegraphics
  % \DeclareGraphicsExtensions{.eps}
\fi
% graphicx was written by David Carlisle and Sebastian Rahtz. It is
% required if you want graphics, photos, etc. graphicx.sty is already
% installed on most LaTeX systems. The latest version and documentation
% can be obtained at:
% http://www.ctan.org/pkg/graphicx
% Another good source of documentation is "Using Imported Graphics in
% LaTeX2e" by Keith Reckdahl which can be found at:
% http://www.ctan.org/pkg/epslatex
%
% latex, and pdflatex in dvi mode, support graphics in encapsulated
% postscript (.eps) format. pdflatex in pdf mode supports graphics
% in .pdf, .jpeg, .png and .mps (metapost) formats. Users should ensure
% that all non-photo figures use a vector format (.eps, .pdf, .mps) and
% not a bitmapped formats (.jpeg, .png). The IEEE frowns on bitmapped formats
% which can result in "jaggedy"/blurry rendering of lines and letters as
% well as large increases in file sizes.
%
% You can find documentation about the pdfTeX application at:
% http://www.tug.org/applications/pdftex






% *** MATH PACKAGES ***
%
%\usepackage{amsmath}
% A popular package from the American Mathematical Society that provides
% many useful and powerful commands for dealing with mathematics.
%
% Note that the amsmath package sets \interdisplaylinepenalty to 10000
% thus preventing page breaks from occurring within multiline equations. Use:
%\interdisplaylinepenalty=2500
% after loading amsmath to restore such page breaks as IEEEtran.cls normally
% does. amsmath.sty is already installed on most LaTeX systems. The latest
% version and documentation can be obtained at:
% http://www.ctan.org/pkg/amsmath





% *** SPECIALIZED LIST PACKAGES ***
%
%\usepackage{algorithmic}
% algorithmic.sty was written by Peter Williams and Rogerio Brito.
% This package provides an algorithmic environment fo describing algorithms.
% You can use the algorithmic environment in-text or within a figure
% environment to provide for a floating algorithm. Do NOT use the algorithm
% floating environment provided by algorithm.sty (by the same authors) or
% algorithm2e.sty (by Christophe Fiorio) as the IEEE does not use dedicated
% algorithm float types and packages that provide these will not provide
% correct IEEE style captions. The latest version and documentation of
% algorithmic.sty can be obtained at:
% http://www.ctan.org/pkg/algorithms
% Also of interest may be the (relatively newer and more customizable)
% algorithmicx.sty package by Szasz Janos:
% http://www.ctan.org/pkg/algorithmicx




% *** ALIGNMENT PACKAGES ***
%
%\usepackage{array}
% Frank Mittelbach's and David Carlisle's array.sty patches and improves
% the standard LaTeX2e array and tabular environments to provide better
% appearance and additional user controls. As the default LaTeX2e table
% generation code is lacking to the point of almost being broken with
% respect to the quality of the end results, all users are strongly
% advised to use an enhanced (at the very least that provided by array.sty)
% set of table tools. array.sty is already installed on most systems. The
% latest version and documentation can be obtained at:
% http://www.ctan.org/pkg/array


% IEEEtran contains the IEEEeqnarray family of commands that can be used to
% generate multiline equations as well as matrices, tables, etc., of high
% quality.




% *** SUBFIGURE PACKAGES ***
%\ifCLASSOPTIONcompsoc
%  \usepackage[caption=false,font=footnotesize,labelfont=sf,textfont=sf]{subfig}
%\else
%  \usepackage[caption=false,font=footnotesize]{subfig}
%\fi
% subfig.sty, written by Steven Douglas Cochran, is the modern replacement
% for subfigure.sty, the latter of which is no longer maintained and is
% incompatible with some LaTeX packages including fixltx2e. However,
% subfig.sty requires and automatically loads Axel Sommerfeldt's caption.sty
% which will override IEEEtran.cls' handling of captions and this will result
% in non-IEEE style figure/table captions. To prevent this problem, be sure
% and invoke subfig.sty's "caption=false" package option (available since
% subfig.sty version 1.3, 2005/06/28) as this is will preserve IEEEtran.cls
% handling of captions.
% Note that the Computer Society format requires a sans serif font rather
% than the serif font used in traditional IEEE formatting and thus the need
% to invoke different subfig.sty package options depending on whether
% compsoc mode has been enabled.
%
% The latest version and documentation of subfig.sty can be obtained at:
% http://www.ctan.org/pkg/subfig




% *** FLOAT PACKAGES ***
%
%\usepackage{fixltx2e}
% fixltx2e, the successor to the earlier fix2col.sty, was written by
% Frank Mittelbach and David Carlisle. This package corrects a few problems
% in the LaTeX2e kernel, the most notable of which is that in current
% LaTeX2e releases, the ordering of single and double column floats is not
% guaranteed to be preserved. Thus, an unpatched LaTeX2e can allow a
% single column figure to be placed prior to an earlier double column
% figure.
% Be aware that LaTeX2e kernels dated 2015 and later have fixltx2e.sty's
% corrections already built into the system in which case a warning will
% be issued if an attempt is made to load fixltx2e.sty as it is no longer
% needed.
% The latest version and documentation can be found at:
% http://www.ctan.org/pkg/fixltx2e


%\usepackage{stfloats}
% stfloats.sty was written by Sigitas Tolusis. This package gives LaTeX2e
% the ability to do double column floats at the bottom of the page as well
% as the top. (e.g., "\begin{figure*}[!b]" is not normally possible in
% LaTeX2e). It also provides a command:
%\fnbelowfloat
% to enable the placement of footnotes below bottom floats (the standard
% LaTeX2e kernel puts them above bottom floats). This is an invasive package
% which rewrites many portions of the LaTeX2e float routines. It may not work
% with other packages that modify the LaTeX2e float routines. The latest
% version and documentation can be obtained at:
% http://www.ctan.org/pkg/stfloats
% Do not use the stfloats baselinefloat ability as the IEEE does not allow
% \baselineskip to stretch. Authors submitting work to the IEEE should note
% that the IEEE rarely uses double column equations and that authors should try
% to avoid such use. Do not be tempted to use the cuted.sty or midfloat.sty
% packages (also by Sigitas Tolusis) as the IEEE does not format its papers in
% such ways.
% Do not attempt to use stfloats with fixltx2e as they are incompatible.
% Instead, use Morten Hogholm'a dblfloatfix which combines the features
% of both fixltx2e and stfloats:
%
% \usepackage{dblfloatfix}
% The latest version can be found at:
% http://www.ctan.org/pkg/dblfloatfix




%\ifCLASSOPTIONcaptionsoff
%  \usepackage[nomarkers]{endfloat}
% \let\MYoriglatexcaption\caption
% \renewcommand{\caption}[2][\relax]{\MYoriglatexcaption[#2]{#2}}
%\fi
% endfloat.sty was written by James Darrell McCauley, Jeff Goldberg and
% Axel Sommerfeldt. This package may be useful when used in conjunction with
% IEEEtran.cls'  captionsoff option. Some IEEE journals/societies require that
% submissions have lists of figures/tables at the end of the paper and that
% figures/tables without any captions are placed on a page by themselves at
% the end of the document. If needed, the draftcls IEEEtran class option or
% \CLASSINPUTbaselinestretch interface can be used to increase the line
% spacing as well. Be sure and use the nomarkers option of endfloat to
% prevent endfloat from "marking" where the figures would have been placed
% in the text. The two hack lines of code above are a slight modification of
% that suggested by in the endfloat docs (section 8.4.1) to ensure that
% the full captions always appear in the list of figures/tables - even if
% the user used the short optional argument of \caption[]{}.
% IEEE papers do not typically make use of \caption[]'s optional argument,
% so this should not be an issue. A similar trick can be used to disable
% captions of packages such as subfig.sty that lack options to turn off
% the subcaptions:
% For subfig.sty:
% \let\MYorigsubfloat\subfloat
% \renewcommand{\subfloat}[2][\relax]{\MYorigsubfloat[]{#2}}
% However, the above trick will not work if both optional arguments of
% the \subfloat command are used. Furthermore, there needs to be a
% description of each subfigure *somewhere* and endfloat does not add
% subfigure captions to its list of figures. Thus, the best approach is to
% avoid the use of subfigure captions (many IEEE journals avoid them anyway)
% and instead reference/explain all the subfigures within the main caption.
% The latest version of endfloat.sty and its documentation can obtained at:
% http://www.ctan.org/pkg/endfloat
%
% The IEEEtran \ifCLASSOPTIONcaptionsoff conditional can also be used
% later in the document, say, to conditionally put the References on a
% page by themselves.




% *** PDF, URL AND HYPERLINK PACKAGES ***
%
%\usepackage{url}
% url.sty was written by Donald Arseneau. It provides better support for
% handling and breaking URLs. url.sty is already installed on most LaTeX
% systems. The latest version and documentation can be obtained at:
% http://www.ctan.org/pkg/url
% Basically, \url{my_url_here}.





% *** Do not adjust lengths that control margins, column widths, etc. ***
% *** Do not use packages that alter fonts (such as pslatex).         ***
% There should be no need to do such things with IEEEtran.cls V1.6 and later.
% (Unless specifically asked to do so by the journal or conference you plan
% to submit to, of course. )


% correct bad hyphenation here
\hyphenation{op-tical net-works semi-conduc-tor}

\usepackage{balance}  % to better equalize the last page
\usepackage{graphics} % for EPS, load graphicx instead
%\usepackage[T1]{fontenc}
\usepackage{txfonts}
\usepackage{times}    % comment if you want LaTeX's default font
\usepackage[pdftex]{hyperref}
% \usepackage{url}      % llt: nicely formatted URLs
\usepackage{color}
\usepackage{textcomp}
\usepackage{booktabs}
\usepackage{ccicons}


\usepackage{cite}
\usepackage{url}
\usepackage{fancybox}
\usepackage{multirow}
\usepackage{flushend}
\usepackage{booktabs}
\usepackage{tabularx}
\usepackage{comment}
\usepackage{array}
\usepackage[flushleft]{threeparttable}
\usepackage{mdframed}
\graphicspath{{}{images/}{dia/}}
\DeclareGraphicsExtensions{.pdf,.png}


\usepackage{listings}
\usepackage{courier}
\usepackage{hyperref}

\usepackage{xpatch}
\xdef\scr{}

\xpatchcmd{\refstepcounter}{%
  \stepcounter{#1}%
}{%
  \stepcounter{#1}%
  \xdef\scr{\number\value{#1}}%
}{\typeout{success}}{\typeout{failure}}


\newcounter{o}
\setcounter{o}{0}

\usepackage{tikz}
%colors
\definecolor{1c1}{RGB}{188,162,6}
\definecolor{1c2}{RGB}{137,129,80}
\definecolor{1c3}{RGB}{239,167,31}
\definecolor{1c4}{RGB}{88,194,241}
\definecolor{1c5}{RGB}{6,180,188}

% stiles used
\tikzset{mynode/.style={draw=white,solid,circle,fill=green,inner sep=1pt, thick,
text=black}}
%draw=black to get a black circle, fill=white so it actually has a
%background and text=black to not get that rendered in the specified color
\tikzset{arrow line/.style={dashed, line width= 2.5pt, color=#1}}

\def\bf{\textbf}
\def\eq {Equation~}
\def\eqm {Eq~}
\def\eqs {Equations~}
\def\fig {Figure~}
\def\figs {Figures~}
\def\tbl {Table~}
\def\tbls {Tables~}
\def\ie{\textit{i.e.,}}
\def\eg{\textit{e.g.,}}
\def\sec {Section~}
\def\secs {Sections~}
\def\alg {Algorithm~}
\def\algs {Algorithms~}
\def\app {Appendix~}
\def\it{\textit}
\def\tr{\textrm}
\def\tt{\mct}
\newcommand{\ib}[1]{{\textbf {\textit { #1}}}}
\newcommand{\ts}[1]{{\textsc {{ #1}}}}
\newcommand{\mct}[1]{{\footnotesize {\texttt {#1}}}}
\newcommand{\Foutse}[1]{\textcolor{red}{{\it [Foutse says: #1]}}}
\newcommand{\qu}[1]{{\it{``#1''}}}
\newcommand{\api}[1]{{\sf{\texttt\small{#1}}}}
\newcommand{\callout}[1]{{\vspace{1mm}\noindent{\fbox{\parbox{0.97\columnwidth}{#1}}}\vspace{1mm}}}
\usepackage{paralist}

\let\labelindent\relax

\newcommand{\nd}{\vspace{1mm}\noindent}
\usepackage[small,bf]{caption}
%\usepackage[toc,page]{appendix}
\usepackage{tikz}
\newcommand*\circled[1]{\tikz[baseline=(char.base)]{
            \node[shape=circle,draw,inner sep=1pt] (char) {#1};}}

 \lstset{
         language=Java,
         basicstyle=\scriptsize\ttfamily, % Standardschrift
         %numbers=left,               % Ort der Zeilennummern
         numberstyle=\tiny,          % Stil der Zeilennummern
         %stepnumber=2,               % Abstand zwischen den Zeilennummern
         numbersep=5pt,              % Abstand der Nummern zum Text
         tabsize=2,                  % Groesse von Tabs
        % extendedchars=true,         %
         breaklines=true,            % Zeilen werden Umgebrochen
%         keywordstyle=\color{black},
 %   	 frame=single,
 %        keywordstyle=[1]\textbf,    % Stil der Keywords
 %        keywordstyle=[2]\textbf,    %
 %        keywordstyle=[3]\textbf,    %
 %        keywordstyle=[4]\textbf,   \sqrt{\sqrt{}} %
         stringstyle=\color{white}\ttfamily, % Farbe der String
         showspaces=false,           % Leerzeichen anzeigen ?
         showtabs=false,             % Tabs anzeigen ?
         xleftmargin=17pt,
         framexleftmargin=17pt,
         framexrightmargin=5pt,
         framexbottommargin=4pt,
         %backgroundcolor=\color{lightgray},
         showstringspaces=false,      % Leerzeichen in Strings anzeigen ?
     %    escapeinside={\%*}{*)}
 }

\lstdefinestyle{inlinecode}{basicstyle={\ttfamily\scriptsize\bfseries}}
\newcommand\code{\lstinline[style=inlinecode]}
\newcommand{\urls}[1]{{\scriptsize\url{#1}}}
\usepackage{tcolorbox}
\newcommand{\emt}[1]{\emph{``#1''}}
\newcommand{\rev}[1]{\textcolor{blue}{#1}}
\newcommand{\anindya}[1] { \textcolor{green}{\\Anindya: #1\\}}
%\newcommand{\Foutse}[1]{\textcolor{blue}{{\it [Foutse says: #1]}}}
\usepackage{paralist}
\usepackage[outercaption]{sidecap}
\usepackage [autostyle, english = american]{csquotes}
\MakeOuterQuote{"}
\newcounter{scn}
\setcounter{scn}{1}
\usepackage[shortlabels]{enumitem}
\usepackage{bchart}
\begin{document}
\title{Can We Detect API Documentation Smells?}
%\author{}% <-this % stops a
%\author{Gias Uddin and Foutse Khomh \\ SWAT Lab, Polytechnique Montr\'{e}al}% <-this % stops a
% space

%\markboth{IEEE TRANSACTIONS ON SOFTWARE ENGINEERING,~Vol.~X, No.~X,
%Month~2018}%
%{Uddin \MakeLowercase{\textit{et al.}}: Opinion Value Analysis in
%API Reviews}

\IEEEtitleabstractindextext{%
\begin{abstract}
We present a catalog of API documentation smells. We develop a benchmark by manually validating the presence of the smells in Java official API reference 
and instructional documentation. We develop a suite of ML classifiers to automatically detect the smells in the documentation.
\end{abstract}


\begin{IEEEkeywords}
API Documentation, Smell, Empirical Study.
\end{IEEEkeywords}}

%
%\ccsdesc[500]{Software and its engineering~Software libraries and repositories}
%\ccsdesc[300]{Computer systems organization~Redundancy}
%\ccsdesc{Computer systems organization~Robotics}
%\ccsdesc[100]{Networks~Network reliability}


%\keywords{API, Usage Scenario, Crowd-Sourced API Documentation, Summarization}


\maketitle



\IEEEdisplaynontitleabstractindextext
% \IEEEdisplaynontitleabstractindextext has no effect when using
% compsoc or transmag under a non-conference mode.



% For peer review papers, you can put extra information on the cover
% page as needed:
% \ifCLASSOPTIONpeerreview
% \begin{center} \bfseries EDICS Category: 3-BBND \end{center}
% \fi
%
% For peerreview papers, this IEEEtran command inserts a page break and
% creates the second title. It will be ignored for other modes.
\IEEEpeerreviewmaketitle

% https://stackoverflow.com/questions/50634335/golang-errors-and-documentation/50634506
% https://stackoverflow.com/questions/7058294/problem-with-ruby-documentation
% https://stackoverflow.com/questions/5361112/io-documentation-question
% https://stackoverflow.com/questions/51895479/documentation-about-error-in-callback-functions-of-mongoose-methods?rq=1
% https://stackoverflow.com/questions/532338/what-to-do-with-star-developers-who-dont-document-their-work
% https://stackoverflow.com/questions/55971626/problem-with-mapping-values-of-document-from-mongodb
% https://stackoverflow.com/questions/58867678/here-maps-errorunauthorized-error-descriptionapikey-invalid-apikey-no
% https://stackoverflow.com/questions/60800757/sortedset-not-working-as-per-documentation
% https://stackoverflow.com/questions/56051905/inexplicable-syntax-error-when-i-write-one-code-from-the-documentation
% https://stackoverflow.com/questions/43368110/interface-builder-where-is-the-documentation?rq=1
% https://stackoverflow.com/questions/48860513/there-is-error-on-python-documentation
% https://www.google.com/search?q=documentation+problem+site:stackoverflow.com&safe=active&rlz=1C1GCEJ_enCA805CA805&sxsrf=ALeKk00-NhuD8gojWObnmu19zxAGhH1BsA:1591039570299&ei=UlbVXq3cEYOytAaDpKSYCQ&start=50&sa=N&ved=2ahUKEwitzaSjrOHpAhUDGc0KHQMSCZM4KBDw0wN6BAgLED8&cshid=1591039584827092&biw=1920&bih=969

% %------------------------------------------
\section{Introduction}\label{sec:introduction}
% %------------------------------------------
APIs (Application Programming Interfaces) are interfaces to reusable software libraries and frameworks. The proper learning of APIs is paramount to support modern day rapid software development. To achieve this goal, APIs typically are supported by official documentation. An API documentation is a product itself, which warrants the creation and maintenance principles similar to any existing software product. A good documentation can facilitate the proper usage of an API, while a bad documentation can severely harm its adoption~\cite{Robillard-APIsHardtoLearn-IEEESoftware2009a}.
Unfortunately, research shows that API official documentation can be often incomplete, incorrect, and outdated~\cite{Uddin-HowAPIDocumentationFails-IEEESW2015}. 

Despite recent efforts to improve API documentation~\cite{Robillard-OndemandDeveloperDoc-ICSME2017,Treude-APIInsight-ICSE2016,Subramanian-LiveAPIDocumentation-ICSE2014}, 
we observe that discussions about API documentation issues continue among developers in online forums and blog posts. 
Consider four such examples in \figs\ref{fig:MotivatingExampleJacksonComplexity} - \ref{fig:MotivatingExampleInsuffientDoc}. The first example (\fig\ref{fig:MotivatingExampleJacksonComplexity}) 
is from Stack Overflow, a popular online technical Q\&A site for software developers. The question asked to compare two popular Java APIs for JSON parsing, Jackson and GSON. 
The question is very popular, 
it is so far viewed more than 164K times. The asker prefers GSON over Jackson. A developer `dongshengcn', however, 
warns against using Jackson, because the documentation of the Jackson is \emt{getting ridiculously complex}. This complexity problem is acknowledged by `StaxMan', the author of the Jackson 
API. This example demonstrates that the \it{usability} of an API documentation suffers due to growing complexity.
%, 
%but it can be challenging as APIs grow bigger and the documentation becomes more complex. 
%Previous research shows that API documentation usability is necessary to support API usage~\cite{Dekel-APIDocumentationUsabilityDirectives-ICSE2009,Stylos-UsabilityObjectConstructor-ICSE2007a}.
\begin{figure}[t]%[!htb]
  \centering
  %\vspace{-6mm}
   \hspace*{-.4cm}%
  \includegraphics[scale=.52]{images/MotivatingExampleJacksonComplexity}
  \vspace{-4mm}
  \caption{Comments complaining about API documentation complexity}

  \label{fig:MotivatingExampleJacksonComplexity}
%\vspace{-4mm}
\end{figure}
\begin{figure}[t]%[!htb]
  \centering
  \vspace{-4mm}
   \hspace*{-.5cm}%
  \includegraphics[scale=.52]{images/MotivatingExampleAutoGenerateCode}
  \vspace{-4mm}
  \caption{Blog post complaining about auto-generated document}

  \label{fig:MotivatingExampleAutoGenerateCode}
%\vspace{-4mm}
\end{figure}

The second example in \fig\ref{fig:MotivatingExampleAutoGenerateCode} is a blog post from Medium, a popular online site to share insights and knowledge about technologies. 
The blog outlines eight reasons why an API can `suck'. The topmost reason is the problem in API documentation, such as the Kubernetes API documentation which 
was auto-generated: \emt{connect POST requests to attach of Pod}. The sentence is grammatically incorrect and hard to understand. 
This example demonstrates lack of \it{readability} in boiler-plate or auto-generated documentation for APIs.             

\begin{figure}[t]%[!htb]
  \centering
  %\vspace{-6mm}
   \hspace*{-.5cm}%
  \includegraphics[scale=.52]{images/MotivatingSmellUnreadableDocumentation}
  \vspace{-4mm}
  \caption{Stack Overflow question complaining unclear documentation}

  \label{fig:MotivatingExampleUnclearDoc}
%\vspace{-4mm}
\end{figure}
\begin{figure}[t]%[!htb]
  \centering
  %\vspace{-6mm}
   \hspace*{-.7cm}%
  \includegraphics[scale=.5]{images/MotivatingSmellInsufficientDoc}
 % \vspace{-4mm}
  \caption{Tweet complaining about lazy documentation of code}

  \label{fig:MotivatingExampleInsuffientDoc}
%\vspace{-4mm}
\end{figure}
The third example in \fig\ref{fig:MotivatingExampleUnclearDoc} is a question in Stack Overflow, where the asker is confused about a particular Ruby documentation syntax. 
The documentation tries to explain an API call by including a set of parameters in/out of brackets without explaining the parameters and the brackets. The answer explains the 
syntax properly, which was gratefully acknowledged by the asker. This problem warrants for more \it{understandable} API documentation. 
This fourth example in \fig\ref{fig:MotivatingExampleInsuffientDoc} is a tweet from a developer Jamie Dixon, who complains about insufficient/barely provided explanation of an API method 
in a commercial API documentation: \emt{Got to love some commercial API documentation: `GetRowEnumerator(): Gets the Rows Enumerator' - No further information}. This problem demonstrates 
the issues with \it{lazy} documentation.



% The above two examples point us to an important problem with API documentation: despite efforts made by the API documentation writes/authors, the 
% documentation still contain problems that may hinder the usage and selection of the API. Unfortunately, despite significant advances in API documentaion research, 
% the problems still remain prevalent and recurrent. The  observation then leads to the question of how much documentation is actually enough. Indeed, 
% the question in \fig\ref{fig:MotivatingExampleExtensiveDocumentation} asks whether extensive documentation in code base should be considered as a `code smell'. 
% The asker argues for source code that should be explanatory by itself, without any explanation. However, such arguments are overwhelmingly rejected by other users, 
% as evidenced in the comments and answers. According to one comment: \emt{`good code needs little or no comment' - this is total rubbish. Code in itself can only tell you `how', the `why' 
% may not be so obvious.} This opinion is then echoed in the accepted answer. The answerer provides a detailed example using a code snippet: \emt{Do you think it's obvious what the 
% method does? \ldots what is the control is not contained in any of \tt{TabPage}? -- Fortunately, the documentation answers this question.} 
% Thus effective API documentation should explain the `why' and `how' parts of using an API with the utmost clarity.     
% \begin{figure}[t]%[!htb]
%   \centering
%   %\vspace{-6mm}
%    \hspace*{-.5cm}%
%   \includegraphics[scale=.5]{images/MotivatingExampleExtensiveDocumentation}
%   \vspace{-4mm}
%   \caption{Discussions in Stack Overflow META on the connection between documentation and code smell}
% 
%   \label{fig:MotivatingExampleExtensiveDocumentation}
% \vspace{-4mm}
% \end{figure}


Despite significant research efforts to improve API documentation~\cite{Subramanian-LiveAPIDocumentation-ICSE2014,Treude-APIInsight-ICSE2016,Robillard-OndemandDeveloperDoc-ICSME2017}, 
the above examples show that API authors and documentation writers still suffer 
from producing `smelly' documentation. API documentation smells are not necessarity errors/faults/incorrectness in the documentation~\cite{Zhong-APIDocError-ACMSigplan2013,Uddin-HowAPIDocumentationFails-IEEESW2015}. 
Rather smells inform us of sub-optimal design/creation of API documentation, which then hinder their proper usage. 
We thus need tools and techniques to guide the documentation writers 
to produce `not-smelly' API documentation. 

In this paper, we make the following contributions:
\begin{enumerate}[leftmargin=10pt]
  \item \bf{Catalog.} We present a list of API documentation smells. We produce the list by consulting literature on API documentation issues~\cite{Uddin-HowAPIDocumentationFails-IEEESW2015,Aghajani-SoftwareDocIssueUnveiled-ICSE2019}, on code and design smells, and by 
  by consulting software developers working on real-world software projects. The catalog shows a list of 8-10 API documentation smells. We provide examples of each smell.
  \item \bf{Benchmark.} We develop a benchmark by manually analyzing Java documentation. Each entry in the benchmark is an API documentation unit, where we found one more API documentation smell based on our catalog. 
  \item \bf{Algorithm.} We develop a suite of Machine learning classifiers that can automatically the presence of smell in a given API documentation unit.
\end{enumerate}

\section{Data Collection}

\section{API Documentation Smell Catalog}
\subsection{Catalog Creation Process}
we produce the catalog as follows: we consult the literature and software developers.
\subsection{List of Documentation Smells}
present the list of API documentation smell catalog.

\section{Documentation Smells Automatic Detection}
\subsection{Benchmark Creation}
We produce the benchmark as follows. First two authors (Junaed and Tawkat) mutually discussed the documentation units. They also consulted with software developers. We report the agreement between the authors. To 
report the agreement, we do this as follows. For each documentation unit, both authors put two labels: whether any smell present in the unit, and if present the type of smell based on our catalog. 
When the two first authors disagreed, they also consulted other co-authors (Anindya and Gias). We use Cohen's kappa to report the agreement between the first two authors.   



\subsection{Features for Machine Learning Models}
discuss the features

\section{Evaluation}
We analyze and report the performance of the each of the techniques developed
and experimented as part of the algorithms development steps. We report
the performance of the techniques using three standard measures of information
retrieval: precision ($P$), recall ($R$) and F1-measure ($F1$)%, Accuracy ($A$)
(\eqs\ref{eq:precision} - \ref{eq:f-score}).
%\usepackage[nodisplayskipstretch]{setspace}
%\setstretch{3}
%{\scriptsize
%\vspace{-1.0mm}
%\noindent\begin{tabularx}{\columnwidth}{@{}XXX@{}}
\begin{equation}
P  = \frac{TP}{TP+FP} \label{eq:precision}
\end{equation} %&
  \begin{equation}
 R = \frac{TP}{TP+FN}\label{eq:recall}
 \end{equation} %&
 \begin{equation}
 F1 = 2*\frac{P*R}{P+R}\label{eq:f-score}
 \end{equation}
%  \begin{equation}\label{eq:acc}
%  A = \frac{TP+TN}{TP+FP+TN+FN}
%  \end{equation}
%\end{tabularx}
%}
{$TP = $ Number of true positives, $FN =$ Number of false
negatives, $FP = $ Number of false positives, $TN =$ Number of true
negatives.}


\section{Threats to Validity}
threats to validity

\section{Related Work}
related work
\section{Conclusions}
conclusions

\begin{small}
\bibliographystyle{abbrv}
%\bibliography{bibtex}
\bibliography{consolidated}
\end{small}


\end{document}
