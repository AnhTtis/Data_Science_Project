\section{Related Work} \label{sec:related-work}

%Related work is divided into studies on understanding the necessity of code examples in documentation and developing techniques to generate code automatically.

\subsection{Necessity of Code Examples in Documentation}
Studies suggest that example-based documentation is more beneficial to the developers than conventional Javadoc-style approaches~\cite{Cai-FrameworkDocumentation-PhDThesis2000,Carroll-MinimalManual-JournalHCI1987a,
Rossen-SmallTalkMinimalistInstruction-CHI1990a,Meij-AssessmentMinimalistApproachDocumentation-SIGDOC1992, DeSouza-DocumentationEssentialForSoftwareMaintenance-SIGDOC2005}. Shull et al. formed 15 teams out of 43 software enginnering students and asked them to perform some programming task in presence of both traditional and example-based documentation \cite{Shull-InvestigatingReadingTechniquesForOOFramework-TSE2000}. They found out that though the participants had complete knowledge about the usage of traditional documentation, they preferred the example-based ones as they were easier to understand. Forward and Lethbridge conducted a survey on software documentation involving 48 software practitioners which revealed that the text descriptions, examples, and their organization are the most important factors for documentation \cite{Forward-RelevanceSoftwareDocumentationTools-DocEng2002}. Nykaza et al. found that one main property of documentation contents desired by the software developers is the availability of easily understandable code examples \cite{Nykaza-ProgrammersNeedsAssessmentSDKDoc-SIGDOC2002}. Robillard and DeLine interviewed 440 software developers at Microsoft to identify the obstacles they face during learning new APIs \cite{Robillard-FieldStudyAPILearningObstacles-SpringerEmpirical2011a}. Their analysis showed that most issues were related to the official documentation such as the lack of code examples and the absence of task-oriented description.


\subsection{Automatic Code Generation}

Thanks to the amazing insight of Hindle et al. \cite{hindle2016naturalness}, language modeling of source code is now an emerging research area. Primarily, some n-gram based models have been proposed for code suggestion or auto-completion \cite{tu2014localness, nguyen2013statistical}. More recently, different NL$\rightarrow$Code generation models have been developed that generally work by training on bi-modal dataset containing NL and code pairs. Motivated by the bimodal models that map between images and natural language, Allamanis et al. considered similar type of mapping between natural language and source code snippets and built such a model leveraging NL and code pairs collected from Stack Overflow \cite{allamanis2015bimodal}. Yin and Neubig proposed a novel neural architecture which was accompanied by a grammar model to capture the underlying syntax of the target programming language \cite{yin2017syntactic}. Different transformer-based pre-trained models have also been employed for NL$\rightarrow$Code task such as CodeBERT \cite{feng2020codebert}, PLBART \cite{ahmad2021unified}, CoTexT \cite{phan2021cotext}. Research also shows that incorporating external knowledge collected from documentation, developers discussion, online forum, etc. can be useful to improve the performance of the code generation models \cite{hayati2018retrieval, xu2020incorporating,parvez2021retrieval, zhou2022docprompting}.   

Although several researches have been done on automatic code generation from a given natural language intent over the years, none of them directly focused on documentation-specific code examples. In this study, we consider this task and primarily evaluated our proposed approach.        

%% Describe the difference between our work and existing ones in a few lines.