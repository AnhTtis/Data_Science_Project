\section{Introduction}
% ML
With the progressive development in data collection, storage and analysis capacities in recent years, data-driven algorithms, especially machine learning (ML) methods, have gained increasing exposure and importance across academia, industry and social life.
% Accelerator
Particle accelerators have a significant impact on a variety of scientific fields, including the hunt for novel physics~\cite{gibney2022upgraded}, nuclear waste transmutation~\cite{bowman1998accelerator,nema2011application}, and cancer treatment~\cite{amaldi1999cancer,amaldi2004future}.
% Accelerator suits ML
Given that they constantly generate large volume of structured data stream throughout operation, particle accelerators are naturally suitable for application of ML techniques which are known to be powerful in handling highly sophisticated input space with precise objectives~\cite{edelen2016neural}.

% Applications - general
In recent years, there has been a significant rise in the use of data-driven methodologies in theoretical and technical research around particle accelerators~\cite{arpaia2020machine, edelen2018opportunities}, such as fast and accurate beam dynamics modelling that aims to assist with future accelerator design~\cite{zhao2020beam}, high-precision surrogate models that significantly save computing cost compared to original simulations~\cite{adelmann2019nonintrusive}, beam energy optimization that complies with safely restrictions~\cite{kirschner2019bayesian, kirschner2022tuning}, and more specific use cases including optics correction~\cite{fol2019optics} and collimator alignment automation~\cite{azzopardi2019operational}. 
% Applications - forecast
Among all prospective areas of application, not limited to what is listed above, the assurance of safe and stable operation has always been a crucial concern in accelerator control. Preemptive detection or forecasting of anomalous events during accelerator operation would greatly contribute to more accurate and timely control, longer beam time and better beam quality for the users. Similar approaches have already been thoroughly researched in the field of predictive maintenance~\cite{kang2021remaining}, which aims to identify potential breakdowns in advance and apply the necessary maintenance actions in time. Because of the complexity of the data and wide variations among different accelerators, forecasting the future behavior of an accelerator facility is more difficult to formulate than predicting that of a deteriorating engine. Despite that, there have been multiple attempts across various institutions that address different types of anomalous events, but jointly focus on failure prediction from a preemptive anomaly detection perspective, and open up possibilities for subsequent mitigation measures~\cite{li2023review, edelen2016jgh, edelen2016neural}.

% List of examples
%% SNS
\citet{revsvcivc2020predicting, revsvcivc2022improvements} from the Spallation Neutron Source (SNS) at Oakridge use beam current pulses to detect beam loss trips via binary classification. By taking pulses \emph{before} the errant pulse as the positive class and pulses in no-trip operation as the negative class, prediction of the next pulse's behaviour should be realized inside a time budget of \SI{16}{\milli\second}, which is the interval between two consequent pulses. The best performing Random Forest (RF) classifier, together with Principle Component Analysis (PCA) techniques to refine the input features, reaches 96\% accuracy and 61\% recall, and all classifiers could perform the prediction inside \SI{4}{\milli\second}, much faster than the available time budget. A follow up study from~\citet{blokland2022uncertainty} using the same beam current pulse data achieves uncertainty aware prediction by establishing a Siamese~\cite{koch2015siamese} network. 

%% CEBAF
\citet{tennant2020superconducting} at the Continuous Electron Beam Accelerator Facility
(CEBAF) at Jefferson Lab have been focusing on  faults in the cryomodules for superconducting radio-frequency (SRF) cavities. Various machine learning models are applied to enable prompt detection of those SRF faults. With a sequential multi-class classification model, the authors further discover that cavity faults which occur quickly are significantly more difficult to anticipate than those that grow gradually~\cite{rahman2022real}. Also dedicated to SRF anomalies, a recent study by \citet{eichler2023anomaly} at the European X-ray Free Electron Laser (EuXFEL) proposes a novel parity space based approach that detects and distinguishes quenches from other anomalies, following the work of ~\citet{nawaz2018anomaly}. A residual signal, obtained from the deviation in RF waveforms and statistically indicated by the generalized likelihood ratio, serves as a distinct measure for various anomaly categories. The method is experimentally implemented, and a detailed analysis  is given on anomalies that are currently mistakenly classified by the existing method.

Following similar methodology, we present in this paper an effective classification-based approach based on a previous study~\cite{li2021novel}, aiming to forecast the beam interruptions, namely \emph{interlocks}, of the proton cyclotron facility --- High Intensity Proton Accelerators (HIPA) at Paul Scherrer Institut (PSI).

\subsection{Facility and Datasets}

HIPA produces a proton beam of nearly \SI{1.4}{\mega\watt} power, which makes it one of the most powerful proton cyclotron facilities in the world~\cite{reggiani2020improving}. To keep track of the accelerator operation while remaining within safety limits, various sensors and monitors are placed across different locations inside the facility, including beam loss monitors, temperature monitors etc. The interlock system is a necessary security measure that immediately shuts off the beam whenever a problem occurs and some signal exceeds their safety limits, such as the loss monitor shown in red in Figure~\ref{fig:interlock}. However, such shut-downs may lead to abrupt operational changes and a substantial loss of beam time. We propose to build a forecasting model for the interlocks as depicted in the bottom right of Figure~\ref{fig:interlock}. Once the model output indicates an incoming interlock, we expect to apply some recovery operation back onto the facility to avoid the interlock from happening, thus save beam time for the users.

\begin{figure}
    \centering
    \includegraphics{tikz/hipa.tikz}
    \caption{A schematic overview of the HIPA facilities~\cite{kovach2017energy}. The protons are pre-accelerated to \SI{870}{\keV} at a Cockroft-Walton accelerator, then fed into the Injector cyclotron to reach \SI{72}{\MeV}. The beam is then accelerated to the maximum energy of \SI{590}{\MeV} by the large 8-sector RING cyclotron before being transferred to the target stations and experimental regions.}
    \label{fig:hipa}
\end{figure}

\begin{figure}
    \centering
    \includegraphics{tikz/interlock.tikz}
    \caption{The HIPA interlock system with the proposed forecasting model. In this example, an interlock is triggered when the beam loss monitor sees a beam loss higher than the safety threshold (shown as red). Without the forecasting model, the beam would be immediately shut off and equivalently \SI{25}{\second} beam time would be lost (shown as gray). Whereas if an interlock is predicted by the model, some possible recovery operation could be implemented to circumvent the sudden shutdowns.}
    \label{fig:interlock}
\end{figure}

The data used across this paper are collected from a 53-days period in October and November of 2019 excluding beam development days, during which there were 1192 interlocks in total. The dataset is composed of 376 Process Variables -- so-called \emph{channels} -- and the interlock signals from the Experimental Physics and Industrial Control System (EPICS), which are recorded as multivariate time series and interpolated with \SI{5}{\hertz} frequency. Details of data collection and preprocessing are presented in section 2.1 of~\cite{li2021novel}. We formulate the interlock forecasting problem as a binary classification of two classes of samples, as shown in Fig.~\ref{fig:problem}. The positive class consists of \emph{interlock samples} that are taken close to the interlocks, which represent unstable states. The negative class consists of \emph{stable samples} that are taken far from interlocks and represent stable operating states. 

\subsection{Problem Formulation}
The behaviour of some example channels right before an interlock is shown in Figure~\ref{fig:problem}. The \emph{interlock samples}, shown as orange blocks, are taken at $T_h$ seconds before the interlock, where $T_h$ is thus the forecasting horizon. The \emph{stable samples}, shown as green blocks, are taken during stable operation away from interlocks. Each of these samples is in principle a snippet of the $d=376$ multivariate time series of length $\Delta t$.
\begin{figure}[htb]
\centering
\includegraphics{tikz/prob.tikz}
\caption{Illustration of example channels, interlock and sample taking. %\emph{AHA:IST:2} records the magnetic strength of a bending magnet; \emph{BHE2LTA:IST:2} is a temperature monitor; and \emph{MHC1:IST:2} is the beam current.
The orange and green blocks show an \emph{interlock sample} and a \emph{stable sample} respectively. $T_h$ denotes the forecasting horizon, i.e. time of advanced alarm; $\Delta t$ is the duration of each sample.}
\label{fig:problem}
\end{figure}
