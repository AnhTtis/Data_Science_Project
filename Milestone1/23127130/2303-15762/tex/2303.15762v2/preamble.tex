
\usepackage{graphicx}
\usepackage[export]{adjustbox}
\usepackage{grffile}
\usepackage{booktabs}
\usepackage[dvipsnames]{xcolor}
\usepackage{tcolorbox}

\usepackage{cellspace}
\usepackage{makecell}
\usepackage{tabularx}
\usepackage{ragged2e}
\usepackage{multicol}

\usepackage{subcaption}
\usepackage{afterpage}
\usepackage{dblfloatfix}

\usepackage{etoolbox}
\usepackage{xparse}
\usepackage{xifthen}
\usepackage{scalerel}
\usepackage{stackengine}

\usepackage{fancyhdr}

\usepackage{upgreek}
\usepackage{bbm}
\usepackage{thmtools}
\usepackage{mathtools}
\usepackage{amsmath}
\usepackage{amsfonts}

\usepackage{xfrac}
\usepackage{interval}
\usepackage{physics}
\usepackage{tensor}
\usepackage{derivative}

\usepackage[linesnumbered,ruled,vlined]{algorithm2e}

\usepackage{relsize}
\usepackage{scalefnt}
\usepackage[inline]{enumitem}
\usepackage[outline]{contour}
\usepackage[normalem]{ulem}

\usepackage[super]{nth}

\usepackage{tikz}
\usepackage{tikz-3dplot}
\usepackage{tkz-euclide}
\usepackage{pgfplots}

\usepackage{xr-hyper}
\usepackage{hyperref}
\usepackage[nameinlink,capitalize]{cleveref}
\crefname{subsection}{Subsection}{Subsections}
\Crefname{subsection}{Subsection}{Subsections}
\crefname{Table}{Table}{Tables}
\Crefname{Table}{Table}{Tables}

\usepackage{siunitx}

\usepackage{nomencl}
\newcommand*\nomenrefpage[1]{\hspace*{\fill} {\footnotesize\color{gray!20!BlueGreen!85!black} [\cpageref{#1}]}}
\newcommand*\nomenref[1]    {\hspace*{\fill} {\footnotesize\color{gray!20!BlueGreen!85!black} [\cref{#1}]}}


\tcbuselibrary{skins}
\tcbsetforeverylayer{shield externalize}


\usetikzlibrary{angles,arrows,arrows.meta,calc,quotes,intersections,shapes,positioning,scopes,shadings,fadings}
\usetikzlibrary{decorations.markings,decorations.text,decorations.pathmorphing,decorations.pathreplacing,patterns,patterns.meta}
\usetikzlibrary{3d,spy}
\usetikzlibrary{pgfplots.colormaps}
\usetikzlibrary{external}
\tikzexternalize[prefix=tikz_figures/]

\pgfplotsset{compat=newest}
\usepgfplotslibrary{fillbetween}
\usepgfplotslibrary{colormaps}
\usepgfplotslibrary{groupplots}
\usepgfplotslibrary{units}
% Colour maps
\pgfplotsset{
	colormap/magma/.style={%
		/pgfplots/colormap={magma}{%
			rgb=(0.001462, 0.000466, 0.013866)
			rgb=(0.035520, 0.028397, 0.125209)
			rgb=(0.102815, 0.063010, 0.257854)
			rgb=(0.191460, 0.064818, 0.396152)
			rgb=(0.291366, 0.064553, 0.475462)
			rgb=(0.384299, 0.097855, 0.501002)
			rgb=(0.475780, 0.134577, 0.507921)
			rgb=(0.569172, 0.167454, 0.504105)
			rgb=(0.664915, 0.198075, 0.488836)
			rgb=(0.761077, 0.231214, 0.460162)
			rgb=(0.852126, 0.276106, 0.418573)
			rgb=(0.925937, 0.346844, 0.374959)
			rgb=(0.969680, 0.446936, 0.360311)
			rgb=(0.989363, 0.557873, 0.391671)
			rgb=(0.996580, 0.668256, 0.456192)
			rgb=(0.996727, 0.776795, 0.541039)
			rgb=(0.992440, 0.884330, 0.640099)
			rgb=(0.987053, 0.991438, 0.749504)
		},
	},
	colormap/inferno/.style={%
		/pgfplots/colormap={inferno}{%
			rgb=(0.001462, 0.000466, 0.013866)
			rgb=(0.037668, 0.025921, 0.132232)
			rgb=(0.116656, 0.047574, 0.272321)
			rgb=(0.217949, 0.036615, 0.383522)
			rgb=(0.316282, 0.053490, 0.425116)
			rgb=(0.410113, 0.087896, 0.433098)
			rgb=(0.503493, 0.121575, 0.423356)
			rgb=(0.596940, 0.154848, 0.398125)
			rgb=(0.688653, 0.192239, 0.357603)
			rgb=(0.775059, 0.239667, 0.303526)
			rgb=(0.851384, 0.302260, 0.239636)
			rgb=(0.912966, 0.381636, 0.169755)
			rgb=(0.956852, 0.475356, 0.094695)
			rgb=(0.981895, 0.579392, 0.026250)
			rgb=(0.987464, 0.690366, 0.079990)
			rgb=(0.973088, 0.805409, 0.216877)
			rgb=(0.947594, 0.917399, 0.410665)
			rgb=(0.988362, 0.998364, 0.644924)
		},
	},
	colormap/plasma/.style={%
		/pgfplots/colormap={plasma}{%
			rgb=(0.050383, 0.029803, 0.527975)
			rgb=(0.186213, 0.018803, 0.587228)
			rgb=(0.287076, 0.010855, 0.627295)
			rgb=(0.381047, 0.001814, 0.653068)
			rgb=(0.471457, 0.005678, 0.659897)
			rgb=(0.557243, 0.047331, 0.643443)
			rgb=(0.636008, 0.112092, 0.605205)
			rgb=(0.706178, 0.178437, 0.553657)
			rgb=(0.768090, 0.244817, 0.498465)
			rgb=(0.823132, 0.311261, 0.444806)
			rgb=(0.872303, 0.378774, 0.393355)
			rgb=(0.915471, 0.448807, 0.342890)
			rgb=(0.951344, 0.522850, 0.292275)
			rgb=(0.977856, 0.602051, 0.241387)
			rgb=(0.992541, 0.687030, 0.192170)
			rgb=(0.992505, 0.777967, 0.152855)
			rgb=(0.974443, 0.874622, 0.144061)
			rgb=(0.940015, 0.975158, 0.131326)
		},
	},
	colormap/inferno,
	colormap/plasma,
	colormap/magma,
}

\sisetup{
	locale=UK,
	per-mode=fraction,
	separate-uncertainty=false,
    range-phrase=-,
}
\DeclareSIUnit\erg{erg}


\intervalconfig{ soft open fences }


\newenvironment{allowdisplaybreak}{\begingroup\allowdisplaybreaks}{\endgroup}


\DeclareMathAlphabet\mathcalcmsy{OMS}{cmsy}{m}{n}
\DeclareMathAlphabet\mathbfcalcmsy{OMS}{cmsy}{b}{n}
\DeclareMathAlphabet\mathbfcal{OMS}{cmsy}{b}{n}
\DeclareMathAlphabet\mathbfscr{OMS}{mdugm}{b}{n}
\DeclareMathAlphabet\mathantt{OMS}{antt}{r}{n}
\DeclareMathAlphabet\mathbfantt{OMS}{antt}{b}{n}
\DeclareMathAlphabet\mathzapf{T1}{pzc}{mb}{it}
\DeclareMathAlphabet\mathpzc{OT1}{pzc}{m}{it}


\newcommand{\raisemath}[1]{\mathpalette{\@raisemath{#1}}}
\newcommand{\@raisemath}[3]{\raisebox{#1}{$#2#3$}}
\newcommand{\scalemath}[1]{\mathpalette{\@scalemath{#1}}}
\newcommand{\@scalemath}[3]{\scalebox{#1}{$#2#3$}}


% Common math stuff
\newcommand\ii{\mathrm{i}}
\newcommand\mpi{\uppi}
\newcommand\ee{\mathrm{e}}

\let\oldva\va
\renewcommand\va[1]  {\oldva{\boldsymbol #1}}
\newcommand\vab[1]   {\oldva{\pmb{#1}}}
\newcommand\vacal[1] {\oldva{\mathbfcal{#1}}}
\newcommand\vascr[1] {\oldva{\mathbfscr{#1}}}
\newcommand\vacalb[1]{\oldva{\pmb{\mathcal{#1}}}}
\newcommand\vascrb[1]{\oldva{\pmb{\mathscr{#1}}}}
\let\oldvb\vb
\renewcommand\vb[1]  {\oldvb{\boldsymbol #1}}
\newcommand\vbb[1]   {\oldvb{\pmb{#1}}}
\newcommand\vbcal[1] {\oldvb{\mathbfcal{#1}}}
\newcommand\vbscr[1] {\oldvb{\mathbfscr{#1}}}
\newcommand\vbcalb[1]{\oldvb{\pmb{\mathcal{#1}}}}
\newcommand\vbscrb[1]{\oldvb{\pmb{\mathscr{#1}}}}
\let\oldvu\vu
\renewcommand\vu[1]  {\oldvu{\boldsymbol #1}}
\newcommand\vub[1]   {\oldvu{\pmb{#1}}}
\newcommand\vucal[1] {\oldvu{\mathbfcal{#1}}}
\newcommand\vuscr[1] {\oldvu{\mathbfscr{#1}}}
\newcommand\vucalb[1]{\oldvu{\pmb{\mathcal{#1}}}}
\newcommand\vuscrb[1]{\oldvu{\pmb{\mathscr{#1}}}}

\newcommand\opr[1]    {\vu{#1}}
\newcommand\oprb[1]   {\vub{#1}}
\newcommand\oprcal[1] {\vucal{#1}}
\newcommand\oprscr[1] {\vuscr{#1}}
\newcommand\oprcalb[1]{\vucalb{#1}}
\newcommand\oprscrb[1]{\vuscrb{#1}}

\newcommand\mat[1]    {{\boldsymbol #1}}
\newcommand\matb[1]   {\pmb{#1}}
\newcommand\matcal[1] {\mathbfcal{#1}}
\newcommand\matscr[1] {\mathbfscr{#1}}
\newcommand\matcalb[1]{\pmb{\mathcal{#1}}}
\newcommand\matscrb[1]{\pmb{\mathscr{#1}}}
\newcommand\matcalcmsy[1] {\mathbfcalcmsy{#1}}

% Common sets
\newcommand\Natural{\mathbbm{N}}
\newcommand\Integer{\mathbbm{Z}}
\newcommand\Real{\mathbbm{R}}
\newcommand\Rational{\mathbbm{Q}}
\newcommand\Complex{\mathbbm{C}}
\newcommand\unitsphere{\mathcal{S}^2}

% Math symbols
\newcommand*\superperp{{\mathpalette\@superperp{}}}
\newcommand*\@superperp[2]{\raisebox{-0.3ex}{\scalebox{.8}{$\m@th#1\perp$}}}
\newcommand*\superparallel{{\mathpalette\@superparallel{}}}
\newcommand*\@superparallel[2]{\raisebox{-0.3ex}{\scalebox{.8}{$\m@th#1\parallel$}}}
\newcommand*\subperp{{\mathpalette\@subperp{}}}
\newcommand*\@subperp[2]{\raisebox{0.2ex}{\scalebox{.8}{$\m@th#1\perp$}}}
\newcommand*\subparallel{{\mathpalette\@subparallel{}}}
\newcommand*\@subparallel[2]{\raisebox{0.2ex}{\scalebox{.8}{$\m@th#1\parallel$}}}

\newcommand\overdarrow[2]{\ensurestackMath{%
	\savestack{\tmpbox}{\stretchto{%
		\scaleto{%
			\scalerel*[\widthof{\ensuremath{#1}}]{\kern.9pt\leftrightarrow\kern.4pt}%
					  {\rule[-\textheight/2]{1ex}{\textheight}}
		}{\textheight}%
	}{.66ex}}%
	\stackon[#2]{#1}{\tmpbox}%
}}
\newcommand*\dvec[1]{\overdarrow{#1}{.45pt}}

\newcommand\overstar[1]{
    \def\useanchorwidth{T}
    \stackon[.02\baselineskip]{\ensuremath{#1}}{$\mkern-.2mu\scriptstyle\star$}
}

\newcommand*\transpose{{\mathpalette\@transpose{}}}
\newcommand*\@transpose[2]{\raisebox{\depth}{$\m@th#1\intercal$}}
\DeclareDocumentCommand\conj{g}{\IfNoValueTF{#1}{\star}{#1^\star}}
\newcommand*\composition{\raisebox{0.5ex}{\scalebox{.375}{$\bigcirc$}}}

% Common operators and functions
\DeclareMathOperator\diag{diag}
\DeclareMathOperator\Cov{Cov}
\DeclareMathOperator\Var{Var}
\DeclareMathOperator\Corr{Corr}
\DeclareMathOperator\ad{ad}
\DeclareMathOperator\refl{refl}
\DeclareMathOperator\tri{tri}
\DeclareMathOperator\sinc{sinc}
\DeclareMathOperator\cond{cond}
\DeclareMathOperator\bernoulli{Bernoulli}
\DeclareMathOperator\unitpshere{S^2}
\DeclareMathOperator\deltaop{\reflectbox{\rotatebox[origin=c]{180}{$\grad$}}}
\DeclareMathOperator\waveop{\scalebox{1.2}{$\triangle$}_{m}}
\DeclareDocumentCommand\rect{g}{\IfNoValueTF{#1}{\operatorname{rect}}{\fbraces{}{}{\operatorname{rect}}{#1}}}
\DeclareDocumentCommand\circ{g}{\IfNoValueTF{#1}{\operatorname{circ}}{\fbraces{}{}{\operatorname{circ}}{#1}}}
\DeclareDocumentCommand\besselJ{m}{\ensuremath{\mathrm{J}_{\mkern-2mu#1}}}
\DeclareDocumentCommand\besselI{m}{\ensuremath{\mathrm{I}_{\mkern-2mu#1}}}

\DeclareMathOperator\dirac{\delta}
\DeclareDocumentCommand\kdelta{g}{\IfNoValueTF{#1}{\operatorname{\dirac}}{\operatorname{\dirac}_{#1}}}

\newcommand\kronprod{\otimes}
\newcommand\cartesianprod{{\boldsymbol\times}}

\DeclareDocumentCommand\wavefunc{}{
	\psiup
}

% Inner product
\DeclarePairedDelimiterX{\definp}[2]{\langle}{\rangle}{#1\,\delimsize\vert\,#2}
\def\inp{\@ifstar{\definp}{\definp*}}

% Statistical analysis
\DeclareDocumentCommand\ensemble{sm}{
	\IfBooleanTF{#1}{\langle#2\rangle}{\left\langle#2\right\rangle}
}
\DeclareDocumentCommand\ensemblew{sm}{
	\IfBooleanTF{#1}{\langle#2\rangle_\omega}{\left\langle#2\right\rangle_\omega}
}
\DeclareDocumentCommand\tavg{sm}{
	\IfBooleanTF{#1}{\langle#2\rangle_\textrm{t}}{\left\langle#2\right\rangle_\textrm{t}}
}
\DeclareDocumentCommand\realization{o m}{\IfNoValueTF{#1}{\tensor*{#2}{}}{\tensor*[^{#1}]{#2}{}}}

% Typesetting linear operators, Fourier, FrFT, etc.
\DeclareDocumentCommand\mathoperator{m m m}{
	\def\op{\mathchoice%
		{\displaystyle 		\scalebox{1.15}{$#1$}}%
		{\textstyle 		\scalebox{1.0} {$#1$}}%
		{\scriptstyle 		\scalebox{0.95}{$#1$}}%
		{\scriptscriptstyle	\scalebox{0.9} {$#1$}}%
	}
	\def\sub{\mathchoice%
		{\displaystyle 		#3}%
		{\textstyle 		#3}%
		{\scriptstyle 		\scalebox{.9}{$#3$}}%
		{\scriptscriptstyle	\scalebox{.8}{$#3$}}%
	}
	\tensor*{\op}{%
		_{\IfNoValueTF{#3}{}{\mkern-1mu\sub}}
		^{\IfNoValueTF{#2}{}{#2}}
	}
}
\DeclareDocumentCommand\fourieroperator{m m}{\mathoperator{\mathscr{F}}{#1}{#2}}
\DeclareDocumentCommand\frft{s o o g}{%
	\def\op{\fourieroperator{#2}{#3}}
	\def\body{#4 \vphantom{\fourieroperator{}{}}}
	\IfNoValueTF{#4}{\fourieroperator{#2}{#3}}{
		\IfBooleanTF{#1}
			{\op\lbrace\body\rbrace}
			{\fbraces{\lbrace}{\rbrace}{\op}{\body}}
	}%
}
\DeclareDocumentCommand\wvdoperator{m m}{\mathoperator{\mathscr{W}}{#1}{#2}}
\DeclareDocumentCommand\wvd{o}{
	\mathscr{W}_{\IfNoValueTF{#1}{}{\mkern-1.2mu#1}}
	\IfNoValueTF{#1}{\mkern2mu}{}
}
\DeclareDocumentCommand\husimiQ{o}{
	\mathscr{Q}_{\IfNoValueTF{#1}{}{\mkern-.5mu#1}}
}
\DeclareDocumentCommand\conv{o}{
    \tensor*{\ast}{_{\IfNoValueTF{#1}{}{\mkern-3.5mu#1}}}
}

\renewcommand{\grad}{\nabla}
\DeclareDocumentCommand\laplacianp{m o d()}{ % Primed Laplacian
	\IfNoValueTF{#1}{%
		\IfNoValueTF{#2}{%
            \IfNoValueTF{#3}{\nabla^{\prime 2}}{\fbraces{\lparen}{\rparen}{\nabla^{\prime 2}}{#3}}%
        }{\fbraces{\lbrack}{\rbrack}{\nabla^{\prime 2}}{#2} \IfNoValueTF{#3}{}{(#3)}}%
    }{\nabla^{\prime 2} #1 \IfNoValueTF{#2}{}{[#2]} \IfNoValueTF{#3}{}{(#3)}}%
}
\DeclareDocumentCommand\gradientp{m o d()}{ % Primed Gradient
	\IfNoValueTF{#1}{%
		\IfNoValueTF{#2}{%
            \IfNoValueTF{#3}{\vnabla^\prime}{\fbraces{\lparen}{\rparen}{\vnabla^\prime}{#3}}%
        }{\fbraces{\lbrack}{\rbrack}{\vnabla^\prime}{#2} \IfNoValueTF{#3}{}{(#3)}}%
    }{\vnabla^\prime #1 \IfNoValueTF{#2}{}{[#2]} \IfNoValueTF{#3}{}{(#3)}}%
}
\DeclareDocumentCommand\gradp{}{\gradientp} % Shorthand for \gradientp

\newcommand\argsep{\ {\scalebox{1.2}{$\boldsymbol ;$}}\ } %Semi-colon separator

% Hyper-geometric functions
\newmuskip\pFqmuskip
\newcommand*\pFq[6][8]{%
    \begingroup %
    \pFqmuskip=#1mu\relax
    \mathcode`\,=\string"8000
    \begingroup\lccode`\~=`\,
    \lowercase{\endgroup\let~}\pFqcomma
    \tensor*[_#2]{F}{_{#3}}
    \ifthenelse{\isempty{#4}}{}{\qty[\genfrac..{0pt}{}{#4}{#5} \argsep #6]}%
    \endgroup
}
\newcommand*\nFq[5][7]{%
    \begingroup %
    \pFqmuskip=#1mu\relax
    \mathcode`\,=\string"8000
    \begingroup\lccode`\~=`\,
    \lowercase{\endgroup\let~}\pFqcomma
    \tensor*[_#2]{F}{_{#3}}
    \ifthenelse{\isempty{#4}}{}{\qty[#4 \argsep #5]}%
    \endgroup
}
\newcommand\pFqcomma{\mskip\pFqmuskip}

\newcommand*\genfactorial[1]{{!}_{\qty(#1)}}

% double and triple primes
\newcommand*\dprime{{\prime\prime\mkern-1.2mu}}
\newcommand*\trprime{{\prime\prime\prime\mkern-1.2mu}}

% pretty fractions
\newcommand\prettyfrac[2]{\fontfamily{ppl}\selectfont\sfrac{#1}{#2}}
\newcommand\half[0]{\prettyfrac{1}{2}}
\newcommand\threeovertwo[0]{\prettyfrac{3}{2}}
\newcommand\third[0]{\prettyfrac{1}{3}}
\newcommand\twothirds[0]{\prettyfrac{2}{3}}
\newcommand\quarter[0]{\prettyfrac{1}{4}}
\newcommand\threequarters[0]{\prettyfrac{3}{4}}


% variation of paragraph that is included in table-of-content
\newcommand\parsection[1]{\paragraph{#1}\addcontentsline{toc}{subsection}{\protect\numberline{}#1}}


\newcommand{\hardsec}{$\scalebox{1.2}{$\boldsymbol{\ast}$}$}


\newenvironment{sizeddisplay}[1]
	{\par\nopagebreak#1\noindent\ignorespaces}
	{\nopagebreak\ignorespacesafterend}
    

% special enumerate envs
\newenvironment{enuminline}
    {\begin{enumerate*}[label=(\roman*)]}
    {\end{enumerate*}}
\newenvironment{enuminlinealph}
    {\begin{enumerate*}[label=(\alph*)]}
    {\end{enumerate*}}
\newenvironment{enumtext}[1]
	{\begin{enumerate}[label=\uline{\textit{#1 \arabic*:}},itemindent=*,wide,leftmargin=4pt,rightmargin=4pt]}
	{\end{enumerate}}
	

% "Derivation" proof env
\newenvironment{derivation}{\paragraph{\normalfont\scshape\selectfont Derivation}}{\hfill\qedsymbol\newline}


\newlist{properties}{enumerate}{1}
\setlist[properties]{label=\textbf{(\Roman*)},
                     ref=(\Roman{*})}
\crefname{propertiesi}{property}{properties}
\Crefname{propertiesi}{Property}{Properties}

% Theorem envs
\newlist{thmlist}{enumerate}{1}
\setlist[thmlist]{label=(\roman{thmlisti}),
                  ref=\thetheorem.(\roman{thmlisti}),
                  noitemsep}
\Crefname{thmlisti}{Theorem}{Theorems}
\addtotheorempostheadhook[theorem]{\crefalias{thmlisti}{listthm}}

\newlist{deflist}{enumerate}{1}
\setlist[deflist]{label=(\roman{deflisti}),
                  ref=\thetheorem.(\roman{deflisti}),
                  noitemsep}
\Crefname{deflisti}{Definition}{Definitions}
\addtotheorempostheadhook[definition]{\crefalias{deflisti}{listdef}}

\newtheorem*{remark}{Remark}
\newtheorem{property}{Property}[section]



% Special symbols
\newcommand\flux{\Phiup}
\newcommand\radiance{L}
\DeclareMathOperator\csd{\mathcalcmsy{C}}
\newcommand\SP{\va{S}}
\newcommand\SUP{\SP_\text{0}}
\newcommand\SLHP{\SP_{\scalebox{.6}{LHP}}}
\newcommand\SLVP{\SP_{\scalebox{.6}{LVP}}}
\DeclareDocumentCommand\SX{g}{
	\SP_{\scalebox{.8}{c}}\IfNoValueF{#1}{\qty(#1)}
}
\DeclareDocumentCommand\gSP{so}{
	\IfBooleanTF{#1}
		{\mathcalcmsy{S}}
		{\vphantom{\widehat{\matcalcmsy{S}}} \smash[t]{\dvec{\matcalcmsy{S}}}}
		^{\refframe{
			\IfNoValueTF{#2}{\vb{\mu}}{#2}
		}}
}
\DeclareDocumentCommand\genrad{so}{
	\IfBooleanTF{#1}
		{\lppscalar}
		{\vphantom{\widehat{\lpp}} \smash[t]{\overdarrow{\lpp}{.3pt}}}
		^{\refframe{
			\IfNoValueTF{#2}{\vb{\mu}}{#2}
		}}
}

\newcommand\Fresnelc{\mathpzc{f}}

\newcommand\LSMautocor{\mat{\rho}_{J\!J}}
\newcommand\LSMstatauto{\mat{R}_{J\!J}}
\newcommand\ACF{\mathpzc{r}}
\newcommand\PSD{\mathpzc{p}}
\DeclareDocumentCommand\lsmM{sg}{
	\bar{\mat{m}}
	\IfNoValueTF{#2}{}{\IfBooleanTF{#1}{(#2)}{\qty(#2)}}
}
\newcommand\pBSDF{\mat{M}}
\newcommand\wBSDF{\matcal{W}}

\newcommand\LSMangspectrum{\widetilde{\mat{M}}}

\DeclareDocumentCommand\anisoGaussian{o}{
	\IfNoValueTF{#1} {\mathpzc{g}}
					 {\mathpzc{g}^{\raisemath{-.55ex}{#1}}}
}

\DeclareDocumentCommand\refframe{sg}{
	\IfBooleanTF{#1} {{[\!#2\!]}}
					 {{\raisemath{-.1em}{\scalemath{.825}{[\!#2\!]}}}}
}
\DeclareDocumentCommand\refframesub{sg}{
	\IfBooleanTF{#1} {{[\!#2\!]}}
					 {{\raisemath{.05em}{\scalemath{.825}{[\!#2\!]}}}}
}
\newcommand\tranfmat[1]{\mat{Q}_{\raisemath{.05em}{\scalemath{.85}{[\!#1\!]}}}}
\newcommand\supo{{\raisemath{-.65em}{\scalemath{.9}{(\mathrm{o})}}}}
\newcommand\supi{{\raisemath{-.65em}{\scalemath{.9}{(\mathrm{i})}}}}
\newcommand\supe{{\raisemath{-.65em}{\scalemath{.9}{(\mathrm{e})}}}}
\newcommand\subo{{\scalemath{.9}{(\mathrm{o})}}}
\newcommand\subi{{\scalemath{.9}{(\mathrm{i})}}}
\newcommand\sube{{\scalemath{.9}{(\mathrm{e})}}}

  