
\begin{figure}[t]%
    \centering
    \tikzset{external/optimize=true}%
    \tikzsetnextfilename{twin_slit}%
    \resizebox{.90\linewidth}{!}{\begin{tikzpicture}[
        ]
    \end{tikzpicture}}
    \centering{\phantomsubcaption\label{fig_twin_slit_a}\phantomsubcaption\label{fig_twin_slit_b}}
    \Description{}%
    \caption{
        \textbf{Diffraction through double slits.}
        (a) Schematic of Young's double slit experiment. 
        A pair of slits, of width $b$ and spaced a distance $d$ apart, are cut in a thin, conductive plate.
        A coherent plane wave (illustrated in green) diffracts through the slits, and is observed upon a screen, placed at a distance $z$ from the plate.
        The superposition of coherent light from both slits results in a rapidly-oscillating phasor $\varphi$ (illustrated in red), producing an interference pattern.
        % See \cref{section_gr_analysis} for more details.
        (b) The experiment is performed with increasing slit distances $d$, and we compare our method (sampling incident light with generalized rays) with a ground truth (explicitly diffracting the plane wave through the slits).
        Differences are plotted in the insets at the bottom right of each pattern (intensity of peak fringe is 1). %, and are negligible.
        The experiment was performed with wavelength $\lambda=1$ (arbitrary units), $z=10000\lambda$ and $b=40\lambda$. 
    }%
    \label{fig_twin_slit}
    \vspace*{-1mm}
\end{figure}
