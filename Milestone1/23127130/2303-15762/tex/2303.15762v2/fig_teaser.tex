
\begin{teaserfigure}
    \centering
	\tikzset{external/optimize=true}
	\tikzsetnextfilename{teaser\imagesmode}%
	\begin{subfigure}{\teasersize\linewidth}%
        \makebox[\textwidth][c]{\resizebox{1\linewidth}{!}{\begin{tikzpicture}[
            ]
        \end{tikzpicture}}}
	\end{subfigure}%
    \vspace*{\teasercaptionoffset}
    \caption{
        \textbf{From ray optics to wave optics.}
        A scene rendered with our technique, showcasing various wave effects: (a) a Bornite ore with an interfering layer of copper oxide; (b) a Brazilian Rainbow Boa, whose scales are biological diffraction-grated surfaces; and (c) a Chrysomelidae beetle, whose colour arises due to multilayered interference reflectors in its elytron.
        The practical contribution of this paper is the ability to render such complex scenes, under rigorous wave-optical light transport, at a performance that surpasses the state-of-the-art by orders-of-magnitude.
        The objective of this paper is not the appearance reproduction of some material (a ``diffractive BRDF''), but an accurate formulation of wave-optical light transport, where light is rigorously treated as waves globally throughout the entire scene.
        We propose the \emph{generalized ray}: an extension of the classical ray to wave optics.
        The generalized ray retains the defining characteristics of the ray-optical ray: \emph{locality} and \emph{linearity}.
        These properties allow the generalized ray to serve as a ``point query'' of light's behaviour, the same purpose that the classical ray fulfils in rendering.
        % By using such generalized rays, we enable the rendering of complex scenes, like the one shown, under rigorous wave-optical light transport.
        % Our formalism serves as a link between path tracing techniques and wave optics, and admits a highly general validity domain. 
        Generalized rays enable the application of backward (sensor-to-source) light transport and sophisticated sampling techniques, which are impossible with the state-of-the-art. %, and achieve performance that surpasses the state-of-the-art by orders-of-magnitude.
        We indicate resolution and samples-per-pixel (spp) count in all figures rendered using our method.
        While these figures showcase converged (high spp) results, our implementation also allows interactive rendering of all these scenes at 1 spp.
        Frame times (at 1 spp) for interactive rendering are indicated.
        See our supplemental material for the implementation, and additional renderings and videos.
    }
	\Description{}
	\label{fig_teaser}
\end{teaserfigure}
