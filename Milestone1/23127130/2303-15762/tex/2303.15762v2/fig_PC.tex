
\begin{figure*}[t]%
    \centering
    \tikzset{external/optimize=true}%
    \tikzsetnextfilename{PC}%
    \resizebox{1\linewidth}{!}{\begin{tikzpicture}[
        ]
    \end{tikzpicture}}
    \vspace*{-4mm}
    \Description{}%
    \caption{
        \textbf{Partially-coherent sampling.}
        (a) A scene contains a compact disk (CD) that rests next an open CD case, upon which another closed CD case is placed.
        A ceiling-mounted light source illuminates the scene.
        While simple, the scene admits interesting light transport.
        (b) We emulate partially-coherent (PC) sampling of BSDFs, in an identical manner to the state-of-the-art, which manifests the sampling problem: 
        the diffraction lobes are very sharp lobes that are difficult to sample when doing backward path tracing (and the coherence properties of light are unknown) using existing tools.
        The close ups show the area marked in orange rendered at various sample counts.
        Note the sharp difference in noise between PC sampling and our proposed light transport formalism with generalized rays.
        (c) Plot of noise as function of sample count, quantifying the drastic improvement in sampling performance: sampling using generalized rays reduces sample count by a factor of about 4000 for similar-quality rendering.
        (a, inset) A difference image between the two sampling strategies. 
        With the exception of the very high-frequency diffraction effects, the differences are minor, and are due to the PC rendering never converging.
        % Note that we formally show that both sampling strategies are equivalent in terms of their accuracy.
        % The rendering runtime of 2 million samples is about \num{21} hours for both approaches.
    }%
    \label{fig_PC}
\end{figure*}
