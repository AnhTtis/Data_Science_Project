
\begin{figure}[t]%
    \centering
    \tikzset{external/optimize=true}%
    \tikzsetnextfilename{sample_solve}%
    \resizebox{.9\linewidth}{!}{\begin{tikzpicture}[
        ]
    \end{tikzpicture}}
    \vspace*{-2mm}
    \Description{}%
    \caption{
        \textbf{Sample-solve.}
        Our path tracing algorithm (a) uses generalized rays (dotted lines) to \emph{sample} paths through the scene.
        Generalized rays are always linear, therefore classical sampling techniques apply essentially unchanged.
        Once a path is sampled (solid red path), we (b) \emph{solve} for the partially-coherent light transport, by applying PLT \cite{Steinberg_practical_plt_2022} from the light source to the sensor across all intermediate interactions.
        % As optical coherence is (inversely) related to the angular spread of rays in a ray bundle (\cref{fig_generalized_rays_coherence}), the \emph{solve} step should be understood as tracing a ray bundle (illustrated in blue) with a statistical distribution of generalized rays over the small solid angle subtended by the sensor from the source (across all intermediate interactions).
    }%
    \label{fig_sample_solve}
    \vspace*{-1mm}
\end{figure}
