
\begin{figure}[t]%
    \centering
    \tikzset{external/optimize=true}%
    \tikzsetnextfilename{ray_sampling_problem}%
    \resizebox{.75\linewidth}{!}{\begin{tikzpicture}[
        ]
    \end{tikzpicture}}
    \Description{}%
    \vspace*{-1mm}
    \caption{
        \textbf{The sampling problem.}
        Existing light transport formalisms, like partially-coherent light transport, work by evolving light's properties from the source, through the scene, until light is sensed by a detector.
        Such formalisms are inherently incompatible with backward (sensor-to-source) models of light transport:
        they are not able to formulate a light-matter interaction (a BSDF) without knowledge of light's wave properties, however these properties depend on the light source and evolve throughout the scene, hence are very difficult to predict or estimate in a backward model \cite{Steinberg_practical_plt_2022}.
        Fundamental path tracing and sampling techniques, like importance sampling of interactions, cannot be applied, greatly hampering the practicality and ability of these formalisms to work with complex, real-world scenes.
        Solving this \emph{sampling problem}, i.e. devising a formalism of backward wave-optical light transport, where a wide-range of sampling techniques can be applied, is the primary motivation for this paper.
        % To importance sample partially-coherent BSDFs (or light-matter interactions in general), the coherence properties of light are needed.
        % However, these properties depend on the light source, and change on propagation and interaction with matter.
        % Therefore, these properties are not available, and are difficult to predict or estimate, when tracing paths \emph{backwards}, i.e. starting from a camera.
        % For a variety of reasons, path tracers trace paths backward-only, or bidirectionally (forward and backward), hence limiting our ability to perform backward sampling is a real problem.
        % This sampling problem applies to PLT, as well as any framework where the descriptor of light quantifies light's optical coherence.
    }%
    \label{fig_sampling_problem}
    \vspace*{-1mm}
\end{figure}
