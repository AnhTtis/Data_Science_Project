
\begin{figure}[t]%
    \centering
    \resizebox{1\linewidth}{!}{\includegraphics{tikz_figures/phase_space_ray.pdf}}
    \vspace*{-5mm}
    \Description{}%
    \caption{
        \textbf{The generalized ray in phase space.}
        It is often convenient to depict the distribution of light in a space known as \emph{phase space}: an artificial space that can be loosely understood as the Cartesian product of position and direction of propagation.
        In this phase space, at a particular time instant, a ray under ray optics is a point---functionally a Dirac delta---as its position and direction are exactly known.
        Under wave optics, such a distribution of light is prohibited by the uncertainty relation (\cref{eq_uncertainty_relation}), which discretizes phase space into overlapping Gaussian cells (size of which is illustrated by the grid).
        While a Dirac delta in phase space is aphysical, the convolution of that Dirac with a \emph{minimum uncertainty Gaussian} (a Gaussian with the least variance that fulfils the uncertainty relation, and occupies a single cell) is a physically-realizable \emph{closest analogue to the ray} under wave optics, which we term the \emph{generalized ray}.
    }%
    \label{fig_phase_space_ray}
\end{figure}
