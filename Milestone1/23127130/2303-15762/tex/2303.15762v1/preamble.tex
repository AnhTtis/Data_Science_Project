
\usepackage{graphicx}
\usepackage[export]{adjustbox}
\usepackage{grffile}
\usepackage{booktabs}
\usepackage[dvipsnames]{xcolor}
\usepackage[most]{tcolorbox}
\usepackage{cellspace}
\usepackage{makecell}
\usepackage{tabularx}
\usepackage{ragged2e}
\usepackage{multicol}

\usepackage{subcaption}
\usepackage{afterpage}
\usepackage{dblfloatfix}

\usepackage{etoolbox}
\usepackage{xparse}
\usepackage{xifthen}
\usepackage{scalerel}
\usepackage{stackengine}

\usepackage{upgreek}
\usepackage{bbm}
\usepackage{thmtools}
\usepackage{mathtools}
\usepackage{amsmath}
\usepackage{amsfonts}

\usepackage{xfrac}
\usepackage{interval}
\usepackage{physics}
\usepackage{tensor}
\usepackage{derivative}

\usepackage[linesnumbered,ruled,vlined]{algorithm2e}

\usepackage{relsize}
\usepackage{scalefnt}
\usepackage[inline]{enumitem}
\usepackage[outline]{contour}
\usepackage[normalem]{ulem}

\usepackage[super]{nth}

\usepackage{xr-hyper}
\usepackage{hyperref}
\usepackage[nameinlink,capitalize]{cleveref}
\crefname{subsection}{Subsection}{Subsections}
\Crefname{subsection}{Subsection}{Subsections}
\crefname{Table}{Table}{Tables}
\Crefname{Table}{Table}{Tables}

\usepackage{siunitx}

\sisetup{
	locale=UK,
	per-mode=fraction,
	separate-uncertainty=false,
    range-phrase=-,
}
\DeclareSIUnit\erg{erg}


\DeclareMathAlphabet\mathcalcmsy{OMS}{cmsy}{m}{n}
\DeclareMathAlphabet\mathbfcalcmsy{OMS}{cmsy}{b}{n}
\DeclareMathAlphabet\mathbfcal{OMS}{cmsy}{b}{n}
\DeclareMathAlphabet\mathbfscr{OMS}{mdugm}{b}{n}
\DeclareMathAlphabet\mathantt{OMS}{antt}{r}{n}
\DeclareMathAlphabet\mathbfantt{OMS}{antt}{b}{n}
\DeclareMathAlphabet\mathzapf{T1}{pzc}{mb}{it}
\DeclareMathAlphabet\mathpzc{OT1}{pzc}{m}{it}


% Common math stuff
\newcommand\ii{\mathrm{i}}
\newcommand\mpi{\uppi}
\newcommand\ee{\mathrm{e}}

\let\oldva\va
\renewcommand\va[1]  {\oldva{\boldsymbol #1}}
\newcommand\vab[1]   {\oldva{\pmb{#1}}}
\newcommand\vacal[1] {\oldva{\mathbfcal{#1}}}
\newcommand\vascr[1] {\oldva{\mathbfscr{#1}}}
\newcommand\vacalb[1]{\oldva{\pmb{\mathcal{#1}}}}
\newcommand\vascrb[1]{\oldva{\pmb{\mathscr{#1}}}}
\let\oldvb\vb
\renewcommand\vb[1]  {\oldvb{\boldsymbol #1}}
\newcommand\vbb[1]   {\oldvb{\pmb{#1}}}
\newcommand\vbcal[1] {\oldvb{\mathbfcal{#1}}}
\newcommand\vbscr[1] {\oldvb{\mathbfscr{#1}}}
\newcommand\vbcalb[1]{\oldvb{\pmb{\mathcal{#1}}}}
\newcommand\vbscrb[1]{\oldvb{\pmb{\mathscr{#1}}}}
\let\oldvu\vu
\renewcommand\vu[1]  {\oldvu{\boldsymbol #1}}
\newcommand\vub[1]   {\oldvu{\pmb{#1}}}
\newcommand\vucal[1] {\oldvu{\mathbfcal{#1}}}
\newcommand\vuscr[1] {\oldvu{\mathbfscr{#1}}}
\newcommand\vucalb[1]{\oldvu{\pmb{\mathcal{#1}}}}
\newcommand\vuscrb[1]{\oldvu{\pmb{\mathscr{#1}}}}

\newcommand\opr[1]    {\vu{#1}}
\newcommand\oprb[1]   {\vub{#1}}
\newcommand\oprcal[1] {\vucal{#1}}
\newcommand\oprscr[1] {\vuscr{#1}}
\newcommand\oprcalb[1]{\vucalb{#1}}
\newcommand\oprscrb[1]{\vuscrb{#1}}

\newcommand\mat[1]    {{\boldsymbol #1}}
\newcommand\matb[1]   {\pmb{#1}}
\newcommand\matcal[1] {\mathbfcal{#1}}
\newcommand\matscr[1] {\mathbfscr{#1}}
\newcommand\matcalb[1]{\pmb{\mathcal{#1}}}
\newcommand\matscrb[1]{\pmb{\mathscr{#1}}}
\newcommand\matcalcmsy[1] {\mathbfcalcmsy{#1}}

% Common sets
\newcommand\Natural{\mathbbm{N}}
\newcommand\Integer{\mathbbm{Z}}
\newcommand\Real{\mathbbm{R}}
\newcommand\Rational{\mathbbm{Q}}
\newcommand\Complex{\mathbbm{C}}
\newcommand\unitsphere{\mathcal{S}^2}

% Common operators and functions
\DeclareMathOperator\diag{diag}
\DeclareMathOperator\Cov{Cov}
\DeclareMathOperator\Var{Var}
\DeclareMathOperator\Corr{Corr}
\DeclareMathOperator\ad{ad}
\DeclareMathOperator\refl{refl}
\DeclareMathOperator\tri{tri}
\DeclareMathOperator\sinc{sinc}
\DeclareMathOperator\cond{cond}
\DeclareMathOperator\bernoulli{Bernoulli}
\DeclareMathOperator\unitpshere{S^2}
\DeclareMathOperator\deltaop{\reflectbox{\rotatebox[origin=c]{180}{$\grad$}}}
\DeclareMathOperator\waveop{\scalebox{1.2}{$\triangle$}_{m}}
\DeclareDocumentCommand\rect{g}{\IfNoValueTF{#1}{\operatorname{rect}}{\fbraces{}{}{\operatorname{rect}}{#1}}}
\DeclareDocumentCommand\circ{g}{\IfNoValueTF{#1}{\operatorname{circ}}{\fbraces{}{}{\operatorname{circ}}{#1}}}
\DeclareDocumentCommand\besselJ{m}{\ensuremath{\mathrm{J}_{\mkern-2mu#1}}}
\DeclareDocumentCommand\besselI{m}{\ensuremath{\mathrm{I}_{\mkern-2mu#1}}}

\DeclareMathOperator\dirac{\delta}
\DeclareDocumentCommand\kdelta{g}{\IfNoValueTF{#1}{\operatorname{\dirac}}{\operatorname{\dirac}_{#1}}}

\newcommand\kronprod{\otimes}
\newcommand\cartesianprod{{\boldsymbol\times}}

\DeclareDocumentCommand\wavefunc{}{
	\psiup
}

% Inner product
\DeclarePairedDelimiterX{\definp}[2]{\langle}{\rangle}{#1\,\delimsize\vert\,#2}
\def\inp{\@ifstar{\definp}{\definp*}}

% Statistical analysis
\DeclareDocumentCommand\ensemble{sm}{
	\IfBooleanTF{#1}{\langle#2\rangle}{\left\langle#2\right\rangle}
}
\DeclareDocumentCommand\ensemblew{sm}{
	\IfBooleanTF{#1}{\langle#2\rangle_\omega}{\left\langle#2\right\rangle_\omega}
}

\DeclareDocumentCommand\mathoperator{m m m}{
	\def\op{\mathchoice%
		{\displaystyle 		\scalebox{1.15}{$#1$}}%
		{\textstyle 		\scalebox{1.0} {$#1$}}%
		{\scriptstyle 		\scalebox{0.95}{$#1$}}%
		{\scriptscriptstyle	\scalebox{0.9} {$#1$}}%
	}
	\def\sub{\mathchoice%
		{\displaystyle 		#3}%
		{\textstyle 		#3}%
		{\scriptstyle 		\scalebox{.9}{$#3$}}%
		{\scriptscriptstyle	\scalebox{.8}{$#3$}}%
	}
	\tensor*{\op}{%
		_{\IfNoValueTF{#3}{}{\mkern-1mu\sub}}
		^{\IfNoValueTF{#2}{}{#2}}
	}
}
\DeclareDocumentCommand\fourieroperator{m m}{\mathoperator{\mathscr{F}}{#1}{#2}}
\DeclareDocumentCommand\frft{s o o g}{%
	\def\op{\fourieroperator{#2}{#3}}
	\def\body{#4 \vphantom{\fourieroperator{}{}}}
	\IfNoValueTF{#4}{\fourieroperator{#2}{#3}}{
		\IfBooleanTF{#1}
			{\op\lbrace\body\rbrace}
			{\fbraces{\lbrace}{\rbrace}{\op}{\body}}
	}%
}
\DeclareDocumentCommand\wvdoperator{m m}{\mathoperator{\mathscr{W}}{#1}{#2}}
\DeclareMathOperator\wvd{\mathscr{W}\!}
\DeclareMathOperator\husimiQ{\mathscr{Q}\!}
\DeclareDocumentCommand\conv{o}{
    \tensor*{\ast}{_{\IfNoValueTF{#1}{}{\mkern-3.5mu#1}}}
}

\renewcommand{\grad}{\nabla}
\DeclareDocumentCommand\laplacianp{m o d()}{ % Primed Laplacian
	\IfNoValueTF{#1}{%
		\IfNoValueTF{#2}{%
            \IfNoValueTF{#3}{\nabla^{\prime 2}}{\fbraces{\lparen}{\rparen}{\nabla^{\prime 2}}{#3}}%
        }{\fbraces{\lbrack}{\rbrack}{\nabla^{\prime 2}}{#2} \IfNoValueTF{#3}{}{(#3)}}%
    }{\nabla^{\prime 2} #1 \IfNoValueTF{#2}{}{[#2]} \IfNoValueTF{#3}{}{(#3)}}%
}
\DeclareDocumentCommand\gradientp{m o d()}{ % Primed Gradient
	\IfNoValueTF{#1}{%
		\IfNoValueTF{#2}{%
            \IfNoValueTF{#3}{\vnabla^\prime}{\fbraces{\lparen}{\rparen}{\vnabla^\prime}{#3}}%
        }{\fbraces{\lbrack}{\rbrack}{\vnabla^\prime}{#2} \IfNoValueTF{#3}{}{(#3)}}%
    }{\vnabla^\prime #1 \IfNoValueTF{#2}{}{[#2]} \IfNoValueTF{#3}{}{(#3)}}%
}
\DeclareDocumentCommand\gradp{}{\gradientp} % Shorthand for \gradientp

\newcommand\argsep{\ {\scalebox{1.2}{$\boldsymbol ;$}}\ } %Semi-colon separator

% Hyper-geometric functions
\newmuskip\pFqmuskip
\newcommand*\pFq[6][8]{%
    \begingroup %
    \pFqmuskip=#1mu\relax
    \mathcode`\,=\string"8000
    \begingroup\lccode`\~=`\,
    \lowercase{\endgroup\let~}\pFqcomma
    \tensor*[_#2]{F}{_{#3}}
    \ifthenelse{\isempty{#4}}{}{\qty[\genfrac..{0pt}{}{#4}{#5} \argsep #6]}%
    \endgroup
}
\newcommand*\nFq[5][7]{%
    \begingroup %
    \pFqmuskip=#1mu\relax
    \mathcode`\,=\string"8000
    \begingroup\lccode`\~=`\,
    \lowercase{\endgroup\let~}\pFqcomma
    \tensor*[_#2]{F}{_{#3}}
    \ifthenelse{\isempty{#4}}{}{\qty[#4 \argsep #5]}%
    \endgroup
}
\newcommand\pFqcomma{\mskip\pFqmuskip}

\newcommand*\genfactorial[1]{{!}_{\qty(#1)}}

% double and triple primes
\newcommand*\dprime{{\prime\prime\mkern-1.2mu}}
\newcommand*\trprime{{\prime\prime\prime\mkern-1.2mu}}

% pretty fractions
\newcommand\prettyfrac[2]{\fontfamily{ppl}\selectfont\sfrac{#1}{#2}}
\newcommand\half[0]{\prettyfrac{1}{2}}
\newcommand\third[0]{\prettyfrac{1}{3}}
\newcommand\quarter[0]{\prettyfrac{1}{4}}



% special enumerate envs
\newenvironment{enuminline}
    {\begin{enumerate*}[label=(\roman*)]}
    {\end{enumerate*}}
\newenvironment{enuminlinealph}
    {\begin{enumerate*}[label=(\alph*)]}
    {\end{enumerate*}}
\newenvironment{enumtext}[1]
	{\begin{enumerate}[label=\uline{\textit{#1 \arabic*:}},itemindent=*,wide,leftmargin=4pt,rightmargin=4pt]}
	{\end{enumerate}}
	


% Special symbols
\newcommand\flux{\Phiup}
\newcommand\radiance{L}
\DeclareMathOperator\csd{\mathcalcmsy{C}}
\newcommand\SP{\va{S}}
\newcommand\SUP{\SP_\text{0}}
\newcommand\SLHP{\SP_{\scalebox{.6}{LHP}}}
\newcommand\SLVP{\SP_{\scalebox{.6}{LVP}}}
\DeclareDocumentCommand\SX{g}{
	\SP_{\scalebox{.8}{c}}\IfNoValueF{#1}{\qty(#1)}
}
