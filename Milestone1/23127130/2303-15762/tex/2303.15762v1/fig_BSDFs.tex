
\begin{figure}[t]%
    \centering
    \resizebox{1\linewidth}{!}{\includegraphics{tikz_figures/BSDFs.pdf}}
    \Description{}%
    \caption{
        \textbf{Light-matter interactions.}
        (a) A diffractive BSDF $f$ formalises the wave-optical interaction of a generalized ray with matter: it diffracts a generalized ray incident with wavevector $\va{k}_\mathrm{i}$ into a generalized ray scattered with $\va{k}_\mathrm{o}$.
        Because generalized rays are perfectly coherent, $f$ does not depend on light's optical coherence and quantifies perfectly-coherent interactions.
        (b) On the other hand, an interaction of a collection of generalized rays---a ray bundle---with matter is understood to be partially-coherent, because optical coherence is directly related to a ray bundle's diffusivity (as illustrated in \cref{fig_generalized_rays_coherence}).
        As generalized rays are mutually incoherent, this interaction is simply the sum of the BSDF acting upon each generalized ray in the bundle, which, in the continuous case, is the BSDF $f$ convolved with the probability density of the bundle's diffusivity.
    }%
    \label{fig_BSDFs}
\end{figure}
