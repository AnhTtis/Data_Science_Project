
\begin{teaserfigure}
    \centering
	\begin{subfigure}{\teasersize\linewidth}%
        \makebox[\textwidth][c]{\resizebox{1\linewidth}{!}{\includegraphics{tikz_figures/teaser}}}
	\end{subfigure}%
    \vspace*{\teasercaptionoffset}
    \caption{
        \textbf{From ray optics to wave optics.}
        In this paper we present the \emph{generalized ray}: an extension of the classical ray to wave optics.
        The generalized ray retains the defining characteristics of the ray-optical ray: \emph{locality} and \emph{linearity}.
        These properties allow the generalized ray to serve as a ``point query'' of light's behaviour---the same purpose that the classical ray fulfils in rendering.
        By using such generalized rays, we enable the rendering of complex scenes, like the one shown, under rigorous wave-optical light transport.
        Materials admitting diffractive optical phenomena are visible: (a) a Bornite ore with a layer of copper oxide causing interference; (b) a Brazilian Rainbow Boa, whose scales are biological diffraction grated surfaces; and (c) a Chrysomelidae beetle, whose colour arises due to naturally-occurring multilayered interference reflectors in its elytron.
        Our formalism serves as a link between path tracing techniques and wave optics, and admits a highly general validity domain. 
        Therefore, we are able to apply sophisticated sampling techniques, and achieve performance that surpasses the state-of-the-art by orders-of-magnitude.
        We indicate resolution and samples-per-pixel (spp) count in all figures rendered using our method.
        While these figures showcase converged (high spp) results, our implementation also allows interactive rendering of all these scenes at 1 spp.
        Frame times (at 1 spp) for interactive rendering are indicated.
        Implementation, as well as additional renderings and videos are available in our supplemental material.
    }
	\Description{}
	\label{fig_teaser}
\end{teaserfigure}
