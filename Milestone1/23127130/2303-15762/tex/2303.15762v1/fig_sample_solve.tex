
\begin{figure}[t]%
    \centering
    \resizebox{1\linewidth}{!}{\includegraphics{tikz_figures/sample_solve.pdf}}
    \vspace*{-4mm}
    \Description{}%
    \caption{
        \textbf{Sample-solve.}
        Our path tracing algorithm (a) uses generalized rays (dotted lines) to \emph{sample} paths through the scene.
        Because generalized rays are mutually incoherent, they superpose linearly, therefore classical sampling techniques apply essentially unchanged.
        Once a path is sampled (solid red path), we (b) \emph{solve} for the partially-coherent transport from the light source to the sensor.
        As optical coherence is (inversely) related to the angular spread of rays in a ray bundle (\cref{fig_generalized_rays_coherence}), the \emph{solve} step should be understood as tracing a ray bundle (illustrated in blue) with a statistical distribution of generalized rays over the small solid angle subtended by the sensor from the source (across all intermediate interactions).
    }%
    \label{fig_sample_solve}
\end{figure}
