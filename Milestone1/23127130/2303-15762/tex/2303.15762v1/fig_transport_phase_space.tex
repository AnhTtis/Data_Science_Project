
\begin{figure}[t]%
    \centering
    \tikzfading
    [
        name=fade out,
        inner color=transparent!0,
        outer color=transparent!100
    ]
    \resizebox{1\linewidth}{!}{\includegraphics{tikz_figures/phase_space_transport.pdf}}
    \vspace*{-5mm}
    \Description{}%
    \caption{
        \textbf{Light transport and dynamics in phase space.}
        A sensor's sensitivity function (gray) and a light source's emission function (orange) are plotted in the wave-optical phase space.
        (a) Generalized rays serve a similar purpose to ray-optical rays: they enable performing point queries of the wave-optical system.
        During the sample stage, a generalized ray (blue), occupying a single phase-space cell, is used to sample the sensor's sensitivity function.
        Then, we backward propagate this generalized ray (reverse its temporal dynamics) throughout the scene, until we hit a light source.
        (b) Solve stage: 
        Once a path connecting a sensor to a light source has been sampled, we source partially-coherent light.
        This can be understood as widening the generalized ray's (that was used to sample the path) phase-space extent, thereby creating a \emph{ray bundle} that occupies multiple phase-space cells.
        We then propagate that partially-coherent ray bundle back over the sampled path to the sensor.
        Observe that free-space propagation constitutes a horizontal shear of the phase space, formalised by \cref{WDF_freespace}, just as in the ray-optical case.
    }%
    \label{fig_phase_space_transport}
\end{figure}
