
\begin{figure*}[t]%
    \centering
    \def\size{.23}
    \begin{subfigure}[b]{\size\textwidth}
        \centering
        \resizebox{1\linewidth}{!}{\includegraphics{tikz_figures/ray_wave_plt_ray.pdf}}
        \Description{}%
        \caption{
            ray optics
        }%
        \label{fig_ray_wave_plt_a_ray}
    \end{subfigure}
    \begin{subfigure}[b]{\size\textwidth}
        \centering
        \resizebox{1\linewidth}{!}{\includegraphics{tikz_figures/ray_wave_plt_wave.pdf}}
        \Description{}%
        \caption{
            wave optics
        }%
        \label{fig_ray_wave_plt_b_wave}
    \end{subfigure}
    \begin{subfigure}[b]{\size\textwidth}
        \centering
        \resizebox{1\linewidth}{!}{\includegraphics{tikz_figures/ray_wave_plt_plt.pdf}}
        \Description{}%
        \caption{
            PLT
        }%
        \label{fig_ray_wave_plt_c_plt}
    \end{subfigure}
    \begin{subfigure}[b]{\size\textwidth}
        \centering
        \resizebox{1\linewidth}{!}{\includegraphics{tikz_figures/ray_wave_plt_gr.pdf}}
        \Description{}%
        \caption{
            generalized rays
        }%
        \label{fig_ray_wave_plt_d_gr}
    \end{subfigure}
    \vspace*{-2mm}
    \Description{}%
    \caption{
        \textbf{Illustration: different pictorial views of light's physics.}
        (a) Most often, light transport is formulated in terms of ray optics, where light is composed of particles---light rays---that can be perfectly \emph{localized} in space: i.e. we may precisely specify the position and direction of propagation of each ray.
        Furthermore, these rays superpose \emph{linearly}, i.e., when two rays arrive at a surface or a sensor, we add up their respective irradiances linearly (no interference takes place).
        (b) Under wave optics---a more rigorous understanding of light---the basic descriptor of light is the \emph{wave function} $\wavefunc$.
        Unlike a light ray under ray optics, perfect localization of light is prohibited by the uncertainty relation, and the wave function now acts as a global descriptor of light.
        Moreover, linearity, in terms of light's observable properties, no longer holds.
        The loss of locality and linearity make wave optics incompatible with path tracing algorithms.
        (c) Physical light transport (PLT) takes advantage of light's limited optical coherence (the spatial extent over which wave-interference effects may arise) to understand light as mutually-incoherent beams.
        Thereby, PLT regains linearity (on superposition of beams), and to some degree locality, up to the extent of light's optical coherence.
        However, as PLT describes partially-coherent light, the \emph{sampling problem} manifests, see \cref{fig_sampling_problem}.
        (d) We solve this problem by drawing upon quantum mechanical tools to decompose light into coherent modes: \emph{generalized rays} that are pair-wise mutually incoherent, thus linear; and local to a greater degree compared to other models, as their extent is only constrained by light's wavelength.
        We use these generalized rays to sample paths effectively.
    }%
    \label{fig_ray_wave_plt}
\end{figure*}
