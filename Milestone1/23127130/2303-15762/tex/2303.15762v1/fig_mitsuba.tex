
\begin{figure*}[t]%
    \centering
    \resizebox{1\linewidth}{!}{\includegraphics{tikz_figures/mitsuba.pdf}}
    % \vspace*{-1mm}
    \Description{}%
    \caption{
        \textbf{Comparison to the state-of-the-art.}
        We render the CD scene using the PLT bidirectional path tracer (BDPT) \cite{Steinberg_practical_plt_2022}.
        Because the illumination of the diffractive CD surface is indirect, convergence is poor, as to be expected given the analysis in \cref{fig_PC}.
        The image was rendered with \num{56000} samples over about \num{315} hours, on an Intel\textsuperscript{\textregistered} Core\texttrademark\ i9-10980XE 18-core CPU.
        (a-c) Their renderer struggles with capturing the high-frequency details of the diffraction grating.
        Furthermore, because they propagate a fixed number (64) of spectral samples, sampled uniformly, clear banding artefacts are reproduced, suggesting that these high-frequency details will fail to converge due to spectral aliasing.
        (right) To analyze and compare the convergence performance to our approach, we plot error per sample and error per rendering-time graphs for the three different regions outlined in red:
        (d) The light transport that arrives to this area (red dotted line) is dominated by light that is not diffracted by the CD.
        Therefore, the per-sample convergence performance of their BDPT renderer (which performs much more work per sample) is significantly superior, as expected.
        (e-f) On the other hand, these regions do sample the CD and the convergence performance of their renderer is considerably inferior.
        A performance improvement is expected, as our renderer is GPU accelerated.
        The performance at region (d), plotted with open circle markers, should then be considered as baseline: up to <100 times faster, depending on sample count.
        The plots pertaining to the diffractive regions (filled markers) show a vastly greater improvement of about \num{1000} to \num{10000} times faster convergence.
    }%
    \label{fig_mitsuba}
\end{figure*}
