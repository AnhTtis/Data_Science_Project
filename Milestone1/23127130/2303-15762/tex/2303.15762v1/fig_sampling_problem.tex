
\begin{figure}[t]%
    \centering
    \resizebox{1\linewidth}{!}{\includegraphics{tikz_figures/ray_sampling_problem.pdf}}
    \Description{}%
    \caption{
        \textbf{The sampling problem.}
        To importance sample partially-coherent BSDFs (or light-matter interactions in general), the coherence properties of light are needed.
        However, these properties depend on the light source, and change on propagation and interaction with matter.
        Therefore, these properties are not available, and are difficult to predict or estimate, when tracing paths \emph{backwards}, i.e. starting from a camera.
        For a variety of reasons, path tracers trace paths backward-only, or bidirectionally (forward and backward), hence limiting our ability to perform backward sampling is a real problem.
        This sampling problem applies to PLT, as well as any framework where the descriptor of light quantifies light's optical coherence.
    }%
    \label{fig_sampling_problem}
\end{figure}
