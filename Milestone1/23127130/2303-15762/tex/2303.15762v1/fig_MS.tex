
\begin{figure}[t]%
    \centering
    \resizebox{1\linewidth}{!}{\includegraphics{tikz_figures/MS.pdf}}
    \Description{}%
    \caption{
        \textbf{Manifold sampling.}
        Example application of an advanced sampling technique.
        When performing next event estimation (NEE), manifold sampling (MS) \cite{Hanika2015-gv,Zeltner2020-xl} enables finding a light path between a surface and a light source across one or more dielectric interfaces.
        In the rendered scene, such a sampled path, that refracts through the dispersive prism (outlined in blue), is visualized via the dotted blue line.
        We also employ MS for NEE on specular reflections: note the reflections off the paint brush's metal handle and off the paint tubes, as well as the thin diffraction grating (outlined in yellow) dispersing light into multiple diffraction lobes. 
        See full high-resolution rendering in our supplemental material.
        (inset) Colour-coded difference compared to a without-MS rendering.
        (bottom) The diffraction grating lobes at increasing samples-per-pixel (spp).
    }%
    \label{fig_MS}
\end{figure}
