
\begin{figure*}[t]%
    \centering
    \resizebox{1\linewidth}{!}{\includegraphics{tikz_figures/FC.pdf}}
    \Description{}%
    \caption{
        \textbf{PLT as variance reduction.}
        During the solve step, we apply PLT to solve for the partially-coherent light transport over a sampled light path.
        (a) Light arrives from the right, illuminating a simple scene. A screen (on the right, red outline) shadows a rectangular area. 
        On the left a pair of thin diffraction gratings (yellow outline) reflect and disperse light.
        (b) When the light is highly collimated (subtends a very small solid angle from the scene), hence moderately coherent, the diffraction lobes are clearly visible.
        (c) As we increase the light's diffusivity (increase its solid angle), the reflection from the direct contribution lobe spreads out, while diffraction lobes mostly disappear, as expected.
        As the only light that arrives to the shadowed region is from the diffraction gratings, this is a good scene to study the benefits of the solve stage, and to do so we render the scene with only fully-coherent transport (no PLT applied).
        (b-c, insets) Difference images between the partially-coherent and fully-coherent sample-solve show that indeed both converge to an identical result.
        (d) However, close ups on the region outlined in orange in (c) show that, while the diffraction lobes are no longer visible, they still induce error, which the partially-coherent solve stage serves to reduce.
        (e) Plot of error in that area as function of sample count suggests that fully-coherent transport requires about 2-5 times the sample count to achieve similar-quality renderings.
    }%
    \label{fig_FC}
\end{figure*}
