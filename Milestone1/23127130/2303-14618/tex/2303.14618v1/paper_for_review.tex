\documentclass[10pt,twocolumn,letterpaper]{article}

\usepackage{cvpr} 
%\usepackage{iccv}


\usepackage{times}
\usepackage{epsfig}
\usepackage{graphicx}
\usepackage{amsmath}
\usepackage{amssymb}
\usepackage{algorithm}
\usepackage{algorithmic}
\usepackage{multirow}
\usepackage{array}
\usepackage{booktabs}
\usepackage{makecell}
\usepackage{float}
\usepackage{diagbox}
\usepackage{changepage}
\usepackage{cuted} % strip
\usepackage{xcolor}
\usepackage{caption}
\usepackage{subcaption}
\usepackage{dsfont}

% Include other packages here, before hyperref.

% If you comment hyperref and then uncomment it, you should delete
% egpaper.aux before re-running latex.  (Or just hit 'q' on the first latex
% run, let it finish, and you should be clear).
\usepackage[pagebackref=true,breaklinks=true,letterpaper=true,colorlinks,bookmarks=false]{hyperref}

\newcommand*{\Scale}[2][4]{\scalebox{#1}{$#2$}}%
\newcommand*{\Resize}[2]{\resizebox{#1}{!}{$#2$}}%

% \iccvfinalcopy % *** Uncomment this line for the final submission

\def\iccvPaperID{2764} % *** Enter the ICCV Paper ID here
\def\httilde{\mbox{\tt\raisebox{-.5ex}{\symbol{126}}}}

% Pages are numbered in submission mode, and unnumbered in camera-ready
% \ificcvfinal\pagestyle{empty}\fi

% Support for easy cross-referencing
\usepackage[capitalize]{cleveref}
\crefname{section}{Sec.}{Secs.}
\Crefname{section}{Section}{Sections}
\Crefname{table}{Table}{Tables}
\crefname{table}{Table}{Tables}

\begin{document}

%%%%%%%%% TITLE
\title{BoxVIS: Video Instance Segmentation with Box Annotations}

 \author{{Minghan Li \quad and \quad Lei Zhang\thanks{Corresponding author.} } \\
    	Department of Computing, Hong Kong Polytechnic University  \\
    	{\tt\small liminghan0330@gmail.com, cslzhang@comp.polyu.edu.hk}
    }

\maketitle
% Remove page # from the first page of camera-ready.
%\ificcvfinal\thispagestyle{empty}\fi


%%%%%%%%% ABSTRACT
\begin{abstract}
The current study investigated possible human-robot kinaesthetic interaction using a variational recurrent neural network model, called PV-RNN, which is based on the free energy principle.
Our prior robotic studies using PV-RNN showed that the nature of interactions between top-down expectation and bottom-up inference is strongly affected by a parameter, called the meta-prior, which regulates the complexity term in free energy.
% The current study examines how the behaviours of robots alter by changing the meta-prior $w$ in human-robot kinaesthetic interaction.
The current study examines how changing the meta-prior $w$ in the interaction phase affects the counter force generated when an experimenter attempts to induce movement pattern transitions familiar to the robot through its prior training.
The study also compares the counter force generated when trained transitions are induced by a human experimenter and when untrained transitions are induced.
Our experimental results indicated that (1) the human experimenter needs more/less force to induce trained transitions when $w$ is set with larger/smaller values, (2) the human experimenter needs more force to act on the robot when he attempts to induce untrained as opposed to trained movement pattern transitions.
Our analysis of time development of essential variables and values in PV-RNN during bodily interaction clarified the mechanism by which gaps in actional intentions between the human experimenter and the robot can be manifested as reaction forces between them.


%% Hiroki writing 2022-11-4
%Current study investigates the dynamics of the latent states during human-robot kinaesthetic interaction using PV-RNN.
%We have achieved to observe and analyse the internal state of an RNN model based on the free energy principle, during real-time human-robot interaction.
%Essential characteristics observed in the previous study of this variational recurrent neural network model, PV-RNN, is that by changing a meta prior $w$, the balance between the top-down intention and the bottom-up perceptual reality changes.
%In the current study, we examined how changing the weighting parameter $w$ between accuracy and complexity in free energy principle affects the humanoid robot's behaviour through human-robot interaction. We have conducted some human-robot kinaesthetic interaction experiments with various $w$ and quantitatively analysed the latent variable and the force applied to the humanoid robot. We have observed that the force required to change the robot's intention has increased, both when the top-down intention was strengthened by changing the $w$ and when corresponding switch of its primitive was against the experience of the RNN during its training. The study confirms through quantitative analysis that by increasing or decreasing the $w$ in PV-RNN, humanoid robot leads or follows the human counterpart during the human-robot kinaesthetic interaction.

\begin{comment}
Comment from Jun #2
・最後にQualitativeな結果(インパクト)が欲しい
・Current study investigates the problem on~と書き出すのが一般的
・最初の一文と最後の一文を対応させる
・最後の一文はもう少しAbstractかつ包括的に
\end{comment}

\begin{comment}
Comment from Jun #1
We investigated how the kinaesthetic human-robot interaction can affect the internal state of a model based on the free energy principle. 
=> how the internal state is affected is not the most important point in this study. This part should be rewritten.

The key function of this variational recurrent neural network model, PV-RNN, is that by changing a meta prior $w$, it takes a balance between the "complexity” term and the ”accuracy” term which corresponds to a top-down intention and a bottom-up perceptual reality in the free energy principle, respectively. 
=> This is not key function of PV-RNN. It is an essential characteristics observed in the previous study. The grammar after $w$ is something strange. Rewrite these.

This research has conducted a human-robot interaction experiment with a robotic agent in a kinaesthetic sense.
=> The sentence is not good. "in a kinaesthetic sense" is grammatically wrong.
MODIFIED => "In the current study human-robot interaction experiments using the kinaesthetic sense were conducted."

We investigated that when human forces the agent to switch primitives from one to another, larger force was required both when the human intention is conflictive against the top-down the intention of the agent and when the agent has a stronger top-down intention by modifying the $w$.
=> You should write the essential results of the experiments rather than what we investigated and also how these results could contribute to the studies on human-robot interaction.
\end{comment}

\end{abstract}
% \begin{figure}[t]
%     % \begin{subfigure}{1\linewidth}
%     %   \centering
%     % %   \includegraphics[width=1\linewidth]{figs/fig_1_moti_textattn.pdf}  
%     % %   \includegraphics[width=1\linewidth]{figs/fig_1_moti_textattn_v2.pdf}  
%     %   \includegraphics[width=1\linewidth]{figs/fig_1_moti_textattn_v5.pdf}  
%     %   \vspace{-0.5cm}
%     %     \caption{Amount of attention added to each video clip from the source video and query text in the self-attention layers of Moment-DETR encoder.}
%     %     % \caption{Distribution of attention for source and query in Moment-DETR encoder}
%     %     % Visualization of video clip's self-attention score in Moment-DETR encoder.
%     %   \label{fig:fig1_text_attn_ex}
%     % \end{subfigure}%\hfill% or  or \hspace{0.3\textwidth}
%     \vspace{0.2cm}
%     % \begin{subfigure}{1\linewidth}
%       \centering
%     %   \includegraphics[width=1\linewidth]{figs/fig1_moti_negattn.pdf}  
%       \includegraphics[width=1\linewidth]{figs/fig1_moti_negattn_v3.pdf}  
%       \vspace{-0.4cm}
%     %   \caption{Correspondence of saliency scores on the relevance between video clips and the text query.}
%     % \caption{Predicted saliency scores against the video relevant positive query and video irrelevant negative query}
%       \label{fig:fig1_neg_attn_ex}
%     % \end{subfigure}%\hfill% or  or \hspace{0.3\textwidth}
%     \caption{
%     % 원준 원본
%     % (a) Comparison between attention scores of source and query for each video clip~(We sum the attention scores from video and text). 
%     % We observe that the attention scores are dominated by other clips in the source video. 
%     % Text queries do not account for much attention regardless of the relevance to the video clips.
%     % \textbf{(a)} Inspection of the query dependency in Moment-DETR encoder.
%     % % We visualize the attention score of video tokens in the transformer encoder and observe that text query accounts for only a low portion of attention.
%     % % This tendency occurs regardless of the relevance between the text query and video clips. 
%     % We visualize the attention score of video tokens in the transformer encoder and observe 1) text query only accounts for a low portion of attention, and 2) relevance between video-query pair does not affect the attention scores ratio of text.
%     \textbf{(b)} Comparison of highlight-ness when relevant and non-relevant queries are input.
%     As observed in , existing work only uses queries to play an insignificant role, thereby may not be capable of detecting false queries and considering the video-query relevance even when the problem in (a) is resolved. 
%     % \SE{} % 이 부분이 "not capable of" 란 용어가 세다는 피드백이 있는 듯 합니다. 이러한 능력이 없다는 것은 굉장히 강한 어조인거 같기는 하고, 이러한 경우들이 종종 있다거나 좀 약화시킬 필요가 있어보이긴 하네요.
%     On the other hand, our QD-DETR yields a query-dependent representation that the relevance between the source video and query text is updated in the saliency scores.
%     There is a large gap between positive and negative saliency scores, and scores are consistent since the clips are all highly correlated to others.
%     }
%     \label{fig:motivation_ex}
%     % \captionsetup{belowskip=13pt}
%     % \setlength{\belowcaptionskip}{-10pt}
% \end{figure}
\begin{figure}
    \centering
    \includegraphics[width=1\linewidth]{figs/fig1_moti_negattn_1111.pdf}
    % \includegraphics[width=1\linewidth]{figs/fig1_moti_negattn_1109.pdf}
    % \includegraphics[width=1\linewidth]{figs/fig1_moti_negattn_stat.pdf}
    \vspace{-0.6cm}
    \caption{
        % \SE{} % 수정 필요
        Comparison of highlight-ness~(saliency score) when relevant and non-relevant queries are given.
        We found that the existing work only uses queries to play an insignificant role, thereby may not be capable of detecting negative queries and video-query relevance; saliency scores for clips in ground-truth~(GT) moments are low and equivalent for positive and negative queries.
        % This also results in mispredicted moments when ground-truth~(GT) moment is dominated by clips unrelated to GT since their prediction is highly focused on the video.
        % \SE{} % 여기 한번 더 보면 좋을 듯 합니다. GT moment에 unrelated한 clip이 많으면? label이 틀렷을 경우를 말씀하시는건지?
        % As observed in saliency graph, existing work only uses queries to play an insignificant role, thereby may not be capable of detecting false queries and considering the video-query relevance.
        On the other hand, query-dependent representations of QD-DETR result in corresponding saliency scores to the video-query relevance and precisely localized moments.
        % On the other hand, our QD-DETR yields a query-dependent representation that the
        % saliency scores are in accordance with the relevance between the video and query.
        % text is in accordance with the saliency scores.
        % There is a large gap between positive and negative saliency scores, and scores are consistent since the clips are all highly correlated to others.
}
    \label{fig:motivation_ex}
\end{figure}


\section{Introduction}
% 원준 원본
% Along with the advance of digital devices and platforms, video is now one of the most desired data type for consumers. However, although the large information capacity of videos may be beneficial in many aspects, e.g., informative and entertaining, on the contrary perspective, videos are time-consuming, and hard to search for desirable moments. 
% This has led many creators to use extra manpower to crop and edit the video to generate highlight clips to gain the consumer’s attention.
Along with the advance of digital devices and platforms, video is now one of the most desired data types for consumers~\cite{apostolidis2021video,wu2017deep}.
% SE: Video aware deep learning application & survey papers?
Although the large information capacity of videos might be beneficial in many aspects, e.g., informative and entertaining, inspecting the videos is time-consuming, so that it is hard to capture the desired moments~\cite{anne2017localizing,apostolidis2021video}. 
% This has led many creators to use extra manpower to crop and edit the video to generate highlight clips to gain the consumer’s attention.


% On the other side, 
Indeed, the need to retrieve user-requested or highlight moments within videos is greatly raised.
Numerous research efforts were put into the search for the requested moments in the video~\cite{anne2017localizing, gao2017tall, liu2015multi, escorcia2019temporal} and summarizing the video highlights~\cite{zhang2016video, mahasseni2017unsupervised, badamdorj2022contrastive, wei2022learning}.
% Numerous research efforts were put into the search for the requested moments in the video~\cite{anne2017localizing, gao2017tall, liu2015multi, escorcia2019temporal}, summarizing the video to generate highlights was another popular topic~\cite{zhang2016video, mahasseni2017unsupervised, badamdorj2022contrastive, wei2022learning}.
Recently, Moment-DETR~\cite{momentdetr} further spotlighted the topic by proposing a QVHighlights dataset that enables the model to perform both tasks, retrieving the moments with their highlight-ness, simultaneously.

% 원준 원본
% To detect the desired moments, previous works employed transformer encoder-decoder architectural designs to fuse the text query into the video representations. Moment-DETR~\cite{mDETR} modified detection transformer to process capture the moment as a set, and UMT~\cite{umt} implemented transformer decoder as to output clip-wise saliency. 
% Yet to their outstanding breakthroughs in the literature of moment retrieval with the seminal architectures, their limitation is that the role of the given text query is insignificant in representing the query-conditioned video representation; the attention mechanism of moment DETR is not explicitly conditioned on the text query, and the text query is conditioned on multi-modal clips where the differences between the clips are smoothed after encoding process in UMT.



% \begin{figure}[t]
% \centering
%     \begin{subfigure}[l]{0.37\linewidth}
%       \centering
%       \vspace{0.20cm}
%     %   \includegraphics[width=1\linewidth]{figs/fig_1_moti_textattn.pdf}  
%     %   \includegraphics[width=1\linewidth]{figs/fig_1_moti_textattn_v2.pdf}  
%       \includegraphics[width=1\linewidth]{figs/fig1_moti_violin_a.pdf}  
%       \vspace{-0.60cm}
%     %   \caption{text attention}
%         \caption{Importance of queries in video representation}
%       \label{fig:fig1_text_attn}
%     \end{subfigure}%\hfill% or  or \hspace{0.3\textwidth}
%     \vspace{0.2cm}
%     \begin{subfigure}[r]{0.61\linewidth}
%       \centering
%     %   \includegraphics[width=1\linewidth]{figs/fig1_moti_negattn.pdf}  
%       \includegraphics[width=1\linewidth]{figs/fig1_moti_violin_b.pdf}  
%     %   \caption{neg attention}
%         % \caption{Relation between the highlight-ness and the relevance between videos and query texts.}
%         \caption{Highlight-ness~(saliency) histogram of positive and negative video-query pairs\SE{}}
%       \label{fig:fig1_neg_attn}
%     \end{subfigure}%\hfill% or  or \hspace{0.3\textwidth}
%     % \vspace{-0.2cm}
%     \caption{Overall statistics for attention scores in Fig.~\ref{fig:motivation_ex} in QVHighlights dataset. 
%     (a) For the attention scores that measure how much the text query is generally involved in video representation, we use violin plots to show the probability density. We plot the score for each layer in the encoder.
%     % (b) Using the histogram, we compare how the baseline and QD-DETR yield different salient scores given the positive and negative video-text pairs.
%     (b) Saliency histogram shows the distributional gap between positive and negative video-text query pairs of baseline~(Moment-DETR) and proposed QD-DETR.\SE{}
%     }
%     \label{fig:motivation}
%     % \captionsetup{belowskip=13pt}
%     % \setlength{\belowcaptionskip}{-10pt}
% \end{figure}

% \begin{figure}[t]
% \centering

%     \begin{subfigure}[r]{1\linewidth}
%       \centering
%       \hspace{-0.2cm}
%     %   \includegraphics[width=1\linewidth]{figs/fig1_moti_negattn.pdf}  
%       \includegraphics[width=1.1\linewidth]{figs/fig1_moti_violin_a_v2.pdf}  
%     %   \caption{neg attention}
%         % \caption{Relation between the highlight-ness and the relevance between videos and query texts.}
%         \vspace{-0.5cm}
%         % \caption{Saliency histogram of positive and negative video-query pairs}
%         \caption{We plot the histograms and its average value~(dotted line) to compare saliency scores when true and false text queries are given for each method. (left) Since the video representations do not include much textual information, both the true and false queries yield similar saliency scores. (Middle) Even when the video representation is enforced to be updated with the textual information, the issue is not much resolved. (Right) By extracting discriminative features in the text query, distributions are differentiated.
%         % \SE{} % R1@0.5 설명
%         Also, R1@0.5 indicates evaluation metric, Recall at 1 with IoU 0.5 threshold on QVhighlight \textit{val} set.
%         }
%       \label{fig:fig1_neg_attn}
%     \end{subfigure}%\hfill% or  or \hspace{0.3\textwidth}
%     \\
%     \begin{tabular}{cc}
%     \hspace{-0.2cm}
%         \begin{minipage}{.4\linewidth}
%             \begin{subfigure}[l]{1\linewidth}
%               \centering
%             %   \vspace{0.20cm}
%             %   \includegraphics[width=1\linewidth]{figs/fig_1_moti_textattn.pdf}  
%             %   \includegraphics[width=1\linewidth]{figs/fig_1_moti_textattn_v2.pdf}  
%               \includegraphics[width=1\linewidth]{figs/fig1_moti_violin_a.pdf}  
%               \vspace{-0.60cm}
%             %   \caption{text attention}
%                 \caption{Importance of queries in video representation}
%               \label{fig:fig1_text_attn}
%             \end{subfigure}%\hfill% or  or \hspace{0.3\textwidth}
%         \end{minipage}
        
%         \begin{minipage}{.6\linewidth}
%             \vspace{-0.2cm}
%             \caption{Overall statistics of Fig.~\ref{fig:motivation_ex} in QVHighlights dataset. 
%             (a) Saliency histogram shows the distributional gap between positive and negative video-text query pairs.
%             % (a) For the attention scores that measure how much the text query is generally involved in video representation, we use violin plots to show the probability density. We plot the score for each layer in the encoder.
%             % (b) Using the histogram, we compare how the baseline and QD-DETR yield different salient scores given the positive and negative video-text pairs.
%             % (b) Text ratio in self-attention layer to  of Moment-DETR
%             % (b) Ratio of text when representing video tokens in self-attention of Moment-DETR.
%             % (b) Magnitude of attention text query involved.
%             % (b) Attention score of video tokens
%             % (b) Magnitude of text query to refine the video tokens in self-attention layer of Moment-DETR.
%             (b) Probability density depicting the weight of the text query in attention score for video clips. Scores are from the self-attention layers in Moment-DETR encoder.
%             % (b) The text query ratio in attention score of video clips (Self-attention layer in Moment-DETR encoder). We use violin plots to show probability density.
%             % 텍스트 쿼리가, 비디오 피쳐에 얼만큼 attend 하는지
%             }
%         \end{minipage}
    
%     \end{tabular}
%     \vspace{-0.5cm}
%     \label{fig:moti}
%     % \captionsetup{belowskip=13pt}
%     % \setlength{\belowcaptionskip}{-10pt}
% \end{figure}


% \begin{figure}
%     \centering
%     % \includegraphics[width=1\linewidth]{figs/fig1_moti_negattn_1109.pdf}
%     \includegraphics[width=1\linewidth]{figs/fig1_moti_negattn_stat_v2.pdf}
%     \vspace{-0.8cm}
%     \caption{
%         Histogram of saliency when the positive and negative queries are given. We plot the histograms and its average value~(dotted line) to compare saliency scores when relevant~(positive) and irrelevant~(negative) text queries are given for each method. (Left) Since the video representations do not properly reflect textual information, both the positive and negative queries yield similar saliency scores. 
%         % (Middle) Even when the video representation is enforced to be updated with the textual information, the issue is not much resolved. 
%         (Right) By representing video clips in query-dependent manner, distributions are differentiated.
%     }
%     \vspace{-0.6cm}
%     \label{fig:motivation}
% \end{figure}


% One of the demanding task is moment retrieval task, which is detecting the desired moments from the given query, typically the text query.
When describing the moment, one of the most favored types of query is the natural language sentence~(text)\cite{anne2017localizing}. 
While early methods utilized convolution networks~\cite{zhang2020learning, gao2021fast, wang2020temporally}, recent approaches have shown that deploying the attention mechanism of transformer architecture is more effective to fuse the text query into the video representation.
% To handle these modalities, previous works simply employed the attention mechanism of transformer architecture to fuse the text query into the video representation.
For example, Moment-DETR~\cite{momentdetr} introduced the transformer architecture which processes both text and video tokens as input by modifying the detection transformer~(DETR), and UMT~\cite{umt} proposed transformer architectures to take multi-modal sources, e.g., video and audio. 
Also, they utilized the text queries in the transformer decoder.
Although they brought breakthroughs in the field of MR/HD with seminal architectures, they overlooked the role of the text query.
To validate our claim, we investigate the Moment-DETR~\cite{momentdetr} in terms of the impact of text query in MR/HD~(Fig.\ref{fig:motivation_ex}).
Given the video clips with a relevant positive query and an irrelevant negative query, we observe that the baseline often neglects the given text query when estimating the query-relevance scores, i.e., saliency scores, for each video clip.
% the output saliency score, i.e. query-relevance scores.
% Based on the observation, we traced the actual saliency prediction of the model against both the video-relevant query and the irrelevant dummy one where we find that the baseline often neglects the given text query when estimating the query-relevance scores of video clips.
% For example, in Fig.~\ref{fig:motivation_ex}, saliency scores are not affected even when the query is substituted with the dummy.
% % General statistics for Fig.~\ref{fig:motivation_ex} is shown in Fig.~\ref{fig:motivation}. 
% General statistics corresponding to Fig.~\ref{fig:motivation_ex} are also shown in Fig.~\ref{fig:motivation}.



% The limitation of the concrete baseline~\cite{momentdetr} is inspected in two different aspects; 1) Utilization of text-query in the encoding process and 2) the output saliency score, i.e. query-relevance scores.
% Firstly, we visualize the attention score when video clips are given as a query in self-attention. 
% We observe that the text queries have relatively small impacts compared to other video features, as shown in Fig.~\ref{fig:fig1_text_attn_ex}.
% That is, the text does not account for much in representing every video clip, although the goal of MR/HD is to detect query-relevant moments.
% Based on the observation, we traced the actual saliency prediction of the model against both the video-relevant query and the irrelevant dummy one where we find that the baseline often neglects the given text query when estimating the query-relevance scores of video clips.
% For example, in Fig.~\ref{fig:motivation_ex}, saliency scores are not affected even when the query is substituted with the dummy.
% % General statistics for Fig.~\ref{fig:motivation_ex} is shown in Fig.~\ref{fig:motivation}. 
% General statistics are also shown in Fig.~\ref{fig:motivation}.

% Consequently, in Fig.~\ref{fig:fig1_neg_attn_ex}~(b), we found that the baseline often neglects the given text query when estimating the query-relevance scores of video clips; 
% For example, 


% We validate the previous work sometimes neglects the given query when estimating the saliency of video clips.
% For example, there is an example that the saliency scores from positive and negative queries cannot be distinguishable, as shown in Fig.~\ref{fig:fig1_neg_attn_ex}.
% % 우리는 추가로 text attention을 추가도 해봤지만, 효과가 있긴 했으나, still 이슈가 있는 것을 확인하였다?
% % Still, we observe that assuring the high attendance of text queries does not resolve the overlap which motivates us to question the quality of the naive use of task-agnostic text representation~\cite{momentdetr, umt}.
% We found that introducing the text-attention for ensuring the high attendance of text queries relieve the overlap, but there still be a severe overlap.


% To validate their limitations, we inspect the impacts of text queries in the concrete baseline~\cite{momentdetr} with the two different aspects, 1) tendency of attention in self-attention layer and 2) saliency score, i.e. query-relevance scores. \SE{} % attention 이 갑자기 등장하는가?
% Firstly, we visualize the attention score when video clips are given as a query in self-attention. We observe the text queries have relatively low attention scores compared to the video features, as shown in Fig.~\ref{fig:fig1_text_attn_ex}.
% That is, the text does not account for much in representing every video clip, although the goal of MR/HD is to detect query-relevant moments.
% Based on this observation, we trace the actual saliency prediction of the model against both positive and negative text queries.
% We validate the previous work sometimes neglects the given query when estimating the saliency of video clips.
% For example, there is an example that the saliency scores from positive and negative queries cannot be distinguishable, as shown in Fig.~\ref{fig:fig1_neg_attn_ex}.
% % 우리는 추가로 text attention을 추가도 해봤지만, 효과가 있긴 했으나, still 이슈가 있는 것을 확인하였다?
% % Still, we observe that assuring the high attendance of text queries does not resolve the overlap which motivates us to question the quality of the naive use of task-agnostic text representation~\cite{momentdetr, umt}.
% We found that introducing the text-attention for ensuring the high attendance of text queries relieve the overlap, but there still be a severe overlap.



% Thus, we 
% query dependency를 높이기 위해 
% Cross-attention? text-attention? detailed explanation on text-attention should be needed?
% By handling these two issues, we find that more precise retrieval can be achieved.
% 
% 
%
% By projecting video-discriminative text features with high text attendance to source video, we f 
% We also find the need to improve the quality of query features since assuring high text attendance also results in...
% pairs are not finetuned to be discriminative that even the similarity within the pairs does not reflect the relevance between the query and the video clips.
% General statistics for Fig.~\ref{fig:motivation_ex} is shown in Fig.~\ref{fig:motivation}. 
% \SE{} % 이거 ??로 뜨는데, 위처럼 figure 그리면 label이 안되는걸까요
% \SE{}
% 형님 아래 사항 생각 좀 해보는게 좋을 거 같아요.
% fig 1. (a) 그림만 봤을 때 모든 clip에 대해 text attention이 일정이상 존재하긴 하니까, 뭔가 not assured to be conditioned가 와닿지 않는거 같아요.
% + 왜 text가 항상 attend 해야하나?
% not assured to be conditioned --> text shows relatively low affects compared to video 같이 실제 나타난 현상까지 같이 적으면 어떨까 싶어요.
% fig 1. (b) 덜 반영한다?

% \SU{}
% 일단 text가 attend 잘 되어야 한다는 것에 좀 궁금점이 생깁니다. 결국에는 text와 관련있는 frame들을 attend해서 higlight를 찾아야 하는게 아닐까요? 그리고, 현제 저희의 모델 구조상 text query가 Key와 Value로 거의 활용되고 있는데 그렇다면 결국에는 해당 모델은 text에 대한 attention이 전혀 없다고 봐도 무방하지 않을까요? 그런 면에서 text attention을 강조하는게 좀 걸리긴 합니다.

% Specifically, the text query is not assured to be explicitly conditioned on every clip of the video, and as the query texts are evenly treated, discriminative keywords may not be spotlighted.
% attention mechanism of Moment-DETR is not explicitly conditioned on the text query as shown in Fig~\ref{}(d), and in UMT, the text are only used for conditioning the queries while the video representation are refined itself by self-attention.

% \begin{figure}[t]
%     \begin{subfigure}{1\linewidth}
%       \centering
%     %   \includegraphics[width=1\linewidth]{figs/fig_1_moti_textattn.pdf}  
%     %   \includegraphics[width=1\linewidth]{figs/fig_1_moti_textattn_v2.pdf}  
%       \includegraphics[width=1\linewidth]{figs/fig_1_moti_textattn_v4.pdf}  
%       \vspace{-0.5cm}
%     %   \caption{text attention}
%         \caption{Distribution of attention scores in Moment-DETR encoder}
%       \label{fig:fig1_text_attn}
%     \end{subfigure}%\hfill% or  or \hspace{0.3\textwidth}
%     \vspace{0.2cm}
%     \begin{subfigure}{1\linewidth}
%       \centering
%     %   \includegraphics[width=1\linewidth]{figs/fig1_moti_negattn.pdf}  
%       \includegraphics[width=1\linewidth]{figs/fig1_moti_negattn_v2.pdf}  
%       \vspace{-0.5cm}
%     %   \caption{neg attention}
%         \caption{Saliency score against positive and negative text queries}
%       \label{fig:fig1_neg_attn}
%     \end{subfigure}%\hfill% or  or \hspace{0.3\textwidth}
%     \vspace{0.2cm}
%     \begin{subfigure}{1\linewidth}
%       \centering
%     %   \includegraphics[width=1\linewidth]{figs/fig1_moti_violin.pdf}  
%       \includegraphics[width=1\linewidth]{figs/fig1_moti_violin_v2.pdf}  
%       \vspace{-0.5cm}
%       \caption{violin}
%       \label{fig:fig1_violin}
%     \end{subfigure}%\hfill% or  or \hspace{0.3\textwidth}
%     \vspace{-0.2cm}
%     \caption{(a) 1. portion of text attention vs. video attention 2. relation with text query and content (e.g. fg, bg) of clip seems not to affect the attention score
%     (b) 1. high variability even though entire clips are highly correlated with the given text query 2. positive and negative query makes overlaps on saliency score distribution
%     (3) actual distribution on validation dataset.}
%     \label{fig:motivation}
%     % \captionsetup{belowskip=13pt}
%     % \setlength{\belowcaptionskip}{-10pt}
% \end{figure}

To this end, we propose Query-Dependent DETR~(QD-DETR) that produces query-dependent video representation.
% Our key focus is to ensure each clip in predicted moments is explicitly conditioned by the query, particularly on the video-descriptive portion of the text query.
% Our key focus is to ensure that query-relevant clips are predicted by enforcing each clip to be explicitly conditioned by the query.
%Our key focus is to ensure that the model prediction for each clip is highly relevant to the query.
Our key focus is to ensure that the model's prediction for each clip is highly dependent on the query.
% by enforcing each clip to be explicitly conditioned by the query. :)
% hmm...
% \SE {} % "query-relevant clips are predicted" 이 문장이 좀 애매한거 같습니다. relevant 클립을 놓지지 않고 찾는 것을 보장한다? 이런 느낌인지 아니면 높은 saliency 를 주는게 목적이다? model prediction이 query-relevance를 반영하는 것을 보장한다?
% Our key focus is to ensure that the model prediction reflects query-relevance of clips by enforcing each clip to be explicitly conditioned by the query.
First, to fully utilize the contextual information in the query, we revise the transformer encoder to be equipped with cross-attention layers at the very first layers.
% 상익's thought :  single video - query간의 관계만 고려 - 같은 word가 더 많이 쓰이는 것을 보고 
% 교수님's thought : neg pair 를 쓰면 쿼리를 보지 않고서는 video clip간만 고려하는 것이 사라짐. 왜냐면 0으로 내보내야 하기 때문. --> SE: relative difference 만 고려하다가, 
By inserting a video as the query and a text as the key and value of the cross-attention layers, our encoder enforces the engagement of the text query in extracting video representation.
% 원준 교수님 코멘트 반영해서 다시
Then, in order to not only inject a lot of textual information into the video feature but also make it fully exploited, we leverage the negative video-query pairs generated by mixing the original pairs.
Specifically, the model is learned to suppress the saliency scores of such  negative~(irrelevant) pairs.
Our expectation is the increased contribution of the text query in prediction since the videos will be sometimes required to yield high saliency scores and sometimes low ones depending on whether the text query is relevant or not.
% \SE{}
% learns to?
% By suppressing the saliency scores of the irrelevant video-query pairs, the model learns to spotlight only the video-specific discriminative words in the query.
% % \SE{} % ====================== 상익 수정 ========================
% However, this architectural design still lacks the capability of identifying the video-descriptive keywords in the query.
% % However, this architectural design still lacks in identifying proper query relevance.
% This is because the current training scheme only focuses on the interactions of video and clips within a single video while neglecting information shared throughout the entire video.
% % We argue the problem of the current training scheme that only focuses on distinguishing the clips in a single video while neglecting information shared throughout the entire video.
% Therefore, we leverage the negative video-query relationships to enhance the capability of identifying the contextual similarity of query and video clips.
% 
% 원준 원본 
% However, this architectural design heavily relies on the quality of the text query.
% Therefore, we leverage the negative video-query relationships to enable the model to emphasize key corresponding query features.
% By suppressing the saliency scores of the irrelevant video-query pairs, the model learns to spotlight only the video-specific discriminative words in the query.
% =========================================================
Lastly, to apply the dynamic criterion to mark highlights for each instance, we deploy a saliency token to represent the entire video and utilize it as an input-adaptive saliency criterion. 
With all components combined, our QD-DETR produces query-dependent video representation by integrating source and query modalities.
This further allows the use of positional queries~\cite{dabdetr} in the transformer decoder.
% Furthermore, we can exploit the advanced DETR decoder architectures using the positional information, e.g., DAB-DETR, since our encoded tokens consist of identical position representations from a single modality.
% \SE{} % ====================== 상익 수정 ========================
% Furthermore, we can exploit the advanced DETR decoder architectures using the positional information, e.g., DAB-DETR, since our video clip tokens consist of identical position representations from a single modality.
% 원준 원본
% It also enables the use of advanced DETR decoder architectures, e.g., DAB-DETR, for the first time, as these works exploit the position information within a single modality.
% =========================================================
Overall, our superior performances over the existing approaches validate the significance of the role of text query for MR/HD.
% Our extensive experiments on QVHighlights, TVSum, and Charades-STA datasets validate the significance of considering the role and the quality of text query.

% All components combined with dynamic anchor moments for the query of decoder, our FOQUE fosters the query-dependent video representation, thereby making the 
% All components combined, our modified transformer encoding process fosters the query-dependent video representation thereby achieving the state-of-the-art results on various benchmarks of moment-retrieval and highlight detection.
	
% -	Video Platform & Streamer & Consumer의 증가. 
% Video는 다른 데이터 타입보다 정보가 많아 유용하지만, 이는 다른 말로 해석하면 video를 보는 것은 time-consuming 하고, 원하는 것을 찾아보기에는 힘들 수 있음.
% 따라서, 많은 매체에서는 사람들의 더 많은 이목을 끌기 위해 highlight 비디오라는 것을 편집하여 공유도 함.
% 하지만, highlight video를 만들기 위해 사람의 노력이 필요한 현 시점에서, This spotlights the need to retrieve the user-requested / Highlight moments in the video.

% -	이전에도 이러한 문제를 해결하기 위해 (asdfasdf) for moment retrieval, (asdfasdf) for highlight detection 등이 제안 되었지만, 이들은 비디오의 특정 영역을 찾는다는 공통된 목적을 가지고 있으면서도, 데이터 셋의 한계로 인해 따로 연구되었음. 이를 문제 삼으며, 최근에는 두 task를 동시에 학습할 수 있는 dataset이 소개 되었는데, 컴퓨터비전에서 최근 각광을 받고 있는 Transformer 모델 도입과 함께 큰 발전을 거듭하고 있음.

% -	구체적으로, 이 두가지 task를 수행하기 위해서는 transformer를 두가지 방법으로 이용할 수 있는데, moment-DETR 처럼 moment 를 clip의 set 단위로 예측할 수 있고, UMT 처럼 clip-wise prediction을 할 수 있음. 하지만, 이들은 query를 condition이 아닌 video와 동등한 레벨로 취급하거나 [mDETR], 매 클립이 self-attention으로 mixing 된 후에 condition을 걸어주어 clip간의 차이를 확실하지 이용하지 못하였고, 또한, 확실하게 condition으로 주지 못하였고, video와 query 사이의 관계를 한정적으로만 이용하였다.

% -	따라서, we explore three different ways to fully exploit query information. First, we design one-way cross-attention layer to condition every clip with the query features. Then, we utilized the negative video-text pairs to better model the relationships between the video and the text embeddings. Lastly, we define the saliency token to be the video-query dependent saliency estimator.


















% ===================== neg pair 부분 ===========================
% Nevertheless, the current training scheme, only considering the given video-query pair, still disturbs the model from identifying proper query-relevance prediction.
% In detail, the model focus on learning the fine-grained discrepancy between video clips, while neglecting the information they share, which contains significant clues to understand the context of video.
% Therefore, we leverage the negative video-query relationships to enhance the capability of identifying the contextual similarity of query and video clips.
% Therefore, we leverage the negative video-query relationships by suppressing those pairs, so that enhance the capability of identifying the contextual similarity of query and video clips.
% We hypothsize the diversity in query-video pairs are insufficient to learn the general relationship between text query and video.
% Therefore, we leverage the negative video-query relationships by suppressing the saliency scores of the irrelevant video-query pairs.
% However, this architectural design still lacks in identifying proper query relevance.
% We argue that the current training scheme only focuses on learning the fine-grained discrepancy between clips in a single video, while neglecting the information they share, which contains significant clues to understand the context of the video.
% Therefore, we leverage the negative video-query relationships to enhance the capability of identifying the contextual similarity of query and video clips.
% However, this architectural design still lacks in identifying proper query relevance.
% We argue the problem of the current training scheme that only focuses on learning the fine-grained discrepancy between clips in a single video.
% That is, the current design neglects the information shared throughout the video, although it contains significant clues to understand the context of the video.
\section{Related Work}

\subsection{Pixel-supervised VIS}
There are two major VIS benchmarks, the YTVIS series \cite{yang2019video} and OVIS \cite{qi2021occluded}, which have very different video types in terms of object motions and scenes. The YTVIS series focus mainly on segmenting sparse objects in shorter videos, while the OVIS aims to segment crowded instances with occlusions in longer videos. Based on these facts, we categorize the pixel-supervised VIS models into YTVIS-oriented ones and OVIS-oriented ones.

\textbf{YTVIS-oriented VIS models.}
By introducing a tracker into the representative IIS methods \cite{he2017mask,bolya2019yolact,tian2020conditional}, the early proposed VIS methods \cite{yang2019video, cao2020sipmask,Li_2021_CVPR,Athar_Mahadevan20stemseg,liu2021sg,yang2021crossover,QueryInst,ke2021pcan} have achieved decent performance on YTVIS series. 
The frame-to-frame trackers \cite{yang2019video,cao2020sipmask,Li_2021_CVPR,QueryInst} integrate the clues such as category score, box/mask IoU and instance embedding similarity. 
The clip-to-clip trackers \cite{Li_2021_CVPR,Athar_Mahadevan20stemseg,bertasius2020classifying,lin2021video, qi2021occluded,wang2020vistr,seqformer,hwang2021video} propagate the predicted instance masks from a key frame to other frames\cite{bertasius2020classifying,dai2017deformable,lin2021video,Li_2021_CVPR,qi2021occluded,wang2021end}.
% which can be implemented by deformable convolution \cite{bertasius2020classifying,dai2017deformable}, non-local block \cite{lin2021video}, correlation \cite{Li_2021_CVPR,qi2021occluded}, graph neural network \cite{wang2021end}, \etc. 
% By exploiting the temporal redundancy among overlapped frames, clip-to-clip trackers improve much the performance over per-frame methods on YTVIS datasets.
Recently, query-based \cite{carion2020end} VIS methods \cite{wang2020vistr,hwang2021video,seqformer,wu2022trackletquery} have achieved impressive progress. VisTR \cite{wang2020vistr} views the VIS task as an end-to-end parallel sequence prediction problem. 
% but it consumes a large amounts of memory to store spatial-temporal features. 
To reduce the storing memory of spatial-temporal features, IFC\cite{hwang2021video} transfers inter-frame information via efficient memory tokens, and SeqFormer \cite{seqformer} locates an instance in each frame and aggregates them to predict video-level instances. 

\textbf{OVIS-oriented VIS models.}
The aforementioned YTVIS-oriented VIS models often fail to handle the challenging long videos with crowded and similar-looking objects in OVIS dataset, resulting in significant performance degradation. Inspired by contrastive learning \cite{chen2020simplectt, pang2021quasi,khosla2020supervisedctt,wang2021densectt}, IDOL \cite{IDOL} learns discriminative embeddings for multiple object tracking frame by frame.
Mask2Former \cite{cheng2021mask2former} achieves impressive performance on IIS tasks by calculating attention only in the region of objects.
M2F-VIS \cite{cheng2021mask2former-video} and MinVIS \cite{huang2022minvis} extend Mask2Former to VIS task, where they respectively take per-frame and per-clip inputs.
% Mask2Former-VIS \cite{cheng2021mask2former-video} with per-clip input takes the mask IoU of the overlapping frames as the tracker, while MinVIS \cite{huang2022minvis} with per-frame input employs instance embedding similarity by using Hungarian matching as the tracker. 
VITA \cite{heo2022vita} integrates object embeddings of all frames in the video to produce video-level instance masks.

\textbf{Remarks.} Though the pixel-supervised VIS methods have achieved much progress, their generalization capability is limited. For example, the VIS models trained on YTVIS21 often fail to handle the challenging videos in OVIS, due to the limited number of type of videos in each dataset. However, it is labour-extensive to label the pixel-wise masks in videos. Inspired by the success of box-supervised IIS methods, we explore VIS task with only box annotations in this paper.

\begin{figure*}[!t]
\begin{center}
    \includegraphics[width=0.98\linewidth]{figs/fig/boxvis.jpg}
\end{center}
\vspace{-5mm}
\caption{(a) Architecture of our proposed BoxVIS, where the Teacher and Student Nets follow the Mask2Former-VIS \cite{cheng2021mask2former-video} framework. The Teacher Net produces high-quality object masks, which are assigned as the pseudo instance masks of the ground-truth bounding boxes by Hungarian matching. 
The predicted masks from the Student Net will be matched with the ground-truth boxes and the generated pseudo masks by taking the assigned pseudo masks into account.
%The Teacher Net is progressively updated via exponential moving average (EMA) without back-propagating gradients. 
(b) Schematic diagram of the proposed spatial-temporal pairwise affinity (STPA) loss, which uses box-center guided shifting to generate temporally paired pixels and employs the color similarity $S_{lab}$ and patch correlation $S_{corr}$ to compute pairwise affinity.} 
%The markers, $\checkmark$ and $\times$, represent above or below than the threshold respectively.}
\label{fig:boxvis_baseline}
\end{figure*}
\subsection{Weakly-supervised Segmentation}

\textbf{Box-supervised IIS.}
A few box-supervised IIS methods have been proposed. BoxCaseg \cite{wang2021boxcaseg} leverages a saliency model to generate pseudo object masks. BoxInst  \cite{tian2020boxinst} achieves impressive performance by proposing a projection loss and a pairwise affinity loss for box-supervision. BoxLevelSet \cite{li2022boxlevelset} utilizes the traditional level set evolution to predict the object boundaries and instance masks. DiscoBox \cite{lan2021discobox} and BoxTeacher \cite{cheng2022boxteacher} employ an exponential moving average (EMA) teacher to produce high-quality pseudo masks and introduce pseudo pixel-wise mask supervision, bringing significant performance improvement. 

\textbf{Weakly-supervised Video Segmentation.}
% It is a challenging task to output pixel-wise segmentation masks with only image-level or box-level supervision. 
For the video object segmentation (VOS) task, BoxVOS \cite{hannan2022box} and QMRA \cite{lin2021query} utilize the motion map and feature aggregation from consecutive frames to segment objects in videos with only box supervision. 
% QMRA \cite{lin2021query} aggregates the rich information in memory frames and box annotations.
For the VIS task, FlowIRM \cite{liu2021weakly} presents a class-supervised VIS baseline with relatively low performance. In this work, we propose the first box-supervised VIS framework and demonstrate its effectiveness. 

\chapter{Methodology}\label{section:method}

This section explains the theoretical details of the models used in the experiments. These are the regular vanilla variational autoencoder (VAE), the Riemannian Hamiltonian variational autoencoder (RH-VAE), the spherical variational autoencoder (\svae) and the roto-equivariant Variational Auto-Encoder (KS-VAE).

\section{Variational Autoencoder} \label{subsec:vae}
At the basis of all models used in this work lies the autoencoder model. 
The general framework of an autoencoder consists of two neural networks: an encoder that encodes an input image $x$ into a lower-dimensional latent representation $z$, and a decoder that decodes the latent representation into a reconstruction $\hat{x}$, with the aim of minimizing the error between the original image and its reconstruction. The Variational Autoencoder (VAE) \citep{maxkingma2013auto} is a generative version of the original autoencoder, that instead of learning the latent representation $z$ directly, learns a distribution describing each data point, from which the latent representation is sampled (see Figure \ref{autoencoders}). The aim of the VAE is therefore to learn a parameterized probability distribution $p_{\theta }$ describing the input data $x$'s true distribution $P(x)$. To do so, we assume that the input data can be characterized by a lower-dimensional latent distribution $z$. The marginal likelihood can then be written as \begin{align} p_\theta(x) = \int p_\theta(x|z)q_{prior}(z)dz\end{align} where $q_{prior}(z)dz$ is a prior distribution over the latent variables, that in case of the vanilla VAE is chosen as a standard normal Gaussian distribution. 
Unfortunately, computing $p_\theta(x)$ involves the posterior $p_\theta(z|x)$, which is computationally expensive and often intractable.
We therefore introduce an approximation $q_\phi(z|x)$ of the true posterior, which is computed by a neural network: the encoder. We can then train a variational autoencoder, consisting of the encoder, which computes the approximate posterior and the decoder, which computes the conditional likelihood $p_\theta(x|z)$.

Within the variational autoencoder framework, the encoder and decoder are optimized in a joint setting. To find the posterior distribution $q_\phi(z|x)$ that best approximates the true posterior $p_\theta(z|x)$, we can use the \textit{Kullback-Leibler} divergence, which measures the difference between two probability distributions. Ideally, we would want to minimize this term, which is given by

\begin{align}
    \text{KL}(q_{\boldsymbol{\mathbf{\phi}}}(\boldsymbol{\mathbf{z}}\mid \boldsymbol{\mathbf{x}})\,\,||\,\,p_{\boldsymbol{\mathbf{\theta}}}(\boldsymbol{\mathbf{z}}\mid \boldsymbol{\mathbf{x}}))
    &= \mathbb{E}_{q_\phi} \big[ \log q_\phi(\mathbf{z}) \big] - \mathbb{E}_{q_\phi} \big[ \log p_\theta(\mathbf{z} | \mathbf{x}) \big]\\
    &= \mathbb{E}_{q_\phi} \big[ \log q_\phi(\mathbf{z}) \big] - \mathbb{E}_{q_\phi} \bigg[ \log \frac{p_\theta(\mathbf{x}, \mathbf{z}) }{p_\theta(\mathbf{x})} \bigg]\\
    &= \mathbb{E}_{q_\phi} \big[ \log q_\phi(\mathbf{z}) \big] - \mathbb{E}_{q_\phi} \big[ \log p_\theta(\mathbf{x}, \mathbf{z}) - \log p_\theta(\mathbf{x}) \big]\\
    &= \mathbb{E}_{q_\phi} \big[ \log q_\phi(\mathbf{z}) - \log p_\theta(\mathbf{x}, \mathbf{z}) \big] + \mathbb{E}_{q_\phi} \big[ \log p_\theta(\mathbf{x}) \big]\\
    &= \mathbb{E}_{q_\phi} \big[ \log q_\phi(\mathbf{z}) - \log p_\theta(\mathbf{x}, \mathbf{z}) \big] + \underbrace{\log p_\theta(\mathbf{x})}_{\text{intractable}}.
\end{align}

% \begin{figure}[h]
%     \centering
%     \subfloat[\centering Autoencoder]{{\includegraphics[width=0.43\textwidth]{images/method/ae.png} }}%
%     \qquad
%     \subfloat[\centering Variational Autoencoder]{{\includegraphics[width=0.5\textwidth]{images/method/vae} }}%
%     \caption{Schematic view of autoencoder and variational autoencoder architectures. The encoder either learns to map the input vector $\mathbf{x}$ to a latent vector $\mathbf{z}$ directly (AE), or learns the parameters of a distribution describing $\mathbf{x}$, from which $\mathbf{z}$ is then sampled (VAE). The decoder in both cases learns to most accurately reconstruct the original input ($\mathbf{\hat{x}}$) from $\mathbf{z}$.}
%     \label{autoencoders}
% \end{figure}
\begin{figure}[h]
    \centering
    \subfloat[\centering Autoencoder]{{\includegraphics[width=0.36\textwidth]{images/method/ae_colored.png} }}%
    \quad \quad \quad
    \subfloat[\centering Variational Autoencoder]{{\includegraphics[width=0.5\textwidth]{images/method/vae_colored1.png} }}%
    \caption{Schematic view of autoencoder and variational autoencoder architectures. The encoder either learns to map the input vector $\mathbf{x}$ to a latent vector $\mathbf{z}$ directly (AE), or learns the parameters of a distribution describing $\mathbf{x}$, from which $\mathbf{z}$ is then sampled (VAE). The decoder in both cases learns to most accurately reconstruct the original input ($\mathbf{\hat{x}}$) from $\mathbf{z}$.}
    \label{autoencoders}
\end{figure}

However, as can be seen when we rewrite the equation, we still have the intractable evidence term $\log p_\theta(\mathbf{x})$. We therefore introduce a lower bound of the log-likelihood using Jensen’s inequality. 
\begin{align}
    \log p_\theta (x) &=\log \int_{\mathbf{z}} p_\theta(\mathbf{x}, \mathbf{z}) \\
    &=\log \int_{\mathbf{z}} p_\theta(\mathbf{x}, \mathbf{z}) \frac{q_\phi(\mathbf{z})}{q_\phi(\mathbf{z})} \\
    &=\log \left(\mathbb{E}_{q_\phi}\left[\frac{p_\theta(\mathbf{x}, \mathbf{z})}{q_\phi(\mathbf{z})}\right]\right) \\
    & \geq \underbrace{\mathbb{E}_{q_\phi}[\log p_\theta(\mathbf{x}, \mathbf{z})]-\mathbb{E}_{q_\phi}[\log q_\phi(\mathbf{z})]}_\text{ELBO} 
\end{align}
This lower bound is called the Evidence Lower BOund (ELBO) \citep{maxkingma2013auto}. Because the evidence is a constant, maximizing the ELBO amounts to minimizing the KL divergence. The ELBO therefore forms the key to variational inference: instead of finding our optimal distribution q by minimizing the KL divergence, requiring us to calculate the intractable evidence term, we find it by maximizing ELBO, which is a tractable operation. We can therefore use the ELBO as our model's loss function. 
To arrive at our final loss function, we rearrange the ELBO term into the following expression

\begin{align}
    \text{ELBO} &= \mathbb{E}_{q_\phi}[\log p_\theta(\mathbf{x}, \mathbf{z})]-\mathbb{E}_{q_\phi}[\log q_\phi(\mathbf{z})] \\
    &= \mathbb{E}_{q_\phi}[\log p_\theta(\mathbf{x}, \mathbf{z}) - \log q_\phi(\mathbf{z})] \\
    &= \mathbb{E}_{q_\phi}[\log p_\theta(\mathbf{x}| \mathbf{z}) + \log p_\theta(\mathbf{z})
    - \log q_\phi(\mathbf{z})] \\
    &= -\mathbb{E}_{q_\phi}[\log p_\theta(\mathbf{x}| \mathbf{z}) - \mathbb{E}_{q_\phi}[\log p_\theta(\mathbf{z}) - \log q_\phi(\mathbf{z})] \\
    &= -\mathbb{E}_{q_\phi}[\log p_\theta(\mathbf{x}| \mathbf{z})] - \text{KL} \left(q_{\boldsymbol{\mathbf{\phi}}}\left(\boldsymbol{\mathbf{z}}\mid \boldsymbol{\mathbf{x}})\,\,||\,\,p_{\boldsymbol{\mathbf{\theta}}}(\boldsymbol{\mathbf{z}}\right)\right),
\end{align}

which consists of a regularization term $\mathbb{E}_{q_\phi}[\log p_\theta(\mathbf{x}| \mathbf{z})]$ and a KL divergence 
term $\text{KL} \left(q_{\boldsymbol{\mathbf{\phi}}}\left(\boldsymbol{\mathbf{z}}\mid \boldsymbol{\mathbf{x}})\,\,||\,\,p_{\boldsymbol{\mathbf{\theta}}}(\boldsymbol{\mathbf{z}}\right)\right)$. The expectation term is also called the \textit{reconstruction loss}. It pushes the model to most accurately reconstruct an image from its encoded latent representation, such that the difference between decoding a sampled latent vector from the learned distribution is as small as possible. Meanwhile, the KL divergence term is also called the \textit{regularization loss}, as it pushes the approximate posterior to more closely resemble the prior distribution. When choosing a normal Gaussian distribution for both the prior and approximate posterior, as is the standard for VAEs, the prior enforces the posterior probability mass to have spread like a Gaussian, therefore adding a form of regularization to the model. Moreover, for a Gaussian prior and posterior, the KL term reduces to a closed-form formula, making computations more efficient. 



Now, all that remains is to solve the problem of the random sampling operation from $\mathbf{z}$ not being differentiable. \citeauthor{maxkingma2013auto} propose to solve this by using the \textit{reparameterization trick}, which suggests that instead of sampling $\mathbf{z}$ directly, some noise $\epsilon$ is sampled from a unit Gaussian distribution. We can then add the learned mean parameter $\mu$ to this noise term and multiply it by the variance $\sigma$ to arrive at a mean and variance as would have been directly sampled from the latent distribution, while still allowing for backpropagation through the neural network. 



% https://mpatacchiola.github.io/blog/2021/01/25/intro-variational-inference.html
% --------------------------------------------
% --------RIEMANNIAN RHVAE--------------------
% --------------------------------------------

\section{Riemannian Variational Autoencoder}\label{subsec:rhvae}
As discussed in Section \ref{bg:dst}, the vanilla variational autoencoder suffers from a distortion in the latent space  as a consequence of the Euclidean manifold assumption and Gaussian prior, which makes geometric notions such as distance unreliable in this space. One way of remedying this problem is to use metrics defined in the non-distorted input space instead, and mapping them to the manifold space. Such a mapping is possible by endowing the manifold with a Riemannian metric. A model that not only does this, but also \textit{learns} a fitting metric from the input data, is the Riemannian Hamiltonian VAE. These qualities make it a promising technique for accurately representing relationships between points and modelling BE progression.

The following section describes the details of RHVAE. Before this can be discussed however, it is important to give the reader a short overview of the basics of Riemannian geometry.

    \subsection{Basics of Riemannian Geometry}
        As discussed earlier, a real, smooth manifold is a space that is locally similar to a linear space. Riemannian geometry allows for defining notions of angles, distances, and volume on such spaces by endowing the manifold with a \textit{Riemannian Metric}. The manifold can then be considered as a \textit{Riemannian manifold}.
        We define an $m$-dimensional Riemannian manifold embedded in an ambient Euclidean space $\mathcal{X} = \mathbf{R}^d$ and endowed with a \textit{Riemannian metric} $\mathbf{G} \triangleq (\mathbf{G}_{\mathbf{x}})_{\mathbf{x} \in \mathcal{M}}$ to be a smooth curved space $(\mathcal{M},G)$. 
        For every point on the manifold $\mathcal{M}$, there exists a tangent vector $\mathbf{v}\in \mathcal{X}$ that is tangent to $\mathcal{M}$ at $\mathbf{x}$ iff there exists a smooth curve $\gamma:[0,1] \mapsto \mathcal{M}$ such that $\gamma(0)=\mathbf{x}$ and $\dot{\gamma}(0)=\mathbf{v}$.
        The velocities of all such curves through $\mathbf{x}$ form the \emph{tangent space} $\mathcal{T}_{\mathbf{x}}\mathcal{M}=\{ \dot{\gamma} (0) \,|\, \gamma:\mathbf{R}\mapsto\mathcal{M} \text{ is smooth around $0$ and } \gamma(0)=\mathbf{x}\}$, which has the same dimensionality as the manifold. The tangent space can be viewed as the collection of all the different ways in which the points on the manifold can be passed. 
        
        \begin{figure}[H]
            \centering
            \includegraphics[width=0.4\textwidth]{images/method/Manifold_Example.png}
            \caption{Schematic example of a 2-D manifold $\mathcal{M}$ and its tangent space $\mathcal{M}_x \mathcal{T}$ at point $x$. The geodesic $\gamma(t)$ starts at $x$ and goes in the direction $\mathbf{v}$.}
            \label{fig:my_label}
        \end{figure}
        
        The Riemannian metric $G(\cdot)$ then equips each point $\mathbf{x}$ on the manifold with an inner product in the tangent space $\mathcal{T}_{\mathbf{x}}\mathcal{M}$, \textit{e}.\textit{g}. $\langle \mathbf{u}, \mathbf{v} \rangle_x = \mathbf{u} ^T \mathbf{G}_{\mathbf{x}} \mathbf{v}$. 
        This induces a norm $\|\mathbf{u}\|_\mathbf{x}\,,\forall \mathbf{u} \in \mathcal{T}_{\mathbf{x}}\mathcal{M}$ locally defining the geometry of the manifold. Given these local notions, we can not only compute local angles, lengths, and areas, but also derive global quantities by integrating over local properties. We can thus compute the length of any curve on the manifold $\gamma : [0,1] \rightarrow \mathcal{M}$, with $\gamma(0) = \mathbf{x}$ and $\gamma(1) = \mathbf{y}$ as the integral of its speed: $\ell(\gamma) = \int_{0}^1 \|\dot{\gamma}(t)\|_{\gamma(t)}dt$.
        The notion of length leads to a natural notion of distance by taking the infimum over all lengths of such curves, giving the \emph{gobal Riemannian distance} on $\mathcal{M}$, $d(\mathbf{x},\mathbf{y})=\inf_{\gamma}\ell(\gamma)$. The constant speed-length that minimizes the distance of a curve between two points is called a \emph{geodesic} on $\mathcal{M}$. VAEs can generate images along such a geodesic path, providing a more geometry-aware alternative to the vanilla VAE's linear interpolations.

           
    \subsection{Riemannian Hamiltonian Variational Autoencoder}
    \citeauthor{chadebec2020geometryaware} propose the Riemannian Hamiltonian VAE (RHVAE), which assumes the latent space to be structured as a Riemannian manifold $\mathcal{M}=\left(\mathbb{R}^{d}, \mathbf{G}\right)$ with $\mathbf{G}$ being the Riemannian metric, as described above. RHVAE attempts to exploit this assumed Riemannian structuring by introducing two main extensions of the vanilla VAE. First, to replace the regular Gaussian posterior distribution with a geometrically-informed posterior through the use of Riemannian Hamiltonian dynamics. Secondly, to find an appropriate Riemannian metric for this space, by learning it with a neural network. 
    
    \subsubsection{Learning the Riemannian Metric}
    As mentioned in section \ref{bg:rie}, while the choice of Riemannian metric is crucial to defining the manifold space, the computation of many proposed metrics involves the Jacobian, which is difficult and expensive to compute. In the RHVAE framework, the metric is therefore proposed to be learned directly from the data. This parameterized metric is defined as follows

    \begin{align}\label{eq:riemanmetric}
            \mathbf{G}^{-1}(z)=\sum_{i=1}^{N} L_{\psi_{i}} L_{\psi_{i}}^{\top} \exp \left(-\frac{\left\|z-c_{i}\right\|_{2}^{2}}{T^{2}}\right)+\lambda I_{d},    
    \end{align}

    
    
    where $N$ is the number of observed data points, $L_{\psi_{i}}$ are parameterized lower triangular matrices with positive diagonal coefficients learned from the data through neural networks, $c_{i}$ are centroids corresponding to the mean of the encoded distributions for every data point, 
    $T$ is a temperature parameter that scales the metric close to the centroids and $\lambda$ is a regularization factor, which allows for scaling the Riemannian volume element further away from the data. The above-defined metric is smooth, differentiable, and allows for computing geodesics easily, which is useful for creating informed interpolations along the geodesic curve on the manifold. 
    % The shape of this metric is very powerful since we have access to a closed-form expression of the inverse metric tensor which is usually useful to compute shortest paths (through the exponential map). Moreover, this metric is very smooth, differentiable everywhere and allows scaling the Riemannian volume element $\sqrt{\operatorname{det} \mathbf{G}(z)}$ far from the data very easily through the regularization factor $\lambda$.

    Training of the metric learning model is done jointly with training the rest of the RHVAE network. Just as with a regular VAE, an input image is encoded by the encoder network, which learns the parameters of a normal Gaussian distribution $\mathcal{N}(\mu, \sigma^2)$. Simultaneously, the metric network learns to map the input image to the lower triangular matrix $L_{\psi_{i}}$, allowing us to compute the Riemannian metric. These serve as input for a sampler, called the RHMC (Riemmanian Hamiltonian Monte-Carlo) sampler, from which a latent vector $z$ defined on the manifold  $z \in \mathcal{M}$ is sampled. The RHMC sampler thus essentially enriches the Gaussian approximate posterior function to be more aware of the underlying geometry of the manifold. 

    \subsubsection{A Geometrically-Aware Posterior through the RHMC Sampler}
    % For RHVAE, we assume that the latent space is the Riemannian manifold $\mathcal{M}=\left(\mathbb{R}^{d}, \mathbf{G}\right)$ with $\mathbf{G}$ being the Riemannian metric. 
    
    % Building upon the Hamiltonian VAE (HVAE) [52], we propose to exploit the assumed Riemannian structure of the latent space by using Riemannian Hamiltonian dynamics [74] instead. The main goal remains the same and consists in using the Riemannian Hamiltonian Monte Carlo (RHMC) sampler to be able to enrich the variational posterior $q_{\phi}(z \mid x)$ such that it targets the true (unknown) posterior $p_{\theta}(z \mid x)$ while exploiting the properties of Riemannian manifolds.
    Given the Riemannian manifold $\mathcal{M}=\left(\mathbb{R}^{d}, \mathbf{G}\right)$ with our metric $\mathbf{G}$, we want to sample our latent $z$ from a distribution that is informed about the geometry of the Riemannian latent space. We therefore want to obtain this target distribution $p_{\text {target }}(z)$ through the Riemannian Hamiltonian dynamics of the RHMC sampler. 
    The core of this sampling process revolves around the concept of seeing the VAE as an energy-based model, where $z$ is seen as the position of a traveling particle in $\mathcal{M}$. We also sample a random variable $v$, which represents the velocity of this particle. Following the view of the $z$ as a particle on a manifold, we aim to essentially simulate the evolution of the traveling particle towards the target density $p_{\text {target }}(z)$ using a Markov Chain.
    We first define the potential energy $U(z)$ and kinetic energy $K(z, v)$ as   

    \begin{align}
    U(z) & =-\log p_{\text {target }}(z) \\
    K(v, z) & =\frac{1}{2}\left[\log \left((2 \pi)^{d}|\mathbf{G}(z)|\right)+v^{\top} \mathbf{G}^{-1}(z) v\right] ,
    \intertext{which together give the Hamiltonian}
    H(z,v) &= U(z) + K(v,z) .
    \end{align}
    
    This Hamiltonian equation is integrated in every step of the Markov chain, which allows us to preserve  the target density and make sure that the chain eventually converges to the stationary target distribution. 
    This essentially creates a flow that is informed both by the target distribution and by the latent space geometry thanks to the Riemannian metric $\mathbf{G}$. The approximate posterior distribution is guided by this flow, leading to better variational posterior estimates.


% -------------------------------------------------------------------------------------------------------
% --------------HYPERSPHERICAL---------------------------------------------------------------------------
% -------------------------------------------------------------------------------------------------------


\section{Hyperspherical Variational Autoencoder}\label{subsec:svae}
Another approach to solving the distortion of the latent space is to structure it as a hyperspherical manifold and assume a uniform prior. 
One of the discussed issues with vanilla variational autoencoders is that the Gaussian prior tends to concentrate points in a cluster around the center of the distribution's probability mass. In the case of multi-class data, this can become problematic, as separate clusters in the latent space will also be pulled towards the origin and therefore become difficult to separate. In an ideal case, we would still have a prior that regularizes the approximate posterior, but that does not enforce the encoded points to be at the center of the probability mass. The probability distribution that does exactly this is the uniform prior. Instead of concentrating points in one location, it spreads them over the latent space. However, the vanilla VAE's Gaussian posterior means that our latent space corresponds to a Euclidean hyperplane, a space on which the uniform prior is not well-defined. 

\subsubsection{Replacing the Gaussian by the von Mises-Fisher Distribution}
By assuming a \textit{hyperspherical} posterior, however, our latent manifold becomes a compact space on which it is possible to define a uniform prior. This is why \svae uses a von Mises-Fisher (vMF) distribution instead of the Gaussian posterior of the vanilla VAE. The vMF distribution is often considered analogous to the Gaussian distribution on a hypersphere of dimensionality $m$. Similarly to the Gaussian, it is parameterized by a mean direction $\mu \in \mathbb{R}^{m}$, but instead of variance, the vMF is parameterized by a concentration parameter around the mean $\kappa \in \mathbb{R}_{\geq 0}$. The parameters $\mu$ and $\kappa$ are called the mean direction and concentration parameter, respectively. The greater the value of $\kappa$, the higher the concentration of the distribution around the mean direction $\mu$. The distribution is unimodal for $\kappa > 0$ and is uniform on the sphere for $\kappa$ = 0. 
The probability density function of the vMF distribution for a random unit vector $\mathbf{z}$ is then defined as

\begin{align}
q(\mathbf{z} \mid \mu, \kappa) & =\frac{\kappa^{m / 2-1}}{(2 \pi)^{m / 2} \mathcal{I}_{m / 2-1}(\kappa)} \exp \left(\kappa \mu^{T} \mathbf{z}\right),
\end{align}

where the mean direction $\mu$ is a unit vector $(\|\mu\| = 1)$ and
% $\|\mu\|^{2}=1, \mathcal{C}_{m}(\kappa)$ is the normalizing constant, and 
$\mathcal{I}_{n}(\kappa)$ denotes the modified Bessel function of the first kind at order $n = (m/2-1)$.

For the special case of $\kappa=0$, the vMF represents a Uniform distribution on the $(m - 1)$-dimensional hypersphere $U\left(\mathcal{S}^{m-1}\right)$. This fact allows us to place the desired uniform prior over the hyperspherical latent space. To incorporate the newly chosen prior and posterior distribution, the KL divergence term to be optimized needs to be rewritten to

\begin{align}
    K L\left(\operatorname{vMF}(\mu, \kappa) \| U\left(\mathcal{S}^{m-1}\right)\right) &= \kappa \cdot \frac{\mathcal{I}_{m / 2}(k)}{\mathcal{I}_{m / 2-1}(k)}+\log \mathcal{C}_{m}(\kappa)-\log \left(\frac{2\left(\pi^{m / 2}\right)}{\Gamma(m / 2)}\right)^{-1},
\end{align}

Using the above as our regularization loss and Mean Squared Error (MSE) as reconstruction loss, we have defined a loss function of the hyperspherical VAE. 
% Notice that since the KL term does not depend on $\mu$, this is only optimized in the reconstruction term. The above expression cannot be handled by automatic differentiation packages because of the modified Bessel function in $\mathcal{C}_{m}(\kappa)$. Thus, to optimize this term we derive the gradient with respect to the 
% concentration parameter $\nabla_{\kappa} K L\left(\operatorname{vMF}(\mu, \kappa) \| U\left(S^{m-1}\right)\right)$ :
% $$
% \begin{align}
% & \frac{1}{2} k\left(\frac{\mathcal{I}_{m / 2+1}(k)}{\mathcal{I}_{m / 2-1}(k)}+\right. \\
% &\left.\quad-\frac{\mathcal{I}_{m / 2}(k)\left(\mathcal{I}_{m / 2-2}(k)+\mathcal{I}_{m / 2}(k)\right)}{\mathcal{I}_{m / 2-1}(k)^{2}}+1\right)
% \end{align}
% $$
% where the modified Bessel functions can be computed without numerical instabilities using the exponentially scaled modified Bessel function.
\subsubsection{Sampling from the von Mises-Fisher Distribution}
Consequently, we need to define a way to sample from the posterior distribution. Sampling from a vMF is not as trivial as from a normal Gaussian distribution, but can be achieved with an algorithm involving an acceptance-rejection scheme, based on \cite{ulrich1984computer} and further defined by \cite{davidson2018hyperspherical}. The entire algorithm for sampling from the vMF distribution is shown in Algorithm \ref{vmf_algorithm}. It consists of sampling a random scalar $\omega$ from $g(\omega \mid \kappa, m) \propto \exp (\kappa \omega)\left(1-\omega^{2}\right)^{(m-3) / 2}, \quad \omega \in[-1,1]$ using an acceptance-rejection scheme. We then sample a random vector $\mathbf{v}$ from the uniform distribution on the sphere. 
Having sampled these independently, we can define a vector $\mathbf{z}' = \left(\omega ;\left(\sqrt{1-\omega^{2}}\right) \mathbf{v}^{\top}\right)^{\top}$. The next step is to construct a Householder reflection matrix $H$, defined as $H=\mathrm{I}-2\mathbf{hh}^T$, where $H=\mathrm{I}$ is the identity matrix and $\mathbf{h} = \frac{\mathbf{e}_{1} - \mu}{\| \mathbf{e}_{1}-\mu \|}$, with modal vector $\mathbf{e}_{1}=$ $(1,0, \cdots, 0)$. Applying this Householder transform to $\mathbf{z}'$ essentially reflects it across the hyperplane that lies between $\mu$ and $\mathbf{e}_1$, resulting in $\mathbf{z} = H\mathbf{z}'$, a direction vector sampled from the vMF distribution. 

\begin{algorithm}
\begin{algorithmic}[1]
\State \textbf{input}: dimension $m$, mean $\mu$, concentration $\kappa$ 
\State Acceptance-rejection sampling:  $\omega \sim g(\omega \mid \kappa, m) \propto \exp (\omega \kappa)\left(1-\omega^{2}\right)^{\frac{1}{2}(m-3)}$ 
\State Sample $\mathbf{v}$ from Uniform distribution: $\mathbf{v} \sim U\left(\mathcal{S}^{m-2}\right)$
\State Householder transform: $\mathbf{z}^{\prime} \leftarrow\left(\omega ;\left(\sqrt{1-\omega^{2}}\right) \mathbf{v}^{\top}\right)^{\top}$
$H \leftarrow$ Householder $\left(\mathbf{e}_{1}, \mu\right)$
\State \Return: $H \mathbf{z}^{\prime}$
\end{algorithmic}
\caption{vMF Sampling}
\label{vmf_algorithm}
\end{algorithm}

The gradient for this sampling procedure can be computed using the reparameterization trick for acceptance-rejection sampling schemes as proposed by \cite{naesseth2017reparameterization} and further defined for the vMF distribution by \cite{davidson2018hyperspherical}. 

% \textcolor{red}{Should I also explain the reparameterization trick for vMFs? It's quite extensive and math heavy. I don't think so, only need sampling procedure because I refer to downsides of numerical instability later.}

    
\section{The Hyperspherical Autoencoder}\label{subsec:sae}
    % Novel idea: increasing the kappa resultls in an autencoder that still follows a hyperpshereical latnet space structuring. We have therefore created a model that retains the benefits of the autoencoder (sharper images), while still also having a structured latent space! Spread loss then makes up for lack of uniform prior
    
    Besides the benefits of the hyperspherical VAE as previously described in the literature by the likes of \cite{davidson2018hyperspherical}, there is another, to our knowledge previously unexplored benefit to the hyperspherical set-up. Namely, we can disregard the variational framework and turn our model into a hyperspherical autoencoder, providing a number of possible benefits.   
    As mentioned,  for the vMF distribution, the greater the value of $\kappa$, the higher the concentration of the distribution around the mean direction $\mu$. 
    This implies that in the limit, as $\kappa \rightarrow \infty$, the probability density will tend to a point mass distribution. We can leverage this fact to use high values of $\kappa$ to effectively turn the variational autoencoder into a regular non-variational autoencoder. The reasons for wanting to do so are twofold. First of all, autoencoders do not require a sampling procedure, which not only makes training the model faster and less computationally expensive, but also circumvents a significant problem present in the sampling procedure as detailed in Algorithm \ref{vmf_algorithm}; namely that the acceptance-rejection scheme becomes highly numerically unstable in higher dimensions \cite{davidson2018hyperspherical}. 
    
    Moreover, autoencoders tend to reconstruct sharper images than VAEs \cite{kovenko2020comprehensive}. As the images used in this project contain a great amount of detail and may be difficult for any model to reconstruct, the increased sharpness of the autoencoder over its variational variant would be a beneficial property indeed. Regular vanilla autoencoders however, are not a generative model.
    The vanilla VAE regularizes the latent space to follow a Gaussian distribution, creating a dense latent space from which we can sample realistic variations of the original input images. However, autoencoders have no such restrictions on the latent vector. The lack of structuring leads to discontinuities in the latent space that don’t result in smooth transitions between encoded points. Decoding a randomly picked vector from the latent space will accordingly likely result in a nonsensical image. 

    \subsection{Hyperspherical Autoencoders as Generative Model}
    However, we can use the hyperspherical nature of the latent space to give the autoencoder generative abilities. In the novel proposed hyperspherical autoencoder (\sae) framework, we still have a von Mises-Fisher distribution, but instead of learning the parameters $\mu$ and $\kappa$, we fix $\kappa$ to a very high value, thereby effectively turning the probability mass into a concentrated peak. Taking the mean of such a “distribution” therefore becomes equal to learning a latent vector $\mathbf{z}$ directly, with the only difference being that $\mathbf{z}$ is still constrained to be on the hypersphere. Having obtained our $\mathbf{z}$ through this method, we avoid expensive and possibly unstable sampling operations and are able to directly decode the representation to a reconstructed image. 
    

    \subsubsection{Spread Loss}
    Having defined an autoencoder with a hyperspherical latent space, we further provide structure to the latent space by introducing \textbf{spread loss}, a custom loss function that encourages points to be spread evenly over the surface of the hypersphere. We hypothesize that for spherical autoencoders specifically, such a loss will enforce a form of regularization that in case of \svae is achieved through KL divergence with a uniform prior.
    In order to achieve such a uniform spread, we maximize the distance between every pair of encoded points on the hypersphere. This distance between two points on a sphere can be computed as

    \begin{align}
        d(z_1, z_2) =   \arccos \left( \frac{\langle z_1, z_2 \rangle}{\|z_1\|\cdot\|z_2\|}\right).
    \end{align}

    However, We can observe that the arccos in this equation function is actually monotonous. When $z_1=z_2$, the inner product $\langle z_1, z_2 \rangle = 1$ and $arccos = 0$. Conversely, if $z_1$ and $z_2$ are as far away from each other on the sphere as possible, their inner product $\langle z_1, z_2 \rangle = -1$ and $arccos = \pi$. Moreover, because all encodings are unitary, we can discard the denominator of the fraction in this equation. Maximizing the distance between two points would therefore be equal to minimizing their inner product. The loss function can then be defined as
    \begin{align}
        L_\text{spread} = \sum^N_{i,j=1} -\mathbf{z}_i^T \mathbf{z}_j,
    \end{align}
    or the sum over all inner products between $N$ vectors. 

    To test the validity of this approach, we perform initial experiments in which we visualize a 3-dimensional latent space for \sae without and with spread loss implemented, as shown in Figure \ref{fig:spread}.
    \begin{figure}[H]
      \centering
      \subfloat[\sae without spread loss.]{\includegraphics[width=0.38\textwidth]{images/experiments/latent_space/latent1.png}} \quad \quad \quad
      \subfloat[\sae with with spread loss.]{\includegraphics[width=0.4\textwidth]{images/experiments/latent_space/latent2.png} \label{fig:b}}
      \caption{Visualization of Latent Space for model \sae with $M=3$ without and with spread loss. The same batch of 200 images was encoded by both models, and different image classes are visualized with different colored points. Although not entirely evenly spaced, the points encoded by the model trained with spread loss cover a significantly larger area of the sphere.} \label{fig:spread}
    \end{figure}

    In this figure, we can see that implementing spread loss has a clear effect on the distribution of encoded images over the latent space. While in the regular \sae, the points are mainly concentrated on one side and the upper half of the sphere, with spread loss, those points are more evenly distributed over the sphere surface, leaving less significant gaps in the informedness of the latent space.     
    We therefore predict that with this loss, a structured and uniformly informed latent space is obtained, which will allow us to use the proposed \sae as a generative model, while still retaining the benefits of autoencoders, such as stability and sharpness of reconstructions. 


% -------------------------------------------------------------------------------------------------------
%  Equivariant Stuff
% -------------------------------------------------------------------------------------------------------

\section{Roto-Equivariant Variational Autoencoder}\label{subsec:ksvae}
    Besides comparing the different geometric latent spaces, we also experiment with learning representations that are orientation-disentangled. CNNs are, by design, equivariant to translation. This means that translating the input will also transform the learned representation accordingly. The same does however not hold for orientation information, causing identical patches in different orientations to result in different learned representation vectors. As biopsies can be scanned in any arbitrary orientation, this redundant information thus becomes entangled in the learned latent space, possibly making the representations harder for the model to process. It would therefore be beneficial to remove this rotation information and create roto-disentangled representations.  
    
    \citeauthor{lafarge2020orientation} propose that learning such representations relies on two factors: extending the encoder and decoder networks to be \textit{group-structured}, which makes the network equivariant to rotations, and then leveraging this structure to separate oriented and non-oriented features in the latent space, which results in disentangled representations. The following section will describe how this is accomplished, providing some necessary mathematical preliminaries of group theory, explaining the concepts of group-convolutional neural networks, and describing how these concepts can be used to achieve roto-disentangled representations. 

    \subsection{Group Convolutional Neural Networks}

    \subsubsection{Group Theory} A group $G$ is a set of elements performing a \textit{group operation}, that together satisfy properties of closure, associativity, presence of the identity element, and that each have an inverse \cite{herstein1991topics}. When such a group operates on a set, this is called the \textit{group action}. Formally, a group action of $G$ on a domain $X$ is defined as a mapping in which every group element is associated with some element in $X$, such that the mapping from $G$ to the permutation group of $X$ is a homomorphism. Such a domain is known as the \textit{G-space}. Each group element can be represented as a matrix that acts on, or transforms, an element of the G-space, a mapping
    also known as the \textit{group representation} $\rho$. Multiple types of group representations exist. For this work, most important are the trivial representation, which maps any vector to itself, and the regular representation, which maps all the axes in a representation space to another basis \cite{weiler2019general}. 

    \subsubsection{Regular Group-Convolutional Neural Networks}
    In theory, regular convolutions can be generalized to such a group framework by viewing the convolutional kernel as the G-space on which elements of the group of translations ($\mathrm{T}$) act. Considering a neural network's convolutional layers in this way, allows us to more easily see how group convolutions can extend the model to be equivariant to rotations as well. Instead of the group $\mathrm{T}$, we simply extend to the finite subgroup $SE(2, N)$ of the continuous translation-rotation group $SE(2)$, where $N$ is the cyclic permutation order. Hence, we can achieve roto-translational equivariance by extending the VAE model encoder and decoder networks from regular convolutional neural networks to Group-Convolutional Neural Networks (G-CNNs) \cite{cohen2016group}. 
    
    G-CNNs generally consist of three main elements that set them apart from a regular CNN: a lifting convolution, the group convolutions, and a projection operation. The lifting convolution discretizes the orientation axis of an image by transforming the image features for every rotation angle $\frac{2\pi}{N} n $, with $n \in \{0, \dots, N-1\}$. 
    The internal feature maps can be treated as \textit{SE}($2$) images $F \in \mathbb{L}_{2}[SE(2,N)]$. Next, these are convolved with image kernels in the group convolution layers of the network, preserving the channels and ensuring equivariance under the action of the $SE(2, N)$ group. Hence, in both group lifting and convolutional layers, information about orientation and translation is preserved. Our goal however, is to obtain an invariant representation, which requires a final projection layer. This layer performs a projection with an operation that is invariant to the group action, such as summing, maxing, or averaging, resulting in a representation from which orientation and translation-information is lost.
        
    \subsubsection{Steerable Group CNNs} 
    One possible downside of the above framework is that the values of the group convolutional feature maps are computed and stored on each element of the group. The computational complexity of the model thus scales with the order of the group that is used. \cite{cohen2016steerable} therefore propose a more general framework through \textit{steerable} G-CNNs. 
    Steerable G-CNNs apply do not learn a signal directly, as is the case for G-CNNs, but instead learn to describe it through functions decomposed by a \textit{Fourier transform}. In our case this is a transform of signals over the orthogonal $SO(2)$ group, which a subgroup of $SE(2)$ that concerns only continuous rotations and no translations. By applying the Fourier transform, steerable group convolutions are expanded to the co-domain,  instead of to an additional axis in the domain as is the case for regular group convolutions. The functions, or feature vectors, in the resulting \textit{feature fields} can be interpreted as Fourier coefficients. The transformation laws of these fields are determined by the group representation type that is associated with it. 
%     Every steerable feature field is associated with corresponding group representation type, that essentially specifies their transformation behavior under transformations of the inpu a transformation law determined by its type ρ. ρ is a group
% % representation (defined in Section 2.1) that specifies how the d channels are combined together (Weiler
% % and Cesa, 2019), e.g. a trivial, regular or quotient representation.
This not only allows for more efficient memory storage, but presents a more precise way of describing signals than in regular G-CNNs.
    


\subsection{Learning Roto-Disentangled Representations}


    Having obtained encoder and decoder networks that are roto-translational equivariant, we can partition the latent space to encode both isotropic and oriented image features, thereby creating disentangled representations. Following \cite{vadgama2022kendall}, we choose to work with the more efficient steerable G-CNN type network as explained above. These networks, like regular G-CNNs, contains three key layers: a lifting layer, which requires the trivial representation type as input in order to lift the input image's feature space, a series of regular representation type steerable convolution layers, and a projection layer.   

    We then have our steerable encoder model learn three different quantities: a latent mean descriptor $\mu$ that is equivariant under the actions of the group $SE(2,N)$, a pose or orientation corresponding to this representation $\mathbf{R}$, and an invariant scalar parameter that either corresponds to $\kappa$ in the case of the hyperspherical framework, or $\sigma$ in the case of the regular Gaussian posterior. Because the architecture is equivariant, rotating the network's input results in a transformation of both the mean descriptor $\mu$, as well as the estimated pose $\mathbf{R}$ via a representation of the group, whereas the predicted parameter $\kappa$ or $\sigma$ stays invariant. 
    The network's equivariance allows us to essentially undo the rotation of the mean descriptor and orient it to a canonical pose via the mapping $\mu_0 = \rho(\mathbf{R}^{-1}) \mu$, with $\rho(R)$ a group-representation\footnote{SO(2) group representations generalize the notation of rotation to vectors other than the usual 2D vectors} of SO(2).
    Thus, our method obtains an invariant descriptor $\mu_0$ that represents a whole equivalence class of images that are just rotated copies of one another. The main learning objective is thus to learn a probability distribution on this equivalence class, which is done as usual through variational inference. For the decoding process, we sample a vector $\hat{z}_0$ from this distribution, map it to its corresponding learned pose $\mathbf{R}$, and feed it through the equivariant decoder network. This results in a reconstructed image that has the same orientation as the original input image.
    





\section{Experimental Setup}
\label{sec:experiments}
\begin{figure}[t]
    \centering 
    \hspace{-.04\columnwidth}
    \includegraphics[width=1.025\columnwidth]{results/VOC/figures/pareto_example.pdf}
    \caption{\textbf{Selecting models for evaluation.} For each configuration, we evaluate every model at every checkpoint and measure its performance across various metrics (\fone, \epg, \iou) on the validation set; \ie every point in the left graph corresponds to one model (for \bcos models optimized via the \epgloss loss at the input layer). Instead of evaluating a single model on the test set, we evaluate \emph{all Pareto-dominant} models, as indicated in the center and right plot.
    % \moritz{Did we not update the results to be consistent with this? I distinctly remember creating the plots for this. (The Pareto front here as a lot more points than those in the result figures...)}
    }
    \label{fig:pareto_example}
\end{figure}

In this section, we describe our experimental setup
and how we select the best models across metrics. {Full training details can be found in the supplement.} We evaluate across the full sweep of combinations of choices for each category, and discuss our results in \cref{sec:results}. 

\myparagraph{Datasets:} We evaluate on \voc \citeMain{everingham2009pascal} and \coco \citeMain{lin2014microsoft} for multi-label image classification. {In \cref{sec:results:waterbirds}, to understand the effectiveness of model guidance in mitigating spurious correlations, we also evaluate on the synthetically constructed Waterbirds-100 dataset \citeMain{sagawa2019distributionally,petryk2022guiding}, where landbirds are perfectly correlated with land backgrounds on the training and validation sets, but are equally likely to occur on land or water in the test set (similar for waterbirds and water). With this dataset, we evaluate model guidance for suppressing undesired features.}

\myparagraph{Attribution Methods and Architectures:} As described in \cref{sec:method:attributions}, we evaluate with \ixg \citeMain{shrikumar2017learning}, \intgrad \citeMain{sundararajan2017axiomatic}, \bcos \citeMain{bohle2022b}, and \gradcam \citeMain{selvaraju2017grad} using models with a \resnet \citeMain{he2016deep} backbone. For \intgrad, we use an \xdnn \resnet \citeMain{hesse2021fast} to reduce the computational cost, and a \bcos \resnet for the \bcos attributions. We optimize the attributions at the input and final layer\footnote{As typically used in \ixg (input) and \gradcam (final) respectively.}; for intermediate layer results, see supplement. Given the similarity of the results between \gradcam and \ixg, and since \bcos attributions performed better than \gradcam for \bcos models, we show \gradcam results in the supplement. 
All models were pretrained on \imagenet \citeMain{imagenet}, and model guidance was performed starting from a baseline model fine-tuned on the target dataset.

\myparagraph{Localization Losses:} As described in \cref{sec:method:losses}, we compare four localization losses in our evaluation: (i) \energyloss, (ii) \loneloss \citeMain{gao2022aligning,gao2022res}, (iii) \ppceloss \citeMain{shen2021human}, and (iv) \rrrloss (cf.~\cref{sec:method:losses}, \citeMain{ross2017right}).

\myparagraph{Evaluation Metrics:} As discussed in \cref{sec:method:metrics}, we evaluate both for classification and localization performance of the models. For classification, we report the F1 scores, similar results with \map scores can be found in the supplement. For localization, we evaluate using the \epg and \iou scores.

\myparagraph{Selecting the best models:} As we evaluate for two distinct objectives (classification and localization), it is non-trivial to decide which models to select during training. \Eg, a model that provides the best classification performance might provide significantly worse localization performance than a model that provides slightly lower classification performance but much better localization. Finding the right balance and deciding which of those models in fact constitutes the `better' model depends on the preference of the end user. 
Hence, instead of selecting models based on a single metric, we select the set of Pareto-dominant models \citeMain{pareto1894massimo,pareto2008maximum,backhaus1980pareto} across three metrics---F1, \epg, and \iou---for each training configuration, as defined by a combination of attribution method, layer, and loss. Specifically, as shown in \cref{fig:pareto_example}, we train for each configuration using three different choices of $\lambda_\text{loc}$, and select the set of Pareto-dominant models among all checkpoints (epochs and $\lambda_\text{loc}$). This provides a more holistic view of the general trends on the effectiveness of model guidance for each configuration.
\section{Discussion and Limitations}

Although we can ablate concepts efficiently for a wide range of object instances, styles, and memorized images, our method is still limited in several ways. First, while our method overwrites a target concept, this does not guarantee that the target concept cannot be generated through a different, distant text prompt. We show an example in \reffig{limitation} (a), where after ablating {\menlo Van Gogh}, the model can still generate {\menlo starry night painting}. However, upon discovery, one can resolve this by explicitly ablating the target concept {\menlo starry night painting}. Secondly, when ablating a target concept, we still sometimes observe slight degradation in its surrounding concepts, as shown in \reffig{limitation} (c). 

\nupur{Our method does not prevent a downstream user with full access to model weights from re-introducing the ablated concept~\cite{ruiz2022dreambooth,kumari2022multi,gal2022image}. Even without access to the model weights, one may be able to iteratively optimize for a text prompt with a particular target concept. Though that may be much more difficult than optimizing the model weights, our work does not guarantee that this is impossible.}

Nevertheless, we believe every creator should have an ``opt-out'' capability. We take a small step towards this goal, creating a computational tool to remove copyrighted images and artworks from large-scale image generative models.


\newpage
{\small
\bibliographystyle{ieee_fullname}
\bibliography{paper_for_review}
}

\onecolumn
\def\thesection{\Alph{section}}

\vspace{+1cm}
\begin{center}
 {\huge Supplementary Materials}
\end{center}
\vspace{+1cm}

\appendix

In this supplementary file, we provide the following materials:
\begin{itemize}
    \item More visual comparison of pixel-supervised and box-supervised VIS methods ($cf.$ Section 4.2 in the main paper);
    \item More visualization of instance masks predicted by BoxVIS ($cf.$ Section 4.2 in the main paper).
\end{itemize}

\noindent\textbf{A. More visual comparison of pixel-supervised and box-supervised VIS methods}

In Figs. { \color{red} 1} - { \color{red} 3}, we provide more visual comparison results of instance masks predicted by M2F-VIS \cite{cheng2021mask2former-video}, M2F-VIS-box and our BoxVIS on videos with crowded scenes, blurred or occluded objects. It can bee seen that compared to the box-supervised baseline M2F-VIS-box, our BoxVIS can predicate instance masks with better spatial and temporal consistency. On the other hand, BoxVIS can yield instance masks of the same high quality as the pixel-supervised M2F-VIS \cite{cheng2021mask2former-video}.

\

% --------------------------------------------------------------------------------
\noindent\textbf{B. Visualization of instance masks predicted by BoxVIS} 

In \cref{fig:visualizer_yt21} and \cref{fig:visualizer_ovis}, we visualize instance masks predicted by BoxVIS on various videos of YTVIS21 and OVIS, demonstrating the model generalization ability.

\begin{figure*}[h]
    \centering
    \includegraphics[width=0.98\textwidth]{figs/fig/multi_objs_comparison.jpg}
\caption{Visual comparison of instance segmentation results by M2F-VIS (top), M2F-VIS-box (middle row) and BoxVIS (bottom row) on videos with crowded scenes.}
\label{fig:visualizer_crowded}
\end{figure*}

\begin{figure*}[h]
    \centering
    \includegraphics[width=0.98\textwidth]{figs/fig/occ_comparison.jpg}
\caption{Visual comparison of instance segmentation results by M2F-VIS (top), M2F-VIS-box (middle row) and BoxVIS (bottom row) on videos with occluded objects.}
\label{fig:visualizer_occ}
\end{figure*}

\begin{figure*}[h]
    \centering
    \includegraphics[width=0.98\textwidth]{figs/fig/motion_comparison.jpg}
\caption{Visual comparison of instance segmentation results by M2F-VIS (top), M2F-VIS-box (middle row) and BoxVIS (bottom row) on videos with blurred objects.}
\label{fig:visualizer_motion}
\end{figure*}


\begin{figure*}[h]
    \centering
    \includegraphics[width=0.98\textwidth]{figs/fig/visual_yt21_sm1.jpg}
    \includegraphics[width=0.98\textwidth]{figs/fig/visual_yt21_sm2.jpg}
\caption{Visualization of instance masks predicted by BoxVIS on YTVIS21 valid set.  }
\label{fig:visualizer_yt21}
\end{figure*}

\begin{figure*}[t]
    \centering
    \includegraphics[width=0.98\textwidth]{figs/fig/visual_ovis_sm.jpg}
    \includegraphics[width=0.98\textwidth]{figs/fig/visual_ovis_sm2.jpg}
\caption{Visualization of instance masks predicted by BoxVIS on OVIS valid set.  }
\label{fig:visualizer_ovis}
\end{figure*}


\end{document}