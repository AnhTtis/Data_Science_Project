
\section{Introduction}

In the past decade, we have witnessed the significant success of convolutional neural networks (CNNs) being applied to various grid-based \cite{simoncelli2001natural} medical imaging applications such as magnetic resonance imaging (MRI) reconstruction \cite{zhang2021efficient,muckley2021results} and lesion segmentation \cite{zhang2021all,kamnitsas2017efficient}.
% The architecture design of CNNs naturally imposes several desirable inductive biases for grid-based image processing \cite{simoncelli2001natural}, e.g. the translation equivariance through spatially invariant convolution filters \cite{ruderman1993statistics,olshausen1996natural}, the translation invariance \cite{kayhan2020translation} with pooling layers and the locality \cite{lenc2015understanding,lecun1989handwritten}.
% These are most general inductive biases in CNNs and have been proved to be useful for many computer vision applications.
Despite the success of general inductive biases such as translation equivariance \cite{kayhan2020translation} and locality \cite{lenc2015understanding}, medical images represent different diseases and require highly domain-specific knowledge. 
Thus, the question of how to inject domain-specific inductive biases (priors) beyond the general ones into neural networks for medical image processing remains to be answered.

In this paper, we attempt to answer though tackling the identification problem of a particular type of multiple sclerosis (MS) lesion, called a chronic active lesion (termed as a rim+ lesion).
A rim+ lesion is characterized by an iron-enriched rim of activated macrophages and microglia in histopathology studies \cite{absinta2016persistent,gillen2021qsm,dal2017slow,kaunzner2019quantitative} and are visible with in-vivo quantitative susceptibility mapping (QSM) \cite{wang2015quantitative,wang2017clinical,de2010quantitative} and phase imaging \cite{absinta2016persistent,absinta2013seven} techniques, where these lesions show a paramagnetic hyperintense rim at the edge (see Fig. \ref{fig:lesion-grads}).
Several attempts \cite{barquero2020rimnet,lou2021fully,zhang2022qsmrim} have been made to address the problem, but a clinically reliable one is not yet available.

\begin{figure}[!t]
	\centering
	\vspace{-1ex}
        \includegraphics[width=0.98\columnwidth]{./lesion_grads.png}

        \caption{
            A visual example of the difference of a rim+ and a rim- lesion. 
            QSM image patches show the the magnetic susceptibility distribution for the lesions.
            Fluid attenuated inversion recovery (FLAIR) image patches show the exact location of the lesions.
            The gradient filed map of QSM images show gradient vectors normalized to unit vectors (the darker of the blue, the larger of the magnitude of a gradient vector).
            The right two columns are gradient magnitude maps $\mathbf{V}_s$ and QSM value maps $\mathbf{V}_u$ processed by DA-TR (see Section \ref{sec:rim}).
            The rim+ lesion shows structured patterns in the accumulator space by aggregating feature values along gradients; however, the rim- lesion possess no such structures. 
        }
	\label{fig:lesion-grads}
\end{figure}

Considering the limited amount of data and the high class imbalance, it is more desirable to encode priors with domain knowledge into the network explicitly.
It can be seen from the Fig. \ref{fig:lesion-grads} that rim+ lesions differ from rim- (non rim+) lesions in three ways. 
First, rim+ lesions have a hyperintense ring-like structure at the edge of the lesion on QSM.
Second, in rim+ lesions, a greater magnitude of gradients is observed near the lesion edge unlike rim- lesions.
Third, rim+ lesions can be characterized by radially oriented gradients on the edge; however, rim- lesions do not possess such structured orientations.

In this paper, we propose \textbf{De}ep \textbf{D}irected \textbf{A}ccumulator (DeDA), an image processing operation symmetric to the grid sampling within the forward-backward neural network framework to explicitly encode the above prior information into networks.
Given a feature map and a set of sampling grids, DeDA creates and quantizes an accumulator space into finite intervals, and accumulates feature values accordingly.
This DeDA operation can also be regarded as a generalized discrete Radon transform, as it maps values between two discrete functional space through accumulation.
The main contribution of this paper are two folds. 
First, we introduce a simple yet effective method DeDA, a generalized image processing operation, for increasing the representation capacity of neural networks by integrating domain-specific prior information explicitly. 
Second, experimental results on rim+ lesion identification show that $10.1\%$ of improvement in a partial (false positive rate less than $10\%$) area under the receiver operating characteristic curve (pROC AUC) and $10.2\%$ of improvement in an area under the precision recall curve (PR AUC) can be achieved respectively comparing to other state-of-the-art methods.  