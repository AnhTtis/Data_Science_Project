\newcommand{\vket}[2]{
\begin{bmatrix} #1 \\ #2 \end{bmatrix}
}

\newcommand{\hbra}[2]{
\begin{bmatrix} #1  \; #2 \end{bmatrix}
}

\onecolumngrid
\appendix
\section{Supplementary information}

\subsection{Bloch wall hamiltonian spectrum with $k_{||} = 0$}
\label{suppblochgammaderivation}
First, one can choose the equivalent block-diagonal $\tau_z$ matrix for the momentum term in Eq.~\ref{eq:PWhamiltonian} and set m = 0. This decouples the left (-) and right (+) helical states $\eta$, and allows one to express the Hamiltonian of the infinitely-periodic hamiltonian with helical exchange field as follows. Let $\Omega = \frac{J}{2\hbar v_f}$, and let $G=\frac{2\pi}{a_S} = \vec{b}_x$ be the superlattice reciprocal vector. We use the definition for $\boldsymbol{\beta}(r)\boldsymbol{\beta}(r)$ in section II: 
\begin{eqnarray}
    \label{FT-exchange}
    \boldsymbol{\beta}_N(r) = J(\cos(\theta_r)\hat{y} + \sin(\theta_r)\hat{x}) = \frac{J}{2}(e^{i G x}\hat{y}+e^{-i G x}\hat{y} + -ie^{i G x}\hat{x}+ie^{-i G x}\hat{x}), \\\boldsymbol{\beta}_B(r) = J(\cos(\theta_r)\hat{y} + \sin(\theta_r)\hat{z}) = \frac{J}{2}(e^{i G x}\hat{y}+e^{-i G x}\hat{y} + -ie^{i G x}\hat{z}+ie^{-i G x}\hat{z}).
\end{eqnarray}

In Fourier space, this can be written as:
\begin{eqnarray}
    \widetilde{\boldsymbol{\beta}}^N_{G'-G} = \frac{J}{2}[(i\hat{x} + \hat{y})\delta(G'-G-\vec{b}_x) + (-i\hat{x} + \hat{y})\delta(G'-G+\vec{b}_x)]; \\
    \widetilde{\boldsymbol{\beta}}^B_{G'-G} = \frac{J}{2}[(i\hat{z} + \hat{y})\delta(G'-G-\vec{b}_x) + (-i\hat{z} + \hat{y})\delta(G'-G+\vec{b}_x)].
\end{eqnarray}

We will now focus on the Bloch wall case. 
\begin{equation}
    \hat{H}_B(k) = \hbar v_f\begin{bmatrix}
    \ddots & \Omega (\sigma_y + i \sigma_z) &    &      &    \\
     \Omega (\sigma_y - i \sigma_z) & \eta(k_x + G)\sigma_x       & \Omega (\sigma_y + i \sigma_z)   &     &    \\
      &         \Omega (\sigma_y - i \sigma_z)& \eta (k_x)\sigma_x         & \Omega (\sigma_y + i \sigma_z)    &     \\
      &         &           \Omega (\sigma_y - i \sigma_z)&     \eta (k_x - G)\sigma_x      & \Omega (\sigma_y + i \sigma_z)   \\
      &         &           &           \Omega (\sigma_y - i \sigma_z)& \ddots
  \end{bmatrix}
\end{equation}

Then, the basis transformation corresponding to a $120^{\circ}$ rotation about $\hat{x} +\hat{y} + \hat{z}$ can be used to transform the spin matrices:
\begin{eqnarray}
\sigma_y \rightarrow -\sigma_x  \nonumber \\
\sigma_z \rightarrow \sigma_y \nonumber \\
\sigma_x \rightarrow -\sigma_z \nonumber
\end{eqnarray}
and, knowing that $\sigma_+ = \sigma_x + i \sigma_y; \sigma_- = \sigma_x - i \sigma_y$, the Hamiltonian can be rewritten as:
\begin{equation}
    \hat{H}_B(k) = -\hbar v_f\begin{bmatrix}
    \ddots & 2\Omega (\sigma_-) &    &      &    \\
     2 \Omega (\sigma_+) & \eta(k_x + G)\sigma_z       & \Omega (\sigma_-)   &     &    \\
      &         2\Omega (\sigma_+)& \eta (k_x)\hat{\sigma}_z         & 2\Omega (\sigma_-)    &     \\
      &         &           2\Omega (\sigma_+)&     \eta (k_x - G)\hat{\sigma}_z      & 2\Omega (\sigma_-)   \\
      &         &           &           \Omega (\sigma_+)& \ddots
  \end{bmatrix}
\end{equation}
\begin{equation}
  = -\hbar v_f \bigoplus_n \left(\begin{bmatrix}
  -\eta(k_x + (n+1)G) & 2\Omega \\
  2\Omega & \eta(k_x + n G)
  \end{bmatrix}\right) = -\hbar v_f \bigoplus_n \left(-\eta k_x \sigma_z + 2 \Omega \sigma_x - \eta n G \sigma_z -\eta G\frac{1}{2}(\sigma_z+\sigma_0)\right)
\end{equation}

\begin{equation}
= -\hbar v_f \bigoplus_n (\icol{2\Omega \\ 0 \\ -\eta(k_x +  G (n + \frac{1}{2}))}\cdot\boldsymbol{\sigma} - \eta G \frac{1}{2}\sigma_0) = \bigoplus_n \hat{H}_n
\end{equation}
with the spectrum of $\hat{H}_n$ (thus, a diagonal subblock of $\hat{H}_B(k_{||}=\Gamma))$ given as:

\begin{equation}
    E_n = -\hbar v_f \left( \pm\sqrt{4\Omega^2 + (k_x + G(n+\frac{1}{2}))^2} - \eta G / 2\right) = \mp \sqrt{J^2 + \hbar^2 v_f^2 (k_x + \frac{\pi}{a_S}(2 n+1))^2} + \hbar v_f  \frac{\eta \pi}{a_S}
\end{equation}


\subsection{Decomposing Bloch exchange helix Hamiltonian}
First, it is useful to rotate the problem with $G=\frac{2\pi}{a_S}$ in $\hat{z}$ while preserving the Bloch magnetization $\boldsymbol{\beta}_B$'s helicity:

\begin{equation}
    \boldsymbol{\beta}_B = \text{J(cos}(G z)\hat{x} + \text{sin}(G z)\hat{y}
\end{equation}

Introducing $G^\pm = e^{\pm i G z}$ and absorbing $\hbar$ into $v_f$, we can rewrite the Hamiltonian $\hat{H}_B$ for one chirality $\eta$:

\begin{equation}
    \hat{H}_B = -i \eta v_f (\frac{\partial}{\partial x} \sigma_x + \frac{\partial}{\partial y} \sigma_y + \frac{\partial}{\partial z} \sigma_z) + J(G^+\sigma_+ + G^-\sigma_-)   
\end{equation}

We can then Fourier transform the Hamiltonian in x and y, and change the basis of $\hat{H}_B$ so as to diagonalize the helical exchange term:

\begin{align}
    \hat{H}'_B = \frac{1}{2}\begin{bmatrix}
        G^- & 1 \\
        -G^- & 1
    \end{bmatrix}
    \begin{bmatrix}
        -i \eta v_f \partial_z & \eta v_f k_x -i \eta v_f k_y + J G^+ \\
        \eta v_f k_x + i \eta v_f k_y + J G^- & i \eta v_f \partial_z \\
    \end{bmatrix} 
    \begin{bmatrix}
        G^+ & -G^+ \\
        1 & 1
    \end{bmatrix}. \\
    = \frac{1}{2}\begin{bmatrix}
        G^- & 1 \\
        -G^- & 1
    \end{bmatrix}
    \begin{bmatrix}
        A & B\\
        C & D \\
    \end{bmatrix}; \\
    A = G G^+\eta v_f - i \eta v_f G^+ \partial_z+JG^+ + \eta v_f \left(k_x-i k_y\right), \nonumber \\
    B = i \eta v_f G^+ \partial_z - G\eta v_f G^+ + JG^+ + \eta v_f \left(k_x - i k_y\right), \nonumber \\
    C = J + \eta v_f G^+\left(k_x +i k_y \right) + i \eta v_f \partial_z \nonumber, \\
    D = -J - \eta v_fG^+\left(k_x + i k_y\right) +i\eta v_f \partial_z;\nonumber \\ 
    \nonumber \\ 
    = \frac{1}{2}
    \begin{bmatrix}
        G^- A+C & G^-B+D\\
        -G^-A+C & -G^-B+D \\
    \end{bmatrix}; \\
    G^- A+C = G \eta v_f  + 2J + k_x \eta v_f \left(G^++G^-\right) + i k_y \eta v_f \left(G^+-G^-\right), \nonumber \\ 
    G^- B+D = 2 i \eta v_f \partial_z  - G \eta v_f  + k_x \eta v_f \left(-G^++G^-\right) - i k_y \eta v_f \left(G^++G^-\right), \nonumber \\ 
    -G^- A+C = - G \eta v_f  + 2 i \eta v_f \partial_z + k_x \eta v_f \left(G^+-G^-\right) + i k_y \eta v_f \left(G^++G^-\right), \nonumber \\ 
    -G^- B+D = G \eta v_f  - 2J - k_x \eta v_f \left(G^++G^-\right) - i k_y \eta v_f \left(G^+-G^-\right). \nonumber \\ 
\end{align}

Switching to cylindrical coordinates in x and y, we can define $k_x = \text{cos}(\theta_k)k_r$ and $k_y = \text{sin}(\theta_k)k_r$. We can also rewrite the above expressions: 

\begin{align}
    G^- A+C = G \eta v_f  + 2J + k_r \text{cos}(\theta_k) \eta v_f \left(G^++G^-\right) + i k_r \text{sin}(\theta_k \eta v_f \left(G^+-G^-\right), \nonumber \\ 
    G^- B+D = 2 i \eta v_f \partial_z  - G \eta v_f  + k_r \text{cos}(\theta_k) \eta v_f \left(-G^++G^-\right) - i k_r \text{sin}(\theta_k) \eta v_f \left(G^++G^-\right), \nonumber \\ 
    -G^- A+C = - G \eta v_f  + 2 i \eta v_f \partial_z + k_r \text{cos}(\theta_k) \eta v_f \left(G^+-G^-\right) + i k_r \text{sin}(\theta_k) \eta v_f \left(G^++G^-\right), \nonumber \\ 
    -G^- B+D = G \eta v_f  - 2J - k_r \text{cos}(\theta_k) \eta v_f \left(G^++G^-\right) - i k_r \text{sin}(\theta_k) \eta v_f \left(G^+-G^-\right); \nonumber \\ 
    \nonumber \\ 
    G^- A+C = G \eta v_f  + 2J + k_r \eta v_f \left( e^{i\theta_k} G^+ + e^{-i\theta_k}G^- \right), \nonumber \\ 
    G^- B+D = 2 i \eta v_f \partial_z  - G \eta v_f  +k_r \eta v_f \left(-e^{i\theta_k}G^+ + e^{-i\theta_k}G^-\right), \nonumber \\ 
    -G^- A+C = - G \eta v_f  + 2 i \eta v_f \partial_z + k_r \eta v_f \left(e^{i\theta_k}G^+ - e^{-i\theta_k}G^-\right), \nonumber \\ 
    -G^- B+D = G \eta v_f  - 2J - k_r \eta v_f \left( e^{i\theta_k} G^+ + e^{-i\theta_k}G^- \right); \nonumber \\ 
    \nonumber \\
    \hat{H}'_B=
    \begin{bmatrix}
        \frac{G \eta v_f}{2} + J + k_r \eta v_f \text{cos}(Gx + \theta_k) &  i \eta v_f \partial_z  - \frac{G \eta v_f}{2}  -i k_r \eta v_f \text{sin}(Gx + \theta_k)\\
        i \eta v_f \partial_z  - \frac{G \eta v_f}{2} + i k_r \eta v_f \text{sin}(Gx + \theta_k) & \frac{G \eta v_f}{2} - J - k_r \eta v_f \text{cos}(Gx + \theta_k) \\
    \end{bmatrix} \nonumber \\
    = \frac{G \eta v_f}{2}\hat{\sigma}_0 + \left(J+k_r\eta v_f \text{cos}(Gx + \theta_k)\right)\hat{\sigma}_z + \left(i \eta v_f \partial_z  - \frac{G \eta v_f}{2}\right)\hat{\sigma}_x +k_r\eta v_f \text{sin}(Gx + \theta_k)\hat{\sigma}_y\label{chiralityfilter}
\end{align}

We can notice that $\hat{H}_B$ permits a continuous set of spatially translated solutions for each value of $\theta_k$. Here, we will also stop and note the correspondence with \ref{suppblochgammaderivation} for $k_r = 0$. We may also notice that a $k_r$ corresponding to the position of the uniformly-magnetized $|L\rangle$-chiral weyl node,

%\begin{align}
%    \hat{H}'_B  = \frac{G \eta v_f}{2}\hat{\sigma}_0 + \left(J+k_r\eta v_f \text{cos}(Gz + \theta_k)\right)\hat{\sigma}_z + \left(i \eta v_f \partial_z  - \frac{G \eta v_f}{2}\right)\hat{\sigma}_x +k_r\eta v_f G \text{sin}(Gz+\theta_k) \hat{\sigma}_y
% \end{align}
\begin{equation}
    k_r = \frac{-\eta J}{v_f},
\end{equation}
will cancel the spatially constant terms of $\hat{\sigma}_z$ using a taylor series around $\theta_k=0$ in z:
\begin{align}
    \hat{H}'_B\left(k=\frac{-\eta J}{v_f}\right)  = \frac{G \eta v_f}{2}\hat{\sigma}_0 + J\frac{\left(Gx\right)^2}{2}\hat{\sigma}_z + \left(i \eta v_f \partial_z  - \frac{G \eta v_f}{2}\right)\hat{\sigma}_x - J G x \hat{\sigma}_y.
\end{align}
Giving us the insight that the Hamiltonian is locally akin to the Dirac Hamiltonian with an additional off-diagonal term which, if the wavefunctions are represented in a hermite-gauss function basis, will generate $|n\pm2\rangle$ states. Squaring $H'_B$ and again changing the basis ($\sigma_x \rightarrow \sigma_z$, $\sigma_z \rightarrow -\sigma_x$) allows us to decouple the spatially-varying terms of the Hamiltonian for each spin, giving us some insight as explained in the main text:

\begin{align}
    H_B^{'2} = \left(\frac{\eta \hbar v_f G}{2}\right)^2 + 2\left(\frac{\eta \hbar v_f G}{2}\right)\Omega + \Omega^2 \nonumber \\  
    = \frac{\hbar^2 v_f^2 G^2}{4}\hat{\sigma}_0+\hbar^2 v_f^2 G k_r\left[\text{sin}(Gz+\theta_k)\hat{\sigma}_y +  \text{cos}(Gz+\theta_k)\hat{\sigma}_z\right]-\hbar^2 v_f^2 G k_r\left[\text{sin}(Gz+\theta_k)\hat{\sigma}_y +  \text{cos}(Gz+\theta_k)\hat{\sigma}_z\right] \nonumber \\ +\left[\frac{\hbar^2 v_f^2 G^2}{4} -\hbar^2 v_f^2 \left(\partial_z^2 +i G \partial_x\right) + 2J^2+2\hbar^2 k_r^2\text{cos}(Gz+\theta_k)\right]\hat{\sigma}_0+i\hbar^2 v_f^2 G \partial_z \hat{\sigma}_x + \eta \hbar v_f G J\hat{\sigma}_z -\frac{\hbar^2 v_f^2 G^2}{4}\hat{\sigma}_x.\\
    \rightarrow \nonumber \\ 
    \hat{H}_B^2 = -\eta \hbar v_f G J \hat{\sigma}_x - i G\hbar^2 v_f^2 \partial_z \left(\hat{\sigma}_0 - \hat{\sigma}_z\right) -\frac{\hbar^2 v_f^2 G^2}{4}\hat{\sigma}_z +\left(2J^2 + 2\eta k_r v_f \text{cos}(Gz+\theta_k)-\hbar^2v_f^2\partial_z^2+\frac{\hbar^2 v_f^2 G^2}{2}\right)\hat{\sigma}_0.
\end{align}

We are unable to solve this exactly, but can manipulate the spin terms of $\left(H_B-E\right)|\Psi\rangle = 0$ to generate constraints on the wavefunctions, defining $\Lambda = \left(\eta \hbar v_f k_r e^{i \theta_{k}} +Je^{-iGx} \right)^{-1}$.
\begin{align}
    \left[\left(\hbar v_f\right)^2 \left(\Lambda^+ \partial_x^2 + \frac{\partial \Lambda^+}{\partial x}\partial_x \right) +i\eta \hbar v_f \frac{\partial \Lambda^+}{\partial x}E - \Lambda^+ E^2+ \frac{1}{\Lambda}\right]|\uparrow\rangle = 0, \\ 
    \left[\left(\hbar v_f\right)^2 \left(\Lambda \partial_x^2 + \frac{\partial \Lambda}{\partial x}\partial_x \right) -i\eta \hbar v_f \frac{\partial \Lambda}{\partial x}E - \Lambda E^2+ \frac{1}{\Lambda^+}\right]|\downarrow\rangle  = 0,\label{downbloch}\\
    |\downarrow\rangle = -\Lambda^+ \left(-i\eta v_f \partial_z - E\right)|\uparrow\rangle,\\
    |\uparrow\rangle = -\Lambda \left(i\eta v_f \partial_z - E\right)|\downarrow\rangle. \label{upbloch}
\end{align}

 Composing Eq. \ref{downbloch} and Eq. \ref{upbloch} at $k_{r} = \frac{-\eta J}{\hbar v_f}$ and $E = \eta \hbar v_f G/4$, we can use the Wolfram Mathematica software to find two solutions for a $|\Psi\rangle$ with normalization coefficients $\mathbb{C}_1, \mathbb{C}_2$:

 \begin{align}
     |\Psi\rangle = \frac{1}{{\sqrt[4]{e^{i G x}}}}\left[\mathbb{C}_1 e^{\frac{2 J \left(1+e^{i G x}\right)}{G \hbar v_f \sqrt{e^{i G x}}}}+\mathbb{C}_2 e^{-\frac{2 J \left(1+e^{i G x}\right)}{G \hbar v_f \sqrt{e^{i G x}}}}\right]|\downarrow\rangle \nonumber \\
     -\eta  \sqrt[4]{e^{i G x}}  \left[\mathbb{C}_1 e^{\frac{2 J \left(1+e^{i G x}\right)}{G \hbar v_f \sqrt{e^{i G x}}}}-\mathbb{C}_2 e^{-\frac{2 J \left(1+e^{i G x}\right)}{G \hbar v_f \sqrt{e^{i G x}}}}\right]|\uparrow\rangle.
 \end{align}
 