% ****** Start of file apssamp.tex ******
%
%   This file is part of the APS files in the REVTeX 4.2 distribution.
%   Version 4.2a of REVTeX, December 2014
%
%   Copyright (c) 2014 The American Physical Society.
%
%   See the REVTeX 4 README file for restrictions and more information.
%
% TeX'ing this file requires that you have AMS-LaTeX 2.0 installed
% as well as the rest of the prerequisites for REVTeX 4.2
%
% See the REVTeX 4 README file
% It also requires running BibTeX. The commands are as follows:
%
%  1)  latex apssamp.tex
%  2)  bibtex apssamp
%  3)  latex apssamp.tex
%  4)  latex apssamp.tex
%

\documentclass[reprint,superscriptaddress,amsmath,amssymb,aps,prl]{revtex4-2}

%\documentclass[%
%reprint,
%superscriptaddress,
%groupedaddress,
%unsortedaddress,
%runinaddress,
%frontmatterverbose, 
%preprint,
%preprintnumbers,
%nofootinbib,
%nobibnotes,
%bibnotes,
% amsmath,amssymb,
% aps,
% prl,
%pra,
%prb,
%rmp,
%prstab,
%prstper,
%floatfix,
%]{revtex4-2}

\usepackage{graphicx}% Include figure files
\usepackage{dcolumn}% Align table columns on decimal point
\usepackage{bm}% bold math

\setlength{\parskip}{0pt}
\setlength{\textfloatsep}{2.0pt plus 2.0pt minus 2.0pt}

\newcommand{\icol}[1]{% inline column vector
  \left(\begin{smallmatrix}#1\end{smallmatrix}\right)%
}

%\usepackage{hyperref}% add hypertext capabilities
%\usepackage[mathlines]{lineno}% Enable numbering of text and display math
%\linenumbers\relax % Commence numbering lines

%\usepackage[showframe,%Uncomment any one of the following lines to test 
%%scale=0.7, marginratio={1:1, 2:3}, ignoreall,% default settings
%%text={7in,10in},centering,
%%margin=1.5in,
%%total={6.5in,8.75in}, top=1.2in, left=0.9in, includefoot,
%%height=10in,a5paper,hmargin={3cm,0.8in},
%]{geometry}

\usepackage[bookmarks=false,linkcolor=blue,urlcolor=blue,colorlinks,citecolor=magenta]{hyperref}

\begin{document}

%\preprint{APS/123-QED}

\title{Flat bands and multi-state memory devices from chiral domain wall \\ superlattices in magnetic Weyl semimetals}

%\title{Multi-state domain wall lattice %racetrack memory devices using\\
%gigantic magnetic Weyl semimetal magnetoresistance}% Force line breaks with \\

\author{Vivian Rogers}
 \email{vivian.rogers@utexas.edu}
 \affiliation{%
 Chandra Department of Electrical and Computer Engineering, The University of Texas at Austin, Austin TX 78712
}%
\author{Swati Chaudhary}%
 \email{swati.chaudhary@austin.utexas.edu}
 \affiliation{Department of Physics, The University of Texas at Austin, Austin, Texas 78712, USA}
\affiliation{Department of Physics, Northeastern University, Boston, Massachusetts 02115, USA}
\affiliation{Department of Physics, Massachusetts Institute of Technology, Cambridge, Massachusetts 02139, USA}
\author{Richard Nguyen}
\affiliation{%
 Chandra Department of Electrical and Computer Engineering, The University of Texas at Austin, Austin TX 78712
}%
\author{Jean Anne Incorvia}%
 \email{incorvia@austin.utexas.edu}
\affiliation{%
 Chandra Department of Electrical and Computer Engineering, The University of Texas at Austin, Austin TX 78712
}%





%\date{\today}% It is always \today, today,
             %  but any date may be explicitly specified

\begin{abstract}


In this work, we study the electronic structure and transport in magnetic Weyl semimetal (MWSM) chiral domain wall (DW) superlattices. We show that the device conductance is strongly modulated by the number of DWs between contacts, providing a means to encode multiple resistance states in a single racetrack device. Additionally, we show that periodic exchange fields can generate tunable flat bands in the direction of transport as well as a chirality filtering mechanism for Bloch DWs. This work elucidates both the physical behavior and nanoelectronic potential for DW-MWSM devices.
%\begin{description}
%\item[Usage]
%Secondary publications and information retrieval purposes.
%\item[Structure]
%You may use the \texttt{description} environment to structure your abstract;
%use the optional argument of the \verb+\item+ command to give the category of each %item. 
%\end{description}
\end{abstract}

%\keywords{Suggested keywords}%Use showkeys class option if keyword
                              %display desired
\maketitle

%\tableofcontents




% INTRO




\section{\label{sec:intro}Introduction}

Topological materials such as topological insulators, Dirac semimetals, and Weyl semimetals have gained much interest for their litany of novel electron transport behaviors and potential utility in nanoelectronic devices~\cite{armitage_weyl_2018,burkov2018weyl}. Weyl semimetals in particular must break either spatial inversion or time-reversal symmetry (TRS). In a magnetic Weyl semimetal (MWSM), a Dirac cone is split into two Weyl cones with opposite chiralities when TRS is broken. This opens up an additional chirality degree of freedom which can significantly influence the electronic transport~\cite{hu2019transport,nagaosa2020transport}. 

%Via broken time-reversal symmetry (TRS), a Dirac cone in a  magnetic Weyl semimetal (MWSM) will be split in k-space into two Weyl cones with opposite chirality, opening up an additional valley (or chirality) degree of freedom~\cite{armitage_weyl_2018}. 

%\begin{figure}[!hb]
%\includegraphics[width=\linewidth]{figures/cat.png}%
%\caption{\label{fig:cartoonbands} a) Band structure of WSM shows chirality and blah blah blah}
%\end{figure}
\begin{figure}[]
\includegraphics[width=\linewidth]{figures/schematic_noexchange.png}%
\caption{\label{fig:DWlat} Diagram of DW lattice MWSM device. $x_0$ represents the start position of a series of chiral DWs (red and blue domains depicted). In the bottom device, few MWSM DWs are present between nonmagnetic electrodes. In the top device, $x_0$ is increased, injecting DWs into the active region, increasing resistance between the grey, nonmagnetic contacts. }

%b) and c) show exchange field textures for varying $x_0$, with 
 %N\'eel and Bloch DW textures, respectively. }
\end{figure}


In intrinsically magnetic Weyl semimetals, the helicity of Weyl fermions can be controlled by changing the magnetization of the material, giving rise to large longitudinal magnetoresistances (MRs) via the helicity mismatch of carriers in differently-magnetized portions of a sample. Previous works~\cite{deSousa2021,kobayashi2018helicity} have looked at this MR associated with transport across MWSM domain walls (DWs) and a magnetic tunnel junction (MTJ) constructed from MWSMs, both of which conclude that on/off ratios could be improved significantly ($10^5\% \text{ vs } 10^2 \%$) over the MRs utilized in traditional CoFeB/MgO MTJs. This giant Weyl MR arising from spatially-varying magnetic textures is expected to be resilient to disorder and large when compared to the DW resistances attributed to anisotropic MR or spin-mistracking in a nontopological DW~\cite{kent_DWR_2001,roxy_DWM_TR_2020}. Experimentally, an anomalous longitudinal MR of $\approx7\%$ was observed by nucleating DWs in a sample of Weyl ferromagnet Co\textsubscript{3}Sn\textsubscript{2}S\textsubscript{2}~\cite{shiogai_co3sn2s2}. Other works~\cite{araki_localized_2018, grushin_inhomogeneous_weyl_2016, kurebayashi_theory_2019} conclude that the chirality-magnetization locking will cause carriers to localize at discontinuities in the magnetization profile, which along with the recently observed topological hall torque~\cite{yamanouchi_srruo3_THT_2022, wang_co3sn2s2_THT_2022}, opens up novel and orders-of-magnitude more efficient means for electronic control of magnetic information in MWSMs. 


%Other works show the enormous and efficient topological Hall torque movement of magnetic DWs with possible observation in recent experiments \cite{yamanouchi_srruo3_THT_2022, wang_co3sn2s2_THT_2022} reaching switching current densities at zero-field as low as $5.1 \times 10^5 \frac{A}{cm^2}$ for Co\textsubscript{3}Sn\textsubscript{2}S\textsubscript{2}. 

 
%\begin{widetext}
The carrier localization and chirality-magnetization locking in MWSMs motivates interest in transport and the electronic structure of the periodic exchange superlattices, with the starting intuition that transport should be strongly forbidden in the direction of alternating magnetic texture. Previous works \cite{araki_localized_2018, grushin_inhomogeneous_weyl_2016} have shown that N\'eel DWs will localize carriers in 1D coplanar to the DW, but so far no works have investigated the connection between periodic magnetic textures and electronic bandstructure in 3D. Experimentally, such a system could potentially be realized in a MWSM helimagnet with or via injection of chiral DWs into a magnetic racetrack \cite{parkin_chiralDW_2018} with mean-free-path larger than the superlattice length. 

In this work, we outline the working principles of a multi-state MWSM DW memory device and investigate the electronic structure of MWSMs in a helical spin texture. We show that the conductance is modulated with the number of DWs between contacts in a MWSM and show the emergence of flat bands in the periodically-magnetized MWSM superlattice. Not only is this of interest for the physics of topological magnetic superlattices, but this highly-tunable readout of magnetic texture could allow for a novel and efficient mechanism to convert magnetization to charge information—a necessity of spintronic circuits, with great utility in emerging neuromorphic computing platforms \cite{leonard_synapses_2022}.


%In this work, we utilize the chirality-magnetization locking in magnetic Weyl semimetals (MWSMs) to outline the working principles of a multi-state magnetic DW memory device and investigate the electronic structure of MWSMs in a helical spin texture. We show that the conductance is modulated with the number of DWs between contacts in a MWSM and discuss the electronic structure of the periodically-magnetized MWSM superlattice.  Not only is this of interest for the physics of topological magnetic superlattices, but this highly-tunable readout of magnetic texture could allow for a novel and efficient mechanism to convert magnetization to charge information—a necessity of spintronic circuits, with great utility in emerging neuromorphic computing platforms \cite{leonard_synapses_2022}. 

%Even with additional achiral bands at the Fermi energy, non-magnetic electrodes, or a modest carrier relaxation length just greater than the width of the DW, a meager DW MR could be improved by chaining DWs along the length of the device to 1) improve the overall on/off ratio and 2) encode multiple bits of information in a single memory device. 



% PLANE-WAVE BASIS CALCULATIONS





\section{\label{sec:PW} Domain wall superlattices and plane-wave analysis}


We consider two models to elucidate the electronic structure and transport in periodically-magnetized MWSM superlattices: one using the continuum plane-wave basis model and another using the standard tight-binding approximation for transport. For both models, we consider the cases of N\'eel and Bloch magnetic DW lattices with the exchange field defined as 
\begin{align}
    \boldsymbol{\beta}_N(\boldsymbol{r}) =& \,J(\cos\theta_r\,\hat{y} + \sin\theta_r\,\hat{x}),\\
    \label{blochprofile}
    \boldsymbol{\beta}_B(\boldsymbol{r}) =& \,J(\cos\theta_r\,\hat{y} + \sin\theta_r\,\hat{z})
\end{align}
where $\theta_r$ is a spatially-dependent magnetization angle. For the infinite superlattice along $\hat{x}$, $\theta_r$ can be defined as $\theta_r = 2\pi x/a_{S} = \vec{b}_x \cdot \vec{R}$, where $a_S = n_x \, a_0$ is the superlattice length. Later, we define $\theta_r$ for a series of chiral DWs. For the continuum model, we consider the four-band model Hamiltonian of a TRS-broken MWSM~\cite{armitage_weyl_2018}:
\begin{equation}
\hat{H}(k)=v_f\tau_x \otimes (\mathbf{k}\cdot \boldsymbol{\sigma}) + m\tau_z \otimes \sigma_0 + \boldsymbol{\beta}\cdot\boldsymbol{\mathbf{\sigma}},
\end{equation}
defined in the spin ($\sigma$) and pseudospin ($\tau$) spaces. Imposing a 1D superlattice structure in $\hat{x}$ and adding the Fourier-transformed exchange field $\widetilde{\boldsymbol{\beta}}$ \footnote{See Supplemental Material at [URL will be inserted by publisher]} to the Hamiltonian off-diagonals results in

\begin{multline}
\label{eq:PWhamiltonian}
\hat{H}(k)=\sum_{G \in  \mathbb{Z} \vec{b}_x} c_{G}^{+}c_{G} (v_f\tau_x \otimes \mathbf{(k-G)}\cdot \boldsymbol{\sigma} + m\tau_z \otimes \sigma_0)  \\
+ \sum_{G', G \in \mathbb{Z}\vec{b}_x}c_{G'}^+c_{G} \tau_0 \otimes \widetilde{\boldsymbol{\beta}}_{(\mathbf{G}'-\mathbf{G})}\cdot\boldsymbol{\mathbf{\sigma}}.
\end{multline}

%\begin{widetext}
%\begin{equation}
%\label{eq:PWhamiltonian}
%\hat{H}(k)=\sum_{G \in  \mathbb{Z} \vec{b}_x} c_{G}^{+}c_{G} (v_f\tau_x \otimes \mathbf{(k-G)}\cdot \boldsymbol{\sigma} + m\tau_z \otimes \sigma_0) + \sum_{G', G \in \mathbb{Z}\vec{b}_x}c_{G'}^+c_{G} \tau_0 \otimes \widetilde{\boldsymbol{\beta}}_{(\mathbf{G}'-\mathbf{G})}\cdot\boldsymbol{\mathbf{\sigma}}.
%\end{equation}
%\end{widetext}
%with 
%\begin{eqnarray}
%\widetilde{\boldsymbol{\beta}}^N_{\mathbf{G'}-\mathbf{G}} = \frac{J}{2}[(i\hat{x} + \hat{y})\delta_{G'-G-\vec{b}_x} + (-i\hat{x} + \hat{y})\delta_{G'-G+\vec{b}_x}];  \nonumber \\ \widetilde{\boldsymbol{\beta}}^B_{\mathbf{G'}-\mathbf{G}} = \frac{J}{2}[(i\hat{z} + \hat{y})\delta_{G'-G-\vec{b}_x} + (-i\hat{z} + \hat{y})\delta_{G'-G+\vec{b}_x}]. \nonumber
%\end{eqnarray}

%\begin{equation}
%\widetilde{\boldsymbol{\beta}}_N[{\mathbf{G'}-\mathbf{G}}]=
%    \begin{cases}
%        \frac{J}{2}(i\hat{x} + \hat{y}) & \text{if } %\mathbf{G'}-\mathbf{G} = \vec{b}_x \\
%        \frac{J}{2}(-i\hat{x} + \hat{y}) & \text{if } %\mathbf{G'}-\mathbf{G} = -\vec{b}_x \\
%        0 & \text{else} 
%    \end{cases},
%\end{equation}

%\begin{equation}
%\widetilde{\boldsymbol{\beta}}_B[{\mathbf{G'}-\mathbf{G}}]=
%    \begin{cases}
%        \frac{J}{2}(i\hat{z} + \hat{y}) & \text{if } %\mathbf{G'}-\mathbf{G} = \vec{b}_x \\
%        \frac{J}{2}(-i\hat{z} + \hat{y}) & \text{if } %\mathbf{G'}-\mathbf{G} = -\vec{b}_x \\
%        0 & \text{else} 
%    \end{cases}.
%\end{equation}

We set the Fermi velocity $v_f = 1.5\times10^6 \frac{m}{s}$ \footnote{$v_f = 1.5 \times 10^6 \frac{m}{s}= \frac{a_0 t}{\hbar} = \frac{1 nm \times 1 eV}{\hbar}$. This is used for comparison with the tight-binding model. In real MWSMs, this is roughly an order of magnitude lower \cite{yamanouchi_srruo3_THT_2022}, but this does not change the underlying physics.} and focus on $m = 0$ in this paper. An exchange splitting magnitude of $J = 0.25 \; eV$ is taken to split the Weyl nodes with a small $\Delta k = 0.04 \frac{2\pi}{a_0}$ with $a_0 = 1$ nm, and for comparison with previous work~\cite{deSousa2021}. 

\begin{figure}[!ht]
\includegraphics[width=\linewidth]{figures/bands.png}%
\caption{\label{fig:bandstructures} Chirality $\langle\gamma^5\rangle$ projected plane-wave band structures for superlattice period $a_s = 100\, a_0, J = 0.25$ eV with differing exchange field textures. $\mathbf{k_0}$ denotes the position of the $|L\rangle$ Weyl point in the uniformly magnetized case. (a) Uniformly magnetized MWSM shows $|L\rangle$-chiral Weyl node with band folding. (b) N\'eel DW superlattices show band flattening and axial Landau levels in $\hat{x} \text{ and } \hat{y}$, while (c) Bloch DW superlattices only show band flattening in $\hat{x}$.}
\end{figure}

%\begin{figure}[!h]
%\includegraphics[width=\linewidth]{figures/N\'eelLLs.png}%
%\caption{\label{fig:axial LLs} Low-energy spectrum of ˆH(Γ)\hat{H}(\Gamma) for sinusoidal %N\'eel lattice shows correspondence with Dirac LL spacing E=±√2nℏe×=±√4πnℏJvf/λE = \pm \sqrt{2n\hbar e \times %\text{max}(|\mathbf{B}^5|)v^2_f} = \pm \sqrt{4\pi n \hbar J v_f/\lambda}. N\'eel lattice %ALL spacing appears in energy window below Lifshitz transition for constantly-magnetized %case.}
%\end{figure}

For a constant magnetization, the exchange field $\beta = J \hat{M}$ will split the $|L\rangle \text{ or } |R\rangle$ ($\text{i.e. }\langle \gamma^5 \rangle$ = -1,+1) Weyl points in momentum space as $\mathbf{k}_\pm = \mp \mathbf{k}_0 = \mp \Delta k \hat{M} = \mp \frac{1}{\hbar v_f}\sqrt{J^2 - m^2} \hat{M}$. Then a low-energy model for the Weyl cone dispersions can be taken as \cite{armitage_weyl_2018,araki_localized_2018,grushin_inhomogeneous_weyl_2016}


\begin{equation}
\hat{H}_{k,\pm} = \pm \hbar v_f [\mathbf{k\mathrm{+}k}_\pm] \cdot \boldsymbol{\sigma} = \pm \hbar v_f [\mathbf{k} \mp \frac{e}{\hbar} \mathbf{A}^5]\cdot \boldsymbol{\sigma},
\end{equation}
%\begin{eqnarray}
%\hat{H}_{k,\pm} = \pm \hbar v_f [\mathbf{k\mathrm{+}k}_\pm] \cdot %\boldsymbol{\sigma} \\
%
%\pm v_f [-i\hbar\mathbf{\nabla} \mp e \mathbf{A}^5]\cdot %\boldsymbol{\sigma} = \pm \hbar v_f [\mathbf{k} \mp %\frac{e}{\hbar} \mathbf{A}_5]\cdot \boldsymbol{\sigma}, \nonumber %\\
%\end{eqnarray}
where $\mathbf{A}^5$ leads to a chirality-dependent magnetic field determined by the exchange field texture:
\begin{equation}
\mathbf{B}^5 = \nabla \times \mathbf{A}^5 = \frac{J}{e v_f} \nabla \times \mathbf{M}(\mathbf{r}).
\end{equation}
%\begin{equation}
%\mathbf{B}^5_N = \frac{-2\pi J}{a_{S} e v_f} (sin(\theta_r)\hat{z})
%\end{equation}
%\begin{equation}
%\mathbf{B}^5_B = \frac{-2\pi J}{a_{S} e v_f} (cos(\theta_r)\hat{y} + %sin(\theta_r)\hat{z})
%\end{equation}
%For the two kind of domain walls, the resulting axial magnetic field is given by $\mathbf{B}^5_N(\boldsymbol{r}) =  \frac{-2\pi J}{a_{S} e v_f} (\sin\theta_r\hat{z})$ and $\mathbf{B}^5_B(\boldsymbol{r}) = \frac{-2\pi J}{a_{S} e v_f} (\cos\theta_r\hat{y} + \sin\theta_r\hat{z})$.

%\begin{align}
%    \mathbf{B}^5_N(\boldsymbol{r}) =&  \frac{-2\pi J}{a_{S} e v_f} (\sin\theta_r\hat{z}) \\
%    %\label{blochprofile}
%    \mathbf{B}^5_B(\boldsymbol{r}) =& \frac{-2\pi J}{a_{S} e v_f} %(\cos\theta_r\hat{y} + \sin\theta_r\hat{z})
%\end{align}
%which indicates that, for the N\'eel wall case, the resulting axial %gauge field always points in $z$ direction with its magnitude changing along the domain wall direction while for the Bloch wall case, it rotates in $y-z$ plane with the magnitude remaining constant. 

%The axial magnetic field generated by the spatial variation of magnetization modifies the band structure significantly. 

This generates $\mathbf{B}^5_N(\boldsymbol{r}) =  \frac{-2\pi J}{a_{S} e v_f} (\sin\theta_r\hat{z})$ and $\mathbf{B}^5_B(\boldsymbol{r}) = \frac{-2\pi J}{a_{S} e v_f} (\cos\theta_r\hat{y} + \sin\theta_r\hat{z})$ for the N\'eel and Bloch wall cases, the inclusion of which significantly modifies the band structure. In Fig~\ref{fig:bandstructures}, we show the chirality-projected bands for the uniformly magnetized, N\'eel, and Bloch helical-exchange superlattices, where the chirality operator $\gamma^5 = \tau_x\otimes\sigma_0$ and $\langle \gamma^5 \rangle = \frac{-1}{\pi}Tr(G_k^R(E)\gamma^5)$ with the time-retarded Green's function $G^R_k = (E+i\eta -H_k)^{-1}$\cite{camsari_nonequilibrium_2023}. While straightforward to project the Hamiltonian eigenstates as $\langle u_n(k)|\gamma^5|u_n(k)\rangle $, it would be misleading in the N\'eel wall case as the $|L\rangle$ and $|R\rangle$ states are chirality-polarized but degenerate in $k_z$, thus obscuring one or the other. In the Bloch wall case, the lowest-energy $|R\rangle \text{ and }|L\rangle$-chiral bands are lowered and raised in energy with $\Delta E_{\pm} = \mp (\sqrt{J^2+\hbar^2 v_f^2 \pi^2/a_S^2} - \hbar v_f \pi/ a_S) \approx \mp J$ at $\Gamma$ (see Eq.~\ref{blochgammaenergies}) for a right-handed magnetization spatial helicity (as opposed to the spin helicity $\hat{\mathbf{p}}\cdot \hat{\boldsymbol{\sigma}}$). It can also be seen that the axial magnetic fields, $\mathbf{B}^5_N$ and $ \mathbf{B}^5_B $, will localize carriers in $\hat{x}$. Notably, the N\'eel case will also localize carriers in $\hat{y}$, while the Bloch wall case remains dispersive in $\hat{y} \text{ and } \hat{z}$, as mentioned in Ref.~\cite{araki_localized_2018}. 

%We note that, at chemical potentials above the Lifshitz transition point\cite{armitage_weyl_2018} (compared to the uniformly-magnetized case) for $H_N$ at $\Gamma$, bands in $k_x$ are no longer localized without the helicity-mismatch mechanism as they move throughout the superlattice, while both the Bloch and N\'eel lattices show band flattening above the Weyl cones when swept at $k_{||} = k_0 \hat{y}$. 

Focusing on the N\'eel wall Hamiltonian $H_N$, In Fig.~\ref{fig:NeelLLs}, we sample the spectrum of $H_{N}(\Gamma)$ and, as also discussed in Ref.~\cite{grushin_inhomogeneous_weyl_2016}, generate axial Landau levels (LLs) corresponding to

\begin{figure}[]
\includegraphics[width=\linewidth]{figures/NeelLLs.png}%
\caption{\label{fig:NeelLLs} 
(a) 3D surfaces of the chirality-projected but degenerate bands are shown for a slice of $H_N(k)$, with $k_x$ fixed to 0. (b) the eigenvalue spacing is sampled at $\Gamma$, plotted with the observed correspondence to the analytical Landau level spacing for bands under $E = J$. 
}
\end{figure}

\begin{eqnarray}
\text{max}(|\textbf{B}^5|) = \frac{2\pi J}{a_S e v_f}; \\
%E_\nu = \text{sgn}(\nu) \sqrt{2 |\nu| \hbar e \; \text{max}(|\textbf{B}^5|) v_f^2} \nonumber \\
E_\nu \approx \text{sgn}(\nu) \sqrt{2|\nu| \frac{J h v_f}{a_S}}. 
\label{neelLLspacing}
% \text{sgn}(\nu) \sqrt{2|\nu| \frac{J h v_f}{a_S}}.
\end{eqnarray}
The good fitting of the low-energy spectrum to max($\mathbf{B}^5$)~$= 10.47$ T implies that the wavefunctions are strongly localized to regions of highest axial magnetic field, i.e. around $x = a_s(2\mathbb{Z}+1)/4$, or where the N\'eel magnetization is collinear with the direction of the superlattice. While an analogous axial LL physics should manifest for the Bloch wall case, the continuously changing direction of $\mathbf{B}^5$ makes it somewhat obscure: $\mathbf{B}^5$ rotates with constant magnitude in the $y-z$ plane as a function of $x$. As a result, the  zeroth Landau level disperses in different directions at different positions which can be seen in Fig.~\ref{fig:bandstructures} (c) for the $k_y$ and $k_z$ directions. Magnetization profile $\boldsymbol{\beta}_B(\mathbf{r})$ (Eq.~\ref{blochprofile}), which decides the separation of two Weyl nodes, leads to a nodal-ring structure in the spectrum in the $k_y-k_z$ plane centered at $\Gamma$. Interestingly, the dispersion for $k_{||} = \Gamma$  can be obtained analytically \footnote{See Supplemental Material at [URL will be inserted by publisher] for full derivation} and is given by:

\begin{equation} \label{blochgammaenergies}
    E_{n}= \mp \sqrt{J^2 + \hbar^2 v_f^2 \left(k_x + \frac{\pi}{a_S}(2 n+1)\right)^2} + \hbar v_f  \frac{\eta \pi}{a_S}.
\end{equation}

% which wraps around $k_{||} = \left[k_y,k_z\right]= \Gamma$, we are able to analytically solve (see \ref{blochgammaderivation}) for the level spacing at $\hat{H}_B(k_{||}=\Gamma)$ for each chirality $\eta$:
%\begin{equation} \label{blochgammaenergies}
%    E_{n}= \mp \sqrt{J^2 + \hbar^2 v_f^2 \left(k_x + \frac{\pi}{a_S}(2 n+1)\right)^2} + \hbar v_f  \frac{\eta \pi}{a_S},
%\end{equation}
%which can give a heuristic explanation for the raising and lowering of the Weyl cones in a Bloch magnetic texture via broken inversion symmetry. We can locally define an effective superlattice length 
%\begin{equation}
%\label{localaS}
%a_S(x) = 2\pi \left(\frac{\partial \theta}{\partial x}\right)^{-1}    
%\end{equation}
%which provides an approximation for the localized Bloch-texture Weyl point energy splitting, given a slowly varying magnetic texture:
%
%\begin{equation} \label{blochgammaenergies}
%    \Delta E_B= -\eta \left[\sqrt{J^2 + \frac{\hbar^2 v_f^2}{4} %\left(\frac{\partial \theta}{\partial x}\right)^2} -\frac{\hbar v_f}%{2}\frac{\partial \theta}{\partial x}\right].
%\end{equation}
%Similarly, the local Neel axial LL spacing can be expected from Eq.\ref{neelLLspacing} and Eq.\ref{localaS}:
%\begin{equation}
%    E_\nu = \text{sgn}(\nu) \sqrt{2|\nu| J \hbar v_f\frac{\partial \theta}{\partial x}},
%\end{equation}
%for energies in the region of the uniformly-magnetized Weyl cones. 

In both cases, we notice that the electronic bands are significantly flattened in the direction of superlattice. Knowing that periodic magnetic textures in MWSMs can modify dispersions in the direction of transport and provide dynamic control over the electronic structure, along with the intuition that DWs will reflect carriers via the helicity-mismatch mechanism, we move to a device picture to assess one application of MWSM DW lattices. 








\section{\label{sec:TB}Quantum transport and tight-binding analysis}

For transport and spatial resolution of Fermi surfaces, we employ the canonical four-band tight-binding model of a MWSM \cite{deSousa2021, araki2018,kobayashi_helicity-protected_2018} to approximate Eq.~\ref{eq:PWhamiltonian}, again choosing the $\tau_x$ and $\tau_z$ matrices for the momentum and mass \footnote{It is necessary to include a k-dependent mass term in order to create a single pair of Weyl cones in the tight-binding model, though $m(k) \propto (1-cos(ka))$ is near-zero in the region of the Weyl points} terms, respectively:

%\begin{eqnarray}
%\hat{H}_{\mathrm{site}\;i} = \mathop{\sum_{d\in\{+,-\},}}_{j\in\{x,y,z\}}[\frac{-itd}{2}(c_{i+d\hat{j}}^+c_i\hat{\tau}_x \hat{\sigma}_j) + \nonumber \\ 
%... \; \frac{t}{2}(c_i^+ c_i \hat{\tau}_0 -  c_{i+d\hat{j}}^+c_i\hat{\tau}_z) \hat{\sigma}_0 ] + \boldsymbol{\beta}(r_i)\cdot\boldsymbol{\mathbf{\sigma}},
%\end{eqnarray}
%\swch{why .... in this equation? Also, the notation 'd' is somewhat confusing. I made some small modifications and instead used a hermitian conjugate }
\begin{multline}
    \hat{H}_{\mathrm{site \, i}} = \sum_{j\in\{x,y,z\}} [\frac{t}{2}(c_\text{i}^+ c_\text{i} \hat{\tau}_0 -  c_{\mathrm{i}+\hat{j}}^+c_\mathrm{i}\hat{\tau}_z) \otimes \hat{\sigma}_0 -  \\ 
    \frac{it}{2}(c_{\mathrm{i}+\hat{j}}^+c_\mathrm{i}\hat{\tau}_x \otimes \hat{\sigma}_j)
    +\text{h.c} ] + \boldsymbol{\beta}(r_\mathrm{i})\cdot\boldsymbol{\mathbf{\sigma}},
\end{multline}
with hopping parameter $t = $ 1 eV, lattice parameter $a_0 = $ 1 nm for simplicity, $n_x = $100 sites in $\hat{x}$, and the Bloch phase prefactors applied to hoppings in $\hat{y}$ and $\hat{z}$ to construct a supercell $H(k_{||})$. Here, $\theta_r$ of a single $180^{\circ}$ DW is defined in \cite{ohandley_modern_1999} and convolved with a semi-infinite Dirac comb to generate the periodic magnetization angle:
%\begin{equation}
%\theta_r = \sum_{i=0}^\infty 2 \;tan^{-1}(\pi e^{(x_0 - x - i n_x a_0/)/\delta}),
%\end{equation}
\begin{equation}
\theta_r = (2\text{tan}^{-1}(\pi e^{-\frac{x+d}{d}})) \star \sum_{\mathrm{i}=0}^\infty \delta(\mathrm{i} n_x a_0-x_0 - x),
\end{equation}
%\begin{equation}
%\theta_r = \frac{\pi}{2}(\text{tanh}(-\frac{x}{d})+1) \star \sum_{n=0}^\infty \delta(x + n a_s-x_0) 
%\end{equation}
%\begin{equation}
%\theta_r = \frac{\pi e^{-x/d}}{1+e^{-x/d}} \star \sum_{i=0}^\infty \delta(i n_x a_0-x_0 - x) 
%\end{equation}
with $\star$ being the convolution operator, $x_0$ referring to the start position of the chiral DW lattice, the DW width $d = 8$ nm, and the superlattice periodicity $a_{s} = 30$ nm. While these parameters are not necessarily physical with relevance to the modelling of real magnetic racetrack systems, they are a minimal model that is computationally tractable and allows us to demonstrate underlying physics. In a real system, thicker domain walls or a longer superlattice would decrease max($\mathbf{B}^5$)~\cite{grushin_inhomogeneous_weyl_2016} thus decreasing the axial Landau Level spacing and zone-folded subband spacing. 


\begin{figure}[!h]
\includegraphics[width=\linewidth]{figures/fermisurfs.png}%
\caption{\label{fig:fermisurfaces} %\textbf{The projection operators} $\gamma^L$ (red) and $\gamma^R$ (blue) show mixed-space 
Chirality-resolved mixed-space Fermi surface with $\mu = 0.1$ eV and $\eta = 10^{-2.5}$ eV are shown with corresponding exchange field textures for $x_0$ fixed to 63 nm. (a) and (b) show the Neel DW lattice, while (c) and (d) show the helical bloch DW lattice.
}
\end{figure}

In the tight-binding model, we consider trivial metallic electrodes \footnote{We take a simple, spin and orbital-degenerate toy metal hamiltonian $\hat{H}_{\mathrm{site}\;i} = \mathop{\sum_{d\in\{+,-\},j\in\{x,y,z\}}}{}\ [t c_{i+d\hat{j}}^+c_i \tau_0\otimes\sigma_0] + (3 t - \varepsilon_0) [c_i^+c_i\tau_0\otimes\sigma_0] $ with an arbitrary $t = 1$ eV and $\varepsilon_0 = 1 eV$} and semi-infinite MWSM electrodes to elucidate the underlying physics~\footnote{The semi-infinite MWSM electrodes are useful in the large scattering limit, where an electron will locally see an infinite lattice in all directions. They are also most useful for resolving the mixed-space Fermi-surfaces, where surface states will confound the fermi surface braiding}. For the case with MWSM electrodes, the magnetizations $\boldsymbol{\beta}(x=-a_0) \text{ and } \boldsymbol{\beta}(x=a_s+a_0)$ are copied for the left and right contact and extend to infinity for the generation of the Sancho-Rubio \cite{sancho_highly_1985} contact self-energies $\Sigma_{L,R}$ to retain the continuity of the magnetization field texture. 



%\begin{widetext}
    
\begin{figure*}[!t]
\includegraphics[width=\linewidth]{figures/transmissions.png}%
\caption{\label{fig:tmaps}  Device conductance is modulated by injecting DWs between contacts via sweeping $x_0$. (a) Conductance modulation by orders of magnitude is shown for all combinations of MWSM and simple metallic electrodes with Bloch and N\'eel DW lattices. (b) A phenomenological broadening term $\eta$ is swept, showing discrete stairsteps in conductance, thus multi-weight behavior, as $\eta$ increases (i.e. carrier lifetime decreases). (c) and (d) show k-resolved transmission maps for the N\'eel and Bloch DW cases, respectively, sweeping $x_0$, over center of surface Brillouin zone. Hotspots in transmission $T(k_{||})$ are visible where the bulk states from the left and right electrodes overlap, connected by twisted Fermi arcs of character determined by the magnetic texture. $x_0$ position is labeled in white for each sub-plot.}
\end{figure*}
%\end{widetext}

 Of particular interest is the braiding of the Fermi surfaces in connection with the band structures of Fig.~\ref{fig:bandstructures}, which are useful to understand transport. Fig.~\ref{fig:fermisurfaces} shows the mixed-space Fermi surface of the periodically magnetized racetrack devices for energies close to the Weyl node. N\'eel devices show a $k_y$-symmetric braiding of the L and R chiral states, corresponding to the degenerate states in Fig.~\ref{fig:fermisurfaces} (b). This can be understood from the profile of $\boldsymbol{\beta}_N$ for the N\'eel case which decides the position of two Weyl nodes in $k_y-k_z$ plane. Additionally, this magnetization pattern generates $\mathbf{B}_N^5$ pointing along $z$ which flattens the bands in $k_y$. It should be noted here that these Fermi surfaces can also be interpreted as zeroth LL~\cite{grushin_inhomogeneous_weyl_2016}. On the other hand, for Bloch DW case, $\mathbf{B}_B^5$ rotates in $y-z$ leading to these twisted Fermi-surfaces which can also be understood as the Fermi contour of the zeroth LL shown in Fig.~\ref{fig:bandstructures} (c) where for an infinitesimal non-zero energy, $|L\rangle$ and $|R\rangle$ states are at a different radius in the $k_y-k_z$ plane. As a result, the $|L\rangle$ chiral states wrap around the expected position of the twisted Weyl cones, while the $|R\rangle$ chiral states wrap closely to $k_{||} = \Gamma$. This twisted Fermi-surface motivates analysis of the chiral anomaly and Weyl orbits with periodic real and axial magnetic fields~\cite{Peri2023}.

For device modeling, we neglect the orbital effects of the magnetic field and consider transport in the Landauer limit of the NEGF quantum transport formalism \footnote{See \cite{camsari_nonequilibrium_2023}; $G^R_{k_{||}} = (E+i\eta - H_{k_{||}} - \Sigma_{R,k_{||}} - \Sigma_{L,k_{||}})^{-1}; \Gamma_i = i(\Sigma_i - \Sigma_i^\dagger); T = tr(\Gamma_L G^R \Gamma_R G^A); G^A = (G^R)^\dagger$}.
With a minimal carrier lifetime broadening of $\eta = 10^{-4}$ eV, in Fig.~\ref{fig:tmaps} (a) we show that device conductance can be modulated by orders of magnitude via the injection of DWs into the active region of the device, and that this effect persists for both magnetization patterns, with both MWSM and simple metal electrodes. In Fig.~\ref{fig:tmaps} (b), we sweep the broadening parameter $\eta$\iffalse$ = \frac{\hbar}{2\tau}$\fi, corresponding to a phenomenological inelastic momentum-preserving scattering self-energy \cite{gilbert_TI_interconnects_2017}, and show that it 1) smooths out conductivity vs. $x_0$ and 2) forms stairsteps in conductance with to the number of DWs in the active region of the device--as one might expect with an increasing number of helicity-mismatch carrier reflections. This form of nonvolatile, texture-dependent conductivity is desirable for use in emerging analog memory devices~\cite{leonard_synapses_2022}. 

Curiously, even a high-resolution 401 x 401 k-grid over the zoomed-in portion of the surface Brillouin zone (SBZ) is unable to converge the value of the Landauer conductance for small broadening parameters ($\eta = 10^{-3}, 10^{-4}$ eV), leading to unphysical spikes in conduction. To explain this, we consider the $k_{||}$-resolved transmission in Fig.~\ref{fig:tmaps} (c) and Fig.~\ref{fig:tmaps} (d) to resolve conductance behavior from the distorted Weyl cones in the periodic magnetization region. We see that the distorted Weyl cones form infinitesimally-thick curtains of conductance in the SBZ which are challenging to capture numerically: In Fig.~\ref{fig:tmaps} (c), the N\'eel DW lattice conduction is dominated by the distorted $|L\rangle$ and $|R\rangle$ Weyl cones in the periodic region of the device, especially where they overlap with the contact electrodes' bulk Weyl cones. In contrast, the Bloch DW device in Fig.~\ref{fig:tmaps} (d) has minimal conduction through the distorted Weyl cones which wrap around the bulk Weyl cones in either contact. Thus, the majority of the conductance tunnels through the periodic region of the device from the Bulk states in both contacts, leading to oscillations in Fig.~\ref{fig:tmaps} (a). Interestingly, due to the broadening parameter providing finite-lifetime states from the contacts, conductance from the contact Weyl cone overlap decreases and the inner $|R\rangle$-chiral transmission ring will begin to plateau in the total conductance, giving rise to a texture-dependent chirality filtering mechanism to generate or reflect chiral currents with potential application in emerging devices~\cite{kharzeev_chiralanomaly_2013}. 



% DISCUSSION



\section{\label{sec:discussion} Discussion}

We have studied the electronic structure and transport in a MWSM superlattice device, have shown magnetization-tunable flat bands in the direction of transport, and have demonstrated that multiple conductance states could potentially be encoded in such a device. We have also shown that the multi-state behavior persists through all electrode and magnetic superlattice configurations we have considered, quite unlike the spin transport in traditional spin filtering systems. With regards to the superlattice picture, more analysis needs to be done to fully understand the connection between exchange textures and topological or electronic structure, the exploration of which is only nascent in systems such as semiconductor nanowires \cite{kammhuber_helical_2017} or magnetic superlattice graphene \cite{wolf_periodicgraphene_2021, paul_skyrmionlattice_2023} under skyrmion exchange superlattices which show topological flat bands. It should be noted that this effect only relies on the mean-field exchange splitting of a periodic ferromagnet, acting on spin space with contributions on the order of 100 to 1000 meV in bulk ferromagnets, as opposed to a real magnetic field or proximity-induced exchange, the effects of which would show with much weaker Zeeman perturbations or contribution to an effective gauge field. Thus, we expect the electronic structure in periodic MWSM lattices to be more robust closer to room temperatures, supposing a MWSM with Curie temperature above 300K as in Mn\textsubscript{3}Sn\cite{muduli_mn3sn_2019}, Co\textsubscript{2}MnGa\cite{saito_co2mnga_2021}, or Fe\textsubscript{3}Sn\textsubscript{2}\cite{yao_fe3sn2_2018}. Nevertheless, other perspectives on scattering at MWSM DWs imply that skew scattering \cite{sorn_skewscattering_2021} or heavily-tilted Weyl cones \cite{xuan_tiltedMR_2021} could significantly decrease the MR. 
%In the future, this could be explored within the NEGF formalism with an incoherent momentum-relaxing transport model\cite{vakili_NLspinvoltage_WSM_2022} or the addition of non-topological bands near the fermi energy, as in a realistic system.

This highly-tunable electronic structure and conductivity via control of magnetization in MWSMs would be of great use to nanoelectronics, which we will discuss briefly. While it is hard to make definitive claims from our model without incoherent momentum relaxation -- we see longitudinal Weyl MRs over $10^{14} \%$ -- we estimate that, using the DW resistances observed in \cite{shiogai_co3sn2s2} an effective MR $= \frac{R_{\text{off}}-R_{\text{on}}}{R_{\text{off}}}=\frac{n_{DW}R_{DW}}{R_0} = (\frac{L}{d}+1)\frac{R_{DW}}{R_{0}}=  32.1\%$ could be achievable for the observed $2 R_{DW}\approx 4 \,\Omega$ and $R_{0}\approx 53 \,\Omega$, for a domain width $d = 80$ nm and device length $L = 600$ nm. This is 14x the 3.93\% read margin observed in CoFeB transverse-read DW memory \cite{roxy_DWM_TR_2020}. In itself, this becomes competitive with other GMR devices (with regards to on/off ratio) or multi-state TMR devices \cite{leonard_synapses_2022}, but without the need of a magnetic tunnel junction to measure the magnetization-induced resistance change. Supposing a deliberate engineering of the chemical potential or DW width could bring about a modest $\text{MR}_{DW}=25\%$ with $L = 1 \mu$m, and $d = 40$ nm, on/off ratios upwards of 650\% become possible.  Another potential advantage of this approach, compared to traditional DW-MTJ devices, is that one could avoid the need for precise MgO deposition to avoid pinholes, thus making high MRs accessible to industry and academic labs without expensive, lengthy fabrication processes. Here, one could construct memory devices using simple metallic thin films, so long as the chirality-magnetization locking in the MWSM is preserved and surface fermi arcs do not dominate the conductance. Predicted orders-of-magnitude-improved readout performance over existing MTJs~\cite{deSousa2021} could further increase device viability, supposing one could amplify the Weyl MR or suppress scattering in a real system. 

\section{Acknowledgements}
The authors acknowledge funding from the UT CDCM MRSEC supported under NSF Award Number DMR-1720595, funding from Sandia National Laboratories, and computational resources from the Texas Advanced Computing Center (TACC) at the University of Texas at Austin. The authors also thank Gregory A. Fiete for helpful discussions, as well as Kerem Camsari and Shuvro Chowdhury for their discussions on the theory and implementation of the NEGF transport formalism.



%\textbf{References:}\\
%magnetization switching, DW nucleation -\cite{Buccheri2022}\\
%charge pumping induced by magnetic texture - \cite{araki2018}\\
%Anomalous conductance scaling in Weyl semimetal NbAs-\cite{kumar2022anomalous}\\
%tilted WSM junction -\cite{xuan2021magnetotransport}
\bibliographystyle{apsrev4-1}
\bibliography{ref.bib}
\appendix
\documentclass[./main.tex]{subfiles}
\begin{document}

\title{Supplemental Material\\From Clean Room to Machine Room: Commissioning of the First-Generation BrainScaleS Wafer-Scale Neuromorphic System}

\DeclareRobustCommand{\enumauthorrefmark}[1]{\smash{\textsuperscript{\footnotesize #1}}}

\newcommand{\contributedSymbol}{\IEEEauthorrefmark{1}}
\newcommand{\uheiSymbol}{\enumauthorrefmark{1}}
\newcommand{\ugoeSymbol}{\enumauthorrefmark{2}}


\author{
	\IEEEauthorblockN{%
		Hartmut Schmidt\contributedSymbol,
		José Montes\contributedSymbol,
		Andreas Grübl,
		Maurice Güttler,
		Dan Husmann,
		Joscha Ilmberger,\\
		Jakob Kaiser,
		Christian Mauch,
		Eric Müller,
		Lars Sterzenbach,
		Johannes Schemmel,
		Sebastian Schmitt\\
	}

	\thanks{
		\IEEEauthorblockA{%
		\contributedSymbol%
		Contributed equally\\
		}
	}
}

\maketitle
Next, we present the Supplementary Materials for the paper ``Re-ReND: Real-time Rendering of NeRFs across Devices''.
Specifically, in addition to the results reported in the paper, we report results of \methodname w.r.t. Image Quality~(Section~\ref{sec:im_qual}) and (Section~\ref{sec:quali}), Rendering Speed~(Section~\ref{sec:fps}), Mesh Size~(Section~\ref{sec:mesh_size} and Section~\ref{sec:meshi}), Disk Space~(Section~\ref{sec:disk_space}), validation of view-dependent effects (Section~\ref{sec:val}),  sensitivity to geometry variations (Section~\ref{sec:geo}) and Photo-metric quality w.r.t. embedding dimensionality $D$ (Section~\ref{sec:dim}).
Furthermore, we encourage the reviewers to watch the \textbf{associated video}, \texttt{Re-ReND.mp4}, demonstrating \methodname's capabilities of real-time rendering across devices.
% In particular, please refer to .
This video demonstrates how \methodname can render, in real time, a scene composed of tens (\Figure{composit}) or even thousands (\Figure{many_objects}) of objects. % , respectively. %  , or even with thousands of . %  in an AR headset.
\Figure{composit} illustrates such a scene, composed of moving chairs, hotdogs, the drumset, and a microphone.


% Finally, we also provide the PyTorch~\cite{NEURIPS2019_9015} and GLSL implementations of our method inside the folders called \texttt{Re-ReND\_Pytorch\_code} and \texttt{Re-ReND\_GLSL\_code}.

% \thispagestyle{empty}
% \appendix

%%%%%%%%% BODY TEXT - ENTER YOUR RESPONSE BELOW
% \section{The PyTorch code and GLSL code}

%  \begin{itemize}
%     \item Clean and README.md
%     \item Should I upload only pur method or MipNeRF and NeRF++?
%     \item Should I upload the generated data and the meshes in a google drive? What happens with anonymity?
% \end{itemize}

% \section{A video showing how we were measuring the FPS}
% \section{A video showing real scenes in comparison with MobileNeRF and SNeRG}
% \section{Qualitative Results}

%  \begin{itemize}
%     \item all objects visualizations 
% \end{itemize}

%-------------------------------------------------------------------------


\begin{figure}
    \centering
    \includegraphics[width=\linewidth]{pics/quantitative.pdf}
    \caption{Box plots of quantitative benchmarks MIG, FactorVAE, Disentanglement, and reconstruction error on dSprites and Shapes3D.}\label{fig:quantitative}
\end{figure}


\bibliographystyle{style/IEEEtran}
\bibliography{bib/vision}

\end{document}

\end{document}
%
% ****** End of file apssamp.tex ******
