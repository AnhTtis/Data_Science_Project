\onecolumngrid
\appendix
\section{Supplementary information}
%\subsection{Terms and definitions}
%\label{definitions}

\subsection{Bloch wall hamiltonian spectrum with k||k_{||} = 0}
\label{suppblochgammaderivation}
First, one can choose the equivalent block-diagonal $\tau_z$ matrix for the momentum term in Eq.\ref{eq:PWhamiltonian} and set m = 0. This decouples the left (-) and right (+) helical states $\eta$, and allows one to express the Hamiltonian of the infinitely-periodic hamiltonian with helical exchange field as follows. Let $\Omega = \frac{J}{2\hbar v_f}$, and let $G = \frac{2\pi}{a_S} = \vec{b}_x$ be the superlattice reciprocal vector. We use the definition for $\boldsymbol{\beta}(r)$ in section II: 
\begin{eqnarray}
    \label{FT-exchange}
    \boldsymbol{\beta}_N(r) = J(\cos(\theta_r)\hat{y} + \sin(\theta_r)\hat{x}) = \frac{J}{2}(e^{i G x}\hat{y}+e^{-i G x}\hat{y} + -ie^{i G x}\hat{x}+ie^{-i G x}\hat{x}), \\\boldsymbol{\beta}_B(r) = J(\cos(\theta_r)\hat{y} + \sin(\theta_r)\hat{z}) = \frac{J}{2}(e^{i G x}\hat{y}+e^{-i G x}\hat{y} + -ie^{i G x}\hat{z}+ie^{-i G x}\hat{z}).
\end{eqnarray}

In Fourier space, this can be written as:
\begin{eqnarray}
    \widetilde{\boldsymbol{\beta}}^N_{G'-G} = \frac{J}{2}[(i\hat{x} + \hat{y})\delta(G'-G-\vec{b}_x) + (-i\hat{x} + \hat{y})\delta(G'-G+\vec{b}_x)]; \\
    \widetilde{\boldsymbol{\beta}}^B_{G'-G} = \frac{J}{2}[(i\hat{z} + \hat{y})\delta(G'-G-\vec{b}_x) + (-i\hat{z} + \hat{y})\delta(G'-G+\vec{b}_x)].
\end{eqnarray}

We will now focus on the Bloch wall case. 
\begin{equation}
    \hat{H}_B(k) = \hbar v_f\begin{bmatrix}
    \ddots & \Omega (\sigma_y + i \sigma_z) &    &      &    \\
     \Omega (\sigma_y - i \sigma_z) & \eta(k_x + G)\sigma_x       & \Omega (\sigma_y + i \sigma_z)   &     &    \\
      &         \Omega (\sigma_y - i \sigma_z)& \eta (k_x)\sigma_x         & \Omega (\sigma_y + i \sigma_z)    &     \\
      &         &           \Omega (\sigma_y - i \sigma_z)&     \eta (k_x - G)\sigma_x      & \Omega (\sigma_y + i \sigma_z)   \\
      &         &           &           \Omega (\sigma_y - i \sigma_z)& \ddots
  \end{bmatrix}
\end{equation}

Then, the basis transformation corresponding to a $120^{\circ}$ rotation about $-\hat{x} +\hat{y} + \hat{z}$ can be used to transform the spin matrices:
\begin{eqnarray}
\sigma_y \rightarrow -\sigma_x  \nonumber \\
\sigma_z \rightarrow \sigma_y \nonumber \\
\sigma_x \rightarrow -\sigma_z \nonumber
\end{eqnarray}
and, knowing that $\sigma_+ = \sigma_x + i \sigma_y; \sigma_- = \sigma_x - i \sigma_y$ (we use the pauli matrices without prefactor 1/2), the Hamiltonian can be rewritten as:
\begin{equation}
    \hat{H}_B(k) = -\hbar v_f\begin{bmatrix}
    \ddots & \Omega (\sigma_-) &    &      &    \\
     \Omega (\sigma_+) & \eta(k_x + G)\sigma_z       & \Omega (\sigma_-)   &     &    \\
      &         \Omega (\sigma_+)& \eta (k_x)\hat{\sigma}_z         & \Omega (\sigma_-)    &     \\
      &         &           \Omega (\sigma_+)&     \eta (k_x - G)\hat{\sigma}_z      & \Omega (\sigma_-)   \\
      &         &           &           \Omega (\sigma_+)& \ddots
  \end{bmatrix}
\end{equation}
\begin{equation}
  = -\hbar v_f \bigoplus_n \left(\begin{bmatrix}
  -\eta(k_x + (n+1)G) & 2\Omega \\
  2\Omega & \eta(k_x + n G)
  \end{bmatrix}\right) = -\hbar v_f \bigoplus_n \left(-\eta k_x \sigma_z + 2 \Omega \sigma_x - \eta n G \sigma_z -\eta G\frac{1}{2}(\sigma_z+\sigma_0)\right)
\end{equation}

\begin{equation}
= -\hbar v_f \bigoplus_n (\icol{2\Omega \\ 0 \\ -\eta(k_x +  G (n + \frac{1}{2}))}\cdot\boldsymbol{\sigma} - \eta G \frac{1}{2}\sigma_0) = \bigoplus_n \hat{H}_n
\end{equation}
with the spectrum of $\hat{H}_n$ (thus, a diagonal subblock of $\hat{H}_B(k_{||}=\Gamma)$) given as:

\begin{equation}
    E_n = -\hbar v_f \left( \pm\sqrt{4\Omega^2 + (k_x + G(n+\frac{1}{2}))^2} - \eta G / 2\right) = \mp \sqrt{J^2 + \hbar^2 v_f^2 (k_x + \frac{\pi}{a_S}(2 n+1))^2} + \hbar v_f  \frac{\eta \pi}{a_S}
\end{equation}
