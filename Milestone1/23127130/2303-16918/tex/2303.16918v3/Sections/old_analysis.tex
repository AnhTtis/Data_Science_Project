For the sinusoidal $\boldsymbol{\beta}_{N,R}$, at a given $|i\rangle = |\mathbf{k-G}_i\rangle = |\mathbf{k} - i\mathbf{b_x}\rangle$, with the mass term set to 0, a column of the Hamiltonian can be rewritten as 


\begin{eqnarray}
    \,  \hat{H}_{i} = |i\rangle \langle i|\otimes v_f \eta \; (\mathbf{k}-i\mathbf{b}_x)\cdot \boldsymbol{\sigma}  \nonumber \\ 
    + |i+1\rangle \langle i| \otimes \widetilde{\boldsymbol{\beta}}^*_1 \cdot \boldsymbol{\sigma} + |i-1\rangle \langle i| \otimes \widetilde{\boldsymbol{\beta}}_1 \cdot \boldsymbol{\sigma}
\end{eqnarray}

For an infinite-dimensional G grid, an analogy can be drawn to the finite-difference expression for $\partial^2_x$ taken along the higher brillouin zones:

\begin{eqnarray}
    \frac{\partial^2}{\partial x^2}= \sum_i\frac{(|i+1\rangle-2|i\rangle+|i-1\rangle)\langle i|}{2\delta x}; \\
    \frac{\delta^2}{\delta G^2}= \sum_i\frac{(|i+1\rangle-2|i\rangle+|i-1\rangle)\langle i|}{2|b_x|}; \\
    \hat{H}(k) =  [2|b_x|(\frac{\delta^2}{\delta \mathbf{G}^2} + 2\mathbb{I}) \otimes \widetilde{\boldsymbol{\beta}}_1  + v_f \eta (\mathbf{k} - \hat{\mathbf{G}})] \cdot \boldsymbol{\sigma}
\end{eqnarray}
