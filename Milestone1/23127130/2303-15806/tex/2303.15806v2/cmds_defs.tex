

% Vectors, Matrices:
\newcommand{\bma}{\begin{bmatrix}}
\newcommand{\ema}{\end{bmatrix}}

\newcommand{\vect}[1]{\mathbf{#1}} % vector
\newcommand{\vectg}[1]{\boldsymbol{#1}} % italic vector (works also for greek)
\newcommand{\refl}[1]{\accentset{\leftarrow}{#1}} % reflection of function
\newcommand{\mat}[1]{\mathsf{#1}} % matrices (capital greek symbols work too)
\newcommand{\boldmat}[1]{\textnormal{\usefont{T1}{ba9}{r}{r}\selectfont #1}} % Humanist970BT-RomanC
\newcommand{\vmat}[1]{\mathbb{#1}} % random matrices
\newcommand{\inv}[1]{#1^{-1}} % inverse  
\newcommand{\trans}[1]{#1^{\textnormal{\textsf{\tiny T}}}} % transpose  
\newcommand{\transsmall}[1]{#1^{\textnormal{\textsf{\fontsize{1pt}{1pt}\selectfont T}}}} % tranpose small  
\newcommand{\invtrans}[1]{#1^{-\textnormal{\textsf{\tiny T}}}} % transpose of inverse
\newcommand{\invtranssmall}[1]{#1^{\scriptscriptstyle -\textnormal{\textsf{\fontsize{1pt}{1pt}\selectfont T}}}} % tranpose small  
\newcommand{\tpt}{\textnormal{\textsf{\tiny T}}} % transpose-t
\newcommand{\conj}[1]{#1^{\textnormal{\textsf{\footnotesize\raisebox{-0.5ex}{*}}}}} % conjugate 
\newcommand{\conjstar}{\textnormal{\textsf{\footnotesize\raisebox{-0.5ex}{*}}}} % star of conjugate 
\newcommand{\hermi}[1]{#1^{\dagger}} % transpose and conjugate
\newcommand{\invhermi}[1]{#1^{-\dagger}} % transpose and conjugate of inverse
\newcommand{\trace}[1]{\operatorname{tr}\left(#1\right)} % trace
\newcommand{\diag}[1]{\operatorname{diag}\left(#1\right)} % diagonal matrix
\newcommand{\rank}[1]{\operatorname{rank}\left(#1\right)} % rank operator
\newcommand{\T}{{\mathsf{T}}} % transpose

% Fields and Spaces
\newcommand{\Reals}{\mathbb{R}}      % real numbers
\newcommand{\Integers}{\mathbb{Z}}   % integers
\newcommand{\Rationals}{\mathbb{Q}}  % rational numbers
\newcommand{\Naturals}{\mathbb{N}}   % natural numbers 1,2,3,...
\newcommand{\Complex}{\mathbb{C}}    % complex numbers
\newcommand{\Field}{\mathbb{F}}      % field
\newcommand{\GF}{\textnormal{GF}}            % Galois-field


% Real and Imaginary Parts of a Complex Number
\renewcommand{\Re}[1]{\mathop{}\!\textnormal{Re}\left\{#1\right\}}
\renewcommand{\Im}[1]{\mathop{}\!\textnormal{Im}\left\{#1\right\}}


% Sets, cardinalities
\newcommand{\set}[1]{\mathcal{#1}} %set
\newcommand{\cset}[1]{\mathcal{#1}^{\textnormal{c}}} %complement of set
\newcommand{\comp}[1]{\left(#1\right)^{\textnormal{c}}} %complement of set
\newcommand{\card}[1]{\left|#1\right|} %cardinality of a set.


% Communication Constants 
\newcommand{\const}[1]{\textnormal{\usefont{U}{eur}{m}{n}\selectfont #1}} % new version Euler
\newcommand{\constb}[1]{\textnormal{\usefont{U}{eur}{b}{n}\selectfont #1}} % bold constant Euler
\newcommand{\Prs}[1]{\operatorname{\textnormal{Pr}}\left(#1\right)}
\newcommand{\Prv}[1]{\operatorname{\textnormal{Pr}}\left[#1\right]}
% \newcommand{\E}[2][]{\mathop{}\!\textnormal{\textsf{E}}_{#1}\left[#2\right]}
\newcommand{\Econd}[3][]{\mathop{}\!\textnormal{\textsf{E}}_{#1}\left[#2\middle|#3\right]}
\newcommand{\Normal}[1]{\mathcal{N}\!\left({#1}\right)} %Gaussian dist.


% Statistical Operators 
\newcommand{\eqlaw}{\stackrel{\mathscr{\scriptscriptstyle L}}{=}} %equivalence in prob. law
\newcommand{\indep}{\mathrel\bot\joinrel\mspace{-8mu}\mathrel\bot}  %independent
\newcommand{\dep}{\centernot\indep} %dependent 
\newcommand{\markov}{\mathrel\multimap\joinrel\mathrel-\mspace{-9mu}\joinrel\mathrel-}

% Various maths:
\newcommand{\dd}{\mathop{}\!\mathrm{d}} %d in integrals and differentiation
\newcommand{\intinf}{\int_{-\infty}^{\infty}}
\newcommand{\ii}{\mathsf{i}} %square root of -1.
\newcommand{\eqdef}{\triangleq} % definition
\newcommand{\Four}[1]{\hat{#1}} %The Fourier trans. of the argument
\newcommand{\InvFour}[1]{\check{#1}} %The Fourier trans. of the argument
\newcommand{\conv}[2]{#1 \star #2}
\newcommand{\argmin}{\operatorname*{argmin}}
\newcommand{\argmax}{\operatorname*{argmax}}
\newcommand{\eps}{\epsilon} 
\newcommand{\veps}{\varepsilon}
\newcommand{\Exp}[1]{\exp \! \left(\! {#1} \! \right)} 

%% Messages (Message Passing)
\makeatletter
\DeclareFontFamily{U}{MnSymbolA}{}
\DeclareSymbolFont{MnSyA}{U}{MnSymbolA}{m}{n}
\DeclareFontShape{U}{MnSymbolA}{m}{n}{
<-6> MnSymbolA5
<6-7> MnSymbolA6
<7-8> MnSymbolA7
<8-9> MnSymbolA8
<9-10> MnSymbolA9
<10-12> MnSymbolA10
<12-> MnSymbolA12}{}
\DeclareMathSymbol{\smallrightarrow}{\mathrel}{MnSyA}{0}
\DeclareMathSymbol{\smallleftarrow}{\mathrel}{MnSyA}{2}
\DeclareMathSymbol{\smallleftrightarrow}{\mathrel}{MnSyA}{16}
\newcommand{\smallrightarrowfill@}{\arrowfill@\relbar\relbar\smallrightarrow}
\newcommand{\smallleftarrowfill@}{\arrowfill@\smallleftarrow\relbar\relbar}
\newcommand{\smallleftrightarrowfill@}
{\arrowfill@\smallleftarrow\relbar\smallrightarrow}
\renewcommand{\overrightarrow}{\mathpalette{\overarrow@\smallrightarrowfill@}}
\renewcommand{\overleftarrow}{\mathpalette{\overarrow@\smallleftarrowfill@}}
\renewcommand{\overleftrightarrow}
{\mathpalette{\overarrow@\smallleftrightarrowfill@}}
\makeatother
%%
\providecommand{\msgf}[2]{\protect\overrightarrow{#1}_{\mspace{-3mu}#2}} % Forward Message
\providecommand{\msgb}[2]{\protect\overleftarrow{#1}_{\mspace{-3mu}#2}} % Backward Message
\providecommand{\msgfb}[2]{\protect\overleftrightarrow{#1}_{\mspace{-3mu}#2}}

\newcommand{\iid}{\overset{\scriptscriptstyle\mathsf{iid}}{\sim}}

\DeclareMathOperator{\E}{\textnormal{\ensuremath{\mathbb{E}}}}
\newcommand{\EE}[1]{\E\!\left[{#1}\right]}
% \newcommand{\Ee}[1]{\E[{#1}]}

\newcommand{\Var}[1]{\operatorname{Var}\left[#1\right]} % trace

\newcommand{\Ident}{\mathbbm{1}}
\newcommand{\cond}{\hspace{0.02em}|\hspace{0.08em}}


% environments
\newcounter{examplecntr}
\newenvironment{example}[1][]%
{\begin{trivlist}\small\item[]\refstepcounter{examplecntr}%
 {\bfseries Example~\theexamplecntr%
  \ifthenelse{\equal{#1}{}}{}{ (#1)}.
}}%
{\end{trivlist}}

\newcounter{definitioncntr}
\newenvironment{definition}%
{\begin{trivlist}\item[]\refstepcounter{definitioncntr}%
{\bfseries Definition~\thedefinitioncntr.}}%
{\hfill$\Box$\end{trivlist}}

\newcounter{theoremcntr}
\newenvironment{theorem}[1][]%
{\begin{trivlist}\item[]\refstepcounter{theoremcntr}%
{\bfseries Theorem~\thetheoremcntr%
  \ifthenelse{\equal{#1}{}}{}{ (#1)}.
}}%
{\hfill$\Box$\end{trivlist}}

\newcounter{propositioncntr}
\newenvironment{proposition}[1][]%
{\begin{trivlist}\item[]\refstepcounter{propositioncntr}%
{\bfseries Proposition~\thepropositioncntr%
  \ifthenelse{\equal{#1}{}}{}{ (#1)}.
}}%
{\hfill$\Box$\end{trivlist}}
%{\hfill$\blacksquare$\end{trivlist}}

\newcounter{lemmacntr}
\newenvironment{lemma}[1][]%
{\begin{trivlist}\item[]\refstepcounter{lemmacntr}%
{\bfseries Lemma~\thelemmacntr%
  \ifthenelse{\equal{#1}{}}{}{ (#1)}.
}}%
{\hfill$\Box$\end{trivlist}}


%\newenvironment{proof}{\begin{trivlist}\item[]{\bfseries Proof: }
% }{\hfill$\Box$\end{trivlist}}
 
\newenvironment{proofof}[1]{\begin{trivlist}\item[]{\bfseries Proof\ifthenelse{\equal{#1}{}}{}{ #1}:}
% }{\hfill$\Box$\end{trivlist}}
 }{\hfill$\blacksquare$\end{trivlist}}

\newcommand{\eproofnegspace}{\\[-1.5\baselineskip]\rule{0em}{0ex}}

% ************************************************************************



% misc, project specific
\newcommand{\restrict}[2]{\left.#1\right|_{#2}}
\newcommand{\sgn}{\operatorname{sgn}}

\newcommand{\imk}{{1 - \kappa(s) z}}
%remark while editing
\newcommand{\rem}[1]{{\color{red} {#1}}}

\def\maxs{\max_{\sigma^2}}
\def\mins{\min_{\sigma^2}}

\def\limb{\lim_{b \rightarrow \infty}}
\def\binf{{b \rightarrow \infty}}

\newcommand{\va}{{\sigma_a^2}}
\newcommand{\vb}{{\sigma_b^2}}
\newcommand{\hva}{{{\hat \sigma_a}^2}}
\newcommand{\hvb}{{{\hat \sigma_b}^2}}
\newcommand{\sda}{{\sigma_a}}
\newcommand{\sdb}{{\sigma_b}}

\newcommand{\vak}{{\sigma_{a,k}^2}}
\newcommand{\vbk}{{\sigma_{b,k}^2}}
\newcommand{\hvak}{{\hat \sigma_{a,k}^2}}
\newcommand{\hvbk}{{\hat \sigma_{b,k}^2}}

\def\saa{\sigma_{1,a}^2}
\def\sab{\sigma_{1,b}^2}
\def\sba{\sigma_{2,a}^2}
\def\sbb{\sigma_{2,b}^2}
\def\hsaa{\hat{\sigma}_{1,a}^2}
\def\hsab{\hat{\sigma}_{1,b}^2}
\def\hsba{\hat{\sigma}_{2,a}^2}
\def\hsbb{\hat{\sigma}_{2,b}^2}
\def\mta{m_1}
\def\mtb{m_2}

\newcommand{\muth}{{\mu_\theta}}
\newcommand{\mth}{{m_\theta}}
\newcommand{\mtha}{{m_{\theta_1}}}
\newcommand{\mthb}{{m_{\theta_2}}}
\def\mtht{{\tilde{m}_{\hat \theta}}}
\def\vtht{{\tilde{\sigma}_{\hat \theta}^2}}

\newcommand{\vth}{{\sigma_\theta^2}}
\newcommand{\vtha}{{\sigma_{\theta_1}^2}}
\newcommand{\vthb}{{\sigma_{\theta_2}^2}}


\newcommand{\markblue}[1]{{\color{blue}#1}}
\newcommand{\markred}[1]{{\color{red}#1}}