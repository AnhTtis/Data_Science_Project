

We formulate the modified atomic norms from the signal model with antenna errors as

\begin{align}
    ||\widetilde{\mathbf{X}}_r||_{\widetilde{\mathcal{A}}_r} = \inf_{\stackrel{[\bsym{\alpha}_r]_\ell \in \mathbb{C}, \boldsymbol{r}_\ell \in [0,1]^3}{||\mathbf{u}||_2 = 1,\Vert\widetilde{\mathbf{E}}\Vert\leq \varepsilon_e}} \Bigg\{\sum_\ell \vert[\bsym{\alpha}_r]_\ell\vert \Big| \widetilde{\mathbf{X}}_r = &\sum_\ell [\bsym{\alpha}_r]_\ell\mathbf{u}\mathbf{w}(\mathbf{r}_\ell)^H\times\nonumber\\&(\widetilde{\mathbf{E}} + \mathbf{I}_{MPN_r})\Bigg\}\nonumber
    \end{align}
    \begin{align}
   ||\widetilde{\mathbf{X}}_c||_{\widetilde{\mathcal{A}}_c} = \inf_{\stackrel{\bsym{\alpha}_c[q]\in \mathbb{C}, \boldsymbol{c}_q \in [0,1]^3}{||\mathbf{v}||_2 = 1,\Vert\widetilde{\mathbf{E}}\Vert\leq \varepsilon_e}} \Bigg\{\sum_q \vert[\bsym{\alpha}_c]_q\vert\Big|  \widetilde{\mathbf{X}}_c =& \sum_q [\bsym{\alpha}_c]_q\mathbf{v}\mathbf{w}(\mathbf{c}_q)^H\times\nonumber\\&(\widetilde{\mathbf{E}} + \mathbf{I}_{MPN_r})\Bigg\}\nonumber.
\end{align}
Different from the optimization problem in \eqref{eq:primal_problem} where we have an equality constraint due to the signal model, here, since a noisy case is considered, the optimization problem is formulated as 

\begin{equation}
    \minimize_{\widetilde{\mathbf{X}}_r,\widetilde{\mathbf{X}}_c} \Vert \mathbf{y} - \mathcal{B}_r(\widetilde{\mathbf{X}}_r) - \mathcal{B}_c(\widetilde{\mathbf{X}}_c)  \Vert_2 + \mu_r \Vert\widetilde{\mathbf{X}}_r\Vert_{\widehat{\mathcal{A}}_r} + \mu_c \Vert\widetilde{\mathbf{X}}_c\Vert_{\widehat{\mathcal{A}}_c}
\end{equation}
here the regularization parameters $\mu_r$ and $\mu_c$ balance the data fidelity term due to the noise scenario and the sparsity promoted by the atomic norms.  Further in this section, we will derive the optimal value of these parameters depending on the noise level $\sigma_{\bsym{\omega}}$ and on the error vector bound. We now analyze  the dual optimization problem. First, the corresponding Lagrangian function is
\begin{align}
    \mathcal{L}(\widetilde{\mathbf{X}}_r,\widetilde{\mathbf{X}}_r,\mathbf{q},\mathbf{x}) = &\Vert\mathbf{x}-\mathbf{y}\Vert + (\mu_r\Vert\widetilde{\mathbf{X}}_r\Vert_{\widetilde{\mathcal{A}}_r}+\mu_c\Vert\mathbf{\widetilde{X}}_c\Vert_{\widetilde{\mathcal{A}}_c})+\nonumber\\&\langle\mathbf{x}-\mathcal{B}(\widetilde{\mathbf{X}}_r)_r-\mathcal{B}(\widetilde{\mathbf{X}}_c)_c,\mathbf{q}\rangle.\nonumber
\end{align}
This yields the dual problem as 
\begin{equation}
    \maximize_{\mathbf{q}}\min_{\widetilde{\mathbf{X}}_r,\widetilde{\mathbf{X}}_c,\mathbf{x}}= \mathcal{L}(\widetilde{\mathbf{X}}_r,\widetilde{\mathbf{X}}_c,\mathbf{q},\mathbf{x})=\max_\mathbf{q}\{\mathcal{L}_1(\mathbf{q})-\mathcal{L}_2(\mathbf{q})\},\nonumber
\end{equation}
where $\mathcal{L}_1 = \min_\mathbf{x}\frac{1}{2}\Vert\mathbf{x-y}\Vert_2^2+\langle\mathbf{x},\mathbf{q}\rangle$ or, equivalently, 
\begin{equation}
    \mathcal{L}_1(\mathbf{q}) = \Vert\mathbf{q}-\mathbf{y}\Vert_2^2  + \frac{1}{2}\Vert\mathbf{y}\Vert_2^2,\nonumber
\end{equation}
and $\mathcal{L}_2(\mathbf{q}) = \max_{\widetilde{\mathbf{X}}_r,\widetilde{\mathbf{X}}_c}\langle\mathcal{B}(\widetilde{\mathbf{X}}_r)_r+\mathcal{B}(\widetilde{\mathbf{X}}_c)_c,\mathbf{q}\rangle -(\rho\Vert\widetilde{\mathbf{X}}_r\Vert_{\mathcal{A}_r}+\Vert\widetilde{\mathbf{X}}_c\Vert_{\mathcal{A}_c}) $
 or, equivalently,
\begin{equation}
    \mathcal{L}_2(\mathbf{q}) = \max_{\widetilde{\mathbf{X}}_r,\widetilde{\mathbf{X}}_c}(\langle\widetilde{\mathbf{X}}_r,\mathcal{B}^*_r(\mathbf{q})\rangle-\rho\Vert\widetilde{\mathbf{X}}_r\Vert_{\mathcal{A}_r})+(\langle\widetilde{\mathbf{X}}_c,\mathcal{B}^*_c(\mathbf{q})\rangle-\Vert\widetilde{\mathbf{X}}_c\Vert_{\mathcal{A}_c}).\nonumber
\end{equation}
It follows from the definition of the dual norm that
\begin{equation}
    \mathcal{L}_2(\mathbf{q}) = \max_{\widetilde{\mathbf{X}}_r,\widetilde{\mathbf{X}}_c}(I_r(\Vert\mathbf{q}\Vert^*_{\widetilde{\mathcal{A}}_r}\leq \rho)+I_c(\Vert\mathbf{q}\Vert^*_{\mathcal{A}_c}\leq 1)),\nonumber
\end{equation}
where 
\begin{equation}
    I_r(\Vert\mathbf{q}\Vert^*_{\widehat{\mathcal{A}}_r}\leq \rho) =  \left\{\begin{array}{cc}
        0 & \text{if }  \Vert\mathbf{q}\Vert^*_{\widetilde{\mathcal{A}}_r}\leq \rho, \\
         \infty & \text{otherwise,} \\
    \end{array}\right. \nonumber
\end{equation}
and
\begin{equation}
    I_c(\Vert\mathbf{q}\Vert^*_{\mathcal{A}_c}\leq 1) =  \left\{\begin{array}{cc}
        0 & \text{if }  \Vert\mathbf{q}\Vert^*_{\mathcal{A}_c}\leq 1, \\
         \infty & \text{otherwise,} \\
    \end{array}\right. \nonumber
\end{equation}
are the indicator functions. Then, the dual problem becomes
which is
\begin{align}
    &\underset{\mathbf{q}}{\textrm{minimize}}\Vert\mathbf{q-y}\Vert_2
    \text{ subject to } \Vert\mathcal{B}_r^\star(\mathbf{q})\Vert^\star_{\widehat{\mathcal{A}}_r}\leq\mu_r, %\nonumber\\
    \Vert \mathcal{B}_c^\star(\mathbf{q})\Vert^\star_{\widehat{\mathcal{A}}_c}\leq\mu_c, 
\label{eq:dual_problem_op_errors}    
\end{align}
The following proposition gives the SDP formulation of this dual optimization problem
