\section{Conclusion}
In this work, we present a novel and effective zero-shot approach for chest X-ray diagnosis prediction, which provides an explanation for the model's decision. We leverage BioVil, a pretrained, domain-specific CLIP model, and use contrastive observation-based prompting to make predictions without label supervision. Our approach significantly outperforms previous zero-shot methods on CheXpert and Chest-Xray14, showcasing the effectiveness of our approach. Furthermore, we show that designing informative prompts is crucial to improve model performance. Our ablation studies demonstrate that adding disease indication and report style formulation to observation-based prompts notably enhances performance, underscoring the importance of aligning prompts with the domain-specific language used in medical reports. Additionally, contrastive prompts significantly boost performance, suggesting that the model can benefit from explicitly contrasting positive and negative examples.

Our work highlights the potential of contrastive pretraining combined with observation-based prompting as a promising avenue for zero-shot medical image classification, where labeled data is scarce or expensive to obtain, and explainability is vital. We envision that our approach can be extended to other medical imaging domains and have practical applications in real-world scenarios. Our findings contribute to the growing body of research to improve the accuracy and interpretability of medical image diagnosis.

\section*{Acknowledgements}
The authors gratefully acknowledge the financial support by the Federal Ministry of Education and Research of Germany (BMBF) under project DIVA (FKZ 13GW0469C) and the Bavarian Research Foundation (BFS) under project PandeMIC (grant AZ-1429-20C).