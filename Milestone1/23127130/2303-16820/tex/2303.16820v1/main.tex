%% Beginning of file 'sample631.tex'
%%
%% Modified 2021 March
%%
%% This is a sample manuscript marked up using the
%% AASTeX v6.31 LaTeX 2e macros.
%%
%% AASTeX is now based on Alexey Vikhlinin's emulateapj.cls 
%% (Copyright 2000-2015).  See the classfile for details.

%% AASTeX requires revtex4-1.cls and other external packages such as
%% latexsym, graphicx, amssymb, longtable, and epsf.  Note that as of 
%% Oct 2020, APS now uses revtex4.2e for its journals but remember that 
%% AASTeX v6+ still uses v4.1. All of these external packages should 
%% already be present in the modern TeX distributions but not always.
%% For example, revtex4.1 seems to be missing in the linux version of
%% TexLive 2020. One should be able to get all packages from www.ctan.org.
%% In particular, revtex v4.1 can be found at 
%% https://www.ctan.org/pkg/revtex4-1.

%% The first piece of markup in an AASTeX v6.x document is the \documentclass
%% command. LaTeX will ignore any data that comes before this command. The 
%% documentclass can take an optional argument to modify the output style.
%% The command below calls the preprint style which will produce a tightly 
%% typeset, one-column, single-spaced document.  It is the default and thus
%% does not need to be explicitly stated.
%%
%% using aastex version 6.3
\documentclass[%linenumbers,
%amsmath,amssymb,
%showkeys,
apjs,
%tighten,
%anonymous,
twocolappendix,
twocolumn
]{aastex631}
%% The default is a single spaced, 10 point font, single spaced article.
%% There are 5 other style options available via an optional argument. They
%% can be invoked like this:
%%
%% \documentclass[arguments]{aastex631}
%% 
%% where the layout options are:
%%
%%  twocolumn   : two text columns, 10 point font, single spaced article.
%%                This is the most compact and represent the final published
%%                derived PDF copy of the accepted manuscript from the publisher
%%  manuscript  : one text column, 12 point font, double spaced article.
%%  preprint    : one text column, 12 point font, single spaced article.  
%%  preprint2   : two text columns, 12 point font, single spaced article.
%%  modern      : a stylish, single text column, 12 point font, article with
%% 		  wider left and right margins. This uses the Daniel
%% 		  Foreman-Mackey and David Hogg design.
%%  RNAAS       : Supresses an abstract. Originally for RNAAS manuscripts 
%%                but now that abstracts are required this is obsolete for
%%                AAS Journals. Authors might need it for other reasons. DO NOT
%%                use \begin{abstract} and \end{abstract} with this style.
%%
%% Note that you can submit to the AAS Journals in any of these 6 styles.
%%
%% There are other optional arguments one can invoke to allow other stylistic
%% actions. The available options are:
%%
%%   astrosymb    : Loads Astrosymb font and define \astrocommands. 
%%   tighten      : Makes baselineskip slightly smaller, only works with 
%%                  the twocolumn substyle.
%%   times        : uses times font instead of the default
%%   linenumbers  : turn on lineno package.
%%   trackchanges : required to see the revision mark up and print its output
%%   longauthor   : Do not use the more compressed footnote style (default) for 
%%                  the author/collaboration/affiliations. Instead print all
%%                  affiliation information after each name. Creates a much 
%%                  longer author list but may be desirable for short 
%%                  author papers.
%% twocolappendix : make 2 column appendix.
%%   anonymous    : Do not show the authors, affiliations and acknowledgments 
%%                  for dual anonymous review.
%%
%% these can be used in any combination, e.g.
%%
%% \documentclass[twocolumn,linenumbers,trackchanges]{aastex631}
%%
%% AASTeX v6.* now includes \hyperref support. While we have built in specific
%% defaults into the classfile you can manually override them with the
%% \hypersetup command. For example,
%%
%% \hypersetup{linkcolor=red,citecolor=green,filecolor=cyan,urlcolor=magenta}
%%
%% will change the color of the internal links to red, the links to the
%% bibliography to green, the file links to cyan, and the external links to
%% magenta. Additional information on \hyperref options can be found here:
%% https://www.tug.org/applications/hyperref/manual.html#x1-40003
%%
%% Note that in v6.3 "bookmarks" has been changed to "true" in hyperref
%% to improve the accessibility of the compiled pdf file.
%%
%% If you want to create your own macros, you can do so
%% using \newcommand. Your macros should appear before
%% the \begin{document} command.
%%

%% Reintroduced the \received and \accepted commands from AASTeX v5.2
%\received{March 1, 2021}
%\revised{April 1, 2021}
%\accepted{\today}

%% Command to document which AAS Journal the manuscript was submitted to.
%% Adds "Submitted to " the argument.
%\submitjournal{ApJS}

%% For manuscript that include authors in collaborations, AASTeX v6.31
%% builds on the \collaboration command to allow greater freedom to 
%% keep the traditional author+affiliation information but only show
%% subsets. The \collaboration command now must appear AFTER the group
%% of authors in the collaboration and it takes TWO arguments. The last
%% is still the collaboration identifier. The text given in this
%% argument is what will be shown in the manuscript. The first argument
%% is the number of author above the \collaboration command to show with
%% the collaboration text. If there are authors that are not part of any
%% collaboration the \nocollaboration command is used. This command takes
%% one argument which is also the number of authors above to show. A
%% dashed line is shown to indicate no collaboration. This example manuscript
%% shows how these commands work to display specific set of authors 
%% on the front page.
%%
%% For manuscript without any need to use \collaboration the 
%% \AuthorCollaborationLimit command from v6.2 can still be used to 
%% show a subset of authors.
%
%\AuthorCollaborationLimit=2
%
%% will only show Schwarz & Muench on the front page of the manuscript
%% (assuming the \collaboration and \nocollaboration commands are
%% commented out).
%%
%% Note that all of the author will be shown in the published article.
%% This feature is meant to be used prior to acceptance to make the
%% front end of a long author article more manageable. Please do not use
%% this functionality for manuscripts with less than 20 authors. Conversely,
%% please do use this when the number of authors exceeds 40.
%%
%% Use \allauthors at the manuscript end to show the full author list.
%% This command should only be used with \AuthorCollaborationLimit is used.

%% The following command can be used to set the latex table counters.  It
%% is needed in this document because it uses a mix of latex tabular and
%% AASTeX deluxetables.  In general it should not be needed.
%\setcounter{table}{1}

\usepackage[T1]{fontenc}
\usepackage{CJKutf8}
\usepackage{hyperref}
\usepackage{amsmath} 
\usepackage{physics}
\usepackage{amssymb}
\usepackage{enumerate}
\usepackage{bm}

%%%%%%%%%%%%%%%%%%%%%%%%%%%%%%%%%%%%%%%%%%%%%%%%%%%%%%%%%%%%%%%%%%%%%%%%%%%%%%%%
%%
%% The following section outlines numerous optional output that
%% can be displayed in the front matter or as running meta-data.
%%
%% If you wish, you may supply running head information, although
%% this information may be modified by the editorial offices.
\shorttitle{
General-relativistic simulations of the formation of a magnetized hybrid star
}
\shortauthors{Yip et al.}
%%
%% You can add a light gray and diagonal water-mark to the first page 
%% with this command:
%% \watermark{text}
%% where "text", e.g. DRAFT, is the text to appear.  If the text is 
%% long you can control the water-mark size with:
%% \setwatermarkfontsize{dimension}
%% where dimension is any recognized LaTeX dimension, e.g. pt, in, etc.
%%
%%%%%%%%%%%%%%%%%%%%%%%%%%%%%%%%%%%%%%%%%%%%%%%%%%%%%%%%%%%%%%%%%%%%%%%%%%%%%%%%
\graphicspath{{./}{figures/}}
%% This is the end of the preamble.  Indicate the beginning of the
%% manuscript itself with \begin{document}.

\begin{document}

\title{General-relativistic simulations of the formation of a magnetized hybrid star}

%% LaTeX will automatically break titles if they run longer than
%% one line. However, you may use \\ to force a line break if
%% you desire. In v6.31 you can include a footnote in the title.

%% A significant change from earlier AASTEX versions is in the structure for 
%% calling author and affiliations. The change was necessary to implement 
%% auto-indexing of affiliations which prior was a manual process that could 
%% easily be tedious in large author manuscripts.
%%
%% The \author command is the same as before except it now takes an optional
%% argument which is the 16 digit ORCID. The syntax is:
%% \author[xxxx-xxxx-xxxx-xxxx]{Author Name}
%%
%% This will hyperlink the author name to the author's ORCID page. Note that
%% during compilation, LaTeX will do some limited checking of the format of
%% the ID to make sure it is valid. If the "orcid-ID.png" image file is 
%% present or in the LaTeX pathway, the OrcID icon will appear next to
%% the authors name.
%%
%% Use \affiliation for affiliation information. The old \affil is now aliased
%% to \affiliation. AASTeX v6.31 will automatically index these in the header.
%% When a duplicate is found its index will be the same as its previous entry.
%%
%% Note that \altaffilmark and \altaffiltext have been removed and thus 
%% can not be used to document secondary affiliations. If they are used latex
%% will issue a specific error message and quit. Please use multiple 
%% \affiliation calls for to document more than one affiliation.
%%
%% The new \altaffiliation can be used to indicate some secondary information
%% such as fellowships. This command produces a non-numeric footnote that is
%% set away from the numeric \affiliation footnotes.  NOTE that if an
%% \altaffiliation command is used it must come BEFORE the \affiliation call,
%% right after the \author command, in order to place the footnotes in
%% the proper location.
%%
%% Use \email to set provide email addresses. Each \email will appear on its
%% own line so you can put multiple email address in one \email call. A new
%% \correspondingauthor command is available in V6.31 to identify the
%% corresponding author of the manuscript. It is the author's responsibility
%% to make sure this name is also in the author list.
%%
%% While authors can be grouped inside the same \author and \affiliation
%% commands it is better to have a single author for each. This allows for
%% one to exploit all the new benefits and should make book-keeping easier.
%%
%% If done correctly the peer review system will be able to
%% automatically put the author and affiliation information from the manuscript
%% and save the corresponding author the trouble of entering it by hand.

%\correspondingauthor{August Muench}
%\email{greg.schwarz@aas.org, gus.muench@aas.org}

\author[0009-0008-8501-3535]{Anson Ka Long Yip}
\email{klyip@phy.cuhk.edu.hk}
\affiliation{Department of Physics, The Chinese University of Hong Kong, Shatin, N.T., Hong Kong}

\author[0000-0003-1449-3363]{Patrick Chi-Kit Cheong \begin{CJK*}{UTF8}{bkai}(張志杰)\end{CJK*}}
%\email{patrick.cheong@berkeley.edu}
\affiliation{Department of Physics \& Astronomy, University of New Hampshire, 9 Library Way, Durham NH 03824, USA}
\affiliation{Department of Physics, University of California, Berkeley, Berkeley, CA 94720, USA}

\author[0000-0003-4297-7365]{Tjonnie Guang Feng Li}
%\email{tgfli@cuhk.edu.hk}
\affiliation{Department of Physics, The Chinese University of Hong Kong, Shatin, N.T., Hong Kong}
\affiliation{Institute for Theoretical Physics, KU Leuven, Celestijnenlaan 200D, B-3001 Leuven, Belgium}
\affiliation{Department of Electrical Engineering (ESAT), KU Leuven, Kasteelpark Arenberg 10, B-3001 Leuven, Belgium }

%% Note that the \and command from previous versions of AASTeX is now
%% depreciated in this version as it is no longer necessary. AASTeX 
%% automatically takes care of all commas and "and"s between authors names.

%% AASTeX 6.31 has the new \collaboration and \nocollaboration commands to
%% provide the collaboration status of a group of authors. These commands 
%% can be used either before or after the list of corresponding authors. The
%% argument for \collaboration is the collaboration identifier. Authors are
%% encouraged to surround collaboration identifiers with ()s. The 
%% \nocollaboration command takes no argument and exists to indicate that
%% the nearby authors are not part of surrounding collaborations.

%% Mark off the abstract in the ``abstract'' environment. 
\begin{abstract}
	Strongly magnetized neutron stars are popular candidates for producing detectable electromagnetic and gravitational-wave signals.
    A rapid density increase in a neutron star core could also trigger the phase transition from hadrons to deconfined quarks and form a hybrid star. 
    This formation process could release a considerable amount of energy in the form of gravitational waves and neutrinos. 
    Hence, the formation of a magnetized hybrid star is an interesting scenario for detecting all these signals.
    These detections may provide essential probes for the magnetic field and composition of such stars.
    Thus far, a dynamical study of the formation of a magnetized hybrid star has yet to be realized.
    Here, we investigate the formation dynamics and the properties of a magnetized hybrid star through dynamical simulations.
    We find that the maximum values of rest-mass density and magnetic field strength increase slightly and these two quantities are coupled in phase during the formation.
    We then demonstrate that all microscopic and macroscopic quantities of the resulting hybrid star vary dramatically when the maximum magnetic field strength goes beyond a threshold of $\sim 5 \times 10^{17}$ G but they are insensitive to the magnetic field below this threshold. 
    Specifically, the magnetic deformation makes the rest-mass density drop significantly, suppressing the matter fraction in the mixed phase.
    Therefore, this work provides a solid support for the magnetic effects on a hybrid star, so it is possible to link observational signals from the star to its magnetic field configuration.
\end{abstract}

%% Keywords should appear after the \end{abstract} command. 
%% The AAS Journals now uses Unified Astronomy Thesaurus concepts:
%% https://astrothesaurus.org
%% You will be asked to selected these concepts during the submission process
%% but this old "keyword" functionality is maintained in case authors want
%% to include these concepts in their preprints.
%\keywords{Classical Novae (251) --- Ultraviolet astronomy(1736) --- History of astronomy(1868) --- Interdisciplinary astronomy(804)}

%% From the front matter, we move on to the body of the paper.
%% Sections are demarcated by \section and \subsection, respectively.
%% Observe the use of the LaTeX \label
%% command after the \subsection to give a symbolic KEY to the
%% subsection for cross-referencing in a \ref command.
%% You can use LaTeX's \ref and \label commands to keep track of
%% cross-references to sections, equations, tables, and figures.
%% That way, if you change the order of any elements, LaTeX will
%% automatically renumber them.
%%
%% We recommend that authors also use the natbib \citep
%% and \citet commands to identify citations.  The citations are
%% tied to the reference list via symbolic KEYs. The KEY corresponds
%% to the KEY in the \bibitem in the reference list below. 

\section{Introduction} \label{sec:intro}

Neutron stars are natural laboratories for studying physics under extreme conditions, which terrestrial experiments cannot reproduce. 
On the one hand, the density of a neutron star reaches above the nuclear saturation density $\rho_{0} \sim 2.8 \times 10^{14}$ g cm$^{-3}$, at which the canonical atomic structure of matter is disrupted. 
The detailed microphysics and the concerning equation of state at supra-nuclear densities still remain elusive. Exotic matter, such as deconfined quark matter and hyperons, could exist in this ultradense regime (See e.g. \citealt{2018ASSL..457.....R} for a review).

Several studies have long proposed compact stars that are partly or wholly composed of deconfined quark matter \citep{1970PThPh..44..291I,1971PhRvD...4.1601B,1984PhRvD..30..272W}. 
These stars are typically interpreted as the products of the phase transition of the hadrons in the original neutron stars. 
In particular, when the density inside a neutron star reaches a threshold, a phase transition converting hadrons into deconfined quarks could happen. 
If this phase transition only occurs in the stellar core, the resulting star is usually called a `hybrid star'. 
A hybrid star generally has a smaller radius and higher compactness than the progenitor neutron star.
Therefore, gravitational potential energy of the order of $\sim 10^{52}$ erg is expected to be released when a hybrid star is formed. 
Significant portions of the released energy could give rise to the emission of neutrinos and gravitational waves. 
Detecting these phase transition signals provides evidence of deconfined quark matter. Newly born neutron stars in supernovae and accreting neutron stars in binary systems are possible hosts for such a phase transition (See e.g. \citealt{1999JPhG...25..195W,2009MNRAS.392...52A} for reviews).  

On the other hand, neutron stars have the strongest magnetic field found in the Universe. 
Dipole spin-down models allow for the estimation of the surface magnetic field strength of neutron stars. 
With the surface field strength $\mathcal{B}_\mathrm{s}$, we can classify neutron stars into millisecond pulsars with $\mathcal{B}_\mathrm{s} \sim 10^{8 - 9}$ G, classical pulsars with $\mathcal{B}_\mathrm{s} \sim 10^{11 - 13}$ G, and magnetars with $\mathcal{B}_\mathrm{s} \sim 10^{14 - 15}$ G.
Although there is still no direct observation of the interior magnetic field of neutron stars, virial theorem suggests that it could reach $10^{18 - 20}$ G (see e.g. \citealt{2010PhRvC..82f5802F}, \citealt{1991ApJ...383..745L}, \citealt{1989ApJ...342..958F}, and \citealt{2001ApJ...554..322C}). 
Furthermore, binary neutron star simulations have demonstrated that the local maximum magnetic field can be amplified up to $\sim 10^{17}$ G during the merger \citep{2006Sci...312..719P,2015PhRvD..92f4034K,2015PhRvD..92l4034K,2020PhRvD.102j3006A}.

Highly magnetized neutron stars are promising candidates for explaining some puzzling astronomical phenomena, including soft gamma-ray repeaters and X-ray pulsars \citep{1998Natur.393..235K,1999ApJ...510L.111H,1995ApJ...442L..17M,2000A&A...361..240M,1995A&A...299L..41V}.
Moreover, neutron stars can be deformed by the magnetic field, depending on the geometry of the magnetic field. 
A purely toroidal field induces prolateness \citep{2008PhRvD..78d4045K,2009ApJ...698..541K,2012MNRAS.427.3406F}, while a purely poloidal field causes oblateness to neutron stars \citep{1995A&A...301..757B,2001A&A...372..594K,2012PhRvD..85d4030Y}. 
These magnetic-field-induced distortions make rotating neutron stars possible sources for the emission of detectable continuous gravitational waves \citep{1996A&A...312..675B}. 
However, the actual field geometry inside neutron stars is still unknown. Stability analyses of magnetized stars suggest that simple geometries are subjected to instabilities \citep{1957PPSB...70...31T,1973MNRAS.161..365T,1973MNRAS.163...77M,1974MNRAS.168..505M,1973MNRAS.162..339W}. Magnetohydrodynamics simulations propose a mixed configuration of toroidal and poloidal fields as the most favored configuration \citep{2006A&A...450.1077B,2006A&A...450.1097B,2009MNRAS.397..763B}. 
This configuration is usually referred to as the `twisted torus'.

Deconfined quarks and strong magnetic fields are expected to be present inside neutron stars, so studying magnetized hybrid stars is necessary to probe the combined effects of these two features.
Previous studies have investigated the properties of magnetized hybrid stars by constructing equilibrium models (e.g. \citealt{2009JPhG...36k5204R,2012EPJA...48..189D,2015JPhCS.607a2013I,2015MNRAS.447.3785C,2016MNRAS.463..571F,2016MNRAS.456.2937F,2022MNRAS.512..517M}). 
In particular, \citet{2015MNRAS.447.3785C} and \citet{2016MNRAS.456.2937F} have demonstrated that the pure field contribution to the energy–momentum tensor primarily contributes to the macroscopic properties of magnetized hybrid stars. 
In contrast, the magnetic effects in the equation of state and the field-matter interactions have negligible effects on these properties. 
Moreover, a magnetic field reduces the central density and prevents the appearance of quark matter. Dynamics of hybrid star have been studied by numerical simulations \citep{2006ApJ...639..382L,2009MNRAS.392...52A,2011PhRvD..84h3002H,2018ApJ...859...57P,2020ApJ...893..151P}. 
Nonetheless, these studies did not take the magnetic field into account.
Since the dynamical stability and the possible observational signal of a magnetized hybrid star could not be thoroughly examined through equilibrium modeling, a dynamical study of this star is still indispensable.

In this work, we numerically study the formation of a magnetized hybrid star through general relativistic magnetohydrodynamics simulations.
Specifically, we first construct magnetized neutron star equilibrium models by the open-sourced code \texttt{XNS} \citep{2011A&A...528A.101B,2014MNRAS.439.3541P,2015MNRAS.447.2821P,2017MNRAS.470.2469P,2020A&A...640A..44S} and then dynamically evolve these models using the new general relativistic magnetohydrodynamics code \texttt{Gmunu} \citep{2020CQGra..37n5015C,2021MNRAS.508.2279C,2022ApJS..261...22C}. 
The details of the initial neutron star models, hybrid star mdoels and evolutions are described in Section \ref{sec:num_method}.
Next, the results of the formation process and the properties of the resulting star are presented in Sections \ref{sec:dynamics} and \ref{sec:properties} respectively. 
Finally, we provide the conclusions in Section \ref{sec:conclusions}.

\section{Numerical methods}\label{sec:num_method}
\subsection{Initial neutron star models}
We construct the non-rotating magnetized neutron star equilibrium models in axisymmetry by the open-sourced code \texttt{XNS}  \citep{2011A&A...528A.101B,2014MNRAS.439.3541P,2015MNRAS.447.2821P,2017MNRAS.470.2469P,2020A&A...640A..44S}. These equilibrium models serve as initial data for our simulations.

The initial neutron star models are constructed with a polytropic equation of state,
    \begin{equation}
    P=K \rho^\gamma,
    \end{equation}
where $P$ is the pressure, $\rho$ is the rest-mass density and we choose a polytropic constant $K=1.6 \times 10^5$ cm$^5$ g$^{-1}$ s$^{-2}$ (which equals to 110 in the unit of $c=G=M_{\odot}=1$) and a polytropic index $\gamma=2$. 

We specify the specific internal energy $\epsilon$ on the initial time-slice by
    \begin{equation}
    \epsilon=\frac{K}{\gamma-1} \rho^{\gamma-1}.
    \end{equation}
We adopt a magnetic polytropic law for the toroidal fields
    \begin{equation}
        \mathcal{B}_{\phi}=\alpha^{-1}K_{\mathrm{m}}(\rho h\varpi^2)^m
    \end{equation}
where $\alpha$ is the laspe function, $K_{\mathrm{m}}$ is the toroidal magnetization constant, $h$ is the specific enthalpy, $\varpi^2=\alpha^2\psi^4r^2\sin^2\theta$, $\psi$ is the conformal factor, $(r,\theta)$ are the radial and angular coordinates in 2D spherical coordinates, and $m\geq1$ is the toroidal magnetization index.

In total, 9 models are constructed, where `REF' is the non-magnetized reference model and the remaining 8 neutron star models are magnetized.
They are part of the models used in \citet{2022CmPhy...5..334L}.
Because we do not intend to perform a comprehensive study of neutron stars with different masses in this work, all models have a fixed baryonic mass $M_0=1.68$ $M_\odot$, which is within the typical range of neutron star mass.
Also, the 8 magnetized models have the same toroidal magnetization index $m=1$ but different values of toroidal magnetization constant $K_{\mathrm{m}}$. 
They are arranged in the order of increasing maximum magnetic field strength $\mathcal{B}_\mathrm{max}$, where the model `T1K1' has the lowest strength, and `T1K2' has the second-lowest strength, so on and so forth.
(`T1' represents the toroidal magnetization index $m=1$ and `K' indicates the toroidal magnetization constant $K_{\mathrm{m}}$).
The configuration of these models allows a phase transition that occurs inside the stellar core and facilitates comparison with \citet{2022CmPhy...5..334L}.
Table~\ref{table1} summarizes the detailed properties of all 9 models. 

\begin{table}
	\centering
	\caption{\label{table1} Properties of the 9 initial neutron star models constructed by the \texttt{XNS} code.
	All numerical values are rounded off to two decimal places.
	$\rho_{\mathrm{c}}$ is the central rest-mass density, $M_{\mathrm{g}}$ is the gravitational mass, $r_{\mathrm{e}}$ is the equatorial radius, and $\mathcal{B}_{\mathrm{max}}$ is the maximum toroidal field strength inside the neutron star. 
    All the models have a fixed baryonic mass $M_{\mathrm{0}}=1.68$ $M_{\odot}$ and the 8 magnetized models also have the same toroidal magnetization index $m=1$.}
	\begin{tabular}{cccccccc}
		 Model & $\rho_{\mathrm{c}}$ & $M_{\mathrm{g}}$ & $r_\mathrm{e}$ & $\mathcal{B}_{\mathrm{max}}$\\
		 & ($10^{14}$ g cm$^{-3}$) & ($M_{\odot}$) & (km) & ($10^{17}$ G)\\
		\hline
		 REF & 8.56 & 1.55  & 11.85 & 0.00\\
		 T1K1 & 8.56 & 1.55 & 11.85 & $3.45\times10^{-2}$\\
		 T1K2 & 8.56 & 1.55 & 11.85 & $6.89\times10^{-2}$\\
		 T1K3 & 8.57 & 1.55 & 11.85 & $3.44\times10^{-1}$\\
		 T1K4 & 8.63 & 1.55 & 11.92 & 1.36\\
		 T1K5 & 8.81 & 1.56 & 12.15 & 2.63\\
		 T1K6 & 9.10 & 1.58 & 14.43 & 5.52\\
		 T1K7 & 8.81 & 1.59 & 16.21 & 6.01\\
		 T1K8 & 8.27 & 1.60 & 18.64 & 6.14\\
	\end{tabular}

\end{table}

\subsection{Hybrid star models and evolution}
The MIT bag model equation of state introduced by \citet{johnson1975bag} has been widely used to model quark matter inside compact stars (see e.g. \citealt{weber1999quark,glendenning2012compact} for a review). 
The MIT bag model equation of state for massless and non-interacting quarks is given by
   \begin{equation}
    P_{\mathrm{q}}=\frac{1}{3}(e-4 B),
    \end{equation}
where $P_{\mathrm{q}}$ is the pressure of quark matter, $e$ is the total energy density and $B$ is the bag constant.

For the normal hadronic matter, we adopt an ideal gas type of equation of state for the evolution
    \begin{equation}
    P_{\mathrm{h}}=(\gamma-1) \rho \epsilon
    \end{equation}
where $P_{\mathrm{h}}$ is the pressure of hadronic matter and $\gamma$ is kept to be 2.

Either two or three parts constitute the hybrid star formed after the phase transition: (i) a hadronic matter region with a rest-mass density below the lower threshold density $\rho_\mathrm{hm}$, (ii) a mixed phase of the deconfined quark matter and hadronic matter for the region with a rest-mass density in between the lower threshold density $\rho_\mathrm{hm}$ and the upper threshold density $\rho_\mathrm{qm}$, and (iii) a region of pure quark matter phase with a rest-mass density beyond $\rho_\mathrm{qm}$ (this might or might not be present in practice, depending on the maximum density reached). 
With this picture, the equation of state for hybrid stars is given by
\begin{equation}
P= \begin{cases}P_{\mathrm{h}} & \text { for } \rho<\rho_{\mathrm{hm}}, \\ \alpha_\mathrm{q} P_{\mathrm{q}}+(1-\alpha_\mathrm{q}) P_{\mathrm{h}} & \text { for } \rho_{\mathrm{hm}} \leq \rho \leq \rho_{\mathrm{qm}}, \\ P_{\mathrm{q}} & \text { for } \rho_{\mathrm{qm}}<\rho,\end{cases}
\label{eqn6}
\end{equation}
where 
\begin{equation}
\alpha_\mathrm{q}=1-\left(\frac{\rho_{\mathrm{qm}}-\rho}{\rho_{\mathrm{qm}}-\rho_{\mathrm{hm}}}\right)^\delta
\end{equation}
is a scale factor to quantify the relative contribution due to hadronic and quark matters
to the total pressure in the mixed phase. 
The exponent $\delta$ adjusts the pressure contribution due to quark matter. 
We set 3 values of $\delta \in \{1, 2, 3\}$ to investigate the dynamical effects of varying quark matter contributions. 
We choose $\rho_\mathrm{hm}= 6.97 \times 10^{14}$ g cm$^{-3}$, $\rho_\mathrm{qm}= 24.3 \times 10^{14}$ g cm$^{-3}$ and $B^{1/4}=170$ MeV. This treatment of phase transition is similar to that of \citet{2009MNRAS.392...52A}.

We employ the new general relativistic magnetohydrodynamics code \texttt{Gmunu} \citep{2020CQGra..37n5015C,2021MNRAS.508.2279C,2022ApJS..261...22C} to evolve the stellar models in dynamical spacetime. 
\texttt{Gmunu} solves the Einstein equations in the conformally flat condition approximation based on the multigrid method.

We perform 2D ideal general-relativistic magnetohydrodynamics simulations in axisymmetry with respect to the $z$-axis and equatorial symmetry using cylindrical coordinates $(R,z)$. 
The computational domain covers [0,100] for both $R$ and $z$, with the base grid resolution $N_{R} \times N_{z} = 32 \times 32$ and allowing 6 AMR levels (effective resolution $= 1024 \times 1024$).
The refinement criteria of AMR is the same as that in \cite{2021MNRAS.508.2279C,2022CmPhy...5..334L}.
Our simulations adopt TVDLF approximate Riemann solver \citep{1996JCoPh.128...82T}, 3rd-order reconstruction method PPM \citep{1984JCoPh..54..174C} and 3rd-order accurate SSPRK3 time integrator \citep{1988JCoPh..77..439S}. 
The region outside the star is filled with an artificial low-density `atmosphere' with rest-mass density $\rho_\mathrm{atm} \sim 10^{-10} \rho_\mathrm{c}$. 
Since we are restricted to purely toroidal field models and axisymmetry for the simulations, we do not use any divergence cleaning method.

\section{Formation dynamics}\label{sec:dynamics}
For each of the 9 equilibrium models, we perform simulations for three times, once for each value of the exponent $\delta \in \{1, 2, 3\}$. 
Consequently, $9 \times 3 = 27$ simulations are performed in total.

Since all simulations exhibit the same behavior, we take one of them as an example to describe the features of the formation dynamics.
Here, we choose the simulation with an initial maximum magnetic field strength $\mathcal{B}_\mathrm{max} = 5.52 \times 10^{17}$ G (i.e. Initial model T1K6) and an exponent $\delta=3$.
The exponent $\delta=3$ corresponds to a more substantial phase transition effect, which favors the demonstration of the formation dynamics.
As illustrated by the radial profiles of the rest-mass density $\rho(r)$ (top panel) and the magnetic field strength $\mathcal{B}(r)$ (bottom panel) at $t=$ 0 ms (grey solid lines), 10 ms (yellow dash-dotted lines), and 20 ms (red dotted lines) in Fig. \ref{fig1}, the resulting hybrid star has a slightly higher central rest-mass density and maximum magnetic field strength after phase transition. 
In addition, the magnetic field inside the star becomes more concentrated towards the core with a tiny shift of the maximum magnetic field strength position $r_{\mathcal{B}}$ (dashed lines) to smaller values.
Furthermore, new configurations of $\rho(r)$ and $\mathcal{B}(r)$ are obtained at $t=10$ ms and remain until at least $t=20$ ms.

Fig. \ref{fig2} shows the time evolution of the maximum values of the rest-mass density $\rho_\mathrm{max}(t)$ (brown solid line) and the magnetic field strength $\mathcal{B}_\mathrm{max}(t)$ (dark cyan dash-dotted line) relative to their initial values $\rho_\mathrm{max}(0)$ and $\mathcal{B}_\mathrm{max}(0)$. 
The equilibrium values obtained at $t=$ 20 ms are plotted with dashed lines. 
We observe similar damped oscillatory behaviors for both quantities and and the star is relaxed into a new equilibrium configuration after the phase transition.
As discussed in \citet{2009MNRAS.392...52A}, this damping is mainly due to numerical dissipation and shock heating.
Importantly, these two quantities are coupled in phase during the formation process.
Moreover, after reaching their peak values at $t \sim 0.5$ ms, the oscillation amplitudes decrease by a factor of $e^{-1}$ at $t\sim 6$ ms. 
This damping explains the minor discrepancy between the radial profiles at $t=10$ ms and $t=20$ ms as demonstrated in Fig. \ref{fig1}.

\begin{figure}
    \centering
    \includegraphics[width=\columnwidth, angle=0]{Fig1.pdf}
    \caption{
	        The radial profile of the rest-mass density $\rho(r)$ (top panel) and the magnetic field strength $\mathcal{B}(r)$ (bottom panel) in the equatorial plane for the simulation with the initial model T1K6 and the exponent $\delta=3$ at $t=0$ ms (grey solid lines), 10 ms (yellow dash-dotted lines), and 20 ms (red dotted lines).
            Model T1K6 has an initial maximum magnetic field strength $\mathcal{B}_\mathrm{max} = 5.52 \times 10^{17}$ G and $\delta$ is an exponent describing the pressure contribution due to quark matter in the mixed phase.
            The dashed lines in the lower panel represent the maximum magnetic field strength position $r_{\mathcal{B}}$.
            Cadet blue region represents the portion of the matter in the mixed phase while light blue region denotes the portion of matter in hadronic phase.
            After phase transition, the resulting hybrid star obtains a slightly higher central rest-mass density and maximum magnetic field strength.
            Also, the magnetic field inside the star becomes more concentrated towards the core with a tiny shift of $r_{\mathcal{B}}$ to smaller values. 
            Moreover, these new configurations of $\rho(r)$ and $\mathcal{B}(r)$ are obtained at $t=10$ ms and remain the same until at least $t=20$ ms.
	         }
    \label{fig1}	
\end{figure}

\begin{figure}
    \centering
    \includegraphics[width=\columnwidth, angle=0]{Fig2.pdf}
    \caption{
            The time evolution of the maximum values of the rest-mass density $\rho_\mathrm{max}(t)$ (brown solid line) and the magnetic field strength $\mathcal{B}_\mathrm{max}(t)$ (dark cyan dash-dotted line) relative to their initial values $\rho_\mathrm{max}(0)$ and $\mathcal{B}_\mathrm{max}(0)$ the simulation with the initial model T1K6 and the exponent $\delta=3$.
            Model T1K6 has an initial maximum magnetic field strength $\mathcal{B}_\mathrm{max} = 5.52 \times 10^{17}$ G and $\delta$ is an exponent describing the pressure contribution due to quark matter in the mixed phase.
            Dashed lines are the equilibrium values of the two quantities obtained at $t=20$ ms. 
            Similar damped oscillatory behaviors are observed for both quantities and the star is relaxed into a new equilibrium configuration after the phase transition.
            Importantly, these two quantities are coupled in phase during the formation process. 
            Moreover, after reaching the peak values at $t \sim 0.5$ ms, the oscillation amplitudes are reduced by a factor of $e^{-1}$ at $t\sim 6$ ms. 
    }
    \label{fig2}	
\end{figure}

\section{Properties of the resulting magnetized hybrid stars}\label{sec:properties}
To better examine the properties of the resulting magnetized hybrid stars, we plot in Fig. \ref{fig3} different microscopic and macroscopic quantities against the maximum magnetic field strength $\mathcal{B}_\mathrm{max}$ of the stars. 
The data points are arranged into 3 sequences with 3 values of $\delta \in \{1,2,3\}$, where $\delta$ is an exponent quantifying the pressure contribution due to quark matter in the mixed phase. 
Here, we define the equatorial radius and polar radius of the resulting hybrid stars as the radial positions where the rest-mass density $\rho$ is less than or equal to $10^{-2}$ of the lower threshold density $\rho_\mathrm{hm}$ (i.e. $\rho \leq 6.97 \times 10^{12}$ g cm$^{-3}$).

We find that for $\mathcal{B}_\mathrm{max} \gtrsim 5 \times 10^{17}$, all microscopic and macroscopic quantities vary strongly with $\mathcal{B}_\mathrm{max}$, irrespective of $\delta$.
When $\mathcal{B}_\mathrm{max} \lesssim 3 \times 10^{17}$ G, all quantities vary slightly with $\mathcal{B}_\mathrm{max}$.
This means that it may be possible to link observational signals from a magnetized hybrid star to the magnetic field of the star.

Specifically, the central rest-mass density $\rho_\mathrm{c}$ (top left panel) and the baryonic mass fraction of the matter in the mixed phase $M_\mathrm{mp} / M_{0}$ (bottom left panel) decrease with $\mathcal{B}_\mathrm{max}$. 
These decreasing behaviors could be understood in terms of magnetic pressure. 
As the magnetic pressure becomes more dominant due to the increasing $\mathcal{B}_\mathrm{max}$, matter is pushed off-center to a greater extent. 
As a result, the rest-mass density $\rho$ in the stellar core reduces, giving a smaller $\rho_\mathrm{c}$. 
Moreover, as described in Eq. (\ref{eqn6}), reducing $\rho$ in the core contributes to a smaller fraction of matter that undergoes the phase transition and thus gives a smaller $M_\mathrm{mp} / M_{0}$.

Moreover, the equatorial radius $r_\mathrm{e}$ (top middle panel), the polar radius to equatorial radius ratio $r_\mathrm{p} / r_\mathrm{e}$ (bottom middle panel) and the gravitational mass $M_\mathrm{g}$ (top right panel) all increase with $\mathcal{B}_\mathrm{max}$. 
The increase in $M_\mathrm{g}$ is due to the increasing contribution of the magnetic field to $M_\mathrm{g}$ (corresponds to the increasing $\mathcal{B}^2$ term of Eq. (B1) in \citealt{2014MNRAS.439.3541P} for example). 
The other two increasing trends could be interpreted in terms of magnetic deformation. As the matter is pushed off-center by the increasing magnetic pressure, both $r_\mathrm{p}$ and $r_\mathrm{e}$ increase and the star then deviates from spherical symmetry. 
Previous studies of magnetized neutron star equilibrium models (e.g. \citealt{2008PhRvD..78d4045K,2009ApJ...698..541K,2012MNRAS.427.3406F}) indicate that a purely toroidal field deforms the stars to prolate shape, corresponding to $r_\mathrm{p} / r_\mathrm{e} > 1$. 
Thus, increasing $\mathcal{B}_\mathrm{max}$ of the toroidal field in our models causes the increase in $r_\mathrm{p} / r_\mathrm{e}$. 

We also examine the effect of pressure contribution due to quark matter $\delta$ on different quantities of the hybrid stars.
$\delta$ has a negligible effect on $r_\mathrm{e}$, $r_\mathrm{p}/r_\mathrm{e}$ and $M_{g}$ for all values of $\mathcal{B}_\mathrm{max}$. 
On the contrary, $\rho_\mathrm{c}$ and $M_\mathrm{mp} / M_{0}$ increase substantially with $\delta$ for $\mathcal{B}_\mathrm{max} \lesssim 3 \times 10^{17}$ G but they become less sensitive to $\delta$ for $\mathcal{B}_\mathrm{max} \gtrsim 5 \times 10^{17}$ G. 
These increasing trends could be interpreted in relation to pressure reduction. 
With the increasing value of $\delta$, the contribution due to quark matter to the total pressure becomes more important and the pressure reduction is enlarged. 
As a result, this enlarged pressure reduction makes the star collapse to a configuration with a higher $\rho_\mathrm{c}$ and $M_\mathrm{mp} / M_{0}$.

We compare our resulting hybrid stars with \citet{2016MNRAS.456.2937F}. 
The magnetized hybrid star models considered in this study are also in axisymmetry, but the magnetic field is purely poloidal. 
A poloidal field would make the stars oblate instead of prolate. 
Also, these models have a different baryonic mass $M_{\mathrm{0}}=2.2$ $M_{\odot}$. 
These equilibrium models are constructed by solving the coupled Maxwell–Einstein equations. 
They also employed a more realistic equation of state with both magnetic and thermal effects taken into account.

We plot the normalized gravitational mass $M_\mathrm{g}/M_\mathrm{g}^{*}$ against $\mathcal{B}_\mathrm{max}$ (bottom right panel) to compare with the models computed in \citet{2016MNRAS.456.2937F} (Red stars), where $M_\mathrm{g}^{*}$ is the gravitational mass of the non-magnetized reference models.
We observe that $M_\mathrm{g} / M_\mathrm{g}^{*}$ increases with $\mathcal{B}_\mathrm{max}$ similarly for the models in our simulations and \citet{2016MNRAS.456.2937F}.
Besides, this previous study also found that the magnetic field reduces the central baryon number density and hinders the appearance of matter in quark and mixed phases. 
These also agree with the trends of $\rho_\mathrm{c}$ and $M_\mathrm{mp} / M_{0}$ for our models.
Accordingly, despite the disparity in field geometry, baryonic mass, and construction method, our models agree qualitatively with the models in the previous study.
This similarity provides additional support that the properties of hybrid the properties of the magnetised hybrid stars presented here are robust.

\begin{figure*}
    \centering
    \includegraphics[width=\textwidth, angle=0]{Fig3.pdf}
    \caption{
            Plots of different microscopic and macroscopic quantities against the maximum magnetic field strength $\mathcal{B}_\mathrm{max}$ of our resulting hybrid star models. 
            In particular, we plot the central rest-mass density $\rho_\mathrm{c}$ (top left panel), the baryonic mass fraction of the matter in the mixed phase $M_\mathrm{mp} / M_{0}$ (bottom left panel), the equatorial radius $r_\mathrm{e}$ (top middle panel), the ratio between polar and the equatorial radii $r_\mathrm{p} / r_\mathrm{e}$ (bottom middle panel), and the gravitational mass $M_\mathrm{g}$ against $\mathcal{B}_\mathrm{max}$ (top right panel).
            The data points are arranged into 3 sequences with 3 values of $\delta \in \{1,2,3\}$, where $\delta$ is an exponent quantifying the pressure contribution due to quark matter in the mixed phase.
            The macroscopic and microscopic quantities of all hybrid star models are not sensitive to the magnetic field for $\mathcal{B}_\mathrm{max} \lesssim 3 \times 10^{17}$ G. 
            However, these quantities noticeably vary with $\mathcal{B}_\mathrm{max}$ for $\mathcal{B}_\mathrm{max} \gtrsim 5 \times 10^{17}$ G.
            In addition, $\delta$ has a negligible effect on $r_\mathrm{e}$, $r_\mathrm{p}/r_\mathrm{e}$ and $M_{g}$ for all values of $\mathcal{B}_\mathrm{max}$. 
            In contrast, $\rho_\mathrm{c}$ and $M_\mathrm{mp} / M_{0}$ increase substantially with $\delta$  for $\mathcal{B}_\mathrm{max} \lesssim 3 \times 10^{17}$ G but they become less sensitive to $\delta$ for $\mathcal{B}_\mathrm{max} \gtrsim 5 \times 10^{17}$ G.
            Furthermore, we plot the normalized gravitational mass $M_\mathrm{g}/M_\mathrm{g}^{*}$ against $\mathcal{B}_\mathrm{max}$  (bottom right panel) as a comparison with the models computed in \citet{2016MNRAS.456.2937F} (Red stars), where $M_\mathrm{g}^{*}$ is the gravitational mass of the non-magnetized reference models. 
            We find that $M_\mathrm{g} / M_\mathrm{g}^{*}$ increases with $\mathcal{B}_\mathrm{max}$ similarly for the models in our simulations and \citet{2016MNRAS.456.2937F}. 
            Hence, we find agreement amongst vastly different methods and thus provide a solid support for the magnetic effects on a hybrid star.
	         }
    \label{fig3}	
\end{figure*}

\section{Conclusions}\label{sec:conclusions}
In this paper, we studied the formation of a magnetized hybrid star by performing 2D axisymmetric general-relativistic magnetohydrodynamics simulations. 
We first found that the maximum values of rest-mass density and magnetic field strength in the stars rise slightly after a phase transition. 
The magnetic field also becomes more concentrated towards the center. In addition, the magnetic field and the rest-mass density are coupled during the process. 
We then investigated the properties of the resulting magnetized hybrid stars. 
Both macroscopic and microscopic quantities of the hybrid stars are not sensitive to the magnetic field until $\mathcal{B}_\mathrm{max} \gtrsim 5 \times 10^{17}$ G, where all quantities change significantly. 
Specifically, the magnetic deformation decreases the rest-mass density dramatically, leading to a substantial reduction in the matter fraction in the mixed phase. 
Similar trends for these quantities are found compared with \citet{2016MNRAS.456.2937F}.

This work takes the first step to dynamically studying magnetized hybrid stars. 
Several natural extensions should be considered to model them more realistically. 
First, a more realistic equation of state, which includes thermal and magnetic effects, should be adopted. 
In addition, since magnetized stars with purely toroidal fields are expected to be unstable, the suppression of instability in this work is mainly due to the restriction to 2D axisymmetry. 
The effects of purely poloidal fields and the twisted torus configurations should also be investigated. 
Since these field geometries extend to the outer region of neutron stars, a force-free/resistive magnetohydrodynamics solver is necessary for more realistic modeling. 
Also, 3D simulations without axisymmetry should be conducted to include the instability of magnetic fields. 
Finally, as most observations suggested that neutron stars rotate, rotations should also be included in future studies.

%% IMPORTANT! The old "\acknowledgment" command has be depreciated. It was
%% not robust enough to handle our new dual anonymous review requirements and
%% thus been replaced with the acknowledgment environment. If you try to 
%% compile with \acknowledgment you will get an error print to the screen
%% and in the compiled pdf.
\begin{acknowledgments}
We acknowledge the support of the CUHK Central High-Performance Computing Cluster, on which the simulations in this work have been performed. 
This work was partially supported by grants from the Research Grants Council of Hong Kong (Project No. CUHK 14306419), the Croucher Innovation Award from the Croucher Foundation Hong Kong, and the Direct Grant for Research from the Research Committee of The Chinese University of Hong Kong. 
P.C.-K.C. acknowledges support from NSF Grant PHY-2020275 (Network for Neutrinos, Nuclear Astrophysics, and Symmetries (N3AS)).
\end{acknowledgments}

%% To help institutions obtain information on the effectiveness of their 
%% telescopes the AAS Journals has created a group of keywords for telescope 
%% facilities.
%
%% Following the acknowledgments section, use the following syntax and the
%% \facility{} or \facilities{} macros to list the keywords of facilities used 
%% in the research for the paper.  Each keyword is check against the master 
%% list during copy editing.  Individual instruments can be provided in 
%% parentheses, after the keyword, but they are not verified.

%\vspace{5mm}
%\facilities{HST(STIS), Swift(XRT and UVOT), AAVSO, CTIO:1.3m,CTIO:1.5m,CXO}

%% Similar to \facility{}, there is the optional \software command to allow 
%% authors a place to specify which programs were used during the creation of 
%% the manuscript. Authors should list each code and include either a
%% citation or url to the code inside ()s when available.

\software{
\texttt{XNS} \citep{2011A&A...528A.101B,2014MNRAS.439.3541P,2015MNRAS.447.2821P,2017MNRAS.470.2469P,2020A&A...640A..44S},
\texttt{Gmunu} \citep{2020CQGra..37n5015C,2021MNRAS.508.2279C,2022ApJS..261...22C}
}

%% Appendix material should be preceded with a single \appendix command.
%% There should be a \section command for each appendix. Mark appendix
%% subsections with the same markup you use in the main body of the paper.

%% Each Appendix (indicated with \section) will be lettered A, B, C, etc.
%% The equation counter will reset when it encounters the \appendix
%% command and will number appendix equations (A1), (A2), etc. The
%% Figure and Table counter will not reset.
%\appendix

%% For this sample we use BibTeX plus aasjournals.bst to generate the
%% the bibliography. The sample631.bib file was populated from ADS. To
%% get the citations to show in the compiled file do the following:
%%
%% pdflatex sample631.tex
%% bibtext sample631
%% pdflatex sample631.tex
%% pdflatex sample631.tex

\bibliography{references}{}
\bibliographystyle{aasjournal}

%% This command is needed to show the entire author+affiliation list when
%% the collaboration and author truncation commands are used.  It has to
%% go at the end of the manuscript.
%\allauthors

%% Include this line if you are using the \added, \replaced, \deleted
%% commands to see a summary list of all changes at the end of the article.
%\listofchanges

\end{document}

% End of file `sample631.tex'.
