\section{Experimental Results}\label{exp}

% \textcolor{gray}{Di: I feel we jumped into these details too quickly. Can we add one sentence to introduce the main focus of experiment part.} 
%This section presents the evaluation of our policy reuse framework. 
% In order to create various traffic scenarios, we collected a proprietary dataset of hourly communication traffic from an existing communication network over one week. In this dataset, there are 21 sectors and each sector has 4 cells with different frequencies and capacities. This dataset was used to tune a proprietary system-level network simulator so that it mimicks the real-world traffic conditions. 


We collected a proprietary dataset of hourly communication traffic from an existing communication network over one week. In this dataset, there are 21 sectors and each sector has 4 cells with different frequencies and capacities. This dataset was used to tune a proprietary system-level network simulator so that it mimics real-world traffic conditions. Details about the dataset and simulator are presented in Section~\ref{sec:traffic-analysis} and~\ref{sec:simulator}, respectively. Section~\ref{sec:baselines} lists the baselines that we use to compare our proposed method. Section~\ref{sec:policy-bank-analysis} constructs and analyses the policy bank that we obtained using simulated scenarios. Finally, Section~\ref{sec:policy-sector-training} and~\ref{sec:policy-sector-result} present our experiment with the policy selector and its performance evaluation.

%Our experiment studies load balancing on these traffic scenarios. Section~\ref{sec:traffic-analysis} presents these traffic scenarios and their clustering in detail. 
%
%To enable the RL algorithm to interact with a simulated environment, we recreate the real-world scenarios within a proprietary system-level network simulator by tuning the simulator's configurations so that it mimics real-world traffic conditions. 
%For each scenario, the configuration that results in a traffic condition matching the closest to the real-world data is kept. 
% More details about the simulator are presented in Section~\ref{sec:simulator}.

%Section~\ref{sec:baselines} presents the baseline methods we use to compare the effectiveness of our proposed framework. In Section~\ref{sec:policy-bank-analysis}, we investigate the performance of the policies in our policy bank on each traffic scenario. The high performance of policies on novel traffic scenarios similar to the traffic scenario seen during training further supports the need for a policy selector. Finally, we explain in detail our experiments using the policy selector in Section~\ref{sec:policy-sector-training} and present its results in Section~\ref{sec:policy-sector-result}.

\begin{figure}[t]
\centering
\begin{tabular}{c}
\includegraphics[width=0.7\columnwidth]{Figures/21sec-rl-analysis_train_group2.png} \\
(a) Average reward on the training scenarios \\
~\\
\includegraphics[width=0.7\columnwidth]{Figures/21sec-rl-analysis_test_group2.png} \\
(b)  Average reward on the testing scenarios \\
\end{tabular}
    \caption{Comparison of the average reward over one week between policies in the policy bank $\Pi$ (orange), the policy $\tilde{\pi}$ trained on all scenarios in $\mathcal{X}$ (blue), and BasicLB (green) across training and testing scenarios. For the policy bank evaluation (orange), we show the mean and indicate the minimum and maximum with error bars.}
    \label{fig:rl_policy_21_sec}
\end{figure}

\subsection{Traffic clustering and analysis}
\label{sec:traffic-analysis}
%\textcolor{gray}{Jimmy: I think we need to add 1 sentence here stating the overall purpose of this subsection. For example, something like "In this section we analyze the three predominant groups of network conditions...blah blah blah, by analyzing the distribution of traffic among various cells....". I think you can write something more accurate than me. } 
% Load balancing aims to move UEs from high traffic cells to low traffic cells to improve the overall efficiency of the network. %Various factors related to the environment, time and the system hardware can describe the traffic condition of a cell. 
%In this section, we conduct preparatory analysis to construct a diverse set of training scenarios with which we use to build our policy bank. 
To identify different types of traffic, we applied the K-Means clustering algorithm to the daily traffic condition described by three traffic-related factors: the number of active UEs, network throughput, and 
% resource utilization in terms of 
the percentage of physical resource block (PRB) used. Three interpretable groups emerge from the clustering process: %After 100 iterations, we divide our dataset into three interpretable groups: 
% The communication traffic can be affected by many factors such as the UE distribution, time of day, and system hardware. For this analysis, we consider the same factors used in state description of our MDP: (1) the number of active UEs, (2) network throughput, and (3) the bandwidth utilization in terms of the percentage of physical resource block (PRB) used. Using K-Means clustering on the daily traffic condition, we can divide our dataset into three interpretable groups:
% \textcolor{blue}{tianyu: I would just remove the "relatively" in all the items. In addition, I think figure 2 should be moved to this page.}
(Group 1) High traffic on the first cell; 
(Group 2) High traffic on the fourth cell;  and 
(Group 3) Low traffic in general.
Figure~\ref{fig:traffic_21_sec} shows how traffic varies across the 21 different traffic scenarios across these three groups. A cell has a high volume of traffic when it has a large number of active UEs, low throughput, and high utilization. 

%We conducted further analysis on the temporal aspect of the traffic condition for each group. %Figure~\ref{fig:avg_traffic_group} shows the traffic condition during low and high utilization hours. 
%We defined low utilization hours to be between midnight and 8 AM, which matches approximately to the time when most human consumers are inactive, and high utilization hours to be the other hours, when they are more likely to be actively accessing the network. From the figure, we observe that the magnitude of each traffic-related factor shifts significantly between the two time periods.

%\begin{figure}[ht]
%\centering
    % \subfigure[Average traffic.]{\includegraphics[width=\linewidth]{Figures/traffic_all_time_21sec.png}\label{fig:avg_traffic_group_all_time}}
%    \hspace{0.01\linewidth}
%    \subfigure[Average traffic during low utilization hours (0:00-8:00)]{\includegraphics[width=\linewidth]{Figures/traffic_non_peak_time_21sec.png}\label{fig:avg_traffic_group_non_peak}}
%    \hspace{0.01\linewidth}
 %   \subfigure[Average traffic during high utilization hours (8:00-24:00)]{\includegraphics[width=\linewidth]{Figures/traffic_peak_time_21sectors.png}\label{fig:avg_traffic_group_peak}}
 %   \caption{Average traffic over one week for each group at low and high utilization hours, illustrating the general daily traffic pattern shared across each group.}\label{fig:avg_traffic_group}
%\end{figure}

\begin{figure}[t]
\centering
\begin{tabular}{c}
\includegraphics[width=0.8\linewidth]{Figures/training-result-reward.png} \\
(a) Average reward on the training scenarios \\
~\\
\includegraphics[width=0.8\linewidth]{Figures/testing-result-reward.png} \\
(b) Average reward on the testing scenarios
\end{tabular}
\caption{Comparison of the average reward over 6 days. Our policy reuse framework with the policy selector (blue) achieves the closest performance to our upper bound BEST-$\pi$ (red) on average. For the training scenarios, it is exactly the same as BEST-$\pi$.}
\label{fig:reward_results}
\end{figure}

\subsection{Simulator}
\label{sec:simulator}
%\textcolor{gray}{Jimmy: Again, adding a high level sentence here would be good. Something like "Key to our experimental framework is a system-level simulator that allows our RL agent to collect interaction experiences for training...."} 
%A key to our experiments is the use of a communication network simulator to mimic real-world data, allowing our RL algorithm to interact with our collected traffic scenarios.
We use a proprietary system-level network simulator, as in~\cite{kang2021hrl}. 
%An example of a simulation scenario is shown in Figure~\ref{fig:sls}. 
This simulator emulates 4G/5G communication network behaviours, and supports various configurations that allow us to customize the traffic condition. In our experiment, we fix the number of base stations to 7, with one base station in the center of the layout. Each base station has 3 sectors and each sector has 4 cells with different carrier frequencies that are identical across all sectors and base stations. We vary the number and distribution of UEs, the packet size and request interval such that the simulation traffic condition at the north-east sector of the center base station matches  the real-world data presented. In our experiments, we aim to balance the load in this particular sector. Our RL policies are only aware of the control parameters and the traffic condition in this sector. 
%Our evaluation is also based on the performance of this sector.

To mimic real-world data, a fraction of the UEs are
% \textcolor{blue}{tianyu:it's what we do right? so just use "a portion of UEs IS ...", also remove the "the" before UEs.} 
uniformly concentrated at specified regions while the remaining are uniformly distributed across the environment. These dense traffic locations change at each hour. All UEs follow a random walk process with an average speed of 3~m/s. The packet arrival follows a Poisson process with variable size between 50~Kb to 2~Mb and inter-arrival time between 10 to 320~ms. Both are specified at each hour to create the desired traffic condition.
%\begin{figure}[ht]
%    \centering
%        \vspace{1mm}
%    \includegraphics[width=0.8\linewidth]{Figures/sls_simulator.png}
%    \caption{An example of a simulation scenario with 7 base stations, each with 3 sectors and 4 cells on each sector. Active and idle UEs connected to different cells are indicated by different shapes and colors.}
%    \label{fig:sls}
%\end{figure}

\subsection{Baselines}
\label{sec:baselines}
To showcase the effectiveness of the proposed method, we compare our solution with the following baselines: 
% \textcolor{gray}{tianyu: Maybe we should include the performance on learning from scratch to further showcase that we perform reasonably well, i.e. not too far away from the learning from scratch method.}
\textbf{Rule-based load balancing (BasicLB)} uses a fixed set of LB parameter values for all traffic scenarios. \textbf{Adaptive rule-based load balancing} (AdaptLB)~\cite{yang2012high} changes the LB parameter values based on the load status of the cells. \textbf{Random policy selection (RAND-$\pi$)} randomly selects a policy in the policy bank $\Pi$ at the beginning of every day. \textbf{Best policy selection (BEST-$\pi$)} selects the best policy based on the performance of all policies in $\Pi$ on the unseen scenarios in ${X}'$ for the whole week. \textbf{New policy trained on the unseen scenario (NEW-$\pi$)} directly trains a new RL policy on the unseen traffic scenario from scratch. BEST-$\pi$ is the best possible performance obtainable form one policy in the policy bank
%, as shown show in Figure~\ref{fig:rl_policy_21_sec_test} 
and it is not a feasible solution to deploy on a real network due to the use of exhaustive search. Similarly, NEW-$\pi$ is another upper bound on performance, and it is also not feasible if the RL agent is not allowed to learn from scratch on an unseen traffic scenario.


\subsection{Policy bank construction and analysis}
% \subsection{Policy bank analysis}
\label{sec:policy-bank-analysis}

%In this section, we conduct preliminary investigations on the performance of the RL policies in the policy bank
% \textcolor{blue}{tianyu: a "what" RL policy? maybe change it to "... performance of the constructed policy bank"?} 
%on unseen traffic scenarios. 
To ensure that our policy bank contains a diverse selection of policies trained from all types of traffic, we randomly select 3 traffic scenarios from each group introduced in Section~\ref{sec:traffic-analysis} to form our set of 9 training scenarios $\mathcal{X}$ and use the remaining 12 scenarios $\mathcal{X}'$ for testing. Following the formulation in Section~\ref{sec:rl-formulation}, we train one PPO policy for each $X\in \mathcal{X}$, creating a policy bank $\Pi$. The reward $R_t$ is the weighted average of the performance metrics defined in Section~\ref{sec:performance-metrics}.
% :
% \[
%     R_t = \frac{0.25}{10.7} G_{avg} + \frac{4}{4.9} G_{min} + \frac{0.5}{2.4(1+G_{sd})} + 0.2 G_{cong}.
% \]
% Because these metrics have values at different scales, we obtain these weights by searching exhaustively through a finite set of weight combinations and these weights were chosen due to their empirical performance for training an RL policy. 
% Note that we use the reciprocal of the shifted $G_{sd}$ so that maximizing the reward minimizes $G_{sd}$.
The weights are selected according to the empirical performance and they are correlated with the magnitude of the metrics. 
Note that we use the reciprocal of $G_{sd}$ so that maximizing the reward minimizes $G_{sd}$.
We also construct another RL policy $\tilde{\pi}$ trained on all scenarios in $\mathcal{X}$ for comparison. This is done by collecting interaction experience on each of the scenarios in parallel at each iteration in the learning process. All policies are trained for 200K interactions.
% with a decay $\lambda=0.97$. 
We use the PPO implementation in the Stable-Baseline 3~\cite{stable-baselines3} Python package. 

% The following was moved from the Method section:
We model the policy selector by a feed-forward neural network classifier with 3 hidden layers (with 128, 64, and 32 neurons, respectively), each preceded by a batch normalization and followed by a rectified linear unit activation. The output layer uses a softmax activation. The architecture hyperparameters were chosen using cross-validation.
% and 1 output layer.
% \textcolor{blue}{tianyu:is it 4 hidden layers or 3 layers with 1 output layer? I think maybe just be a bis specific here by saying "3 hidden layers and 1 output layer" to avoid confusion.)} 
% Each of the first three layers follows a batch normalization. The number of neurons for these layers are 128, 64 and 32, respectively.  
% \textcolor{gray}{tianyu: did we use cross validation to choose the network structure? if so maybe add it.} 
% We use rectified linear unit activation for the first three layers and softmax for the last layer. The number of layers, the number of neurons for each layer and the activation function were chosen using cross-validation.

%\textcolor{gray}{Jimmy: I feel the wording ``picking any policy" is not very clear, because it sounds like you are choosing a policy in the policy bank. I think it would be better to say something along the lines of ``the average performance of executing each policy in the policy bank on the target scenario"} 
Figure~\ref{fig:rl_policy_21_sec} illustrates the performance of executing each policy in the policy bank $\Pi$ in comparison with $\tilde{\pi}$ and the BasicLB method. We observe that, for some scenarios, the minimum possible average reward resulted from an RL policy in $\Pi$ lies
% \textcolor{blue}{tianyu: lies} 
much lower than the average reward resulted from the BasicLB. This supports the assumption that an RL policy trained on one scenario may not generalize well to another, and implies that randomly choosing a policy from the policy bank can significantly degrade the performance for some scenarios. On the other hand, the maximum possible average reward from an RL policy in $\Pi$ is always higher than the average reward resulted by $\tilde{\pi}$ and BasicLB even for the test scenarios. Furthermore, $\tilde{\pi}$ under-performs BasicLB for some test scenarios such as 12 and 14. This indicates that with careful selection of a policy trained from an individual scenario, we can achieve significant improvement over a policy trained on multiple scenarios and BasicLB. The next sections will present our results with the policy selector.



%\begin{figure}[ht]
%\centering
%    \hspace{0.01\linewidth}
%    \subfigure[Average reward on the training scenarios]{\includegraphics[width=\linewidth]{Figures/21sec-rl-analysis_train_group2.png}\label{fig:rl_policy_21_sec_train}}
%    \hspace{0.01\linewidth}
%    \subfigure[Average reward on the testing scenarios]{\includegraphics[width=\linewidth]{Figures/21sec-rl-analysis_test_group2.png}\label{fig:rl_policy_21_sec_test}}
    %\caption{Comparison of the average reward over one week for choosing any policy in the policy bank $\Pi$ against a policy $\tilde{\pi}$ trained on all scenarios in $\mathcal{X}$ and BasicLB. The blank line shows the minimum and maximum average reward.}
%    \caption{Comparison of the average reward over one week between policies in the policy bank $\Pi$ (orange), the policy $\tilde{\pi}$ trained on all scenarios in $\mathcal{X}$ (blue), and BasicLB (green) across training and testing scenarios. For the policy bank evaluation (orange), we show the mean, and indicate the minimum and maximum with error bars.}
%    \label{fig:rl_policy_21_sec}
%\end{figure}

%To further understand the correlation between the RL policy and the traffic condition, we conduct an  experiment that verifies whether policies trained on similar traffic conditions result in higher performance. The similarity between individual sectors' traffic patterns is non-trivial to define due to their high volatility. Hence, for simplicity, we use the L2 distance on the average traffic condition during the low and high utilization hours. Table~\ref{tab:similarity-performance} show the smallest number $k$ such that the policy trained on the $k$th most similar traffic scenario is within the top $k$ best performance for the unseen scenarios in $\mathcal{X}'$. Interestingly, we observe that group 2 has the smallest $k$ on average and group 1 has the largest $k$ on average. Referring back to Figure~\ref{fig:rl_policy_21_sec_test}, we see that any policy in $\Pi$ can easily outperform the rule-based method in a group 1 scenario, while it is much harder in a group 2 scenario. This demonstrates the benefit of selecting a policy based on the similarity of the traffic condition between the new scenario and the seen scenarios. 



%\begin{table}[!htbp]
%\centering
%\caption{The smallest number $k$ such that the policy trained on the $k$th most similar traffic scenario is within the top $k$ best performance for the unseen scenarios.}
%\label{tab:similarity-performance}
%\begin{tabular}{|c | c | c|  c | c | c |}
%\hline
%\multicolumn{2}{|c|}{\bf Group 1} & %\multicolumn{2}{c|}{\bf Group 2} &
%\multicolumn{2}{c|}{\bf Group 3}\\
%\hline
%Scenario  & $k$  & Scenario  & $k$  & Scenario &  $k$ %\\
%\hline
%3 & 3 & 8 & 1 & 16 & 1\\
%4 & 4 & 12 & 2 & 17 & 2 \\
%5 & 3 & 13 & 1 & 18 & 2 \\
%7 & 3 & 14 & 1 & 19 & 3\\
%\hline
%\end{tabular}
%\end{table}

\subsection{Policy selector training}
% \subsection{Policy selector experiment setup}
\label{sec:policy-sector-training}
We now describe the training process of our policy selector. After constructing the policy bank $\Pi$ as in Section~\ref{sec:policy-bank-analysis}, we run BasicLB and each policy $\pi\in \Pi$ on each of the training scenarios in $\mathcal{X}$ to collect the data used to train the policy selector.
% We now describe implementation details of our policy selection method. 
%On the very first day in the simulation scenarios, our policy selector has no data that can be used to select a policy. Therefore, we use a rule-based method to balance the load on the first day. 
% The data that we have collected to train our policy selector is generated by running BasicLB and each policy in the policy bank $\Pi$ on each of the 9 traffic scenarios in $\mathcal{X}$. 
Specifically, we run each policy $\pi\in \Pi$ on each scenario $X\in\mathcal{X}$ for one week and we collect the traffic condition data at each hour. In addition, we repeat this process by running BasicLB on each each scenario $X\in\mathcal{X}$ for one week. We use the data generated by BasicLB as part of the training set since we need to rely on the rule-based method to perform load balancing on the first day, as there is no data that can be used to select a policy. 
% \textcolor{gray}{tianyu: I think here we need to make it clear on how we used both rule-based lb and the policy in the policy bank to collect data, i.e. when do we use what policy} 
In total, we have gathered 15.12K samples corresponding to the hourly traffic condition. These samples are reformatted using a sliding window algorithm to create $T=24$ hour data samples. By randomly selecting 30\% of the samples as our validation set, we use cross-validation to choose
% \textcolor{blue}{tianyu:no the here} 
hyperparameters of the 
% \textcolor{blue}{tianyu:, and need to add an "of the" here} 
policy selector, as discussed in~\ref{sec:policy-selector}.

During evaluation, we bring our policy selector online. For each evaluation scenario,
% in our test set $\mathcal{X}'$
we first run BasicLB to obtain one day of data to initiate the policy selection process. Then, at the beginning of each new day, we feed the data from the previous day to the policy selector to obtain a selected policy to run on that new day. 

\subsection{Performance evaluation}
\label{sec:policy-sector-result}
We evaluate our proposed policy reuse framework and the policy selector on fixed and transient traffic scenarios.
%This section shows the performance evaluation results of our proposed policy reuse framework and the policy selector in fixed and transient traffic scenarios. 
% The last part explores the possibility of improving the performance by fine-tuning the select policy with a small amount of experience interaction under the new scenario. 

\subsubsection{Fixed traffic scenario}
\label{sec:fix-traffic-scenario}

\begin{table}[t]
\centering
\caption{Average performance over 6 days and all training scenarios.}
\label{tab:train-res-kpis}
\begin{tabular}{l|c c c c c}
\toprule
& Reward & {\bf $G_{avg}$} & {\bf $G_{min}$} & {\bf$ G_{sd}$} & {\bf $G_{cong}$}\\
\midrule
{\bf BEST/NEW-$\pi$} & 0.479 & 3.600 & 2.246 & 1.487 & 0.889\\
\midrule
{\bf BasicLB} & 0.401 & 3.033 & 1.680 & 2.190 & 0.837\\
{\bf AdaptLB} & 0.438 & 3.228 & 1.990 & 1.851 & 0.847\\
{\bf RAND-$\pi$} & 0.447 & 3.425 & 2.013 & 1.724 & 0.862\\
{\bf Policy selector} & {\bf 0.479} & {\bf 3.600} & {\bf 2.246} & {\bf 1.487} & {\bf 0.889}\\
\bottomrule
\end{tabular}
\end{table}

\begin{table}[t]
\centering
\caption{Average performance over 6 days and all testing scenarios.}
\label{tab:test-res-kpis}
\begin{tabular}{l|c c c c c}
\toprule
& Reward & {\bf $G_{avg}$} & {\bf $G_{min}$} & {\bf $G_{sd}$} & {\bf $G_{cong}$}\\
\midrule
{\bf BEST-$\pi$} & 0.452 & 3.399 & 2.016 & 1.680 & 0.887\\
{\bf NEW-$\pi$} & 0.456 & 3.365 & 2.057 & 1.631 &	0.889\\
\midrule
{\bf BasicLB} & 0.403 & 3.036 & 1.646 & 2.204 & 0.854\\
{\bf AdaptLB} & 0.422 & 3.144 & 1.834 & 1.936 & 0.847\\
{\bf RAND-$\pi$} & 0.426 & 3.245 & 1.847 & 1.822 & 0.855\\
{\bf Policy selector} & {\bf 0.446} & {\bf 3.355} & {\bf 2.010} & {\bf 1.692} & {\bf 0.867}\\
\bottomrule
\end{tabular}
\end{table}

This experiment tests each scenario in $\mathcal{X}\bigcup\mathcal{X}'$ independently for a simulation period of one week.
% This experiment is conducted on all training and testing traffic scenarios for one week. 
For all methods, including the baselines, BasicLB is applied on the first day. Tables~\ref{tab:train-res-kpis} and~\ref{tab:test-res-kpis} shows the comparison of the average performance over the remaining 6 days. Overall, our policy selector outperforms BasicLB or AdaptLB by 20.33\% and 9.84\%, respectively, on the training scenarios ($\mathcal{X}$), and by 10.26\% and 5.24\%, respectively, on the test scenarios ($\mathcal{X}'$). Furthermore, it achieves on average the closest performance to BEST-$\pi$ and NEW-$\pi$ upper bounds compared to the other baselines. 

Recall that BEST-$\pi$ is not a feasible solution to be deployed in a real network as it requires all policies in $\Pi$ to be applied to the scenario. It can be considered as a performance upper bound for the policy reuse framework. Similarly, NEW-$\pi$, which trains a new RL policy on the unseen traffic scenario, can also be considered as another performance upper bound. For the training scenarios, NEW-$\pi$ and BEST-$\pi$ are equivalent since the policy with the best performance in $\Pi$ for any scenario $X\in \mathcal{X}$ is also the policy trained on $X$. For the testing scenarios, as expected, NEW-$\pi$ is better than BEST-$\pi$, but only by 0.94\% in terms of reward as shown in Table~\ref{tab:test-res-kpis}. Compared to our policy selector, our policy selector achieves an accuracy of 100\%, reaching the two upper bound performance for all training scenarios in $\mathcal{X}$. For the testing scenarios, BEST-$\pi$ and NEW-$\pi$ are on average only 1.21\% and 2.16\% higher than our proposed method, respectively. This demonstrates that our policy reuse framework can efficiently be used to avoid training on unseen scenarios without significant loss in performance.

Figure~\ref{fig:reward_results} shows the detailed performance comparison of the average reward for each scenario.
For certain test scenarios in $\mathcal{X}'$, especially in Group 2, BasicLB or AdaptLB achieves the best performance. 
%As follows from the discussion of Figure~\ref{fig:rl_policy_21_sec_test}, 
Group 2 includes some scenarios that are relatively more difficult to optimize. However, our policy selector can outperform RAND-$\pi$ for all scenarios in Group 2, demonstrating the effectiveness of choosing the policy based on the similarity of the traffic condition. 

% the comparison of the average reward over the remaining 6 days. 
%We observe in Figure~\ref{fig:train-reward-result} that with our policy selector,
% We achieve the upper bound performance, same as Best-$\pi$,
% We achieve the same performance as Best-$\pi$ 
% in the scenarios in the training set $\mathcal{X}$. In fact, our policy selector has near 100\% accuracy on the training and validation dataset mentioned in Section~\ref{sec:policy-sector-training}. In this online experiment, it achieves an accuracy of 100\% for the scenarios in $\mathcal{X}$. 
% However, for the scenarios in $\mathcal{X}'$, 
%as we observe in Figure~\ref{fig:test-reward-result}, 
% and hence obtained a slightly lower average reward compared to BEST-$\pi$.



%\begin{figure}[ht]
%\centering
%    \hspace{0.01\linewidth}
%    \subfigure[Average reward on the training scenarios]{\includegraphics[width=\linewidth]{Figures/training-result-reward.png}\label{fig:train-reward-result}}
%    \hspace{0.01\linewidth}
%    \subfigure[Average reward on the testing scenarios]{\includegraphics[width=\linewidth]{Figures/testing-result-reward.png}\label{fig:test-reward-result}}
%    \caption{Comparison of the average reward over 6 days. Our policy reuse framework with the policy selector (blue) achieves the closest performance to our upper bound BEST-$\pi$ (red) on average. For the training scenarios, it is exactly the same as BEST-$\pi$.}\label{fig:reward_results}
%\end{figure}



 
% Furthermore, our policy selector outperforms RAND-$\pi$ by 7.38\% and 4.36\% on $\mathcal{X}$ and $\mathcal{X}'$, respectively, demonstrating the effectiveness of choosing the policy based on the similarity of the traffic condition. 

\subsubsection{Transient traffic scenario}

This experiment evaluates how our policy reuse framework adapts to a changing traffic condition. We construct a transient traffic scenario $\tilde{X}$ by consecutively running a sequence of random scenarios picked from $\mathcal{X}\bigcup\mathcal{X}'$. Each scenario $\mathcal{X}\bigcup\mathcal{X}'$ is run for 3 consecutive days. We compare our proposed framework, which selects a policy on each day, against its variation which selecting a policy on the first day only. Both use the policy selector to select the policy. Again, BasicLB is applied on the first day. 

Figure~\ref{fig:transient-scenario-result} plots the average reward on each day for 24 days. The vertical grid shows the day on which the scenario changes. As shown in this figure, our framework can chose a suitable policy after it has experienced a new traffic for a day, and its performance compared to BasicLB and AdaptLB is consistent with the result in Section~\ref{sec:fix-traffic-scenario}.
% Depending on the scenario, our framework may not outperform BasicLB or AdaptLB, like at the beginning where scenario 8 was chosen. This is consistent with the result in Section~\ref{sec:fix-traffic-scenario}. 
Although compared to selecting a policy on the first day only, our proposed framework occasionally gets a lower reward on the days when the scenario changes, like on day 7 and 13, it can quickly recover on the next day and achieves a higher performance overall. This demonstrates the merit of our framework, in particular for real traffic scenarios where changes in daily traffic patterns may occur, but not as frequent as in this synthetic scenario $\tilde{X}$. 


\begin{figure}[t]
    \centering
        \vspace{1mm}
    \includegraphics[width=0.8\linewidth]{Figures/online-policy-change.png}
    \caption{The average reward for each day with transient traffic scenario. We change the traffic scenario every 3 days. Our method (blue) may not perform optimally on the first day when the scenario changes, but it can recover quickly on the next day and it outperforms the other baselines overall.}
    \label{fig:transient-scenario-result}
\end{figure}


% \subsubsection{Fast adaptation}

% In this experiment, we allow the selected policy on the first day to be fine-tuned to the new traffic scenario. Specifically, for each test traffic scenario in $\mathcal{X}'$, we run BasicLB on the first day and use the data on the first day to select a policy through our policy selector. We then train a new policy on the first day of the new scenario by initializing the weights \textcolor{blue}{tianyu: of the policy networks} with the weights from the selected policy. 

% Figure~\ref{fig:fast-adapt} shows the average and standard deviation of the learning curves for training a new policy on the new scenario with the policy weights initialized by the weights from the selected policy (selected) and by the worst policy (worst) according to our experiment described in Figure~\ref{fig:rl_policy_21_sec_test}. We also compare this result with training from scratch, which initializes the weights by a random orthogonal matrix. We observe that, in general, the curves for the selected policy start at a much higher reward value, and they converge much faster compared to the other two. Although the curves form the worst policy start also at a higher reward value compared to the curves from learning from scratch, it improves much slower at the first 10K interactions and eventually becomes the one with the lowest reward. As expected, a policy trained on a similar traffic scenario will have more relevant learning that can be transferred to the new scenario, resulting in a faster adaptation to the new scenario. This can be very helpful considering the 

% \begin{figure}[ht]
%     \centering
%         \vspace{1mm}
%     \includegraphics[width=\linewidth]{Figures/fast_adapt_smooth.pdf}
%     \caption{Learning curves for fine-tuning on the test scenarios. We experiment with using the policy selector to choose a trained model for initialization (selected) and using the model with the worst performance on the target test scenario for initialization (worst). The learning curve for training from scratch (scratch) is also shown. For each case, we show the mean curve across all test scenarios as the percentage increase with respect to the initial reward achieved by training from scratch. The channel denotes $\pm0.25$ standard deviation from the mean.}
%     \label{fig:fast-adapt}
% \end{figure}

% Figure~\ref{fig:fast-adapt-avg-reward-test} shows the improvement made by fine-tuning the selected policy after 4k, 8k, 50k, 100k and 200k interaction experiences. We observe that after 4k, the fine-tuned policy can already achieves close if not better reward on all traffic scenarios in $\mathcal{X}'$. 

% \begin{figure}[ht]
%     \centering
%         \vspace{1mm}
%     \includegraphics[width=\linewidth]{Figures/fast-adapt-sectors.png}
%     \caption{Average reward over the first day on each testing scenarios.}
%     \label{fig:fast-adapt-avg-reward-test}
% \end{figure}

% Table~\ref{tab:fast-adapt-avg-kpis-test} shows that, on average, FastAdapt-4k can already outperform BasicLB and AdaptLB by a large margin. In particular, for the reward, it achieves 13.57\% and 11.24\% percent improvement over BasicLB and AdaptLB, respectively.

% \begin{table}[htb]
%     \centering
%     \caption{Average performance over over \textcolor{blue}{tianyu: extra over} the first day and all testing scenarios.\textcolor{blue}{tianyu: Can we make this consistant with table 2 and 3, namely average over 6 days instead of only first day? This way we can compare with the results in table 3}}
%     \label{tab:fast-adapt-avg-kpis-test}
%     \begin{tabular}{l|c c c c c}
% \toprule
% & Reward & {\bf $G_{avg}$} & {\bf $G_{min}$} & {\bf $G_{sd}$} & {\bf $G_{cong}$}\\
% \midrule
% BasicLB & 0.402 & 3.027 & 1.655 & 2.282 & 0.850\\
% AdaptLB & 0.411 & 3.080 & 1.771 & 2.132 & 0.832\\
% FastAdapt-4k & 0.457 & 3.409 & 2.076 & 1.764 & 0.887\\
% % FastAdapt-8k & 0.459 & 3.412 & 2.093 & 1.744 & 0.891\\
% % FastAdapt-50k & 0.466 & 3.443 & 2.166 & 1.707 & 0.890\\
% % FastAdapt-100k & 0.469 & 3.460 & 2.180 & 1.701 & 0.895\\
% % FastAdapt-200k & 0.472 & 3.471 & 2.198 & 1.687 & 0.899\\
% \bottomrule
%     \end{tabular}
% \end{table}