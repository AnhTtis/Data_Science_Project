\documentclass[AMA,STIX1COL]{WileyNJD-v2}
\usepackage{float}

\articletype{Research Article}%

\received{26 April 2016}
\revised{6 June 2016}
\accepted{6 June 2016}

\raggedbottom

\usepackage{blindtext}
\usepackage{subfiles} % Best loaded last in the preamble
\usepackage{algorithm}
\usepackage{algpseudocode}
\usepackage{hyperref}

\usepackage{comment}

% \usepackage[maxbibnames=99,style=numeric,doi=false,isbn=false,url=true,eprint=false,sorting=none]{biblatex}
% \addbibresource{wileyNJD-AMA.bib}
% \usepackage[numbers]{natbib}

\begin{document}
\title{Enrollment Forecast for Clinical Trials at the Planning Phase with Study-Level Historical Data}

\author[1]{Mengjia Yu}

\author[1]{Sheng Zhong*}

\author[1]{Yunzhao Xing}

\author[1]{Li Wang*}

\authormark{Yu \textsc{et al.}}



\address{\orgdiv{Statistical Innovation Group, Data and Statistical Sciences}, \orgname{AbbVie Inc.}, \orgaddress{\state{Illinois}, \country{United States}}}

\corres{Li Wang,
\email{wangleelee@gmail.com}\\
Sheng Zhong, \email{zhongever@gmail.com}
}


\presentaddress{1 N Waukegan Rd, North Chicago, IL, 60064}

\abstract[Abstract]{
Given progressive developments and demands on clinical trials, accurate enrollment timeline forecasting is increasingly crucial for both strategic decision-making and trial execution excellence. 
Naïve approach assumes flat rates on enrollment using average of historical data, while traditional statistical approach applies simple Poisson-Gamma model using time-invariant rates for site activation and subject recruitment. Both of them are lack of non-trivial factors such as time and location.
We propose a novel two-segment statistical approach based on Quasi-Poisson regression for subject accrual rate and Poisson process for subject enrollment and site activation.
The input study-level data is publicly accessible and it can be integrated with historical study data from user's organization to prospectively predict enrollment timeline. The new framework is neat and accurate compared to preceding works. We validate the performance of our proposed enrollment model and compare the results with other frameworks on 7 curated studies. 
}

\keywords{Quasi-Poisson Regression, Poisson Process, Enrollment Forecast, Bootstrap}
\maketitle

\section{INTRODUCTION}

\subfile{sections/introduction}

\section{INPUT DATA AND FORECAST WORKFLOW}

\subfile{sections/dataworkflow}

\section{METHODOLOGY DETAILS}

\subfile{sections/methods}

\section{PERFORMANCE EVALUATION}

\subfile{sections/performance}

\section{DISCUSSION}

\subfile{sections/discussion}

\section*{Disclosure}
This manuscript was sponsored by AbbVie. AbbVie contributed to the design, research, and interpretation of data, writing, reviewing, and approved the content.   All authors are employees of AbbVie Inc. and may own AbbVie stock.

\section*{Data availability statement}

The authors elect to not share data.

%This publication was neither originated nor managed by AbbVie, and it does not communicate results of AbbVie-sponsored Scientific Research. Thus, it is not in scope of the AbbVie Publication Procedure (PUB-100).

%\nocite{*}% Show all bib entries - both cited and uncited; comment this line to view only cited bib entries;
\newpage
% \bibliographystyle{ieeetr}
% \bibliographystyle{plainnat} 
\bibliography{wileyNJD-AMA}
% \printbibliography
\clearpage

\end{document}