\documentclass[../main.tex]{subfiles}

\begin{document}

\subsection{General Consideration and Data Description}

We will apply our study-level prediction model to 7 recently completed studies within our organization and compared the prediction to their actual enrollment duration, which is the period between LSFD (last subject first dose) and
the protocol approval. 

The data for estimating site activation are internally available CTMS data within the author's organization.
In the subject enrollment modelling process, we filtered out interventional studies that are within the same therapeutic area (oncology, neuroscience, immunology, and general medicine) and the same study phases (Phase 2 and/or 3) as the target study to forecast. 
The data for modelling subject enrollment contain both internal and external data that are available on Citeline. The information include therapeutic area, 
study phase,
study duration,
the total number of subjects, 
the total number of sites, 
disease indication 
and the patient population. 
Here, the selected studies must be interventional, Phase 2 and/or 3, completed and with the same therapeutic area, disease indication and patient population as the study to be tested. Our 7 candidate studies for performance testing purpose are curated and prepared by an independent data management team to ensure data objectivity and integrity. The study numbers are masked due to confidential reason.

We also compare the performance of our proposed modeling framework to a previous internally-developed
modeling pipelines, based on the classical Poisson-Gamma models and a new site-level modeller \cite{zhong2022sitelevel}. 

\subsection{Model Performance}
\label{sec:model_performance}
\subsubsection{The approach with linear (fixed) estimates of site opening}
We first apply our model and use the linear (fixed) estimation of $t_{i}$ and $\beta_{i}$ as described in Section~\ref{sec:fixed_estimate}. 
In Table~\ref{tab:table1}, we report the actual enrollment duration followed by the predicted value and the corresponding prediction error. We also provide the 95\% Prediction Interval (PI) and the coverage status within 95\% PI, +/- 1 month, +/- 2 months and +/- 3 months. In order to better investigate the source of modelling validity, we also list the actual FSFD (first subject first dose) and the predicted FSFD together with the corresponding 95\% PI to reflect the prediction accuracy on the site initiation $t_i$. 

From the predictions of enrollment we can find that the linear approach gives more accurate estimation on Studies 2-7 compared to that of Study 1.
The 95\% PI covers 4 studies, but their PI windows are also wide and may be lack of usability. Hence, we should look at the fixed-window coverage as well.
The 3-month coverage overall shows that our prediction of enrollment performs well. When we compare the actual FSFD (first subject first dose) with the predicted FSFD, the forecasts from the linear (fixed) approach are generally aggressive, which indicates a negative bias from the site activation portion. 

\begin{table}[h]
% \small
\begin{tabular}{p{0.04\linewidth}|p{0.065\linewidth}p{0.065\linewidth}p{0.065\linewidth}p{0.072\linewidth}p{0.056\linewidth}p{0.056\linewidth}p{0.056\linewidth}p{0.056\linewidth}|p{0.055\linewidth}p{0.055\linewidth}p{0.06\linewidth}}
\hline
Study & Actual Enrollment (mo) & Predicted Enrollment  (mo) & Prediction Error (mo) & 95\%PI (mo)    & Within 95\%PI & Within +/-1 month & Within +/-2 month & Within +/-3 month & Actual FSFD (mo) & Predicted FSFD (mo) & 95\%PI (mo)   \\ \hline
1     & 26.4                            & 19.3                               & -7.0             & (15.5,24.0) & NO       & NO         & NO         & NO         & 5.3          & 4.6                   & (4.1,6.2) \\
2     & 17.5                            & 18.9                               & 1.4              & (15.8,22.4) & YES        & NO         & YES          & YES          & 7.9          & 4.2                   & (3.1,5.3) \\
3     & 11.3                            & 9.2                                & -2.1             & (8.1,10.3)  & NO       & NO         & NO         & YES          & 6.3          & 3.1                   & (3.0,3.8) \\
4     & 10.0                            & 8.7                                & -1.3             & (7.7,9.9)   & NO       & NO         & YES          & YES          & 7.0          & 3.2                   & (3.0,4.0) \\
5     & 21.1                            & 20.1                               & -1.0             & (16.0,26.5) & YES        & NO         & YES          & YES          & 8.4          & 4.1                   & (3.1,5.2) \\
6     & 16.1                            & 19.0                               & 2.9              & (14.9,24.1) & YES        & NO         & NO         & YES          & 6.0          & 5.4                   & (4.1,6.7) \\
7     & 10.8                            & 11.2                               & 0.3              & (8.0,18.1)  & YES        & YES          & YES          & YES          & 5.4          & 4.5                   & (4.0,5.6) \\ \hline
\end{tabular}
\caption{\label{tab:table1} Study-level prediction performance using the linear (fixed) approach.}
\end{table}

\subsubsection{The approach with perturbation-based approaches of site opening}
Next we apply the model using perturbation-based simulation introduced in Section ~\ref{sec:perturbation_estimate}. A similar display of prediction on the 7 studies is provided in Table~\ref{tab:table2}.

The predictions of enrollment shows that our perturbation-based approach generally performs well on Studies 2-7 compared to Study~1. The 95\% PI coverage and 2-month coverage show that our prediction captures variability from data meanwhile providing accurate estimates. 
When we compare the actual FSFD with the predicted FSFD, the perturbation-based predictions are still aggressive, but the corresponding 95\% PIs are wider compared to the linear (fixed) approach.   

\begin{table}[h]
% \small
\begin{tabular}{p{0.04\linewidth}|p{0.065\linewidth}p{0.065\linewidth}p{0.065\linewidth}p{0.072\linewidth}p{0.056\linewidth}p{0.056\linewidth}p{0.056\linewidth}p{0.056\linewidth}|p{0.055\linewidth}p{0.055\linewidth}p{0.065\linewidth}}
\hline
Study & Actual Enrollment  (mo) & Predicted Enrollment  (mo) & Prediction Error (mo) & 95\%PI (mo)    & Within 95\%PI & Within +/-1 month & Within +/-2 month & Within +/-3 month & Actual FSFD (mo) & Predicted FSFD (mo) & 95\%PI (mo)   \\ \hline
1     & 26.4                            & 18.5                               & -7.9             & (14.7,24.9) & NO       & NO         & NO         & NO         & 5.3          & 3.3                   & (1.7,6.9) \\
2     & 17.5                            & 18.8                               & 1.3              & (15.3,23.1) & YES        & NO         & YES          & YES          & 7.9          & 4.1                   & (1.3,7.3) \\
3     & 11.3                            & 9.7                                & -1.5             & (6.7,15.6)  & YES        & NO         & YES          & YES          & 6.3          & 3.3                   & (1.1,9.5) \\
4     & 10.0                            & 9.4                                & -0.6             & (6.1,15.9)  & YES        & YES          & YES          & YES          & 7.0          & 4.1                   & (1.2,10.5) \\
5     & 21.1                            & 20.1                               & -1.1             & (16.0,26.5) & YES        & NO         & YES          & YES          & 8.4          & 4.2                   & (1.2,7.2) \\
6     & 16.1                            & 19.3                               & 3.2              & (15.4,25.2) & YES        & NO         & NO         & NO         & 6.0          & 5.5                   & (4.1,9.4) \\
7     & 10.8                            & 10.3                               & -0.5             & (6.0,19.7)  & YES        & YES          & YES          & YES          & 5.4          & 3.6                   & (1.4,7.7)
\\ \hline
\end{tabular}
\caption{\label{tab:table2} Study-level prediction performance using perturbed approach.}
\end{table}

Table~\ref{tab:table3} provides a comparison between the two approaches in a summary level. The median length of 95\% PI for the perturbed approach is wider, and its prediction is more accurate as the median prediction error is smaller. They demonstrate the improvement of the perturbed approach by capturing the positive correlation between site initiation time and site opening rate. Besides, except for the 3-month coverage (of Study 6), the perturbation approach has better coverage rates than the fixed approach. We conclude that the two approaches are comparable, but the perturbation model turns to provide slightly better point prediction and wider PI on enrollment. 

In practice, a slightly narrower prediction interval that is about +/-3 month may be preferable to provide guidance for operational functions. One can use a lower level for PI calculation. For example, the median lengths of 70\% and 80\% PIs from the perturbed estimation approach are 5.3 and 6.9, respectively, while 5 out of 7 studies are covered by their own 70\% PIs as well as 80\% PIs. Hence, from application perspective, the trade-off between PI coverage and PI length need to be evaluated case by case. 


\begin{table}[h]
\centering
\begin{tabular}{c|c|c|cccc}
\hline
                   & 95\%PI length                    & Prediction Error                 & \multicolumn{4}{c}{Coverage Rate}    \\
                   & \multicolumn{1}{c|}{median (mo)} & \multicolumn{1}{c|}{median (mo)} & 95\%PI & +/-1 month & +/-2 month & +/-3 month \\ \hline
Fixed estimation    & 8.46                             & -1.02                            & 57\%   & 14\%    & 57\%    & 86\%    \\
Perturbed estimation & 9.78                             & -0.60                            & 86\%   & 29\%    & 71\%    & 71\%    \\ \hline
\end{tabular}
\caption{\label{tab:table3} Comparison of prediction performance between the two proposed approaches.}
\end{table}


\subsection{Comparison with existing models}

We compare our proposed perturbation approach with the site-level model and the traditional model described in Zhong et al. \cite{zhong2022sitelevel}. 
To ensure fairness, we will focus on the same 7 studies that are Studies 6, 13, 14, 15, 16, 24 and 25 in Zhong et al. \cite{zhong2022sitelevel}.
Table~\ref{tab:table4} lists the means and medians of 95\% PI lengths and absolute prediction errors in month (mo), which are followed by the coverage rates of 95\%PI, +/-1 month, +/-2 month and +/-3 month. Detailed predictions of the Site-level model and the traditional model can be found in Table~\ref{tab:table5} and Table~\ref{tab:table6}, respectively.

\begin{table}[h]
\begin{tabular}{c|c|c|c|c|cccc}
\hline
                   & \multicolumn{2}{c|}{95\%PI length (mo)} & \multicolumn{2}{c|}{\begin{tabular}[c]{@{}c@{}}Absolute Prediction \\ Error (mo)\end{tabular}} & \multicolumn{4}{c}{Coverage Rate}    \\\cline{2-9} 
                   & median              & mean              & median      & mean  & 95\%PI & +/-1 month & +/-2 month & +/-3 month \\ \hline
Perturbation approach & 9.8                 & 10.1              & 1.3                       & 2.3                     & 86\%   & 29\%    & 71\%    & 71\%    \\
Site-level model   & 17.6                & Inf               & 1.1                       & 4.3                     & 100\%   & 29\%    & 57\%    & 71\%    \\
Traditional model  & 1.6                 & 2.4               & 4                         & 5.3                     & 29\%   & 29\%    & 29\%    & 43\%    \\ \hline
\end{tabular}
\caption{\label{tab:table4} Comparison among the proposed Study-level model, the Site-level model and the traditional model in Zhong et al. \cite{zhong2022sitelevel}.}
\end{table}

From Table~\ref{tab:table4}, we can draw the following observation.
The traditional model has the narrowest 95\% PI length, but its prediction deviation (in terms of absolute prediction error) is the largest and the coverage rates in terms of all 4 criteria are the lowest. So this model tends to provide biased predictions and fails to capture the variability from data.
The site-level model provides the widest 95\% PI so that the corresponding coverage rate reaches 100\%. However, as the authors mentioned, the site-level model take into account many sources of randomness, which is accumulated from the three-segment framework consisting of country start-up, site initiation and mixed-effect Poisson regression for subject enrollment, as well as Monte-Carlo simulation of recruitment. Hence, a particular simulation of site-level model may accidentally fall into a stagnant side and cannot enroll enough samples within a long enough period (2000 days in this case). So it leads to the infinity of the upper PI bound for Study 6. But the accuracy with respect to median absolute prediction error and 1$\sim$3 month coverage rates indicate that the site-level model in general can provide good point estimation.
Our proposed model using the perturbation approach can generate moderate 95\% PI while its absolute prediction error is comparable with the site-level model. Our 2$\sim$3 month coverage rates are the highest among all three methods. Therefore, our proposed method preserves prediction precision and offers a good balance between PI length and PI coverage.

\begin{remark}
We would like to compare the site-level with the study-level framework in depth from the models themselves. First, it needs to emphasize that the difference from data source should be considered. In fact, it is difficult to get access to site-level data in practice, so the number of underlying historical studies for the site-level model is often much less than that for the study-level model. Therefore, the 95\% PI from site-level model is wide. Second, the site-level model is more complex from methodology perspective. The site-level framework consists of three segments (i.e., country start-up, site activation and subject enrollment), which can fit for a broad-sense enrollment process. Due to the lack of studies with site-level data in practice, the numeric performance may not have been pushed to the extreme. As a result, practitioners should evaluate data availability before determining which model to use.
\end{remark}


\begin{table}[h]
\begin{tabular}{p{0.05\linewidth}|p{0.095\linewidth}p{0.095\linewidth}p{0.095\linewidth}p{0.095\linewidth}p{0.07\linewidth}p{0.07\linewidth}p{0.07\linewidth}p{0.07\linewidth}}
\hline
Study & Actual Enrollment  (mo) & Predicted Enrollment  (mo) & Prediction Error (mo) & 95\%PI (mo) & Within 95\%PI & Within +/-1 month & Within +/-2 month & Within +/-3 month \\ \hline
1     & 26.4                    & 20.5                       & -5.9                  & (13.7,31.3) & YES           & NO             & NO                & NO                \\
2     & 17.5                    & 20.2                       & 2.7                   & (13.5,35.4) & YES           & NO             & NO                & YES               \\
3     & 11.3                    & 10.1                       & -1.1                  & (6.4,17.3)  & YES           & NO             & YES               & YES               \\
4     & 10.0                    & 10.0                       & -0.0                  & (6.9,17.3)  & YES           & YES            & YES               & YES               \\
5     & 21.1                    & 22.1                       & 1.0                   & (15.7,36.7) & YES           & NO             & YES               & YES               \\
6     & 16.1                    & 35.0                       & 18.9                  & (16.2,Inf)  & YES           & NO             & NO                & NO                \\
7     & 10.8                    & 11.6                       & 0.8                   & (8.11,17.7) & YES           & YES            & YES               & YES               \\ \hline
\end{tabular}
\caption{\label{tab:table5} Performance of the Site-level model Zhong et al. \cite{zhong2022sitelevel}.}
\end{table}



\begin{table}[h]
\begin{tabular}{p{0.05\linewidth}|p{0.095\linewidth}p{0.095\linewidth}p{0.095\linewidth}p{0.095\linewidth}p{0.07\linewidth}p{0.07\linewidth}p{0.07\linewidth}p{0.07\linewidth}}
\hline
Study & Actual Enrollment  (mo) & Predicted Enrollment  (mo) & Prediction Error (mo) & 95\%PI (mo) & Within 95\%PI & Within +/-1 month & Within +/-2 month & Within +/-3 month \\ \hline
1     & 26.4                    & 17.7                       & -8.7                  & (16.9,18.4) & NO            & NO                & NO                & NO                \\
2     & 17.5                    & 13.5                       & -4.0                  & (12.7,14.2) & NO            & NO                & NO                & NO                \\
3     & 11.3                    & 9.1                        & -2.2                  & (8.5,9.9)   & NO            & NO                & NO                & YES               \\
4     & 10.0                    & 9.6                        & -0.4                  & (8.8,10.4)  & YES           & YES               & YES               & YES               \\
5     & 21.1                    & 15.3                       & -5.8                  & (14.5,16.1) & NO            & NO                & NO                & NO                \\
6     & 16.1                    & 31.1                       & 15.0                  & (27.4,35.0) & NO            & NO                & NO                & NO                \\
7     & 10.8                    & 11.7                       & 0.9                   & (10.8,12.7) & YES           & YES               & YES               & YES               \\ \hline
\end{tabular}
\caption{\label{tab:table6} Performance of the traditional model in Zhong et al.\cite{zhong2022sitelevel}.}
\end{table}

\end{document}