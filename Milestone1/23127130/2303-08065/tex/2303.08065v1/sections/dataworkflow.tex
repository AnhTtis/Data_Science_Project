\documentclass[../main.tex]{subfiles}
\usepackage{graphicx}
\graphicspath{ {./sections/images/} }

\begin{document}
\label{sec:data_flow}
% \subsection{Enrollment framework}
The whole enrollment procedure in clinical trials is a complex process with multiple steps. Previous research usually concentrated on the key part of subject enrollment \cite{anisimov2020modern, anisimov2011statistical, anisimov2007recruitment}. Nevertheless, subject enrollment is just one factor in the clinical trial enrollment procedure. In real-world practice, the country/site preparation step (such as contracting and training investigators, preparing doses, etc) that is prior to subject enrollment can also be a compelling impact on the enrollment timeline. This paper proposes a comprehensive enrollment framework based on study-level historical trial data, which covers site activation and the subject enrollment processes. Models with proper parameter estimation and simulation will be developed (depending on historical data availability) to cover each part of the enrollment framework. Unless stated otherwise, the entire enrollment process in a study, as illustrated in Figure~\ref{framework_figure}, is referred to as two sequential segments from the date of final protocol approval to the date of subject enrollment completion. 

\begin{figure}[h]
\centering
\caption{Two Sequential Segments of Enrollment Framework}
\label{framework_figure}
\includegraphics[width=16cm]{./sections/images/enrollment_framework2}
\end{figure}

The first segment of the enrollment framework is the site activation process, defined as the time between the study start date and the opening date of each site. We assume all the selected sites and corresponding countries in a clinical trial will share the same study start date defined as the final protocol approval date. 
In addition, we assume sites are indistinguishable within each country and allow each country's first site activation date to be modeled separately from the others. The first site activation is also referred as country start-up, a country-level procedure involving country approval and preparation activities such as country/region regulatory approval and IRB approval process. In real world, it can be significantly affected by a couple of factors, including but not limited to sponsors, countries/regions, therapeutic areas, and phases. Therefore, dedicated integrated data from multiple sources are required to predict the country's start-up and site activation. From a modeling perspective, data availability is also a critical factor, since final protocol approval dates and site activation dates are typically unavailable in the most common public databases. Our proposed model will rely on an internally available data set to develop a forecasting process for site activation.

The second part of the enrollment framework is the subject enrollment process. It is a country-specific procedure that starts from the country start-up date to the enrollment completion date of the entire clinical trial. All the sites in the same country start to recruit patients after the country's start-up date. No particular site start-up patterns are considered due to the limitation of study-level historical data. The enrollment rate of historical trials will be adjusted based on the simulated country start-up pattern to avoid bias. 


\end{document}