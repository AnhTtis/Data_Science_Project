\documentclass[../main.tex]{subfiles}

\begin{document}
\label{sec:discussion}
For the problem of predicting study duration at early planning stage, the proposed study-level forecasting model is neat and accurate to characterize entire enrollment by the subject recruitment and the site initiation parts. Through numeric analysis, our model conducts a balanced forecast between PI length and prediction error, which essentially stand for the trade-off between variance and bias. 

In terms of determining the number of sites in each country, our framework assumes they are specified at the beginning. However, at early stage of study preparation, it is common to only have a list of countries to carry out the trial and a total number of sites to initiate. As a consequence, there exist a need to get more insights from historical data to guide the start-up of counties as well as sites. 
In our performance evaluation, we used the number of sites from a given list of countries and approximated the remaining sites by corresponding median estimates of model parameters based on all countries on the list.
Similarly, practitioners can use historical data to obtain a recommendation of the number of sites of each country based on, for instance, maximum of opened sites in each study, site opening rate and country start-up days.

With regard to the prediction of site initiation, we found the FSFD seemed to have a negative bias in our performance evaluation in Section~\ref{sec:model_performance}, though it was not conclusive from validation of only 7 studies. Despite that, the overall predictions of study duration were generally good. This phenomenon indicates a potential improvement on the site activation modelling. One possible remedy is to build a separate country start-up model in addition to site initiation as in Zhong et al. \cite{zhong2022sitelevel} to better model $t_i$. Another way is to consider zero-inflation Poisson model for $u_{i,j}$, where an additional underlying process is introduced to determine whether a count is zero or non-zero. Then, the underlying process brings more chance to zero observations while a large rate parameter will accompany to balance the overall modelling. 
Since it is out the scope of this project, we will leave the training of zero-inflation model to the future. 

For the subject accrual rate parameter, $\mu$, is set to be a common parameter across country in our model. A natural variation may be to estimate the rate along with country. However, as mentioned in the quasi-Poisson model set-up, historical data availability does not encourage an integrated structure incorporating too many factors like random effect and country effect. Hence, we would leave such modification as a humble remark here.

\end{document}