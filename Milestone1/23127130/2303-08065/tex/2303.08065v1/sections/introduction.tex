\documentclass[../main.tex]{subfiles}

\begin{document}
In clinical development, forecast of trial duration especially at the portfolio planning stage is extremely important. It facilitates the informed decision making for senior management and provides an initial idea of the potential spend on time and cost for any new trial proposal. An advanced predictive modeling providing accurate enough enrollment forecast across all portfolios is in starving demand. There are several quantitative approaches developed in statistical literature.  

A naive approach is based on Patients (subjects) per Site per Month, often abbreviated as $psm$, which is defined as 
% $psm = \frac {\text{number of enrolled subjects}} {\text{number of sites } \times \text{ enrollment time}}$. 
$psm = \textit{number of enrolled patients or subjects} / (\textit{number of sites} \times \textit{enrollment time})$.
Assuming that every site in a trial will have such a constant enrollment rate over time, the projected enrollment duration for a planned trial is simply 
% $\frac{\text{planned sample size}}{\text{number of sites proposed } \times \text{ $psm$}}$, 
$\textit{planned sample size}/(\textit{number of sites proposed} \times psm)$,
where the $psm$ rate in the denominator can be directly taken from the average rate in historical studies or user past experience.  This approach is simple and empirical. It relies on strong assumption on the constant enrollment rate for all sites overtime, which is not usually the case in reality. With often unsatisfactory predictive performance from this naive approach, a more sophisticated statistics modeling is needed to fill the gap.

There are various statistical approaches proposed in the literature to model and predict patient accrual. The papers \cite{anisimov2016discussion, HEITJAN201526} are provoking systematic literature review for patient accrual models. %\citep{https://doi.org/10.1002/sim.843} 
Bagiella and Heitjan \cite{https://doi.org/10.1002/sim.843} proposed to use a homogenous Poisson process to model the patient recruitment. %\citeauthor{https://doi.org/10.1002/sim.2956} 
Anisimov and Fedorov \cite{https://doi.org/10.1002/sim.2956} improved the recruitment
model by applying a Poisson-Gamma mixture model to handle the variation in recruitment rate across multiple centers.
As pointed out by succeeding works \cite{anisimov2020modern, anisimov2011statistical, anisimov2007recruitment}, such type of methods is mainly random effects models. The intrinsic goal is to capture the heterogeneity in enrollment
rates across different centers whereas the enrollment pattern within each center is described by a homogeneous
Poisson process with gamma distributed rate. %\citeauthor{https://doi.org/10.1002/sim.8036} 
Lan, Tang, and Heitjan \cite{https://doi.org/10.1002/sim.8036} first proposed a time-varying rate function that allows modeling
the time decay trend in recruitment while taking site initiation into consideration. %\citeauthor{deng2017bayesian} 
Deng, Zhang, and Long \cite{deng2017bayesian} investigated a Bayesian approach using a non-homogeneous Poisson process where region-specific accrual is accounted in their framework. %\citeauthor{zhang2010stochastic} 
Zhang and Long \cite{zhang2010stochastic,zhang2012joint} employed a non-homogeneous Poisson process to model patient accrual where the underlying accrual rates are allowed to change over time. %\citeauthor{wang2022real} 
Wang et al. \cite{wang2022real} defined time to endpoint maturation framework and linked the concept to key milestone dates in clinical trials. They proposed a simulation based non-homogeneous Poisson process with a normal kernel enrollment rate which can capture the up-and-down enrollment trend in reality and provided improved prediction performance in both simulated and real study enrollment data.   

Motivated by %\citeauthor{wang2022real} 
Wang et al. \cite{wang2022real} and others work, %\citeauthor{zhong2022sitelevel} 
Zhong et al. \cite{zhong2022sitelevel} proposed a novel statistical framework based on generalized linear mixed-effects model (GLMM) and the use of non-homogeneous Poisson processes through Bayesian hierarchical modeling framework to predict trial duration at the portfolio planning stage. It utilizes site level enrollment information from proprietary data to predict trial duration more accurately based a set of pre-selected validation studies. It also shows that their modeling and simulation approach calibrates the data variability appropriately and gives correct coverage rates for prediction intervals of various nominal levels.

One of the challenges for 
%\citeauthor{zhong2022sitelevel} 
Zhong et al. \cite{zhong2022sitelevel} is the availability and the size of the site level enrollment information. Some site-level data is proprietary and may be not publicly available, where the number of studies from that data is just limited to several sponsor companies who contributed to the data set. To make the prediction algorithm beneficial to broader users in pharmaceutical industry, we proposed a new advanced statistical modeling algorithm utilizing publicly available study-level information that can be generalized from \url{ClinicalTrials.gov}. This data source is free and it contained extensive number of studies compared to proprietary data. 
The study-level information from \url{ClinicalTrials.gov} together with historical data within users organization can bring prospective evaluations on subjects from new sites and for new indications. 
Our methodology includes different homogeneous Poisson process for subject enrollment and site activation. Quasi-Poisson regression and Monte Carlo sampling are proposed to estimate and simulate subject enrollment, and a linear (fixed) approach and a perturbed approach are proposed to model site activation. 

The rest of this paper is organized as follows. The input data and entire forecast workflow are described in Section 2. In Section 3, we derive the details of methodology. Section 4 provides the prediction performance based on real case studies. Discussions are given in Section 5. 
\end{document}