% This is samplepaper.tex, a sample chapter demonstrating the
% LLNCS macro package for Springer Computer Science proceedings;
% Version 2.20 of 2017/10/04
%
\documentclass[runningheads]{llncs}
%
\usepackage{graphicx}
 \usepackage{subcaption}
\usepackage{amsfonts, amssymb, amsmath,pifont,float,caption,hyperref}
\usepackage[ruled,vlined,linesnumbered]{algorithm2e}
\SetKwComment{Comment}{/* }{ */}
% Used for displaying a sample figure. If possible, figure files should
% be included in EPS format.
% to display URLs in blue roman font according to Springer's eBook style:
% \renewcommand\UrlFont{\color{blue}\rmfamily}
\newtheorem{Corollary}{Corollary}

\begin{document}
%
\title{Arbitrary Pattern Formation on a Continuous Circle by Oblivious Robot Swarm\thanks{Supported by organization x.}}
%
%\titlerunning{Abbreviated paper title}
% If the paper title is too long for the running head, you can set
% an abbreviated paper title here
%
\author{Brati Mondal\orcidID{0009-0001-3017-9924} \and
Pritam Goswami \orcidID{0000-0002-0546-3894} \and
Avisek Sharma\orcidID{0000-0001-8940-392X}\and
Buddhadeb Sau\orcidID{0000-0001-7008-6135}
}
%
\authorrunning{B.Mondal, P.Goswami, A.Sharma and B.Sau}
% First names are abbreviated in the running head.
% If there are more than two authors, 'et al.' is used.
%
\institute{Jadavpur University, Kolkata-700032\\
\email{\{bratim.math.rs,pritamgoswami.math.rs,aviseks.math.rs\}@jadavpuruniversity.in}}
%
\maketitle              % typeset the header of the contribution
%
\begin{abstract}
In the field of distributed system, Arbitrary Pattern Formation (APF) problem is an extensively studied problem. The purpose of APF is to design an algorithm to move a swarm of robots to a particular position on an environment (discrete or continuous) such that the swarm can form a specific but arbitrary pattern given previously to every robot as an input. In this paper the APF problem is discussed on continuous circle by a swarm of homogeneous, anonymous, oblivious, silent mobile robots with no chirality. The algorithm discussed here, can solve APF problem deterministically within $\mathcal{O}(n)$ epochs without collision under Semi Synchronous scheduler with $n$ (where $n$ is odd) mobile robots if the initial configuration is rotationally asymmetric. For any even number of robots (grater than three) the algorithm can be modified to a probabilistic one to solve APF. Also if the initial configuration is rotationally symmetric then APF problem can't be solved by any deterministic algorithm.

\keywords{APF  \and Continuous circle \and Oblivious \and mobile robots \and distributed algorithm.}
\end{abstract}
\section{Introduction}

Applications of distributed systems and their relevant problems have gained substantial importance in the last two decades. Unlike a centralized system, using a swarm of inexpensive, simple robots to do a task is more cost-effective, robust, and scalable. These swarms of robots have many applications like rescue operations, military operations, search and surveillance, rescue operations, disaster management, cleaning of large surfaces, and so on. 

Researchers are interested in studies about using swarm robots with minimum capabilities to do some specific tasks like \textsc{Gathering}, \textsc{Arbitrary Pattern Formation}, \textsc{Dispersion}, \textsc{Exploration}, \textsc{Scattering} etc. The robots are \textit{autonomous} (have no centralized controller), \textit{anonymous} (have no IDs), and \textit{homogeneous} (have same capabilities and execute same algorithm). Depending on the capabilities of robots there are four types of robot models: $\mathcal{OBLOT}$, $\mathcal{FSTA}$, $\mathcal{FCOM}$, $\mathcal{LUMI}$. In the $\mathcal{OBLOT}$ model robots do not have any persistent memory of their previous round i.e. \textit{oblivious} and they can't communicate with each other i.e. \textit{silent}. In the $\mathcal{FSTA}$ model robots are not oblivious but silent. In the $\mathcal{FCOM}$ model robots are oblivious but not silent. In the $\mathcal{LUMI}$ model robots are neither oblivious nor silent. 

Each robot executes a \textsc{Look-Compute-Move} (LCM) cycle after activation. In \textsc{Look} phase robot takes a snapshot of its surroundings and collects the required information. Then the robot calculates the target using that information in the \textsc{Compute} phase and moves to the destination in \textsc{Move} phase. A scheduler is the controller of the activation of robots. There are three types of schedulers:  \textit{Fully synchronous} (\textsc{FSync}) scheduler, \textit{Semi synchronous} (\textsc{SSync}) scheduler, \textit{Asynchronous} (\textsc{ASync}) scheduler. In \textsc{FSync} scheduler, time is divided into global rounds of the same duration and each robot is activated in every round and executes the LCM cycle. In \textsc{SSync} scheduler, here also time is divided into global rounds of the same duration like \textsc{FSync}, but all robots may not be activated at the beginning of each round. In \textsc{ASync} scheduler robots are activated independently and the LCM cycle is not synchronized here.

The problem considered here is the Arbitrary Pattern Formation (APF) problem in which a swarm of robots is deployed in an environment (discrete or continuous domain). APF problem aims to design an algorithm such that robots move to a particular position and form a specific but arbitrary pattern which is already given to every robot as input. There is a vast literature of APF in both discrete and continuous domains (\cite{SY1996,YS2010,FPSW2008,DPV2010,BQTS2016,BQTS2018,SGA2019,BDS2021,BOSEKAS21,BAKS20,CFSN20,KGGS22,SAGA2023}). Most of the works of APF in continuous domain considered on the Euclidean plane. There are other sorts of environments which are included continuous domain, e.g. any closed curve embedded on the plane where robots can only move on that curve. In real life, such environments also exist and are hugely applicable in different scenarios such as roads, railway tracks, tunnels, waterways, etc. Another example of this kind of environment can be a circle of fixed radius embedded on the plane. Studying this problem is interesting because the solution for this problem can be extended to all other closed curves. Thus in this paper, we have considered the problem of Arbitrary Pattern Formation (APF) on a circle. 


\section{Related Works and Our Contribution}

\subsection{Related work}
In swarm robotics, Arbitrary Pattern Formation (APF) problem is a hugely studied problem. This problem was first introduced by Suzuki and Yamashita in \cite{SY1996} on the Euclidean plane. Later they characterized the geometric patterns formable by oblivious and anonymous robots in \cite{YS2010} for fully synchronous and asynchronous scheduler. After that this problem has been considered on different environments on continuous and discrete domains (\cite{FPSW2008,DPV2010,BQTS2016,BQTS2018,SGA2019,BDS2021,BOSEKAS21,BAKS20,CFSN20,KGGS22,SAGA2023}). 

In the continuous domain, most of the works which consider arbitrary pattern formation are done on the Euclidean plane under different settings. In \cite{FPSW2008} Flocchini discussed the solvability of the pattern formation problem by considering oblivious robots with fully asynchronous schedulers. They showed that if the robots have no common agreement with their environment, they are unable to form an arbitrary pattern. Moreover, if the robots have one axis agreement, then any odd number of robots can form an arbitrary pattern, whereas the even number can't. Further, if the robots have both axis agreement, then any set of robots can form any pattern. They proved that it is possible to elect a leader if it is possible to form any pattern for $n \ge 3$ robots. The converse of this result is proved and a relationship between leader election and arbitrary pattern formation for oblivious robots in the asynchronous scheduler is studied in \cite{DPV2010}. The authors showed that for $n \ge 4$ robots with chirality (respectively for $n \ge 5$ without chirality), the Arbitrary pattern formation problem is solvable if and only if leader election is possible. In \cite{BQTS2016} the authors proposed a probabilistic pattern formation algorithm for oblivious robots under asynchronous scheduler without chirality. Their protocol is the combination of two phases, one is a probabilistic leader election phase and another is a deterministic pattern formation phase. Later in \cite{BQTS2018} they proposed a new geometric invariant that exists in any configuration with four oblivious anonymous mobile robots to solve arbitrary pattern formation problems with or without the common chirality assumption. In \cite{SGA2019} authors studied Embedded Pattern Formation without chirality with oblivious robots. They characterized when the problem can be solved by a deterministic algorithm and when it is unsolvable. In \cite{BDS2021} authors studied the APF problem for the robots whose movement can be inaccurate and the formed pattern is very close to the given pattern. In \cite{BOSEKAS21} the authors provided a deterministic algorithm in the Euclidean plane with asynchronous opaque robots.


% In g, arbitrary pattern formation was first introduced in \cite{BAKS20} on an infinite grid by oblivious mobile robots in asynchronous scheduler with full visibility. In \cite{SAGA2023} the authors solved the APF problem for oblivious robots in asynchronous scheduler on a triangular grid which also works for square and hexagonal grids. 

Note that all the works in arbitrary pattern formation considering continuous domain have been done only for the Euclidean plane where the robots can arbitrarily move from one point to another via infinitely many paths. But there are some environments in continuous domain in which the movements of robots from one point to another are restricted to a finite number of possible paths. A continuous circle of fixed radius is one such environment. To the best of our knowledge, there are some works (\cite{DUVY20,FKKSY19,GSGS23,CGKK11,CKPT14,FKKR17}) which considered continuous circle as their corresponding environment. The problem of patrolling, gathering, and rendezvous are the main focus of these works. But none of them considered the problem of APF on the continuous circle.
 
\subsection{Our Contribution}
In this work, our aim is to solve the problem of Arbitrary Pattern Formation (APF) on a continuous circle by oblivious and silent mobile robots with full visibility under semi-synchronous scheduler. To the best of our knowledge, the APF problem has not yet been considered on a continuous circle. So in this paper, we have considered this problem for the first time. Here the robots do not agree with a particular direction i.e. robots have no chirality. The movements of robots are restricted only in two directions, clockwise and anti-clockwise from any point. So, avoiding collision in a circle is more difficult than avoiding collision on a plane.
% There are some works of Uniform Circle Formation, which is a sub-case of this problem.
The robot model considered here is the weakest $\mathcal{OBLOT}$ model. In this problem, there is no particular landmark or door from which the robots enter. Here we discussed that, if the initial configuration is rotationally asymmetric then, for any odd number of robots the proposed deterministic algorithm $APF\_CIRCLE$ can solve the problem of Arbitrary Pattern Formation under Semi Synchronous (\textsc{SSync}) scheduler within $O(n)$ epochs avoiding collision, where $n$ is the number of robots in the swarm. For any even number of robots, which is greater than three, the algorithm can be modified to a probabilistic one that solves APF on a continuous circle within $O(n)$ epochs. We have also shown that the APF problem can't be solved deterministically if the initial configuration is rotationally symmetric.


\section{Model and Problem Definition }
\subsection{Problem Definition}
 Let $\mathcal{CIR}$ be a continuous circle of fixed radius. Let $n$ robots resides on the perimeter of $\mathcal{CIR}$ such a way that the configuration formed by them is rotationally asymmetric. The robots can move freely on the circle. A sequence of angular distances $\beta_0,\beta_1, \beta_2\dots, \beta_{n-1}$ is given to the robots such that the sum of the angles of the sequence is equal to $2\pi$. The problem is to design a distributed algorithm for the robots so that by finite execution of the algorithm, the robots move in such locations on $\mathcal{CIR}$, such that the final configuration has the following property:
 
 \begin{itemize}
     \item [\ding{71}] There exist a robot, say $r_0$ and a direction $\mathcal{D} \in \{$ clockwise, anticlockwise$\}$ such that the angular distance between the $i$-th and $(i+1)$-th robot in the direction $\mathcal{D}$ (denoted as $r_i$ and $r_{i+1}$ respectively) is $\beta_i$ where all the indices are considered under modulo $n$.
 \end{itemize}
 
 % there exist one robot (say $r_0$) and a direction $\mathcal{D} \in \{$ clockwise, anticlockwise$\}$, starting from that the sequence of angular distances between robots is equal to the given sequence either in clockwise or in anticlockwise direction .

\subsection{Model}

    \subsubsection{Robot Model:}  All robots are placed on the perimeter of a circle, say $\mathcal{CIR}$. Here the robots can move only on the perimeter of the circle. Robots have no particular orientation (i.e., no agreement on clockwise or anticlockwise direction). Robots have full visibility of the circle. The initial configuration is rotationally asymmetric. The movements of the robots are rigid i.e. robots always moves to its destination in a particular round. Robots have following properties-
\begin{itemize}
    \item[] \textit{Autonomous:} Robots don't have any centralised controller.
    \item[] \textit{Anonymous:} Robots have no IDs.
    \item[] \textit{Homogeneous:} All robots have same capabilities and execute the same algorithm.
    \item[] \textit{Oblivious:} Robots have no persistent memory.
    \item[] \textit{Silent:} Robots have no means of communications.
    \item[] \textit{Visibility:} Robots have full visibility of the circle i.e., robots can see all other robots on the circle.
\end{itemize}

\paragraph{\textbf{LCM cycle:}} Each robot executes a cycle of \textsc{Look-Compute-Move}(LCM) phases upon activation.

\begin{itemize}
    \item[] \textit{LOOK:} In look phase a robot takes a snapshot of its surroundings and gets the location of other robots on the circle according to its own coordinate .
    \item[] \textit{COMPUTE:} Robot determines target location using the input from the snapshot of the look phase by executing the provided algorithm.
    \item[] \textit{MOVE:}  In move phase robot moves to the destination point calculated in the compute phase.
\end{itemize}

\subsubsection{Scheduler Model:}The activation of robots are controlled by an entity called scheduler. Depending on the activation timing there are three types of schedulers: \\
\begin{itemize}
    \item[] \textit{Fully synchronous} (\textsc{FSync}): In fully synchronous scheduler time is divided into rounds of equal length and all robots are activated at the beginning of every round and performs the LCM cycle synchronously.\\
    
    \item[] \textit{Semi synchronous} (\textsc{SSync}): Similar to fully synchronous scheduler here also time is divided into rounds of equal length. But all robots might not get activated at the beginning of a particular round. In a particular round the activated robots perform the LCM cycle synchronously. Note that semi synchronous scheduler is more generalised than fully synchronous scheduler.\\
    
    \item[] \textit{Asynchronous} (\textsc{ASync}): In asynchronous scheduler time is not divided into rounds like fully synchronous and semi synchronous scheduler. Robots are activated independently. In a particular moment of time some robots may be in Look phase, some in Compute phase, some in Move phase or some may be idle. Asynchronous scheduler is the most general among all the schedulers. 
    
    In this paper, Semi Synchronous (\textsc{SSync}) scheduler has been considered to solve the problem.
\end{itemize}


\section{Prelimineries}
In this section, we first justify the reason for assuming the initial configuration is rotationally asymmetric. Then, we define some terminologies here in this section that will be needed to describe the algorithm provided in the next section.
\begin{proposition}
    \label{Prop:rotationallySymmetricImpossible} There is no deterministic distributed algorithm that solves arbitrary pattern formation problem on a continuous circle if the initial configuration is rotationally symmetric.
\end{proposition}
\begin{proof}
Let $\mathcal{C}(0)$ be the initial configuration which is rotationally symmetric. Let $\mathcal{C}(0)$ has a $k-$fold symmetry i.e., for a rotation of $\frac{2\pi}{k}$ along the center, the configuration remains the same. By proposition 2.4 in \cite{DUVY20} if the scheduler is fully synchronous then for any algorithm $\mathcal{A}$, the new configuration will have a $k'-$fold symmetry after one execution of $\mathcal{A}$, where $k' \ge k$. Thus for any finite execution of $\mathcal{A}$, the configuration will always have a $k_1-$fold symmetry where $k_1 \ge k$. So if the target pattern is asymmetric it can not be formed by the robot swarm by finite execution of $\mathcal{A}$. Hence the result.\qed
\end{proof}
\begin{definition}[$(A, B)_{\mathcal{D}}$]
    \label{def:angle}
    Let $A$ and $B$ be two points on the circle $\mathcal{CIR}$. Let $\mathcal{D}$ be a direction either clockwise or anticlockwise. Then $(A,B)_{\mathcal{D}}$  denotes the angular distance from point $A$ to point $B$ in the direction $\mathcal{D}$.
\end{definition}

\begin{definition}[ Set of Angle Sequences of a robot $r$] \\Let, $\mathcal{R}=$ $\{r_0, r_1, r_2,\dots,r _{n-1}\}$ be the set of robots placed consecutively in a fixed direction, say $\mathcal{D}$ (either clockwise or anticlockwise) on the circle $\mathcal{CIR}$. Let $\theta_{i \pmod{n}}$ be the angular distance from the location of robot $r_{i \pmod{n}}$ to the location of robot $r_{{i+1} \pmod{n}}$ in the direction $\mathcal{D}$. Then the set of angle sequences of the robot $r=r_0$, denoted as $\mathcal{AS}(r)$, is the set $\{\mathcal{AS}_D (r),\mathcal{AS}_{D'}(r)\}$ where, $\mathcal{D}'$ is the opposite direction of $\mathcal{D}$ and  $\mathcal{AS}_D (r)=(\theta_0, \theta_1,\dots,\theta_{n-1}$), $\mathcal{AS}_{D'} (r)=(\theta_{n-1}, \theta_{n-2},\dots,\theta_{0}$) are two angle sequences  in the direction $\mathcal{D}$ and $\mathcal{D'}$ respectively.
\end{definition}

Since the initial configuration is rotationally asymmetric then, for a fixed particular orientation (either clockwise or anti-clockwise) all the robots have different angle sequences (??). So, note that, for two robots, say $r_1$ and $r_2$ if $\mathcal{AS}_{\mathcal{D}_1}(r_1) \in \mathcal{AS}(r_1)$ is equal to $\mathcal{AS}_{\mathcal{D}_2}(r_2) \in \mathcal{AS}(r_2)$ then, $\mathcal{D}_1$ must be equals to $\mathcal{D}_2'$

 
\begin{definition}[Nominee] A robot $r_l$ is considered as the nominee if \\
min$(\cup_{r\in \mathcal{R}} \mathcal{AS}(r)) \in \mathcal{AS}(r_l)$    
\end{definition}

\begin{figure}
    \centering
    \includegraphics[height=4.5cm]{Fig/initConfigFinial.eps}
    \caption{ Here, $\mathcal{AS}(r) = \{(30^{\circ} 90^{\circ} 60^{\circ} 75^{\circ} 105^{\circ}), (105^{\circ} 75^{\circ} 60^{\circ} 90^{\circ} 30^{\circ})\}$ is the set of angle sequences for the robot $r$. Note that $\mathcal{AS}(r)$ contains the minimum angle sequence $(30^{\circ} 90^{\circ} 60^{\circ} 75^{\circ} 105^{\circ})$ so, $r$ is a nominee. This configuration is also a single nominee configuration.} 
    \label{fig:SingleNominee}
\end{figure}




\begin{proposition}
\label{prop:atLeastoneAtmostTwo}
In the initial configuration, there are at least one and at most two nominees.
\end{proposition}
\begin{proof}
Since the initial configuration is rotationally asymmetric, by a result stated in \cite{DUVY20} it can be said that all the robots have distinct angle sequences in a particular direction. Thus $n$ robots have $n$ distinct angle sequences in a particular direction. Similarly in opposite direction, there exists $n$ distinct angle sequence. Among those $2n$ angle sequences, at least one angle sequence must be minimum. If this minimum angle sequence belongs to $\mathcal{AS}(r)$, then $r$ is selected as the nominee. So the initial configuration must have at least one nominee.
\par Now, let it be assumed that there are more than two nominees in the initial configuration. Without loss of generality let there be three nominees in the initial configuration. Let the first nominee whose angle sequence is minimum, has the minimum angle sequence in a particular direction say, $\mathcal{D}$. Then the second nominee must get its minimum angle sequence in the direction $\mathcal{D}'$, the opposite direction of $\mathcal{D}$ (as no two robots can have minimum angle sequence in the same direction). Now, the third nominee must have its minimum angle sequence in the direction of either $\mathcal{D}$ or $\mathcal{D}'$. But this can't be possible because, in a particular direction, no two robots have the same angle sequence. So there can not be more than two nominees in the initial configuration.
\qed
\end{proof}

\begin{figure}[h]
    \centering
    \includegraphics[height=4.5cm]{Fig/InitConfigDoubleNominee.eps}
    \caption{ A double nominee configuration where both $r$ and $r'$ are nominees. The angle bisector $\mathcal{AB}$ contains a robot. $Arc(r)$ is highlighted with red dotted line and $Arc(r')$ is highlighted with green dotted line.} 
    \label{fig:DoubleNominee}
\end{figure}

\begin{definition}[Single nominee configuration]
This is a configuration where there is only one nominee.    
\end{definition}

\begin{definition}[Double nominee configuration]
 This is a configuration where there are two nominees in the configuration.  
\end{definition}



\begin{definition}[Angle Bisector in a double nominee configuration] \label{def:angleBisector}
Let $r$ and $r'$ be two nominees in a double nominee configuration. The angle bisector of this configuration is defined as the straight line that bisects the angles formed by the robots $r$ and $r'$ and is denoted as $\mathcal{AB}$.
\end{definition}
In the future, the term ``angle bisector of a configuration" or the symbol $\mathcal{AB}$  will always be used for a double nominee configuration even if it is not mentioned explicitly. 

\begin{definition}[Arc of a nominee in a double nominee configuration]
    Let $r$ and $r'$ be two nominees in a double nominee configuration. Let $\mathcal{AB}$ be the angle bisector of the angle between $r$ and $r'$. Now, $\mathcal{AB}$ divides the circle into two arcs. Among these two arcs, the arc on which the robot $r$ is located except the points of $\mathcal{AB}$ is called the arc of the robot $r$ and is denoted as $Arc(r)$.
\end{definition}



\begin{definition}[Leader] \label{def:leader}
 We call a robot, say $r$, the leader if any one of the following statements is true for  the robot $r$:
 \begin{enumerate}
     \item $r$ is the nominee in a single nominee configuration.
     \item $r$ is a nominee in a double nominee configuration and $Arc(r)$ has more robot than $Arc(r')$ where $r'$ is another nominee.
 \end{enumerate}
\end{definition}

\begin{definition}[Leader configuration]\label{def:leaderConfig}
A configuration is called a leader configuration if it has a leader.    
\end{definition}

\begin{definition}[Pivotal Direction]
In a leader configuration, the direction in which the leader has the minimum angle sequence is called the pivotal direction. The pivotal direction is denoted as $\mathcal{D}_p$.
\end{definition}

\begin{figure}[h]
    \centering
    \includegraphics[height=4.5cm]{Fig/LeaderConfig.eps}
    \caption{Here the configuration is a leader configuration and $r$ is the leader. The direction $\mathcal{D}_p$ is the direction in which $r$ has the minimum angle sequence.} 
    \label{fig:leader}
\end{figure}

\begin{proposition}
    \label{prop:notLeaderConfigMeansRobotOnBisector}
    If an asymmetric configuration with an odd number of robots is not a leader configuration then the configuration must be a double nominee configuration with exactly one robot on the angle bisector $\mathcal{AB}$.
\end{proposition}
\begin{proof}
    Let $C$ be an asymmetric configuration then by Proposition~\ref{prop:atLeastoneAtmostTwo} $C$ must be either a single nominee configuration or a double nominee configuration. By Definition~\ref{def:leader} and Definition~\ref{def:leaderConfig} any single nominee configuration must be a leader configuration. So, $C$ must be a double nominee configuration. let $r$ and $r'$ be the two nominees in $C$. Let $\mathcal{AB}$ be the angle bisector of the configuration $C$. Now since the configuration $C$ is not a leader configuration, the number of robots on $Arc(r) =$ number of robots on $ Arc(r')$. Now if there is no robot or two robots on $\mathcal{AB}$  then the total number of the robots becomes even in the configuration, contrary to our assumption. So, $\mathcal{AB}$ must contain exactly one robot. 
    \qed
\end{proof}

\begin{definition}[Move Ready Robot]
    In a leader configuration, let $r$ be the first robot from leader, say $r_0$ in the direction $\mathcal{D}_p$ which satisfies the following condition:
    \begin{enumerate}
        \item $r$ is not the first or second neighbour of leader $r_0$ in the direction $\mathcal{D}_p$.
        \item  If $D$ is the destination of $r$ in direction $\mathcal{D}$ and $r'$ be the neighbour of $r$ in the direction $\mathcal{D}$. Then $(R,R')_{\mathcal{D}}-(R,D)_{\mathcal{D}}>\alpha_1$, where $\alpha_1$ is the angular distance between first and second neighbour of $r_0$ in the direction $\mathcal{D}_p$ and $R$ and $R'$ are the locations of $r$ and $r'$ respectively on the circle.
    \end{enumerate}
    Then $r$ is defined as the Move Ready robot of the configuration
   
\end{definition}

\section{Algorithm $APF\_CIRCLE$}

Let $r_0, r_1, \dots r_{n-1}$ be $n$ robots ($n$ is odd and $n>2$)  on a continuous circle $\mathcal{CIR}$. The robots can move freely on $\mathcal{CIR}$. Let for a robot $r_i$ the location of $r_i$ on $\mathcal{CIR}$ is denoted as $R_i$.  Let initially the configuration is rotationally asymmetric. Let $\mathcal{D}$ be the direction in which $r_{i+1}$ is a neighbor of $r_i$. Also, let $(R_i, R_{i+1})_{\mathcal{D}}=  \alpha_i$ where all indices are considered in modulo $n$. Note that, $\sum_{i=0}^{n-1} \alpha_i = 2\pi$. Now, each robot has been provided with a sequence of angle $\beta_1, \beta_2,\dots,\beta_{n-1}$ as input such that, $\sum_{i=0}^{n-1} \beta_i = 2\pi$. The robots are oblivious, silent, and have no chirality. The circle is not oriented. In this setting, the algorithm provided in this section guides the robots to form the sequence of angles provided to them on the circle $\mathcal{CIR}$. 

The first challenge to do so is to find a robot and a direction from which this formation will start. This will be done by leader election. Now since the robots are oblivious the next problem that arises is to make the leader robot remain leader and the direction in which the pattern will be formed (pivotal direction) fixed throughout the execution of the algorithm. All of these challenges are considered and handled in the provided algorithm. In the following, the description of the algorithm and its correctness are discussed in detail.

\subsection{Description and Correctness of $APF\_CIRCLE$}

Let $\mathcal{C}(0)$ be the initial configuration. According to Proposition \ref{prop:atLeastoneAtmostTwo}, there can be at most two nominees and at least one nominee in the configuration $\mathcal{C}(0)$. If there is a single nominee then it is the leader and the configuration is a leader configuration (Definition~\ref{def:leader} and Definition~\ref{def:leaderConfig}). Otherwise, there must be two nominees in $\mathcal{C}(0)$. Let $r_1$ and $r_2$ be two nominees in $\mathcal{C}(0)$. Now there are two possibilities. Either $Arc(r_1)$ and $Arc(r_2)$ have same number of robots or they don't. Let, $Arc(r_1)$ and $Arc(r_2)$ do not have equal number of robots then without loss of generality Let $Arc(r_1)$ has more robots. In this case, $r_1$ would be the leader and the configuration is again a leader configuration. So, let us consider the case where $Arc(r_1)$ and $Arc(r_2)$ have equal numbers of robots. According to Proposition~\ref{prop:notLeaderConfigMeansRobotOnBisector}, there must be exactly one robot on the angle bisector $\mathcal{AB}$. Let us denote that robot as $r_d$. Thus, when $\mathcal{C}(0)$ has two nominees and it is not a leader configuration then, the robot $r_d$ moves a positive angular distance $\epsilon_d$ towards any of its neighbor, say $r$, in the direction $\mathcal{D}$ such that, $0 < \theta-\epsilon_d < \alpha_0$ ($\alpha_0$ is the smallest angle in the initial configuration and $\theta = (R_d, R)_{\mathcal{D}}$ )  and also, after $r_d$ moves the angular distance $\epsilon_d$ towards $r$, there is no robot on the angle bisector of $r_d$ and $r$ (Fig.\ref{fig:Lemma1}). In the following two lemmas, we first prove the existence of such $\epsilon_d$ and then prove that this move of $r_d$ converts the configuration into a leader configuration. 
\begin{lemma}
    \label{lemma:existenceEpsilond}
    If the initial configuration is not a leader configuration then, the robot $r_d$ on the angle bisector $\mathcal{AB}$ always finds an $\epsilon_d > 0$ such that after moving $\epsilon_d$ towards a neighbor $r$ in the direction $\mathcal{D}$, there is no robot on the angle bisector of $R_d$ and $R$. Also if in the initial configuration $ (R_d, R)_{\mathcal{D}} = \theta$ then, $0 < \theta - \epsilon_d < \alpha_0$. 
\end{lemma}
\begin{proof}
    Let initially the angle bisector $\mathcal{AB}$ intersect the circle $\mathcal{CIR}$ at two points $A$ and $B$. Without loss of generality let $r_d$ be on $A$. Without loss of generality let $r_d$ decides to move towards its neighbor $r$ in the direction say $\mathcal{D}$. Let initially $(R_d, R)_{\mathcal{D}} = \theta \ge \alpha_0$ as $\alpha_0$ is the minimum angle of $\mathcal{C}(0)$. We will show that there exists a positive $\epsilon_d < \theta $ such that after $r_d$ moves $\epsilon_d$ towards $r$, in the direction $\mathcal{D}$, $r_d$ and $r$ will not have any robot on the angle bisector of the angle formed by the points $R_d$ and $R$. Also, angular distance between $R_d$ and $R$ will be strictly less than $\alpha_0$ after the move of $r_d$ (i.e., $\theta - \epsilon_d < \alpha_0$). Let $P_{\alpha}$ and $Q_{\alpha}$ be the points of intersection of the circle $\mathcal{CIR}$ and the angle bisector of the angle formed by $R_d$ and $R$, after $r_d$ moves towards $r$ by an angular distance $\alpha$ in the direction $\mathcal{D}$. Here without loss of generality, it is assumed that $P_{\alpha}$ is the point on the major arc formed by $R_d$ and $R$ and $Q_{\alpha}$ is the point on the corresponding minor arc (Fig.\ref{fig:Lemma1}).
    
    \textbf{Case 1:} First, let us consider the case when $\theta = \alpha_0$. Let, $0< \alpha < \alpha_0$. Now, if $Q_{\alpha}$ is empty then we are done. So, let us assume for all positive $\alpha < \alpha_0$, $Q_{\alpha}$ is non-empty. Now consider the one-one correspondence $\alpha \mapsto Q_{\alpha}$. Now since $\alpha$ can have infinitely many values, there must be infinitely many $Q_{\alpha}$ which must be occupied by a robot in the configuration contrary to our assumption that there are finitely many robots on the circle $\mathcal{CIR}$. Hence, there must be some $\alpha$, say $ \epsilon_d$, in the open interval $(0, \alpha_0) = (0, \theta)$ such that, $Q_{\epsilon_d}$ is empty. Now since $\theta > \epsilon_d > 0 =  \theta - \alpha_0$, the condition, $0 < \theta -\epsilon_d < \alpha_0$ is satisfied.

    \textbf{Case 2:} Now let us consider the case when $\theta > \alpha_0$. Here $\epsilon_d$ must be in the open interval $(\theta-\alpha_0, \theta)$ for the condition $0 < \theta - \epsilon_d < \alpha_0$ to satisfy. Now let us consider the one-one correspondence $\alpha \mapsto Q_{\alpha}$ that maps all points which are at an angular distance $\alpha \in (\theta - \alpha_0, \theta)$. Again by a similar argument as in the previous case it can be shown that there is an $\alpha$, say $\epsilon_d \in (\theta - \alpha_0, \theta)$ such that $Q_{\epsilon_d}$ is empty. Also since $\theta - \alpha_0 < \epsilon_d < \theta$, the condition $0 < \theta -\epsilon_d <\alpha_0$ is also satisfied. Hence the result.
    \qed
    \end{proof}
\begin{figure}
    \centering
    \includegraphics[height=4.5cm]{Fig/Lemma1.eps}
    \caption{$r_d$ is the robot on $\mathcal{AB}$ of the double nominee configuration with two nominees $r_1$ and $r_2$. Here in this scenario the robot $r_d$ moves an angular distance $\epsilon_d$ to $D$.}
    \label{fig:Lemma1}
\end{figure}
\begin{Corollary}
\label{cor:LeaderConfigInOneEpoch}
 If the initial configuration is not the leader configuration, then within one epoch the configuration will be the leader configuration.
\end{Corollary}
\begin{proof}
% Since $\mathcal{C}(0)$ is not an leader configuration, proposition~\ref{prop:notLeaderConfigMeansRobotOnBisector} there must be two nominees, say $r_1$ and $r_2$, on $\mathcal{C}(0)$ and there must be a robot, say $r_d$, on the angle bisector $\mathcal{AB}$ of the angle formed by the robots $r_1$ and $r_2$. In this scenario, by Algorithm~\ref{} the robot $r_d$ must move an angular distance $\epsilon_d$ towards any of its neighbours, say $r$ in the direction $\mathcal{D}$, without colliding with in such a way that after the move the angle bisector of the angle formed by $r_d$ and $r$ does not contain any robot. Also after the move of angular distance $\epsilon_d$ by $r_d$, the angle formed by $r_d$ and $r$ in direction  $\mathcal{D}$ must be less than $\alpha_0$. Lemma~\ref{lemma:existenceEpsilond} guarantees the existence of such $\epsilon_d$. Now, if the configuration is a single nominee configuration after the move of $r_d$ then the configuration is also a leader configuration (by the Definition~\ref{def:leaderConfig}). So, let us consider that after $r_d$ moves an angular distance of $\epsilon_d$ towards $r$, the configuration is still a double nominee configuration. Note that in this configuration the nominees must be $r_d$ and $r$ as the angle between them in the direction $\mathcal{D}$ is the minimum and unique in the whole configuration. Also since $\epsilon_d$ in the previous round was chosen in such a way that there is no robot on $Q_{\epsilon_d}$ ( i.e., the angle bisector of the angle formed by $r_d$ and $r$ after $r_d$ moved $\epsilon_d$ towards $r$), the configuration after $r_d$ moves, must be a leader configuration.
Since $\mathcal{C}(0)$ is not a leader configuration, the robot $r_d$ on $\mathcal{AB}$  can move an angular distance $\epsilon_d > 0$ satisfying the condition $0 < \theta - \epsilon_d < \alpha_0$, towards its neighbor $r$ in the direction $\mathcal{D}$ such that after the move, the angle bisector of the angle formed by $R_d$ and $R$ does not contain any robot (By Lemma~\ref{lemma:existenceEpsilond}). Here, $\theta = (R_d, R)_{\mathcal{D}}$. Now note that after this move by $r_d$, if the configuration is a single nominee configuration then, by Definition~\ref{def:leader} and Definition~\ref{def:leaderConfig}, the configuration must be a leader configuration. Otherwise, if the configuration is not a single nominee configuration then, $r_d$ and $r$ must be the two nominees in the new configuration. But since they do not have any robot on $\mathcal{AB}$, the configuration must be a leader configuration.   
    \qed
\end{proof}

Thus, if the initial configuration is not a leader configuration, it will become one within one epoch. 

 \textbf{Taget embedding:} To embed the target pattern on the circle $\mathcal{CIR}$, a point and a direction on $\mathcal{CIR}$ must be agreed upon by all the robots. let $T_0$ be the point and $\mathcal{D}$ be the direction. Let the $j$-th target locations on $\mathcal{CIR}$ from $T_0$ in the direction $\mathcal{D}$ is denoted by $T_j$, where $j \in \{0, 1, \dots, n-1\}$. The points are embedded in such a way on $\mathcal{CIR}$ that $(T_j, T_{j+1})_{\mathcal{D}} = \beta_j$ where the sequence $\beta_0, \beta_1, \dots, \beta_{n-1}$ is lexicographically smallest upto rotation of the input pattern given to the robots (all the indices are considered in modulo $n$).

\begin{figure}
    \centering
    \includegraphics[height=4.5cm]{Fig/targetembedding.eps}
    \caption{ $(\beta_0, \beta_1,\dots \beta_{n-1})$ is the smallest in lexicographic ordering of all possible sequences that can be formed from the input upto rotation. The sequence is embedded on the circle starting from the location of $r_0$ and in the direction $\mathcal{D}_p$. }
    \label{fig:targetembedding}
\end{figure}
 
 In a leader configuration, the location of leader $r_0$, denoted as $R_0$ serves the purpose of $T_0$. We also claim that the pivotal direction $\mathcal{D}_p$ of $r_0$ must be unique, which serves the purpose of the direction on which all robots can agree. Thus, the target locations $T_j$ can be embedded on $\mathcal{CIR}$ as described above (Fig.~\ref{fig:targetembedding}).
 
 The next lemma ensures the claim that the leader $r_0$ must have a unique pivotal direction $\mathcal{D}_p$.
\begin{lemma}
    \label{lemma:uniquePivotalDirectionInLeaderConfig}
    In a leader configuration, the pivotal direction is unique.
\end{lemma}
\begin{proof}
    If possible let, the leader $r_0$ has two pivotal directions $\mathcal{D}$ and $\mathcal{D'}$. Then both of $\mathcal{AS}_{\mathcal{D}}(r_0)$ and $\mathcal{AS}_{\mathcal{D}'}(r_0)$ are minimum and equal. Let, $R_{1}$ and $R_{n-1}$ are the location of the neighbours of $r_0$ in the direction $\mathcal{D}$ and $\mathcal{D'}$ respectively. Then both the angles $(R_0,R_{1})_\mathcal{D}$ and $(R_0,R_{n-1})_\mathcal{D'}$ must be $\alpha_0$, where $\alpha_0$ is the minimum angle of the configuration. In this case $\mathcal{AS}_{\mathcal{D}}(r_{n-1})$ and $\mathcal{AS}_{\mathcal{D}'}(r_{1})$ both are strictly less than $\mathcal{AS}_{\mathcal{D}}(r_0) = \mathcal{AS}_{\mathcal{D'}}(r_0)$. So $r_0$ can't be the leader in the first place, which is not possible. Thus the pivotal direction of the leader must be unique.
    
    \qed
\end{proof}


Now in a leader configuration, let us denote the $i$-th robot from $r_0$, (i.e., the leader) in the direction $\mathcal{D}_p$ as $r_i$ and $(R_i, R_{i+1})_{\mathcal{D}_p}$ as $\alpha_i$, where all the indices are considered in modulo $n$. Note that $\alpha_0 \le \alpha_i$, for all $i$. Now, if there exists some $i \ne 0$ for which $\alpha_i = \alpha_0$ or, $\alpha_0 \ge \beta_0$ ($\beta_0$ is the smallest angle in the input target pattern according to the embedding), i.e., if $\alpha_0 \ge \underset{i \ne 0}{\min}\{\alpha_i, \beta_0\}$ then, the leader $r_0$  can always move in the direction $\mathcal{D}_p$, a positive angular distance $\epsilon < \alpha_0$ in such a way that $\alpha_0 - \epsilon < \underset{i \ne 0}{\min}\{\alpha_i, \beta_0\}$ and there is no robot on the angle bisector of the angle formed by $R_0$ and $R_1$, after $r_0$ moves (Fig.\ref{fig:Leadermove}). Observe that, $(R_0,R_1)_{\mathcal{D}_p} = \alpha_0-\epsilon$, after $r_0$ moves which is strictly less than all other angles in the current configuration (Here, $\alpha_0 = (R_0, R_1)_{\mathcal{D}_p}$ before $r_0$ moves). In the following lemma, we have proved the existence of such $\epsilon$. Note that after $r_0$ moves the configuration remains a leader configuration though the leader and the pivotal direction might change. Note that, if the leader changes then, it must be none other than the robot $r_1$ in the configuration before $r_0$ moved. But after the move we can assure that the new $\alpha_0$ must be strictly less than $  \underset{i\ne 0}{\min}\{\alpha_i, \beta_0\}$ i.e., strictly less than any other angle except the new $\alpha_0$ in the current configuration as well as the minimum angle of the target configuration.
\begin{figure}
    \centering
    \includegraphics[height=4.5cm]{Fig/leadermove.eps}
    \caption{Here $\alpha_i=\alpha_0$, so the leader $r_0$ moves towards $r_1$ and angular distance $\epsilon$ such that after the move the angular distance between $r_0$ and $r_1$ is strictly smallest in the current configuration and the inputs, also after moving the angle bisector of the angle formed by $r_0$ and $r_1$ doesn't contain any robot. }
    \label{fig:Leadermove}
\end{figure}

\begin{lemma}
\label{lemma:existenceofEpsilon}
 In a leader configuration, if $\alpha_0 \ge \underset{i\ne 0}{\min}\{\alpha_i,\beta_0\}$ where $\alpha_0$ is the angle between the leader $r_0$ and the neighbour of $r_0$, say $r_1$, in the direction $\mathcal{D}_p$ then, there exists a positive $\epsilon < \alpha_0 $ such that $\epsilon>\alpha_0-\underset{i\ne 0}{\min}\{\alpha_i,\beta_0\}$ and after moving $\epsilon$ angular distance towards $r_1$, the angle bisector of $r_0$ and $r_1$ does not contain any other robot.  
 \end{lemma}
\begin{proof}
Let us consider the open interval $(\alpha_0 - \underset{i \ne 0}{\min}\{\alpha_i, \beta_0\}, \alpha_0)$. Now for any $\alpha \in (\alpha_0 - \underset{i \ne 0}{\min}\{\alpha_i, \beta_0\}, \alpha_0)$, let $P_{\alpha}Q_{\alpha}$ be the angle bisector of the angle formed by $R_0$ and $R_1$, after $r_0$ moves an angular distance of $\alpha$ towards $r_1$. without loss of generality, it is assumed that $P_{\alpha}$ is the point on the major arc joining $R_0$ and $R_1$ and $Q_{\alpha}$ be the same on the minor arc after the move of $r_0$ by an angular distance of $\alpha$ towards $r_1$. Now considering the one-one correspondence $\alpha \mapsto Q_{\alpha}$ and with similar arguments as in Lemma~\ref{lemma:existenceEpsilond} we can prove that there is a $\alpha \in (\alpha_0 - \underset{i \ne 0}{\min}\{\alpha_i, \beta_0\}, \alpha_0)$, say $\epsilon$, such that the angle bisector of the angle formed by $R_0$ and $R_1$ after the move of $r_0$, does not have any robot on it. Also, since $ \in (\alpha_0 - \underset{i \ne 0}{\min}\{\alpha_i, \beta_0\}, \alpha_0)$, the conditions $\epsilon < \alpha_0$ and  $\epsilon>\alpha_0-\underset{i\ne 0}{\min}\{\alpha_i,\beta_0\}$ both are satisfied.
    \qed
\end{proof}


So, from a leader configuration within at most one epoch, the configuration changes into another leader configuration where the smallest angle $\alpha_0$ of the configuration is unique and also strictly less than $\beta_0$, the smallest angle of the target configuration. In this configuration if $\alpha_1 > \alpha_0$ is strictly less than all other $\alpha_j$ for all $j \ne 0, 1$ and also strictly less than $\beta_0$, i.e., $\alpha_1 < \underset{j \ne 0, 1}{\min}\{\alpha_j, \beta_0\}$ then we can ensure that the leader $r_0$ and the pivotal direction $\mathcal{D}_p$ never changes even if the robots $r_3,r_4,\dots, r_{n-1}$ moves according to the algorithm~\ref{algo:apfcircle}. This is because Algorithm  $APF\_CIRCLE$ ensures that no angle less or equal to $\alpha_1$ is formed during the moves by the above-mentioned robots.

 
  Now if in a leader configuration $\alpha_0 < \underset{i \ne 0}{\min}\{\alpha_i, \beta_0\}$ but $\alpha_1 \ge \underset{j \ne 0, 1}{\min}\{\alpha_j, \beta_0\}$ then, $r_2$ can always move an angular distance $\epsilon_1$ towards $r_1$ such that the resultant configuration is again a leader configuration and $\mathcal{D}_p$ does not change due to this movement and also, the new $\alpha_1$ in the resultant configuration after the move by $r_2$ must be strictly less than all $\alpha_j$ and $\beta_0$ where $j \ne 0,1$. So, the angle $(R_1,R_2)_{\mathcal{D}_p}$, after the move by $r_2$ becomes old $\alpha_1-\epsilon_1$ which is strictly less than $\underset{i\ne0,1}{\min}\{\alpha_i,\beta_0\}$ (Fig.\ref{fig:r2Moveanglemin}). The following lemma guarantees the existence of such $\epsilon_1$.

\begin{figure}[H]
     \centering
     \includegraphics[height=4cm]{Fig/rSecondMove.eps}
     \caption{Here $\alpha_1$ is not strictly smaller than other $\alpha_i$s (except $i=0$) or $\beta_j$s. So $r_2$ moves an angular distance $\epsilon_1$ towards $r_1$ such that the new $\alpha_1$ after the move becomes the second uniquely minimum angle of the configuration and also less than all $\beta_j$s. }
     \label{fig:r2Moveanglemin}
 \end{figure} 
\begin{lemma}
 If in a leader configuration $\alpha_0 < \underset{i \ne 0}{\min}\{\alpha_i, \beta_0\}$ and $\alpha_1\ge \underset{j \ne 0,1}{\min}\{\alpha_j,\beta_0\}$ then there exists $\epsilon_1>0$ such that $\alpha_1-\underset{j\ne0,1}{\min}\{\alpha_j,\beta_0\}<\epsilon_1<\alpha_1-\alpha_0$
\end{lemma}
\begin{proof}
    From the given conditions, 
    \begin{equation}\label{l1}
    \alpha_0<\min_{i\ne0}\{\alpha_i,\beta_0\}
    \ \& \ \alpha_1\ge\underset{j\ne0,1}{\min}\{\alpha_j,\beta_0\}.
    \end{equation}
    We have to find an $\epsilon_1>0$ such that it follows the following conditions:
    $$\
      \alpha_0<\alpha_1-\epsilon_1 \ \& \ \alpha_1-\epsilon_1<\underset{j\ne0,1}{\min}\{\alpha_j,\beta_0\}.$$
    Let us consider,
    $ \label{l4}
        \epsilon_1=\alpha_1-\frac{\alpha_0+\underset{i\ne0}{\min}\{\alpha_i,\beta_j\}}{2}
    $.
    Then from the inequality in (\ref{l1}), $\frac{\alpha_0+\underset{i\ne0}{\min}\{\alpha_i,\beta_j\}}{2}>\alpha_0$ and thus $\alpha_1-\epsilon_1>\alpha_0$.
    Hence first part is proved.
    \vspace{1cm}
    \\
    To prove the second part, if possible let, 
    $$
       \alpha_1-\epsilon_1\ge\underset{i\ne0,1}{\min}\{\alpha_i,\beta_0\} 
    $$
    Also since
    $\underset{i\ne 0}{\min}\{\alpha_i,\beta_0\}\le \underset{j\ne 0,1}{\min}\{\alpha_j,\beta_0\}$.
    We have the following inequality,
    $$
    \frac{\alpha_0+\underset{i\ne0}{\min}\{\alpha_i,\beta_j\}}{2} = \alpha_1-\epsilon_1\ge\underset{j\ne0,1}{\min}\{\alpha_j,\beta_0\}\ge \underset{i\ne 0}{\min}\{\alpha_i,\beta_0\}
    $$
    Which implies, $\alpha_0\ge \underset{i\ne 0}{\min}\{\alpha_i,\beta_0\}$. 
    % \begin{align*}
    %    ~& \frac{\alpha_0+\underset{i\ne0}{\min}\{\alpha_i,\beta_j\}}{2} = \alpha_1-\epsilon_1\ge\underset{j\ne0,1}{\min}\{\alpha_j,\beta_0\}\ge \underset{i\ne 0}{\min}\{\alpha_i,\beta_0\}\\
    %    i.e. & \frac{\alpha_0+\underset{i\ne0}{\min}\{\alpha_i,\beta_j\}}{2} \ge \underset{i\ne 0}{\min}\{\alpha_i,\beta_0\}\\
    %    i.e. & \alpha_0\ge \underset{i\ne 0}{\min}\{\alpha_i,\beta_0\}
    % \end{align*}
    Which is a contradiction to the inequality in (\ref{l1}).
    Thus $\alpha_1-\epsilon_1<\underset{j\ne0,1}{\min}\{\alpha_j,\beta_0\}$
    \qed 
\end{proof}

So, even after $r_2$ moves an angular distance $\epsilon_1$ towards $r_1$, the configuration still remains a leader configuration which satisfies the conditions that, $\alpha_0< \underset{i\ne0}{\min}\{\alpha_i, \beta_0\}$ and $\alpha_1< \underset{j\ne0,1}{\min}\{\alpha_j, \beta_0\}$. We call this kind of configuration a $RFC$. The definition of $RFC$ has been mentioned formally in the following:
\begin{definition}[Ready to Form Configuration ($RFC$)]
    A leader configuration is said to be a Ready to Form configuration, or a $RFC$, if it satisfies the conditions $\alpha_0< \underset{i\ne0}{\min}\{\alpha_i, \beta_0\}$ and $\alpha_1< \underset{j\ne0,1}{\min}\{\alpha_j, \beta_0\}$.
\end{definition}
% \begin{figure}
%     \centering
%     \includegraphics[height=4.5cm]{Fig/rfc.eps}
%     \caption{Ready to Form Configuration. Here $\alpha_0$ is uniquely smallest and $\alpha_1$ is uniquely second smallest among $\alpha_i$s. Also, both the angles are strictly smaller than all $\beta_j$s.}
%     \label{fig:rfc}
% \end{figure}

Other than $RFC$ we have another type of configuration called $PFC$ that need to be defined first.


% \begin{definition}[Partially Formed Configuration ($PFC$)]
%     A leader configuration is said to be a Partially Formed configuration ($PFC$) if all robots $r_p$, $p\ge3$ are on their target $T_p$.
% \end{definition}
% \begin{figure}
%     \centering
%     \includegraphics[height=4cm]{Fig/pfc.eps}
%     \caption{Partially Formed configuration, where all the robots $r_i$, $i\ge3$ are in target.}
%     \label{fig:pfc}
% \end{figure}
\begin{figure}
\centering

\begin{subfigure}{.45\textwidth}
  \centering
  \includegraphics[height=5.5cm]{Fig/rfc.eps}
  \caption{Ready to Form Configuration. Here $\alpha_0< \underset{i\ne0}{\min}\{\alpha_i, \beta_0\}$ and $\alpha_1<~ \underset{j\ne0,1}{\min}\{\alpha_j, \beta_0\}$.}
  \label{fig:rfc}
\end{subfigure}%
\hfill
\begin{subfigure}{.45\textwidth}
  \centering
  \includegraphics[height=5.5cm]{Fig/pfc.eps}
  \caption{Partially Formed configuration, where all the robots $r_i$, $i\ge3$ are in target.}
  \label{fig:pfc}
\end{subfigure}
\caption{RFC and PFC}
\label{fig:rfcpfc}
\end{figure}

Next when $RFC$ is achieved the Move Ready robot $r_i$ ($i \in {3, 4, \dots, n-1}$) moves to its corresponding target destination $T_i$ and terminates. Note that since $\alpha_0 <\alpha_1 < \underset{j \ne 0,1}{\min}\{\alpha_j,\beta_0\}$ and  the move of $r_i$ does not make any angle less or equal to $\alpha_1$, the configuration still remains a $RFC$ and the leader and pivotal direction $\mathcal{D}_p$ does not change. We now show that from an $RFC$ within at most $n-3$ epochs, the configuration must become a $PFC$. For that, we need to prove the following lemmas.

\begin{lemma}
\label{lemma:NotMoveReadySameDirection}
    In a $RFC$ if a robot $r_i$, $i\ge 3$ is not a Move Ready robot and the destination of $r_i$ i.e., $T_i$ is in the direction $\mathcal{D}$ from $R_i$ then, the neighbor of $r_i$ in the direction $\mathcal{D}$, say $r_k$, must also have its target destination $T_k$ in direction $\mathcal{D}$ from $R_k$.
\end{lemma}
\begin{proof}
    Let $r_i$ ($i \ge 3$) be a robot that is not Move Ready and its destination $T_i$ is in direction $\mathcal{D}$ from $R_i$. Let $r_k$ be the neighbour of $r_i$ in the direction $\mathcal{D}$ ($k$ can be either $i+1$ or, $i-1$ in modulo n). Since $r_i$ is not move ready, $(R_i,R_k)_{\mathcal{D}}-(R_i, T_i)_{\mathcal{D}} \le \alpha_1$. Now there can be two possibilities. Either, $(R_i,R_k)_{\mathcal{D}}-(R_i, T_i)_{\mathcal{D}} \le 0$ or, $0 < (R_i,R_k)_{\mathcal{D}}-(R_i, T_i)_{\mathcal{D}} \le \alpha_1$. Now, if possible let the destination of $r_k$ i.e., $T_k$ be in the direction $\mathcal{D}'$ from $R_k$.

    \textbf{Case 1:} Let $(R_i,R_k)_{\mathcal{D}}-(R_i, T_i)_{\mathcal{D}} \le 0$. This implies $T_i$ is further than $R_k$ in the direction $\mathcal{D}$, from $R_i$ (Fig.\ref{fig:lemma5a}). Now consider $k = i+1$ and thus the $\mathcal{D}= \mathcal{D}_p$. Note that $i$ then can not be $n-1$ as $T_{n-1}$ can not be further than $R_0=T_0$ in the direction $\mathcal{D}_p$ from $R_{n-1}$ according to the target embedding. Now, for all other values for $i \ge 3$, if $T_{i+1}$ is in the direction $\mathcal{D}_p'$ from $R_{i+1}$, then $T_{i+1}$ appears before $T_i$ in the direction $\mathcal{D}_p$ in the embedding which is contradiction. Similarly, let us consider $k= i-1$ and thus $\mathcal{D} = \mathcal{D}_p'$. Here note that $i$ can not be $3$ as otherwise $r_3$ is Move Ready. This is because $T_3$ and $T_2$ must be on the arc joining from $R_2$  to $R_3$ in the direction $\mathcal{D}_p$. Thus $T_3$ can not be further than $R_2$ from $R_3$ in the direction $\mathcal{D}_p'$ as, $(T_3,R_2)_{\mathcal{D}_p'} > (T_3, T_2)_{\mathcal{D}_p'} =\beta_2 > \alpha_1 >0$ . Now for all other values of $i >3$, it can be shown that we will arrive at a contradiction by a similar argument as in the case where $k=i+1$ has been considered.

\begin{figure}
\centering
\begin{subfigure}{.5\textwidth}
  \centering
  \includegraphics[height=7cm]{Fig/lemma5a.eps}
  \caption{case:1}
  \label{fig:lemma5a}
\end{subfigure}%
\hfill
\begin{subfigure}{.5\textwidth}
  \centering
  \includegraphics[height=7cm]{Fig/lemma5b.eps}
  \caption{case:2}
  \label{fig:lemma5b}
\end{subfigure}
\caption{If the target destination $T_i$ of the robot $r_i$ is in the direction $\mathcal{D}$, then the target destination $T_{i+1}$ of its neighbour $r_{i+1}$ is also in the same direction $\mathcal{D}$. }
\label{fig:test}
\end{figure}


    \textbf{Case 2:} Let $0 < (R_i,R_k)_{\mathcal{D}}-(R_i, T_i)_{\mathcal{D}} \le \alpha_1$. This implies $R_k$ is further than $T_i$ from $R_i$ in the direction $\mathcal{D}$ but, $(T_i, R_k)_{\mathcal{D}} \le \alpha_1$ (Fig.~\ref{fig:lemma5b}). Let $k= i+1$ and hence $\mathcal{D}= \mathcal{D}_p$ ($i$ can not be $n-1$ as shown earlier in case 1). Now, according to the embedding $T_i$ can not be further than $T_{i+1}$ from $T_0$ in the direction $\mathcal{D}_p$. Hence, $T_{i+1}$ must be on the arc joining from $T_i$ to $R_{i+1}$ in the direction $\mathcal{D}_p$. This implies $\beta_0 \le (T_i, T_{i+1})_{\mathcal{D}_p} \le (T_i, R_{i+1})_{\mathcal{D}_p} \le \alpha_1 \implies \beta_0 \le \alpha_1$, a contradiction due to the fact that the configuration is a $RFC$. Similarly if $k=i-1$ and hence the direction $\mathcal{D}= \mathcal{D}_p'$ then again we will arrive at a contradiction by a similar argument. 

    Since for both the possibilities we arrive at a contradiction, $T_k$ must also be in the direction of $\mathcal{D}$ from $R_k$.
    \qed
\end{proof}
\begin{lemma}
\label{lemma:ATleastoneMoveReady}
    If a $RFC$, is not a $PFC$ then there exists a robot $r_p$ which is Move Ready.
\end{lemma}
\begin{proof}
A robot is called terminated if it has already reached its target. If possible let in a $RFC$ the robots $r_i$ ($i \ge 3$) are either terminated or not Move Ready. Let $r_k$ be a  robot from $R_0$ in the direction $\mathcal{D}_p$ which has not terminated and is not Move Ready. Let the target of $r_k$ i.e., $T_k$ be in a direction $\mathcal{D}$ from $R_k$. Observe that if $r_k = r_3$, then $\mathcal{D} = \mathcal{D}_p$. Otherwise, since $T_2$ is in the direction $\mathcal{D}_p$ from $R_2$, $(R_3,R_2)_{\mathcal{D}_p'}- (R_3, T_3)_{\mathcal{D}_p'} = (R_2,T_3)_{\mathcal{D}_p}\ge (T_2,T_3)_{\mathcal{D}_p} = \beta_2 > \alpha_1$ and hence $r_3$ becomes Move Ready. Similarly if $r_k = r_{n-1}$ then, $T_{n-1}$ must be in the direction $\mathcal{D}_p'$ from $R_{n-1}$. Otherwise, $T_{n-1}$ must lie on the arc joining $R_{n-1}$ and $T_0 = R_0$ in the direction $\mathcal{D}_p$ which implies $(R_{n-1},R_0)_{\mathcal{D}_p}-(R_{n-1},T_{n-1})_{\mathcal{D}_p} = (T_{n-1},T_0)_{\mathcal{D}_p} = \beta_{n-1} > \alpha_1$ a contradiction. 

Now we claim that, For a robot $r_i, (i \ge 3)$ which has not terminated and is not move ready, if the direction of its target is in the direction $\mathcal{D}$ from $r_i$, then the neighbor of $r_i$ , say $r_j$ in the direction $\mathcal{D}$ must have not terminated also. Otherwise, if $r_j$ is terminated then it must be on $T_j$. Also, $T_i$ must be on the arc joining the points from $R_i$ to $T_j$ in the direction $\mathcal{D}$. This implies $(R_i,R_j)_{\mathcal{D}}-(R_i,T_i)_{\mathcal{D}} = (T_i,T_j)_{\mathcal{D}} = \beta_t >\alpha_1$ ($t \in \{i,j\}$) and thus $r_i$ becomes move ready contrary to the assumption. 

So, now for a robot $r_{k_1}$ which is not Move Ready and has not terminated yet, let $\mathcal{D}$ be the direction of $T_{k_1}$ from $R_{k_1}$. Also let $r_{k_2}$ be the neighbour of $r_{k_1}$ in the direction $\mathcal{D}$. By Lemma~\ref{lemma:NotMoveReadySameDirection} and the above claim $r_{k_2}$ must have not terminated yet and the direction of $T_{k_2}$ must be in the direction $\mathcal{D}$ from $R_{k_2}$. Now by mathematical induction, it can be shown that all robots $r_i$ ($i \in \{3,4,\dots,n-1\}$) in the direction $\mathcal{D}$ from $r_k$, must have not terminated and are not Move Ready. So either $r_3$ or $r_{n-1}$ must be not Move Ready and has not terminated. if $r_3$ is not move ready and has not terminated then the direction of $T_3$ must be $\mathcal{D}_p$ from $R_3$ and then by induction it can be shown that $r_{n-1}$ must also be not Move ready and has not terminated and direction of $T_{n-1}$ must be in $\mathcal{D}_p$ from $R_{n-1}$ which is a contradiction. Similarly, if $r_{n-1}$ is not Move ready and has not terminated then $T_{n-1}$ must be in the direction $\mathcal{D}_p'$ from $R_{n-1}$ which will imply $r_3$ is not Move Ready and is not terminated and $T_3$ must be in direction $\mathcal{D}_p'$ from $R_3$. which is again a contradiction. Hence the result. 

    \qed
\end{proof}

Now as a direct corollary of the above Lemma~\ref{lemma:ATleastoneMoveReady}, we can conclude the following.
\begin{Corollary}
\label{Cor:EachRoundARobotOnTarget}
    If a $RFC$ is not a $PFC$ then, in each epoch at least one of the robots, say $r_p$, $p\ge3$ must reach its target $T_p$.
\end{Corollary}
Now we are in a shape to establish the claim of a $RFC$ becoming a $PFC$ within $n-3$ epochs using the above Corollary~\ref{Cor:EachRoundARobotOnTarget}. Since in each epoch, at least one of $r_i$ ($i \ge 3$) is moving to its target $T_i$, and there can be at most $n-3$ such robots, we can ensure that within at most $n-3$ epochs, an $RFC$ must become a $PFC$. We formalize this in the following theorem.
\begin{theorem}
    From the initial configuration, within $n$ epochs the configuration will become a $PFC$.
\end{theorem}
\begin{proof}
    From initial configuration, $RFC$ is achieved in at most 3 epochs and from $RFC$ to $PFC$ within $n-3$ epochs. Hence the theorem.
    \qed
\end{proof}

Now when the $PFC$ has been achieved, then only $r_1$ and $r_2$ are not in $T_1$ and $T_2$. Note that since $\alpha_1 < \underset{j \ne 0,1}{\min}\{\alpha_j, \beta_0\} \le \beta_1$, $r_1$ can not move even if it is activated earlier or together with $r_2$. On the other hand when activated $r_2$ first checks if its move to $T_2$ might change the leader and the pivotal direction. Note that if $r_2$ finds that $\beta_0+\beta_1-\alpha_0 < \beta_{n-1}$ then moving to its target $T_2$ does not change the leader and the pivotal direction. So, in this case, it moves to $T_2$. Otherwise, $r_2$ moves to a point, say $D_2$ such that $(R_1,D_2)_{\mathcal{D}_p} =\beta_{n-1}-\delta$, where $\delta \in (0, \beta_{n-1}-\beta_0+\alpha_0)$ (Fig.\ref{fig:lemma7dscrp.eps}). In the following lemma, we have justified that any of such moves by $r_2$ does not change the leader and the pivotal direction. 

\begin{figure}
    \centering
    \includegraphics[height=8cm]{Fig/lemma7dscrp.eps}
    \caption{If $\beta_0+\beta_1-\alpha_0 \ge \beta_{n-1}$, then $r_2$ moves to the point $D_2$ in $\mathcal{D}_p$ such that $(R_1,D_2)_{\mathcal{D}_p}$ marked here with red arc, equals to $\beta_{n-1}-\delta$.}
    \label{fig:lemma7dscrp.eps}
\end{figure}

\begin{lemma}
\label{lemma:MoveofR2dontChangeLeader}
    In a $PFC$, after $r_2$ moves for the first time, the leader and its pivotal direction do not change.
\end{lemma}
\begin{proof}
    In a $PFC$, $r_2$ always moves before $r_1$ even if $r_1$ is activated earlier or together with $r_2$. When $r_2$ moves for the first time in a $PFC$, $r_1$ is not on $T_1$, and $r_2$ moves to $D_2$. Now, either $D_2=T_2$ or, it is a point on the circle such that,  $(R_1,D_2)_{\mathcal{D}_p}=\beta_{n-1}-\delta$, where $0 <\delta<\beta_{n-1}-\beta_0+\alpha_0$. As $\alpha_0<\alpha_1<\beta_0\le\beta_1$, then the destination $D_2$ of $r_2$ must be on the arc joining from point $R_2$ to point $R_3=T_3$ in the direction $\mathcal{D}_p$. 

\textit{Case-I:} If $(R_1,T_2)_{\mathcal{D}_p}=\beta_0+\beta_1-\alpha_0<\beta_{n-1}$, then $r_2$ moves to its target $T_2$. Then the angle sequence of $r_0$ in the pivotal direction remains uniquely minimum, as $(R_1,T_2)_{\mathcal{D}_p}<\beta_{n-1}$. Thus in this case the leader and the pivotal direction will not change.

\textit{Case-II:} If $(R_1,T_2)_{\mathcal{D}_p}=\beta_0+\beta_1-\alpha_0 \ge \beta_{n-1}$, then $r_2$ moves to a point $D_2$ in the direction $\mathcal{D}_p$ such that $(R_1,D_2)_{\mathcal{D}_p}$ must be $\beta_{n-1}-\delta$. Then the minimum angle sequence of the configuration remains unique and belongs to $\mathcal{AS}(r_0)$ and the pivotal direction also remains same as $\beta_{n-1}>\beta_{n-1}-\delta$, for any $\delta \in (0, \beta_{n-1}-\beta_0+\alpha_0)$.\qed  
\end{proof}

% Now we have to show the existence of such $\delta$.
After $r_2$ moves for the first time in a $PFC$, $r_1$ will now move to its target $T_1$. Note that until $r_1$ reaches $T_1$, $r_2$ will not move even if it is activated according to the algorithm~\ref{algo:apfcircle}. It has to be ensured that after $r_2$ moves for the first time $r_1$ will move when activated and after $r_1$ moves to $T_1$ the leader and pivotal direction will remain the same. The following lemmas will establish these claims. 


\begin{lemma}
\label{lemma:r1Moves}
    In a $PFC$, after $r_2$ moves for the first time, the condition $(R_1,R_2)_{\mathcal{D}_p} > \beta_1$ must becomes true.
\end{lemma}
\begin{proof}
    In a $PFC$, $r_2$ if moves for the first time either moves to $T_2$ or, moves to $D_2$ such that $(R_1,D_2)_{\mathcal{D}_p} = \beta_{n-1}-\delta$ for some $\delta \in (0, \beta_{n-1}-\beta_0+\alpha_0)$.
    
    \textit{Case-I:} Let $r_2$ has moved to $T_2$. We now have to show that $(R_1, T_2)_{\mathcal{D}_p} > \beta_1$.
    If $(R_1, T_2)_{\mathcal{D}_p} \le \beta_1$ then, $\beta_0+\beta_1-\alpha_0 \le \beta_1 \implies  \alpha_0\ge \beta_0 $ which is a contradiction. Hence  $(R_1, T_2)_{\mathcal{D}_p} > \beta_1$.

    \textit{Case-II:} Now let us consider the case where $r_2$ has moved to $D_2$ such that $(R_1, D_2)_{\mathcal{D}_p} = \beta_{n-1}-\delta$, where $\delta \in (0, \beta_{n-1}-\beta_0+\alpha_0)$. Observe that according to the target embedding $\beta_{n-1} \ge \beta_1$. This implies for any $\delta \in (0, \beta_{n-1}-\beta_0+\alpha_0)$, $(R_1,D_2)_{\mathcal{D}_p} = \beta_{n-1}-\delta \ge \beta_1 -\delta > \beta_1$ and hence the result.
    \qed
\end{proof}

% \begin{figure}
%     \centering
%     \includegraphics[height=4.5cm]{Fig/r1move.eps}
%     \caption{After $r_2$ moves for the first time $r_1$ moves to its target without collision.}
%     \label{fig:r1move}
% \end{figure}

\begin{figure}
\centering
\begin{subfigure}{.45\textwidth}
  \centering
  \includegraphics[height=5cm]{Fig/r1move.eps}
  \caption{After $r_2$ moves for the first time $r_1$ moves to its target without collision.}
  \label{fig:r1moves}
\end{subfigure}%
\hfill
\begin{subfigure}{.45\textwidth}
  \centering
  \includegraphics[height=5cm]{Fig/lemma8.eps}
  \caption{When $r_1$ is on target $r_2$, if already not on target, moves to target $T_2$.}
  \label{fig:r2finalMove}
\end{subfigure}
\caption{$r_1$ and $r_2$ moves to target.}
\end{figure}
In a $PFC$ after $r_2$ moves for the first time, it will never move again until $r_1$ reaches its target $T_1$. By the Lemma~\ref{lemma:r1Moves}, if $r_2$ moves for the first time, then the condition $(R_1,R_2)_{\mathcal{D}_p} > \beta_1$ must becomes true, which makes sure that $r_1$ can move to $T_1$ when activated again. We now ensure that to reach $T_1$, $r_1$ does not have to cross $D_2$. For that we have to show that $(T_1,D_2)_{\mathcal{D}_p} >0$. If possible let, $(T_1,D_2)_{\mathcal{D}_p} = \beta_{n-1}-\delta-\beta_0+\alpha_0 \le 0$, then, $\delta \ge \beta_{n-1}-\beta_0+\alpha_0$ contradicting the choice of $\delta$. Hence to reach $T_1$, $r_1$ does not have to cross $D_2$. Thus $r_1$ moves to $T_1$ (Fig.\ref{fig:r1moves}). We now show that this move by $r_1$ does not change the leader and the pivotal direction if the target is already not formed.

\begin{lemma}
\label{lemma:AlmostFormed}
    If in a $PFC$, $r_1$ moves to its target $T_1$, then either pattern formation is complete or, the leader and the pivotal direction don't change.
\end{lemma}
\begin{proof}
    If $r_2$ already moved to $T_2$ during the first time it moved in a $PFC$ then, after $r_1$ moves to $T_1$, the target is formed already. So let us consider, in a $PFC$, $r_2$ during its first move moved to $D_2$ in such a way that $(R_1, D_2)_{\mathcal{D}_p} = \beta_{n-1}-\delta$, where $\delta \in (0, \beta_{n-1}-\beta_0+\alpha_0)$. This implies $\beta_0+\beta_1-\alpha_0 \ge \beta_{n-1}$. Now, by Lemma~\ref{lemma:r1Moves}, $r_1$ moves to $T_1$ and $r_2$ is on $D_2$. So after the move of $r_1$ the angle between them will be  $(T_1,D_2)_{\mathcal{D}_p}$. If possible let the leader or the pivotal direction has changed by this move. This implies $(T_1,D_2)_{\mathcal{D}_p} = \beta_{n-1}-\delta-\beta_0+\alpha_0 \ge \beta_1 \implies \beta_{n-1}-\delta \ge \beta_0+\beta_1-\alpha_0 \ge \beta_{n-1}$ which is a contradiction. Hence the leader and pivotal direction do not change.
    \qed
\end{proof}

\begin{algorithm}
\caption{$APF\_CIRCLE$}
\label{algo:apfcircle}
\LinesNumbered

 $r \gets$ myself.\;
 \uIf{the configuration is not leader configuration}
     {
        $r_d \gets $ robot on the angle bisector $\mathcal{AB}$\;
        \If{I am $r_d$}
         {
             move towards any neighbour by a small angle $\epsilon$ such that, $\epsilon < \alpha_0$ and $r_d$ and its neighbours don,t have any other robot on their angle bisector\;
            
         }
     }
\Else{
      $r_i \gets$ the $i$-th robot from leader in the direction $\mathcal{D}_p$\;  
       $T_i \gets$ target destination of $r_i$\; 
     \uIf{I am the leader}
          {
            \If{$\alpha_0\ge\underset{i \ne 0}{\min}\{\alpha_i,\beta_0\}$}
                {
                     choose $\epsilon \in (\alpha_0 - \underset{i \ne 0}{\min}\{\alpha_i, \beta_0\})$ such that even after leader move $\epsilon$ angular distance towards $r_1$ the configuration remains a leader configuration \;
                     move by an angular distance $\epsilon$ towards $r_1$\;
                }
          }
      \Else
          {
            
            \uIf{I am $r_2$}
               {
                    \uIf{all robots $r_p$ are in $T_p$, for all $p\ge 3$}
                    {
                        \uIf{ $\beta_0+\beta_1-\alpha_0 < \beta_{n-1}$}
                        {
                            move to $T_2$\;
                        }
                        \Else
                        {
                            \uIf{$r_1$ is not on $T_1$}
                          {
                            choose $\delta  \in (0,\beta_{n-1}-\beta_0+\alpha_0)$\; 
                            move to an angular distance $\alpha_0+\beta_{n-1}-\delta$ from $r_0$ in direction $\mathcal{D}_p$\;
                          }

                         \Else{
                            move to $T_2$.\;
                          }
                        }
                       
                    }
                    \Else 
                    {
                        \If{$\alpha_0 < \underset{i\ne 0}{\min}\{\alpha_i,\beta_0\}$}
                        {
                            
                            \If{$\alpha_1\ge\underset{j \ne 0,1}{\min}\{\alpha_j,\beta_0\}$}
                           {
                                choose $\epsilon_1 \in (\alpha_1- \underset{j \ne 0,1}{\min}\{\alpha_j,\beta_0\}, \alpha_1-\alpha_0)$\;
                                move by an angular distance $\epsilon_1$ towards $r_1$\;
                           }
                        }
                    }     
               }
            \uElseIf{I am $r_1$}
              {
                \If{all robots $r_p$ are in $T_p$, for all $p\ge 3$}
                    {   
                        \If{$(R_1,R_2)_{\mathcal{D}_p}>\beta_1$}
                           {move to $T_1$\;}
                        % move to an angle $\beta_0$ from leader in the direction $\mathcal{D}_p$\; 
                    }
              }
            \Else{
                \If{$\alpha_0 <  \alpha_i, \forall i \ne 0$ \textbf{and} $\alpha_0 < \alpha_1 < \alpha_i, \forall i \ne 0,1$ \textbf{and} $\alpha_0,\alpha_1 < \beta_j, \forall j$}
                {
                  \If{I am $r_p$ and I am Move Ready}
                  {
                    % $\theta \gets \sum_{j=p}^{n-1} \beta_{j}$\;
                    % $\theta' \gets \sum_{i=p}^{n-1} \alpha_{i}$\;
                    % $\theta_{res} \gets |\theta-\theta'|$\;
                    % \uIf{$\theta > \theta'$}
                    % {
                    %     \If{there is no robot within $\theta_{res}$ from me in the direction $\mathcal{D}_p'$}
                    %     {
                    %         move $\theta_{res}$ in the direction $\mathcal{D}_p'$\;
                    %     }
                    % }
                    % \Else
                    % {
                    %     \If{there is no robot within $\theta_{res}$ from me in the direction $\mathcal{D}_p$}
                    %     {
                    %         move $\theta_{res}$ in the direction $\mathcal{D}_p$\;
                    %     }
                    % }
                    Move to $T_p$\;
                    
                  }
                }
            }
            
           
          }
     

}     
\end{algorithm}

 Now if the pattern formation is already done then upon activation again, if a robot sees that the smallest angle sequence of the configuration is the given lexicographically smallest angle sequence (upto rotation) then it terminates. Note that after pattern formation is done within one epoch all robots will terminate.
  Now if the pattern is not formed even after $r_1$ moved in a $PFC$, that implies $r_2$ is not on $T_2$. Note that since the leader and the pivotal direction remain the same the target embedding also remains the same. In this scenario, upon activation, $r_2$ moves to $T_2$ (Fig.\ref{fig:r2finalMove})target pattern is formed. So from a $PFC$ within at most $4$ more epochs, all the robots terminate knowing the pattern has been formed. Now, From all of these above descriptions, we can now conclude the following theorem.

% \begin{figure}
%     \centering
%     \includegraphics[height=4cm]{Fig/lemma8.eps}
%     \caption{Caption}
%     \label{fig:lemma8}
% \end{figure}

  \begin{theorem}
      \label{thm:finalthm}
      $n=2k+1$ oblivious and silent robots placed on a circle can solve arbitrary pattern formation problem on a circle by executing the algorithm $APF\_CIRCLE$ within at most $n+4$ epochs, even without chirality agreement under a semi-synchronous scheduler iff the initial configuration is rotationally asymmetric.
  \end{theorem}
% \begin{lemma}
%     If $\alpha_0< \underset{i\ne0}{\min\{\alpha_i. \beta_0\}}$ and $\alpha_1< \underset{i\ne0,1}{\min\{\alpha_i. \beta_0\}}$, then any movement of $r_p$, $p>2$ does not change the leader and the pivotal direction.
% \end{lemma}
% \begin{lemma}
%     If $\alpha_0< \underset{i\ne0}{\min\{\alpha_i. \beta_0\}}$ and $\alpha_1< \underset{i\ne0,1}{\min\{\alpha_i. \beta_0\}}$, then by a single move any robot $r_p$,$p>2$ reaches to its destination.
% \end{lemma}

%  \begin{lemma}
%      If a robot is the second neighbor of the leader in the pivotal direction, then within at most two moves it will terminate.
%  \end{lemma}


\section{Randomized $APF$ for any $n \ge3$}
In the arbitrary pattern formation problem on a circle, if the leader configuration can be achieved for any $n$ number of robots, then the algorithm $APF\_CIRCLE$ can solve the problem of pattern formation.  
\par Now let there are $n=2k$ numbers of robots on the circle in the initial configuration $\mathcal{C}(0)$. If there is a single nominee then the configuration is leader configuration. Otherwise, the configuration must be a double nominee configuration. Let, $r_{n1}$ and $r_{n2}$ be two nominees of the configuration $\mathcal{C}(0)$ which have their minimum angle sequences in the directions $\mathcal{D}_{n1}$ and $\mathcal{D}_{n2}$ respectively. $\mathcal{D}_{n1}$ and $\mathcal{D}_{n2}$ can be same or different. Let, $\mathcal{AB}$ be the angle bisector of the angle between $r_{n1}$ and $r_n2$, which divides the circle into two arcs. Now if $Arc(r_{n1})$ contains more robots than $Arc(r_{n2})$, then $Arc(r_{n1})$ becomes the leader and leader configuration will be achieved. But if $Arc(r_{n1})$ and $Arc(r_{n2})$ contain the same number of robots then leader configuration can't be achieved from double nominee configuration.
\par Let us consider the case where $Arc(r_{n1})$ and $Arc(r_{n2})$ contain same number of robots. As the number of robots is even, then on the angle bisector $\mathcal{AB}$ either there are two robots or no robot. Then the two nominees $r_{n1}$ and $r_{n2}$ randomly choose a positive $\epsilon$ independently in the directions of their minimum angle sequences and move the angular distance $\epsilon$ towards its neighbor in the direction of its minimum angle sequence. For each nominee, $\epsilon$ is chosen independently. So the probability of their choice of the same $\epsilon$ is $0$. Thus for a different choice of $\epsilon$, one of the angle sequences of a nominee must be minimum and that will be considered as the leader, and leader configuration will be achieved. So a probabilistic algorithm can be given to solve the arbitrary pattern formation problem on the circle for an even number of robots. 
\section{Conclusion}
The arbitrary pattern formation problem is a classical problem in the field of swarm robotics. Till now it has been studied considering the euclidean plane and some discrete domains mostly. In continuous domains, there are certain environments that restrict the movement of the robot in any direction. Any closed curve embedded on a plane is an example of this. In the real world, this kind of environment can be seen everywhere, for example, train lines, road networks, etc. It can be argued that a problem solvable in a continuous circle can be solved on any closed curve. So here, in this paper, this problem has been introduced on a continuous circle for the first time.  Here, considering a rotationally asymmetric initial configuration having an odd number of oblivious and silent robots, a distributed deterministic algorithm $APF\_CIRCLE$ has been provided that solves the arbitrary pattern formation problem on a continuous circle of fixed radius within $O(n)$ epochs of time under a semi-synchronous scheduler. To justify the initial rotationally asymmetric configuration, it has been also been shown that if the initial configuration is rotationally symmetric then, no deterministic algorithm exists that solves the arbitrary pattern formation problem on a circle. Furthermore, by tweaking the deterministic algorithm a little bit, a probabilistic algorithm has been described that solves arbitrary pattern formation problem for any $n$ number of robots on a circle, where $n \ge 3$. 

For the days ahead, it would be really interesting if this problem can be solved under an asynchronous scheduler. Also, another interesting thing to check would be to find out if there is an initial configuration and a target configuration such that for any embedding of the target the time taken by the $n$ robots to form the target is $O(n)$ or not. If this lower bound is $O(n)$ then the algorithm presented here is time optimal otherwise another time the optimal algorithm has to be designed. Furthermore, for a swarm of an even number of robots our algorithm is not deterministic. So, it would be another way to carry forward this research to find out if there is any deterministic algorithm for an even number of robots.  
%
%
%

% ---- Bibliography ----
%
% BibTeX users should specify bibliography style 'splncs04'.
% References will then be sorted and formatted in the correct style.
%
   \bibliographystyle{splncs04}
   \bibliography{samplepaper}
%

\end{document}
