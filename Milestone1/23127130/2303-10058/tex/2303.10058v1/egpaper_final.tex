\documentclass[10pt,twocolumn,letterpaper]{article}

\usepackage{iccv}
\usepackage{times}
\usepackage{epsfig}
\usepackage{graphicx}
\usepackage{amsmath}
\usepackage{amssymb}


% Added packages
\usepackage{graphicx}
% \usepackage{subfigure}
\usepackage{booktabs} % for professional tables
\usepackage{mathtools}
\usepackage{amsthm}
\usepackage{newfloat}
\usepackage{listings}
% \usepackage{mathtools}
\usepackage{multirow}
\usepackage{bbm}
\usepackage{enumitem}
% \usepackage{subcaption}
\usepackage[T1]{fontenc}
\usepackage[table]{xcolor}
\usepackage{subfigure}
\usepackage{cancel}
% \usepackage{ulem}
% \newcommand*{\fullref}[1]{\namecref{#1} \nameref*{#1}}
\usepackage{wrapfig}
\usepackage{appendix}
% \usepackage{lineno}
\usepackage{units}
\usepackage{xspace}
\usepackage{xcolor}
\usepackage{enumerate}
\usepackage{authblk}
\usepackage[ruled]{algorithm2e}

\theoremstyle{plain}
\newtheorem{theorem}{Theorem}[section]
\newtheorem{proposition}[theorem]{Proposition}
\newtheorem{lemma}[theorem]{Lemma}
\newtheorem{corollary}[theorem]{Corollary}
\theoremstyle{definition}
\newtheorem{definition}[theorem]{Definition}
\newtheorem{assumption}[theorem]{Assumption}
\theoremstyle{remark}
\newtheorem{remark}[theorem]{Remark}

% for notations
\newcommand{\x}{\mathbf{x}}
\newcommand{\y}{\mathbf{y}}
\newcommand{\z}{\mathbf{z}}
\newcommand{\f}{\mathbf{f}}
\newcommand{\h}{\mathbf{h}}
\newcommand{\w}{\mathbf{w}}
\newcommand{\m}{\mathbf{m}}

\newcommand{\F}{{\mathbf{F}}}
\newcommand{\W}{{\mathbf{W}}}
\newcommand{\U}{{\mathbf{U}}}
\newcommand{\M}{{\mathbf{M}}}
\renewcommand{\H}{{\mathbf{H}}}

\newcommand{\X}{\bm{\mathbf{X}}}
\newcommand{\Y}{\bm{\mathbf{Y}}}
\newcommand{\A}{\bm{\mathbf{A}}}
\renewcommand{\P}{\bm{\mathbf{P}}}

\newcommand{\<}{\left\langle}
\renewcommand{\>}{\right\rangle}
\newcommand{\prox}{\mbox{prox}}
\newcommand{\sign}{\mbox{sign}}
\newcommand{\proj}{\operatorname{Proj}}

\newcommand{\FedAvg}{\textsc{FedAvg}\xspace}
\newcommand{\FedNH}{\textsc{FedNH}\xspace}
\newcommand{\FedProto}{\textsc{FedProto}\xspace}
\newcommand{\Ditto}{\textsc{Ditto}\xspace}
\newcommand{\FedRep}{\textsc{FedRep}\xspace}
\newcommand{\FedRoD}{\textsc{FedRoD}\xspace}
\newcommand{\FedETF}{\textsc{FedETF}\xspace}
\newcommand{\CCVR}{\textsc{CCVR}\xspace}
\newcommand{\FedDyn}{\textsc{FedDyn}\xspace}
\newcommand{\FedProx}{\textsc{FedProx}\xspace}

\def\bx{\mathbf{x}}
\def\bs{\mathbf{s}}
\def\bt{\mathbf{t}}
\def\bw{\mathbf{w}}
\def\bz{\mathbf{z}}
\def\bv{\mathbf{v}}
\def\DD{\mathcal{D}}
\def\CC{\mathcal{C}}
\def\DS{\mathcal{S}}
\def\LL{L}
\def\bu{\mathbf{u}}
\def\bv{\mathbf{v}}
\def\bg{\mathbf{g}}
\def\bh{\mathbf{h}}
\def\bp{\mathbf{p}}
\def\bmu{\boldsymbol{\mu}}
\def\bbeta{\boldsymbol{\beta}}

% Include other packages here, before hyperref.

% If you comment hyperref and then uncomment it, you should delete
% egpaper.aux before re-running latex.  (Or just hit 'q' on the first latex
% run, let it finish, and you should be clear).
\usepackage[breaklinks=true,bookmarks=false]{hyperref}

\iccvfinalcopy % *** Uncomment this line for the final submission

\def\iccvPaperID{****} % *** Enter the ICCV Paper ID here
\def\httilde{\mbox{\tt\raisebox{-.5ex}{\symbol{126}}}}

% Pages are numbered in submission mode, and unnumbered in camera-ready
% \ificcvfinal\pagestyle{empty}\fi

\begin{document}

%%%%%%%%% TITLE
% \vspace{-20.7cm}
\title{No Fear of Classifier Biases: Neural Collapse Inspired Federated Learning with Synthetic and Fixed Classifier}

% \vspace{-10.7cm}
\author{Zexi Li$^1$\qquad Xinyi Shang$^2$\qquad Rui He$^1$\qquad Tao Lin$^3$\thanks{Corresponding authors.}\qquad Chao Wu$^{1*}$\\
$^1$Zhejiang University\qquad $^2$Xiamen University\qquad $^3$Wesklake University\\
{\tt\small \{zexi.li,ruihe,chao.wu\}@zju.edu.cn}\quad {\tt\small shangxinyi@stu.xmu.edu.cn}\quad {\tt\small lintao@westlake.edu.cn}
}


\maketitle
% Remove page # from the first page of camera-ready.
% \ificcvfinal\thispagestyle{empty}\fi

%%%%%%%%% ABSTRACT
\begin{abstract}
Data heterogeneity is an inherent challenge that hinders the performance of federated learning (FL). Recent studies have identified the biased classifiers of local models as the key bottleneck. Previous attempts have used classifier calibration after FL training, but this approach falls short in improving the poor feature representations caused by training-time classifier biases. Resolving the classifier bias dilemma in FL requires a full understanding of the mechanisms behind the classifier. Recent advances in neural collapse have shown that the classifiers and feature prototypes under perfect training scenarios collapse into an optimal structure called simplex equiangular tight frame (ETF). Building on this neural collapse insight, we propose a solution to the FL's classifier bias problem by utilizing a synthetic and fixed ETF classifier during training. The optimal classifier structure enables all clients to learn unified and optimal feature representations even under extremely heterogeneous data. We devise several effective modules to better adapt the ETF structure in FL, achieving both high generalization and personalization. Extensive experiments demonstrate that our method achieves state-of-the-art performances on CIFAR-10, CIFAR-100, and Tiny-ImageNet.
\end{abstract}

%%%%%%%%% BODY TEXT
\vspace{-0.7cm}
\section{Introduction}
Federated learning (FL) \cite{mcmahan2017communication,li2022towards,luo2021no} is a distributed training paradigm that enables collaborative training from massive multi-source datasets without transferring the raw data, reserving data ownership \cite{li2022can} while relieving communication burdens \cite{mcmahan2017communication}. FL facilitates broad applications in medical images \cite{adnan2022federated,guo2022auto}, the internet of things \cite{khan2021federated,nguyen2021federated}, mobile services \cite{hard2020training,kang2020reliable}, and so on; it shows promising prospects in data collaboration.
However, clients in FL training may hold heterogeneous data, in other words, clients' datasets are in Non-IID distributions\footnote{We use ``data heterogeneity'' and ``Non-IID data'' interchangeably.}, which causes a huge degradation to the global model's generalization \cite{dai2022tackling,luo2021no,chen2021bridging}.

Numerous recent studies have shown that \textit{\textbf{classifier biases}} in clients' local models caused by Non-IID data are the primary cause of degradation in FL \cite{luo2021no,zhou2022fedfa,li2022partial}. It has been discovered that the classifier layer is more biased than other layers \cite{luo2021no}, and classifier biases will create a \textit{vicious cycle} between biased classifiers and misaligned features across clients \cite{zhou2022fedfa}. Figure \ref{fig:motivation} illustrates the issue of classifier bias in FL, where Non-IID data leads to poor pairwise cosine similarities among clients' classifiers and feature prototypes. Furthermore, class-wise classifier vectors of clients are scattered in the embedding space, leading to significant generalization declines.

\begin{figure*}[t]
 \vspace{-1.7cm}
\label{fig:motivation}
\centering
\includegraphics[width=1.83\columnwidth]{figs/motiv_fig_new.pdf}
 \vspace{-0.3cm}
\caption{ \textbf{How data heterogeneity causes classifier biases in FL.} Smaller $\alpha$ corresponds to higher Non-IID. Experiments are conducted on CIFAR-10 with vanilla \textsc{FedAvg}. Columns from left to right: (1) Non-IID data results in poor generalization, biased classifiers, and misaligned features. (2) Clients' data distributions. (3) Clients with Non-IID data have smaller pair-wise classifier cosine similarities. (4) t-SNE visualization of clients' class-wise classifier vectors (represented by colors), which are more scattered in Non-IID data. }
 \vspace{-0.3cm}
\end{figure*}

Previous research has attempted to mitigate classifier biases through classifier retraining via generated virtual features at the end of FL training \cite{luo2021no,shang2022federated}. However, these methods fail to address classifier biases during training. Biased classifiers during the training phase lead to inadequate feature extractors and poor representations of generated features, negatively affecting the retrained classifiers. Our experiments have also shown the limitations of classifier retraining methods (Table \ref{table:first_table}). Therefore, we wonder:

\vspace{0.15cm}
%\emph{What is the optimal classifier?}\\
\noindent\emph{Can we break the classifier bias dilemma during training, improving both the classifiers and feature representations?}
\vspace{0.15cm}

To resolve the classifier bias dilemma, it is essential to fully understand the mechanisms behind the classifier. We further wonder: \textit{what are the properties of a well-trained (i.e. good) classifier?} An emerging discovery called neural collapse \cite{papyan2020prevalence,yang2022we,li2022principled,yangneural} has shed light on this matter. It describes the phenomenon that, in the perfect training scenario, where the dataset is balanced and sufficient, the feature prototypes and classifier vectors converge to an optimal simplex equiangular tight frame (ETF) with maximal pairwise angles \cite{papyan2020prevalence}. Insights from balanced training inspire us to tackle the challenges in Non-IID FL.

Thus, in this paper, \textit{we \textbf{fundamentally solve the FL's classifier bias problem} with a neural-collapse-inspired approach. }
Knowing the optimal classifier structure, we make the first attempt to introduce a synthetic simplex ETF as a fixed classifier for all clients so that the clients can learn unified and optimal feature representations even under high heterogeneity. We devise \textsc{FedETF} which incorporates several effective modules that better adapt the ETF structure in FL training, reaching strong results on \textit{\textbf{both}} \textit{\textbf{generalization}} \textbf{\textit{and}} \textit{\textbf{personalization}}.

Specifically, we employ a projection layer that maps the features to a space where neural collapse is more likely to occur. We also implement a balanced feature loss with a learnable temperature to minimize entropy between the features and the fixed ETF classifier. These techniques enable us to achieve high generalization performance of the global model during FL training.
To further improve personalization, we introduce a novel fine-tuning strategy that adapts the global model locally after FL training.
Extensive experiments have strongly supported the effectiveness of our method. Our contributions are summarized as follows.
\begin{itemize}[leftmargin=*,nosep]
    \item To the best of our knowledge, this is the first paper that tackles the data heterogeneity problem in FL from the perspective of neural collapse.
    \item We devise \textsc{FedETF}, which takes the simplex ETF as a fixed classifier, and it fundamentally solves the classifier biases brought by Non-IID data, reaching high generalization of the global model.
    \item We propose a local fine-tuning strategy in \FedETF to boost personalization in each client after FL training.
    \item Our method is validated on three vision datasets: CIFAR-10, CIFAR-100, and Tiny-ImageNet. Our proposed method outperforms strong baselines and achieves sota in both generalization and personalization.
\end{itemize}

%-------------------------------------------------------------------------
\section{Related Works}
\vspace{-0.3cm}
\noindent \textbf{Data Heterogeneity in Federated Learning.}
A variety of solutions have been proposed to tackle data heterogeneity in FL. Recent works \cite{li2022partial,luo2021no,shang2022federated,zhou2022fedfa} have revealed that the biased classifier is the main cause leading to poor performance of the global model, and they use classifier retraining \cite{luo2021no,shang2022federated} or classifier variance reduction \cite{li2022partial} to calibrate the classifier. In particular, \CCVR \cite{luo2021no} finds that there exists a greater bias in the classifier than in other layers, and only calibrating the classifier via virtual features after FL training can improve the global model performance. However, this approach cannot resolve the misaligned representations of local models caused by biased classifiers during FL training. Consequently, the backbone feature extractor cannot be improved. Moreover, some concurrent works use classifier variance reduction \cite{li2022partial} or feature anchors \cite{zhou2022fedfa} to relieve classifier biases. However, when data is highly Non-IID, variance-reduced classifiers and aggregated feature anchors are also biased and far from optimal (i.e., trained on IID data). Although these methods can alleviate classifier biases to some extent, they cannot completely solve them.

To improve the misaligned features, another line of works use prototypical methods to aid client training. \FedProto \cite{tan2022fedproto} only transmits and aggregates prototypes on the server to deal with data heterogeneity and model heterogeneity. A concurrent work named \FedNH \cite{dai2022tackling} adopts a prototypical classifier and uses a smoothing aggregation strategy to update the classifier based on clients' local prototypes.
However, the aggregated prototypes will also be biased in extreme Non-IID settings, and these methods did not use the classifier's ETF optimality to tackle this problem, which is where our contribution lies.

\noindent \textbf{Neural Collapse.}
% \cite{zhu2021geometric} prove that Simplex ETFs are the only global minimizers of the cross-entropy training loss with weight decay and bias.
The neural collapse was firstly observed in \cite{papyan2020prevalence} that at the terminal phase of training on a balanced dataset, the feature prototypes and the classifier vectors will converge to a simplex ETF where the vectors are normalized and the pair-wise angles are maximized. Afterward, there are some works trying to figure out the mechanism behind neural collapse \cite{ji2021unconstrained,zhu2021geometric,tirer2022extended,kothapalli2022neural} and in which conditions neural collapse will happen \cite{li2022principled,tirer2022extended,kothapalli2022neural}. Recent works use neural-collapse-inspired methods to solve the problems in imbalanced training \cite{yang2022we,xie2023neural,thrampoulidis2022imbalance}, incremental learning \cite{yangneural}, and transfer learning \cite{li2022principled}.
Despite the neural-collapse-inspired methods' success in centralized learning, deep insights and effective solutions regarding neural collapse are missing in distributed training. In this paper, we find neural collapse is also the key to success in FL and we show that inducing ETF optimality can inherently solve the classifier bias and feature misalignment problem in FL and largely improve performance.
%-------------------------------------------------------------------------
\section{Preliminaries}
%-------------------------------------------------------------------------
\subsection{Federated Learning}
\noindent\textbf{Basic Settings.} We introduce a typical FL setting with $K$ clients holding potentially Non-IID data partitions $\DD_1, \DD_2, ..., \DD_K$, respectively. A supervised classification task with $C$ classes is considered. Let $n_{k,c}$ be the number of samples of class $c$ on client $k$, and $n_k=\sum_{c=1}^Cn_{k,c}=|\DD_k|$ denotes the number of training samples held by client $k$. FL aims to realize generalization or personalization by distributed training under the coordination of a central server without any data transmission.
For \textit{generalization} in FL, the goal is to learn a global model over the whole training data $\DD\triangleq \bigcup_{k} \DD_k$ and the global model is expected to be generalized to the distribution of the whole data $\DD$. For \textit{personalization}, the goal is to learn $K$ personalized local models by FL training, and the local model $k$ is expected to have better adaptation in local data distribution $\DD_k$ than the independently trained one.
For the model in FL, we typically consider a neural network $\phi_{\bw}$ with parameters $\bw=\{\bu, \bv\}$. It has two main components: 1) a feature extractor $f_{\bu}$ parameterized by $\bu$, mapping each input sample $\bx$ to a $d$-dim feature vector; 2) a classifier $h_{\bv}$ parameterized by $\bv$. The parameters of client $k$'s local model are denoted as $\bw_k$.
%\subsection{Basic Algorithm of Federated Learning}\label{fl}

\noindent\textbf{Model updates.} There are two iterative steps in FL, client local training and server global aggregation, and the iteration lasts for $T$ rounds. In round $t$, the server first sends a global model $\bw^t$ to clients.

\noindent\textit{\textbf{Client local training:}} For each client $k,~ \forall k \in [K]$, it conducts SGD updates on the local data $\DD_k$:
\begin{align}\label{equ:client_update}
\bw_k^{t}\gets \bw_k^t-\eta\nabla_{\bw}\ell(\bw_k^t;\mathcal{B}^i),
\end{align}
where $\eta$ is the learning rate, $\mathcal{B}^i$ is the mini-batch sampled from $\DD_k$ at the $i$-th local iteration. The local epoch is $E$.

\noindent\textit{\textbf{Server global aggregation:}}
After local training, the server samples a set of clients $\mathcal{A}^t$ and the sampled clients send their updated models to the server. Then the server performs the weighted aggregation to update the global model for round $t+1$. The vanilla aggregation strategy is \FedAvg \cite{mcmahan2017communication} where the aggregation weights are proportional to clients' data sizes.
\begin{align}\label{equ:global_agg}
\bw^{t+1}&= \sum_{k\in \mathcal{A}^t}\frac{n_k}{\sum_{j\in\mathcal{A}^{t}}n_j}\bw_k^{t}.
\end{align}

%-------------------------------------------------------------------------
\subsection{Neural Collapse} \label{sect:pre_nc}
Neural collapse refers to a phenomenon about the last-layer features and classifier vectors at the terminal phase of training (zero training loss) on a balanced dataset \cite{papyan2020prevalence}. We first give the definition of simplex ETF in neural collapse.
\begin{definition}[Simplex Equiangular Tight Frame]
	\label{def:ETF}
A collection of vectors $\mathbf{v}_i\in\mathbb{R}^d$, $i \in [C]$, $d\ge C-1$, is said to be a simplex equiangular tight frame if:
\begin{equation}\label{equ:ETF_definition}
    \mathbf{V}=\sqrt{\frac{C}{C-1}}\mathbf{U}\left(\mathbf{I}_C-\frac{1}{C}\mathbf{1}_C\mathbf{1}_C^T\right),
\end{equation}
where $\mathbf{V}=[\mathbf{v}_1,\cdots,\mathbf{v}_C]\in\mathbb{R}^{d\times C}$, $\mathbf{U}\in\mathbb{R}^{d\times C}$ allows a rotation and satisfies $\mathbf{U}^T\mathbf{U}=\mathbf{I}_C$, $\mathbf{I}_C$ is the identity matrix, and $\mathbf{1}_C$ is an all-ones vector. All vectors in a simplex ETF have an equal $\ell_2$ norm and the same pair-wise angle, i.e.
\begin{equation}\label{equ:mimj}
	\mathbf{v}_i^T\mathbf{v}_j=\frac{C}{C-1}\delta_{i,j}-\frac{1}{C-1}, \forall i, j\in[C],
\end{equation}
where $\delta_{i,j}$ equals to $1$ when $i=j$ and $0$ otherwise. The pair-wise angle $-\frac{1}{C-1}$ is the maximal equiangular separation of $C$ vectors in $\mathbb{R}^d$ \cite{papyan2020prevalence}.
\end{definition}

We highlight three key properties of the neural collapse (NC) phenomenon below.
% Then the neural collapse (NC) phenomenon can be formally described in the following three properties.

\noindent\textbf{NC1} (Features collapse to the class prototypes).
The last-layer features will collapse to their within-class mean (prototypes), i.e. for any class $c,~ \forall c \in [C]$, the covariance $\Sigma_W^c\rightarrow\mathbf{0}$, where $\Sigma_W^c:=\frac{1}{n_c}\sum_{i=1}^{n_c}(\h_{c,i}-\h_c)(\h_{c,i}-\h_c)^T$. $\h_{c,i}=f(\bu;\bx_{c,i})$ is the feature of the $i$-th sample in the class $c$, and $\h_c=\frac{1}{n_c}\sum_{i=1}^{n_c}\h_{c,i}$ is the class $c$'s prototype.

\noindent\textbf{NC2} (Prototypes collapse to simplex ETFs).  $\tilde{\h}_c = (\h_c-\h_G)/||\h_c-\h_G||, \forall c\in[C]$, collapses to a simplex ETF which satisfies Eq. (\ref{equ:mimj}). $\h_G$ is the global mean of the last-layer features, that $\h_G=\sum_{c=1}^C\sum_{i=1}^{n_c}\h_{c,i}$.

\noindent\textbf{NC3} (Classifiers collapse to the same simplex ETFs). The normalized feature prototype $\tilde{\h}_c$ is aligned with their corresponding classifier weights\footnote{For simplicity, we omit the bias term in a linear classifier layer.}, which means that the classifier weights collapse to the same simplex ETF, i.e. $\tilde{\h}_c=\bv_c/||\bv_c||$, where $\bv_c$ refers to the vectorized classifier weights of class $c$.

% \noindent\textbf{NC4} (Simplification to the nearest prototype prediction). Owing to \textbf{NC1-NC3}, the model prediction using logits can be simplified to the nearest prototype prediction, which means that $\arg\max_c\langle\h, \bv_c\rangle={\arg\min}_c||\h-\h_c||$, where $\h=f(\bu;\bx)$ is the last-layer feature of a sample $\bx$ to predict for classification.

%-------------------------------------------------------------------------
\begin{figure*}[t]
    \centering
    \includegraphics[width=2.2\columnwidth]{figs/fedetf_framework.pdf}
    \caption{\textbf{Proposed \FedETF during FL training.} (a) In vanilla FL training, the feature extractor and linear classifier are both learned at clients and aggregated at server. (b) In our \FedETF, only the feature extractor and projection layer are learned and aggregated, and we adopt the same synthetic and fixed ETF classifier for all clients throughout the FL training process. Instead of prediction logit loss in vanilla FL, we use a novel balanced feature loss for the ETF classifier.}
    \label{fig:framework}
\end{figure*}

\section{Methods}
Neural collapse tells us the optimal structure (i.e. simplex ETF) of classifiers and feature prototypes in a perfect training setting. It inspires us to use a synthetic simplex ETF as a fixed classifier from the start to mitigate the classifier bias and feature misalignment problems (see Figure~\ref{fig:motivation}) brought by clients' data heterogeneity. Therefore, we propose \FedETF, a novel FL algorithm inspired by neural collapse. Concretely, as elaborated in Section \ref{sect:fedetf_generalization}, to promote the generalization of the global model, we reformulate the model architecture by replacing the learnable classifier with a fixed ETF classifier and devise a tailored loss for robust learning during FL training. Moreover, as described in Section \ref{sect:fedetf_personalization}, to improve local personalization after FL training, we propose a finetuning strategy for both finetuning the model and the formerly fixed ETF classifier.

\subsection{Improving Generalization by ETF Classifier}
\label{sect:fedetf_generalization}
\noindent\textbf{Reformulation of the model architecture.} In previous works of FL, a model architecture that consists of a learnable feature extractor and a learnable linear classifier is adopted \cite{mcmahan2017communication,chen2021bridging,collins2021exploiting,li2020federated}, as shown in Figure \ref{fig:framework} (a). However, due to clients' data heterogeneity, the classifiers will be more biased than other layers \cite{luo2021no,li2022partial}, and a vicious cycle between classifier biases and feature misalignment will exist \cite{zhou2022fedfa}. In this paper, we reformulate the model in FL into the combination of a learnable feature extractor and a fixed ETF classifier, as demonstrated in Figure \ref{fig:framework} (b).

\noindent\textit{\textbf{ETF classifier initialization:}} At the beginning of the FL training, we first randomly synthesize a simplex ETF $\mathbf{V}_{ETF} \in \mathbb{R}^{d\times C}$ by Eq. (\ref{equ:ETF_definition}), where $d$ denotes the feature dimension of the ETF and $C$ is the number of classes. The feature dimension $d$ should require $d\ge C-1$ in Definition \ref{def:ETF}, and we will discuss in Figure \ref{fig:understanding_fedetf_featuredim_personalization} (a) that a relatively low dimension is beneficial to neural collapse. For each class's classifier vector $\bv_i,~ \forall i\in [C]$ in the ETF $\mathbf{V}_{ETF}$, it requires $\Vert\bv_i \Vert_2 = 1$; and any pair of classifier vectors $(\bv_i,\bv_j),~ i \neq j,~ \forall i,j\in [C]$ satisfies $\cos(\bv_i,\bv_j) = -\frac{1}{C-1}$ according to Eq. (\ref{equ:mimj}).

\noindent\textit{\textbf{Projection layer:}} Given a data sample $\bx$, we first use the feature extractor $f_\bu$ to transform the data into the raw feature $\mathbf{h}$ and then use a projection layer $g_{\mathbf{p}}$ to map this raw feature to the ETF feature space and normalize it into $\bmu$.
\begin{equation} \label{equ:etf_feature}
    \bmu = \hat{\bmu} / \Vert \hat{\bmu} \Vert_2,\quad \hat{\bmu} = g(\bp;\bh),\quad \bh = f(\bu;\bx),
\end{equation}
where $\bu$ and $\bp$ denote the parameters of the feature extractor and the projection layer. We note that the projection layer is essential in our \FedETF design: 1) If the last layer of the feature extractor is the non-linear activation, e.g. the ReLU, the raw feature $\bh$ will be sparse with zeros (or near zero values), and it is hard for $\bh$ to be close to the dense ETF classifier vectors. 2) The raw features always have high dimensions, and high-dimensional vectors are more prone to be orthogonal, which is harder to collapse into the ETF with maximal angles. It is necessary to use the projection layer to map the features into a suitable dimension $d$. 3) The projection layer is helpful in the local finetuning stage for personalization.

\noindent\textbf{Balanced feature loss with learnable temperature.} In neural-collapse-inspired imbalanced learning \cite{yuan2021we}, it is found that when the ETF classifier is used, the gradients of cross entropy (CE) will be biased towards the head class, and the authors proposed a dot regression loss to tackle this problem. In FL, clients' local datasets are also class-imbalanced due to data heterogeneity, so techniques tackling the imbalanced problem are also needed in our design. Following previous work \cite{chen2021bridging} which induces balanced softmax loss to logit-prediction-based CE in FL, in this paper, we also incorporate the balanced loss \cite{ren2020balanced} into our feature-based CE. Instead of using former dot regression loss \cite{yuan2021we,yangneural}, it is found that our balanced feature CE loss also can solve the imbalanced gradient problem in learning with the ETF classifier.

Moreover, the softmax function's input of the vanilla CE loss is the logits, generated by MLP, while that of our method is the features' product $\bv_{y}^T\bmu$. The logits have a wide range of value since the output of MLP has no constraints, but our features' product has a limited range [-1, 1]\footnote{Knowing the fact that both the vectors are normalized.}, which is sensitive to scaling. Therefore, we add a temperature\footnote{The term ``temperature'' is borrowed from a similar concept in knowledge distillation \cite{hinton2015distilling}.} scalar $\beta$ to scale the features' product. Further, we found that in different stages of training, it requires different $\beta$, and fixed $\beta$ is hard to tune and may impede performance if not appropriate. To solve this, we take $\beta$ as one of the parameters in the model and update it by SGD during training. This learnable temperature $\beta$ will capture the learning dynamics in each client under various heterogeneity.

We define the model parameters in our \FedETF as $\bw=\{\bu, \bp, \beta\}$, which consists of the feature extractor, the projection layer, and the learnable temperature. For a given sample $(\bx, y)$ of client $k,~ \forall k \in [K]$, we define the loss function for generalization in Eq. (\ref{equ:g_sample_loss}), where the \textcolor{orange}{orange} term is for balanced feature loss and the \textcolor{blue}{blue} term is for learnable temperature.
\begin{align} \label{equ:g_sample_loss}
\small
    \ell^g(\bw, \mathbf{V}_{ETF}; \bx, y) = -\log\frac{\textcolor{orange}{n_{k,y}^{\gamma}}\exp(\textcolor{blue}{\beta}\cdot\bv_{y}^T\bmu)}{\sum_{c\in[C]}\textcolor{orange}{n_{k,c}^{\gamma}}\exp(\textcolor{blue}{\beta}\cdot\bv_{c}^T\bmu)},
\end{align}
where $n_{k,c}$ refers to the number of samples in class $c$ of client $k$, $\beta$ is the learnable temperature, $\bmu$ is the normalized feature in Eq. (\ref{equ:etf_feature}), $\bv_{c}$ is the class $c$'s classifier vector in $\mathbf{V}_{ETF}$, and $\gamma$ is the hyperparameter for balanced loss. Below, we give the learning objective of client $k,~ \forall k \in [K]$, and solve the objective by SGD in Eq. (\ref{equ:client_update}).
\begin{align}
    \bw_k^t &= \arg\min_{\bw} \mathcal{L}_k^g(\bw), \\
    \text{where }\mathcal{L}_k^g(\bw) &= \frac{1}{n_k}\sum_{(\bx_i, y_i)\in \DD_k}\ell^g(\bw, \mathbf{V}_{ETF}; \bx_i, y_i).
\end{align}
We adopt the vanilla aggregation strategy on the server, formulated in Eq. (\ref{equ:global_agg}).


\subsection{Personalized Adaptation by Local Finetuning}
\label{sect:fedetf_personalization}
\begin{figure}[t]
    \centering
    \includegraphics[width=0.9\columnwidth]{figs/personalization_framework.pdf}
    \caption{\textbf{Local finetuning stage of proposed \FedETF for personalization.} This stage is after the whole FL training stage when clients receive the final global model. For each client, we first finetune the feature extractor and then we finetune the ETF prototypical classifier and projection layer alternately.}
    \label{fig:fedetf_personalization_framework}
\end{figure}
% 讲下fedrod的观点:良好的global model是personalization的基础
As discovered in \cite{chen2021bridging}, local adaptation of a more generalized global model will reach stronger personalization. We will also verify this finding in our \FedETF. After the FL training, we obtain a global model $\bw^g$ with better generalization, and we use $\bw^g$ as the initialization in each client for personalization. We will show that by our tailored local finetuning of $\bw^g$, we will also reach the state-of-the-art in personalization.

Our personalized local finetuning consists of two parts: \textit{local feature adaptation} and \textit{classifier finetuning}.
In the \textit{local feature adaptation}, we fix the projection layer and ETF classifier and finetune the feature extractor to let the feature extractor be more customized to the features of clients' local data.
In the \textit{classifier finetuning period}, we finetune the ETF classifier and projection layer alternately for several iterations to make the classifier more biased to the local class distributions. We note that the simplex ETF is not an ideal classifier for local personalization, since the clients may have imbalanced class distributions or even have missing classes. Biased classifiers are needed for personalization to take the local class distributions as prior knowledge and maximize the prediction likelihood. We alternately finetune the ETF classifier and projection layer to make the projected features and classifier vectors converge to be aligned.

The process of local finetuning is illustrated in Figure \ref{fig:fedetf_personalization_framework}. The learned model parameters in the personalization stage are $\bw=\{\bu, \bp, \beta, \mathbf{V}_{ETF}\}$, and we split $\bw$ into the tuned parameters $\hat{\bw}$ and the fixed parameters $\overline{\bw}$. When finetune the feature extractor, $\hat{\bw}=\{\bu, \beta\},~ \overline{\bw}=\{\bp, \mathbf{V}_{ETF}\}$; when finetune the ETF classifier, $\hat{\bw}=\{\mathbf{V}_{ETF}, \beta\},~ \overline{\bw}=\{\bu, \bp\}$; when finetune the projection layer, $\hat{\bw}=\{\bp, \beta\},~ \overline{\bw}=\{\bu, \mathbf{V}_{ETF}\}$. We use the vanilla CE loss without balanced softmax in each stage of finetuning.
\begin{align}
    \ell^p(\hat{\bw},\overline{\bw}; \bx, y) = -\log\frac{\exp(\beta\cdot\bv_{y}^T\bmu)}{\sum_{c\in[C]}\exp(\beta\cdot\bv_{c}^T\bmu)}.
\end{align}
The learning objective in each stage is shown as follows.
\begin{align}
    \bw_k^p &= \{\overline{\bw}, \arg\min_{\hat{\bw}} \mathcal{L}_k^p(\hat{\bw})\}, \\
    \text{where }\mathcal{L}_k^p(\hat{\bw}) &= \frac{1}{n_k}\sum_{(\bx_i, y_i)\in \DD_k}\ell^p(\hat{\bw},\overline{\bw}; \bx_i, y_i).
\end{align}
Personalization will be reached after several iterative stages of finetuning in Figure \ref{fig:fedetf_personalization_framework}.

\begin{table*}[h]
    \footnotesize
    \centering
      \vspace{-0.5cm}
    \caption{\textbf{Results in terms of generalization (General.) accuracy (\%) of global models and personalization (Personal.) accuracy (\%) of local models on three datasets under different heterogeneity.} Best two methods in each setting are highlighted in \textbf{bold} fonts.}
    % \vspace{-1em}
    \resizebox{\linewidth}{!}{
    \begin{tabular}{l|cc|cc|cc|cc|cc|cc}
    \toprule
    Dataset&\multicolumn{4}{c}{CIFAR-10}&\multicolumn{4}{c}{CIFAR-100}&\multicolumn{4}{c}{Tiny-ImageNet}\\
    \cmidrule(lr){1-5}
    \cmidrule(lr){6-9}
    \cmidrule(lr){10-13}
    % \midrule
    NonIID ($\alpha$) &\multicolumn{2}{c}{0.1}&\multicolumn{2}{c}{0.05}&\multicolumn{2}{c}{0.1}&\multicolumn{2}{c}{0.05}&\multicolumn{2}{c}{0.1}&\multicolumn{2}{c}{0.05}\\
    \midrule
    Methods/Metrics &General.&Personal.&General.&Personal.&General.&Personal.&General.&Personal.&General.&Personal.&General.&Personal.\\
    \midrule
    \textsc{FedAvg} \cite{mcmahan2017communication}&52.76{\tiny±6.08}	&83.85{\tiny±0.89}	
    &44.48{\tiny±6.19}	&89.80{\tiny±0.39}
    &24.77{\tiny±1.19}	&49.93{\tiny±1.17}	&\textbf{22.53{\tiny±0.40}}	&58.85{\tiny±0.33}	&28.93{\tiny±0.52}	&40.81{\tiny±0.35}	&24.88{\tiny±0.34}	&46.90{\tiny±0.44}\\
    \midrule
    \textsc{FedProx} \cite{li2020federated}&46.59{\tiny±3.04}	&82.08{\tiny±0.27}	
    &40.95{\tiny±5.75}	&87.69{\tiny±2.85}
    &23.33{\tiny±1.72}	&46.44{\tiny±1.64}	&19.12{\tiny±0.77}	&57.01{\tiny±2.17}	&25.93{\tiny±0.27}	&31.90{\tiny±1.91}	&23.06{\tiny±0.68}	&32.43{\tiny±0.65}\\
    \textsc{FedDyn} \cite{acar2020federated}&36.35{\tiny±5.33}	&85.39{\tiny±0.77}	
    &23.90{\tiny±1.40}	&88.72{\tiny±1.59}
    &\textbf{25.53{\tiny±2.39}}	&51.79{\tiny±2.12}	&20.71{\tiny±2.83}	&\textbf{61.77{\tiny±0.32}}	&26.42{\tiny±0.56}	&45.84{\tiny±0.34}	&23.63{\tiny±1.55}	&52.27{\tiny±1.06}\\
    \midrule
    \textsc{Ditto} \cite{li2021ditto}&52.76{\tiny±6.08}	&79.81{\tiny±1.89}	
    &44.48{\tiny±6.19}	&85.17{\tiny±3.47}
    &24.77{\tiny±1.19}	&38.06{\tiny±1.26}	&22.53{\tiny±0.40}	&50.18{\tiny±1.22}	&28.93{\tiny±0.52}	&33.00{\tiny±1.01}	&24.88{\tiny±0.34}	&40.31{\tiny±0.12}\\
    \textsc{FedRep} \cite{collins2021exploiting}&26.85{\tiny±10.13}	&\textbf{87.76{\tiny±0.87}}	
    &15.79{\tiny±3.68}	&90.71{\tiny±2.25}
    &5.47{\tiny±0.20} &\textbf{53.62{\tiny±1.49}}	&4.18{\tiny±0.85}	&61.51{\tiny±0.61}	&4.10{\tiny±0.22}	&43.66{\tiny±0.48}	&2.20{\tiny±0.19}	&49.52{\tiny±1.64}\\
    \midrule
    \textsc{CCVR} \cite{luo2021no}&52.50{\tiny±6.31}	&55.62{\tiny±5.89}	&47.98{\tiny±6.24}	&73.52{\tiny±7.49}	&24.54{\tiny±0.71}	&34.01{\tiny±2.01}	&22.28{\tiny±0.43}	&39.16{\tiny±1.41}	&\textbf{32.78{\tiny±0.24}}	&\textbf{54.00{\tiny±0.46}}	&\textbf{29.27{\tiny±0.25}}	&\textbf{59.29{\tiny±0.30}}	\\
    \textsc{FedProto} \cite{tan2022fedproto}&-	&83.34{\tiny±0.71}	
    &-	&88.21{\tiny±1.77}
    &- &43.31{\tiny±0.70}	
    &- &54.87{\tiny±0.52}	
    &-	&40.74{\tiny±0.87}	&-	&48.05{\tiny±0.82}\\
    \textsc{FedRoD} \cite{chen2021bridging}&\textbf{55.72{\tiny±2.40}}	&86.19{\tiny±0.91}	
    &\textbf{49.89{\tiny±3.64}}	&88.83{\tiny±4.14}
    &24.49{\tiny±1.05} &51.78{\tiny±1.16}	&21.63{\tiny±0.42}	&59.44{\tiny±0.45}	&32.17{\tiny±0.41}	&38.27{\tiny±1.00}	&28.45{\tiny±0.58}	&44.09{\tiny±0.44}\\
    \textsc{FedNH} \cite{dai2022tackling}&55.37{\tiny±4.48}	&85.98{\tiny±0.15}	
    &47.96{\tiny±2.59}	&\textbf{91.06{\tiny±3.13}}
    &24.67{\tiny±0.68} &52.09{\tiny±0.78}	&21.95{\tiny±0.85}	&\textbf{62.71{\tiny±0.22}}	&17.51{\tiny±0.62}	&36.53{\tiny±0.29}	&14.00{\tiny±0.17}	&41.80{\tiny±1.78}\\
    \midrule
    \rowcolor{gray!20}\textbf{Our \textsc{FedETF}}&\textbf{59.56{\tiny±1.84}}	&\textbf{87.89{\tiny±1.19}}	&\textbf{56.08{\tiny±3.44}}	&\textbf{92.62{\tiny±0.54}}
    &\textbf{26.24{\tiny±1.78}}	&\textbf{52.86{\tiny±1.53}}	&\textbf{24.17{\tiny±0.54}}	
    &60.68{\tiny±0.91}	
    &\textbf{33.49{\tiny±0.82}}	&\textbf{55.82{\tiny±0.60}}	&\textbf{29.15{\tiny±1.03}}	&\textbf{62.36{\tiny±0.13}}\\
    \bottomrule
    \end{tabular}
    }
    \label{table:first_table}
    % \vspace{-0.2cm}
\end{table*}
%------------------------------------------------------------------------
\section{Experiments and Results} \label{sect:exp}
\subsection{Settings}
\noindent\textbf{Datasets and Models.} Following previous works \cite{dai2022tackling,lin2020ensemble}, we use three vision datasets to conduct experiments: CIFAR-10 \cite{krizhevsky2009learning}, CIFAR-100 \cite{krizhevsky2009learning}, and Tiny-ImageNet \cite{deng2009imagenet,le2015tiny}. Tiny-ImageNet is a subset of ImageNet with 100k samples of 200 classes. Following \cite{li2018visualizing}, we adopt ResNet20 \cite{li2018visualizing,he2016deep} for CIFAR-10/100 and use ResNet-18 \cite{li2018visualizing,he2016deep} for Tiny-ImageNet. We use a linear layer as the classifier for the baselines and as the projection layer for our method.

\noindent\textbf{Compared Methods.} We take three lines of methods as baselines.
\textit{\textbf{1) Classical FL with Non-IID data:}} \FedAvg \cite{mcmahan2017communication} with vanilla local training, a simple but strong baseline; \FedProx \cite{li2020federated}, FL with proximal regularization at clients; \FedDyn \cite{acar2020federated}, FL based on dynamic regularization.
\textit{\textbf{2) Personalized FL:}} \Ditto \cite{li2021ditto}, personalization through separated local models; \FedRep \cite{collins2021exploiting}, personalization by only aggregating feature extractors; \FedRoD \cite{chen2021bridging}, personalization through decoupling models.
\textit{\textbf{3) FL methods most relevant to ours:}} \CCVR \cite{luo2021no}, FL with classifier retraining; \FedProto \cite{tan2022fedproto}, FL with only prototype sharing; \FedRoD \cite{chen2021bridging}, generalization through decoupling and balanced softmax loss; \FedNH \cite{dai2022tackling}, FL with smoothing aggregation of prototypical classifiers.

\noindent\textbf{Client Settings.} We adopt the Dirichlet sampling to generate Non-IID data for each client. We note that the Dirichlet-sampling-based data heterogeneity is widely used in FL literature \cite{lin2020ensemble,chen2021bridging,dai2022tackling,luo2021no}. It considers a class-imbalanced data heterogeneity, controlled by hyperparameter $\alpha$, and smaller $\alpha$ refers to more Non-IID data of clients.
% When $\alpha < 1$, the data are considered to be rather Non-IID, which means that most of the training samples of one class are likely assigned to a small portion of clients \cite{chen2021bridging}.
In our experiments, we evaluate the methods under strong Non-IID settings with $\alpha \in \{0.1,0.05\}$ \cite{luo2021no,zhang2022federated}.
% and the visualization of class distributions is demonstrated in the Appendix.
If without mentioning otherwise, the number of clients $K=20$ and we adopt full client participation.
% We will also implement experiments by changing $K$ and conducting partial client sampling.
For CIFAR-10 and CIFAR-100 the number of local epochs $E=3$ and the number of communication rounds $T=200$, while for Tiny-ImageNet, considering the high computation costs, we set $E=1$ and $T=50$.

\noindent\textbf{Evaluation Metrics.} We test the generalization of the aggregated global model (General.) and the personalization of clients' local models (Personal.). Generalization performance is validated on the balanced testset of each dataset after the global model is generated on the server. For each client, we split 70\% of the local data for the trainset and 30\% for the testset. Following \cite{chen2021bridging}, we validate the personalization performance on each client's local testset after the local training and average the personalized accuracies.

\noindent\textbf{Implementation.} In all the experiments, we conduct three trials for each setting and present the mean accuracy and the standard deviation in the tables. For more implementation details, please refer to the Appendix.

\subsection{Main Results}

\begin{figure}[t]
 % \vspace{-1.5cm}
\centering
\subfigure[]{\includegraphics[width=0.48\columnwidth]{figs/acc_curves_cifar10_dir0.05_seed8.pdf}}
\subfigure[]{\includegraphics[width=0.48\columnwidth]{figs/acc_curves_cifar100_dir0.05_seed8.pdf}}
% \subfigure[]{\includegraphics[width=0.48\columnwidth]{figs/acc_curves_cifar10_dir0.1_seed8.pdf}}
% \subfigure[]{\includegraphics[width=0.48\columnwidth]{figs/acc_curves_cifar100_dir0.1_seed8.pdf}}
\caption{\textbf{Global models' test accuracy curves of the methods.} (a) CIFAR-10 with $\alpha=0.05$. (b) CIFAR-100 with $\alpha=0.05$.
% (c) CIFAR-10 with $\alpha=0.1$. (d) CIFAR-100 with $\alpha=0.1$.
}
\label{fig:acc_curves}
% \vspace{-0.2cm}
\end{figure}

\begin{table*}[t]
    \footnotesize
    \centering
      \vspace{-0.5cm}
    \caption{\textbf{Results (\%) under various numbers of clients with partial client sampling.} The dataset is CIFAR-10 with Non-IID $\alpha=0.1$.}
    % \vspace{-1em}
    % \resizebox{0.85\textwidth}{!}{
    \setlength\tabcolsep{10.76pt}
    \begin{tabular*}{0.99\linewidth}{l|cc|cc|cc|cc}
    % \begin{tabular}{c|cc|cc|cc|cc}
    \toprule
    Number of Clients&\multicolumn{4}{c}{50}&\multicolumn{4}{c}{100}\\
    \cmidrule(lr){1-5}
    \cmidrule(lr){6-9}
    % \midrule
    Sampling Rate &\multicolumn{2}{c}{0.4}&\multicolumn{2}{c}{0.6}&\multicolumn{2}{c}{0.4}&\multicolumn{2}{c}{0.6}\\
    \midrule
    Methods/Metrics &General.&Personal.&General.&Personal.&General.&Personal.&General.&Personal.\\
    \midrule
    \textsc{FedAvg} \cite{mcmahan2017communication}&38.13{\tiny±5.12}	&77.28{\tiny±2.17}	
    &42.68{\tiny±6.28}	&74.99{\tiny±2.34}
    &42.15{\tiny±1.61}	&71.52{\tiny±1.88}	&41.42{\tiny±3.31}	&70.40{\tiny±2.13}\\
    \midrule
    \textsc{CCVR} \cite{luo2021no}&44.59{\tiny±11.4}	&\textbf{78.93{\tiny±3.26}}	&52.49{\tiny±6.73}	&\textbf{82.33{\tiny±1.72}}	&50.07{\tiny±0.80}	&\textbf{76.27{\tiny±2.08}}	&50.41{\tiny±3.93}	&\textbf{77.27{\tiny±1.22}}\\
    \textsc{FedRoD} \cite{chen2021bridging}&\textbf{55.84{\tiny±3.96}}	&76.60{\tiny±0.13}	
    &\textbf{53.04{\tiny±2.54}}	&74.42{\tiny±1.99}
    &\textbf{52.62{\tiny±1.68}} &71.27{\tiny±0.69}	&\textbf{52.34{\tiny±0.11}}	&72.41{\tiny±0.74}\\
    \textsc{FedNH} \cite{dai2022tackling}&39.97{\tiny±6.90}	&76.59{\tiny±0.59}	
    &45.36{\tiny±3.58}	&78.17{\tiny±1.15}
    &42.77{\tiny±0.65} &73.47{\tiny±1.38}	&45.85{\tiny±2.98}	&73.15{\tiny±0.95}\\
    \midrule
    \rowcolor{gray!20}\textbf{Our \textsc{FedETF}}&\textbf{58.05{\tiny±4.63}}	&\textbf{85.82{\tiny±0.86}}	&\textbf{58.75{\tiny±1.72}}	&\textbf{85.05{\tiny±0.87}}
    &\textbf{56.67{\tiny±0.88}}	&\textbf{83.47{\tiny±0.45}}	&\textbf{55.96{\tiny±0.23}}	
    &\textbf{83.38{\tiny±0.72}}\\
    \bottomrule
    \end{tabular*}
    % }
    \label{table:partial_select}
    \vspace{-0.2cm}
\end{table*}


\noindent\textbf{Results under various vision datasets and data heterogeneity.} Table \ref{table:first_table} shows the results of all methods on three vision datasets with Non-IID $\alpha \in \{0.1, 0.05\}$. \textit{Our method achieves state-of-the-art performances in 11 out of 12 settings in both generalization and personalization. It is notable that except for our \FedETF, there is no comparable baseline that can achieve high results in all datasets.} Generally, our method has more significant improvement under more heterogeneous settings ($\alpha=0.05$), especially in CIFAR-10. We also visualize the learning curves in Figure \ref{fig:acc_curves}. \textit{Our \FedETF not only has higher accuracies but also has faster and more steady convergence. }

For generalization, in most cases, \FedRoD can improve the accuracies upon \FedAvg, which showcases the effectiveness of the balanced loss. However, the balanced loss cannot thoroughly solve the classifier bias problem, and \textit{our method \FedETF which adopts the optimal classifier structure has large-margin gains over \FedRoD}. We notice that the classifier retraining algorithm \CCVR is not effective in all cases, which indicates that the retraining method is not practical enough for solving classifier biases.

For personalization, we find the personalized FL \FedRep and \FedRoD are strong baselines. Compared with these personalized FL methods, our \FedETF also reaches the state-of-the-art in almost all cases. \textit{The success of \FedETF in personalization is the result of training a better generalized global model and the effective local adaption of such a global model.}

\noindent\textbf{Results under different $K$ with partial client sampling.}
We select the best baselines in Table \ref{table:first_table} and conduct experiments on CIFAR-10 under various numbers of clients $K$ with partial client sampling in Table \ref{table:partial_select}. We set $K \in \{50,100\}$ and set the sampling rate as 0.4 and 0.6. To ensure fair comparisons, we randomly generate and save a sampling list in advance and let every method load the same sampling list during training. It is obvious that our \FedETF is the best method in both generalization and personalization. Compared with \FedAvg, the advantage of our method is more dominant under the smaller sampling rate, i.e. 0.4. With respect to generalization, when $K=50$, the improvement is 5.2\% for the sampling rate 0.4, and 3.8\% for the sampling rate 0.6. \textit{It indicates that the fixed ETF classifier is more robust than the learnable classifier to tackle \textbf{system heterogeneity}.} Moreover, our method has smaller variances of accuracies, which also verifies its robustness.

\begin{table}[h]\footnotesize
%\vspace{-15pt}
\centering
\caption{\textbf{Evaluation (\%) of \textsc{FedETF} using different model architectures.} The dataset is CIFAR-10 with Non-IID $\alpha=0.1$.}
% \resizebox{0.5\textwidth}{!}{%
% \begin{tabular}{lcc|cc}
\begin{tabular*}{0.99\linewidth}{lcc|cc}
    \toprule
    Methods& \multicolumn{2}{c}{\textsc{FedAvg}} & \multicolumn{2}{c}{\textsc{FedETF}} \\
    \midrule
    Models/Metrics& General.& Personal.& General.& Personal.\\
    \midrule
     DenseNet121   & 62.87{\tiny±2.23}  & 85.66{\tiny±3.70}& \textbf{74.92{\tiny±2.75}}  & \textbf{91.83{\tiny±0.67}} \\
     MobileNetV2   & 43.20{\tiny±4.45}  & 87.07{\tiny±1.04}& \textbf{57.43{\tiny±12.0}}  & \textbf{89.88{\tiny±0.40}} \\
     EfficientNet   & 35.92{\tiny±4.47}  & 84.69{\tiny±1.26}& \textbf{56.70{\tiny±5.52}}  & \textbf{87.50{\tiny±0.74}} \\
     \midrule
     ResNet20   & 52.76{\tiny±6.08}  & 83.85{\tiny±0.89}& \textbf{59.56{\tiny±1.84}}  & \textbf{87.89{\tiny±1.19}} \\
     ResNet32   & 53.22{\tiny±7.73}  & 82.90{\tiny±4.31}& \textbf{60.71{\tiny±2.67}}  & \textbf{87.97{\tiny±1.17}} \\
     ResNet56   & 57.09{\tiny±6.10}  & 83.70{\tiny±4.65}& \textbf{60.44{\tiny±3.57}}  & \textbf{88.23{\tiny±0.99}} \\
     % WRN56\_2   & 43.75{\small±0.42}  & 38.47{\small±1.95}& \textbf{43.75{\small±0.42}}  & \textbf{38.47{\small±1.95}} \\
     WRN56\_4   & 63.64{\tiny±4.91}  & 86.76{\tiny±1.08}& \textbf{66.30{\tiny±3.88}}  & \textbf{89.94{\tiny±0.52}} \\
    \bottomrule
  \end{tabular*}
% }
 \vspace{-0.2cm}
\label{table:model_architecture}
\end{table}

\begin{figure}[h]
 % \vspace{-1.5cm}
\centering
\subfigure[]{\includegraphics[width=0.48\columnwidth]{figs/understand_fedetf_nc_feature_proto_cosine.pdf}}
\subfigure[]{\includegraphics[width=0.48\columnwidth]{figs/understand_fedetf_nc_globalmodel_neural_collapse.pdf}}
\caption{\textbf{Understanding feature alignment and neural collapse of \textsc{FedETF}.} Test accuracy of final global models: \FedAvg 29.73\%, \FedETF 54.95\%. Experiments on CIFAR-10 with $\alpha=0.05$. (a) Feature prototype consistency of clients' local models. (b) Neural collapse error of the aggregated global model.}
\label{fig:understanding_fedetf_nc}
 \vspace{-0.4cm}
\end{figure}

\subsection{Understanding \textbf{\FedETF}}
\noindent\textbf{Evaluation using different model architectures.}
We consider two scenarios: \textbf{\textit{1) Different backbones.}} DenseNet121 \cite{huang2017densely}, MobileNetV2 \cite{howard2018inverted,sandler2018mobilenetv2}, and EfficientNet \cite{tan2019efficientnet}. \textbf{\textit{2) Deeper and wider models.}}
Deeper models: ResNet20, ResNet32, and ResNet56 \cite{li2018visualizing}  (the larger number refers to the deeper model); wider models: ResNet56 and WRN56\_4 \cite{li2018visualizing} (WRN: the abbreviation for Wide ResNet). The results are shown in Table \ref{table:model_architecture}. \textit{Our method can also improve performance under various model architectures.} For models with different depths, we observe that \FedETF has larger superiority in shallower models.
Specifically, \FedETF can release the full potential of ResNet20 (shallower and smaller model) to let it has even better performance than \FedAvg with ResNet56 (deeper and larger model). \textit{It showcases the applicability of \FedETF that it can enable smaller models to have better performances than the larger ones, saving both computation and communication costs in FL.}


\noindent\textbf{Why does \FedETF work well?} We explore how the features are learned in \FedETF compared with \FedAvg. We first examine the feature alignment of local models. In each round, after local training, we compute class prototypes (feature mean of each class) in each client and calculate the cosine similarities of clients' class-wise prototypes, which is analogous to NC1 in Section \ref{sect:pre_nc}. Then we average all the cosine similarities to indicate the feature alignment, a larger value reveals more aligned clients' features. The results are in Figure \ref{fig:understanding_fedetf_nc} (a). \textit{It shows that only after a few rounds, \FedETF has constantly stronger clients' feature alignment than \FedAvg, showing the fixed ETF classifier is effective to align local features.}

We also study whether \FedETF can help the global model reach neural collapse in terms of NC2 in Figure \ref{fig:understanding_fedetf_nc} (b).
In each round, we first compute the class prototypes of the global model and calculate the pair-wise cosine similarities of these prototypes. In neural collapse optimality (Definition \ref{def:ETF}), the pair-wise cosine similarities of prototypes are $-\frac{1}{K-1}$. Hence, we calculate the mean square error between the global model's cosines and $-\frac{1}{K-1}$ to indicate the neural collapse error. Results display that \FedETF has a much smaller neural collapse error than \FedAvg, and the error of \FedETF is decreasing along the training. \textit{It indicates that \FedETF can help the global model reach neural collapse.} Note that \FedAvg has 29.73\% in global test accuracy while \FedETF has 54.95\%.
\textit{It also verifies that the global model's generalization is connected with neural collapse optimality in FL, which is consistent with the observations in centralized training \cite{li2022principled}.}

\begin{table}[!t]\footnotesize
 % \vspace{-0.5cm}
\centering
\caption{\textbf{Ablation study of \textsc{FedETF} in terms of global model's generalization.} The dataset is CIFAR-10.}
% \resizebox{0.43\textwidth}{!}{%
% \setlength{\tabcolsep}{3mm}{
% \begin{tabular}{lcc}@{}lcc@{}
\setlength\tabcolsep{11.55pt}
\begin{tabular*}{0.99\linewidth}{lcc}
    \toprule
    Methods/NonIID($\alpha$) & 0.1 & 0.05\\
    \midrule
     \textsc{FedAvg}   & 43.75{\tiny±0.42}  & 38.47{\tiny±1.95} \\
     \midrule
    Ours w/o Projection Layer &44.92{\tiny±6.22} &  41.91{\tiny±1.47}\\
    Ours w/o Balanced Loss &46.06{\tiny±0.75} &  37.63{\tiny±4.45} \\
    Ours w/o Learnable Temperature  &49.80{\tiny±4.52} & 46.07{\tiny±1.61} \\
    \midrule
     \rowcolor{gray!20}Ours    & \textbf{56.46{\tiny±4.18}} & \textbf{53.98{\tiny±1.29}}\\
    \bottomrule
  \end{tabular*}
% }
 \vspace{-0.35cm}
\label{table:ablation}
\end{table}

\begin{figure}[t]
 % \vspace{-0.5cm}
\centering
\subfigure[]{\includegraphics[width=0.48\columnwidth]{figs/understand_featuredim.pdf}}
\subfigure[]{\includegraphics[width=0.48\columnwidth]{figs/fedetf_per_vis.pdf}}
\vspace{-0.1cm}
\caption{\textbf{Understanding feature dimension and local personalization in \FedETF.} Experiments on CIFAR-10 with $\alpha=0.1$. (a) How feature dimension affects \FedETF's generalization and neural collapse. (b) How personalization is reached in each iteration of \FedETF's local finetuning. }
\label{fig:understanding_fedetf_featuredim_personalization}
 \vspace{-0.5cm}
\end{figure}

\noindent\textbf{How feature dimension affects \FedETF.} We analyse how the feature dimension $d$ of ETF affects \FedETF's performance on CIFAR-10 in Figure \ref{fig:understanding_fedetf_featuredim_personalization} (a). We find that smaller $d$ will cause smaller neural collapse errors and are slightly beneficial to generalization. Random high-dimensional vectors are more prone to be orthogonal, so we suppose that prototypes in higher dimensions are more likely to be orthogonal. In CIFAR-10, the number of classes $K=10$, and the ETF angles are obtuse with -0.11 pair-wise cosines. Therefore, it is hard for high-dimensional features to collapse into the obtuse angle's structure. \textit{We suggest setting the feature dimension $d$ in \FedETF according to the number of classes $K$.} If $K$ is small, it also requires a relatively small $d$ to improve neural collapse.

\noindent\textbf{How personalization is reached during local finetuning.} We visualize the averaged personalized accuracies of different iterations during \FedETF's local finetuning in Figure \ref{fig:understanding_fedetf_featuredim_personalization}~(b). At first, when clients' local models are initialized as the final global model, the personalization is poor. After finetuning the feature extractor, \FedETF has better results than the baseline \FedAvg with finetuning. Then alternatively finetuning the ETF classifier and projection layer further improves the personalization and makes the accuracy converge to a higher point.

 \vspace{-0.15cm}
\subsection{Ablation Study}
 \vspace{-0.25cm}
We conduct the ablation study of \FedETF in terms of generalization\footnote{Figure \ref{fig:understanding_fedetf_featuredim_personalization} (b) can be viewed as the ablation study of the personalized local finetuning stage.} on CIFAR-10 with Non-IID $\alpha \in \{0.1, 0.05\}$. \textit{It is found that every module in \FedETF plays an important role and the modules strengthen each other to realize better performances.} If taking one module off, the performance will meet severe declines, but the results are still better than \FedAvg in general. We notice the balanced loss module is more important under a more heterogeneous environment, and this observation is consistent with previous works in neural collapse \cite{yang2022we} and FL \cite{chen2021bridging}. It is also notable to emphasize the significance and necessity of the projection layer. Our method without a projection layer only has marginal gains over \FedAvg. We also find that \FedNH has relatively poor performances in Table \ref{table:first_table}, especially on CIFAR-100 and Tiny-ImageNet, and we suppose the main cause may be that \FedNH does not adopt a projection layer to map the raw features into a space where neural collapse is more prone to happen. Moreover, the learnable temperature is also crucial for \FedETF to adaptively adjust the softmax temperature so as to meet the learning dynamics of feature representations.


%------------------------------------------------------------------------
 \vspace{-0.15cm}
\section{Conclusion}
 \vspace{-0.25cm}
In this paper, we fundamentally solve the classifier biases caused by data heterogeneity in FL by proposing a neural-collapse-inspired solution. Specifically, we employ a simplex ETF as a fixed classifier for all clients during federated training, which allows them to learn unified and optimal feature representations. Afterward, we introduce a novel finetuning strategy to enable clients to have more personalized local models.
Our method achieves the state-of-the-art performance regarding both generalization and personalization compared to strong baselines, as shown by extensive experimental results on CIFAR-10/100 and Tiny-ImageNet. Furthermore, we gained insights into understanding the effectiveness and applicability of our approach.

\documentclass[10pt,twocolumn]{article}

\usepackage{iccv}
\usepackage{times}
\usepackage{epsfig}
\usepackage{graphicx}
\usepackage{amsmath}
\usepackage{amssymb}
% \usepackage{eso-pic}
% \usepackage{everyshi}

\usepackage{algorithm}  
\usepackage{algorithmic} 
% \usepackage{algpseudocode}  
\usepackage{mathtools}
\usepackage{amsthm}
\usepackage{multirow}
% \usepackage{multicol}
\usepackage{makecell}
\usepackage{graphicx}
\usepackage{booktabs}
\usepackage{subfigure}
\usepackage{authblk}

% Include other packages here, before hyperref.

% If you comment hyperref and then uncomment it, you should delete
% egpaper.aux before re-running latex.  (Or just hit 'q' on the first latex
% run, let it finish, and you should be clear).
% \usepackage[breaklinks=true,bookmarks=false]{hyperref}

\usepackage[pagebackref=true,breaklinks=true,colorlinks,bookmarks=false]{hyperref}

\iccvfinalcopy % *** Uncomment this line for the final submission

\def\iccvPaperID{****} % *** Enter the ICCV Paper ID here
\def\httilde{\mbox{\tt\raisebox{-.5ex}{\symbol{126}}}}

% Pages are numbered in submission mode, and unnumbered in camera-ready
\ificcvfinal\pagestyle{empty}\fi

\begin{document}

%%%%%%%%% TITLE
\title{CroSel: Cross Selection of Confident Pseudo Labels \\ 
for Partial-Label Learning}

\author[1]{Shiyu Tian}
\author[2]{Hongxin Wei}
\author[1]{Yiqun Wang}
\author[1,2]{Lei Feng \thanks{Corresponding author: feng0093@e.ntu.edu.sg}}
\affil[1]{College of Computer Science, Chongqing University}
\affil[2]{School of Computer Science and Engineering, Nanyang Technological University}


% \author{Shiyu Tian$^1$, Hongxin Wei$^2$, Yiqun Wang$^1$ and Lei Feng$^1$\\
%     $^1$College of Computer Science, Chongqing University\\
%     $^2$School of Computer Science and Engineering, Nanyang Technological University\\
%     \{first.author, second.author, third.author\}@cu-tipaza.dz}

% \authornote{*Corresponding authors}
% \author{Shiyu Tian\\
% Chongqing University\\
% Institution1 address\\


% \author{Shiyu Tian\\
% Chongqing University\\
% Institution1 address\\
% {\tt\small firstauthor@i1.org}
% % For a paper whose authors are all at the same institution,
% % omit the following lines up until the closing ``}''.
% % Additional authors and addresses can be added with ``\and'',
% % just like the second author.
% % To save space, use either the email address or home page, not both
% \and
% Yiqun Wang\\
% Institution2\\
% First line of institution2 address\\
% {\tt\small secondauthor@i2.org}
% }
% \and
% Hongxin Wei\\
% Institution2\\
% First line of institution2 address\\
% {\tt\small secondauthor@i2.org}

% \and
% Lei Fe\\
% Institution2\\
% First line of institution2 address\\
% {\tt\small secondauthor@i2.org}




\maketitle
% Remove page # from the first page of camera-ready.
\ificcvfinal\thispagestyle{empty}\fi

%%%%%%%%% ABSTRACT
\begin{abstract}
Partial-label learning (PLL) is an important weakly supervised learning problem, which allows each training example to have a candidate label set instead of a single ground-truth label. Identification-based methods have been widely explored to tackle label ambiguity issues in PLL, which regard the true label as a latent variable to be identified. However, identifying the true labels accurately and completely remains challenging, causing noise in pseudo labels during model training. In this paper, we propose a new method called CroSel, which leverages historical prediction information from models to identify true labels for most training examples. First, we introduce a cross selection strategy, which enables two deep models to select true labels of partially labeled data for each other. Besides, we propose a novel consistent regularization term called co-mix to avoid sample waste and tiny noise caused by false selection. In this way, CroSel can pick out the true labels of most examples with high precision. Extensive experiments demonstrate the superiority of CroSel, which consistently outperforms previous state-of-the-art methods on benchmark datasets. Additionally, our method achieves over 90\% accuracy and quantity for selecting true labels on CIFAR-type datasets under various settings.
\end{abstract}

%%%%%%%%% BODY TEXT
\section{Introduction}

The past few years have seen an increased interest in deep learning due to its outstanding performance in various application domains, including image processing~\cite{chen2021imageprocession}, automatic driving~\cite{xiong2019autodriving}, and medical diagnosis~\cite{park2018medical}. The success of deep learning heavily relies on a massive amount of fully labeled data. However, it is challenging to obtain a large-scale dataset with completely accurate annotations in the real world. To address this challenge, many researchers have explored a promising weakly supervised learning problem called partial-label learning (PLL)~\cite{cour2011learningPLL2,feng2020provably,wang2022pico,wen2021LWS,wu2022CRDPLL}, which allows each training example to have a set of candidate labels that includes the true label. This problem arises in many real-world tasks such as automatic image annotation~\cite{chen2017autoimageannotation} and facial age estimation~\cite{panis2015faceage}.

As PLL focuses on multi-class classification, there is only one ground-truth label for each training example, and other labels in the candidate label set are actually wrong (false positive) labels, which would have a negative impact on model training. Therefore, there exists the challenge of \emph{label ambiguity} in PLL. To address this challenge, the current mainstream solution is to disambiguate the candidate labels so as to figure out the true label for each training instance~\cite{jin2002learningmultiplelabels,yu2016pllidentimaximum,liu2012PLL-idnbase,feng2020provably,lv2020proden}.
 
However, most of the existing disambiguation methods normally leverage simple heuristics to iteratively update the labeling confidences or pseudo labels~\cite{feng2020provably,lv2020proden,wang2022pico,wu2022CRDPLL}, which could not achieve convincing performance in identifying the true label during the training phase. Generally, if more true labels of training instances can be identified, we can train a better model. This motivates us to focus on identifying the true labels of training instances as many as possible, thereby training a desired model.

In this paper, we propose a method called \textbf{CroSel} (Cross Selection of Confident Pseudo Labels), which leverages historical prediction information from deep neural networks to accurately identify true labels for most training examples. Our selecting criteria are based on the assumption that if a model consistently predicts the same label for an input image with high confidence and low volatility, then that label has a high probability of being the true label for that example. Using the cross selection strategy, the true labels of the vast majority of training examples can be accurately identified, with only negligible noise. Moreover, in order to avoid sample waste and tiny noise resulting from the selection, we also proposed a co-mix consistency regularization to generate trainable targets for all examples. This regulation term serves as an essential complement to our method, which can further enhance the scope and accuracy of our selection of ``true" labels. The algorithm details are shown in Section 3.

Our main contributions are summarized as follows:
\begin{itemize}
\item We propose a cross selection strategy to select the confident pseudo labels in the candidate label set based on historical prediction information. This strategy shows high selection precision and selection ratio of ``true" labels in the experiment.
\item We propose a new consistency loss regularization term that can leverage MixUp~\cite{zhang2017mixup} to enhance the data and generate trainable targets as an important supplement to our method.
\item We experimentally show that CroSel achieves state-of-the-art performance on common benchmark datasets. We also provide extensive ablation studies to examine the effect of the different components of CroSel.

\end{itemize}

\section{Related Work}
\begin{figure*}[ht]
\begin{center}
\centerline{\includegraphics[width=1.0\textwidth]{frame.changeorder.png}}
\caption{The left side of the figure is a brief example of our memory bank that stores the softmax output of the model for the last $t$ epochs, which is updated by the FIFO (First In First Out) principle; the middle is the cross selection strategy: within each epoch, data subsets $\mathcal{D}_{\mathrm{sel}}$ with confident pseudo-labels are selected from the MB of each network, which produce loss function $\mathcal{L}_{\mathrm{l}}$ to the training process for the other network;
the right side illustrates our co-mix regularization term and the corresponding loss function $\mathcal{L}_{\mathrm{cr}}$ .}
\label{icml-historical}
\end{center}
\end{figure*}

\noindent \textbf{Partia-Label Learning.\quad} This setting allows each training example to be annotated with a set of candidate labels for which the ground truth label is guaranteed to be included. However, \emph{label ambiguity} can pose a significant challenge in PLL. Early methods used an averaging strategy, which tends to treat each candidate label equally.~\cite{cour2011learningPLL2,zhang2017plldisambiguation} But such methods are easily affected by negative labels in the candidate label set, thus forming wrong classification boundaries. Afterwards, the identification-based method~\cite{jin2002learningmultiplelabels,yu2016pllidentimaximum,liu2012PLL-idnbase} has received more attention from the community, which regard the ground-truth label as a latent variable, and will maintain a confidence level for each candidate label. For instance, Yu \emph{et al.}~\cite{yu2016pllidentimaximum} introduce the maximum margin constraint to PLL problems, trying to optimize the margin between the model outputs from candidate labels and other negative labels. This method has shown better performance in disambiguating labels.

Recently, partial-label learning has been combined with deep networks, leading to significant improvements in performance. Feng \emph{et al.}~\cite{feng2020provably} assume that partial labels come from a uniform generation model and give a mathematical formulation, which is adopted by most of the algorithms proposed later. Based on this, they also propose an algorithm for classification consistency and risk consistency. PRODEN~\cite{lv2020proden} assumes the true label should be the one with the smallest loss among the candidate labels, and improve the classification risk algorithm accordingly. Wen \emph{et al.}~\cite{wen2021LWS} propose a risk-consistent leveraged weighted loss with label-specific candidate label sampling.
PiCO~\cite{wang2022pico} innovatively introduced contrastive learning~\cite{oord2018contrastive} to the field and provided a solid theoretical analysis based on EM. CRDPLL~\cite{wu2022CRDPLL} proposed a new consistency loss item in this field and treat the parts that are not selected as candidate labels as supervision information. SoLar~\cite{wang2022solar} focuses on solutions to PLL problems in more realistic scenarios such as class-imbalanced settings.

\noindent\textbf{Sample selection.\quad} Sample selection is a popular technique in deep learning, especially for datasets with noisy labels or incomplete annotations. Then an obvious idea is to separate the clean samples and noisy samples in the mixed dataset. 
To address this issue, many existing works adopt the small loss criterion~\cite{han2018coteaching,jiang2018mentornet}, which assumes that clean samples tend to have a smaller loss than noisy samples during training.
MentorNet~\cite{jiang2018mentornet} is a representative work that let the teacher model pick up clean samples for the student model. Co-teaching~\cite{han2018coteaching} constructs a double branches network to select clean samples for each branch, which is different from the teacher-student approach since none of the models supervise the other but rather help each other out. This idea was improved by some research later~\cite{yu2019coteachingimprove1,wang2019coteachingimprove2} to achieve better performance.
Curriculum learning is also applied to this field~\cite{han2018CLlabelnoise1}, which considers clean labeled data as an easy task, while noisily labeled data as a harder task. 
Guo \emph{et al.}~\cite{guo2018curriculumnet} split data into subgroups according to their complexities, in order to optimize the training objectives in the early stage of course learning. 
OpenMatch~\cite{saito2021openmatch} trains $n$ OVA classifiers to select the in-distribution samples under the open set setting. 
Generative models such as Beta Mixture Model ~\cite{arazo2019betamixture} and Gaussian Mixture Model~\cite{li2020dividemix} are also used to fit loss functions to distinguish clean labels from noisy labels. Recently, the fluctuation magnitude of the output of the same example is also considered as an important credential to judge whether the label is clean~\cite{wei2022self}.

\section{Our methods}
In this section, we provide a detailed explanation of how our algorithm works. Our method is composed of two main components: a cross selection strategy that utilizes two models to select confident pseudo labels for each other, and a consistency regularization term that is applied across different data augmentation versions. The latter part not only addresses the issue of label waste resulting from the selection process but also enhances the quantity and accuracy of selections. The pseudo-code for our algorithm is presented in Algorithm 1.

\subsection{Problem setting}
Suppose the feature space is $\mathcal{X}\in\mathbb{R}^d$ with $d$ dimensions and the label space is $\mathcal{Y}=\{1,2,\dots,k\}$ with $k$ classes. We are given a dataset $\mathcal{D} =\{(\boldsymbol{x}_i,S_i)\}_{i=1}^n$ with n examples,
where the instance $\boldsymbol{x}_i\in\mathcal{X}$ and the candidate label set $S_i\subset\mathcal{Y}$. Same as previous studies, we assume that the true label $y_i\in\mathcal{Y}$ of each input $\boldsymbol{x}_i$ is concealed in $S_i$.

Our aim is to train a multi-class classifier $f:\mathbb{R}^d\rightarrow\mathbb Y$ that minimizes the classification risk on the given dataset. For our classifier $f$, we use $f(\boldsymbol{x})$ to represent the output of classifier $f$ on given input $\boldsymbol{x}$. And we use $\hat{y}=\mathrm{argmax}_{y\in\mathcal{Y}}f_y(\boldsymbol{x})$ to denote the prediction of our classifier, where $f_y(\boldsymbol{x})$ is the $y\text{-}$th coordinate of $f(\boldsymbol{x})$.


\begin{algorithm*}[tb]
    \centering
   \caption{Pseudo-code of CroSel}
   \label{alg:ours}
\begin{algorithmic}
   \STATE {\bfseries Input:} Training dataset $\mathcal{D}=\{(\boldsymbol{x}_i,S_i)\}_{i=1}^n$, consistency regularization parameter $\lambda_{\mathrm{cr}}$, sharpen parameter $T$, confidence threshold $\gamma$, memory bank $\mathrm{MB}^{(1)}$, $\mathrm{MB}^{(2)}$, network $\Theta^{(1)},\Theta^{(2)}$, epoch $E$, iteration $I$.
   \STATE {\bfseries Procedure:}
   \STATE $\mathrm{MB}^{(1)},\mathrm{MB}^{(2)},\Theta^{(1)},\Theta^{(2)}=\mathrm{WarmUp}(\mathrm{MB}^{(1)},\mathrm{MB}^{(2)},\Theta^{(1)},\Theta^{(2)},\mathcal{D})$. \qquad // \quad CC algorithm
   \FOR{$e=1$ {\bfseries to} $E$}
   \STATE Select labeled dataset $\mathcal{D}_{\mathrm{sel}}^1$ through $\mathrm{MB}^{(1)}$.     
   \STATE Select labeled dataset $\mathcal{D}_{\mathrm{sel}}^2$ through $\mathrm{MB}^{(2)}$.\qquad // \quad Eq. (4)
    \FOR{$k=1,2$ }
    % \qquad // Repeat training process on two models
    \FOR{$i=1$ {\bfseries to} $I$}
    \STATE   Fetch a labeled batch $\hat{B}_i$ from the opposite selected dataset $\mathcal{D}_{\mathrm{sel}}^{\thicksim k}$.
    \STATE   Fetch a  batch $B_i$ from the training dataset $\mathcal{D}$.
    \STATE   Compute the loss $L$ among the two batches $\hat{B}_i$ and $B_i$ through Eq. (13).
    \STATE   Update the weight of $\Theta^k$ by optimizer.
    \ENDFOR    
    \STATE  Update the memory bank $\mathrm{MB}^{(k)}$ through the FIFO principle.
    \ENDFOR
   \ENDFOR
\end{algorithmic}
\end{algorithm*}

\subsection{Selection Strategy}
In the current partial-label learning task, the large number of candidate labels can confuse classifier, making it difficult for the classifier to capture specific features belonging to a certain label. Therefore, our goal is to identify the most likely true label among the candidate labels and eliminate the interference of other negative labels during model training. By selecting these ``true" labels, we can train the data with a supervised learning approach.

\noindent\textbf{Warm up.\quad}Before selecting, we warm up the network using the entire training set. The goal of this stage is to reduce the classification risk of the input $x$ to the whole set of candidate labels $S$, and obtain some historical information that can be used to select. Therefore, here we use CC algorithm~\cite{feng2020provably} to warm up models for 10 epochs.
At the same time, we'll update the value of memory bank $\mathrm{MB}$.

\noindent\textbf{Selecting criteria.\quad}We have three criteria for selecting the confident pseudo labels from the candidate label set.
Depending on the setup of our problem, the true label of each example must be in its candidate label set. This is our first criterion. In addition, we believe that an example's predicted label is likely to be the true label if the model predicts it with high confidence and low fluctuation. To determine the latter, we maintain a memory bank $\mathrm{MB}$ to store the historical prediction information of this neural network.

The size of the memory bank is $t\times n \times k$, where $t$ denotes the length of time it stores, $n$ is the length of dataset, and $k$ is the number of categories in the classification. In other words, $\mathrm{MB}$ stores the output after softmax of the model in the last $t$ epochs. 
$\mathrm{MB}$ is structured as a queue, with each element being a $k$-dimensional vector $\boldsymbol{q}$ representing the output of an example in a particular epoch. We update $\mathrm{MB}$ with the FIFO principle. 
These selection criteria can be summarized as follows.
\begin{align}
\beta_1 &= (\mathrm{argmax}(\boldsymbol{q}^i) \in S),\\
\beta_2 &= (\mathrm{argmax}(\boldsymbol{q}^i) == \mathrm{argmax}(\boldsymbol{q}^{i+1})),\\
\beta_3 &= (\frac{1}{t} \sum_{i=1}^{t} \mathrm{max}(\boldsymbol{q}^i) > \gamma).
\end{align}
where $i=1,2...t$, $t \geq 2 $, $\gamma$ is the confidence threshold of selection.
$\beta_1$ lets our selected label be in the candidate label set. $\beta_2$ limits that the label we picked has not flipped in the past $t$ epochs, which is a volatility consideration and $\beta_3$ ensures that the label we selected has a high confidence level. The final selected dataset is $\mathcal{D}_{\mathrm{sel}}$:
\begin{equation}
\mathcal{D}_{\mathrm{sel}} = {((\boldsymbol{x}_i,\mathrm{argmax}(\boldsymbol{q}^t_i)) | (\beta_1^i \wedge \beta_2^i \wedge \beta_3^i) =1, \boldsymbol{x}_i \in \mathcal{D})}
\end{equation}

\noindent\textbf{Selected label loss.\quad}
After selecting the high-confidence examples, we obtain a dynamically updated data subset $\mathcal{D}_{\mathrm{sel}}=(X,\hat{Y})$. Each tuple of the subset has an instance $\boldsymbol{x}$ and a selected label $\hat{y}=\mathrm{argmax}(\boldsymbol{q}^t)$. In this configuration, we can use the basic cross-entropy loss to deal with this part of samples.
\begin{equation}
\mathcal{L}_{\mathrm{l}} =\frac{1}{|\mathcal{D}_{\mathrm{sel}}|} { \sum_{\boldsymbol{x} \in \mathcal{D}_{\mathrm{sel}}}} \mathrm{CE}(f(\boldsymbol{x}_{\mathrm{w}}),\hat{y})
\end{equation}
where $\mathrm{CE}(\cdot,\cdot)$ denotes the cross entropy loss, $\boldsymbol{x}_{\mathrm{w}}$ denotes the weak augmented version of example $\boldsymbol{x}$.

\noindent\textbf{Cross Selection.\quad}
However, In the process of selecting labels, it is difficult to guarantee that the selected labels are 100\% accurate.
In order to maximize the accuracy of selection, we propose a cross selection framework based on the idea of ensemble learning. Specifically, we train two identical models $\Theta^{(1)}$ and $\Theta^{(2)}$ through the same training and label selection process. By forming different decision boundaries, the two models can adaptively correct most of the errors even if there is noise in the selected confident pseudo labels. 

To maximize the accuracy of the selected labels, we employ a cross-supervised training process. This involves using the selected dataset $\mathcal{D}_{\mathrm{sel}}^1$ from $\mathrm{MB}^{(1)}$ to train $\Theta^{(2)}$, and vice versa, using the selected dataset $\mathcal{D}_{\mathrm{sel}}^2$ from $\mathrm{MB}^{(2)}$ to train $\Theta^{(1)}$. By doing so, the two models can learn from each other and further improve their ability to select true labels.
In the test process, we will average the output of the two models to reduce the variance.
\begin{equation}
f^{\prime}(\boldsymbol{x}) =\frac{1}{2} (f^1(\boldsymbol{x})+f^2(\boldsymbol{x}))
\end{equation}
where $f^{\prime}(\boldsymbol{x})$ denotes the final output of our method in the test process, $f^{1}(\boldsymbol{x})$ denotes the output of $\Theta^{(1)}$, $f^{2}(\boldsymbol{x})$ denotes the output of $\Theta^{(2)}$.

\subsection{Co-mix Consistency Regulation}
\noindent\textbf{Motivation.\quad} When dealing with complex partial-label learning tasks, our label selection strategy may not accurately select the true labels for all examples. If we only use the selected examples with their corresponding labels, it will result in a significant amount of wasted data, which contradicts our goal of utilizing as many examples as possible. Therefore, we aim to provide a trainable target for the remaining examples that are not selected. 
However, our setting differs from traditional semi-supervised learning as the proportion of unlabeled examples is relatively small. So it is unreasonable to directly transfer the existing semi-supervised learning tools to unlabeled data like other weakly supervised learning methods. 

Motivated by this, we hope to propose a regularization term that can serve as an important supplement to our method and help us select examples.
We proposed the co-mix regularization term, which employs two widely used data augmentation methods: weak augmentation and strong augmentation, to generate pseudo-labels as training targets for consistency regularization. We further employ MixUp to further enhance the data. It is worth mentioning that the term "pseudo labels" in this part specifically refers to the soft labels generated from different data augmentation versions, rather than the confident hard labels selected during the selection process.

% We assign a pseudo target $\boldsymbol{p}_i$ for each augmented image $\boldsymbol{x}{aug1}$, which is generated by another augment transformation of the image $\boldsymbol{x}{aug2}$. With the help of these pseudo-labels, we can process samples like normal supervised learning. This can also be seen as another form of consistent regularization loss. 

\noindent\textbf{Pseudo label generating.\quad}Consistency loss is a simple but effective idea in weakly supervised learning, whose key point is to reduce the gap between the output of two perturbed examples after passing through the model. As discussed in the previous section, we used two widely used data augmentations: 'weak' and 'strong', and crossed them to generate pseudo labels. Specifically, as stated in Figure 1, the pseudo label corresponding to weak augmented example is generated by strong augmented example, while the pseudo label corresponding to strong augmented example is generated by weak augmented example.

To generate these pseudo labels, we fix the parameters of the neural network, and pass the augmented images through the model to get the logits output. And we perform two operations on the logits, sharpening and normalization. For sharpening operation, we use a hyper parameter $T$, the more $T$ goes to zero, the more logits tend to become a one-hot distribution. These two operations can be summarized by the following formula.
\begin{equation}
\boldsymbol{p}_{i}=
\begin{cases}
\frac{\exp(f_j(\boldsymbol{x})^{\frac{1}{T} })}{ {\textstyle \sum_{j\in S} \exp(f_j(\boldsymbol{x})^{\frac{1}{T} })} }& \text{$i \in S$}\\
0& \text{$i \notin S$}
\end{cases}
\end{equation}
where $\boldsymbol{p}_{i}$ denotes the $i\text{-}_{th}$ coordinate of pseudo label.

\noindent\textbf{MixUp.\quad} After generating the pseudo labels of each example, we end up with two datasets that can be trained: $(X_{\mathrm{w}},P_{\mathrm{s}})$ and $(X_{\mathrm{s}},P_{\mathrm{w}})$. The subscripts \emph{w} and \emph{s} indicate the type of data augmentation used, i.e., weak or strong. Then, We spliced the two datasets together to further enhance the data with MixUp.~\cite{zhang2017mixup} For a pair of two examples with their corresponding pseudo labels $(\boldsymbol{x}_1, \boldsymbol{p}_1),(\boldsymbol{x}_2, \boldsymbol{p}_2)$, we compute $(\boldsymbol{x}^{\prime}, \boldsymbol{p}^{\prime})$ by the following formula: 
\begin{equation}
\lambda  \sim{\mathrm{Beta}(\alpha,\alpha)}
\end{equation}
\begin{equation}
\lambda^{\prime} = \mathrm{max}(\lambda,1-\lambda)
\end{equation}
\begin{equation}
\boldsymbol{x}^{\prime}  =\lambda^{\prime}\boldsymbol{x}_1+(1-\lambda^{\prime})\boldsymbol{x}_2
\end{equation}
\begin{equation}
\boldsymbol{p}^{\prime}  =\lambda^{\prime}\boldsymbol{p}_1+(1-\lambda^{\prime})\boldsymbol{p}_2
\end{equation}
Then we can use the typical cross-entropy loss on every example, the consistency regulation loss will be:
\begin{equation}
\mathcal{L}_{\mathrm{cr}} = \frac{1}{2n} { \sum_{i=1}^{2n}} \mathrm{CE}(f(\boldsymbol{x}^{\prime}),{\boldsymbol{p}^{\prime}})
\end{equation}
where $\mathrm{CE}(\cdot,\cdot)$ denotes the softmax cross entropy loss, $n$ denotes the length of a dataset.

\begin{table*}[t]
\small
\caption{Accuracy(mean ± std) comparisons on benchmark datasets.}
\label{Main reults}
\begin{center}
\begin{tabular}{l|c|cccccc}
\toprule
Dataset & $q$ & Ours & CRDPLL & PiCO & PRODEN & LWS & CC  \\
\midrule
\multicolumn{1}{c|}{\multirow{3}{*}{CIFAR-10}}    & $q=0.1$& 97.31 ± 0.04\% &\textbf{97.41 ± 0.06}\% & 96.10 ± 0.06\% &95.66 ± 0.08\% &91.20 ± 0.07\% &  90.73 ± 0.10\% \\
\multicolumn{1}{c|}{} & $q=0.3$& \textbf{97.50 ± 0.05}\% & 97.38 ± 0.04\%  & 95.74 ± 0.10\% & 95.21 ± 0.07\% & 89.20 ± 0.09\% & 88.04 ± 0.06\% \\
\multicolumn{1}{c|}{}    & $q=0.5$& \textbf{97.34 ± 0.05}\% & 96.76 ± 0.05\% & 95.32 ± 0.12\% & 94.55 ± 0.13\% & 80.23 ± 0.21\% & 81.01 ± 0.38\% \\
\hline
\multicolumn{1}{c|}{\multirow{3}{*}{SVHN}}    & $q=0.1$& \textbf{97.71 ± 0.05}\% &97.63 ± 0.06\% & 96.58 ± 0.04\% &96.20 ± 0.07\% &96.42 ± 0.09\% &  96.99 ± 0.17\% \\
\multicolumn{1}{c|}{} & $q=0.3$& \textbf{97.96 ± 0.05}\% & 97.65 ± 0.07\%  & 96.32 ± 0.09\% & 96.11 ± 0.05\% & 96.15 ± 0.08\% & 96.67 ± 0.20\% \\
\multicolumn{1}{c|}{}    & $q=0.5$& \textbf{97.86 ± 0.06}\% & 97.70 ± 0.05\% & 95.78 ± 0.05\% & 95.97 ± 0.03\% & 95.79 ± 0.05\%  & 95.83 ± 0.23\% \\
\hline
\multicolumn{1}{c|}{\multirow{3}{*}{CIFAR-100}}    & $q=0.01$& \textbf{84.24 ± 0.09}\% &82.95 ± 0.10\% & 74.89 ± 0.11\% &72.24 ± 0.12\% &62.03 ± 0.21\% &  66.91 ± 0.24\% \\
\multicolumn{1}{c|}{} & $q=0.05$& \textbf{83.92 ± 0.24}\% & 82.38 ± 0.09\%  & 73.26 ± 0.09\% & 70.03 ± 0.18\% & 57.10 ± 0.17\% & 64.51 ± 0.37\% \\
\multicolumn{1}{c|}{}    & $q=0.10$& \textbf{84.07 ± 0.16}\% & 82.15 ± 0.20\% & 70.03 ± 0.10\% & 69.82 ± 0.11\% & 52.60 ± 0.54\%  & 61.50 ± 0.36\%\\
\hline
\end{tabular}
\end{center}
\end{table*}

\subsection{Algorithm overview}

\noindent\textbf{Overall loss.\quad} In the formal training phase, our loss function will be composed of two parts, the supervised loss $\mathcal{L}_l$ in the selected label set $\mathcal{D}_{\mathrm{sel}} $ and the consistency regularization item loss $\mathcal{L}_{\mathrm{cr}}$. The two will be dynamically combined into the final loss function by a hyperparameter.
\begin{equation}
\mathcal{L}_{\mathrm{all}} =  \mathcal{L}_{\mathrm{l}} + \mathcal{L}_{\mathrm{cr}} * \lambda_{\mathrm{d}}
\end{equation}
$\lambda_{\mathrm{d}}$ is a dynamically changing parameter, $\mathcal{L}_l$ and $\mathcal{L}_{\mathrm{cr}}$ can be calculated by Eq. (5) and Eq. (12), respectively. The use of MixUp can cause significant changes to the original feature space, making it necessary to adjust the weight of the regularization term in the loss function as the number of selected samples increases. 
To achieve this, we proposed a gradually decreasing $\lambda_{\mathrm{d}}$ with the increase of selected samples. We can control the magnitude of $\lambda_{\mathrm{d}}$ with a set hyperparameter $\lambda_{\mathrm{cr}}$. The updated rules of $\lambda_{\mathrm{d}}$ are as follows.
\begin{equation}
\lambda_{\mathrm{d}} = (1- r_{\mathrm{s}})* \lambda_{\mathrm{cr}}
\end{equation}
where $r_{\mathrm{s}}$ denotes the percentage of labeled data that we picked out, $\lambda_{\mathrm{cr}}$ is a hyperparameter that collaboratively adjusts the ratio of two loss items.

\section{Experiments}
In this section, we provide a detailed description of the experiments we conducted. We introduce the experimental setup and present our main experimental results. Additionally, we present the findings of our ablation experiments.

\subsection{Experimental Setup}
\noindent\textbf{Datasets.\quad}We used three widely used benchmarks in this field: SVHN~\cite{svhn}, CIFAR-10~\cite{krizhevsky2009cifar} and CIFAR-100~\cite{krizhevsky2009cifar}.  The way we generate partial labels is by flipping the negative labels $\overline{y} \neq y$ of the example with a set probability $q=P(\overline{y} \in S|\overline{y} \neq y)$. With the increase of $q$, the noise of the dataset increases gradually. Following PiCO~\cite{wang2022pico}, we consider $q=\{0.01,0.05,0.1\}$ for CIFAR-100 and $q=\{0.1,0.3,0.5\}$ for other datasets.


\noindent\textbf{Compared methods.\quad} We choose five well-performed partial-label learning algorithms to compare: 
\begin{itemize}
    \item CRDPLL~\cite{wu2022CRDPLL}, an algorithm that takes non-candidate labels as supervision information and proposes a new consistency loss term between augmented images.
    \item PiCO~\cite{wang2022pico}, a theoretical solid framework that combines contrastive learning and prototype-based label disambiguation algorithm.
    \item LWS~\cite{wen2021LWS}, an algorithm that wants to balance the risk error between the candidate label set and the non-candidate label set.
    \item PRODEN~\cite{lv2020proden}, a self-training algorithm that dynamically updates the confidence of candidate labels.
    \item CC~\cite{feng2020provably}, an algorithm that wants to minimize the classification error of the whole candidate label sets.
    % \item RC~\cite{feng2020provably}, a algorithm that based on risk-consistency.
\end{itemize}

\begin{table}[htpb]
\caption{Selected ratio and selected accuracy(mean ± std) on benchmark datasets. S-ratio represents the selected ratio and S-acc represents selected accuracy in $\mathcal{D}_{\mathrm{sel}}$.}
\label{Co-selct results:Sratio and Sacc}
\begin{center}
\begin{tabular}{c|c|c|c}
\toprule
\multicolumn{1}{l|}{Datasets} & Setting                  & Index          &  Performance\\ 
\midrule
\multirow{6}{*}{CIFAR-10}     & \multirow{2}{*}{$q=0.1$} & S-ratio &  99.09 ± 0.07\% \\
                              &                          & S-acc   &  99.79 ± 0.05\%\\ \cline{2-4}
                              & \multirow{2}{*}{$q=0.3$} & S-ratio  &  98.10 ± 0.10\%\\
                              &                          & S-acc   &  99.55 ± 0.03\%\\ \cline{2-4}
                              & \multirow{2}{*}{$q=0.5$} &  S-ratio&  96.25 ± 0.12\%\\
                              &                          & S-acc  &  99.44 ± 0.06\%\\ \hline
\multirow{6}{*}{SVHN}     & \multirow{2}{*}{$q=0.1$} & S-ratio &  97.25 ± 0.14\% \\
                              &                          & S-acc   &  99.84 ± 0.06\% \\ \cline{2-4}
                              & \multirow{2}{*}{$q=0.3$} & S-ratio  &  76.42 ± 0.21\%\\
                              &                          & S-acc   &  99.77 ± 0.06\%\\ \cline{2-4}
                              & \multirow{2}{*}{$q=0.5$} &  S-ratio&  73.21 ± 0.15\%\\
                              &                          & S-acc  &  99.34 ± 0.02\%\\ \hline
\multirow{6}{*}{CIFAR-100}     & \multirow{2}{*}{$q=0.01$} & S-ratio &  96.58 ± 0.13\% \\
                              &                          & S-acc   &  99.71 ± 0.06\% \\ \cline{2-4}
                              & \multirow{2}{*}{$q=0.05$} & S-ratio  &  95.45 ± 0.21\%\\
                              &                          & S-acc   &  98.29 ± 0.15\%\\ \cline{2-4}
                              & \multirow{2}{*}{$q=0.10$} &  S-ratio&  93.61 ± 0.12\%\\
                              &                          & S-acc  &  97.93 ± 0.11\%\\ \hline

\end{tabular}
\end{center}
\end{table}

\noindent\textbf{Implementations.\quad} Our implementation is based on PyTorch~\cite{paszke2019pytorch}. We use WRN-34-10 (short for Wide-ResNet-34-10) as the backbone model with a weight decay of 0.0001 on all the datasets for all methods. We set the batch size as 64 and total epochs as 200, using SGD as optimizer with a momentum of 0.9, and set the initial learning rate as 0.1, which is divided by 10 after 100 and 150 epochs respectively. For the hyper parameter in our method, We set $t=3$, $\alpha=0.75$, $T=0.5$ for all datasets, and $\lambda_{\mathrm{cr}}=1$ for CIFAR-100, $\lambda_{\mathrm{cr}}=4$ for others. For the selection threshold, we set $\gamma=0.9$ for CIFAR-type datasets, and $\gamma=0.85$ for SVHN. We present the mean and standard deviation in each case based on three independent runs with different random seeds. 


\begin{table}[!t]
\caption{Results for the study of scope of co-mix regularization.}
\label{Ablation comix results}
\begin{center}
\begin{tabular}{c|c|c|c}
\toprule
\multicolumn{1}{l|}{Setting} & Scope         & Index          &  Performance\\ 
\midrule
\multirow{9}{*}{\makecell[c]{CIFAR-10 \\ $q=0.5$}}     & \multirow{3}{*}{All data} & acc &  97.34\% \\
                              &                          & S-ratio   &  96.25\%\\ 
                              &                          & S-acc   &  99.44\%\\ \cline{2-4}
                              & \multirow{3}{*}{Unseleted data} & acc  &  90.32\%\\
                              &                          & S-ratio   &  93.27\%\\ 
                              &                          & S-acc   &  95.72\%\\ \cline{2-4}
                              & \multirow{3}{*}{None} &  acc&  81.01\%\\
                              &                          & S-ratio   &  90.23\%\\ 
                              &                          & S-acc  &  89.72\%\\ \hline
\multirow{9}{*}{\makecell[c]{CIFAR-100 \\ $q=0.1$}}     & \multirow{3}{*}{All data} & acc &  84.07\% \\
                              &                          & S-ratio   &  93.61\%\\ 
                              &                          & S-acc   & 97.93\%\\ \cline{2-4}
                              & \multirow{3}{*}{Unseleted data} & acc  &  77.61\%\\
                              &                          & S-ratio   &  90.12\%\\ 
                              &                          & S-acc   &  97.63\%\\ \cline{2-4}
                              & \multirow{3}{*}{None} &  acc&  70.68\%\\
                              &                          & S-ratio   &  78.65\%\\ 
                              &                          & S-acc  &  96.22\%\\ \hline
\end{tabular}
\end{center}
\end{table}

\subsection{Main empirical results}

As shown in Table 1, our methods achieve state-of-the-art results on all the settings except CIFAR-10 with $q=0.1$. Notably, on the complex dataset CIFAR-100, our method significantly improves performance.

However, a counter-intuitive phenomenon appears in our experiment, that is, the performance of our method does not necessarily decline strictly with the increase of noise $q$; on the contrary, it may perform best in the case of moderate noise. We believe the possible reason for this phenomenon is that: On the one hand, when generating pseudo labels, we  normalize them according to Eq. (7). That is to say, as the number of candidate labels increases, the ingredients involved in MixUp will also increase, leading to better-enhanced data interpolation. On the other hand, the increase of candidate labels also represents the increase of noise, which can have a negative impact on the algorithm's performance. Therefore, CroSel performs well in cases of intermediate noise, representing an optimal solution found in such a trade-off problem.

Table 2 presents the selection ratio and accuracy of CroSel in selecting the true labels for the selected label dataset $\mathcal{D}_{\mathrm{sel}}$. It is evident that, CroSel can accurately select the true labels of most examples in the dataset under various noise conditions. Notably, for CIFAR-10 and CIFAR-100, the selection ratio and accuracy are over $90\%$ both. However, in SVHN, the selection ratio drops significantly with the increase of noise. This could be attributed to the fact that digital images in SVHN have relatively simple shape features, and MixUp may significantly disturb the feature space.

\begin{figure*}[!t]
\centering
\subfigure[CTAFR-10 Accuracy]{
	\label{fig:subfig:a} %% label for first subfigure
	\includegraphics[width=2.2in,height=1.9in]{cifar10acc70-100.png}}
\subfigure[CTAFR-10 Selected Ratio]{
	\label{fig:subfig:c} %% label for second subfigure
	\includegraphics[width=2.2in,height=1.9in]{cifar10sratio.png}} 
\subfigure[CTAFR-10 Selected Accuracy]{
	\label{fig:subfig:b} %% label for second subfigure
	\includegraphics[width=2.2in,height=1.9in]{cifar10sacc99-100.png}}
 \subfigure[CTAFR-100 Accuracy]{
	\label{fig:subfig:d} %% label for first subfigure
	\includegraphics[width=2.2in,height=1.9in]{cifar100acc60-85.png}}
\subfigure[CTAFR-100 Selected Ratio]{
	\label{fig:subfig:e} %% label for second subfigure
	\includegraphics[width=2.2in,height=1.9in]{cifar100sratio.png}}
\subfigure[CTAFR-100 Selected Accuracy]{
	\label{fig:subfig:f} %% label for second subfigure
	\includegraphics[width=2.2in,height=1.9in]{cifar100sacc.png}}
\caption{Parameter test for $\lambda_{\mathrm{cr}}$ on CTAFR-10 and CTAFR-100.}
\label{fig:Parameter test} %% label for entire figure
\end{figure*}

\subsection{Ablation study}

\noindent\textbf{Parameter test on $\lambda_{\mathrm{cr}}$.\quad}In this experiment, we test the parameter $\lambda_{\mathrm{d}}$, which weights the contribution of the consistency regularization term to the training loss. As mentioned in  Eq. (14), the parameter $\lambda_{\mathrm{d}}$ is directly influenced by the hyperparameter $\lambda_{\mathrm{cr}}$. Therefore, We test $\lambda_{\mathrm{cr}}=\{1,2,4\}$ on CIFAR-10 ($q=0.5$) and CIFAR-100 ($q=0.1$). At the same time, we also try to fix the parameter $\lambda_{\mathrm{d}}$, that is, its value is not related to the selection ratio $r_{\mathrm{s}}$. This setting we denote by $\lambda_{\mathrm{cr}}(fix)$, we test $\lambda_{\mathrm{cr}}(fix)=\{0.5,1,2\}$. The results are visualized in Figure 2, and detailed data can be found in Appendix B.3.

As mentioned in Table 2 above, CroSel achieves a very high selection ratio. When using a dynamically changing $\lambda_{\mathrm{d}}$, the contribution made by the regularization term would be quite small in the later stages as the learning rate decays. In contrast, with a fixed value of $\lambda_{\mathrm{d}}$, the regularization term would still have a significant contribution in the later stages. Although the accuracy rate on the test set does not show a significant difference between different parameters, the selection effect is critical. A too large contribution of the regularization item will slightly improve the selection ratio at the cost of a decrease in the selection accuracy, which goes against the original intention of our algorithm design. Therefore, we finally decided to use dynamically changing parameters. In other words, we want the algorithm to return to a supervised learning setting with minimal noise as much as possible at the end of the training process.


\noindent\textbf{The influence on the scope of consistency regulation term.\quad}The preceding ablation experiment explored the magnitude of the co-mix regularization, while this experiment focused on the scope of the regularization term. As described in Section 3, our co-mix regularization is designed to avoid sample waste. As such, a natural idea is to apply the regularization term to the unselected samples, as in traditional semi-supervised learning. However, our setting differs from traditional semi-supervised learning in that our unselected data only constitutes a small portion of all the data. So we conducted experiments by trying three cases: no regularization term, using the regularization term only for the unselected dataset, and using the regularization term for all examples.

The results in Table 3 suggest that with the expansion of the application scope of co-mix regularization term, the model's performance steadily improves, and the number of selected labeled examples also gradually increases. This also shows that our co-mix regularization term and selection strategy can achieve a mutually beneficial effect.

\begin{table}[htpb]
\small
\caption{Accuracy(mean ± std) on ablation study on select criteria.}
\label{Ablation t results}
\begin{center}
\begin{tabular}{c|c|c|c|c}
\toprule
Setting & $t$ & Accuracy & $\gamma$ & Accuracy\\
\midrule
\multicolumn{1}{c|}{\multirow{3}{*}{\makecell[c]{CIFAR-10 \\ $q=0.3$}}}    & $t=2$ &97.03\%& $\gamma=0.8$& 96.24\%  \\
\multicolumn{1}{c|}{} & $t=3$ & 97.50\%& $\gamma=0.9$ & 97.50\% \\
\multicolumn{1}{c|}{} & $t=4$ & 96.15\% & $\gamma=0.95$ & 97.38\%\\
\hline
\multicolumn{1}{c|}{\multirow{3}{*}{\makecell[c]{CIFAR-100 \\ $q=0.1$}}}    & $t=2$ & 82.74\%& $\gamma=0.8$& 80.20\% \\
\multicolumn{1}{c|}{} & $t=3$ & 84.07\%& $\gamma=0.9$ & 84.07\% \\
\multicolumn{1}{c|}{} &$ t=4$ & 83.56\%& $\gamma=0.95$ & 83.56\% \\
\bottomrule
\end{tabular}
\end{center}
\end{table}

\noindent\textbf{Whether the selection criteria are strict or lenient.\quad}
The two parameters $t$ and $\gamma$ in Eq. (2) and Eq. (3) determine the strictness of our selection criteria. $t$ represents the length of historical information stored in $\mathrm{MB}$, while $\gamma$ represents the select threshold for the average prediction confidence of the model for the example prediction in the past $t$ epochs. A larger $t$ and a higher $\gamma$ represent a stricter selection criterion, resulting in a smaller $\mathcal{D}_{\mathrm{sel}}$ size but higher precision, which can affect model training. During the early stages, it may be difficult to select enough examples under the condition of using stricter selection criteria. Therefore, in this and the next ablation experiment, we set the label flipping probability $q$ to 0.3 on CIFAR-10. However, in the later stages, the impact of selection criteria on the final accuracy rate is not significant if the initial stage is passed smoothly. Our experiment shows that $t=3$ and $\gamma=0.9$ are suitable values that can be applied to most experimental environments. The experimental results are presented in Table 4 and Appendix B.3.


\begin{table}[!htpb]
\caption{Results for Augmentations test on $\mathcal{D}_{\mathrm{sel}}$.}
\label{Results for Parameter test on $D_{sel}$}
\begin{center}
\begin{tabular}{c|c|ccc}
\toprule
Setting & Aug type & Accuracy & Selected ratio\\
\midrule
\multicolumn{1}{c|}{\multirow{3}{*}{\makecell[c]{CIFAR-10 \\ $q=0.3$}}}    &None& 96.91\% & 97.44\%  \\
\multicolumn{1}{c|}{} & Weak& 97.50\% & 98.10\%  \\
\multicolumn{1}{c|}{} & Strong& 97.22\% & 96.26\% \\
\hline
\multicolumn{1}{c|}{\multirow{3}{*}{\makecell[c]{CIFAR-100 \\ $q=0.1$}}}    & None&83.74\% & 90.48\%  \\
\multicolumn{1}{c|}{} & Weak& 84.07\% & 93.61\%  \\
\multicolumn{1}{c|}{} & Strong& 81.01\% & 79.16\% \\
\bottomrule
\end{tabular}
\end{center}
\end{table}

\noindent\textbf{The influence on data augmentation on $\mathcal{D}_{\mathrm{sel}}$.\quad}Data augmentations play a crucial role in weakly supervised learning. However, our selection criteria rely on historical prediction information to select examples with high confidence, and we cannot guarantee that stronger data augmentations will necessarily lead to better results and selection effects. So, in this part, we explore the impact of data augmentation on $\mathcal{D}_{\mathrm{sel}}$ and its influence on label selection and overall training. As shown in Table 5, weak augmentation is a more appropriate and effective choice. However, because of the impact of regular items, even if data augmentation is not used on $\mathcal{D}_{\mathrm{sel}}$, there is no significant performance degradation. Strong data augmentation has an adverse effect on the selection of examples, especially in CIFAR-100, which may also be related to the fact that the historical predictions stored in $\mathrm{MB}$ are produced by data that have not been augmented.

\noindent\textbf{The influence on double model.\quad}As recognized by the community, dual models tend to achieve better performance than single models. We are curious about how effective our selection criteria would be without the adaptive error correction capability of cross selection.  Figure 3 visualizes the gap in the selection ratio between dual-model and single-model training. It shows that our cross selection strategy can select samples more comprehensively, and on average, about 10\% more training examples can be selected on each dataset. However, the effectiveness of our algorithm is not solely due to the dual model. Even when using a single model with our selection criteria, the accuracy rate on the test set only decreases by 0.83\% and 2.68\% on CIFAR-10 and CIFAR-100 respectively. Furthermore, the high precision of selection is reflected in both settings. Detailed results can be found in Table 6 and Appendix.B.3.

\begin{figure}[!t]
\centering
\subfigure[CTAFR-10 comparison]{
	\label{fig:subfig:cifar10} %% label for second subfigure
	\includegraphics[width=1.5in,height=1.6in]{cifar10comparation_sratio.png}}
\subfigure[CTAFR-100 comparison]{
	\label{fig:subfig:cifar100} %% label for second subfigure
	\includegraphics[width=1.5in,height=1.6in]{cifar100comparation_sratio.png}}
\caption{Selection ratio comparison between dual model and single model on CTAFR-10 and CTAFR-100.}
\label{fig:Selection ratio comparison} %% label for entire figure
\end{figure}

\begin{table}[!htpb]
\caption{Comparison between Dual model and Single model. \\
S-acc represents selected accuracy in $\mathcal{D}_{\mathrm{sel}}$.}
\label{comparison between dual model and single model}
\begin{center}
\begin{tabular}{c|c|cc}
\toprule
Setting & Model & Accuracy & S-acc\\
\midrule
\multicolumn{1}{c|}{\multirow{2}{*}{\makecell[c]{CIFAR-10 \\ $q=0.5$}}}    &Single model& 96.51\%  & 99.72\%  \\
\multicolumn{1}{c|}{} & Dual model& 97.34\% & 99.44\%  \\
\hline
\multicolumn{1}{c|}{\multirow{2}{*}{\makecell[c]{CIFAR-100 \\ $q=0.1$}}}    & Single model&81.39\%  & 98.35\%  \\
\multicolumn{1}{c|}{} & Dual model& 84.07\%& 97.93\%  \\
\bottomrule
\end{tabular}
\end{center}
\end{table}

\section{Conclusion}
This work introduces CroSel, a novel partial-label learning method that leverages historical prediction information to select the confident pseudo labels from candidate label sets. The proposed method consists of two parts: a cross selection strategy that enables two deep models to select ``true" labels for each other, and a consistent regularization term co-mix that avoids sample waste and tiny noise caused by false selection. Empirically, extensive experiments demonstrate the superiority of CroSel, which consistently outperforms previous state-of-the-art methods on multiple benchmark datasets.

{\small
\bibliographystyle{ieee_fullname}
\bibliography{egpaper_final}
}

\appendix

\onecolumn
\section{Notation and definitions}
\begin{table}[htpb]
\caption{Notation and definitions}
\label{Notations}
\begin{center}
\begin{tabular}{c|c}
\toprule
Notations & Definitions\\
\midrule
$\boldsymbol{x}$ & A picture representing an input example of a neural network\\
\hline
$S$ & A candidate label set of an example\\
\hline
$\mathrm{MB}$ & A memory bank that stores historical prediction information\\
\hline
$\boldsymbol{q}$ & a $k$-dimensional vector representing the output of an example in a particular epoch\\
\hline
$\Theta$ & The model’s parameters\\
\hline
$\mathrm{CE}(\cdot,\cdot)$ & Cross-entropy between two distributions \\
\hline
$f(\boldsymbol{x})$ & the output of classifier $f$ on given input $x$\\
\hline
$\mathcal{D}_{\mathrm{sel}}$ & The selected dataset\\
\hline
$\mathcal{D}$ & The initial dataset\\
\hline
$\hat{y}$ & The selected label of an example\\
\hline
$\beta$ & A sign that complies with the criteria for selecting labels\\
\hline
$T$ & Temperature parameter for sharpening used in MixUp\\
\hline
$\boldsymbol{p}$ & The pseudo-label generated by co-mix before Mixup\\
\hline
$\boldsymbol{p}^{\prime}$ & The pseudo-label generated by co-mix after Mixup\\
\hline
$\boldsymbol{x}^{\prime}$ & An input example after Mixup\\
\hline
$\lambda_{\mathrm{cr}}$ & A hyperparameter set to the co-mix regularization\\
\hline
$\lambda_{\mathrm{d}}$ & A parameter weighting the contribution of the co-mix regularization to the training loss\\
\hline
$r_{\mathrm{s}}$ & The percentage of labeled data that we picked out \\
\hline
$\gamma$ & A hyperparameter that represents the threshold of confidence for picking the true label\\
\hline
$t$ & A hyperparameter representing the length of time the $\mathrm{MB}$ stores\\





\bottomrule
\end{tabular}
\end{center}
\end{table}

\section{Experimental details}
\subsection{Datasets}
\begin{itemize}
    \item \textbf{CIFAR-10:} It contains 60,000 32 X 32 RGB color pictures in a total of 10 categories. Including 50,000 for the training set and 10,000 for the test set.
    \item \textbf{CIFAR-100:} It has 100 classes, each containing 600 32 X 32 RGB color images. Each category has 500 training images and 100 test images. The 100 classes in CIFAR-100 are divided into 20 superclasses. Each image has a "fine" tag (the class to which it belongs) and a "rough" tag (the superclass to which it belongs).
    \item \textbf{SVHN:} It derived from Google Street View door numbers, each image contains a set of Arabic numbers' 0-9 '. The training set contained 73,257 numbers, the test set 26,032 numbers, and 531,131 additional numbers. Each number is a 32 X 32 color picture.
\end{itemize}

\subsection{Data augmentations}
Data augmentation is widely used in weakly supervised learning algorithms. There are two types of data augmentations used in our algorithm: "weak" and "strong". For "weak" augmentation, it is just a standard flip-and-shift augmentation strategy consisting of Randomcrop and RandomHorizontalFlip. For "strong" augmentations, We use RandAugment strategy for all, which randomly selects the type and magnitude of data augmentation with the same probability.

\newpage
\subsection{Detailed results}

\subsubsection{Detailed results for Parameter test on $\lambda_{cr}.$}

\begin{table}[!htpb]
\caption{Detailed results for Parameter test on $\lambda_{cr}.$}
\label{Ablation lambda-cr results}
\begin{center}
\begin{tabular}{l|c|ccc}
\toprule
Setting & $\lambda_{cr} $& Accuracy & Selected ratio &Selected accuracy\\
\midrule
\multicolumn{1}{c|}{\multirow{6}{*}{\makecell[c]{CIFAR-10 \\ $q=0.5$}}}    &$\lambda_{cr}=1$& 95.94\% & 91.57\% &99.51\% \\
\multicolumn{1}{c|}{} & $\lambda_{cr}=2$ & 96.80\% & 97.32\% & 99.17\% \\
\multicolumn{1}{c|}{} & $\lambda_{cr}=4$ & 97.33\% & 96.25\% & 99.44\% \\
\multicolumn{1}{c|}{} & $\lambda_{cr}=1(fixed)$ & 96.88\% & 87.88\% & 99.73\% \\
\multicolumn{1}{c|}{} & $\lambda_{cr}=2(fixed)$ & 96.95\% & 92.19\% & 99.68\% \\
\multicolumn{1}{c|}{} & $\lambda_{cr}=0.5(fixed)$ & 96.16\% & 95.21\% & 99.44\% \\
\hline
\multicolumn{1}{c|}{\multirow{6}{*}{\makecell[c]{CIFAR-100 \\ $q=0.1$}}}    &$\lambda_{cr}=1$& 84.02\% & 93.61\% &97.93\% \\
\multicolumn{1}{c|}{} & $\lambda_{cr}=2$ & 83.61\% & 94.15\% & 97.78\% \\
\multicolumn{1}{c|}{} & $\lambda_{cr}=4$ & 83.88\% & 95.63\% & 97.59\% \\
\multicolumn{1}{c|}{} & $\lambda_{cr}=1(fixed)$ & 83.91\% & 99.35\% & 96.12\% \\
\multicolumn{1}{c|}{} & $\lambda_{cr}=2(fixed)$ & 84.03\% & 99.17\% & 96.15\% \\
\multicolumn{1}{c|}{} & $\lambda_{cr}=0.5(fixed)$ & 83.48\% & 97.02\% & 96.81\% \\
\bottomrule
\end{tabular}
\end{center}
\end{table}

% \subsection{Detailed results for ablation study on select criteria}
\subsubsection{Detailed results for the influence of parameter $t$.}\

\begin{table}[!htpb]
\caption{Detailed results for Parameter test on $t$}
\label{Detailed results for Parameter test on $t$}
\begin{center}
\begin{tabular}{l|c|ccc}
\toprule
Setting & $t$ & Accuracy & Selected ratio &Selected accuracy\\
\midrule
\multicolumn{1}{c|}{\multirow{3}{*}{\makecell[c]{CIFAR-10 \\ $q=0.3$}}}    &$t=2$& 97.03\% & 99.07\% &99.62\% \\
\multicolumn{1}{c|}{} & $t=3$ & 97.50\% & 98.10\% & 99.55\% \\
\multicolumn{1}{c|}{} & $t=4$ & 96.15\% & 89.23\%& 99.76\% \\
\hline
\multicolumn{1}{c|}{\multirow{3}{*}{\makecell[c]{CIFAR-100 \\ $q=0.1$}}}    & $t=2$ &82.74\% & 87.32\% &98.50\% \\
\multicolumn{1}{c|}{} & $t=3$ & 84.07\% & 93.61\% & 97.93\% \\
\multicolumn{1}{c|}{} & $t=4$ & 83.56\% & 93.24\% &98.23\% \\
\bottomrule
\end{tabular}
\end{center}
\end{table}

\subsubsection{Detailed results for the influence of parameter $\gamma$.}\

\begin{table}[!htpb]
\caption{Detailed results for Parameter test on 
$\gamma$}
\label{Detailed results for Parameter test on gamma}
\begin{center}
\begin{tabular}{l|c|ccc}
\toprule
Setting & $\gamma$ & Accuracy & Selected ratio &Selected accuracy\\
\midrule
\multicolumn{1}{c|}{\multirow{3}{*}{\makecell[c]{CIFAR-10 \\ $q=0.3$}}}    &$\gamma=0.8$& 96.24\% & 95.29\% &99.08\% \\
\multicolumn{1}{c|}{} & $\gamma=0.9$ & 97.50\% & 98.10\% & 99.55\% \\
\multicolumn{1}{c|}{} & $\gamma=0.95$ & 97.38\% & 93.10\%& 99.87\% \\
\hline
\multicolumn{1}{c|}{\multirow{3}{*}{\makecell[c]{CIFAR-100 \\ $q=0.1$}}}    & $\gamma=0.8$ &80.20\% & 82.11\% &98.63\% \\
\multicolumn{1}{c|}{} & $\gamma=0.9$ & 84.07\% & 93.61\% & 97.93\% \\
\multicolumn{1}{c|}{} & $\gamma=0.95$ & 81.23\% & 67.32\% & 99.26\% \\
\bottomrule
\end{tabular}
\end{center}
\end{table}

\newpage
\subsubsection{Detailed results for the data augmentation for $\mathcal{D}_{\mathrm{sel}}$}

\begin{table}[!htpb]
\caption{Detailed results for Parameter test on $\mathcal{D}_{\mathrm{sel}}.$}
\label{Detailed results for Parameter test on $D_{sel}$}
\begin{center}
\begin{tabular}{l|c|ccc}
\toprule
Setting & Data augmentation type & Accuracy & Selected ratio &Selected accuracy\\
\midrule
\multicolumn{1}{c|}{\multirow{3}{*}{\makecell[c]{CIFAR-10 \\ $q=0.3$}}}    &No augmentation& 96.91\% & 97.44\% &99.60\% \\
\multicolumn{1}{c|}{} & Weak augmentation & 97.50\% & 98.10\% & 99.55\% \\
\multicolumn{1}{c|}{} & Strong augmentation & 97.22\% & 96.26\%& 99.75\% \\
\hline
\multicolumn{1}{c|}{\multirow{3}{*}{\makecell[c]{CIFAR-100 \\ $q=0.1$}}}    & No augmentation &83.74\% & 90.48\% &98.32\% \\
\multicolumn{1}{c|}{} & Weak augmentation & 84.07\% & 93.61\% & 97.93\% \\
\multicolumn{1}{c|}{} & Strong augmentation & 81.01\% & 79.16\% & 98.98\% \\
\bottomrule
\end{tabular}
\end{center}
\end{table}

\subsubsection{Detailed results for comparison between Dual model and Single model}
\begin{table}[!htpb]
\caption{Detailed results of comparison between Dual model and Single model.}
\label{Detailed results of comparison between dual model and single model}
\begin{center}
\begin{tabular}{c|c|ccc}
\toprule
Setting & Model & Accuracy & Selected ratio & Selected accuracy\\
\midrule
\multicolumn{1}{c|}{\multirow{2}{*}{\makecell[c]{CIFAR-10 \\ $q=0.5$}}}    &Single model& 96.51\% &87.61\% & 99.72\%  \\
\multicolumn{1}{c|}{} & Dual model& 97.34\% &96.25\%& 99.44\%  \\
\hline
\multicolumn{1}{c|}{\multirow{2}{*}{\makecell[c]{CIFAR-100 \\ $q=0.1$}}}    & Single model&81.39\% &85.39\% & 98.35\%  \\
\multicolumn{1}{c|}{} & Dual model& 84.07\%& 93.61\%& 97.93\%  \\
\bottomrule
\end{tabular}
\end{center}
\end{table}

\end{document}

{\small
\bibliographystyle{ieee_fullname}
\bibliography{egpaper_final}
}

\newpage
\appendix
{\LARGE \textbf{Appendix}}
\section{Pseudo Codes of the Proposed Method}
To better present our approach and demonstrate the workflow, we give the pseudo codes of \FedETF. The pseudo code of the federated learning (FL) training is shown in Algorithm \ref{algorithm1}, and the pseudo code of the personalized local finetuning is shown in Algorithm \ref{algorithm2}.

\begin{algorithm}[th]\label{algorithm1}
\caption{ \FedETF FL Training }
\KwIn{Clients $\{1,\dots,K\}$, communication round $T$, local epoch $E$, initial model $\bw^1=\{\bu, \bp, \beta\}$, feature dimension $d$, balanced loss hyperparameter $\gamma$.}
\KwOut{Gloabl model $\bw^g$.}
Synthesize a simplex ETF $\mathbf{V}_{ETF} \in \mathbb{R}^{d\times C}$ by Eq. (3) as the fixed classifier for all clients;\\
\For{$t=1,\dots, T$}{
\For{{\rm client} $k =1,\dots, K$ \textbf{in parallel}}{
$\bw_k^{t} \leftarrow \bw^{t}$; \\
\For{{\rm local epoch} $e=1,\dots, E$}{
Obtain $\mathcal{L}_k^g$ by Eq. (6,~7,~8);
$\bw_k^{t} \leftarrow \bw_k^{t} -\eta \nabla \mathcal{L}_k^g(\bw_k^{t})$;
}
}
The server updates $\bw^{t+1}$ by Eq. (2).
}
The final global model $\bw^g$ = $\bw^T$.
\end{algorithm}

\begin{algorithm}[th]
\caption{ \FedETF Personalized Finetuning }\label{algorithm2}
\KwIn{Clients $\{1,\dots,K\}$, iteration round $T_p$, epoch for each stage $E$, final global model $\bw^g=\{\bu, \bp, \beta, \mathbf{V}_{ETF}\}$.}
\KwOut{Personalized local models $\{\bw_k^p\}_{k=1}^{K}$.}
Assign the final global model $\bw^g$ as clients' initial local models.\\
\textit{Finetune the feature extractor.} \\
\For{{\rm client} $k =1,\dots, K$ \textbf{in parallel}}{
$\hat{\bw}_k=\{\bu, \beta\},~ \overline{\bw}_k=\{\bp, \mathbf{V}_{ETF}\}$; \\
\For{{\rm local epoch} $e=1,\dots, E$}{
Obtain $\mathcal{L}_k^p$ by Eq. (9,~10,~11);
$\hat{\bw}_k \leftarrow \hat{\bw}_k -\eta \nabla \mathcal{L}_k^p(\hat{\bw}_k)$;
}}
\For{$t=1,\dots, T_p$}{
\For{{\rm client} $k =1,\dots, K$ \textbf{in parallel}}{
\textit{Finetune the ETF classifier.} \\
$\hat{\bw}_k=\{\mathbf{V}_{ETF}, \beta\},~ \overline{\bw}_k=\{\bp, \bu\}$; \\
\For{{\rm local epoch} $e=1,\dots, E$}{
Obtain $\mathcal{L}_k^p$ by Eq. (9,~10,~11);
$\hat{\bw}_k \leftarrow \hat{\bw}_k -\eta \nabla \mathcal{L}_k^p(\hat{\bw}_k)$;
}
\textit{Finetune the projection layer.} \\
$\hat{\bw}_k=\{\bp, \beta\},~ \overline{\bw}_k=\{\mathbf{V}_{ETF}, \bu\}$; \\
\For{{\rm local epoch} $e=1,\dots, E$}{
Obtain $\mathcal{L}_k^p$ by Eq. (9,~10,~11);
$\hat{\bw}_k \leftarrow \hat{\bw}_k -\eta \nabla \mathcal{L}_k^p(\hat{\bw}_k)$;
}
}
}
The personalized local models are $\{\bw_k^p = \hat{\bw}_k \cup \overline{\bw}_k\}_{k=1}^{K}$.
\end{algorithm}

\begin{figure*}[t]
 % \vspace{-0.7cm}
\label{fig:data_distr}
\centering
\includegraphics[width=1.5\columnwidth]{figs/cifar10_num20_dir0.1_seed8_classdistr.pdf}
\includegraphics[width=1.5\columnwidth]{figs/cifar10_num20_dir0.05_seed8_classdistr.pdf}
 % \vspace{-0.3cm}
\caption{ \textbf{Visualization of clients' data distributions.} Random seed is 8. Left: data distributions of Non-IID $\alpha=0.1$ with 20 clients. Right: data distributions of Non-IID $\alpha=0.05$ with 20 clients.}
\end{figure*}


\begin{figure*}[t]
 % \vspace{-1.5cm}
\centering
\subfigure[]{\includegraphics[width=0.75\columnwidth]{figs/acc_curves_cifar10_dir0.1_seed8.pdf}}
\subfigure[]{\includegraphics[width=0.75\columnwidth]{figs/acc_curves_cifar100_dir0.1_seed8.pdf}}
% \subfigure[]{\includegraphics[width=0.48\columnwidth]{figs/acc_curves_cifar10_dir0.1_seed8.pdf}}
% \subfigure[]{\includegraphics[width=0.48\columnwidth]{figs/acc_curves_cifar100_dir0.1_seed8.pdf}}
\caption{\textbf{Global models' test accuracy curves of the methods.} (a) CIFAR-10 with $\alpha=0.1$. (b) CIFAR-100 with $\alpha=0.1$.
}
\label{fig:acc_curves_app}
% \vspace{-0.2cm}
\end{figure*}

\section{Implementation Details}
\noindent\textbf{Models and Data.} Our model implementations of the ResNet series and the DenseNet are referred from the codes of \cite{li2018visualizing}. The model implementation of the EfficientNet is referred from the official code in \cite{tan2019efficientnet}, and the implementation of the MobileNetv2 is referred from \cite{sandler2018mobilenetv2,howard2018inverted}. For the data, we use the Dirichlet-sampling-based data partition adopted in \cite{lin2020ensemble,chen2021bridging,dai2022tackling,luo2021no}. It considers a class-imbalanced data heterogeneity, controlled by hyperparameter $\alpha$, and smaller $\alpha$ refers to more Non-IID data of clients. When $\alpha < 1$, the data are considered to be rather Non-IID, which means that most of the training samples of one class are likely assigned to a small portion of clients \cite{chen2021bridging}. In our Dirichlet implementation, when $\alpha$ goes smaller, the number of samples in each client along with the class distribution of each client both become more heterogeneous, which is realistic in practical scenarios.
We use the same Tiny-ImageNet dataset as in \cite{dai2022tackling}.

\noindent\textbf{Local learning rate and optimizer.} For CIFAR-10 the local learning rate (LR) $\eta=0.04$, and for CIFAR-10 and Tiny-ImageNet, $\eta=0.01$. For clients, we use SGD optimizer with momentum 0.9 and weight decay $5\times10^{-4}$. Following \cite{DBLP:conf/nips/GhoshCYR20}, we adopt a learning rate decaying scheduler, which decays the local LR by 0.99 in each round.

\begin{table*}[t]
    \footnotesize
    \centering
      % \vspace{-0.5cm}
    \caption{\textbf{Results (\%) under different local epochs ($E$).} The dataset is CIFAR-10 with Non-IID $\alpha=0.1$.}
    % \vspace{-1em}
    % \resizebox{0.85\textwidth}{!}{
    \setlength\tabcolsep{10.76pt}
    \begin{tabular*}{0.99\linewidth}{l|cc|cc|cc|cc}
    % \begin{tabular}{c|cc|cc|cc|cc}
    \toprule
    % \midrule
    $E$ &\multicolumn{2}{c}{1}&\multicolumn{2}{c}{2}&\multicolumn{2}{c}{4}&\multicolumn{2}{c}{8}\\
    \midrule
    Methods/Metrics &General.&Personal.&General.&Personal.&General.&Personal.&General.&Personal.\\
    \midrule
    \textsc{FedAvg} \cite{mcmahan2017communication}&45.75{\tiny±1.97}	&74.18{\tiny±2.22}	
    &43.02{\tiny±2.16}	&77.45{\tiny±0.79}
    &36.30{\tiny±2.88}	&80.66{\tiny±1.37}	&32.58{\tiny±4.87}	&84.24{\tiny±0.92}\\
    \midrule
    \textsc{CCVR} \cite{luo2021no}&59.82{\tiny±4.35}	&\textbf{79.83{\tiny±1.66}}	&53.73{\tiny±8.48}	&80.43{\tiny±2.54}	&55.73{\tiny±4.00}	&81.83{\tiny±1.32}	&\textbf{55.00{\tiny±2.29}}	&\textbf{85.61{\tiny±1.12}}\\
    \textsc{FedRoD} \cite{chen2021bridging}&\textbf{60.34{\tiny±3.22}}	&77.10{\tiny±1.97}	
    &\textbf{56.74{\tiny±5.59}}	&76.96{\tiny±4.26}
    &\textbf{57.85{\tiny±5.22}} &81.08{\tiny±1.74}	&50.63{\tiny±9.49}	&84.75{\tiny±0.96}\\
    \textsc{FedNH} \cite{dai2022tackling}&39.14{\tiny±9.56}	&77.27{\tiny±1.53}	
    &45.13{\tiny±1.60}	&\textbf{81.84{\tiny±0.88}}
    &40.28{\tiny±2.78} &\textbf{82.90{\tiny±1.00}}	&39.18{\tiny±3.66}	&83.95{\tiny±1.23}\\
    \midrule
    \rowcolor{gray!20}\textbf{Our \textsc{FedETF}}&\textbf{62.76{\tiny±2.90}}	&\textbf{88.00{\tiny±0.65}}	&\textbf{62.34{\tiny±4.10}}	&\textbf{88.20{\tiny±0.93}}
    &\textbf{61.78{\tiny±3.21}}	&\textbf{88.46{\tiny±0.75}}	&\textbf{54.23{\tiny±10.8}}	
    &\textbf{88.40{\tiny±0.98}}\\
    \bottomrule
    \end{tabular*}
    % }
    \label{table:local_epoch}
    % \vspace{-0.2cm}
\end{table*}


\noindent\textbf{Hyperparameters.}
For \FedETF, we set the feature dimension to the number of classes, i.e. $d = C$; the initial temperature $\beta=1$; $\gamma=1$.
We set $\mu_{FedProx} = 0.001$ in \FedProx and $\alpha_{FedDyn} = 0.01$ in \FedDyn as suggested in their official implementations or papers. For \Ditto, the learning setting of the personalized model is the same as the one of the global model. For \FedRep, the epoch number of the classifier training and the epoch number of the feature extractor training are the same and are set as $E$. For \FedRoD, we set $\gamma=1$. For \CCVR, the number of virtual features is 10 per class, and the number of classifier calibration training epochs is 100. For \FedNH, the smoothing hyperparameter $\rho=0.9$ as suggested in the paper \cite{dai2022tackling}.

\noindent\textbf{Randomness.} We set the same random seeds for all methods in the same setting. The random seed list is $\{7, 8, 9, 10\}$. For the extremely Non-IID settings when $\alpha=0.05$, we use the random seeds that can ensure all clients can be assigned a proportion of training data (on the contrary, some random seeds will generate a data partition where particular clients have zero data samples).

\noindent\textbf{Environments.} All experiments are conducted in PyTorch with Quadro RTX 8000 GPUs.


\section{More Results and Illustrations}
\subsection{Visualization}
\noindent\textbf{Visualization of clients' data distributions.} Here, we additionally visualize the clients' data distributions mainly adopted in the main paper. In the main paper, we adopt $\alpha \in \{0.1, 0.05\}$ with 20 clients in Tables 1, 3, and 4. We visualize the data distributions in Figure \ref{fig:data_distr}. It shows that in both settings, the clients have extremely heterogeneous data distributions. Especially when $\alpha=0.05$, all clients have some classes missing, and some clients have extremely rare data (e.g. client 1). We note that these settings are very realistic in practical FL scenarios.

\noindent\textbf{More visualization of learning curves.} We additionally visualize the test accuracy curves of the methods when $\alpha=0.1$ in Figure \ref{fig:acc_curves_app}. It also demonstrates that our algorithm has superior convergence over the compared methods.

\subsection{More Experimental Results}
We add experimental results under different local epochs $E$ in Table \ref{table:local_epoch}. Generally, when the number of local epochs varies, our \FedETF also achieves constantly state-of-the-art performances.
For \FedAvg and \FedRoD, when $E$ is larger, the generalization performance will decline. However, for our \FedETF, the declines are weaker, showing its robustness and effectiveness. For personalization, \FedETF has more steady and promising results under different $E$.

\end{document}
