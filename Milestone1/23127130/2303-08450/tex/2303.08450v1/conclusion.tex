\section{Conclusion}
In conclusion, this paper presents a significant contribution to the field of repetitive action counting by introducing the novel approach, Pose Saliency Representation, which efficiently represents each action using only two salient poses. The proposed pose-level method, PoseRAC, based on this representation, achieves state-of-the-art performance on two new version datasets by utilizing Pose Saliency Annotation for training. Our lightweight model requires only 20 minutes for training on a GPU and infers nearly 10x faster compared to previous methods, making it highly efficient for practical use. Moreover, our approach significantly outperforms the previous state-of-the-art TransRAC, achieving an OBO metric of 0.56 compared to the 0.29 of TransRAC, demonstrating the effectiveness of our proposed method. The code and new version dataset are publicly available, enabling the research community to reproduce our results and conduct further experiments. Overall, our approach shows promising results and opens up new avenues for future research in the field of repetitive action counting.