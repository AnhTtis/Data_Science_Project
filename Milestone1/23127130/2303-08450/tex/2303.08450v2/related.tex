\section{Related Works}

\noindent{\bf Repetitive action counting.} Early methods \cite{cutler2000robust, pogalin2008visual, tsai1994cyclic, albu2008generic, laptev2005periodic, lu2004repetitive, panagiotakis2018unsupervised, tralie2018quasi, chetverikov2006motion} focus on compressing the motion field into one-dimensional signals to recover the repetition, where Fourier analysis\cite{pogalin2008visual, tsai1994cyclic, briassouli2007extraction}, peak detection\cite{thangali2005periodic}, classification\cite{davis2000categorical, levy2015live} can be used. However, they are limited to stationary situations, so \cite{runia2019repetition, runia2018real} collect a dataset with non-stationary repetitions. As they all analyze the visual information, \cite{zhang2021repetitive} utilizes the sound for the first time. Moreover, \cite{zhang2020context} proposes a context-aware model and constructs \emph{UCFRep} dataset with 526 videos. Similarly, \cite{dwibedi2020counting} creates \emph{Countix} which contains over 6000 videos. However, they only have coarse-grained annotation, so \cite{hu2022transrac} introduces a large-scale \emph{RepCount} dataset with fine-grained annotation of the actions. This work also encodes multi-scale temporal correlation to improve the performance and efficiency.

Different from them, we propose a simple pose-level method which outperforms all previous work in the speed and performance. Meanwhile, as current datasets lack salient annotations, we re-annotate \emph{RepCount} and \emph{UCFRep} to contribute to the future pose-level work on this task.

\noindent{\bf Human pose estimation.} Convolutional nerual network is mainstream in early works\cite{xiao2018simple, rafi2016efficient, wei2016convolutional, sun2019deep}, but when vision transformer emerged in various visual tasks\cite{li2022exploring, liu2022video}, it began to be used more\cite{yang2021transpose, li2021tokenpose}. Here we focus on two works. One is Vitpose\cite{xu2022vitpose} with a plain vision transformer, which achieves good performance. Another is BlazePose\cite{bazarevsky2020blazepose}, which is a lightweight architecture designed for mobile devices. In this paper, we directly use them as our Pose Estimation Network.