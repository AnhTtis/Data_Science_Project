% CVPR 2023 Paper Template
% based on the CVPR template provided by Ming-Ming Cheng (https://github.com/MCG-NKU/CVPR_Template)
% modified and extended by Stefan Roth (stefan.roth@NOSPAMtu-darmstadt.de)

\documentclass[10pt,twocolumn,letterpaper]{article}

%%%%%%%%% PAPER TYPE  - PLEASE UPDATE FOR FINAL VERSION
%\usepackage[review]{cvpr}      % To produce the REVIEW version
% \usepackage{cvpr}              % To produce the CAMERA-READY version
\usepackage[pagenumbers]{cvpr} % To force page numbers, e.g. for an arXiv version

% Include other packages here, before hyperref.
\usepackage[accsupp]{axessibility}  % Improves PDF readability for those with disabilities.
\usepackage{graphicx}
\usepackage{amsmath}
\usepackage{amssymb}
\usepackage{booktabs}
\usepackage{makecell}
\usepackage[export]{adjustbox}
\usepackage[normalem]{ulem}
\usepackage{textpos}  %
\usepackage[table]{xcolor}

\usepackage{tabulary,multirow,overpic,xcolor,subfloat}

% It is strongly recommended to use hyperref, especially for the review version.
% hyperref with option pagebackref eases the reviewers' job.
% Please disable hyperref *only* if you encounter grave issues, e.g. with the
% file validation for the camera-ready version.
%
% If you comment hyperref and then uncomment it, you should delete
% ReviewTempalte.aux before re-running LaTeX.
% (Or just hit 'q' on the first LaTeX run, let it finish, and you
%  should be clear).
\usepackage[pagebackref,breaklinks,colorlinks, citecolor=citecolor, linkcolor=linkcolor]{hyperref}
%\usepackage[pagebackref=false, breaklinks=true, letterpaper=true, colorlinks, citecolor=citecolor, linkcolor=linkcolor, bookmarks=false]{hyperref}
\definecolor{citecolor}{HTML}{0071BC}
\definecolor{linkcolor}{HTML}{ED1C24}

\newcommand{\bd}[1]{\textbf{#1}}
\newcommand{\app}{\raise.17ex\hbox{$\scriptstyle\sim$}}
\newcommand{\ncdot}{{\mkern 0mu\cdot\mkern 0mu}}
\def\x{\times}
\newcolumntype{x}[1]{>{\centering\arraybackslash}p{#1pt}}
\newcolumntype{y}[1]{>{\raggedright\arraybackslash}p{#1pt}}
\newcommand{\dt}[1]{\fontsize{5pt}{0.1em}\selectfont (#1)}
\newlength\savewidth\newcommand\shline{\noalign{\global\savewidth\arrayrulewidth
		\global\arrayrulewidth 1pt}\hline\noalign{\global\arrayrulewidth\savewidth}}
\newcommand{\tablestyle}[2]{\setlength{\tabcolsep}{#1}\renewcommand{\arraystretch}{#2}\centering\footnotesize}
\makeatletter\renewcommand\paragraph{\@startsection{paragraph}{4}{\z@}
	{.5em \@plus1ex \@minus.2ex}{-.5em}{\normalfont\normalsize\bfseries}}\makeatother

\newcommand\blfootnote[1]{\begingroup\renewcommand\thefootnote{}\footnote{#1}\addtocounter{footnote}{-1}\endgroup}

\DeclareMathAlphabet\mathbfcal{OMS}{cmsy}{b}{n}

\newcommand{\std}[1]{\fontsize{5pt}{0.1em}\selectfont #1}
\definecolor{Gray}{gray}{0.5}
\newcommand{\demph}[1]{\textcolor{Gray}{#1}}

\newcommand{\modelname}{FastInst\xspace}

\newcommand{\sota}[0]{state-of-the-art\xspace}

\definecolor{gain}{HTML}{34a853}
\newcommand{\gain}[1]{\textcolor{gain}{#1}}
\definecolor{lost}{HTML}{ea4335}
\newcommand{\lost}[1]{\textcolor{lost}{#1}}

\newcommand{\rg}[1]{{\color{red} R: #1}}
\newcommand{\im}[1]{{\color{magenta} I: #1}}
\newcommand{\rgedit}[1]{{\color{pink}#1}}
\newcommand{\todo}[1]{{\color{red} TODO: #1}}

\newcommand{\figref}[1]{Figure~\ref{#1}}
\newcommand{\secref}[1]{Section~\ref{#1}}
\newcommand{\tabref}[1]{Table~\ref{#1}}
\newcommand{\appref}[1]{Appendix~\ref{#1}}

\newcommand{\apb}{AP${^\text{bbox}}$}
\newcommand{\apm}{AP}
\newcommand{\aps}{AP$_\text{S}$ & AP$_\text{M}$ & AP$_\text{L}$}
\newcommand{\apl}{AP$_\text{50}$ & AP$_\text{75}$ & AP$_\text{S}$ & AP$_\text{M}$ & AP$_\text{L}$}

\newcommand{\gr}{\rowcolor[gray]{.95}}
 
% Support for easy cross-referencing
\usepackage[capitalize]{cleveref}
\crefname{section}{Sec.}{Secs.}
\Crefname{section}{Section}{Sections}
\Crefname{table}{Table}{Tables}
\crefname{table}{Tab.}{Tabs.}

%%%%%%%%% PAPER ID  - PLEASE UPDATE
\def\cvprPaperID{3555} % *** Enter the CVPR Paper ID here
\def\confName{CVPR}
\def\confYear{2023}


\begin{document}

%%%%%%%%% TITLE - PLEASE UPDATE
\title{\modelname: A Simple Query-Based Model for Real-Time Instance Segmentation}
\author{Junjie He, Pengyu Li, Yifeng Geng, Xuansong Xie  \\
DAMO Academy, Alibaba Group\\
{\tt\small hejunjie.hjj@alibaba-inc.com, lipengyu007@gmail.com, cangyu.gyf@alibaba-inc.com} \\
{\tt\small xingtong.xxs@taobao.com}
% For a paper whose authors are all at the same institution,
% omit the following lines up until the closing ``}''.
% Additional authors and addresses can be added with ``\and'',
% just like the second author.
% To save space, use either the email address or home page, not both
% \and
% Second Author\\
% Institution2\\
% First line of institution2 address\\
% {\tt\small secondauthor@i2.org}
}
\maketitle

\begin{abstract}
Contrastive Language-Image Pre-training, benefiting from large-scale unlabeled text-image pairs, has demonstrated great performance in open-world vision understanding tasks. 
However, due to the limited Text-3D data pairs, adapting the success of 2D Vision-Language Models (VLM) to the 3D space remains an open problem. 
Existing works that leverage VLM for 3D understanding generally resort to constructing intermediate 2D representations for the 3D data, but at the cost of losing 3D geometry information. To take a step toward open-world 3D vision understanding, we propose \textbf{C}ontrastive \textbf{L}anguage-\textbf{I}mage-\textbf{P}oint Cloud \textbf{P}retraining (CLIP$^2$) to directly learn the transferable 3D point cloud representation in realistic scenarios with a novel proxy alignment mechanism. 
Specifically, we exploit naturally-existed correspondences in 2D and 3D scenarios, and build well-aligned and instance-based text-image-point proxies from those complex scenarios.
On top of that, we propose a cross-modal contrastive objective to learn semantic and instance-level aligned point cloud representation.
% \xd{too vague description, any more specific name?} objective for 3D point cloud representation learning. 
Experimental results on both indoor and outdoor scenarios show that our learned 3D representation has great transfer ability in downstream tasks, including zero-shot and few-shot 3D recognition, which boosts the state-of-the-art methods by large margins.   %\hjh{give some numbers}. \xh{comparing to XXX}
Furthermore, we provide analyses of the capability of different representations in real scenarios and present the optional ensemble scheme. %\jg{cannot understand the last sentence}. 



\end{abstract}

%We build the proxy alignment based on the observations that the data collections of real-scenarios are usually conducted in a cross-modal way, which naturally align 2D and 3D data by sensor calibration. 
%Specifically, we set a generic caption list as text proxies, then align the relative image proxies by 2D Vision-Language Model and obtain the corresponding 3D proxies, which eventually form the alignment between text-image-3D proxies.
%\jg{We exploit naturally-existed correspondences in paired 2D and 3D scenarios, and build well-aligned and instance-based text-image-point proxies from those complex scenarios.} 


\section{Introduction}

The increasing complexity of source code poses a key challenge to the reliability of large-scale software systems. Software bugs in these systems can lead to safety issues~\cite{bug_safety} for users around the world as well as cause non-negligible financial losses~\cite{bug_loss}. As such, developers have to spend a large amount of time and effort on bug fixing. Consequently, \aprfull (\apr), designed to automatically generate patches to fix software bugs, has attracted wide attention from both academia and industry~\cite{long2016prophet, legoues2012genprog, long2015spr, lou2020can, tufano2018empstudy}. 


To achieve \apr, one popular approach is known as Generate-and-Validate (G\&V)~\cite{qi2015gv, ghanbari2019prapr, lou2020can, le2016hdrepair, legoues2012genprog, wen2018capgen, hua2018sketchfix, martinez2016astor, koyuncu2020fixminder, liu2019tbar, liu2019avatar}, which is typically based on the following pipeline: First, fault localization techniques~\cite{wong2016fl, abreu2007ochiai, zhang2013injecting, papadakis2015metallaxis, li2019deepfl, li2017transforming} are applied to determine the suspicious locations in programs where bugs are likely to exist. Then, the buggy locations are used by the \apr tools to generate a list of patches that replace buggy lines with correct lines. Afterward, each patch is validated against the original test suite to identify any \emph{plausible patches} (i.e., passing all tests in the test suite). Finally, to determine the \emph{correct patches}, developers examine the list of plausible patches to see if any of them can correctly fix the bug. 

Traditional \apr tools can mainly be categorized into heuristic-based~\cite{legoues2012genprog, le2016hdrepair, wen2018capgen}, constraint-based~\cite{mechtaev2016angelix, le2017s3, demacro2014nopol, long2015spr} and \template~\cite{ghanbari2019prapr, hua2018sketchfix, martinez2016astor, liu2019tbar, liu2019avatar}. Among these traditional tools, \template \apr tools~\cite{ghanbari2019prapr, liu2019tbar, benton2020effectiveness} have been able to achieve state-of-the-art results. \Template \apr tools typically leverage pre-defined templates (e.g., adding a nullness check) for bug fixing. However, since these fix templates are typically handcrafted, the number and types of bugs they are able to fix can be limited. 



To address the limitations of traditional \apr, researchers have proposed various \learning \apr tools~\cite{li2020dlfix, chen2018sequencer, jiang2021cure, lutellier2020coconut, zhu2021recoder, ye2022rewardrepair} based on the \nmtfull (\nmt) architecture~\cite{sutskever2014mt} where the input is the buggy code snippets and the goal is to translate the buggy code snippets into a fixed version. To accomplish this, \learning \apr tools require supervised training datasets with pairs of both buggy and fixed code snippets in order to learn how to perform this translation step. These training data are usually obtained by mining historical bug fixes using heuristics/keywords~\cite{dallmeier2007benchmark}, which can be imprecise for identifying bug-fixing commits; even the actual bug-fixing commits can include irrelevant code changes, leading to further pollution in the dataset~\cite{xia2022alpharepair}.
% 
Moreover, it can be hard for such \apr tools to generalize and fix bug types unseen during training. 



To better leverage recent advances in \plmfull{s} (\plm{s}), researchers~\cite{xia2022alpharepair, xia2023repairstudy, kolak2022patch, prenner2021codexws} have directly applied \plm{s} to generate patches without bug-fixing datasets. These \llm-based \apr tools work by either directly generating a complete code function~\cite{prenner2021codexws, xia2023repairstudy} or predict/infill the correct code snippet given its surrounding context~\cite{xia2022alpharepair, xia2023repairstudy}. By directly using \llm{s} that are pre-trained on billions of open-source code snippets, \llm-based \apr tools can achieve state-of-the-art performance on many repair datasets~\cite{xia2022alpharepair}. 


% 
%
%

Traditional \apr tools have long used the insight of the \emph{plastic surgery hypothesis}~\cite{barr2014plastic} where it states that the code ingredients to fix a bug already exist within the same project. Traditional \apr tools have manually designed pattern-~\cite{ghanbari2019prapr, saha2017elixir} or heuristic-based~\cite{jiang2018simfix, legoues2012genprog} approaches to finding and using such relevant code ingredients to generate fixes for bugs. However, the plastic surgery hypothesis has been largely ignored in \llm-based \apr. In fact, \llm provides a unique opportunity to fully automate the plastic surgery hypothesis idea via fine-tuning (learning project-specific information via model updates from the buggy project) and prompting (directly providing relevant code ingredients to the model), and make it directly applicable to different languages (since the \llm{s} are typically multi-lingual).%
Moreover, despite the intensive manual efforts involved, traditional \apr tools still cannot fully leverage project-specific information due to large search space for leveraging/composing existing code ingredients. In contrast, the project-specific information can effectively leveraged by \llm{s} due to their power in code understanding/vectorization, e.g., even partial/imprecise information may still guide \llm{s} in correct patch generation!
 To this end, we ask the question: \emph{How useful is the plastic surgery hypothesis in the era of \plm{s}}?








\mypara{Our Work.} To answer the question, we present \ourtech{\xspace} -- a \llm-based approach that automatically utilizes the plastic surgery hypothesis by systematically combining multiple fine-tuning and prompting strategies for \apr. \ourtech fine-tunes \plm{s} using two novel domain-specific training strategies: \textbf{\epfinetune} -- we fine-tune using the original buggy project by aggressively masking out a high percentage of tokens, which allows \plm to learn project-specific code tokens and programming styles; and \textbf{\rofinetune} -- which only masks out a single continuous code sequence per training sample, allowing the model to get used to the final \csapr task of predicting a single continuous code sequence. Furthermore, we directly leverage the ability for \plm{s} to understand natural language instructions and introduce a novel prompting strategy, \textbf{\idprompting}, which uses information retrieval and static analysis to obtain a list of relevant identifiers for the buggy lines. While such relevant identifiers are critical for fixing some difficult bugs, they may not be seen by the \llm during inference due to limited context window size. Through the use of prompting, we directly tell the model to use these extracted identifiers (relevant code ingredients) to generate the correct code. Finally, to perform repair, we combine all four model variants (including the base model, both fine-tuned models and the base model with prompting) for the final repair.





While our insight of leveraging the plastic surgery hypothesis for \llm-based \apr is generalizable across different types of \plm{s}, to implement \ourtech, we choose a recent \plm{\xspace}, \ctfive~\cite{wang2021codet5}, which is pre-trained on millions of open-source code snippets. \ctfive is an encoder-decoder model trained using \mspfull (\msp) objective where a percentage of tokens are masked out and each continuous masked token sequence is referred to as a masked span. Also, although we only extract relevant identifiers from the current buggy project (since this paper focuses on the plastic surgery hypothesis), our work can be easily extended to obtain other code information (such as relevant statements or functions) from other sources, such as  the massive pre-training corpora~\cite{husain2020codesearchnet} or historical bug-fixing datasets~\cite{jiang2019infer}, which can provide more coding knowledge for \llm{s}. Besides, although we mainly focus on using traditional string comparison algorithms for information retrieval in this paper, these techniques can be easily replaced by other frequency-based retrieval~\cite{robertson2009probabilistic} and neural search (or embedding-based search)~\cite{reimers2019sentence}.
  In summary, this paper makes the following contributions:


%


\begin{itemize}[noitemsep, leftmargin=*, topsep=0pt]
    \item \textbf{Dimension.} This paper is the first to revisit the important plastic surgery hypothesis in the era of \llm{s}. It opens up a new dimension for \llm-based \apr to incorporate previously neglected information from the buggy project itself to boost \apr performance. Furthermore, it demonstrates the promising future of retrieval-based prompting for modern \llm-based \apr.
    \item \textbf{Implementation.} We implement \ourtech based on the recent \ctfive model. We augment the model using two novel fine-tuning strategies: \epfinetune and \rofinetune, along with a novel prompting strategy based on information retrieval and static analysis: \idprompting. We combine the patches generated by all four models together and perform patch ranking to speed up \apr.% 
    \item \textbf{Evaluation Study.} We conduct an extensive evaluation against state-of-the-art \apr tools. On the widely studied \dfj 1.2 and 2.0 datasets~\cite{just2014dfj}, \ourtech is able to achieve the new state-of-the-art results of 89 and 44 correct bug fixes (15 and 8 more than best baseline) respectively.  Furthermore, we perform a broad ablation study to justify our design. \ourtech demonstrates for the first time that the plastic surgery hypothesis can substantially boost \llm-based \apr and advance state-of-the-art \apr, while being fully automated and general. Moreover, even partial/imprecise code ingredients may still effectively guide \llm{s} for \apr!
\end{itemize}


\section{Related work}

\paragraph{Cloud \vtpm{}s}
Cloud providers offering \cvm{}s typically provide virtual TPM device that
would serve as a root-of-trust and could also be used for remote
attestation.
%
%
Google cloud only offers plain \sev{} \cvms{} and offers measured boot
attestation via a \vtpm{} managed by the
hypervisor~\cite{vtpm:gcp-shielded-vms}.
%
%
Microsoft Azure cloud relies on azure attestation service for attesting
\cvms{}~\cite{vtpm:azure} that generates a token to decrypt the \vtpm{}
state and the disk, hinting that 
%
Microsoft may have their custom firmware based on \svsm{}
specification~(i.e., inside \vmpl{0}) with a persistent \vtpm{} for attesting
\snp{} VMs.
%
Alibaba cloud offers \vtpm{} support on their elastic compute service
VMs~\cite{vtpm:alibaba}.
%
Amazon AWS provides Nitro TPM, a virtual TPM implementation conforming to
the TPM 2.0 specification as part of their EC2
offering~\cite{vtpm:aws-nitro}.
%
Some of these providers use a qemu-backed \vtpm{} that runs on the host,
which requires trusting the cloud provider.
%
Also, there is very limited public knowledge on how these cloud \vtpm{}s
are designed and the security guarantees of it.
%
In contrast, we plan to publish the source code of \svtpm{} implementation
that is built on top of other standard opensource components~(i.e., Qemu,
Linux, and Keylime).
%
As our \svtpm{} rely only on the hardware-protected isolation environment
offered by the AMD-SP hardware, by bringing their own \svsm{} firmware, a
user can completely eliminate the need for trusting the cloud provider.
%
\paragraph{TEE-based \vtpm{}s}
\cocotpm{} proposes a unified architecture for attestation of
\cvms{} where the hypervisor launches a \cvm{} that acts as a \vtpm{}
manager and handles all the \vtpm{} instances~\cite{cocotpm}.
They require TLS for securing the communication channel between a \cvm{}
and its \vtpm{}.
%
Though the \vtpm{} is running under a TEE, a central \vtpm{} manager
suffers from several attacks ranging from denial of service to colluding
with other \cvm{}s,
%
on the other hand launching a dedicated \cocotpm{} for every \cvm{} results
in wastage of architectural resources as the number of
address space identifiers~(ASIDs) are limited.
%

Several projects rely on running \vtpm{} under isolation provided by other
hardware TEE mechanisms such as Intel SGX~\cite{svtpm, eTPM,
vtpm-for-cloud} and ARM Trustzone~\cite{fTPM}.
%
SvTPM aims to protect against NVRAM replacement, and rollback
attacks~\cite{svtpm} by running the \vtpm{} inside an SGX enclave for
KVM-based VMs, whereas
%
eTPM manages several enclave \vtpm{}s in a Xen environment and relies on a
physical TPM to provide root-of-trust~\cite{eTPM}, similar to Berger et
al.~\cite{vtpm:berger}.
%
In contrast, our \svtpm{} architecture equips each \cvm{} with their own
private \vtpm{} instance by leveraging the \svsm{} architecture that
implements VM privilege levels.
%
Also, by implementing an ephemeral \vtpm{}, we completely eliminate the
classes of attacks that come with state protection.

\paragraph{Trusted execution environments}
Arm introduced confidential compute architecture~(CCA) with their Armv9-A
architecture where the processor provides an isolated hardware execution
environment called \emph{Realms}, for hosting entire VMs in a secure
space~\cite{wp:arm-cca}.
%
Similar to other TEEs~\cite{wp:amd-sev, spec:intel-tdx} they offer
pre-attestation of realms and can do measured boot with their hardware
enforced security~(HES) module specification~\cite{spec:arm-cca-sec-model}
which serves as the root-of-trust~\cite{arm-cca:rss, arm-cca:rss-talk}.

Intel, with their trust domain extensions~(TDX) introduced their own
version of hardware-isolated encrypted virtual machines called trusted
domains~(TDs).
%
Intel TDX relies on an SGX-based quoting enclave called the TD-quoting
enclave to perform remote attestation of trusted domains~\cite{spec:intel-tdx}.
%
However, SGX suffered from numerous vulnerabilities in the
past~\cite{sgx-attacks:survey} where researchers were able to extract the SGX
quoting enclave's attestation keys through micro-architectural side-channel
attacks to forge attestation reports~\cite{sgx-attack:foreshadow}.
%
The attestation keys used by these quoting enclave are long-lived, and when
leaked, affect millions of devices.
%
In our design, we do not have any secrets to guard as the attestation keys
are ephemeral.


\subsection{Approach}
This paper describes three concepts with the aim 
of a robust, energy-efficient robot control. 
While these concepts are rather straightforward and intuitive, they 
are not yet utilised in mainstream manipulator control.
It is not argued that all of these principles 
need to be used, but if the (sub)task allows it, using any 
of these principles can have a positive impact on the energy-efficiency.

\vspace{-4mm}
\subsubsection{Contextual prior knowledge:}
When humans perform a transportation task, they do  
not perform strict PTP motions such as traditional industrial 
robots. Instead, movements with a certain tolerance on the position 
are performed. 
This allows the natural dynamics of the system to be exploited, 
as will be explained in the following subsection. Typically, the 
tolerances come from 
knowledge about both the environment and the task context. For example, 
the spatial constraints, 
fragility of the payload, 
if a certain part of the task requires a higher precision, 
etc. 
It is clear that this knowledge precedes the task execution and 
determines how the human will perform the task.
The execution is generally done in multiple states, e.g., picking up 
the payload, moving and placing near the target position, making small 
adjustments when necessary.

This knowledge is used to split up the task in multiple 
subtasks and identify the different requirements. Robust 
controllers and monitors are then developed to perform and coordinate 
between these subtasks. 
Examples of such requirements are crane like operations such as:
 lifting the load to a certain 
height, transporting it without colliding, and lowering the load 
until contact is made.

The task also does not require high control precision throughout, 
but only for the initial grasping and final placement. 
In addition, this does not need to come only from the 
control.
Geometric constraints such as the environment or a previously placed
payload can be used to achieve this accuracy by sliding against them. 
This is further explained in section \ref{sec:discrete_control}.

By using this knowledge, lower-cost (and often also lower-weight) 
hardware can be used, so that a more robust, 
energy efficient execution can be developed. 
Thus, for a repetitive task, the cost of 
designing and implementing a task-specific controller is not 
necessarily higher than a generic, less energy-efficient controller.\\

\vspace{-8mm}
\subsubsection{Exploiting natural dynamics}
In this work, the natural dynamics of the system are used to inject 
as little energy as possible, resulting in energy-efficient 
motions.
However, precise control of the timing is lost when the system freely
follows its natural dynamics.

Due to the layout of the used cable robot (Fig.1), when the end effector is
in a fully constrained position, releasing the power of one (or more) 
of the motors, will result in a pendulum-like swing around the 
cables that are still powered, or braked. 
This swing is used in the control strategy to cover the horizontal 
distance while consuming a minimal amount of energy. 

\vspace{-4mm}
\subsubsection{Active use of brakes}
Based on the context, certain subtasks may occur where a joint 
does not need to move. Instead of producing a constant standstill
torque, it can also be opted to brake the joint.
Another case occurs when the demanded motion is in line with 
external forces such as gravity. In case of a continuous brake,
the brake force can be directly controlled to achieve a certain 
resulting force. 
With a discrete brake, a tolerance region can be determined between  
which the brake switches on-and-off to achieve a similar effect.
Section \ref{sec:continuous_control} utilises this concept to drop the
payload without driving the motor. The brakes can also be used to stop 
the natural dynamics, if necessary.
\section{Experiments}

In this section, we describe how we adapt various NLP tasks to the in-context learning setting. We describe the prompting strategies we use for the benchmark and the models, tasks and datasets included in our initial study.

\subsection{Problem Formulation}

In order to solve different tasks via in-context learning we adopt the prompt-based few-shot learning strategy as defined in \cite{brown-etal-2020-language}. We define four main components of the prompts that we use in our experiments as follows: i) a \textbf{test example} $x_{test}$ for which the predictions are to be made; ii) $K$ \textbf{few-shot exemplars} $\{(x_i, y_i)\}_{i = 1}^{K}$, that are used to provide in-context supervision to the model; iii) a \textbf{prompt template} $f_{temp}(x)$ which turns a dataset input example into a text format that can 
 be used for prompting, containing the task description; and iv) an \textbf{answer verbalizer} $f_{verb}(y)$ that maps the label $y$ to a textual representation. In our evaluation framework we often consider the template and verbalizer as a single entity, and from now on will denote the template to encapsulate both the template and verbalizer unless specified separately. Some examples of $f_{temp}$ and $f_{verb}$ are given in table \ref{tab:promptsource}.



Given these components, the final prompt $f_{prompt}(x_{test}; \{(x_i, y_i)\}_{i = 1}^{K}, f_{temp}, f_{verb})$ or $f_{prompt}(x_{test})$ for short for a test input $x_{test}$ can be defined as:
\begin{align*}
f_{prompt}(x_{test}) =\mathbin\Vert_{i = 1}^{K} \big\{f_{temp}(x_i)&\mathbin\Vert f_{verb}(y_i)\big\}\\
&\mathbin\Vert f_{temp}(x_{test})
\end{align*}

where $\mathbin\Vert$ denotes the string concatenation operator.

The prompt can then be provided as input to the LLM $P(.;\theta)$ to obtain the prediction $z_{test}$

\begin{align*}
    z_{test} = \argmax_{z \in \mathcal{Z}} P(z | f_{prompt}(x_{test}); \theta)
\end{align*}

where $\mathcal{Z}$ is the space of possible answers, which in all of our experiments is taken to be the entirety of the language as modeled by the LLM. We approximate the $\argmax$ by sampling from the probability distribution predicted by the LLM.

The predicted answer $z_{test}$ is compared with the verbalized label using $f_{metric}(z_{test}, f_{verb}(y_{test})) \in [0, 1]$ that measures the extent of similarity between the ground truth and predicted answer. For our experiments, we use the exact-match score to determine accuracy for classification tasks and use the exact-match and F1-score for QA tasks. Formally, the evaluation score $s$ for an LLM $P(.;\theta)$ on a task $\mathcal{T}$ can be defined as:

\begin{equation*}
    s = {\E_{(x_{test}, y_{test}) \in \mathcal{T}}}[f_{metric}(z_{test}, f_{verb}(y_{test}))]
\end{equation*}


\subsection{Prompting Strategies}
\label{sec:prompt_strategies}
The choice of prompt significantly influences the performance of Large Language Models. Generative models have been shown to be brittle to simple prompting variations, such as ordering of examples, number of few-shot examples and the choice of words in the prompt. There are many variations to consider for our setup: the choice of the language of the prompt examples, the language of the prompt template, and the language of the test examples. In this work we evaluate all the models using three types of prompts:

\iffalse
The choice of prompt can greatly influence the performance of generative models, and models have been shown in the past to be brittle to prompting variations such as the words used in the prompt, number of few-shot examples, ordering of examples etc CITE. Our setup in particular involves the choice of the few-shot examples $\{(x_i, y_i)\}_{i = 1}^{K}$ as well as  choice of different template $f_{temp}$ and verbalizer $f_{verb}$ functions, for defining the prompt. For our evaluation framework we consider two higher level decisions which stem from the choice of language in which the few-shot examples and the text examples are represented and the language in which the templates are written. For the former in particular we consider three setups:

\fi
\noindent
\begin{itemize}
    \item \textbf{Monolingual Prompting}: In this setup, the k\footnote{k=8, unless specified} randomly selected examples are of the same language as the test examples. Figure \ref{fig:monoprompting} illustrates an example of monolingual prompting in hindi.
    \item \textbf{Zero-Shot Cross-Lingual Prompting}: Pre-trained multilingual models are effective at zero-shot cross-lingual transfer \cite{pires-etal-2019-multilingual, wu-dredze-2019-beto}, that is on fine-tuning them for a task in one language leads to reasonable performance on unseen languages. In this section, we evaluate the model's zero-shot cross-lingual transfer ability after in-context learning. In this experiment, we use k-shot examples from a pivot language\footnote{We use english as the pivot language in this paper} which is different from the language of the test example. Figure \ref{fig:zsprompting} illustrates the setup for a hindi test query.
    \item \textbf{Translate-Test Prompting}: This setup is similar to the Zero-Shot Cross-Lingual setup in the fact that the few-shot examples are sampled from English data. However, here we modify the test example itself by translating it to English.  Translate-test has been shown to be often better than cross-lingual transfer for both fine-tuning \cite{ponti-etal-2021-modelling} and in-context learning \cite{lin-etal-2022-shot, shi-etal-2022-language} and hence we explore its effectiveness for our benchmarking exercise as well. An example for this setup for Hindi is given in Figure \ref{fig:translate_test}, we use Bing Translator to translate the test examples to English.
\end{itemize}

\iffalse
\noindent
\textbf{2. Zero-Shot Cross-Lingual Prompting}: Pre-trained multilingual models have been shown to be surprisingly effective at zero-shot cross lingual transfer \cite{pires-etal-2019-multilingual, wu-dredze-2019-beto}, where fine-tuning them for a task in one language leads to reasonable performance on unseen languages. In this setup, we try to probe to what extent LLMs can exhibit this behavior via in-context learning. Hence, the few-shot examples here are selected from a language (we call it pivot language) different from the language of the test example and for the purposes of our experiments we use English as the pivot language. Refer to Figure \ref{fig:zsprompting} for an example of this setup for Hindi.

\noindent
\textbf{3. Translate-Test Prompting}: This setup is similar to the Zero-Shot Cross-Lingual setup in the fact that the few-shot examples are sampled from English data. However, here we modify the test example itself by translating it to English.  Translate-test has been shown to be often better than cross-lingual transfer for both fine-tuning \cite{ponti-etal-2021-modelling} and in-context learning \cite{lin-etal-2022-shot, shi-etal-2022-language} and hence we explore its effectiveness for our benchmarking exercise as well. An example for this setup for Hindi is given in Figure \ref{fig:translate_test}. We use Bing Translator to translate the test examples to English in our experiments.
\fi

For the choice of language for the prompt template, we consider the following two setups:
\noindent
\begin{itemize}
    \item \textbf{English-Template}: Here the prompt templates are written in English, irrespective of the language of the few-shot and test examples. As has been shown in \citet{shi-etal-2022-language}, English instructions can often perform on par or even better than providing them in the native language. 
    \item \textbf{Native-Language-Template}: Here the prompt templates are written in the language of the test example $(x_{test}, y_{test})$. For our experiments, we use Bing Translator to translate the prompts from English to the native language.
\end{itemize}

In our initial experiments, we used different languages for task templates in few-shot examples and the test example but it performs poorly. We speculate such a poor performance of the model, as it may be getting confused about which language to generate the predictions in. We also found English templates to perform better than native language prompts. Hence we use English prompts for all our experiments.

\iffalse
\begin{table}[]
    \centering
    \begin{tabular}{p{2.5cm}p{2cm}p{2cm}}
        \toprule
         &  \textbf{English-Template} & \textbf{Native-Language-Template}\\
         \midrule
         Monolingual & \checkmark & \checkmark\\
         Zero-Shot Cross-Lingual & \checkmark & \xmark \\
         Translate-Test & \checkmark & \xmark\\
         \bottomrule
    \end{tabular}
    \caption{Combinations of the two higher level prompting setups.}
    \label{tab:prompting_setups}
\end{table}

Table \ref{tab:prompting_setups} provides the possible combinations of the two classes of prompting strategies. We only allow Native-Language templates for Monolingual setup, as in the other two setups the few-shot examples are in English. In our initial experiments, we tried using different language templates for few-shot examples and the test example but found that it performs poorly, as it would often lead to the model being confused about which language to generate the predictions in.
\fi
\subsection{Models}

We conduct all benchmarking experiments on OpenAI's GPT text-davinci-003 \cite{brown-etal-2020-language} model, which is available via API access. We do not conduct any fine-tuning of the model for our benchmarking experiments or carry out hyperparameter tuning for temperature or other settings.

We compare the performance of DV003 with the following models: BLOOMZ \cite{muennighoff2022crosslingual}, a fine-tuned version of the BLOOM \cite{scao2022bloom} model, which is a 176 parameter model created by the BigScience community trained on 46 natural languages and 13 programming languages. We also compare DV003's performance against SOTA non-autoregressive models such as TULRv6 \cite{patra2022beyond} and MuRIL \cite{khanuja2021muril} for the Indic benchmarks. The TULRv6 model is trained with a novel sampling strategy and bitexts in multiple languages and is currently at the top position on the XTREME \cite{hu2020xtreme} benchmark as of writing this paper. MuRIL is multilingual BERT model trained on 16 Indic languages and obtains SOTA performance on some Indic benchmarks. 

\subsection{Tasks and Datasets}

In our experiments, we consider two broad families of NLU tasks, i) Classification and ii) Question Answering. Below we review the experimental setups and datasets used for benchmarking for these two tasks. A list of all the datasets with the languages covered by them can be found in Table \ref{tab:datasets}.

\subsubsection{Classification}
These tasks involve classifying a single sentence or a group of sentences into a finite number of discrete labels. For each dataset, we measure the performance of different models in terms of classification accuracy. For prompt-based models in particular, since we add no constraint on the output space of the LLM we compute the exact match between the generated output and a verbalized label to determine if the example was classified correctly. We run experiments for all the prompting strategies that we discussed in the previous sections for each dataset. The details of each dataset that we use for benchmarking are given below:

 \begin{table*}[h]
 \small
     \centering
     \begin{tabular}{ccc}
     \toprule
    Dataset&Task&Languages\\
    \midrule
    XNLI&Natural Language Inference&en, fr, es, de, el, bg, ru, tr, ar, vi, th, zh, hi, sw, ur\\
    Indic-XNLI&Natural Language Inference&as, bn, gu, hi, kn, ml, mr, or, pa, ta, te\\
    %Indic-WNLI&Natural Language Inference&gu, hi, mr\\
    %GLUECoS-NLI&Natural Language Inference&hi-en\\
    %GLUECoS-En-Es-Sentiment&Sentiment Analysis&es-en\\
    PAWS-X&Paraphrase Identification&zh, fr, de, ko, ja, es\\
    XCOPA&Commonsense Reasoning&et, ht, id, it, qu, sw, ta, th, tr, vi, zh\\
    TyDiQA&Question Answering&en, ar, bn, fi, id, ja, sw, ko, ru, te, th\\
    MLQA&Question Answering&de, es, ar, zh, vi, hi\\
    XQUAD&Question Answering&en, es, de, el, ru, tr, ar, vi, th, zh, hi\\
    IndicQA&Question Answering&as, bn, gu, hi, kn, ml, mr, or, pa, ta, te\\
  \bottomrule
     \end{tabular}
     \caption{Datasets and languages}
     \label{tab:datasets}
     \vspace{-0.4cm}
 \end{table*}

\noindent{\textbf{1. Natural Language Inference}}: XNLI \cite{Conneau2018xnli} is a dataset for cross-lingual Natural Language Inference, which consists of professional translations of the MNLI \cite{wang2018glue} corpus into 14 languages. We also consider IndicXNLI \cite{aggarwal2022indicxnli} that translates the XNLI dataset into 11 Indic languages by using Machine Translation, followed by validation by native speakers.

% Indic-WNLI \cite{doddapaneni2022indicxtreme} is a translation of the Winograd NLI dataset \cite{wang2018glue}, an NLI version of the Winograd Schema Challenge into three Indic languages. 

%remove gluecos-nli if not done

% GLUECoS-NLI \cite{khanuja2020new} is a code-mixed NLI dataset in Hindi-English, consisting of Bollywood (Hindi) movie conversations as premises, with manually created hypotheses. 

% \subsubsection{Sentiment Analysis}

% %remove section if not done

% The EN-ES-CS Sentiment Analysis dataset \cite{vilares2016cs}, part of the GLUECoS benchmark \cite{khanuja2020gluecos} is a code-mixed dataset consisting of English-Spanish Tweets annotated with SentiStrength \cite{thelwall2017heart} scores. 

\noindent{\textbf{2. Paraphrase Identification}}: PAWS-X \cite{yang2019paws} is a paraphrase identification dataset professionally translated from the PAWS \cite{zhang2019paws} dataset into six typologically diverse languages. 

\noindent{\textbf{3. Commonsense Reasoning}}: XCOPA \cite{ponti2020xcopa} is a commonsense reasoning dataset, which is a translation of the COPA \cite{roemmele2011choice} dataset into 11 typologically diverse languages, including very low-resource languages such as Eastern Apurímac Quechua and Haitian Creole.

\subsubsection{Question Answering}
We focus on the Span Prediction type of Question Answering (QA) tasks in our experiments, where given a context and a question the task is to predict the answer within the context. One major challenge that we come across for multilingual evaluation of QA tasks is that for many languages we often cannot fit the context and question pairs for the few-shot and text examples in the maximum context size of 4096 for the DV003 model.

To overcome this issue we follow two steps. First, for the few-shot examples we only provide the line within the paragraph containing the answer as the context. Second, for the test example, we index the chunks of the context using the embeddings from the \texttt{text-embedding-ada-002} model. Given the question, the closest chunk in the full context is retrieved and used in the prompt for the test example. We use a maximum chunk size of 100 in our experiments and use the implementation for retrieval provided in the \textbf{LangChain}\footnote{\url{https://github.com/hwchase17/langchain}} library. By doing this,we minimize the space taken by the context tokens in our prompt.

For each task, we calculate the Exact Match and F1 score as defined in \citet{rajpurkar-etal-2016-squad}.  For our experiments we 
 consider the following three tasks:

\noindent \textbf{1. TyDiQA} \cite{clark2020tydi} is a QA dataset covering 11 typologically diverse languages. The task consists of two sub-tasks - passage selection and minimum answer span (Gold-P). For our experiments, we consider the Gold-P task and evaluate Monolingual and Zero-Shot Cross-Lingual prompting strategies. Since the labels do not directly transfer one-to-one across translation for QA tasks as they do for classification and require the use of alignment algorithms, we skip translate-test prompting for this task.

\noindent \textbf{2. MLQA} \cite{lewis2020mlqa} is an extractive QA dataset translated into 7 languages by professional translators. The task has two variants, the first where the question, context, and answer are all in the same language; and the second, where the question is in a different language than the context and answer. We consider the former variant of the task in our experiments. For MLQA, translate-test splits are also available, where each language's test data has been translated into English with answers aligned using the attention scores. There is no training data available for MLQA, and we use SQuAD's\citet{rajpurkar-etal-2016-squad} training data for selecting few-shot examples in English and validation data for MLQA in other languages to get their few-shot examples. This way, we are able to evaluate for all three prompting setups.

\noindent \textbf{3. XQuAD} \cite{artetxe2020cross} consists of professional translations of a subset of the SQuaD dataset \cite{rajpurkar2016squad} into 10 languages. XQuAD only has validation datasets available publicly, hence we evaluate the models on them. Like MLQA we use English SQuAD data for few-shot examples and since we cannot use validation data in other languages for few-shot, we only evaluate for zero-shot cross-lingual setup for this task.


\noindent \textbf{4. IndicQA} \cite{doddapaneni2022indicxtreme} is a manually curated cloze-style reading comprehension dataset that can be used for evaluating question-answering models in 11 Indic languages. The context paragraphs are chosen from Wikipedia articles whose topics are closely related to Indic culture, history,etc. The publicly available test set has about 2000 sentences that we carry out our evaluation on. 

% IndicQA \cite{doddapaneni2022indicxtreme} is a cloze-style reading comprehension dataset with context taken from Wikipedia articles on Indian culture and history, manually created in 11 Indic languages. 


% the few-shot examples are selected to belong to the same language as the test example to be evaluated. An example of the same for Hindi is shown in Figure \ref{fig:monoprompting}.

% Figure \ref{fig:zsprompting} illustrates the zero-shot prompting technique, in which the few-shot examples are in English, while the test example is in the target language. In this case, the model learns how to perform the task using few-shot examples in English and generates a response for the task in the target language.

% As shown in Figure \ref{fig:monoprompting}, in the monolingual prompting strategy, the entire prompt is in the target language, with the few-shot examples also coming from the target language. 

% The choice of prompt can greatly influence the performance of generative models, and models have been shown in the past to be brittle to prompting variations such as the words used in the prompt, number of few-shot examples, ordering of examples etc CITE. We used four prompting strategies for all our experiments - translate-test, zero-shot prompting, cross-lingual translated prompting and monolingual prompting. To illustrate the differences between the prompting methods, we use the following terminology: the \textit{instructions} part of the prompt contains the instructions on how to perform the task, along with \textit{few-shot examples}. The \textit{test example}
 % part of the prompt contains the data point for which we want the response from the model. We used the Bing Translator to perform the automatic translation in all our experiments. 

% \subsubsection{Translate-test}

% As shown in Figure \ref{fig:translate_test}, the Translate-test setting translates test example into English and uses the English instructions along with few-shot examples from English data.

\begin{figure*}[h!]
    \centering
    \includegraphics[width=18cm]{figures/translate_test.jpg}
    \caption{Translate-test}
    \label{fig:translate_test}
\end{figure*}

% \subsubsection{Zero-shot prompting}

% Figure \ref{fig:zsprompting} illustrates the zero-shot prompting technique, in which the few-shot examples are in English, while the test example is in the target language. In this case, the model learns how to perform the task using few-shot examples in English and generates a response for the task in the target language.

\begin{figure*}[h!]
    \centering
    \includegraphics[width=18cm]{figures/zero_shot_prompting.jpg}
    \caption{Zero-shot prompting}
    \label{fig:zsprompting}
\end{figure*}

% \subsubsection{Monolingual translated prompting}

 

% \subsubsection{Monolingual prompting}

% As shown in Figure \ref{fig:monoprompting}, in the monolingual prompting strategy, the entire prompt is in the target language, with the few-shot examples also coming from the target language. 

\begin{figure*}[h!]
    \centering
    \includegraphics[width=18cm]{figures/monolingual_prompting.jpg}
    \caption{Monolingual prompting}
    \label{fig:monoprompting}
\end{figure*}


\subsection{Few-shot examples}

In all our experiments, we choose few-shot examples randomly from the development set available in the dataset, unless specified. Better choices of few-shot examples for the tasks can lead to higher performance, which we leave for future work.

% how we chose the few shot examples and the different design choices that can be made here

% Issues with prompt length - cannot stuff more few shot examples in some languages. just mention here and can go into more detail in the tokenizer discussion

\subsection{Choice of prompts}

% Task-specific choice of prompts

% how we went about choosing the prompt for each task, using promptsource, optimizing for english, can add the potential drawbacks of doing so here or in discussion.

For each task that we consider in our benchmarking study, we need to come up with a prompt that specifies the instruction that the model should follow. We use PromptScource from the BigScience community to use the existing prompts or to create new prompts for the tasks \footnote{Hosted version: \url{https://huggingface.co/spaces/bigscience/promptsource}}. PromptSource is a toolkit for creating, sharing, and using natural language prompts. Prompts are saved in standalone structured files and written in a simple templating language called Jinja. 

For all datasets, we evaluate the performance of all English prompts on 10\% of the English test set. The 10\% of the test set is sampled randomly. We select the prompt that gives best performance on English. This prompt is then used for the entire test set for all prompt strategies as described in \ref{sec:prompt_strategies}. The selected English prompt is also translated to the corresponding target language using the Bing translator. Table \ref{tab:promptsource} shows the final English prompt for each dataset. 

There is a possibility that the best prompt for English is not necessarily the best prompt for other or all languages. Additionally, translation errors may propagate in the form of incorrect syntax and semantics in the prompts, which may influence task performance negatively. To avoid this, we manually inspect and edit prompts for languages that we know (mainly Indian languages and Swahili). We recommend that all translated prompt templates should be verified by a native speaker and plan to do so in future work.

% \myworries{ToDo: Discuss drawbacks of English finetuning here?}

\begin{table*}[h]
\begin{tabular}{p{3cm}p{2cm}p{5cm}p{3cm}}
\toprule
Dataset                 & Prompt Name & Template $f_{temp}$ & Verbalizer $f_{verb}$ \\ \midrule
XNLI, Indic-XNLI                    & Based on previous passage & \{premise\} Based on previous passage, is it true that \{hypothesis\} ? Yes, No, or Maybe? & Entailment -> Yes, Contradiction -> No, Neutral -> Maybe \\ \midrule
PAWS-X                  & Concatenation                                          & Sentence 1: \{sentence1\} Sentence 2: \{sentence2\} Question: Does Sentence 1 paraphrase Sentence 2? Yes or No? & Positive -> Yes, Negative -> No  \\ \midrule
XCOPA                   & Discrete version of plausible alternatives prompt &
\{ premise \} \{\% if question == "cause" \%\} This happened  because \{\% else \%\} As a consequence... \{\% endif \%\} Help  me pick the more plausible option:- choice1: \{choice1\}, choice2: \{choice2\} & \{choice1\} -> choice1 , \{choice2\} -> choice2       \\ \midrule
TyDiQA, MLQA, XQUAD, IndicQA &
  Answer given context and question &
  \{context\} Q: \{question\} Referring to the passage above, the correct answer to the given question is & Identity\\
  \bottomrule
\end{tabular}
\caption{Prompt type and prompt used for each dataset.}
\label{tab:promptsource}
\end{table*}

\section{Conclusion}
We propose \modelname for real-time instance segmentation. Built on a query-based segmentation framework~\cite{cheng2021mask2former} and three designed efficient components, \ie, instance activation-guided queries, dual-path update strategy, and ground truth mask-guided learning, \modelname achieves excellent performance on the popular COCO dataset while maintaining a fast inference speed. Extensive experiments demonstrate the effectiveness of core ideas and the superiority of \modelname over previous state-of-the-art real-time counterparts. We hope this work can serve as a new baseline for real-time instance segmentation and promote the development of query-based image segmentation algorithms. 

\noindent\textbf{Limitations.} (1) Like general query-based models~\cite{detr, cheng2021mask2former,li2021panopticsegformer}, \modelname is not good at small targets. Even though using stronger pixel decoders or larger feature maps improves it, it introduces heavier computational burdens, and the result is still unsatisfactory. We look forward to an essential solution to handle this problem. (2) although GT mask-guided learning improves the performance of masked attention, it increases training costs. We hope a more elegant method can be proposed to replace it.



%%%%%%%%% REFERENCES
{\small
\bibliographystyle{ieee_fullname}
\bibliography{egbib}
}

\clearpage
\appendix
\begin{center}{\bf \Large Appendix}\end{center}\vspace{-2mm}
\renewcommand{\thetable}{\Roman{table}}
\renewcommand{\thefigure}{\Roman{figure}}
\setcounter{table}{0}
\setcounter{figure}{0}
\section{Appendix for Proofs}

\paragraph{Proof of Theorem \ref{thm:main}.}

\begin{proof}
\label{proof:main}
Our proof has two steps. In Step 1, we will show that SimCLR is equivalent to minimizing the cross entropy loss defined in Eqn.~(\ref{eqn:cross-entropy}). 
In Step 2, we will show  that minimizing the cross-entropy loss 
is equivalent to spectral clustering on $\bfpi$. 
Combining the two steps together, we have proved our theorem. 

\textbf{Step 1: } SimCLR is equivalent to minimizing the cross entropy loss.

The cross-entropy loss takes expectation over 
$\bfW_\bfX\sim \mathbb{P}(\cdot ; \bfpi)$, 
which means $\bfW_\bfX$ has exactly one non-zero entry in each row $i$. By Lemma~\ref{lem:multinomial}, we know every row $i$ of $\bfW_\bfX$ is independent of other rows. Moreover, 
$\bfW_{\bfX,i}\sim \mathcal{M}(1, \bfpi_i/\sum_j \bfpi_{i,j})=\mathcal{M}(1, \bfpi_i)$, because $\bfpi_i$ itself is a probability distribution.
Similarly, we know $\bfW_\bfZ$ also has the row-independent property by sampling over $\mathbb{P}(\cdot;\bfK_\bfZ)$.
Therefore, by Lemma~\ref{lem:cross_split}, we know Eqn.~(\ref{eqn:cross-entropy}) is equivalent to:
\[
 -\sum_{i=1}^n \mathbb{E}_{\bfW_{\bfX,i}}[\log \mathbb{P}(\bfW_{\bfZ,i}=\bfW_{\bfX,i};\bfK_\bfZ)],
\]

This expression takes expectation over $\bfW_{\bfX,i}$ for the given row $i$. Notice that 
$\bfW_{\bfX,i}$ has exactly one non-zero entry, which equals $1$ (same for $\bfW_{\bfZ,i}$). 
As a result
we expand the above expression to be:
\begin{equation}
 -\sum_{i=1}^n \sum_{j\neq i} \Pr(\bfW_{\bfX,i,j}=1)\log \Pr(\bfW_{\bfZ,i,j}=1).
\label{eqn:detailed-expansion}    
\end{equation}


By Lemma~\ref{lem:multinomial}, $\Pr(\bfW_{\bfZ,i,j}=1)=\bfK_{\bfZ,i,j}/\|\bfK_{\bfZ,i}\|_1$ for $j\neq i$. Recall that $\bfK_\bfZ=(k(\bfZ_i-\bfZ_j))_{(i,j)\in[n]^2}$, which means 
$\bfK_{\bfZ,i,j}/\|\bfK_{\bfZ,i}\|_1=\frac{\exp(-\|\bfZ_i-\bfZ_j\|^2/{2\tau})}{\sum_{k\neq i}
\exp(-\|\bfZ_i-\bfZ_k\|^2/{2\tau})
}$ for $j\neq i$, when $k$ is the Gaussian kernel with variance $\tau$. 

Notice that $\bfZ_i=f(\bfX_i)$, so we know
\begin{equation}
-\log \Pr(\bfW_{\bfZ,i,j}=1)=
-\log \frac{\exp(-\|f(\bfX_i)-f(\bfX_j)\|^2/{2\tau})}{\sum_{k\neq i}
\exp(-\|f(\bfX_i)-f(\bfX_k)\|^2/{2\tau}),
}
\label{eqn:infonce-equivalence}    
\end{equation}


The right hand side is exactly the InfoNCE loss defined in Eqn.~(\ref{eqn:infonce}).
Inserting Eqn.~(\ref{eqn:infonce-equivalence}) into Eqn.~(\ref{eqn:detailed-expansion}), we get the SimCLR algorithm, which first samples augmentation pairs $(i,j)$ with $\Pr(\bfW_{\bfX,i,j}=1)$ for each row $i$, and then optimize the InfoNCE loss. 

\textbf{Step 2: } minimizing the cross entropy loss 
is equivalent to spectral clustering on $\bfpi$.


By Lemma~\ref{lem:convert_to_spectral}, we may further convert the loss to 
\begin{equation}
\label{eqn:main-theorem-repul-attr}
\min_{\bfZ}
-\sum_{(i,j)\in [n]^2} \mathbf{P}_{i,j}
\log k (\bfZ_i-\bfZ_j)+\log \mathbf{R}(\bfZ).
\end{equation}
Since $k$ is the Gaussian kernel, this reduces to \[
\min_\bfZ \mathrm{tr}(\bfZ^\top \mathbf{L}(\bfpi) \bfZ)
+\log \mathbf{R}(\bfZ),
\]

where we use the fact that $\mathbb{E}_{\bfW_\bfX\sim \mathbb{P}(\cdot; \bfpi)}[\mathbf{L}(\bfW_\bfX)]
=\mathbf{L}(\bfpi)
$, because the Laplacian operator is linear and $
\mathbb{E}_{\bfW_\bfX\sim \mathbb{P}(\cdot; \bfpi)}(\bfW_\bfX)=\bfpi
$.
\end{proof}

\paragraph{Proof of Theorem \ref{thm:clip}.}
\begin{proof}
Since $\bfW_\bfX\sim \mathbb{P}(\cdot;\bfpi_{\mathbf{A}, \mathbf{B}})$, we know 
$\bfW_\bfX$ has exactly one non-zero entry in each row, denoting the pair that got sampled. 
A notable difference compared to the previous proof is we now have $n_\mathcal{A}+n_\mathcal{B}$ objects in our graph. CLIP deals with this by taking a mini-batch of size $2N$, 
such that $n_\mathcal{A}=n_\mathcal{B}=N$, and adding the $2N$ InfoNCE losses together. We label the objects in $\mathcal{A}$ as $[n_\mathcal{A}]$, and the objects in $\mathcal{B}$ as $\{n_\mathcal{A}+1, \cdots, n_\mathcal{A}+n_\mathcal{B}\}$. 

Notice that $\bfpi_{\mathbf{A}, \mathbf{B}}$ is a bipartite graph, so the edges of objects in $\mathcal{A}$ will only connect to object in $\mathcal{B}$ and vice versa. We can define the similarity matrix in $\cZ$ as $\bfK_\bfZ$, 
where $\bfK_\bfZ(i, j+n_\mathcal{A})=\bfK_\bfZ(j+n_\mathcal{A},i)= k(\bfZ_i-\bfZ_j)$ for $i\in [n_\mathcal{A}], j\in [n_\mathcal{B}]$, and otherwise we set $\bfK_\bfZ(i,j)=0$. 
The rest is same as the previous proof. 
\end{proof}

\paragraph{Proof of Theorem \ref{thm:exponential}.}

\begin{proof}
\label{proof:exponential}
Since the objective function consists of a linear term combined with an entropy regularization, which is a strongly concave function, the maximization problem is a convex optimization problem. Owing to the implicit constraints provided by the entropy function, the problem is equivalent to having only the equality constraint. We then introduce the Lagrangian multiplier $\lambda$ and obtain the following relaxed problem:

$$
\widetilde{E}(\boldsymbol{\alpha})=\psi_{1}-\sum_{i=1}^n \alpha_{i} \psi_{i}+\tau \sum_{i=1}^n \alpha_{i}\log \alpha_{i}+\lambda\left(\boldsymbol{\alpha}^{\top} \mathbf{1}_n-1\right).
$$

As the relaxed problem is unconstrained, taking the derivative with respect to $\alpha_{i}$ yields

$$
\frac{\partial \widetilde{E}(\boldsymbol{\alpha})}{\partial \alpha_{i}}=-\psi_{i}+\tau\left(\log \alpha_{i}+\alpha_{i} \frac{1}{\alpha_{i}}\right)+\lambda=0.
$$

Solving the above equation implies that $\alpha_{i}$ takes the form
$
\alpha_{i}=\exp \left(\frac{1}{\tau} \psi_{i}\right) \exp \left(\frac{-\lambda}{\tau}-1\right).
$ Since $\alpha_{i}$ lies on the probability simplex, the optimal $\alpha_{i}$ is explicitly given by
$
\alpha^{*}_{i}=\frac{\exp \left(\frac{1}{\tau} \psi_{i}\right)}{\sum_{i^{\prime}=1}^n \exp \left(\frac{1}{\tau} \psi_{i^{\prime}}\right)} .
$ Substituting the optimal point into the objective function, we obtain
$$
\begin{aligned}
E\left(\boldsymbol{\alpha}^*\right)  &=\psi_1-\sum_{i=1}^n \frac{\exp \left(\frac{1}{\tau} \psi_{i}\right)}{\sum_{i^{\prime}=1}^n \exp \left(\frac{1}{\tau} \psi_{i^{\prime}}\right)} \psi_{i}+\tau \sum_{i=1}^n \frac{\exp \left(\frac{1}{\tau} \psi_{i}\right)}{\sum_{i^{\prime}=1}^n \exp \left(\frac{1}{\tau} \psi_{i^{\prime}}\right)}\log \frac{\exp \left(\frac{1}{\tau} \psi_{i}\right)}{\sum_{i^{\prime}=1}^n \exp \left(\frac{1}{\tau} \psi_{i^{\prime}}\right)} \\
& =\psi_1 - \tau \log \left(\sum_{i=1}^n \exp \left(\frac{1}{\tau} \psi_{i}\right)\right).
\end{aligned}
$$
Thus, the Lagrangian dual function is given by
\begin{equation*}
-E\left(\boldsymbol{\alpha}^*\right)= -\tau \log \frac{\exp \left(\frac{1}{\tau} \psi_{1}\right)}{\sum_{i=1}^n \exp \left(\frac{1}{\tau} \psi_{i}\right)}.\qedhere
\end{equation*}
\end{proof}



\section{More on Experiments} \label{section: experiment_details}

\paragraph{CIFAR-10 and CIFAR-100} CIFAR-10 ~\citep{krizhevsky2009learning} and CIFAR-100 ~\citep{krizhevsky2009learning} are well-known classic image classification datasets. Both CIFAR-10 and CIFAR-100 contain a total of 60k $32 \times 32$ labeled images of different classes, with 50k for training and 10k for testing. CIFAR-10 is similar to CIFAR-100, except there are 10 different classes in CIFAR-10 and 100 classes in CIFAR-100.

\paragraph{TinyImageNet} TinyImageNet ~\citep{le2015tiny} is a subset of ImageNet ~\citep{deng2009imagenet}. There are 200 different object classes in TinyImageNet, with 500 training images, 50 validation images, and 50 test images for each class. All the images in TinyImageNet are colored and labeled with a size of $64 \times 64$.

\textbf{Pseudo-code.} Algorithm \ref{alg:Training Procedure} presents the pseudo-code for our empirical training procedure.

\begin{algorithm}[!htbp]
\caption{Training Procedure}
\label{alg:Training Procedure}
\begin{algorithmic}[1]
\REQUIRE trainable encoder network $f$, batch size $N$, augmentation strategy \textit{aug}, loss function $L$ with hyperparameters \textit{args}
\FOR {sampled minibatch ${x_i}_{i=1}^N$}
\FORALL{$i \in { 1, ..., N }$}
\STATE draw two augmentations $t_i = \textit{aug}\left(x_i\right) $, $t_i' = \textit{aug}\left(x_i\right) $
\STATE $z_i = f\left(t_i\right)$, $z_i' = f\left(t_i'\right)$
\ENDFOR
\STATE compute loss $\mathcal{L} = L(N, z, z', \textit{args})$
\STATE update encoder network $f$ to minimize $\mathcal{L}$
\ENDFOR
\STATE \textbf{Return} encoder network $f$
\end{algorithmic}
\end{algorithm}

We also provide the pseudo-code for our core loss function used in the training procedure in Algorithm \ref{alg:Core loss}. The pseudo-code is almost identical to SimCLR's loss function, with the exception of an extra parameter $\gamma$.

\begin{algorithm}[!htbp]
\caption{Core loss function $\mathcal{C}$}
\label{alg:Core loss}
\begin{algorithmic}[1]
\REQUIRE batch size $N$, two encoded minibatches $z_1, z_2$, $\gamma$, temperature $\tau$
\STATE $z = \textit{concat}\left(z_1, z_2\right)$
\FOR {$i \in {1, ..., 2N }, j \in {1, ..., 2N}$ }
\STATE $s_{i,j} = \Vert z_i - z_j \Vert_2^{\gamma}$
\ENDFOR
\STATE \textbf{define} $l(i, j)$ \textbf{as} $l(i, j) = - \log \frac{exp\left(s_{i,j}/\tau \right)}{\sum_{k=1}^{2N} \mathbf{1}{[k \ne i]} exp\left(s{i, j} / \tau \right)} $
\STATE \textbf{Return} $\frac{1}{2N} \sum_{k=1}^N\left[l(i, i+N) + l(i+N, i)\right]$
\end{algorithmic}
\end{algorithm}

Utilizing the core loss function $\mathcal{C}$, we can define all kernel loss functions used in our experiments in Table \ref{table: loss definition}. For all $z_i \in z$ with even dimensions $n$, we define $z_{L_i} = z_i\left[0:n/2\right]$ and $z_{R_i} = z_i\left[n/2:n\right]$.

\begin{table}[ht]
\centering
\begin{tabular}{{@{}l|l@{}}}
Kernel  &  Loss function \\ \midrule
Laplacian & $\mathcal{C}\left(N, z, z', \gamma=1, \tau\right)$\\ \midrule
Sum       & $\lambda * \mathcal{C}\left(N, z, z', \gamma=1, \tau_1\right) + (1-\lambda) * \mathcal{C}\left(N, z, z', \gamma=2, \tau_2\right)$  \\ \midrule
Concatenation Sum&$\lambda * \mathcal{C}\left(N, z_L, z'_L, \gamma=1, \tau_1\right) + (1-\lambda) * \mathcal{C}\left(N, z_R, z'_R, \gamma=2, \tau_2\right)$\\ \midrule
$\gamma = 0.5$ & $\mathcal{C}\left(N, z, z', \gamma=0.5, \tau\right)$          \\ 

\end{tabular}

\caption{Definition of kernel loss functions in our experiments}
\label {table: loss definition}
\end{table}

\textbf{Baselines.} We reproduce the SimCLR algorithm using PyTorch Lightning~\citep{PytorchLightning}.

\textbf{Encoder details.}
The encoder $f$ consists of a backbone network and a projection network. We employ ResNet50~\citep{ResNet} as the backbone and a 2-layer MLP (connected by a batch normalization~\citep{ioffe2015batch} layer and a ReLU \cite{nair2010rectified} layer) with hidden dimensions 2048 and output dimensions 128 (or 256 in the concatenation kernel case).

\textbf{Encoder hyperparameter tuning.}
For each encoder training case, we randomly sample 500 hyperparameter groups (sample details are shown in Table \ref{table: Hyperparameter sample}) and train these samples simultaneously using Ray Tune ~\citep{RayTune}, with the ASHA scheduler~\citep{li2018massively}. Ultimately, the hyperparameter group that maximizes the online validation accuracy (integrated in PyTorch Lightning) within 5000 validation steps is chosen for the given encoder training case.

\begin{table}[ht]
\centering

\begin{tabular}{@{}l|l|l@{}}
\midrule
Hyperparameter  & Sample Range & Sample Strategy \\ \midrule
start learning rate & $\left[10^{-2}, 10\right]$ & log uniform \\ \midrule
$\lambda$       & $\left[0, 1\right]$ & uniform \\ \midrule
$\tau$, $\tau_1$, $\tau_2$ & $\left[0, 1\right]$ & log uniform \\ \midrule
\end{tabular}

\caption{Hyperparameters sample strategy}
\label {table: Hyperparameter sample}
\end{table}

\textbf{Encoder training.} 
We train each encoder using the LARS optimizer~\citep{LARSOptimizer}, LambdaLR Scheduler in PyTorch, momentum 0.9, weight decay $10^{-6}$, batch size 256, and the aforementioned hyperparameters for 400 epochs on a single A-100 GPU.

\textbf{Image transformation.} The image transformation strategy, including augmentation, is identical to the default transformation strategy provided by PyTorch Lightning.

\textbf{Linear evaluation.}
The linear head is trained using the SGD optimizer with a cosine learning rate scheduler, batch size 64, and weight decay $10^{-6}$ for 100 epochs. The learning rate starts at $0.3$ and ends at $0$.

\textbf{Moco Experiments.} We also tested our method based on MoCo~\citep{he2019moco}. The results are summarized in Table \ref{tab:results-moco}. Here we choose ResNet18~\citep{ResNet} as the backbone and set a temperature of $0.1$ as default. For our simple sum kernel, we set $\lambda=0.8$. The results show that our method outperforms the original MoCo method.

\begin{table}[thb]
\centering
\caption{MoCo Experiment Results on CIFAR-10 and CIFAR-100.}
\label{tab:results-moco}
\resizebox{\textwidth}{!}{%
\begin{tabular}{@{}c|ccc|ccc@{}}
\toprule
\multirow{3}{*}{Method} & \multicolumn{3}{c|}{CIFAR-10} & \multicolumn{3}{c}{CIFAR-100} \\ \cmidrule(lr){2-4} \cmidrule(lr){5-7} 
                        & 200 epochs & 400 epochs    & 1000 epochs   & 200 epochs & 400 epochs & 1000 epochs         \\ \midrule
MoCo (repro.)         & $76.41 \pm 0.12$    & $80.01 \pm 0.15$          & $84.45 \pm 0.08$    & $\mathbf{47.02 \pm 0.11}$ & $52.50 \pm 0.07$ & $57.62 \pm 0.15$            \\
\midrule
Laplacian Kernel        & ${78.09 \pm 0.10}$    & $\mathbf{83.85 \pm 0.09}$          & $\mathbf{88.34 \pm 0.16}$    & $46.12 \pm 0.22$   & $53.44 \pm 0.17$ & $59.10 \pm 0.14$        \\
Simple Sum Kernel & $\mathbf{78.12 \pm 0.15}$   & $83.23 \pm 0.18$ & $87.50 \pm 0.20$ & $46.65 \pm 0.06$ & $\mathbf{53.62 \pm 0.19}$ & $\mathbf{59.83 \pm 0.12}$\\
\bottomrule
\end{tabular}
}
\end{table}



\section{More Experiments on Synthetic Data}


Consider a scenario with $n$ clusters, each containing $k$ vertices. Let the probability of vertices $u$ and $v$ from the same cluster belonging to $\bfpi$ be $p$. Conversely, for vertices $u$ and $v$ from different clusters, let the probability of belonging to $\pi$ be $q$. We generate the graph $\bfpi$ randomly, based on $p$ and $q$. We experiment with values of $k=100$ and $n=6$ for ease of visualization, embedding all points in a two-dimensional space. Each vertex's initial position originates from a normal distribution. In each iteration, we sample a subgraph of $\bfpi$ uniformly, ensuring each vertex has an out-degree of $1$. We then optimize the corresponding vectors using InfoNCE loss with an SGD optimizer and iterate until convergence. Our experimental setup consists of an SGD learning rate of $1$, an InfoNCE loss temperature of $0.5$, and a batch size of $50$. We evaluate two scenarios with different $p$ and $q$ values: $p=1$, $q=0$, and $p=0.75$, $q=0.2$. The results of these experiments are visualized in Figure \ref{fig:vis-spectral-cluster}. The obtained embeddings exhibit the hallmark pattern of spectral clustering of graph $\bfpi$.

\begin{figure}[!tb]
\centering
\subfigure{
\includegraphics[width=1\textwidth]{Figures/cluster_pi.png}
\label{fig:vis-cluster}
}
\subfigure{
\includegraphics[width=1\textwidth]{Figures/noised_cluster_pi.png}
\label{fig:vis-noised-cluster}
}
\caption{Visualizations of the optimization process using InfoNCE Loss on the vectors corresponding to $\bfpi$. Points of identical color belong to the same cluster within $\bfpi$. To showcase the internal structure of $\bfpi$, we randomly select 10 vertices from each cluster to display the edge distribution of $\bfpi$.}
\label{fig:vis-spectral-cluster}
\end{figure}




\end{document}
