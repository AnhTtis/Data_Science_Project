\section{Related Work}

Existing instance segmentation techniques can be grouped into three classes, \ie,  region-based methods, instance activation-based methods, and query-based methods.

\noindent\textbf{Region-based methods} first detect object bounding boxes and then apply RoI operations such as RoI-Pooling~\cite{Ren2015a} or RoI-Align~\cite{he2017mask} to extract region features for object classification and mask generation. As a pioneering work, Mask R-CNN~\cite{he2017mask} adds a mask branch on top of Faster R-CNN~\cite{Ren2015a} to predict the segmentation mask for each object. Follow-up methods either focus on improving the precision of detected bounding boxes~\cite{cai2018cascade,chen2019hybrid} or address the low-quality segmentation mask arising in Mask R-CNN~\cite{kirillov2020pointrend,tang2021look,chengwhl20}. Although the performance has been advanced on several benchmarks, these region-based methods suffer from a lot of duplicated region proposals that hurt the model's efficiency. 


\noindent\textbf{Instance activation-based methods} employ some meaningful pixels to represent the object and train the features of these pixels to be activated for the segmentation during the prediction. A typical class of such methods is based on the center activation~\cite{zhang2022e2ec,wang2020solov2,yolact-plus-tpami2020,tian2020conditional}, which forces the center pixels of the object to correspond to the segmentation and classification. For example, SOLO~\cite{wang2020solo, wang2020solov2} exploits the center features of the object to predict a mask kernel for the segmentation. MEInst~\cite{zhang2020MEInst} and CondInst~\cite{tian2020conditional} build the model upon the center-activation-based detector FCOS~\cite{tian2021fcos} with an additional branch of predicting mask embedding vectors for dynamic convolution. Recently, SparseInst~\cite{cheng2022sparseInst} learns a weighted pixel combination to represent the object. The proposed \modelname exploits the pixels located in the object region with the high class semantics as the representation of the object and extracts their features as the queries.


\noindent\textbf{Query-based methods} have emerged with DETR~\cite{detr} and show that a convolutional backbone with an end-to-end set prediction-based Transformer encoder-decoder~\cite{vaswani2017attention} can achieve good performance on the instance segmentation task. SOLQ~\cite{dong2021solq} and ISTR~\cite{hu2021istr} exploit the learned object queries to infer mask embeddings for instance segmentation. 
Panoptic SegFormer~\cite{li2021panopticsegformer} adds a location decoder to provide object position information.
Mask2Former~\cite{cheng2021maskformer,cheng2021mask2former} introduces masked attention for improved performance and faster convergence.
Mask DINO~\cite{li2022mask} unifies object detection and image segmentation tasks, obtaining great results on instance segmentation.
Despite the outstanding performance, query-based models are usually too computationally expensive to be applied in the real world. Compared with convolutional networks~\cite{cheng2022sparseInst,wang2020solov2}, their advantages on fast, efficient instance segmentation have not been well demonstrated. Our goal is to leverage the powerful modeling capabilities of the Transformer while designing an efficient, concise, and real-time instance segmentation scheme to promote the application of query-based segmentation methods. In addition, many works~\cite{wang2021max,yu2022cmt} also utilize the dual-path Transformer architecture in image segmentation tasks. However, their designs are generally complex and hard to deploy. We build our dual-path architecture simply upon plain Transformer layers for improved efficiency.



