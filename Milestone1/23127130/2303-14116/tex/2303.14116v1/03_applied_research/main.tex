In terms of applied research for this work, we describe research on applications of ML/DL models in NLP tasks.
In particular, we focus on applications to the field of computational advertising~\cite{dave2014computational}, which has a significant impact on business.
The field of computational advertising is a relatively new and important business-related research topic, dealing with very large online data volumes of the order of 100 million.
Display advertising is a type of online advertising in which an advertiser pays a publisher to display graphical material on the publisher's web page or application.
This graphical material, which primarily contains images and text, is commonly referred to as ad creative and serves to effectively provide product information to consumers who are willing to buy.
The market for digital advertising has been expanding enormously and is expected to grow further in the future.
In fact, the IAB internet advertising revenue report 2021~\cite{iab2021internet} states that ``digital advertising revenue increased 35.4\% year over year, the highest growth since 2006.''
Since it is difficult to operate ad data due to its large scale manually, a methodology is expected to provide operational support by means of a computer to distribute highly effective ads and discontinue ineffective ads.

Mainstream research in online advertising estimates click-through rate (CTR) and conversion rates (CVR) for users of the ads being served.
For such tasks, various methods based on ML/DL have recently emerged~\cite{chakrabarti2008contextual,richardson2007predicting,chen2016deep,covington2016deep,cheng2016wide}, including the variants of factorization machines~\cite{rendle2010factorization,juan2016field,juan2017field,guo2017deepfm} that can consider feature interactions.
While these have been evaluated and assessed on anonymized ad-serving benchmark datasets such as Criteo\footnote{\url{http://labs.criteo.com/2013/12/download-terabyte-click-logs/}} and Avazu\footnote{\url{https://www.kaggle.com/c/avazu-ctr-prediction}} and have reported some improvement in prediction performance, very few studies have evaluated them on other data, including real-world data~\cite{mishra2019guiding,zhou2020recommending}.
This is because ad data is generally very complex in terms of rights involving a large number of stakeholders, making it difficult to disclose.
Additionally, there are very few studies on the prediction of effectiveness for ads with no delivery performance, the so-called cold start setting~\cite{gope2017survey}.

The applied aspect of this work describes the practical application of NLP technology to the computational advertising field and discusses the prediction performance and model interpretability of the proposed frameworks.
We describe a framework for evaluating ad creatives that are more effective in terms of serving.
Although it is important to evaluate in advance what kind of ad creatives should be served, it is also important to provide evidence as to why the ad creatives are effective in order to improve newly created ad creatives and judge the validity of such evaluations.
We also describe a framework for automating the discontinuation of ad creatives with less serving effectiveness.
Predicting the timing of ad discontinuation itself is important.
Additionally, providing some evidence as to why the ad creative is discontinued at a certain time is an important indicator for advertisers and ad operators to consider whether to trust the prediction.
This is a major factor in confidence in the DL model.

\begin{figure}[t]
    \centering
    \includegraphics[width=\linewidth]{figures/pdf/ad_conversion_prediction.pdf}
    \caption[Outline of the proposed framework]{
        Outline of the proposed framework.
        In the framework, we propose two strategies: multi-task learning, which simultaneously predicts conversions and clicks, and a conditional attention mechanism, which detects important representations in ad creative text according to the text's attributes.
    }
    \label{fig:kitada2019conversion/proposed_architecture}
\end{figure}

\subsection{Operational Support for More Effective Ad}

Accurately predicting conversions in advertisements is generally a challenging task because such conversions do not occur frequently.
We propose a new framework to support creating high-performing ad creatives, including the accurate prediction of ad creative text conversions before serving them to the consumer, as shown in Fig.~\ref{fig:kitada2019conversion/proposed_architecture}.
The proposed framework includes the key ideas needed to train the model: multi-task learning and conditional attention mechanism.
The multi-task learning is an idea to improve prediction performance by simultaneously predicting clicks and conversions in order to overcome the difficulty of prediction due to data imbalance.
The conditional attention mechanism is a new mechanism that can take into account the attribute values of the ad creative, such as the genre of the ad creative and the gender of the delivery target, to further improve the performance of conversion prediction.
We evaluated the proposed framework on actual large-scale serving history data and confirmed that these ideas improved the performance of conversion prediction.

The conditional attention mechanism incorporated in the proposed framework is capable of interpreting word expressions that contribute to the effectiveness of ad serving.
Attention highlighting using the learned conditional attention mechanism predicts conversions in advance, taking into account the attribute values set at the time of ad submission, and simultaneously visualizes the importance of the words that contributed to the prediction.
By observing the attention highlighting when the proposed method predicts a high number of conversions for a prototype ad creative, it is possible to confirm the word expressions that better fit the target ad attributes.

\begin{figure*}[t]
    \centering
    \includegraphics[width=0.95\linewidth]{figures/pdf/ad_discontinuation_prediction.pdf}
    \caption[Outline of our framework]{
        Outline of our framework that exploits a hazard function, which draws on the idea of survival prediction, to predict the discontinuation of ad creatives.
        The input includes the four types of features: text, categorical, image, and numerical features.
        The output is the hazard probability, which includes whether the target ad creative has been discontinued in each time interval.
    }
    \label{fig:kitada2022ad/proposed_framework}
\end{figure*}

\subsection{Operational Support for Less Effective Ad}

Discontinuing ad creatives at an appropriate time is one of the most important ad operations that can have a significant impact on sales. 
Such operational support for ineffective ads has been less explored than that for effective ads. 
After pre-analyzing 1,000,000 real-world ad creatives, we found that there are two types of discontinuation: short-term (i.e., cut-out) and long-term (i.e., wear-out).
We propose a practical prediction framework for the discontinuation of ad creatives with a hazard function-based loss function inspired by survival prediction, as shown in Fig.~\ref{fig:kitada2022ad/proposed_framework}. 
Our framework accurately predicts the appropriate timing for the discontinuation of two types of digital advertisements, short-term and long-term. 
The framework consists of two main techniques: (1) a two-term estimation technique with multi-task learning and (2) a click-through rate-weighting technique for the loss function. 
We evaluated our framework using 1,000,000 real-world ad creatives, including 10 billion scale impressions. 

The attention mechanism incorporated in the proposed framework allows us to interpret the word expressions that contribute to ad discontinuation.
While various factors determine ad discontinuation, we expect that short- and long-term discontinuation in the ad text will show different trends. 
Specifically, in the short-term discontinuation, it is possible to identify word expressions in which the user did not show interest.
On the other hand, in the long-term discontinuation, it is possible to confirm word expressions that are no longer in season.
Based on the interpretation of these word expressions, the operator can make a final decision to discontinue the ad creatives.
Unfortunately, due to restrictions on information disclosure, quantitative evaluation of interpretability by specific attention visualization is not possible.
Meanwhile, we report that the performance of the framework is sufficient to support ad operators, and the word-by-word visualization the framework provides has significant advantages that could support discontinuation.
