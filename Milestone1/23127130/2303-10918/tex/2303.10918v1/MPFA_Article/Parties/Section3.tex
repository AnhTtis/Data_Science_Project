\section{Numerical Results on the Stokes Problem}\label{sec:numstokes}
In this section, we give some $2D$ numerical results which compare the $\bP^1_{nc}-P^0_{Mps}$ scheme to the $\bP^1_{nc}-P^0$ and $\bP^1_{nc}-(P^0+P^1)$ schemes. \\
Consider Problem \eqref{eq:StokesVF} with prescribed solution such that: $ (\uvec,p)=(\bm{0},\varphi)$. \\
When $\varphi$ is some affine function, then both  $\bP^1_{nc}-(P^0+P^1)$ and  $\bP^1_{nc}-P^0_{Mps}$ schemes give exactly $\uvec_h= \bm{0}$.\\
When $\varphi$ is some quadratic function, then $\bP^1_{nc}-(P^0+P^1)$ scheme gives exactly $\uvec_h= \bm{0}$, as a consequence of Property \ref{thm:P1nc-P0P1}. \\
In what follows, we set $\Om=(0,1)^2$. We denote the $L^2(\Om)$ errors estimates of the discrete velocity and pressure by:
$$\eps_0^X(\uvec_h) := \left\{ 
\begin{array}{rl}
\ds \|\uvec_h\|_{\bL^2(\Om)}&\mbox{ if }\uvec=0\\ \\
 \ds \frac{\|\uvec_h-\uvec\|_{\bL^2(\Om)}}{ \|\uvec \|_{\bL^2(\Om)}} &\mbox{ otherwise}\end{array}\right.  \quad  \quad   \textrm{and} \quad \quad    \eps_0^X(p_h) := \frac{\| p_h-p\|_{L^2(\Om)}}{\|p\|_{L^2(\Om)}}, $$
 where: $X=CR$ (resp. $X=Trio$ and $X=Mps$) refers to the solution computed with the $\bP^1_{nc}-P^0$ (resp. $\bP^1_{nc}-(P^0+P^1)$ and $\bP^1_{nc}-P^0_{Mps}$) scheme. \\
 We first consider Problem \eqref{eq:StokesVF} with prescribed solution $(\uvec,p) =(\mathbf{0},\sin(2 \pi x )\,\sin( 2 \pi y)\,)$. On Fig. \ref{fig:NoFlowu1} (resp. \ref{fig:NoFlowp1}), we plot $\eps_0^X(\uvec_h)$ (resp. $\eps_0^X(p_h)$) against the meshstep $h$ in the logarithmic scale, for $\nu=1$ and $\nu=10^{-3}$.
 
  We notice that $\eps_0^{X}(\uvec_h)\propto\nu^{-1}$ for the three schemes. Concerning the $\bP^1_{nc}-P^0_{Mps}$ scheme, we first remark that $\eps_0^{Mps}(\uvec_h)$ gives intermediate results between $\eps_0^{CR}(\uvec_h)$ and $\eps_0^{Trio}(\uvec_h)$ (see Fig. \ref{fig:NoFlowu1}). Second, we notice that $\eps_0^{Mps}(p_h)\approx\eps_0^{Trio}(p_h)$. Finally, we notice that the $\bP^1_{nc}-P^0_{Mps}$ scheme returns a convergence rate of order $3$ for $\eps_0^{Mps}(\uvec_h)$ and $2$ for $\eps_0^{Mps}(p_h)$.
 



 % Plot de la vitesse 
\begin{figure}[!ht]
	\centering
	\begin{tikzpicture}
	\begin{axis}[
	height = 7cm, % standard 10cm
	width = 10cm,  % standard 10cm
	xlabel = {$h$},
	ylabel = {$\eps_0(\uvec_h)$},
%	grid=both,
	ymajorgrids=true,
	major grid style={black!50},
	xmode=log, ymode=log,
	xmin=1e-2, xmax=1e-1,
	ymin=1e-9, ymax=1e1,
	yticklabel style={
		/pgf/number format/fixed,
		/pgf/number format/precision=0},
	scaled y ticks=false,
	xticklabel style={
		/pgf/number format/fixed,
		/pgf/number format/precision=0},
	scaled y ticks=false,  
	legend style={at={(1.5,0)},anchor=south east},
	legend columns=1
	], 
	\addplot[color=mygreen,mark=square] table [x=h, y=eUCRP0nu1] {Plots/Ctu0psin.txt};
	\addlegendentry{$\eps_0^{CR}(\uvec_h) ,\, \nu=1$ }
	\addplot[color=myred,mark=square] table [x=h, y=eUCRP0P1nu1] {Plots/Ctu0psin.txt};
	\addlegendentry{$\eps_0^{\textrm{Trio}}(\uvec_h),\, \nu=1 $}
	\addplot[color=myblue,mark=square] table [x=h, y=eUCRMPSnu1] {Plots/Ctu0psin.txt};
	\addlegendentry{$\eps_0^{Mps}(\uvec_h) ,\, \nu=1$}
	\addplot[color=mygreen,mark=triangle*] table [x=h, y=eUCRP0nu0001] {Plots/Ctu0psin.txt};
	\addlegendentry{$\eps_0^{CR}(\uvec_h),\, \nu=10^{-3} $ }
	\addplot[color=myred,mark=triangle*] table [x=h, y=eUCRP0P1nu0001] {Plots/Ctu0psin.txt};
	\addlegendentry{$\eps_0^{\textrm{Trio}}(\uvec_h) ,\, \nu=10^{-3}$}
	\addplot[color=myblue,mark=triangle*] table [x=h, y=eUCRMPSnu0001] {Plots/Ctu0psin.txt};
	\addlegendentry{$\eps_0^{Mps}(\uvec_h) ,\, \nu=10^{-3}$}
	\logLogSlopeTriangle{0.9}{0.1}{0.42}{3}{myblue};
	\logLogSlopeTriangle{0.9}{0.1}{0.58}{2}{mygreen};
	\logLogSlopeTriangle{0.9}{0.1}{0.30}{4}{myred};
	\end{axis}       
	\end{tikzpicture}
	\caption{$\eps_0^X(\uvec_h)$ for $(\uvec,p) =(\mathbf{0},\sin(2 \pi x )\,\sin( 2 \pi y)\,)$}
	\label{fig:NoFlowu1}
\end{figure}

 % Plot de la pression
\begin{figure}[!ht]
	\centering
	\begin{tikzpicture}
	\begin{axis}[
	height = 7cm, % standard 10cm
	width = 10cm,  % standard 10cm
	xlabel = {$h$},
	ylabel = {$\eps_0(p_h)$},
%	grid=none,
	ymajorgrids=true,
	major grid style={black!50},
	xmode=log, ymode=log,
	xmin=1e-2, xmax=1e-1,
	ymin=1e-4, ymax=1e-0,
	yticklabel style={
		/pgf/number format/fixed,
		/pgf/number format/precision=0},
	scaled y ticks=false,
	xticklabel style={
		/pgf/number format/fixed,
		/pgf/number format/precision=0},
	scaled y ticks=false,  
	legend style={at={(1.5,0)},anchor=south east},
	legend columns=1
	],
	\addplot[color=mygreen,mark=square] table [x=h, y=ePCRP0nu1] {Plots/Ctu0psin.txt};
		\addlegendentry{$\eps_0^{CR}(p_h) ,\, \nu=1$ }
		\addplot[color=myred,mark=square] table [x=h, y=ePCRP0P1nu1] {Plots/Ctu0psin.txt};
		\addlegendentry{$\eps_0^{\textrm{Trio}}(p_h),\, \nu=1 $}
		\addplot[color=myblue,mark=square] table [x=h, y=ePCRMPSnu1] {Plots/Ctu0psin.txt};
		\addlegendentry{$\eps_0^{Mps}(p_h) ,\, \nu=1$}
	\addplot[color=mygreen,mark=triangle*] table [x=h, y=ePCRP0nu0001] {Plots/Ctu0psin.txt};
		\addlegendentry{$\eps_0^{CR}(p_h),\, \nu=10^{-3} $ }
	\addplot[color=myred,mark=triangle*] table [x=h, y=ePCRP0P1nu0001] {Plots/Ctu0psin.txt};
		\addlegendentry{$\eps_0^{\textrm{Trio}}(p_h) ,\, \nu=10^{-3}$}
	\addplot[color=myblue,mark=triangle*] table [x=h, y=ePCRMPSnu0001] {Plots/Ctu0psin.txt};
		\addlegendentry{$\eps_0^{Mps}(p_h) ,\, \nu=10^{-3}$}
		\logLogSlopeTriangle{0.9}{0.1}{0.43}{2}{myred};
		\logLogSlopeTriangle{0.9}{0.1}{0.70}{1}{mygreen};
		\logLogSlopeTriangle{0.9}{0.1}{0.32}{2}{myblue};
	\end{axis}       
	\end{tikzpicture}
	\caption{$\eps_0(p_h)$ for $(\uvec,p) =(\mathbf{0},\sin(2 \pi x )\,\sin( 2 \pi y)\,)$}
	\label{fig:NoFlowp1}
\end{figure}

\newpage


We notice that, compared to the $P^1_{NC}-P^0$ scheme, the errors are greatly reduced by $\bP^1_{nc}-(P^0+P^1)$ and $\bP^1_{nc}-P^0_{Mps}$ schemes. These schemes allow to attenuate the amplitude of spurious velocities and hence provide a better simulation. This is illustrated by the resolution of \eqref{eq:Stokes} with $(\uvec,p)$ defined by \eqref{Test case 4 }. In this case, as $\uvec$ is not an affine function, the three schemes return a convergence rate of order $2$ for $\eps_0^{Mps}(\uvec_h)$ and $1$ for $\eps_0^{Mps}(p_h)$. The errors resulted for $h=0.1$ and $h=0.0125$ are plotted against viscosity in Figures \ref{fig:testvisco_vit} and \ref{fig:testvisco_press}. In these plots, we see that the $\bP^1_{nc}-P^0_{Mps}$ scheme gives intermediate results. Also, we notice that the spurious velocities errors become overriding when: 
\begin{itemize}
	\item $\nu\leq 10^0$    with $h=0.1$  and $\nu\leq 10^{-3}$ with $h=0.0125$ for the $P^1_{NC}-P^0$. 
	\item $\nu\leq 10^{-2}$ with $h=0.1$  and $\nu\leq 10^{-3}$ with $h=0.0125$ for the $\bP^1_{nc}-P^0_{Mps}$. 
	\item $\nu\leq 10^{-3}$ with $h=0.1$  and $\nu\leq 10^{-5}$ with $h=0.0125$ for the $\bP^1_{nc}-(P^0+P^1)$. 
\end{itemize}

The tipping viscosity point, where the spurious velocities errors become dominant, depends on the velocity error generated by the gradient approximation and therefore the mesh size. As these schemes converge with different orders when $\uvec=0$, it can be seen that decreasing the mesh size reduces the viscosity at which this point is reached more or less depending on the order. 

\begin{equation} %\textbf{ Test\,  3: \quad }
\label{Test case 4 }
 (\uvec,p)= \begin{pmatrix} \begin{matrix}
(\cos(2 \pi x)-1) \, \sin(2 \pi y) \\ 
-(\cos(2 \pi y)-1)\, \sin(2 \pi x)
\end{matrix} \, 
,  \sin(2 \pi x ) \sin( 2 \pi y)
\end{pmatrix}
\end{equation}


 % Plot de l'erreur de vitesse  en fonction de la viscosité
\begin{figure}[!ht]
	\centering
	\begin{tikzpicture}
	\begin{axis}[
	height = 7cm, % standard 10cm
	width = 10cm,  % standard 10cm
	xlabel = {$\nu$},
	ylabel = {$\eps_0(\uvec_h)$},
%	grid=none,
	yminorgrids=true,
	major grid style={black!50},
	xmode=log, ymode=log,
	xmin=1e-7, xmax=1e-0,
	ymin=1e-4, ymax=1e03,
	yticklabel style={
		/pgf/number format/fixed,
		/pgf/number format/precision=0},
	scaled y ticks=false,
	xticklabel style={
		/pgf/number format/fixed,
		/pgf/number format/precision=0},
	scaled y ticks=false,  
	legend style={at={(1.6,0)},anchor=south east},
	legend columns=1
	],
	\addplot[color=mygreen,mark=*] table [x=visco, y=eUCRP0] {Plots/visco/visco_Stokes_num30_h1TrioCFDconvnu.txt};
	\addlegendentry{$\eps_0^{CR}(\uvec_h),\, h=0.1 $ }
	\addplot[color=myred,mark=*] table [x=visco, y=eUCRP0P1] {Plots/visco/visco_Stokes_num30_h1TrioCFDconvnu.txt};
	\addlegendentry{$\eps_0^{\textrm{Trio}}(\uvec_h),\, h=0.1 $}
	\addplot[color=myblue,mark=*] table [x=visco, y=eUCRMPS] {Plots/visco/visco_Stokes_num30_h1TrioCFDconvnu.txt};
	\addlegendentry{$\eps_0^{Mps}(\uvec_h),\, h=0.1 $}	
	\addplot[color=mygreen,mark=square] table [x=visco, y=eUCRP0] {Plots/visco/visco_Stokes_num30_h4TrioCFDconvnu.txt};
	\addlegendentry{$\eps_0^{CR}(\uvec_h),\, h=0.0125 $ }
	\addplot[color=myred,mark=square] table [x=visco, y=eUCRP0P1] {Plots/visco/visco_Stokes_num30_h4TrioCFDconvnu.txt};
	\addlegendentry{$\eps_0^{\textrm{Trio}}(\uvec_h), \, h=0.0125 $}
	
	\addplot[color=myblue,mark=square] table [x=visco, y=eUCRMPS] {Plots/visco/visco_Stokes_num30_h4TrioCFDconvnu.txt};
	\addlegendentry{$\eps_0^{Mps}(\uvec_h),\, h=0.0125 $}
	\logLogSlopeTriangle{0.75}{-0.15}{0.65}{-1}{black};
	\end{axis}       
	\end{tikzpicture}
	\caption{$\eps_0(\uvec_h)$ for $\uvec$ and $p$ sinusoidal functions against viscosity with different mesh sizes}
	\label{fig:testvisco_vit}
\end{figure}



 % Plot de l'erreur de pression  en fonction de la viscosité
\begin{figure}[!ht]
	\centering
	\begin{tikzpicture}
	\begin{axis}[
	height = 7cm, % standard 10cm
	width = 10cm,  % standard 10cm
	xlabel = {$\nu$},
	ylabel = {$\eps_0(p_h)$},
%	grid=none,
	yminorgrids=true,
	major grid style={black!50},
	xmode=log, ymode=log,
	xmin=1e-7, xmax=1e-0,
	ymin=1e-4, ymax=1e0,
	yticklabel style={
		/pgf/number format/fixed,
		/pgf/number format/precision=0},
	scaled y ticks=false,
	xticklabel style={
		/pgf/number format/fixed,
		/pgf/number format/precision=0},
	scaled y ticks=false,  
	legend style={at={(1.5,0)},anchor=south east},
	legend columns=1
	],
	\addplot[color=mygreen,mark=*] table [x=visco, y=ePCRP0] {Plots/visco/visco_Stokes_num30_h1TrioCFDconvnu.txt};
	\addlegendentry{$\eps_0^{CR}(p_h),\, h=0.1  $ }
	\addplot[color=myred,mark=*] table [x=visco, y=ePCRP0P1] {Plots/visco/visco_Stokes_num30_h1TrioCFDconvnu.txt};
	\addlegendentry{$\eps_0^{\textrm{Trio}}(p_h),\, h=0.1  $}
	\addplot[color=myblue,mark=*] table [x=visco, y=ePCRMPS] {Plots/visco/visco_Stokes_num30_h1TrioCFDconvnu.txt};
	\addlegendentry{$\eps_0^{Mps}(p_h),\, h=0.1  $}
	\addplot[color=mygreen,mark=square] table [x=visco, y=ePCRP0] {Plots/visco/visco_Stokes_num30_h4TrioCFDconvnu.txt};
	\addlegendentry{$\eps_0^{CR}(p_h),\, h=0.0125  $ }
	\addplot[color=myred,mark=square] table [x=visco, y=ePCRP0P1] {Plots/visco/visco_Stokes_num30_h4TrioCFDconvnu.txt};
	\addlegendentry{$\eps_0^{\textrm{Trio}}(p_h),\, h=0.0125  $}
	\addplot[color=myblue,mark=square] table [x=visco, y=ePCRMPS] {Plots/visco/visco_Stokes_num30_h4TrioCFDconvnu.txt};
	\addlegendentry{$\eps_0^{Mps}(p_h),\, h=0.0125  $}
	\logLogSlopeTriangle{0.9}{0.1}{0.15}{1}{black};
	\end{axis}       
	\end{tikzpicture}
	\caption{$\eps_0(p_h)$ for $\uvec$ and $p$ sinusoidal functions against viscosity with different mesh sizes}
	\label{fig:testvisco_press}
\end{figure}



 

 
 
\newpage
We are also interested in the sensitivity to the mesh deformation. Indeed, nowadays, mesh refinement techniques based on a posteriori error estimators or industrial constraints can generate anisotropic meshes. In this subsection, we show that the three schemes have the same behaviour with respect to the regularity of the mesh.
To illustrate this property, we propose to use the Kershaw meshes presented in the benchmark \cite{HeHu08} ( see Fig. \ref{fig:kershaw}) with $(\uvec,p)$ in \eqref{Test case 4 }. As the mesh is composed of quadrilaterals, we cut them with along one of the diagonal, which allows us to remain within reasonable convergence assumptions. The mesh is represented in Fig. \ref{fig:kershaw} and we plot the results in Fig \ref{fig:kershawu} and \ref{fig:kershawp}. We can see that the schemes have a convergence rate of order $2$ for $\eps_0^{X}(\uvec_h)$ and $1$ for $\eps_0^{X}(p_h)$.

\begin{figure}[!ht]
	\centering
	\includegraphics[width=7.5cm]{Image_rap/kershaw.png}
	\caption{Kershaw mesh.}
	\label{fig:kershaw}
\end{figure} 
% Resultat en traçant en fct des inconnues : 
\begin{figure}[!ht]
	\centering
	\begin{tikzpicture}
	\begin{axis}[
	height = 7cm, % standard 10cm
	width = 10cm,  % standard 10cm
	xlabel = {$N_K$ (number of cells)},
	ylabel = {$\eps_0(\uvec_h)$},
%	grid=none,
	yminorgrids=true,
	major grid style={black!50},
	xmode=log, ymode=log,
	xmin=5e2, xmax=3e4,
	ymin=1e-3, ymax=1e0,
%	ymin=1e-6, ymax=1e0,
	yticklabel style={
		/pgf/number format/fixed,
		/pgf/number format/precision=0},
	scaled y ticks=false,
	xticklabel style={
		/pgf/number format/fixed,
		/pgf/number format/precision=0},
	scaled y ticks=false,  
	legend style={at={(1.3,0)},anchor=south east},
	legend columns=1
	], 
	\addplot[color=mygreen,mark=square] table [x=NCR, y=eUCRP0] {Plots/anisotropic/Stokes_meshtype8_num30ef0_2_7.txt};
	\addlegendentry{$\eps_0^{CR}(\uvec_h) $ }
	\addplot[color=myred,mark=square] table [x=NCR, y=eUCRP0P1] {Plots/anisotropic/Stokes_meshtype8_num30ef0_2_7.txt};
	\addlegendentry{$\eps_0^{\textrm{Trio}}(\uvec_h) $}
	\addplot[color=myblue,mark=square] table [x=NCR, y=eUCRMPS] {Plots/anisotropic/Stokes_meshtype8_num30ef0_2_7.txt};
	\addlegendentry{$\eps_0^{Mps}(\uvec_h) $}
	\end{axis}       
	\end{tikzpicture}
	\caption{$\eps_0(\uvec_h)$ for $\uvec$ and $p$ sinusoidal functions against viscosity with different kershaw meshes and $\nu =1$. }
	\label{fig:kershawu}
\end{figure}

% Plot de la pression
\begin{figure}[!ht]
	\centering
	\begin{tikzpicture}
	\begin{axis}[
	height = 7cm, % standard 10cm
	width = 10cm,  % standard 10cm
	xlabel = {$N_K$ (number of cells)},
	ylabel = {$\eps_0(p_h)$},
	yminorgrids=true,
%	grid=none,
	major grid style={black!50},
	xmode=log, ymode=log,
	xmin=5e2, xmax=3e4,
	ymin=1e-1, ymax=1e1,
%	ymin=1e-4, ymax=1e1,
	yticklabel style={
		/pgf/number format/fixed,
		/pgf/number format/precision=0},
	scaled y ticks=false,
	xticklabel style={
		/pgf/number format/fixed,
		/pgf/number format/precision=0},
	scaled y ticks=false,  
	legend style={at={(1.3,0)},anchor=south east},
	legend columns=1
	],
	\addplot[color=mygreen,mark=square] table [x=NCR, y=ePCRP0] {Plots/anisotropic/Stokes_meshtype8_num30ef0_2_7.txt};
	\addlegendentry{$\eps_0^{CR}(p_h) $ }
	\addplot[color=myred,mark=square] table [x=NCR, y=ePCRP0P1] {Plots/anisotropic/Stokes_meshtype8_num30ef0_2_7.txt};
	\addlegendentry{$\eps_0^{\textrm{Trio}}(p_h)$}
	\addplot[color=myblue,mark=square] table [x=NCR, y=ePCRMPS] {Plots/anisotropic/Stokes_meshtype8_num30ef0_2_7.txt};
	\addlegendentry{$\eps_0^{Mps}(p_h) $}
	\end{axis}       
	\end{tikzpicture}
	\caption{$\eps_0(p_h)$ for $\uvec$ and $p$ sinusoidal functions against viscosity with different kershaw meshes and $\nu =1$.}
	\label{fig:kershawp}
\end{figure}

\newpage
\newpage



\section{Numerical Results on the Navier-Stokes Problem}\label{sec:numNstokes}
 We choose the convection scheme initially presented in \cite[Eq. 2.8]{GaHeLa10}. This choice is motivated by the result of the Benchmark \cite{CHEY21} where the scheme presents good convergence and stability results  without increasing the stencil of the scheme. To resolve efficiently the Navier-Stokes equations, we use a prediction correction time-scheme  \cite{Chorin67} \cite{Teman68},  which consists in calculating a predicted velocity (with non-zero divergence), then solving the pressure at the next time and correcting the velocity to ensure divergence-free flow. With this approach, the velocity and pressure resolutions are decoupled.

	 As the approach is not classical, it is interesting to check the convergence of the scheme on Navier-Stokes. This leads to study the Green-Taylor vortex solution which is a well-known analytical solution to \eqref{eq:NS}.

% How we solve it 
Let first introduce the space-discretization of the Navier-Stokes equations \eqref{eq:NS} : 

\begin{equation}
\left\{
\begin{aligned}
 M \partial_t U +   \nu KU +L(U)+GP &=F\\
DU&=0
\end{aligned}
\right.
\end{equation}


where $U$, $P$ contains the velocity and pressure unknowns and $F$ is the right hand side. The matrices $M$ and $K$ are respectively the mass and stiffness matrices. Also, the matrices $G$ and $D$ represent the gradient and divergence operators. Finally, the matrix $L(U)$ is associated with the convection term and $\partial_t U$ is the time derivative of $U$.

We first present the prediction-correction time scheme : 

 \begin{enumerate}
 	\item Prediction :
 	\begin{equation}
 	\label{eq:pred_corr1}
 	M \frac{U^*-U^n}{\delta t} +\nu KU^* + L(U^n)U^n+GP^n =F^n \quad \mbox{ and } \quad DU^*\neq 0
 	\end{equation}
 	\item Pressure solver:
 	\begin{equation}
 	\label{eq:pred_corr2}
 	\delta t (D \tilde{M}^{-1} G) \delta P = D U^*
 	\end{equation}
 	with $\delta P = P^{n+1}-P^n$
 	\item Correction:
 	\begin{equation}
 	\label{eq:pred_corr3}
 	U^{n+1} = U^* +\delta t \tilde{M}^{-1} G \delta P
 	\end{equation}
 \end{enumerate}
with $\tilde{M}=M+\delta t \, \nu  K$. The CFL of the global system is then: 
\begin{equation}
	\label{eq:CFL_ORDER1}
	\delta t < C h 
\end{equation}


\begin{rmrk}
	In Equation \eqref{eq:pred_corr2}, we can approximate $\tilde{M} \simeq M $. This leads to a linear system which is faster to solve but less accurate. %and lead to a system with a CFL of order 2 ($\delta t < C h^2$)
\end{rmrk} %as $\delta t \, \nu  K\ll M$ 



\begin{rmrk}
	As we did in section 6, to determine the auxiliary pressures on the boundary of the MPFA scheme, we impose a condition for the pressure gradient at the edge. On each half-edge $\tilde{F}_i$ related to the vertex $S_0$ and edge $F_i \in \partial \Om$:
	\begin{equation}
	\label{eq:grap.n=f.n_for_NS}
	\int_{\tilde{F}_i}  \Gcal_i\cdot\nvec_{|\tilde{F}_i}= \int_{\tilde{F}_i} \big(\fvec + (\uvec_h^{n+1}-\uvec_h^{n})/\delta t + (\uvec_h^n \cdot \grad) \uvec_h^n \big)\cdot \nvec_{|\tilde{F}_i}.
	\end{equation}
\end{rmrk}

Let $\Om=(0,1)^2$. We prescribe the solution to Equation \eqref{eq:NS} with $\fvec=\bm{0}$, to be:
%We assume that the analytical solution of the Green-Taylor vortex is given by $\fvec=0$ and 
 
 \begin{equation}
 \label{Test case Green-Taylor }
 \left\{
 \begin{array}{rcll}
 u_x &=& - \, \cos\big(2 \pi (x+\frac{1}{2})\big) \,\sin\big(2 \pi (y+\frac{1}{2})\, \big)exp(-8\pi^2t) \vspace{0.05cm}\\
 u_y &=&  \, \sin\big(2 \pi (x+\frac{1}{2})\big)\, \cos \big(2 \pi (y+\frac{1}{2})\, \big)exp(-8\pi^2t) \vspace{0.05cm} \\
 p &=&   -\frac{1}{4} \, \cos(4 \pi (x+\frac{1}{4}) ) + \cos( 4 \pi (y+\frac{1}{2}))\, exp(-16\pi^2t) \vspace{0.05cm}
 \end{array}
 \right.
 \end{equation}
 
 We set $t_{max}=0.01 $ the final time of the simulation. The time step is chosen with respect to the CFL \eqref{eq:CFL_ORDER1} with $C=\frac{1}{2}$. The errors in velocity and pressure at the final time are plotted in Figures \ref{fig:NSgreentaylorU} \ref{fig:NSgreentaylorP} against the mesh step. We can see that the three schemes converge with order $2$ for $\eps_0^{Mps}(\uvec_h)$ and $1$ for $\eps_0^{Mps}(p_h)$ as expected. 
 


\begin{figure}[!ht]
	\centering
	\begin{tikzpicture}
	\begin{axis}[
	height = 7cm, % standard 10cm
	width = 10cm,  % standard 10cm
	xlabel = {$h$},
	ylabel = {$\eps_0(\uvec_h)$},
	%	grid=both,
	ymajorgrids=true,
	major grid style={black!50},
	xmode=log, ymode=log,
	xmin=1e-2, xmax=1e-1,
	ymin=1e-4, ymax=1e-1,
	yticklabel style={
		/pgf/number format/fixed,
		/pgf/number format/precision=0},
	scaled y ticks=false,
	xticklabel style={
		/pgf/number format/fixed,
		/pgf/number format/precision=0},
	scaled y ticks=false,  
	legend style={at={(1.3,0)},anchor=south east},
	legend columns=1
	], 
	\addplot[color=mygreen,mark=triangle*] table [x=h, y=eUCRP0] {Plots/NavierStokes_num100ef0_2_7.txt};
	\addlegendentry{$\eps_0^{CR}(\uvec_h) $ }
	\addplot[color=myred,mark=triangle*] table [x=h, y=eUCRP0P1] {Plots/NavierStokes_num100ef0_2_7.txt};
	\addlegendentry{$\eps_0^{\textrm{Trio}}(\uvec_h)$}
	\addplot[color=myblue,mark=triangle*] table [x=h, y=eUCRMPS] {Plots/NavierStokes_num100ef0_2_7.txt};
	\addlegendentry{$\eps_0^{Mps}(\uvec_h)$}
	\logLogSlopeTriangle{0.9}{0.2}{0.23}{2}{black};
	\end{axis}       
	\end{tikzpicture}
 	\caption{$\eps_0(\uvec_h)$ for $(\uvec,p) \in $ \eqref{Test case Green-Taylor }. }
\label{fig:NSgreentaylorU}
\end{figure}

% Plot de la pression
\begin{figure}[!ht]
	\centering
	\begin{tikzpicture}
	\begin{axis}[
	height = 7cm, % standard 10cm
	width = 10cm,  % standard 10cm
	xlabel = {$h$},
	ylabel = {$\eps_0(p_h)$},
	%	grid=none,
	ymajorgrids=true,
	major grid style={black!50},
	xmode=log, ymode=log,
	xmin=1e-2, xmax=1e-1,
	ymin=1e-2, ymax=1e1,
	yticklabel style={
		/pgf/number format/fixed,
		/pgf/number format/precision=0},
	scaled y ticks=false,
	xticklabel style={
		/pgf/number format/fixed,
		/pgf/number format/precision=0},
	scaled y ticks=false,  
	legend style={at={(1.3,0)},anchor=south east},
	legend columns=1
	],
	\addplot[color=mygreen,mark=triangle*] table [x=h, y=ePCRP0] {Plots/NavierStokes_num100ef0_2_7.txt};
	\addlegendentry{$\eps_0^{CR}(p_h) $ }
	\addplot[color=myred,mark=triangle*] table [x=h, y=ePCRP0P1] {Plots/NavierStokes_num100ef0_2_7.txt};
	\addlegendentry{$\eps_0^{\textrm{Trio}}(p_h)$}
	\addplot[color=myblue,mark=triangle*] table [x=h, y=ePCRMPS] {Plots/NavierStokes_num100ef0_2_7.txt};
	\addlegendentry{$\eps_0^{Mps}(p_h)$}
	\logLogSlopeTriangle{0.9}{0.2}{0.23}{1}{black};
	\end{axis}       
	\end{tikzpicture}
 	\caption{$\eps_0(p_h)$ for $(\uvec,p) \in $ \eqref{Test case Green-Taylor }.}
\label{fig:NSgreentaylorP}
\end{figure}
\newpage
