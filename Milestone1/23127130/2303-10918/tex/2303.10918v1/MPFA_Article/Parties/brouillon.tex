%
\begin{equation}
\label{Test case 3 }
\left\{
\begin{array}{rcll}
\uvec &=& \bm{0}\\
p &=&sin(2 \pi x)\, sin(2 \pi y)
\end{array}
\right.
\end{equation}

\begin{table}[!ht]
	\begin{center} 
		\begin{tabular}{c|c|c|c} 
			$h$ & $\eps^{CR}(\uvec)$ & $\eps^{\textrm{Trio}}(\uvec)$ & $\eps^{Mps}(\uvec)$\\ 
			\hline 
			$\num{1.00e-01 }$ &$\num{ 3.01e-03 }$ & $\num{1.04e-05}$  & $\num{1.59e-04}$  \\ 
			$\num{5.00e-02 }$ &$\num{ 6.93e-04 }$ & $\num{6.52e-07 }$ & $\num{1.78e-05}$  \\ 
			$\num{ 2.50e-02}$  &$\num{ 1.63e-04}$  & $\num{4.02e-08}$  & $\num{2.10e-06 }$  \\ 
			$\num{1.25e-02 }$ & $\num{4.14e-05 }$ & $\num{2.54e-09 }$ & $\num{3.91e-07}$   \\ 
			\hline  & & \\
			$EOC$  & $\tau^{CR}_\uvec=2.06 $ & $\tau^{\textrm{Trio}}_\uvec=4.00$  & $\tau^{Mps}_\uvec=2.89 $
		\end{tabular} 
	\end{center} 
	\begin{center}
		\begin{tabular}{c|c|c|c}
			$h$ & $\eps^{CR}(p)$ & $\eps^{\textrm{Trio}}(p)$ & $\eps^{Mps}(p)$\\ 
			\hline 
			$\num{1.00e-01 }$ & $\num{1.88e-01 }$ & $\num{2.48e-02}$  & $\num{1.95e-02 }$  \\ 
			$\num{ 5.00e-02 }$ &$\num{ 9.06e-02 }$ &$\num{ 5.49e-03}$  & $\num{4.28e-03 }$  \\ 
			$\num{ 2.50e-02}$  &$\num{ 4.40e-02}$  & $\num{1.37e-03}$  &$\num{ 1.09e-03}$   \\ 
			$\num{ 1.25e-02 }$ &$\num{ 2.22e-02 }$ & $\num{3.50e-04 }$ &$\num{ 2.78e-04 }$  \\ 
			\hline & & \\
			$EOC$  & $\tau^{CR}_p=1.03 $ & $\tau^{\textrm{Trio}}_p=2.05$  & $\tau^{Mps}_p=2.04 $
		\end{tabular} 
		\caption{Numerical errors in the case of $\uvec = \bm{0}, \, p=sin(2 \pi x ) sin( 2 \pi y)$ with $\nu=1$. }
		\label{test:conv1}
	\end{center} 
\end{table}

\begin{table}[!ht]
	\begin{center} 
		\begin{tabular}{c|c|c|c} 
			$h$ & $\eps^{CR}(\uvec)$ & $\eps^{\textrm{Trio}}(\uvec)$ & $\eps^{Mps}(\uvec)$\\ 
			\hline 
			$\num{1.00e-01 }$ &$\num{ 3.01e-03 }$ & $\num{1.04e-05}$  & $\num{1.59e-04}$  \\ 
			$\num{5.00e-02 }$ &$\num{ 6.93e-04 }$ & $\num{6.52e-07 }$ & $\num{1.78e-05}$  \\ 
			$\num{ 2.50e-02}$  &$\num{ 1.63e-04}$  & $\num{4.02e-08}$  & $\num{2.10e-06 }$  \\ 
			$\num{1.25e-02 }$ & $\num{4.14e-05 }$ & $\num{2.54e-09 }$ & $\num{3.91e-07}$   \\ 
			\hline  & & \\
			$EOC$  & $\tau^{CR}_\uvec=2.06 $ & $\tau^{\textrm{Trio}}_\uvec=4.00$  & $\tau^{Mps}_\uvec=2.89 $
		\end{tabular} 
	\end{center} 
	\begin{center}
		\begin{tabular}{c|c|c|c}
			$h$ & $\eps^{CR}(p)$ & $\eps^{\textrm{Trio}}(p)$ & $\eps^{Mps}(p)$\\ 
			\hline 
			$\num{1.00e-01 }$ & $\num{1.88e-01 }$ & $\num{2.48e-02}$  & $\num{1.95e-02 }$  \\ 
			$\num{ 5.00e-02 }$ &$\num{ 9.06e-02 }$ &$\num{ 5.49e-03}$  & $\num{4.28e-03 }$  \\ 
			$\num{ 2.50e-02}$  &$\num{ 4.40e-02}$  & $\num{1.37e-03}$  &$\num{ 1.09e-03}$   \\ 
			$\num{ 1.25e-02 }$ &$\num{ 2.22e-02 }$ & $\num{3.50e-04 }$ &$\num{ 2.78e-04 }$  \\ 
			\hline & & \\
			$EOC$  & $\tau^{CR}_p=1.03 $ & $\tau^{\textrm{Trio}}_p=2.05$  & $\tau^{Mps}_p=2.04 $
		\end{tabular} 
		\caption{Numerical errors in the case of $\uvec = \bm{0}, \, p=sin(2 \pi x ) sin( 2 \pi y)$ with $\nu=10^{-3}$ }
		\label{test:conv1}
	\end{center} 
\end{table}
\newpage



\begin{equation}
\label{Test case 1 }
\left\{
\begin{array}{rcll}
\uvec &=& \bm{0}\\
p &=&\varphi 
\end{array}
\right.
\end{equation}

If we take $\varphi$ as 
\begin{itemize}
	\item an affine function, then the $P^1_{NC}-(P^0+P^1)$ and the MPFA scheme find the exact solution. 
	\item a quadratic function, only the $P^1_{NC}-(P^0+P^1)$ give the exact solution for $d=2$, this is a consequence of theorem \ref{Theoreme fortin}.
\end{itemize}

Then, if we take $\varphi= sin(2 \pi x ) sin( 2 \pi y)$ we have the following result : 

\begin{table}[!ht]
	\begin{center} 
		\begin{tabular}{c|c|c|c} 
			$h$ & $\eps^{CR}(\uvec)$ & $\eps^{\textrm{Trio}}(\uvec)$ & $\eps^{Mps}(\uvec)$\\ 
			\hline 
			$\num{1.00e-01 }$ &$\num{ 3.01e-03 }$ & $\num{1.04e-05}$  & $\num{1.59e-04}$  \\ 
			$\num{5.00e-02 }$ &$\num{ 6.93e-04 }$ & $\num{6.52e-07 }$ & $\num{1.78e-05}$  \\ 
			$\num{ 2.50e-02}$  &$\num{ 1.63e-04}$  & $\num{4.02e-08}$  & $\num{2.10e-06 }$  \\ 
			$\num{1.25e-02 }$ & $\num{4.14e-05 }$ & $\num{2.54e-09 }$ & $\num{3.91e-07}$   \\ 
			\hline  & & \\
			$EOC$  & $\tau^{CR}_\uvec=2.06 $ & $\tau^{\textrm{Trio}}_\uvec=4.00$  & $\tau^{Mps}_\uvec=2.89 $
		\end{tabular} 
	\end{center} 
	\begin{center}
		\begin{tabular}{c|c|c|c}
			$h$ & $\eps^{CR}(p)$ & $\eps^{\textrm{Trio}}(p)$ & $\eps^{Mps}(p)$\\ 
			\hline 
			$\num{1.00e-01 }$ & $\num{1.88e-01 }$ & $\num{2.48e-02}$  & $\num{1.95e-02 }$  \\ 
			$\num{ 5.00e-02 }$ &$\num{ 9.06e-02 }$ &$\num{ 5.49e-03}$  & $\num{4.28e-03 }$  \\ 
			$\num{ 2.50e-02}$  &$\num{ 4.40e-02}$  & $\num{1.37e-03}$  &$\num{ 1.09e-03}$   \\ 
			$\num{ 1.25e-02 }$ &$\num{ 2.22e-02 }$ & $\num{3.50e-04 }$ &$\num{ 2.78e-04 }$  \\ 
			\hline & & \\
			$EOC$  & $\tau^{CR}_p=1.03 $ & $\tau^{\textrm{Trio}}_p=2.05$  & $\tau^{Mps}_p=2.04 $
		\end{tabular} 
		\caption{Numerical errors in the case of $\uvec = \bm{0}, \, p=sin(2 \pi x ) sin( 2 \pi y)$ with $\nu=1$. }
		\label{test:conv1}
	\end{center} 
\end{table}

\begin{table}[!ht]
	\begin{center} 
		\begin{tabular}{c|c|c|c} 
			$h$ & $\eps^{CR}(\uvec)$ & $\eps^{\textrm{Trio}}(\uvec)$ & $\eps^{Mps}(\uvec)$\\ 
			\hline 
			$\num{1.00e-01 }$ &$\num{ 3.01e-03 }$ & $\num{1.04e-05}$  & $\num{1.59e-04}$  \\ 
			$\num{5.00e-02 }$ &$\num{ 6.93e-04 }$ & $\num{6.52e-07 }$ & $\num{1.78e-05}$  \\ 
			$\num{ 2.50e-02}$  &$\num{ 1.63e-04}$  & $\num{4.02e-08}$  & $\num{2.10e-06 }$  \\ 
			$\num{1.25e-02 }$ & $\num{4.14e-05 }$ & $\num{2.54e-09 }$ & $\num{3.91e-07}$   \\ 
			\hline  & & \\
			$EOC$  & $\tau^{CR}_\uvec=2.06 $ & $\tau^{\textrm{Trio}}_\uvec=4.00$  & $\tau^{Mps}_\uvec=2.89 $
		\end{tabular} 
	\end{center} 
	\begin{center}
		\begin{tabular}{c|c|c|c}
			$h$ & $\eps^{CR}(p)$ & $\eps^{\textrm{Trio}}(p)$ & $\eps^{Mps}(p)$\\ 
			\hline 
			$\num{1.00e-01 }$ & $\num{1.88e-01 }$ & $\num{2.48e-02}$  & $\num{1.95e-02 }$  \\ 
			$\num{ 5.00e-02 }$ &$\num{ 9.06e-02 }$ &$\num{ 5.49e-03}$  & $\num{4.28e-03 }$  \\ 
			$\num{ 2.50e-02}$  &$\num{ 4.40e-02}$  & $\num{1.37e-03}$  &$\num{ 1.09e-03}$   \\ 
			$\num{ 1.25e-02 }$ &$\num{ 2.22e-02 }$ & $\num{3.50e-04 }$ &$\num{ 2.78e-04 }$  \\ 
			\hline & & \\
			$EOC$  & $\tau^{CR}_p=1.03 $ & $\tau^{\textrm{Trio}}_p=2.05$  & $\tau^{Mps}_p=2.04 $
		\end{tabular} 
		\caption{Numerical errors in the case of $\uvec = \bm{0}, \, p=sin(2 \pi x ) sin( 2 \pi y)$ with $\nu=10^{-3}$ }
		\label{test:conv1}
	\end{center} 
\end{table}
\newpage
\begin{rmq}
	The additional order of convergence in speed for TrioCFD is explained in \cite{Heib03}. 
	This one comes from a quadrature formula that uses the degrees of freedom of the scheme. As this formula is not true in dimension 3, this super-convergence will not appear. To keep this relationship, we will have to add degrees of freedom on the edges as it is done in \cite{Fort06}. This method lead to add approximately as many additional unknowns as there are cells. 
\end{rmq}



Let consider : 
\begin{equation}
\label{Test case 2}
\left\{
\begin{array}{rcll}
\uvec&=&\left(\begin{array}{c}  sin(2 \pi y) (cos(2 \pi x )-1)\\-sin(2 \pi x) (cos(2 \pi y )-1)\end{array}\right), \\
p&=&  \quad  \quad sin(2\pi x) sin(2 \pi y)
\end{array}
\right.
\end{equation}

\begin{table}[!ht]
	\begin{center} 
		\begin{tabular}{c|c|c|c} 
			$h$ & $\eps^{CR}(\uvec)$ & $\eps^{\textrm{Trio}}(\uvec)$ & $\eps^{Mps}(\uvec)$\\ 
			\hline 
			$\num{1.00e-01  }$& $\num{2.69e-02 }$ & $\num{2.45e-02  }$& $\num{2.96e-02 }$  \\ 
			$\num{ 5.00e-02}$  &$\num{ 6.92e-03}$  & $\num{5.84e-03 }$ &$\num{ 7.40e-03}$   \\ 
			$\num{ 2.50e-02 }$ & $\num{1.70e-03 }$ & $\num{1.45e-03 }$ & $\num{1.71e-03 }$  \\ 
			$\num{ 1.25e-02 }$ & $\num{4.28e-04 }$ & $\num{3.65e-04 }$ & $\num{4.31e-04 }$  \\ 
			\hline 
			$\tau$  & $\tau^{CR}_\uvec=1.99 $ & $\tau^{\textrm{Trio}}_\uvec=2.02 $ & $\tau^{Mps}_\uvec=2.03 $
		\end{tabular} 
	\end{center} 
	\begin{center}
		\begin{tabular}{c|c|c|c}
			$h$ & $\eps^{CR}(p)$ & $\eps^{\textrm{Trio}}(p)$ & $\eps^{Mps}(p)$\\ 
			\hline 
			$\num{1.00e-01} $&$\num{ 8.35e-01  }$& $ 1.14 \times 10^{-0}$ &  $ \num{8.00e-01} $  \\ 
			$\num{ 5.00e-02 }$ & $\num{4.06e-01 }$ & $\num{6.08e-01 }$ & $\num{4.20e-01}$   \\ 
			$\num{ 2.50e-02 }$ & $\num{1.88e-01 }$ & $\num{2.89e-01 }$ & $\num{2.08e-01 }$  \\ 
			$\num{1.25e-02}$  & $\num{9.50e-02 }$ & $\num{1.45e-01 }$ & $\num{1.07e-01  }$ \\ 
			\hline 
			$\tau$  & $\tau^{CR}_p=1.05 $ & $\tau^{\textrm{Trio}}_p=0.99 $ & $\tau^{Mps}_p=0.96 $
		\end{tabular} 
		\caption{Numerical errors in the case of $\uvec$ and $p$ sinusoidal functions}
		\label{test:conv2}
	\end{center} 
\end{table}
The convergence orders shown in \ref{test:conv2} are those that were intended for non-conforming Crouzeix-Raviart finite element. 
\newpage
\subsection{Viscosity robustness}
In this section, we illustrate the viscosity robustness obtain with $P^1_{NC}-(P^0+P^1)$ and $P^1_{NC}-MPS$ schemes. To this end, simulations are proposed on $\Omega=[0,1]\times[0,1]$ with Dirichlet condition for the velocity and a mesh with a mesh-size $h=0.05$
\\

If we consider a pressure solution of our Stokes problem $p \in P^1$. Then the $P^1_{NC}-MPS$ is fully robust to the viscosity i.e. the velocity errors is invariant to $\nu$. We illustrate this property with the following results. Let's consider the problem :
\begin{equation} 
\label{Test nu 1}
\left\{
\begin{array}{rcll}
\uvec&=&\left(\begin{array}{c}  sin(2 \pi y) (cos(2 \pi x )-1)\\-sin(2 \pi x) (cos(2 \pi y )-1)\end{array}\right), \\
p&=&  \quad  \quad x+y-1
\end{array}
\right.
\end{equation}


\begin{table}[!ht]
	\begin{center} 
		\begin{tabular}{c|c|c|c} 
			$\nu [m^2s^{-1}]$ & $\eps^{CR}(\uvec)$ & $\eps^{\textrm{Trio}}(\uvec)$ & $\eps^{Mps}(\uvec)$\\ 
			\hline 
			$10^{-0}$& $\num{2.68e-02 }$ & $\num{2.45e-02  }$& $\num{2.61e-02 }$  \\ 
			$10^{-1}$  &$\num{ 2.82e-02}$  & $\num{2.45e-02  }$ & $\num{2.61e-02 }$    \\ 
			$10^{-2}$ & $\num{8.70e-02 }$ & $\num{2.45e-02  }$ & $\num{2.61e-02 }$  \\ 
			$10^{-3}$ & $\num{8.24e-01 }$ & $\num{2.45e-02  }$ & $\num{2.61e-02 }$ 
		\end{tabular} 
	\end{center} 
	\begin{center}
		\begin{tabular}{c|c|c|c}
			$\nu [m^2s^{-1}]$ & $\eps^{CR}(p)$ & $\eps^{\textrm{Trio}}(p)$ & $\eps^{Mps}(p)$\\ 
			\hline 
			$10^{-0}$&$\num{ 9.99e-01  }$& $1.40\times 10^{-0}$ & $2.59 \times 10^{-0}$ \\ 
			$10^{-1}$ & $\num{1.25e-01 }$ & $\num{1.40e-01}$ & $\num{2.59e-01}$   \\ 
			$10^{-2}$ & $\num{7.66e-02 }$ & $\num{1.40e-02 }$ & $\num{2.59e-02}$  \\ 
			$10^{-3}$  & $\num{7.59e-02 }$ & $\num{1.40e-03 }$ & $\num{2.59e-03  }$ 
		\end{tabular} 
	\end{center} 
	\caption{Numerical errors according viscosity with $\uvec$ sinusoidal function and $p$ affine function. } 
	\label{viscosity test 1}
\end{table} 
\newpage 

This property result to the exact approximation of the gradient of pressure. No source term are generated by the pressure approximation and therefore the error does not longer depends on the viscosity. 
\begin{rmq}
	For $P^1_{NC}-(P^0+P^1)$, as $\grad p$ is still approximated exactly if $p \in P^2$ then the robustness is conserved.  
\end{rmq}

Now consider $p\notin P^2$, then this property doesn't stand anymore for both scheme. 


\begin{equation} 
\label{Test nu 2}
\left\{
\begin{array}{rcll}
\uvec&=& \bm{0}\\
p&=& x^5+x^4y^3+x^2y+y^4-\frac{7}{12}
\end{array}
\right.
\end{equation}


\begin{table}[!ht]
	
	\begin{center} 
		\begin{tabular}{c|c|c|c} 
			$\nu [m^2s^{-1}]$ & $\eps^{CR}(\uvec)$ & $\eps^{\textrm{Trio}}(\uvec)$ & $\eps^{Mps}(\uvec)$\\ 
			\hline 
			$10^{-0}$& $\num{5.84e-04 }$ & $\num{6.01e-08  }$& $\num{5.30e-06}$  \\ 
			$10^{-1}$  &$\num{ 5.84e-03}$  & $\num{ 6.01e-07  }$ & $\num{5.30e-05 }$    \\ 
			$10^{-2}$ & $\num{5.84e-02}$ & $\num{6.01e-06  }$ & $\num{5.30e-04 }$  \\ 
			$10^{-3}$ & $\num{5.84e-01}$ & $\num{6.01e-05  }$ & $\num{5.30e-03 }$ 
		\end{tabular} 
	\end{center} 
	\begin{center}
		\begin{tabular}{c|c|c|c}
			$\nu [m^2s^{-1}]$ & $\eps^{CR}(p)$ & $\eps^{\textrm{Trio}}(p)$ & $\eps^{Mps}(p)$\\ 
			\hline 
			$10^{-0}  $&$ \num{5.82e-02}  $& $\num{1.37e-03} $ & $\num{1.16e-03} $ \\ 
			$10^{-1} $ & $\num{5.82e-02} $ & $\num{1.37e-03}$ & $\num{1.16e-03}$   \\ 
			$ 10^{-2} $ & $\num{5.82e-02} $ & $\num{1.37e-03} $ & $\num{1.16e-03}$  \\ 
			$10^{-3}$  & $\num{5.82e-02}$ & $\num{1.37e-03} $ & $\num{1.16e-03}  $ 
		\end{tabular} 
	\end{center} 
	\caption{Numerical errors according viscosity with $\uvec=\bm{0}$ and $p \in P^5$. } 
	\label{viscosity test 2}
\end{table}



Even though the robustness of $\nu$ is no longer observed, we notice that the errors are greatly reduced by the more accurate approximation of the gradient. This allows to reduce the amplitude of spurious velocities and hence provide a better simulation. This is illustrated by the test case \eqref{viscosity test 2} where the spurious velocities errors become dominant when: 
\begin{itemize}
	\item $\nu\leq 10^0$ for the $P^1_{NC}-P^0$. 
	\item $\nu\leq 10^{-3}$ for the $P^1_{NC}-MPS$. 
	\item $\nu\leq 10^{-4}$ for the $P^1_{NC}-(P^0+P^1)$. 
\end{itemize}


\begin{table}[!ht]
	\begin{center} 
		\begin{tabular}{c|c|c|c} 
			$\nu [m^2s^{-1}]$ & $\eps^{CR}(\uvec)$ & $\eps^{\textrm{Trio}}(\uvec)$ & $\eps^{Mps}(\uvec)$\\ 
			\hline 
			$10^{-1}$  &$\num{  6.93e-03}$  & $\num{ 5.84e-03  }$ & $\num{7.41e-03 }$    \\ 
			$10^{-2}$ & $\num{8.48e-03 }$ & $\num{5.84e-03   }$ & $\num{7.41e-03 }$  \\ 
			$10^{-3}$ & $\num{4.84e-02}$ & $\num{5.83e-03  }$ & $\num{7.48e-03  }$ \\
			$10^{-4}$ & $\num{4.77e-01}$ & $\num{5.83e-03  }$ & $\num{9.11e-03 }$ 
		\end{tabular} 
	\end{center} 
	\begin{center}
		\begin{tabular}{c|c|c|c}
			$\nu [m^2s^{-1}]$ & $\eps^{CR}(p)$ & $\eps^{\textrm{Trio}}(p)$ & $\eps^{Mps}(p)$\\ 
			\hline 
			$10^{-1} $ & $\num{3.16e-01} $ & $\num{4.78e-01}$ & $\num{4.44e-01}$   \\ 
			$ 10^{-2} $ & $\num{6.60e-02} $ & $\num{4.78e-02} $ & $\num{4.44e-02}$  \\ 
			$10^{-3}$  & $\num{5.83e-02}$ & $\num{ 4.97e-03} $ & $\num{4.55e-03}  $ \\
			$10^{-4}$  & $\num{5.83e-02}$ & $\num{ 1.45e-03} $ & $\num{1.23e-03}  $ 
		\end{tabular} 
	\end{center} 
	\caption{Numerical errors according viscosity with $\uvec$ sinusoidal function and $p \in P^5$. } 
	\label{viscosity test 2}
\end{table}



\begin{rmq} \textbf{Comparison of computation times}

We have plot in Figure \ref{fig:ctime} the total time of solving the global linear system for the Stokes problem for different meshes. 
\begin{figure}[!h]
	\centering
	\tiny
	\includegraphics[width=9cm]{figs/ctime.png}
	\label{fig:ctime}
\end{figure}
\newpage
\end{rmq}
%\subsection{Comparison on the forward facing step test in the 2D Navier-Stokes problem}
