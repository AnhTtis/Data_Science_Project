



\section{Conclusion and perspectives}


The purpose of this work is to present a new discretisation for the gradient of pressure. This scheme presents similar result to $\bP^1_{nc}-(P^0+P^1)$ discretization.   
However, some points have been left out of the scope of this work and deserve further investigation:
\begin{itemize}
%	\item To eliminate the auxiliary unknowns on the boundary, we suppose a regularity to $\fvec \in \bH(\dive;\Om)$. If we change the type of limit condition, or the regularity of $f$, one should consider the influence in the system. For example, we can easily impose pressure Dirichlet boundary condition, by imposing strongly the value of the auxiliary unknowns.
	\item On the boundary, the continuity of the gradient flows can not be applied. We need boundary conditions to complete the system of elimination of the auxiliary unknowns \eqref{eq:continuitefacesnormales}. If the problem has, for the pressure:
	\begin{itemize}
		\item[-] Dirichlet boundary condition: we can evaluate the value of the auxiliary unknowns on the boundary.
		\item[-] Neumann boundary condition: we can evaluate the value of normal component of the pressure gradient on the boundary. 
	\end{itemize}
	Otherwise, we can keep the auxiliary unknowns and complete the problem with other equations.\\
	
	\item The section 3 shows that our scheme provides a benefit to the classic $\bP^1_{NC}-P^0$ discretisation but $\bP^1_{nc}-(P^0+P^1)$ has an additional superconvergence case. This property disappears in 3D, unless we add pressure degree of freedom on edges which turns out to be costly in computer memory. In that case, the MPFA scheme and $\bP^1_{nc}-(P^0+P^1)$ give comparable results but with a duality between scheme stencil and memory footprint. A study will be carried out to compare the efficiency of the two schemes.
	\item The scheme seems numerically stable but the inf-sup condition has still not been proven. 
	\item The scheme is currently in development in the CEA thermohydraulic code TrioCFD and its implementation will allow to realise more test.
	\item The FECC scheme is an other gradient discretization scheme, which has similar properties to the MPFA scheme and can handle more general meshes \cite{lepotier12}. The same approach can be used to develop a new scheme on polyhedral meshes for the Navier-Stokes problem.
	
\end{itemize}



