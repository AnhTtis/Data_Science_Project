%-----------------------------------%
\section{Motivation}\label{sec:intro}
%-----------------------------------%
The TrioCFD code is a computational fluid dynamics (CFD) simulation software
developed at the CEA. It is open source, object-oriented and massively parallel. It is
dedicated to the numerical simulation of turbulent flows for scientific and industrial
applications, particularly in the nuclear field. Let $\Om$, the domain of study, be an open connected bounded domain of $\R^d$, $d=2,\,3$, with a polygonal $(d=2)$ or Lipschitz polyhedral $(d=3)$ boundary $\pa\Om$ with constant physical properties. Let $T>0$ be a simulation time. The TrioCFD code solves the incompressible Navier-Stokes equations which read: Find $(\uvec(\xvec,t),p(\xvec,t))$ such that $\forall(\xvec,t)\in\Om\times(0,T)$,
%-------------------------------%
\begin{equation}\label{eq:NS}
\left\{
\begin{array}{rcl}
\ds\pa_t\uvec-\nu\Deltavec\uvec+(\uvec\cdot\grad)\uvec+\grad p&=&\fvec,\\
\dive\uvec&=&0,\\
u(\xvec,0)&=&u_0(\xvec).
\end{array}\right.
\end{equation}
%------------%
We consider here Dirichlet boundary conditions for the velocity $\uvec$ and we impose a normalization condition for the pressure $p$:
\[
\uvec=0\mbox{ on }\pa\Om,\quad\int_\Om p=0.
\]
The vector field $\uvec$ represents the velocity of the fluid and the scalar field $p$ represents its pressure divided by the fluid density which is supposed to be constant. 
The first equation of \eqref{eq:NS} corresponds to the momentum balance equation and the second one corresponds to the mass conservation. The constant parameter $\nu>0$ is the kinematic viscosity of the fluid. The vector field $\fvec$ represents the body force divided by the fluid density. We first consider the steady Stokes problem which reads:
\begin{equation}\label{eq:Stokes}
\mbox{Find }(\uvec,p)\mbox{  such that }\forall\xvec\in\Om:\,\left\{\begin{array}{rcl}
-\nu\Delta\uvec+\grad p&=&\fvec,\\
\dive\uvec&=&0.
\end{array}\right.
\end{equation}
Before stating the variational formulation of Problem \eqref{eq:Stokes}, we provide some definition and reminders. Let us set $\bL^2(\Om)=(L^2(\Om))^d$, $\bH^1_0(\Om)=(H^1_0(\Om))^d$, $\bH^{-1}(\Omega)=(H^{-1}(\Om))^d$ its dual space and $L^2_{zmv}(\Om)=\{q\in L^2(\Om)\,|\,\int_{\Om}q=0\}$. We recall that $\bH(\dive;\,\Om)=\{\vvec\in\bL^2(\Om)\,|\,\dive\vvec\in\, L^2(\Om)\}$. Let us first recall Poincaré-Steklov inequality:
%---------------------------------%
\begin{equation}\label{eq:Poincare}
\exists C_{PS}>0\,|\,\forall v\in H^1_0(\Om),\quad \|v\|_{L^2(\Om)}\leq C_{PS}\|\grad v\|_{\bL^2(\Om)}.
\end{equation}
%------------%
Thanks to this result, in $H^1_0(\Om)$, the semi-norm is equivalent to the natural norm, so that the scalar product reads $(v,w)_{H^1_0(\Om)}=(\grad v,\grad w)_{\bL^2(\Om)}$ and the norm is $\|v\|_{H^1_0(\Om)}=\|\grad v\|_{\bL^2(\Om)}$. Let $\vvec,\,\wvec\in\bH^1_0(\Om)$, we denote by $(v_i)_{i=1}^d$ (resp. $(w_i)_{i=1}^d$) the components of $\vvec$ (resp. $\wvec$), and we set $\Grad\vvec=(\pa_j v_i)_{i,j=1}^d\in\bbL^2(\Om)$, where $\bbL^2(\Om)=[L^2(\Om)]^{d\times d}$. We have: 
\[
(\Grad\vvec,\Grad\wvec)_{\bbL^2(\Om)}=(\vvec,\wvec)_{\bH^1_0(\Om)}=\ds\sum_{i=1}^d(v_i,w_i)_{H^1_0(\Om)}\mbox{ and }\|\vvec\|_{\bH^1_0(\Om)}=\|\Grad\vvec\|_{\bbL^2(\Om)}.\]
Let us set $\bV=\left\{\vvec\in\bH^1_0(\Om)\,|\,\dive\vvec=0\right\}$. The space $\bV$ is a closed subset of $\bH^1_0(\Om)$. We denote by $\bV^\perp$ the orthogonal of $\bV$ in $\bH^1_0(\Om)$. We recall that \cite[cor. I.2.4]{GiRa86}: %Let $\nu_p>0$ be a kinematic viscosity
%----------%
\begin{prop}\label{prop:diviso}
	The operator $\dive:\,\bH^1_0(\Om)\rightarrow L^2(\Om)$ is an isomorphism of $\,\bV^{\perp}$ onto $L^2_{zmv}(\Om)$. We call $C_{\dive}$ the constant such that:
	%----------%
	\begin{equation}\label{eq:BAI}
	\forall p\in L^2_{zmv}(\Om),\,\exists!\vvec\in\bV^{\perp}\,|\,\dive\vvec=p\mbox{ and }
	\|\vvec\|_{\bH^1_0(\Om)}\leq C_{\dive}\|p\|_{L^2(\Om)}.
	\end{equation}
\end{prop}
%----------%
Let us set~:
\begin{equation}
\label{eq:BilinForms}
a_\nu:\left\{\begin{array}{rcl}\bH^1_0(\Omega)\times\bH^1_0(\Omega)&\rightarrow&\R\\
(\uvec', \vvec)&\mapsto&\nu\,(\uvec', \vvec)_{\bH^1_0(\Om)}
\end{array}\right.\mbox{ and }b:\left\{\begin{array}{rcl}\bH^1_0(\Omega)\times L^2_{zmv}(\Om)&\rightarrow&\R\\
(\vvec, q)&\mapsto&(\dive\vvec, q)_{L^2(\Om)}
\end{array}\right..
\end{equation}
Classically, the variational formulation of Problem \eqref{eq:Stokes} reads: 
\begin{equation}
\label{eq:StokesVF}
\mbox{Find }(\uvec, p) \in \bH^1_0(\Omega)\times L_{zmv}^2(\Omega)\,|\,\left\{
\begin{array}{rcll}
a_\nu(\uvec, \vvec)_{\bH^1_0(\Om)}-b(\vvec,p)&=&\langle\fvec, \vvec\rangle & \forall \vvec \in  \bH^1_0(\Omega),\\
b(\uvec,q)&=&0 &\forall q \in L_{zmv}^2(\Omega).
\end{array}
\right.
\end{equation}
This saddle point problem is well-posed. Indeed, the bilinear form $a_\nu(\cdot,\cdot)$ is continuous and coercive. Moreover, the bilinear form $b(\cdot,\cdot)$ is continuous and due to Proposition \ref{prop:diviso}, it satisfies the following inf-sup condition:
\begin{equation}
\label{eq:CIS}
\forall q\in L^2_{zmv}(\Om)\backslash\{0\},\quad\exists\,\vvec_q\in\bH^1_0(\Omega)\backslash\{0\}\,|\quad\frac{b(\vvec_q,q)}{\|\vvec_q\|_{\bH^1_0(\Om)}\,\|q\|_{L^2(\Om)}}\geq C_{\dive}.
\end{equation}
In TrioCFD code, the spatial discretization of Problem \eqref{eq:Stokes} is based on  first order nonconforming Crouzeix-Raviart finite element method.\\
The outline of this article is as follows: in section \ref{sec:disc}, we provide some notations for the discretization. Next, in section \ref{sec:P1nc-P0}, we recall the first order nonconforming finite element method, that we call the $\bP^1_{nc}-P^0$ scheme. Then in section \ref{sec:P1nc-P0P1}, we describe the spatial discretization of TrioCFD code for simplicial meshes. We call this discretization the $\bP^1_{nc}-(P^0+P^1)$ scheme. This discretization is very precise in $2D$. It is also precise in $3D$, except when the source term is a strong gradient. In order to obtain the same accuracy in $3D$ than in $2D$, one must increase the number of degrees of freedom of the discrete pressure space, which leads to a more expensive numerical scheme. Our aim is to develop a new numerical scheme that would be precise both in $2D$ and $3D$, but at a lower cost. We present such a scheme in section \ref{sec:MPFA} and numerical illustration in section \ref{sec:numstokes} and \ref{sec:numNstokes}.
%------------------------------------------------%
\section{Discrete notations}\label{sec:disc}
%------------------------------------------------%
We call $(O,(x_{d'})_{d'=1}^d)$ the Cartesian coordinates system, of orthonormal basis $(\evec_{d'})_{d'=1}^d$. %Suppose that the computational domain $\Om$ is bounded by a Lipschitz polyhedral boundary $\pa\Om$. 
Consider $(\Tcal_h)_h$ a simplicial triangulation sequence of $\Om$. For a triangulation $\Tcal_h$, we use the following index sets:
\begin{itemize}
	\item $\Ical_K$ denotes the index set of the elements, such that $\Tcal_h:=\ds\bigcup_{\ell\in\Ical_K}K_\ell$  is the set of elements.
	\item $\Ical_F$ denotes the index set of the facets\footnote{The term facet stands for face (resp. edge) when $d=3$ (resp. $d=2$).}, such that $\Fcal_h:=\ds\bigcup_{f\in\Ical_F}F_f$ is the set of facets.
	%\overset{\circ}{\Om}$ $\stackrel{\circ}{\Omega}$
%	\item[]Let $\Ical_F=\Ical_F^i\cup\Ical_F^b$, where $\forall f\in\Ical_F^i$, \textcolor{blue}{$F_f\in\Om$} and $\forall f\in\Ical_F^b$, $F_f\in\pa\Om$.
	\item[]Let $\Ical_F=\Ical_F^i\cup\Ical_F^b$, where $\forall f\in\Ical_F^i$, $F_f\in\Om$ and $\forall f\in\Ical_F^b$, $F_f\in\pa\Om$.
	\item $\Ical_S$ denotes the index set of the vertices, such that $(S_j)_{j\in\Ical_S}$ is the set of vertices.
	\item[]Let $\Ical_S=\Ical_S^i\cup\Ical_S^b$, where $\forall j\in\Ical_S^i$, $S_j\in\Om$ and $\forall j\in\Ical_S^b$, $S_j\in\pa\Om$.
\end{itemize} 
We also define the following index subsets:
\begin{itemize}
	\item $\forall\ell\in\Ical_K$, $\Ical_{F,\ell}=\{f\in\Ical_F\,|\,F_f\in K_\ell\},\quad\Ical_{S,\ell}=\{j\in\Ical_S\,|\,S_j\in K_\ell\}$.
	\item $\forall j\in\Ical_S$, $\Ical_{K,j}=\{\ell\in\Ical_K\,|\,S_j\in K_\ell\}$, $\quad N_{K,j}:=\mathrm{card}(\Ical_{K,j})$.
	\item $\forall j\in\Ical_S$, $\Ical_{S,j}=\{k\in\Ical_S\,|\,S_kS_i\in\Fcal_h\}$, $\quad N_{S,j}:=\mathrm{card}(\Ical_{S,j})$.
\end{itemize}
Notice that in $2D$, $N_{K,j}=N_{S,j}$.\\
For all $f\in\Ical_F$, $M_f$ denotes the barycentre of $F_f$, and by $\nvec_f$ a unit normal (outward oriented if $F_f\in\pa\Omega$). For all $j\in\Ical_S$, for all $\ell\in\Ical_{K,j}$, $\lambda_{j,\ell}$ denotes the barycentric coordinate of $S_j$ in $K_\ell$; $F_{j,\ell}$ denotes the face opposite to vertex $S_j$ in element $K_\ell$. We call $\Scal_{j,\ell}$ the outward normal vector of $F_{j,\ell}$ and of norm $|\Scal_{j,\ell}|=|F_{j,\ell}|$. Let introduce spaces of piecewise regular elements:\\
We set $\Pcal_h H^1=\left\{v\in L^2(\Om)\,;\quad\forall \ell\in\Ical_K,\,v_{|K_\ell}\in H^1(K_\ell)\right\}$, endowed with the scalar product~:
\[
(v,w)_h:=\sum_{\ell\in\Ical_K}(\grad v,\grad w)_{\bL^2(K_\ell)}\quad
\|v\|_h^2=\sum_{\ell\in\Ical_K}\|\grad v\|^2_{\bL^2(K_\ell)}.
\]
We set $\Pcal_h\bH^1=[\Pcal_hH^1]^d$, endowed with the scalar product~:
\[
(\vvec,\wvec)_h:=\sum_{\ell\in\Ical_K}(\Grad\vvec,\Grad\wvec)_{\bbL^2(K_\ell)}\quad
\|\vvec\|_h^2=\sum_{\ell\in\Ical_K}\|\Grad \vvec\|^2_{\bbL^2(K_\ell)}.
\]
Let $f\in\Ical_F^i$ such that $F_f=\pa K_L\cap\pa K_R$ and let $\nvec_f$ the unit normal that is outward $K_L$ oriented.\\
The jump (resp. average) of a function $v\in \Pcal_h H^1$ across the facet $F_f$, in $\nvec_f$ direction, is defined as follows: $[v]_{F_f}:=v_{|K_L}-v_{|K_R}$ (resp. $\{v\}_{F_f}:=\frac{1}{2}(v_{|K_L}+v_{|K_R})\,$). For $f\in\Ical_F^b$, we set: $[v]_{F_f}:=v_{|F_f}$ and $\{v\}_{F_f}:=v_{|F_f}$.\\
We set $\Pcal_h\bH(\dive)=\left\{\vvec\in \bL^2(\Omega)\,;\quad\forall\ell\in\Ical_K,\,\vvec_{|K_\ell}\in\bH(\dive;\,K_\ell)\right\}$, and we define the operator $\divh$ such that: 
\[
\forall\vvec\in\Pcal_h\bH(\dive),\,\forall q\in L^2(\Om),\quad
(\divh\vvec,q)=\sum_{\ell\in\Ical_K}(\dive\vvec,q)_{L^2(K_\ell)}.
\]
For all $D\subset\R^d$, and $k\in\N^*$, we call $P^k(D)$ the set of order $k$ polynomials on $D$, $\bP^k(D)=(P^k(D))^d$, and we consider the space of the broken polynomials:
\[
P^k_{disc}(\Tcal_h)=\left\{q\in L^2(\Om);\quad \forall\ell\in\Ical_K,\,q_{|K_\ell}\in P^k(K_\ell)\right\},\quad\bP^k_{disc}(\Tcal_h):=(P^k_{disc}(\Tcal_h))^d.
\]
We let $P^0(\Tcal_h)$ be the space of piecewise constant functions on $\Tcal_h$.% We now have the tools to describe our discretization, which comes from Crouzeix-Raviart nonconforming finite element method. We set:
\begin{equation}
\label{eq:Qhk}
\forall k \in \mathbb{N}, \quad Q_{k,h}:= P^k(\Th) \cap L^2_{zmv}(\Om)
\end{equation}
We will now describe three numerical scheme to solve \eqref{eq:Stokes} for which the components of the velocity is discretized with the first order nonconforming Crouzeix-Raviart finite element method \cite[\S 5, Example 4]{CrRa73}. For simplicity, we suppose now that $\fvec\in\bL^2(\Om)$.
%------------------------------------------------------%
\section{The $\bP^1_{nc}-P^0$ scheme}\label{sec:P1nc-P0}
%------------------------------------------------------%
The first order nonconforming finite element method was introduced by Crouzeix and Raviart in the seminal paper \cite{CrRa73} to solve Stokes Problem \eqref{eq:Stokes}. We call it the $\bP^1_{nc}-P^0$ scheme. Let us consider $X_{h}$ (resp. $X_{0,h}$), the space of nonconforming approximation of $H^1(\Om)$ (resp. $H^1_0(\Om)$) of order $1$:
\begin{equation}
\label{eq:Xh}
X_{h}=\left\{v_h\in P^1_{disc}(\Tcal_h)\,;\quad\forall f\in\Ical_F^i,\,\int_{F_f} [v_h]=0\right\}.
\end{equation}
\begin{equation}
\label{eq:X0h}
X_{0,h}=\left\{v_h\in X_{h} \,;\quad\forall f\in\Ical_F^b,\,\int_{F_f} [v_h]=0\right\}.
\end{equation}
%-----------%
\begin{prop}\label{pro:broknorm}
	The broken norm $v_h\rightarrow\|v_h\|_h$ is a norm over $X_{0,h}$. 
\end{prop}
%
The following discrete Poincaré–Steklov inequality holds \cite[Lemma 36.6]{ErGu21-II}: there exists a constant $C_{PS}^{nc}>0$ such that
\begin{equation}\label{eq:cps-constant}
\forall v_h\in X_{0,h},\quad\|v_h\|_{L^2(\Om)}\leq C_{PS}^{nc}\,\|v_h\|_h.
\end{equation}
The constant $C_{PS}^{nc}$ is independent of the triangulation $\Tcal_h$ and it is proportional to the diameter of $\Om$.
\\
We can endow $X_{0,h}$ with the basis $(\psi_f)_{f\in\Ical_F^i}$ such that: $\forall \ell\in\Ical_K$,
\[
\psi_{f|K_\ell}=\left\{\begin{array}{cl}1-d\lambda_{i,\ell}&\mbox{if }f\in\Ical_{F,\ell},\\ 0&\mbox{ otherwise,}\end{array}\right.
\]
where $S_i$ is the vertex opposite to $F_f$ in $K_\ell$.  We then have $\psi_{f|F_f}=1$, so that $[\psi_f]_{F_f}=0$ if $f\in\Ical_F^i$ (i.e. $F_f\in \Om$), and $\forall f'\neq f$, $\int_{F_{f'}}\psi_f=0$. We have: $X_{0,h}=\mathrm{vect}\left((\psi_f)_{f\in\Ical_F^i}\right)$.\\ 
The Crouzeix-Raviart interpolation operator $\pi_h$ for scalar functions is defined by:
\[
\pi_h: \left\{\begin{array}{rcl} H^1(\Om)&\rightarrow&X_h\\ v&\mapsto&\ds\sum_{f\in\Ical_F}\pi_fv\,\psi_{f}\end{array}\right.,\mbox { where }\pi_fv=\frac{1}{|F_f|}\int_{F_f} v.
\]
Notice that $\forall f\in\Ical_F$, $\int_{F_f} \pi_hv=\int_{F_f} v$. Moreover, the Crouzeix-Raviart interpolation operator preserves the constants, so that $\pi_h\ul{v}_\Om=\ul{v}_\Om$ where $\ul{v}_\Om=\int_\Om v/|\Om|$. %We recall the following result \cite[Lemma 2]{ApNS01}):

The space of nonconforming approximation $\bH^1_0(\Om)$ of order $1$ is $\bX_{0,h}=(X_{0,h})^d$. For a vector $\vvec\in\bH^1(\Om)$ of components $(v_{d'})_{d'=1}^d$, the Crouzeix-Raviart interpolation operator is such that: $\Pi_h\vvec=\left(\pi_hv_{d'}\right)_{d'=1}^d$. We recall the following result:
%-------------------------%
\begin{prop}\label{lem:PiCR}
	The Crouzeix-Raviart interpolation operator $\Pi_h$ can play the role of the Fortin operator:
	\begin{align}
	\label{eq:WellPosed-CR-1}
	\forall\vvec\in\bH^1(\Om)\hspace*{5mm}\|\Pi_h\vvec\|_h&\leq\|\Grad\vvec\|_{\bbL^2(\Om)},\\
	\label{eq:WellPosed-CR-2}
	\forall\vvec\in\bH^1(\Om)\quad(\divh\Pi_h\vvec,q_h)&=(\dive\vvec,q_h)_{L^2(\Om)},\quad\forall q\in Q_h.
	\end{align}
	Moreover, for all $\vvec\in\bP^1(\Om)$, $\Pi_h\vvec=\vvec$.
\end{prop}
%--------%
Notice that the stability constant of the bound on $\|\Pi_h\vvec\|_h$ is equal to $1$ \cite[Lemma 2]{Apel01}~: it is independent of the mesh.\\
Let us set $Q_h= Q_{0,h}$. We now define the following bilinear forms~:
\begin{equation}
\label{eq:DiscBilinForms}
a_{\nu,h}:\left\{\begin{array}{rcl}\bX_{0,h}\times\bX_{0,h}&\rightarrow&\R\\
(\uvec'_h, \vvec_h)&\mapsto&\nu\,(\uvec'_h, \vvec_h)_h
\end{array}\right.\mbox{ and }b_h:\left\{\begin{array}{rcl}\bX_{0,h}\times Q_h&\rightarrow&\R\\
(\vvec_h, q_h)&\mapsto&-(\divh\vvec_h, q_h)
\end{array}\right..
\end{equation}
We suppose here that $\fvec\in\bL^2(\Om)$. The discretization of variational formulation \eqref{eq:StokesVF} reads:
\begin{equation}
\label{eq:StokesVFh}
\mbox{Find }(\uvec_h, p_h) \in\bX_{0,h}\times Q_h\,|\,\left\{
\begin{array}{rcll}
a_{\nu,h}(\uvec_h, \vvec_h)_h+b_h(\vvec_h,p)&=&(\fvec, \vvec_h)_{L^2(\Om)} & \forall \vvec_h \in  \bX_{0,h},\\
b_h(\uvec_h,q_h)&=&0 &\forall q_h \in Q_h.
\end{array}
\right.
\end{equation}
This saddle point problem is well-posed. Indeed, the bilinear form $a_{\nu,h}(\cdot,\cdot)$ is continuous and coercive. Moreover, the bilinear form $b_h(\cdot,\cdot)$ is continuous and due to Proposition \ref{prop:diviso} and Lemma \ref{lem:PiCR}, it satisfies the following discrete inf-sup condition:
\begin{equation}
\label{eq:CISh}
\forall q_h\in Q_h\backslash\{0\},\quad\exists\,\vvec_h,q\in\bX_{0,h}\backslash\{0\}\,|\quad\frac{b_h(\vvec_h,q_h)}{\|\vvec_h\|_h\,\|q_h\|_{L^2(\Om)}}\geq C_{\dive}.
\end{equation}
Suppose that it exists $\phi\in H^1(\Om)\cap L^2_{zmv}(\Om)$ such that $\fvec=\grad\phi$. In that case, the solution to Problem \eqref{eq:Stokes} is $(\uvec,p)=(0,\phi)$. By integrating by parts, we have:
\[
\forall\vvec_h\in\bX_{0,h},\quad
(\fvec, \vvec_h)_{L^2(\Om)}=-(\divh\vvec_h,\phi)+\sum_{f\in\Ical_F^i}\int_{F_f}[\vvec_h\cdot\nvec_f]\,\phi.
\]
The term with the jump acts as a numerical source, which numerical influence is proportional to $1/\nu$. Hence, we cannot obtain exactly $\uvec_h=0$. There are different strategies to cure this well-known problem:
\begin{itemize}
	\item Using a polynomial approximation of higher degree \cite{FoSo83}.
	\item Increasing the space of the discrete pressures \cite{Heib03,Fort06}.
	\item Projecting the test-function on a discrete subspace of $\bH(\dive;\Om)$ \cite{Link14}.
\end{itemize}
We propose in the next section to give details on the second strategy.
%--------------------------------------------------------------%
\section{The $\bP^1_{nc}-(P^0+P^1)$ scheme}\label{sec:P1nc-P0P1}
%--------------------------------------------------------------%
In his thesis \cite{Heib03}, Heib proposed to use the following space discrete pressures space (cf. \eqref{eq:Qhk}):
\begin{equation}\label{eq:Qh}
\tilde{Q}_h=Q_{0,h}\oplus Q_{1,h}.
\end{equation}
For any $\tilde{q}_h\in\tilde{Q}_h$, we write: $\tilde{q}_h=q_{0,h}+q_{1,h}$, where $q_{0,h}\in Q_{0,h}$ and $q_{1,h}\in Q_{1,h}$. \\
Let us consider the following bilinear form:
\begin{equation}
\label{eq:btilde}
\tilde{b}_h:\left\{
\begin{array}{rcl}\bX\times \tilde{Q}_h&\rightarrow&\R \\ 
(\vvec_h,\tilde{q}_h)&\mapsto&-(\divh\vvec_h,q_{0,h})+(\vvec_h,\grad q_{1,h})_{\bL^2(\Om)}
\end{array}
\right..
\end{equation}
The discretization of Problem \eqref{eq:StokesVF} with $\bP^1_{nc}-(P^0+P^1)$ finite elements reads:
\begin{equation}
\label{eq:StokesVFhTrio}
\mbox{Find }(\uvec_h, p_h) \in\bX_{0,h}\times \tilde{Q}_h\,|\,\left\{
\begin{array}{rcll}
a_{\nu,h}(\uvec_h, \vvec_h)_h+\tilde{b}_h(\vvec_h,p_h)&=&(\fvec, \vvec_h)_{L^2(\Om)} & \forall \vvec_h \in  \bX_{0,h},\\
\tilde{b}_h(\uvec_h,\tilde{q}_h)&=&0 &\forall \tilde{q}_h \in \tilde{Q}_h.
\end{array}
\right.
\end{equation}
We will need the following Hypothesis \cite[Hyp.~4.1]{BeHe00}:
\begin{hypo}\label{hyp:cis}
	We suppose that the triangulation $\Tcal_h$ is such that the boundary $\pa\Om$ contains at most one edge in dimension $d=2$ and at most two faces in dimension $d=3$, of the same element $K_\ell$, $\ell\in\Ical_K$.
\end{hypo}
Under Hypothesis \eqref{hyp:cis}, one can prove that the bilinear form $\tilde{b}_h(\cdot,\cdot)$ is continuous that it satisfies the following discrete inf-sup condition \cite[\S 4.2]{Heib03}:
\begin{equation}
\label{eq:CIShTrio}
\forall \tilde{q}_h\in \tilde{Q}_h\backslash\{0\},\quad\exists\,\vvec_h\in\bX_{0,h}\backslash\{0\}\,|\quad\frac{\tilde{b}_h(\vvec_{h,\tilde{q}_h},\tilde{q}_h)}{\|\vvec_{h,\tilde{q}_h}\|_h\,\|\tilde{q}_h\|_{L^2(\Om)}}\geq \tilde{C}_{\dive},
\end{equation}
where the constant $\tilde{C}_{\dive}$ is independent of the mesh size. Compared to $\bP^1_{nc}-P^0$ scheme, the $\bP^1_{nc}-(P^1+P^0)$ scheme gives a better approximation of the velocity in the sense that the discrete mass conservation equation is strengthened. Indeed, one can show, for $d=2$ that \cite[Theorem 4.3.2]{Fort06}:
\begin{prty}\label{thm:P1nc-P0P1}
	Let $\vvec_h\in\bV_h:=\{\wvec_h\in\bX_h\,|\quad\forall q_h\in\tilde{Q}_h,\quad\tilde{b}_h(\wvec_h,q_h)=0\}$. \\
	Then for $d=2$, we have: for all $q_{2,h}\in Q_{2,h}$, $\tilde{b}_h(\wvec_h,q_{2,h})=(\grad q_{2,h},\vvec_h)_{\bL^2(\Om)}=0$.
\end{prty}
The proof of Property \ref{thm:P1nc-P0P1} relies on a quadrature formula which uses the degrees of freedom of the discrete pressure. As this formula cannot be extended in $3D$, this property does not hold. To recover Property \ref{thm:P1nc-P0P1} in $3D$, we must introduce $P^2$ discrete pressure degrees of freedom, located on the edges of the mesh, as detailed in \cite{Fort06}. This increases the number of unknowns by the number of cells, which leads to an expensive linear system. Hence, we look for a numerical scheme which could be as precise in $3D$ than in $2D$, but at a lower cost. In the next section, we propose a new strategy, which relies on the multi-points flux approximation to discretize the pressure gradient term in \eqref{eq:Stokes}.

