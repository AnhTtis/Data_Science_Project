\usepackage[utf8]{inputenc} % police encodee en UTF-8 %
%\usepackage[british,UKenglish,USenglish,american]{babel}
\usepackage{amsmath,amssymb}
\usepackage{tikz}
\usepackage{pgfplots} % permet de tracer des figures 
\usepackage{graphicx,subcaption}
\graphicspath{{./figs/}{./}{./Images/}}
\usetikzlibrary{patterns}
\usepackage{dsfont}
\usepackage{xcolor}
\usepackage{csquotes} % AVOID A WARNING WITH BABEL 
%Biblio:
\usepackage[backend=biber,style=numeric,sorting=none,language=french,maxbibnames=99]{biblatex}
\addbibresource{22-P1ncMPFA.bib}

\usepackage{url}            % Pour citer les adresses web
\usepackage[T1]{fontenc}    % Encodage des accents
\usepackage[utf8]{inputenc} % Lui aussi
% \usepackage[english]{babel} % Pour la traduction française
% \usepackage{numprint}       % Histoire que les chiffres soient bien

\usepackage{amsmath}        % La base pour les maths
\usepackage{mathrsfs}       % Quelques symboles supplémentaires
\usepackage{amssymb}        % encore des symboles.
\usepackage{amsfonts}       % Des fontes, eg pour \mathbb.
\usepackage{cancel}

%\usepackage[svgnames]{xcolor} % De la couleur

%%% Si jamais vous voulez changer de police: décommentez les trois 
%\usepackage{tgpagella}
%\usepackage{tgadventor}
%\usepackage{inconsolata}



\usepackage{graphicx} % inclusion des graphiques
\usepackage{wrapfig}  % Dessins dans le texte.

\usepackage{tikz}     % Un package pour les dessins (utilisé pour l'environnement {code})
\usepackage[framemethod=TikZ]{mdframed}


%\usepackage{subfigure} pas compatible avec subcaption
\usepackage{color}
\usepackage{dsfont}
\usepackage{bbm}
\usepackage{bm}


\usepackage{transparent} % permet de faire du transparant


\usepackage[retain-zero-exponent=true]{siunitx} % Ecriture scientifique 

