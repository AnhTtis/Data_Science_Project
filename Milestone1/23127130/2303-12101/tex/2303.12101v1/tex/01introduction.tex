\section{Introduction}\label{sec:introduction}

% HII regions in the context of the baryon cycle
The formation of stars on galactic scales is a continuous cycle in which material from previous generations is recycled into new stars. 
This so-called \emph{baryon cycle} is regulated by the feedback from massive ($>\SI{8}{\Msun}$), short-lived stars \citep{Hopkins+2014,Kim+2017}. 
They produce ultraviolet (UV) radiation that ionizes the surrounding gas, forming \HII regions. Stellar winds and supernovae will further deposit energy in the cloud and enrich the gas, but the latter also mark the end of the short life \citep[\SIrange{3}{30}{\mega\year};][]{Ekstroem+2012} of the most massive O and B stars. 
If these feedback mechanisms are strong enough to overcome gravity, they are able to disperse the host cloud \citep{Dale+2014,Rahner+2017,Haid+2018,Kim+2018,Kruijssen+2019,Chevance+2022}, but if not, the \HII region will cease to exist once the last B stars explode. 
% evolution time scale is ill-constrained
The exact ages of \HII regions are difficult to pin down, as line emission can vary as a function of multiple local physical conditions (age, but also e.g.\ metallicity or density). 
One approach is to determine the ages of underlying star clusters \citep[e.g.\ ][]{Whitmore+2011,Hollyhead+2015,Hannon+2019,Hannon+2022,Stevance+2020}, which can be estimated by fitting the observed \emph{spectral energy distribution} (SED) with theoretical models \citep[e.g.][]{Turner+2021}. 
Direct constraints from ionized nebulae on the other hand are rather rare, and existing studies are almost exclusively on $\si{\kilo\parsec}$ scales. One possibility is to use the $\HA$ \emph{equivalent width} $\EW$ \citep{Dottori+1981,Copetti+1986,Fernandes+2003,Levesque+2013} or the $\HA/\FUV$ ratio. 
Both fluxes are extensively used, most commonly as tracers for star formation \citep{Lee+2009} or to constrain the initial mass function \citep{Meurer+2009, Hermanowicz+2013}, but they have also been used on cloud scales as age indicators \citep[e.g.][]{SanchezGil+2011,Faesi+2014}.
$\HA$ is only created by the emission of the most massive stars in the stellar population. 
The stellar continuum in the $\FUV$ and underlying $\HA$, on the other hand, have a significant contribution from lower mass stars. 
Hence, once the most massive stars are gone, both $\EW$ and $\HA/\FUV$ start to decline. 
Assuming an instantaneous burst of star formation, population synthesis models like \textsc{starburst99} \citep{Leitherer+2014} or \textsc{bpass} \citep{Eldridge+2009}, the latter in combination with the photoionization model \textsc{cloudy} \citep{Ferland+2017}, also predict that after \SIrange{2}{3}{\mega\year} the ratio decreases monotonically with the age of the cluster.

Another potential age tracer is the \emph{ionization parameter} $q$, defined as the ratio of the incident ionizing photon flux $\Phi_\mathrm{H^0}$ to the local hydrogen density $n_\mathrm{H}$ \citep{Kewley+2002}. 
In the case of a spherical geometry, one can write the ionization parameter as
\begin{equation}
    q = \frac{\Phi_\mathrm{H^0}}{n_\mathrm{H}} = \frac{Q_\mathrm{H^0}}{4\pi R^2 n_\mathrm{H}},
\end{equation}  
where $Q_\mathrm{H^0}$ is the rate of ionizing photons striking the cloud at a distance $R$. Assuming the model of a partially-filled \emph{Strömgren sphere} \citep{Charlot+2001}, this becomes
\begin{equation}\label{eq:ionisation_stroemgren}
    q = \left( \alpha_\mathrm{B} \epsilon \right)^{2/3} \left( \frac{3\,Q_\mathrm{H^0}\, n_\mathrm{H}}{4\pi} \right)^{1/3},
\end{equation}
where $\alpha_\mathrm{B}$ is the case-B hydrogen recombination coefficient and $\epsilon$ is the volume-filling factor of the gas. 
As the stellar population ages, the rate at which it produces ionizing photons decreases \citep{Smith+2002} and so does the density as the \HII region expands (if the shell does not sweep up additional material), leading to a decrease of the ionization parameter over time. 
\citet{Dopita+2006} present a more realistic scenario and also conclude that the ionization parameter decreases as a function of time, with secondary dependencies on the ambient pressure, the cluster mass and the metallicity. 
Unlike the other two quantities, $\log q$ can not be measured directly, but the $\SIII/\SII$ ratio is a good proxy for it \citep{Diaz+1991}.

The physical conditions of the interstellar medium (ISM, e.g.\ metallicity, ionization parameter, density) regulate future star formation, and can be diagnosed via optical emission line ratios \citep{Kewley+2019b}. 
However some ratios can be sensitive to multiple properties, and careful work is needed to break those degeneracies  \citep{Kewley+2002,Kewley+2019b}. 
One particular issue is the relation between the chemical abundance and the ionization parameter. 
Some studies find an anti-correlation between the two \citep[e.g.][]{PerezMontero+2014,EspinosaPonce+2022}, while others find a positive correlation \citep[e.g.][]{Kreckel+2019,Grasha+2022}. 
This discrepancy is sometimes attributed to the resolution of the observations \citep{Kewley+2019b}, the type of galaxy that is studied \citep[star-forming or not,][]{Dopita+2014} or  to the underlying photoionization models that are used to measure the metallicity and ionization parameter \citep{Ji+2022}.

With an established age sequence, it becomes possible to directly track how conditions in the ISM evolve as a result of stellar feedback processes. 
However, it is challenging to identify an ionizing source for each \HII region and the sample for which we are able to is therefore biased. 
Direct age constraints from the ionized gas would allow us to include \HII regions without an ionizing source and hence study an unbiased populations.
With this, it becomes possible to study the evolutionary timescales, using large representative samples of \HII regions to probe the different stages in their evolution.
Modern integral field unit (IFU) spectroscopic surveys enable us to isolate individual \HII regions at $<\SI{100}{\parsec}$ scale and observe thousands of \HII regions across individual galaxies. 
Previous work has typically focused on detailed case studies of individual galaxies \citep{Niederhofer+2016,McLeod+2020,McLeod+2021,DellaBruna+2021}. 
In order to investigate the dependence on galactic properties like environments (morphological features), metallicities or star formation rates, a more comprehensive sample is required. 
The Physics at High Angular resolution in Nearby GalaxieS (PHANGS)\footnote{\url{http://www.phangs.org}} collaboration studies star formation in nearby galaxies, using large samples of giant molecular clouds (GMCs), \HII regions and star clusters. 
It combines observations from ALMA \citep{Leroy+2021b}, MUSE \citep{Emsellem+2022}, \textit{HST} \citep{Lee+2022} and \textit{JWST} \citep{Lee+2023} for a sample of 19 nearby galaxies.
This provides an unprecedented sample that allows us to study the ionized gas in \HII regions together with the massive stars that ionize them. 
We produce for the first time a catalogue of cross-matched ionizing sources and ionized nebulae, well suited for addressing how  stellar feedback evolves (empirically) and constraining stellar feedback models.

This paper is organised as follows: in \cref{sec:data}, we present the data and the existing catalogues that are used in the analysis. 
In \cref{sec:matched_catalogue} we match the \HII regions to their ionizing sources. 
In \cref{sec:evolutionary_sequence} we establish the \HII region evolutionary sequence and discuss our findings and conclude in \cref{sec:conclusion}.
