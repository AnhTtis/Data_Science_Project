% Appendix

\section{Model predictions}\label{sec:model_predictions}

In \cref{fig:age_vs_Ha_over_FUV} we show the age evolution of $\HA/\FUV$ and $\EW$ as predicted by \textsc{starburst99} \citep{Leitherer+2014}, following a \citet{Kroupa+2001} IMF.
Independent of metallicity, we see a plateau for the first $\sim\kern-2pt\SI{3}{\mega\year}$ after which both the $\HA/\FUV$ and the $\EW$ decrease almost monotonically. 

\begin{figure}
    \centering
    \includegraphics{age_vs_Ha_over_FUV}
    \caption{$\HA/\FUV$ flux ratio and $\EW$ as a function of cluster age for different metallicities as predicted by \textsc{starburst99}.}
    \label{fig:age_vs_Ha_over_FUV}
\end{figure}


\section[Background corrected EW]{Background corrected $\mathrm{EW}$}\label{sec:bkg_EW}
We provide the raw and background corrected $\EW$ and $\mathrm{EW}(\HB)$ for the full nebula catalogue from \citet{Groves+2023}. 
The fluxes are measured in the rest frame wavelength intervals listed in \cref{tbl:ew_intervals}, following the procedure from \citet{Westfall+2019}.
To account for the contribution of older stars in the stellar disk, we subtracted the background from an annulus with three times the size of the \HII region. 
For roughly one third of the \HII regions, the background subtraction is complicated by neighbouring \HII regions, and we mask pixels that fall in other nebulae. 
Because the continuum is relatively smooth, this should not be an issue, except for the $\SI{0.5}{\percent}$ \HII regions that are completely surrounded by other nebulae. 
Those objects are hence excluded from the analysis of the $\mathrm{EW}$.

\begin{table}
    \centering
    \caption{Intervals for the equivalent width measurement.}
    \begin{tabular}{lll}\toprule
         & \multicolumn{1}{c}{$\HA$} & \multicolumn{1}{c}{$\HB$} \\\midrule
       Line  & \SIrange{6557.6}{6571.35}{\angstrom} &  \SIrange{4847.9}{4876.6}{\angstrom} \\
       Continuum low  & \SIrange{6483.0}{6513.0}{\angstrom} & \SIrange{4827.9}{4847.9}{\angstrom} \\
       Continuum high  & \SIrange{6623.0}{6653.0}{\angstrom} & \SIrange{4876.6}{4891.6}{\angstrom} \\\bottomrule
    \end{tabular}
    \label{tbl:ew_intervals}
\end{table}


\section{Density and temperature}\label{sec:CrossTemDen}

As discussed in \cref{sec:data}, we assume a fixed temperature of $\SI{8000}{\kelvin}$ to derive the density.
To validate this assumption, we compare the values derived with a fixed temperature versus those fitted simultaneously with the temperature.
Auroral lines are faint, and only \num{840} \HII regions are detected with a $\StoN>10$ in the temperature sensitive $\NII[5754]$ line.
For this sub-sample, we use \textsc{pyneb} \citep{Luridiana+2015} to derive the electron density and the temperature simultaneously from the $\SII[6731/6717]$ and $\NII[5755/6548]$ ratios.
We also derive the density with a fixed temperature of $\SI{8000}{\kelvin}$ and compare the two in \cref{fig:getCrossTemDen}.
As evident in the figure, the small variations in temperature do not affect the measured density significantly. 
While the majority of these \HII regions are significantly different from the low density limit, the sample is only half the size of the full catalogue. 
Due to the large difference in sample size and the small difference in the derived density we opt to use the densities derived with a fixed temperature in our analysis.

\begin{figure}
    \centering
    \includegraphics[width=0.8\columnwidth]{fig/density_getCrossTemDen}
    \caption{Comparison between the densities derived at a fixed temperature of $\SI{8000}{\kelvin}$ and those derived by fitting density and temperature simultaneously. The grey points are indistinguishable from the low-density limit.}
    \label{fig:getCrossTemDen}
\end{figure}


\section[Objects in unmatched HII regions]{Objects in unmatched \HII regions}\label{sec:single_stars}

As stated in \cref{sec:unmatched_HII_regions}, almost \num{9000} \HII regions do neither contain a stellar association nor a compact star cluster. 
However, we do find a \textsc{dolphot} peak in $\SI{63.7}{\percent}$ of them.
In \cref{fig:dolphot_peaks_stellar_models}, we compare the $\HA$ luminosity of those unmatched \HII regions with the \textit{V}-band magnitudes of the \textsc{dolphot} peaks and overplot models for single stars from \citet{Martins+2005}. 
We find that most unmatched \HII regions fall in a regime where they could be ionized by a single massive star.

\begin{figure}
    \centering
    \includegraphics{dolphot_peaks_stellar_models}
    \caption{$\HA$ luminosity and absolute \textit{V} band magnitude for the \textsc{dolphot} peaks that are matched to our previously unmatched \HII regions. Overplotted are models for different class O-type stars from \citet{Martins+2005}. The ionizing photon flux is converted to an $\HA$ luminosity via $Q(\HA)=\num{7.31e11} L(\HA)\, \si{\per \second}$ \citep{Kennicutt+1998a}. The solid lines shows their theoretical calibration of $T_\mathrm{eff}$ and the dashed lines the observational calibration. The insert shows a colour-colour diagram of the sample with the track of a single stellar population at solar metallicity for reference \citep{Bruzual+2003}.}
    \label{fig:dolphot_peaks_stellar_models}
\end{figure}


\section{Binned age histograms}\label{sec:age_hist}

\Cref{fig:age_hist,fig:age_hist_eq_width} show the distribution of stellar association ages for different sub-samples. 
In \cref{fig:age_hist} we separate the sample based on the overlap with the \HII regions. We find that associations that overlap with an \HII region are significantly younger than those that are isolated. 
In \cref{fig:age_hist_eq_width} we separate the sample based on the first, second and third percentile in the un-corrected $\EW$. 
The sample with the highest $\EW$ has on average the lowest SED ages.

% --- age histograms ---
\begin{figure*}
    \centering
    \includegraphics[width=\textwidth]{age_hist_contained}
    \caption{Age distribution of the associations with \texttt{isolated}, \texttt{partial} and \texttt{contained} overlap (only the first $\SI{10}{\mega\year}$ are shown here). For each subsample we show the median age with the uncertainty taken from the $\SI{68}{\percent}$ interval of the data and the number of young ($\leq\SI{2}{\mega\year}$), intermediate (between $\SI{2}{\mega\year}$ and $\SI{10}{\mega\year}$) and old ($>\SI{10}{\mega\year}$) associations. We only include the \num{6027} associations that are more massive than $\SI{e4}{\Msun}$. }
    \label{fig:age_hist}
\end{figure*}

\begin{figure*}
    \centering
    \includegraphics[width=\textwidth]{age_hist_eq_width}
    \caption{Age distribution, similar to \cref{fig:age_hist}, but based on the $\EW$. We only include associations that are more massive than $\SI{e4}{\Msun}$ and apply the same $\StoN$ cut described in \cref{sec:correlation_sed}, leaving us with \num{854} objects.}
    \label{fig:age_hist_eq_width}
\end{figure*}


%\begin{figure}
%    \centering
%    \includegraphics{fig/tmp_mass_HA}
%    \caption{$\HA$ luminosity as a function of cluster mass. The associations are binned by cluster age.}
%    \label{fig:mass_ha_age}
%\end{figure}

