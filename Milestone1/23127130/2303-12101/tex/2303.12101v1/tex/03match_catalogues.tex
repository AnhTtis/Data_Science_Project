\section{Matching nebulae and stellar associations}
\label{sec:matched_catalogue}

To relate the \HII regions to the source of the ionizing radiation, we match the nebula catalogue with the association catalogue. 
Both have spatial masks, but with different pixel scales ($\SI{0.2}{\arcsec}$ per pixel for MUSE and $\SI{0.04}{\arcsec}$ per pixel for \textit{HST}). 
The coarser resolution of the nebula masks can result in some larger \HII regions that are not correctly partitioned \citep[see][]{Barnes+2022}. 
However we would still correctly map them to their ionizing sources.
The absolute astrometric accuracy is excellent for both data sets. 
\textit{HST} achieves $0.1$ pixel accuracy while MUSE reaches $0.5$ pixel, corresponding to two and a half \textit{HST} pixels.
Considering the uncertainties in deriving the exact boundaries of the \HII regions, this is negligible when matching the datasets.
We reproject the nebula masks to the association masks and determine the fractional overlap as measured in terms of the association or nebula pixel areas. 
In \cref{fig:overlap} we showcase examples for the overlap between the \HII regions and associations. 
For each of the stellar associations we define the \texttt{overlap} to be one of the following three:
\begin{itemize}[leftmargin=*]
    \item \texttt{isolated}: the association does not overlap with any nebula (fractional overlap of $\SI{0}{\percent}$). 
    \item \texttt{partial}: part of the association overlaps with one or more nebula, but part of it extends beyond it (fractional overlap between $0$ and $\SI{100}{\percent}$). 
    \item \texttt{contained}: the association is fully contained in the nebulae. It can be contained within multiple nebulae (fractional overlap of $\SI{100}{\percent}$).
 \end{itemize}
 For the \HII regions we determine the following properties:
 \begin{itemize}[leftmargin=*]
    \item \texttt{neighbors}: number of neighbouring \HII regions (i.e.\ regions that share a common boundary)
    \item \texttt{Nassoc}: number of stellar associations that overlap with the \HII region. 
\end{itemize}
Finally, we construct a catalogue of matched objects where each \HII region and stellar association overlap with exactly one stellar association and \HII region respectively. 
It contains \num{4177} objects that constitute the parent sample that we will be working with and to which we refer to as the \texttt{one-to-one sample}. This catalogue is available in the supplementary online material and incorporates a number of columns from the nebula and association catalogues. \Cref{tbl:matched_catalogue_columns} gives an overview of the properties that are included.

\begin{figure}
\centering
\includegraphics[width=\columnwidth]{overlap_rgb}
\caption{Examples for the overlap between the \HII regions and stellar associations in \galaxyname{NGC}{1365}. The cutouts show a three colour composite images, based on the 5 available \textit{HST} bands, overlaid with the $\HA$ line emission of MUSE in red. The boundaries of the \HII regions are shown in red and the stellar associations in blue. Flags that characterise the overlap between the two catalogues are showcased in this figure.} 
\label{fig:overlap}
\end{figure}

\begin{table}
    \centering
    \caption{Columns of the matched catalogue (\texttt{one-to-one sample}).}
    \begin{tabular}{ll}\toprule
        Column & Description  \\\midrule
        \verb|gal_name| & Name of the Galaxy \\
        \verb|region_ID| & ID of the \HII region \\
        \verb|assoc_ID| & ID of the stellar association \\
        \verb|ra_neb| & Right ascension of the nebulae \\
        \verb|dec_neb| & Declination of the nebulae \\
        \verb|ra_asc| & Right ascension of the stellar association \\
        \verb|dec_asc| & Declination of the stellar association \\ 
        \verb|overlap_neb| & Overlap percentage with stellar association  \\
        \verb|overlap_asc| & Overlap percentage with \HII region \\
        \verb|overlap| & Flag for overlap (isolated, partial, contained) \\
        \verb|environment| & Galactic environment (e.g.~bar,centre,disc)\\
        \verb|neighbors| & Number of neighbouring \HII regions \\  
        \verb|HA6562_lum|$^\dag$ & Extinction corrected $\HA[6562]$ luminosity \\
        \verb|EBV_balmer|$^\dag$ & Colour excess from the Balmer decrement \\
        \verb|EBV_stellar|$^\dag$ & Colour excess from the SED fit \\
        \verb|age|$^\dag$ & Age of the stellar association in $\si{\mega\year}$ \\
        \verb|mass|$^\dag$ & Mass of the stellar association in $\si{\Msun}$  \\
        \verb|{filter}_flux|$^{\dag\ddag}$ & Flux in \textit{HST} band in mJy \\
        \verb|Ha/FUV|$^\dag$ & $\HA$ to $\FUV$ flux ratio (extinction corrected) \\
        \verb|EW_HA|$^\dag$ & $\HA$ equivalent width in $\si{\angstrom}$ \\
        \verb|EW_HA_corr|$^\dag$ & Background corrected $\EW$ in $\si{\angstrom}$ \\
        %\verb|[SIII]/[SII]|$^\dag$ & $\SIII/\SII$ line ratio \\
        \verb|logq|$^\dag$ & Ionisation parameter $\log q$\\
        \verb|Delta_met_scal| & Local metallicity offset $\DeltaOH$ \\
        \verb|density|$^\dag$ & Electron density in $\si{\per\cm\cubed}$ \\
        \verb|GMC_sep| & Distance to nearest GMC in $\si{\parsec}$ \\
        \bottomrule\addlinespace
        \multicolumn{2}{l}{${}^\dag$ associated errors are included as \texttt{*\_err}.}\\
        \multicolumn{2}{l}{${}^\ddag$ for \texttt{filter} in \textit{NUV}, \textit{U}, \textit{B}, \textit{V} and \textit{I}.}
    \end{tabular}
    \label{tbl:matched_catalogue_columns}
\end{table}

\subsection{Statistics of the matched catalogue}

% like Table 1, this table only includes associations and HII regions that fall in the FoV of MUSE and HST respectively. 
\begin{table}
\centering
\caption{Level of spatial correlation between \HII regions and associations. We list the total number and percentage of \HII regions (associations) that overlap with $N$ associations (\HII regions) respectively.}
\begin{tabular}{r
                S[table-format=5]
                r
                S[table-format=5]
                r
                }
\toprule
$N$ & \multicolumn{2}{c}{\HII regions} & \multicolumn{2}{c}{Stellar associations} \\
\midrule
0 & 11687 & (57.5\%) & 4596 & (28.2\%) \\
1 & 5673 & (27.9\%) & 9288 & (57.0\%) \\
2 & 1702 & (8.4\%) & 2011 & (12.3\%) \\
3 & 643 & (3.2\%) & 319 & (2.0\%) \\
4 & 291 & (1.4\%) & 57 & (0.3\%) \\
>5 & 345 & (1.7\%) & 24 & (0.1\%) \\
\bottomrule
\end{tabular}
\label{tbl:overlap_statistics}
\end{table}

From the initial nebula catalogue, \num{20341} \HII regions fall inside the \textit{HST} FoV and \num{8654} of them ($\SI{42.5}{\percent}$) overlap with at least one stellar association. 
Conversely, the association catalogue contains \num{16295} associations in the MUSE FoV and \num{11699} of them ($\SI{71.8}{\percent}$) overlap with at least one \HII region. 
\cref{tbl:overlap_statistics} provides a detailed breakdown of how the \HII regions and associations overlap. 

In cases where a nebula contains multiple associations, it is difficult to assess the contribution of the individual objects and ambiguous to assign a single age, as they are not necessarily coeval \citep{Efremov+1998}. 
In fact, when we find multiple stellar associations in one \HII region, they rarely have the same age. 
The age spread is usually small ($<\SI{10}{\mega\year}$), but there are also more extreme cases, with $\sim\kern-2pt\SI{25}{\percent}$ having larger age spreads.
Therefore, to simplify our analysis, in this paper we require a one-to-one match (this can be with either \texttt{partial} or \texttt{contained} overlap) between the associations and \HII regions, leaving us with a sample of \num{4177} objects in our parent \texttt{one-to-one sample} (see \cref{tbl:sample} for a detail breakdown by galaxy). 
Depending on the application, we also apply further cuts in age, mass or overlap to obtain cleaner samples. 

The nebula masks are dense and cover most of the spiral arms and hence it is likely that some of the overlaps are coincidental. 
To estimate how many of our matches this might concern, we run a test where we rotate the full set of nebula masks by $\SI{90}{\deg}$ around the galaxy centre, before matching with the associations. 
While previously one third of the \HII regions overlapped with an association, this number drops to one sixth. 
The number of \texttt{partial} overlapped associations decreases only slightly to $\SI{81}{\percent}$. 
The difference is more apparent with the \texttt{contained} sample: we estimate that only around $\SI{16}{\percent}$ of the \texttt{contained} associations could be chance alignments.
This motivates us to only use the \texttt{contained} objects for our analysis, reducing the sample to \num{1918} \HII regions and stellar associations.

Another concern is stochastic sampling of the IMF \citep{Fouesneau+2010,Hannon+2019}. 
Below a certain cluster mass, the presence or absence of a single massive star can have a significant impact on the observed colours and hence the derived properties of the association, as well as on whether it is able to ionize hydrogen and has an associated detectable \HII region. 
We assume that stellar associations that are more massive than $\SI{e4}{\Msun}$ will fully sample the IMF \citep{daSilva+2012}.
We also do not expect significant $\HA$ emission to be associated with old stellar populations and therefore introduce an age cut at $\SI{8}{\mega\year}$.
The distribution of masses and ages of the matched catalogue is shown in \cref{fig:catalogue_properties_2D_hist_v2}, with both cuts illustrated by black lines. 
Note that the gap at \SIrange{2}{3}{\mega\year} is inherited from the stellar association catalogue, and reflects specific features in the stellar population tracks \citep{Larson+2023}. 
In total, \num{1041} objects pass the mass cut and \num{3531} pass the age cut. 
Applying both of these criteria, and further requiring the stellar associations to be \texttt{contained}, leaves \num{469} objects, constituting our \texttt{robust sample}.

\begin{figure}
    \centering
    \includegraphics{catalogue_properties_2D_hist}
    \caption{Distribution of masses and ages of the stellar associations in the matched catalogue (the \texttt{one-to-one} sample). The black lines mark the cuts that we apply to the sample to ensure a fully sampled IMF (more massive than $>\SI{e4}{\Msun}$) and to only include young clusters that should be associated with ionized gas (younger than $\leq\SI{8}{\mega\year}$). The mass cut leaves us with a sample of \num{1014} objects and the age cut with \num{3531}. Applying both cuts results in a sample of \num{756} objects.}
    \label{fig:catalogue_properties_2D_hist_v2}
\end{figure}

\subsection[The physical nature of HII regions without associations]{The physical nature of \HII regions without associations}
\label{sec:unmatched_HII_regions}

\begin{figure}
    \centering
    \includegraphics{Halpha_luminosity_function}
    \caption{$\HA$ luminosity function of the \HII regions that overlap with one or more associations and those that do not. The matched \HII regions are on average a factor 10 brighter than the unmatched ones.}
    \label{fig:luminosity_function}
\end{figure}

% outline for the argument
% association catalogue by design does not include single peaks
% those objects can appear in compact cluster catalogue
% V band is used for detection: not all ionising sources are detected
% concentration index removes compact sources (single stars or clusters dominated by a single star).
% there is no compact cluster catalogue for NGC4254 and NGC1300

As shown in \cref{tbl:overlap_statistics}, the majority of our \HII regions do not overlap with an association, which raises the question of the origin of their ionizing radiation. 
We consider several scenarios to explain this discrepancy. 
First of all, the unmatched \HII regions could host deeply embedded, highly extincted stars that we are unable to observe.
However their reddening distribution is almost identical to the matched \HII regions with a mean of $E(B-V)_\mathrm{Balmer}=0.26$, making it unlikely that we miss many objects due to extinction.

Secondly, we restricted our analysis to the $\SI{32}{\parsec}$ scale associations, but there are some variations between the different scales and $\SI{20.6}{\percent}$ of the unmatched \HII regions overlap with an $\SI{16}{\parsec}$ or $\SI{64}{\parsec}$ association. By design, the stellar association catalogue only includes objects with multiple peaks, excluding other ionizing sources like compact star clusters or individual stars. 
Looking at the $\HA$ luminosity function in \cref{fig:luminosity_function}, we see that \HII regions that are matched to an association are on average brighter by a factor of 10 compared to those that are unmatched, meaning that the unmatched regions are ionized by fainter sources.
We consider the previously mentioned PHANGS--\textit{HST} compact cluster by \citep[for example, the unmatched \HII region in \cref{fig:overlap} contains an object that is classified as a compact cluster]{Thilker+2022}. 
Using the machine learning classified catalogue \citetext{\citealp{Wei+2020}; Hannon et al.\ in preparation}, we find that \num{2569} of all \HII regions contain a compact cluster ($\SI{15.3}{\percent}$, comparable to the $\SI{13}{\percent}$ reported by \citealt{DellaBruna+2022b} for \galaxyname{M}{83}). 
The majority of these clusters are already contained in an association and only $\SI{6.1}{\percent}$ of the previously unmatched \HII regions contain a compact cluster. 
By design, the compact cluster identification pipeline disfavours the selection of looser association-like structures that are young. 
Considering that we are primarily interested in young and massive clusters, the number of potentially interesting compact clusters is only of the order of a few dozen.

Even after accounting for different scale associations and compact clusters, $\SI{43.5}{\percent}$ of the \HII regions are still without a catalogued ionizing source. However, looking at the \textit{HST} $\NUV$ images, we find peaks that are clearly associated with those unmatched \HII regions. 
Their absence in the aforementioned catalogues can have a few reasons. 
First, the selection of cluster candidates is based on the \textit{V} band and hence does not necessarily trace the ionizing sources as well as the $\NUV$. 
Secondly, they are often too compact to be classified as a star cluster according to their concentration index. 
As for the association catalogue, multiple peaks are required and the objects here are mostly single peaks. 
Using the point source catalogue, created with \textsc{dolphot}, that forms the base for both the stellar association and compact cluster catalogues \citep{Thilker+2022}, we find peaks in most \HII regions. 
If we require that a peak is detected with a $\StoN>5$ in both the $\NUV$ and at least one other filter, around $\SI{63.7}{\percent}$ of the unmatched \HII regions contain a peak. 

% could those objects be a single star?
The fainter unmatched \HII regions ($L_{\mathrm{H}\alpha}\lesssim\SI{e37}{\erg\per\second}$) fall in a regime where they could be ionized by a single O-star \citep{Martins+2005}. 
The apparent magnitudes of the \textsc{dolphot} sources inside those unmatched \HII regions are also comparable to the luminosity of a single O-star (see also \cref{sec:single_stars} with \cref{fig:dolphot_peaks_stellar_models} for a comparison with models). 
This subsample of \HII regions that are potentially ionized by a single star (or a very small star cluster, dominated by a single massive star) is interesting in its own right, however beyond the scope of this paper. 

% summary
For now, we leave this with the understanding that most of the unmatched \HII regions in our catalogue have an ionizing source that is visible in the \textit{HST} data, but that these sources do not meet the criteria for our catalogue of stellar associations and hence are not included in our analysis.
Overall, we are able to identify $\NUV$-bright ionizing sources for the vast majority of our sample. 
After accounting for the different sources of ionizing radiation, only $\SI{18.6}{\percent}$ of the \HII regions are left without an ionizing counterpart. 
This number could be even further decreased by lowering the S/N requirement for the \textsc{dolphot} peaks (to $\SI{9.9}{\percent}$ with $\StoN>3$). 


\subsection{Validating the sample}
\label{sec:validate}

\begin{figure}
    \centering
    \includegraphics[width=0.9\columnwidth]{mass_HA_EBV}
    \caption{Top: comparison between stellar association mass and extinction corrected $\HA$ luminosity. The 68 and 98 percentile are shaded in grey. Bottom: Comparison between the stellar $E(B-V)$ derived from the SED fit and the one from the Balmer decrement. The black lines indicate one-to-one and 0.44 to 1 correlation \citep{Calzetti+2000}. The average uncertainties are indicated in the bottom right corner. Both panels use the \texttt{contained} objects in the \texttt{one-to-one} sample.}
    \label{fig:matched_catalogues}
\end{figure}

We validate our cross-matched catalogue by comparing physical properties between the nebulae and stellar associations. 
If the associations are the origin of the ionizing radiation, the $\HA$ luminosity of the \HII region is expected to be related to the mass of the association. 
As shown in \cref{fig:matched_catalogues}, we find a strong correlation between the two. 
For this comparison we utilise the sub-sample of fully contained associations (\num{1918} objects).
There is some scatter, as the $\HA$ flux is also dependent on the age of the stellar association and the amount of radiation that escapes the cloud.

We also compare the dust content, as traced by the colour excess $E(B-V)$, derived from the SED fit to the one calculated from the Balmer decrement. 
We find that for young ages, the two methods are in good agreement, while for older associations, the reddening from the Balmer decrement is systematically higher than that from the SED fit. 
This could be due to problems with the SED fit, where young reddened cluster are identified by the fitting routine as old and un-reddened \citep{Hannon+2022}.
The age of some associations ($>\SI{8}{\mega\year}$) raises some doubts if they are the origin of the ionizing radiation. 
For most of those objects, the reddening derived from the Balmer decrement is higher than the one derived from the SED fit, suggesting that the age/reddening degeneracy is not correctly resolved and we therefore exclude those objects from further analysis.
Alternatively, for young clusters the gas and the stars are closely associated, while they decouple over time \citep{Charlot+2000}, which would also result in lower reddening of the older clusters relative to the nebular emission. 
Future inclusion of \textit{JWST} bands and refined age determinations will help us distinguish these scenarios \citep{Whitmore+2023}.

Given the above discussion, we are confident that the stellar associations are clearly linked to the nebulae, enabling us to investigate more deeply evolutionary trends with stellar age. 

% those comments belong to the next section, but overleaf did not copy them when I moved the content to a new file