% just to be consistent with the correlation coefficients:
% 0.0--0.1: None; 
% 0.1--0.3: Weak; 
% 0.3--0.5: Moderate; 
% 0.5--0.7:Strong; 
% 0.7--1.0:Very Strong

\section[HII region evolutionary sequence]{\HII region evolutionary sequence}
\label{sec:evolutionary_sequence}

In this section we use the matched catalogue to establish different nebular properties as proxies for the age and then explore how other \HII region properties evolve as the nebula ages.
 
\subsection{Correlations with SED ages}
\label{sec:correlation_sed}

\begin{figure}
    \centering
    \includegraphics[width=1\columnwidth]{age_tracers}
    \caption{Comparison between stellar associations ages with MUSE \HII region properties. The age is derived from the SED fit and compared to $\EW$, $\EW_\mathrm{corr}$ (the continuum is corrected for background emission), $\HA/\FUV$ and $\log q$. The binned median is shown with a black line and the 68 and 98 percentile are shaded in grey. The colour of the points indicates the density of the points. The Spearman's rank correlation coefficient is shown in the top right corner. The $\EW$, $\EW_\mathrm{corr}$ and $\log q $ have $p$-value$ <\num{e-7}$ and the $\HA/\FUV$ $p$-value$ <\num{e-2}$.}
    \label{fig:age_tracers}
\end{figure}

In \cref{sec:introduction}, we introduced $\EW$, $\HA/\FUV$ and $\log q$ as potential age tracers. 
With the matched catalogue, we can assess how well they trace the age of the stellar population that is powering the nebula. 
Because accurate ages are of utmost importance to this application, we only use the \texttt{robust sample} and also apply further signal-to-noise cuts. 
For the $\HA/\FUV$ and $\log q$ we only use objects with $\StoN\geq 5$ and for the ages we require that they are at least as large as their uncertainties.
This mainly affects $\HA/\FUV$ and choice of a higher $\StoN$ would reduce the sample too much.
For the $\EW$, the largest uncertainty comes from the continuum, which is often indistinguishable from the background. 
We therefore compute $\EW_\mathrm{corr}$ and require that the background subtracted continuum has $\StoN\geq 5$ (this cut is used for the corrected and un-corrected equivalent width). 
% this is how much each age proxy decreases in the observations:
% 1 Myr   2 Myr   3 Myr   4 Myr   5 Myr   6 Myr   7 Myr   8 Myr
%100.0%,  86.2%,  75.9%,  53.0%,  52.1%,  77.6%,  56.7%,  76.7% (EW_HA)
%100.0%,  89.3%,  84.3%,  44.6%,  41.5%,  81.3%,  65.7%,  76.8% (EW_HA_CORR)
%100.0%,  84.1%,  90.5%,  86.1%,  54.3%,  64.5%,  71.6%,  91.8% (HA/FUV_corr)
%100.0%,  95.6%,  83.1%,  55.5%,  47.6%,  57.8%,  63.0%,  83.1% (q)
In \cref{fig:age_tracers} we compare the age from the SED fit to all four properties.
We observe weak to moderate anti-correlations between the SED ages and all proposed age tracers. 
To test how robust the trends are, we apply a Monte Carlo approach, where we repeatedly sample our data, based on the associated uncertainties and assuming a normal distribution. 
The observed trends are robust and persist, with a scatter around $\SI{10}{\percent}$, except for the $\HA/\FUV$, which shows a larger scatter of $\SI{20}{\percent}$.  

The $\EW$ drops by $\SI{47}{\percent}$ within the first $\SI{4}{\mega\year}$. 
This is caused by the death of the most massive stars and in line with the $\SI{42}{\percent}$ decrease that is predicted by models like \textsc{starburst99} \citep{Leitherer+2014}. 
However, after that, the observed $\EW$ stays roughly constant and even increases slightly towards $\SI{8}{\mega\year}$, compared to the models that predict further decrease to only $\SI{6}{\percent}$ at the same age. 
Also conspicuous is the smaller range of values covered by the observations, compared to the models (see \cref{sec:model_predictions}). 
An easy explanation for this is that we neglected the background. For the stellar continuum around $\HA$, we observe a significant contribution from older stellar populations throughout the galaxy. 
Accounting for this, the $\EW_\mathrm{corr}$ increases by a factor of 10, bringing the measured values closer to the model predictions.
The correlation with the background corrected equivalent width $\EW_\mathrm{corr}$ is slightly stronger than the un-corrected one.
However it is quite challenging to correctly disentangle the contribution of the background.

The flux ratio of $\HA$ to $\FUV$ stays roughly constant for the first $\SI{4}{\mega\year}$, after which it drops to $\sim\kern-2pt\SI{65}{\percent}$.
Again, after that, contrary to the model predictions, the ratio does not further decrease. 
Similar to the equivalent width, the observed values are smaller than the model predictions, although the difference is not quite as large. 
Applying the same reasoning as with the equivalent width to resolve this discrepancy does not work. 
In the case of the $\FUV$, we do not observe a global background, but rather other events of previous nearby star formation, which should not bias our measurements.

Lastly, we look at the ionization parameter. 
For reference we also show $\SIII/\SII$ based on \cref{eq:logq_D91}.
We observe a similar behaviour with an initial drop and a flattening after $\SI{4}{\mega\year}$, corresponding to the decrease in ionizing photon flux and expansion of the nebula.
However it should be noted that the range of ionization parameters that we observe only covers the first few $\si{\mega\year}$ in the models \citep[this could be due to systematic offsets between observations and model grids as highlighted in][]{Mingozzi+2020}.

Given that the uncertainties on the ages are often similar to the ages themselves, we consider a broader statistical analysis by binning our sample by the fiducial age tracers and looking at the median SED ages. 
We find that associations that overlap with an \HII region are significantly younger than those that are isolated. 
When binning our sample based on the $\EW$ we find that the associations with high $\EW$ are younger than their counterparts with low $\EW$ (see \cref{sec:age_hist} for more details). 
A similar but less pronounced trend can also be observed when binning with the $\HA/\FUV$ and $\log q$. 

Even though the observed decrease with SED ages are weaker than expected, $\EW$, $\EW_\mathrm{corr}$, $\HA/\FUV$ and $\log q$ are all consistent in their evolution. 
In \cref{fig:age_tracers_corner} we use the full \HII region sample of $\sim\kern-2pt\num{20000}$ \HII regions (applying the same signal-to-noise cuts as to the previous sample) to compare the proposed age tracers with each other. 
Across the entire galaxy sample, spanning a wide range of stellar masses, SFRs and metallicities, we find moderate to strong correlations between all properties, consistent with the idea of an evolutionary sequence. 

\begin{figure}
    \centering
    \includegraphics[width=1\columnwidth]{nebulae_age_tracers_corner}
    \caption{Comparison between the proposed age tracers for the full \HII region catalogue. The nebulae are grouped by their host galaxy, sorted by stellar mass $M_{\star}$, with the median of the entire sample indicated by a black line. The 68 and 98 percentile ranges are shaded in grey.}
    \label{fig:age_tracers_corner}
\end{figure}

\subsection{Age trends in the nebular catalogue}\label{sec:trends_nebulae}

\begin{figure*}
\centering
\includegraphics[width=0.9\textwidth]{trends_with_age_tracers}
\caption{Correlations between the physical properties in the \HII region catalogue and the proposed age tracers. We show trends with metallicity offset, $\HA$ luminosity, electron density, colour excess and separation to the nearest GMC. The nebulae are grouped by their host galaxies, sorted by their stellar mass $M_{\star}$, with the average of the entire sample indicated by a black line. The 68 and 98 percentile are shaded in grey.}
\label{fig:trends_with_age}
\end{figure*}

In the previous section, we showed that $\EW$, $\HA/\FUV$ and $\log q$ all exhibit anti-correlations with age, albeit weaker than expected. 
However, for this analysis we were limited to the few hundred \HII regions that contained a massive stellar association, severely limiting our ability to study the evolution of the \HII regions in terms of different ISM properties. 
In this section we use the full catalogue of \num{23736} \HII regions and look for correlations with the proposed age tracers. 
\Cref{fig:trends_with_age} shows trends between $\EW$, $\EW_\mathrm{corr}$, $\HA/\FUV$ and $\log q$ with various nebula properties. 
All trends have $p$-values$<\num{e-6}$, with the exception of $\HA/\FUV$ versus $\delta_\mathrm{GMC}$. 
Following the same procedure as in \cref{sec:correlation_sed}, we find that the trends are robust when varying the properties within their uncertainties.

First, the metallicity variations $\DeltaOH$ are the difference between the local metallicity and the global metallicity gradient. 
We recover the same correlation between $\log q$ and the $\DeltaOH$ that were also found by \citet[][based on the same data]{Kreckel+2019} and \citet{Grasha+2022}. 
The metallicity offset also shows a correlation with $\EW$ and $\HA/\FUV$, indicating that younger regions are more enriched. 
This suggests that the natal environment, not yet dispersed and associated with the youngest regions, is more metal rich.
Over time, this material could mix on larger scales \citep{Kreckel+2020}, decreasing the metallicity offset.

Next, we observe correlations with the $\HA$ luminosity of the \HII regions with our age tracers. The strong correlation with $\EW$ is likely due to the significant contamination of the stellar continuum by (unrelated) older stellar populations. Due to this, the variations could be attributed to the smaller variations in both the $\FUV$ and the stellar continuum compared to $\HA$.

For the electron density of the ionized gas, the majority of our \HII regions fall in the low-density limit, making it difficult to robustly measure a density. 
Only \num{1375} \HII regions are significantly different from the low-density limit which limits our statistical capabilities for individual galaxies.
We find no strong correlations with density, and in most tracers we observe weak anti-correlations, contradicting expectations that newly formed star clusters are still embedded in a dense cloud of gas. 
Focusing on the ionization parameter, we do recover a positive trend, consistent with the simplistic Str\"{o}mgren sphere assumptions from \cref{eq:ionisation_stroemgren}. 

The extinction on the other hand, traced by the colour excess from the Balmer decrement, shows a correlation with $\EW$ and $\log q$, fitting the picture of a young embedded cluster. 
However, by contrast, we observe an anti-correlation with $\HA/\FUV$.
It should be noted that especially for the lower mass galaxies, the trends are very flat.

Finally, we compare with the separation to the nearest \emph{giant molecular cloud} (GMC). 
For this we use the GMC catalogue from \citet{Rosolowsky+2021} and Hughes et al.\ in preparation, based on the PHANGS--ALMA survey \citep{Leroy+2021b}, and cross-match with the \HII region catalogue. 
Given that the typical intercloud and interregion distances in these catalogues are a few hundred $\si{pc}$ (e.g.\ \citealt{Kim+2022}, \citealt{Groves+2023}, Machado et al.\ in preparation) we consider separations of this order to be unphysical but closer clouds may indicate a real separation. 
Therefore we exclude \HII regions with a GMC separation larger than $\SI{300}{\parsec}$ from this analysis. 
We expect the youngest \HII regions to still be closely associated to their birth environment and indeed we find an anti-correlation between separation and age.

For this investigation we presented both the observed and corrected $\EW$. Overall, these two measures show good correlation with each other (\cref{fig:age_tracers_corner}), and similar qualitative trends in \cref{fig:age_tracers}. 
For most properties we find stronger correlations with the un-corrected $\EW$, which could hint at complications with the background subtraction.

\subsection{Understanding the weak correlations with SED age}

% here is a list of things that could go wrong
%% HERE IS A LIST OF SCIENTIFICALLY INTERESTING IMPLICATIONS OF OUR WORK!
%
% 1) the evolution happens so quick that the age resolution is not sufficient to trace it
% 2) there is no evolutionary sequence (-> extended star formation, find reference)
% 3) impact of environment 
% 4) ionisation parameter has secondary dependence on density and metallicity. this should smear out the correlation
% 5) impact of escape fraction
% 6) we do not account for background 
% 7) something is wrong with the measured cloud properties (hopefully not, do not mention)
% 8) something is wrong with the cluster ages (we also don't mention this)

Although we do see some trends of $\EW$, $\HA/\FUV$ and $\log q$ with age, they are weaker than predicted by models. 
There are a number of reasons why this could be the case. 

First, looking at \cref{fig:catalogue_properties_2D_hist_v2}, we find that roughly half of our sample falls in the youngest $\SI{1}{\mega\year}$ age bin. Given that our median reported uncertainty in the ages is $\SI{1}{\mega\year}$,  the age resolution of the SED fit is not sufficient to probe the evolution of these \HII regions or (realistically) changes that happen within the first \SIrange{1}{2}{\mega\year}. 
Beyond model limitations, another major issue is that we are using only broad bands with limited UV coverage. 
This makes it very difficult to achieve higher precision at very young ages. 
However, the $\sim\kern-2pt\SI{50}{\percent}$ drop in intensity of the nebular age indicators within the first $\SI{4}{\mega\year}$ is broadly consistent with models, and suggests that generally the expected age trends are recovered (see \cref{sec:evolutionary_sequence}).

If we only consider the youngest objects ($\leq \SI{2}{\mega\year}$), we still recover the same trends between $\log q$, $\EW$ and $\HA/\FUV$. 
We note that we also observe the same trends in the older subsample ($> \SI{2}{\mega\year}$), albeit somewhat weaker. 
This suggests that the evolution is not bound to a fixed or absolute timescale, but can vary greatly between clouds. 
In many ways, this is unsurprising, and depending on the environment, the \HII regions can evolve differently.
The timescales over which natal molecular gas clouds are cleared are estimated to be \SIrange{2}{3}{\mega\year} \citep{Kim+2021,Chevance+2022}, suggesting that significant morphological changes in the local environment are occurring. 
Depending on the initial density of the cloud and the ionizing cluster mass, this can lead to large variations in the evolution time scale between different \HII regions \citep{Kruijssen+2019}. 
In this way, we would still expect to see trends between these nebular age tracers, as they are all reflecting the clearing of the cloud, but the absolute timescales (traced by the age of the underlying stellar cluster) would not necessarily agree across different clouds. 

% 4)
There are also clearly secondary dependencies beyond age, which may or may not significantly impact the trends we explore but are challenging to account for. 
For example, it is likely that different stellar population models would give different results for the age and mass of the stellar associations, and this is not something that has been explored. 
In the case of the ionization parameter, variations will also  arise due to changes in metallicity\citep{Dopita+2014, Kreckel+2019, Grasha+2022} and density \citep{Dopita+2006}, and smear out the trend with age. 
However these relations, particularly with metallicity, are poorly understood from observations or modeling, and it is not even clear if we expect a correlation or anti-correlation \citep[and references therein]{Ji+2022}. 

One key assumption is that we are looking at nebular emission associated with an instantaneous burst of star formation. 
This assumption is not only important for the SED fit, but if the star formation is extended over a larger period of time, it will also be reflected in the observed flux ratios. 
The decrease of both $\HA/\FUV$ and $\EW$ will appear to be slower and much less pronounced, due to the additional underlying stellar continuum emission from the pre-existing older population \citep{Smith+2002,Levesque+2013}. 
Meanwhile, the $\HA$ flux and $\log q$ will largely trace more directly the evolution of the latest burst. 
The exact duration of star formation is actively discussed in the literature, and can vary from very short duration of less than $\SI{1}{\mega\year}$ \citep{Povich+2010} to multiple episodes of star formation across tens of $\si{\mega\year}$ \citep{Ramachandran+2018b}. 
A prominent example here are the two clusters at the centre of 30 Doradus that are thought to have an age spread of a few $\si{\mega\year}$ \citep{Sabbi+2012,Rahner+2018}. 
By studying only a sample where the \HII region and stellar association are matched one-to-one, we have attempted to minimise the impact of blending of different stellar populations.  

% 5)
One aspect we have neglected is the impact of leaking radiation, which we know must occur as it is largely responsible for the $\si{\kilo\parsec}$ scale ionization of the diffuse ionized gas \citep{Belfiore+2022}.  
Ionizing photons that escape the cloud result in a lower $\HA$ flux measured within the nebula, while the measurement of the $\FUV$ and stellar continuum is mostly unaffected. 
Existing studies measure a wide range of \emph{escape fractions}, ranging from only a few $\si{\percent}$ \citep{Pellegrini+2012} all the way up to \SIrange{60}{70}{\percent} \citep{McLeod+2019,DellaBruna+2021}.
Because the fraction of leaking radiation is likely to change as the cloud ages and dissolves, this can introduce considerable scatter, depending on how much the escape fraction varies between clouds. 
However this process will only decrease the measured $\EW$ or $\HA/\FUV$, and hence the true value should form an upper envelope above the data. 
Escape fractions within the range from \SIrange{40}{80}{\percent} would result in a factor of 5 scatter in $\EW$ at fixed age, roughly consistent with our result.

 % summary
Overall, a combination of the effects listed above are probably responsible for the observed trends in age being less pronounced than expected. 
However, we emphasise that our observations are still consistent with an evolutionary sequence.
 
\subsection{Recommendations on nebular age tracers}
 
The equivalent width measurement has most commonly been used as an age tracer in the literature, and suffers from relatively few systematics or uncertainties due to extinction corrections and is easy to measure. 
However, the contribution of stellar light from the galaxy disk is problematic. 
The stellar continuum emission arising from old stars in the disk is not physically associated with the nebulae, does not contribute to its ionization, and hence should not compromise its role as an age tracer. 
As suggested in \cref{fig:age_tracers_corner}, the raw $\EW$ appears to correlate strongly with the corrected one $\EW_\mathrm{corr}$, however robust use of this tracer is only possible if we include $\StoN$ cuts on the sample that require an understanding of the background contribution. 

On average, the background contribution makes up $\SI{90}{\percent}$ of the light, such that the uncertainties introduced when converting between observed and galaxy-corrected $\EW$ introduce a lot of scatter. 
This makes this age proxy particularly problematic to use for individual \HII regions. 
As the background correction we currently implement relies heavily on identifying nearby (but not co-spatial) contributions to the continuum stellar light, this could be improved by utilising constraints directly at the location of the \HII regions themselves. 
This should be possible by carrying out physically motivated stellar population fits to the \HII region spectrum itself, separating the contribution of older stars and refining this age tracer.  
However, challenges in identifying robust young stellar templates \citep{Emsellem+2022} make such an analysis beyond the scope of this paper. 

The ionization parameter also shows a lot of promise as an evolutionary tracer, however it is strongly dependent on local physical conditions and is perhaps less robust as a direct age tracer. 
Here we are deriving ionizing parameter based on the $\SIII/\SII$ ratio, which shows a very strong primary correlation and minimal secondary dependencies in photoionization modeling \citep{Kewley+2002}. 
While the wide wavelength separation of the lines means they are susceptible to uncertainties in the extinction correction,  the lines are at relatively red wavelengths and therefore attenuation effects are minimised. 
As $\SIII/\SII$ can be detected in the largest number ($\SI{86}{\percent}$) of our nebulae, it also holds the most promise for exploring  evolutionary trends statistically. 

We deem $\HA/\FUV$ to be the least trustworthy age tracer. 
We note that the trends with other physical properties measured in the nebulae are weaker with $\HA/\FUV$ than with the other two nebula age tracers. 
In addition, there are large uncertainties arising from the extinction correction, as $\HA$ and $\FUV$ are at vastly different wavelengths. 
At these physical scales it is not entirely clear at what stage the reddening in the stars begins to deviate from the reddening measured in the gas (via the Balmer decrement) and we find that our choice of $X$ in ${E(B-V)}_\mathrm{stellar} = X \cdot {E(B-V)}_\mathrm{Balmer}$ can alter some of the observed trends. 
Our comparison with the \textit{HST} SED derived $E(B-V)$ in \cref{fig:matched_catalogues} suggests most of the nebulae are best fit by $X=1$, however this changes at ages roughly older than $\SI{5}{\mega\year}$. 
Another issue is the sensitivity of the observations: we detect $\HA/\FUV$ in less than $\SI{23}{\percent}$ of our \HII regions with a $\StoN\geq5$. 
For these reasons, we are hesitant to recommend $\HA/\FUV$ as an age tracer, but note that it does still encode information about the early evolutionary state of these nebulae. 
