\section{Conclusion}
\label{sec:conclusion}

We combine observations from MUSE, \textit{HST} and \textit{AstroSat} to study \HII regions and their ionizing sources together. 
This enables us to use the SED age from the ionizing stellar associations as clocks and time the evolution of the \HII regions.
%
\begin{enumerate}[leftmargin=*]
    \item We present a catalogue of \num{4177} \HII regions and stellar associations that are clearly matched to each other. This catalogue is well suited for placing empirical constraints on models.
    \item We find strong correlations between the masses of the stellar associations and the $\HA$ fluxes of the \HII regions as well as the colour excess derived from the Balmer decrement and the SED fit.
    \item We search for age trends and find weak to moderate correlations with the $\EW$, $\HA/\FUV$ and $\log q$.
    \item All three properties show consistent trends among each other, hinting at an evolutionary sequence.
    \item We find similar trends with the raw and corrected $\EW$. However the $\StoN$ cuts that we apply necessitate an understanding of the background in both cases.  
    \item Using our nebular age indicators, we find tentative trends for younger \HII regions to exhibit higher densities, higher reddening and smaller separations to GMCs. Interestingly, we also find strong correlations with local metallicity, with the youngest \HII regions exhibiting locally elevated metallicities.
\end{enumerate}
%
The catalogue presented in this paper provides a novel statistical base for further investigations of the interaction between newly formed stars and their natal gas cloud. 
For example it can be used to validate models like \citet{Kang+2022} that use emission line diagnostics of \HII regions to predict the properties of the underlying star cluster. 
It also allows us to study the impact of stellar feedback on the ISM and in a follow up paper, we will use this catalogue to measure escape fractions of the \HII regions. 
The addition of PHANGS--\textit{JWST} observations for all 19 targets \citep{Lee+2023} will further serve to refine our view of the early embedded phase of star formation, completing our view of the nebular evolutionary sequence and baryon cycle in these nearby galaxies. 
