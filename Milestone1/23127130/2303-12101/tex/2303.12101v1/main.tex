% mnras_template.tex 
%
% LaTeX template for creating an MNRAS paper
%
% v3.0 released 14 May 2015
% (version numbers match those of mnras.cls)
%
% Copyright (C) Royal Astronomical Society 2015
% Authors:
% Keith T. Smith (Royal Astronomical Society)

% Change log
%
% v3.0 May 2015
%    Renamed to match the new package name
%    Version number matches mnras.cls
%    A few minor tweaks to wording
% v1.0 September 2013
%    Beta testing only - never publicly released
%    First version: a simple (ish) template for creating an MNRAS paper

%%%%%%%%%%%%%%%%%%%%%%%%%%%%%%%%%%%%%%%%%%%%%%%%%%
% Basic setup. Most papers should leave these options alone.
\documentclass[fleqn,usenatbib]{mnras}

% MNRAS is set in Times font. If you don't have this installed (most LaTeX
% installations will be fine) or prefer the old Computer Modern fonts, comment
% out the following line
\usepackage{newtxtext,newtxmath}

% Use vector fonts, so it zooms properly in on-screen viewing software
% Don't change these lines unless you know what you are doing
\usepackage[T1]{fontenc}

% Allow "Thomas van Noord" and "Simon de Laguarde" and alike to be sorted by "N" and "L" etc. in the bibliography.
% Write the name in the bibliography as "\VAN{Noord}{Van}{van} Noord, Thomas"
\DeclareRobustCommand{\VAN}[3]{#2}
\let\VANthebibliography\thebibliography
\def\thebibliography{\DeclareRobustCommand{\VAN}[3]{##3}\VANthebibliography}


%%%%% AUTHORS - PLACE YOUR OWN PACKAGES HERE %%%%%

% Only include extra packages if you really need them. 
\usepackage{graphicx}	% Including figure files
\usepackage{amsmath}	% Advanced maths commands
\usepackage{booktabs}   % for nicer tables
\usepackage{siunitx}    % quantities with units
\usepackage{pgf}        % vector graphics (for distance plots)
\usepackage{xspace}     % spaces after custom commands
\usepackage{ifthen}     % used to define own commands
\usepackage{enumitem}   % customise lists
\usepackage{flushend}   % balance the columns on the last page
\usepackage[all]{hypcap} % fix location of links (\ref{})
\usepackage[noabbrev,
            nameinlink,
            capitalise]{cleveref} 	 % easier references 

%%%%%%%%%%%%%%%%%%%%%%%%%%%%%%%%%%%%%%%%%%%%%%%%%%

%%%%% AUTHORS - PLACE YOUR OWN COMMANDS HERE %%%%%

% Please keep new commands to a minimum, and use \newcommand not \def to avoid
% overwriting existing commands. Example:
\newcommand{\uncertainty}[3]{#1^{+#2}_{-#3}}
\newcommand{\StoN}{\mathrm{S}/\mathrm{N}}
\newcommand\galaxyname[2]{#1\,#2}
%\newcommand{\galaxyname}[2]{\textsc{\lowercase{#1}}{\small#2}}

\graphicspath{{fig/}}

% implement guidelines from  https://academic.oup.com/mnras/pages/general_instructions for ionised species
\newcommand\ionized[2]{[\mathrm{#1}\,\textsc{#2}]}
\newcommand\OI[1][]{\ifthenelse{\equal{#1}{}}{\ionized{O}{i}}{\ionized{O}{i}\,\uplambda#1}}
\newcommand\OII[1][]{\ifthenelse{\equal{#1}{}}{\ionized{O}{ii}}{\ionized{O}{ii}\,\uplambda#1}}
\newcommand\OIII[1][]{\ifthenelse{\equal{#1}{}}{\ionized{O}{iii}}{\ionized{O}{iii}\,\uplambda#1}}
\newcommand\NII[1][]{\ifthenelse{\equal{#1}{}}{\ionized{N}{ii}}{\ionized{N}{ii}\,\uplambda#1}}
\newcommand\SII[1][]{\ifthenelse{\equal{#1}{}}{\ionized{S}{ii}}{\ionized{S}{ii}\,\uplambda#1}}
\newcommand\SIII[1][]{\ifthenelse{\equal{#1}{}}{\ionized{S}{iii}}{\ionized{S}{iii}\,\uplambda#1}}
\newcommand\HA[1][]{\ifthenelse{\equal{#1}{}}{\mathrm{H}\,\upalpha}{\mathrm{H}\,\upalpha\,\uplambda#1}}
\newcommand\HB[1][]{\ifthenelse{\equal{#1}{}}{\mathrm{H}\,\upbeta}{\mathrm{H}\,\upbeta\,\uplambda#1}}

% typeset R.A. and Dec. in a nice way
\newcommand\RA[4]{$#1^\mathrm{h}#2^\mathrm{m}#3^\mathrm{s}\kern -3pt.#4$}
\newcommand\DEC[4]{$#1^\mathrm{d}#2^\mathrm{m}#3^\mathrm{s}\kern -3pt.#4$}

% keep abbreviations consistent throughout the document
\newcommand\HII{H\,\textsc{ii}\xspace}
\newcommand\pn{\textsc{PN}\xspace}
\newcommand\snr{\textsc{SNR}\xspace}
\newcommand\FUV{\mathrm{FUV}}
\newcommand\NUV{\mathrm{NUV}}
\newcommand\logOH{12+\log (\mathrm{O}/\mathrm{H})}
\newcommand\DeltaOH{\Delta (\mathrm{O}/\mathrm{H})}
\newcommand\EW{\mathrm{EW}(\HA)}

\sisetup{%
    per-mode=reciprocal,
    exponent-base = 10,
	exponent-product = \cdot ,
	group-digits = true,
	group-separator = {,},
	group-minimum-digits = 4,
	inter-unit-product = {\;},
    parse-numbers = true, 
	range-units = single,
	separate-uncertainty = true,
	input-uncertainty-signs = \pm,
    uncertainty-separator =  {\,}, 
	multi-part-units=single,
	}

% define non-SI units
\DeclareSIUnit\angstrom{\text {Å}}
\DeclareSIUnit\percent{per\,cent}
\DeclareSIUnit\arcsec{arcsec}
\DeclareSIUnit\arcmin{arcmin}
\DeclareSIUnit\parsec{pc}
\DeclareSIUnit\lightyear{ly}
\DeclareSIUnit\year{yr}
\DeclareSIUnit\mag{mag}
\DeclareSIUnit\erg{erg}
\DeclareSIUnit\Msun{M_\odot}
\DeclareSIUnit\Lsun{L_\odot}

% insert ORCID icon with link to to ORCID record
\newcommand*{\orcidlink}[1]{%
    \href{https://orcid.org/#1}{\,\raisebox{0.2em}{%
        \includegraphics[height=0.6em,width=0.6em]{orcid.png}%
    }}}

% mark changes in the text
\newcommand{\change}[1]{{\color{orange}#1}}

%%%%%%%%%%%%%%%%%%%%%%%%%%%%%%%%%%%%%%%%%%%%%%%%%%

%%%%%%%%%%%%%%%%%%% TITLE PAGE %%%%%%%%%%%%%%%%%%%

% Title of the paper, and the short title which is used in the headers.
% Keep the title short and informative.
%
\title[\HII region evolutionary sequence]{Stellar associations powering \HII regions -- I. Defining an evolutionary sequence}

% The list of authors, and the short list which is used in the headers.
% If you need two or more lines of authors, add an extra line using \newauthor
\author[Scheuermann et al.]{Fabian Scheuermann\orcidlink{0000-0003-2707-4678},$^{\hyperlink{ARI}{1}}$\thanks{E-mail: f.scheuermann@uni-heidelberg.de} %0000-0003-2707-4678
Kathryn Kreckel\orcidlink{0000-0001-6551-3091},$^{\hyperlink{ARI}{1}}$ %0000-0001-6551-3091
Ashley T.~Barnes\orcidlink{0000-0003-0410-4504},$^{\hyperlink{Bonn}{2}}$ %0000-0003-0410-4504
\newauthor
Francesco Belfiore\orcidlink{0000-0002-2545-5752},$^{\hyperlink{INAF}{3}}$ %0000-0002-2545-5752
Brent Groves\orcidlink{0000-0002-9768-0246},$^{\hyperlink{Perth}{4}}$ %0000-0002-9768-0246
Stephen Hannon\orcidlink{0000-0001-9628-8958},$^{\hyperlink{Riverside}{5},\hyperlink{Gemini}{6}}$ %0000-0001-9628-8958
Janice C.~Lee\orcidlink{0000-0002-2278-9407},$^{\hyperlink{Gemini}{6},\hyperlink{Steward}{7}}$ %0000-0002-2278-9407
\newauthor
Rebecca Minsley$^{\hyperlink{Arizona}{8}}$,
Erik Rosolowsky\orcidlink{0000-0002-5204-2259},$^{\hyperlink{Alberta}{9}}$ %0000-0002-5204-2259
Frank Bigiel\orcidlink{0000-0003-0166-9745},$^{\hyperlink{Bonn}{2}}$ %0000-0003-0166-9745
Guillermo A.~Blanc\orcidlink{0000-0003-4218-3944},$^{\hyperlink{Carnegie}{10},\hyperlink{Chile}{11}}$ %0000-0003-4218-3944
\newauthor
M\'ed\'eric Boquien\orcidlink{0000-0003-0946-6176},$^{\hyperlink{Antofagasta}{12}}$ %0000-0003-0946-6176
Daniel A. Dale\orcidlink{0000-0002-5782-9093},$^{\hyperlink{Wyoming}{13}}$ %0000-0002-5782-9093
Sinan Deger\orcidlink{0000-0003-1943-723X},$^{\hyperlink{Stockholm}{14}}$ %0000-0003-1943-723X
Oleg Egorov\orcidlink{0000-0002-4755-118X},$^{\hyperlink{ARI}{1}}$ %0000-0002-4755-118X
\newauthor
Eric Emsellem\orcidlink{0000-0002-6155-7166},$^{\hyperlink{ESO}{15},\hyperlink{Lyon}{16}}$ %0000-0002-6155-7166
Simon C.~O.~Glover\orcidlink{0000-0001-6708-1317},$^{\hyperlink{ITA}{17}}$ %0000-0001-6708-1317
Kathryn Grasha\orcidlink{0000-0002-3247-5321},$^{\hyperlink{Canberra}{18},\hyperlink{ARC}{19}}$ %0000-0002-3247-5321
Hamid Hassani\orcidlink{0000-0002-8806-6308},$^{\hyperlink{Alberta}{9}}$ %0000-0002-8806-6308
\newauthor
Sarah M.~R.~Jeffreson\orcidlink{0000-0002-4232-0200},$^{\hyperlink{Harvard}{20}}$ %0000-0002-4232-0200
Ralf S.~Klessen\orcidlink{0000-0002-0560-3172},$^{\hyperlink{ITA}{17},\hyperlink{IWR}{21}}$ %0000-0002-0560-3172
J.~M.~Diederik~Kruijssen\orcidlink{0000-0002-8804-0212},$^{\hyperlink{cool}{22}}$ %0000-0002-8804-0212
\newauthor
Kirsten L. Larson\orcidlink{0000-0003-3917-6460},$^{\hyperlink{STsci}{23}}$ %0000-0003-3917-6460
Adam K. Leroy\orcidlink{0000-0002-2545-1700},$^{\hyperlink{Ohio}{24}}$ %0000-0002-2545-1700
Laura A. Lopez\orcidlink{0000-0002-1790-3148},$^{\hyperlink{Ohio}{24},\hyperlink{Colombus}{25}}$ %0000-0002-1790-3148
Hsi-An Pan\orcidlink{0000-0002-1370-6964},$^{\hyperlink{Taiwan}{26}}$ %0000-0002-1370-6964
\newauthor
Patricia S\'anchez-Bl\'azquez\orcidlink{0000-0003-0651-0098},$^{\hyperlink{Cantoblanco}{27}}$ %0000-0003-0651-0098
Francesco Santoro\orcidlink{0000-0002-6363-9851},$^{\hyperlink{MPIA}{28}}$ %0000-0002-6363-9851
Eva Schinnerer\orcidlink{0000-0002-3933-7677},$^{\hyperlink{MPIA}{28}}$ %0000-0002-3933-7677
\newauthor
David A. Thilker\orcidlink{0000-0002-8528-7340},$^{\hyperlink{JohnHopkins}{29}}$ %0000-0002-8528-7340
Bradley C. Whitmore\orcidlink{0000-0002-3784-7032},$^{\hyperlink{Stsci}{23}}$ %0000-0002-3784-7032
Elizabeth J. Watkins\orcidlink{0000-0002-7365-5791},$^{\hyperlink{ARI}{1}}$ %0000-0002-7365-5791
\newauthor
Thomas G. Williams\orcidlink{0000-0002-0012-2142}\,$^{\hyperlink{MPIA}{28}, \hyperlink{Oxford}{30}}$ %0000-0002-0012-2142
\vspace*{0.2em} \\
Affiliations are listed at the end of the paper.
}

% These dates will be filled out by the publisher
\date{Accepted 2023 March 16. Received 2023 February 23; in original form 2022 November 5}

% Enter the current year, for the copyright statements etc.
\pubyear{2023}

% Don't change these lines
\begin{document}
%\setcounter{page}{30}
\label{firstpage}
\pagerange{\pageref{firstpage}--\pageref{lastpage}}
\maketitle

\begin{abstract}
Connecting the gas in \HII regions to the underlying source of the ionizing radiation can help us constrain the physical processes of stellar feedback and how \HII regions evolve over time. 
With PHANGS--MUSE we detect nearly \num{24000} \HII regions across 19 galaxies and measure the physical properties of the ionized gas (e.g.\ metallicity, ionization parameter, density). 
We use catalogues of multi-scale stellar associations from PHANGS--\textit{HST} to obtain constraints on the age of the ionizing sources. 
We construct a matched catalogue of \num{4177} \HII regions that are clearly linked to a single ionizing association. 
A weak anti-correlation is observed between the association ages and the $\HA$ equivalent width $\EW$, the $\HA/\FUV$ flux ratio and the ionization parameter, $\log q$. 
As all three are expected to decrease as the stellar population ages, this could indicate that we observe an evolutionary sequence. 
This interpretation is further supported by correlations between all three properties. 
Interpreting these as evolutionary tracers, we find younger nebulae to be more attenuated by dust and closer to giant molecular clouds, in line with recent models of feedback-regulated star formation. 
We also observe strong correlations with the local metallicity variations and all three proposed age tracers, suggestive of star formation preferentially occurring in locations of locally enhanced metallicity. 
Overall, $\EW$ and $\log q$ show the most consistent trends and appear to be most reliable tracers for the age of an \HII region.
\end{abstract}

% Select between one and six entries from the list of approved keywords.
% Don't make up new ones.
\begin{keywords}
galaxies: ISM -- ISM: \HII regions -- galaxies: star clusters: general
\end{keywords}

%%%%%%%%%%%%%%%%%%%%%%%%%%%%%%%%%%%%%%%%%%%%%%%%%%

%%%%%%%%%%%%%%%%% BODY OF PAPER %%%%%%%%%%%%%%%%%%

%size for figures: (one column=3.321") (two columns=6.974"). 

% the content is split into separate files
\input{"tex/01introduction"}
\input{"tex/02data"}
\input{"tex/03match_catalogues"}
\input{"tex/04evolutionary_sequence"}
\input{"tex/05conclusion"}

\section*{Acknowledgements}

We thank the anonymous referee for the helpful comments that improved this work. 
This work was carried out as part of the PHANGS collaboration. 
Based on observations collected at the European Southern Observatory under ESO programmes 094.C-0623 (PI: Kreckel), 095.C-0473,  098.C-0484 (PI: Blanc), 1100.B-0651 (PHANGS--MUSE; PI: Schinnerer), as well as 094.B-0321 (MAGNUM; PI: Marconi), 099.B-0242, 0100.B-0116, 098.B-0551 (MAD; PI: Carollo) and 097.B-0640 (TIMER; PI: Gadotti). 
Based on observations made with the NASA/ESA \textit{Hubble Space Telescope}, obtained from the data archive at the Space Telescope Science Institute. STScI is operated by the Association of Universities for Research in Astronomy, Inc. under NASA contract NAS 5-26555. Support for Program number 15654 was provided through a grant from the STScI under NASA contract NAS5-26555. This publication uses data from the \textit{AstroSat} mission of the Indian Space Research Organisation (ISRO), archived at the Indian Space Science Data Centre (ISSDC).
FS, KK and OE gratefully acknowledges funding from the German Research Foundation (DFG) in the form of an Emmy Noether Research Group (grant number KR4598/2-1, PI Kreckel).
KK and EW acknowledge support from the DFG via SFB 881 ‘The Milky Way System’ (project-ID 138713538; subproject P2).
ATB and FB would like to acknowledge funding from the European Research Council (ERC) under the European Union’s Horizon 2020 research and innovation programme (grant agreement No. 726384/Empire).
ER and HH acknowledges the support of the Natural Sciences and Engineering Research Council of Canada (NSERC), funding reference number RGPIN-2017-03987, and the Canadian Space Agency funding reference numbers SE-ASTROSAT19 and 22ASTALBER.
GAB acknowledges support from the ANID BASAL FB210003 project.
MB gratefully acknowledges support by the ANID BASAL project FB210003 and from the FONDECYT regular grant 1211000.
SD is supported by funding from the European Research Council (ERC) under the European Union’s Horizon 2020 research and innovation programme (grant agreement no. 101018897 CosmicExplorer).
RSK and SCOG acknowledge funding from the Deutsche Forschungsgemeinschaft (DFG) via SFB 881 ‘The Milky Way System’ (subprojects A1, B1, B2 and B8) and from the Heidelberg Cluster of Excellence STRUCTURES in the framework of Germany’s Excellence Strategy (grant EXC-2181/1-390900948). They also acknowledge support from the European Research Council in the ERC synergy grant ‘ECOGAL’ Understanding our Galactic ecosystem: From the disk of the Milky Way to the formation sites of stars and planets’ (project ID 855130).
SMRJ is supported by Harvard University through the ITC.
JMDK gratefully acknowledges funding from the European Research Council (ERC) under the European Union’s Horizon 2020 research and innovation programme via the ERC Starting Grant MUSTANG (grant agreement number 714907). COOL Research DAO is a Decentralised Autonomous Organisation supporting research in astrophysics aimed at uncovering our cosmic origins.
The work of AKL was partially supported by the National Science Foundation (NSF) under Grants No.~1653300 and 2205628.
HAP acknowledges support by the National Science and Technology Council of Taiwan under grant 110-2112-M-032-020-MY3.
ES and TGW acknowledges funding from the European Research Council (ERC) under the European Union’s Horizon 2020 research and innovation programme (grant agreement No. 694343).
This research made use of \textsc{astropy} \citep{Astropy+2013,Astropy+2018,Astropy+2022},  \textsc{numpy} \citep{Harris+2020}, \textsc{matplotlib} \citep{Hunter+2007} and \textsc{pyneb} \citep{Luridiana+2015}.
The distances in \cref{tbl:sample} were compiled by \citet{Anand+2021a} and are based on \citet{Freedman+2001,Nugent+2006,Jacobs+2009,Kourkchi+2017,Shaya+2017,Kourkchi+2020,Anand+2021a,Scheuermann+2022}.

\section*{Data availability}
The MUSE data underlying this article are presented in \citet{Emsellem+2022} and are available from the ESO archive\footnote{\url{https://archive.eso.org/scienceportal/home?data_collection=PHANGS}} and the CADC\footnote{\url{https://www.canfar.net/storage/vault/list/phangs/RELEASES/PHANGS-MUSE}}. 
The \textit{HST} data are presented in \citet{Lee+2022} and are available on the PHANGS--\textit{HST} website\footnote{\url{https://archive.stsci.edu/hlsp/phangs-hst}}. 
The catalogue with the background subtracted equivalent width and the matched catalogue with the \texttt{one-to-one sample} are both available in the online supplementary material of the journal. The code for this project can be found at
\begin{center}
\url{https://github.com/fschmnn/gas-around-stars}
\end{center}

%%%%%%%%%%%%%%%%%%%%%%%%%%%%%%%%%%%%%%%%%%%%%%%%%%

%%%%%%%%%%%%%%%%%%%% REFERENCES %%%%%%%%%%%%%%%%%%

\bibliographystyle{mnras}
\bibliography{paper} 

%%%%%%%%%%%%%%%%%%%%%%%%%%%%%%%%%%%%%%%%%%%%%%%%%%

%%%%%%%%%%%%%%%%% APPENDICES %%%%%%%%%%%%%%%%%%%%%

\appendix
\input{"tex/06appendix"}

%%%%%%%%%%%%%%%%%%%%%%%%%%%%%%%%%%%%%%%%%%%%%%%%%%

\vspace{4mm}

\noindent {\itshape
\hypertarget{ARI}{$^{1}$Astronomisches Rechen-Institut, Zentrum f\"{u}r Astronomie der Universit\"{a}t Heidelberg, M\"{o}nchhofstra\ss e 12-14, 69120 Heidelberg, Germany} \\
\hypertarget{Bonn}{$^{2}$Argelander-Institut f\"{u}r Astronomie, Universit\"{a}t Bonn, Auf dem H\"{u}gel 71, D-53121 Bonn, Germany} \\
\hypertarget{INAF}{$^{3}$INAF -- Osservatorio Astrofisico di Arcetri, Largo E. Fermi 5, I-50157 Firenze, Italy} \\
\hypertarget{Perth}{$^{4}$International Centre for Radio Astronomy Research, University of Western Australia, 7 Fairway, Crawley, 6009 WA, Australia} \\
\hypertarget{Riverside}{$^{5}$Department of Physics and Astronomy, University of California, Riverside, CA 92501, USA} \\
\hypertarget{Gemini}{$^{6}$Gemini Observatory / NSF’s NOIRLab, 950 N. Cherry Avenue, Tucson, AZ 85719, USA} \\
\hypertarget{Steward}{$^{7}$Steward Observatory, University of Arizona, Tucson, AZ 85721, USA} \\
\hypertarget{Arizona}{$^{8}$Department of Physics and Astronomy, University of Arizona, USA} \\
\hypertarget{Alberta}{$^{9}$Department of Physics, University of Alberta, Edmonton, AB T6G2E1, Canada} \\
\hypertarget{Carnegie}{$^{10}$Observatories of the Carnegie Institution for Science, 813 Santa Barbara Street, Pasadena, CA 91101, USA} \\
\hypertarget{Chile}{$^{11}$Departamento de Astronom\'ia, Universidad de Chile, Camino del Observatorio 1515, Las Condes, Santiago, Chile} \\
\hypertarget{Antofagasta}{$^{12}$Centro de Astronomía (CITEVA), Universidad de Antofagasta, Avenida Angamos 601, Antofagasta, Chile} \\
\hypertarget{Wyoming}{$^{13}$Department of Physics \& Astronomy, University of Wyoming, Laramie, WY 82071, USA} \\
%$^{14}$Caltech-IPAC, 1200 E. California Blvd., Pasadena, CA 91125, USA} \\
\hypertarget{Stockholm}{$^{14}$The Oskar Klein Centre for Cosmoparticle Physics, Department of Physics, Stockholm University, AlbaNova, Stockholm, SE-106 91, Sweden} \\
\hypertarget{ESO}{$^{15}$European Southern Observatory, Karl-Schwarzschild Stra\ss e 2, D-85748 Garching bei M\"{u}nchen, Germany} \\
\hypertarget{Lyon}{$^{16}$Univ Lyon, Univ Lyon 1, ENS de Lyon, CNRS, Centre de Recherche Astrophysique de Lyon UMR5574, F-69230 Saint-Genis-Laval, France} \\
\hypertarget{ITA}{$^{17}$Universit\"{a}t Heidelberg, Zentrum f\"{u}r Astronomie, Institut f\"{u}r Theoretische Astrophysik, Albert-Ueberle-Str. 2, 69120, Heidelberg, Germany} \\
\hypertarget{Canberra}{$^{18}$Research School of Astronomy and Astrophysics, Australian National University, Canberra, ACT 2611, Australia} \\
\hypertarget{ARC}{$^{19}$ARC Centre of Excellence for All Sky Astrophysics in 3 Dimensions (ASTRO 3D), Australia} \\
\hypertarget{Harvard}{$^{20}$Center for Astrophysics, Harvard \& Smithsonian, 60 Garden St, Cambridge, MA 02138, USA} \\
\hypertarget{IWR}{$^{21}$Universit\"{a}t Heidelberg, Interdisziplin\"{a}res Zentrum f\"{u}r Wissenschaftliches Rechnen, Im Neuenheimer Feld 205, D-69120 Heidelberg, Germany} \\
\hypertarget{cool}{$^{22}$Cosmic Origins Of Life (COOL) Research DAO, coolresearch.io} \\
\hypertarget{STsci}{$^{23}$Space Telescope Science Institute, 3700 San Martin Dr., Baltimore, MD 21218, USA} \\
\hypertarget{Ohio}{$^{24}$Department of Astronomy, The Ohio State University, 140 West 18th Avenue, Columbus, OH 43210, USA} \\
\hypertarget{Colombus}{$^{25}$Center for Cosmology and AstroParticle Physics, 191 West Woodruff Ave., Columbus, OH 43210, USA} \\
\hypertarget{Taiwan}{$^{26}$Department of Physics, Tamkang University, No.151, Yingzhuan Road, Tamsui District, New Taipei City 251301, Taiwan} \\
\hypertarget{Cantoblanco}{$^{27}$Departamento de F\'isica de la Tierra y Astrof\'isica, Universidad Complutense de Madrid, 28040, Madrid, Spain} \\
\hypertarget{MPIA}{$^{28}$Max Planck Institut für Astronomie, K\"{o}nigstuhl 17, 69117 Heidelberg, Germany} \\
\hypertarget{JohnHopkins}{$^{29}$Department of Physics and Astronomy, The Johns Hopkins University, Baltimore, MD 21218, USA} \\
\hypertarget{Oxford}{$^{30}$Sub-department of Astrophysics, Department of Physics, University of Oxford, Keble Road, Oxford OX1 3RH, UK} 
}

% Don't change these lines
\bsp	% typesetting comment
\label{lastpage} % does not work because only a figure is on the last page
\end{document}

% End of mnras_template.tex