\begin{table}[t]
    \centering
    \begin{adjustbox}{width=0.99\linewidth,center}
    \begin{tabular}{@{}c|c|ccc|ccc@{}}
        \hline
        & \multirow{2}{*}{Method} & \multicolumn{3}{c|}{Car $\text{AP}_{3D}$ $\text{IoU}\geqslant0.7$} & \multicolumn{3}{c}{Ped. $\text{AP}_{3D}$ $\text{IoU}\geqslant0.5$} \\
        & & Easy & Mod. & Hard & Easy & Mod. & Hard \\ \hline
        \ding{172} & Baseline & 22.71 & 17.56 & 14.68 & 9.35 & 6.77 & 5.58 \\
        \ding{173} & + 3D bbox jitter filter ~\cite{xu2021softteacher}& 22.50 & 16.71 & 14.40 & 8.36 & 6.30 & 5.07 \\
        \ding{174} & + cls.\&loc. weight filter & 21.86 & 17.20 & 14.31 & 9.04 & 7.25 & 5.75 \\
        % \ding{175} & + CRS filter w/o critical module & 22.01 & 16.18 & 13.95 & 7.05 & 5.58 & 4.26 \\
        \ding{175} & + CRS filter w/o critical module & 21.41 & 16.58 & 14.05 & 8.05 & 6.23 & 5.01 \\
        \ding{176} & + CRS filter w/ critical module& \textbf{22.87} & \textbf{17.65} & \textbf{14.83} & \textbf{10.99} & \textbf{8.25} & \textbf{6.72} \\
        \hline
    \end{tabular}
    \end{adjustbox}
    \vspace{0.2cm}
    \caption{\textbf{Effectiveness of CRS.} We compare our critical module with other counterparts in filtering pseudo labels. The prominent improvement on pedestrian indicates that the CRS can effectively find informative samples during training.}
\label{tab:crs}
\end{table}
