%%%%%%%%% ABSTRACT
% In this paper, we handle the challenging semi-supervised monocular 3D object detection problem with a general framework.

\begin{abstract}

    In this paper, we improve the challenging monocular 3D object detection problem with a general semi-supervised framework. Specifically, having observed that the bottleneck of this task lies in lacking reliable and informative samples to train the detector, we introduce a novel, simple, yet effective `Augment and Criticize' framework that explores abundant informative samples from unlabeled data for learning more robust detection models. In the `Augment' stage, we present the Augmentation-based Prediction aGgregation (APG), which aggregates detections from various automatically learned augmented views to improve the robustness of pseudo label generation. Since not all pseudo labels from APG are beneficially informative, the subsequent `Criticize' phase is presented. In particular, we introduce the Critical Retraining Strategy (CRS) that, unlike simply filtering pseudo labels using a fixed threshold (\textit{e.g.,} classification score) as in 2D semi-supervised tasks, leverages a learnable network to evaluate the contribution of unlabeled images at different training timestamps. This way, the noisy samples prohibitive to model evolution could be effectively suppressed. To validate our framework,  we apply it to MonoDLE~\cite{ma2021monodle} and MonoFlex~\cite{zhang2021monoflex}. The two new detectors, dubbed 3DSeMo$_{\text{DLE}}$ and 3DSeMo$_{\text{FLEX}}$, achieve state-of-the-art results with remarkable improvements for over 3.5\% $AP_{3D/BEV} (Easy)$ on KITTI, showing its effectiveness and generality. Code and models will be released.
    
    % and 2.96\% $AP$ on nuScenes
    
    % Code and models will be released.
    
    % which applies automatically learned transformations to unlabeled images and aggregates detections from various augmented views as pseudo labels
    
    
    % Interestingly, in addition to pseudo label generation, APG provides uncertainty measurement on a sample as an indicator of its label quality via maximum likelihood estimation, further enhancing the reliability of samples.
    
    % In this paper, we tackle the challenging problem of semi-supervised monocular 3D object detection with a general framework. Specifically, having observed that the bottleneck of this task lies in lacking of reliable and informative samples from unlabeled data for detector learning, we propose a novel simple yet effective `Augment-Criticize' pipeline that mines abundant informative samples for robust detection. To be more specific, in the `Augment' stage, we present Augmentation-based Prediction aGgregation (APG), which applies automatically learned transformations to unlabeled images and aggregates detections from various augmented views as pseudo labels. Interestingly, in addition to the pseudo label generation, APG provides uncertainty measurement of a sample as an indicator of its pseudo label quality via maximum likelihood estimation, further enhancing reliability of samples. Since not all the pseudo labels in APG are beneficially informative, the subsequent `Criticize' phase is introduced. In particular, we present a Critical Retraining Strategy (CRS) that, unlike simply filtering pseudo labels with a fixed threshold (\textit{e.g.,} classification score) as in 2D semi-supervised tasks, leverages a learnable network to evaluate the contribution of unlabeled images at different training timestamps. By doing so, noisy samples prohibitive for model evolution can be effectively suppressed. To validate the proposed approach, we apply `Augment-Criticize' framework to MonoDLE~\cite{ma2021monodle} and MonoDETR~\cite{zhang2022monodetr} and achieve state-of-the-art results with consistent improvements, evidencing its effectiveness and generality. Code will be released.
    
    % The goal of our work is to establish a general pipeline for employing semi-supervised learning (SSL) to monocular 3D (Mono3D) object detection. The effective adoption of SSL in Mono3D detection faces two main challenges, \textit{i.e.,} robustly generating pseudo labels for unlabeled data and adaptively selecting informative samples meanwhile discarding noisy ones in training. However, it's non-trivial to achieve this in Mono3D due to the unreliable detections of de facto methods. To fulfill this purpose, we propose a framework based on ``augment and criticize'' policy. Augment policy applies automatically learned transformations to an unlabeled image and aggregates the detection results from different augmented views. This method not only improves the recall of the generated pseudo labels, but also provides uncertainty measurement of a sample via maximum likelihood estimation, which can be used as the indicator of pseudo label quality. Criticize policy aims at chopping off noisy samples that are prohibitive for model evolution. Unlike simply filtering pseudo labels with a fixed threshold (\textit{e.g.,} classification score) as in 2D SSL methods, we construct a learnable network to adaptively evaluate the contribution of unlabeled images, for which the importance of a sample changes along different training timestamps. Experimental results on MonoDLE~\cite{ma2021monodle} and MonoDETR~\cite{zhang2022monodetr} demonstrate the effectiveness and generality of our framework. Code and models will be released upon publication.
    
   
\end{abstract}