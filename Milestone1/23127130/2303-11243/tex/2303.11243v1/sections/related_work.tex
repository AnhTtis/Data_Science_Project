\section{Related Work}
\subsection{Monocular 3D Object Detection}
3D object detection serves as a fundamental yet crucial task for agents perceiving the surrounding 3D world and making reasonable planning~\cite{arnold2019survey}. Given the practical demand to apply detectors on portable devices and high-redundancy systems, monocular 3D object detection from cameras (Mono3D) has drawn much more attention in recent years~\cite{brazil2020kinematic,ding2020d4lcn,zhang2021monoflex,shi2021monorcnn,ma2021monodle,lu2021gup,huang2022monodtr}. Caused by the ill-posed property, earlier work adopts auxiliary depth networks~\cite{weng2019plpc} to generate pseudo point cloud or lift 2D features into 3D space~\cite{roddick2018orthographic}, and then apply 3D detectors to achieve localization and recognition. These kinds of methods focus on the quality of depth estimation~\cite{wang2020taskawaredp,park2021dd3d} and strategy to inv-project image features into the 3D space~\cite{reading2021caddn,zhou2022persdet}. While achieving promising results, the highly dependence on consuming depth labels and onerous pipelines make them hard to practical deployment. Recently, another line of research ~\cite{liu2020smoke,ma2021monodle,zhang2021monoflex,wang2021fcos3d} proposes to design a neater framework in an end-to-end manner like 2D detection~\cite{zhou2019centernet,tian2019fcos}, where the depth estimation is designed as an isolated branch of regression heads. These methods predict object centers on 2D images paired with the instance depth. Associated with camera parameters, 3D object attributes can be recovered on the fly. Representative methods like SMOKE~\cite{liu2020smoke} and MonoDLE~\cite{ma2021monodle} adopt the CenterNet-based architecture~\cite{zhou2019centernet} whereas FCOS3D~\cite{wang2021fcos3d} and PGD~\cite{wang2022pgd} extend FCOS~\cite{tian2019fcos} into a 3D version. In this paper, we aim to design a semi-supervised framework based on this prevalent kind of methods~\cite{ma2021monodle,zhang2021monoflex} to further investigate the potential of monocular 3D object detection.

\subsection{Semi-Supervised Learning}

Semi-Supervised Learning (SSL) attracts growing interest by virtue of its aim to benefit from massive unlabeled data meanwhile preserving the nutrition in a small amount of human-annotated samples~\cite{van2020survey}. A number of approaches have been proposed to address SSL. Due to the space limitation, we only review self-training based methods, one of the most engaging directions in recent SSL.

The idea behind self-training is to train a teacher model with the human-annotated data, utilize it to generate pseudo labels for unlabeled data, and retrain a student model with the combination of human- and self-annotated data. It has been widely adopted to various tasks such as image classification~\cite{tarvainen2017meanteacher,berthelot2019mixmatch,sohn2020fixmatch,xie2020self}, semantic segmentation~\cite{yang2022st++,liu2022perturbed,yuan2022semi}, and 2D object detection~\cite{xu2021softteacher,zhou2022denseteacher,zhang2022semi}. While with distinct task properties, the critical insight in these approaches is to design better strategies for high-quality pseudo-label generation and effective student model retraining. For instance, Mean-Teacher~\cite{tarvainen2017meanteacher} proposes a temporal ensembling strategy to facilitate retraining. Soft-Teacher~\cite{xu2021softteacher} applies the classification score to reweight supervision on the student model and propose a 2D bbox jitter strategy to filter unreliable labels in 2D object detection. ST++~\cite{yang2022st++} adopts strong augmentations on pseudo-labeled segmentation samples and leverages evolving stability during training to prioritize reliable labels. In this paper, we attempt to investigate the effectiveness of self-training in Mono3D. Compared with aforementioned tasks, approaches for Mono3D are lagging extremely far behind in performance, making more elaborate designs on the pseudo-label generation and ratraining strategy \textit{de facto} requirements for an effective Mono3D semi-supervise framework.