\section{Introduction}

Monocular 3D (Mono3D) object detection plays an essential role for agents to understand the real world, but reasoning object's 3D location from a single image is an ill-posed problem~\cite{eigen2014depth}. Fortunately with the flourishing of deep learning, the community can seek to circumvent the mathematical depth calculation and solve the task in a data-driven manner. A plethora of models have been designed following this scheme and demonstrate remarkable performance. But current data volume in Mono3D object detection training is nowhere enough for a human-level ability. Considering the tremendous cost for manually annotating 3D boxes in larger-scale images, we argue that semi-supervised learning could be an economical substitute.

\begin{figure}[t]
    \centering
    \footnotesize
    \includegraphics[width=0.98\linewidth]{images/teaser_v.pdf}
    \caption{\textbf{Motivation and Proposal.} The differences between our method (\textcolor{red}{red}) and previous semi-supervised learning framework (\textcolor{green}{green}) in pseudo label (PL) generation and student model retraining. The introduced framework can improve detection recall by observing different views of an image (\textcolor{red}{red} dots in (a)), and dynamically determine when to discard an unlabeled sample during training (line-chart in (b)) by the learnable critical module.} 
    \label{fig::teaser}
\end{figure}

% cure for further unveiling the power of monocular 3D object detection methods. 

Semi-supervised learning benefits from massive unlabeled data meanwhile preserving the nutrition in a small amount of human-annotated samples. It has been widely explored in 2D tasks like classification~\cite{tarvainen2017meanteacher,berthelot2019mixmatch,sohn2020fixmatch,xie2020self}, detection~\cite{xu2021softteacher,zhou2022denseteacher,zhang2022semi} and segmentation~\cite{yang2022st++,liu2022perturbed,yuan2022semi}, yet surprisingly less exposed to Mono3D object detection. One possible reason is that the ill-posed task definition of Mono3D object detection makes algorithms not robust to complex environments and eventually incurs noisy pseudo-label generation on the unlabeled data. For example, the state-of-the-art method MonoDLE~\cite{ma2021monodle} only achieves 13.66 AP@0.7 on KITTI~\cite{geiger2012kitti}, leading to an unpleasant detection recall. Previous work~\cite{lian2022semimono3d} constructs multi-view consistency learning to improve the effectiveness and resilience of semi-supervised training in Mono3D object detection, however, stereo or multi-view requirement is overstrict and maybe not easily met in practical scenarios. Instead, in this work, we tend to construct a semi-supervised Mono3D detection framework using only single-camera, single-view, and single modality input, without relaying stereo, multi-view, or any other modality (like LIDAR) information.    

% Although decreasing the threshold to filter predicted boxes can alleviate this problem, it inevitably brings detrimental false-positive (FP) noise. Besides, carefully

Revisiting the roadmap of semi-supervised learning, the essence always goes to \textit{how to robustly generate high-quality pseudo labels on unlabeled data} and \textit{how to properly take advantage of these pseudo labels.} The former determines the qualities of training samples, while the latter controls the evolving direction of a model. Yet as aforementioned, it's non-trivial to collect precise pseudo labels under the purely monocular 3D premise. Although carefully choosing a threshold to select the detection boxes may alleviate this issue, as done in some works~\cite{xu2021softteacher,chen2022labelmatch}, it would overfit a specific model, limiting the resilience of the semi-supervised method. \textbf{A more robust pseudo label generation strategy is thus necessary.} Another essential challenge for semi-supervised Mono3D object detection is the lack of adaptive training strategy to handle pseudo labels with different qualities. Intuitively the high-quality pseudo labels should contribute more to model updating, and the influence of low-quality samples should be suppressed. The jitter score of a detection box is adopted by some 2D approaches to filter the samples~\cite{xu2021softteacher}. But such a handcrafted rule-based manner cannot guarantee the model evolves along the right direction, since a sample could show diverse effects to the model at different training steps. \textbf{Therefore, a more adaptive strategy is required to guide the training on unlabeled data.} 

% The core idea is replacing the one-pass pseudo label generation to aggregating multi-pass predictions. APG uses multiple observations of a image, it can thus reduce the detection bias and improve the robustness of pseudo label generation.

To this end, we propose the ``Augment and Criticize'' framework to approach the  two challenges in semi-supervised monocular 3D object detection. In particular, the augmented-based prediction aggregation, dubbed \textbf{APG}, aims at robustly generating pseudo labels on the unlabeled data. The core idea is aggregating predictions from a image's different observations to reduce the detection bias and improve the robustness of pseudo label generation. The transformation parameters for an observation are automatically learned by an efficient reward-based Tree-Parzen-Window algorithm~\cite{bergstra2011algorithms}. We  find that content-based transformations like color-jitter are useless, but geometry-based transformations like resize and crop show much affirmative effect for improving detection recall. Moreover, critical retraining strategy, dubbed \textbf{CRS}, is imposed to adaptively update the model with noisy pseudo labels. It consists of a memory bank to preserve evaluation images and a critical module to determine which pseudo label benefits to update the model. At each training step, the loss of each pseudo label corresponds to a update choice of the model. The critical module sampling images from the memory bank to evaluate whether this update improves model capability. If the model resembles to a worse one, the update would be discarded (self-criticise). During the cyclical updating of the memory bank, the critical module gradually encodes the knowledge of the whole evaluation set to its weight parameters, and it can thus be more and more powerful along training period. We apply our semi-supervised framework to classical methods MonoDLE~\cite{ma2021monodle} and MonoDETR~\cite{zhang2022monodetr}. Experimental results demonstrate that the proposed methods can improve both of them for about 3\% AP@40 on KITTI, which shows its effectiveness and versatility.

% This is impressively different from the conclusions claimed by the auto-augmentation works in other fields~\cite{???}.

%% Cite paper: Learning Data Augmentation with Online Bilevel Optimization for Image Classification

% the optimal image transformation parameters for augmenting the raw image usually dramatically change the object distribution to help the model locate objects that are not easily detected. 
% This is distinct from the generally adopted data augmentation in deep net training which avoiding change the input distribution too much. 

Our contributions are summarized as three-fold: 1) We propose the augmented-based prediction aggregation (APG) method to improve the quality of the generated pseudo labels in Mono3D detection. 2) We introduce a critical retraining strategy (CRS) to adaptively evaluate the contribution of each pseudo label and effectively guide the direction of the model evolving. 3) We integrate our semi-supervised framework into different methods, and the experimental results have evidenced its effectiveness.  

% via cyclically memory bank updating. The memory bank is a queue that contains all images from evaluation set.  

% For example, as in Fig.X, the white car away from the camera are successfully located after moving to the image center.