\section{Introduction}

\begin{figure}[!t]
    \centering
    \footnotesize
    \includegraphics[width=\linewidth]{images/teaser_v.pdf}
    \caption{\textbf{Motivation and Proposal.} The differences between our method (\textcolor{red}{red}) and previous semi-supervised learning framework (\textcolor{green}{green}) in pseudo label (PL) generation and student model retraining. The introduced framework can improve detection recall by observing different views of an image (\textcolor{red}{red} dots in (a)), and dynamically determine when to discard an unlabeled sample during training (line-chart in (b)) by the learnable critical module.} 
    \label{fig::teaser}
\end{figure}

Monocular 3D (Mono3D) object detection plays an essential role for agents in understanding the real world. However, due to its ill-posed property~\cite{eigen2014depth}, Mono3D detection remains an open problem. Recently, with the flourishing of deep learning, the community seeks to circumvent the mathematical depth estimation and solve the task in a data-driven manner. A plethora of deep models have been designed and demonstrated remarkable performance~\cite{liu2020smoke,ma2021monodle,zhang2022monodetr,zhang2021monoflex,wang2021fcos3d}. Despite this, current data volume in Mono3D object detection is nowhere enough for achieving a human-level 3D sensing ability. Since manually annotating 3D boxes in larger-scale data is costly, we argue that semi-supervised learning could be an economical substitute.

% Despite this, these methods heavily rely on a large volume of labeled data for Mono3D object detection training, which is costly. To remedy this, we argue that \emph{semi-supervised learning} has great potential for economical Mono3D object detection. 

Semi-supervised learning, given a \emph{small} amount of manually annotated training samples, aims to explore beneficial information from massive \emph{unlabeled} data for training. It has been extensively studied in 2D vision tasks, such as classification~\cite{tarvainen2017meanteacher,berthelot2019mixmatch,sohn2020fixmatch,xie2020self}, detection~\cite{xu2021softteacher,zhou2022denseteacher,zhang2022semi} and segmentation~\cite{yang2022st++,liu2022perturbed,yuan2022semi}, yet surprisingly less explored for Mono3D detection. One possible reason is that its ill-posed task definition makes algorithms suffer from complex environments and eventually incurs noisy pseudo-label generation on the unlabeled data, degrading performance. A previous method of~\cite{lian2022semimono3d} attempts to construct the multi-view consistency for semi-supervised learning of Mono3D object detection. It is, however, restricted to stereo or multi-view requirement that may be not easily guaranteed in practical scenarios. Instead, in this work, we tend to construct a semi-supervised Mono3D detection framework using only single-camera, -view, and -modality inputs, without relying on the multi-view, or any other modality (\eg, LiDAR) information.

Revisiting the road map of semi-supervised learning, we observe that, the bottleneck always lies in the lack of abundant reliable and informative samples
from unlabeled data for training. Specifically, two challenges are faced: \emph{How to robustly generate high-quality pseudo labels for unlabeled data} \textbf{and} \emph{How to properly leverage these pseudo labels for effective learning}. The former mainly focuses on the quality
of training samples, while the latter is related to the evolving direction of the detection model. Yet as discussed earlier, it is \emph{non-trivial} to generate precise pseudo labels under the pure monocular 3D premise. Previous methods in 2D tasks (\eg,~\cite{xu2021softteacher,chen2022labelmatch}) try to alleviate this issue by carefully choosing a threshold for detection box selection (see Fig.~\ref{fig::teaser}). This handcraft selection, however, may cause overfit to a specific model, limiting the resilience of the semi-supervised method. Addressing this problem, \textbf{\emph{a robust pseudo label generation strategy is thus necessary}}. In addition, another essential challenge for semi-supervised Mono3D object detection is the lack of an adaptive training strategy to deal with pseudo labels with different qualities. Intuitively the high-quality pseudo labels should contribute more to model updating, and meanwhile the influence of low-quality samples should be suppressed. The classification score (or its variants) of a detection box is adopted in 2D methods (\eg,~\cite{xu2021softteacher}) to filter out the low-quality samples. But such a handcrafted rule-based manner is hard to guarantee that the model evolves along the right direction, because a sample could show diverse effects to the model at different training steps. Therefore, \textbf{\emph{a more adaptive mechanism is needed to guide the training on unlabeled data}}.

% The core idea is replacing the one-pass pseudo label generation to aggregating multi-pass predictions. APG uses multiple observations of a image, it can thus reduce the detection bias and improve the robustness of pseudo label generation.

\vspace{0.5em}
\noindent
\textbf{Contribution.} Motivated by the above, we propose a novel `Augment-Criticize' framework to approach the two challenges in semi-supervised monocular 3D object detection.

In specific in `Augment' stage, we introduce a simple yet effective Augmentation-based Prediction aGgregation strategy, dubbed \textbf{APG}, that aims at robustly generating pseudo labels for unlabeled data (\ie, the first challenge). The core idea is to aggregate predictions from different observations of an image, which we find effectively reduces the detection bias and improves the robustness of pseudo label generation (see Fig.~\ref{fig::teaser} again). In order to avoid handcrafted selection, the transformation parameters for generating an observation are automatically learned by using an efficient reward-based Tree-Parzen-Window algorithm~\cite{bergstra2011algorithms}. Interestingly, we find that, content-based transformations such as color-jitter are not helpful, yet geometry-based transformations like resize and crop exhibit much affirmative effect, for improving detection recall, providing guidance for future research.

Since not all the pseudo labels from APG is beneficially informative, an adaptive strategy is desired to exploit these pseudo labels for effective model training (\ie, the second challenge), which motivates the proposed `Criticize' stage. More specifically, in this stage, a Critical Retraining Strategy (\textbf{CRS}) is imposed to adaptively update the model with noisy pseudo labels. Particularly, CRS contains a memory bank to preserve evaluation images and a critical module to determine which pseudo label benefits to update the model. At each training step, the loss of each pseudo label corresponds to an update choice of the model. The critical module samples images from the memory to determine whether this update improves model capability. If the model resembles to a worse one, the update would be discarded (self-criticise). During the cyclical updating of the memory bank, the critical module gradually encodes the knowledge of the whole evaluation set to its weight parameters, and therefore it becomes more and more powerful along training period. 

To verify our `Augment-Criticize' framework, we apply it to MonoDLE~\cite{ma2021monodle} and MonoDETR~\cite{zhang2022monodetr}. In experiments, compared with baselines, our semi-supervised detectors, 3DSeMo$_{\text{DLE}}$ and 3DSeMo$_{\text{DETR}}$, achieve consistent improvements for about 3\% $\mathtt{CAR (Mod.)~ AP_{3D}}$ on KITTI, which shows the effectiveness and versatility of our method.



% This is impressively different from the conclusions claimed by the auto-augmentation works in other fields~\cite{???}.

%% Cite paper: Learning Data Augmentation with Online Bilevel Optimization for Image Classification

% the optimal image transformation parameters for augmenting the raw image usually dramatically change the object distribution to help the model locate objects that are not easily detected. 
% This is distinct from the generally adopted data augmentation in deep net training which avoiding change the input distribution too much. 

In summary, we make the following contributions: 

\noindent
(\textbf{1}) \emph{We propose a novel `Augment-Criticize' framework for semi-supervised Mono3D object detection.}

\noindent
(\textbf{2}) \emph{We propose an augmented-based prediction aggregation to improve the quality of pseudo labels for unlabeled data.}

\noindent
(\textbf{3}) \emph{We propose a critical retraining strategy that adaptively evaluates each pseudo label for effective model training. }

\noindent
(\textbf{4}) \emph{We integrate our semi-supervised framework into different methods, and results evidence its effectiveness. }


% via cyclically memory bank updating. The memory bank is a queue that contains all images from evaluation set.  

% For example, as in Fig.X, the white car away from the camera are successfully located after moving to the image center.