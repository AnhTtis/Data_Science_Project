\begin{figure}[t]
\centering
\subcaptionbox{\scriptsize Disclosed Positions (\textbf{\datavec{}})}{\includegraphics[width=0.50\linewidth,trim={0cm 0.8cm 0cm, 1.4cm},clip]{figures/early_leakage/chair_leak_d2v_0_gray_crop.png}}%
\hfill
\subcaptionbox{\scriptsize Concealed Positions (\textbf{\name{}})}{\includegraphics[width=0.50\linewidth,trim={0cm 0.8cm 0cm, 1.4cm},clip]{figures/early_leakage/chair_leak_p2v_0_gray_crop.png}}%
\caption{\textbf{Leakage Of Positional Information.}
The center points\,\colordot{black} are associated with the corresponding masked embeddings of masked-out patches (\reffig{model}, \protect\maskembedding{}).
\textbf{(a)} As \datavec{} feeds the masked embeddings to the student, the positions of masked patches are disclosed to the student.
Consequently, the %
overall shape of the chair remains clearly visible to the student, thus making the learning objective less challenging.
\textbf{(b)} To overcome this limitation, \name{} excludes masked embeddings from the student and subsequently feeds them only to a decoder.
As a result, several parts of the chair are truly masked from the student input. 
}
\label{fig:early_leakage}
\end{figure}