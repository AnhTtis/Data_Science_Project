\section{Experiments}\label{sec:experiments}
In this section, we describe the self-supervised pre-training of \name{} on ShapeNet\,\cite{chang2015shapenet} (\refsec{pretraining}).
Next, we compare \name{} with top-performing self-supervised approaches and our baseline method \datavec{} on three well-established datasets and four downstream tasks (\refsec{main_results}).
Finally, we put the spotlight on the architectural changes from our data2vec adaptation for point clouds to our proposed model \name{} which address the unique challenges of 3D point clouds (\refsec{analysis}).
In the supplementary material, we provide detailed hyperparameters of our model.
Code and checkpoints will be made available.


\subsection{Self-Supervised Pre-training}
\label{sec:pretraining}
Following the pre-training protocol propagated by Point-BERT\,\cite{yu2021pointbert}, Point-MAE\,\cite{pang2022pointmae} and Point-M2AE\,\cite{zhang2022pointm2ae}, we pre-train \name{} on the training split of ShapeNet\,\cite{chang2015shapenet} consisting of \num{41952} synthetic 3D meshes of $55$ categories, \eg `\emph{chair}', `\emph{guitar}', `\emph{airplane}'.
We set the number of Transformer blocks to $12$ with an internal dimension of $384$. %
To pre-train our point-based approach, we uniformly sample \num{8192} points from the surfaces of the objects and then resample \num{1024} points using farthest point sampling\,\cite{qi2017pointnetplusplus}.
During the point cloud embedding step we sample $n$$=$$64$ center points and $k$$=$$32$ nearest neighbors.
We train \name{} with a batch size of $512$ for $800$ epochs using the AdamW\,\cite{loshchilov2018adamw} optimizer and a cosine learning rate decay\,\cite{loshchilov2017ICLR} with a maximal learning rate of $10^{-3}$ after $80$ epochs of linear warm-up.
For \datavec{}, we increase the batch size and learning rate to $2048$ and $2$$\times$$10^{-3}$, respectively, as this empirically led to better results.
Following data2vec\,\cite{baevski2022data2vec}, we set $\beta$$=$$2$ for the Smooth L1 loss and average the last $K$$=$$6$ blocks of the teacher.
We use minimal data augmentations during pre-training: we randomly scale the input with a factor between $[0.8, 1.2]$ and rotate around the gravity axis.
Pre-training takes roughly $18$\,hours on a single V100 GPU.
\begin{table}
    \centering
    \caption{
        \textbf{Shape Classification on ModelNet40\,\cite{wu2015modelnet40}.}
        We report the overall accuracy with and without voting, as well as the mean per-class accuracy (mAcc).
    }
    \label{tab:modelnet_results}
    \begin{tabu}{l@{\hskip 0.06in}cp{0.001cm}@{\hskip 0.06in}cp{0.001cm}cp{0.001cm}}
        \toprule
         & \multicolumn{4}{c}{Overall Accuracy} & \multicolumn{2}{c}{mAcc}         \\
        \cmidrule(lr){2-5}
         Method        & $+$Voting                 && $-$Voting         &&                   \\
        \midrule
        Transf.-OcCo\,\cite{yu2021pointbert} & 92.1 && -- && -- \\
        ParAE\,\cite{eckart2021parae} & -- && 92.9 && --\\
        STRL\,\cite{huang2021strl} & 93.1 && -- && -- \\
        Point-BERT\,\cite{yu2021pointbert}          & 93.2                  && 92.7             && --                 \\
        PointGLR\,\cite{rao2020pointglr} & -- && 93.0 && -- \\
        OcCo\,\cite{wang2021occo} & 93.0 && -- && -- \\
        MaskPoint\,\cite{liu2022maskpoint} & 93.8 && -- && -- \\
        Point-MAE\,\cite{pang2022pointmae}                           & 93.8                  && 93.2             && --             \\
       Point-M2AE\,\cite{zhang2022pointm2ae}              & 94.0                  && 93.4             && --                 \\
        \arrayrulecolor{black!10}\midrule\arrayrulecolor{black}
        from scratch                         & 93.3                  &\multirow{2}{*}{\hspace{-0.2cm}\ArrowDown{\footnotesize \textcolor{darkgreen}{$+0.3$}}}& 93.0             & \multirow{2}{*}{\hspace{-0.2cm}\ArrowDown{\footnotesize \textcolor{darkgreen}{$+0.3$}}} & 89.8 & \multirow{2}{*}{\hspace{-0.0cm}\ArrowDown{\footnotesize \textcolor{darkgreen}{$+0.7$}}}             \\
        \datavec{}     & 93.6 &\multirow{2}{*}{\hspace{-0.2cm}\ArrowDown{\footnotesize \textcolor{darkgreen}{$+1.2$}}}                 & 93.3             &\multirow{2}{*}{\hspace{-0.2cm}\ArrowDown{\footnotesize \textcolor{darkgreen}{$+1.4$}}}& 90.5 & \multirow{2}{*}{\hspace{-0.0cm}\ArrowDown{\footnotesize \textcolor{darkgreen}{$+1.5$}}}             \\
        \textbf{\name{}} (Ours)   & $\textbf{94.8}$       && \textbf{94.7}    && \textbf{92.0}    \\
        \bottomrule
    \end{tabu}
\end{table}
\begin{table*}[t]
	\caption{\textbf{Part Segmentation Results on ShapeNetPart\,\cite{yi16siggraph}}.
		We report the mean IoU across all part categories mIoU$_C$ and the mean IoU across all instance mIoU$_I$, as well as the IoU for each categories.
	}
	\label{tab:ShapeNetPart}
	\newcommand{\fn}{\footnotesize}
	\centering
	\setlength{\tabcolsep}{2pt}
	\begin{tabularx}{\textwidth}{lcccYYYYYYYYYYYYYYYY}
		\toprule
		Methods                               & mIoU$_C$      & mIoU$_I$      &  & \fn aero & \fn bag  & \fn cap  & \fn car  & \fn chair & \fn earph & \fn guitar & \fn knife & \fn lamp & \fn laptop & \fn motor & \fn mug  & \fn pistol & \fn rocket & \fn skateb & \fn table \\
		\midrule
            Transf.-OcCo\,\cite{yu2021pointbert} & 83.4 & 85.1 && \fn 83.3 & \fn 85.2 & \fn 88.3 & \fn 79.9 & \fn 90.7 & \fn 74.1 & \fn 91.9 & \fn 87.6 & \fn 84.7 & \fn 95.4 & \fn 75.5 & \fn 94.4 & \fn 84.1 & \fn 63.1 & \fn 75.7 & \fn 80.8 \\
		Point-BERT\,\cite{yu2021pointbert}    & 84.1          & 85.6          &  & \fn 84.3 & \fn 84.8 & \fn 88.0 & \fn 79.8 & \fn 91.0  & \fn 81.7  & \fn 91.6   & \fn 87.9  & \fn 85.2 & \fn 95.6   & \fn 75.6  & \fn 94.7 & \fn 84.3   & \fn 63.4   & \fn 76.3   & \fn 81.5  \\
            MaskPoint\,\cite{liu2022maskpoint} & 84.4 & 86.0 & & \fn 84.2 & \fn 85.6 & \fn 88.1 & \fn 80.3 & \fn 91.2 & \fn 79.5 & \fn 91.9 & \fn 87.8 & \fn 86.2 & \fn 95.3 & \fn 76.9 & \fn 95.0 & \fn 85.3 & \fn 64.4 & \fn 76.9 & \fn 81.8 \\
		Point-MAE\,\cite{pang2022pointmae}    & 84.1          & 86.1          &  & \fn 84.3 & \fn 85.0 & \fn 88.3 & \fn 80.5 & \fn 91.3  & \fn 78.5  & \fn 92.1   & \fn 87.4  & \fn 86.1 & \fn 96.1   & \fn 75.2  & \fn 94.6 & \fn 84.7   & \fn 63.5   & \fn 77.1   & \fn 82.4  \\
		Point-M2AE\,\cite{zhang2022pointm2ae} & \textbf{84.9} & \textbf{86.5} &  & \fn --   & \fn --   & \fn --   & \fn --   & \fn --    & \fn --    & \fn --     & \fn --    & \fn --   & \fn --     & \fn --    & \fn --   & \fn --     & \fn --     & \fn --     & \fn --    \\
		\arrayrulecolor{black!10}\midrule\arrayrulecolor{black}
		from scratch                          & 84.1          & 85.7          &  & \fn 84.2 & \fn 84.8 & \fn 87.9 & \fn 80.2 & \fn 91.2  & \fn 79.3  & \fn 91.6   & \fn 87.8  & \fn 85.4 & \fn 96.1   & \fn 75.8  & \fn 94.7 & \fn 85.0   & \fn 64.7   & \fn 75.4   & \fn 81.1  \\
		\datavec{}  & 84.1          & 85.9          &  & \fn 84.7 & \fn 85.8 & \fn 87.4 & \fn 80.7 & \fn 91.4  & \fn 79.3  & \fn 91.7   & \fn 87.9  & \fn 85.2 & \fn 95.8   & \fn 76.2  & \fn 94.9 & \fn 83.6   & \fn 61.6   & \fn 78.5   & \fn 80.7  \\
		\textbf{\name{}} (Ours)               & 84.6          & 86.3          &  & \fn 85.2 & \fn 85.7 & \fn 88.5 & \fn 80.3 & \fn 91.5  & \fn 77.3  & \fn 91.8   & \fn 88.5  & \fn 85.7 & \fn 96.1   & \fn 77.3  & \fn 95.5 & \fn 84.9   & \fn 66.2   & \fn 77.0   & \fn 82.0  \\
		\bottomrule
	\end{tabularx}
\end{table*}

\subsection{Main Results on Downstream Tasks}
\label{sec:main_results}
In order to evaluate the effectiveness of \name{}'s self-supervised learning capabilities, we test \name{} against top-performing self-supervised methods on four different downstream tasks on well-established benchmarks.
To that end, we discard the teacher network as well as the decoder and append a task-specific head to the student network.
We then fine-tune the full network end-to-end for the specific task.
We provide detailed hyperparameters for all downstream tasks in the supplementary material.

\paragraph{Synthetic Shape Classification.}
\begin{figure}[t!]
\begin{tikzpicture}

\colorlet{crimson2143940}{m_red_border}
\definecolor{darkorange25512714}{RGB}{255,127,14}
\definecolor{darkslategray38}{RGB}{38,38,38}
\colorlet{forestgreen4416044}{m_green_border}
\definecolor{lightgray204}{RGB}{204,204,204}
\colorlet{steelblue31119180}{m_blue_border}

\begin{axis}[
width=1.1\linewidth,height=6.4cm,
axis x line=bottom,
axis y line=left,
x axis line style={-},
ytick distance=0.5,
minor y tick num=1,
ytick={89, 90, 91, 92, 93, 94.0},
y axis line style={-},
axis line style={white!80!black},
legend cell align={left},
legend style={
  fill opacity=0.8,
  draw opacity=1,
  text opacity=1,
  at={(0.98,0.03)},
  anchor=south east,
  draw=lightgray204,
  legend image post style={line width =1.5pt}
},
tick align=outside,
xlabel=\textcolor{darkslategray38}{\small Finetuning Epochs},
xmin=0, xmax=150,
xtick={10, 30, 50, 70, 90, 110, 130},
xtick style={color=darkslategray38},
y grid style={lightgray204},
x grid style={lightgray204},
every axis y label/.style={at={(0.001,1.02)},anchor=north west},
ylabel=\textcolor{darkslategray38}{\small Overall Accuracy (\%)},
ymin=88.8, ymax=94.2,
ytick style={color=darkslategray38}
]
\path [draw=steelblue31119180, fill=steelblue31119180, opacity=0.2, line width=0.32pt]
(axis cs:0,6.0916395423313)
--(axis cs:0,3.05240408827861)
--(axis cs:1,47.5413631896178)
--(axis cs:2,62.5199209650358)
--(axis cs:3,72.6565023760001)
--(axis cs:4,81.2533756097158)
--(axis cs:5,86.5207999944687)
--(axis cs:6,88.5602374871572)
--(axis cs:7,90.2415922284126)
--(axis cs:8,90.5456513166428)
--(axis cs:9,90.937332312266)
--(axis cs:10,91.0318752129873)
--(axis cs:11,91.3492699464162)
--(axis cs:12,91.3761139412721)
--(axis cs:13,91.240883966287)
--(axis cs:14,91.7747169733047)
--(axis cs:15,92.0043230056763)
--(axis cs:16,91.9773081938426)
--(axis cs:17,91.8962726990382)
--(axis cs:18,91.6327318300804)
--(axis cs:19,92.2812004884084)
--(axis cs:20,92.2676950693131)
--(axis cs:21,91.9433753192425)
--(axis cs:22,92.4297680457433)
--(axis cs:23,92.4095074335734)
--(axis cs:24,92.3824965953827)
--(axis cs:25,92.21333074073)
--(axis cs:26,92.4701154977083)
--(axis cs:27,92.4566110720237)
--(axis cs:28,92.4095074335734)
--(axis cs:29,92.3757423957189)
--(axis cs:30,92.6458656787872)
--(axis cs:31,92.5714125980934)
--(axis cs:32,92.517559727033)
--(axis cs:33,92.1051106850306)
--(axis cs:34,92.5445695718129)
--(axis cs:35,92.9291597008705)
--(axis cs:36,92.5847501556079)
--(axis cs:37,92.6188548405965)
--(axis cs:38,92.8214480479558)
--(axis cs:39,92.6254382232825)
--(axis cs:40,92.5850897779067)
--(axis cs:41,92.8282022476196)
--(axis cs:42,92.8957323233287)
--(axis cs:43,92.8145269801219)
--(axis cs:44,92.6931401093801)
--(axis cs:45,92.8957323233287)
--(axis cs:46,92.4836238722007)
--(axis cs:47,92.4566110471884)
--(axis cs:48,92.7876830101013)
--(axis cs:49,92.8412001828353)
--(axis cs:50,92.5175577402115)
--(axis cs:51,92.9767688115438)
--(axis cs:52,92.6318537940582)
--(axis cs:53,92.7404115597407)
--(axis cs:54,92.9497569799423)
--(axis cs:55,92.9561705638965)
--(axis cs:56,93.0713136990865)
--(axis cs:57,92.9632623990377)
--(axis cs:58,92.9630945622921)
--(axis cs:59,92.9294983545939)
--(axis cs:60,92.6792940994104)
--(axis cs:61,92.7469949424267)
--(axis cs:62,92.9565091927846)
--(axis cs:63,92.9225763181845)
--(axis cs:64,93.1455959876378)
--(axis cs:65,92.8957303365072)
--(axis cs:66,92.6661252975464)
--(axis cs:67,93.0645594994227)
--(axis cs:68,93.1455959876378)
--(axis cs:69,93.0848171313604)
--(axis cs:70,93.0172880242268)
--(axis cs:71,93.4359798828761)
--(axis cs:72,92.9565101613601)
--(axis cs:73,93.1185851494471)
--(axis cs:74,93.2941655317942)
--(axis cs:75,93.1388457367817)
--(axis cs:76,93.0578062931697)
--(axis cs:77,92.9497569799423)
--(axis cs:78,93.1455969810486)
--(axis cs:79,92.9495871812105)
--(axis cs:80,93.2198812564214)
--(axis cs:81,93.0172880490621)
--(axis cs:82,93.1591033935547)
--(axis cs:83,92.6998913288116)
--(axis cs:84,93.2669818898042)
--(axis cs:85,93.3144251505534)
--(axis cs:86,93.1793630123138)
--(axis cs:87,93.2063748439153)
--(axis cs:88,93.2131270567576)
--(axis cs:89,93.2468940814336)
--(axis cs:90,93.1455979744593)
--(axis cs:91,93.1455969810486)
--(axis cs:92,93.1388427813848)
--(axis cs:93,93.0846483260393)
--(axis cs:94,93.1861142317454)
--(axis cs:95,93.2806601126989)
--(axis cs:96,92.9092387358348)
--(axis cs:97,93.2198802630107)
--(axis cs:98,93.1861162185669)
--(axis cs:99,93.0172880490621)
--(axis cs:100,92.6862171043953)
--(axis cs:101,92.5648311773936)
--(axis cs:102,92.6053464412689)
--(axis cs:103,92.9765990376472)
--(axis cs:104,93.0778971066078)
--(axis cs:105,92.7066465218862)
--(axis cs:106,92.8687204917272)
--(axis cs:107,93.2467222958803)
--(axis cs:108,93.233385682106)
--(axis cs:109,92.9497569799423)
--(axis cs:110,93.266983901461)
--(axis cs:111,93.0105358362198)
--(axis cs:112,93.2266344626745)
--(axis cs:113,93.1658565998077)
--(axis cs:114,93.4832513332367)
--(axis cs:115,93.0510550240676)
--(axis cs:116,93.0983265240987)
--(axis cs:117,93.4292276700338)
--(axis cs:118,93.3547735959291)
--(axis cs:119,93.4494853019714)
--(axis cs:120,93.2131280501684)
--(axis cs:121,93.3144241571426)
--(axis cs:122,93.6318208773931)
--(axis cs:123,93.4629927078883)
--(axis cs:124,92.9835210243861)
--(axis cs:125,93.233385682106)
--(axis cs:126,93.4359808762868)
--(axis cs:127,93.4157212575277)
--(axis cs:128,93.5102641582489)
--(axis cs:129,93.1726078192393)
--(axis cs:130,93.4764971335729)
--(axis cs:131,93.4290588398774)
--(axis cs:132,93.2604004939397)
--(axis cs:133,93.5776264220476)
--(axis cs:134,93.334685737888)
--(axis cs:135,93.3211793502172)
--(axis cs:136,93.4292276700338)
--(axis cs:137,93.2604014873505)
--(axis cs:138,93.523770570755)
--(axis cs:139,93.5305237521728)
--(axis cs:140,93.3414379755656)
--(axis cs:141,93.4089680512746)
--(axis cs:142,93.3009187380473)
--(axis cs:143,93.4359808762868)
--(axis cs:144,93.233385682106)
--(axis cs:145,93.3076709508896)
--(axis cs:146,93.226632475853)
--(axis cs:147,93.4089680512746)
--(axis cs:148,93.4562385082245)
--(axis cs:149,93.4427350759506)
--(axis cs:149,93.6318218708038)
--(axis cs:149,93.6318218708038)
--(axis cs:148,93.7128573656082)
--(axis cs:147,93.7331170092027)
--(axis cs:146,93.604977875948)
--(axis cs:145,93.6588307221731)
--(axis cs:144,93.6184842884541)
--(axis cs:143,93.8141534725825)
--(axis cs:142,93.6655869086583)
--(axis cs:141,93.8074002663295)
--(axis cs:140,93.7668830404679)
--(axis cs:139,93.8820232202609)
--(axis cs:138,93.7398711840312)
--(axis cs:137,93.6385740836461)
--(axis cs:136,93.7263618161281)
--(axis cs:135,93.7331179777781)
--(axis cs:134,93.8749323536952)
--(axis cs:133,93.7803874909878)
--(axis cs:132,93.8276608784993)
--(axis cs:131,93.8276608784993)
--(axis cs:130,93.87493232886)
--(axis cs:129,93.550953194499)
--(axis cs:128,93.712858359019)
--(axis cs:127,93.8681801160177)
--(axis cs:126,93.8884387413661)
--(axis cs:125,94.0842777738969)
--(axis cs:124,93.7062729646762)
--(axis cs:123,93.8209066788356)
--(axis cs:122,93.8209066788356)
--(axis cs:121,93.8344150781631)
--(axis cs:120,93.5777972141902)
--(axis cs:119,93.7331159909566)
--(axis cs:118,93.7668840090434)
--(axis cs:117,93.7466234217087)
--(axis cs:116,93.5642908016841)
--(axis cs:115,93.7331159909566)
--(axis cs:114,93.9086963733037)
--(axis cs:113,93.3076719443003)
--(axis cs:112,93.6520795027415)
--(axis cs:111,93.7938958406448)
--(axis cs:110,93.6790923277537)
--(axis cs:109,93.4293965001901)
--(axis cs:108,93.6318188905716)
--(axis cs:107,93.942462404569)
--(axis cs:106,93.7128563970327)
--(axis cs:105,93.4765011072159)
--(axis cs:104,93.4359808762868)
--(axis cs:103,93.5781338314215)
--(axis cs:102,93.5237685839335)
--(axis cs:101,93.3279305696487)
--(axis cs:100,93.1861152251561)
--(axis cs:99,93.456240495046)
--(axis cs:98,93.5171872129043)
--(axis cs:97,93.7398691972097)
--(axis cs:96,93.395459651947)
--(axis cs:95,93.4832523266474)
--(axis cs:94,93.5035109519958)
--(axis cs:93,93.4157212575277)
--(axis cs:92,93.3819552262624)
--(axis cs:91,93.2671546936035)
--(axis cs:90,93.4629917144775)
--(axis cs:89,93.4765001138051)
--(axis cs:88,93.4832513332367)
--(axis cs:87,93.9154495795568)
--(axis cs:86,93.3819562196732)
--(axis cs:85,93.5237715641658)
--(axis cs:84,93.6655859400829)
--(axis cs:83,93.3752030134201)
--(axis cs:82,93.4292276700338)
--(axis cs:81,93.5305217901866)
--(axis cs:80,93.5644566516082)
--(axis cs:79,93.5511210312446)
--(axis cs:78,93.2874113321304)
--(axis cs:77,93.2605673372746)
--(axis cs:76,93.7331170092026)
--(axis cs:75,93.5710410277049)
--(axis cs:74,93.7601308027903)
--(axis cs:73,93.4361496567726)
--(axis cs:72,93.5913026332855)
--(axis cs:71,93.685844540596)
--(axis cs:70,93.5102631648382)
--(axis cs:69,93.4089660892884)
--(axis cs:68,93.780388434728)
--(axis cs:67,93.6115602900585)
--(axis cs:66,93.3481891949971)
--(axis cs:65,93.4967567771673)
--(axis cs:64,93.5170183579127)
--(axis cs:63,93.5170173645019)
--(axis cs:62,93.5372769832611)
--(axis cs:61,93.3213462183873)
--(axis cs:60,93.2809967547655)
--(axis cs:59,93.2198822498322)
--(axis cs:58,93.6048100392024)
--(axis cs:57,93.4363195051749)
--(axis cs:56,93.3414369821548)
--(axis cs:55,93.462993701299)
--(axis cs:54,93.6522483328978)
--(axis cs:53,93.6049768825372)
--(axis cs:52,93.2131280501684)
--(axis cs:51,93.5507833957672)
--(axis cs:50,93.1727796296279)
--(axis cs:49,93.361696600914)
--(axis cs:48,93.5982256382704)
--(axis cs:47,92.9565101861954)
--(axis cs:46,93.0510530869166)
--(axis cs:45,93.2874123255412)
--(axis cs:44,93.233386700352)
--(axis cs:43,93.4157212575277)
--(axis cs:42,93.3957983056704)
--(axis cs:41,93.1050797303518)
--(axis cs:40,92.9227431615194)
--(axis cs:39,93.1861162185669)
--(axis cs:38,93.1525209546089)
--(axis cs:37,93.0375476678212)
--(axis cs:36,93.2198802630107)
--(axis cs:35,93.3414360135794)
--(axis cs:34,92.9159899552663)
--(axis cs:33,92.7674233913422)
--(axis cs:32,92.8822259108226)
--(axis cs:31,92.9902742306391)
--(axis cs:30,92.9159909735123)
--(axis cs:29,92.7336593468984)
--(axis cs:28,92.9769386102756)
--(axis cs:27,92.90923876067)
--(axis cs:26,92.9362505674362)
--(axis cs:25,92.7539179722468)
--(axis cs:24,93.0915713310242)
--(axis cs:23,92.8148626536131)
--(axis cs:22,92.9634331911802)
--(axis cs:21,92.4569467206796)
--(axis cs:20,92.5175577402115)
--(axis cs:19,92.8621370842059)
--(axis cs:18,92.3960020393133)
--(axis cs:17,92.456778883934)
--(axis cs:16,92.355654562513)
--(axis cs:15,92.382665425539)
--(axis cs:14,92.4297680457433)
--(axis cs:13,92.0515944560369)
--(axis cs:12,91.9165323177973)
--(axis cs:11,91.9367889563243)
--(axis cs:10,91.6666666915019)
--(axis cs:9,91.3222581148148)
--(axis cs:8,91.0453796386719)
--(axis cs:7,90.5660787721475)
--(axis cs:6,89.573202530543)
--(axis cs:5,87.0073615262906)
--(axis cs:4,82.4353376527627)
--(axis cs:3,73.9267615973949)
--(axis cs:2,63.4461112568776)
--(axis cs:1,48.9532689253489)
--(axis cs:0,6.0916395423313)
--cycle;

\path [draw=forestgreen4416044, fill=forestgreen4416044, opacity=0.2, line width=0.32pt]
(axis cs:0,4.31523496905962)
--(axis cs:0,2.16774720077713)
--(axis cs:1,38.0601700146993)
--(axis cs:2,52.2487839063009)
--(axis cs:3,66.35602414608)
--(axis cs:4,75.6685574849447)
--(axis cs:5,81.9624533007542)
--(axis cs:6,85.2579683065414)
--(axis cs:7,86.6085906823476)
--(axis cs:8,87.6418153444926)
--(axis cs:9,88.1888171037038)
--(axis cs:10,88.891139626503)
--(axis cs:11,88.5330558319886)
--(axis cs:12,89.0194495519002)
--(axis cs:13,89.6542419989904)
--(axis cs:14,89.7420326868693)
--(axis cs:15,89.559360469381)
--(axis cs:16,90.0189071893692)
--(axis cs:17,90.1809831460317)
--(axis cs:18,90.3361360977093)
--(axis cs:19,90.3025398651759)
--(axis cs:20,90.8292800188065)
--(axis cs:21,90.7144794861476)
--(axis cs:22,90.9439147512118)
--(axis cs:23,90.9710973501205)
--(axis cs:24,90.6537006298701)
--(axis cs:25,90.7482445240021)
--(axis cs:26,90.9641733268897)
--(axis cs:27,91.1534309387207)
--(axis cs:28,90.7887617746989)
--(axis cs:29,91.1534309387207)
--(axis cs:30,91.2142078081767)
--(axis cs:31,90.8022691806157)
--(axis cs:32,91.0582117487987)
--(axis cs:33,91.1939511696498)
--(axis cs:34,91.2007013956706)
--(axis cs:35,91.0858998696009)
--(axis cs:36,91.0723934570948)
--(axis cs:37,91.3627763589223)
--(axis cs:38,91.4434781918923)
--(axis cs:39,91.4843340714772)
--(axis cs:40,91.2408829977115)
--(axis cs:41,91.585459386309)
--(axis cs:42,91.3830359776815)
--(axis cs:43,91.6936794916789)
--(axis cs:44,91.7409499486287)
--(axis cs:45,91.7542856186628)
--(axis cs:46,91.5316025416056)
--(axis cs:47,91.7679637422164)
--(axis cs:48,91.7206913232803)
--(axis cs:49,91.6666666666667)
--(axis cs:50,91.4640734593074)
--(axis cs:51,91.4843330780665)
--(axis cs:52,91.8422480424245)
--(axis cs:53,91.5044238666693)
--(axis cs:54,91.619395216306)
--(axis cs:55,91.6869252920151)
--(axis cs:56,91.5180991093318)
--(axis cs:57,91.7544583479563)
--(axis cs:58,91.6801740725835)
--(axis cs:59,91.9097791115443)
--(axis cs:60,92.0313328504562)
--(axis cs:61,91.9975697994232)
--(axis cs:62,91.9097781181335)
--(axis cs:63,91.8490002552668)
--(axis cs:64,91.9435421625773)
--(axis cs:65,91.8353240191936)
--(axis cs:66,91.8895175059636)
--(axis cs:67,91.8557554235061)
--(axis cs:68,91.8895204861959)
--(axis cs:69,92.0380880435308)
--(axis cs:70,91.8422480424245)
--(axis cs:71,92.0988659063975)
--(axis cs:72,91.7341967423757)
--(axis cs:73,92.294704914093)
--(axis cs:74,92.010904426376)
--(axis cs:75,92.098863919576)
--(axis cs:76,92.0110742251078)
--(axis cs:77,92.0581778635582)
--(axis cs:78,91.9906477878491)
--(axis cs:79,92.0178264379501)
--(axis cs:80,91.8895184993744)
--(axis cs:81,91.781469186147)
--(axis cs:82,92.0718520879745)
--(axis cs:83,92.0650998751322)
--(axis cs:84,92.1663979440928)
--(axis cs:85,92.2202518333991)
--(axis cs:86,92.0921136687199)
--(axis cs:87,92.0313348372777)
--(axis cs:88,92.0988659063975)
--(axis cs:89,92.28795170784)
--(axis cs:90,91.9973980138699)
--(axis cs:91,92.2811985015869)
--(axis cs:92,92.1056181192398)
--(axis cs:93,92.1731511751811)
--(axis cs:94,92.3217167456945)
--(axis cs:95,92.2744472821554)
--(axis cs:96,92.058010995388)
--(axis cs:97,92.2472666203976)
--(axis cs:98,92.3419773578644)
--(axis cs:99,92.2474344571431)
--(axis cs:100,91.3761159032583)
--(axis cs:101,91.322257121404)
--(axis cs:102,91.6666646798452)
--(axis cs:103,92.1258767197529)
--(axis cs:104,91.8219883988301)
--(axis cs:105,92.0110751936833)
--(axis cs:106,92.0784354954958)
--(axis cs:107,92.1934098005295)
--(axis cs:108,92.0853594938914)
--(axis cs:109,92.031333843867)
--(axis cs:110,92.2676940759023)
--(axis cs:111,92.5038825472196)
--(axis cs:112,92.2810287028551)
--(axis cs:113,91.9840623935064)
--(axis cs:114,92.0921127001444)
--(axis cs:115,92.2001620133718)
--(axis cs:116,92.2136694192886)
--(axis cs:117,92.3419763644536)
--(axis cs:118,92.1393851439158)
--(axis cs:119,92.2204196453094)
--(axis cs:120,92.3419753710429)
--(axis cs:121,92.3554847637812)
--(axis cs:122,92.6458656787872)
--(axis cs:123,92.3082133134206)
--(axis cs:124,92.5173899034659)
--(axis cs:125,92.4567798773448)
--(axis cs:126,92.3487295707067)
--(axis cs:127,92.5445695718129)
--(axis cs:128,92.4702843030294)
--(axis cs:129,92.5445725272099)
--(axis cs:130,92.463363284866)
--(axis cs:131,92.6188538471858)
--(axis cs:132,92.5850888093313)
--(axis cs:133,92.5918429593245)
--(axis cs:134,92.5985942284266)
--(axis cs:135,92.5781677663326)
--(axis cs:136,92.618855809172)
--(axis cs:137,92.6323612531026)
--(axis cs:138,92.5715823968251)
--(axis cs:139,92.49054590861)
--(axis cs:140,92.5445695718129)
--(axis cs:141,92.5513217846553)
--(axis cs:142,92.4566110720237)
--(axis cs:143,92.5108055273692)
--(axis cs:144,92.4363504598538)
--(axis cs:145,92.6389466226101)
--(axis cs:146,92.564660385251)
--(axis cs:147,92.6053464412689)
--(axis cs:148,92.6256060600281)
--(axis cs:149,92.6053484280904)
--(axis cs:149,92.9497549931208)
--(axis cs:149,92.9497549931208)
--(axis cs:148,92.8484588861465)
--(axis cs:147,92.8417086601257)
--(axis cs:146,92.8891479223967)
--(axis cs:145,92.882227897644)
--(axis cs:144,92.8486277163029)
--(axis cs:143,92.7066445350647)
--(axis cs:142,92.9971982290347)
--(axis cs:141,92.9024835427602)
--(axis cs:140,92.8417066733042)
--(axis cs:139,92.7944372097651)
--(axis cs:138,92.9227431615194)
--(axis cs:137,92.9632623990377)
--(axis cs:136,93.0309632668892)
--(axis cs:135,92.8687214851379)
--(axis cs:134,92.929667159915)
--(axis cs:133,92.9902742306391)
--(axis cs:132,92.7809298038483)
--(axis cs:131,93.0848191430171)
--(axis cs:130,93.0172890424728)
--(axis cs:129,92.8754727045695)
--(axis cs:128,93.0240402619044)
--(axis cs:127,93.0174568792184)
--(axis cs:126,92.7405803650618)
--(axis cs:125,92.9904440045357)
--(axis cs:124,92.8687185049057)
--(axis cs:123,92.6595438768466)
--(axis cs:122,92.8756405413151)
--(axis cs:121,92.6931381225586)
--(axis cs:120,92.9294983545939)
--(axis cs:119,92.8484588861465)
--(axis cs:118,92.4434422701597)
--(axis cs:117,92.6798015087843)
--(axis cs:116,92.5445715586344)
--(axis cs:115,92.7608389904102)
--(axis cs:114,92.8687214851379)
--(axis cs:113,92.9902752240499)
--(axis cs:112,92.5918420155843)
--(axis cs:111,92.8957313547532)
--(axis cs:110,92.7876830101013)
--(axis cs:109,92.5648282468319)
--(axis cs:108,92.6664649198453)
--(axis cs:107,92.6933079212904)
--(axis cs:106,92.6121016343435)
--(axis cs:105,92.4837907155355)
--(axis cs:104,92.4837907403708)
--(axis cs:103,92.5378153969844)
--(axis cs:102,92.7741756041845)
--(axis cs:101,91.9300367434819)
--(axis cs:100,92.0110762119293)
--(axis cs:99,92.6391154527664)
--(axis cs:98,92.7336593468984)
--(axis cs:97,92.5985962152481)
--(axis cs:96,92.6593720912933)
--(axis cs:95,92.8351242591937)
--(axis cs:94,92.6728794972102)
--(axis cs:93,92.6458676656087)
--(axis cs:92,92.5310641527176)
--(axis cs:91,92.4635340770086)
--(axis cs:90,92.4500276893377)
--(axis cs:89,92.6394521196683)
--(axis cs:88,92.7133977413177)
--(axis cs:87,92.5445715586344)
--(axis cs:86,92.6458656787872)
--(axis cs:85,92.652619878451)
--(axis cs:84,92.3892498016357)
--(axis cs:83,92.4905459334453)
--(axis cs:82,92.5783356030782)
--(axis cs:81,92.098863919576)
--(axis cs:80,92.5378183523814)
--(axis cs:79,92.4027562141418)
--(axis cs:78,92.5987640519937)
--(axis cs:77,92.4770394961039)
--(axis cs:76,92.4567798773448)
--(axis cs:75,92.7336563666662)
--(axis cs:74,92.7135645846526)
--(axis cs:73,92.5108045587937)
--(axis cs:72,92.4702852964401)
--(axis cs:71,92.585089802742)
--(axis cs:70,92.186655625701)
--(axis cs:69,92.5715833902359)
--(axis cs:68,92.2947059075038)
--(axis cs:67,92.3757443825404)
--(axis cs:66,92.3419753710429)
--(axis cs:65,92.4839595456918)
--(axis cs:64,92.2676921139161)
--(axis cs:63,92.5648301839828)
--(axis cs:62,92.3014601071676)
--(axis cs:61,92.2069162130356)
--(axis cs:60,92.3082113265991)
--(axis cs:59,92.3960010210673)
--(axis cs:58,92.3017967740695)
--(axis cs:57,92.2406812508901)
--(axis cs:56,92.1935756504536)
--(axis cs:55,92.0988649129868)
--(axis cs:54,92.0112430552642)
--(axis cs:53,92.03808705012)
--(axis cs:52,92.1596417824427)
--(axis cs:51,92.1663949886958)
--(axis cs:50,92.166397968928)
--(axis cs:49,92.098863919576)
--(axis cs:48,92.0786062876384)
--(axis cs:47,92.2408500562111)
--(axis cs:46,91.8827652931213)
--(axis cs:45,92.166397968928)
--(axis cs:44,92.4501945326726)
--(axis cs:43,92.1191245317459)
--(axis cs:42,92.1258767445882)
--(axis cs:41,91.8964415043592)
--(axis cs:40,91.8287396430969)
--(axis cs:39,91.9435451428096)
--(axis cs:38,91.9030259052912)
--(axis cs:37,91.9232855240504)
--(axis cs:36,91.8761809170246)
--(axis cs:35,91.6261484225591)
--(axis cs:34,91.5451109409332)
--(axis cs:33,91.7477031548818)
--(axis cs:32,91.6531612475713)
--(axis cs:31,91.4169718325138)
--(axis cs:30,91.869259874026)
--(axis cs:29,91.6870950659116)
--(axis cs:28,91.8017278363307)
--(axis cs:27,91.6801730791728)
--(axis cs:26,91.6261464357376)
--(axis cs:25,91.578874985377)
--(axis cs:24,91.2142078081767)
--(axis cs:23,91.3629471510649)
--(axis cs:22,91.6396548350652)
--(axis cs:21,91.3222591330608)
--(axis cs:20,91.2751544763644)
--(axis cs:19,90.9238259245952)
--(axis cs:18,90.7820105552673)
--(axis cs:17,90.8225288242102)
--(axis cs:16,90.7617509365082)
--(axis cs:15,90.741491317749)
--(axis cs:14,90.7820105552673)
--(axis cs:13,90.4781192541122)
--(axis cs:12,90.0530118991931)
--(axis cs:11,89.4516477982203)
--(axis cs:10,89.2693142096202)
--(axis cs:9,88.5467301060756)
--(axis cs:8,88.6142621686061)
--(axis cs:7,87.3583554724852)
--(axis cs:6,86.1088593800863)
--(axis cs:5,82.7797481914361)
--(axis cs:4,76.6815215349197)
--(axis cs:3,67.8957313299179)
--(axis cs:2,54.0923833847046)
--(axis cs:1,40.0999456644058)
--(axis cs:0,4.31523496905962)
--cycle;

\path [draw=crimson2143940, fill=crimson2143940, opacity=0.2, line width=0.32pt]
(axis cs:0,11.2034036467473)
--(axis cs:0,8.75202603638172)
--(axis cs:1,45.0766479969025)
--(axis cs:2,62.9708603769541)
--(axis cs:3,70.8799294382334)
--(axis cs:4,74.0410586198171)
--(axis cs:5,76.438243240118)
--(axis cs:6,78.2347380866607)
--(axis cs:7,80.7717102020979)
--(axis cs:8,82.9889247814814)
--(axis cs:9,81.3614249229431)
--(axis cs:10,83.3400865395864)
--(axis cs:11,83.7923745810986)
--(axis cs:12,83.7857911984126)
--(axis cs:13,85.2579683065414)
--(axis cs:14,86.1966500679652)
--(axis cs:15,85.0070897241433)
--(axis cs:16,85.8860055605571)
--(axis cs:17,86.7368976275126)
--(axis cs:18,86.2709333499273)
--(axis cs:19,87.3512616505226)
--(axis cs:20,86.8652075529099)
--(axis cs:21,87.7765398472548)
--(axis cs:22,87.7566168705622)
--(axis cs:23,88.2766077915827)
--(axis cs:24,87.6958400011063)
--(axis cs:25,88.3573076377312)
--(axis cs:26,88.2631013790766)
--(axis cs:27,87.8309031327565)
--(axis cs:28,88.7692442784707)
--(axis cs:29,89.1342520713806)
--(axis cs:30,89.5596961180369)
--(axis cs:31,88.4454349676768)
--(axis cs:32,89.6137227614721)
--(axis cs:33,89.0800605465968)
--(axis cs:34,89.9444550772508)
--(axis cs:35,89.4854128360748)
--(axis cs:36,89.9444550772508)
--(axis cs:37,89.5191798607508)
--(axis cs:38,89.6745006243388)
--(axis cs:39,90.0324145952861)
--(axis cs:40,89.8095627625783)
--(axis cs:41,90.160725514094)
--(axis cs:42,90.9170707066854)
--(axis cs:43,90.3025378783544)
--(axis cs:44,90.4777816186349)
--(axis cs:45,90.2890334278345)
--(axis cs:46,90.3768241405487)
--(axis cs:47,90.5051320791245)
--(axis cs:48,90.2822802464167)
--(axis cs:49,89.9918963263432)
--(axis cs:50,90.6064281860987)
--(axis cs:51,90.7009720802307)
--(axis cs:52,90.7009720802307)
--(axis cs:53,91.1802729715904)
--(axis cs:54,90.545650323232)
--(axis cs:55,90.842788418134)
--(axis cs:56,90.801593909661)
--(axis cs:57,90.896812081337)
--(axis cs:58,91.4909164607525)
--(axis cs:59,91.1871979633967)
--(axis cs:60,90.8630470434825)
--(axis cs:61,91.1264171202977)
--(axis cs:62,91.1264191071192)
--(axis cs:63,91.3627753655116)
--(axis cs:64,91.2410518030326)
--(axis cs:65,90.8090213934581)
--(axis cs:66,90.6537016232808)
--(axis cs:67,90.7009720802307)
--(axis cs:68,91.2479748328527)
--(axis cs:69,91.6193942228953)
--(axis cs:70,91.0924833019574)
--(axis cs:71,90.8966432511806)
--(axis cs:72,91.3019994894663)
--(axis cs:73,91.558449541529)
--(axis cs:74,91.3627763589223)
--(axis cs:75,91.6126430034637)
--(axis cs:76,91.7544563611348)
--(axis cs:77,91.3019984960556)
--(axis cs:78,91.5248523155848)
--(axis cs:79,91.545111934344)
--(axis cs:80,91.2074565887451)
--(axis cs:81,91.3019994894664)
--(axis cs:82,91.3290113210678)
--(axis cs:83,90.9569136550029)
--(axis cs:84,91.5246805548668)
--(axis cs:85,91.6396538416545)
--(axis cs:86,91.2408830225468)
--(axis cs:87,91.3762817780177)
--(axis cs:88,91.2273775786161)
--(axis cs:89,91.5314337362846)
--(axis cs:90,91.4235552151998)
--(axis cs:91,91.5383577346802)
--(axis cs:92,91.7544563611348)
--(axis cs:93,91.5586173534393)
--(axis cs:94,91.8219894170761)
--(axis cs:95,91.5113439162572)
--(axis cs:96,91.5451089541117)
--(axis cs:97,91.5920437375704)
--(axis cs:98,91.450568040212)
--(axis cs:99,91.4168020089467)
--(axis cs:100,91.6126400232315)
--(axis cs:101,91.5717851370573)
--(axis cs:102,91.707185904185)
--(axis cs:103,91.8555846313636)
--(axis cs:104,91.5381888796886)
--(axis cs:105,91.3560231526693)
--(axis cs:106,91.7544553677241)
--(axis cs:107,91.8557554235061)
--(axis cs:108,91.8017268180847)
--(axis cs:109,91.7544583479563)
--(axis cs:110,91.9097781181335)
--(axis cs:111,91.9097771247228)
--(axis cs:112,91.6869252920151)
--(axis cs:113,91.8962717056274)
--(axis cs:114,92.0853604873021)
--(axis cs:115,91.8354938427607)
--(axis cs:116,91.7341977357864)
--(axis cs:117,92.159644762675)
--(axis cs:118,91.9028560817242)
--(axis cs:119,91.8488304316998)
--(axis cs:120,91.6734188546737)
--(axis cs:121,91.7004336913427)
--(axis cs:122,91.8692568937937)
--(axis cs:123,91.9435431559881)
--(axis cs:124,92.2541856765747)
--(axis cs:125,91.943374350667)
--(axis cs:126,92.0043220122655)
--(axis cs:127,92.004322980841)
--(axis cs:128,91.7949736118317)
--(axis cs:129,91.9973980138699)
--(axis cs:130,92.1999932080507)
--(axis cs:131,92.0244108388821)
--(axis cs:132,91.9638017813365)
--(axis cs:133,92.0244118322929)
--(axis cs:134,91.8961038937171)
--(axis cs:135,91.9840633869171)
--(axis cs:136,91.9705559810003)
--(axis cs:137,91.9838925699393)
--(axis cs:138,92.0515954246124)
--(axis cs:139,92.1663959821065)
--(axis cs:140,92.2676920890808)
--(axis cs:141,92.0988659063975)
--(axis cs:142,92.2405114521583)
--(axis cs:143,92.0581788569689)
--(axis cs:144,92.1326319376628)
--(axis cs:145,92.1799033880234)
--(axis cs:146,92.0786062876384)
--(axis cs:147,92.0446714262168)
--(axis cs:148,92.1663969755173)
--(axis cs:149,92.1663969755173)
--(axis cs:149,92.5582467764616)
--(axis cs:149,92.5582467764616)
--(axis cs:148,92.5850888341665)
--(axis cs:147,92.3824936399857)
--(axis cs:146,92.3286417375008)
--(axis cs:145,92.5175587336222)
--(axis cs:144,92.6053484280904)
--(axis cs:143,92.5310651461283)
--(axis cs:142,92.6931381225586)
--(axis cs:141,92.5715823968251)
--(axis cs:140,92.6256080468496)
--(axis cs:139,92.4905449151993)
--(axis cs:138,92.4500266710917)
--(axis cs:137,92.4231826514006)
--(axis cs:136,92.2879517326752)
--(axis cs:135,92.321718732516)
--(axis cs:134,92.4230138460795)
--(axis cs:133,92.3419773578644)
--(axis cs:132,92.4432724714279)
--(axis cs:131,92.49054590861)
--(axis cs:130,92.5040523211161)
--(axis cs:129,92.5175567468007)
--(axis cs:128,92.0652677118778)
--(axis cs:127,92.4432744582494)
--(axis cs:126,92.4770394961039)
--(axis cs:125,92.3689891894658)
--(axis cs:124,92.5108045339584)
--(axis cs:123,92.3622379700343)
--(axis cs:122,92.5175567468007)
--(axis cs:121,92.3824946333965)
--(axis cs:120,92.2406792889039)
--(axis cs:119,92.396002014478)
--(axis cs:118,92.4501974135637)
--(axis cs:117,92.5850888093313)
--(axis cs:116,92.3222242295742)
--(axis cs:115,92.652619878451)
--(axis cs:114,92.2204206387202)
--(axis cs:113,92.3824946085612)
--(axis cs:112,92.0448392877976)
--(axis cs:111,92.200161019961)
--(axis cs:110,92.2069142510494)
--(axis cs:109,92.1326319376628)
--(axis cs:108,92.0583446820577)
--(axis cs:107,92.3151333381732)
--(axis cs:106,92.3960010210673)
--(axis cs:105,92.1528915564219)
--(axis cs:104,92.1463051935037)
--(axis cs:103,92.4230128526688)
--(axis cs:102,92.180071224769)
--(axis cs:101,92.2204216321309)
--(axis cs:100,92.1191255251567)
--(axis cs:99,91.8827672799428)
--(axis cs:98,92.0313348372777)
--(axis cs:97,92.1326299508413)
--(axis cs:96,92.03808705012)
--(axis cs:95,92.0650998999675)
--(axis cs:94,92.2408490628004)
--(axis cs:93,92.2069162130356)
--(axis cs:92,92.1731491883596)
--(axis cs:91,92.0043230056763)
--(axis cs:90,92.2271738449733)
--(axis cs:89,91.8017268180847)
--(axis cs:88,91.8356646597385)
--(axis cs:87,92.4837907155355)
--(axis cs:86,92.0515944560369)
--(axis cs:85,92.4230128526688)
--(axis cs:84,92.1799014012019)
--(axis cs:83,92.098863919576)
--(axis cs:82,91.9300377368927)
--(axis cs:81,91.774715979894)
--(axis cs:80,91.8895184993744)
--(axis cs:79,92.0650988817215)
--(axis cs:78,92.1596437692642)
--(axis cs:77,91.5788759787877)
--(axis cs:76,92.2541856765747)
--(axis cs:75,92.1663969755173)
--(axis cs:74,91.8559222668409)
--(axis cs:73,91.9435451676448)
--(axis cs:72,91.8557544549306)
--(axis cs:71,92.132630944252)
--(axis cs:70,91.6464070727428)
--(axis cs:69,92.0315036425988)
--(axis cs:68,91.8625056743622)
--(axis cs:67,91.5383587280909)
--(axis cs:66,91.6261474291484)
--(axis cs:65,91.6193942228953)
--(axis cs:64,91.9502973556519)
--(axis cs:63,91.8424168229103)
--(axis cs:62,91.7071849107742)
--(axis cs:61,91.6666666666667)
--(axis cs:60,91.4912560582161)
--(axis cs:59,91.7949755986532)
--(axis cs:58,91.8017278363307)
--(axis cs:57,91.5383557478587)
--(axis cs:56,91.5518631537755)
--(axis cs:55,91.5385265648365)
--(axis cs:54,91.4775788784027)
--(axis cs:53,91.545111934344)
--(axis cs:52,91.3492699464162)
--(axis cs:51,91.2682324647903)
--(axis cs:50,91.2481426944336)
--(axis cs:49,91.4032946030299)
--(axis cs:48,90.8495406309764)
--(axis cs:47,91.1804447571437)
--(axis cs:46,90.8294498175383)
--(axis cs:45,90.9508387247721)
--(axis cs:44,91.234806055824)
--(axis cs:43,91.1129126946131)
--(axis cs:42,91.5451109409332)
--(axis cs:41,91.1466797192891)
--(axis cs:40,90.8157745997111)
--(axis cs:39,90.5726651350657)
--(axis cs:38,90.2417620023092)
--(axis cs:37,90.626857628425)
--(axis cs:36,90.2687738339106)
--(axis cs:35,90.2620216210683)
--(axis cs:34,90.8157746245464)
--(axis cs:33,89.6069685618083)
--(axis cs:32,90.1742299646139)
--(axis cs:31,89.8433287938436)
--(axis cs:30,90.0866090506315)
--(axis cs:29,89.924534112215)
--(axis cs:28,89.2966626832882)
--(axis cs:27,89.6812538305918)
--(axis cs:26,89.4249716897805)
--(axis cs:25,89.5934621493022)
--(axis cs:24,88.8708810011546)
--(axis cs:23,88.9384130636851)
--(axis cs:22,88.9249046643575)
--(axis cs:21,88.85737657547)
--(axis cs:20,88.019988934199)
--(axis cs:19,88.3576442797979)
--(axis cs:18,87.4664037923018)
--(axis cs:17,87.5474402308464)
--(axis cs:16,87.5406860808531)
--(axis cs:15,86.8591296424468)
--(axis cs:14,87.0070219039917)
--(axis cs:13,86.507632235686)
--(axis cs:12,86.1021081606547)
--(axis cs:11,84.7717444101969)
--(axis cs:10,85.1971924304962)
--(axis cs:9,84.9145727107922)
--(axis cs:8,85.2579673131307)
--(axis cs:7,83.7182601292928)
--(axis cs:6,81.3074012597402)
--(axis cs:5,79.8961728314559)
--(axis cs:4,77.3163139820099)
--(axis cs:3,74.4394938151042)
--(axis cs:2,67.9718737055858)
--(axis cs:1,49.5340352257093)
--(axis cs:0,11.2034036467473)
--cycle;

\addplot [line width=0.48pt, steelblue31119180]
table {%
0 4.50432191913326
1 48.3725016315778
2 62.9592110713323
3 73.2779572407405
4 81.8408956130346
5 86.7909242709478
6 89.053213596344
7 90.4038359721502
8 90.7685031493505
9 91.1264181137085
10 91.3627773523331
11 91.6261474291484
12 91.6531602541606
13 91.6329016288121
14 92.1056191126506
15 92.2474344571431
16 92.1663969755173
17 92.1596427758535
18 92.0178284247716
19 92.5580769777298
20 92.4095084269841
21 92.2069142262141
22 92.7133987347285
23 92.5985952218374
24 92.7201509475708
25 92.49054590861
26 92.6863849163055
27 92.6796327034632
28 92.6998913288116
29 92.5580769777298
30 92.7809288104375
31 92.7741746107737
32 92.6661262909571
33 92.4635330835978
34 92.7606691916784
35 93.1388437747955
36 92.8889781236649
37 92.8349544604619
38 92.9902762174606
39 92.9227441549301
40 92.7741756041845
41 92.9632643858592
42 93.1253373622894
43 93.1320905685425
44 92.9632633924484
45 93.0848191181819
46 92.747163772583
47 92.7133977413177
48 93.1928694248199
49 93.0915723244349
50 92.8619672854741
51 93.2806591192881
52 92.909237742424
53 93.1320915619532
54 93.3481891949971
55 93.2131280501684
56 93.2131280501684
57 93.2131280501684
58 93.294166525205
59 93.0848201115926
60 92.9835220177968
61 93.0240412553151
62 93.2536462942759
63 93.1996216376622
64 93.3481901884079
65 93.1726078192393
66 93.0172870556514
67 93.3346837759018
68 93.4089670578639
69 93.2401388883591
70 93.260399500529
71 93.5642898082733
72 93.267152706782
73 93.2874123255412
74 93.5237715641658
75 93.3684488137563
76 93.3684498071671
77 93.0915713310242
78 93.2063748439153
79 93.2536462942759
80 93.3887084325155
81 93.2739049196243
82 93.2874123255412
83 93.0915723244349
84 93.4764981269836
85 93.4157212575277
86 93.2739069064458
87 93.5710420211156
88 93.3481891949971
89 93.3549433946609
90 93.3211773633957
91 93.206375837326
92 93.260399500529
93 93.260399500529
94 93.3684488137563
95 93.3819562196732
96 93.1185841560364
97 93.4900045394897
98 93.3346847693125
99 93.2468940814336
100 92.9497569799423
101 92.9024855295817
102 93.0645575126012
103 93.2739059130351
104 93.2941655317942
105 93.0983265240987
106 93.267152706782
107 93.591300646464
108 93.4359788894653
109 93.2131280501684
110 93.4764991203944
111 93.4089690446854
112 93.4224744637807
113 93.2401408751806
114 93.685844540596
115 93.3684498071671
116 93.3414379755656
117 93.5980558395386
118 93.5777962207794
119 93.584547440211
120 93.3954626321793
121 93.5575366020203
122 93.7263637781143
123 93.6655859152476
124 93.3414359887441
125 93.6385730902354
126 93.6723391215007
127 93.6318208773931
128 93.618314464887
129 93.3616956075033
130 93.6723381280899
131 93.6385740836461
132 93.5710430145264
133 93.6790923277537
134 93.6115612586339
135 93.5170193513234
136 93.5845484336217
137 93.4359808762868
138 93.6318208773931
139 93.7061041593552
140 93.5440311829249
141 93.6318198839823
142 93.4900055329005
143 93.6250666777293
144 93.4562395016352
145 93.490003546079
146 93.4427330891291
147 93.5710430145264
148 93.5777952273687
149 93.5440311829249
};
\addlegendentry{\textbf{\name{}} (Ours)}
\addplot [line width=0.48pt, forestgreen4416044]
table {%
0 3.23473801836371
1 39.0329549709956
2 53.241491317749
3 67.0583466688792
4 76.1615335941315
5 82.2730958461761
6 85.656401515007
7 86.9800120592117
8 88.114532828331
9 88.371150692304
10 89.1072392463684
11 88.9721771081289
12 89.5394384860992
13 90.0729338328044
14 90.2620216210683
15 90.1539713144302
16 90.3970827658971
17 90.4916266600291
18 90.572664141655
19 90.6199355920156
20 91.058886051178
21 91.0183697938919
22 91.2682334582011
23 91.1736905574799
24 90.9440845251083
25 91.2209610144297
26 91.3290103276571
27 91.4100488026937
28 91.3087517023087
29 91.4573202530543
30 91.4910852909088
31 91.1061594883601
32 91.3830369710922
33 91.4708276589711
34 91.3897891839345
35 91.349270939827
36 91.4438138405482
37 91.6734198729197
38 91.6801740725835
39 91.7341987291972
40 91.5788759787877
41 91.7477031548818
42 91.7544563611348
43 91.9030249118805
44 92.0786052942276
45 91.9502973556519
46 91.7071839173635
47 92.0313348372777
48 91.9300377368927
49 91.8827652931213
50 91.8219884236654
51 91.8287406365077
52 92.0043220122655
53 91.7882223924001
54 91.7949755986532
55 91.8895184993744
56 91.8490002552668
57 91.9908165931702
58 92.0178284247716
59 92.1799023946126
60 92.1799023946126
61 92.0786072810491
62 92.1056191126506
63 92.2339270512263
64 92.1056171258291
65 92.1191235383352
66 92.1056171258291
67 92.1123723189036
68 92.0988659063975
69 92.2879527012507
70 92.0245816310247
71 92.3419773578644
72 92.1123703320821
73 92.4027552207311
74 92.355481783549
75 92.3824946085612
76 92.2339270512263
77 92.2609408696492
78 92.3014591137568
79 92.2204206387202
80 92.2136684258779
81 91.9435431559881
82 92.3352241516113
83 92.2744462887446
84 92.2879527012507
85 92.4297680457433
86 92.3622359832128
87 92.2744472821554
88 92.4567798773448
89 92.4770394961039
90 92.2271738449733
91 92.3757423957189
92 92.3014581203461
93 92.4162626266479
94 92.5108045339584
95 92.5310651461283
96 92.3757433891296
97 92.4297690391541
98 92.5378183523814
99 92.4432744582494
100 91.6734198729197
101 91.6126410166423
102 92.1393831570943
103 92.3419763644536
104 92.1663959821065
105 92.2474334637324
106 92.3689891894658
107 92.4500276645025
108 92.4027552207311
109 92.3149645328522
110 92.5378183523814
111 92.6998923222224
112 92.4365202585856
113 92.5310641527176
114 92.4635340770086
115 92.4500266710917
116 92.3824965953827
117 92.5108045339584
118 92.2947059075038
119 92.5783346096675
120 92.6188538471858
121 92.5445705652237
122 92.7606691916784
123 92.4973001082738
124 92.6863859097163
125 92.7404095729192
126 92.5445705652237
127 92.7809288104375
128 92.7606691916784
129 92.7133987347285
130 92.7269041538239
131 92.8619662920634
132 92.6931391159693
133 92.7876820166906
134 92.7539169788361
135 92.7471647659938
136 92.8282012542089
137 92.7876820166906
138 92.7404095729192
139 92.6526208718618
140 92.6931381225586
141 92.72014995416
142 92.747163772583
143 92.6121016343435
144 92.6188538471858
145 92.780930797259
146 92.74041056633
147 92.7201509475708
148 92.7269031604131
149 92.7876820166906
};
\addlegendentry{\datavec{}}
\addplot [line width=0.48pt, crimson2143940]
table {%
0 9.90005408724149
1 47.4338193734487
2 65.5118852853775
3 72.5351164738337
4 75.6280382474263
5 78.2347391049067
6 79.7811994949977
7 82.4081579844157
8 84.1166933377584
9 83.1374923388163
10 84.2382500569026
11 84.2787673075994
12 84.8933011293411
13 85.9062671661377
14 86.6018364826838
15 86.0413283109665
16 86.7233912150065
17 87.1826032797496
18 86.8719607591629
19 87.8646671772003
20 87.4324689308802
21 88.2833609978358
22 88.3238782485326
23 88.6142631371816
24 88.3171260356903
25 88.9384120702744
26 88.8573745886485
27 88.8438691695531
28 89.0262007713318
29 89.5732045173645
30 89.8298213879267
31 89.1680171092351
32 89.863587419192
33 89.3368452787399
34 90.3565635283788
35 89.8770938316981
36 90.120205283165
37 90.059428413709
38 89.958131313324
39 90.3295526901881
40 90.3092920780182
41 90.6266897916794
42 91.2277152140935
43 90.7752563556035
44 90.9170717000961
45 90.6537016232808
46 90.6334410111109
47 90.842788418134
48 90.572664141655
49 90.7549967368444
50 90.9170717000961
51 90.9710963567098
52 91.0251210133235
53 91.35602414608
54 91.0116146008174
55 91.1736905574799
56 91.1804437637329
57 91.2277142206828
58 91.6464070479075
59 91.518098115921
60 91.1871959765752
61 91.3762827714284
62 91.4370616277059
63 91.6531592607498
64 91.6464070479075
65 91.2682324647903
66 91.2209620078405
67 91.1129126946131
68 91.5518631537755
69 91.8354948361715
70 91.3627773523331
71 91.5451109409332
72 91.5721237659454
73 91.7612105607986
74 91.6126410166423
75 91.9030259052912
76 92.0178274313609
77 91.4438128471375
78 91.8287416299184
79 91.8017288049062
80 91.5788759787877
81 91.5383577346802
82 91.6329016288121
83 91.5991346041362
84 91.8084810177485
85 92.0786052942276
86 91.6531612475713
87 91.9773081938426
88 91.5518641471863
89 91.673418879509
90 91.8625066677729
91 91.8152352174123
92 91.9705559810003
93 91.8827662865321
94 92.0313348372777
95 91.774715979894
96 91.7882213989894
97 91.8692588806152
98 91.7274455229441
99 91.6464080413183
100 91.8962717056274
101 91.9232845306396
102 91.9502973556519
103 92.1596427758535
104 91.8827652931213
105 91.7477041482925
106 92.0718520879745
107 92.0786062876384
108 91.9232835372289
109 91.9232855240504
110 92.0583466688792
111 92.0583456754684
112 91.8760120868683
113 92.1326299508413
114 92.1528905630112
115 92.2339270512263
116 92.0043220122655
117 92.3689901828766
118 92.2136674324671
119 92.1056181192398
120 91.9705559810003
121 92.0853594938914
122 92.1866546074549
123 92.1596437692642
124 92.3689891894658
125 92.1191245317459
126 92.227174838384
127 92.1799033880234
128 91.9165303309758
129 92.2609388828278
130 92.3622369766235
131 92.2406802574794
132 92.2069142262141
133 92.2001620133718
134 92.1528905630112
135 92.1596437692642
136 92.132630944252
137 92.200161019961
138 92.2406812508901
139 92.3217177391052
140 92.4230138460795
141 92.3352241516113
142 92.4635330835978
143 92.3419783512751
144 92.3689901828766
145 92.3554837703705
146 92.2001620133718
147 92.200161019961
148 92.4027552207311
149 92.3487305641174
};
\addlegendentry{from scratch}
\end{axis}

\end{tikzpicture}

\vspace{-0.5cm}
\caption{
\textbf{Learning Curves for ModelNet40\,\cite{wu2015modelnet40}.}
We show the mean (solid line) and the standard deviation (shaded background) over $6$ independent runs of \name{}, \datavec{} as well as the model trained \emph{from scratch} on ModelNet40.
\emakefirstuc{\name{}} consistently outperforms the baselines by a large margin.}
\label{fig:oa_learning_curve_modelnet40}
\end{figure}
After pretraining on ShapeNet, we finetune our model for shape classification on ModelNet40\,\cite{wu2015modelnet40} consisting of \num{12311} \emph{synthetic} 3D models of $40$ semantic categories.
We obtain the semantic class label by passing the concatenated mean- and max-pooled output of the Transformer encoder into a $3$-layer MLP and finetune the whole network end-to-end.
We use minimal data augmentations consisting only of resampling $1024$ points with farthest point sampling, applying random anisotropic scaling of up to $40\%$, centering at the origin, and rescaling to the unit sphere.
Other commonly used augmentations did not improve performance, \eg random rotations around the axis of gravity and random translations are detrimental as ModelNet40 instances are canonically oriented.
During the point cloud embedding step we sample $n$$=$$64$ center points and $k$$=$$32$ nearest neighbors.
In~\reftab{modelnet_results}, we report a new state-of-the-art for shape classification on ModelNet40\,\cite{wu2015modelnet40} among self-supervised methods by a large margin of $+1.3\%$ without voting\,\cite{zhang2022pointm2ae, pang2022pointmae, yu2021pointbert}.
Interestingly, pre-training with \datavec{} results only in marginal improvements ($+0.3\%$ without voting) over the same model trained \emph{from scratch} on ModelNet40.
Unlike \datavec{}, we observe that \name{} unleashes the full potential of data2vec-like pre-training on ModelNet40 by achieving substantial performance gains of $+1.7\%$ over the baseline trained from scratch.
In~\reffig{oa_learning_curve_modelnet40}, 
we plot the accuracy per training epoch of \name{}, \datavec{}, as well as our baseline trained \emph{from scratch} on ModelNet40.
We observe that \name{} outperforms our strong baselines by a consistent margin throughout the entire training.
We conclude that \name{} effectively learns strong feature representations on ShapeNet, resulting in a significantly accelerated adaptation to the fine-tuning task (\reffig{oa_learning_curve_modelnet40}) as well as strong performance gains of $+2.2\%$\,mAcc over the baseline only trained on ModelNet40 (\reftab{modelnet_results}).

\paragraph{Real-World Shape Classification.}
\begin{table}
    \centering
    \setlength{\tabcolsep}{1.8pt}
    \caption{
        \textbf{Shape Classification on ScanObjectNN\,\cite{uy2019scanobjectnn}.}
        We report the overall accuracy over the three subsets \texttt{OBJ-BG}, \texttt{OBJ-ONLY} and the most challenging variant \texttt{PB-T50-RS}.
    }
    \label{tab:scanobjectnn_results}
    \begin{tabularx}{\linewidth}{lcp{0.001cm}Yp{0.001cm}Yp{0.001cm}}
        \toprule
         & \multicolumn{5}{c}{Overall Accuracy} \\
         \cmidrule(lr){2-7}
        Method & \footnotesize \texttt{OBJ-BG} && \footnotesize \texttt{OBJ-ONLY} && \footnotesize \texttt{PB-T50-RS} \\
        \midrule
        Transf.-OcCo\,\cite{yu2021pointbert} & 84.9 && 85.5 && 78.8 \\
        Point-BERT\,\cite{yu2021pointbert}  & 87.4 && 88.1 &&  83.1                       \\
        MaskPoint\,\cite{liu2022maskpoint} & 89.3 && 89.7 && 84.6 \\
        Point-MAE\,\cite{pang2022pointmae}                        & 90.0 && 88.3 &&  85.2                  \\
        Point-M2AE\,\cite{zhang2022pointm2ae} & \textbf{91.2} && 88.8 &&  86.4  \\
        \arrayrulecolor{black!10}\midrule\arrayrulecolor{black}
        from scratch                      & 88.1 & \multirow{2}{*}{\hspace{-0.0cm}\ArrowDown{\footnotesize \textcolor{darkgreen}{$+1.6$}}} & 88.8 & \multirow{2}{*}{\hspace{-0.3cm}\ArrowDown{\footnotesize \textcolor{red}{$-0.7$}}} &  84.3 & \multirow{2}{*}{\hspace{-0.3cm}\ArrowDown{\footnotesize \textcolor{darkgreen}{$+1.2$}}}                     \\
        \datavec{} & 89.7 & \multirow{2}{*}{\hspace{-0.0cm}\ArrowDown{\footnotesize \textcolor{darkgreen}{$+1.5$}}} & 88.1 & \multirow{2}{*}{\hspace{-0.3cm}\ArrowDown{\footnotesize \textcolor{darkgreen}{$+2.3$}}}&  85.5 & \multirow{2}{*}{\hspace{-0.3cm}\ArrowDown{\footnotesize \textcolor{darkgreen}{$+2.0$}}}                     \\
        \textbf{\name{}} (Ours) & \textbf{91.2} && \textbf{90.4} && \textbf{87.5}\\
        \bottomrule
    \end{tabularx}
\end{table}
Next, we fine-tune \name{} on ScanObjectNN\,\cite{uy2019scanobjectnn} containing \num{2902} \emph{real-world} object scans of $15$ semantic classes.
In contrast to shape classification on ModelNet40, we do not resample points but use all \num{2048} points and sample $n$$=$$128$ center points for the point cloud embedding step. %
We found more aggressive scaling to be detrimental and use random anisotropic scaling of up to $10\%$. %
Although pre-trained on synthetic data, \reftab{scanobjectnn_results} shows that \name{} generalizes well to cluttered real-world data and achieves state-of-the-art performance among self-supervised methods by a significant margin of $+1.1\%$ on \texttt{PB-T50-RS}, the most difficult variant of the dataset.
Furthermore, we observe that pre-training \name{} on ShapeNet plays a crucial role to its strong performance.
Compared to the baseline trained from scratch on ScanObjectNN, pre-training with \name{} achieves an impressive performance gain of $+3.2\%$.
We again report significant improvements of \name{} over \datavec{} of up to $+2.3\%$.


\paragraph{Few-Shot Classification.}
\begin{table}[t]
    \caption{
        \textbf{Few-Shot Classification on ModelNet40\,\cite{wu2015modelnet40}.}
        We report mean and standard deviation over $10$ runs.
    }
    \label{tab:modelnet_fewshot_results}
    \centering
    \setlength{\tabcolsep}{3pt}
    \resizebox{\linewidth}{!}{
    \begin{tabular}{lcccccc}
        \toprule
         && \multicolumn{2}{c}{5-way}  && \multicolumn{2}{c}{10-way}\\
        \cmidrule(lr){3-4} \cmidrule(lr){6-7}
        Method && 10-shot & 20-shot && 10-shot & 20-shot \\
        \midrule
        OcCo\,\cite{wang2021occo} && $91.9 $\footnotesize $\pm 3.6$ & $93.9 $\footnotesize $\pm 3.1$ && $86.4 $\footnotesize $\pm 5.4$ & $91.3 $\footnotesize $\pm 4.6$ \\
        Transf.-OcCo\,\cite{yu2021pointbert} && $94.0 $\footnotesize $\pm 3.6$ & $95.9 $\footnotesize $\pm 2.3$ && $89.4 $\footnotesize $\pm 5.1$ & $92.4 $\footnotesize $\pm 4.6$ \\
        Point-BERT\,\cite{yu2021pointbert}       && $94.6 $\footnotesize $\pm 3.1$             & $96.3 $\footnotesize $\pm 2.7$               && $91.0 $\footnotesize $\pm 5.4$               & $92.7 $\footnotesize $\pm 5.1$               \\
        MaskPoint\,\cite{liu2022maskpoint} && $95.0 $\footnotesize $\pm 3.7$ & $97.2 $\footnotesize $\pm 1.7$ && $91.4 $\footnotesize $\pm 4.0$ & $93.4 $\footnotesize $\pm 3.5$ \\
        Point-MAE\,\cite{pang2022pointmae}                     && $96.3 $\footnotesize $\pm 2.5$           & $97.8 $\footnotesize $\pm 1.8$ && $92.6 $\footnotesize $\pm 4.1$ & $95.0 $\footnotesize $\pm 3.0$ \\
        Point-M2AE\,\cite{zhang2022pointm2ae}                     && $96.8 $\footnotesize $\pm 1.8$           & $98.3 $\footnotesize $\pm 1.4$ && $92.3 $\footnotesize $\pm 4.5$ & $95.0 $\footnotesize $\pm 3.0$ \\
        \arrayrulecolor{black!10}\midrule\arrayrulecolor{black}
        from scratch                    && $93.8 $\footnotesize $\pm 3.2$           & $97.1 $\footnotesize $\pm 1.9$             && $90.1 $\footnotesize $\pm 4.6$             & $93.6 $\footnotesize $\pm 3.9$             \\
        \datavec{}  && $96.2 $\footnotesize $\pm 2.6$           & $97.8 $\footnotesize $\pm 2.2$             && $92.6 $\footnotesize $\pm 4.9$             & $95.0 $\footnotesize $\pm 3.2$ \\
        \textbf{\name{}} (Ours) && $\mathbf{97.0} $\footnotesize $\pm 2.8$  & $\mathbf{98.7} $\footnotesize $\pm 1.2$    && $\mathbf{93.9} $\footnotesize $\pm 4.1$    & $\mathbf{95.8} $\footnotesize $\pm 3.1$    \\
        \bottomrule
    \end{tabular}
    }
\end{table}
Following the standard evaluation protocol proposed by Sharma \etal\,\cite{sharma2020ssl}, we test the few-shot capabilities of \name{} in a $m$-way, $n$-shot setting.
To this end, we randomly sample $m$ classes and select $n$ instances for training at random for each of these classes. 
For testing, we randomly pick $20$ unseen instances from each of the $m$ support classes.
We provide the standard deviation over $10$ independent runs.
In \reftab{modelnet_fewshot_results}, we report a new state-of-the-art by improvements up to $+1.3\%$ in the most difficult $10$-way $10$-shot setting.
\emakefirstuc{\name{}} clearly outperforms the \datavec{} baseline in all settings.
We conclude that \name{} learns rich feature representations which are also well suited for transfer learning in a low-data regime.

\paragraph{Part Segmentation.}
Finally, we address the task of part segmentation, which assigns a semantic part label to each point in a 3D point cloud of a single object.
For this purpose, we employ a simple segmentation head that is similar to the segmentation head in Point-MAE\,\cite{pang2022pointmae}.
First, we average the outputs of the 4th, 8th, and 12th Transformer blocks to incorporate features from multiple levels of abstraction.
We then concatenate the mean- and max-pooling of the $n$ averaged token outputs, along with the one-hot encoded class label of the object, to obtain a global feature vector.
At the same time, we up-sample the $n$ averaged outputs from the corresponding center points to all points using a PointNet\texttt{++}\,\cite{qi2017pointnetplusplus} \emph{feature propagation layer}, which uses inverse distance weighting and a shared MLP to produce a local feature vector for each point.
Finally, we concatenate the global feature vector with each local feature vector and a shared MLP predicts a part label for each point.
In~\reftab{ShapeNetPart}, we report competitive results on ShapeNetPart\,\cite{yi16siggraph} which consists of \num{16881} 3D models of $16$ semantic categories.
Apart from Point-M2AE\,\cite{zhang2022pointm2ae}, \name{} outperforms all other self-supervised methods.
We hypothesize that Point-M2AE's multi-scale U-Net like architecture\,\cite{ronneberger2015unet} enables to learn more expressive spatially localized features which results in slightly better scores ($+0.2$\,mIoU$_I$).
Since \name{} relies on a standard single-scale Transformer backbone, we see multi-scale Transformers for 3D point clouds as an interesting orthogonal improvement, similar to the advances in 2D vision\,\cite{zhang2021longformer,fan2021msvit,li2022msvit2,chen2021crossvit} extending vision Transformers\,\cite{dosovitskiy2020vit} with multi-scale capabilities.
\subsection{Analysis}
\label{sec:analysis}
\begin{table}
    \centering
    \setlength{\tabcolsep}{3pt}
    \caption{
        \textbf{Ablation.}
        \emakefirstuc{\name{}} outperforms \datavec{} by a significant margin.
        We find that a deferred shallow decoder~(\textbf{D}) (\reffig{model}~\colorsquare{m_red}) predicting the teacher's representations for masked patches shows consistent improvements but we identify that concealing positional information (\textbf{no \protect\maskembedding{}}) from the student is key.
        We report the overall accuracy on ModelNet40 and ScanObjectNN.
        }
    \label{tab:architecture_ablation}
    \begin{tabular}{lcccccc}
        \toprule
        &&         & \multicolumn{3}{c}{Overall Accuracy}                                                 \\
        \cmidrule(lr){4-6}
                       &&            & \multicolumn{2}{c}{\textbf{ModelNet40}} & \textbf{ScanObjNN}                         \\
        \cmidrule(lr){4-5} \cmidrule(lr){6-6}
                &no \protect\maskembedding{}        & D          &  $+$Voting                         & $-$Voting         & \small \texttt{PB-T50-RS}                 \\
        \midrule
                    \datavec{} &\xmark  & \xmark           & 93.6                          & 93.3             & 85.5             \\
                     &\xmark & \cmark & 94.0              & 93.6 & 86.8             \\
                \textbf{\name{}} &\cmark & \cmark & \textbf{94.8}                 & \textbf{94.7}    & \textbf{87.5}    \\
        \bottomrule
    \end{tabular}
\end{table}

\paragraph{Leakage of Positional Information.}
\begin{figure}[t]
\centering
\subcaptionbox{\scriptsize Disclosed Positions (\textbf{\datavec{}})}{\includegraphics[width=0.50\linewidth,trim={0cm 0.8cm 0cm, 1.4cm},clip]{figures/early_leakage/chair_leak_d2v_0_gray_crop.png}}%
\hfill
\subcaptionbox{\scriptsize Concealed Positions (\textbf{\name{}})}{\includegraphics[width=0.50\linewidth,trim={0cm 0.8cm 0cm, 1.4cm},clip]{figures/early_leakage/chair_leak_p2v_0_gray_crop.png}}%
\caption{\textbf{Leakage Of Positional Information.}
The center points\,\colordot{black} are associated with the corresponding masked embeddings of masked-out patches (\reffig{model}, \protect\maskembedding{}).
\textbf{(a)} As \datavec{} feeds the masked embeddings to the student, the positions of masked patches are disclosed to the student.
Consequently, the %
overall shape of the chair remains clearly visible to the student, thus making the learning objective less challenging.
\textbf{(b)} To overcome this limitation, \name{} excludes masked embeddings from the student and subsequently feeds them only to a decoder.
As a result, several parts of the chair are truly masked from the student input. 
}
\label{fig:early_leakage}
\end{figure}
The main limitation of \datavec{} is that it directly feeds masked embeddings, along with their positional information, to the student network, which undermines the effectiveness of masking.
To visualize this problem, we show a representative example in \reffig{early_leakage}(a).
Revealing the positions of masked patches %
of the chair inadvertently weakens the learning objective because it allows the student to rely on the positional information instead of truly learning to predict the teacher's representations of the corresponding masked-out patches.
To mitigate this issue, \name{} excludes masked embeddings from the student and only subsequently feeds them to the decoder.
As a result, %
several sections of the chair in \reffig{early_leakage}(b) are effectively concealed from the student network, leading to a more resilient learning framework.
In \reftab{architecture_ablation}, we report that \name{} outperforms our baseline \datavec{} by a significant margin of up to $+2.0\%$.
In particular, we observe that the decoder itself provides consistent improvements, but the key contribution of \name{} is to conceal positional information from the student. 
Complementary to our findings, He \etal\,\cite{he2022mae} show that moving masked embeddings to a deferred shallow decoder reduces memory requirements and training time significantly.
Our findings align with those of Pang \etal\,\cite{pang2022pointmae}, who found similar benefits for masked autoencoders on point clouds.





\begin{figure*}[t!]
    \includegraphics[width=1.0\linewidth,trim={0cm 0.65cm 0cm 0.8cm},clip]{figures/representation_qualitative/representation_quality.pdf}
    \vspace{-20pt}
    \caption{
        \textbf{Visualization of Learned Representations.}
        We use PCA to project the learned representations into RGB space.
        Both a random initialization and \datavec{} pre-training show a fairly strong positional bias, whereas \name{} exhibits a stronger semantic grouping without being trained on downstream dense prediction tasks.
    }
    \label{fig:quali_representations}
    \vspace{-15pt}
\end{figure*}
\parag{Masking Strategy.}
\begin{table}
    \centering
    \caption{
        \textbf{Masking Strategy.}
        We explore two variants for masking the input of the student.
        For \textbf{(a)} random masking, we uniformly mask out a given ratio of all embeddings.
        For \textbf{(b)} block masking, we mask out a random embedding and its nearest neighbors. 
        We report the overall accuracy on ModelNet40 and ScanObjectNN.}
    \label{tab:ablation_masking_ratio}
    \vspace{-8pt}
    \hfill
    \subcaptionbox{65\% \emph{random} masking\vspace{-5pt}}
    {\includegraphics[width=0.35\linewidth]{tables/ablation_masking/rand_65_0_gray_crop.png}}%
    \hfill
    \subcaptionbox{65\% \emph{block} masking\vspace{-5pt}}{\includegraphics[width=0.35\linewidth]{tables/ablation_masking/block_65_0_gray_crop.png}}%
    \hfill
    \vspace{-5pt}
    \setlength{\tabcolsep}{2pt}
    \begin{tabular}{lcccc}
        \toprule
         & & \multicolumn{3}{c}{Overall Accuracy}                                           \\
        \cmidrule(lr){3-5}
                                    &       & \multicolumn{2}{c}{\textbf{ModelNet40}} & \textbf{ScanObjNN}                   \\
        \cmidrule(lr){3-4} \cmidrule(lr){5-5}
         Strategy & Masking Ratio   & $+$Voting                         & $-$Voting      & \small \texttt{PB-T50-RS}                \\
        \midrule
        random                      & \input{tables/tikz/ratio_bar_45.tex}    & 94.5                          & 94.3          & 86.8          \\ %
        random & \input{tables/tikz/ratio_bar_65.tex}    & \textbf{94.8}                 & \textbf{94.7} & \textbf{87.5} \\ %
        random                      & 
\definecolor{euc_red}{RGB}{250, 250, 250}
\definecolor{bar_border}{RGB}{200,200,200}

\tikzset{
horizontal fill/.style 2 args={fill=#2, path picture={
\fill[#1, sharp corners] (path picture bounding box.south west) -|
(path picture bounding box.north west) --
($(path picture bounding box.north) + (0.4, 0)$) --
($(path picture bounding box.south) + (0.4, 0)$) -- cycle;}}
}

\adjustbox{valign=c}{
\begin{tikzpicture}[shorten >=0pt,auto,node distance=0.0cm,thin, transform shape, scale=1.0]
\tikzstyle{every state}=[           rectangle,
           rounded corners,
           draw=bar_border, thin,
           minimum height=0em,
           inner sep=0pt,
           text centered]

  \node (geo) [align=center, inner sep=2pt] at (-0.85,0) {\small $75\%$};
  \node (euc) [align=center, inner sep=2pt] at (0.45,0) {};
  
  \node (bb) [state, minimum height=0cm, horizontal fill={m_blue}{euc_red}, inner sep=0pt, fit={(geo) (euc)}] {};
  
  \draw[draw=bar_border] ($(bb.north) + (0.4, 0)$) --
($(bb.south) + (0.4, 0)$);
  
  \node (geo) [align=center] at (-0.85,0) {\small $85\%$};
  \node (euc) [align=center] at (0.45,0) {};
\end{tikzpicture}
}    & 94.5                          & 93.8           & 86.7          \\ %
        \arrayrulecolor{black!10}\midrule\arrayrulecolor{black}
        block                       & \input{tables/tikz/ratio_bar_25.tex}    & 93.9                          & 93.7          & 86.3          \\ %
        block  & \input{tables/tikz/ratio_bar_45.tex}    & 94.5                & 93.8       & 87.4 \\ %
        block                       & \input{tables/tikz/ratio_bar_65.tex}    & 94.0                          & 93.9 & 86.1          \\ %
        \bottomrule
    \end{tabular}
    \vspace{-18pt}
\end{table}
The masking strategy defines which of the student's input embeddings are masked (\reffig{model}, \protect\inlinegraphics{tables/misc/masked_token.pdf}).
In this study, we investigate two variants of masking strategies with different masking ratios: \emph{random} masking and \emph{block} masking.
For random masking, we mask out a specified ratio of embeddings for the student.
In contrast, block masking masks out a random embedding and its nearest neighbors such that the specified masking ratio is achieved.
This strategy puts focus on masking out spatially contiguous regions of the point cloud whereas random masking is independent of position.
Our findings, summarized in \reftab{ablation_masking_ratio}, reveal that random masking with a $65\%$ masking ratio performs best for both ModelNet40 and ScanObjectNN, while block masking lags behind.
We attribute this to the high level of ambiguity that arises when masking a spatially contiguous region, resulting in several potential point clouds that could have given rise to the masked input.
While we seek a challenging pretext task to learn rich representations, ambiguity should not be the primary source of difficulty.

\parag{Visualization of representations learned by \name{}.}
In \reffig{quali_representations}, we show qualitative examples of representations of ModelNet40 instances after pre-training on ShapeNet.
Both a random initialization and \datavec{} pre-training show a strong positional bias, whereas \name{} exhibits a stronger semantic grouping without being trained on downstream dense prediction tasks.
Unlike \datavec{}, \name{} conceals positional information from the student, forcing it to learn more about the semantics of the data, resulting in more semantically meaningful representations.%
