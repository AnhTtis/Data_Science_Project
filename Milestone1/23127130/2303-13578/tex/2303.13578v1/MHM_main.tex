\documentclass[%
 reprint,
%superscriptaddress,
%groupedaddress,
%unsortedaddress,
%runinaddress,
%frontmatterverbose,
%preprint,
%preprintnumbers,
%nofootinbib,
%nobibnotes,
%bibnotes,
 amsmath,amssymb,
 aps,
prl,longbibliography,
%pra,
% prx,
%rmp,
%prstab,
%prstper,
%floatfix,
]{revtex4-2}

\usepackage{graphicx}% Include figure files
\usepackage{dcolumn}% Align table columns on decimal point
\usepackage{bm}% bold math
\usepackage{hyperref}% add hypertext capabilities
\hypersetup{colorlinks=true,breaklinks,linkcolor=blue,urlcolor=blue,citecolor=blue}
\usepackage[mathlines]{lineno}% Enable numbering of text and display math
\usepackage{algorithm}
\usepackage[noend]{algpseudocode}
\usepackage{xcolor}
\usepackage{pifont}
\usepackage{tikz}
\usetikzlibrary{matrix}
\usepackage[normalem]{ulem}
\usepackage{verbatim}
\usepackage{soul}
\usepackage{mathtools}
\usepackage{ulem}
\usepackage[utf8]{inputenc}\DeclareUnicodeCharacter{2212}{-}

\newcommand{\ts}[1]{{\color{blue} [TS: #1]}}
\newcommand{\pt}[1]{{\color{green} [Patrick: #1]}}
\newcommand{\mf}[2]{{\color{pink}\sout{#1} #2}}
\newcommand{\ajm}[1]{{\color{red} [AJM: #1]}}
\newcommand{\jz}[1]{{\color{orange} [Jiawei: #1]}}
\newcommand{\todo}[1]{{\color{red} [TODO: #1]}}

\begin{document}
\preprint{arXiv}

\title{Magnetism and Metallicity in Moir{\'e} Transition Metal Dichalcogenides}

\author{Patrick Tscheppe$^{1}$}
\author{Jiawei Zang$^{2}$}
\author{Marcel Klett$^{1}$}
\author{Seher Karakuzu$^{3}$}
\author{Armelle Celarier$^{5}$}
\author{Zhengqian Cheng$^{2}$}
\author{\\Thomas A. Maier$^{4}$}
\author{Michel Ferrero$^{5,6}$}
\author{Andrew J. Millis$^{2,3}$}
\author{Thomas Sch{\"a}fer$^{1}$}
\email{t.schaefer@fkf.mpg.de}
 
\affiliation{
$^1$Max-Planck-Institut für Festkörperforschung, Heisenbergstraße 1, 70569 Stuttgart, Germany}
\affiliation{
$^2$Department of Physics, Columbia University, 538 West 120th Street, New York, New York 10027, USA}
\affiliation{
$^3$Center for Computational Quantum Physics, Flatiron Institute, 162 5th Avenue, New York, New York 10010, USA}
\affiliation{$^4$Computational Sciences and Engineering Division, Oak Ridge National Laboratory, Oak Ridge, Tennessee 37831-6164, USA}
\affiliation{$^5$CPHT, CNRS, {\'E}cole Polytechnique, Institut Polytechnique de Paris, Route de Saclay, 91128 Palaiseau, France}
\affiliation{$^6$Coll\`ege de France, 11 place Marcelin Berthelot, 75005 Paris, France \\}

\date{\today}

\begin{abstract}
The ability to control the properties of twisted bilayer transition metal dichalcogenides in situ makes them an ideal platform for investigating the  interplay of strong correlations and geometric frustration. Of particular interest are the low energy scales, which make it possible to experimentally access both temperature and magnetic fields that are of the order of the bandwidth or the correlation scale. In this manuscript we analyze the moir{\'e} Hubbard model, believed to describe the low energy physics of an important subclass of the twisted bilayer compounds. We establish its magnetic and the metal-insulator phase diagram for the full range of magnetic fields up to the fully spin polarized state. We find a rich phase diagram including fully and partially polarized insulating and metallic phases of which we determine the interplay of magnetic order, Zeeman-field, and metallicity, and make connection to recent experiments.
\end{abstract}

\maketitle

{\it Introduction.}
The correlation-driven Mott metal-insulator transition --in other words, under what circumstances can electrons move through a material-- is one of the central issues in modern day condensed matter physics. Developments over the past five years in moir{\'e} transition metal dichalcogenides, including the observation of a continuous Mott transition \cite{Li2021} and quantum criticality \cite{Ghiotto2021}, have opened a new experimental frontier in this area \cite{Tang2020,Wang2020,Kennes2021,Wu2018}. Moir{\'e} materials consist of two or more atomically thin layers, perhaps with slightly different lattice constants, stacked at a small twist angle. The lattice mismatch and twist angle, combined with a weak but nonzero interlayer tunnelling, produce experimental platforms whose low energy physics is described by a few-band model with a very large unit cell and therefore very low bandwidth and interaction scales, which moreover are tunable by twist angle, pressure, and the choice of materials in which the moir{\'e} system is embedded \cite{Kennes2021,Tang2020}. One particularly widely studied class of moir{\'e} materials are bilayers comprised of transition metal dichalcogenide materials such as WSe$_2$ and MoTe$_2$ which in appropriate circumstances realize the {\sl moir{\'e} Hubbard model}: a two-dimensional triangular lattice hosting a single band of electrons correlated by an  interaction that to a good approximation may be taken to be site-local. Importantly, the magnitude and form of the interaction and the electronic band structure can be varied over wide ranges {\sl in situ} by changing gate potentials and twist angles \cite{Wu2018,Wu2019,Pan2020,Wang2020,Kennes2021} while all electronic scales are small enough that temperatures and magnetic fields spanning the whole range from very low to higher than the effective bandwidth are experimentally accessible.

While the metal-insulator transition in two dimensional Hubbard models has been studied, both in general \cite{Georges1996,Parcollet2004,Moukouri2001,Schaefer2015,Schaefer2016b,Wietek2021,Klett2020,Downey2022} and in connection to moir{\'e} systems \cite{Morales2021,Zang2022}, the effect of a magnetic field seems (apart from one notable exception \cite{Laloux94}) not to have been investigated, perhaps in part because for most conventional materials the experimentally accessible fields are a tiny fraction of the bandwidth so that linear response theory suffices. Motivated by the wide range of field strengths experimentally accessible in moir{\'e} systems,  in this paper we use state of the art single-site and cluster dynamical mean-field methods to study the metal-insulator phase diagram of the moir{\'e} Hubbard model over the full magnetic field range, assuming that the primary coupling is the Zeeman-coupling to the electronic spin. Orbital effects will be considered in a forthcoming paper.  We reveal full and partially polarized insulating and metallic phases, as well as canted antiferromagnetically ordered phases.

\begin{figure}
		\centering
		\includegraphics{Figures/phasediagram.pdf}
		\caption{Phase diagram of the moir{\'e} Hubbard model (Eq.~\ref{eq:mhm}) for $\phi=\pi/6$ at half-filling and zero external magnetic field $B\!=\!0$ calculated by the dynamical mean-field methods indicated in the legends. Solid reddish lines denote magnetic phase transition lines from a paramagnetic to a 120$^\circ$ ordered antiferromagnetic state, solid blueish lines denote a crossover from a metallic to an insulating region. Dashed blueish lines mark a metal-insulator crossover when the calculation is restricted to a non-magnetic (metastable) solution.}
		\label{fig:phasediagram}
\end{figure}

{\it Model and methods. \label{sec:model}}
We study the  moir{\'e} Hubbard model (MHM), a modification of the well-known Hubbard model \cite{Hubbard1963, Hubbard1964, Gutzwiller1963, Kanamori1963}, the fruit fly of electronic correlations \cite{Qin2022,Arovas2022}. The Hamiltonian is
\begin{eqnarray}
    H&=&\!-\!\sum\limits_{\left<ij\right>,\sigma={\uparrow\downarrow}}c_{i,\sigma}^{\dagger}t_{\sigma}^{ij}c_{j,\sigma}+U\sum\limits_{i}n_{i,\uparrow}n_{i,\downarrow}\nonumber\\&&- g\mu_B B \sum_{i,\alpha,\beta} c^{\dagger}_{i,\alpha} \sigma_z^{\alpha, \beta} c_{i,\beta}.
    \label{eq:mhm}
\end{eqnarray}  Here, $i,j$ represent nearest-neighbor sites on a two-dimensional triangular lattice, $U$ is the (purely local) Coulomb interaction, and $t_{\sigma}^{ij}\!=\!\left|t\right|e^{i\sigma\phi_{ij}}$ is a spin-dependent hopping parameter \cite{Zang2021}, which can be parameterized by a complex phase $\phi$ arising from the strong spin-orbit coupling of the constituent layers and a magnitude $t$. $g$ is the gyromagnetic factor of an electron, $\mu_\text{B}$ is the Bohr magneton, and $B$ an externally applied field in $z$-direction. The structure of the model is such that at $\phi=\pi/6$ at half-filling the model has a particle-hole symmetry, a nested Fermi surface, and a third-order van Hove point, implying that at  $T\!=\!B\!=\!0$ the system is a $120^{\circ}-$antiferromagnetic insulator at even infinitesimal coupling; while for $\phi\neq \pi/6$ the model at $T=0$ is a paramagnetic metal at small interaction strengths, with a first order magnetic and metal-insulator transition as the relative interaction $U/t$ is increased above a critical value \cite{Zang2021}. Both $t$ and $\phi$ may be experimentally tuned {\sl in situ} by the application of appropriate gate voltages. In this work we analyze the half-filled situation $\left<n_\uparrow\right>\!+\!\left<n_\downarrow\right>\!=\!1$, considering both paramagnetic and $120^\circ$ magnetically ordered states. 

We investigate this model by means of the dynamical mean-field theory (DMFT \cite{Metzner1989,Georges1992,Georges1996}),  in its single site and cluster forms. We employ two flavors of  cluster DMFT: the cellular DMFT (CDMFT \cite{Maier2005} with center-focused post-processing \cite{Klett2020}) and the dynamical cluster approximation (DCA \cite{Maier2005}). We use cluster sizes $N_c\!\in\!\left\{1,3,7,9\right\}$. For our calculations at non-zero temperatures we use continuous-time quantum Monte Carlo in its interaction expansion (CT-INT), using the TRIQS package \cite{TRIQS}, to solve the dynamical mean-field equations \cite{Rubtsov2005,Gull2011}. These methods provide results only above a certain low temperature limit, which is typically low enough that a reliable extrapolation to the $T=0$ physics is possible. For some of our calculations we employ the recently developed Variational Discrete Action (VDAT) method \cite{Cheng21} which provides an extremely computationally efficient estimate of ground state properties of the single-site model.  Details of the solvers,  cluster geometries and implementations are given in the Supplemental Material \cite{Supplemental}.

\begin{figure*}[t!]
		\centering
		\includegraphics{Figures/T0.pdf}
		\caption{(a) Zero temperature phase diagram for the half-filled perfectly nested MHM ($\phi=\pi/6$), calculated in the interaction strength ($U$)-magnetic field ($B$) plane in the single-site dynamical mean-field approximation at $T=0$ using the VDAT method in the paramagnetic phase, i.e., without permitting spontaneous ordering. A phase boundary separates a large $B$, fully spin-polarized trivially insulating phase (green shaded region) from a small $B$ partially spin-polarized phase. For $U/t\lesssim 6$ and $U/t\gtrsim 12$ the transition to the fully polarized phase is second order (green circles) while for ($6\lesssim U/t\lesssim 12$)  the transition is first order (green squares). The dashed line shows the Hartree-Fock approximation to the full-polarization transition line. The dotted line shows the mean-field approximation to the full polarization transition of a nearest-neighbor Heisenberg model with $J\propto t^2/U$. Also shown is the critical coupling  $U_c$ of the Mott-Hubbard metal-insulator transition, which is seen to be reentrant as a function of field (purple pentagons). The vertical line at $U/t=4$ indicates $B$ fields at which a dynamical mean-field calculation at a temperature $T/t=1/17\approx 0.06$ leads to an antiferromagnetic insulator (red triangles), paramagnetic metal (blue stars) or trivial fully spin-polarized insulator (green circles). (b)-(g) Single-site dynamical mean-field results for the Matsubara frequency dependence of the imaginary part of the Green function (averaged over spin) and the two spin components of the dynamical self energy $\Delta \Sigma=\Sigma(\omega)-\Sigma(\omega\!\rightarrow\!\infty)$ obtained at the nonzero temperature $T/t=1/17$ for several  magnetic fields along the vertical $U/t=4$ line of (a).}
		\label{fig:T0phasediagram}
\end{figure*}


{\sl Zero field phase diagram.} For orientation and to demonstrate the robustness of our methods we present in Fig.~\ref{fig:phasediagram} the zero-field phase diagram of the fully nested ($\phi=\pi/6$) model in the temperature ($T$)-interaction ($U$) plane obtained from single-site and cluster dynamical mean-field methods. Paramagnetic insulator, paramagnetic metal and antiferromagnetic insulator phases are found. The phase boundaries determined by the different methods are quantitatively similar almost everywhere, strongly suggesting that the results we find are insensitive to cluster effects. The only important difference is that, as is well known, the single-site DMFT method strongly overestimates the low $T$ critical $U$ needed to drive a paramagnetic metal-paramagnetic insulator phase transition; but it is important to note that the region of large difference occurs within the $120^\circ$-antiferromagnetic phase (i.e., below $T_N$) where the paramagnetic phase single-site DMFT calculation is irrelevant.

We remark that the calculations involve a mean-field approximation, so at finite $N_c$ the calculations do not capture the long-wavelength fluctuations that convert the transition to one of the Kosterlitz-Thouless type (for $\phi\neq0$) or push the transition temperature to zero (for the Heisenberg-symmetry $\phi=0$ case \cite{Mermin1966}). The mean-field temperature found here should be interpreted as setting the scale at which magnetic fluctuations become both strong and long ranged.

\begin{figure*}
		\centering
		\includegraphics{Figures/magnetization_spectralweight.pdf}
		\caption{(a) Magnetic field (B) dependence of antiferromagnetic order parameter $m_{xy}$ (blue triangles) and spin polarization $m_z$ (orange squares). (b) shows $\Xi=-dA/dT$ which is positive (negative) in metallic (insulating) phases (see text). The local spectral functions (c)-(f), obtained with MaxEnt analytic continuation \cite{Kraberger2017}, confirm this classification. All quantities were computed at perfect nesting ($\phi=\pi/6$) for $U/t=4$ and $T/t\!=\!1/17$ with single-site DMFT.}
		\label{fig:observables}
\end{figure*}

{\sl Applied magnetic field at $T\!=\!0$.} Turning now to the effects of a magnetic field, in panel (a) of Fig.~\ref{fig:T0phasediagram} we show the phase diagram  in the magnetic field-interaction plane. This phase diagram was obtained at $T=0$ using the VDAT method to solve the single-site DMFT equations in the paramagnetic phase; however, spot checking the result with cluster methods and by allowing for magnetic order reveals that the VDAT/single-site result for the full polarization line is quantitatively accurate.  A fully spin polarized, trivially insulating high-field phase is separated from a partially polarized phase by a transition line (sharp at $T=0$). At small-to-intermediate interaction the transition is continuous in the sense that the magnetization $m_z$ in the paramagnetic phase evolves smoothly up to the saturation value $m=1$, and our computed phase boundary agrees precisely with the Hartree-Fock result \cite{Zang2021} and results from exact diagonalization \cite{Wietek2022}. At larger interactions $U/t \gtrsim 6$ the transition changes to first order (meaning that $m_z$ jumps from a value less than $1$ to the saturated value) while the line deviates from the Hartree-Fock result and rolls over to the $\sim t^2/U$ saturation field expected for a Heisenberg magnet.

Also shown in the phase diagram is the paramagnetic (Mott) metal-insulator phase boundary. At the $\phi=\pi/6$ value used to construct Fig.~\ref{fig:T0phasediagram} the metallic phase is reentrant: in the small range $12.5\lesssim U/t\lesssim 14$ the zero $B$-field Mott insulator becomes metallic as the field is increased, before again becoming insulating. This reentrance is absent for  $\phi=0$   (see \cite{Supplemental} for the corresponding phase diagram) and it should be noted that cluster effects substantially change the single-site results for the Mott transition.


{\it Applied field at nonzero temperatures.} 
We now incorporate magnetic ordering and non-zero temperatures in the analysis. We focus on the interaction strength $U/t\!=\!4$, believed to be a suitable value for the description of the homobilayer WSe$_2$ \cite{Zang2022}. Panels (b)-(g) of Fig.~\ref{fig:T0phasediagram} show the Green functions and self-energies (minus their Hartree contributions) on the Matsubara axis, obtained from single-site DMFT at $T/t=1/17$. At low field strengths $g\mu_\text{B}B/t\!=\!1$, the system is insulating, indicated by the decrease in the imaginary part of the Green function at low Matsubara frequencies. This gap is opened by a strongly spin-dependent self-energy. At large field strengths $g\mu_\text{B}B/t\! \gtrsim \!3.4$ the system is fully polarized,  the lower $\sigma=\uparrow$ band is completely filled, and the combination of the magnetic field and the interaction opens a gap between the spin up and spin down bands. In between, e.g., at $g\mu_\text{B}B/t\!=\!3$ and temperature $T/t=1/17$, the $xy$-ordering is suppressed by the magnetic field, however, the system is not yet fully $z$-polarized, and, hence, the system is metallic.

We now turn to physical observables that can be obtained from these raw Green function data. Panel~(a) of Fig.~\ref{fig:observables} shows that as the Zeeman-field is increased from zero the staggered magnetization $m_{xy}$ decreases and the uniform  magnetization $m_z$ increases, indicative of the spin canting expected for a Heisenberg-symmetry magnet and sketched on the figure. At the value $g\mu_BB/t=2.8$ the staggered magnetization vanishes, but at this field the uniform magnetization $m_z<1$, indicating at this temperature a small window of paramagnetic partially polarized phase separating the antiferromagnet from the fully polarized state. 


Panel~(b) of Fig.~\ref{fig:observables} examines the evolution of the electronic properties of these states, plotting $\Xi\!\coloneqq\!-dA/dT$, where $A \!=\!-\frac{1}{\pi T}G(\tau=\beta/2)$ is an estimate for the many-body density of states at the Fermi level. For the metallic state $\Xi\!>\!0$, whereas for the gapped  states $\Xi\!<\!0$. The magnetic state at different fields is shown as colored symbols. Densities of states obtained by analytically continuing the Matsubara axis Green function are shown for several points in panels~(c)-(f), confirming the identification of the different phases. We see that at this value of $U$ insulating and magnetically ordered behavior are closely linked.


{\it $T-B$ phase diagrams and generalizations.} Fig.~\ref{fig:BTphasediagram} presents the phase diagram in the field-temperature plane at $U/t=4$ for a different phase angle ($\phi=\pi/8$), for which the nesting is imperfect and the van Hove singularity is removed from the Fermi surface. The analogous phase diagram for perfect nesting can be found in the Supplemental Material \cite{Supplemental}. Although some quantitative features change compared to $\phi\!=\!\pi/6$ (like a smaller onset of the magnetic ordering temperature at zero $B$-field), the two phase diagrams are qualitatively very similar: both show two types of magnetic ordering as well as an intermediate metallic phase (see inset of the right panel for a zoom into this regime).

The ordering temperature $T_\text{N}$ (red triangles), denoting the second order phase transition from a paramagnetic metal to a $120^\circ$ ordered insulator, is reduced upon the application of the external field. Interestingly, as already pointed out before (and in contrast to $T=0$), at finite temperatures an intermediate metallic phase (blue stars) appears with partial $z$-polarization. At even larger fields, the system  enters the fully polarized regime, which is insulating (green circles). At non-zero temperatures the $z$-magnetization is never completely saturated, hence we distinguish the partially polarized state from a `fully polarized' one using the criterion $m_{z}(B, T_{\text{pol}}) = 0.997$, which defines the boundary curve $T_{\text{pol}}(B)$ shown in Fig.~\ref{fig:BTphasediagram}.

We have found that as the temperature is decreased, the range of $B$ over which an intermediate metallic regime is observed decreases; the available evidence implies that at $T=0$ the entire range from $B=0$ up to the saturation field is $xy$-ordered and insulating, consistent with previous Hartree-Fock \cite{Zang2021} and exact diagonalization results \cite{Wietek2022}. Finally, let us note that non-local (spatial) correlations, neglected by DMFT, do not change the picture drastically. This can be inferred from the comparison (and qualitative agreement) of DMFT with 9-site CDMFT calculations in panel (b) of Fig.~\ref{fig:BTphasediagram}.

\begin{figure*}
		\centering
		\includegraphics{Figures/BT.pdf}
		\caption{(a) Magnetic and metallic phases in the presence of an externally applied Zeeman-field $B$ for fixed $U/t\!=\!4$ and $\phi\!=\!\pi/8$ (imperfect nesting), calculated by DMFT. (b) Comparison of DMFT and 9-site CDMFT for $T/t=1/17$ and $\phi\!=\!\pi/6$.}
		\label{fig:BTphasediagram}
\end{figure*}

{\it Conclusions and outlook.} In this paper we have investigated the full magnetic field dependence of the metal-insulator and magnetic field phase diagram of the moir{\'e} Hubbard model (two dimensional triangular lattice Hubbard model with $xy$-magnetic anisotropy and nontrivial hopping phase). Our results refine, extend and generalize the important early work of Laloux and Georges on the infinite dimensional Hubbard model \cite{Laloux94}. Our comparison of single-site and cluster dynamical mean-field approximations confirms that once magnetic ordering is allowed for, the single-site approximation provides a reasonably accurate solution even though the model is two-dimensional and that the VDAT method provides an extremely computationally efficient and highly accurate solution for ground state properties. Within this approximation we generally find, for both nested and non-nested  cases, that at $T=0$, the $B=0$ magnetic order and insulating behavior persists over the entire $B>0$ field range until the system becomes fully polarized, with the order parameter and transition temperature being gradually reduced by the spin canting. Metallic behavior is only found at nonzero temperatures for magnetic fields that suppress the magnetically ordered state to lower temperatures but are too weak to yield a fully spin-polarized state. Our results may be qualitatively compared to recent experiments on twisted WSe$_2$  which indicate that as the field is increased from zero the half filled material undergoes an insulator to metal transition at a field of about 2T, and becomes a fully polarized insulator at a field of $\approx 30$T \cite{Ghiotto23}. A precise comparison is difficult because the bandwidth cannot be unambiguously determined, but estimates from band theory \cite{Wang2020} and quantum oscillation measurements on a sample with a Moire lattice constant of $6.3$nm suggest at low hole density a mass of $\approx 0.4$m$_\text{e}$ implying $t\approx 2$meV. Use of the band theory $g$-factor $\approx 6$ and $U/t=4$ would then lead to a fully polarized state at about $30T$, consistent with experiment. However, the wide field range yielding metallic behavior is inconsistent with the theory. Whether this calls for a multi-orbital or non-local interaction extension of the MHM (analogously to an extended Hubbard model, see, e.g., \cite{Gneist2022,Zhou2022}) is to be clarified by future studies, as well as the role of superconductivity \cite{Belanger2022,Klebl2023} and its connection to experiments.

{\it Acknowledgements.}
The authors are grateful for fruitful discussions with Sabine Andergassen, Laura Classen, Lorenzo Del Re, Antoine Georges, Henri Menke, Michael Scherer, and Nils Wentzell. T.S., M.K., and M.F. acknowledge the hospitality of the Center for Computational Quantum Physics at the Flatiron Institute. The authors acknowledge the computer support teams at CPHT {\'E}cole Polytechnique and at the Flatiron Institute. We thank the computing service facility of the MPI-FKF for their support and we gratefully acknowledge use of the computational resources of the Max Planck Computing and Data Facility.  J.~Z. acknowledges support from the NSF MRSEC program through the Center for Precision-Assembled Quantum Materials (PAQM) NSF-DMR-2011738 and A.J.M. was supported by Programmable Quantum Materials, an Energy Frontier Research Center funded by the U.S. Department of Energy (DOE), Office of Science, Basic Energy Sciences (BES), under award DE-SC0019443. The work by T.A.M. was supported by the U.S. Department of Energy, Office of Science, Basic Energy Sciences, Materials Sciences and Engineering Division. An award of computer time was provided by the INCITE program. This research also used resources of the Oak Ridge Leadership Computing Facility, which is a DOE Office of Science User Facility supported under Contract DE-AC05-00OR22725. The Flatiron Institute is a division of the Simons Foundation.

\bibliography{MHM_main.bib}


\end{document}

