\documentclass[prb,aps,amsmath,amssymb,amsfonts,floatfix,showpacs,singlecolumn,longbibliography]{revtex4-2} %,footinbib,superscriptaddress
\usepackage{graphicx}
\usepackage{epstopdf}
\usepackage[colorlinks=true,urlcolor=blue,citecolor=blue,linkcolor=blue,breaklinks=true]{hyperref}
\usepackage{color}
\usepackage{colortbl}
\usepackage{xcolor}
\usepackage{setspace}
\usepackage[utf8]{inputenc}
\usepackage{amsmath}
%\usepackage{float}
\usepackage{placeins}
\usepackage{amssymb}
\usepackage{nicefrac}
\usepackage{physics}
\usepackage{subcaption}
\usepackage[font=small]{caption}

\newcommand{\remark}[1]{{ \bf [\footnotesize #1]}}

%\newcommand{\Tr}{\operatorname{Tr}}
%\newcommand{\Res}{\operatorname*{Res}}
%\renewcommand{\Im}{\operatorname{Im}}
%\renewcommand{\Re}{\operatorname{Re}}

\usepackage{hyperref}
\hypersetup{colorlinks=true,breaklinks,linkcolor=blue,urlcolor=blue,citecolor=blue}
\begin{document}

\title{Magnetism and Metallicity in Moir{\'e} Transition Metal Dichalcogenides\\
{\it -- Supplemental Material --}}



\author{Patrick Tscheppe}
\author{Jiawei Zang}
\author{Marcel Klett}
\author{Seher Karakuzu}
\author{Armelle Celarier}
\author{Zhengqian Cheng}
\author{Chris A. Marianetti}
\author{Thomas A. Maier}
\author{Michel Ferrero}
\author{Andrew J. Millis}
\author{Thomas Sch{\"a}fer}
\maketitle
In this Supplemental Material we detail our calculation procedures and show additional data. In Sec.~\ref{sec:magnetism} we define different magnetizations and show how we calculated the magnetic ordering temperatures. Sec.~\ref{sec:dispersion} gives the non-interacting dispersion for different $\phi$s. Sec.~\ref{sec:mit} details the procedure of determining the interaction-driven metal-insulator crossover. Sec.~\ref{sec:polarized} defines the procedure for calculations with applied field. Sec.~\ref{sec:methods} gives details about the used algorithms, especially cluster layouts. Sec.~\ref{sec:plots} shows additional phase diagrams referred to in the main text.

\clearpage
\section{Magnetic order}
\label{sec:magnetism}
\begin{figure*}[b!]
		%\centering
		\includegraphics[width=\textwidth]{Figures/supplemental_chi.pdf}
        \captionsetup{justification=raggedright, singlelinecheck=false}
		\caption{Extraction of the N{\'e}el temperature $T_N$ for $xy$ ordering (we show data at $B=0$). (a) Staggered magnetization $m_{xy}$ as a function of applied in-plane field $B_{xy}$ for three different temperatures at $U/t = 7$ in 3-site CDMFT. From the slopes $\chi = \text{d}m_{xy}/\text{d}B_{xy}$ we extract the static susceptibility $\chi$. (b) Close to the transition the behavior of $\chi$ is well captured by a mean-field exponent $\chi^{-1} \propto \abs{T-T_N}$. By linear extrapolation we obtain the N{\'e}el temperature $T_N$.}
		\label{fig:chi}
\end{figure*}
At half-filling and for $\phi=\pi/6$ the low temperature magnetic ground state is characterized by a $120^\circ$ antiferromagnetic ordering pattern \cite{Pan2020, Zang2021, Zang2022}. As a consequence of broken $\mathrm{SU}(2)$ symmetry of the model the spins always align in the bilayer $xy$-plane and we define a corresponding staggered magnetization
\begin{equation}
    m_{xy} = e^{-i\vb{Q}\cdot \vb{R}_j} \qty(\expval{S_j^x} + i\expval{S_j^y}).
\end{equation}
Here $\vb{Q} = (- 4\pi/3a_\mathrm{M}, 0)$ and $a_\mathrm{M}$ denotes the moiré lattice spacing. Treatment of this order is greatly simplified by first applying a unitary transformation $c^\dagger_{j,\sigma} \rightarrow e^{i\sigma \vartheta_j/2}c^\dagger_{j,\sigma}$ mapping $S^x_j + i S^y_j \rightarrow e^{i\vartheta_j} (S^x_j + i S^y_j)$. We choose the phases $\vartheta_j$ such that $\vb{Q} = 0$ after the transformation, which is implemented by the replacement $\phi\rightarrow\phi + \pi/3$ \cite{Zang2021}. Competition between $xy$-order and strong $z$-polarization is then captured by the simpler order parameters $m_{xy}^2 = \expval{S_j^x}^2 + \expval{S_j^y}^2$ and $m_z = \expval{S_j^z}$. \par
At non-zero temperatures we use a response calculation to an externally applied field within the single-site DMFT and CDMFT approximations to obtain the ordering temperature $T_N$ for the $120^\circ$ state. The temperature dependence of the static susceptibility is well described by a mean-field exponent $\chi \propto \abs{T - T_N}^{-1}$ for all cluster sizes considered in this study. We therefore compute the staggered magnetization $m_{xy}$ at weak applied staggered fields $B_{xy}$ and at various temperatures and determine the susceptibility from the slope  $\chi = \text{d}m_{xy}/\text{d}B_{xy}$. The N{\'e}el temperature can then be extracted by linear extrapolation of $\chi^{-1}$. In Fig.~\ref{fig:chi} we show data for a 3-site CDMFT calculation to illustrate this procedure. 

\clearpage
\section{Dispersion}
\label{sec:dispersion}
\begin{figure}[b!]
    \centering
    \begin{subfigure}[b]{0.45\textwidth}
        \caption{$\phi = 0$ \linebreak$\sigma=\,\uparrow$\qquad\qquad\qquad\qquad\qquad\;\;\; $\sigma=\,\downarrow$}
        \includegraphics[width=\textwidth]{Figures/supplemental_phi0.png}
        \vspace{-16pt}
        \label{fig:dispersion(a)}
        \caption*{\scalebox{0.8}{$\textstyle\qquad(\varepsilon_{\vb{k},\sigma} - \varepsilon_{\mathrm{F}})/t$}}
        \vspace{-5pt}
    \end{subfigure}
    \begin{subfigure}[b]{0.45\textwidth}
        \caption{$\phi = \pi/8$\linebreak$\sigma=\,\uparrow$ \qquad\qquad\qquad\qquad\qquad\;\;\; $\sigma=\,\downarrow$}
        \includegraphics[width=\textwidth]{Figures/supplemental_phi8.png}
        \vspace{-16pt}
        \label{fig:dispersion(b)}
        \caption*{\scalebox{0.8}{$\textstyle\qquad(\varepsilon_{\vb{k},\sigma} - \varepsilon_{\mathrm{F}})/t$}}
        \vspace{-5pt}
    \end{subfigure}
    \begin{subfigure}[b]{0.45\textwidth}
        \caption{$\phi = \pi/6$\linebreak$\sigma=\,\uparrow \qquad\qquad\qquad\qquad\qquad\;\;\; \sigma=\,\downarrow$}
        \includegraphics[width=\textwidth]{Figures/supplemental_phi6.png}
        \vspace{-16pt}
        \label{fig:dispersion(c)}
        \caption*{\scalebox{0.8}{$\textstyle\qquad(\varepsilon_{\vb{k},\sigma} - \varepsilon_{\mathrm{F}})/t$}}
        \vspace{-5pt}
    \end{subfigure}
    \captionsetup{justification=raggedright, singlelinecheck=false}
    \caption{Non-interacting dispersion $\varepsilon_{\vb{k},\sigma} - \varepsilon_{\mathrm{F}}$ of the moiré Hubbard model for (a) $\phi=0$, (b) $\phi=\pi/8$ and (c) $\phi=\pi/6$ at half-filling. The Fermi surface $\varepsilon_{\vb{k},\sigma} = \varepsilon_{\mathrm{F}}$ is indicated in white.}
    \label{fig:dispersion}
\end{figure}
Fig.~\ref{fig:dispersion} shows the non-interacting dispersion $\varepsilon_{\vb{k},\sigma}$ of the MHM over the Brillouin zone for several values of $\phi$. At $\phi=0$ we recover the nearest-neighbor tight-binding dispersion on a triangular lattice. However, as $\phi$ is increased toward $\phi=\pi/6$ nesting develops in the spin up and down Fermi surfaces, which are now no longer identical and are related by time reversal symmetry. At $\phi=\pi/6$ there is both perfect nesting and a third-order van Hove singularity directly on the Fermi surface \cite{Zang2021, Zang2022}. For this reason the $120^\circ$ magnetic state persists down to infinitesimally small $U\to 0$. 

\clearpage
\section{Metal-insulator crossover}
\label{sec:mit}
For all temperatures above the magnetic dome (see Fig.~1 in the main text) there is no sharp transition between the metallic and insulating state but rather a smooth crossover. As a criterion for this crossover we use the inflection point in the temperature dependence of the local (spin-summed) spectral function at zero frequency
\begin{equation}
    A_\mathrm{loc}(\omega=0) = -\frac{1}{\pi}\sum_{\sigma =\uparrow/\downarrow} \Im G_{\mathrm{loc},\sigma}(\omega=0)
\end{equation}
as a function of the interaction strength $U$. This quantity is not directly accessible from our imaginary time data, but it can be approximated by
\begin{equation}\label{eq:A_loc}
    A_\mathrm{loc}(\omega=0)\approx-\frac{1}{\pi T}\, G_{\mathrm{loc}}(\tau=\beta/2),
\end{equation}
which we found to agree well with a direct Matsubara frequency extrapolation $i\omega_n \rightarrow i0^+$ of $G_\mathrm{loc}(i\omega_n)$. In Fig.~\ref{fig:mott} we show this quantity both for the single site DMFT and 3-site CDMFT. Inside the magnetic dome the true solution of these methods is always insulating, which necessitates a restriction to the metastable paramagnetic solution in order to track the evolution of the crossover line to lower temperatures. In this study we compute $U_{c2}$ only, i.e. the critical interaction strength obtained by gradually increasing $U$ from small values within the metallic phase. At low temperatures the metal-insulator crossover can turn into a first-order transition, characterized by a hysteresis behavior of $A_\mathrm{loc}(\omega=0)$. We then expect to find a different value $U_{c1}$ for the transition when decreasing the interaction $U$. \\ While there is good quantitative agreement between all cluster sizes (including $N_c = 1$) as to the location of the high temperature crossover, larger differences become apparent at lower temperatures. In particular, as has been extensively studied in previous literature, the slope of the crossover line $U_{c2}(T)$ has a different sign in DMFT as compared to all clusters with $N_c > 1$ (see Fig.~1 of the main text).
\begin{figure*}[b!]
		%\centering
		\includegraphics[width=\textwidth]{Figures/supplemental_mott.pdf}
        \captionsetup{justification=raggedright, singlelinecheck=false}
		\caption{Metal-insulator crossover at non-zero temperature in the perfectly nested moir{\'e} Hubbard model from single-site DMFT and 3-site CDMFT. (a) Zero-frequency spectral weight in DMFT computed using Eq.~\eqref{eq:A_loc}. (b) On the triangular 3-site cluster Eq.~\eqref{eq:A_loc} is applied to the local Green function of an arbitrary site since they are all equivalent by symmetry. In both cases the position $U_{c2}$ of the inflection point is used as a criterion for the metal-insulator crossover at non-zero temperature (vertical dashed lines). We observe that the inclusion of non-local correlations reverses the temperature-dependence of $U_{c2}$.}
		\label{fig:mott}
\end{figure*}

\clearpage
\section{Metallicity in the polarized state}
\label{sec:polarized}
Once magnetic order is included,  at half filling the zero temperature ground state of the moir{\'e} Hubbard model is  insulating for $U$ greater than a critical value which vanishes for the perfectly nested case ($\phi=\pi/6$). However, for $U$ less than about $9t$, even if the $B=0$ ground state is insulating, at $T\neq 0$  we observe a reentrant insulator-metal-insulator transition as a function of the applied Zeeman-field at non-zero temperatures (Fig.~4 of the main text). As described in the main text we determine the metallic region by considering the temperature dependence of the Matsubara frequency estimator of the zero frequency spectral weight Eq.~\eqref{eq:A_loc}: if $A$ decreases to zero as $T$ is lowered we identify the phase as insulating; if it is temperature independent or tends to a nonzero value we identify the phase as metallic. In Fig.~\ref{fig:spectral_weight} this quantity is shown for applied Zeeman-fields $B$ close to the fully $z$-polarized region and at different temperatures. While the spectral weight decreases for small and very large values of $B$ when the temperature is lowered, we observe the opposite behavior in a small intermediate window at around $g\mu_\mathrm{B}B/t \approx 3.1$. Below a temperature of $T/t \approx 0.033$ this metallic phase is absent for $\phi=\pi/6$ in agreement with our zero temperature calculations which show no signs of metallicity. \\
Elaborating on the impact of non-local correlations on the insulator-metal transition, here we also compare the local spectral weights of single-site DMFT and 9-site CDMFT (Fig.~\ref{fig:A_cluster}). The transition between antiferromagnetic insulating and weakly temperature dependent metallic phase is concomitant with a change in the slope $\dd{A}/\dd{B}$ (at $g\mu_B B/t \approx 2.7$ and $\approx 2.3$ in DMFT and CDMFT, respectively) and these slope changes are much more pronounced in the cluster calculations (at $T/t = 0.059$ the slopes in the metallic regime are comparable in both cases, however right before the kink $\dd{A}/\dd{B} \approx 0.11$ in CDMFT v. the much smaller $\dd{A}/\dd{B} \approx 0.04$ in DMFT). \\
The currently available data indicate that the reentrant metallic phase is a temperature effect, with the phase remaining insulating at all $B$ for $T=0$; However, especially for the larger cluster calculations, further studies at lower temperatures (not technically feasible with currently computational resources) would be needed for providing a reliable extrapolation to the ground state.

\begin{figure*}[b!]
		%\centering
		\includegraphics[width=\textwidth]{Figures/supplemental_A.pdf}
        \captionsetup{justification=raggedright, singlelinecheck=false}
		\caption{Metallicity in single-site DMFT at $U/t=4$. (a) Local spectral weight at the Fermi energy for $\phi=\pi/6$ and four representative temperatures. For low and very large values of the applied Zeeman-field $B$ the spectral weight decreases monotonically with temperature. There is an intermediate metallic region around $g\mu_B B/t \approx 3.1$ where the spectral weight increases as the temperature is lowered. However, this region vanishes when $T/t < 0.033$. (b) Shows the same quantity at $\phi=\pi/8$, off of the perfect nesting condition. The qualitative behavior remains unchanged, while the metallic phase persists until lower temperatures.}
		\label{fig:spectral_weight}
\end{figure*}

\begin{figure*}[b!]
		%\centering
		\includegraphics[width=0.5\textwidth]{Figures/A_cluster.pdf}
        \captionsetup{justification=raggedright, singlelinecheck=false}
		\caption{Comparison of the local spectral weight at $\phi=\pi/6$ between single-site DMFT (filled symbols) and 9-site CDMFT (open symbols) as a function of the applied field and for different temperatures. While the slope $\dd{A}/\dd{B}$ changes almost smoothly at these temperatures in DMFT (at $g\mu_B B/t \approx 2.7$) we observe pronounced kinks at  $g\mu_B B/t \approx 2.3$ on the cluster.}
		\label{fig:A_cluster}
\end{figure*}

\clearpage
\section{Numerical methods}
\label{sec:methods}
\begin{figure*}[b!]
		%\centering
		\includegraphics[width=0.8\textwidth]{Figures/supplemental_clusters.png}
        \captionsetup{justification=raggedright, singlelinecheck=false}
		\caption{Illustration of the various cluster geometries of size $N_c \in \{3, 7, 9\}$ used in the CDMFT calculations. (a) Due to lattice symmetry all sites in the three-site cluster are equivalent. (b), (c) In the seven and nine-site geometries there is a central site (red dots) whose nearest neighbors all belong to the cluster. By periodic repetition along superlattice vectors $\vb{A}_1, \vb{A}_2$ (dark red arrows) the clusters form a tiling of the triangular lattice.}
		\label{fig:clusters}
\end{figure*}

\subsection{Cellular dynamical mean-field theory (CDMFT)}
The single-site DMFT includes all local dynamical correlations between electrons. In order to systematically study the corrections due the to non-local correlations we here employ its cellular cluster extension CDMFT \cite{Kotliar2001, Maier2005} for three different cluster sizes having $N_c = 3,7$ and $9$ sites, respectively. In this approach the cluster degrees of freedom take the place of the DMFT single-site impurity, which is used to self-consistently approximate the lattice dynamics. We show the geometry of these clusters along with their corresponding superlattice vectors $\vb{A}_1, \vb{A}_2$ in Fig.~\ref{fig:clusters}. \\ Local quantities (such as magnetization and spectral weight) extracted from the center converge faster with cluster size \cite{Klett2020}. For this reason we show only data at the central site of the 7-site and 9-site clusters in the main text. \\
We solved the self-consistent cluster impurity model using the interaction expansion based continuous time quantum Monte Carlo solver \textsc{CT-INT} as implemented in the TRIQS framework \cite{TRIQS}.
\clearpage
\subsection{Dynamical cluster approximation (DCA)}
DCA is an embedding method where an infinite size lattice is mapped onto a finite size cluster embedded in a dynamic mean-field, which is determined self-consistently~\cite{DCA_ref1}. In contrast to the CDMFT, this mapping is performed in momentum space by tiling the first Brillouin zone into $N_c$ patches as shown in Fig.~\ref{fig:DCAclusters}. This leads to an approximation in which short-range correlations within the cluster are treated accurately while longer-ranged correlations are approximated at the mean-field level. Here we use an $N_c\!=\!3\times 3\!=\!9$ cluster, for which the momentum space patches are shown in Fig.~\ref{fig:DCAclusters}.
To solve the effective cluster problem, we use a continuous-time auxiliary field quantum Monte Carlo (CT-AUX) solver ~\cite{CTAUX_ref1} as described in Ref.~\cite{DCA_code}. Unlike the CDMFT cluster, the DCA cluster is translationally invariant and the local single-particle Green function used to determine the critical $U_c$ shown in Fig.~1 in the main text is calculated as an average over the momentum space Green function. 

\begin{figure*}[b!]
		%\centering
		\includegraphics[width=0.4\textwidth]{Figures/DCA3x3.png}
        \captionsetup{justification=raggedright, singlelinecheck=false}
		\caption{Illustration of the triangular lattice Brillouin zone and the $N_c\!=\!3\times 3\!=\!9$ momentum space patches used in the DCA calculations.}
		\label{fig:DCAclusters}
\end{figure*}

\subsection{Variational discrete action theory (VDAT)}
VDAT is a variational method to solve the ground state of a many-body system with an ansatz for the many-body density matrix, known as the sequential product density matrix (SPD), and a technique to evaluate the ansatz, known as the discrete action theory. The accuracy of the SPD is controlled by an integer $\mathcal{N}$. For the Hubbard model, the SPD with  $\mathcal{N}=1$ recovers the Hartree-Fock wave-function,  $\mathcal{N}=2$ recovers the Gutzwiller wave-function, and the large $\mathcal{N}$ limit recovers the exact ground state. In $d=\infty$, the SPD with any integer  $\mathcal{N}$ can be exactly evaluated within the discrete action theory via the self-consistent canonical discrete action theory (SCDA), which also provides a robust approximation for finite dimensions. The details are described in Refs.~\cite{Cheng2021,Cheng21,Cheng2022}. In this study, we used VDAT within the SCDA at $\mathcal{N}=3$, which is referred to as ``single-site VDAT" for brevity in the main manuscript. The code used to perform VDAT calculations is available \cite{VDATCode}. 


\clearpage
\section{Additional plots}
\label{sec:plots}
In Fig.~\ref{fig:zero_temperature} we compare the zero temperature phase diagram of the perfectly nested moir{\'e} Hubbard model [$\phi=\pi/6$, panel (b)] with the spin-symmetric limit of the model [$\phi=0$, panel (a)]. Please note that, unlike in the perfect nesting situation discussed in the main text, the Mott transition as a function of the applied magnetic field is not reentrant at $\phi=0$. Still, in both cases the boundary of the fully $z$-polarized state follows the mean-field Heisenberg result $g\mu_\mathrm{B}B_{\mathrm{pol}} = 6t^2/U$ for $U$ beyond the Mott transition. \\
Fig.~\ref{fig:BT_phi_pi_6} shows the analogous plot to panel (a) of Fig.~4 of the main text also for $\phi=\pi/6$, i.e., the $B-T$ phase diagram at $U=4t$. Qualitatively the fully nested situation is very similar to the phase diagram at $\phi=\pi/8$. However, closer inspection reveals that the intermediate metallic phase is already completely suppressed at temperatures $T/t < 0.033 $, while it persists down to the lowest investigated temperature $T/t = 0.025$ for $\phi=\pi/8$.

\begin{figure*}[b!]
		%\centering
		\includegraphics[width=0.9\textwidth]{Figures/supplemental_T0.pdf}
        \captionsetup{justification=raggedright, singlelinecheck=false}
		\caption{Zero temperature phase diagram in the $U-B$-plane for the triangular lattice Hubbard model ($\phi=0$) and the perfectly nested moir{\'e} Hubbard model ($\phi=\pi/6$).}
		\label{fig:zero_temperature}
\end{figure*}
\begin{figure*}
		%\centering
		\includegraphics[width=0.9\textwidth]{Figures/supplemental_BT.pdf}
        \captionsetup{justification=raggedright, singlelinecheck=false}
		\caption{Magnetic and metallic phases in the presence of an externally applied Zeeman-field $B$ for fixed $U/t\!=\!4$ and two different values of $\phi\!=\!\pi/6$ and $\phi\!=\!\pi/8$, calculated by DMFT.}
		\label{fig:BT_phi_pi_6}
\end{figure*}

\clearpage

\bibliography{MHM_Supplemental_Material.bib}

\end{document}
