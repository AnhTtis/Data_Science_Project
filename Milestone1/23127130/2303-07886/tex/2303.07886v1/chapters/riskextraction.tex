
\section{Map-Based Risk Prediction}
\label{sec:riskextr}
%Representation of urban situations requires a suitable environment model. 
The target is to represent driving situations with a flexible and efficient environment model. Generalized ADAS for intersections require clear data management functions.  %Representation of driving situations requires a flexible environment model. 
The following section \ref{subsec:ldm} shows how a graph-based local dynamic map can be leveraged for this purpose. Additionally, risk-relevant data within a certain virtual horizon is retrieved to allow predictive trajectory evaluations and planning in section \ref{subsec:trajeval}. %We combine simple but diverse 

\subsection{Relational Local Dynamic Map (R-LDM)}
\label{subsec:ldm}
Underlying road geometry significantly influences and constrains all traffic participants. %in urban cities multi-faceted  is available for the ego driver.  %Additionally, there are many different informations available. 
From the ego driver, multi-faceted sensory inputs have to be processed in real-time.
Since important meta data lies in the connectivity of environment entities,
a graph representation, consiting of nodes, attributes and relations, is beneficial. 
% In this sense, it has items as nodes and more importantly attributes as connections. 
% This also yields a natural link to road structure data, since the road map interconnections and geometry data build a graph. 
The Relational Local Dynamic Map (R-LDM) presents an abstraction layer between sensor data and higher level functions~\cite{ldm2017}.
Such functions include e.g. risk prediction and visualization as required by our RNS. 
With the R-LDM, we support the paradigm shift from single sensor-controller loops towards technologies with enhanced understanding.

\subsubsection{Graph Structure}

A R-LDM graph stores data on four layers based on their dynamicity (compare Fig. \ref{fig:rldm} top right).
The map is represented as nodes with additional data of shape, orientation or type saved as attributes. % at intersections, lane marking types and other information like height and curvature.
On the lowest layer, the street geometry presents one of the most crucial data types. Generally, \textbf{static data} is stored in three levels of granularity, ranging from detailed lanes via half roads to the whole road.
In this context, half roads are the sum of all lanes pointing in the same direction.
Furthermore, the road subgraph consists of alternating segments and junctions, while all junctions at one location forms an intersection.

The second layer comprises \textbf{quasi-static data} with changes on the scale of several days. 
This implies traffic lights, roadside infrastructure, traffic signs or construction sites. 
% The mentioned data falls into static map data which is expected to change on a very slow timescale and quasi-static data with changes on a scale of several days and longer. Here, we have besides traffic lights, also roadside infrastructure, traffic signs or construction sites. 
On the third layer, \textbf{transient data} is stored changing on hours scale or faster (i.e., traffic light states, car density or slippery road conditions). 
Finally, the fourth layer contains highly \textbf{dynamic data} about the state of traffic participants, including vehicles and pedestrians. 
Detected dynamic objects are connected to specific nodes of the lower layers, which can be used for risk extraction in the ego vicinity.
The R-LDM is consequently maintained or updated in real-time.

%, which allow further behavior prediction, planning and support.
With regard to lower layers, available map data sources such as the crowd-sourced OpenStreetMap (OSM)\footnote{See www.openstreetmap.org for further explanation.} can be applied. % \cite{OSM}. %~\cite{OSM}.
% While the lowest layer relies mostly on offline data from OSM, it is the highest layer which comes from post-processed sensor information. 
%STATIC DATA INPUT
% For the download of the map data, we rely on OpenStreetMap (OSM), a crowd-sourced database that is freely available to the public \cite{OSM}. 
OSM mainly consists of navigational information represented in the form of road centerlines and intersection points. 
Therefore, we add topological information on lane level and estimate the actual lane geometry under assumption of a concrete lane width. 
% Afterwards, by combining the lane semantics with an estimate of the lane and road width, we obtain the actual geometric road shape. 
Besides, stop lines are manually added and linked to incoming lanes, while the positions of traffic lights, crosswalks or buildings are similarly extracted from OSM. 

\vspace{0.06cm}
\subsubsection{Virtual Horizon}
\label{sub:horizon}

The graph structure with its connected nodes suits itself to a wide scope of potential queries with regard to path finding, routing or tree search. 
In this way, extraction of behaviorally relevant situations is possible in a straightforward way.
One crucial information for risk estimation is the knowledge of the upcoming driving path of traffic participants.
When considering start and end points in the map, we can intuitively query a \textit{sequence of nodes} (segments and junctions) along the route by traversing its relations. 

Fig. \ref{fig:rldm} shows an example of an intersection scenario. 
Therein, we obtain lane nodes (:LaneSegment, :LaneJunction) and traffic light nodes (:TrafficLight), while the graph is explored based on the current position and orientation of a chosen vehicle (:Vehicle). 
Relevant data is stored in the attributes of the individual nodes. 
Essentially, we can query any processed information that has been acquired online (e.g. from sensor data) or offline (e.g. from OSM).

The left part of Fig. \ref{fig:rldm} depicts the corresponding excerpt of the graph concept that highlights layers (:StaticEntities, :QuasiStaticEntities, :TransientEntities, :DynamicEntities) and granularity.
In order to successfully calculate risk, not only one but several \textit{possible paths} have to be queried. 
In particular, a virtual horizon can be determined which extracts predicted paths for the vehicles in the traffic scene.

\begin{figure}
  \centering
  \resizebox{\linewidth}{!}{
  \import{img/}{rldm.pdf_tex}}
  \caption{Nodes hierarchy and interconnection graph structure within R-LDM for an intersection with traffic lights.}
  \label{fig:rldm}
\end{figure}
\vspace{-0.05cm}

\vspace{0.05cm}
To illustrate this virtual horizon, Fig. \ref{fig:virtualhorizon} addresses a scenario with two approaching cars. 
First, we map the GNSS-based ego vehicle position (in green) onto the center of the corresponding lane element.
% This can be achieved, because the OSM information is given in world coordinates. In this process, we also use . 
Afterwards, we retrieve other cars from lidar (in red) up to a certain distance with R-LDM queries. More stored elements, for instance, quasi-static data with crosswalks can be equally obtained.

By scanning lane nodes for all cars, concatenating their polylines and ordering the branches (straight, left and right, 
etc.) according to a \textit{tree structure}, we obtain distinct paths. In Fig. \ref{fig:virtualhorizon}, every car posseses three possible paths.
The root of the tree is the actual position and the length increases with the number of segments we traverse.
With the virtual horizon, we can directly estimate simple risks. Crossings of the paths reflect conflict zones, where accidents might happen. These are unique and critical for the given road.

\subsection{Trajectory Evaluation}
\label{subsec:trajeval}
After querying the R-LDM for upcoming map structures, traffic elements and sensed objects, we predict trajectories from the vehicle dynamics and evaluate their concrete risk time-courses. Proper safe maneuvers or velocities for recommendation are related to our behavior planning. We divide risks for this purpose into three generic types: 1) collision risk, 2) curvature risk and 3) regulatory risk. In RNS, we combine the risk types to holistically support the driver.
%-Risk sources vs actual risk \\
%-Collision risk (next intersection, intersection zone, intersecting possible paths or intersection lanes) $\rightarrow$ from LDM queries \\

\begin{figure*}
  \centering
  \resizebox{0.82\linewidth}{!}{
  \import{img/}{vhorizon.pdf_tex}}
  \caption{Steps of the virtual horizon shown exemplarily for collision risk. Left: Filtered relevant cars and their positions. Middle: Extraction of possible future paths. Right: Map-based risk extraction, with crosses indicating future path intersections.}
  \label{fig:virtualhorizon} 
\end{figure*} 

\begin{figure}
  \centering
  \resizebox{0.80\linewidth}{!}{
  \import{img/}{type_of_risks.pdf_tex}}
  \vspace{0.08cm}
  \caption{Three risk types of RNS with time as well as space metrics (collision, curve and regulatory risks).}
  \label{fig:risktypes} 
\end{figure} 

\vspace{0.06cm}
\subsubsection{Collision Risk} %\ref{sub:horizon}
Besides coarse conflict zones from the last section \ref{subsec:ldm}, risk metrics of quantitative collision probability can be obtained as well.
In an initial step, paths are transformed with constant velocity models into trajectories. For a car-to-car encounter (indexed 1,2), this gives us position sequences $\mathbf{x}_1(s)$ and $\mathbf{x}_2(s)$ over the predicted time $s$ with assumed constant velocities $v_1$ and $v_2$. 

The distance between the vehicles represents the pointwise trajectory difference and can be written as 
\begin{equation}
d(s)=\left\| \mathbf{x}_2(s) - \mathbf{x}_1(s) \right\|.
\end{equation}
For time risks, we next consider the time of maximal criticality along the predicted trajectories. In other words, we filter out the minimal distance, which is 
the event distance $d_E$ and get the corresponding Time-To-Closest-Encounter (TTCE) as $s_E$. The two indicators are in summary retrieved with

\begin{equation}
d_E=\text{min}\{d(s)\} \text{ and } s_E=\text{argmin}\{d(s)\}.
\end{equation}
\vspace{0.05cm}
\hspace{-0.114cm}Coarsely, collision risk corresponds to the inverse $1/s_E$ and is only considered if $d_E \hspace{-0.02cm} < \hspace{-0.02cm} d_{\text{min}}$ holds true. The minimal value $d_{\text{min}}$ defines the risk sensitivity. 

Once there are multiple trajectory pairs as in Fig. \ref{fig:virtualhorizon}, we would iterate over the combinations $\text{\textbf{x}}_{ij}$ (i.e., variation of ego paths $i$ and other paths $j$) and extract the highest risk by $\text{max}\{1/s_{E,ij}\}$. In this method, we therefore capture time and space of passing vehicles. %\textcolor{red}{emphasize is basic risk, but can also use more 

\subsubsection{Curvature Risk}
From quantitative risks, we can also infer behavior recommendations as target velocities that minimize criticality and maximize ego utility. This is essentially a harder problem, since the driver should now understand the warning as well as interpret and process the advice. 

Consider the situation that the ego user is driving with velocity $v_0$ at the current time $t$. We define a turn segment with curvatures $[\kappa_{\text{start}}, \kappa_{\text{end}}]$ in the extracted path, if a thresholding condition
\begin{equation}
\kappa_{\text{start}} > \kappa_{\text{th}} \text{ and } \kappa_{\text{end}} < \kappa_{\text{th}}
\end{equation}
is valid. Here, the threshold $\kappa_{\text{th}}$, similarly as $d_{\text{min}}$, lets us adjust the turn detection along the future profile $\kappa(s)$. 

In curves, the vehicle is prone to drive off the lane. For higher velocities $v_0$, increased lateral acceleration is exerted. When $a_y$ depicts the dynamics limit that the vehicle can follow and $\kappa_{\text{max}} \hspace{-0.07cm} = \hspace{-0.07cm} \text{max}\{\kappa(s)\}$ is the maximal curvature in the segment, we can derive a target velocity via
\begin{equation}
v_{\text{tar}} = \sqrt{a_y/\kappa_{\text{max}}}.
\end{equation}
The variable $v_{\text{tar}}$ indicates the speed that should be maximally reached at the curve.  For low risk, we should accordingly move slower than $v_{\text{tar}}$, which leads to the condition $v_0 < v_{\text{tar}}$. Driving with $v_0 \ll v_{\text{tar}}$ is also not recommended, because of the reduction in utility.

%Curve risk is extracted solely based on the road. Another structural risk that we could retrieve with the R-LDM is occlusion risk. The visibility of drivers decreases around buildings and relevant cars are overlooked.

\subsubsection{Regulatory Risk}
Quasi-static elements of a stop line, traffic light or crosswalk create risks at rule violation. The reason behind this circumstance is that generally not obeying the norms can lead to unexpected situations with accidents. Only because the vehicles are rule-conform, we can drive safely. %and estimate the intentions of other vehicles.

In this sense, we may want to stop directly in front of the traffic data. Mathematically, we assume a soft braking trajectory with constant deceleration $a_c$ to the stopping point in the distance $d_c$ along the ego path. With the time $s$, we initially obtain the basic kinematic equation
%stopping distance ds and dc
\begin{equation}
d = \frac{a \cdot s^2}{2}.
\label{eq:distancestop}
\end{equation}
Substituting $s$ with $v_{\text{tar}}/a_c$ and inserting $a=a_c$ as well as $d=d_c$ in Eq. (\ref{eq:distancestop}) ultimately results into
\begin{equation}
v_{\text{tar}} = \sqrt{2a_c \cdot d_c},
\label{eq:vtar}
\end{equation}
whereby $v_{\text{tar}}$ is the target velocity to be able to stop at the intersection with certainty.\footnote{For Eq. (\ref{eq:vtar}), we do not account for the reaction time $t_r$. However, $t_r$ can be added for large distances $d_c$ with $v_{\text{tar}} \approx \sqrt{2a_c \cdot d_c} - a_c\cdot t_r$.} 
%wolframalpha https://www.wolframalpha.com/input/?i=%5Bsqrt%282x%29%2C+sqrt%282x%29*sqrt%281%2B%280.8%5E2%29%2F%282*x%29%29-0.8%2C+sqrt%282x%29-0.8%5D+from+0+to+10

Fig. \ref{fig:risktypes} illustrates the corresponding variables for the three types of risk from this section. We want to stress at this point that our approach employs explicit trajectory prediction along the map paths from the R-LDM. Concretely, we assume constant velocity for the trajectory prediction in the collision case, an acceleration or deceleration trajectory to a fixed reduced velocity in the curve case and a smooth braking trajectory to a fixed position for the regulatory case. Therewith, we eventually filter and consider single time points in the trajectory for the risks.
