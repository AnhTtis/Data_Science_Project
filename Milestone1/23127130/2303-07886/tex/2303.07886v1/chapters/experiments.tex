\section{Field Tests}
\label{sec:fite}
This section presents an evaluation of the complete Risk Navigation System (RNS) from Fig.~\ref{fig:blockdiagram}. %Section \ref{sub:quali} shows qualitative ou
We tested RNS offline on real-world recordings using the middleware RTMaps.\footnote{For the RTMaps software, please consult www.intempora.com.}
The mobile platform is a modified Honda CR-V, equipped with an OXTS localization device\footnote{Specifications of OXTS gear are described in www.oxts.com.}  
and six Ibeo Lux lidars allowing $\unit[360]{^\circ}$ perception.\footnote{At last, details of Ibeo sensors can be found on www.ibeo-as.com.} 
Additionally for debugging, a front-facing camera provides images. %of $1240 \times 1080$ pixels 
%%%%%%%%%%%%%discuss about bold and italic%%%%%%%
In the current implementation, RNS runs with a frequency of $\unit[10]{Hz}$ while employing the library OpenGL for environment rendering.
%with a range of up to $\unit[200]{m}$

\begin{figure*}
  \centering
  \begin{overpic}[width=0.9853\linewidth]{./img/results/collision_case_row1_all.jpg}
    \put(24.5,-1){\color{white}\rule[0pt]{3.63pt}{110pt}}
    \put(49.675,-1){\color{white}\rule[0pt]{3.48pt}{110pt}}
    \put(74.85,-1){\color{white}\rule[0pt]{3.42pt}{110pt}}
  \end{overpic}
  \resizebox{0.24135\linewidth}{!}{\import{img/results/}{collision_case_row2_column1.pdf_tex}}
  \resizebox{0.2411\linewidth}{!}{\import{img/results/}{collision_case_row2_column2.pdf_tex}}
  \resizebox{0.2411\linewidth}{!}{\import{img/results/}{collision_case_row2_column3.pdf_tex}}
  \resizebox{0.24085\linewidth}{!}{\import{img/results/}{collision_case_row2_column4.pdf_tex}}
  %\resizebox{0.8\linewidth}{!}{\import{img/results/}{experiments_legend_horizontal.pdf_tex}}
  \caption{Collision risk example. At intersections, RNS calculates TTCE for combinations of ego plus other vehicle's path and informs about the critical encounter point (red line: recorded trace from ego car, green line: trace projected on navigation path). Top: Stream of front camera images. Bottom: Screen layout of system.}
  \label{fig:scenario1}
\end{figure*} 
%Collision risk example -

\begin{figure*}
  \centering
  \begin{overpic}[width=0.981\linewidth]{./img/results/curve_case_row1_all.jpg}
    \put(24.49,-1){\color{white}\rule[0pt]{3.4pt}{110pt}}
    \put(49.65,-1){\color{white}\rule[0pt]{3.48pt}{103pt}}
    \put(74.83,-1){\color{white}\rule[0pt]{3.42pt}{110pt}}
  \end{overpic} \\
  \vspace{-0.015cm}
  \resizebox{0.24\linewidth}{!}{\import{img/results/}{curve_case_row2_column2.pdf_tex}}
  \resizebox{0.24\linewidth}{!}{\import{img/results/}{curve_case_row2_column3.pdf_tex}}
  \resizebox{0.24\linewidth}{!}{\import{img/results/}{curve_case_row2_column4.pdf_tex}}
  \resizebox{0.24\linewidth}{!}{\import{img/results/}{curve_case_row2_column5.pdf_tex}}
  \caption{Curve risk example. RNS detects a sharp turn and recommends to decrease the velocity, which the driver is shortly after abiding to. The velocity scale goes from red (difference of actual to desired velocity) to green (small difference).}
  \label{fig:scenario2}
\end{figure*} 

\begin{figure*}
  \centering
  \begin{overpic}[width=0.9816\linewidth]{./img/results/crosswalk_case_row1_all.jpg}
    \put(24.49,-1){\color{white}\rule[0pt]{3.45pt}{110pt}}
    \put(49.635,-1){\color{white}\rule[0pt]{3.45pt}{110pt}}
    \put(74.81,-1){\color{white}\rule[0pt]{3.96pt}{110pt}}
  \end{overpic}
  \\
  \resizebox{0.24025\linewidth}{!}{\import{img/results/}{crosswalk_case_row2_column1.pdf_tex}}
  \resizebox{0.24\linewidth}{!}{\import{img/results/}{crosswalk_case_row2_column2.pdf_tex}}
  \resizebox{0.24025\linewidth}{!}{\import{img/results/}{crosswalk_case_row2_column3.pdf_tex}}
  \resizebox{0.24025\linewidth}{!}{\import{img/results/}{crosswalk_case_row2_column4.pdf_tex}}
  \caption{Regulatory risk example. In the situation, the driver does not let a prioritized pedestrian pass first on the crosswalk. RNS warns of the upcoming traffic object and displays the distance and colors the ego lane according to the risk.}
  \label{fig:scenario3}
\end{figure*} 

Multiple recordings have been acquired in Offenbach am Main, Germany. 
Necessary inputs of the system are positions, velocity estimates and angle measurements of both ego car and surrounding vehicles. 
Regarding localization of the ego vehicle, we project the GNSS signal onto its navigation route. 
Therefore, we have to know the route of the user in advance.
Alternatively, a filter has to be added that selects the most likely ego path.  
% In this sense, we replayed the recordings and saved the position trace.
%The given examples are from one data stream that is 35 minutes long.
Throughout the course of the experiments, the R-LDM is queried for upcoming risks-related data around the current position. 
% We dynamically reload the virtual horizon after the end of an intersection. 
% The map has to be moderately sized so that RNS can run in real-time with a frequency of $\unit[10]{Hz}$. 

\subsection{Qualitative Outputs}
\label{sub:quali}
In three intersection cases, we assess the support of RNS for \textit{approaching}, \textit{turning} and \textit{crossing} tasks. 
% \subsection{Qualitative Outputs}
%We want to highlight that instead one could let the test driver follow a preset route. 
Qualitative outputs of RNS visualizations are shown. 
Furthermore, we provide values of distances $d_I$, $d_c$, $d_r$ and times $s_E$ as well as driven $v_0$ and target $v_{\text{tar}}$ velocity. 
If needed, the quantitative values can be turned off (slim mode) in order to reduce the workload for the user. 
%After the description of recordings and relation to support functionality of RNS, we  give a short discussion with section \ref{sub:discuss}. 

The first test run is an X-shaped junction with one oncoming other car. 
Four chronologically sorted scenes are given in Fig. \ref{fig:scenario1} with front camera images on top and the synchronized RNS output on the bottom. 
% The ego paths are plotted and the 
We show the recorded ego trace with a red line and projected position as a green arrow tip.
For the prediction horizon, we set $\Delta l_h = \unit[50]{m}$. 
Other detected cars are visualized as yellow dots, while the most critical other vehicle is depicted in blue with its possible paths being displayed as well. 

In the initial image, the ego car is approaching the intersection. 
As can be seen, the junction lies in a distance of $\unit[30]{m}$ for the next picture. 
RNS shows the point of closest encounter and indicates the event time $s_E\hspace{-0.02cm}=\hspace{-0.02cm}\unit[3]{s}$ under the given traffic sign. 
At the time of the third image, we nearly passed the other vehicle with $s_E\hspace{-0.05cm}=\hspace{-0.05cm}\unit[1]{s}$ and are close to the intersection, i.e. distance $d_I\hspace{-0.06cm}=\hspace{-0.06cm}\unit[15]{m}$. 
Finally, no critical situation is present anymore in the last image.
% Lastly, we cross the intersection and drive straight. 
The scenario shows how RNS can \textbf{inform} the driver about possible situations and improves the prediction capability.
% the ego possible paths are plotted in the environment

The second test represents a T-junction with a sharp right turn (see Fig. \ref{fig:scenario2}). 
In this context, $v_{\text{tar}}\hspace{-0.03cm}=\hspace{-0.03cm} \unit[4]{m/s}$ describes the velocity the ego driver should have adopted when reaching the curve.
Since the vehicle exceeds $v_{\text{tar}}$, the velocity scale turns red. 
However, due to appropriate behavior of the driver, i.e., reduction of speed $v_0$, the scale changes to yellow on the second image from the left. 
Here, the distance to the turn $d_r$ decreased simultaneously from $\unit[40]{m}$ to $\unit[10]{m}$. 
% , the driver draws closer to $v_{\text{tar}}$ so that the scale changes to yellow. 
When arriving at the curve, the driver now matches the RNS target velocity  $v_0\hspace{-0.04cm}=\hspace{-0.04cm}v_{\text{tar}}$, denoted in green. 
This example shows how users can leverage \textbf{recommendations} from curve risks using the velocity scale of the RNS.

In the third experiment, a pedestrian intends to use a cross-walk and the car driver ignores the regulatory risk. 
It should be considered that this crosswalk does not have traffic lights, as it is common e.g. in Germany. Since the R-LDM stores traffic elements, we can warn the driver already $d_c=\unit[90]{m}$ in advance with a pop-up symbol. In the following sequence of Fig. \ref{fig:scenario3}, the driver keeps its velocity $v_0$ nearly constant while the suggested stopping trajectory $v_{\text{tar}}$ changes from $\unit[12]{m/s}$ with a green warning, via $\unit[7]{m/s}$ with a yellow velocity scale to $\unit[3]{m/s}$ at $\unit[10]{m}$ distance to the waiting pedestrian. Because of the deviation between the driver and RNS, the \textbf{warning} became critically red in the end. In turn, we are able to guide the driver's awareness towards the most relevant risks.

\subsection{Performance Discussion}
\label{sub:dicuss}
We showed the general applicability of our concept and focused on urban scenarios with single-lane streets.
When handling complex roads, accurate lane-level localization becomes more important.
On one side, the ego vehicle has to be projected on the right lane, which puts demands on the lateral localization precision. 
On the other side, obstacles that are sensed relative to the ego vehicle need to be assigned to their own proper lane, which adds further requirements to both lateral error and ego orientation estimation. 
To tackle this, we refer to e.g. \cite{flade2017}.

The results underline the potential of RNS to proactively support the driver. 
% Active While we do not target actuation or takeover, 
Its visual display allows us to consistently analyze the situational points of interests. In this sense, we provide a navigation on strategical and manoevring level with time horizons of several seconds.

