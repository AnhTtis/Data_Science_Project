\section{Experimental Results}
\label{sec:experimentalResults}
Once both models are trained, these models are tested on a similar but another track called 'E-Track 4' (refer to figure \ref{fig:et4}). The selection of the track is based on twists and turns and complexity in the track profile.  We set the target speed of the car to 60 miles/hrs. For the first experiment, we used both controllers to complete the track without any manual interventions.  The task was to finish one complete lap of this track. One complete lap of this track is 7.041 Km long with a track width of 15 meters.  The random forest controller was able to complete the task without any crash or off-the-track navigation, while the deep neural network-based controller failed to complete the track and crashed multiple times and could not complete even $10\%$ of the track without a crash.  

\begin{figure}[tbhp]
\centering
   \includegraphics[width=1.0
   \linewidth]{images/et-4.png}
   \caption{track: E-track 4}
   \label{fig:et4}
\end{figure}
In the second experiment, we let the experimenter intervene whenever the random forest controller gives a prediction with a high coefficient of variance. A high coefficient of variance (CoV) reflects less confidence in prediction and consequently, its prediction cannot rely upon. In such cases, the control action is shifted to manual control. We observed that the trained random forest controller produced high CoV in scenarios which was very complex and never encountered during training and also when we tried to run it on a significantly different track. This feature is missing with the Deep Neural network controller. A snapshot of the random forest controller driving the car on the test track is shown in figure \ref{fig:drive}. 

\begin{figure*}[h!]
\centering
   \includegraphics[width=1.0
   \linewidth]{images/drive_snap.png}
   \caption{RF controller in Torcs and related statistics, the statistics of our interest are: \textit{odd counts} -number of decision trees out of 2 standard deviations from the mean, \textit{total}- total number of odd counts during whole simulation, \textit{steer}- steering value predicted by regressor, \textit{std}- standard deviation of prediction, \textit{CoV}- coefficient of variance, that explain the spread of the distribution. }
   \label{fig:drive}
\end{figure*}

