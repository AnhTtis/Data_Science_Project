\section{Introduction}
\label{sec:introduction}
A self-driving car has many machine learning and AI-based components involved that act as subsystems in a complex driving system like a perception system, obstacle detection system, etc. However, there are very few examples of an end-to-end driving system\cite{bojarski2016end}  by using ML and AI models. ML and AI have shown extraordinary success in the field of controls\cite{abbeel2010autonomous,vardhan2021rare}, prediction or modeling of complex behavior like cancer detection, stock market prediction, etc \cite{al2019comparative,ghazanfar2017using}, to design automation \cite{vardhan2022deepal}.  In this work , we take two famous and extensively used machine learning model and train it to control the vehicle lateral control. The goal of this study is to understand the comparison between different trained model when used in controlling a system  when an adequate amount of data is available. This comparison study can be used as a baseline and can provide guidelines for AI based control designer an idea about selection of AI-ML model and their respective strength and weakness. For vehicle lateral control, we utilize the open-source car racing simulator called TORCS\cite{wymann2000torcs}. On a given set speed , the data is generated using an traditional PID controller designed by experimenter. The PID controller produces brake value, acceleration and steering to control and drive the car on a given track. During training , for lateral control, we are only interested in steering value and brake and acceleration value is taken using the PID controller. The collected data is distance measured by LIDAR sensor suite and the steering value. For training the controller ,we select decision tree based ML model called Random forest\cite{breiman2001random} and Deep neural network model\cite{ivakhnenko1968group,fukushima1988neocognitron}. Both are trained on same data set and tested in same scenario (track and velocity). 
Our experimentation shows following :  
 \begin{enumerate}
     \item  Random forest-based controller provides better generalization capability than Deep neural network when we done have access of very large data set.
     \item When both trained model is deployed on another track, the random forest-based controller was able to complete the track without crashing, while the deep neural network-based controller  failed to complete the track.
     \item The random forest controller can quantify uncertainty in prediction that can help to decide the failure probability of the controller and override the AI control command if needed.  
 \end{enumerate}

The rest of the paper is organised as follow. Section \ref{sec:methods} formulate the problem and provide background, approach and training details of controller. Section \ref{sec:experimentalResults} provides the experimental results of the experimentation. The related work is discussed in section \ref{sec:relatedWorks} and at the end we produce our conclusion and future direction of research\ref{sec:conclusionFutureWork}.
