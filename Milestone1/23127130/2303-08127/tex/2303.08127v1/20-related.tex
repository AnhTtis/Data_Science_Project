\section{Related Work}\label{sec:related}

\gamename is a re-implementation and extension of 
\cerealbar, a scalable platform to study natural language instruction collaboration~\cite{Suhr2019:cerealbar}. 
\cerealbar was used to study  instruction following~\cite{Suhr2019:cerealbar,Suhr2022:continualfollowing}, instruction generation~\cite{Kojima2021:gen-learn}, and linguistic change~\cite{Effenberger2021:cerealbar-analysis}. 

\gamename is related to instruction following environments, such as SAIL~\cite{MacMahon:06}, R2R~\cite{Anderson:18r2r}, RxR~\cite{Ku2020:room-across-room}, and ALFRED~\cite{Shridhar2020:alfred}. 
In contrast, \gamename is focused on embodied multi-agent collaborations, including with human agents.  

Symbol grounding~\cite{Harnad1990:symbol-grounding-problem}, a core challenge in \gamename, was studied extensively in the single-agent context of instruction following~\cite[e.g.,][]{Chen:11, Artzi:13, Fried:17pragmatic-models, Blukis:18drone} and generation~\cite[e.g.,][]{daniele2016natural, Kojima2021:gen-learn, Wang2021:generatingInstructions}.
The \gamename scenario emphasizes multi-agent collaboration, an aspect that is significantly less studied with natural language instruction. 
The Cards corpus~\cite{Djalali12:cards-preference, Potts:12} presents a related scenario, which has been used for linguistic analysis. 
A related problem is studied by the emergent communication literature~\cite{Lazaridou:17, Andreas:17, Lazaridou2020:emegent-lang-survey}, but with less focus on collaboration with human agents. 
Natural language collaboration between agents with asymmetric capabilities has also been studied with Minecraft-based scenarios~\cite{Narayan2019:collaborative-minecraft-diaglogue, Jayannavar2020:instructions-minecraft-dialogue, kiseleva2022iglu}. 
\gamename differs from these in allowing both agents to affect changes on the environment, enabling ad-hoc modification and delegation of tasks.











