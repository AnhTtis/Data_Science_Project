\section{Crowdsourcing Process}\label{sec:crowdsourcing}

\begin{table*}[t]
    \centering
    \footnotesize
    \begin{tabular}{@{}lccccc@{}}
        \toprule
        \textbf{Dataset} & \textbf{\# Games} & \textbf{\# Instructions} & \textbf{Mean Score} & \textbf{Vocabulary} & \textbf{Mean Instruction Length}\\ 
        \midrule
        Training Data & 185 & 3{,}439 & 6.42 $\pm$ 4.88 & 714 & 10.95 $\pm$ 5.29  \\
        Human-Human Deployment & 187 & 3{,}404 & 6.69 $\pm$ 4.51 & 728 & 11.73 $\pm$ 6.09 \\
        Human-Model Deployment & 188 & 2{,}869 & 3.15 $\pm$ 3.29 & 542 & 9.62 $\pm$ 5.28 \\ 
        \bottomrule
    \end{tabular}
    \vspace{-5pt}
    \caption{Data and interaction statistics for the human-human training data, and the two side-by-side deployments.}\label{tab:game_stats}
    \vspace{-10pt}
\end{table*}

\gamename poses several relatively demanding crowdsourcing tasks. 
Human-human interactions require pairing two workers for real-time play over extended time. 
We design a process to collect \gamename interactions via crowdsourcing, either for games where both roles are controlled by human players, or with one of the sides controlled by a learned model. 
The task-focused design of \gamename allows an effective incentive structure by tying game performance with compensation. 



The key to our process is gradual training of workers. 
A new worker first starts with a tutorial and a qualifier quiz that covers the relatively simple role of the follower. 
The follower role requires following the leader instructions by controlling the character in the game. 
The worker is then qualified to the follower role only, and is paired by joining a dedicated follower-only queue in the lobby. 
Focusing on the follower role only simplifies the learning curve, and much of the learning required for the leader role takes place on the job, as the worker collaborates with more experienced leaders. 


Once the worker displays sufficient level of performance for several games, they are invited to qualify as a leader by taking a leader tutorial and a quiz. 
The second tutorial is both longer and more complex than the follower tutorial, and includes both planning and instruction writing. 
Once the worker completes the tutorial and passes the quiz, they are qualified to the leader role, and can then participate in tasks as both leader or follower. 

We design the lobby to pair workers based on experience. 
Because the leader role is significantly more critical to the effectiveness of the interaction and the quality of language data, we prioritize workers with better performance for it. 
We measure worker performance, keeping track of the mean game score in the most recent games. If two leader-qualified players are waiting in the lobby for matching, we will assign the leader role to the higher performing of the two. 
















