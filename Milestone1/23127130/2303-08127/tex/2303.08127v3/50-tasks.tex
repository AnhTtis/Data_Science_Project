
\section{Example Task Formulations}\label{sec:tasks}

\gamename is well suited to study a variety of tasks, with emphasis on learning and evaluation in collaborative interactions with human agents, such as:


\paragraph{Instruction Following}

The task of instruction following is to map a start state observation from the follower perspective and a leader instruction to a sequence of actions. 
After each action, the agent receives a new observation. 
\citet{Suhr2019:cerealbar} studied this problem with \cerealbar by learning from recorded human-human interactions, and \citet{Suhr2022:continualfollowing} studied it within a continual learning from human feedback scenario. Both approaches were evaluated by deploying follower agents to interact with human leaders. 






\paragraph{Instruction Generation}

The task of instruction generation is to generate a leader instruction for the follower to execute given an observation of the world state from the leader perspective. This requires planning the cards the two agents should select, divide the tasks, plan trajectories, and express the intended follower trajectory in a natural language instruction. 
\citet{Kojima2021:gen-learn} focused on the problem of mapping deterministically generated plans to natural language instructions, and proposed a continual learning approach for learning by observing human follower behavior. 


\paragraph{Emergent Communication}

\gamename is particularly well suited to study emergent communication in multi-agent systems~\cite{Lazaridou2020:emegent-lang-survey}. 
The goal is to jointly learn separate models for the leader and follower. 
The two models generate actions to move in the world. The leader model additionally generates instructions, which the follower model is conditioned on. The learning can be driven by performance in the game. 
\gamename easily allows to integrate human agents into the learning and evaluation processes, bringing natural human language into the process. 
Alternating between interaction between agent-agent and agent-human interactions has the potential to address the language drift problem~\cite{Lee2019:countering-lang-drift}.


