\documentclass[a4paper,11pt,final]{article}


%\usepackage{graphicx,amsmath,amssymb,algorithm,algorithmic} % Add all your packages here
\usepackage{epsfig}
\usepackage[cmex10]{amsmath}
\usepackage{amssymb,amsthm}
\usepackage{mathtools}
\usepackage{nccmath}
\usepackage{soul}
\usepackage[normalem]{ulem}
\usepackage{cite}
\usepackage{todonotes}
\usepackage{tikz}
\usepackage{hyperref}
\usetikzlibrary{hobby,decorations.markings,arrows,intersections,shapes}
\usepackage{pgfplots}
\usepgfplotslibrary{fillbetween}
\pgfplotsset{compat=newest}
\usepgfplotslibrary{fillbetween}
\usepackage{verbatim}



\oddsidemargin=-24pt
\textwidth 180mm   
\textheight 240mm  





\begin{document}





% paper title: Must keep \ \\ \LARGE\bf in it to leave enough margin.
\title{\ \\ \LARGE\bf Do Random and Chaotic Sequences Really Cause Different PSO Performance?}

\author{Paul Moritz N\"orenberg and Hendrik~Richter \\
HTWK Leipzig University of Applied Sciences \\ Faculty of
Engineering \\
        Postfach 301166, D--04251 Leipzig, Germany. \\ Email:
        paul\_moritz.noerenberg@stud.htwk-leipzig.de\\ 
hendrik.richter@htwk-leipzig.de. }



\maketitle



\begin{abstract} 


Our topic is PSO performance differences between  random and chaos. We take random sequences with different  probability distributions and compare them to chaotic sequences with different but also with  same density distributions. This enables us to differentiate between differences in the origin of the sequences (random number generator or chaotic nonlinear system) and statistical differences expressed by the underlying distributions. Our findings obtained by evaluating the PSO performance for various benchmark problems cast considerable doubt on previous results which compared random to chaos and suggested that the choice leads to intrinsic differences in performance. 





\end{abstract}



\section{Introduction}
A main driving force of metaheuristic, nature-inspired, population-based optimization methods is random. This is particularly true for particle swarm optimization (PSO). The  particles of  PSO are supposed to move towards potential solutions of an optimization problem. Their movement and consequently their search dynamics depends on random as a particle's new position is calculated from its current position as well as from randomly weighted cognitive and social components. The most common way for obtaining  the needed random sequences
in practical implementations is by employing a pseudo-random number generator (PRNG)~\cite{lec12}. Starting with a seed value such a generator produces a sequence as a realization of (or at least an imitation of) a random variable with prescribed probability distribution, for instance a normal or an uniform distribution. A popular alternative is using the time evolution of chaotic maps as a substitute for PRNGs. As chaotic dynamics is known to be ergodic, certain maps with appropriate parameters give  sequences with random-like properties comparable to those obtained by PRNGs. Moreover, as chaotic trajectories show a sensitive dependence on initial conditions, sequences with different initial states have different time evolutions but the same underlying statistical properties. Thus, they can be seen as equivalent to realizations of a random process.  Statistical properties of chaotic sequences can be described by the natural invariant density~\cite{ott93,diak96}. The invariant density expresses the distribution of values over the chaotic map's state space, which is equivalent to the probability distribution for random sequences over a sample space.   

If and under what circumstances chaotic sequences give different performance results than random sequences is a topic of intensive  debate~\cite{capo03,chen18,gag21,xu18,ma19,gan13,tian19,plu14,zel22}. This debate has mainly been conducted by taking different chaotic maps and comparing for a set of benchmark problems the performance results of PSO driven by sequences from theses maps  with the results involving PRNGs. In most of these works we find the claim that chaos somehow improves the performance as compared to random but different results have also been reported~\cite{ala09,chen18,gan13,tian19,plu14,zel22,kuang14,rong10,yang12,zheng13}. 
This paper adds to this discussion but proposes a slightly different approach. 

Suppose there are differences in PSO performance between using chaotic and random sequences and given the PSO implementations and parameters are the same except the sequences used. Then, these differences in performance must correspond to differences between the sequences, and  a question worth asking is how these sequences could possibly differ. We put aside the difference in origin of both types of sequences and
consider two major reasons which could make chaotic sequences different from random ones. 
A first reason concerns the sequences as a whole  and involves differences in the underlying distribution or density. A  second reason concerns the timely order of the elements in the sequence and involves  differences in short-term
correlations between successive values. 



Our addition to the discussion about performance differences between chaotic and random sequences is to 
disentangle the effect of the underlying distribution and the effect of short-term correlations between successive value. 
It is well-known that PRNGs can produce
random sequences associated with different  probability distributions, thus offering to study the effect that  different distributions have on PSO performance. A first observation is that
chaotic sequences obtained from a certain variety of nonlinear maps also have different  invariant densities, thus enabling to study the same effect.   Another key observation is that certain chaotic sequences have a density which has the same algebraic description as the distribution of known random sequences.
For instance, the 
time evolution of the logistic map with $r=4$ has a density which is the same function as the Beta distribution with the parameters 
$\alpha=\beta=\frac{1}{2}$.  In other words, for certain parameter settings chaotic sequences from the logistic map and random sequences from the Beta distribution are statistically equivalent as they can be described by one and the same distribution. These observations suggest the following experimental setup. We take chaotic and random sequences which have either the same  distribution or different ones.  We test these sequences in otherwise identical PSO implementations. In this way, we can differentiate  the effect that different or same  distributions have on performance for either random or chaotic sequences. 
Our main result is that the underlying distribution is the key factor in performance, while the origin (either chaotic or random) is secondary.  

The paper is structured as follows. In Section 2  we briefly review generating random and chaotic sequences and compare their invariant and probability densities as well as their short-term correlations. The PSO algorithm and the experimental setup are discussed in Section 3. The results are presented in Section 4 and we close the paper with a discussion of the results and concluding remarks.  


\section{Random and chaotic sequences}
The logistic map
\begin{equation}z(k+1)=rz(k)(1-z(k)) \label{eq:logist} \end{equation}
is defined on the (state space) interval $[0,1]$. 
For certain parameter values of $r$ and  initial states $z(0)\in\mathbb{R}$ it yields  a chaotic trajectory $z=(z(0),z(1),\ldots,z(k),z(k+1))$ which can be interpreted as a sequence of real valued numbers. For $r=4$ and  $k \rightarrow \infty$, the sequence has the
natural invariant density, see~\cite{ott93}, p. 33
\begin{equation}\varrho(z)=\frac{1}{\pi \sqrt{z(1-z)}}. \label{eq:log_dens} \end{equation}
Thus, a chaotic sequence $z$ of the logistic map (\ref{eq:logist}) with $r=4$ can be interpreted as being statistically equivalent to  a realization of a random variable with a distribution defined by $\varrho(z)$.



The Beta distribution
\begin{equation} \mathcal{B}(z,\alpha,\beta)=\frac{z^{\alpha-1}(1-z)^{\beta-1}}{B(\alpha,\beta)} \end{equation}
constitutes a family of continuous probability distributions. It is defined on the (sample space) interval $[0,1]$  and  its form is shaped by the Beta function $\mathbb{B}(\alpha,\beta)$  with two positive parameters $\alpha$ and $\beta$. 
 For $\alpha=\beta=0.5$, the Beta function $\mathbb{B}(0.5,0.5)=\pi$ and consequently the Beta distribution and the natural invariant density of the logistic map (\ref{eq:logist}) equal as $\mathcal{B}(z,0.5,0.5)=\mathcal{B}(0.5/0.5)=\varrho(z)$. In other words, for $\alpha=\beta=0.5$, chaotic sequences of the logistic map and realizations of the Beta distribution have the same distribution and are statistically equivalent~\cite{diak96}. This opens up comparing the effect of random and chaotic sequences on the performance of PSO.   
 
 

Apart from the logistic map, we consider two more chaotic maps. These are the cubic logistic map
\begin{equation}z(k+1)=rz(k)(1-z(k)^2) \label{eq:cubiclogist} \end{equation}
 exhibiting chaotic behaviour for $r=2.62$ and the Bellows map
\begin{equation}z(k+1)=\frac{rz(k)}{1+z(k)^6}, \label{eq:bellows} \end{equation}
which is chaotic for $r=2$. Both maps are defined on the interval $[-1,1]$ and 
as we want to use them analogously to the previously described logistic map for providing chaotic sequences with certain statistical properties, the time evolutions from \eqref{eq:cubiclogist} and \eqref{eq:bellows} are re-scaled to the sample space interval $[0,1]$.
\begin{figure}[htb]
\begin{center}
\includegraphics[trim = 35mm 90mm 42mm 100mm,clip, width=6.2cm, height=5.2cm]{figures/chaotic.pdf}
\includegraphics[trim = 35mm 90mm 42mm 100mm,clip, width=6.2cm, height=5.2cm]{figures/random.pdf}

(a) \hspace{4cm} (b)
\caption{Distribution of sequences used in the study (a) Natural invariant densities of chaotic maps. (b) Probability density functions of random variables. }
\label{fig_chaotic}
\end{center}
\end{figure}

Contrary to Equation~\eqref{eq:log_dens} for the logistic map \eqref{eq:logist} with $r=4$, analytic  descriptions of the invariant densities of the maps \eqref{eq:cubiclogist} and  \eqref{eq:bellows}  are not known.  To have an illustration of the differences in the statistical properties of the sequences, Figure~\ref{fig_chaotic}(a) shows numerically calculated approximations of the invariant densities for all 3 chaotic maps considered in the study. The results are averages over 10.000 different initial states and 400 iterations each. We see that although there is also some similarity between the densities (all have maxima for values close to 0 and close to 1), there are also some notable differences. For instance, the cubic map produces sequences which have another peak at the value 0.5, and the Bellows map has a peak for values between 0.6 and 0.8.

The random sequences used in this study are generated with a Mersenne Twister PRNG~\cite{mat98}. Figure~\ref{fig_chaotic}(b) shows the probability density functions of $\mathcal{B}(0.5/0.5)$, $\mathcal{B}(1/5)$, $\mathcal{B}(5/1)$ and $\mathcal{N}(0.5/0.1)$ and thus provides a comparison of the distributions for the random sequences considered here. The functions of      
$\mathcal{B}(0.5,0.5)$ and the logistic map are the same. The other 3 random sequences are selected  to study the effect of different distributions on PSO performance.

\begin{figure}[htb]
\begin{center}
\includegraphics[trim = 35mm 90mm 42mm 100mm,clip, width=6.2cm, height=5.2cm]{figures/chaotic_corr.pdf}
\includegraphics[trim = 35mm 90mm 42mm 100mm,clip, width=6.2cm, height=5.2cm]{figures/rand_corr.pdf}

(a) \hspace{4cm} (b)
\caption{Estimation $r(\ell)$, Equation \eqref{eq:corr}, of the autocorrelation of the sequences with time lag $\ell$.    (a) Chaotic maps. (b) Random variables. }
\label{fig_corr}
\end{center}
\end{figure}



 To finally have a measure for differences in short-term correlations between successive values, we  estimate the  autocorrelation of a sequence with time lag $\ell$ by 
\begin{equation} r(\ell)=\frac{\sum_{t=0}^{T-\ell}(z(t+1)-\bar{z})(z(t+\ell)-\bar{z})}{\sum_{t=0}^{T-1} (z(t+1)-\bar{z})^2} \label{eq:corr}\end{equation}
with $\bar{z}=\frac{1}{T} \sum_{t=0}^{T-1} z(t+1)$, see e.g~\cite{kan97}. The estimation of autocorrelation $r(\ell)$ compares sections of the sequences and shows how they correlate, see Figure \ref{fig_corr} for results of the same chaotic and random sequences as in Figure \ref{fig_chaotic}. 
The results are obtained by averaging 10.000 samples from different initial states or seeds and $T=400$. 
For $\ell=1$, we compare the same section, thus obtaining $r(1)=1$.   For $\ell>1$, the correlation decays rapidly, but the rate of decay differs between the chaotic sequences, but not between the random sequences obtained by the PRNG. For chaotic sequences, the correlation decays slowest for the Bellows map, and fastest for the cubic map, with the logistic map between them. In Figure \ref{fig_corr}(b), we see that the short-term correlation of the logistic map is very similar to pseudo random sequences, but for $\ell=2$ and $\ell=3$ we find small but statistically significant differences.      

 
 \section{Experimental setup}
 \subsection{PSO algorithm}
\textit{Particle swarm optimization} (PSO) is a generational population-based algorithm used for calculating optima of single- and multidimensional functions. A PSO particle holds 4 variables for each  generation $t$: its position $x(t) \in \mathbb R^n$ and velocity $v(t)\in \mathbb R^n$ in search space, its own best position (local best position) $p^l(t)$ and the best position of any particle of the population that occurred during all generations (global best position) $p^g(t)$. The particle movement is described by  
\begin{align}
x(t+1) =& x(t) + v(t+1) \label{eq:pos}\\
v(t+1) =&wv(t)+c_1r_1(p^l(t) - x(t))+c_2r_2(p^g(t) - x(t)) \label{eq:vel}.
\end{align}
The parameters  $w$, $c_1$ and $c_2$ are the inertial,  cognitive and social weights. The random variables $r_1,r_2$ are taken from the sequences produced by either the PRNGs or chaotic maps. The performance of the PSO algorithm can be significantly affected by inertia, cognitive and social weight.
At $t=0$, each particle $i$, $i=1,2,\ldots I$,  of  swarm size $I$ is initialized randomly in position $x(0)$ with velocity $v(0)$. The fitness function is evaluated for the position of every particle. The position with the highest fitness that occurred in this evaluation is stored as global best position in each particle. The following steps are repeated for $t>0$ until some stop criterion is fulfilled:
\begin{enumerate}
\setlength\itemsep{0.8em}
    \item update velocity for each particle by Equation \eqref{eq:vel}
    \item update position of each particle by Equation \eqref{eq:pos}
    \item evaluate fitness function for each particle
    \item set new local  and global bests, $p^l(t)$ and $p^g(t)$, if needed.
\end{enumerate}

 %It is important, that inertia weight to make sure that the influence on $v_n(k+1)$ of $v_n(k-m)$ decreases with growing $m$. Cognitive and social components have significant influence on the performance and behaviour of the particle-population.

% maybe later use
% Higher numbers in the sequence correspond to the particles taking larger steps in the searchspace. Thus a broad search called exploration is anticipated. Smaller numbers contrariwise promote the particles to do small movements in searchspace leading to a fine scouting called exploitation.
\noindent  In the simulation we take standard values of PSO parameters frequently used:
 \begin{itemize}
     \item $w = 0.79$
     \item $c_1 = 1.49$
     \item $c_2 = 1.49$
     \item $200$ generations per run
     \item $1000$ runs per sample.
 \end{itemize}
\subsection{Test functions} \label{sec:test}
To compare the impact of chaotic and random sequences on PSO the following test functions $f(x)$ from the IEEE CEC 2013 test suite~\cite{li13} are employed:
\begin{itemize}
    \item 1 Equal Maxima (1 dimension)
    \item 2 Uneven Decreasing Maxima (1 dimension)
    \item 3 Himmelblau (2 dimension)
    \item 4 Six-Hump Camel Back (2 dimension)
    \item 5 Shubert (2 dimension)
    \item 6 Vincent (2 dimension)
    \item 7-9 Rastrigin (10, 20 and 30 dimensions)
    \item 10-12 Rosenbrook (10, 20 and 30 dimensions)
    \item 13-15 Sphere (10, 20 and 30 dimensions).
\end{itemize}
\hfill

\subsection{Distributions}
\label{sec:dist}
Random sequences with the following distributions from PRNGs are used: 
\begin{itemize}
    \item $\mathcal{B}(0.5/0.5)$ distribution
    \item $\mathcal{B}(1/5)$ distribution
    \item $\mathcal{B}(5/1)$ distribution
    \item $\mathcal{N}(0.5/0.1)$ distribution. 
\end{itemize} 
Chaotic sequences from the following chaotic maps are taken:
\begin{itemize}
    \item Logistic Map, Equation \eqref{eq:logist}
    \item Cubic Map, Equation \eqref{eq:cubiclogist}
    \item Bellows map, Equation \eqref{eq:bellows}.
 \end{itemize}
\hfill 




\begin{table*}[tb]
  \centering
  \caption{Comparison between chaotic and random sequences for the test function according to Section \ref{sec:test} and the distributions in Section \ref{sec:dist}. Mean and standard deviation of the PSO performance over 1000 runs each. Bold numbers indicate the best performance for each of the 15 test functions. }
    \scalebox{0.7}{
  \begin{tabular}{|c||c|c||c|c||c|c||c|c||c|c||c|c||c|c||}
    \hline
   Test \\ function &  \multicolumn{2}{c}{Logistic Map} & \multicolumn{2}{c}{Cubic Map} & \multicolumn{2}{c}{Bellows Map} & \multicolumn{2}{c}{$\mathcal{B}(0.5/0.5)$} & \multicolumn{2}{c}{$\mathcal{N}(0.5/0.1)$} & \multicolumn{2}{c}{$\mathcal{B}(1/5)$} & \multicolumn{2}{c}{$\mathcal{B}(5/1)$} \\
    \hline
    & $\mu$ & $\sigma$ & $\mu$ & $\sigma$ & $\mu$ & $\sigma$ & $\mu$ & $\sigma$ & $\mu$ & $\sigma$ & $\mu$ & $\sigma$ & $\mu$ & $\sigma$ \\
    \hline
    1 &  0.44e-4 & 0.16e-3 & 0.32e-4 & 0.18e-3 & 0.03e-4 & 0.04e-3 & 0.41e-4 & 0.16e-3 & \textbf{0.00e-4} & 0.00e-3 & 0.02e-4 & 0.03e-3 & \textbf{0.00e-4} & 0.00e-3\\
    \hline
    2 & \textbf{0.10e-2} & 1.05e-2 & 0.26e-2 & 1.95e-2 & 0.15e-2 & 1.38e-2 &  0.16e-2 & 1.48e-2 & 1.12e-2 & 4.12e-2 & 1.13e-2 & 4.24e-2 & 1.19e-2 & 4.24e-2\\
    \hline
    3 &  \textbf{1.58e-1} & 0.69e-2 & \textbf{1.58e-1} & 0.70e-2 & 1.60e-1 & 0.94e-2 & \textbf{1.58e-1} & 0.68e-2 & 1.63e-1 & 1.17e-2 & 1.59e-1 & 0.87e-2 & 1.59e-1 & 0.86e-2\\
    \hline
    4 &  \textbf{1.66e-1} & 2.61e-2 & \textbf{1.66e-1} & 2.59e-2 & \textbf{1.66e-1} & 2.62e-2 & \textbf{1.66e-1} & 2.60e-2 & \textbf{1.66e-1} & 2.61e-2 & 1.68e-1 & 2.54e-2 & \textbf{1.66e-1} & 2.61e-2\\ 
    \hline
    5 & 5.16e-2 & 3.26e-2 & 5.22e-2 & 3.06e-2 & 5.23e-2 & 3.19e-2 &  5.22e-2 & 3.21e-2 &4.93e-2 & 3.09e-2 & \textbf{4.82e-2} & 2.88e-2 & 4.98e-2 & 3.07e-2\\
    \hline
    6 &  3.62e-6 & 0.22e-5 & 3.50e-6 & 0.22e-5 & 4.11e-6 & 1.41e-5 & 3.68e-6 & 0.22e-5 & \textbf{3.44e-6} & 0.22e-5 & 3.49e-6 & 0.22e-5 & 3.45e-6 & 0.22e-5\\
    \hline
    7 &7.74e-2 & 2.11e-2 & \textbf{7.27e-2} & 1.94e-2 & 7.73e-2 & 2.11e-2 &  7.67e-2 & 2.22e-2 & 7.39e-2 & 1.92e-2 & 7.37e-2 & 1.88e-2 & 7.51e-2 & 2.02e-2\\
    \hline
    8 &  8.03e-2 & 1.75e-2 & 7.72e-2 & 1.77e-2 &8.09e-2 & 1.99e-2 & 7.95e-2 & 1.79e-2 & 7.65e-2 & 1.74e-2 & \textbf{7.60e-2} & 1.54e-2 & 7.84e-2 & 1.85e-2\\
    \hline
    9 &  8.38e-2 & 1.78e-2 & 8.02e-2 & 1.73e-2 & 8.47e-2 & 1.98e-2 & 8.35e-2 & 1.77e-2 & 8.02e-2 & 1.84e-2 & \textbf{7.75e-2} & 1.44e-2 & 8.39e-2 & 2.11e-2\\
    \hline
    10 &  7.28e-2 & 1.73e-2 & 7.29e-2 & 1.79e-2 & 8.05e-2 & 1.58e-2 & \textbf{7.20e-2} & 1.80e-2 & 8.93e-2 & 1.28e-2 & 8.87e-2 & 1.21e-2 & 8.65e-2 & 1.35e-2\\
    \hline
    11 & 9.14e-2 & 0.74e-2 & 9.18e-2 & 0.69e-2 & 9.07e-2 & 0.78e-2 & 9.12e-2 & 0.74e-2 &  9.05e-2 & 0.77e-2 & 9.06e-2 & 0.70e-2 & \textbf{9.04e-2} & 0.76e-2\\
    \hline
    12 &  9.06e-2 & 0.63e-2 & 9.05e-2 & 0.66e-2 & 9.10e-2 & 0.66e-2 & 9.12e-2 & 0.61e-2 & \textbf{9.03e-2} & 0.67e-2 & 9.08e-2 & 0.64e-2 & 9.04e-2 & 0.70e-2\\
    \hline
    13 & \textbf{0.01e-4} & 0.02e-4 & 0.03e-4 & 0.05e-4 & 0.05e-4 & 0.38e-4 &  \textbf{0.01e-4} & 0.02e-4 & 0.36e-4 & 1.73e-4 & 0.36e-4 & 1.72e-4 & 1.36e-4 & 2.26e-4\\
    \hline
    14 &  0.64e-2 & 0.28e-2 & \textbf{0.51e-2} & 0.27e-2 & 0.81e-2 & 0.33e-2 & 0.68e-2 & 0.29e-2 & 0.73e-2 & 0.30e-2 & 0.86e-2 & 0.33e-2 & 0.94e-2 & 0.35e-2\\
    \hline
    15 &  1.67e-2 & 0.41e-2 & \textbf{1.55e-2} & 0.40e-2 & 1.83e-2 & 0.43e-2 & 1.71e-2 & 0.42e-2 & 1.67e-2 & 0.43e-2 & 1.91e-2 & 0.44e-2 & 1.97e-2 & 0.44e-2\\
    \hline
  \end{tabular}
  }
    \label{tab:1a}
\end{table*}


\subsection{Performance evaluation}
The performance of PSO runs is quantified by the mean distance error, which is the mean of the absolute difference in search space between the found best fitness value and the global optimum of the function $f(x)$. 
 A good performance means a low mean distance of the found best result to the actual optimum of the fitness function.

 \subsection{Swarm diversity measure}
 To track the behaviour of the particle swarm during its search after every iteration a diversity measure is used. It  estimates how widely spread in the search space the particles are:
 \begin{equation}
\Delta = \frac{1}{I}\cdot\sum\limits_{i=1}^{I}\cdot\sqrt{\sum\limits_{j=1}^{n}(x_{i\cdot j}- \overline{x}_{j})^2}. \label{eq:divers}
\end{equation}
$\Delta$ is the mean distance of every particle to the swarm center. Swarms that are more evenly spread around the search space have a higher diversity than swarms being concentrated on a specific location. It is to expect that with ongoing iterations the diversity declines during a PSO run.


\section{Results}
Table \ref{tab:1a} presents a comparison between chaotic and random sequences for the test functions according to Section \ref{sec:test} and the distributions in Section \ref{sec:dist}. For each combination of test function and distribution the mean performance and the standard deviation of the performance are given for 1000 PSO runs each. Bold numbers indicate the best performance for each of the 15 test functions. The results vary not much for a given test function; there are several test functions where more than one distribution scores best results, but there  are  also subtle differences in performance. For instance,  PSO driven by sequences associated with the cubic map performs best for 5 out of the 15 test functions, but Bellows map only scores 1 out of 15  and the logistic map has 4 out of 15. For the random sequences the results are a little more balanced with PSO driven by sequences from  $\mathcal{B}(0.5/0.5)$ and $\mathcal{N}(0.5/0.1)$ performing best 4 times out of 15, while $\mathcal{B}(1/5)$ and $\mathcal{B}(5/1)$ come up with 3 best results. At least for the selection considered, every distribution is best for at least one test function. Moreover, if we were to compare chaotic sequences from the cubic map and random sequences from $\mathcal{B}(1/5)$ (or $\mathcal{B}(5/1)$), we might be inclined to conclude that chaos and random give different performances with chaos somewhat being better. In the following, we analyse the performance results and argue that these differences are rather caused by differences in the underlying distribution and to a lesser degree by the fact that the sequences come from either PRNGs or chaotic maps.

A first step is by displaying the distribution of the 1000 PSO runs as boxplots to evaluate not only mean and standard deviation of the result data, but also quartiles and outliers. We take 2 examples which are typical for all test functions considered. For the case of the two dimensional shubert function some systematic differences of the PSO performance can be seen in Figure \ref{fig_shubert}(a). In addition to the data in Table~\ref{tab:1a}, we also give the result of a re-run with $\mathcal{B}(0.5/0.5)$, for which we get the same mean and standard deviation, but a slight difference in the lower whisker. Although the means and standard deviations and also the median, first and third quartiles for all distributions are almost equal,
for PSO run samples that were driven by $\mathcal{B}(0.5/0.5)$, the logistic map and Bellows map, the lower quartile of the performance data is significantly narrower than their counterparts resulting from  $\mathcal{B}(1/5)$, $\mathcal{B}(5/1)$, $\mathcal{N}(0.5/0.1)$ and the cubic map. In other words, we may conclude that  PSO performs slightly worse when using $\mathcal{B}(0.5/0.5)$, logistic Map or Bellows Map, respectively, as for sequences from the other 3 distributions very good results fall into the quartile group 1. Moreover,  for these 3 distributions the performance is almost the same. Furthermore,  $\mathcal{B}(0.5/0.5)$ in both runs and the logistic map are most similar. 
The performances for decreasing maxima also show interesting effects, see Figure \ref{fig_shubert}(b). Again PSO samples  driven by $\mathcal{B}(0.5/0.5)$, the logistic map, Bellows map or $\mathcal{B}(1/5)$ show very similar performances. For the other PSO setups several runs with outlying performance occur. Again the performances of PSO driven by $\mathcal{B}(0.5/0.5)$ and the logistic map are not distinguishable.

\begin{figure}[htb]
\begin{center}
\includegraphics[width=8cm,height=6cm]{figures/shubert_8_dists_final.png}
\includegraphics[width=8cm,height=6cm]{figures/decreasing_maxima_8_dist_final.png}

(a) \hspace{4cm} (b)


\caption{Box plot for PSO performance data for the shubert and the decreasing maxima function.}
\label{fig_shubert}
\end{center}
\end{figure}
%\FloatBarrier

%\begin{figure}[!h]
%\begin{center}
%\includegraphics[width=0.5\textwidth]{figures/decreasing_maxima_8_dist_final.png}
%\caption{decreasing maxima}
%\label{fig_decreasing_maxima}
%\end{center}
%\end{figure}
%\FloatBarrier







We next analyse boxplots for 3 test function with varying dimension, Rastrigin, Rosenbrock and Sphere with dimension $n=\{10,20,30\}$.  In particular,  we compare chaotic and random sequences with the focus of differences and sameness in distribution and 
consider:
\begin{itemize}
    \item chaotic sequences from the logistic map vs. random sequences with $\mathcal{B}(0.5/0.5)$, which have the same distribution, Figure \ref{fig_CPSO_Beta0505}.
    \item  random sequences from $\mathcal{B}(0.5/0.5)$ vs. $\mathcal{N}(0.5/0.1)$,  which have different distributions, Figure \ref{fig_B0505_N0501}. 
    \item chaotic sequences from the logistic map vs. Bellows map, which have different distributions, Figure \ref{fig_log_bell}. 
\end{itemize}
It can generally be seen that significant differences occur when the underlying distributions differ. See, for instance, in Figure \ref{fig_B0505_N0501} the performance differences for the 30D Rastrigin or 10D Rosenbrock for the comparison between $\mathcal{B}(0.5/0.5)$ and $\mathcal{N}(0.5/0.1)$),  or in Figure \ref{fig_log_bell} the performance difference for 10D Rosenbrock for the comparison between logistic map and Bellows map. 
Furthermore, no significant differences in performance has been found for the comparison between chaotic sequences from the logistic map vs. random sequences with $\mathcal{B}(0.5/0.5)$. There are some differences in the quartile groups, mainly in group 3 and the upper whisker, and in the outliers. This is an indication that while the fundamental  sameness in performance corresponds to a general match in distribution,  the subtle differences in performance could be traced to differences in short-term correlation, as shown for the logistic map vs. $\mathcal{B}(0.5/0.5)$, see Figure \ref{fig_corr}(b). 

\begin{figure}[tb]
\begin{center}
\includegraphics[width=8cm,height=6cm]{figures/boxRRS_200_logistic_Map_B0505.png}
\caption{Comparison of PSO driven by the logistic map vs. $\mathcal{B}(0.5/0.5)$.}
\label{fig_CPSO_Beta0505}
\end{center}
\end{figure}

\begin{figure}[tb]
\begin{center}
\includegraphics[width=8cm,height=6cm]{figures/boxRRS_200_B0505_N0501.png}
\caption{Comparison of  PSO driven by $\mathcal{B}(0.5/0.5)$ vs. $\mathcal{N}(0.5/0.1)$.}
\label{fig_B0505_N0501}
\end{center}
\end{figure}

\begin{figure}[tb]
\begin{center}
\includegraphics[width=8cm,height=6cm]{figures/boxRRS_200_logistic_Map_Bellows_Map.png}
\caption{Comparison of PSO driven by the logistic map vs. Bellows map.}
\label{fig_log_bell}
\end{center}
\end{figure}


The next analysis serves to study a possible mechanism of how differences in  random or chaotic sequences may cause differences in performance.  Therefore, we evaluate the swarm diversity as defined by Equation \eqref{eq:divers} over the run time of the PSO with 200 iterations and averaged over 1000 runs.

\begin{figure*}[h]
\begin{center}
\includegraphics[width=1\textwidth]{figures/diversity_2by_B0505N0501.png}

(a)
\includegraphics[width=1\textwidth]{figures/diversity_2by_B0505logistic_Map.png}

(b)
\caption{Decline of averaged swarm diversity $\bar{\Delta}(t)$ over 200 iterations . (a) PSO with $\mathcal{B}(0.5/0.5)$ and  $\mathcal{N}(0.5/0.1)$. (b) PSO with $\mathcal{B}(0.5/0.5)$ and  logistic map.}
\label{fig_div_Beta_Norm}
\end{center}
\end{figure*}

\clearpage


In Figure \ref{fig_div_Beta_Norm} we see how sequences with a given distribution influence the behaviour of the particle swarm during a PSO run. Figure \ref{fig_div_Beta_Norm}(a) shows for all test functions considered the average swarm diversity $\bar{\Delta}(t)$ on semi-logarithmic plots over run time $t$ for $\mathcal{B}(0.5/0.5)$ and $\mathcal{N}(0.5/0.1)$, while Figure \ref{fig_div_Beta_Norm}(b) does the same for $\mathcal{B}(0.5/0.5)$ and the logistic map.  In all cases we observe a decline in swarm diversity which is a characteristic feature of population-based metaheuristic search and reflects the PSO search dynamics. More specific, the search dynamics  undergoes exploration and exploitation~\cite{liu05,tian19}. However, comparing $\mathcal{B}(0.5/0.5)$ and $\mathcal{N}(0.5/0.1)$, Figure \ref{fig_div_Beta_Norm}(a),   we notice that the curves showing the characteristic decline of swarm diversity  differ considerably. For the multimodal test functions such as decreasing maxima, six hump camel back, vincent and all dimensions of the Rastrigin function the swarm contracts less strongly for using $\mathcal{N}(0.5/0.1)$ as compared to $\mathcal{B}(0.5/0.5)$. In contrast, there are only small differences in the characteristic decline of swarm diversity for the PSO using $\mathcal{B}(0.5/0.5)$ as compared to the logistic map, Figure \ref{fig_div_Beta_Norm}(b). In other words,  
when the used sequences do not differ in distribution, but only in their respective source, the characteristic decline of swarm diversity is most similar. In fact, the swarms contract for every test function in almost the same way.








Our final analysis step is to compare the performance  with a nonparametric test of the null hypothesis. For this task, the Wilcoxon rank-sum test is widely used to verify performance differences between variants of metaheuristic algorithms. It is here taken to evaluate performance differences between chaotic and random sequences. The two sided test evaluates the null hypothesis that two given data samples are samples from continuous distributions with the same median. We use a 0.05 significance level.
The Wilcoxon rank-sum test is used on every combination of PSO performance samples. Table \ref{tab:wilcoxon} states for each comparison between differently driven PSO the following tuple $[+,-,\thickapprox]$ with:
\begin{itemize}
    \item $+$ \dots number of test functions where the PSO driven by the sequence in the \textit{row} performs better
    \item $-$ \dots number of test functions where the PSO driven by the sequence in the \textit{column} performs better
    \item $\thickapprox$ \dots number of test functions where the  PSO performances are not distinguishable by the Wilcoxon rank-sum test.
\end{itemize}
The results show that performance differences primarily occur when the used sequences differ in the underlying probability distribution. For instance, compare the results for the logistic map with $\mathcal{B}(0.5/0.5)$, which gives 5 superior, 6 inferior and 4 equal performances, or the results for  $\mathcal{B}(0.5/0.5)$ vs.  $\mathcal{B}(1/5)$ with 4 superior, 5 inferior and 6 equal.
In contrast  the PSO using sequences from  $\mathcal{B}(0.5/0.5)$ and the logistic map have very small deviations in performance with 1 superior, 2 inferior and 12 equal. This is almost on a par with a re-run with another set of realizations of $\mathcal{B}(0.5/0.5)$, which gives 1 inferior and 14 equal. Another interesting comparison is the overall performance of chaotic sequences vs. random sequences, see the bold numbers in Table \ref{tab:wilcoxon} and the summation in the last row and column. Here, we find for chaos vs. random that the cubic map gives 22 superior, 11 inferior and 27 equal performances, while the logistic and Bellows map score 14 superior vs. 18 (and 22) inferior and 28 (and 24) equal performances. Similar results  can be recorded from the perspective of random sequences vs. chaotic sequences. For instance,  $\mathcal{B}(0.5/0.5)$ has 7 superior, 9 inferior and 29 equal performances, but  $\mathcal{N}(0.5/0.1)$ scores 16 superior, 11 inferior and 18 equal. To summarize some chaotic sequences perform better than random, but there are also some random sequences which perform better than chaos, which agrees with previously reported result~\cite{ala09,chen18,gan13,tian19,plu14,zel22,kuang14,rong10,yang12,zheng13}. In view of the results presented here, however, it does not appear plausible to assume the presence of general and systematic differences in performance between chaotic and random sequences irrespective and  independent of taking into account the underlying distribution. Given the fact that for random and chaotic sequences with the same distribution, logistic map and $\mathcal{B}(0.5/0.5)$, we get the smallest difference in performance, while for all other combinations of different sources, we get different performances, the results rather support the conclusion that the underlying distribution rather than the origin is the main influential factor   in PSO performance.

\begin{table*}[tb]
  \centering
   \caption{Performance comparison between chaotic and random sequences by Wilcoxon rank-sum test for all 15 test function according to Section \ref{sec:test} and the 7 distributions in Section \ref{sec:dist}. For each cell the tuple $[+,-,\thickapprox]$ indicates for $+$ that the sequences from the distribution in the row performs better, for $-$ that the sequences from the distribution in the column performs better, while for $\thickapprox$  there is no difference in performance. Bold numbers indicate comparisons between chaotic and random sequences. }
  \scalebox{0.65}{
  \begin{tabular}{|c||c|c|c|c|c|c||c||}
    \hline
    & $\mathcal{B}(0.5/0.5)$  & \parbox{1cm}{\centering cubic\\map} & \parbox{1cm}{\centering Bellows\\map} & $\mathcal{N}(0.5/0.1)$ & $\mathcal{B}(1/5)$ & $\mathcal{B}(5/1)$ & chaotic vs. random\\
    \hline
    $\mathcal{B}(0.5/0.5)$ & [1,0,14] &   &    &  [4,7,4] &  [5,6,4] &  [5,4,6] &  \\
    \hline
    logistic map & \textbf{[2,1,12]} &    [1,5,9] &  [5,2,8] &  \textbf{[3,6,6]} &  \textbf{[5,6,4]} &  \textbf{[4,5,6]} & 14 ($+$), 18 ($-$), 28 ($\thickapprox$)\\
    \hline
    cubic map & \textbf{[5,2,8]} &   &     [9,1,4] &  \textbf{[5,3,7]} &  \textbf{[5,4,6]} &  \textbf{[7,2,6]} & 22 ($+$), 11 ($-$), 27 ($\thickapprox$)\\
    \hline
    Bellows map & \textbf{[2,4,9]} &   &      &  \textbf{[3,7,5]} &  \textbf{[5,6,4]} &  \textbf{[4,5,6]} & 14 ($+$), 22 ($-$), 24 ($\thickapprox$)\\
    \hline
    $\mathcal{N}(0.5/0.1)$ &  &      &   &  &  [2,3,10] &  [5,3,7] &\\
    \hline
    $\mathcal{B}(1/5)$ &  &   &     &  &  &  [5,2,8] &\\
    \hline \hline
    random vs. chaotic & 7 ($+$), 9 ($-$), 29 ($\thickapprox$)& & & 16 ($+$), 11 ($-$), 18 ($\thickapprox$) & 16 ($+$), 15 ($-$), 14 ($\thickapprox$) & 12 ($+$), 15 ($-$), 18 ($\thickapprox$)& \\ \hline
    %$\mathcal{B}(5/1)$ &  &   &   &   &  &  & \\
    %\hline
  \end{tabular}
  }
   \label{tab:wilcoxon}
\end{table*}
\section{Discussion}
 There is an ongoing debate about using random and chaotic sequences for driving population-based metaheuristic search algorithms and the effect on behaviour and performance such a selection has~\cite{capo03,chen18,gag21,xu18,ma19,gan13,tian19,plu14,zel22}. This paper adds to this discussion with the focus on PSO and proposed the following approach. We not only use random sequences associated with different  probability distributions, but also compare to chaotic sequences with different  density distributions and complete the experimental setup by random and chaotic sequences with the same distribution. This enables us to differentiate between differences in the origin of the sequences (PRNGs or chaotic maps) and statistical differences expressed by the underlying distributions. 
 
  These differences in statistical properties can potentially have profound influences on PSO  search dynamics, and thus on behaviour and performance. The particle movement directly depends on the sequences as successive particle positions are calculated via cognitive and social components  weighted by the elements from the sequences. The distribution associated with the sequences defines global, long-term search preferences of the particle swarm. If, for instance, the distribution is rather bell shaped, as a normal distribution, 
then the search is more likely centered to the interior of the search space region where the swarm is currently situated.
If, on the other hand, the distribution is rather U-shaped, as Beta distributions with $\alpha=\beta<1$, the search more likely concentrates on the exterior of the current region.  How such a varying search bias interacts with unimodal and multimodal objective functions of the optimization problem and thus causes different types of swarm diversity declines can be seen in Figure \ref{fig_div_Beta_Norm}(a). By this mechanism does the
distribution of the random numbers in a sequence influence the ratio of exploration and exploitation during the PSO run and thus behavior and performance. 

In the experimental setup, we considered 3 sources of chaotic sequences, logistic, cubic and Bellows map, and 4 sources of random sequences, a normal distribution
$\mathcal{N}(0.5/0.1)$ and 3 Beta distributions, $\mathcal{B}(0.5/0.5)$, $\mathcal{B}(1/5)$ and $\mathcal{B}(5/1)$. With this selection, 
we took distributions which are not constant but have shape over the sample space. This is in contrast to an uniform distribution, which is constant and produces random sequences which are equally spread on a fixed interval. For such a type of random distribution results similar to those discussed here are known.
The invariant density of the tent map equals an uniform distribution and    
the tent map has been used for generating chaotic sequences~\cite{kuang14,rong10}.  These works also showed differences in performance between chaotic sequences generated by the tent map as compared to sequences generated by the logistic map~\cite{kuang14,rong10,zheng13}, which is consistent with and supplements our results.








\section{Conclusions}
We considered PSO and address the question if  using chaotic or random sequences really causes different performances. Thus, our discussion is not so much about what sequence performs best, but rather about identifying reasons why  some sequences are better than others.
Our findings cast considerable doubt on previous results which compared random or chaos and suggested that there are intrinsic differences in PSO performance,   irrespective and independent of the associated distributions. We may choose either random or chaotic sequences to drive PSO and observe differences in performance, but most likely this is because the underlying distributions of the chosen  sequences were different.  In other words, if chaotic sequences for a certain optimization problem (or also for a certain collection  of optimization problems) lead to  better performance then this is primary the case because the distribution is more fitting to the problem.

Although our results suggests that the distribution is the main influential factor in PSO performance, there are smaller differences which may also be attributable to  short-term correlations between successive values.  Successive values of a realization of a random variable should, in theory,  be completely uncorrelated. However, in implementations of PRNGs a complete absence of correlation  most likely cannot be guaranteed, but as shown in Figure~\ref{fig_chaotic}(b) these short-term correlations are small and mainly constant for varying random distributions (and the same type of PRNG). For chaotic sequences, short-term correlations vary considerably, see Figure~\ref{fig_chaotic}(a).  Thus, an interesting question for future work is to study the relationships between these correlations and performance. In this context, the idea of mixed chaotic search~\cite{zheng13}, which gives sub-swarms chaotic sequences from different maps, might be a promising  starting point.   











\begin{thebibliography}{1}

\bibitem{ala09} Bilal Alatas, Erhan Akin, and A. Bedri  Ozer. 2009. Chaos embedded particle swarm optimization algorithms. Chaos, Solitons \& Fractals, 40(4), 1715-1734. 

\bibitem{chen18}  Ke Chen,  Fengyu Zhou, and Aling Liu. 2018. Chaotic dynamic weight particle swarm optimization for numerical function optimization. Knowledge-Based Systems, 139, 23-40.

\bibitem{capo03} Riccardo Caponetto,  Luigi Fortuna,  Stefano Fazzino, and
Maria Gabriella Xibilia. 2003. Chaotic sequences to improve the performance of evolutionary algorithms. IEEE Trans. Evolut. Comp. 7, 289-304.

\bibitem{diak96}
Fotis K. Diakonos and Peter Schmelcher. 1996. On the construction of one-dimensional iterative maps from their variant  density:  The  dynamical  route  to  the  beta distribution. Phys. Lett. A211, 199--203.

 
\bibitem{gag21} Iannick Gagnon, Alain April, and Alain  Abran. 2021. An investigation of the effects of chaotic maps on the performance of metaheuristics. Engineering Reports, 3(8), e12369.

\bibitem{gan13} Amir Hossein Gandomi, Gun Jin Yun, Xin-She Yang, and Siamak Talatahari. 2013. Chaos-enhanced accelerated particle swarm optimization. Communications in Nonlinear Science and Numerical Simulation, 18(2), 327-340.

\bibitem{kan97} Holger Kantz and Thomas Schreiber. 1997. Nonlinear Time Series Analysis.  Cambridge University Press, Cambridge, UK. 


\bibitem{kuang14} Fangjun Kuang,  Zhong Jin, Weihong Xu, and Siyang Zhang. 2014.  A novel chaotic artificial bee colony algorithm based on tent map. In: Proc. 2014 IEEE Congress on Evolutionary Computation (CEC),  IEEE, Pistacaway,  235--241.

\bibitem{li13} Xiaodong Li, Andries Engelbrecht, and Michael G. Epitropakis. 2013. Benchmark functions for CEC’2013 special session and competition on niching methods for multimodal function optimization. RMIT University, Evolutionary Computation and Machine Learning Group, Australia, Tech. Rep.

\bibitem{liu05} Bo Liu, Ling Wang, Yi-Hui Jin, Fang Tang,  and De-Xian Huang. 2005. Improved particle swarm optimization combined with chaos. Chaos, Solitons \& Fractals, 25(5), 1261-1271.

\bibitem{lec12}
Pierre L’Ecuyer. 2012.  Random number generation. In:   James E. Gentle, Wolfgang Karl H\"ardle, and Yuichi Mori (Eds.). Handbook of Computational Statistics.  Springer, Berlin, Heidelberg, 35–71.


\bibitem{ma19} Zhiteng Ma, Xianfeng Yuan, Sen Han, Deyu Sun,
and Yan Ma. 2019. Improved chaotic particle swarm optimization algorithm with more symmetric distribution for numerical function optimization. Symmetry, 11(7), 876.

\bibitem{mat98} Makoto
Matsumoto and  Takuji  Nishimura. 1998. Mersenne twister: a 623-dimensionally equidistributed uniform pseudo-random number generator. ACM Transactions on Modeling and Computer Simulation (TOMACS), 8(1), 3-30.

\bibitem{ott93} Edward Ott. 1993. Chaos in Dynamical Systems. Cambridge University Press, Cambridge, UK. 


\bibitem{plu14} Michal Pluhacek, Roman Senkerik, and Ivan  Zelinka. 2014. Particle swarm optimization algorithm driven by multichaotic number generator. Soft Comput. 18(4), 631--639. 





\bibitem{rong10} Hua Rong. 2010. Study of adaptive chaos embedded particle swarm optimization algorithm based on skew tent map. In Proc. 2010 International Conference on Intelligent Control and Information Processing,  IEEE, Piscataway, 316-321.


%\bibitem{song07} Ying Song, Zengqiang Chen, and Zhuzhi Yuan. 2007. New chaotic PSO-based neural network predictive control for nonlinear process. IEEE Trans. Neural Networks, 18(2), 595-601.

\bibitem{tian19} Dongping Tian, Xiaofei Zhao, and Zhongzhi Shi. 2019.
Chaotic particle swarm optimization with sigmoid-based acceleration coefficients for numerical function optimization.
Swarm and Evolutionary Computation,
51,
100573.




\bibitem{xu18} Xiaolong Xu, Hanzhong Rong, Marcello Trovati, Mark Liptrott, and  Nik Bessis. 2018. CS-PSO: chaotic particle swarm optimization algorithm for solving combinatorial optimization problems. Soft Comput. 22(3), 783-795.

\bibitem{yang12} Cheng-Hong Yang, Sheng-Wei Tsai, Li-Yeh Chuang, and Cheng-Huei Yang. 2012. An improved particle swarm optimization with double-bottom chaotic maps for numerical optimization. Applied Mathematics and Computation, 219(1), 260-279.


\bibitem{zel22} Ivan Zelinka, Quoc Bao Diep, Vaclav Snasel, Swagatam Das, Giacomo Innocenti, Alberto Tesi, Fabio Schoen, and Nikolai V. Kuznetsov. 2022. Impact of chaotic dynamics on the performance of metaheuristic optimization algorithms: An experimental analysis. Information Sciences, 587, 692-719.
























\bibitem{zheng13} Hui Zheng, Jing Jie, and Yongping Zheng. 2013. Multi-swarm chaotic particle swarm optimization for protein folding. Journal of Bionanoscience, 7(6), 643-648.



\end{thebibliography}



\end{document}


