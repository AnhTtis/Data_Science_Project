%& math_commands

% CVPR 2023 Paper Template
% based on the CVPR template provided by Ming-Ming Cheng (https://github.com/MCG-NKU/CVPR_Template)
% modified and extended by Stefan Roth (stefan.roth@NOSPAMtu-darmstadt.de)

\documentclass[10pt,twocolumn,letterpaper]{article}

%%%%%%%%% PAPER TYPE  - PLEASE UPDATE FOR FINAL VERSION
%\usepackage[review]{cvpr}      % To produce the REVIEW version
%\usepackage{cvpr}              % To produce the CAMERA-READY version
\usepackage[pagenumbers]{cvpr} % To force page numbers, e.g. for an arXiv version

% Include other packages here, before hyperref.
\usepackage{graphicx}
\usepackage{amsmath}
\usepackage{amssymb}
\usepackage{booktabs}


% It is strongly recommended to use hyperref, especially for the review version.
% hyperref with option pagebackref eases the reviewers' job.
% Please disable hyperref *only* if you encounter grave issues, e.g. with the
% file validation for the camera-ready version.
%
% If you comment hyperref and then uncomment it, you should delete
% ReviewTempalte.aux before re-running LaTeX.
% (Or just hit 'q' on the first LaTeX run, let it finish, and you
%  should be clear).
\usepackage[pagebackref,breaklinks,colorlinks]{hyperref}


% Support for easy cross-referencing
\usepackage[capitalize]{cleveref}
\crefname{section}{Sec.}{Secs.}
\Crefname{section}{Section}{Sections}
\Crefname{table}{Table}{Tables}
\crefname{table}{Tab.}{Tabs.}

% our additions
\usepackage{wrapfig}
\usepackage{multirow, makecell}
\usepackage{url}            % simple URL typesetting
\usepackage{booktabs}       % professional-quality tables
\usepackage{amsfonts}       % blackboard math symbols
\usepackage{nicefrac}       % compact symbols for 1/2, etc.
\usepackage{microtype}      % microtypography
\usepackage{xcolor}         % colors
\usepackage{url}         % colors
\newcommand{\bbox}{\text{bbox}}
\newcommand{\alphapck}{\alpha_\bbox}
\newcommand{\kcycle}{\text{k-CyPCK}}
\newcommand{\cycle}{\text{-CyPCK}}

\newcommand{\I}{\mathbf{I}}
\newcommand{\Ia}{\I^\text{a}}
\newcommand{\Ib}{\I^\text{b}}
\newcommand{\Iatob}{\I^\text{a $\rightarrow$ b}}
\newcommand{\F}{\mathbf{F}}
\newcommand{\Fa}{\F^\text{a}}
\newcommand{\Fb}{\F^\text{b}}
\newcommand{\f}{\mathbf{f}}
\newcommand{\fa}{\f^\text{a}}
\newcommand{\fb}{\f^\text{b}}
\newcommand{\p}{\mathbf{p}}
\newcommand{\pa}{\p^\text{a}}
\newcommand{\pb}{\p^\text{b}}
\newcommand{\A}{\boldsymbol{\Phi}_\text{align}}
\newcommand{\G}{\mathbf{G}}
\newcommand{\C}{\mathbf{C}}
\newcommand{\Ca}{\C^\text{a}}
\newcommand{\Cb}{\C^\text{b}}
\newcommand{\cc}{\mathbf{c}}
\newcommand{\cca}{\cc^\text{a}}
\newcommand{\ccb}{\cc^\text{b}}
\newcommand{\Irec}{\I_\text{Recon}}
\newcommand{\M}{\mathbf{M}}
\newcommand{\Mrec}{\M_\text{Recon}}
\newcommand{\loss}{\mathcal{L}}
\newcommand{\T}{\mathcal{T}}
\newcommand{\W}{\mathcal{W}}
\newcommand{\Id}{\mathcal{I}}

\usepackage[sort, numbers]{natbib}


%%%%%%%%% PAPER ID  - PLEASE UPDATE
\def\cvprPaperID{43} % *** Enter the CVPR Paper ID here
\def\confName{CVPR}
\def\confYear{2023}


\begin{document}

%%%%%%%%% TITLE - PLEASE UPDATE
\title{Visual Chain-of-Thought Diffusion Models}

\author{William Harvey\\
University of British Columbia\\
{\tt\small wsgh@cs.ubc.ca}
% For a paper whose authors are all at the same institution,
% omit the following lines up until the closing ``}''.
% Additional authors and addresses can be added with ``\and'',
% just like the second author.
% To save space, use either the email address or home page, not both
\and
Frank Wood\\
University of British Columbia\\
{\tt\small fwood@cs.ubc.ca}
}
\maketitle

%%%%%%%%% ABSTRACT
\begin{abstract}
   Recent progress with conditional image diffusion models has been stunning, and this holds true whether we are speaking about models conditioned on a text description, a scene layout, or a sketch. Unconditional image diffusion models are also improving but lag behind, as do diffusion models which are conditioned on lower-dimensional features like class labels. We propose to close the gap between conditional and unconditional models using a two-stage sampling procedure. In the first stage we sample an embedding describing the semantic content of the image. In the second stage we sample the image conditioned on this embedding and then discard the embedding. Doing so lets us leverage the power of conditional diffusion models on the unconditional generation task, which we show improves FID by $25 - 50\%$ compared to standard unconditional generation.
\end{abstract}

\section{Introduction}
\label{sec:intro}
Recent text-to-image diffusion generative models (DGMs) have exhibited stunning sample quality~\citep{saharia2022photorealistic} to the point that they are now being used to create art~\citep{oppenlaender2022creativity}. 
%
Further work has explored conditioning on scene layouts~\citep{zhang2023adding}, segmentation masks~\citep{zhang2023adding,hu2022self}, or the appearance of a particular object \citep{ma2023unified}. We broadly lump these methods together as ``conditional'' DGMs to contrast them with ``unconditional'' image DGMs which sample an image without dependence on text or any other information.
%
Relative to unconditional DGMs, conditional DGMs typically produce more realistic samples~\citep{ho2022classifier,bao2022conditional,hu2022self} and work better with few sampling steps~\citep{meng2022distillation}. Furthermore our results suggest that sample realism grows with ``how much'' information the DGM is conditioned on: as hinted at in \cref{fig:stable-diffusion-example} an image is likely to be more realistic if conditioned on being ``an aerial photograph of a road between green fields'' than if it is if simply conditioned on being ``an aerial photograph.''

\begin{figure}
    \centering
    \includegraphics[width=0.49\columnwidth]{figs/uncond-aerial-photo.jpg}
    \includegraphics[width=0.49\columnwidth]{figs/cond-aerial-photo.jpg}
    \caption{\textbf{Left:} Output from Stable Diffusion~\citep{rombach2022high} prompted to produce ``aerial photography''. \textbf{Right:} Using a more detailed prompt\protect\footnotemark with the same random seed removes the ``smudged'' road artifact that appears on the left. VCDM builds on this observation.}
    \label{fig:stable-diffusion-example}
\end{figure}

This gap in performance is problematic. In the spirit of this workshop, imagine you have been tasked with sampling a dataset of synthetic aerial photos which will be used to train a computer vision system. A researcher doing so would currently have to either (a) make up a scene description before generating each dataset image, and ensure these cover the entirety of the desired distribution, or (b) accept the inferior image quality gleaned by conditioning just on each image being ``an aerial photograph''. \footnotetext{We used the prompt ``Aerial photography of a patchwork of small green fields separated by brown dirt tracks between them. A large tarmac road passes through the scene from left to right.''}

To close this gap, we take inspiration from ``chain of thought'' reasoning~\citep{wei2022chain} in large language models (LLMs). Consider using an LLM to answer a puzzle: \textit{Roger has 5 tennis balls. He buys 2 more cans of tennis balls. Each can has 3 tennis balls. How many tennis balls does he have now?}
%
If the LLM is prompted to directly state the answer, it must perform all reasoning and computation in a single step. If instead it is prompted to explain its reasoning as it computes an answer, it can first conclude that the answer is given by the expression $5+2\times3$, and then output an answer \textit{conditioned} on it arising from such an expression. Printing this expression in an intermediate step dramatically improves accuracy~\citep{wei2022chain}.

Let us imagine an image generative model along these lines. When prompted to sample ``an aerial photograph'', it may start by sampling a more detailed description: ``an aerial photograph of a patchwork of small green fields [...]''. 
Given this detailed description, it can leverage the full power of a conditional DGM to generate a high-quality image. Our approach follows these lines but, instead of operating on language, our intermediate space consists of a semantically-meaningful embedding from a pretrained CLIP embedder~\citep{radford2021learning}.

Specifically we train a DGM to model the distribution of CLIP embeddings of images in our dataset. From this we achieve improved unconditional image generation by first sampling a CLIP embedding  and then feeding this CLIP embedding into a conditional image DGM.
% Specifically our methodology learns a diffusion model trained on CLIP embeddings to model the conditioning marginal.  From this we achieve improved unconditional image generation by sampling a CLIP embedding from this generative model of CLIP embeddings and then feeding this CLIP embedding to a conditional image diffusion model.  
Note that, while this technique is related to text-conditional image generation, we are instead applying it to improved {\em unconditional} image generation. We call the resulting model a Visual Chain-of-Thought Diffusion Model (VCDM).
%and moreover requires no test-time language processing.

% The history of AI research is one of competition between approaches which leverage human knowledge and approaches which rely on general-purpose methods and ample data. Richard Sutton points out in his ``bitter lesson'' that, as the cost of compute falls, the general purpose methods will inevitably win out by better leveraging computation. This has repeatedly proven to be true. In the vision domain hand-crafted features were surpassed by convolutional neural networks (CNNs) leveraging only the inductive bias on convolution. Even CNNs are now being replaced as state-of-the-art by vision transformers~\cite{peebles2022scalable} with yet weaker inductive biases. That is not to say that realizing the potential of AI is simply a matter of adding more computation. Large language models (LLMs) can ``hallucinate'' facts without warning~\cite{}, and it is not clear whether more computation will fix this fundamental fallibility. 

% Current image generative models produce extremely convincing low-level detail but often still struggle to produce coherent high-level concepts like multi-legged animals~\citep{saharia2022photorealistic}. These failure modes are even more apparent in video generative models~\citep{ho2022imagen}. And purely data-driven approaches scale poorly with problem dimension for structured tasks like matrix factorization~\cite{weilbach2022graphically}. While scaling the model complexity is likely to help in these cases, much larger (and computationally cheaper) gains may come from new approaches with ``learned'' structure.

% For language models, ``chain of thought'' reasoning provides an intriguing hybrid~\cite{}. Consider using an LLM to answer a puzzle: \textit{Roger has 5 tennis balls. He buys 2 more cans of tennis balls. Each can has 3 tennis balls. How many tennis balls does he have now?}. If the LLM is asked to directly state the answer, it must perform all reasoning and computation in a single step. If it is asked to describe how it reaches the answer, it can take multiple steps to first reduce the problem to the computation $5+2\times3$, and then perform this simple computation. Prompting the model to state its computations in this manner can dramatically improve accuracy~\cite{}. The LLM is leveraging knowledge of the problem structure but, instead of being imposed by human designers, this knowledge has been learned from its training data.

% We propose an analogue of chain-of-thought-prompting for image generation. An analogous representation of intermediate computation in this domain is a description of an image, which we obtain with a pretrained image embedder. We show that first sampling this representation and then sampling an image conditioned on it leads to better performance as measured by FID, ..., and ....


\section{Background}
\paragraph{Conditional DGMs}
We provide a high-level overview of conditional DGMs that is sufficient to understand our contributions, referring to \citet{karras2022elucidating} for a more complete description and derivation. A conditional image DGM~\citep{tashiro2021csdi} samples an image $\rvx$ given a conditioning input $\rvy$, where $\rvy$ can be, for example, a class label, a text description, or both of these in a tuple. We can recover an unconditional DGM by setting $\rvy$ to a null variable in the below. Given a dataset of $(\rvx,\rvy)$ pairs sampled from $\pdata(\cdot,\cdot)$, a conditional DGM $p_\theta(\rvx|\rvy)$ is fit to approximate $\pdata(\rvx|\rvy)$. It is parameterized by a neural network $\hat{\rvx}_\theta(\cdot)$ trained to optimize
\begin{align}\label{eq:diffusion-loss}
    % \mathcal{L}(\theta) = 
    \mathbb{E}_{u(\sigma)p_\sigma(\rvx_\sigma|\rvx,\sigma)\pdata(\rvx,\rvy)} \left[ \lambda(\sigma) \lvert\lvert \rvx - \hat{\rvx}_\theta(\rvx_\sigma, \rvy, \sigma) \rvert\rvert^2 \right]
\end{align}
where $\rvx_\sigma \sim p_\sigma(\cdot|\rvx,\sigma)$ is a copy of $\rvx$ corrupted by Gaussian noise with standard deviation $\sigma$; $u(\sigma)$ is a broad distribution over noise standard deviations; and $\lambda(\sigma)$ is a weighting function. During inference, samples from $p_\theta(\rvx|\rvy)$ are drawn via a stochastic differential equation with dynamics dependent on $\hat{\rvx}_\theta(\cdot)$.

\begin{figure}[t]
    \centering
    \includegraphics[width=0.49\columnwidth]{figs/afhq.png}
    \includegraphics[width=0.49\columnwidth]{figs/ffhq.png}
    \caption{CLIP-conditional samples on AFHQ and FFHQ. Each row shows three samples conditioned on the same CLIP embedding.}
    \label{fig:samples}
\end{figure}


\begin{figure}[t]
    \centering
    \includegraphics[width=\columnwidth]{figs/cond-results-vs-nclusters.pdf}
    \vspace{-4mm}
    \caption{FID versus dimensionality of $\rvy$ on AFHQ~\citep{choi2020stargan} and FFHQ~\citep{karras2018style}. With small training budgets (brown line), it is harmful when $\rvy$ is too informative. With larger training budgets (purple line), it is helpful to make $\rvy$ much more high dimensional.}
    \label{fig:fid-vs-ncomp}
\end{figure}


\paragraph{CLIP embeddings}
CLIP (contrastive language-image pre-training)~\cite{radford2021learning} consists of two neural networks, an image embedder $e_i(\cdot)$ and a text embedder $e_t(\cdot)$, trained on a large captioned-image dataset. Given an image $\rvx$ and a caption $\rvy$, the training objective encourages the cosine similarity between $e_i(\rvx)$ and $e_t(\rvy)$ to be large if $\rvx$ and $\rvy$ are a matching image-caption pair and small if not.
% The embedders are trained by separately embedding images and captions, and encouraging the cosine similarity between embeddings to be large if they are of matching image-caption pairs and small if not.
The image embedder therefore learns to map from an image to a semantically-meaningful embedding capturing any features that may be included in a caption. We use a CLIP image embedder with the ViT-B/32 architecture and weights released by \citet{radford2021learning}. We can visualize the information captured by the CLIP embedding by showing the distribution of images produced by our conditional DGM given a single CLIP embedding; see \cref{fig:samples}.

% The diffusion framework is based on a family of joint distributions ${p_\sigma(\rvx_\sigma,\rvx,\rvy)=\pdata(\rvx,\rvy)\gN(\rvx_\sigma|\rvx,\sigma^2\mI)}$ for $\sigma>0$.  Marginalizing over $\rvx$ and conditioning on $\rvy$ yields the corresponding family of marginal distributions $p_\sigma(\rvx_\sigma|\rvy)$. For large $\sigma$, $p_\sigma(\rvx_\sigma|\rvy)$ is smooth and well-approximated by $\gN(\mathbf{0}, \sigma^2\mI)$. As $\sigma$ tends to zero, $p_\sigma(\rvx_\sigma|\rvy)$ tends to $\pdata(\rvx|\rvy)$. We can therefore approximate samples from $\pdata$ by first drawing samples $\rvx_\sigma \sim \gN(\mathbf{0},\sigma^2\mI)$ for large $\sigma$ and then, roughly speaking, nudging them to keep approximating $p_\sigma(\cdot|\rvy)$ as $\sigma$ is reduced. When $\sigma$ reaches zero, we have our final samples from $p_\theta(\rvx|\rvy)\approx\pdata(\rvx|\rvy)$. This process can be formalized as either a continuous-time stochastic differential equation~\cite{song2020score} or as a discrete-time process~\cite{ho2020denoising}. In either case, the mechanism to ``nudge'' the samples uses a neural network $\hat{\rvx}_\theta$ trained to minimize the ``reconstruction'' loss
% \begin{align}\label{eq:diffusion-loss}
%     % \mathcal{L}(\theta) = 
%     \mathbb{E}_{u(\sigma)p_\sigma(\rvx,\rvx_\sigma,\rvy)} \left[ \lambda(\sigma) \lvert\lvert \rvx - \hat{\rvx}_\theta(\rvx_\sigma, \rvy, \sigma) \rvert\rvert^2 \right]
% \end{align}
% where the distribution $u(\sigma)$ and weighting function $\lambda(\sigma)$ are hyperparameters~\cite{ho2020denoising,song2020score,karras2022elucidating,tashiro2021csdi}. \Cref{eq:diffusion-loss} is sometimes written so that the $\rvx_\theta$ is trained to predict $\frac{\rvx_\sigma-\rvx}{\sigma}$, but these perspectives are identical given an appropriate transformation of $\lambda$ and a linear transformation of the network output. Also note that for simplicity we have used the ``variance-exploding'' definition of $p_\sigma$; the other commonly-used ``variance-preserving'' process is similar but scales $\rvx_\sigma$ by a $\sigma$-dependent constant.


% Let $p_\sigma(\rvx_\sigma|\rvy)$ be the distribution obtained by sampling $\rvx_0 \sim \pdata(\cdot|\rvy)$ and setting $\rvx_\sigma := \rvx_0+\sigma\epsilon$ with $\epsilon \sim \gN(\mathbf{0},\mI)$. For large $\sigma$, $p_\sigma(\rvx_\sigma|\rvy)$ is smooth and well-approximated by $\gN(\mathbf{0}, \sigma^2\mI)$. As $\sigma$ tends to zero, $p_\sigma$ becomes the $\pdata$. We can therefore approximate $\pdata$ by first drawing samples $\rvx_\sigma \sim \gN(\mathbf{0},\sigma^2\mI)$ for large $\sigma$ and then, roughly speaking, nudging them to stay within $p_\sigma(\cdot|\rvy)$ as $\sigma$ is reduced. When $\sigma$ reaches zero, we have our final samples from $p_\theta(\rvx_\rvy)\approx\pdata(\rvx|\rvy)$. This ca+n be formalized as either a continuous-time stochastic differential equation~\cite{} or as a discrete-time process~\cite{}.


% When $\sigma$ is sufficiently large, the distribution of corrupted data $\rvx+\sigma\epsilon$ is smooth and approximates $\gN(\mathbf{0}, \sigma^2\mI)$. As $\sigma$ is reduced, this distribution over $\rvx+\sigma\epsilon$ becomes progressively less smooth and eventually matches $\pdata(\rvx)$ when $\sigma$ is zero. This is the intuition behind procedures to sample from a diffusion model, which begin with a sample from a zero-mean Gaussian.

% %Sampling from $p_\theta(\rvx|\rvy)$ after training involves an iterative process in which Gaussian noise is first used to approximate $\rvx+\sigma\eps$ for some large $\sigma$

% initialize $\rvx$ with Gaussian noise and then perform updates using $\hat{\rvx}_\theta$ to gradually reduce the signal-to-noise ratio to zero. Depending on the problem formulation, 

% conditional diffusion model represents the distribution $p_\theta(\rvx|\rvy)\approx\pdata(\rvx|\rvy)$ over a variable $\rvx$ (e.g. an image) given a set of conditioning inputs $\rvy$ (which can e.g. contain a class-label for class-conditional generation or simply be the empty set for unconditional generation). For simplicity we describe here a continuous variance-exploding diffusion process. It is defined via a family of probability distributions at various noise levels $\sigma$ which we denote $q_\sigma(\rvx_\sigma, \rvy)$. These interpolate between the data distribution when $\sigma=0$ and an approximation of a high variance Gaussian when $\sigma$ is large. Concretely, let
% \begin{equation}
%     q_\sigma(\rvx_\sigma,\rvy) = \int \mathcal{N}(\rvx_\sigma|\rvx_0,\sigma^2) \pdata{}(\rvx_0, \rvy) \mathrm{d}\rvx_0
% \end{equation}
% where $\gN(\rvx_\sigma|\rvx_0,\sigma^2)$ is a Gaussian with variance $\sigma^2$ and mean $\rvx_0$. Intuitively, $q_\sigma(\rvx_\sigma,\rvy)$ is sampled from by sampling $\rvx_0$ and $\rvy$ from the data distribution and then corrupting $\rvx_0$ with Gaussian noise of variance $\sigma^2$. No noise is added to $\rvy$.

% \citet{} show that we can draw approximate samples from $q_0$ as long as we can approximate the score function $\nabla_{\rvx_\sigma}q_\sigma(\rvx_\sigma, \rvy)$ at all $\sigma$. Learning the score function is possible with a mean-squared error loss:
% \begin{equation}
%     \mathcal{L}(\theta) = \mathbb{E}_{u(\sigma)q(\rvx_\sigma,\rvx_0,\rvy)} \left[ \lvert\lvert \rvs_\theta(\rvx_\sigma) - \nabla_{\rvx_\sigma}q_\sigma(\rvx_\sigma, \rvy) \rvert\rvert^2 \right]
% \end{equation}
% where $u(\sigma)$ is a distribution defining how often each $\sigma$ should be seen in training. This mean-squared error can be computed down to a constant term.

% Assuming that there exists a dataset of $(\rvx,\rvy)$ pairs sampled from the data distribution $\pdata{}(\rvx,\rvy)$, the conditional diffusion model is fit to the conditional distribution $\pdata{}(\rvx|\rvy)$.

\section{Conditional vs. unconditional DGMs}
\paragraph{What does it mean to say that conditional DGMs beat unconditional DGMs?} A standard procedure to evaluate unconditional DGMs is to start by sampling a set of $N$ images independently from the model: ${\rvx^{(1)},\ldots,\rvx^{(N)} \sim p_\theta(\cdot)}$. We can then compute the Fr\'echet Inception distance (FID)~\citep{heusel2017gans} between this set and the dataset. If the generative model matches the data distribution well, the FID will be low.
%
For conditional DGMs the standard procedure has one extra step: we first independently sample ${\rvy^{(1)},\ldots,\rvy^{(N)} \sim \pdata(\cdot)}$. We then sample each image given the corresponding $\rvy^{(i)}$ as ${\rvx^{(i)} \sim p_\theta(\cdot|\rvy^{(i)})}$. 
%
Then, as in the unconditional case, we compute the FID between the set of images $\rvx_1,\ldots,\rvx_N$ and the dataset, without reference to $\rvy_1,\ldots,\rvy_N$. Even though it does not measure alignment between $\rvx, \rvy$ pairs, conditional DGMs beat comparable unconditional DGMs on this metric in many settings: class-conditional CIFAR-10 generation~\citep{karras2022elucidating}, segmentation-conditional generation~\cite{hu2022self}, or bounding box-conditional generation~\citep{hu2022self}.

% \citet{bao2022conditional} hypothesize that the gains from conditioning are because ``the data distribution conditioned on a specific class has fewer modes and is easier to fit than the original data distribution''. \citet{hu2022self} found, however, that performance could deteriorate again when conditioning on ``too much'' information. We present results in \cref{fig:fid-vs-ncomp} to shed some further light on when conditioning is helpful. We vary the number of PCA components of am embedding vector which can be conditioned on between zero (meaning the model is unconditional), 16, and 512 (the full vector). For small training budgets, it is best to condition on only 16 components and conditioning on more is actively harmful. For larger training budgets, it is best to condition on as many components as possible. We hypothesize that conditioning on something very informative slows down training (for reasons that are easier to explain if we add conditioning on clusters as well - TODO).

\begin{figure*}[ht]
    \centering
    \includegraphics[width=\textwidth]{figs/cond-results.pdf}
    \vspace{-5mm}
    \caption{FID throughout training. We show results for each method trained from scratch and, on AFHQ and FFHQ, for finetuning a pretrained EDM model (which was trained for the equivalent of 32 GPU days). VCDM quickly outperforms EDM when trained from scratch and quickly improves on the pretrained model when used for finetuning.}
    \vspace{-2mm}
    \label{fig:fid_vs_training}
\end{figure*}

\paragraph{Why do conditional DGMs beat unconditional DGMs?}
% Practitioners often observe that conditional image diffusion models produce significantly more realistic images than their unconditional counterparts~\citep{meng2022distillation}. This has been observed for both simple $\rvy$, \eg a class label, and more complex $\rvy$, \eg a text caption.

Conditional DGMS ``see'' more data during training than their unconditional counterparts because updates involves $\rvy$ as well as $\rvx$. \citet{bao2022conditional,hu2022self} prove that this is not the sole reason for their successes because the effect holds up even when $\rvy$ is derived from an unconditional dataset through self-supervised learning.
%
To our knowledge, the best explanation for their success is, as stated by \citet{bao2022conditional}, that conditional distributions typically have ``fewer modes and [are] easier to fit than the original data distribution.''

\paragraph{When do conditional DGMs beat unconditional DGMS?}
% Conditional DGMs performance can deteriorate when conditioning on ``too much'' information~\citet{hu2022self}. 
%
We present results in \cref{fig:fid-vs-ncomp} to answer this question. We show FID scores for conditional DGMs trained to condition on embeddings of varying information content. 
%
We produce $\rvy$ by starting from the CLIP embedding of each image in our dataset and using either principal component analysis to reduce their dimensionality (left two panels) or K-means clustering to discretize them (right two panels)~\citep{hu2022self}.
%$d$-dimensional embedding as the first $d$ principal components of a CLIP embedding. 
%
We see that, given a small training budget, it is best to condition on little information. With a larger training budget, performance appears to improve steadily as the dimensionality of $\mathbf{y}$ is expanded. We hypothesize that \textbf{(1)} conditioning on higher-dimensional $\mathbf{y}$ slows down training because it means that points close to any given value of $\mathbf{y}$ will be seen less frequently and $\textbf{(2)}$ with a large enough compute budget, any $\mathbf{y}$ correlated with $\mathbf{x}$ will be useful to condition on. This suggests that, as compute budgets grow, making unconditional DGM performance match conditional DGM performance will be increasingly useful.
%
% We train conditional models on AFHQ and FFHQ vary the number of PCA components of am embedding vector which can be conditioned on between zero (meaning the model is unconditional), 16, and 512 (the full vector). For small training budgets, it is best to condition on only 16 components and conditioning on more is actively harmful. For larger training budgets, it is best to condition on as many components as possible. We hypothesize that conditioning on something very informative slows down training (for reasons that are easier to explain if we add conditioning on clusters as well - TODO).

% In order to leverage this phenomenon in the setting where we do not wish to specify a CLIP embedding to condition each image on, we introduce an \textit{auxiliary model} of the distribution of CLIP embeddings. 


\section{Method} \label{sec:method}
We have established that conditioning on CLIP embeddings improves DGMs. We now introduce VCDM which leverages this phenomenon to benefit the unconditional setting (in which the user does not wish to specify any input to condition on) and the ``lightly-conditional'' setting in which the input is low-dimensional, e.g. a class-label. We will denote any such additional input $\rva$ (letting $\rva$ be a null variable in the unconditional setting) and from now on always use $\rvy := e_i(\rvx)$ to refer to a CLIP embedding. VCDM approximates the target distribution $\pdata(\rvx|\rva)$ as
% and so denote the auxiliary model $p_\phi(\rvy)$.  Doing so lets us model the marginal distribution over images as
%
% We establish in \cref{fig:fid-vs-ncomp} that conditioning on auxiliary variables in the form of CLIP embeddings improves DGMs. %
% We now introduce VCDM which leverages this phenomenon to benefit the unconditional setting (in which $\rvy$ is a null variable) and the ``lightly-conditional'' setting in which $\rvy$ is low-dimensional.
%
% VCDM uses an auxiliary variable defined as an image's CLIP embedding, $\rva := e_i(\rvx)$, to approximate the target distribution $\pdata(\rvx|\rvy)$ as
% \begin{align} \label{eq:no-a}
%     p_{\theta,\phi}(\rvx|\rva) &= \mathbb{E}_{p_\phi(\rvy|\rva)} \left[ p_\theta(\rvx|\rvy,\rva) \right] \\
%      &\approx \mathbb{E}_{\pdata(\rvy|\rva)} \left[ \pdata(\rvx|\rvy,\rva) \right] = \pdata(\rvx|\rva)
% \end{align}
\begin{align} \label{eq:no-a}
    \pdata(\rvx|\rva) &= \mathbb{E}_{\pdata(\rvy|\rva)} \left[ \pdata(\rvx|\rvy,\rva) \right] \\
    &\approx \mathbb{E}_{p_\phi(\rvy|\rva)} \left[ p_\theta(\rvx|\rvy,\rva) \right] % = p_{\theta,\phi}(\rvx|\rva)
\end{align}
where $p_\phi(\rvy|\rva)$ is a second DGM modeling the CLIP embeddings. We can sample from this distribution by sampling $\rvy\sim p_\phi(\cdot|\rva)$ and then leveraging the conditional image DGM to sample $\rvx \sim p_\theta(\cdot|\rvy,\rva)$ before discarding $\rvy$.
%
% \begin{align}
%     \pdata(\rvx|\rvy) &= \mathbb{E}_{\pdata(\rva|\rvy)} \left[ \pdata(\rvx|\rva,\rvy) \right] \\
%     &\approx \mathbb{E}_{p_\phi(\rva|\rvy)} \left[ p_\theta(\rvx|\rva,\rvy) \right]  \label{eq:combined-dist}
% \end{align}
%where $p_\phi$ is a second DGM modeling the auxiliary variables. 
% Conditioning on CLIP embeddings can be useful even if we are also conditioning on other inputs like class-labels and so we now generalize \cref{eq:no-a} slightly. Denoting these inputs $\rva$, we can incorporate them with
% \begin{align} \label{eq:with-a}
%     \mathbb{E}_{p_\phi(\rvy|\rva)} \left[ p_\theta(\rvx|\rvy,\rva) \right] \approx \pdata(\rvx|\rva)
% \end{align}
% We can sample from this distribution by sampling $\rvy\sim p_\phi(\rvy)$ and then leveraging the conditional DGM $p_\theta(\rvx|\rvy)$ to sample the image given $\rvy$.
From now on we will call $p_\theta(\rvx|\rvy,\rva)$ the \textit{conditional image model} and $p_\phi(\rvy|\rva)$ the \textit{auxiliary model}. In our experiments the auxiliary model uses a small architecture relative to the conditional image model and so adds little extra cost.

\paragraph{Auxiliary model}
Our auxiliary model is a conditional DGM targeting $\pdata(\rvy|\rva)$, where $\rvy$ is a 512-dimensional CLIP embedding. We follow the architectural choice of \citet{ramesh2022hierarchical} and use a DGM with a transformer architecture. It takes as input a series of 512-dimensional input tokens: an embedding of $\sigma$; an embedding of $\rva$ if this is not null; an embedding of $\rva_\sigma$; and a learned query. These are passed through six transformer layers and then the output corresponding to the learned query token is used as the output. Like \citet{ramesh2022hierarchical}, we parameterize the DGM to output an estimate of the denoised $\rva$ instead of estimating the added noise as is more common in the diffusion literature.
%
On AFHQ and FFHQ we find that data augmentation is helpful to prevent the auxiliary model overfitting. We perform augmentations (including rotation, flipping and color jitter) in image space and feed the augmented image through $e_i(\cdot)$ to obtain an augmented CLIP embedding. Following \citet{karras2022elucidating}, we pass a label describing the augmentation into the transformer as an additional input token so that we can condition on there being no augmentation at test-time.

\paragraph{Conditional image model}
% We use the same conditional DGM architectures as \citet{karras2022elucidating} on each dataset. 
Our diffusion process hyperparameters and samplers build on those of \citet{karras2022elucidating}.  For AFHQ and FFHQ, we use the U-Net architecture originally proposed by \citet{song2020score}. For ImageNet, we use the slightly larger U-Net architecture originally proposed by \citet{dhariwal2021diffusion}. We match the data augmentation scheme to be the same as that of \citet{karras2022elucidating} on each dataset. There are established conditional variants of both architectures we use~\citep{dhariwal2021diffusion,karras2022elucidating}, both of which incorporate $\rvy$ via a learned linear projection that is added to the embedding of the noise standard deviation $\sigma$. Our conditional image model needs to additionally incorporate $\rva$; we can do so by simply concatenating it to $\rvy$ and learning a projection for the resulting vector.

% In order to ease adoption of our method we show that it is possible to train $p_\theta(\rvx|\rva,\rvy)$ starting from a pretrained $p_\theta(\rvx|\rvy)$.

% In addition to training a conditional image model from scratch, we consider adapting an existing model to condition on $\rva$. To do so, we simply add an additional layer which projects $\rva$ onto the timestep embedding and initialize its weights to zero.

\section{Experiments}
We experiment on three datasets: AFHQ~\citep{choi2020stargan}, FFHQ~\citep{karras2018style} and ImageNet~\citep{deng2009imagenet}, all at $64\times64$ resolution. We target unconditional generation for AFHQ and FFHQ, and class-conditional generation for ImageNet. As well as training networks from scratch on each dataset, we report results with the model checkpoints released by \citet{karras2022elucidating} on AFHQ and FFHQ, which we finetune to be conditional on the CLIP embedding. To do so, we simply add a learnable linear projection of the CLIP embedding and initialize its weights to zero.
%
\Cref{fig:fid_vs_training} reports the FID on each setting and dataset throughout the training of the conditional image model.\footnote{Each FID is estimated using $20\,000$ images, each sampled with the SDE solver proposed by \citet{karras2022elucidating} using 40 steps, $S_\text{churn}=50$, $S_\text{noise}=1.007$, and other parameters set to their default values.} 
%
In each case, the auxiliary model is trained for one day on one V100 GPU. We compare VCDM to three other approaches: \textbf{EDM}~\citep{karras2018style} is a standard DGM directly modeling $\pdata(\rvx|\rva)$. \textbf{VCDM with oracle} uses our conditional image model but uses the ground-truth $\rvy$ for each test $\rva$ instead of sampling from the learned auxiliary model, i.e. it is the performance that VCDM would achieve with a perfect auxiliary model. \textbf{Class-cond} is an ablation of VCDM that applies to unconditional tasks where $\rva$ is null. It uses discrete $\rvy$ (as on the right of \cref{fig:fid-vs-ncomp}) so that $\pdata(\rvy|\rva)=\pdata(\rvy)$ is a simple categorical distribution which we can sample from exactly, but we see that it is outperformed by VCDM.

VCDM consistently outperforms unconditional generation after 1-2 GPU-days and this performance gap continues for as long as we train the networks. Comparing VCDM's performance with and without the oracle we see that they are close. For networks trained from scratch we show in \cref{tab:results} that VCDM always has an improvement over EDM at least $80\%$ as large as that of VCDM with an oracle, indicating that $p_\phi(\rvy|\rva)$ is a good approximation of $\pdata(\rvy|\rva)$. We can therefore leverage almost the full power of conditional DGMs for unconditional sampling. 


\begin{table} \label{tab:results}
\centering
\caption{Final FID score for the models we train from scratch and a comparison of their improvements over EDM.}
\begin{tabular}{lccc}
\midrule
Dataset & AFHQ & FFHQ & ImageNet \\
\midrule
$\mathbf{y}$ & null & null & class label \\
\midrule
VCDM & $1.83$ & $4.73$ & $18.1$ \\
VCDM + oracle & $1.57$ & $4.35$ & $19.7$ \\
Class-cond. & $2.56$ & $5.24$ & - \\
EDM & $3.53$ & $6.39$ & $26.5$ \\
\midrule
\begin{tabular}{@{}l@{}}Improv. w/ VCDM\end{tabular} & $48.2\%$ & $26.0\%$ & $31.5\%$ \\
\begin{tabular}{@{}l@{}}Improv. w/ oracle\end{tabular} & $55.6\%$ & $31.9\%$ & $25.6\%$ \\
\midrule
%\begin{tabular}{@{}l@{}}$\%$ improvement \\ captured by VCDM\end{tabular} & $86.6\%$ & $81.3\%$ & - \\
$\frac{\text{Improv. w/ VCDM}}{\text{Improv. w/ oracle}}$ & $86.6\%$ & $81.3\%$ & $123\%$ \\
\end{tabular}
\end{table}



% \begin{table}
% \centering
% \begin{tabular}{cccc}
% & AFHQ & FFHQ & ImageNet \\
% \hline
% $\mathbf{y}$ & null & null & class label \\
% \end{tabular}
% \end{table}

% We compare samples drawn with three methods. The one we denote \textbf{baseline} simply samples images from a learned model $p_\theta(\rvx|\rvy)$ without considering $\rva$. We denote our approach as described in \cref{sec:method} chain-of-thought, or \textbf{CoT}. The \textbf{oracle} approach is an upper-bound on CoT's performance obtained by sampling the auxiliary variables from the dataset at test-time instead of using the learned auxiliary model. This is possible for unconditional and class-conditional generation but would become infeasible if $\rvy$ was more complex and we wished to extrapolate to unseen values of $\rvy$ at test time.  We also compare against an ablation \textbf{discrete aux} with discretised $\rva$, since using $\rva$ allow us to use a simple classifier as the auxiliary model instead of a diffusion model.


\section{Related work}
Several existing image generative models leverage CLIP embeddings for better text-conditional generation~\citep{nichol2021glide,ramesh2022hierarchical}. We differ by suggesting that CLIP embeddings are not only useful for text-conditioning, but also as a general tool to improve the realism of generated images. We demonstrate this for unconditional and class-conditional generation. Our work takes inspiration from \citet{weilbach2022graphically}, % who use diffusion models to perform approximate inference in graphical models. They 
who show improved performance in various approximate inference settings by modeling problem-specific auxiliary variables (like $\rvy$) in addition to the variables of interest ($\rvx$) and observed variables ($\rva$). We apply these techniques to the image domain and incorporate pretrained CLIP embedders to obtain auxiliary variables. 
%
VCDM also relates to methods which perform diffusion in a learned latent space~\citep{rombach2022high}: our auxiliary model $p_\phi(\rvy|\rva)$ is analogous to a ``prior'' in a latent space and our conditional image model $p_\theta(\rvx|\rva,\rvy)$ to a ``decoder'' Such methods typically use a near-deterministic decoder and so their latent variables must summarize all information about the image. Our conditional DGM decoder on the other hand will function reasonably however little information is stored in $\rvy$ and so VCDM provides an additional degree of freedom in terms of what to store. This is an interesting design space for future exploration. Classifier~\citep{song2020score} and classifier-free guidance~\citep{ho2022classifier} are two alternative methods for conditional sampling from DGMs. Both have a ``guidance strength'' hyperparameter to trade fidelity to $\pdata(\rvx|\rvy)$ against measures of alignment between $\rvx$ and $\rvy$. A possible extension to VCDM could parameterize $p_\theta(\rvx|\rvy,\rva)$ with either of them.

% is sometimes used to improve conditional sampling from diffusion models. It takes a ``guidance strength'' parameter and reduces to the conditional DGM described above if this is set to zero. Setting a positive guidance strength typically leads to samples of $\rvx$ that are more obviously related to $\rvy$ but that do not necessarily better fit the data distribution as measured by FID scores (see e.g. Table ... of \citet{meng2022distillation}). In this paper we focus on matching the data distribution and so do not experiment with classifier-free guidance. It is likely that our contributions could be later combined with classifier-free guidance to improve more subjective measures of sample quality.




% \citet{bao2022conditional,hu2022self} both use self-supervised learning to obtain auxiliary variables and then training a diffusion model $p(\rvx|\rva)$. However, they do not model $\rva$ and therefore are not able to sample $\rvx$ without an oracle that can provide $\rva$. Their success when given an oracle, however, provides reason to believe that our approach is likely to yield benefits even if the embedder that produces $\rva$ is obtained through self-supervised learning and without access to additional data as our CLIP embedder had.

% \citet{deja2023learning} relate unconditional image models with additional data with an architecture that allows weight-sharing between an image classifier and a diffusion model. They do not use this to improve performance for unconditional image generation.

% Concretely, we add a first stage which samples a CLIP embedding given $\rvy$, before sampling $\rvx$ given both the CLIP embedding and $\rvy$, and finally discarding the sampled CLIP embedding and returning $\rvx$. Denoting the CLIP embedding as $\rva$, we implement this procedure using the two following steps:
% \begin{enumerate}
%     \item We modify a pretrained diffusion model $p_\theta(\rvx|\rvy)$ to additionally condition on CLIP embeddings, yielding an augmented model $p_\theta(\rvx|\rvy,\rva)$. 
%     \item Train a separate diffusion model of CLIP embeddings, $p_\theta(\rva|\rvy)$.
% \end{enumerate}
% We detail each step.


\section{Discussion and conclusion}
We have presented VCDM, a method for unconditional or lightly-conditional image generation which harnesses the impressive performance of conditional DGMs. A massive unexplored design space remains: there are almost certainly more useful quantities that we could condition on than CLIP embeddings. It may also help to condition on multiple quantities, or ``chain'' a series of conditional DGMs together. An alternative direction is to simplify VCDM's architecture by, for example, learning a single diffusion model over the joint space of $\rvx$ and $\rvy$ instead of generating them sequentially. A drawback of VCDM is that it relies on the availability of a pretrained CLIP embedder. While this is freely available for natural images, it could be a barrier to other applications; an alternative would be to explore the self-supervised representations used by \citet{bao2022conditional,hu2022self}.


%%%%%%%%% REFERENCES
{\small
\bibliographystyle{ieee_fullname_natbib}
\bibliography{references}
}

\end{document}
