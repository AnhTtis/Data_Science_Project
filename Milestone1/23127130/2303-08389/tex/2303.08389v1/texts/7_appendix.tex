%\section{Related works}
%\input{resource/Relatedworks}

\section{Pre-training Details}
Like \citet{multilingualCLIP2022}'s method, our multilingual CLIP is trained by pre-training through teacher learning using MSE loss as shown on the left of the Figure \ref{fig2}. The datasets used for pre-training are GCC \cite{wang2019learning}, VizWiz, and MSCOCO with total 2.2M sentences. Each English sentence is translated into German, Spanish, French, and Japanese using MBART-50. The pre-trained CLIP model used as the teacher model is the RN50X4 model, and the Distill-Multilingual BERT \cite{sanh2019distilbert} is used as the student text encoder. The model is trained in a total of 5 epochs, and it takes about 20 hours per epoch with our computing power . 
\label{pretraining_details}

\section{Experimental Details}
\subsection{Reproductabilty checklists} 
\noindent \textbf{Dataset and Source code} We provide our pre-training, fine-tuning, and evaluation source code along with configuration code for perturbations as supplementary materials.
We will publicly release our dataset~\dataset, and the full codes with weight parameters. \\

\noindent \textbf{Computing Resources} AMD Ryzen Threadripper 2950X (3.50 GHz) with GeForce GTX 2080 Ti is used for the experiments. All codes are implemented on Python 3.6.15 and PyTorch 1.7.1. The fine-tuning of each model trains 5 epochs, and takes about 6 hours per epoch.\\

\noindent \textbf{Number of Parameters} The number of parameter of our multilingual CLIP is about 66M as like as Distill-Multilingual BERT. \\

\noindent \textbf{Train-Valid-Test split} MSCOCO used for fine-tuning consists of 414k training set and 25k validation set. We split the training set by 9:1 and used it for fine-tuning and validation. We also randomly extracted 3k samples from the existing validation set and used it as a test set.

\subsection{Hyper-parameters}
\noindent \textbf{Hyper-parameters for fine-tuning} In order to find the best-performing model, we conducted an experiment on 16 hyper-parameter combinations ($\lambda_{1}:0\sim 0.5, \lambda_{2}:0\sim0.1, \lambda_{3}:0\sim0.1$). 
The hyper-parameter was manually tuned based on the effective detection of lexical noise while maintaining high human correlation, and finally, the best-performing $\lambda$ values of the objective function for fine-tuning are as follows:
$\lambda_{1}=0.1, \lambda_{2}=0.05, \lambda_{3}=0.05$. \\ 

\noindent \textbf{Hyper-parameters for optimizer} We use AdamW \cite{loshchilov2017decoupled} optimizer with $\beta_{1}=0.9,\beta_{2}=0.999, \epsilon=1e-8$. The initial learning rate is $5e-5$.

\section{\dataset eval set examples}
The examples of the \dataset eval set for languages other than English can be seen in Figure~\ref{fig7}.
\begin{figure}[h]
\centering
\includegraphics[width=1.0\columnwidth]{resource/ex1-4.pdf} 
\caption{\dataset~eval set examples for each languages.}
\label{fig7}
\end{figure}

% \section{Eval set Examples}

% \begin{figure}[h]
% \centering
% \includegraphics[width=1.0\columnwidth]{resource/ex1-4.pdf} 
% \caption{\dataset~eval set examples for each languages.}
% \label{fig5}
% \end{figure}
% \clearpage

\section{Perturbed caption examples}
The examples of the perturbed captions for languages other than English can be seen in Figure~\ref{fig8}-Figure~\ref{fig11}.
The critical objects shuffled for in-sentence substitution perturbation are displayed using each color.

\section{Implementation Details}
In Alg.~\ref{alg1}, we show the Python implementation of each perturbation criterion: \textit{"Repetition"}, \textit{"Removal"}, \textit{"Masking"}, \textit{"Jumble"}, and \textit{"Substitution"}.

\section{All results tables}
\label{appendix:all_results_table}

\noindent \textbf{MSCOCO}
The results for all perturbation of all languages for MSCOCO 3k eval set can be found in Table \ref{table4}.\\
\noindent \textbf{Flickr8k}
The results for all perturbation of all languages for Flickr8k eval set can be found in Table \ref{table5}.\\
\noindent \textbf{VizWiz}
The results for all perturbation of all languages for Vizwiz eval set can be found in Table \ref{table6}.\\
\noindent \textbf{\dataset}
The results for all perturbation of all languages for~\dataset eval set can be found in Table \ref{table7}.

\begin{algorithm*}
\caption{Python implementation of perturbation}\label{alg1} 
\vspace{2mm}

\lstset{language=Python}
\lstset{label={lst:code_direct}}
\lstset{basicstyle=\footnotesize}
\begin{lstlisting}
def rep_rem_mask(caption_list): # Repetition, Removal, and Masking
    caption_rp = []
    caption_rm = []
    caption_rm_mask = []

    for i in range(len(caption_list)):
        words = caption_list[i].split()
        substitued_rp = []
        substitued_rm = []
        substitued_rm_mask = []
        substitued_rmrp = []
        for j in range(len(words)):
            substitued_rp.append(words[j])
            substitued_rm_mask.append(words[j])
            if random.random() > threshold:
                substitued_rp.append(words[j])
                substitued_rm.append(words[j])
                substitued_rm_mask[-1] = '[MASK]'
            elif random.random() > threshold:
                substitued_rmrp.append(words[j])  
        caption_rp.append(" ".join(substitued_rp))
        caption_rm.append(" ".join(substitued_rm))
        caption_rm_mask.append(" ".join(substitued_rm_mask))
        
    return caption_rp, caption_rm, caption_rm_mask

def jumble(caption_list): # Jumble
    caption_jumble = []
    for i in range(len(caption_list)):
        words = caption_list[i].split()
        random.shuffle(words)
        caption_jumble.append(" ".join(words))
    return caption_jumble
    
def sub_in_sent(caption_list, critical_obj_list): # In-sentence substitution
    caption_sub_in = []
    for i in range(len(caption_list)):
        current_caption = caption_list[i]
        current_critical_obj_list = critical_obj_list[i]
        shuffled = current_critical_obj_list.copy()
        words = caption_list[i].split()
        
        if len(current_critical_obj_list) < 2:
            caption_sub_in.append(" ".join(words))
        else :
            while current_critical_obj_list == shuffled:
                random.shuffle(shuffled)
            
            target = current_caption
            for j in range(len(shuffled)):
                target = shuffled[j].join(target.rsplit(current_critical_obj_list[j],1))
            caption_sub_in.append(target)
            
    return caption_sub_in

\end{lstlisting}




\end{algorithm*}

\clearpage

\begin{figure}[t]
\centering
\includegraphics[width=1.0\columnwidth]{resource/evalset1.pdf} 
\caption{Eval set perturbed captions example (FR).}
\label{fig8}
\end{figure}

\begin{figure}[t]
\centering
\includegraphics[width=1.0\columnwidth]{resource/evalset2.pdf} 
\caption{Eval set perturbed captions example (DE).}
\label{fig9}
\end{figure}

\begin{figure}[t]
\centering
\includegraphics[width=1.0\columnwidth]{resource/evalset3.pdf} 
\caption{Eval set perturbed captions example (JA).}
\label{fig10}
\end{figure}


\begin{figure}[t]
\centering
\includegraphics[width=1.0\columnwidth]{resource/evalset4.pdf} 
\caption{Eval set perturbed captions example (ES).}
\label{fig11}
\end{figure}

\clearpage

\begin{table*}[h]
\tabcolsep7.5pt
\caption{Binding energy (BE) of carbon-chain species}
\label{tab:Ebind}
\begin{center}
\begin{tabular}{cc|cc|cc}
\hline
Species & BE (K) & Species& BE (K) & Species& BE (K) \\
\hline
%HC$_3$N &\\
%HC$_3$N &\\
C$_2$& 10000& C$_8$N&7200&C$_3$O&2750 \\
%C$_{2}$&10000$^{b}$\\
C$_3$&2500&C$_9$N&8000&C$_5$O&4350 \\
%C$_{3}$&2500$^{b}$&&\\
C$_{4}$&3200&C$_{10}$N&8800 & C$_7$O&5950  \\ 
C$_5$&4000&C$_2$H$_2$&2587 &C$_9$O&7550 \\
C$_6$&4800&C$_2$H$_4$&2500 & HC$_2$O&2400 \\
C$_7$&5600&C$_2$H$_5$&3100 &SiC$_2$&4300 \\
%C$_{5}$&4000$^{b}$&&\\
%C$_{6}$&4800$^{b}$&&\\
%C$_{7}$&5600$^{b}$&&\\
C$_{8}$&6400&C$_2$H$_6$&1600 &SiC$_3$&5100 \\
C$_{9}$&7200&C$_4$H$_2$&4187& SiC$_4$&5900 \\
C$_{10}$&8000&C$_5$H$_2$&4987 \\
C$_{11}$&9600&C$_6$H$_2$&5787\\
C$_2$H&3000&C$_7$H$_2$&6587\\
$l$-C$_3$H&4000&C$_2$P&4300 & \\
$c$-C$_3$H&5200&C$_3$P&5900\\
C$_4$H&3737&C$_4$P&7500\\
C$_5$H&4537&C$_2$S&2700\\
C$_6$H&5337&C$_3$S&3500\\
C$_7$H&6137&C$_4$S&4300\\
C$_8$H&6937&HC$_3$N&4580\\
$c$-$\rm{C_3H_2}$&5900&HC$_4$N&5380\\
C$_2$N&2400&HC$_5$N&6180\\
C$_3$N&3200&HC$_6$N&7780\\
C$_4$N&4000&HC$_7$N&7780\\
C$_5$N&4800&HC$_8$N&9380\\
C$_6$N&5600&HC$_9$N&9380\\
C$_7$N&6400&C$_2$O&1950\\

%HC$_3$O&3111$^{b}$

%C$_7$N&6400$^{a}$&H$_2$C$_3$N&3133$^{b}$\\
%C$_8$N&7200$^{b}$&\\
%C$_9$N&8000$^{b}$&\\
%C$_2$O&1950$^{b}$&&\\
%C$_3$O&4208$^{a}$&&\\
%C$_5$O&4350$^{b}$&&\\
%C$_7$O&5950$^{b}$&&\\
%C$_9$O&7550$^{b}$&&\\
% &&C$_2$S&2943$^{a}$\\
% &&C$_3$S&3500$^{b}$\\
% &&C$_4$S&4300$^{b}$\\
% &&HC$_3$N&3475$^{a}$\\
% &&HC$_4$N&5380$^{b}$\\
% &&HC$_5$N&6180$^{b}$\\
% &&HC$_6$N&7780$^{b}$\\
% &&HC$_7$N&7780(it should change)$^{b}$\\
% &&HC$_8$N&9380$^{b}$\\
% &&HC$_9$N&9380(it should change)$^{b}$\\
% &&HC$_2$O&2400$^{b}$\\
% &&HC$_3$O&3111$^{a}$\\
% &&SiC$_2$&4300$^{b}$\\
% &&SiC$_3$&5100$^{b}$\\
% &&SiC$_4$&5900$^{b}$\\
% %&&\\
\hline
\end{tabular}
\end{center}
%\begin{tabnote}
{Taken from the KIDA (\url{https://kida.astrochem-tools.org/}), and also see \citet{wake17}, \citet{pent17}, \citet{das18}.}
% higher-order-chain values are estimated with lower order chain plus one carbon atom's BE (KIDA has used 800 K for these estimations since updated binding energy of C atom is 1300 K, thus need to update those values. HC$_{2n+1}$ show different binding energy, however, we think the BE of HC$_4$N, HC$_6$N, and HC$_8$N needs to revisit. Especially for C$_n$P group, C$_n$+P is the rule for the estimation of binding energy. Also, include BE values of HC$_n$O ($n>4$) following a similar carbon addition method. Check some estimation if possible for recently detected higher-order carbon chains.
%\end{tabnote}
\end{table*}
\begin{table*}[tb]
\centering
\scalebox{0.95}{
\setlength{\tabcolsep}{0.8mm}
\begin{tabular}{c|c|ccccc|ccc}
\toprule[1pt]
Method & LA & AP(1$\times$) & AR(1$\times$) & AP(3$\times$) & AR(3$\times$) & NMS & forward(ms) & NMS(ms) & FPS \\
 \hline
FCOS~\cite{fcos} & o2m & 38.6 & 57.2 & 41.4 & 59.1 & \Checkmark & 27.7 & 0.7 & 35.2 \\
FCOS~\cite{fcos} & o2m & 17.7 & 52.9 & 19.1 & 54.2 & \XSolidBrush & 27.7 & 0 & 36.1 \\
POTO~\cite{poto} & o2o & 36.5 & 58.9 & 40.2 & 61.1 & \XSolidBrush & 27.7 & 0 & 36.1 \\
POTO+3DMF~\cite{poto} & o2o & 37.0 & 58.8 & 40.5 & 60.9 & \XSolidBrush & 30.3 & 0 & 33.0 \\
POTO+3DMF+Aux~\cite{poto} & o2o+o2m(ATSS) & 37.6 & 58.7 & 41.2 & 61.2 & \XSolidBrush & 30.3 & 0 & 33.0 \\
POTO+3DMF+Aux~\cite{poto} & o2o+o2m(Top-k) & 37.6 & 59.0 & 41.1 & 61.3 & \XSolidBrush & 30.3 & 0 & 33.0 \\
POTO+3DMF+Aux~\cite{poto} & o2o+o2m(FCOS) & 36.5 & 58.0 & 40.3 & 60.6 & \XSolidBrush & 30.3 & 0 & 33.0 \\
Ours-1conv & o2f & 38.4 & 60.5 & 41.6 & 63.3 & \XSolidBrush & 28.1 & 0 & 35.6 \\
Ours-2convs & o2f & 38.7 & 60.6 & 42.0 & 63.5 & \XSolidBrush & 28.2 & 0 & 35.5 \\
Ours-3convs & o2f & 39.0 & 61.2 & 42.2 & 63.5 & \XSolidBrush & 28.2 & 0 & 35.5 \\
%Ours-3convs+DCN & o2f & 39.4 & 61.8 & & & \XSolidBrush & 29.9 & 0 & 33.4 \\
\bottomrule[1pt]
\end{tabular}
}
\caption{Comparisons with state-of-the-art end-to-end dense detectors on COCO \texttt{val} set. All experiments are conducted with ResNet-50 backbone. `LA' means label assignment. `o2o' means one-to-one. `o2m' means one-to-many. `o2f' means one-to-few. `ATSS', `Top-k' and `FCOS' in the `LA' column are the different o2m label assignment strategies used in the auxiliary loss in POTO. The reported runtime (ms) are all evaluated on a Tesla-V100 GPU under the MMDetection toolbox.}
%\vspace{-2mm}
\label{table5}
\end{table*}
\begin{table}[t]
\caption{\textbf{(a)} Ablation on NeFeS architecture. \textbf{(b)} Ablation on the proposed training scheduling}
\label{table:6}
\centering
\resizebox{\linewidth}{!}{

\begin{tabular}{lc}
        \multicolumn{2}{c}{\textbf{(a) NeFeS Architecture Ablation}}\\
        \\
            \toprule
            Method                               & Shop Facade \\
            \midrule
            NeFeS (ours) & 0.14m/0.47\degree  \\
            - Exposure-adaptive ACT & 0.14m/1.20\degree  \\
            - Feature Fusion & 0.37m/1.62\degree\\
            \midrule
            NeRF RGB+CNN & 0.15m/0.87\degree\\
            \bottomrule
        \end{tabular}

\begin{tabular}{lc}
        \multicolumn{2}{c}{\textbf{(b) Training Scheduling Ablation}}\\
        \\
            \toprule
            Method              & Church \\
            \midrule
            Combined & 0.35m/1.13\degree  \\
            Progressive & 0.32m/0.99\degree\\
            \bottomrule
        \end{tabular}

}
\end{table}

\begin{table}[tb]
\centering
\scalebox{0.85}{
\setlength{\tabcolsep}{0.8mm}
\begin{tabular}{c|ccccc}
\toprule[1pt]
Method & AP(1$\times$) & AR(1$\times$) & AP(3$\times$) & AR(3$\times$) & NMS \\
 \hline
FCOS~\cite{fcos} & 43.6 & 61.1 & 46.8 &  63.7& \Checkmark \\
FCOS~\cite{fcos} & 20.7 & 57.2 & 23.0 & 59.6 & \XSolidBrush \\
POTO~\cite{poto} & 40.8 & 61.6 & 44.0 & 64.7 & \XSolidBrush \\
POTO+3DMF+Aux~\cite{poto} & 41.8 & 61.7 &  44.8 & 63.9 & \XSolidBrush \\
Ours & 42.9 & 63.7 & 46.1 & 66.7  & \XSolidBrush \\
\bottomrule[1pt]
\end{tabular}
}
\caption{The object detection results with Swin-T backbone.}
\vspace{-4mm}
\label{swin}
\end{table}
