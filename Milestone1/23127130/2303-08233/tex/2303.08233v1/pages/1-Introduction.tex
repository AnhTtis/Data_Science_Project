\section{Introduction}
% 2017 version right now, need to update

Operations research (OR) tools can be leveraged to model and solve many real-world decision-making problems analytically and efficiently. OR is a field of applied mathematics that has been proven beneficial in many applications such as supply chain management \citep{supplychain}, production planning \citep{productionplanning}, bike-share ridership and efficiency in urban cities \citep{bikeshare, bicycleGuangzhou}, managing wastewater collection and treatment systems \citep{wastewater}, and finding a revenue-maximizing pricing strategy for businesses \citep{revenue}. Different types of optimization problems can be solved using standard optimization algorithms such as the simplex \citep{simplex} or interior-point method \citep{karmarkar}. However, modeling real-world problems into proper formulations as input to optimization solvers is still an iterative and strenuous process. First, the problem must be described by the stakeholder in the language of a domain expert. Then, an OR expert must extract the decision variables, objective, and the constraints from the description. Finally, the problem must be re-written in an algebraic modeling language that solvers can interpret.

Through the \textsc{NL4Opt Competition}, we investigate the feasibility of learning-based natural language interfaces for optimization solvers. To do so, we explored the practicality of partially automating the formulation of optimization problems. In particular, semantic parsing is a general task for extracting machine-interpretable meaning representations from natural language utterances. They have been well-studied for designing NLP systems that interact with database systems \citep{zhong2017seq2sql-custom, gan-etal-2020-review}, Unix machines \citep{lin-etal-2018-nl2bash}, knowledge base systems \citep{berant-liang-2014-semantic, dong-lapata-2016-language} or dialog systems \citep{guo-2018-custom}. However, extracting the formulation of optimization problems is still an under-explored problem. Meanwhile, solving math word problems with NLP has seen sustained research activity \citep{koncel-kedziorski-etal-2016-mawps, hopkins-etal-2019-semeval, miao-etal-2020-diverse, patel-etal-2021-nlp} with researchers focused on finding the correct answers to elementary algebraic and arithmetic problems. In contrast, rather than exploring methods of producing the solution to the problem, we focus on converting optimization problems into a form that can be passed to commercial optimization solvers to efficiently find optimal solutions. 

Lately, a few related challenges have been created for analyzing scientific texts. For instance, \cite{harper-etal-2021-semeval} proposed the MeasEval challenge focused on extracting counts and measurements from clinical documents and finding the attributes of those quantities. Another popular challenge was MultiCoNER \citep{multiconer} which focused on detecting semantically ambiguous and complex entities from documents written in 11 languages spanning 13 tracks. The \textsc{NL4Opt Competition} expands this task by not only detecting complex entities from optimization problems but also generating the equivalent mathematical formulation. 