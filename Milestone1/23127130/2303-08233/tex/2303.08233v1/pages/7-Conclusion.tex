\section{Conclusion}

We hosted the \textsc{NL4Opt Competition} at NeurIPS 2022 to draw attention towards the potential of machine learning in augmenting the user experience of OR tools. This competition presented two engaging tasks that successfully attracted many unique solutions. The two tasks (NER and generation) combine to take a linear programming word problem, tag its relevant entities, and generate a canonical representation that can be easily converted into a format that optimization solvers can interpret. 

To summarize, many winning teams of sub-task 1 reported a significant improvement in performance when leveraging ensemble learning and various augmentation techniques. Winning teams of sub-task 2 reported the main contributor for improved performances resulted from redesigning the input prompt. These solutions improved upon the baseline (up to 3.3\% for sub-task 1 and 29\% for sub-task 2) and will be explored for their use in making commercial solvers more accessible to non-experts by accepting natural language problem descriptions. ChatGPT achieved a 2.8\% improvement over the top-performing submission for sub-task 2 without the need for intermediate entity tagging. Future research should investigate the generalizability of large language models and the potential benefits of fine-tuning them for specific applications. 

In addition to the impact of providing an alternative input format to solvers, the labelled dataset from this competition has been released and may be used to evaluate methods for multi-sentence inputs, low-resource learning (eg. zero-shot/few-shot learning), and generalizability to unseen domains. We encourage and look forward to continual applications of the open-sourced dataset and the subsequent exciting new research interests that may stem from the solutions of the \textsc{NL4Opt Competition}.