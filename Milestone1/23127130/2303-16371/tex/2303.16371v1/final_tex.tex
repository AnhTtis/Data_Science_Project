\documentclass[11pt]{article}
\usepackage{amsmath}
\usepackage{amsfonts}
\usepackage{amssymb}
\usepackage{bbm}

%%%%%%%%%%%% Final List %%%%%%%%%%%%%%%%%%%%%%%%%%%%%%%% Questions%%%%%%%%%%%%%%%%%%%%%%%
% 1. Removed Table 3 from 1/30 version to IA


%%%%%%%%%%%%%%%%%%%%%%%%%%%%%%%%%%%%%%%%%%%% OR revision check list
% Bakshi 0000-0003-0838-4128
% crosby 0000-0003-4179-9464
% Xiaohui 0000-0001-8303-0675

%%%%%%%%%%%%%%%%%%%%%%%%%%%%%%%%%%%%%%%%%%%%%%%%%%%%%%%%%%%%%%%%%%%%%%%%%%%%%%

%%%%%%%%%%%%%%%% Reasonable Margins
\setlength{\textwidth}{6.5in} \setlength{\textheight}{8.8in}
\setlength{\topmargin}{-0.0in} \setlength{\oddsidemargin}{0in}
\setlength{\parskip}{1mm}
\setlength{\widowpenalty}{20000}      %/*150*/ No Widows at bottom of page
\setlength{\displaywidowpenalty}{20000}       %/*50*/
\setlength{\clubpenalty}{100000}               % No orphans at top of page

%\usepackage{setspace} JC commented out



\usepackage{color}
\usepackage[usenames,dvipsnames]{xcolor}

%% John needs to comment the next line
                     %\RequirePackage[colorlinks=true,urlcolor=blue, citecolor=BlueViolet, linkcolor=blue]{hyperref} %gsb commented out

% John needs to comment the next line
%\usepackage{hyperref}
% John needs to comment the next line
%\usepackage[width=1.0\textwidth, textfont=bf, font=onehalfspacing,labelfont=bf]{caption}
%\usepackage{pgf}
%\usepackage{tikz}
\usepackage{setspace}
\usepackage[margin=1in]{geometry}
\usepackage{authblk}
% JC added 10th April 2013
\usepackage{graphicx}


%\usepackage[sort, longnamesfirst]{natbib}
\usepackage[sort, round, longnamesfirst]{natbib} % Added 5th May 2015


%%%%%%%%%%%%%%%%%%%%%%%%%%%%%%%%%%%%%%%%%%%%%%%%%%%%%%%%%%%%%%%%%%%%%%%%%%%%%%%%%%%%%%%%%%%%%%%%%%%%%%%%%%%%%%%%%%%%%%%%%%%%%%%%%%%%%%%%%%%%%%%%%%%%%%%%%%%%%%%%%%%%%%%%%%%%%%%%%%%%%%%%%%%%%%%%%%%%%%%%%%%%%%%%%%%%%%%%%%%%%%%%%%%%%%%%%%%%%%%%%%%%%%%%%%%%
%TCIDATA{OutputFilter=LATEX.DLL}
%TCIDATA{Version=5.00.0.2606}
%TCIDATA{<META NAME="SaveForMode" CONTENT="1">}
%TCIDATA{BibliographyScheme=Manual}
%TCIDATA{Created=Tuesday, June 21, 2011 16:48:55}
%TCIDATA{LastRevised=Tuesday, June 21, 2011 18:04:08}
%TCIDATA{<META NAME="GraphicsSave" CONTENT="32">}
%TCIDATA{<META NAME="DocumentShell" CONTENT="Standard LaTeX\Blank - Standard LaTeX Article">}
%TCIDATA{CSTFile=40 LaTeX article.cst}

\newtheorem{theorem}{Theorem}
\newtheorem{acknowledgement}[theorem]{Acknowledgement}
\newtheorem{algorithm}[theorem]{Algorithm}
\newtheorem{axiom}[theorem]{Axiom}
\newtheorem{case}[theorem]{Case}
\newtheorem{claim}[theorem]{Claim}
\newtheorem{conclusion}[theorem]{Conclusion}
\newtheorem{condition}[theorem]{Condition}
\newtheorem{conjecture}[theorem]{Conjecture}
\newtheorem{corollary}[theorem]{Corollary}
\newtheorem{criterion}[theorem]{Criterion}
\newtheorem{definition}[theorem]{Definition}
\newtheorem{example}[theorem]{Example}
\newtheorem{exercise}[theorem]{Exercise}
\newtheorem{fact}[theorem]{Fact}
\newtheorem{illustration}[theorem]{Illustration}
\newtheorem{lemma}[theorem]{Lemma}
\newtheorem{notation}[theorem]{Notation}
\newtheorem{problem}[theorem]{Problem}
\newtheorem{proposition}[theorem]{Proposition}
\newtheorem{assumption}[theorem]{Assumption}
\newtheorem{remark}[theorem]{Remark}
\newtheorem{result}[theorem]{Result}
\newtheorem{solution}[theorem]{Solution}
\newtheorem{summary}[theorem]{Summary}
\newenvironment{proof}[1][Proof]{\noindent\textbf{#1.} }{\ \rule{0.5em}{0.5em}}
%\input{tcilatex}   JC commented out
%        \newtheorem{problem}{Problem}

%\usepackage{bbm}
\usepackage{bm}


%\begin{document}
%%%%%%%%%%%%%%%%%%%%%%%%%%%%%%%%%%% GSB added
%\usepackage{setspace}
%                       \setstretch{1.80}


%%%%%%%%%%%%%%%%%%%%%%%%%%%%%%%%%%%%%%%%%%%%%%%%%%%%%%%%%%%%%%%%%%%%%%%%%%%%%%%%%%%%%%%%%%%%%%%%%%%%%%%%%%%
%\title{A Result and Finding
%to
%%Envision Smarter Term-Structure
%%and Macro-Finance
%%Models}
%%{\color{blue} Discipline}
%Differentiate Among Term Structure Models}


\title{
%The Role of
%Market Incompleteness and
%, Volatility Uncertainty, and
%Unspanned Stochastic Volatility and the Trifecta of Option Pricing Puzzles
%The Dark Matter in Equity Option Risk Premiums and its Implications for Asset Pricing
%The Dark Matter in Equity {\color{red}[Index]} Volatility Dynamics: Assessing the Economic Rationales for Unspanned Risks
Dark Matter in (Volatility and) Equity Option Risk Premiums \\  \textbf{(\underline{Operations Research} December 2022)} }


%Strome College of Business,
%Old Dominion University,
%Norfolk VA
%johnc2205@yahoo.com

\author[1]{Gurdip Bakshi}
\affil[1]{Fox School of Business, Temple University,~~~gurdip.bakshi@temple.edu}
\author[2]{John Crosby}
\affil[2]{Strome College of Business, Old Dominion University,~~johnc2205@yahoo.com}
\author[3]{Xiaohui Gao}
\affil[3]{Fox School of Business, Temple University,~~~xiaohui.gao.bakshi@temple.edu}
\date{}




%%%%%%   date  26th March 2023   at  09.50 am



\begin{document}
\maketitle
\thispagestyle{empty}

%\vspace{-12mm}
\vspace{-6mm}
\begin{abstract}

\noindent Emphasizing the statistics of jumps crossing the strike and local time, we develop a
%{\color{red} [novel]}
decomposition of
equity option risk premiums. Operationalizing this theoretical treatment, we equip the pricing kernel process with unspanned risks,
embed (unspanned) jump risks, and allow equity return volatility to contain unspanned risks. Unspanned risks are
consistent with negative risk premiums for jumps crossing the strike and local
time and imply negative risk premiums for out-of-the-money call options
and straddles. The empirical evidence from weekly and farther-dated index options is supportive of
our theory of economically relevant unspanned risks and reveals ``dark matter" in option risk premiums.



%{\color{green} If the evolution of the equity price index and the pricing kernel were
%%to be
%absent of unspanned
%risks,
%%{\color{red}[[[ I still find ``If the equity price index ... were to be absent of unspanned risks" slightly baffling?  ]]]}
%%[[[[Another way to write the first part would be:
%%``If the pricing kernel were to have no unspanned risks,"
%%
%%or ``If the evolution of the pricing kernel were to be absent of unspanned
%%risks,"
%%]]]]}
%then this would counterfactually imply that (i) the expected excess return of out-of-the-money calls on
%equity is positive and (ii) the expected excess return of straddles is zero.}
%Remedying these contradictions,


%\noindent {\color{magenta} If the evolution of the equity price index {\color{green}[, its volatility,]} and the pricing kernel were to be absent of unspanned
%risks, it would counterfactually imply that (i) the expected excess return of out-of-the-money calls on
%equity is positive and (ii) the expected excess return of straddles is zero.}
%Remedying these contradictions, we equip the pricing kernel process with unspanned risks,
%embed (unspanned) jump risks, and allow return volatility to contain unspanned risks.
%The
%empirical evidence from weekly and farther-dated options is supportive of our
%theory of economically relevant unspanned risks and reveals ``dark matter" in option risk premiums.

\end{abstract}


\begin{flushleft}
\textbf{Keywords}:
Unspanned equity volatility and jump risks,
unspanned risks in the pricing kernel,
%semimartingales,
dark matter, option risk premiums
\vspace{2 mm}\\
%\end{flushleft}

\vspace{3mm}

%\textbf{JEL classification codes}: G12, G15, E44, F31, F36. \vspace{15 mm} %\vspace{25 mm}
\end{flushleft}


\vspace{13mm}
\thanks{
\noindent
We are extremely grateful for the feedback of Steven Kou (the editor), the
associate editor,
and the two referees.
First posted on SSRN: November 23, 2020. Participants at the
Canadian Derivatives Institute Conference (September 2020),
Midwest Finance Association (March 2021),
EFA (April 2021),
ITAM conference (May 2021),
Sofie conference,
the Econometrics Society meeting (Melbourne, Australia),
the University of Maryland,
Temple,
Case Western University, and
University of Manitoba (Asper School) provided many insightful
comments that we incorporated in the paper.
The paper has improved
from the feedback of Winston Dou (discussant),
Jeroen Dalderop (discussant),
Olivier Scaillet (discussant),
and Dmitry Makarov (discussant).
We are indebted to Francis Longstaff for urging us to explore the connection of local time risk premiums to gamma risk premiums. Peter Carr suggested that we make precise how the variance/dispersion risk premium is related to the local time risk premium.
Liuren Wu alerted us to a result that implied variance equates to the expected time integral of instantaneous variance weighted by dollar gamma.
Zhenzhen Fan,
Bjorn Eraker,
Steve Heston,
Pete Kyle,
Bingxin Li,
Mark Loewenstein,
Dilip Madan,
Peter Ritchken,
Oleg Rytchkov,
Ivan Shaliastovich,
Tobias Sichert,
Jinming Xue,
Justin Vitanza,
Li Wang,
Liuren Wu,
and Wei Zhou provided comments that improved our paper.
The authors are grateful to colleagues at Temple University and the University of
Maryland for
discussions.
An earlier version was circulated under the
title ``The Dark Matter in Equity Index Volatility Dynamics: Assessing
the Economic Rationales for Unspanned Risks."} % Any remaining errors are our responsibility alone.}



%%%%%%%%%%%%%%%%%%%%%%%%%%%%%%%%TABLE OF CONTENTS%%%%%%%%%%%%%%%%%%%%%%%%%%%%%%%%%%%%%%%%%%%%%%%%%%%%%%%%%%%
%                                      \newpage
%                                        \tableofcontents
%                                           \thispagestyle{empty}
                                           %%%%%%%%%%%%%%%%%%%%%%%%%%%%%%%%%%%%%%%%%%%%%%%%%%%%%%%%%%%%%%%%%%%%%%%%%%%%%%%%%%%%%%%%%%%
%

\newpage
\doublespacing
\setcounter{page}{1}

%%%%%%%%%%%%%%%%%%%%%%%%%%%%%%%%%%%%%%%%%%%%%%%%%%%%%%%%%%%%%%%%%%%%%%%%

%%%%%%%%%%%%%%%%%%%%%%%%%%%%%%%%%%%%%%%%%%%%%%%%%%%%%%%%%%%%%%%%%%%%%%%%
\section{Introduction}
\label{sec:introduction}

%Is there dark matter embedded in volatility and
%{\color{red}[[ ``in" added ]]]}
%{\color{blue}in} equity {\color{red}[[[ options? ]]]}
%{\color{blue} option prices?}

Is there dark matter embedded in volatility and equity options?
We present a semimartingale theoretical approach that allows
us to introduce the constructs of risk premiums on \emph{jumps crossing}
the strike (from
above and below
(details later)) and on \emph{local time}. A semimartingale is the most general
type of process suitable for modeling equity prices.

The treatment of jumps crossing the strike and local time is integral to our
theory, because their absence would be counterfactual from an empirical standpoint. We label such
abstract uncertainties --- driven by unspanned risk components --- ``dark matter," as they can
be hard to identify, but their presence is implied in options data,
and the workings of dark matter can be economically influential.

Through our theoretical characterizations, we reveal the manner in which call option risk premiums can be decomposed into
dark matter risk premiums and \emph{upside} equity risk premiums.
Our empirical exercises are based on weekly equity index options (the ``weeklys"), in addition to the farther-dated (index and futures) options up to 88 days maturity.

%{\color{red}[[Through our theoretical characterizations, we reveal the manner in which call option risk premiums can be decomposed into
%dark matter risk premiums and \emph{upside} equity risk premiums.
%{\color{green} Our theoretical treatment predicts positive call risk premiums
%and a zero straddle risk premium \emph{only if} there are no unspanned risks
%in the pricing kernel.} The latter is contrary to our empirical evidence.]]}


\textbf{Elements of our approach.}  We propose a theory with three tenets. First, equity volatility is impacted
by both spanned and unspanned risks. Unspanned risks refer to uncertainties not spanned by equity futures but possibly
spanned by
options.

Second, the jump structure is unspecified, and no stance is taken about the exact nature of discontinuities
(e.g., \citet[Figure 7]{AitsahaliaJacod:2012}). Akin to \citet*{Merton:76} and \citet*{Kou:2002}, the jumps
constitute unspanned risks that are unhedgeable.

Third, we highlight pricing kernel evolution
that incorporates both unspanned and spanned risks. Essential to our decompositions is Tanaka's formula
for semimartingales, which gives rise to the analytical forms of (i) jumps crossing the strike and (ii) local time.

Rooted in our theory is the notion that unspanned risks differentially impact the physical and risk-neutral expectations of (i) jumps
crossing the strike and (ii) local time. To reproduce data traits, the properties of unspanned risks
in the pricing kernel, price jumps, and volatility dynamics must be such that the
risk premiums for jumps crossing the strike and local time are
negative. We formalize
how the concept of local time risk premium is distinct from volatility risk premium.

\textbf{Implications of a theory with unspanned risks and dark matter.} The implications of dark matter permeate
the spectrum of claims on equity and volatility, on both the downside
and the upside. For instance, risk premiums of out-of-the-money (OTM) calls can only be negative, as
supported by our empirical work from weeklys,
if the dark matter risk premiums --- the sum of risk premiums for jumps crossing the strike
and local time --- are negative.
Negative dark matter risk premiums stem from
unspanned risks impacting the pricing kernel, volatility dynamics, and price jumps.\footnote{Our investigation favors return volatility dynamics that cohabit
with unspanned risks. To our knowledge, the
scope of this feature has not been appreciated in the theoretical and empirical
equity pricing
literature.}


%{\color{blue} Zero dark matter risk premium counterfactually stems from
%there being
%%essentially
%no unspanned risks impacting
%the pricing kernel or volatility dynamics} and no price jumps.

%Motivating our study in
\textbf{Relation to the dark matter literature.} The work of \citet*{Chen_Dou_Kogan:JF2020} formalizes
a theory for measuring dark matter in asset pricing models.
Their
approach
is founded in the observation
that some
models rely on a form of dark matter,
by which they mean
economic
components or parameters that are
difficult to measure directly.
%{\color{red}[[[ a possible alternative to ``that are
%difficult to verify and measure directly" might be
%``that are
%difficult to estimate directly" or ``that are
%difficult to estimate" or even ``that are
%difficult to estimate, with precision, "? ]]]}
%in the data.
Complementing,
we depict
dark matter as
variables whose
dominant source is unspanned risks
in volatility and (price) jumps crossing the strike, and we use it to
summarize the properties of option returns.
We additionally show that dark matter risk premiums
take
center stage in the construction of the volatility
risk premium.

%The insight of
%To understand economic mechanisms of dark matter uncertainty,
Paving the way for a better appreciation of
dark matter uncertainties,
%{\color{red}[[[ a nicer way might be to write this as:
%``Paving the way for a better appreciation of dark matter,
%\citet*{Cheng_Dou_Liao:ECMTA2021}
%develop
%robust ....." -- i.e. leaving out ``uncertainties" ]]]]}
\citet*{Cheng_Dou_Liao:ECMTA2021}
develop
model evaluation procedures for testing asset pricing models.
%\citet*{Cheng_Dou_Liao:ECMTA2021}
%%propose
%%are after the question of
%develop
%robust model evaluation procedures for testing asset pricing models,
%especially those that embed
%%are subject
%%prone
%%to
%%confounding effects of
%dark matter uncertainties. % (the rare-disaster model).
%\citet*{Cheng_Dou_Liao:ECMTA2021}
%%propose
%%are after the question of
%develop
%robust model evaluation procedures for testing asset pricing models,
%especially those that embed
%%are subject
%%prone
%%to
%%confounding effects of
%dark matter uncertainties. % (the rare-disaster model).
%especially with the dark matter feature.
%those that are vulnerable to dark matter.
%subject to  with dark matter.
Their
proposed
econometric
%synthesis,
methodology,
while not implemented in this paper,
can be adapted to
probe %understand}
%the inscrutability of
%forms of
%dark matter implicit in option returns and the
%their interpretation in models with unspanned risks.}
%(i.e., the
the dark matter restrictions
of option pricing models with unspanned volatility and jump risks.
%using option data.}


The subject of our paper invites
connections
with \citet*{Chen_Dou_Kogan:JF2020} and
\citet*{Cheng_Dou_Liao:ECMTA2021}. Like them, we utilize the dark matter link, consistent with the notion
from cosmology:
The dynamics of the local time, the jumps crossing the strike, and the properties of the
pricing kernel may be hard
to identify directly using equity index returns. Instead,
their relevance can
be inferred only from option returns through the standpoint of the
model-implied restrictions.

%An object which must exist but whose presence can only be indirectly inferred.

Our contributions complement, yet differ from,
\citet*{Chen_Dou_Kogan:JF2020} and  \citet*{Cheng_Dou_Liao:ECMTA2021}.
First, they consider dark matter as the degree of fragility for
potentially
misspecified models
formulated under the data-generating
measure $\mathbb{P}$,
whereas our usage pertains to local time and
jumps crossing the strike under
both $\mathbb{P}$ and an
equivalent martingale measure $\mathbb{Q}$.
Second, we
develop the notion of risk premiums for dark matter and
economically isolate
their sign by taking cues from
option excess returns differentiated by
strikes and maturities. Third, we employ option data
to analyze
the presence of
dark matter --- specifically, to unravel
%analyze
%establish
%recognize
%unravel
%identify
the workings of unspanned risks in the pricing kernel.
%to be consistent with the data.
%Fourth,
%%\citet*{Chen_Dou_Kogan:JF2020} and \citet*{Cheng_Dou_Liao:ECMTA2021} detail {\color{blue} the measurement of
%%dark matter} under models with jumps (specifically, rare disasters),
%we add the new dimension of jumps across the strike, that is, jumps in equity index prices that result in options
%going directly from in-the-money (respectively, out-of-the-money) to out-of-the-money (respectively, in-the-money).}



\textbf{Empirical takeaways informed by option excess returns.}
Although we do not observe dark matter,
we can infer the effect
of negative dark matter risk premiums from call risk
premiums
getting more negative deeper
OTM.
The empirical setting of weeklys
aids
in decoupling the effect of jumps crossing the strike from that of
local time.
Our bootstrap exercises show
that risk premiums
for jumps crossing the strike are equally pronounced on the upside as
they are on the downside. With
weeklys,
the dark matter and its risk premium are shaped by jumps crossing the strike. This is gauged by the size
of the risk premiums for puts, and calls, deeper OTM.


Our evidence from negative straddle risk premiums
undermines
the ``no unspanned risks" hypothesis. We infer negative risk
premiums for local time from farther-dated options. Our findings are consistent with a dislike for jumps
crossing the strike (as inferred from weeklys)
and a dislike for unspanned volatility risks (as
inferred from farther-dated options).
A rationale is
that jump movements across
actively traded strike thresholds are pertinent to
traders
and to the exposures of option writers.
Our conclusions stem from the behavior of option returns and
%are notable since
%stems from the feature that
%our approach is free of
do not hinge on parametric assumptions about the evolution of the pricing kernel,
returns, and volatility.




\textbf{Theoretical and empirical context for why our approach
is relevant.}
We present an explanation
that conforms with
data features from
the equity options market.
%s in the equity index, futures, and options markets.
If there were no unspanned jump risks in the pricing kernel,
              %{\color{red}[then the risk premium for local time would be identically zero]},
then no risk premium would be elicited for jumps crossing the strike,
refuting
empirical evidence.
Imparting
direct
theoretical and empirical content, our predictions are devised using Tanaka's formula for
semimartingales.
This framework
gives economic footings to the concepts of jumps
crossing the strike terms and local time
and yields the context for the salient
data features of option returns.
%traded  convex payoffs}.
%\footnote{Efforts to understand options data is ongoing, and
%work is voluminous. In what way do we offer differentiation?
%\citet*{AndersenBollerslevDieboldLabys:ecma2003} present
%a theory in which the
%price process can be decomposed
%into a continuous-sample path part and a jump part.
%They consider volatility
%concepts with a focus on measuring and forecasting volatility.
%Essential to \citet*{CarrWu:2003bJF}
%is the question of what type of risk-neutral
%processes underlie options. Using
%short-dated options,  they {\color{blue} discern
%the relevance
%of both
%continuous} and jump components.}

Efforts to understand options data are ongoing.
%and work is voluminous.
%{\color{red}[In what way do we offer differentiation?]}
\citet*{AndersenBollerslevDieboldLabys:ecma2003} present
a theory in which the
price process can be decomposed
into a continuous-sample path part and a jump part.
%They consider volatility
%concepts with a focus on measuring and forecasting volatility.
Essential to \citet*{CarrWu:2003bJF}
is the question of what type of risk-neutral processes underlie options, and
%. Using short-dated options,
they discern
the relevance
of both
continuous and jump components.
The treatment of \citet*{Bollerslev_todorov:JF_2011} shows that
the compensation for rare events accounts for a large fraction of the
%average
equity and variance
risk premiums.
%What emerges from
%%the analysis of
%\citet*{Todorov_Tauchen:2011JBES} is a volatility process with jumps of infinite variation.
\citet*{Todorov_Tauchen:2011JBES} favor a volatility process with jumps of infinite variation.
%%and  that jumps in volatility and price move in the opposite direction.
Using high-frequency data,
%the methods of
\citet*{AitsahaliaJacod:2012}
%imply
show that models are amiss if they fail
to simultaneously incorporate the continuous, small, and large jump
components of
%equity
returns.
\citet*{Andersen_fusari_todorov:2015_JFE} identify a factor driving the left jump tail of the risk-neutral
distribution. They show
that option markets embody
critical
information about the
%market
risk premium and its dynamics.
%{\color{red}[Essential to \citet*{Bollerslev_Todorov_Xu:2015_JFE} is that the variance risk premium helps predict subsequent
%market returns, and that much of this predictability arises from the part of the variance risk premium associated with tail risk.]}


%Departing from others,
Our approach is about distilling the effects of
unspanned
risks relevant to
trading
options. Our interest is not modeling the
volatility or
price
jump distributions but rather, it is on
uncovering the
properties that unspanned risks --- in the pricing kernel, price jumps, and volatility --- must possess
to be compatible with option returns.
While dissecting the channel of unspanned risks, we propose model-free
%results.
characterizations.
All in all,
we offer
differentiation
%inroads
by framing theoretical and empirical
questions using the constructs
of local time and jumps crossing the strike and synthesizing
economic mechanisms by
combining short- and farther-dated
option prices. \vspace{-4mm}




%**************************
%{\color{red} [Thisexplicit, strong message will highlight the contribution of this paper and add a lot of value].}
%{\color{red} [[[ However, the current draft stops before allowing the researchers who work on option pricing
%and trading to understand so what if option pricing models feature excessive economic dark
%matter. ]]]}

%{\color{red}[Complementing the works of \citet*{Chen_Dou_Kogan:JF2020} and \citet*{Cheng_Dou_Liao:ECMTA2021}, our
%paper has consequences for understanding dark matter embedded in
%[[[[[ the ]]]]]] option pricing theory.]}
%{\color{red} These unspanned risks are typically assumed away in extant macro-finance models.}
%{\color{red} [[[ Our core idea is to test whether option prices are consistent with the time series properties of the
%underlying asset price. ]]]]}

%{\color{red} [[[[ Perhaps because of the lack of a strong and consistent focus on the core issue
%of economic dark matter, at times the motivation is hard to follow while I was reading the paper, and the contributions are likely
%to be hard for the researchers and practitioners who are not familiar with the concept of economic dark matter.}
%{\color{red} [[[[[ I strongly suggest that the revised draft focuses on the dark matter property of option pricing
%models and provides more guidance on how the researchers and practitioners should act to use the findings of this paper. ]]]]}


%%%%%%%%%%%%%%%%%%%%%%%%%%%%%%%%%%%%%%%%%%%%%%%%%%%%%%%%%%%%%%%%%%%%%%%%%%%%%%%%%%%%%%%%%%%%%%%%%%%%%%%%%%%%%%
%%%%%%%%%%%%%%%%%%%%%%%%%%%%%%%%%%%%%%%%%%%%%%%%%%%%%%%%%%%%%%%%%%%%%%%%%%%%%%%%%%%%%%%%%%%%%%%%%%%%%%%%%%%%%%
\section{Dark matter, unspanned risks, and
option risk premiums} % and the Trifecta of Option Pricing Puzzles}
\label{sec-GeneralDiffusionDynamics}


% The path consists of continuous motion interspersed with diverse form of jump continuities of random size
% appearing at random time.

Consider a theoretical framework
% is adapted from \citet*{Bakshi_Crosby_Gao:2019wp}, with four differences.}
%First,
%considers a market
in which an equity index is tradeable and written upon
it is a futures contract.
%{\color{red}[[[ Do we even need
%``interest-rates are constant"? ]]]]}
%Second, interest-rates are constant. Third, and
Essential for interpretations in the market for equities,
we consider the setting of
a general \emph{semimartingale}
(that encompasses diverse forms of discontinuities (jumps)).
%{\color{red}[[[ We need to be careful here. We
%cannot and should not claim to be in
%a general \emph{semimartingale} setting if we don't make our proofs to that
%level of generality. ]]]]}

There are certain risks that are spanned by equity index futures and risks that are, by definition,
unspanned. The sources of risks are allied to movements in volatility as
well as jump discontinuities. % (in price and/or return volatility).

In what follows, let $(\Omega,\mathcal{F},(\mathcal{F}_t)_{0 \leq t \leq \mathfrak{T}},\mathbb{P})$ be a filtered probability space, with $\mathfrak{T}$ being
a fixed
finite time.
The filtration $(\mathcal{F}_t)_{0 \leq t \leq \mathfrak{T}}$ satisfies the usual
conditions.
%{\color{red}[[[  We should be careful here with this statement.
%Local time $\mathbb{L}^{T_O}_t[k]$ has a subscript $t$ and is not
%$\mathcal{F}_t$-measurable. I always wondered if we should change
%notation for $\mathbb{L}^{T_O}_t[k]$ or take out this statement. ]]]]}
Stochastic processes are assumed to be right continuous with left limits.

Let $\mathbb{P}$ denote the physical probability measure.
Since markets are not complete,
there is neither a unique equivalent martingale measure
nor a unique pricing
kernel.
We consider an equivalent martingale measure $\mathbb{Q}$ and a pricing kernel $M_t$
consistent with the absence of arbitrage.

Additionally, we assume that $M_t$ is a \emph{semimartingale}.
Fixing notation,
$\mathbb{E}^{\mathbb{P}}_{t}( \bullet ) \equiv  \mathbb{E}^{\mathbb{P}}( \bullet | \mathcal{F}_t )$ (respectively, $\mathbb{E}^{\mathbb{Q}}_{t}( \bullet ) \equiv  \mathbb{E}^{\mathbb{Q}}( \bullet | \mathcal{F}_t )$) is the expectation under $\mathbb{P}$ (respectively, $\mathbb{Q}$), \emph{conditional} on $\mathcal{F}_t$. Furthermore,  $r$ is the spot interest-rate, assumed constant.


\noindent \textbf{Equity
%{\color{green}[risk]}
premium.}
The (cum dividend) equity index price, at time $t$,
is denoted by $S_t$,
and is a semimartingale.
We maintain that
the time
$t$
conditional equity
%{\color{green}[risk]}
premium is positive over any holding period ${T}_O-t$; that is,
$\mathbb{E}_{t}^{\mathbb{P}}( \frac{S_{{T}_O}}{S_t} ) - e^{r ({T}_O-t)}>0$.

\noindent \textbf{Gross equity futures return.}
We denote the time $t$ equity futures price by $F_{t}^{T_F}$, where $T_F$ denotes the expiration date of the futures contract.
It holds that
\begin{align}
F_t^{T_F} &~=
\mathbb{E}_{t}^{\mathbb{P}} ( \frac{M_{\ell}}{M_{t} e^{ -r (\ell-t)}}  F_{\ell}^{T_F} ) \, = \mathbb{E}_{t}^{\mathbb{Q}}( F_{\ell}^{T_F} ),
&~\mbox{ \, \, for all $t$ and $\ell$ satisfying $t \leq \ell \leq T_F$},\mbox{ \, \, }~~
\label{eq:RelationsInGeneralDynamics}  \\
 &~=  S_t \, e^{r (T_F - t)},
 &\mbox{ \, \,(i.e., cost of carry with $S_{T_F} = F_{T_F}^{T_F}$), \, \, }~
 \label{fuut}
\end{align}
where
$\frac{M_{\ell}}{M_{t} e^{ -r (\ell-t)}}$ represents the Radon-Nikodym derivative. Hence, the
process $(G_\ell)$ defined by %, for $\ell \geq t$, by
\begin{eqnarray}
G_\ell \equiv \frac{F_{\ell}^{T_F}}{F_{t}^{T_F}},~\mbox{ }~\mathrm{represents~the~\emph{gross~futures~return},~from}~t~\mathrm{to}~\ell,~\mathrm{for}~\ell~\mathrm{satisfying}~t \leq \ell \leq T_F.
%\mbox{ \, \quad }
\label{eq:DefinitionOfYProcess}
\end{eqnarray}


\noindent \textbf{Futures risk premium on the downside and upside.}
The %equity
futures risk premium, with $G_t=1$, is given by
$\mathbb{E}_{t}^{\mathbb{P}}(\frac{F_{T_O}^{T_F}}{F_{t}^{T_F}}) -
\mathbb{E}_{t}^{\mathbb{Q}}(\frac{F_{T_O}^{T_F}}{F_{t}^{T_F}}) =
\mathbb{E}_{t}^{\mathbb{P}}(\frac{F_{T_O}^{T_F}}{F_{t}^{T_F}}) -
1 =  \mathbb{E}_{t}^{\mathbb{P}}( G_{T_O}) - G_{t} = \mathbb{E}_{t}^{\mathbb{P}}( \int_{t}^{T_O} dG_{\ell} )$. Define $k$ as
\begin{equation}
 k \, \equiv \, \frac{K}{F_{t}^{T_F}} \, \in \, (0,\infty),
%\mbox{ \, }
~
\text{\small
which~is~the~option~moneyness~for~strike~price}~K. \mbox{ \, } ~
\end{equation}
In light of their connection to option risk premiums,
we define the following futures risk premiums:
\begin{align}
&\mathbb{E}_{t}^{\mathbb{P}}( \int_{t+}^{{T}_O} \mathbbm{1}_{\{G_{\ell-} < k\}} \,dG_{\ell} )&
&&
&\mbox{(downside~risk premium, $k<1$)}~~~\mathrm{and}& \label{a.x2} \\
&\mathbb{E}_{t}^{\mathbb{P}}( \int_{t+}^{{T}_O} \mathbbm{1}_{\{G_{\ell-} > k\}} \,dG_{\ell} )&
&&
&\mbox{(upside~risk premium, $k>1$).}~& \label{a.x1}
\end{align}
Additionally, $\mathbbm{1}_{\{G_{\ell-} > k\}}=1$ if $G_{\ell-} > k$ and is zero otherwise.
In equations (\ref{a.x2})--(\ref{a.x1}), $G_{\ell-}$ can be thought of as the value ``just an instant before time $\ell$."

Both of the terms in
equations (\ref{a.x2})--(\ref{a.x1}) reflect risk premiums since
$\mathbb{E}_{t}^{\mathbb{Q}}( \int_{t+}^{{T}_O} \mathbbm{1}_{\{G_{\ell-} < k\}} \, dG_{\ell} )=0$ and $\mathbb{E}_{t}^{\mathbb{Q}}( \int_{t+}^{{T}_O} \mathbbm{1}_{\{G_{\ell-} > k\}} \,dG_{\ell} )=0$.
This is because
$(F_{\ell}^{T_F})$ and $(G_\ell)$ are martingales under
$\mathbb{Q}$.

%sant 1/25/2022
%{\color{red} [[[[[[[[[ We assume that the downside and upside futures risk premiums in (\ref{a.x2})--(\ref{a.x1}) are both positive.]}
%{\color{red} [For example, $\mathbb{E}_{t}^{\mathbb{P}}( \int_{t+}^{{T}_O} \mathbbm{1}_{\{G_{\ell-} > k\}} dG_{\ell} )>0$ is equivalent to
%assuming that $\mathbb{E}_{t}^{\mathbb{P}}( \int_{t+}^{{T}_O} \mathbbm{1}_{\{F^{T_F}_{\ell-} > K \}} \frac{dF^{T_F}_{\ell}}{F^{T_F}_t} )>0$. ]]]]]]]]]]}


\noindent \textbf{Options on the equity futures price with moneyness $k$.}
Consider an option written on the equity futures price over
%the time period
$t$ to ${T}_O$ with strike price $K$ (or moneyness $k$).
Therefore,
\begin{equation}
\mathrm{for ~ OTM ~ (at\mbox{-}the\mbox{-}money) ~calls} ~ k \, > \, 1 ~ ~ (k \, = \, 1) ~ ~ \mathrm{and ~ for ~ OTM ~ puts}, ~ k \, < \, 1. ~ ~ ~ ~ ~
~ \mbox{ \, \quad \, }
\end{equation}
It is understood that $t \leq {T}_O \leq T_F$, where $T_O$ is the maturity of the option.
The expected return of holding a call option on equity futures over $t$ to ${T}_O$ with moneyness $k$,
denoted $\mu^{{T}_O}_{t,{\tiny \mathrm{call}}}[k]$, satisfies
\begin{eqnarray}
1 + \mu^{{T}_O}_{t,{\tiny \mathrm{call}}}[k]  &\equiv&  \frac{\mathbb{E}_{t}^{\mathbb{P}}( \max (F_{{T}_O}^{T_F} - K, 0) )}{ e^{-r ({T}_O - t)}\,  \mathbb{E}_{t}^{\mathbb{Q}}(
\max(F_{{T}_O}^{T_F} - K, 0) )}
%\, \,\\
~=~  \, \frac{\mathbb{E}_{t}^{\mathbb{P}}( \max(G_{{T}_O} - k,0) )}
{ e^{-r ({T}_O - t)}\, \mathbb{E}_{t}^{\mathbb{Q}}(
\max(G_{{T}_O} - k,0)  )}.~~~\mbox{ \, \, \, \, }
\label{eq:ExpectedHoldingReturn1GneralDynamics}
\end{eqnarray}

%Table~\ref{tab:notation} summarizes the notation and the description of the terms, for ease of reference.

\noindent \textbf{Tanaka's formula for semimartingales.}
Our Theorem~\ref{claimm:claim1call_jump}
will rely
on Tanaka's formula for (general) semimartingales.
Specifically (and relevant for call option payoffs),
Tanaka's formula for semimartingales --- as in \citet*[Theorem 68, page 216]{Protter:2013} ---
implies (mapping
his notation of $x^{+}=\max(x,0)$ and $x^{-}=-\min(x,0)=\max(-x,0)$)
\begin{eqnarray}
\max( G_{T_O} - k, 0 ) ~-~ \overbrace{ {\underbrace{\max( G_{t} - k, 0 )}_{\tiny =0,~\mbox{for~OTM~calls}}}}^{\tiny \mbox{intrinsic~value}} & = &
\int_{t+}^{T_O} \mathbbm{1}_{\{G_{\ell -} > k\}} dG_{\ell} ~+~
\overbrace{\mathbb{L}^{T_O}_t[k]}^{\tiny \mbox{local~time}}~~\mbox{ \, \, } \nonumber \\
& & +~~ \underbrace{\sum_{t < \ell \leq T_O} \mathbbm{1}_{\{G_{\ell -} \leq k\}} \, \max( G_{\ell} - k, 0 )}_{~\equiv~a_t^{T_O}[k]~~\tiny \mbox{(jumps~crossing~the~strike~from~below)}} \nonumber \\
%& & ~-~~~ \underbrace{\sum_{t < \ell \leq T_O} \mathbbm{1}_{\{G_{\ell \, -} > k\}} \, \min( G_{\ell} - k, 0 )}_{~\equiv~b_t^{T_O}[k]}.\\
& & +~~~ \underbrace{\sum_{t < \ell \leq T_O} \mathbbm{1}_{\{G_{\ell -} > k\}} \, \max( k - G_{\ell}, 0 ).}_{~\equiv~b_t^{T_O}[k]~~\tiny \mbox{(jumps~crossing~the~strike~from~above)}}
~~\mbox{ \quad}~~
\label{eq:TanakaJumps}
\end{eqnarray}

The summand terms on the second and third lines characterize \emph{large deviations} or significant events
and do not appear in the absence of jumps. We interpret
them as follows (presuming $k>1$):
\begin{description}
\item $\mathbbm{1}_{\{G_{\ell -} \leq k\}} \max( G_{\ell} - k, 0 )$
%(appearing in $a_t^{T_O}[k]$)
is only nonzero when $G_{\ell -} \leq k$ and $G_{\ell} > k$ --- loosely speaking, when
a jump at time $\ell$ results in $G$ jumping from below $k$ to above $k$ (i.e., the equity futures price jumps
upward
and crosses the level of the strike).

\item $\mathbbm{1}_{\{G_{\ell \, -} > k\}} \max( k - G_{\ell}, 0 )$
is only nonzero when $G_{\ell -} > k$ and $G_{\ell} < k$ --- loosely speaking, when
a jump at time $\ell$ results in $G$ jumping from above $k$ to below $k$.
\end{description}
In a continuous semimartingale setting, $a_t^{T_O}[k]$ and $b_t^{T_O}[k]$ vanish.
Finally, the term $\int_{t+}^{T_O} \mathbbm{1}_{\{G_{\ell -} > k\}} \,dG_{\ell}$ is a stochastic integral representing the gains/losses to a dynamic trading strategy that takes a
long position of magnitude $\frac{1}{F_{t}^{T_F}}$ at time $\ell$, in the futures, if, and only if, $G_{\ell  -} > k$ (i.e., $F_{\ell -}^{T_F} > K$).

Likewise, and relevant for put option payoffs, Tanaka's formula for semimartingales implies
\begin{eqnarray}
\max( k - G_{T_O}, 0 ) -
\overbrace{ {\underbrace{\max( k- G_{t}, 0 )}_{\tiny =0,~\mbox{for~OTM~puts}}}}^{\tiny \mbox{intrinsic~value}}
  & = & ~ - ~
\int_{t+}^{T_O} \mathbbm{1}_{\{G_{\ell -} < k\}} dG_{\ell} ~+~
\overbrace{\mathbb{L}^{T_O}_t[k]}^{\tiny \mbox{local~time}}~~\mbox{ \, \, } \nonumber \\
& &+~~ \underbrace{\sum_{t < \ell \leq T_O} \mathbbm{1}_{\{G_{\ell  -} \geq k\}} \, \max( k - G_{\ell}, 0 )}_{~\equiv~c_t^{T_O}[k]~\tiny~\mbox{(jumps~crossing~the~strike~from~above)}} \nonumber \\
%& & ~-~~~ \underbrace{\sum_{t < \ell \leq T_O} \mathbbm{1}_{\{G_{\ell \, -} < k\}} \, \min( k - G_{\ell}, 0 )}_{~\equiv~d_t^{T_O}[k]}.\\
& & +~~~ \underbrace{\sum_{t < \ell \leq T_O} \mathbbm{1}_{\{G_{\ell -} < k\}} \, \max(  G_{\ell}- k, 0 ).}_{~\equiv~d_t^{T_O}[k]~\tiny~\mbox{(jumps~crossing~the~strike~from~below)}}
~~\mbox{ \quad}~~ \label{eq:TanakaPutCaseJumps}
\end{eqnarray}
We interpret the terms in our context as follows (presuming~$k <1$):
\begin{description}
\item $\mathbbm{1}_{\{G_{\ell -} \geq k\}} \max( k-G_{\ell}, 0 )$
is only nonzero when $G_{\ell -} \geq k$ and $G_{\ell} < k$ --- loosely speaking, when
a jump at time $\ell$ results in $G$ jumping from above $k$ to below $k$.


\item $\mathbbm{1}_{\{G_{\ell -} < k\}} \max( G_{\ell} - k, 0 )$
%(appearing in $a_t^{T_O}[k]$)
is only nonzero when $G_{\ell -} < k$ and $G_{\ell} > k$ --- loosely speaking, when
a jump at time $\ell$ results in $G$ jumping from below $k$ to above $k$.

\end{description}
In a continuous semimartingale setting, $c_t^{T_O}[k]$ and $d_t^{T_O}[k]$
are identically zero.
The
term $-\int_{t+}^{T_O} \mathbbm{1}_{\{G_{\ell -} < k\}} \,dG_{\ell}$ reflects
the
gains/losses to a dynamic trading strategy
that takes a short futures  position of magnitude
$\frac{1}{F_{t}^{T_F}}$ at time $\ell$, if and only if, $G_{\ell  -} < k$ (i.e., $F_{\ell -}^{T_F} < K$).

\noindent \textbf{Local time and
risk premiums on local time.}
In
equations
(\ref{eq:TanakaJumps}) and (\ref{eq:TanakaPutCaseJumps}),
the term
\begin{eqnarray}
\mathbb{L}^{T_O}_t[k]=\frac{1}{2} \int_{t}^{T_O} \delta_{\{G_\ell ~-~ k\}}\,d [ G^\mathrm{c}, G^\mathrm{c} ]_{\ell} ~~ \mathrm{is~the~\emph{local~time}.}
%~as~defined~in~}
~~~\text{\small (\citet*[Theorem~71, page~221]{Protter:2013})}  ~ ~
\label{ltt.1}
\end{eqnarray}
In
(\ref{ltt.1}), $\delta_{\{\bullet\}}$ is the Dirac delta function,
and $[ G^\mathrm{c}, G^\mathrm{c} ]_{\ell}$ denotes the path-by-path  continuous part of
the
quadratic variation, defined
(see \citet*[page~70]{Protter:2013})
as
\begin{equation}
[ G^\mathrm{c}, G^\mathrm{c} ]_{\ell} ~~ \equiv ~ \underbrace{~ [ G, G ]_{\ell} ~ }_{\tiny \mbox{quadratic~variation}} ~ - ~ \underbrace{\sum_{t \leq h \leq \ell} (G_{h} - G_{h -})^2.}_{\tiny \mbox{sum~of~squares~of~the~jumps}} ~~ \mbox{ \, \quad } \label{eq:RelationQVToPathByPathContinuousQV}
\end{equation}

Intuitively,
$\mathbb{L}^{T_O}_t[k]$ captures the slice of uncertainty associated with the
time that $G_{\ell}$ spends at the level $k$. In economic terms, one may contemplate $\mathbb{L}^{T_O}_t[k]$ as a form of
volatility
uncertainty. Continuous semimartingales
imply $\sum_{t \leq h \leq \ell} (G_{h} - G_{h -})^2  = 0$, for all $h$,
so, in this case, one
may view \emph{local time} as a measure of integrated variance over $T_O-t$ computed
when $(G_{\ell})$ is
\emph{exactly} $k$.
%\emph{precisely equal}  to $k$.

The local time reflects sample path properties that do not vary according to the
measures $\mathbb{P}$ or $\mathbb{Q}$.
At the same time, the expectations of $\mathbb{L}^{T_O}_t[k]$ under $\mathbb{P}$ and $\mathbb{Q}$ may differ. We define
\begin{equation}
\mathbb{E}^{\mathbb{P}}_{t}( \mathbb{L}^{T_O}_t[k] ) ~-~ \mathbb{E}^{\mathbb{Q}}_{t}( \mathbb{L}^{T_O}_t[k] )
~\mbox{ \, } ~\mathrm{as~the~\emph{local~time~risk~premium}~for~moneyness}~k. ~~~ \mbox{ \,  \, }~
\end{equation}
We interpret the local time risk premium, between $t$ and $T_O$, as conveying the risk premium for the strip of
volatility
uncertainty associated with $k$.

Local time risk premiums corresponding to the downside ($k <1$) can be distinct from those to the upside
($k >1$). We will show the manner in which the local time risk premium at $k=1$ associates with
the risk premium on straddles (under some mild assumptions). This analytical association is concrete for continuous semimartingales.

\noindent \textbf{Dark matter, unspanned risks, and dark matter risk premiums.} Before we present our theoretical results and explore their empirical implications, we emphasize that the dynamics of the pricing kernel
and futures return volatility may contain
both spanned and unspanned diffusive risks as well as
jump risks. In other words,
they may contain risks that are spanned by equity futures
as well as risks that are
not spanned by equity futures but may be spanned by options.
%{\color{blue} There is jump risk in the price process, which is unspanned.}

The complexity of local time and of the ``jumps crossing the strike" terms (i.e., $a_t^{T_O}[k]$,
$b_t^{T_O}[k]$, $c_t^{T_O}[k]$, and $d_t^{T_O}[k]$)
gives rise to the following definition of dark matter:
\begin{gather}
\underbrace{\mathrm{Dark~Matter}}_{\tiny(\mbox{over}~t~\mbox{to}~T_{O})}~=~
\begin{cases}
  \mathrm{D}^{d, T_O}_t[k]\equiv\underbrace{\mathbb{L}^{T_O}_t[k]}_{\tiny \mbox{local~time}} ~+~
\underbrace{c_t^{T_O}[k] + d_t^{T_O}[k],}_{\tiny \mbox{jumps~crossing~the~strike~terms~(eq.~(\ref{eq:TanakaPutCaseJumps}))}}
 & \text{for } k<1, \\
  \mathrm{D}^{\tiny \mbox{atm}, T_O}_t[1]\equiv~
\mathbb{L}^{T_O}_t[1] ~~+~~ a_t^{T_O}[1] + b_t^{T_O}[1]
%\sum_{t < \ell \leq T_O} \{\mathbbm{1}_{\{G_{\ell \, -} < 1\}} \, \max(  G_{\ell}- 1, 0 )+
% \mathbbm{1}_{\{G_{\ell  -} > 1\}} \, \max( 1 - G_{\ell}, 0 )\}
 & \text{for } k=1, \\
  \mathrm{D}^{u, T_O}_t[k] ~\equiv~ \underbrace{\mathbb{L}^{T_O}_t[k]}_{\tiny \mbox{local~time}} ~+~
\underbrace{a_t^{T_O}[k]+b_t^{T_O}[k],}_{\tiny \mbox{jumps~crossing~the~strike~terms~(eq.~(\ref{eq:TanakaJumps}))}}
 & \text{for } k>1.
\end{cases}
\label{darkk.1}
\end{gather}
Then, we can define as follows:
\begin{gather}
\underbrace{\mathrm{Dark~Matter~Risk~Premium}}_{\tiny (\mbox{over}~t~\mbox{to}~T_{O})}
~\equiv~
\begin{cases}
\mathbb{E}^{\mathbb{P}}_{t}( \mathrm{D}^{d, T_O}_t[k] ) ~-~ \mathbb{E}^{\mathbb{Q}}_{t}( \mathrm{D}^{d, T_O}_t[k] ), & \text{for } k<1,
%~~\mathrm{and}
~ \mbox{ \, \, } ~ \\
\mathbb{E}^{\mathbb{P}}_{t}( \mathrm{D}^{\tiny \mbox{atm}, T_O}_t[1] ) ~-~ \mathbb{E}^{\mathbb{Q}}_{t}( \mathrm{D}^{\tiny \mbox{atm}, T_O}_t[1] ), & \text{for } k=1,~~\mathrm{and} ~ \mbox{ \, \, } ~ \\
\mathbb{E}^{\mathbb{P}}_{t}( \mathrm{D}^{u, T_O}_t[k] ) ~-~ \mathbb{E}^{\mathbb{Q}}_{t}( \mathrm{D}^{u, T_O}_t[k] ), & \text{for } k>1. ~ \mbox{ \, } ~ \mbox{ \, \, } ~
\end{cases}
\label{darkk.rp}
\end{gather}

We note that, due to the convexity of $a_t^{T_O}[k]$, $b_t^{T_O}[k]$, $c_t^{T_O}[k]$, and $d_t^{T_O}[k]$ in
$G_\ell$, we have $\mathbb{E}^{\mathbb{Q}}_{t}(a_t^{T_O}[k])>0$,
$\mathbb{E}^{\mathbb{Q}}_{t}(b_t^{T_O}[k])>0$,
$\mathbb{E}^{\mathbb{Q}}_{t}(c_t^{T_O}[k])>0$, and
$\mathbb{E}^{\mathbb{Q}}_{t}(d_t^{T_O}[k])>0$.
%Thus, $\mathbb{E}^{\mathbb{Q}}_{t}( \mathrm{D}^{d, T_O}_t[k] )>0$ and
%$\mathbb{E}^{\mathbb{Q}}_{t}( \mathrm{D}^{u, T_O}_t[k] )>0$.

The source of risk premiums
on $a_t^{T_O}[k]$, $b_t^{T_O}[k]$, $c_t^{T_O}[k]$, and $d_t^{T_O}[k]$ is, by definition,
unspanned jump risks (one
may not be able to trade during a jump). In other words, the risk associated with jumps crossing the strike cannot be
eliminated.
%{\color{red} [We establish that the source of
%negative call premiums is unspanned risks in the pricing kernel.]--saying Below THM 1}
Now we state: \vspace{-3mm}
\setcounter{theorem}{0}
\begin{theorem}[Negative risk premiums for dark matter]
\label{claimm:claim1call_jump}
%{\color{red} [[[[[[[[ Yes, but we need to cross-check the proof because
%the statement of the theorem (as it pertains to calls -- straddles is okay) presumes that the upside risk premium is positive. ]]]]]}
The call risk premium at $k>1$ \emph{can} be negative only
if the dark matter risk premium at $k>1$, as defined in (\ref{darkk.rp}),
is negative. The
straddle risk premium is negative only if the dark matter risk premium at
$k=1$
is negative.
\vspace{-3mm}
\end{theorem}
\noindent {\bf Proof:} See Appendix~\ref{appsec:jumppps}. $\blacksquare$


Using Tanaka's formula for semimartingales (details in
Appendix~\ref{appsec:jumppps}), we derive
the following expression for
the \emph{call risk premium} (for $k>1$) as follows:
\begin{equation}
\underbrace{1 + \mu^{{T}_O}_{t,{\tiny \mathrm{call}}}[k] - e^{r ({T}_O - t)}}_{\tiny \mbox{expected~excess~return~of~calls}}
 =  \underbrace{\frac{e^{r ({T}_O - t)} }{ \mathbb{E}_{t}^{\mathbb{Q}}( \mathrm{D}^{u, T_O}_t[k] ) }}_{>0}
\{
\underbrace{ \mathbb{E}_{t}^{\mathbb{P}}( \mathrm{D}^{u, T_O}_t[k] )
~-~ \mathbb{E}_{t}^{\mathbb{Q}}( \mathrm{D}^{u, T_O}_t[k] )}_{\tiny \mbox{risk~premium~for~dark~matter}~(k>1)}
~+~
\underbrace{ \mathbb{E}_{t}^{\mathbb{P}}( \int_{t+}^{{T}_O} \mathbbm{1}_{\{G_{\ell-} > k\}} \,dG_{\ell} )}_{\tiny \mbox{upside~risk~premium}}
 \}.
\label{eq:excess_mucall1x}
\end{equation}
%sant 1/25/2022
%{\color{red} [[[[[[[[ The upside equity futures risk premium,
%$\mathbb{E}_{t}^{\mathbb{P}}( \int_{t+}^{{T}_O} \mathbbm{1}_{\{G_{\ell-} > k\}} \,dG_{\ell} )$, is presumed to be positive. ]]]]]]}
Theorem~\ref{claimm:claim1call_jump} establishes when call risk premiums can be negative. Negative
call (or straddle) risk premiums imply the relevance of unspanned risks. % and are consistent with negative dark matter risk premiums.}
%They further infer the presence of unspanned risks in the pricing kernel.}


%{\color{red} [[Specifically, if there are no unspanned risks in the pricing kernel, then the risk premiums for OTM calls must
%be positive.--We show this later before section 3.}
%LINK Appendix A to IA II and III}
%{\color{red} [[[[[ There is some poor english here. One can't start
%the next sentence with ``Thus" ?
%Also, Theorem 1 isn't showing the presence of unspanned
%risks in the pricing kernel. We infer that separately -- although
%only in DPS. This leads to a very noisy
%statement.
%]]]]] [[[
%Thus, negative
%call (or straddle) risk premiums  imply the presence of unspanned
%risks in the pricing kernel and negative dark matter risk premiums. ]]]]}
%{\color{red} [[[[[[[[[[ ]]]]]]]]]]]]]]]]]]]]]]]}
%{\color{red}[[[ The upside equity futures risk premium,
%$\mathbb{E}_{t}^{\mathbb{P}}( \int_{t+}^{{T}_O} \mathbbm{1}_{\{G_{\ell-} > k\}} \,dG_{\ell} )$, is presumed to be positive. ]]]]}


The \emph{put risk premium} (for $k<1$) can be determined (details in
Appendix~\ref{appsec:jumppps}) as follows:
\begin{equation}
\underbrace{1 + \mu^{{T}_O}_{t,{\tiny \mathrm{put}}}[k] - e^{r ({T}_O - t)}}_{\tiny \mbox{expected~excess~return~of~puts}}
 =  \underbrace{\frac{e^{r ({T}_O - t)} }{ \mathbb{E}_{t}^{\mathbb{Q}}( \mathrm{D}^{d, T_O}_t[k] ) }}_{ >0}
\{ \underbrace{ \mathbb{E}_{t}^{\mathbb{P}}( \mathrm{D}^{d, T_O}_t[k] )
~-~ \mathbb{E}_{t}^{\mathbb{Q}}( \mathrm{D}^{d, T_O}_t[k] )}_{\tiny \mbox{risk~premium~for~dark~matter}~(k<1)}
~-~ \underbrace{ \mathbb{E}_{t}^{\mathbb{P}}( \int_{t+}^{{T}_O} \mathbbm{1}_{\{G_{\ell-} < k\}} \,dG_{\ell} )}_{\tiny \mbox{downside~risk~premium}}
\}.
\label{eq:excess_muputs}
\end{equation}
If $\mathbb{E}_{t}^{\mathbb{P}}( \mathrm{D}^{d, T_O}_t[k] )
- \mathbb{E}_{t}^{\mathbb{Q}}( \mathrm{D}^{d, T_O}_t[k] ) < 0$, then
the
put risk premium is
%{\color{red} [[[[[[[[ do we need ``unequivocably"? ]]]]]]}
%unequivocably
%{\color{red} [[[[[[[[ Yes, but we need to cross-check the proof
%in the appendix because
%the statement presumes that the downside risk premium is positive. ]]]]]}
negative. This implication is empirically supported in return data of OTM
puts.
%sant 1/25/2022

%sant 1/25/2022
%{\color{red} [[[[ To the extent that (i) the futures risk premiums to
%the upside and downside are both positive and (ii) the dark matter risk premium can be negative,
%the existence and relevance of
%dark matter can be detected from
%call (or straddle) risk premiums. ]]]]]]}


\noindent \textbf{Dark matter risk premium ($k=1$) and straddle risk premium.}
In
Appendix~\ref{appsec:jumppps} (equation (\ref{eq:StraddleInterim1Jumps})),
we develop the link of the local time risk
premium (when $k=1$) and risk premium for jumps crossing the strike
(from below and above $k=1$)
to the straddle risk premium.
The latter risk premium effect can be traced
to the quantity $a_t^{T_O}[1] + b_t^{T_O}[1]=\sum_{t < \ell \leq T_O} \{\mathbbm{1}_{\{G_{\ell  -} < 1\}}  \max(  G_{\ell}- 1, 0 )+  \mathbbm{1}_{\{G_{\ell  -} > 1\}}  \max( 1 - G_{\ell}, 0 )\}$, which
represents jumps that cross $k=1$ in either direction. Importantly, the existence and relevance of
dark matter can be detected from straddle risk premiums.


\noindent \textbf{Linking dark matter risk premiums
to the risk premium on volatility uncertainty.}
To formalize this notion, suppose $\{\log \frac{F_{T_O}^{T_F}}{F_t^{T_F}}\}^2$ represents uncertainty
related to the volatility
of futures returns over $t$ to $T_O$.
Then the risk premium on dark matter
is a building block for constructing the risk premium on
volatility uncertainty.
It is seen that (Internet Appendix~(Section~\ref{appsec:dispersion})) \small
\begin{eqnarray}
\underbrace{\mathbb{E}_t^{\mathbb{P}} ( \big\{\log \frac{F_{T_O}^{T_F}}{F_t^{T_F}}\big\}^2 )
- \mathbb{E}_t^{\mathbb{Q}} ( \big\{\log \frac{F_{T_O}^{T_F}}{F_t^{T_F}}\big\}^2 )}_{\mathrm{risk~premium~on~squared~log~contract}}
& = & -\mathrm{e}_t^{\mathbb{P}} ~+~ \int\limits_{0}^\infty ~ \omega[k] \, \underbrace{\{ \mathbb{E}_t^{\mathbb{P}} ( \mathbb{L}^{T_O}_t[k] )-\mathbb{E}_t^{\mathbb{Q}} ( \mathbb{L}^{T_O}_t[k] )\}}_{\mathrm{risk~premium~for~local~time}}\,dk~\mbox{ \, \, \, } \nonumber \\
&+& \int\limits_{0}^1 \omega[k]\, \underbrace{ \{
\mathbb{E}_t^{\mathbb{P}}( c_t^{T_O}[k] + d_t^{T_O}[k] )
- \mathbb{E}_t^{\mathbb{Q}}( c_t^{T_O}[k] + d_t^{T_O}[k] ) \}}_{\mathrm{risk~premium~for~jumps~crossing~the~strike}~(k<1)} \, dk  \nonumber \\
&+& \int\limits_{1}^\infty \omega[k]\, \underbrace{\{
\mathbb{E}_t^{\mathbb{P}}( a_t^{T_O}[k]+ b_t^{T_O}[k] )
- \mathbb{E}_t^{\mathbb{Q}}( a_t^{T_O}[k]+ b_t^{T_O}[k]) \}}_{\mathrm{risk~premium~for~jumps~crossing~the~strike}~(k>1)} \, dk,
~~\mbox{ \, \, }\label{eq:LogFuturesSqasb11InResult}  \nonumber\\
&&~~~~\mbox{with}~ \omega[k] ~ \equiv ~ \frac{2}{k^2} ( 1 - \log k ),
~~\mbox{ \, \, }
~\mbox{ \, \, }
~~\mbox{ \, }
\end{eqnarray}  \normalsize
where $\mathrm{e}_t^{\mathbb{P}}$ has the economic interpretation of the expected total gain/loss, over $t$ to $T_O$, from
a  dynamic equity futures trading strategy (details in Internet Appendix~\ref{appsec:dispersion} (equation (\ref{eq:DefMuP}))).

\noindent \textbf{Absence of unspanned risks in the pricing kernel and a continuous semimartingale model setting
with stochastic return volatility.}
The final question is: Is it possible to obtain negative risk premiums for OTM calls if there
are unspanned diffusive risks in volatility dynamics but not in the pricing kernel?
This continuous semimartingale
environment is revealing for two reasons. First, the jumps crossing the strike terms --- $a_t^{T_O}[k]$, $b_t^{T_O}[k]$, $c_t^{T_O}[k]$, and $d_t^{T_O}[k]$ --- \emph{vanish}. Second, one can delineate the distinction between spanned and unspanned \emph{diffusive} risks.
%Third, the risk premium adjustments that link $\mathbb{P}$ to $\mathbb{Q}$ are explicit through Girsanov's change of measure theorem.

Reconciling intuition,
we establish the takeaway that
OTM call option risk premiums will be positive if
there are no unspanned risks in
the pricing kernel.\footnote{This analysis is presented in
Internet Appendix (Section~\ref{app:LTRPSpannedDiffusiveVolRisk}) to save on space.}
The model studied in Section~\ref{sec:furtherrr} ascribes
clear-cut roles for spanned and unspanned risks,
and we show that unspanned risks can generate negative local time risk premiums and negative risk premiums of
calls and straddles.
\vspace{-4mm}


%{\color{blue} The distinct role played by unspanned risks are developed in the parametric model of Section~\ref{sec:furtherrr}.}


%The {\color{blue} model in
%Section~\ref{sec:furtherrr}}
%%{\color{red}[[[[[ This seems rather poor style. The model does not show it as such --- we are showing it. ]]]] [[[
%%contributes by showing ]]]]]]]]}
%{\color{blue} is illuminating because we show}
%that {\color{blue} unspanned risks can generate} negative local time risk premiums and negative risk premiums of
%%OTM
%calls and straddles. %{\color{blue} The distinct role played by unspanned risks are developed in the parametric model of Section~\ref{sec:furtherrr}.}
%{\color{blue} These links are elaborated in the context of a model with spanned and unspanned risks in Section~\ref{sec:furtherrr}.}




%\noindent \textbf{A parametric model with stochastic volatility and jumps.}
%Whilst our modeling framework is non-parametric,
%it is still illuminating to consider a
%parametric model which we do in the Internet Appendix.
%There, we consider a
%parametric model with stochastic variance and jumps
%in both the price process and
%variance process. This parametric model allows
%one to see the various moving parts of our
%general theory. First, one can delineate the distinction between spanned and unspanned
%\emph{diffusive} risks. \textbf{Second, the call option risk premium must be positive for $k>1$ if there are no unspanned risks in the pricing kernel.}
%Third, if jumps in the price process
%are absent, the jumps crossing the strike terms --- $a_t^{T_O}[k]$, $b_t^{T_O}[k]$, $c_t^{T_O}[k]$, and $d_t^{T_O}[k]$ --- \emph{vanish}.


% can be traced to negative local time risk premiums due to .}

%Third, the local
%time risk premium is economically and conceptually \emph{distinct} from
%the variance risk premium.}




%{\color{red} [[[[  This next sentence is in the wrong place. It interrupts the flow of this paragraph. I moved it to the
%previous page. ]]]] [[[[
%In the setting of continuous semimartingales, one has $a_t^{T_O}[k] = b_t^{T_O}[k] \equiv  0$. ]]]}
%{\color{red} [[[[ With continuous semimartingales,  ]]]]]]]}
%{\color{red}[[[[[ This read awkwardly to me. ]]]] [[[[[[ The further takeaway is that if unspanned diffusive risks in return volatility were absent, or are unpriced,
%%and the local time risk premium were to be negligible [zero]
%then the risk premiums for OTM calls would be positive. ]]]]]}
%%This is shown in Corollary~\ref{claimm:claim1call} (Internet Appendix (Section~\ref{gggs})).}

%{\color{magenta} Next we examine the empirical implications of the derived option risk premium decomposition into the risk compensation for
%dark matter -- terms related to local time and for jumps across the strike.} % Dark matter is a key component of the option premium.}

%%%%%%%%%%%%%%%%%%%%%%%%%%%%%%%%%%%%%%  empirical section %%%%%%%%%%%%%%%%%%%%%%%%%%%%%%%%%%
%%%%***************************

%%%%%%%%%%%%%%%%%%%%%%%%%%%%%%%%%%%%%%%%%%%%%%%%%%%%%%%%%%%%%%%%%%%%%%%%%%%%%%%%%%%%%%%%%%%%%%%%%%%%%%%%%%%%%%

%%%%%%%%%%%%%%%%%%%%%%%%%%%%%%%%%%%%%%  empirical section %%%%%%%%%%%%%%%%%%%%%%%%%%%%%%%%%%
\section{Supportive empirical evidence on dark matter}
\label{sec:EmpiricalDataSupportUnspannedIndexVol}
%{\color{blue} By definition, one may not be able to trade during the jump.}

%{\color{red}[If volatility risks were to be spanned
%and if risk premiums for jumps crossing the strike were to be absent,
%OTM call risk premiums would be positive (assuming that
%the
%risk premium to the upside, $\mathbb{E}_{t}^{\mathbb{P}}( \int_{t+}^{{T}_O} \mathbbm{1}_{\{G_{\ell-} > k\}} \,dG_{\ell} )$, is positive).
%On the other hand, suppose volatility risks are unspanned and
%there is potential for
%jumps crossing the strike, that is, the dark matter is relevant, then this may give rise to negative
%call risk premiums, as suggested by our theory.]
%}

%{\color{red}[[[ I still worry about this. Why not say
%``Our theory shows that if there were no unspanned
%risks in the pricing kernel, OTM call risk premiums
%would be positive ....... ". ]]]]]}


%{\color{red} [[[[[[[ Our theory shows that if risk premiums for jumps crossing the strike were absent and {\color{red} if
%{\color{red}[[[[ volatility ]]]]}
%underlying risks were spanned,}
%OTM call risk premiums would be positive (assuming that the upside equity risk premium, $\mathbb{E}_{t}^{\mathbb{P}}( \int_{t+}^{{T}_O} \mathbbm{1}_{\{G_{\ell-} > k\}} \,dG_{\ell} )$, is positive).
%On the other hand, ]]]]]]]}
Suppose there is potential for jumps crossing the strike
and
%{\color{red}[[[[ We suddenly change our language?
%Previously, we emphasized the issue of whether there unspanned risks or
%no unspanned risks in the pricing kernel. However, now it is
%about ``volatility risks are unspanned". This is not correct.
%One could have volatility risks that are unspanned by the futures
%but if these risks are not priced (i.e., contained in the pricing kernel), they can't contribute to
%dark matter risk premiums. ]]]]}
%{\color{red} [[[[[ volatility risks are unspanned, ]]]]}
the pricing kernel contains risks that are not spanned by equity futures but do
correlate with risks that intersect local time,
then this attribute may give rise to negative call
option
risk premiums. Such a feature speaks to the relevance of dark matter.

The risk premium on dark matter is implied to
be negative if the straddle risk premium
is negative or if the call option risk premiums are negative at some $k>1$.
Our goal is to detect dark matter and probe its workings.

% expose

%We wish to provide evidence that dark matter is relevant.

%For our purposes,
\noindent \textbf{A. Implication-rich weeklys (short-dated
options).}
Notably, weeklys
%(with an average maturity of eight days)
are considered gamma plays, whereas long-dated options are vega plays.
%Prices of
%very short expiration options react to
%fluctuations
%in the underlier
%far more strongly
%than those with farther expirations.
With no more than eight days to maturity,
%expiration,
the delta
of such options can move quickly along directional movement.

%{\color{red}[[[[[[[ --REMOVE The aspect of option maturity {\color{red}[can]} offers {\color{red}[key]} distinctions to assess ingredients of our
%theory and guides
%%%steers
%our empirical emphasis on \emph{weeklies.} ]]]]]]]]}

In conjunction with shrinking
time value for weeklys,
the insight to exploit
is that the source of dark matter risk premiums is
predominantly risk premiums for jumps crossing the strike, pertinently so for deep OTM options.
Complementing this channel, straddle risk premiums are linked to risk  premiums on price
jumps without regard to
their direction.


%For straddles on weeklies, the effect has a slightly different interpretation:
%Straddle risk premiums are
%linked to risk  premiums on
%price jumps
%without regard of their direction.
%Our arguments are intuitive as trading activity in weeklies
%often spikes over high impact windows.
%PAJC

%(for example, major macro-economic data announcements
%or FOMC meetings).} %events.

\noindent \textbf{B. Framing the theoretical predictions.} Our theory allows us to
formulate the following predictions about equity option risk premiums: \vspace{-2mm}
\begin{enumerate}

\item[\textbf{H1.}] \textbf{No unspanned risks hypothesis.}
If there are no unspanned risks in the pricing kernel,
%{\color{red}[[[[[[[ the dark matter risk premium (for all $k$)
%is zero, ]]]]]]]]]}
the risk premium of OTM calls is  \emph{positive} and the risk premium of
straddles is \emph{zero}.
%%%%%%%%%%%%%%%%%%% detached

\item[\textbf{H2.}] \textbf{Negative risk premiums for \emph{jumps} crossing the strike hypothesis for
\emph{short-dated} options.}
Deep OTM weekly
options exhibit negative risk premiums, in line with negative risk premiums for jumps
crossing the strike.
%{\color{red} [[[[ A negative risk premium for jumps crossing the strike for
%weekly options with strike closest to the current price
%%{\color{red}[[[[ But this is not quite what the asymptotic theory points to? ]]]}
%is also imputed if the straddle risk
%premium is negative. ]]]]}

%{\color{blue} Weekly options with strikes closest to the current price exhibit strong sensitivity to return jumps.}}
%more sensitive {\color{blue} to equity  price deviations.}}

            %PAJC
%{\color{red}[[[[ I am not sure I get this? For ATM, local time risk premium may be of comparable size.
%Unless I missed
%something?..... ]]]]}

\item[\textbf{H3.}] \textbf{Negative risk premiums on dark matter hypothesis.} If there are unspanned risks,
%{\color{red}[and, hence, dark matter,]}
the risk premium on dark matter
(for moneyness $k$) can be negative. Then, the risk premiums of straddles and OTM calls can be \emph{negative}.


\end{enumerate}
We examine these predictions using option returns computed over expiration cycles.
Our focus is on option maturities that are actively traded: weeklys
(eight days), 28 days, and 88 days.

\noindent \textbf{C. Excess returns of weeklys}. Weekly options are instrumental
in identifying
and isolating
the jumps crossing the strike component of dark matter.
Motivated by questions concerning our hypotheses, we first construct the time-series of excess returns of options on the
S\&P 500 index over the \emph{weekly} expiration cycles.

Specifically, for $T_{O}-t=$ 8 days (on average), and setting $k= \frac{K}{S_t}$, %{\color{red} [[[[[ What is this in green? Change of notation? ]]]}
\begin{equation}
%{\color{green} \mathbbm{r}_{t,{\tiny \mathrm{call}} }^{{T}_O}[k] }
{q_{t,{\tiny \mathrm{call}} }^{{T}_O}[k]} ~=~ \frac{ \max( S_{{T}_O} - k \,S_t,0)}{\mathrm{call}_{t}[k \,S_{t}]}~-~e^{r(T_{O}-t)},
~~~\mathrm{where}~\log(k)~\mathrm{is}~1\%, 2\%,~\mathrm{and}~3\%~\mathrm{OTM},
\end{equation}
and $\mathrm{call}_{t}[k \, S_{t}]$ is the
ask price of an OTM call with strike $K=k \, S_{t}$.
Anchoring our discussions,
the selected
$\log(k)$
are allied
to a delta of 27, 12, and  6 (in \%, likewise for puts).
The  straddle excess return is
\begin{equation}
q_{t,{\tiny \mathrm{straddle}} }^{{T}_O} ~=~\frac{\max( S_t - S_{{T}_O},0) ~+~ \max(S_{{T}_O} - S_t,0)}{
\mathrm{put}_{t}[S_{t}]
~+~ \mathrm{call}_{t}[S_{t}]  }~-~e^{r(T_{O}-t)},
\end{equation}
where $\mathrm{put}_{t}[S_{t}]$ is the ask price of an at-the-money (ATM) put with strike $K= S_{t}$.

Weekly options initiate on a Thursday and expire on the Friday of the
following week.
The first (final) expiration cycle is 1/13/2011
(12/20/2018). Hence, our analysis covers 415 weekly expiration cycles. These weekly options are
associated with sizable open interest and volume.



%(we focus on options with maturity of (on average) 28 days and 88 days).} \vspace{2mm}
%We affirm, for robustness, our findings using options on the S\&P 500 index.} \vspace{2mm}

%Andersen et al. (2017, JF, p 1350, equation 5) argue that, for very short-dated options (such as weekly),
%the variance of the continuous component and the jump intensity rate of the discontinuous component of futures returns will vary by very little over the life of the option and thus, in particular,
%the variance of the continuous component can be treated Andersen et al. (2017, JF, p 1350) as a constant over very short time horizons. Furthermore, for very short-dated OTM options,
%the futures price will only very, very rarely trade at the level of the strike.
%With both these considerations in mind, we form the hypothesis that
%the local time risk premium for deep OTM options may be approximately zero.
%Thus, the risk premiums of deep OTM options will consist of two terms:
%First, the upside (respectively, downside) risk premium for calls (puts) and
%second, the risk premium for jumps crossing the strike.
%For very short-dated OTM options, the first term is also likely to be not large (in absolute value).
%This permits us to formulate the following prediction:
%


\noindent \textbf{D. Drawing inferences from empirical measures of
option risk premiums.}
Our theoretical results pertain to the expectation of option returns conditional on the
filtration
$\mathcal{F}_t$; that is, $\mathbb{E}^{\mathbb{P}}_{t}( \bullet ) =
\mathbb{E}^{\mathbb{P}}( \bullet | \mathcal{F}_t )$. To measure this
object empirically,
we construct
average excess option returns (over expiration cycles)
conditional on $\{{\cal F}_t \in \mathfrak{s}\}$, for some
variable $\mathfrak{s}$.

We are guided by the implication that historically generated excess returns conform with ex-ante
expected excess returns.
Our criteria for $\mathfrak{s}$ are that they connect to
time $t$ information
tracked by market participants.
Each $\mathfrak{s}$ is  arranged so as to be in one of the following three categories:
\begin{gather}
{\cal F}_t \in \mathfrak{s}=
\begin{cases}
\mathfrak{s}_{\tiny \mbox{bad}}& \mathrm{(when~the~equity~premium~is~presumably~high)}, \\
\mathfrak{s}_{\tiny \mbox{normal}}&\mathrm{(when~the~equity~premium~is~presumably~normal)},~~\mathrm{and} \\
\mathfrak{s}_{\tiny \mbox{good}}& \mathrm{(when~the~equity~premium~is~presumably~low)}. \\
\end{cases}
\label{filt.1}
\end{gather}
Thus, we draw inferences based on partitioned average excess option returns.
Pertinent to our exercise for \emph{weekly option returns}, we
consider the following variables
to surrogate $\mathfrak{s}$:

\begin{enumerate}

\item \textbf{Change in the Weekly Economic Index}$_{t}$. Reflects the weekly innovation in the WEI
index (source: New York Fed). A decline indicates a weakening economy.


\item \textbf{Quadratic Variation}$_{t}$. Sum of daily squared (log) returns over
the \emph{prior} expiration cycle (eight days). A high $\mathrm{QV}_{t}$ corresponds
to unfavorable economic states.

\item \textbf{Risk Reversal}$_t$. The negative skew, reflected in $\log(\frac{\mathrm{IV_t^{\tiny \mbox{put}}}[k]}{\mathrm{IV_t^{\tiny \mbox{call}}}[k]}$), mirrors
    downside protection concerns. The implied volatility (IV$_{t}$) for puts (calls) uses $\log(k)$ equal
    to $-2\%$ (2\%).

\item \textbf{Change in Volatility}$_{t}$ ($\log(\frac{\mathrm{IV}^{\tiny \mbox{atm}}_{t}}{\mathrm{IV}^{\tiny \mbox{atm}}_{t-1}})$).
A positive change in ATM implied volatility, over the prior expiration cycle, coincides
with rising market uncertainty (and wary investors). The implied volatility is the average across
ATM puts and calls of weekly options.


%\item \textbf{Dividend Yield}$_{t}$: A high
%%value of
%dividend yield aligns with bad
%%economic
%states and high expected excess returns (e.g.,
%\citet*{Cochrane:2008RFS}).
%The data is from the web site of Robert Shiller.

\item \textbf{Recent Market}$_{t}$: Log relative of the S\&P 500 index over the prior
expiration cycle.
 \vspace{-3mm}

%\item \textbf{Butterfly Spread}$_t$. The flatness of average implied volatility of OTM puts and calls relative to average ATM puts and calls
%is consistent with calmer economic outlook.  We measure this
%{\color{blue}variable as} $\log(\frac{\mathrm{average~put~and~call~IV}_{t}[k\%~\mathrm{OTM}]}{\mathrm{average~ATM~put~and~call~IV}_{t}})$.


%\item \textbf{Jump Fear}$_t$.
%{\color{blue}The} log ratio of {\color{blue}the} 3\% (delta of -6\%) OTM put price divided by {\color{blue}the} index price (\citet*{Bollerslev_todorov:JF_2011}). This variable is high during {\color{blue} bearish} states.

%\item \textbf{Yield~Spread}$_{t}$: Difference between the 30-year and 1-year Treasury yields at
%the start of the expiration cycle (source: CRSP (daily) Fixed Term Indices Files).
%Flattening of the yield curve conveys {\color{blue}a deteriorating economic environment and concerns about discounting.}

%Jump Fear ($-\log(\frac{\mathrm{put}^{\mathrm{futures}}_{t}[F_t^{T_{F}} e^{-0.05}]}{F_t^{T_{F}}}$): This is the jump fear variable of
%    \citet*{Bollerslev_todorov:JF_2011} such that low values of
%    $-\log(\frac{\mathrm{put}^{\mathrm{futures}}_{t}[F_t^{T_{F}} e^{-0.05}]}{F_t^{T_{F}}})$ reflects bad economic states.

\end{enumerate}

Our rationale for considering these variables is that they may be correlated with subsequent variation in
equity premiums
and may influence dark matter risk premiums.

\noindent \textbf{E. Support for our predictions about dark matter from \emph{weeklys}}.
We consider a regression framework, where excess returns of calls
is the dependent variable (likewise for
straddles and
puts), as follows:
\begin{equation}
q_{t,{\tiny \mathrm{call}} }^{{T}_O}[k]
=
\underbrace{\mu_{\{ {\cal F}_{t} \in \mathfrak{s}_{\tiny \mbox{bad}} \} } \mathbbm{1}_{\{ {\cal F}_{t} \in\mathfrak{s}_{\tiny \mbox{bad}} \}}
+ \mu_{\{ {\cal F}_{t} \in \mathfrak{s}_{\tiny \mbox{normal}} \} }
\mathbbm{1}_{\{ {\cal F}_{t} \in\mathfrak{s}_{\tiny \mbox{normal}} \}}
+ \mu_{\{ {\cal F}_{t} \in \mathfrak{s}_{\tiny \mbox{good}} \} }
\mathbbm{1}_{\{ {\cal F}_{t} \in\mathfrak{s}_{\tiny \mbox{good}} \}}}_{\tiny \mbox{Dichotomizing~expected~excess~returns~across~economic~states}} +
\underbrace{\epsilon_{T_{O}}.}_{\tiny \mbox{error~term}}
\label{eq:regress_three_states}
\end{equation}

Table~\ref{tab:weekly} reports
the
estimates of partitioned average excess returns of puts, straddles, and calls,
without
making
distributional assumptions about $\epsilon_{T_{O}}$.
For instance, $\mu_{\{ {\cal F}_{t} \in \mathfrak{s}_{\tiny \mbox{bad}} \} }$ reflects the call
risk premium in bad economic states, which, in turn, tends to be associated with higher equity
premiums.
The
%specification (\ref{eq:regress_three_states}) dichotomizes return behavior across economic states, whereas
presence of $\epsilon_{T_{O}}$
%merely
recognizes the departures between observed option excess
returns and ex-ante expected option excess returns.


The superscripts ***, **, and * on estimates indicate statistical significance
at 1\%, 5\%, and 10\%, respectively. We rely on the HAC estimator
of \citet*{NeweyWest:87} with the lag
selected automatically. The reported partitioned average weekly
option returns are \emph{not} annualized.

Having laid the groundwork, we have hypothesized
that the local time component of the dark matter risk premium for $k<1$ and $k>1$
will be negligible in the case of weeklys. This is because, for small $T_{O}-t$, concerns about jump risks outweigh concerns about
volatility risks.\footnote{The size of the local time risk premiums for very short horizon options can also be understood
from the standpoint of
\citet*{Andersen_Fusari_Todorov:2017JFweekly}.
They suggest
the possibility that the variance of the continuous
component of equity returns is effectively almost constant over small $T_{O}-t$.}

Mindful of these
considerations, \emph{for~small}~$T_O - t$, we, hence, posit %PAJC
\begin{gather*}
\underbrace{\mathrm{Dark~Matter}}_{\tiny \mathrm{for~weeklys}}~\approx \sum_{t < \ell \leq T_O}
\begin{cases}
\mathbbm{1}_{\{G_{\ell  -} \geq k\}}  \max( k - G_{\ell}, 0 ) +
\mathbbm{1}_{\{G_{\ell  -} < k\}}  \max(  G_{\ell}- k, 0 )
&~~~~\text{puts}, k<1 \\
\mathbbm{1}_{\{G_{\ell  -} \leq k\}} \max( G_{\ell} - k, 0 ) +
\mathbbm{1}_{\{G_{\ell -} > k\}} \max( k - G_{\ell}, 0 )
 &~~~~\text{calls},~k>1.
\end{cases}
\label{emp.1}
\end{gather*}

Viewed through the prism of our theory, what do the weekly options data tell us? The empirical pattern
that emerges from Table~\ref{tab:weekly} is fourfold.
First,
the
partitioned
average excess returns of
straddles are negative (14 out of 15 estimates).
The weekly straddle return is $-10\%$ unconditionally.

Second, the partitioned average excess returns of
3\% OTM calls are
%all
negative.  Consistent with our predictions, the negative effect of the risk premium for
$\sum_{t < \ell \leq T_O} \{ \mathbbm{1}_{\{G_{\ell  -} \leq k\}} \max( G_{\ell} - k, 0 ) +
\mathbbm{1}_{\{G_{\ell -} > k\}} \max( k - G_{\ell}, 0 )\}$
dominates
the
%{\color{red}[[[[[[[[ positive ]]]]]]]]} %sant 1/25/2022
effect of
$\mathbb{E}_{t}^{\mathbb{P}}( \int_{t+}^{{T}_O} \mathbbm{1}_{\{G_{\ell-} > k\}}\, dG_{\ell} )$ at
high $k>1$.
%(low delta).
The OTM call excess return is $-59\%$ unconditionally. The upshot from
the model-derived restrictions is that
the risk premiums for jumps crossing the strike are implied to
be negative at high $k>1$.

Third, the
difference in the partitioned average excess returns  of 3\% and 1\% OTM calls
are significantly negative. Our bootstrap-based exercise (Table~\ref{tab:bootstrap} (Panel A)) furnishes a finding that
the associated 95\% lower and upper confidence intervals do not tend to bracket zero.

Fourth, all estimates of partitioned average excess returns of OTM puts are negative.
The unconditional return
of $-59\%$ for 3\% OTM call, as opposed to $-61\%$ for 3\% OTM put, with the same
absolute delta, is revealing. Based on the 95\% bootstrap confidence intervals shown in
Table~\ref{tab:bootstrap} (Panel B), the risk premium for jumps crossing the strike for
$k>1$ (i.e., on the upside) is statistically
at par with that for $k<1$ (i.e., on the downside).
This finding stands out across three bootstrap procedures (IID, stationary, and circular block)
that we employ to safeguard inference.


Our Theorem~\ref{claimm:claim1call_jump}, in conjunction with the analytical link
in equation
(\ref{eq:StraddleInterim1Jumps}), \emph{for~small}~$T_O-t$, can
be considered
as a form of
specification test for the absence of unspanned risks. This is because of the correspondence between
the straddle risk premium and the risk premium for jumps crossing the strike
and local time. Stated differently,
the negative partitioned average weekly excess straddle returns mimic the sign and magnitude of the
dark matter risk premium at $k=1$.\footnote{The dichotomy
observed between partitioned average excess call returns for $\mathfrak{s}_{\tiny \mbox{bad}}$ and
$\mathfrak{s}_{\tiny \mbox{good}}$ can be understood in the context of our theory. To be specific, $\mathfrak{s}_{\tiny \mbox{bad}}$
may reflect \emph{high} prevailing $\mathbb{E}_{t}^{\mathbb{P}}( \int_{t+}^{{T}_O} \mathbbm{1}_{\{G_{\ell-} > k\}}\, dG_{\ell} )$, which
translates into positive partitioned average excess returns for
1\% OTM calls. This
dimension
may further help to explain the outcome that partitioned average excess call returns for $\mathfrak{s}_{\tiny \mbox{bad}}$ are
typically higher compared to those in $\mathfrak{s}_{\tiny \mbox{good}}$.}


%The merit of our theory is that it is amenable to assessment at multiple horizons. While it is
%plausible
%that $\mathbb{E}_{t}^{\mathbb{P}}( \int_{t}^{{T}_O} \mathbbm{1}_{\{G_{\ell} > k\}} \,dG_{\ell} )$ is magnified at
%longer $T_O$ (due to the
%long-term upward drift in equity prices),
%it is also plausible that the investor's dislike for volatility may be
%exacerbated as well. % (for example, VIX futures are mostly in contango).
%Beyond the prediction about the cross-section of moneyness, our theory, thus,
%generates an
%implication about the nature of local time risk premiums beyond the 28-day
%horizon.

%options {\color{blue} reinforce } our theory.}

Reinforcing the view that jumps crossing
%across
the strike are a pertinent component of dark matter, we report the returns of crash-neutral
straddles %(e.g., \citet*{CovalShumway:2001})
%and \citet*{Driessen:2009ROF})
in Table~\ref{tab:weekly} (final column).
%In this calculation, we allow for CBOE collateral requirements.
Our treatment of the short put position
accounts for the posting of required collateral as per \citet*[page 22]{CBOE:2000}.
The salient finding is that average returns of crash-neutral straddles are small (and close to zero).
This outcome
supports a view that the risk premium for the jumps crossing
the strike component of shorting puts
%{\color{red}[[[[[[[[[[ counteracts ]]]]]]]]}
balances out
%exerts an offsetting influence on
the negative risk premium component for
long straddle positions.
%position in straddles.the risk premium for jumps across the strike (when $k<1$)
%
%are a countervailing source of the
%dark matter risk premium of long straddle positions at short-horizons.}

%{\color{red} [[[ This explanation doesn't account for the possibility
%that straddles have negative risk premiums due to
%negative local time risk premiums while
%deep OTM options have negative risk premiums
%due to jumps across the strike? Might we be better writing:
%
%{\color{green} The average returns of crash-neutral straddles are close to zero, imparting the insight
%that the risk premium for jumps across the strike on the downside are a key source of the dark matter risk premium at short-horizons.}
%]]]]}
%effectively offset the risk premiums
%for jumps across the strike on
%the downside.}  %the is consistently negative across the relevant strikes.}


%%%%%%%%%%%%%%%%%%%% %ORchange%
The
negative average option excess returns for ultra-short maturities
further corroborate the relevance of jumps crossing the strike
(as noted in Internet Appendix (Table~\ref{tab:2and3day})).
These maturities of two- and three-day
manifest option prices that are higher
than the minimum tick size and
reflect positive likelihood of expiring in-the-money (i.e.,
$\mathbbm{1}_{\{ q_{t, {T}_O} >0 \}}$).
%\footnote{This evidence is reported in Internet Appendix (Table~\ref{tab:2and3day}).}
In sum, our evidence highlights hurdles
facing option models looking to match the behavior of ultra-short maturity option payoffs under both
$\mathbb{P}$ and $\mathbb{Q}$.

%Since many option pricing models also face hurdles fitting option data with very short
%maturity, Table~\ref{tab:2and3day} affirms that average option returns, computed over 3 days, all tend to be negative. These
%sizable nature of the risk premiums --- which can be entirely attributable due to jumps across the strike --- contradict option
%models that may produce small risk premiums for short-dated options.}


%Finally,
In what
ways
could
liquidity
considerations, margin requirements, and heterogeneous trading
contribute to outcomes of
negative returns of deep OTM calls? We address this issue from
three %concrete
angles.
First, alleviating concerns that lack of liquidity may overly influence option returns, we
report (i) open interest and (ii) trading volume in Table~\ref{tab:weekly}
(and also Tables~\ref{tab:equity_options},
\ref{tab:table2}, and \ref{tab:opretspfutx}). In line with \citet*{Muravyev_Pearson:RFS2020},
deep OTM options do not appear to come with sharply lower open interest or
thin trading volume.

Second, we
consider OTM
calls with as small as
\textit{1 delta} and these strikes
maintain
positive
%elicit noticeable
%discernible
%commensurately adequate
%meaningful
open interest and trading volume.\footnote{This feature is noted in Internet Appendix~(Table~\ref{tab:deep_weekly}).}
%Summarizing,
%{\color{red} [[[[ The call risk premiums becoming more negative at higher strikes appears to be a robust data property. ]]]]}
Our evidence indicates
that call option risk premiums are
negative
at progressively higher strikes.

%appears to be a data property.}
%The negative risk premium for calls that appears to be a data property.}
%{\color{blue} In sum, it is a {\color{red}[robust]} data property that call risk premiums become more negative at higher strikes.}
%%appears to be a data property.}
%The negative risk premium for calls that appears to be a data property.}

Third,  it is plausible that bid-ask spreads widen
%are pressured
%during times of
%market distress. In this situation, certain
when market participants are adversely exposed to large price jumps.
Taking cues from \citet*{Christoffersen_Goyenko_Jacobs_Karoui:RFS2018},
we recompute option returns
using the midpoint of bid and ask prices.
The pattern of negative returns to buying deep OTM call options
remains qualitatively unchanged.\footnote{We display this evidence in Internet Appendix
(Table~\ref{tab:bid-ask}).}

\noindent \textbf{F. Composition of dark matter from farther-dated options.} To study
the nature of
dark matter risk premiums,
we examine
evidence from farther-dated options, with
${T}_O-t$ equal to 28 and 88 days (on average).
%\footnote{{\color{red}[[[[[[ --IN NOTES [[The first (final) expiration cycle for
%28-day futures options is 01/18/1988 (05/23/2016), with 341 %expiration
%cycles. These
%one-month futures options were discontinued and only the three-month
%options were traded after that. The first (final) expiration cycle
%for the 88-day futures options is 03/21/1988 (03/18/2019), while for 28-day options on the S\&P 500
%it is 01/22/1990 (12/24/2018).
%The history of expiration cycles
%encompasses a variety of market conditions. ]]]]]}}
%{\color{blue} Farther-dated options can provide cues to the viability of local time risk premiums because of lessened concerns
%associated with any imminent jumps crossing the strike.}
Farther-dated options can highlight
the relevance
of local time risk premiums because concerns
associated with jumps crossing the strike may be, relatively speaking, lessened.


Tables \ref{tab:equity_options}, \ref{tab:table2},  and  \ref{tab:opretspfutx} uncover negative partitioned average
excess returns of straddles. The uniformly negative estimates, in particular, for 88-day options, attests to the
notion of negative local time risk premiums.
Essentially, this exercise identifies the dark matter risk premium (for $k=1$) as being negative and
significant.
Our findings are an acknowledgment of
a viewpoint that aversion to unspanned risks
is implied within farther-dated options.

Consistent with our theoretical predictions, the negative effect of $\mathbb{E}_{t}^{\mathbb{P}}( \mathrm{D}^{u,T_O}_t[k] ) - \mathbb{E}_{t}^{\mathbb{Q}}( \mathrm{D}^{u, T_O}_t[k] )$ overcomes
the
%{\color{red}[[[[[[[[[[ positive ]]]]]]]}  %sant 1/25/2022
effect of $\mathbb{E}_{t}^{\mathbb{P}}( \int_{t+}^{{T}_O} \mathbbm{1}_{\{G_{\ell-} > k\}}\, dG_{\ell} )$ at high $k>1$.
Specifically, based on Tables \ref{tab:equity_options}
and \ref{tab:table2} --- which cover 28-day options --- we garner that risk premiums for
5\% OTM calls exhibit partitioned average excess returns that are negative in 21 out of 30 entries.



Complementary evidence comes from Table~\ref{tab:opretspfutx}, which covers 88-day options, and shows that
partitioned average excess returns of 12\% (i.e.,
6 delta) OTM calls are  negative.
These estimates are statistically significant in 10 out of 15 entries. Additionally, the unconditional call
risk premiums get more negative deeper OTM (i.e., going from 32 to
6
delta). This outcome reflects
the interaction
%interplay
between dark matter risk premiums --- which may get more negative with higher $k>1$ ---
and upside equity risk premiums.
%{\color{red}[[[[[[[ which apparently become less positive with higher $k>1$. ]]]]]]}}
%sant 1/25/2022

Our theoretical results were designed in terms of equity futures, and the expected returns of their options, to exploit the
analytical convenience of the property that the futures price is a martingale under the $\mathbb{Q}$-measure.
This
aspect
is not essential, as noted in the context of Tables \ref{tab:equity_options} and~\ref{tab:table2}
and
because $F_{T_{O}}^{T_{O}}=S_{T_{O}}$.
First, there is agreement on negative straddle risk premiums and negative risk premiums for calls 5\% OTM. Second,
the evidence for negative put risk premiums is mutually consistent. Taken together, our
evidence favors dark matter risk premiums that tend to be more pronounced at both low $k<1$ and high $k>1$.

\noindent \textbf{G. Reconciling the various pieces of evidence and our hypotheses.}
The implication from straddle risk premiums
across the three maturities is that one can reject
the ``No unspanned risks" hypothesis.
Also, essential is
the data outcome
that the partitioned average excess returns of calls,
which depict call risk premiums, are negative at high $k>1$,
which is indicative of dark matter.

What is the foundation
of these findings? Connecting to
equation (\ref{eq:excess_mucall1x}),
$\mathbb{E}_{t}^{\mathbb{P}}( \int_{t+}^{{T}_O} \mathbbm{1}_{\{G_{\ell-} > k\}} \,dG_{\ell} )$
is likely to be small at higher $k$
and is conceivably dominated by the magnitude of the dark matter risk premium.
Accompanying
these effects across option maturities,
the straddle risk premiums being negative is
a further indication that  dark matter is relevant. %active.
The negative dark matter risk premiums --- imputed from traded options ---
support our theory that there are unspanned risks  and that they are economically pertinent.

Our theoretical predictions are free of parametric
assumptions about the evolution of the pricing kernel, price jumps, and return volatility.
Dark matter is needed to explain the behavior of call option risk premiums.
Although
we do not observe dark
matter directly, we are able to detect the workings of dark matter, and its risk premium, in the turning point of the call risk premiums
computed at rising $k$, as reflected in partitioned average excess return of calls \emph{switching sign} from positive to negative.

The
consequences
of our approach are
compatible with unspanned volatility risks
being disliked and jumps crossing the strike being disliked. The latter finding is informed by our
evidence from the weeklys. It is with the OTM weeklys that we can decouple the effects of jumps crossing the strike
from local time. These effects would otherwise be blended
within dark matter.



%
%we find that trading activity is not substantially lower for
%deep OTM options.
%
%Could liquidity considerations and heterogeneous agent trading impact the observed average
%option returns, particularly the deep OTM calls? We address this issue from two angles. First,
%we tabulate the open interest and trading volume of the OTM puts and calls, all observed on the
%first day of the expiration cycle. Overall, deep OTM calls do not appear to suffer from significantly
%lower open interest and thin trading volume.

%Second, we compute the returns using the mid price instead of the ask price. We observe that
%all return patterns are retained. Both exercise mitigate the concern that liquidity plays a significant
%role in the observed option returns.

%Using daily bid and ask prices of euro interest rate caps and floors,
%we find that illiquid options trade at higher prices relative to liquid options, controlling for other
%effects,

%In equity option markets, traders face margin requirements both for the options themselves and for hedging-related positions in the
%underlying stock market.

%We provide evidence of a strong effect of the un-
%derlying stocks illiquidity on option returns. By conditioning on end user demand, we find that
%the corresponding illiquidity premiums are negative and decrease in stock illiquidity if there is net
%buying pressure, while premiums are positive and tend to increase otherwise.

%Muravyev and Pearson (2020) Conventional estimates of the costs of taking liquidity in options
%markets are large. Nonetheless, options trading volume is high. We resolve this puzzle by showing
%that options price changes are predictable at high frequency, and many traders time executions by
%buying (selling) when the option fair value is close to the ask (bid).

%We show, both theoretically and empirically, that liquidity
%creation induces negative exposure to volatility risk. Intuitively, liquidity creation involves taking
%positions that can be exploited by privately informed investors. These investors ability to predict
%future price changes makes their payoff resemble a straddle (a combination of a call and a put).
%By taking the other side, liquidity providers are implicitly short a straddle, suffering losses when
%volatility spikes. Empirically, we show that short-term reversal strategies, which mimic liquidity
%creation by buying stocks that go down and selling stocks that go up, have a large negative exposure
%to volatility shocks. This exposure, together with the large premium investors demand for bearing
%volatility risk, explains why liquidity creation earns a premium, why this premium is strongly
%increasing in volatility, and why times of high volatility like the 2008 financial crisis trigger a
%contraction in liquidity.
\vspace{-3mm}


%%%%%%%%%%%%%%%%%%%%%%%%%%%%%%%%% final section
%%%%%%%%%%%%%%%%%%%%%%%%%%%%%%%%% final section
%\section{Links to previous works and further discussion} %Comparisons and further discussions}
%\newpage
\section{Dark matter in option pricing models}
\label{sec:furtherrr}
%The distinguishing feature of our theory is that it maps the option risk premium to the risk premiums
%for (i) jumps crossing the strike, (ii) local time, and (iii) equity.

The distinguishing feature of our theory is that it maps
option risk premiums to the risk premiums for dark matter while
emphasizing
the statistics
of jumps crossing the strike and local time. What
are the consequences of dark matter embedded in an option pricing model? We explore
the dark matter property, as elaborated
in \citet*{Chen_Dou_Kogan:JF2020},
by parameterizing
uncertainties related to
unspanned diffusive risks and price and volatility jump risks
%unspanned volatility and jumps %(under $\mathbb{P}$ and $\mathbb{Q}$)
in option pricing models.
\vspace{2mm}

%We explore this issue in the context of the dark matter property of asset pricing
%{\color{blue}models
%{\color{blue} exemplified}
%elucidated}
%advertised}
%in
%lens of
%In this For this, we make connections to
%{\color{red} [[[\citet*{Chen_Dou_Kogan:JF2020},
%%who show that models often rely upon the dark matter property,
%a concept we
%also utilize]]]}
%\citet*{Chen_Dou_Kogan:JF2020}.
%In our setting, dark matter is exemplified by unspanned risks and option pricing models

%\noindent \textbf{Option pricing models with unspanned diffusive risks and price and volatility jump risks}.
Consider a
parametric option pricing model that arises from the following setup under $\mathbb{P}$:
\begin{eqnarray}
\underbrace{\frac{dM_t}{M_{t-}}}_{\underset{\tiny \mbox{kernel}}{\tiny \mbox{pricing}}} & = & -r\, dt
+ \eta[t,\mathrm{v}_t] \underbrace{d z_t^{\mathbb{P}}}_{\underset{\tiny \mbox{risks}}{\tiny \mbox{spanned}}}
+~ \theta[t,\mathrm{v}_t] \underbrace{du_t^{\mathbb{P}}}_{\underset{\tiny \mbox{risks}}{\tiny \mbox{unspanned}}}
+ \underbrace{(e^{\mathbbm{x}_m} - 1) \, d \mathbb{N}_t^{\mathbb{P}}
}_{\underset{\tiny \mbox{risks}}{\text{\tiny unspanned~jump}}}
- {{\bm \lambda}^{\mathbb{P}}_{\tiny \mbox{jump}}} \, \mathbb{E}^{\mathbb{P}}( e^{\mathbbm{x}_m} - 1 ) \, dt,
\label{eq:dou1}\\
& &
\eta[t,\mathrm{v}_t] \, = \, - \frac{1}{ \sqrt{\mathrm{v}_t}}(
\alpha_{\tiny \mbox{vol}} +\lambda_{\tiny \mbox{vol}} \, \mathrm{v}_t),
%~~\mbox{ \, }
~~~~~~~~
%~~\mathrm{and}~~~~
\theta[t,\mathrm{v}_t]  =  - \theta_{\mathrm{LT}}\, \sqrt{\mathrm{v}_t},
~\mbox{ \, }
\label{eq:dou2}
%\nonumber
\\
\frac{d F_{t}^{T_F}}{F_{t-}^{T_F}} & = &  \overbrace{
(\alpha_{\tiny \mbox{vol}} + \lambda_{\tiny \mbox{vol}} \, \mathrm{v}_t +
\mu_{\tiny \mbox{jump}})}^{\text{\tiny futures~risk~premium}}  \, dt
+\sqrt{\mathrm{v}_t}
%\underbrace{d z_t^{\mathbb{P}}}_{\underset{\tiny \mbox{risks}}{\tiny \mbox{spanned}}}
d z^{\mathbb{P}}_t
~+~\underbrace{(e^{\mathbbm{x}_s} - 1) \, d \mathbb{N}_t^{\mathbb{P}}}_{\underset{\tiny \mbox{jump~risks}}{\text{\tiny unspanned~price}}}
 -
{\bm \lambda}^{\mathbb{P}}_{\tiny \mbox{jump}} \, \mathbb{E}^{\mathbb{P}}( e^{\mathbbm{x}_s} - 1 ) \, dt,
\label{eq:dou3} \\
\underbrace{d\mathrm{v}_t}_{\tiny \mbox{variance}}  &=&
 ( \phi_{\tiny \mbox{vol}}^{\mathbb{P}} - \kappa_{\tiny \mbox{vol}}^{\mathbb{P}} \,\mathrm{v}_t )\, dt +
  \sigma_{\tiny \mbox{vol}} \, \sqrt{\mathrm{v}_t} \,\rho_{\tiny \mbox{vol}}
\underbrace{d z_t^{\mathbb{P}}}_{\underset{\tiny \mbox{risks}}{\tiny \mbox{spanned}}}
+~ \sigma_{\tiny \mbox{vol}} \sqrt{\mathrm{v}_t} \, \sqrt{1-\rho^2_{\tiny \mbox{vol}}} \,
\underbrace{du_t^{\mathbb{P}}}_{\underset{\tiny \mbox{risks}}{\tiny \mbox{unspanned}}} +
%  du_t^{\mathbb{P}} +
\underbrace{\mathbbm{x}_{\mathrm{v}} \, d \mathbb{N}^{\mathbb{P}}_{t}}_{\underset{\tiny \mbox{(additive)}}{\tiny \mbox{jumps~in}~\mathrm{v}_t}},~~ \mbox{ \, \,  }~ \label{eq:dou4}\\
\underbrace{d \mathbb{N}^{\mathbb{P}}_{t}}_{\tiny \mbox{Poisson~jump}} &=& \left \{
\begin{array}{ll}
1
& \hspace{10 mm} \mbox{with~probability}~{\bm \lambda}^{\mathbb{P}}_{\tiny \mbox{jump}}\,dt \\
0
& \hspace{10 mm} \mbox{with~probability}~1- {\bm \lambda}^{\mathbb{P}}_{\tiny \mbox{jump}}\,dt \\
\end{array}
\right.  \label{eq:dou5} \\
\mathbbm{x}_{\mathrm{v}} &&\text{variance~jumps~follow~spectrally~positive~i.i.d.~distribution~under}~\mathbb{P} ~ ~~ ~~~ ~ \label{eq:dou6a} \\
(\mathbbm{x}_m, \mathbbm{x}_s) &&\text{jumps in $M_t$ and $F_{t}^{T_F}$ have i.i.d. distributions~under}~\mathbb{P}.
\label{eq:dou6}
\end{eqnarray}
%{\color{green}conditional~on}}~\mathbbm{x}_{\mathrm{v}}~


%{\color{red}[[[ This is now awkward -- I actually prefer
%the way we had before? The issue is that all the Brownian motions
%are uncorrelated but $\mathbbm{x}_{\mathrm{v}}$,
%$\mathbbm{x}_m$ and $\mathbbm{x}_s$ are also
%sources of uncertainty and they can be correlated.
%I think we should take out the statement
%``All sources of uncertainty are uncorrelated." and go back to
%{\color{blue}``$z_t^{\mathbb{P}}$ and $u_t^{\mathbb{P}}$ are each
%uncorrelated standard Brownian motions."} ]]]]]]]}
%{\color{green} All sources of uncertainty are uncorrelated.}

In this model, $\mathrm{v}_t$ denotes
the variance of the diffusive component of the
equity (futures) return,
and  $z_t^{\mathbb{P}}$ and $u_t^{\mathbb{P}}$ are each
independent
standard Brownian motions.
Unspanned risks are commingled with spanned risks in both the $M_t$ and $\mathrm{v}_t$ dynamics.

%{\color{blue} The interpretation of $\mathrm{v}_t$ parameters, absent volatility jumps,} is standard from \citet*{Heston:1993}, except for
%$\theta_{\mathrm{LT}}$ introduced in (\ref{eq:dou2}). This parameter controls the contribution of priced unspanned diffusive
%volatility risks over $T_{O}-t$.
%%The consequence is that the
%Local time risk premium is negative
%%(and hence, the call risk premium is negative)
%%if and only
%only if $\theta_{\mathrm{LT}}<0$ (Internet Appendix (Corollary~\ref{claimm:SV} of Section~\ref{subsec:model_sv})).
%The takeaway is that any potential mis-specification of models with $\theta_{\mathrm{LT}}\equiv0$ (that is, absent of unspanned
%volatility risks) may be hard to disentangle without additional data on option returns.


How does this model ---
which
traverses  the dimension of unspanned diffusive volatility risks and unspanned price and volatility jump risks --- fare in summarizing
option risk premiums? First,
the risk premiums associated with jumps crossing the strike vary across alternative
jump specifications under $\mathbb{P}$ and $\mathbb{Q}$.
Our model analysis, pursued in Internet Appendix (Section~\ref{app:jumps_across}),
%for small $T_{O}-t$,
shows that the risk premiums for jumps
crossing the strike can
rationalize negative risk premiums
of OTM calls.
We establish this attribute for
jump specifications of \citet*{Merton:76}, \citet*{Kou:2002}, and \citet*{DuffiePanSingleton:2000}.
However, akin to the dark matter property, reconciliation
between option models and data requires
a stand on the
%magnitude of ${\bm \lambda}^{\mathbb{P}}_{\tiny \mbox{jump}}$ and
properties of jumps under $\mathbb{P}$ and $\mathbb{Q}$.\footnote{Our approach
aims to understand
the  differences in option risk
premiums
across strikes. Additionally,
we emphasize weekly options, which allow us to draw the distinctions between
risk premiums for jumps crossing the strike on the downside versus on the
upside. While \citet*{Merton:76} emphasizes
\emph{downward} jumps in equities, \citet*{Kou:2002} presents a model with
\emph{both upward} and downward jumps.
See also \citet*{Aitsahalia_Yacine:2004}.
We refer the reader to works
that
consider
%estimates
%jump
%diffusion
models with jumps
(in price and volatility) and/or
stochastic volatility. See, among others,
\citet*{BakshiCaoChen:97},
\citet*{Bates:2000},
\citet*{Pan:2002},
\citet*{ErakerJohannesPolson:2003},
\citet*{Eraker:2004},
\citet*{Kou_Wang:2004},
\citet*{BroadieChernovJohannes:2007},
%\citet*{DrechslerYaron:RFS2011},
and \citet*{Cai_Kou:2011}.}


%{\color{magenta} Throughout draw on the link between the variations in conditional option returns and modeling ingredients.}
%{\color{blue}
%The approach in our paper relies on
%a dynamic model with unspanned risks.}

%What can we say about local time risk premiums?

Three sources contribute to local time risk premiums in this model: (i) unspanned diffusive risks,
(ii) unspanned volatility jump risks, and (iii) spanned diffusive risks. This
analysis is rather lengthy
and is presented in Internet Appendix (Section~\ref{app:var_jumps}).

Notably, we show
that the parameter $\theta_{\mathrm{LT}}$  --- introduced in (\ref{eq:dou2}) --- controls the contribution of priced unspanned diffusive volatility risks over $T_{O}-t$ to local time risk
premiums.\footnote{We show this in Internet Appendix (Section~\ref{app:unspannedDiffusiveVolRisk}).} The overall consequence is that the
local time risk premiums for unspanned diffusive risks can be negative
(provided $\theta_{\mathrm{LT}} < 0$), which contributes to negative call option risk premiums.

Additionally, the local time risk premiums due to unspanned jump volatility risks
can be negative.\footnote{These restrictions are identified in Internet Appendix (Section~\ref{app:unspannedVolJumpRisk}).}
The takeaway is that any potential
misspecification of models with $\theta_{\mathrm{LT}} \equiv 0$, or absent of unspanned
volatility jump risks, may be hard to disentangle without data on option returns.

Finally, if spanned risks were the only source of uncertainty in the pricing kernel (i.e., if $\mathbbm{x}_m \equiv 0$ and
$\theta_{\mathrm{LT}} \equiv 0$), then
local time risk premiums are such that the risk premiums for OTM calls
would be \emph{positive}. The
implication is that unspanned risks, and hence, dark matter,
are relevant to capturing realities of option risk premiums.


Our decomposition of option risk premiums
provides additional perspectives. First, option models rely upon
variables and parameters that may be hard to reliably extract from equity and volatility dynamics. Second, some parameter restrictions required for empirical consistency may not be directly verifiable.
For example,
%we deduce that option model parameterizations must be such that large positive jumps in volatility associate
%with large positive jumps in the pricing kernel
to align negative local time risk premiums for volatility jumps, we deduce that option
model parameterizations must be such that large positive jumps in volatility associate with large positive
jumps in the pricing kernel. However, the pricing kernel is not a directly inferable quantity.
% We recognize that the pricing kernel




\noindent \textbf{Connections with other option modeling frameworks.}
Through Tanaka's formula, we emphasize the analyticity of local time and jumps crossing the strike
and this angle deviates from others.

\citet*{CovalShumway:2001}
feature a theory in which the call option risk premium is \emph{positive and increasing} in the strike price. Our
prediction, with unspanned risks, with or without jumps, is for the opposite, when the dark matter risk premium is sufficiently negative, and we pose this as a testable implication at high $k>1$
(i.e., farther OTM
calls).
The work of \citet*{Christoffersen_Jacobs_Heston:2013RFSa}
considers
a log stochastic discount factor (SDF),
affine in the return of the equity
and its variance, but the SDF's projection onto
returns is nonmonotonic.
Their
framework does not
formalize a theory of option risk premiums across strikes.

%{\color{magenta} [[[The key central point
%advertised in \citet*{Chen_Dou_Kogan:JF2020} is that an asset pricing model with a large amount of dark matter will be hard
%to refute. These models may convey acceptable insample overfit but may, in fact, perform poorly upon external validation exercises.
%raises at least two concerns regarding its robustness.
%First, it will be difficult to detect any potential misspecification in the dark matter elements
%of a model due to a lack of direct evidence to measure them based on the underlying asset
%returns. Consequently, economic dark matter raises a model effective degrees of freedom and
%leads to low refutability by the standard optimal test procedures. Second, the high effective
%degrees of freedom will likely cause the model to overfit the data in sample and lead to poor
%expected out-of-sample fit. It is a main contribution of this paper (probably the most impor-
%tant contribution)]]]]}



%{\color{magenta}
%To show We defer to Section~\ref{sec:furtherrr} for discussion of model structures that map
%the risk premiums for local time and jumps crosses in the context of the decomposition of option returns.}
%%Based on the asymptotic behavior of short-dated options,
%%\citet*{CarrWu:2003bJF} develop tools to distinguish the relevance of continuous versus jump components in the \emph{risk-neutral} price process.



On the other hand, {\citet*{BakshiMadanPanayotov:2010JFE}}
%{\color{red} [[[[ The theory, which underpins Theorem~1,
%%Results~\ref{claimm:claim1call} and \ref{claimm:straddles},
%is
%different from
%that
%pursued in \citet*{BakshiMadanPanayotov:2010JFE}. ]]]]}
%They
consider a model with heterogeneity in beliefs with personalized
change of measure
for investors, long and short
equity.
In this setting, it is shown
that the risk premium of
OTM calls
%on equity
can be negative when the SDF
%stochastic discount factor (SDF)
admits an increasing region to the
upside.
The approach in our paper relies
on a dynamic model with unspanned risks,
and it does
not take a stand on whether the
SDF is nonmonotonic.


\citet*{Andersen_Fusari_Todorov:2017JFweekly} explore the merits of using weekly options.
They formalize the argument that the jump intensity rate of the discontinuous component and
the return variance of the continuous component will vary little for short-dated options.
In particular, the variance of the continuous component can be regarded as a constant
over very short horizons.
Complementing their approach,
we uncouple, using weeklys (analogous to small $T_{O}-t$), the
effects of risk premiums on local time from risk premiums on
jumps crossing the strike.
%{\color{red} [\citet*{CarrWu:2003bJF} consider short maturity asymptotic behavior, but not
%for option risk premiums.]} Our empirical work
%provides evidence
%on option risk premiums
%extracted from weekly, 28-, and 88-day options.




Our perspective about local time risk premiums --- gleaned from option returns --- intersects with work
on volatility.
%\citet*{DrechslerYaron:RFS2011} present
%a  long-run risks model to
%explain the difference between the squared VIX index and expected realized variance.
\citet*{Carr_Wu:2016_JFE} model
implied volatility dynamics and then derive implications for the shape of the
volatility surface.
\citet*{Eraker_Wu:2017} show
negative average returns to holding volatility products.
%The study of \citet*{Giglio:2017variance} considers variance swap
%contracts to show that investors pay substantially for perceived exposures to negative economic shocks.
%\citet*{Cheng_IH:2019}
%shows that ex-ante estimates of the volatility risk premium embedded in VIX futures fall or stay flat when ex-ante measures
%of risks rise and argues that these facts provide a puzzle for theories as to why investors hedge volatility.
What emerges from the analysis of \citet*{Aitsahalia_Karaman_Mancini:JOE2019} is
that variance swap rates incorporate
a significant price jump component.
%{\color{magenta} Our emphasis is on mechanism of unspanned volatility risk components.}

The driving mechanism of our theory of
option risk premiums is dark matter.
\citet*{Jonesc:2006} considers
%both linear and non-linear
factor models of index option returns but
does not emphasize jumps, and
%with the factors including the
%returns of the S\&P~500 index, changes in the VIX and changes in interest-rates.
%However, as he himself emphasizes,
%even multi-factor non-linear models
%are still insufficient to explain the magnitudes of the
%observed excess returns, particularly for short-term out-of-the-money puts.
%Seen through the lens of our approach, we make two comments.
%First, \citet*{Jonesc:2006} does not emphasize jumps. Second,
%To the extent that he
%incorporates the returns of the S\&P~500 index as a factor, he
the setup does not offer differentiation
between diffusive and discontinuous return components. %and returns generated by large price discontinuities.}
%\citet*{Jonesc:2006}
%probes
%the pricing of volatility and jump risks and concludes that such effects fall short of reconciling
%put returns.
%{\color{red}[The treatment of \citet*{Bondarenko_Oleg:QJF2014}
%centers on the expensiveness of index puts.]}
\citet*{Broadie_Chernov_Ghysels:2009RFS} explore option mispricing and examine unconditional
returns to writing puts on the S\&P 500 index futures.
Essential to \citet*{Bollerslev_Todorov_Xu:2015_JFE} is that the variance risk premium helps predict
market returns and that much of this predictability arises from the part of the variance risk premium associated with tail risk.
\vspace{-4mm}

%{\color{magenta} and \citet*{Bondarenko_Oleg:QJF2014}.}}
%%%%%%%%%%%%%%%%%%%%% elciits

%In this paper, we use option pricing models as a benchmark to assess the evidence for index option mispricing


%Furthermore, for very short-dated OTM options,
%the futures price will only very, very rarely trade at the level of the strike.
%With both these considerations in mind, we form the hypothesis that
%the local time risk premium for deep OTM options may be approximately zero.
%Thus, the risk premiums of deep OTM options will consist of two terms:
%First, the upside (respectively, downside) risk premium for calls (puts) and
%second, the risk premium for jumps crossing the strike.
%For very short-dated OTM options, the first term is also likely to be not large (in absolute value).
%This permits us to formulate the following prediction:


 %%%%%%%%%% hamper

%\subsection{Evidence from S\&P index options reinforces the relevance of dark matter}


%\citet*{Cremers_Halling_Weinbaum:JF2015} construct
%a portfolio of options
%which is
%exposed either to jump risk or to
%volatility risk, revealing that investors dislike both jump and volatility risk.


%%Theoretical
%\section{Extension to general semimartingales and comparisons}
%
%Having seen the findings from option returns on the equity index and futures, one may inquire: Under what restrictions
%do our Results~\ref{claimm:claim1call} and \ref{claimm:straddles} extend to the case when the equity index futures price is a semimartingale, encompassing diverse forms of discontinuities.
%The most prominent and empirically studied example of this general class of price processes
%is the model of \citet*{Merton:76} with orthogonal return jumps, with or without stochastic equity volatility.
%
%Laying out the general semimartingale model of price dynamics is a bit more involved with a broader definition of quadratic variation and local time that internalizes the effect of discontinuities. In what follows, the term $\mathbb{L}^{T_O}_t[k]^{J}$ is local time as defined in
%\citet*[page 216]{Protter:2013} (which, when jumps are present, differs from the definition for continuous semimartingales given in Section~\ref{sec-GeneralDiffusionDynamics}, but the intuition is portable; see \citet*[Remark 8.4, page 225]{Klebaner:2012book}).
%\begin{result}[Extension to semimartingales]
%\label{claimm:claim1call_jump}
%Suppose now $(F_{\ell}^{T_F})$, and thus $(G_\ell)$, for $\ell \geq t$, are semimartingales.
%Then the expected excess return of an OTM call on the equity index futures, with moneyness $k$, \emph{can} be negative only if the
%local time risk premium corresponding to $k$
%%{\color{red} [[[ , $\mathbb{E}^{\mathbb{P}}_{t}( \mathbb{L}^{T_O}_t[k]^{J} ) - \mathbb{E}^{\mathbb{Q}}_{t}( \mathbb{L}^{T_O}_t[k]^{J} )$,]]]}
%is negative.
%\end{result}
%\noindent {\bf Proof:} See Appendix~\ref{appsec:jumppps}. $\blacksquare$
%
%%We formulate the implications of this theory as Result~\ref{claimm:claim1call_jump} of} Appendix~\ref{appsec:jumppps}.
%The takeaway is that the negative expected excess returns of OTM calls and straddles can still be traced to negative local time risk premiums.
% \vspace{-3mm}

%\section{Theoretical and empirical connections to the literature}
%\label{connections}


%\noindent \textbf{VI. Relation to the empirical options literature.}

%\citet*{Song_Xiu:JOE2016} employ options on the S\&P 500 and VIX to show that investors have
%a high marginal utility in both high and low volatility states.}



%{\color{red} [While the literature has made progress, we explore a theory that includes reconciling the negative average excess returns of
%straddles and call options on an equity index or index futures.
%We draw the link between the variations in conditional
%option returns and modeling ingredients.]}

%%%%%%%%%%%%%%%%%%%%%%%%%%%%%%%%%%%%%%%%%%%%%%%%%%%%%%%%%%%%%%%%%%%%%%%%%%% check later
%
%, estimates of the degree of activity of jumps
%were first proposed by Carr et al. (2002). Cont and Mancini
%(2011) use threshold or truncation-based estimators of the
%continuous component of the quadratic variation, originally
%proposed in Mancini (2001), in order to test for the presence
%of a continuous component in the price process.

%
%
%Other models are based on infinite activity
%jumps: see for example Madan and Seneta
%(1990), Madan and Milne (1991), Eberlein
%and Keller (1995), Barndorff-Nielsen (1997,
%1998), Carr et al. (2002), Carr and Wu
%(2003a), Carr and Wu (2004), and Schoutens
%(2003), although with the exception of Carr
%et al. (2002) models of this type are justified
%primarily by their ability to produce interesting
%pricing formulae rather than necessarily
%an attempt at empirical realism.
%So, which is it, based on the data? Our
%objective is now to discriminate between
%finite and infinite activity jumps using again
%the same set of tools.
%

% jump risk not proportional to volatility will be outsized
% semiparametric way to extract tail risks

%%%%%%%%%%%%%%%%%%%%%%%%%%%%%%%%%%%%%%%%%%%%%%%%%%%%%%%%%%%%%%%%%%%%%%%%%%% check later



%The semimartingale setting allows for processes whose path consists of continuous motion interspersed with diverse form of
%jump continuities of random size appearing at random time.

%%%%%%%%%%%%%%%%%%%%%%%%%%%%%%%%%%%%%%%%%%%%%%%%%%%%%%%%%%%%%%%%%%%%%%%%%%%%%%%%%%%%%%%%%%%%%%%%%%%%%%%%%%%%
%%%%%%%%%%%%%%%%%%%%%%%%%%%%%%%%%%%%%%%%%%%%%%%%%%%%%%%%%%%%%%%%%%%%%%%%%%%%%%%%%%%%%%%%%%%%%%%%%%%%%%%%%%%%

%%%%%%%%%%%%%%%%%%%%%%%%%%%%%%%%%%%%%%%%%%%%%%%%%%%%%%%%%%%%%%%%%%%%%%%%%%%%%%%%%%%%%%%%%%%%%%%%%%%%%%%%%%%%
%%%%%%%%%%%%%%%%%%%%%%%%%%%%%%%%%%%%%%%%%%%%%%%%%%%%%%%%%%%%%%%%%%%%%%%%%%%%%%%%%%%%%%%%%%%%%%%%%%%%%%%%%%%%


%%%%%%%%%%%%%%%%%%%%%%%%%%%%%%%%%%%%%%%%%%%%%%%%%%%%%%%%%%%%%%%%%%%%%%%%%%%%%%%%%%%%%%%%%%%%%%%%%%%%%%%%%%%%








%%%%%%%%%%%%%%%%%%%%%%%%%%%%%%%%%%%%%%%%%%%%%%%%%%%%%%%%%%%%%%%%%%%%%%%%%%%%%%%%%%%%%%%%%%%%%%%%%%%%%%%%%%%%%%%%%
%%%%%%%%%%%%%%%%%%%%%%%%%%%%%%%%%%%%%%%%%%%%%%%%%%%%%%%%%%%%%%%%%%%%%%%%%%%%%%%%%%%%%%%%%%%%%%%%%%%%%%%%%%%%%%%%%
%%%%%%%%%%%%%%%%%%%%%%%%%%%%%%%%%%%%%%%%%%%%%%%%%%%%%%%%%%%%%%%%%%%%%%%%%%%%%%%%%%%%%%%%%%%%%%%%%%%%%%%%%%%%%%%%%
%%%%%%%%%%%%%%%%%%%%%%%%%%%%%%%%%%%%%%%%%%%%%%%%%%%%%%%%%%%%%%%%%%%%%%%%%%%%%%%%%%%%%%%%%%%%%%%%%%%%%%%%%%%%%%%%%
%\section{Testing the implications of dark matter and unspanned risks} %Does empirical data support unspanned index volatility}



%%%%%%%%%%%%%%%%%%%%%%%%%%%%%%%%%%%%%%%%%%%%%%%%%%%%%%%%%%%%%%%%%%%%%%%%%%%%%%%%%%%%%%%%%%%%%%%%%%%%%%%%%%%%%%%%%
%%%%%%%%%%%%%%%%%%%%%%%%%%%%%%%%%%%%%%%%%%%%%%%%%%%%%%%%%%%%%%%%%%%%%%%%%%%%%%%%%%%%%%%%%%%%%%%%%%%%%%%%%%%%%%%%%
%%%%%%%%%%%%%%%%%%%%%%%%%%%%%%%%%%%%%%%%%%%%%%%%%%%%%%%%%%%%%%%%%%%%%%%%%%%%%%%%%%%%%%%%%%%%%%%%%%%%%%%%%%%%%%%%%
%%%%%%%%%%%%%%%%%%%%%%%%%%%%%%%%%%%%%%%%%%%%%%%%%%%%%%%%%%%%%%%%%%%%%%%%%%%%%%%%%%%%%%%%%%%%%%%%%%%%%%%%%%%%%%%%%
%\section{Testing the implications of dark matter and unspanned risks} %Does empirical data support unspanned index volatility}





%%%%%%%%%%%%%%%%%%%%%%%%%%%%%%%%%%%%%%%%%%%%%%%%%%%%%%%%%%%%%%%%%%%%%%%%%%%%%%%%%%%%%%%%%%%%%%%%%%%%%%%%%%%%%%%%%
%%%%%%%%%%%%%%%%%%%%%%%%%%%%%%%%%%%%%%%%%%%%%%%%%%%%%%%%%%%%%%%%%%%%%%%%%%%%%%%%%%%%%%%%%%%%%%%%%%%%%%%%%%%%%%%%%
%%%%%%%%%%%%%%%%%%%%%%%%%%%%%%%%%%%%%%%%%%%%%%%%%%%%%%%%%%%%%%%%%%%%%%%%%%%%%%%%%%%%%%%%%%%%%%%%%%%%%%%%%%%%%%%%%
%%%%%%%%%%%%%%%%%%%%%%%%%%%%%%%%%%%%%%%%%%%%%%%%%%%%%%%%%%%%%%%%%%%%%%%%%%%%%%%%%%%%%%%%%%%%%%%%%%%%%%%%%%%%%%%%%
%\section{Testing the implications of dark matter and unspanned risks} %Does empirical data support unspanned index volatility}





%%%%%%%%%%%%%%%%%%%%%%%%%%%%%%%%%%%%%%%%%%%%%%%%%%%%%%%%%%%%%%%%%%%%%%%%%%% check later
\section{Conclusion}
\label{eq:concluding_remarks}

Is there dark matter embedded in volatility and in equity options?
That is, are
there unspanned risks
that
are hard to observe but elicit
risk premiums on equity options?
Building on this question,
our answer is ``yes," and we provide supportive empirical evidence.

%The semimartingale setting allows for processes whose path consists of continuous motion interspersed with diverse form of
%jump continuities of random size appearing at random time.

We present a semimartingale theoretical approach that allows us to study
the constructs of \emph{jumps crossing the strike}
(from below and above) and of \emph{local time}. Our treatment of jumps crossing
the strike and of local time is essential to our theory, because their
absence would go against our empirical evidence.
We label such abstract uncertainties dark matter,
as they can be hard to identify, but their presence
is inferred
in options data.
Dark matter generates statistically significant risk premiums,
and the workings of dark matter can be economically influential.


Developing this line of inquiry, we reveal the manner in which call option risk premiums can be decomposed into
dark matter risk premiums and \emph{upside} equity risk premiums.
%{\color{red}[[[[ Why are we suddenly turning
%everything around? How we have written it here is wrong and confusing.
%Let us look at Theorem 1.
%Our theory predicts \textbf{\underline{negative}} call risk premiums and \textbf{\underline{negative}} straddle risk premium \emph{only if} there are
%unspanned risks in the pricing kernel (and with the right sign eg
%$\theta_{\mathrm{LT}} < 0$).
%One would have (with positive equity risk premium presumed)
%\textbf{\underline{positive}} call risk premiums under 3 conditions:
%1./ Unspanned risks are absent in the pricing kernel.
%2./ Unspanned risks are present in the pricing kernel but they
%generate for example a positive local time risk premium (eg
%$\theta_{\mathrm{LT}} > 0$).
%3./ Unspanned risks are present in the pricing kernel
%(with the right sign eg $\theta_{\mathrm{LT}} < 0$)
%but
%the magnitude is insufficient to counteract the positive equity risk premium.
%\textbf{Hence, the next sentence is incorrect.}
%]]]]]
%[[[[[ Our theoretical
%treatment predicts positive call risk premiums and a zero straddle risk premium \emph{only if}
%unspanned risks are absent in the pricing kernel. The latter is contrary to our empirical evidence. ]]]]]]]]}
Our theoretical
treatment predicts negative call option risk premiums and a negative straddle risk premium \emph{only if}
there are unspanned risks in the pricing kernel.
%Our empirical evidence indicates that partitioned average excess returns of OTM calls
%%risk premiums (at high strikes)
%and straddles
%%risk premiums
%are negative.
Our empirical findings are
consistent with
%Thus, there must be
the relevance of unspanned risks and dark matter in option risk premiums.
%in the pricing kernel --- that is, there must be dark matter in option risk premiums.}
%the dark matter risk premium is zero.
%{\color{red} [[[[ Our investigation of the risk
%premium on dark matter is pertinent, as it is a feature that is missed by
%%{\color{red}[macrofinance]}
%models
%that do not incorporate information from equity and equity futures
%options. ]]]]]}

We develop theoretical results with testable implications.
The key to attaining consistency with data attributes lies in equipping
the pricing kernel dynamics, and the
volatility dynamics with unspanned risks (the jump risks are intrinsically unspanned),
which ends up inducing negative
dark matter risk premiums. What stands out from our analysis is the compatibility between
negative dark matter risk premiums and negative risk premiums of straddles and deep out-of-the-money call options.
Our empirical investigation substantiates these implications,
thus, aligning with our theory of unspanned risks and dark matter in equity markets.



%%%%%%%%%%%%%%%%%%%%%%%%%%%%%%%%%%%%%%%%%%%%%%%%%%%%%%%%%%%%%%%%%%%%%%%%%%%%%%%%%%%%%%%%%%%%%%%%%%%%%%%%%%%%%%%%%%%%%%%%%%%%%%%%%%5
\newpage
\clearpage
%%%%%%%%% to fix the references for submission
%\bibliographystyle{plainnat}
%%%%%%%%%%%%%%%%%%%%%%%%%%%%%%%%%%%%%%%%%%%% use below for fixing inline
                                            \bibliographystyle{abbrvnat}

%%%%%%%%%%%%%%% DONT; for references below; but use to create the 2 pages of references
%                                 \bibliographystyle{apalike}


\bibliography{masterbib_gao}


\newpage
%%%%%%%%%%%%%%%%%%%%%%%%%%%%%%%%%%%%%%%%%%%%%%%%%%%%%%%%% APPENDIX
\appendix

\begin{center}
{\Large \bf Appendix}
\end{center}
%We employ the following notations.
%{\color{red} [\begin{description}
%%\item[-] $r$ denotes the (constant) interest-rate.
%\item[-] $T_{F}$ is the maturity of the equity index futures contract.
%\item[-] $T_{O}$ is the maturity of the option on the equity index futures (with ${T}_O \leq T_F$).
%\item[-] $F_{t}^{T_F}$ is the time $t$ price of the equity index futures with maturity $T_{F}$.
%\item[-] $G_s \equiv \frac{F_{s}^{T_F}}{F_{t}^{T_F}}$ is the gross return on the equity index futures price over the time period $t$ to $s$.
%\item[-] $\mathbbm{1}_{\{ G_{\ell} > k \}}$ is an indicator function that takes a value of 1 if $G_{\ell} > k$ and zero otherwise.
%\end{description}
%]}
%%%%%%%%%%%%%%%%%%%%%%%%%%%%%%%%%%%%%%%%%%%%%%%%%%%%%%%%%%%%%%%%%%%%%%%%%%%%%%%%%%%%%%%%%%%%%%
\vspace{-3mm}
%%%%%%%%%%%%%%%%%%%%%%%%%%%%%%%%%%%%%%%%%%%%%%%%%%%%%%%%%%%%%%%%%%%%%%%%%%%%%%%%%%%%%%%%%%%%%%%%%%%%%%%%%%%%%%%%%%%%%%%
\setcounter{equation}{0}
\renewcommand{\theequation}{A\arabic{equation}}
%\setcounter{section}{0}
%\renewcommand{\thesection}{\Alph{section}}
%\renewcommand{\thesubsection}{\Alph{subsection}}

\section{\bf \small Appendix A: Proof of Theorem~\ref{claimm:claim1call_jump} ((general) semimartingales)} % Proof of Result~\ref{claimm:claim1call_jump}}
\label{appsec:jumppps}

Suppose $(F_{\ell}^{T_F})$, and thus $(G_\ell)$, for $\ell \geq t$, are semimartingales.
This theoretical
environment
%Here we consider an economic setting in which the price process is a {\color{blue}general semimartingale,
allows for the possibility of jumps in
the futures price, as well as for stochastic volatility effects (including accommodating jumps in volatility).

Henceforth, the
term $\mathbb{L}^{T_O}_t[k]$ is local time (as defined in  (\ref{ltt.1})).

By construction,  $G_t=1$.
Since the stochastic processes $(F_{\ell}^{T_F})$ and $(G_\ell)$ are $\mathbb{Q}$ martingales,
\begin{align}
&\mathbb{E}_{t}^{\mathbb{Q}}( \int_{t+}^{T_O} \mathbbm{1}_{\{G_{\ell \, -} < k\}} dG_{\ell} ) ~=~0 &
&\mathrm{and}&
&\mathbb{E}_{t}^{\mathbb{Q}}( \int_{t+}^{T_O} \mathbbm{1}_{\{G_{\ell \, -} > k\}} dG_{\ell} ) ~=~0.&
\end{align}
%\begin{eqnarray}
%
%~~and~~
%~~\mbox{ \, \, \, \, \, \, %({\color{blue}this is true} with or without jumps) \, \, \, \, \, \, } ~ ~
%\label{eq:ExpectUnderQOfFuturesTradingIsZeroJumps}
%%\mathbb{E}_{t}^{\mathbb{Q}}( \int_{t+}^{T_O} \mathbbm{1}_{\{G_{\ell \, -} > k\}} dG_{\ell} ) ~=~0. ~~\mbox{ \, \, \, \, \, \, ({\color{blue}this is %true} with or without jumps) \, \, \, \, \, \, } ~ ~ \label{eq:ExpectUnderQOfFuturesTradingIsZeroJumpsd}
%\end{eqnarray}

\noindent \textbf{I. OTM call option risk premium.}
We employ Tanaka's formula in
%{\color{red}[equation]}
(\ref{eq:TanakaJumps}).

Using
the definition of the expected return of a call option, the fact that $(F_{\ell}^{T_F})$ is a martingale
under $\mathbb{Q}$, and considering OTM calls, that is $k> 1$, so that $\max( G_{t} - k, 0 ) = 0$, we obtain
\begin{equation}
1 + \mu^{{T}_O}_{t,{\tiny \mathrm{call}}}[k] \, = \,
\frac{\mathbb{E}_{t}^{\mathbb{P}}( \int_{t+}^{T_O} \mathbbm{1}_{\{G_{\ell  -} > k\}} dG_{\ell})
~+~ \mathbb{E}_{t}^{\mathbb{P}}( \mathbb{L}^{T_O}_t[k])
~+~
\mathbb{E}_{t}^{\mathbb{P}}(a_t^{T_O}[k]) + \mathbb{E}_{t}^{\mathbb{P}}(b_t^{T_O}[k]) }
{ e^{-r ({T}_O - t)} \{ \mathbb{E}_{t}^{\mathbb{Q}}( \mathbb{L}^{T_O}_t[k] )
~+~ \mathbb{E}_{t}^{\mathbb{Q}}( a_t^{T_O}[k] )
~+~ \mathbb{E}_{t}^{\mathbb{Q}}( b_t^{T_O}[k]) \} }.~~~\mbox{ \, \, }
\label{eq:ExpectedHoldingReturn1GneralDynamicsWithJumps}
\end{equation}
From the definition of $\mathrm{D}^{u, T_O}_t[k] = \mathbb{L}^{T_O}_t[k] + a_t^{T_O}[k] + b_t^{T_O}[k]$ in
%{\color{red}[equation]}
(\ref{darkk.1}), we note that
\begin{equation}
\mathbb{E}_{t}^{\mathbb{Q}}( \mathrm{D}^{u, T_O}_t[k]) ~=~ \mathbb{E}_{t}^{\mathbb{Q}}(\mathbb{L}^{T_O}_t[k] + a_t^{T_O}[k] + b_t^{T_O}[k])~>0,~~~~\mathrm{for}~k>1.
\end{equation}
This follows, since
\begin{align}
&\mathbb{E}_{t}^{\mathbb{Q}}( \mathbb{L}^{T_O}_t[k] ) > 0 &(\mathbb{L}^{T_O}_t[k]~\mathrm{is~a~nonnegative~random~variable).} \\
&\mathbb{E}_{t}^{\mathbb{Q}}( a_t^{T_O}[k]) > 0~~\mathrm{and}~~\mathbb{E}_{t}^{\mathbb{Q}}( b_t^{T_O}[k]) > 0
&(a_t^{T_O}[k]~\mathrm{and}~b_t^{T_O}[k]~\mbox{are~each~{convex in $G_\ell$).}}
\end{align}

Subtracting $e^{r ({T}_O - t)}$ from both sides of
%{\color{red}[equation]}
(\ref{eq:ExpectedHoldingReturn1GneralDynamicsWithJumps}),
we obtain the following: %have {\color{blue}that the expected excess return to holding the option over $t$ to $T_O$ is}
\begin{equation*}
1 + \mu^{{T}_O}_{t,{\tiny \mathrm{call}}}[k] - e^{r ({T}_O - t)} =
\frac{e^{r ({T}_O - t)}}{\mathbb{E}_{t}^{\mathbb{Q}}( \mathrm{D}^{u, T_O}_t[k])  }
\{
\underbrace{\mathbb{E}_{t}^{\mathbb{P}}( \int_{t+}^{T_O} \mathbbm{1}_{\{G_{\ell  -} > k\}} dG_{\ell})}_{\tiny~\mbox{upside~risk~premium}}
~+~ \underbrace{\mathbb{E}_{t}^{\mathbb{P}}( \mathrm{D}^{u, T_O}_t[k] ) - \mathbb{E}_{t}^{\mathbb{Q}}( \mathrm{D}^{u, T_O}_t[k] )}_{\tiny \mbox{risk~premium~for~dark~matter}} \}.
%~~~\mbox{ \, \, }
\label{eq:ExpectedHoldingReturn1call}
\end{equation*}

%Under the {\color{red}[[[[[ maintained ]]]]} assumption that
If the upside risk premium $\mathbb{E}_{t}^{\mathbb{P}}( \int_{t+}^{{T}_O} \mathbbm{1}_{\{G_{\ell -} > k\}} dG_{\ell} )$
were
positive, the expected excess return of an OTM call on the equity futures \emph{can} be negative only if
\begin{equation}
\mathbb{E}^{\mathbb{P}}_{t}( \mathrm{D}^{u, T_O}_t[k] ) ~-~ \mathbb{E}^{\mathbb{Q}}_{t}( \mathrm{D}^{u, T_O}_t[k] )~\mathrm{is~negative~for}~k>1.
\end{equation}
The following case is instructive:
\begin{itemize}
\item Suppose $(F_{\ell}^{T_F})$ is a continuous semimartingale. Then  $a_t^{T_O}[k]=b_t^{T_O}[k]=0$ and the source of
the risk premium for dark matter is the risk premium for local time (for $k>1$).

%\item {\color{red} [Suppose $F_{t}^{T_F}$ is a continuous semimartingale and
%return volatility and/or the pricing kernel
%is \emph{absent} of unspanned risks. Then, the risk premium for $\mathbb{L}^{T_O}_t[k]$
%is zero (see Lemma~\ref{eq:lemmon} (Internet Appendix~(Section~\ref{appsec:SV1}))).}
\end{itemize}
We have verified the statement of Theorem~\ref{claimm:claim1call_jump} with respect to OTM calls. $\square$

\noindent \textbf{II. OTM put option risk premium.} With the definition, for $k<1$, in (\ref{darkk.1})
that $\mathrm{D}^{d, T_O}_t[k]= \mathbb{L}^{T_O}_t[k] + c_t^{T_O}[k] + d_t^{T_O}[k]$, and Tanaka's formula in (\ref{eq:TanakaPutCaseJumps}),
we obtain the following:
\begin{equation*}
\underbrace{1 + \mu^{{T}_O}_{t,{\tiny \mathrm{put}}}[k] - e^{r ({T}_O - t)}}_{\tiny \mbox{expected~excess~return~of~puts}} =
\frac{e^{r ({T}_O - t)}}{\mathbb{E}_{t}^{\mathbb{Q}}( \mathrm{D}^{d,T_O}_t[k])  }
\{~
- \underbrace{\mathbb{E}_{t}^{\mathbb{P}}( \int_{t+}^{T_O} \mathbbm{1}_{\{G_{\ell -} < k\}} dG_{\ell})}_{~\tiny \mbox{downside~risk~premium}}
+ \underbrace{\mathbb{E}_{t}^{\mathbb{P}}( \mathrm{D}^{d,T_O}_t[k] ) - \mathbb{E}_{t}^{\mathbb{Q}}( \mathrm{D}^{d,T_O}_t[k] )}_{\tiny \mbox{risk~premium~for~dark~matter}} \}.
%~~~\mbox{ \, \, }
\label{eq:ExpectedHoldingReturn1put}
\end{equation*}
If the downside risk premium $\mathbb{E}_{t}^{\mathbb{P}}( \int_{t+}^{{T}_O} \mathbbm{1}_{\{G_{\ell -} < k\}} dG_{\ell} )$
were
positive, the put risk premium is negative when the risk premium for dark matter is negative. $\square$

\noindent \textbf{III. Straddle risk premium.}
Since at $k = 1$, $a_t^{T_O}[1] = d_t^{T_O}[1]$, and
$b_t^{T_O}[1] = c_t^{T_O}[1]$, it holds that
%{\small
\begin{eqnarray}
&a_t^{T_O}[1] + b_t^{T_O}[1] + c_t^{T_O}[1] + d_t^{T_O}[1] ~ = ~ 2 \,(a_t^{T_O}[1] + b_t^{T_O}[1])~\equiv~ 2 \, \mathbb{A}_t^{T_O}[1], ~ \mbox{ \, \, \, \, \, } ~  &  \\
& \mathrm{where} ~~ ~ \mathbb{A}_t^{T_O}[1] ~ \equiv ~  \sum_{t < \ell \leq T_O} \underbrace{\{
 \mathbbm{1}_{\{G_{\ell \, -} < 1\}} \, \max(  G_{\ell}- 1, 0 ) \, + \,
 \mathbbm{1}_{\{G_{\ell  -} > 1\}} \, \max( 1 - G_{\ell}, 0 )\}}_{\tiny \mbox{jumps~crossing~the~strike~from~below~and~above,}~ k = 1}. ~ \mbox{ \, \, \, \, } ~ &
%\nonumber
\end{eqnarray} %}

Suppose further that, for $k=1$, the futures risk premium to the
upside is approximately equal to the futures risk premium to the downside.
This is akin to an assumption that return movements
(anchored to $F_{t}^{T_F}$)
to the downside or upside
are equally probable and unforecastable.

Specifically,
\begin{equation}
\overbrace{\mathbb{E}_{t}^{\mathbb{P}}( \int_{t+}^{T_{O}} \mathbbm{1}_{\{G_{\ell -} > 1 \}} dG_{\ell} ) }
^{\tiny \mbox{upside~risk~premium~for}~\tiny {k=1}}
~~ - ~~
\overbrace{\mathbb{E}_{t}^{\mathbb{P}}( \int_{t+}^{T_{O}} \mathbbm{1}_{\{G_{\ell -} < 1 \}} dG_{\ell} )}^{\tiny \mbox{downside~risk~premium~for}~{k=1}}  ~~ \approx ~ 0. ~~ \mbox{ \, \, \, \, \, \, }~~~~ \label{eq:RiskPremiumUpsideEqualsRiskPremiumDownsideJumps}
\end{equation}
Then we have
\begin{eqnarray}
& & \overbrace{1 + \mu^{T_{O}}_{t,{\tiny \mbox{straddle}}} - e^{r (T_{O}-t)}}^{\tiny \mbox{straddle~risk~premium}} \nonumber \\
& & ~ = ~ e^{r (T_{O}-t)} (
\frac{\overbrace{\mathbb{E}_{t}^{\mathbb{P}}( \int_{t+}^{T_{O}} \mathbbm{1}_{\{G_{\ell -} > 1 \}} dG_{\ell}
- \int_{t+}^{T_{O}} \mathbbm{1}_{\{G_{\ell -} < 1 \}} dG_{\ell}}^{\approx 0}
~+~ 2 \, \mathbb{L}_t^{{T}_O}[1] + 2\, \mathbb{A}_t^{T_O}[1] )}{ \mathbb{E}_{t}^{\mathbb{Q}}( 2 \, \mathbb{L}_t^{{T}_O}[1]  + 2\, \mathbb{A}_t^{T_O}[1] )} - 1) ~ \mbox{ \, \, \, } ~ \nonumber \\
& & ~=~\frac{e^{r (T_{O}-t)}}{\mathbb{E}_{t}^{\mathbb{Q}}( \mathbb{L}_t^{{T}_O}[1] +\mathbb{A}_t^{{T}_O}[1] )} ~
\{ \underbrace{\mathbb{E}_{t}^{\mathbb{P}}( \mathbb{L}_t^{{T}_O}[1]) -\mathbb{E}_{t}^{\mathbb{Q}}( \mathbb{L}_t^{{T}_O}[1] )}_{\underset{(k=1)}{\tiny \mbox{local~time~risk~premium}}} ~+~
\underbrace{\mathbb{E}_{t}^{\mathbb{P}}( \mathbb{A}_t^{{T}_O}[1]) -\mathbb{E}_{t}^{\mathbb{Q}}( \mathbb{A}_t^{{T}_O}[1] )}_{\underset{ \mathrm{strike~from~below~and~above}~k=1}{\tiny \mbox{risk~premium~for~jumps~crossing~the}}} \}.
\label{eq:StraddleInterim1Jumps}
\end{eqnarray}
%Equation (\ref{eq:RiskPremiumUpsideEqualsRiskPremiumDownsideJumps}) is of the view that .
%The straddle risk premium result in Corollary~\ref{claimm:straddles} is
The continuous semimartingale analog of (\ref{eq:StraddleInterim1Jumps}) is obtained by setting
$\mathbb{A}_t^{T_O}[1]=0$ (because, in this case, there are no jumps).  $\square$

Internet Appendix (Section~\ref{app:jumps_acrossPutsStraddles}) further shows that when there are no unspanned risks in the pricing kernel, the straddle risk premium is zero.

%When there are no unspanned risks in the pricing kernel, the risk premium on straddles is zero.

We have the proof of Theorem~\ref{claimm:claim1call_jump}. $\blacksquare$ %\vspace{3mm}



%%%%%%%%%%%%%%%%%%%%%%%%%%%%%%%%%%%%%%%%%%%%%%



%%%%%%%%%%%%%%%%%%%%%%%%%%%%%%%%%%%%%%%%%%%%%%%%%%%%%%%%%%%%%%%%%%%%%%%%%%%%%%%%%%%%%%%%%%%%%%%%%%%%%%%%%%%%%%%%%%%%%%%%
%\setcounter{equation}{0}
%\renewcommand{\theequation}{B\arabic{equation}}
%\section{\bf \small Appendix B: Risk premium on dispersion uncertainty
%and its link to risk premium on local time and risk premium on jumps crossing the strike}
%\label{appsec:dispersion}
%Consider the time $T_O$ payoff $\{ \log G_{T_O} \}^2= \{\log \frac{F_{T_O}^{T_F}}{F_t^{T_F}}\}^2$. This payoff represents \emph{dispersion uncertainty}.
%
%Define the function
%\begin{equation}
%\mathfrak{f}[K] ~\equiv~ \frac{2}{K^2} ( 1 - \log \frac{K}{F_{t}^{T_F}}).
%%= \frac{d^2 \{\log \frac{F_{T_O}^{T_F}}{F_t^{T_F}}\}^2}{d (F_{T_O}^{T_F})^2} {\big |}_{F_{T_O}^{T_F}=K}.
%\end{equation}
%Since  $\{\log \frac{F_{T_O}^{T_F}}{F_t^{T_F}}\}^2$ is twice continuously differentiable, it may be expressed, following
%%\citet*{BakshiMadan:2000} and
%\citet{CarrMadan:2001QF}, as
%\begin{eqnarray}
%\big\{\log \frac{F_{T_O}^{T_F}}{F_t^{T_F}}\big\}^2 & = &
% \int\limits_0^{F_{t}^{T_F}} \mathfrak{f}[K] \max(K-F_{T_O}^{T_F},0)\,dK  ~+~
% \int\limits_{F_{t}^{T_F}}^\infty \mathfrak{f}[K] \max(F_{T_O}^{T_F}-K,0)\,dK~~\mbox{ \, \, }\label{eq:asb1} \\
%% & = &
%% \int\limits_{F_{T_O}^{T_F}}^\infty y[K] F_{t}^{T_F} [\frac{F_{T_O}^{T_F}}{F_{t}^{T_F}}-\frac{K}{F_{t}^{T_F}}]^{+}dK ~+~
%% \int\limits_0^{F_{T_O}^{T_F}} y[K] F_{t}^{T_F} [\frac{K}{F_{t}^{T_F}}- \frac{F_{T_O}^{T_F}}{F_{t}^{T_F}}]^{+}dK~~\mbox{ \, \, \, \, }~ \label{eq:asb2}\\
% & = &
% \int\limits_0^{1} \omega[k] \max(k - \frac{F_{T_O}^{T_F}}{F_{t}^{T_F}},0)\,dk~+~
%\int\limits_{1}^\infty \omega[k] \max(\frac{F_{T_O}^{T_F}}{F_{t}^{T_F}}- k,0)\,dk
%~~\mbox{ \, \, }\label{eq:asb3} \\
%\mbox{ where \, }~ \omega[k] & \equiv & \frac{2}{k^2} ( 1 - \log k ), ~~\mbox{ \, with \, }~~k \, = \, \frac{K}{F_{t}^{T_F}},~~\mbox{ \, and \, }~ d k \, = \, \frac{d K}{F_{t}^{T_F}}.~~~\mbox{ \, }
%\end{eqnarray}
%In moving from
%%{\color{red}[equation]}
%(\ref{eq:asb1}) to (\ref{eq:asb3}), we have performed a change of variable.
%% of $dK = dk\, F_{t}^{T_F}$.
%
%We can now substitute Tanaka's formula for semimartingales into the expression for $\max(k-G_{T_O},0)$ and
%$\max(G_{T_O}- k,0)$ in the right-hand side of
%%{\color{red}[equation]}
%(\ref{eq:asb3}). Therefore, we obtain the following:
%\begin{eqnarray}
%\big\{\log \frac{F_{T_O}^{T_F}}{F_t^{T_F}}\big\}^2 &=& \int\limits_0^{1} \omega[k] \max(k - G_{T_O},0) dk
%+  \int\limits_{1}^\infty \omega[k] \max(G_{T_O}- k,0) dk
%~~\mbox{ \, \, }
%%\label{eq:asb4}
%\nonumber
%\\
%&=& \int\limits_0^{1} \omega[k] \{ - \int_{t}^{T_O} \mathbbm{1}_{\{G_\ell- < k\}} \,dG_{\ell}
%~+~ \mathbb{L}^{T_O}_t[k] ~+~ c_t^{T_O}[k] ~+~ d_t^{T_O}[k] \} dk \nonumber \\
%&& +  \int\limits_{1}^\infty \omega[k] \{\int_{t}^{T_O} \mathbbm{1}_{\{G_{\ell-} > k\}} \,dG_{\ell}
%~+~ \mathbb{L}^{T_O}_t[k] ~+~ a_t^{T_O}[k] ~+~ b_t^{T_O}[k] \} dk
%~~\mbox{ \, \, }\label{eq:asb5} \\
%&=& \int_{t}^{T_O} \big( \int\limits_{1}^\infty \omega[k] \, \mathbbm{1}_{\{G_{\ell-} > k\}} \, dk - \int\limits_0^{1} \omega[k] \, \mathbbm{1}_{\{G_{\ell-} < k\}} \ dk \big) \, dG_{\ell}
%%{\color{magenta} +\int\limits_{0}^\infty \omega[k]\,\mathbb{L}^{T_O}_t[k]\,dk }
%\nonumber \\
%&&~+~ \int\limits_{0}^\infty \omega[k]\,\mathbb{L}^{T_O}_t[k]\,dk  \nonumber \\
%&& ~+~
%\int\limits_{0}^{1} \omega[k]\, ( c_t^{T_O}[k] + d_t^{T_O}[k] ) \,dk
%%\nonumber \\ &&
% ~+~
%\int\limits_{1}^\infty \omega[k]\, ( a_t^{T_O}[k] + b_t^{T_O}[k]) \,dk.
%~\mbox{ \, }
%~~\mbox{ \, \, }\label{eq:asb6}
%\end{eqnarray}
%Using $\mathbb{P}$ and $\mathbb{Q}$ measure expectations, we consequently obtain the following:
%\begin{eqnarray}
%\mathbb{E}_t^{\mathbb{P}} ( \big\{\log \frac{F_{T_O}^{T_F}}{F_t^{T_F}}\big\}^2 )
%&=&
%-~\mathbb{E}_t^{\mathbb{P}} ( \int_{t}^{T_O} \big\{ -\int\limits_{1}^\infty \omega[k] \, \mathbbm{1}_{\{G_{\ell-} > k\}} \, dk +
%\int\limits_0^{1} \omega[k] \, \mathbbm{1}_{\{G_{\ell-} < k\}} \ dk \big\} \, dG_{\ell} )
%~~\mbox{ \, \, \, \, }~~ \nonumber \\
%&+&
%\int\limits_{0}^\infty \omega[k] \, \mathbb{E}_t^{\mathbb{P}} ( \mathbb{L}^{T_O}_t[k] ) \, dk   \nonumber \\
%&+& ~
%\int\limits_{0}^1 \omega[k]\, \mathbb{E}_t^{\mathbb{P}}( c_t^{T_O}[k]+ d_t^{T_O}[k] ) \,dk  +
%\int\limits_{1}^\infty \omega[k]\, \mathbb{E}_t^{\mathbb{P}}( a_t^{T_O}[k] + b_t^{T_O}[k]) \,dk.
%\\
%\mathbb{E}_t^{\mathbb{Q}} ( \big\{\log \frac{F_{T_O}^{T_F}}{F_t^{T_F}}\big\}^2 ) &=&
%\int\limits_{0}^\infty \omega[k] \, \mathbb{E}_t^{\mathbb{Q}} ( \mathbb{L}^{T_O}_t[k] ) \, dk \nonumber \\
%&+&
%\int\limits_{0}^1 \omega[k]\, \mathbb{E}_t^{\mathbb{Q}}( c_t^{T_O}[k] + d_t^{T_O}[k] ) \,dk
%+ \int\limits_{0}^\infty \omega[k]\, \mathbb{E}_t^{\mathbb{Q}}( a_t^{T_O}[k] + b_t^{T_O}[k] ) \,dk.
%%~~\mbox{ \, }~ ~~\mbox{ \, \, }
%\label{eq:asb8}
%\end{eqnarray}
%This is because
%$\mathbb{E}_t^{\mathbb{Q}} ( \int_{t}^{T_O} \big\{\int\limits_{1}^\infty \omega[k] \, \mathbbm{1}_{\{G_{\ell-} > k\}} dk
%\big\} dG_{\ell} ) =0$
%and $\mathbb{E}_t^{\mathbb{Q}} ( \{\int\limits_0^{1} \omega[k]\, \mathbbm{1}_{\{G_{\ell-} < k\}} \ dk \big\} dG_{\ell} ) =0$.
%
%The expression for the risk premium for dispersion uncertainty
%is as follows:
%\begin{eqnarray}
%& & \underbrace{\mathbb{E}_t^{\mathbb{P}} ( \big\{\log \frac{F_{T_O}^{T_F}}{F_t^{T_F}}\big\}^2 )
%- \mathbb{E}_t^{\mathbb{Q}} ( \big\{\log \frac{F_{T_O}^{T_F}}{F_t^{T_F}}\big\}^2 )}_{\mathrm{risk~premium~for~dispersion~uncertainty}}
%~=~  -\mathrm{e}_t^{\mathbb{P}} +
%\int\limits_{0}^\infty ~ \omega[k] \, \underbrace{\{ \mathbb{E}_t^{\mathbb{P}} ( \mathbb{L}^{T_O}_t[k] )-\mathbb{E}_t^{\mathbb{Q}} ( \mathbb{L}^{T_O}_t[k] )\}}_{\mathrm{risk~premium~for~local~time}} \,dk   \nonumber \\
%&&+~ \int\limits_{0}^1 \omega[k]\, \underbrace{ \{
%\mathbb{E}_t^{\mathbb{P}}( c_t^{T_O}[k] ~+~ d_t^{T_O}[k] )
%- \mathbb{E}_t^{\mathbb{Q}}( c_t^{T_O}[k] ~+~ d_t^{T_O}[k] ) \}}_{\mathrm{risk~premium~for~jumps~crossing~the~strike}~(k<1)} \, dk \nonumber \\
%&&+~ \int\limits_{1}^\infty \omega[k]\, \underbrace{\{
%\mathbb{E}_t^{\mathbb{P}}( a_t^{T_O}[k] ~+~ b_t^{T_O}[k] )
%- \mathbb{E}_t^{\mathbb{Q}}( a_t^{T_O}[k] ~+~ b_t^{T_O}[k]) \}}_{\mathrm{risk~premium~for~jumps~crossing~the~strike}~(k>1)} \, dk,
%%,~\mbox{ \, }
%%~~\mbox{ \,\,\,\, }
%\nonumber \\
%%\label{eq:asb9} \\
%& &~~~~ \mbox{ where \, \, }~ \mathrm{e}_t^{\mathbb{P}} ~=~ \mathbb{E}_t^{\mathbb{P}} ( \int_{t}^{T_O}
%\big\{ -\int\limits_{1}^\infty \omega[k] \, \mathbbm{1}_{\{G_{\ell-} > k\}} \, dk +  \int\limits_0^{1} \omega[k] \, \mathbbm{1}_{\{G_{\ell-} < k\}} \ dk \big\} \, dG_{\ell} ).
%%~~~\mbox{ \, }
%\label{eq:DefMuP}
%\end{eqnarray}
%The term inside the $dG_{\ell}$ integral inside the expectation
%in
%%{\color{red}[equation]}
%(\ref{eq:DefMuP}) is the gain/loss from a dynamic trading strategy,
%which, at time $\ell$, takes a position in the futures
%proportional to the quantity $\big( -\int\limits_{1}^\infty \omega[k] \, \mathbbm{1}_{\{G_{\ell-} > k\}} \, dk + \int\limits_0^{1} \omega[k] \, \mathbbm{1}_{\{G_{\ell-} < k\}} \ dk \big)$. In essence, $\mathrm{e}_t^{\mathbb{P}}$ is the expected total gain/loss, over $t$ to $T_O$, from this futures trading strategy.
%
%Finally, %we note that
%$\omega[k]>0$ for $0< k<\exp(1)=2.71828$, and $\omega[k]$ is declining for high enough $k$.
%%Equation (\ref{eq:LogFuturesSqasb11InResult}) follows.
%$\blacksquare$ \vspace{-3mm}
%%%%%%%%%%%%%%%%%%%%%%%%%%%%%%%%%%%%%%%%%%%%%%%%%%%%%%%%%%%%%%%%%%%%%%%%%%%%%%%%%%%%%%%%%%%%%%%%%%%%%%%%%%%%%%%%%%%%%%%%%%%%%%%%%%%%%%%%%%%%


%%%%%%%%%%%%%%%%%%%%%%%%%%%%%%%%%%%%%%% below moved 10/1/2020

%                                                        \setcounter{equation}{0}
%                                                        \renewcommand{\theequation}{C\arabic{equation}}


%%%%%%%%%%%%%%%%%%%%%%%%%%%%%%%%%%%%%%% above moved 10/1/2020


%%%%%%%%%%%%%%%%%%%%%%%%%%%%%%%%%%%%%%%%%%%%%%%%%%%%%%%%%%%%%%%%%%%%%%%%%%%%%%%%%%%%%%%%%%%%%%%%%%%%%%%%%%%%%%%%%%%%%%%%%%%%%%%%%%%%%%%%%%%%%
\newpage

%%%%%%%%%%%%%%%%%%%%%%%%%%%%%%%%%%%%%%%%%%%%%%%%%%%%%%%%%%%%%%%%%%%%%%%%%%%%%%%%%%%%%%%%%%%%%%%%%%%%%%%%%%%%%%%%%%%%%%%%%%%%%%%%%%%%%%%%%%%%%
%\newpage
%%%%%%%%%%%%%%%%%%%%%%%%%%%%%%%%%%%%%%%%% Table 1
%\begin{table}[h!]
%%\small %footnotesize
%\caption{\textbf{{Notation and description of terms}}} \vspace{2mm}
%\label{tab:notation}
%\small
%\begin{description}
%\item[-] $M_t$ denotes the pricing kernel, and $r$ denotes the (constant) interest-rate.
%\item[-] $S_t$ is the (cum dividend) equity index price at time $t$.
%\item[-] $\mathbb{E}^{\mathbb{P}}_{t}( \bullet ) \equiv  \mathbb{E}^{\mathbb{P}}( \bullet | \mathcal{F}_t )$ is the expectation under $\mathbb{P}$ conditional on $\mathcal{F}_t$.
%\item[-] $\mathbb{E}^{\mathbb{Q}}_{t}( \bullet ) \equiv  \mathbb{E}^{\mathbb{Q}}( \bullet | \mathcal{F}_t )$ is the expectation under $\mathbb{Q}$ conditional on $\mathcal{F}_t$.
%\item[-] $T_{F}$ is the maturity of the equity futures contract.
%\item[-] $F_{t}^{T_F}$ is the time $t$ price of the equity index futures with maturity $T_{F}$.
%\item[-] $G_s \equiv \frac{F_{s}^{T_F}}{F_{t}^{T_F}}$ is the gross return on the equity index futures price over the time period $t$ to $s$, with $s \geq t$.
%\item[-] $T_{O}$ is the maturity of the option on the equity index futures (with ${T}_O \leq T_F$).
%\item[-] $k=\frac{K}{F_{t}^{T_F}}$ is~option~moneyness~for~strike~price~$K$.
%\item[-] $\mathbbm{1}_{\{ G_{\ell-} > k \}}$ is an indicator function that takes a value of 1 if $G_{\ell-} > k$ and zero otherwise.
%\item[-]  Equity risk premium is $\mathbb{E}_{t}^{\mathbb{P}}( \int_{t+}^{T} \frac{dS_{\ell}}{S_t} )-
%\mathbb{E}_{t}^{\mathbb{Q}}( \int_{t+}^{T} \frac{dS_{\ell}}{S_t} )$.
%\item[-] Equity futures risk premium is $\mathbb{E}_{t}^{\mathbb{P}}( \int_{t+}^{T_O} dG_{\ell} ) =  \mathbb{E}_{t}^{\mathbb{P}}(\frac{F_{T_O}^{T_F}}{F_{t}^{T_F}})  -
%\mathbb{E}_{t}^{\mathbb{Q}}(\frac{F_{T_O}^{T_F}}{F_{t}^{T_F}})  = \mathbb{E}_{t}^{\mathbb{P}}(\frac{F_{T_O}^{T_F}}{F_{t}^{T_F}})
%-  1$.
%\item[-] Downside~futures risk premium is $\mathbb{E}_{t}^{\mathbb{P}}( \int_{t+}^{{T}_O} \mathbbm{1}_{\{G_{\ell-} < k\}} \,dG_{\ell} )$~for $k<1$.
%\item[-] Upside~ futures risk premium is $\mathbb{E}_{t}^{\mathbb{P}}( \int_{t+}^{{T}_O} \mathbbm{1}_{\{G_{\ell-} > k\}} \,dG_{\ell} )$ ~for $k>1$.
%\item[-] $\mathbf{Y}_t$ is the vector of (state) variables. \vspace{-4mm}
%\end{description}
%%%%%%%%%%%%%%%%%%%%%%%%%%%%%%%%%%%%%%%%%%%%%%%%%%%%%%%%%%%%%%%%%%%%%%%%%%%%%%%%%%%%%%%%%%%%%%
%\begin{center}
%\setlength{\tabcolsep}{0.07in}
%\begin{tabular}{ll} \hline
%          &  \\
%\multicolumn{1}{c}{Terms} & \multicolumn{1}{c}{Description} \\
%          &  \\ \hline
%%& \\
%\multicolumn{1}{l}{Semimartingale} &  A semimartingale is the most general type of process  suitable for modeling\\
%          & asset prices, and can be decomposed into a local martingale and a finite\\
%          &  variation process. They are the most general type that allows one to compute \\
%          & stochastic integrals and thus the
%          gains/losses from dynamic trading strategies.  \\ \\
%\multicolumn{1}{l}{Continuous semimartingale} & A special case of a semimartingale that has continuous sample paths \\
%& (i.e., no jumps). \\ \\
%
%\multicolumn{1}{l}{Spanned risks}   &       Risks that can be hedged away by dynamic trading in an underlying \\
%  &asset or the futures contract written on the asset. \\ \\
%\multicolumn{1}{l}{Unspanned risks}   &       Risks that are not spanned risks (as just defined) but may be hedgeable
%\\
%          & by trading in, for example, options. \\
%          \\
%\multicolumn{1}{l}{Local time, $\mathbb{L}^{T_O}_t[k]$}          &  Nonnegative process associated with a semimartingale that characterizes \\
%&           the amount of time spent by the process at a given level $k$. For a continuous \\
%& semimartingale, it can be viewed as a measure of integrated variance    \\
%& computed (only) when the process is at given level $k$. \\
%
%           \\
%          Jumps~crossing~the~strike& We define this quantity as $\sum_{t < \ell \leq T_O} \mathbbm{1}_{\{G_{\ell \, -} \leq k\}} \max( G_{\ell} - k, 0 ).$ \\
%(from below, $k>1$) & \\ \\
%          Jumps~crossing~the~strike& We define this quantity as $\sum_{t < \ell \leq T_O} \mathbbm{1}_{\{G_{\ell \, -} > k\}} \max( k - G_{\ell}, 0 )$. \\
%(from above, $k>1$) & \\
%\hline
%\end{tabular}
%\end{center}
%\end{table}
%%%%%%%%%%%%%%%%%%%%%%%%%%%%%%%%%%%%%%%%% end Table 1
%%%%%%%%%%%%%%%%%%%%%%%%%%%%%%%%%%%%%%%%%%%%%%%%%%%%%%%%%%%%%%%%%%%%%%%%%%%%%%%%%%%%%%%%%%%%%%%%





%%%%%%%%%%%%%%%%% Table 2: 8 day options %%%%%%%%%%%%%%%%%%%%%%%%%%%%%%%%%%%%%%%%%%%%%%%%%%%%%%%%
\newpage
\begin{table}[h!]
%\scriptsize
\footnotesize
\caption{\textbf{{
%Estimates of the
Risk premiums for \emph{weekly options} on the S\&P 500 index
}}}
\vspace{2mm}
\label{tab:weekly}
The sample period is 01/13/2011 to 12/20/2018, with 415 weekly option expiration cycles (8 days to maturity (on average)).
%The options data on S\&P 500 futures (respectively, S\&P 500 index) is from the CME (respectively, CBOE).
The weekly options data on S\&P 500 index is from the CBOE.
We construct the excess return of OTM puts, OTM calls, and straddles (ATM and crash-neutral) over weekly
expiration cycles. These calculations are done at the ask option price. The returns of a crash-neutral
straddle combines a long straddle position and a short 3\% OTM put position.
%Presented are the results from the following regression specification (analogously for puts and straddles):
The following is the regression specification (analogously for puts and straddles):
\begin{align*}
&q_{t,{\tiny \mathrm{call}} }^{{T}_O}[k] =
\mu_{\{ {\cal F}_{t} \in \mathfrak{s}_{\tiny \mbox{bad}} \} } \mathbbm{1}_{\{ {\cal F}_{t} \in\mathfrak{s}_{\tiny \mbox{bad}} \}}
+ \mu_{\{ {\cal F}_{t} \in \mathfrak{s}_{\tiny \mbox{normal}} \} }
\mathbbm{1}_{\{ {\cal F}_{t} \in\mathfrak{s}_{\tiny \mbox{normal}} \}}
+ \mu_{\{ {\cal F}_{t} \in \mathfrak{s}_{\tiny \mbox{good}} \} }
\mathbbm{1}_{\{ {\cal F}_{t} \in\mathfrak{s}_{\tiny \mbox{good}} \}} ~+~ \underbrace{\epsilon_{T_{O}}.}_{\mathrm{error~term}}&
\end{align*}
We use proxies for the variable $\mathfrak{s}$, known at the beginning of the expiration cycle.
The variable construction for this weekly exercise is described
in the text.
For example, WEI is the weekly economic index. % (Federal Reserve Bank of New York).
\\ %\vspace{2mm}

We indicate statistical significance at 1\%, 5\%, and 10\% by the superscripts ***, **, and *, respectively,
where the $p$-values rely on the \citet*{NeweyWest:87} HAC estimator (with the lag selected automatically).
%according to \citet*{NeweyWest:1994RES}).
The reported
put (respectively, call) delta is $-{\cal N}(-d_1)$ (respectively, ${\cal N}(d_1)$), where $d_1=  \frac{1}{ \sigma \sqrt{T_O-t}} \{ - \log k + r (T_O-t) + \frac{1}{2} \sigma^2 (T_O-t)\}$. %(following \citet*[page 718]{BollenWhaley:2004}).
SD is the standard deviation, and $\mathbbm{1}_{\{ q_{t, {T}_O} >0 \}}$ is the proportion (in \%) of
option positions that generate positive returns.
We tabulate the average open interest and trading volume, all observed on the first day of the weekly option expiration cycle.
The average number of strikes across puts and calls is 112.
\begin{center}
\setlength{\tabcolsep}{0.06in}
\begin{tabular}{lll c ccc ccc cc ccc} \hline
       &   & &           &           &           &           &           &           &  \\
       &   & &\multicolumn{3}{c}{OTM puts on equity} &           &  \multicolumn{3}{c}{OTM calls on equity}& &\multicolumn{2}{c}{Straddle} \\
       &   & &\multicolumn{3}{c}{$\log(k)\times 100$} &        & \multicolumn{3}{c}{$\log(k)\times 100$} &    &\multicolumn{2}{c}{on equity} \\
          \cline{4-6} \cline{8-10} \cline{12-13}
Moneyness (\%)   &  &      & -3         & -2         & -1         &          & 1         & 2         & 3 &    &  ATM  &     Crash-\\
Delta (\%)       &  & & -6         & -12         & -26         &           & 27         & 12         & 6 &    &   &     Neutral\\ \\
Open Interest ($\times 1,000$)    &  & & 10.2         & 9.3         & 7.4         &           & 9.1         & 7.9         & 6.9 &    &   &    \\
Volume ($\times 1,000)$     &  & & 2.5        & 2.5         & 2.6         &           & 3.1         & 2.4        & 1.8 &    &   &    \\
 &         &       &    &           &           &           &                      &  &    &    &\\ \hline
 &         &       &    &           &           &           &                      &  &    &    &\\

\multicolumn{1}{l}{Change in WEI} & \multicolumn{1}{l}{L} & \multicolumn{1}{l}{$\mathfrak{s}_{\tiny \mbox{bad}}$} & \multicolumn{1}{c}{-44} & \multicolumn{1}{c}{-36} & \multicolumn{1}{c}{-30} &           & \multicolumn{1}{c}{60} & \multicolumn{1}{c}{11} & \multicolumn{1}{c}{-53***} &    &  -2  & 0\\
\multicolumn{1}{l}{ } & \multicolumn{1}{l}{M} & \multicolumn{1}{l}{$\mathfrak{s}_{\tiny \mbox{normal}}$} & \multicolumn{1}{c}{-81***} & \multicolumn{1}{c}{-69***} & \multicolumn{1}{c}{-53***} &           & \multicolumn{1}{c}{-16} & \multicolumn{1}{c}{-46***} & \multicolumn{1}{c}{-64***}  &    & -23***   & -1***\\
\multicolumn{1}{l}{ } & \multicolumn{1}{l}{H} & \multicolumn{1}{l}{$\mathfrak{s}_{\tiny \mbox{good}}$} & \multicolumn{1}{c}{-58***} & \multicolumn{1}{c}{-32} & \multicolumn{1}{c}{-16} &           & \multicolumn{1}{c}{-8} & \multicolumn{1}{c}{-38**} & \multicolumn{1}{c}{-59***}  &    & -5   & 0\\
          &           &           &           &           &           &           &           &           &   &    &    &\\
\multicolumn{1}{l}{Quadratic Variation} & \multicolumn{1}{l}{H} & \multicolumn{1}{l}{$\mathfrak{s}_{\tiny \mbox{bad}}$} & \multicolumn{1}{c}{-42} & \multicolumn{1}{c}{-27} & \multicolumn{1}{c}{-19} &           & \multicolumn{1}{c}{3} & \multicolumn{1}{c}{-2} & \multicolumn{1}{c}{-22} &    &  -7  & 0 \\
\multicolumn{1}{l}{ } & \multicolumn{1}{l}{M} & \multicolumn{1}{l}{$\mathfrak{s}_{\tiny \mbox{normal}}$} & \multicolumn{1}{c}{-50**} & \multicolumn{1}{c}{-24} & \multicolumn{1}{c}{-9} &           & \multicolumn{1}{c}{32} & \multicolumn{1}{c}{7} & \multicolumn{1}{c}{-56***}  &    & 2   & 0\\
\multicolumn{1}{l}{ } & \multicolumn{1}{l}{L} & \multicolumn{1}{l}{$\mathfrak{s}_{\tiny \mbox{good}}$} & \multicolumn{1}{c}{-91***} & \multicolumn{1}{c}{-86***} & \multicolumn{1}{c}{-71***} &           & \multicolumn{1}{c}{0} & \multicolumn{1}{c}{-77***} & \multicolumn{1}{c}{-98***} &    &  -25***  &-2*** \\
          &           &           &           &           &           &           &           &           &  &    &    & \\
\multicolumn{1}{l}{Risk Reversal} & \multicolumn{1}{l}{H} & \multicolumn{1}{l}{$\mathfrak{s}_{\tiny \mbox{bad}}$} & \multicolumn{1}{c}{-70***} & \multicolumn{1}{c}{-51***} & \multicolumn{1}{c}{-35**} &           & \multicolumn{1}{c}{24} & \multicolumn{1}{c}{-53***} & \multicolumn{1}{c}{-92***} &    & -10   & 0\\
\multicolumn{1}{l}{ } & \multicolumn{1}{l}{M} & \multicolumn{1}{l}{$\mathfrak{s}_{\tiny \mbox{normal}}$} & \multicolumn{1}{c}{-94***} & \multicolumn{1}{c}{-75***} & \multicolumn{1}{c}{-56***} &           & \multicolumn{1}{c}{18} & \multicolumn{1}{c}{2} & \multicolumn{1}{c}{-36} &    &  -14**  & 0\\
\multicolumn{1}{l}{ } & \multicolumn{1}{l}{L} & \multicolumn{1}{l}{$\mathfrak{s}_{\tiny \mbox{good}}$} & \multicolumn{1}{c}{-19} & \multicolumn{1}{c}{-10} & \multicolumn{1}{c}{-8} &           & \multicolumn{1}{c}{-6} & \multicolumn{1}{c}{-22} & \multicolumn{1}{c}{-49***} &    & -7   &  -1\\
          &           &           &           &           &           &           &           &           &  &    &    &\\
\multicolumn{1}{l}{Change in Volatility} & \multicolumn{1}{l}{H} & \multicolumn{1}{l}{$\mathfrak{s}_{\tiny \mbox{bad}}$} & \multicolumn{1}{c}{-41} & \multicolumn{1}{c}{-38*} & \multicolumn{1}{c}{-37**} &           & \multicolumn{1}{c}{2} & \multicolumn{1}{c}{-7} & \multicolumn{1}{c}{-55***}  &    & -15**   & -1\\
\multicolumn{1}{l}{ } & \multicolumn{1}{l}{M} & \multicolumn{1}{l}{$\mathfrak{s}_{\tiny \mbox{normal}}$} & \multicolumn{1}{c}{-71***} & \multicolumn{1}{c}{-51***} & \multicolumn{1}{c}{-30*} &           & \multicolumn{1}{c}{49} & \multicolumn{1}{c}{-6} & \multicolumn{1}{c}{-43*}  &    &  -5  &  0\\
\multicolumn{1}{l}{ } & \multicolumn{1}{l}{L} & \multicolumn{1}{l}{$\mathfrak{s}_{\tiny \mbox{good}}$} & \multicolumn{1}{c}{-71***} & \multicolumn{1}{c}{-48***} & \multicolumn{1}{c}{-32**} &           & \multicolumn{1}{c}{-15} & \multicolumn{1}{c}{-59***} & \multicolumn{1}{c}{-78***} &    & -10*   & 0 \\
          &           &           &           &           &           &           &           &           &  &    &    & \\
\multicolumn{1}{l}{Recent Market} & \multicolumn{1}{l}{L} & \multicolumn{1}{l}{$\mathfrak{s}_{\tiny \mbox{bad}}$} & \multicolumn{1}{c}{-45} & \multicolumn{1}{c}{-39*} & \multicolumn{1}{c}{-28*} &           & \multicolumn{1}{c}{9} & \multicolumn{1}{c}{-21} & \multicolumn{1}{c}{-53***} &    & -13*   & -1\\
\multicolumn{1}{l}{ } & \multicolumn{1}{l}{M} & \multicolumn{1}{l}{$\mathfrak{s}_{\tiny \mbox{normal}}$} & \multicolumn{1}{c}{-56***} & \multicolumn{1}{c}{-38*} & \multicolumn{1}{c}{-24} &           & \multicolumn{1}{c}{6} & \multicolumn{1}{c}{-33} & \multicolumn{1}{c}{-59***} &    & -5   & 0\\
\multicolumn{1}{l}{ } & \multicolumn{1}{l}{H} & \multicolumn{1}{l}{$\mathfrak{s}_{\tiny \mbox{good}}$} & \multicolumn{1}{c}{-82***} & \multicolumn{1}{c}{-60***} & \multicolumn{1}{c}{-47***} &           & \multicolumn{1}{c}{21} & \multicolumn{1}{c}{-18} & \multicolumn{1}{c}{-65***} &    & -12*   & -1 \\
          &           &           &           &           &           &           &           &           &  &    &    & \\ \hline
          &           &           &           &           &           &           &           &           &  &    &    &\\
\multicolumn{1}{l}{\textbf{Unconditional}} &           & \multicolumn{1}{l}{Average} & \multicolumn{1}{c}{-61} & \multicolumn{1}{c}{-46} & \multicolumn{1}{c}{-33} &           & \multicolumn{1}{c}{12} & \multicolumn{1}{c}{-24} & \multicolumn{1}{c}{-59} &    &  -10  & -1 \\
\multicolumn{1}{l}{\textbf{Estimates}} &           & \multicolumn{1}{l}{SD} & \multicolumn{1}{c}{240} & \multicolumn{1}{c}{216} & \multicolumn{1}{c}{181} &           & \multicolumn{1}{c}{273} & \multicolumn{1}{c}{255} & \multicolumn{1}{c}{210} &    & 78   & 7 \\
\multicolumn{1}{l}{ } &           & \multicolumn{1}{l}{
$\mathbbm{1}_{\{ q_{t, {T}_O} >0 \}}$}
%$\mathbbm{1}_{\{ \mathbbm{r} >0 \}}$}
& \multicolumn{1}{c}{6\%} & \multicolumn{1}{c}{10\%} & \multicolumn{1}{c}{17\%} &           & \multicolumn{1}{c}{26\%} & \multicolumn{1}{c}{12\%} & \multicolumn{1}{c}{5\%} &    &   40\% & 43\%\\
          &           &           &           &           &           &           &           &           & &    &    & \\ \hline
\end{tabular}%
\end{center}
\end{table}
%%%%%%%%%%%%%%%%%%%%%%%%%%%%%%%%%%%%%%%%%%%%%%%%%%%%%%%%%%%%%%%%%%%%%%%%%%%%%%
%%%%%%%%%%%%%%%%%%%%%%%%%%%%%%%%%%%%%% END %%%%%%%%%%%%%%%%%%%%%%%%%%%%%%%%%%%

%%%%%%%%%%%%%%%%% Table 2: 8 day options %%%%%%%%%%%%%%%%%%%%%%%%%%%%%%%%%%%%%%%%%%%%%%%%%%%%%%%%
\newpage
\begin{table}[h!]
%\scriptsize
\footnotesize
\caption{\textbf{Disparities in option risk premiums
with weekly options}}
\vspace{2mm}
\label{tab:bootstrap}
The sample period is 01/13/2011 to 12/20/2018, with 415 weekly option expiration cycles (8 days to maturity (on average)).
We construct the excess return of OTM puts and OTM calls over weekly expiration cycles (as in Table~\ref{tab:weekly}).
These calculations are done at the ask option price.
Then we compute
\begin{align*}
&q_{t,{\tiny \mathrm{call}} }^{{T}_O}[k]{\Big |}_{\log(k)=3\%} ~~-~~ q_{t,{\tiny \mathrm{call}} }^{{T}_O}[k]{\Big |}_{\log(k)=1\%}&
&\mbox{(3\% OTM call minus 1\% OTM call)}& \mathrm{and} &\\
&q_{t,{\tiny \mathrm{call}} }^{{T}_O}[k]{\Big |}_{\log(k)=3\%} ~~-~~ q_{t,{\tiny \mathrm{put}} }^{{T}_O}[k]{\Big |}_{\log(k)=-3\%}.&
&\mbox{(3\% OTM call minus 3\% OTM put)}&  &
\end{align*}
Reported are the option risk premium differentials, partitioned according to ${\cal F}_{t} \in \mathfrak{s}_{\tiny \mbox{bad}}$, ${\cal F}_{t} \in \mathfrak{s}_{\tiny \mbox{normal}}$, and ${\cal F}_{t} \in \mathfrak{s}_{\tiny \mbox{good}}$. We employ
proxies for $\mathfrak{s}$ known at the beginning of the expiration cycle (as outlined in Table~\ref{tab:weekly}).
Reported are the two-sided $p$-values for these option risk premium differentials, relying on
the \citet*{NeweyWest:87} HAC estimator (with the lag selected
automatically).
We jointly bootstrap --- via an i.i.d, stationary, or
circular block bootstrap procedures ---
the returns of the options with
replacement and report the 95\% lower and upper confidence intervals. Bootstrap
confidence intervals --- shown as $\lfloor . \rfloor$ --- that bracket zero
imply that the disparity in the option risk premiums is indistinguishable from zero.
We perform 10,000 bootstraps. \vspace{-2mm}
\begin{center}
\setlength{\tabcolsep}{0.06in}
\begin{tabular}{ll l cc ccc ccc } \hline
          &           &           &           &           &           &           &           &           &           &                    \\
          &           &           &           &           &           \multicolumn{6}{c}{\textbf{Bootstrap procedure}}
 \\
 \cline{6-11}
          &           &           &  Estimate          &   NW[$p$]        &           & \multicolumn{1}{c}{\textbf{IID}} &           & \multicolumn{1}{c}{\textbf{Stationary}} &           & \multicolumn{1}{c}{\textbf{Circular}} \\
          &           &           &           &           &           & \multicolumn{1}{c}{$\lfloor$Lower~Upper$\rfloor$} &           & \multicolumn{1}{c}{$\lfloor$Lower~Upper$\rfloor$} &  & \multicolumn{1}{c}{$\lfloor$Lower~Upper$\rfloor$} \\
          \cline{7-7} \cline{9-9} \cline{11-11}%\cline{10-11} \cline{13-14}
          &           &           &           &           &           &           &           &           &           &                    \\ \hline
          &           &           &           &           &           &           &           &           &           &                    \\
%          &           &           & \multicolumn{1}{c}{Estimate} & \multicolumn{1}{c}{NW[$p$]} &           & \multicolumn{1}{c}{Lower} & \multicolumn{1}{c}{Upper} &           & \multicolumn{1}{c}{Lower} & \multicolumn{1}{c}{Upper} &           & \multicolumn{1}{c}{Lower} & \multicolumn{1}{c}{Upper} \\
%          &           &           &           &           &           &           &           &           &           &           \\
 & &  &        \multicolumn{8}{l}{\textbf{Panel A: Risk premium differentials}}  \\
 & &  &        \multicolumn{8}{l}{~~~~~~~~~~~~~~\textbf{(3\% OTM call minus 1\% OTM call)}}  \\
% & &  &        \multicolumn{8}{l}{\textbf{Panel A: 3\% OTM call minus 1\% OTM call}}  \\
%          &           &           &           &           &           &           &           &           &           &           &           \\
%          &           &           &           &           &           &           &           &           &           &                    \\
\multicolumn{1}{l}{Change in WEI} & L         &  $\mathfrak{s}_{\tiny \mbox{bad}}$         & -113      & \textbf{0.01}&     &$\lfloor$-181~ -51$\rfloor$       & & $\lfloor$-206 ~ -47$\rfloor$ & & $\lfloor$-207 ~ -47$\rfloor$ \\
                                 & M         &   $\mathfrak{s}_{\tiny \mbox{normal}}$        & -48       & \textbf{0.00}      &         & $\lfloor$-78~ -16$\rfloor$       &           & $\lfloor$-73~ -22$\rfloor$      &           & $\lfloor$-73~ -22$\rfloor$\\
          & H         &   $\mathfrak{s}_{\tiny \mbox{good}}$        & -51       & \textbf{0.00}      &           & $\lfloor$-85~ -16$\rfloor$       &           & $\lfloor$-68~ -34$\rfloor$       &           & $\lfloor$-67~  -35$\rfloor$ \\ \hline
\multicolumn{1}{l}{Quadratic Variation} & H         &  $\mathfrak{s}_{\tiny \mbox{bad}}$         & -26       & \textbf{0.09}      &           & $\lfloor$-55 ~ 1$\rfloor$         &           & $\lfloor$-50 ~ -1$\rfloor$        &           & $\lfloor$-51 ~ 0$\rfloor$ \\
          & M         &   $\mathfrak{s}_{\tiny \mbox{normal}}$        & -88       & \textbf{0.00}      &           & $\lfloor$-138 ~ -38$\rfloor$       &           & $\lfloor$-135~ -45$\rfloor$       &           & $\lfloor$-137 ~ -45$\rfloor$ \\
          & L         &    $\mathfrak{s}_{\tiny \mbox{good}}$       & -98       & \textbf{0.00}      &           & $\lfloor$-164 ~ -49$\rfloor$       &           & $\lfloor$-154 ~ -55$\rfloor$      &           & $\lfloor$-154~ -55$\rfloor$ \\ \hline
\multicolumn{1}{l}{Risk Reversal} & H         &    $\mathfrak{s}_{\tiny \mbox{bad}}$       & -116      & \textbf{0.00}      &           & $\lfloor$-179 ~ -65$\rfloor$       &           & $\lfloor$-170 ~ -71$\rfloor$       &           & $\lfloor$-172 ~ -71$\rfloor$ \\
          & M         &   $\mathfrak{s}_{\tiny \mbox{normal}}$        & -54       & \textbf{0.02}      &           & $\lfloor$-97  ~ -14$\rfloor$       &           & $\lfloor$-74 ~ -33$\rfloor$       &           & $\lfloor$-74 ~ -33$\rfloor$ \\
          & L         &    $\mathfrak{s}_{\tiny \mbox{good}}$       & -43       & \textbf{0.03}      &           & $\lfloor$-81 ~ -8$\rfloor$        &           & $\lfloor$-76 ~ -14$\rfloor$      &           & $\lfloor$-75 ~ -14$\rfloor$ \\ \hline
\multicolumn{1}{l}{Change in Volatility} & H         &     $\mathfrak{s}_{\tiny \mbox{bad}}$      & -56       & \textbf{0.00}      &           & $\lfloor$-88 ~ -27$\rfloor$       &           & $\lfloor$-83 ~ -32$\rfloor$       &           & $\lfloor$-83 ~ -32$\rfloor$ \\
          & M         &   $\mathfrak{s}_{\tiny \mbox{normal}}$        & -92       & \textbf{0.04}      &           & $\lfloor$-171 ~ -30$\rfloor$       &           & $\lfloor$-178 ~ -25$\rfloor$       &           & $\lfloor$-179 ~ -24$\rfloor$ \\
          & L         &    $\mathfrak{s}_{\tiny \mbox{good}}$       & -64       & \textbf{0.00}      &           & $\lfloor$-90 ~ -37$\rfloor$       &           & $\lfloor$-84 ~ -43$\rfloor$       &           & $\lfloor$-84 ~ -44$\rfloor$ \\ \hline
\multicolumn{1}{l}{Recent Market} & L         &   $\mathfrak{s}_{\tiny \mbox{bad}}$        & -62       & \textbf{0.03}      &           & $\lfloor$-119 ~ -20$\rfloor$       &           & $\lfloor$-116 ~ -22$\rfloor$       &           & $\lfloor$-115 ~ -23$\rfloor$ \\
          & M         &   $\mathfrak{s}_{\tiny \mbox{normal}}$        & -65       & \textbf{0.00}      &           & $\lfloor$-110 ~ -23$\rfloor$       &           & $\lfloor$-103 ~ -27$\rfloor$       &           & $\lfloor$-103 ~ -27$\rfloor$ \\
          & H         &  $\mathfrak{s}_{\tiny \mbox{good}}$         & -85       & \textbf{0.00}      &           & $\lfloor$-130 ~ -45$\rfloor$       &           & $\lfloor$-123 ~ -50$\rfloor$       &           & $\lfloor$-125 ~ -48$\rfloor$ \\
%                    &           &           &           &           &           &           &           &           &           &                    \\
 \hline \hline
          &           &           &           &           &           &           &           &           &           &            \\
 & &  &        \multicolumn{8}{l}{\textbf{Panel B: Risk premium differentials}}  \\
 & &  &        \multicolumn{8}{l}{~~~~~~~~~~~~~~\textbf{(3\% OTM call minus 3\% OTM put)}}  \\
% & &  &        \multicolumn{8}{l}{\textbf{Panel B: 3\% OTM call minus 3\% OTM put}}  \\
%          \multicolumn{11}{l}{\textbf{Panel B: Risk premium differentials for jumps crossing the strike, 3\% OTM call minus 3\% OTM put}}  \\
%          &           &           &           &           &           &           &           &           &           &            \\
          \multicolumn{1}{l}{Change in WEI} & L         &  $\mathfrak{s}_{\tiny \mbox{bad}}$         & -9        & \textbf{0.80}      &           & $\lfloor$-85 ~ 54$\rfloor$        &           & $\lfloor$-70 ~ 46$\rfloor$        &           & $\lfloor$-70 ~ 46$\rfloor$ \\
          & M         &  $\mathfrak{s}_{\tiny \mbox{normal}}$         & 17        & \textbf{0.36}      &           & $\lfloor$-18 ~ 54$\rfloor$        &           & $\lfloor$-12 ~ 48$\rfloor$        &           & $\lfloor$-12 ~ 48$\rfloor$ \\
          & H         &  $\mathfrak{s}_{\tiny \mbox{good}}$         & -1        & \textbf{0.97}      &           & $\lfloor$-53 ~ 53$\rfloor$        &           & $\lfloor$-26 ~ 23$\rfloor$        &           & $\lfloor$-25 ~ 22$\rfloor$ \\ \hline
\multicolumn{1}{l}{Quadratic Variation} & H         & $\mathfrak{s}_{\tiny \mbox{bad}}$          & 20        & \textbf{0.58}      &           & $\lfloor$-52 ~ 85$\rfloor$        &           & $\lfloor$-31 ~ 68$\rfloor$        &           & $\lfloor$-31 ~ 68$\rfloor$ \\
          & M         &  $\mathfrak{s}_{\tiny \mbox{normal}}$         & -6        & \textbf{0.85}      &           & $\lfloor$-63 ~ 58$\rfloor$        &           & $\lfloor$-48 ~ 36$\rfloor$        &           & $\lfloor$-49 ~ 35$\rfloor$ \\
          & L         &   $\mathfrak{s}_{\tiny \mbox{good}}$        & -7        & \textbf{0.31}      &           & $\lfloor$-24 ~ 4$\rfloor$         &           & $\lfloor$-20 ~  2$\rfloor$         &           & $\lfloor$-20 ~ 2$\rfloor$ \\ \hline
\multicolumn{1}{l}{Risk Reversal} & H         &    $\mathfrak{s}_{\tiny \mbox{bad}}$       & -22       & \textbf{0.20}      &           & $\lfloor$-56 ~ 9$\rfloor$         &           & $\lfloor$-52 ~ 4$\rfloor$         &           & $\lfloor$-52 ~ 4$\rfloor$ \\
          & M         &   $\mathfrak{s}_{\tiny \mbox{normal}}$        & 58        & \textbf{0.02}      &           & $\lfloor$15 ~ 106$\rfloor$       &           & $\lfloor$21 ~ 99$\rfloor$        &           & $\lfloor$22 ~ 100$\rfloor$ \\
          & L         &  $\mathfrak{s}_{\tiny \mbox{good}}$         & -30       & \textbf{0.40}      &           & $\lfloor$-110 ~ 36$\rfloor$        &           & $\lfloor$-98 ~ 27$\rfloor$        &           & $\lfloor$-97 ~ 29$\rfloor$ \\ \hline
\multicolumn{1}{l}{Change in Volatility} & H         &   $\mathfrak{s}_{\tiny \mbox{bad}}$        & -14       & \textbf{0.66}      &           & $\lfloor$-87 ~ 44$\rfloor$        &           & $\lfloor$-71 ~ 37$\rfloor$        &           & $\lfloor$-69 ~ 34$\rfloor$ \\
          & M         & $\mathfrak{s}_{\tiny \mbox{normal}}$          & 28        & \textbf{0.33}      &           & $\lfloor$-28 ~ 86$\rfloor$        &           & $\lfloor$1 ~ 55$\rfloor$        &           & $\lfloor$3 ~ 54$\rfloor$ \\
          & L         & $\mathfrak{s}_{\tiny \mbox{good}}$          & -8        & \textbf{0.66}      &           &$\lfloor$-43 ~ 27$\rfloor$        &           & $\lfloor$-36 ~ 21$\rfloor$        &           & $\lfloor$-36 ~ 21$\rfloor$ \\ \hline
\multicolumn{1}{l}{Recent Market} & L         &   $\mathfrak{s}_{\tiny \mbox{bad}}$        & -8        & \textbf{0.78}      &           & $\lfloor$-75 ~ 46$\rfloor$        &           & $\lfloor$-65 ~ 40$\rfloor$        &           & $\lfloor$-61 ~ 37$\rfloor$ \\
          & M         &   $\mathfrak{s}_{\tiny \mbox{normal}}$        & -3        & \textbf{0.92}      &           & $\lfloor$-58 ~ 54$\rfloor$        &           & $\lfloor$-28 ~ 23$\rfloor$        &           & $\lfloor$-28 ~ 23$\rfloor$ \\
          & H         &   $\mathfrak{s}_{\tiny \mbox{good}}$        & 18        & \textbf{0.38}      &           & $\lfloor$-16 ~ 57$\rfloor$        &           & $\lfloor$-12 ~ 53$\rfloor$        &           & $\lfloor$-12 ~ 53$\rfloor$ \\
%                    &           &           &           &           &           &           &           &           &           &                    \\
\hline \hline
\end{tabular}%
\end{center}
\end{table}
%%%%%%%%%%%%%%%%%%%%%%%%%%%%%%%%%%%%%%%%%%%%%%%%%%%%%%%%%%%%%%%%%%%%%%%%%%%%%%
%%%%%%%%%%%%%%%%%%%%%%%%%%%%%%%%%%%%%% END %%%%%%%%%%%%%%%%%%%%%%%%%%%%%%%%%%%






%%%%%%%%%%%%%%%%%%%%%%%%%%%%%%%%%%%%%% END %%%%%%%%%%%%%%%%%%%%%%%%%%%%%%%%%%%


%%%%%%%%%%%%%%%%% Table 3: 28 day options %%%%%%%%%%%%%%%%%%%%%%%%%%%%%%%%%%%%%%%%%%%%%%%%%%%%%%%%
\newpage
\begin{table}[h!]
%\scriptsize
\footnotesize
\caption{\textbf{{Risk premiums for \emph{28-day} options on the S\&P 500 \emph{index}}}} \vspace{2mm}
\label{tab:equity_options}
The sample period is 01/22/1990 to 12/24/2018, with 348 option expiration cycles (28 days to maturity (on average)).
The 28-day options data on S\&P 500 index is from the CBOE.
We construct the excess return of OTM puts, OTM calls, and straddles (ATM and crash-neutral) over expiration cycles. These calculations are done at the ask option price.
The returns of a crash-neutral
straddle combines a long straddle position and a short 5\% OTM put position.
The following is the regression specification (analogously for puts and straddles):
\begin{align*}
&q_{t,{\tiny \mathrm{call}} }^{{T}_O}[k] =
\mu_{\{ {\cal F}_{t} \in \mathfrak{s}_{\tiny \mbox{bad}} \} } \mathbbm{1}_{\{ {\cal F}_{t} \in\mathfrak{s}_{\tiny \mbox{bad}} \}}
+ \mu_{\{ {\cal F}_{t} \in \mathfrak{s}_{\tiny \mbox{normal}} \} }
\mathbbm{1}_{\{ {\cal F}_{t} \in\mathfrak{s}_{\tiny \mbox{normal}} \}}
+ \mu_{\{ {\cal F}_{t} \in \mathfrak{s}_{\tiny \mbox{good}} \} }
\mathbbm{1}_{\{ {\cal F}_{t} \in\mathfrak{s}_{\tiny \mbox{good}} \}} +
\epsilon_{T_{O}}.&
\end{align*}
We use the following proxies for the variable $\mathfrak{s}$, known at the beginning of the expiration cycle.
\begin{description}
\item[-] \textit{Dividend Yield}$_{t}$: A high dividend yield (from Robert Shiller's website)
aligns with bad states. % (\citet*{Cochrane:2008RFS}).

\item[-] \textit{Quadratic Variation}$_t$. Sum of daily squared (log) returns over
the \emph{prior} expiration cycle.



\item[-] \textit{Risk Reversal$_t$} ($\log(\frac{\mathrm{IV_t^{\mathrm{put}}}[k]}{\mathrm{IV_t^{\mathrm{call}}}[k]}$)). The 28-day implied volatility for puts (calls) uses $\log(k)$ equal to $-3\%$ (3\%).
    %\vspace{-3mm}

\item[-] \textit{Change in Volatility$_t$} ($\log(\frac{\mathrm{IV}^{\mathrm{atm}}_{t}}{\mathrm{IV}^{\mathrm{atm}}_{t-1}})$).
The 28-day implied volatility (IV$_{t}$) is the average across
ATM puts and calls.

%\item[-] \textit{Yield~Spread}$_{t}$: Difference between the 30-year and one-year Treasury yields at the start of the expiration cycle.
%(source: CRSP (daily) Fixed Term Indices Files).

\item[-] \textit{Recent Market}$_{t}$: Log relative of the S\&P 500 index over the prior expiration cycle.


\end{description}
%The variable construction is described in the text.
We indicate statistical significance at 1\%, 5\%, and 10\% by the superscripts ***, **, and  *, respectively,
where the $p$-values rely on
the \citet*{NeweyWest:87} HAC
estimator (with the lag selected automatically).
The reported put (respectively, call) delta is $-{\cal N}(-d_1)$ (respectively, ${\cal N}(d_1)$), where $d_1=  \frac{1}{ \sigma \sqrt{T_O-t}} \{ - \log k +r (T_O-t) + \frac{1}{2} \sigma^2 (T_O-t)\}$.
SD is the standard deviation, and
$\mathbbm{1}_{\{ q_{t, {T}_O} >0 \}}$ is the proportion (in \%) of option
positions that generate positive returns.
We tabulate the average open interest and trading volume, all observed on the first day of the option expiration cycle.
%\vspace{-3mm}
\begin{center}
\setlength{\tabcolsep}{0.053in}
\begin{tabular}{lll ccc ccc ccc c} \hline
          &           &           &           &           &           &           &           &           &           &           &           &  \\
          & &&\multicolumn{3}{c}{OTM puts on equity} &     & \multicolumn{3}{c}{OTM calls on equity} &  &  \multicolumn{2}{c}{Straddle}\\
          & & &\multicolumn{3}{c}{$\log(k)\times 100$} &   & \multicolumn{3}{c}{$\log(k)\times 100$}  &  &\multicolumn{2}{c}{on equity} \\
          \cline{4-6} \cline{8-10} \cline{12-13}
\multicolumn{1}{l}{Moneyness (\%)} &           &           & -5         & -3         & -1         &           & 1         & 3         & 5         &           & \multicolumn{1}{l}{ATM} & \multicolumn{1}{l}{Crash-} \\
\multicolumn{1}{l}{Delta (\%)} &           &           & -9         & -18         & -35         &           & 41         & 22         & 11         &           & & Neutral\\ \\
\multicolumn{1}{l}{Open interest ($\times 1,000$)} &           &           & 19.3      & 18.2      & 17.0      &           & 16.1      & 15.9      & 14.2      &           &           &  \\
  Volume ($\times 1,000$)       &           &           & 2.6       & 2.5       & 2.9       &           & 2.3       & 2.4       & 1.9       &           &           &  \\
          &           &           &           &           &           &           &           &           &           &           &           &  \\ \hline
          &           &           &           &           &           &           &           &           &           &           &           &  \\
\multicolumn{1}{l}{Dividend Yield} & \multicolumn{1}{l}{H} & \multicolumn{1}{l}{$\mathfrak{s}_{\tiny \mbox{bad}}$ } & \multicolumn{1}{c}{-83***} & \multicolumn{1}{c}{-75***} & \multicolumn{1}{c}{-64***} &           & \multicolumn{1}{c}{7} & \multicolumn{1}{c}{-4} & \multicolumn{1}{c}{-21} &           & \multicolumn{1}{c}{-26***} & \multicolumn{1}{c}{-4***} \\
\multicolumn{1}{l}{} & \multicolumn{1}{l}{M} &   $\mathfrak{s}_{\tiny \mbox{normal}}$         & \multicolumn{1}{c}{-61***} & \multicolumn{1}{c}{-42***} & \multicolumn{1}{c}{-34**} &           & \multicolumn{1}{c}{27} & \multicolumn{1}{c}{12} & \multicolumn{1}{c}{-19} &           & \multicolumn{1}{c}{-7} & \multicolumn{1}{c}{0} \\
\multicolumn{1}{l}{} &      L     &   $\mathfrak{s}_{\tiny \mbox{good}}$         & \multicolumn{1}{c}{-57***} & \multicolumn{1}{c}{-37**} & \multicolumn{1}{c}{-28*} &           & \multicolumn{1}{c}{-21**} & \multicolumn{1}{c}{-38***} & \multicolumn{1}{c}{-53***} &           & \multicolumn{1}{c}{-21***} & \multicolumn{1}{c}{-4***} \\
          &           &           &           &           &           &           &           &           &           &           &           &  \\
\multicolumn{1}{l}{Quadratic Variation} &    H       &  $\mathfrak{s}_{\tiny \mbox{bad}}$         & \multicolumn{1}{c}{-55***} & \multicolumn{1}{c}{-49***} & \multicolumn{1}{c}{-45***} &           & \multicolumn{1}{c}{7} & \multicolumn{1}{c}{13} & \multicolumn{1}{c}{29} &           & \multicolumn{1}{c}{-19***} & \multicolumn{1}{c}{-3*} \\
\multicolumn{1}{l}{} &    M       &   $\mathfrak{s}_{\tiny \mbox{normal}}$        & \multicolumn{1}{c}{-60***} & \multicolumn{1}{c}{-47***} & \multicolumn{1}{c}{-41***} &           & \multicolumn{1}{c}{9} & \multicolumn{1}{c}{-7} & \multicolumn{1}{c}{-22} &           & \multicolumn{1}{c}{-15***} & \multicolumn{1}{c}{-2**} \\
\multicolumn{1}{l}{} &   L        &  $\mathfrak{s}_{\tiny \mbox{good}}$         & \multicolumn{1}{l}{-87***} & \multicolumn{1}{l}{-58***} & \multicolumn{1}{l}{-40***} &           & \multicolumn{1}{c}{-3} & \multicolumn{1}{c}{-37*} & \multicolumn{1}{c}{-100***} &           & \multicolumn{1}{c}{-21***} & \multicolumn{1}{c}{-3**} \\
          &           &           &           &           &           &           &           &           &           &           &           &  \\
\multicolumn{1}{l}{Risk Reversal} &   H        &    $\mathfrak{s}_{\tiny \mbox{bad}}$       & \multicolumn{1}{c}{-71***} & \multicolumn{1}{c}{-50***} & \multicolumn{1}{c}{-39***} &           & \multicolumn{1}{c}{20} & \multicolumn{1}{c}{-4} & \multicolumn{1}{c}{-37} &           & \multicolumn{1}{c}{-14**} & \multicolumn{1}{c}{-1} \\
\multicolumn{1}{l}{} &    M       &  $\mathfrak{s}_{\tiny \mbox{normal}}$         & \multicolumn{1}{l}{-83***} & \multicolumn{1}{l}{-67***} & \multicolumn{1}{l}{-59***} &           & \multicolumn{1}{c}{15} & \multicolumn{1}{c}{7} & \multicolumn{1}{c}{-16} &           & \multicolumn{1}{c}{-19***} & \multicolumn{1}{c}{-2} \\
\multicolumn{1}{l}{} &     L      &     $\mathfrak{s}_{\tiny \mbox{good}}$      & \multicolumn{1}{c}{-45**} & \multicolumn{1}{c}{-35*} & \multicolumn{1}{c}{-26} &           & \multicolumn{1}{c}{-25***} & \multicolumn{1}{c}{-42***} & \multicolumn{1}{c}{-56***} &           & \multicolumn{1}{c}{-21***} & \multicolumn{1}{c}{-4***} \\
          &           &           &           &           &           &           &           &           &           &           &           &  \\
\multicolumn{1}{l}{Change in Volatility} &   H        &   $\mathfrak{s}_{\tiny \mbox{bad}}$        & \multicolumn{1}{c}{-52***} & \multicolumn{1}{c}{-33**} & \multicolumn{1}{c}{-25*} &           & \multicolumn{1}{c}{12} & \multicolumn{1}{c}{10} & \multicolumn{1}{c}{-6} &           & \multicolumn{1}{c}{-8} & \multicolumn{1}{c}{-1} \\
\multicolumn{1}{l}{} &     M      &   $\mathfrak{s}_{\tiny \mbox{normal}}$        & \multicolumn{1}{c}{-83***} & \multicolumn{1}{c}{-62***} & \multicolumn{1}{c}{-53***} &           & \multicolumn{1}{c}{2} & \multicolumn{1}{c}{-18} & \multicolumn{1}{c}{-29} &           & \multicolumn{1}{c}{-25***} & \multicolumn{1}{c}{-3***} \\
\multicolumn{1}{l}{} &    L       &  $\mathfrak{s}_{\tiny \mbox{good}}$         & \multicolumn{1}{l}{-65***} & \multicolumn{1}{l}{-57***} & \multicolumn{1}{l}{-46***} &           & \multicolumn{1}{c}{-4} & \multicolumn{1}{c}{-33*} & \multicolumn{1}{c}{-81***} &           & \multicolumn{1}{c}{-22***} & \multicolumn{1}{c}{-3***} \\
%\multicolumn{1}{l}{Yield Spread} &           &           & \multicolumn{1}{l}{-74***} & \multicolumn{1}{l}{-47***} & \multicolumn{1}{l}{-40***} &           & \multicolumn{1}{l}{8} & \multicolumn{1}{l}{19} & \multicolumn{1}{l}{-5} &           & \multicolumn{1}{l}{-15**} & \multicolumn{1}{l}{-2} \\
%\multicolumn{1}{l}{} &           &           & \multicolumn{1}{l}{-61***} & \multicolumn{1}{l}{-47***} & \multicolumn{1}{l}{-38**} &           & \multicolumn{1}{l}{3} & \multicolumn{1}{l}{-29} & \multicolumn{1}{l}{-53*} &           & \multicolumn{1}{l}{-19***} & \multicolumn{1}{l}{-2**} \\
%\multicolumn{1}{l}{} &           &           & \multicolumn{1}{l}{-67***} & \multicolumn{1}{l}{-59***} & \multicolumn{1}{l}{-48***} &           & \multicolumn{1}{l}{2} & \multicolumn{1}{l}{-20} & \multicolumn{1}{l}{-35} &           & \multicolumn{1}{l}{-22***} & \multicolumn{1}{l}{-2*} \\
          &           &           &           &           &           &           &           &           &           &           &           &  \\
\multicolumn{1}{l}{Recent Market} &  L         &  $\mathfrak{s}_{\tiny \mbox{bad}}$         & \multicolumn{1}{c}{-58***} & \multicolumn{1}{c}{-48***} & \multicolumn{1}{c}{-43***} &           & \multicolumn{1}{c}{14} & \multicolumn{1}{c}{7} & \multicolumn{1}{c}{8} &           & \multicolumn{1}{c}{-17**} & \multicolumn{1}{c}{-2} \\
\multicolumn{1}{l}{} &    M       & $\mathfrak{s}_{\tiny \mbox{normal}}$          & \multicolumn{1}{c}{-75***} & \multicolumn{1}{c}{-48***} & \multicolumn{1}{c}{-38***} &           & \multicolumn{1}{c}{17} & \multicolumn{1}{c}{11} & \multicolumn{1}{c}{-18} &           & \multicolumn{1}{c}{-13**} & \multicolumn{1}{c}{-2*} \\
\multicolumn{1}{l}{} &     H      &  $\mathfrak{s}_{\tiny \mbox{good}}$         & \multicolumn{1}{c}{-68***} & \multicolumn{1}{c}{-58***} & \multicolumn{1}{c}{-44***} &           & \multicolumn{1}{c}{-18} & \multicolumn{1}{c}{-48***} & \multicolumn{1}{c}{-82***} &           & \multicolumn{1}{c}{-25***} & \multicolumn{1}{c}{-4***} \\
          &           &           &           &           &           &           &           &           &           &           &           &  \\ \hline
          &           &           &           &           &           &           &           &           &           &           &           &  \\
\multicolumn{1}{l}{\textbf{Unconditional}} &           &   Average        & \multicolumn{1}{c}{-67} & \multicolumn{1}{c}{-51} & \multicolumn{1}{c}{-42} &           & \multicolumn{1}{c}{4} & \multicolumn{1}{c}{-10} & \multicolumn{1}{c}{-31} &           & \multicolumn{1}{c}{-19} & \multicolumn{1}{c}{-3} \\
\multicolumn{1}{l}{\textbf{Estimates}} &           &    SD       & \multicolumn{1}{c}{151} & \multicolumn{1}{c}{159} & \multicolumn{1}{c}{146} &           & \multicolumn{1}{c}{151} & \multicolumn{1}{c}{253} & \multicolumn{1}{c}{360} &           & \multicolumn{1}{c}{72} & \multicolumn{1}{c}{14} \\
\multicolumn{1}{l}{} &           &     $\mathbbm{1}_{\{ q_{t, {T}_O} >0 \}}$      & \multicolumn{1}{c}{6\%} & \multicolumn{1}{c}{11\%} & \multicolumn{1}{c}{16\%} &           & \multicolumn{1}{c}{38\%} & \multicolumn{1}{c}{19\%} & \multicolumn{1}{c}{8\%} &           & \multicolumn{1}{c}{30\%} & \multicolumn{1}{c}{38\%} \\
          &           &           &           &           &           &           &           &           &           &           &           &  \\
\hline
\end{tabular}%
\end{center}
\end{table}
%%%%%%%%%%%%%%%%%%%%%%%%%%%%%%%%%%%%%%%%%%%%%%%%%%%%%%%%%%%%%%%%%%%%%%%%%%%%%%

%%%%%%%%%%%%%%%%% Table 2: 28 day options %%%%%%%%%%%%%%%%%%%%%%%%%%%%%%%%%%%%%%%%%%%%%%%%%%%%%%%%
\newpage
\begin{table}[h!]
%\scriptsize
\footnotesize
\caption{\textbf{{%Estimates of the
Risk premiums for \emph{28-day} options on the S\&P 500 \emph{futures}}}} \vspace{2mm}
%\caption{\textbf{{Estimates of the equity futures
%option risk premiums (S\&P 500 futures index, 28 days to maturity (on average))}}} \vspace{2mm}
\label{tab:table2}
The sample period is 01/18/1988 to 05/23/2016, with 341 option expiration cycles (28 days to maturity (on average)).
These one-month futures options were discontinued and only the three-month
options were traded after that.
We construct the excess return of OTM puts, OTM calls, and straddles (ATM and crash-neutral) over option expiration cycles. The option settlement price is provided by the CME.
The returns of a crash-neutral straddle combines a long straddle position and a short 5\% OTM put position.
The following is the regression specification (analogously for puts and straddles):
\begin{align*}
&q_{t,{\tiny \mathrm{call}} }^{{T}_O}[k] =
\mu_{\{ {\cal F}_{t} \in \mathfrak{s}_{\tiny \mbox{bad}} \} } \mathbbm{1}_{\{ {\cal F}_{t} \in\mathfrak{s}_{\tiny \mbox{bad}} \}}
+ \mu_{\{ {\cal F}_{t} \in \mathfrak{s}_{\tiny \mbox{normal}} \} }
\mathbbm{1}_{\{ {\cal F}_{t} \in\mathfrak{s}_{\tiny \mbox{normal}} \}}
+ \mu_{\{ {\cal F}_{t} \in \mathfrak{s}_{\tiny \mbox{good}} \} }
\mathbbm{1}_{\{ {\cal F}_{t} \in\mathfrak{s}_{\tiny \mbox{good}} \}} +
\epsilon_{T_{O}}.&
\end{align*}
%We use the following proxies for the variable $\mathfrak{s}$, known at the beginning of the expiration cycle.
The proxies for the variable $\mathfrak{s}$, which are known at the beginning of the option expiration cycle, are as described in
the note to Table~\ref{tab:equity_options}.
%\begin{description}
%\item[-] \textit{Dividend Yield}$_{t}$: A high dividend yield aligns with bad states.
%
%\item[-] \textit{Quadratic Variation}$_t$. Sum of daily squared (log) returns over
%the \emph{prior} expiration cycle.
%
%\item[-] \textit{Risk Reversal$_t$} ($\log(\frac{\mathrm{IV_t^{\mathrm{put}}}[k]}{\mathrm{IV_t^{\mathrm{call}}}[k]}$)). The 28-day implied volatility for puts (calls) uses $\log(k)$ equal to $-3\%$ (3\%).
%    %\vspace{-3mm}
%
%\item[-] \textit{Change in Volatility$_t$} ($\log(\frac{\mathrm{IV}^{\mathrm{atm}}_{t}}{\mathrm{IV}^{\mathrm{atm}}_{t-1}})$).
%The 28-day implied volatility (IV$_{t}$) is the average across
%ATM puts and calls.
%
%%\item[-] \textit{Yield~Spread}$_{t}$: Difference between the 30-year and one-year Treasury yields at the start of the expiration cycle.
%%(source: CRSP (daily) Fixed Term Indices Files).
%
%\item[-] \textit{Recent Market}$_{t}$: Log relative of the S\&P 500 futures price over the prior expiration cycle.
%
%
%\end{description}
We indicate statistical significance at 1\%, 5\%, and 10\% by the superscripts ***, **, and *, respectively,
where the $p$-values rely on
the \citet*{NeweyWest:87} HAC estimator (with the lag selected automatically).
 The reported put (respectively, call) delta is $-e^{- r (T_O-t)} {\cal N}(-d_1)$ (respectively, $e^{- r (T_O-t)} {\cal N}(d_1)$),
where $d_1=  \frac{1}{ \sigma \sqrt{T_O-t}} \{ - \log k + \frac{1}{2} \sigma^2 (T_O-t)\}$.
SD is the standard deviation,
and $\mathbbm{1}_{\{ q_{t, {T}_O} >0 \}}$ is the proportion (in \%) of option positions
that generate positive returns. % \vspace{-2mm}
We tabulate the average open interest and trading volume, all observed on the first day of the option expiration cycle.

\begin{center}
\setlength{\tabcolsep}{0.065in}
\begin{tabular}{lll ccc ccc ccc c} \hline
          &           &           &           &           &           &           &           &           &           &           &           &  \\
          & &&\multicolumn{3}{c}{OTM puts on futures} &     & \multicolumn{3}{c}{OTM calls on futures} &  &  \multicolumn{2}{c}{Straddle}\\
          & & &\multicolumn{3}{c}{$\log(k)\times 100$} &   & \multicolumn{3}{c}{$\log(k)\times 100$}  &  &\multicolumn{2}{c}{on  futures} \\
          \cline{4-6} \cline{8-10} \cline{12-13}
\multicolumn{1}{l}{Moneyness (\%)} &           &           & -5         & -3         & -1         &           & 1        & 3        & 5        &           & \multicolumn{1}{l}{ATM} & \multicolumn{1}{l}{Crash-} \\
\multicolumn{1}{l}{Delta (\%)} &           &           & -9         & -18         & -35         &           & 41        & 22         & 11         &           & & Neutral\\ \\
\multicolumn{1}{l}{Open Interest} &           &          & 1708      & 1544      & 1254      &           & 1560      & 1974      & 1792      &           &           &  \\
Volume          &           &           & 260       & 218       & 204       &           & 142       & 291       & 255       &           &           &  \\
          &           &           &           &           &           &           &           &           &           &           &           &  \\ \hline
          &           &           &           &           &           &           &           &           &           &           &           &  \\
\multicolumn{1}{l}{Dividend Yield } & \multicolumn{1}{l}{H} & \multicolumn{1}{l}{$\mathfrak{s}_{\tiny \mbox{bad}}$} & -77***    & -70***    & -60***    &           & 5         & -15       & -39       &           & -30***    & -4*** \\
\multicolumn{1}{l}{} &     M      &     $\mathfrak{s}_{\tiny \mbox{normal}}$      & -57***    & -41***    & -31**     &           & 23        & 17        & 31        &           & -4        & 1 \\
\multicolumn{1}{l}{} &    L       &     $\mathfrak{s}_{\tiny \mbox{good}}$      & -51***    & -33*      & -25*      &           & -19**     & -35***    & -46***    &           & -18***    & -2** \\
          &           &           &           &           &           &           &           &           &           &           &           &  \\
\multicolumn{1}{l}{Quadratic Variation} &     H      &    $\mathfrak{s}_{\tiny \mbox{bad}}$       & -57***    & -53***    & -45***    &           & 12        & 19        & 25        &           & -16**     & -1 \\
\multicolumn{1}{l}{} &      M     &    $\mathfrak{s}_{\tiny \mbox{normal}}$       & -44**     & -33*      & -32**     &           & 12        & 1         & 10        &           & -10       & -1 \\
\multicolumn{1}{l}{} &      L     &    $\mathfrak{s}_{\tiny \mbox{good}}$       & -84***    & -58***    & -40***    &           & -15       & -52***    & -90***    &           & -25***    & -3*** \\
          &           &           &           &           &           &           &           &           &           &           &           &  \\
\multicolumn{1}{l}{Risk Reversal} &   H        & $\mathfrak{s}_{\tiny \mbox{bad}}$          & -52***    & -31*      & -22       &           & 8         & -4        & -7        &           & -7        & 0 \\
\multicolumn{1}{l}{} &      M     &    $\mathfrak{s}_{\tiny \mbox{normal}}$       & -88***    & -73***    & -60***    &           & 15        & 3         & 7         &           & -21***    & -1 \\
\multicolumn{1}{l}{} &      L     &    $\mathfrak{s}_{\tiny \mbox{good}}$       & -42*      & -38*      & -34*      &           & -18       & -35***    & -63***    &           & -24***    & -4*** \\
          &           &           &           &           &           &           &           &           &           &           &           &  \\
\multicolumn{1}{l}{Change in Volatility} &     H      &  $\mathfrak{s}_{\tiny \mbox{bad}}$         & -60***    & -43***    & -34**     &           & 6         & 15        & 46        &           & -15**     & -1 \\
\multicolumn{1}{l}{} &     M      &     $\mathfrak{s}_{\tiny \mbox{normal}}$      & -67***    & -46***    & -39***    &           & -1        & -11       & -29       &           & -20***    & -2 \\
\multicolumn{1}{l}{} &      L     &    $\mathfrak{s}_{\tiny \mbox{good}}$       & -58***    & -56***    & -44***    &           & 4         & -36**     & -73***    &           & -17***    & -2 \\
          &           &           &           &           &           &           &           &           &           &           &           &  \\
%\multicolumn{1}{l}{Yield Spread} &  L         &  $\mathfrak{s}_{\tiny \mbox{bad}}$         & -73***    & -51***    & -44***    &           & 0         & -18       & -36       &           & -22***    & -2* \\
%\multicolumn{1}{l}{} &   M        &  $\mathfrak{s}_{\tiny \mbox{normal}}$         & -46**     & -35*      & -27       &           & 2         & -2        & 16        &           & -12       & -1 \\
%\multicolumn{1}{l}{} &   H        &   $\mathfrak{s}_{\tiny \mbox{good}}$        & -66***    & -58***    & -46***    &           & 7         & -13       & -35       &           & -18**     & -1 \\
%          &           &           &           &           &           &           &           &           &           &           &           &  \\
\multicolumn{1}{l}{Recent Market} &   L        &    $\mathfrak{s}_{\tiny \mbox{bad}}$       & -55***    & -42**     & -38***    &           & 22        & 18        & 19        &           & -11       & 0 \\
\multicolumn{1}{l}{} &    M       &   $\mathfrak{s}_{\tiny \mbox{normal}}$        & -71***    & -51***    & -40***    &           & 3         & -1        & 2         &           & -19***    & -2 \\
\multicolumn{1}{l}{} &     H      &   $\mathfrak{s}_{\tiny \mbox{good}}$        & -59***    & -52***    & -39***    &           & -15       & -49***    & -76***    &           & -22***    & -3** \\
          &           &           &           &           &           &           &           &           &           &           &           &  \\ \hline
          &           &           &           &           &           &           &           &           &           &           &           &  \\
\multicolumn{1}{l}{\textbf{Unconditional}} &           &     Average      & -62       & -48       & -39       &           & 3         & -11       & -18       &           & -17       & -2 \\
\multicolumn{1}{l}{\textbf{Estimates}} &           &      SD     & 167       & 166       & 148       &           & 145       & 241       & 441       &           & 74        & 14 \\
\multicolumn{1}{l}{} &           &     $\mathbbm{1}_{\{ q_{t, {T}_O} >0 \}}$       & 6\%         & 12\%        & 17\%        &           & 37\%        & 20\%        & 7\%         &           & 32\%        & 42\% \\
          &           &           &           &           &           &           &           &           &           &           &           &  \\ \hline
\end{tabular}
\end{center}
\end{table}
%%%%%%%%%%%%%%%%%%%%%%%%%%%%%%%%%%%%%%%%%%%%%%%%%%%%%%%%%%%%%%%%%%%%%%%%%%%%%%
%%%%%%%%%%%%%%%%%%%%%%%%%%%%%%%%%%%%%% END %%%%%%%%%%%%%%%%%%%%%%%%%%%%%%%%%%%


%%%%%%%%%%%%%%%%% Table 2: 28 day options %%%%%%%%%%%%%%%%%%%%%%%%%%%%%%%%%%%%%%%%%%%%%%%%%%%%%%%%
\newpage
\begin{table}[h!]
%\scriptsize
\footnotesize
\caption{\textbf{{%Estimates of the equity
Risk premiums for \emph{88-day} options on the S\&P 500 \emph{futures}}}} \vspace{2mm}
\label{tab:opretspfutx}
The sample period is 03/21/1988 to 03/18/2019,
with 125 option expiration cycles (88 days to maturity (on average)). We construct the excess return of OTM puts, OTM calls, and straddles (ATM and crash-neutral) over option expiration cycles.
The option settlement price is provided by the CME.
The returns of a crash-neutral
straddle combines a long straddle position and a short 12\% OTM put position.
The following is the
regression specification (analogously for puts and straddles):
\begin{align*}
&q_{t,{\tiny \mathrm{call}} }^{{T}_O}[k] =
\mu_{\{ {\cal F}_{t} \in \mathfrak{s}_{\tiny \mbox{bad}} \} } \mathbbm{1}_{\{ {\cal F}_{t} \in\mathfrak{s}_{\tiny \mbox{bad}} \}}
+ \mu_{\{ {\cal F}_{t} \in \mathfrak{s}_{\tiny \mbox{normal}} \} }
\mathbbm{1}_{\{ {\cal F}_{t} \in\mathfrak{s}_{\tiny \mbox{normal}} \}}
+ \mu_{\{ {\cal F}_{t} \in \mathfrak{s}_{\tiny \mbox{good}} \} }
\mathbbm{1}_{\{ {\cal F}_{t} \in\mathfrak{s}_{\tiny \mbox{good}} \}} +
\epsilon_{T_{O}}.&
\end{align*}
The proxies for the variable $\mathfrak{s}$, which are known at the beginning of the option expiration cycle, are as described in
the note to Table~\ref{tab:equity_options}.
%We use the following proxies for the variable $\mathfrak{s}$, known at the beginning of the expiration cycle.
%\begin{description}
%\item[-] \textit{Dividend Yield}$_{t}$: A high dividend yield aligns with bad states.
%
%\item[-] \textit{Quadratic Variation}$_t$. Sum of daily squared (log) returns over
%the \emph{prior} expiration cycle.
%
%\item[-] \textit{Risk Reversal}$_t$ ($\log(\frac{\mathrm{IV_t^{\mathrm{put}}}[k]}{\mathrm{IV_t^{\mathrm{call}}}[k]}$)). The 88-day implied volatility for puts (calls) uses $\log(k)$ equal to $-8\%$ (8\%).
%    %\vspace{-3mm}
%
%\item[-] \textit{Change in Volatility$_t$} ($\log(\frac{\mathrm{IV}^{\mathrm{atm}}_{t}}{\mathrm{IV}^{\mathrm{atm}}_{t-1}})$).
%The 88-day implied volatility (IV$_{t}$) is the average across ATM puts and calls.
%
%%\item[-] \textit{Yield~Spread}$_{t}$: Difference between the 30-year and one-year Treasury yields at the start of the expiration cycle.
%%(source: CRSP (daily) Fixed Term Indices Files).
%
%\item[-] \textit{Recent Market}$_{t}$: Log relative of the S\&P 500 futures price over the prior expiration cycle.
%
%
%\end{description}
We indicate statistical significance at 1\%, 5\%, and 10\% by the superscripts ***, **, and *, respectively,
where the $p$-values rely on
the \citet*{NeweyWest:87} HAC estimator (with the lag selected automatically).
 The reported put (respectively, call) delta is $-e^{- r (T_O-t)} {\cal N}(-d_1)$ (respectively, $e^{- r (T_O-t)} {\cal N}(d_1)$),
where $d_1=  \frac{1}{ \sigma \sqrt{T_O-t}} \{ - \log k + \frac{1}{2} \sigma^2 (T_O-t)\}$.
SD is the standard deviation, and $\mathbbm{1}_{\{ q_{t, {T}_O} >0 \}}$ is the proportion (in \%) of option positions
that generate positive returns.
We tabulate the average open interest and trading volume, all observed on the first day of the option expiration cycle.
\vspace{-2mm}
\begin{center}
\setlength{\tabcolsep}{0.058in}
\begin{tabular}{lll ccc ccc ccc c} \hline
          &           &           &           &           &           &           &           &           &           &           &           &  \\
          & &&\multicolumn{3}{c}{OTM puts on futures} &     & \multicolumn{3}{c}{OTM calls on futures} &  &  \multicolumn{2}{c}{Straddle}\\
          & & &\multicolumn{3}{c}{$\log(k)\times 100$} &   & \multicolumn{3}{c}{$\log(k)\times 100$}  &  &\multicolumn{2}{c}{on  futures} \\
          \cline{4-6} \cline{8-10} \cline{12-13}
\multicolumn{1}{l}{Moneyness (\%)} &           &           & -12         & -8         & -3         &           & 3         & 8         & 12         &           & \multicolumn{1}{l}{ATM} & \multicolumn{1}{l}{Crash-} \\
\multicolumn{1}{l}{Delta (\%)} &           &           & -5         & -11         & -30         &           & 32         & 13         & 6         &           & & Neutral\\ \\
\multicolumn{1}{l}{Open Interest} &           &           & 969       & 1047      & 959       &           & 839       & 577       & 653       &           &           &  \\
\multicolumn{1}{l}{Volume}          &           &           & 43        & 76        & 78        &           & 50        & 44        & 26        &           &           &  \\
          &           &           &           &           &           &           &           &           &           &           &           &  \\ \hline
          &           &           &           &           &           &           &           &           &           &           &           &  \\
\multicolumn{1}{l}{Dividend Yield} & \multicolumn{1}{l}{H} & \multicolumn{1}{l}{$\mathfrak{s}_{\tiny \mbox{bad}}$} & -73***    & -70***    & -70***    &           & 20        & -48*      & -76***    &           & -22*      & -5 \\
\multicolumn{1}{l}{} &       M    &    $\mathfrak{s}_{\tiny \mbox{normal}}$       & -95***    & -90***    & -76***    &           & 43        & -26       & -34       &           & -14*      & -1 \\
\multicolumn{1}{l}{} &     L      &  $\mathfrak{s}_{\tiny \mbox{good}}$         & -38       & -24       & -21       &           & -41***    & -78***    & -88***    &           & -23*      & -7** \\
\multicolumn{1}{l}{} &           &           &  &      &        &           &        &       &       &           &           &  \\

\multicolumn{1}{l}{Quadratic Variation} &    H       &  $\mathfrak{s}_{\tiny \mbox{bad}}$         & -49**     & -41       & -39       &           & -1        & 1         & -13       &           & -17*      & -3 \\
\multicolumn{1}{l}{} &    M       &    $\mathfrak{s}_{\tiny \mbox{normal}}$       & -83***    & -73***    & -61***    &           & 19        & -58**     & -90***    &           & -21**     & -4 \\
\multicolumn{1}{l}{} &     L      &   $\mathfrak{s}_{\tiny \mbox{good}}$        & -74***    & -71**     & -68***    &           & 4         & -94***    & -94***    &           & -21       & -6 \\
          &           &           &           &           &           &           &           &           &           &           &           &  \\
\multicolumn{1}{l}{Risk Reversal} &   H        &   $\mathfrak{s}_{\tiny \mbox{bad}}$        & -75***    & -72**     & -70***    &           & 18        & -96***    & -101***   &           & -20*      & -5 \\
\multicolumn{1}{l}{} &   M        &   $\mathfrak{s}_{\tiny \mbox{normal}}$        & -85***    & -75***    & -61***    &           & -1        & -39       & -83***    &           & -21*      & -4 \\
\multicolumn{1}{l}{} &   L        &   $\mathfrak{s}_{\tiny \mbox{good}}$        & -45*      & -37       & -37       &           & 6         & -14       & -11       &           & -17       & -4 \\
          &           &           &           &           &           &           &           &           &           &           &           &  \\
\multicolumn{1}{l}{Change in Volatility} &    H       &   $\mathfrak{s}_{\tiny \mbox{bad}}$        & -66***    & -58***    & -52**     &           & 19        & -31       & -53       &           & -15       & -2 \\
\multicolumn{1}{l}{} &     M      &   $\mathfrak{s}_{\tiny \mbox{normal}}$        & -65**     & -56*      & -48*      &           & 2         & -65***    & -82***    &           & -17       & -4 \\
\multicolumn{1}{l}{} &    L       &   $\mathfrak{s}_{\tiny \mbox{good}}$        & -75***    & -71***    & -68***    &           & 2         & -55**     & -62**     &           & -27***    & -6** \\
          &           &           &           &           &           &           &           &           &           &           &           &  \\
%\multicolumn{1}{l}{Yield Spread} &    L       &   $\mathfrak{s}_{\tiny \mbox{bad}}$        & -61**     & -48       & -42       &           & 26        & -60***    & -48       &           & -7        & 0 \\
%\multicolumn{1}{l}{} &      M     &    $\mathfrak{s}_{\tiny \mbox{normal}}$       & -60**     & -57**     & -56**     &           & 6         & -51*      & -70***    &           & -25**     & -7** \\
%\multicolumn{1}{l}{} &    H       &    $\mathfrak{s}_{\tiny \mbox{good}}$       & -85***    & -80***    & -71***    &           & -8        & -41*      & -80***    &           & -26***    & -5 \\
%          &           &           &           &           &           &           &           &           &           &           &           &  \\
\multicolumn{1}{l}{Recent Market} &    L       &  $\mathfrak{s}_{\tiny \mbox{bad}}$         & -66***    & -54**     & -40*      &           & 11        & -29       & -51       &           & -10       & 0 \\
\multicolumn{1}{l}{} &       M    &    $\mathfrak{s}_{\tiny \mbox{normal}}$       & -66***    & -59**     & -55**     &           & -3        & -57**     & -71***    &           & -25**     & -6 \\
\multicolumn{1}{l}{} &     H      &   $\mathfrak{s}_{\tiny \mbox{good}}$        & -74***    & -71**     & -73***    &           & 15        & -64***    & -76***    &           & -23**     & -7* \\
          &           &           &           &           &           &           &           &           &           &           &           &  \\ \hline
          &           &           &           &           &           &           &           &           &           &           &           &  \\
\multicolumn{1}{l}{\textbf{Unconditional}} &           & Average          & -69       & -62       & -56       &           & 7         & -51       & -66       &           & -20       & -5 \\
\multicolumn{1}{l}{\textbf{Estimates}} &           &     SD      & 143       & 158       & 139       &           & 172       & 153       & 171       &           & 68        & 23 \\
\multicolumn{1}{l}{} &           &   $\mathbbm{1}_{\{ q_{t, {T}_O} >0 \}}$         & 6\%         & 6\%         & 11\%        &           & 31\%        & 11\%        & 5\%         &           & 34\%        & 38\% \\
          &           &           &           &           &           &           &           &           &           &           &           &  \\ \hline
\end{tabular}%
\end{center}
\end{table}

%%%%%%%%%%%%%%%%%%%%%%%%%%%%%%%%%%%%%% END %%%%%%%%%%%%%%%%%%%%%%%%%%%%%%%%%%%




%%%%%%%%%%%%%%%%%%%%%%%%%%%%%%%%%%%%%% END %%%%%%%%%%%%%%%%%%%%%%%%%%%%%%%%%%%
%%%%%%%%%%%%%%%%%%%%%%%%%%%%%%%%%%%%%% END %%%%%%%%%%%%%%%%%%%%%%%%%%%%%%%%%%%
\newpage
%==*==*==*==*==*==*==*==*==*==*==*==*==*==*==*==*==*== Tables
 %\setcounter{table}{0}  % reset counter

%\renewcommand{\thetable}{\arabic{table}}

 %\setcounter{figure}{0}  % reset counter

%\renewcommand{\thefigure}{\arabic{figure}}

%  \renewcommand{\theequation}{\arabic{equation}}
%  \setcounter{equation}{0}  % reset counter
%\renewcommand{\thetheorem}{\arabic{theorem}}
%  \setcounter{theorem}{0}  % reset counter

\newpage
\thispagestyle{empty}
\clearpage
\begin{center}
{\Large{Dark Matter in (Volatility and) Equity Option Risk Premiums}} \\
\vspace{0.04in}
%Gurdip Bakshi~~~John Crosby~~~Xiaohui Gao \\
\textbf{\underline{Internet Appendix: Not Intended for Publication}}
\end{center}
%\vspace{1mm}
\begin{center}
\textbf{Abstract}
\end{center}

Section~\ref{appsec:dispersion} outlines how the risk premium on volatility
uncertainty relates to the risk premiums on local time and jumps crossing the strike.


Section~\ref{app:jumps_across} develops the
analysis that links jump model assumptions under $\mathbb{P}$ and
$\mathbb{Q}$ to the risk premium for jumps crossing
the strike over small $T_{O}-t$. Our focus here is on the setting of a general
semimartingale that admits jumps. Our analysis incorporates the models
of \citet*{Merton:76}, \citet*{Kou:2002}, and \citet*{DuffiePanSingleton:2000}.


Section~\ref{app:var_jumps}
provides the expressions for the local time risk premiums when there are unspanned risks,
dichotomized in the form of diffusive volatility risks and jump volatility risks.


%{\color{red}[[[ Are we not going to mention the
%other things here? The DPS dynamics? ]]]]}
%{\color{blue} Section~\ref{seimimartingales_continuous} focuses on a continuous semimartingale theoretical environment in which there are both spanned and unspanned diffusive risks. As there are no jumps,
%the terms relating to jumps crossing the strike  --- $a_t^{T_O}[k]$, $b_t^{T_O}[k]$, $c_t^{T_O}[k]$, and $d_t^{T_O}[k]$ --- are all zero.}
%


%Corollary~\ref{claimm:claim1call} depicts option risk premiums when there are unspanned diffusive risks in the dynamics of the pricing kernel
%and the volatility.

%Corollary~\ref{claimm:SV} shows that a suitably motivated stochastic volatility model (under the $\mathbb{P}$ measure)
%can generate
%negative call risk premiums provided that certain restrictions are imposed on unspanned risks.

%{\color{red}[[[ Lemma~\ref{eq:lemmon} will have to
%be significantly re-jigged or removed. ]]] [[[  Finally, we prove
%Lemma~\ref{eq:lemmon}
%(stated as a part of the proof of Theorem~\ref{claimm:claim1call_jump}
%in Appendix~\ref{appsec:jumppps}). The essence of this lemma is
%that if unspanned risks  are irrelevant, then the local time risk premium is \emph{zero}. ]]]}

%{\color{blue}Finally, we prove
%Lemma~\ref{eq:lemmon}
%(stated after the proof of Theorem~\ref{claimm:claim1call_jump}
%in Appendix~\ref{appsec:jumppps}). The essence of this lemma is
%that if unspanned risks are irrelevant, then the risk premium for call options is positive for every $k$.}





%\newpage
\thispagestyle{empty}
%\clearpage

\thispagestyle{empty}
% *************** start of text ****************************************
\newpage
\setcounter{page}{1}
\renewcommand{\thefootnote}{\arabic{footnote}}
\setcounter{footnote}{0}

%\setcounter{equation}{0}
%\renewcommand{\theequation}{B\arabic{equation}}

\setcounter{section}{0}
\renewcommand{\thesection}{\Roman{section}}
%\renewcommand{\thesection}{\Roman{section}}
%\renewcommand{\thesubsection}{\thesection.\Roman{subsection}}
\renewcommand{\thesubsection}{\thesection.\arabic{subsection}}
%\renewcommand{\thesubsection}{\Roman{subsection}}

%\section{ \bf \large Internet Appendix}

                                                        \setcounter{equation}{0}
                                                        \renewcommand{\theequation}{IA-\arabic{equation}}
                                                        %\setcounter{section}{0}
%\numberwithin{equation}{section}
\numberwithin{table}{section}
\numberwithin{theorem}{section}
\numberwithin{figure}{section}
%                                            \numberwithin{theorem}{subsection}

%%%%%%%%%%%%%%%%%%%%%%%%%%%%%%%%%%%%%%%% Begin I %%%%%%%%%%%%%%%%%%%%%%%%%%%%%%%%%%%%%%%%%%%%%%%%%%%%%
%%%%%%%%%%%%%%%%%%%%%%%%%%%%%%%%%%%%%%%%%%%%%%%%%%%%%%%%%%%%%%%%
                                                        \setcounter{equation}{0}
                                                        \renewcommand{\theequation}{I\arabic{equation}}




%\begin{center}
%\textbf{Internet Appendix}
%\end{center}





%%%%%%%%%%%%%%%%%%%%%%%%%%%%%%%%%%%%%%%%%%%%%%%%%%%%%%%%%%%%%%%%%%%%%%%%%%%%%%%%%%%%%%%%%%%%%%%%%%%%%%%%%%%%
%%%%%%%%%%%%%%%%%%%%%%%%%%%%%%%%%%%%%%%%%%%%%%%%%%%%%%%%%%%%%%%%%%%%%%%%%%%%%%%%%%%%%%%%%%%%%%%%%%%%%%%%%%%%
\section{Risk premium for volatility
uncertainty
and its link to risk premiums on (i) local time and (ii)
jumps crossing the strike}
\label{appsec:dispersion}
Consider the time $T_O$ payoff $\{ \log G_{T_O} \}^2= \{\log \frac{F_{T_O}^{T_F}}{F_t^{T_F}}\}^2$. This payoff represents \emph{volatility
%dispersion
uncertainty}.
Define the function
\begin{equation}
\mathfrak{f}[K] ~\equiv~ \frac{2}{K^2} ( 1 - \log \frac{K}{F_{t}^{T_F}}).
%= \frac{d^2 \{\log \frac{F_{T_O}^{T_F}}{F_t^{T_F}}\}^2}{d (F_{T_O}^{T_F})^2} {\big |}_{F_{T_O}^{T_F}=K}.
\end{equation}
Since  $\{\log \frac{F_{T_O}^{T_F}}{F_t^{T_F}}\}^2 \in {\cal C}^2$,
%is twice continuously differentiable,
it may be expressed
%, following
%\citet*{BakshiMadan:2000} and
%\citet{CarrMadan:2001QF},
as
\begin{eqnarray}
\big\{\log \frac{F_{T_O}^{T_F}}{F_t^{T_F}}\big\}^2 & = &
 \int\limits_0^{F_{t}^{T_F}} \mathfrak{f}[K] \max(K-F_{T_O}^{T_F},0)\,dK  ~+~
 \int\limits_{F_{t}^{T_F}}^\infty \mathfrak{f}[K] \max(F_{T_O}^{T_F}-K,0)\,dK~~\mbox{ \, \, }\label{eq:asb1} \\
% & = &
% \int\limits_{F_{T_O}^{T_F}}^\infty y[K] F_{t}^{T_F} [\frac{F_{T_O}^{T_F}}{F_{t}^{T_F}}-\frac{K}{F_{t}^{T_F}}]^{+}dK ~+~
% \int\limits_0^{F_{T_O}^{T_F}} y[K] F_{t}^{T_F} [\frac{K}{F_{t}^{T_F}}- \frac{F_{T_O}^{T_F}}{F_{t}^{T_F}}]^{+}dK~~\mbox{ \, \, \, \, }~ \label{eq:asb2}\\
 & = &
 \int\limits_0^{1} \omega[k] \max(k - \frac{F_{T_O}^{T_F}}{F_{t}^{T_F}},0)\,dk~+~
\int\limits_{1}^\infty \omega[k] \max(\frac{F_{T_O}^{T_F}}{F_{t}^{T_F}}- k,0)\,dk,
~~\mbox{ \, \, }\label{eq:asb3} \\
\mbox{ where \, }~ \omega[k] & \equiv & \frac{2}{k^2} ( 1 - \log k ), ~~\mbox{ \, with \, }~~k \, = \, \frac{K}{F_{t}^{T_F}},~~\mbox{ \, and \, }~ d k \, = \, \frac{d K}{F_{t}^{T_F}}.~~~\mbox{ \, }
\end{eqnarray}
%In moving from (\ref{eq:asb1}) to (\ref{eq:asb3}), we have performed a change of variable.
We can now substitute Tanaka's formula for semimartingales into the expression for $\max(k-G_{T_O},0)$ and
$\max(G_{T_O}- k,0)$ in the right-hand side of
%{\color{red}[equation]}
(\ref{eq:asb3}). Therefore, we obtain the following:
\begin{eqnarray}
\big\{\log \frac{F_{T_O}^{T_F}}{F_t^{T_F}}\big\}^2 &=& \int\limits_0^{1} \omega[k] \max(k - G_{T_O},0) dk
+  \int\limits_{1}^\infty \omega[k] \max(G_{T_O}- k,0) dk
~~\mbox{ \, \, }
%\label{eq:asb4}
\nonumber
\\
&=& \int\limits_0^{1} \omega[k] \{ - \int_{t}^{T_O} \mathbbm{1}_{\{G_{\ell-} < k\}} \,dG_{\ell}
~+~ \mathbb{L}^{T_O}_t[k] ~+~ c_t^{T_O}[k] ~+~ d_t^{T_O}[k] \} dk \nonumber \\
&& +  \int\limits_{1}^\infty \omega[k] \{\int_{t}^{T_O} \mathbbm{1}_{\{G_{\ell-} > k\}} \,dG_{\ell}
~+~ \mathbb{L}^{T_O}_t[k] ~+~ a_t^{T_O}[k] ~+~ b_t^{T_O}[k] \} dk
~~\mbox{ \, \, }\label{eq:asb5} \\
&=& \int_{t}^{T_O} \big( \int\limits_{1}^\infty \omega[k] \, \mathbbm{1}_{\{G_{\ell-} > k\}} \, dk - \int\limits_0^{1} \omega[k] \, \mathbbm{1}_{\{G_{\ell-} < k\}} \ dk \big) \, dG_{\ell}
%{\color{magenta} +\int\limits_{0}^\infty \omega[k]\,\mathbb{L}^{T_O}_t[k]\,dk }
\nonumber \\
&&~+~ \int\limits_{0}^\infty \omega[k]\,\mathbb{L}^{T_O}_t[k]\,dk  \nonumber \\
&& ~+~
\int\limits_{0}^{1} \omega[k]\, ( c_t^{T_O}[k] + d_t^{T_O}[k] ) \,dk
%\nonumber \\ &&
 ~+~
\int\limits_{1}^\infty \omega[k]\, ( a_t^{T_O}[k] + b_t^{T_O}[k]) \,dk.
~\mbox{ \, }
~~\mbox{ \, \, }\label{eq:asb6}
\end{eqnarray}

Using $\mathbb{P}$ and $\mathbb{Q}$ measure expectations, we consequently obtain the following:
\begin{eqnarray}
\mathbb{E}_t^{\mathbb{P}} ( \big\{\log \frac{F_{T_O}^{T_F}}{F_t^{T_F}}\big\}^2 )
&=&
-~\mathbb{E}_t^{\mathbb{P}} ( \int_{t}^{T_O} \big\{ -\int\limits_{1}^\infty \omega[k] \, \mathbbm{1}_{\{G_{\ell-} > k\}} \, dk +
\int\limits_0^{1} \omega[k] \, \mathbbm{1}_{\{G_{\ell-} < k\}} \ dk \big\} \, dG_{\ell} )
~~\mbox{ \, \, \, \, }~~ \nonumber \\
&+&
\int\limits_{0}^\infty \omega[k] \, \mathbb{E}_t^{\mathbb{P}} ( \mathbb{L}^{T_O}_t[k] ) \, dk   \nonumber \\
&+& ~
\int\limits_{0}^1 \omega[k]\, \mathbb{E}_t^{\mathbb{P}}( c_t^{T_O}[k]+ d_t^{T_O}[k] ) \,dk  +
\int\limits_{1}^\infty \omega[k]\, \mathbb{E}_t^{\mathbb{P}}( a_t^{T_O}[k] + b_t^{T_O}[k]) \,dk.
\\
\mathbb{E}_t^{\mathbb{Q}} ( \big\{\log \frac{F_{T_O}^{T_F}}{F_t^{T_F}}\big\}^2 ) &=&
\int\limits_{0}^\infty \omega[k] \, \mathbb{E}_t^{\mathbb{Q}} ( \mathbb{L}^{T_O}_t[k] ) \, dk \nonumber \\
&+&
\int\limits_{0}^1 \omega[k]\, \mathbb{E}_t^{\mathbb{Q}}( c_t^{T_O}[k] + d_t^{T_O}[k] ) \,dk
+ \int\limits_{0}^\infty \omega[k]\, \mathbb{E}_t^{\mathbb{Q}}( a_t^{T_O}[k] + b_t^{T_O}[k] ) \,dk.
%~~\mbox{ \, }~ ~~\mbox{ \, \, }
\label{eq:asb8}
\end{eqnarray}
This is because
$\mathbb{E}_t^{\mathbb{Q}} ( \int_{t}^{T_O} \big\{\int\limits_{1}^\infty \omega[k] \, \mathbbm{1}_{\{G_{\ell-} > k\}} dk
\big\} dG_{\ell} ) =0$
and $\mathbb{E}_t^{\mathbb{Q}} ( \{\int\limits_0^{1} \omega[k]\, \mathbbm{1}_{\{G_{\ell-} < k\}} \ dk \big\} dG_{\ell} ) =0$.

The expression for the risk premium for volatility uncertainty
is as follows:
\begin{eqnarray}
& & \underbrace{\mathbb{E}_t^{\mathbb{P}} ( \big\{\log \frac{F_{T_O}^{T_F}}{F_t^{T_F}}\big\}^2 )
- \mathbb{E}_t^{\mathbb{Q}} ( \big\{\log \frac{F_{T_O}^{T_F}}{F_t^{T_F}}\big\}^2 )}_{\tiny \mbox{risk~premium~for~volatility~uncertainty}}
~=~  -\mathrm{e}_t^{\mathbb{P}} +
\int\limits_{0}^\infty ~ \omega[k] \, \underbrace{\{ \mathbb{E}_t^{\mathbb{P}} ( \mathbb{L}^{T_O}_t[k] )-\mathbb{E}_t^{\mathbb{Q}} ( \mathbb{L}^{T_O}_t[k] )\}}_{\tiny \mbox{risk~premium~for~local~time}} \,dk   \nonumber \\
&&+~ \int\limits_{0}^1 \omega[k]\, \underbrace{ \{
\mathbb{E}_t^{\mathbb{P}}( c_t^{T_O}[k] ~+~ d_t^{T_O}[k] )
- \mathbb{E}_t^{\mathbb{Q}}( c_t^{T_O}[k] ~+~ d_t^{T_O}[k] ) \}}_{\tiny \mbox{risk~premium~for~jumps~crossing~the~strike}~(k<1)} \, dk \nonumber \\
&&+~ \int\limits_{1}^\infty \omega[k]\, \underbrace{\{
\mathbb{E}_t^{\mathbb{P}}( a_t^{T_O}[k] ~+~ b_t^{T_O}[k] )
- \mathbb{E}_t^{\mathbb{Q}}( a_t^{T_O}[k] ~+~ b_t^{T_O}[k]) \}}_{\tiny \mbox{risk~premium~for~jumps~crossing~the~strike}~(k>1)} \, dk,
%,~\mbox{ \, }
%~~\mbox{ \,\,\,\, }
\nonumber \\
%\label{eq:asb9} \\
& &~~~~ \mbox{ where \, \, }~ \mathrm{e}_t^{\mathbb{P}} ~=~ \mathbb{E}_t^{\mathbb{P}} ( \int_{t}^{T_O}
\big\{ -\int\limits_{1}^\infty \omega[k] \, \mathbbm{1}_{\{G_{\ell-} > k\}} \, dk +  \int\limits_0^{1} \omega[k] \, \mathbbm{1}_{\{G_{\ell-} < k\}} \ dk \big\} \, dG_{\ell} ).
%~~~\mbox{ \, }
\label{eq:DefMuP}
\end{eqnarray}
The term inside the $dG_{\ell}$ integral inside the expectation
in
(\ref{eq:DefMuP}) is the gain/loss from a dynamic trading strategy,
which, at time $\ell$, takes a position in the equity futures
proportional to the quantity $\big( -\int\limits_{1}^\infty \omega[k] \, \mathbbm{1}_{\{G_{\ell-} > k\}} \, dk + \int\limits_0^{1} \omega[k] \, \mathbbm{1}_{\{G_{\ell-} < k\}} \ dk \big)$. In essence, $\mathrm{e}_t^{\mathbb{P}}$ is the expected total gain/loss, over $t$ to $T_O$, from this futures trading strategy.

Finally, %we note that
$\omega[k]>0$ for $0< k<\exp(1)=2.71828$, and $\omega[k]$ is declining for high enough $k$.
%Equation (\ref{eq:LogFuturesSqasb11InResult}) follows.
$\blacksquare$ \vspace{-3mm}



%%%%%%%%%%%%%%%%%%%%
%%%%%%%%%%%%%%%%%%%%
%%%%%%%%%%%%%%%%%%%%
%%%%%%%%%%%%%%%%%%%%


\section{Option models and risk premiums for jumps crossing the strike}
\label{app:jumps_across}

In this section, we draw on the link between the variations in option risk premiums and modeling ingredients.
Specifically,
we investigate parametric restrictions under which the risk premium for jumps crossing the strike  can be
negative for $k>1$
(i.e., pertaining to OTM calls). Analogous steps apply for $k<1$ (for puts).

%{\color{red}[[[[ We should make more explicit our
%notation for the dynamics under the
%$\mathbb{Q}$ measure. ]]]]]]}
We consider the option model based on the price dynamics in (\ref{eq:dou1})--(\ref{eq:dou6}). This
model has price and  volatility jump risks, and spanned and unspanned (diffusive and jump) risks in the volatility dynamics.
The
risk premium adjustments that link $\mathbb{P}$ to $\mathbb{Q}$ are explicit through Girsanov's change of measure
theorem for jump-diffusions (e.g., \citet*{Runggaldier:2003} and \citet*{ContTankov:2004}).

Let ${\bm \lambda}_{\tiny \mbox{jump}}^{\mathbb{P}}$ (${\bm \lambda}_{\tiny \mbox{jump}}^{\mathbb{Q}}$) be the constant
intensity rate of the Poisson process and $\nu^{\mathbb{P}}[\mathbbm{x}_s]$ ($\nu^{\mathbb{Q}}[\mathbbm{x}_s]$) be the density of price jumps under
$\mathbb{P}$ ($\mathbb{Q}$).  The risk premium for jumps crossing the strike $\mathbbm{rp}_{t}^{T_O}[k]$ is
\begin{eqnarray}
\mathbbm{rp}_{t}^{T_O}[k] & \equiv & \mathbb{E}^{\mathbb{P}}_t( a_t^{T_O}[k] + b_t^{T_O}[k] ) ~ - ~ \mathbb{E}^{\mathbb{Q}}_t( a_t^{T_O}[k] + b_t^{T_O}[k] )
~~~~~~~~~~~~~ ~~~~ \mbox{ \, \, }
\nonumber \\
%\label{eq:PMinusQJumpsAcrossStrikeMerton76}\\
&=& \mathbb{E}^{\mathbb{P}}_t( \sum_{t < \ell \leq T_O} \mathbbm{1}_{\{G_{\ell \, -} \leq k\}} \, \max( G_{\ell} - k, 0 ) ) \, - \, \mathbb{E}^{\mathbb{Q}}_t( \sum_{t < \ell \leq T_O} \mathbbm{1}_{\{G_{\ell \, -} \leq k\}} \, \max( G_{\ell} - k, 0 ) ) ~ \mbox{ \, } \nonumber \\
&+&
\mathbb{E}^{\mathbb{P}}_t( \sum_{t < \ell \leq T_O} \mathbbm{1}_{\{G_{\ell \, -} > k\}} \, \max( k - G_{\ell}, 0 ) ) \, - \, \mathbb{E}^{\mathbb{Q}}_t( \sum_{t < \ell \leq T_O} \mathbbm{1}_{\{G_{\ell \, -} > k\}} \, \max( k - G_{\ell}, 0 ) ). ~ \mbox{ \, \, \, } \label{eq:GeneralDefPMinusQRP}
\end{eqnarray}
Given our focus on the returns of weekly options,
%we exploit analytical tractability of small $T_O - t \equiv \Delta T$.
we emphasize analytical tractability and economic insight by developing
our analysis in the limit of small $\Delta T$, where
$\Delta T \equiv T_O - t$.

For small $\Delta T$, the probability, under $\mathbb{P}$ (respectively, $\mathbb{Q}$) of one jump over the time
period $t$ to $t+\Delta T$ approximates to ${\bm \lambda}_{\tiny \mbox{jump}}^{\mathbb{P}}\,\Delta T$ (respectively, ${\bm \lambda}_{\tiny \mbox{jump}}^{\mathbb{Q}} \, \Delta T$). The probability of two or more jumps is negligible for small $\Delta T$.
Therefore, in the limit of small $\Delta T$,
\begin{equation}
G_{\ell-} ~ \mathrm{tends~to} ~ G_t=1 ~ (\mathrm{since}~ G_t = 1 ~\mathrm{(by~construction))}.
~ \mathrm{So} ~ G_{\ell} ~\mathrm{tends~to}~
\underbrace{G_{t}}_{= 1} \, e^{\mathbbm{x}_s} = \ e^{\mathbbm{x}_s}. ~ \mbox{ \, \, \, }
\end{equation}
Simplifying  (\ref{eq:GeneralDefPMinusQRP}), the risk premium for jumps crossing the strike
approximates to
\begin{eqnarray}
\mathbbm{rp}_{t}^{t+\Delta T}[k] & = & {\bm \lambda}_{\tiny \mbox{jump}}^{\mathbb{P}}
\Delta T \int_{-\infty}^{\infty}
\mathbbm{1}_{\{ 1 \leq k\}}
( e^{\mathbbm{x}_s} - k )^{+}  \nu^{\mathbb{P}}[d \mathbbm{x}_s]
- {\bm \lambda}_{\tiny \mbox{jump}}^{\mathbb{Q}}  \Delta T  \int_{-\infty}^{\infty}
\mathbbm{1}_{\{ 1\leq k \}}
( e^{\mathbbm{x}_s} - k )^{+}  \nu^{\mathbb{Q}}[d \mathbbm{x}_s]
 \mbox{ \, \, } \nonumber \\
& +& {\bm \lambda}_{\tiny \mbox{jump}}^{\mathbb{P}}  \Delta T  \int_{-\infty}^{\infty}  \mathbbm{1}_{\{1 > k \}}
( k - e^{\mathbbm{x}_s} )^{+} \, \nu^{\mathbb{P}}[d \mathbbm{x}_s]
-  {\bm \lambda}_{\tiny \mbox{jump}}^{\mathbb{Q}}\Delta T \int_{-\infty}^{\infty}  \mathbbm{1}_{\{ 1> k \}}
( k - e^{\mathbbm{x}_s} )^{+} \nu^{\mathbb{Q}}[d \mathbbm{x}_s],
\mbox{ \, }
\nonumber
\end{eqnarray}
where the error in the approximation is $O[\{\Delta T\}^2]$ and, for brevity, $x^{+} \, \equiv \, \max(x, 0 )$.
Equivalently, since we focus on $k>1$ (pertaining to OTM calls), the task is to compute the following expression:
\begin{eqnarray}
\frac{1}{\Delta T} \mathbbm{rp}_{t}^{t+\Delta T}[k] & = & {\bm \lambda}_{\tiny \mbox{jump}}^{\mathbb{P}} \, \int_{\log(k)}^{\infty}
( e^{\mathbbm{x}_s} - k ) \, \nu^{\mathbb{P}}[\mathbbm{x}_s] \,d \mathbbm{x}_s
~-~ {\bm \lambda}_{\tiny \mbox{jump}}^{\mathbb{Q}} \, \int_{\log(k) }^{\infty}
( e^{\mathbbm{x}_s} - k ) \, \nu^{\mathbb{Q}}[\mathbbm{x}_s] \,d \mathbbm{x}_s.
~ \mbox{ \, \, \, \, }
\label{eq:InterimPMinusQApprox2a}
\end{eqnarray}  \vspace{3mm}
\noindent \textbf{Case~1 (Normally distributed jumps (\citet{Merton:76})).}  For this exercise, we posit
\begin{eqnarray}
\underbrace{\nu^{\mathbb{P}}[\mathbbm{x}_s]}_{\tiny \mbox{density~of~price~jump~under}~\mathbb{P}} & = & \frac{1}{\sqrt{2 \pi {(\sigma^{\mathbb{P}}_{\mathbbm{x}})}^{2}}} \, \,
\exp( -\frac{(\mathbbm{x}_s - \{\mu^{\mathbb{P}}_{\mathbbm{x}} - \frac{1}{2} {(\sigma^{\mathbb{P}}_{\mathbbm{x}}})^2\})^2}{2\, {(\sigma^{\mathbb{P}}_{\mathbbm{x}})}^{2}}) ~~~~~~ ~~ \mbox{ \, } ~~ \mathrm{and} ~~ ~~  \\
\underbrace{\nu^{\mathbb{Q}}[\mathbbm{x}_s]}_{\tiny \mbox{density~of~price~jump~under}~\mathbb{Q}} & = & \frac{1}{\sqrt{2 \pi {(\sigma^{\mathbb{Q}}_{\mathbbm{x}})}^{2}}} \, \,
\exp( -\frac{(\mathbbm{x}_s - \{\mu^{\mathbb{Q}}_{\mathbbm{x}} - \frac{1}{2} {(\sigma^{\mathbb{Q}}_{\mathbbm{x}})}^{2} \})^2}{2\, {(\sigma^{\mathbb{Q}}_{\mathbbm{x}})}^{2}}).
\label{eq:meeerton1}
\end{eqnarray}
Then $\mathbb{E}^{\mathbb{P}}( e^{\mathbbm{x}_s}) =  \exp( \mu^{\mathbb{P}}_{\mathbbm{x}} )$ and
$\mathbb{E}^{\mathbb{Q}}( e^{\mathbbm{x}_s}) = \exp( \mu^{\mathbb{Q}}_{\mathbbm{x}} )$.
It follows from (\ref{eq:InterimPMinusQApprox2a}) that
\begin{equation}
\frac{1}{\Delta T} \mathbbm{rp}_{t}^{t+\Delta T}[k] =
{\bm \lambda}_{\tiny \mbox{jump}}^{\mathbb{P}} \{ e^{\mu^{\mathbb{P}}_{\mathbbm{x}}} {\cal N}(d_1^{\mathbb{P}}[k]) - k \, {\cal N}(d_2^{\mathbb{P}}[k]) \}
~-~
 {\bm \lambda}_{\tiny \mbox{jump}}^{\mathbb{Q}}
 \{ e^{\mu^{\mathbb{Q}}_{\mathbbm{x}}} {\cal N}(d_1^{\mathbb{Q}}[k]) - k \,{\cal N}(d_2^{\mathbb{Q}}[k]) \},
\label{eq:meeerton2}
\end{equation}
where ${\cal N}(.)$ denotes the standard normal cumulative distribution function, and
\begin{eqnarray}
d_1^{\mathbb{P}}[k] &=& \frac{ - \log(k) + \mu^{\mathbb{P}}_{\mathbbm{x}} + \frac{1}{2}{(\sigma^{\mathbb{P}}_{\mathbbm{x}})^{2}}}{\sigma^{\mathbb{P}}_{\mathbbm{x}}},~~ \mbox{ \, \, } ~~ \mathrm{and} ~~~~ d_2^{\mathbb{P}}[k] \, = \, d_1^{\mathbb{P}}[k] \, - \, \sigma^{\mathbb{P}}_{\mathbbm{x}}, ~~  \mbox{ \, \, } ~~ \\
d_1^{\mathbb{Q}}[k] &=& \frac{ - \log(k) + \mu^{\mathbb{Q}}_{\mathbbm{x}} + \frac{1}{2}{(\sigma^{\mathbb{Q}}_{\mathbbm{x}})^{2}}}{\sigma^{\mathbb{Q}}_{\mathbbm{x}}},
 ~~ \mbox{ \, \, } ~~ \mathrm{and} ~~~~ d_2^{\mathbb{Q}}[k] \, = \,  d_1^{\mathbb{Q}}[k] \, - \, \sigma^{\mathbb{Q}}_{\mathbbm{x}}. ~~
 \mbox{ \, \, } ~~
\end{eqnarray}
The ensuing restrictions yield negative risk premiums for jumps crossing the strike and, thus, is an intermediate
step
to supporting negative risk premiums for OTM calls:
\begin{align}
&{\bm \lambda}_{\tiny \mbox{jump}}^{\mathbb{Q}} > {\bm \lambda}_{\tiny \mbox{jump}}^{\mathbb{P}},&
&\mu^{\mathbb{Q}}_{\mathbbm{x}} < \mu^{\mathbb{P}}_{\mathbbm{x}},&
&\mathrm{and}&
&\sigma^{\mathbb{Q}}_{\mathbbm{x}} > \sigma^{\mathbb{P}}_{\mathbbm{x}}.&
&\blacksquare&
\end{align}



\noindent \textbf{Case~2 (Double exponentially distributed jumps (\citet{Kou:2002})).} Under the assumption that the jump distribution under $\mathbb{P}$ and $\mathbb{Q}$ is of the same parametric form, we have
\begin{gather}
\nu^{\mathbb{P}}[\mathbbm{x}_s]~=~
\begin{cases}
p_{+}^{\mathbb{P}} \, \, \eta_{+}^{\mathbb{P}} \, \, e^{- \eta_{+}^{\mathbb{P}} \, \mathbbm{x}_s} \mbox{ \, } &
~ \mbox{ \, } \mathrm{for} ~ ~ \mathbbm{x}_s >0, \\
p_{-}^{\mathbb{P}} \, \, \eta_{-}^{\mathbb{P}} \, \, e^{\eta_{-}^{\mathbb{P}} \, \mathbbm{x}_s} \mbox{ \, }
 & ~ \mbox{ \, } \mathrm{for} ~ ~ \mathbbm{x}_s < 0, ~~~ ~~ \mbox{ \, \, \, where \, $p_{-}^{\mathbb{P}} \, \equiv \, 1 - p_{+}^{\mathbb{P}}$, \, \, \, \, } ~ ~ ~ \mbox{ \, } ~
\end{cases}
\label{kou.1}
\end{gather}
and analogously under $\mathbb{Q}$ (replacing
each superscript $\mathbb{P}$ by a superscript $\mathbb{Q}$ in
equation (\ref{kou.1})).

We assume
that
$0 < p_{+}^{\mathbb{P}} < 1$,
$\eta_{+}^{\mathbb{P}} > 1$,
$\eta_{-}^{\mathbb{P}} > 0$,
$0 < p_{+}^{\mathbb{Q}} < 1$,
$\eta_{+}^{\mathbb{Q}} > 1$,
and $\eta_{-}^{\mathbb{Q}} > 0$.
The mean jump sizes are, respectively, $\frac{1}{\eta_{+}^{\mathbb{P}}}$,  $\frac{1}{\eta_{-}^{\mathbb{P}}}$,
$\frac{1}{\eta_{+}^{\mathbb{Q}}}$, and $\frac{1}{\eta_{-}^{\mathbb{Q}}}$ (\citet[page 1087]{Kou:2002}).

Direct evaluation implies the following expression:
\begin{eqnarray}
\frac{1}{\Delta T} \mathbbm{rp}_{t}^{t+\Delta T}[k] & = &
\frac{{\bm \lambda}_{\tiny \mbox{jump}}^{\mathbb{P}} \, p_{+}^{\mathbb{P}} \, e^{- \log(k) \{\eta_{+}^{\mathbb{P}}-1\} }}{\eta_{+}^{\mathbb{P}}-1}
\, \, - \, \, \frac{{\bm \lambda}_{\tiny \mbox{jump}}^{\mathbb{Q}} \, p_{+}^{\mathbb{Q}} \, e^{- \log(k) \{\eta_{+}^{\mathbb{Q}}-1\}}}{\eta_{+}^{\mathbb{Q}}-1}.
\label{eq:InterimPMinusQApprox2d}
\end{eqnarray}

The following restrictions support negative risk premiums for jumps crossing the strike:
\begin{align}
&{\bm \lambda}_{\tiny \mbox{jump}}^{\mathbb{Q}}
> {\bm \lambda}_{\tiny \mbox{jump}}^{\mathbb{P}},
&
& \frac{1}{\eta_{+}^{\mathbb{P}}} < \frac{1}{\eta_{+}^{\mathbb{Q}}},&
&\mathrm{and}&
& p_{+}^{\mathbb{P}} = p_{+}^{\mathbb{Q}}.~~~~~~\blacksquare&
\label{eq:InterimPMinusQApprox2de}
\end{align}
%{\color{magenta} These restrictions translate into jumps being perceived more frequent under $\mathbb{Q}$ than under $\mathbb{P}$. $\blacksquare$}
%\vspace{3mm}

\noindent \textbf{Case~3 (Normally distributed jumps in equity prices
conditional on exponential jumps in variance (\citet*{DuffiePanSingleton:2000})).} This model admits jumps
in return variance, and the distribution
of price jumps is conditioned on (one-sided) variance jumps.

The consequence is an altered functional form of $\nu^{\mathbb{P}}[\mathbbm{x}_s]$ (and $\nu^{\mathbb{Q}}[\mathbbm{x}_s]$) and
{is amenable to evaluating $\int_{\log(k)}^{\infty}
( e^{\mathbbm{x}_s} - k ) \nu^{\mathbb{P}}[\mathbbm{x}_s] \,d \mathbbm{x}_s$ and $\int_{\log(k)}^{\infty}
( e^{\mathbbm{x}_s} - k ) \nu^{\mathbb{Q}}[\mathbbm{x}_s] \,d \mathbbm{x}_s$.
The model specifies the following:
\begin{eqnarray}
&\mathrm{Jumps}~\mathbbm{x}_\mathrm{v}~\mathrm{in}~\mathrm{v}_t~\mathrm{are~exponentially~and~independently~distributed~with~mean}~ \mu_\mathrm{v}^{\mathbb{P}}, ~ & \mbox{ \quad } \\
&\mathbbm{x}_s \mid \mathbbm{x}_\mathrm{v}  \sim  {\cal N} \left (\beta_{0}^{\mathbb{P}} + \beta_{s,\mathrm{v}}^{\mathbb{P}} \, \mathbbm{x}_\mathrm{v},
(\sigma^{\mathbb{P}}_{s,\mathrm{v}})^2 \right). ~ & \mbox{ \quad }
\label{DPSv-jv.4}
\end{eqnarray}
Equation (\ref{DPSv-jv.4}) allows for simultaneous and correlated jumps in equity price and variance.



Completing the square in the density function of the conditional normal distribution,
we obtain the following density function for price jumps (the form of integral in (\ref{sdff.1}) resembles
\citet*[page 384]{GradshteynRyzhik:1994book}):
\begin{eqnarray}
\nu^{\mathbb{P}}[\mathbbm{x}_s] & = & \int_{0}^{\infty} \, \frac{1}{\sqrt{2 \pi (\sigma^{\mathbb{P}}_{s,\mathrm{v}})^2}} \, \,
\exp( -\frac{(\mathbbm{x}_s - \{ \beta_0^{\mathbb{P}} + \beta_{s,\mathrm{v}}^{\mathbb{P}} \, \mathbbm{x}_\mathrm{v} \})^2}{2 \, (\sigma^{\mathbb{P}}_{s,\mathrm{v}})^2} )
~
\, \, \frac{1}{\mu_\mathrm{v}^{\mathbb{P}}} \, \, e^{- \frac{1}{\mu_\mathrm{v}^{\mathbb{P}}} \mathbbm{x}_\mathrm{v}} \, d \mathbbm{x}_\mathrm{v} ~ \mbox{ \, \, \, \quad } ~ ~ ~ \mbox{ \, \, } ~ \label{sdff.1} \\
%& = & {\color{red}[[[[ ~ \frac{{\cal N}(\frac{(\sigma^{\mathbb{P}}_{s,\mathrm{v}})^2}{\mu_\mathrm{v}^{\mathbb{P}} (\beta_{s,\mathrm{v}}^{\mathbb{P}})^2} + \frac{(\mathbbm{x}_s - \beta_0^{\mathbb{P}})}{(\beta_{s,\mathrm{v}}^{\mathbb{P}})^2})}{\mu_\mathrm{v}^{\mathbb{P}} (\beta_{s,\mathrm{v}}^{\mathbb{P}})^2} \, ~ \,
%\exp \big( - \, \frac{(\mathbbm{x}_s - \beta_0^{\mathbb{P}})}{2 \, \mu_\mathrm{v}^{\mathbb{P}} (\beta_{s,\mathrm{v}}^{\mathbb{P}})^2} ~ ~  - \frac{1}{2}  (\frac{\sigma^{\mathbb{P}}_{s,\mathrm{v}}}{\mu_\mathrm{v}^{\mathbb{P}} \, \beta_{s,\mathrm{v}}^{\mathbb{P}}})^2 \big), ~ ]]]]]} \mbox{ \, \, \, \, } ~~
%\mbox{ \, } \nonumber \\
& = &\frac{{\cal N}(\frac{\, - \, \sigma^{\mathbb{P}}_{s,\mathrm{v}}}{\mu_\mathrm{v}^{\mathbb{P}} \, \beta_{s,\mathrm{v}}^{\mathbb{P}}} + \frac{(\mathbbm{x}_s - \beta_0^{\mathbb{P}})}{\sigma^{\mathbb{P}}_{s,\mathrm{v}}})}{\mu_\mathrm{v}^{\mathbb{P}} \beta_{s,\mathrm{v}}^{\mathbb{P}}} \, ~ \,
\exp \big( - \, \frac{(\mathbbm{x}_s - \beta_0^{\mathbb{P}})}{\mu_\mathrm{v}^{\mathbb{P}} \, \beta_{s,\mathrm{v}}^{\mathbb{P}}} ~ + ~ \frac{1}{2}  (\frac{\sigma^{\mathbb{P}}_{s,\mathrm{v}}}{\mu_\mathrm{v}^{\mathbb{P}} \, \beta_{s,\mathrm{v}}^{\mathbb{P}}})^2 \big),  \mbox{ \, \, \, \, } ~~
\mbox{ \, }
\label{DPSDensity1.1}
\end{eqnarray}
and analogously under $\mathbb{Q}$ (replacing each superscript $\mathbb{P}$ by a superscript $\mathbb{Q}$ in
%equations
(\ref{DPSv-jv.4})--(\ref{DPSDensity1.1})).
%sant \footnote{For constants $a$, $b$, $c$, $f$, and $g$, let $I \equiv \int_{0}^{\infty} \frac{1}{g} \exp(- \frac{(a-bv)^2}{2c^2}) \exp(-f v) dv$ $= \int_{0}^{\infty} \frac{1}{g} \exp(- \frac{(bv - (ab-c^2 f)/b)^2}{2c^2})  \exp(a^2 - \frac{(ab - c^2 f)^2}{b^2}) dv$, after completing the square. After substituting $z= \frac{1}{c}(bv - (ab-c^2 f)/b)$ and $z^{\star}= \frac{1}{c}(- (ab-c^2 f)/b)$, we find
%$I = \int_{z^{\star}}^{\infty} \frac{1}{g} \exp(- \frac{z^2}{2} ) \exp(a^2 - \frac{(ab - c^2 f)^2}{b^2}) dz$
%$= \sqrt{2 \pi} \frac{c}{gb} {\cal N}( \frac{(ab-c^2 f)}{cb})
%\exp( - \frac{a f}{b} + \frac{1}{2} \frac{c^2 f^2}{b^2} )$.
%Equation (\ref{DPSDensity1.1}) follows
%with $g = \sqrt{2 \pi (\sigma^{\mathbb{P}}_{s,\mathrm{v}})^2} \,
%\, \mu_\mathrm{v}^{\mathbb{P}}$, $ \, $
%$c = \sigma^{\mathbb{P}}_{s,\mathrm{v}}$,
%$a = \mathbbm{x}_s - \beta_0^{\mathbb{P}}$,
%$b = \beta_{s,\mathrm{v}}^{\mathbb{P}}$, and
%$f = \frac{1}{\mu_\mathrm{v}^{\mathbb{P}}}$.}


%{\color{red} [[[[[[Completing the square in the density function of the conditional normal distribution,
%we obtain the following density function for price jumps:
%\begin{eqnarray}
%\nu^{\mathbb{P}}[\mathbbm{x}_s] & = & \int_{0}^{\infty} \, \frac{1}{\sqrt{2 \pi (\sigma^{\mathbb{P}}_{s,\mathrm{v}})^2}} \, \,
%\exp( -\frac{(\mathbbm{x}_s - \{ \beta_0^{\mathbb{P}} + \beta_{s,\mathrm{v}}^{\mathbb{P}} \, \mathbbm{x}_\mathrm{v} \})^2}{2 \, (\sigma^{\mathbb{P}}_{s,\mathrm{v}})^2} )
%~
%\, \, \frac{1}{\mu_\mathrm{v}^{\mathbb{P}}} \, \, e^{- \frac{1}{\mu_\mathrm{v}^{\mathbb{P}}} \mathbbm{x}_\mathrm{v}} \, d \mathbbm{x}_\mathrm{v} ~ \mbox{ \, \, \, \quad } ~ ~ ~ \mbox{ \, \, } ~ \nonumber \\
%& = & \frac{{\cal N}(\frac{(\sigma^{\mathbb{P}}_{s,\mathrm{v}})^2}{\mu_\mathrm{v}^{\mathbb{P}} (\beta_{s,\mathrm{v}}^{\mathbb{P}})^2} + \frac{(\mathbbm{x}_s - \beta_0^{\mathbb{P}})}{(\beta_{s,\mathrm{v}}^{\mathbb{P}})^2})}{\mu_\mathrm{v}^{\mathbb{P}} (\beta_{s,\mathrm{v}}^{\mathbb{P}})^2} \, ~ \,
%\exp \big( - \, \frac{(\mathbbm{x}_s - \beta_0^{\mathbb{P}})}{2 \, \mu_\mathrm{v}^{\mathbb{P}} (\beta_{s,\mathrm{v}}^{\mathbb{P}})^2} ~ ~  - \frac{1}{2}  (\frac{\sigma^{\mathbb{P}}_{s,\mathrm{v}}}{\mu_\mathrm{v}^{\mathbb{P}} \, \beta_{s,\mathrm{v}}^{\mathbb{P}}})^2 \big), \mbox{ \, \, \, \, } ~~
%\mbox{ \, }
%\label{DPSDensity1.1}
%\end{eqnarray}
%and analogously under $\mathbb{Q}$ (replacing each superscript $\mathbb{P}$ by a superscript $\mathbb{Q}$ in
%%equations
%(\ref{DPSv-jv.4})--(\ref{DPSDensity1.1})).]]]}


The tractability of the jump densities
enables the determination of
the risk premium for jumps crossing the strike in
(\ref{eq:InterimPMinusQApprox2a})
(via numerical integration).
Setting $\sigma^{\mathbb{P}}_{s,\mathrm{v}}=\sigma^{\mathbb{Q}}_{s,\mathrm{v}}$, the following parameter
restrictions facilitate
the outcome of negative risk premiums for jumps crossing the strike:
\begin{align}
&\beta_{0}^{\mathbb{P}} <0,&
& \beta_{s,\mathrm{v}}^{\mathbb{P}} < 0,&
&\beta_{0}^{\mathbb{Q}} < \beta_{0}^{\mathbb{P}},&
& \beta_{s,\mathrm{v}}^{\mathbb{Q}} < \beta_{s,\mathrm{v}}^{\mathbb{P}},&
&\mathrm{and}&
&\mu_\mathrm{v}^{\mathbb{Q}} > \mu_\mathrm{v}^{\mathbb{P}}.&
\end{align}


In other words, the option model imposes inequality restrictions to match the empirical patterns.
These parametric restrictions
have not, to our knowledge, been tested, and may be difficult
to validate.
Similar to the emphasis in \citet*{Chen_Dou_Kogan:JF2020} and \citet*{Cheng_Dou_Liao:ECMTA2021}, these
restrictions highlight the dark matter property of option models.
$\blacksquare$ \vspace{-4mm}

%%%end Section II %%%%%%%%%%%%%%%%%%%%%%%%%%%%%%%%%%%%%%%%%%%%%%%%%%%%%%%%%%%%%%%%%%%%%%%%%%%%%%%%%%%%%
%%%%%%%%%%%%%%%%%%%%%%%%%%%%%%%%%%%%%%%%%%%%%%%%%%%%%%%%%%%%%%%%%%%%%%%%%%%%%%%%%%%%%%%%

%Many asset pricing models require subtle and carefully calibrated dynamics for the fundamental processes to generate desired asset pricing %predictions.

%%%%%%%%%%%%%%%%%%%%%%%%
%%%%%%%%%%%%%%%%%%%%%%%%
%%%%%%%%%%%%%%%%%%%%%%%%
%%%%%%%%%%%%%%%%%%%%%%%%
%%%%%%%%%%%%%%%%%%%%%%%%

\section{Option models and local time risk premiums for moneyness $k$}
\label{app:var_jumps}

We outline restrictions that generate negative local time risk premiums,
%(for moneyness $k$),
in the context of the model in
(\ref{eq:dou1})--(\ref{eq:dou6}).
{By the definition of covariance, and using
$\mathbb{E}_{t}^{\mathbb{Q}}( \frac{M_{t}}{M_{T_{O}} e^{r ({T}_O - t)}} ) \, = \, 1$, we have \small
\begin{eqnarray}
\mathrm{cov}_t^{\mathbb{Q}}( \frac{M_{t}}{M_{{T}_O} e^{r ({T}_O - t)}}, \mathbb{L}^{T_O}_t[k] )
& = &
\mathbb{E}_{t}^{\mathbb{Q}}( \frac{M_{t}}{M_{{T}_O} e^{r ({T}_O - t)}} \, \mathbb{L}^{T_O}_t[k] ) ~ - ~
\overbrace{\mathbb{E}_{t}^{\mathbb{Q}}( \frac{M_{t}}{M_{{T}_O} e^{r ({T}_O - t)}} )}^{= \, 1} \,
\mathbb{E}_{t}^{\mathbb{Q}}( \mathbb{L}^{T_O}_t[k] ) ~~ \mbox{ \, \, \, }  \nonumber \\
&=& \mathbb{E}_{t}^{\mathbb{Q}}( \frac{M_{t}}{M_{{T}_O} e^{r ({T}_O - t)}} \mathbb{L}^{T_O}_t[k] ) ~-~
\mathbb{E}_{t}^{\mathbb{Q}}( \mathbb{L}^{T_O}_t[k] ) ~ \mbox{ \, } \label{c5.2} \\
%&=& e^{r ({T}_O - t)} \mathbb{E}_{t}^{\mathbb{P}}( \frac{M_{{T}_O}}{M_{t}} \times \{ \frac{M_{t}}{M_{{T}_O} e^{r ({T}_O - t)}} \mathbb{L}^{T_O}_t[k] \} ) ~ - ~
%\mathbb{E}_{t}^{\mathbb{Q}}( \mathbb{L}^{T_O}_t[k] ) \label{c5.3}\\
&=& \underbrace{\mathbb{E}_{t}^{\mathbb{P}}( \mathbb{L}^{T_O}_t[k] ) ~-~ \mathbb{E}_{t}^{\mathbb{Q}}( \mathbb{L}^{T_O}_t[k] ).}_{\tiny \mbox{local~time~risk~premium}} ~~~ \mbox{ \, }
\label{eq:covqgsbstatement}
\end{eqnarray} \normalsize
To evaluate $\mathrm{cov}_t^{\mathbb{Q}}( \frac{M_{t}}{M_{{T}_O} e^{r ({T}_O - t)}}, \mathbb{L}^{T_O}_t[k] )$,
we consider the dynamics of $\frac{M_{t}}{M_{{T}_O} e^{r ({T}_O - t)}}$ under $\mathbb{Q}$
as well as those of local time $\mathbb{L}^{T_O}_t[k]$. \vspace{-4mm}

\subsection{Expression for $\frac{M_{t}}{M_{{T}_O} e^{r ({T}_O - t)}}$ dynamics under $\mathbb{Q}$}
\label{app:OptionModelsAndOptionRiskPremiums}
Using equation (\ref{eq:dou1}), we have the following representation:\footnote{In light of Girsanov's theorem, ${z}^{\mathbb{Q}}_t$ and ${u}^{\mathbb{Q}}_t$ are independent standard Brownian motions under the probability measure $\mathbb{Q}$, linked to ${z}^{\mathbb{P}}_t$ and ${u}^{\mathbb{P}}_t$, by
$d {z}^{\mathbb{P}}_t-d {z}^{\mathbb{Q}}_t = {\eta}[t,\mathrm{v}_t] \,dt$ and
$d {u}^{\mathbb{P}}_t-d {u}^{\mathbb{Q}}_t =  {\theta}[t,\mathrm{v}_t] \, dt$.}
\begin{eqnarray}
\frac{M_{t}}{M_{{T}_O} e^{r ({T}_O - t)}}
&=& \overbrace{e^{ \int_{t}^{{T}_O} \{ -\frac{1}{2} (\eta[s,\mathrm{v}_s])^2 ds - \eta[s,\mathrm{v}_s] d z_s^{\mathbb{Q}} \}}}^{\tiny \mbox{spanned~diffusive~component}} ~ \times ~ \, \,
\overbrace{e^{ \int_{t}^{{T}_O} \{
- \frac{1}{2} (\theta[s,\mathrm{v}_s])^2 ds - \theta[s,\mathrm{v}_s]
d u^{\mathbb{Q}}_s \}}}^{\tiny \mbox{unspanned~diffusive~component}} \, \, ~ \times \mbox{ \, \, } ~ ~ \mbox{ \, \, } \nonumber \\
& & ~ \mbox{ \, \, } ~ \mbox{ \, \, } \underbrace{e^{
\{
\sum_{t < \ell \leq T_O} (-\mathbbm{x}_m)
~ - \, \int_{t}^{{T}_O} {{\bm \lambda}^{\mathbb{Q}}_{\tiny \mbox{jump}}} \, \mathbb{E}^{\mathbb{Q}}( e^{-\mathbbm{x}_m} - 1 ) \, ds \}}.}_{\tiny \mbox{unspanned~jump~component}}
~ ~ ~ \mbox{ \, \, \, \, } \label{eq:RecipPK5TermsDPSLT1}
%\\
%& \equiv &
%\mathbb{H}_{T_O}^{\tiny \mbox{span~diffusive}} \,
%\times \mathbb{H}_{T_O}^{\tiny \mbox{unspan~diffusive}} \,
%\times \,\mathbb{H}_{T_O}^{\tiny \mbox{unspan~jump}}. \mbox{ \, \, }
%\label{eq:RecipPK5TermsDPSLT1cv}
\end{eqnarray}
For compactness of equation presentation, define as follows:
\begin{eqnarray}
\mathcal{R}_{T_O}^{\tiny \mbox{span~diffusive}} &\equiv&
e^{ \int_{t}^{{T}_O} \{ -\frac{1}{2} (\eta[s,\mathrm{v}_s])^2 ds - \eta[s,\mathrm{v}_s] d z_s^{\mathbb{Q}} \}}, \label{ah.1}\\
\mathcal{R}_{T_O}^{\tiny \mbox{unspan~diffusive}} &\equiv&
e^{ \int_{t}^{{T}_O} \{
- \frac{1}{2} (\theta[s,\mathrm{v}_s])^2 ds - \theta[s,\mathrm{v}_s]
d u^{\mathbb{Q}}_s \}}, ~ ~ \mbox{ \, \, } ~~~ \mathrm{and} \mbox{ \, \, } \\
\mathcal{R}_{T_O}^{\tiny \mbox{unspan~jump}} & \equiv &
e^{
\{
\sum_{t < \ell \leq T_O} (-\mathbbm{x}_m)
~ - \, \int_{t}^{{T}_O} {{\bm \lambda}^{\mathbb{Q}}_{\tiny \mbox{jump}}} \, \mathbb{E}^{\mathbb{Q}}( e^{-\mathbbm{x}_m} - 1 ) \, ds \}}. ~ \mbox{ \, \, }
\end{eqnarray}
Then, we can write the reciprocal of the Radon-Nikodym derivative as  follows:
\begin{eqnarray}
\frac{M_{t}}{M_{{T}_O} e^{r ({T}_O - t)}} \, \, \, = \, \,
\mathcal{R}_{T_O}^{\tiny \mbox{span~diffusive}} \, \,
\times \, \mathcal{R}_{T_O}^{\tiny \mbox{unspan~diffusive}} \, \,
\times \, \mathcal{R}_{T_O}^{\tiny \mbox{unspan~jump}}. ~ \mbox{ \quad \, \, } ~
\label{eq:RecipPK5TermsDPSLT1cv}
\end{eqnarray}
Thus, $\frac{M_{t}}{M_{{T}_O} e^{r ({T}_O - t)}}$ is multiplicative in three positive (orthogonal) martingales under $\mathbb{Q}$.
%\footnote{As noted,
%$\mathbb{H}_{T_O}^{\tiny \mbox{span~diffusive}}$ captures the
%component of the pricing kernel which is diffusive and
%perfectly instantaneously correlated
%with the diffusive component of the futures price dynamics
%(and therefore captures spanned risks);
%$\mathbb{H}_{T_O}^{\tiny \mbox{unspan~diffusive}}$
%captures the
%component of the pricing kernel which is diffusive and
%orthogonal to the futures price dynamics
%(and therefore captures unspanned risks);
%$\mathbb{H}_{T_O}^{\tiny \mbox{unspan~jump}}$ captures the
%discontinuous component of the pricing kernel.
%Each of $(\frac{1}{M_{s} e^{r s}})$,
%$(\mathbb{H}_{s}^{\tiny \mbox{span~diffusive}})$,
%$(\mathbb{H}_{s}^{\tiny \mbox{unspan~diffusive}})$, and
%$(\mathbb{H}_{s}^{\tiny \mbox{unspan~jump}})$ is a martingale
%under $\mathbb{Q}$.}
%since, for example,
%\begin{eqnarray}
%\mathbb{E}_{t}^{\mathbb{Q}}( \frac{M_{t}}{M_{{T}_O} e^{r ({T}_O - t)}} ) ~ = ~
%\mathbb{E}_{t}^{\mathbb{P}}( \frac{M_{{T}_O}}{M_{t}} e^{ r ({T}_O - t)} ~ \frac{M_{t}}{M_{{T}_O} e^{r ({T}_O - t)}} ) ~ = ~
%\mathbb{E}_{t}^{\mathbb{P}}( 1 ) ~ = ~ 1.  ~~ \mbox{ \quad } ~ \label{eq:eachComponentQMartingale}
%\end{eqnarray}


\subsection{Characterizing the
%local time risk premium Objects determining the
sign of the local time risk premiums}

For the results that follow, we define the
following. \small
\begin{equation}
\mathrm{Let~} \, \mathcal{I}_{s} ~ \mbox{be~the~sub-filtration~of} ~ \mathcal{F}_{s}~\mathrm{generated~by} ~ \mathcal{R}_{s}^{\tiny \mbox{span~diffusive}},\mathrm{~that~is,}~\mathrm{by}~\eta[s,\mathrm{v}_s]~\mathrm{and}~\eta[s,\mathrm{v}_s] d z_s^{\mathbb{Q}}. ~ \mbox{ \, \, \, } ~
 \label{eq:SubFiltrationDefinition}
\end{equation} \normalsize
Exploiting the law of total covariance, the risk premium for local time, with moneyness $k$, is \small
\begin{eqnarray}
\mathrm{cov}_t^{\mathbb{Q}}( \overbrace{\frac{M_{t}}{M_{{T}_O} e^{r ({T}_O - t)}} }^{\mathrm{from~\tiny(\ref{eq:RecipPK5TermsDPSLT1}})}, \, \mathbb{L}_t^{{T}_O}[k] )&=&  \mathbb{E}_{t}^{\mathbb{Q}}( \mathrm{cov}_t^{\mathbb{Q}}( \frac{M_{t}}{M_{{T}_O} e^{r ({T}_O - t)}},
\, \mathbb{L}_t^{{T}_O}[k] {\Big |} \, \mathcal{I}_{T_O} ) ) \nonumber \\
&&  + \, \mathrm{cov}_t^{\mathbb{Q}}( \mathbb{E}_{t}^{\mathbb{Q}}( \frac{M_{t}}{M_{{T}_O} e^{r ({T}_O - t)}} \,  {\Big |} \mathcal{I}_{T_O} ), \, \, \mathbb{E}_{t}^{\mathbb{Q}}( \mathbb{L}_t^{{T}_O}[k] \,{\Big |} \, \mathcal{I}_{T_O} ) ) ~ \mbox{ \, \, \, }
%\label{eq:NewConditionalCovariance1Eq}
\nonumber
\\
&=& \mathbb{E}_{t}^{\mathbb{Q}}( \mathrm{cov}_t^{\mathbb{Q}}(
\mathcal{R}_{T_O}^{\tiny \mbox{span~diffusive}}
\times \mathcal{R}_{T_O}^{\tiny \mbox{unspan~diffusive}}
\times \mathcal{R}_{T_O}^{\tiny \mbox{unspan~jump}},
\, \mathbb{L}_t^{{T}_O}[k] {\Big |} \, \mathcal{I}_{T_O} ) ) ~~ \mbox{ \, \, } ~~~ \nonumber \\
& & \, + \, \mathrm{cov}_t^{\mathbb{Q}}( \mathbb{E}_{t}^{\mathbb{Q}}( \frac{M_{t}}{M_{{T}_O}} e^{-r ({T}_O - t)} \, \, {\Big |} \mathcal{I}_{T_O} ), \, \, \mathbb{E}_{t}^{\mathbb{Q}}( \mathbb{L}_t^{{T}_O}[k] \, {\Big |}\, \mathcal{I}_{T_O} ) )
\nonumber
\\
&=&\mathbb{E}_{t}^{\mathbb{Q}}( \mathcal{R}_{T_O}^{\tiny \mbox{span~diffusive}} \times
\mathrm{cov}_t^{\mathbb{Q}}(
\mathcal{R}_{T_O}^{\tiny \mbox{unspan~diffusive}}
\times \mathcal{R}_{T_O}^{\tiny \mbox{unspan~jump}},
\, \mathbb{L}_t^{{T}_O}[k] {\Big |} \, \mathcal{I}_{T_O} ) ) ~~ \mbox{ \, \, } ~~~ \nonumber \\
& &
\, +\,
\mathrm{cov}_t^{\mathbb{Q}}( \mathbb{E}_{t}^{\mathbb{Q}}( \frac{M_{t}}{M_{{T}_O}} e^{-r ({T}_O - t)} \, \, {\Big |} \mathcal{I}_{T_O} ), \, \, \mathbb{E}_{t}^{\mathbb{Q}}( \mathbb{L}_t^{{T}_O}[k] \, {\Big |} \, \mathcal{I}_{T_O} ) ). \mbox{ \, }~\mbox{ \, \, }~
%\nonumber
\label{eq:ConditCovarSpanning2Linesa}
\end{eqnarray} \normalsize

To reproduce the empirical finding of negative risk premiums of OTM calls, one
may require negative local time
risk premiums, which in view of equation (\ref{eq:ConditCovarSpanning2Linesa}) leads us to assess when the
two terms appearing in (\ref{eq:ConditCovarSpanning2Linesa}) can be negative.

In particular, examining (\ref{eq:ConditCovarSpanning2Linesa}), we are interested in when the term involving unspanned risks, specifically,
\begin{eqnarray}
\mathrm{cov}_t^{\mathbb{Q}}( \mathcal{R}_{T_O}^{\tiny \mbox{unspan~diffusive}} \times \mathcal{R}_{T_O}^{\tiny \mbox{unspan~jump}},\, \mathbb{L}_t^{{T}_O}[k] {\Big |} \,
\mathcal{I}_{T_O} ) ~ ~ \mbox{ \quad \, is negative. \quad \, } ~ ~~ \label{eq:CovarianceTermWhoseSign}
\end{eqnarray}

To keep the analysis contained, our approach is twofold,
as follows:
\begin{enumerate}

\item Assess the economic implications of the sign of $\mathrm{cov}_t^{\mathbb{Q}}( \mathcal{R}_{T_O}^{\tiny \mbox{unspan~diffusive}} \times \mathcal{R}_{T_O}^{\tiny \mbox{unspan~jump}},\, \mathbb{L}_t^{{T}_O}[k] {\Big |} \,
\mathcal{I}_{T_O} )$ in Section~\ref{app:unspannedDiffusiveVolRisk}
(see (\ref{eq:FinalCovarianceTerm3DPS})--(\ref{eq:FinalCovarianceTermisNegIfThetaNeg4DPS})
and in Section~\ref{app:unspannedVolJumpRisk}
(see
(\ref{c5.10DPS3})--(\ref{eq:FinalCovarianceTermisNegIfThetaNeg4DPSSecondTerm})).

\item Then, assess the economic implications of the sign of
$\mathrm{cov}_t^{\mathbb{Q}}( \mathbb{E}_{t}^{\mathbb{Q}}( \frac{M_{t}}{M_{{T}_O}} e^{-r ({T}_O - t)} \,  {\Big |} \mathcal{I}_{T_O} ), \, \, \mathbb{E}_{t}^{\mathbb{Q}}( \mathbb{L}_t^{{T}_O}[k] \, {\Big |} \mathcal{I}_{T_O} ) )$ in
Section~\ref{app:LTRPSpannedDiffusiveVolRisk}.

\end{enumerate}
We turn to these tasks in turn. \vspace{-4mm}

\subsection{Evolution of  diffusive component of the futures return
%equity return variance
under $\mathbb{Q}$}

First, we note that the evolution of $\mathrm{v}_{\ell}$
%return variance
under $\mathbb{Q}$ is
\begin{eqnarray}
\mathrm{v}_{\ell} & =  & \mathrm{v}_{t}\, e^{ \kappa_{\mathrm{vol}}^{\mathbb{Q}} ( t - \ell ) }
~+~
\int_{t}^{\ell} \phi_{\tiny \mbox{vol}}^{\mathbb{Q}} e^{ \kappa_{\tiny \mbox{vol}}^{\mathbb{Q}} ( s - \ell ) } ds
~+~ \sigma_{\tiny \mbox{vol}}\,\rho_{\tiny \mbox{vol}} \overbrace{\int_{t}^{\ell}  e^{ \kappa_{\tiny \mbox{vol}}^{\mathbb{Q}} ( s - \ell ) } \sqrt{\mathrm{v}_{s}} \,  dz_s^{\mathbb{Q}}}^{\tiny \mbox{spanned~diffusive~volatility~risk}} \nonumber \\
&+&\sigma_{\tiny \mbox{vol}} \, \sqrt{1-\rho^2_{\tiny \mbox{vol}}}
\underbrace{\int_{t}^{\ell}  e^{ \kappa_{\tiny \mbox{vol}}^{\mathbb{Q}} ( s - \ell ) } \sqrt{\mathrm{v}_{s}} \,  du_s^{\mathbb{Q}}}_{\tiny \mbox{unspanned~diffusive~volatility~risk}}
\, + \,
\underbrace{\int_{t}^{\ell}  e^{ \kappa_{\tiny \mbox{vol}}^{\mathbb{Q}} ( s - \ell ) }\,
\mathbbm{x}_{\mathrm{v}} \,
d \mathbb{N}^{\mathbb{Q}}_{s}}_{\tiny \mbox{unspanned~volatility~jump~risk}}~~~~~~\mbox{for $\ell \geq t$.\, \, \, }~~
\mbox{ \, \, } ~
\label{cv.sDPS}
\end{eqnarray}

To obtain an expression for $\mathbb{L}^{T_O}_t[k]$, we note that the path-by-path continuous part of the quadratic variation $[ G^\mathrm{c}, G^\mathrm{c} ]_{s} =
\int_{t}^s \{ \sqrt{\mathrm{v}_{\ell}} \, G_{\ell} \}^2 \, d\ell = \int_{t}^s \, \mathrm{v}_{\ell} \,G_\ell^2 \,d\ell$.
We deduce the form of $\mathbb{L}^{T_O}_t[k]$ as
\begin{eqnarray}
\mathbb{L}^{T_O}_t[k] & = & \frac{1}{2} \int_{t}^{T_O} \delta_{\{G_\ell- k\}} d [ G^\mathrm{c}, G^\mathrm{c} ]_{\ell}
%~ ~ \, \mbox{ \, }  \nonumber \\
~=~ \frac{1}{2} \, \int_{t}^{{T}_O} \, \delta_{\{G_\ell - k\}} \, \mathrm{v}_{\ell} \, G_\ell^2 \, d\ell ~ \mbox{ \, } ~ ~~ ~ \nonumber \\
& = & \frac{1}{2} \int_{t}^{T_O} \delta_{\{G_\ell - k\}}
\big\{ \, \overbrace{\mathrm{v}_{t}\, e^{ \kappa_{\tiny \mbox{vol}}^{\mathbb{Q}} ( t - \ell ) } + \int_{t}^{\ell} \phi_{\tiny \mbox{vol}}^{\mathbb{Q}} e^{ \kappa_{\tiny \mbox{vol}}^{\mathbb{Q}} ( s - \ell ) } ds }^{\tiny \mbox{irrelevant~for~conditional~covariance~in~(\ref{eq:CovarianceTermWhoseSign})}}
~ + ~ \sigma_{\tiny \mbox{vol}}\,\rho_{\tiny \mbox{vol}} \int_{t}^{\ell}  e^{ \kappa_{\tiny \mbox{vol}}^{\mathbb{Q}} ( s - \ell ) } \sqrt{\mathrm{v}_{s}} \,  dz_s^{\mathbb{Q}} ~
\nonumber \\
& + & \underbrace{\sigma_{\tiny \mbox{vol}} \, \sqrt{1-\rho^2_{\tiny \mbox{vol}}} \int_{t}^{\ell}  e^{ \kappa_{\tiny \mbox{vol}}^{\mathbb{Q}} ( s - \ell ) } \sqrt{\mathrm{v}_{s}} \,  du_s^{\mathbb{Q}}}_{\tiny \mbox{covaries~(relevant~in~(\ref{eq:CovarianceTermWhoseSign}))}}
\, + \,
\underbrace{\int_{t}^{\ell}  e^{ \kappa_{\tiny \mbox{vol}}^{\mathbb{Q}} ( s - \ell ) }\,
\mathbbm{x}_{\mathrm{v}} \,
d \mathbb{N}^{\mathbb{Q}}_{s}}_{\tiny \mbox{covaries~(relevant~in~(\ref{eq:CovarianceTermWhoseSign}))}} \big\}
\, G_\ell^2 \, d\ell.
\label{eq:longLDPS1}
\end{eqnarray}
%\subsection{$\mathrm{cov}_t^{\mathbb{Q}}( \mathbb{H}_{T_O}^{\tiny \mbox{unspan~diffusive}} \times \mathbb{H}_{T_O}^{\tiny \mbox{unspan~jump}}, \, \mathbb{L}_t^{{T}_O}[k] {\Big |} \,
%\mathcal{I}_{T_O} )$ in equation (\ref{eq:CovarianceTermWhoseSign})}
Using (\ref{eq:longLDPS1}), we now substitute
for $\mathbb{L}^{T_O}_t[k]$ into
(\ref{eq:CovarianceTermWhoseSign}). Recognizing that some terms are irrelevant in the computation
of
$\mathrm{cov}_t^{\mathbb{Q}}( \mathcal{R}_{T_O}^{\tiny \mbox{unspan~diffusive}} \times \mathcal{R}_{T_O}^{\tiny \mbox{unspan~jump}},\, \mathbb{L}_t^{{T}_O}[k] {\Big |} \,
\mathcal{I}_{T_O} )$, we determine as follows:
\begin{eqnarray}
& &
\mathrm{cov}_t^{\mathbb{Q}}( \mathcal{R}_{T_O}^{\tiny \mbox{unspan~diffusive}} \times \mathcal{R}_{T_O}^{\tiny \mbox{unspan~jump}}, \, \mathbb{L}_t^{{T}_O}[k] {\Big |} \,
\mathcal{I}_{T_O} ) ~
\label{asssd.1}
\nonumber \\
& & ~ = \, \mathrm{cov}_t^{\mathbb{Q}}( \mathcal{R}_{T_O}^{\tiny \mbox{unspan~diffusive}} ~
\times ~ \mathcal{R}_{T_O}^{\tiny \mbox{unspan~jump}},
\nonumber \\
& & ~ \mbox{ \, \, } ~~~~ ~ (\frac{1}{2} \int_{t}^{T_O} \delta_{\{G_\ell - k\}} \sigma_{\mathrm{vol}} \, \sqrt{1-\rho^2_{\tiny \mbox{vol}}} \int_{t}^{\ell}  e^{ \kappa_{\tiny \mbox{vol}}^{\mathbb{Q}} ( s - \ell ) } \sqrt{\mathrm{v}_{s}} \,  du_s^{\mathbb{Q}} \, G_\ell^2 \, d\ell \, \, + \, \nonumber \\
& & ~ \mbox{ \, \, \, \, } ~~~~ ~ ~ \frac{1}{2} \int_{t}^{T_O} \delta_{\{G_\ell- k\}} \int_{t}^{\ell}  e^{ \kappa_{\mathrm{vol}}^{\mathbb{Q}} ( s - \ell ) }
\mathbbm{x}_{\mathrm{v}} \,
d \mathbb{N}^{\mathbb{Q}}_{s} \, G_\ell^2 \, d\ell \, \big)
\, \, {\Big |} \mathcal{I}_{T_O} ) ~
~ \mbox{ \, \, \, \, \, }
\nonumber \\
& & ~ = \, \mathrm{cov}_t^{\mathbb{Q}}(
\mathcal{R}_{T_O}^{\tiny \mbox{unspan~diffusive}},
\frac{1}{2} \int_{t}^{T_O} \delta_{\{G_\ell- k\}} \sigma_{\tiny \mbox{vol}} \sqrt{1-\rho^2_{\tiny \mbox{vol}}} \int_{t}^{\ell}  e^{ \kappa_{\tiny \mbox{vol}}^{\mathbb{Q}} ( s - \ell ) } \sqrt{\mathrm{v}_{s}} du_s^{\mathbb{Q}} G_\ell^2 \, d\ell \, {\Big |} \mathcal{I}_{T_O} )
\mbox{ \, \, } \,
\nonumber \\
& & ~ \mbox{ \, \, }
~ + ~ \mathrm{cov}_t^{\mathbb{Q}}(
\mathcal{R}_{T_O}^{\tiny \mbox{unspan~jump}}, \frac{1}{2} \int_{t}^{T_O} \delta_{\{G_\ell- k\}} \int_{t}^{\ell}  e^{ \kappa_{\mathrm{vol}}^{\mathbb{Q}} ( s - \ell ) }
\mathbbm{x}_{\mathrm{v}} \,
d \mathbb{N}^{\mathbb{Q}}_{s}
\, G_\ell^2 \, d\ell \, \, {\Big |} \mathcal{I}_{T_O} ),
~ ~ \mbox{ \, \, \, \, \, } \label{eq:JumpRelatedLT2}
\end{eqnarray}
where we have exploited independence between $d u^{\mathbb{Q}}_s$ and $d \mathbb{N}^{\mathbb{Q}}_{s}$.

Thus, the conditional covariance in
(\ref{eq:JumpRelatedLT2}) consists of two
parts: (i) an unspanned diffusion-related term and
(ii) an unspanned jump-related term.

Next, we elaborate the economic rationale under which these derived terms can be signed. \vspace{-4mm}

%%%%%%%%%%%%%%%%%%%%%%

\subsection{Negative local time risk premium for unspanned diffusive volatility risk}
\label{app:unspannedDiffusiveVolRisk}

The sign of the first term in (\ref{eq:JumpRelatedLT2}) (after substituting from
(\ref{eq:RecipPK5TermsDPSLT1})) is the sign of
{\small \begin{equation}
\mathrm{cov}_t^{\mathbb{Q}}( e^{ \int_{t}^{{T}_O} \{
- \frac{1}{2} (\theta[s,\mathrm{v}_s])^2 ds - \theta[s,\mathrm{v}_s]
d u^{\mathbb{Q}}_s \}},
\frac{1}{2} \int_{t}^{T_O} \delta_{\{G_\ell-k\}} \sigma_{\mathrm{vol}} \, \int_{t}^{\ell}  e^{ \kappa_{\mathrm{vol}}^{\mathbb{Q}} ( s - \ell ) } \sqrt{\mathrm{v}_{s}} \,  du_s^{\mathbb{Q}} \, G_\ell^2 \, d\ell \, \, {\Big |} \mathcal{I}_{T_O}  ), ~ ~
\end{equation}}
which is the same (in light of Stein's lemma) as the sign of
\small
\begin{eqnarray}
& & \mathrm{cov}_t^{\mathbb{Q}}( \int_{t}^{{T}_O} - \theta[s,\mathrm{v}_s]
d u^{\mathbb{Q}}_s, \, \frac{1}{2} \int_{t}^{T_O} \delta_{\{G_\ell- k\}} \sigma_{\mathrm{vol}} \, \int_{t}^{\ell}  e^{ \kappa_{\mathrm{vol}}^{\mathbb{Q}} ( s - \ell ) } \sqrt{\mathrm{v}_{s}} \,  du_s^{\mathbb{Q}} \, G_\ell^2 \, d\ell \, \, {\Big |} \mathcal{I}_{T_O} ) \nonumber \\
& & \, \, = ~ \mathrm{cov}_t^{\mathbb{Q}}( \int_{t}^{{T}_O} -
\{-\theta_{\mathrm{LT}} \, \sqrt{\mathrm{v}_s} \, \, d u^{\mathbb{Q}}_{s}\}, \, \frac{1}{2} \int_{t}^{T_O} \delta_{\{G_\ell - k\}}
\sigma_{\tiny \mbox{vol}} \, \int_{t}^{\ell}  e^{ \kappa_{\tiny \mbox{vol}}^{\mathbb{Q}} ( s - \ell ) } \sqrt{\mathrm{v}_{s}} \,  du_s^{\mathbb{Q}} \, G_\ell^2 \, d\ell
\, {\Big |} \mathcal{I}_{T_O} )  ~ \mbox{ \, \, \, } ~ \nonumber \\
& & \, \, = ~ \mathrm{cov}_t^{\mathbb{Q}}( \int_{t}^{{T}_O} \theta_{\mathrm{LT}} \, \sqrt{\mathrm{v}_s} \, \, d u^{\mathbb{Q}}_{s},
\, \int_{t}^{{T}_O} \sqrt{\mathrm{v}_{s}} \, \{\ \int_{s}^{{T}_O} \frac{\sigma_{\tiny \mbox{vol}}}{2} e^{\kappa_{\tiny \mbox{vol}}^{\mathbb{Q}} ( s - \ell ) } \, \delta_{\{G_\ell - k\}} \, G_\ell^2 \, d\ell \} \, du^{\mathbb{Q}}_{s} \,{\Big |} \mathcal{I}_{T_O} )  \, \, \nonumber \\
& & \, \, = ~ \mathbb{E}_{t}^{\mathbb{Q}}( \int_{t}^{{T}_O} \theta_{\mathrm{LT}} \, \sqrt{\mathrm{v}_s} \,  \sqrt{\mathrm{v}_s} \,
\, \{\ \int_{s}^{{T}_O} \frac{\sigma_{\tiny \mbox{vol}} }{2} e^{\kappa_{\tiny \mbox{vol}}^{\mathbb{Q}} ( s - \ell ) } \,
\delta_{\{G_\ell- k\}}
\, G_\ell^2 \, d\ell \} \,  ds \, {\Big |} \mathcal{I}_{T_O})  \, \nonumber \\
& & \, \, = ~ \theta_{\mathrm{LT}} \, \, \,
\underbrace{\mathbb{E}_{t}^{\mathbb{Q}}(  \int_{t}^{{T}_O} \, \mathrm{v}_s \, \, \{\ \int_{s}^{{T}_O} \frac{\sigma_{\tiny \mbox{vol}}}{2}
e^{ \kappa_{\tiny \mbox{vol}}^{\mathbb{Q}} ( s - \ell ) } \, \delta_{\{G_\ell - k\}}
\, G_\ell^2 \, d\ell \} \, ds \,{\Big |} \mathcal{I}_{T_O} ) }_{~\geq ~0}.
\label{eq:FinalCovarianceTerm3DPS}
\end{eqnarray} \normalsize

Inspection of (\ref{eq:FinalCovarianceTerm3DPS}) shows that  \vspace{-3mm}
\begin{eqnarray}
\mbox{ the diffusion-related term in (\ref{eq:JumpRelatedLT2}) is negative if ~$\theta_{\mathrm{LT}} \, <  \, 0$. \, \, } ~ ~~ ~ ~~
\mbox{ \, \quad \quad \quad } %\blacksquare
\label{eq:FinalCovarianceTermisNegIfThetaNeg4DPS}
\end{eqnarray}
This is the restriction required for negative local time risk premiums for unspanned diffusive volatility risks. $\blacksquare$ \vspace{-4mm}

\subsection{Negative local time risk premiums due to unspanned volatility jump risks}
\label{app:unspannedVolJumpRisk}

With the term $\int_{t}^{{T}_O} {{\bm \lambda}^{\mathbb{Q}}_{\tiny \mbox{jump}}} \, \mathbb{E}^{\mathbb{Q}}( e^{-\mathbbm{x}_m} - 1 ) \, ds$
not relevant for the conditional covariance, the second term in (\ref{eq:JumpRelatedLT2}) is \vspace{-3mm}
\begin{eqnarray}
& & \mathrm{cov}_t^{\mathbb{Q}}(
e^{ \{ \sum_{t < \ell \leq T_O} (-\mathbbm{x}_m) \} },
\frac{1}{2} \int_{t}^{T_O} \delta_{\{G_\ell-k\}}
\int_{t}^{\ell}  e^{ \kappa_{\tiny \mbox{vol}}^{\mathbb{Q}} ( s - \ell ) }
\mathbbm{x}_{\mathrm{v}} \,
d \mathbb{N}^{\mathbb{Q}}_{s} \, G_\ell^2 \, d\ell \, {\Big |} \mathcal{I}_{T_O} )
 % end color blue
%~ \mbox{\small (change order of integration) \quad } \,\,\,\,\,\,\,
\nonumber \\
& & = \mathrm{cov}_t^{\mathbb{Q}}(
e^{ \{ \sum_{t < \ell \leq T_O} (-\mathbbm{x}_m) \}}, \frac{1}{2} \int_{t}^{T_O} \{\ \int_{s}^{T_O} \delta_{\{G_\ell- k\}}
e^{ \kappa_{\tiny \mbox{vol}}^{\mathbb{Q}} ( s - \ell ) }\,
\mathbbm{x}_{\mathrm{v}} \,
G_\ell^2 \, d\ell \} \, ~ d \mathbb{N}^{\mathbb{Q}}_{s}
\, {\Big |} \mathcal{I}_{T_O} )
 ~ ~ \mbox{ \, \quad } ~
 % end color blue
\nonumber \\
& &  = ~ \mathrm{cov}_t^{\mathbb{Q}}(
e^{ \{ \sum_{t < \ell \leq T_O} (-\mathbbm{x}_m)  \}},~\frac{1}{2} \sum_{t < \ell \leq T_O} \{\ \int_{s}^{T_O}
\delta_{\{G_\ell - k\}}
e^{ \kappa_{\tiny \mbox{vol}}^{\mathbb{Q}} ( s - \ell ) }\,
\mathbbm{x}_{\mathrm{v}} \,
G_\ell^2 \, d\ell \}
\, {\Big |} \mathcal{I}_{T_O}).
% end color blue
 ~ ~ \mbox{ \, \quad } ~
\label{c5.10DPS3}
\end{eqnarray}
%{\color{red}[[[[ I still think ``The principal determinant" would be
%better here? ]]]]]]}
Among the determinants of the sign of equation (\ref{c5.10DPS3}) and, thus, of
(\ref{eq:JumpRelatedLT2}) is the sign of $\mathrm{cov}^{\mathbb{Q}}( e^{-\mathbbm{x}_m},
\mathbbm{x}_{\mathrm{v}} )$.
In particular, for a
negative contribution to the local time risk premium, one is
led
to postulate the following restriction: \vspace{-3mm}
\begin{eqnarray}
\mathrm{cov}^{\mathbb{Q}}( e^{-\mathbbm{x}_m},
\mathbbm{x}_{\mathrm{v}} ) \, < \, 0.
%~~ ~
%{\color{red}[[[ (condition for jump-related term in (\ref{eq:JumpRelatedLT2}) to be negative) ]]]}} ~ \mbox{ \, \, }
\label{eq:FinalCovarianceTermisNegIfThetaNeg4DPSSecondTerm}
\end{eqnarray}
Equation (\ref{eq:FinalCovarianceTermisNegIfThetaNeg4DPSSecondTerm}) holds
when model parameters under $\mathbb{Q}$ are such that
large jumps in volatility associate with large up jumps in the pricing kernel.
$\blacksquare$ \vspace{-4mm}

\subsection{Local time risk premiums due to spanned risks}
\label{app:LTRPSpannedDiffusiveVolRisk}

In light of the fact that
\begin{align}
&\mathbb{E}_{t}^{\mathbb{Q}}( \mathcal{R}_{T_O}^{\tiny \mbox{unspan~diffusive}} \, {\Big |}\, \mathcal{I}_{T_O} ) = 1&
&\mathrm{and}&
&\mathbb{E}_{t}^{\mathbb{Q}}( \mathcal{R}_{T_O}^{\tiny \mbox{unspan~jump}} \, {\Big |}\, \mathcal{I}_{T_O} ) = 1, &
\end{align}
we consider the final term $\mathrm{cov}_t^{\mathbb{Q}}( \mathbb{E}_{t}^{\mathbb{Q}}( \frac{M_{t}}{M_{{T}_O}} e^{-r ({T}_O - t)} \, \, {\Big |} \mathcal{I}_{T_O} ), \, \, \mathbb{E}_{t}^{\mathbb{Q}}( \mathbb{L}_t^{{T}_O}[k] \, {\Big |} \mathcal{I}_{T_O} ) )$ in equation (\ref{eq:ConditCovarSpanning2Linesa}).

Direct evaluation of the covariance is unrevealing.
Therefore,  we cast this final term in terms of economic variables, specifically, expectations of option payoffs.

To see our rationale, we work through the covariance as follows:
\begin{eqnarray}
& &
\mathrm{cov}_t^{\mathbb{Q}}( \mathbb{E}_{t}^{\mathbb{Q}}( \frac{M_{t}}{M_{{T}_O}} e^{-r ({T}_O - t)} \, \, {\Big |} \mathcal{I}_{T_O} ), \, \, \mathbb{E}_{t}^{\mathbb{Q}}( \mathbb{L}_t^{{T}_O}[k] \, {\Big |} \mathcal{I}_{T_O} ) )  \nonumber \\
&& ~ = ~
\mathrm{cov}_t^{\mathbb{Q}}( \mathbb{E}_{t}^{\mathbb{Q}}( e^{ \int_{t}^{{T}_O} \{ -\frac{1}{2} (\eta[s,\mathrm{v}_s])^2 ds - \eta[s,\mathrm{v}_s] d z_s^{\mathbb{Q}} \}} \, \, {\Big |} \mathcal{I}_{T_O} ), \, \, \mathbb{E}_{t}^{\mathbb{Q}}( \mathbb{L}_t^{{T}_O}[k] \, {\Big |}\, \mathcal{I}_{T_O} ) ) ~ \mbox{ \, \, \quad } ~~ \label{eq:CovarianceForSpanned1} \\
& & ~ = \, \mathbb{E}_{t}^{\mathbb{Q}}( \mathbb{E}_{t}^{\mathbb{Q}}( e^{ \int_{t}^{{T}_O} \{ -\frac{1}{2} (\eta[s,\mathrm{v}_s])^2 ds - \eta[s,\mathrm{v}_s] d z_s^{\mathbb{Q}} \}} \, {\Big |} \mathcal{I}_{T_O} ) \, \, \mathbb{E}_{t}^{\mathbb{Q}}( \mathbb{L}_t^{{T}_O}[k] \, {\Big |} \mathcal{I}_{T_O} ) )
%sant ~ - ~ ~***
\nonumber \\
& &~~~~~~-~~
%sant \mbox{ \, \quad } ~ ~ \, ~
%\mbox{ \, \, }
\overbrace{\mathbb{E}_{t}^{\mathbb{Q}}( \mathbb{E}_{t}^{\mathbb{Q}}( e^{ \int_{t}^{{T}_O} \{ -\frac{1}{2} (\eta[s,\mathrm{v}_s])^2 ds - \eta[s,\mathrm{v}_s] d z_s^{\mathbb{Q}} \}} \, {\Big |} \mathcal{I}_{T_O} ) )}^{= \, \, \mathbb{E}_{t}^{\mathbb{Q}}( e^{ \int_{t}^{{T}_O} \{ -\frac{1}{2} (\eta[s,\mathrm{v}_s])^2 ds - \eta[s,\mathrm{v}_s] d z_s^{\mathbb{Q}} \}} ) \, ~ = ~ 1} ~
\overbrace{\mathbb{E}_{t}^{\mathbb{Q}}( \mathbb{E}_{t}^{\mathbb{Q}}( \mathbb{L}_t^{{T}_O}[k] \,{\Big |} \mathcal{I}_{T_O} ) )}^{= \, \mathbb{E}_{t}^{\mathbb{Q}}( \mathbb{L}_t^{{T}_O}[k] )} ~ \mbox{ \, \, \quad } ~
\nonumber \\
& & ~ = \, \mathbb{E}_{t}^{\mathbb{Q}}( \underbrace{ ~ \mathbb{E}_{t}^{\mathbb{Q}}( e^{ \int_{t}^{{T}_O} \{ -\frac{1}{2} (\eta[s,\mathrm{v}_s])^2 ds - \eta[s,\mathrm{v}_s] d z_s^{\mathbb{Q}} \}} \, {\Big |} \mathcal{I}_{T_O} ) ~ }_{= \, \, e^{ \int_{t}^{{T}_O} \{ -\frac{1}{2} (\eta[s,\mathrm{v}_s])^2 ds - \eta[s,\mathrm{v}_s] d z_s^{\mathbb{Q}} \}}} \, \, \mathbb{E}_{t}^{\mathbb{Q}}( \mathbb{L}_t^{{T}_O}[k] \, {\Big |} \mathcal{I}_{T_O} ) ) ~ - ~ \mathbb{E}_{t}^{\mathbb{Q}}( \mathbb{L}_t^{{T}_O}[k] ) ~ \mbox{ \, \, \, \, \, \, } ~ ~ \nonumber \\
& & ~ = \, \mathbb{E}_{t}^{\mathbb{Q}}( \mathbb{E}_{t}^{\mathbb{Q}}( e^{ \int_{t}^{{T}_O} \{ -\frac{1}{2} (\eta[s,\mathrm{v}_s])^2 ds - \eta[s,\mathrm{v}_s] d z_s^{\mathbb{Q}} \}} \, \, \mathbb{L}_t^{{T}_O}[k] \,{\Big |} \mathcal{I}_{T_O} ) ) ~ - ~ \mathbb{E}_{t}^{\mathbb{Q}}( \mathbb{L}_t^{{T}_O}[k] ) ~ \mbox{ \, \, \, \, \, \, } ~ ~
\nonumber \\
& & ~ = \, \mathbb{E}_{t}^{\mathbb{Q}}( e^{ \int_{t}^{{T}_O} \{ -\frac{1}{2} (\eta[s,\mathrm{v}_s])^2 ds - \eta[s,\mathrm{v}_s] d z_s^{\mathbb{Q}} \}} \, \, \mathbb{L}_t^{{T}_O}[k] ) ~ - ~ \mathbb{E}_{t}^{\mathbb{Q}}( \mathbb{L}_t^{{T}_O}[k] ) ~ ~ ~ \mbox{ \, \, \, \, \, \, (now use (\ref{ah.1})) \, \, \, } \nonumber \\
& & ~ = \, \mathbb{E}_{t}^{\mathbb{Q}}( \mathcal{R}_{T_O}^{\tiny \mbox{span~diffusive}} \, \, \mathbb{L}_t^{{T}_O}[k] ) ~ - ~ \mathbb{E}_{t}^{\mathbb{Q}}( \mathbb{L}_t^{{T}_O}[k] ) ~ \mbox{ \, \, \, \, \, \, } \label{eq:Interim1} \\
%& & {\color{red} [[[[ ~ = \, \mathbb{E}_{t}^{\mathbb{Q}}( \{ e^{ \int_{t}^{{T}_O} \{ -\frac{1}{2} (\eta[s,\mathrm{v}_s])^2 ds - \eta[s,\mathrm{v}_s] d z_s^{\mathbb{Q}} \}} \, - \, 1 \} \, \, \mathbb{L}_t^{{T}_O}[k] ) ~ ]]]] ~ } \mbox{ \, \, \, \, \, \, } \label{eq:Interim2} \\
& & ~ = \,
\mathrm{cov}_t^{\mathbb{Q}}(
e^{ \int_{t}^{{T}_O} \{ -\frac{1}{2} (\eta[s,\mathrm{v}_s])^2 ds - \eta[s,\mathrm{v}_s] d z_s^{\mathbb{Q}} \}}, \mathbb{L}_t^{{T}_O}[k] ).
\label{eqa.contt0}
\end{eqnarray}
We now use Tanaka's formula in equation (\ref{eq:Interim1}) to
substitute out $\mathbb{L}_t^{{T}_O}[k]$ and
re-express our quantity of interest in terms of option payoffs.
From the definition of expected call returns in equation (\ref{eq:ExpectedHoldingReturn1GneralDynamics}), we note that the
expected \emph{excess} return of holding a call option over $t$ to ${T}_O$ is
\begin{eqnarray}
\underbrace{1 + \mu^{{T}_O}_{t,{\tiny \mathrm{call}}}[k] - e^{r ({T}_O - t)}}_{\tiny \mbox{expected~excess~return~of~calls}} ~ = ~ e^{r ({T}_O - t)} ~ \, \frac{\mathbb{E}_{t}^{\mathbb{P}}( \max(G_{{T}_O} - k, 0) ) \, - \, \mathbb{E}_{t}^{\mathbb{Q}}( \max(G_{{T}_O} - k,0) )}
{ \mathbb{E}_{t}^{\mathbb{Q}}(
\max(G_{{T}_O} - k, 0)  )}. ~ \mbox{ \, \, \, } \label{eq:ExpectedExcessHoldingReturn1GeneralDynamics}
\end{eqnarray} } % end color blue
Thus, the call option risk premium inherits the sign of (using Tanaka's formula)
\begin{eqnarray}
\mathbb{E}_{t}^{\mathbb{P}}( \max( G_{T_O} - k, 0 ) ) -
\mathbb{E}_{t}^{\mathbb{Q}}( \max( G_{T_O} - k, 0 ) )
&=& \mathbb{E}_{t}^{\mathbb{P}}( \int_{t+}^{T_O} \mathbbm{1}_{\{G_{\ell  -} > k\}} dG_{\ell})
%~ ~ ~ \mbox{ \, \, (use Tanaka's formula) \, }
~~ \nonumber \\
&+& ~ \mathbb{E}_{t}^{\mathbb{P}}( \mathbb{L}^{T_O}_t[k]) ~ -
~ \mathbb{E}_{t}^{\mathbb{Q}}( \mathbb{L}^{T_O}_t[k])  \nonumber \\
&+&
\underbrace{\mathbb{E}_{t}^{\mathbb{P}}( a_t^{T_O}[k] + b_t^{T_O}[k] ) - \mathbb{E}_{t}^{\mathbb{Q}}( a_t^{T_O}[k]
+b_t^{T_O}[k]).}_{\text{ \tiny \, (risk premium for jumps crossing the strike (already signed in
Section~\ref{app:jumps_across})) }} ~~~ \mbox{ \, \, \, \, \, \, }. \nonumber
\end{eqnarray}
%\begin{eqnarray}
%& & \mathbb{E}_{t}^{\mathbb{P}}( \max( G_{T_O} - k, 0 ) ) ~ - ~
%\mathbb{E}_{t}^{\mathbb{Q}}( \max( G_{T_O} - k, 0 ) ) ~~ ~ ~ ~ \mbox{ \, \, (use Tanaka's formula) \, } ~~ \nonumber \\
%& & ~ ~ = ~ \mathbb{E}_{t}^{\mathbb{P}}( \int_{t+}^{T_O} \mathbbm{1}_{\{G_{\ell  -} > k\}} dG_{\ell})
%~ + ~ \overbrace{ ~ \mathbb{E}_{t}^{\mathbb{P}}( \mathbb{L}^{T_O}_t[k]) ~ -
%~ \mathbb{E}_{t}^{\mathbb{Q}}( \mathbb{L}^{T_O}_t[k])  ~ }^{\text{next use (\ref{eq:covqgsbstatement}),
%(\ref{eq:ConditCovarSpanning2Linesa}), and
%(\ref{eq:Interim1}) }} \nonumber \\
%& & ~~~ ~ ~ \mbox{ \, \, \, \, } ~ + ~
%\underbrace{\mathbb{E}_{t}^{\mathbb{P}}( a_t^{T_O}[k] + b_t^{T_O}[k] ) - \mathbb{E}_{t}^{\mathbb{Q}}( a_t^{T_O}[k]
%+b_t^{T_O}[k]).}_{\text{ \tiny \, (risk premium for jumps crossing the strike (already signed in
%Section~\ref{app:jumps_across})) }} ~~~ \mbox{ \, \, \, \, \, \, }. \nonumber
%\end{eqnarray}
To further reduce the problem to what we have already derived based on conditioning on $\mathcal{I}_{T_O}$, note the following simplification
steps:
 \begin{eqnarray}
& & \mathbb{E}_{t}^{\mathbb{P}}( \max( G_{T_O} - k, 0 ) ) ~ - ~
\mathbb{E}_{t}^{\mathbb{Q}}( \max( G_{T_O} - k, 0 ) )  \nonumber \\
& & ~ ~ = ~ \mathbb{E}_{t}^{\mathbb{P}}( \int_{t+}^{T_O} \mathbbm{1}_{\{G_{\ell  -} > k\}} dG_{\ell}) ~ \nonumber \\
&&~~~ ~ ~ \mbox{ \, \, \, \, }
+\overbrace{\mathbb{E}_{t}^{\mathbb{Q}}( \mathcal{R}_{T_O}^{\tiny \mbox{span~diffusive}} \, \, \mathbb{L}_t^{{T}_O}[k] ) ~ - ~ \mathbb{E}_{t}^{\mathbb{Q}}( \mathbb{L}_t^{{T}_O}[k] )}^{\text{ from
(\ref{eq:Interim1})}}  ~ ~
\nonumber \\
%\label{eq:ReexpressCallRiskPremiumMinus3} \\
& & ~~~ ~ ~ \mbox{ \, \, \, \, } ~ + ~ \overbrace{\mathbb{E}_{t}^{\mathbb{Q}}( \mathcal{R}_{T_O}^{\tiny \mbox{span~diffusive}} \times
\mathrm{cov}_t^{\mathbb{Q}}(
\mathcal{R}_{T_O}^{\tiny \mbox{unspan~diffusive}}
\times \mathcal{R}_{T_O}^{\tiny \mbox{unspan~jump}},
\, \mathbb{L}_t^{{T}_O}[k] {\Big |} \, \mathcal{I}_{T_O} ) )}^{\text{ from
(\ref{eq:ConditCovarSpanning2Linesa}) }} ~ \mbox{ \, \, \, \, } ~ \mbox{ \, \, } \nonumber \\
& & ~~~ ~ ~ \mbox{ \, \, \, \, } ~ + ~
\mathbb{E}_{t}^{\mathbb{P}}( a_t^{T_O}[k]  +  b_t^{T_O}[k] )
- \mathbb{E}_{t}^{\mathbb{Q}}( a_t^{T_O}[k] + b_t^{T_O}[k]).
~~~ \mbox{ \, \, \, \, \, \, } \label{eq:ReexpressCallRiskPremiumMinus2}
\end{eqnarray}

Rearranging for clarity and to see the term that is left to be signed, we then have
 \begin{eqnarray}
& & \mathbb{E}_{t}^{\mathbb{P}}( \max( G_{T_O} - k, 0 ) ) ~ - ~
\mathbb{E}_{t}^{\mathbb{Q}}( \max( G_{T_O} - k, 0 ) )  \nonumber \\
& & ~ = ~
\overbrace{\mathbb{E}_{t}^{\mathbb{P}}( a_t^{T_O}[k] ) + b_t^{T_O}[k] )
- \mathbb{E}_{t}^{\mathbb{Q}}( a_t^{T_O}[k] + b_t^{T_O}[k])}^{\text{ \tiny \, (already signed in
Section~\ref{app:jumps_across})}} ~~~ \mbox{ \, \, \, \, \, \, } \nonumber \\
& & ~ ~ ~ + ~ \overbrace{\mathbb{E}_{t}^{\mathbb{Q}}( \mathcal{R}_{T_O}^{\tiny \mbox{span~diffusive}} \times
\mathrm{cov}_t^{\mathbb{Q}}(
\mathcal{R}_{T_O}^{\tiny \mbox{unspan~diffusive}}
\times \mathcal{R}_{T_O}^{\tiny \mbox{unspan~jump}},
\, \mathbb{L}_t^{{T}_O}[k] {\Big |} \, \mathcal{I}_{T_O} ) )}^{\text{ \tiny \, (already signed by equations (\ref{eq:ConditCovarSpanning2Linesa}) and
(\ref{eq:JumpRelatedLT2})) }} ~ \mbox{ \, \, \, \, }
\nonumber \\
& & ~ ~ ~ + ~ \mathbb{E}_{t}^{\mathbb{P}}( \int_{t+}^{T_O} \mathbbm{1}_{\{G_{\ell  -} > k\}} dG_{\ell} ) ~ - ~ \overbrace{\mathbb{E}_{t}^{\mathbb{P}}( \frac{1}{\mathcal{R}_{T_O}^{\tiny \mbox{unspan~diffusive}} \, \mathcal{R}_{T_O}^{\tiny \mbox{unspan~jump}}} \, \, \int_{t+}^{T_O} \mathbbm{1}_{\{G_{\ell  -} > k\}} dG_{\ell} )}^{= ~ \mathbb{E}_{t}^{\mathbb{Q}}(  \mathcal{R}_{T_O}^{\tiny \mbox{span~diffusive}} \, \int_{t+}^{T_O} \mathbbm{1}_{\{G_{\ell  -} > k\}} dG_{\ell} )} ~ ~  \mbox{ \, } ~
%\label{eq:ReexpressCallRiskPremiumMinus1}
\nonumber
\\
& & ~ ~ ~ + ~
\underbrace{ ~ \mathbb{E}_{t}^{\mathbb{Q}}( \mathcal{R}_{T_O}^{\tiny \mbox{span~diffusive}} \, \max( G_{T_O} - k, 0 ) ) ~ - ~
\mathbb{E}_{t}^{\mathbb{Q}}( \max( G_{T_O} - k, 0 ) ).}_{\text{ \tiny
$= \, \mathbb{E}_{t}^{\mathbb{Q}}(  \mathcal{R}_{T_O}^{\tiny \mbox{span~diffusive}} \, \int_{t+}^{T_O} \mathbbm{1}_{\{G_{\ell  -} > k\}} dG_{\ell} ) ~ + ~ \mathbb{E}_{t}^{\mathbb{Q}}( \mathcal{R}_{T_O}^{\tiny \mbox{span~diffusive}} \, \, \mathbb{L}_t^{{T}_O}[k] ) ~ - ~ \mathbb{E}_{t}^{\mathbb{Q}}( \mathbb{L}_t^{{T}_O}[k] )$ }}
\label{eq:ReexpressCallRiskPremium}
\end{eqnarray}
The last two terms in (\ref{eq:ReexpressCallRiskPremium}) are adding and subtracting the same quantity. This
%quantity is
a consequence of using Tanaka's formula to
%express
reverse engineer
the local time $\mathbb{L}_t^{{T}_O}[k]$ in terms of the call payoff.
Further recognize that
$\mathbb{E}_{t}^{\mathbb{Q}}(  \mathcal{R}_{T_O}^{\tiny \mbox{span~diffusive}} \, \int_{t+}^{T_O} \mathbbm{1}_{\{G_{\ell  -} > k\}} dG_{\ell} )$
and $\mathbb{E}_{t}^{\mathbb{P}}( \frac{1}{\mathcal{R}_{T_O}^{\tiny \mbox{unspan~diffusive}} \, \mathcal{R}_{T_O}^{\tiny \mbox{unspan~jump}}} \, \, \int_{t+}^{T_O} \mathbbm{1}_{\{G_{\ell  -} > k\}} dG_{\ell} )$ are identical (by Girsanov's Theorem and the definitions in equations (\ref{eq:RecipPK5TermsDPSLT1}) and (\ref{ah.1})).




%{\color{blue}Above, in going
%from the line
%with the equation number
%(\ref{eq:ReexpressCallRiskPremiumMinus3}), we have substituted for
%$\mathbb{L}_t^{{T}_O}[k]$ from Tanaka's formula.
%In the line with the equation number
%(\ref{eq:ReexpressCallRiskPremiumMinus1}),
%we have subtracted \\
%$\mathbb{E}_{t}^{\mathbb{Q}}(  \mathcal{R}_{T_O}^{\tiny \mbox{span~diffusive}} \, \int_{t+}^{T_O} \mathbbm{1}_{\{G_{\ell  -} > k\}} dG_{\ell} )$ while in the line with the equation number
%(\ref{eq:ReexpressCallRiskPremium}), we have added the same quantity.
%The line with the equation number
%(\ref{eq:ReexpressCallRiskPremiumMinus1})

\vspace{1mm}

The following feature is
evident from equation (\ref{eq:ReexpressCallRiskPremium}):
\begin{itemize}

\item In the special case that \emph{there are no unspanned risks in the pricing kernel}, we would have
(i) $\mathcal{R}_{T_O}^{\tiny \mbox{unspan~diffusive}} \equiv 1$ and (ii) $\mathcal{R}_{T_O}^{\tiny \mbox{unspan~jump}} \equiv 1$
(state-by-state). Hence,
\emph{the call option risk premium would inherit the same sign as that of the final line}, specifically
of $\mathbb{E}_{t}^{\mathbb{Q}}( \mathcal{R}_{T_O}^{\tiny \mbox{span~diffusive}} \max( G_{T_O} - k, 0 ) ) -
\mathbb{E}_{t}^{\mathbb{Q}}( \max( G_{T_O} - k, 0 ) )$, since the first three lines of equation (\ref{eq:ReexpressCallRiskPremium})
would vanish.

\end{itemize}
\vspace{1mm}
We will now show that the sign of the final line of equation (\ref{eq:ReexpressCallRiskPremium}) is
positive regardless of whether or not there are unspanned risks in the
pricing kernel. \vspace{1mm}

%%%%%%%%%%%%%%%%%%%%%%%% %%%%%%%%%%%%%%%%%%%%%%%%
\noindent \textbf{Result.} The following result is true:
\begin{eqnarray}
\mathbb{E}_{t}^{\mathbb{Q}}( \mathcal{R}_{T_O}^{\tiny \mbox{span~diffusive}} \, \max( G_{T_O} - k, 0 ) ) ~ - ~
\mathbb{E}_{t}^{\mathbb{Q}}( \max( G_{T_O} - k, 0 ) ) ~ > ~ 0.
\label{eq:InequalityInBlackScholesTypeFormula3}
\end{eqnarray}
%{\color{red} [[[[ It does not logically fit here.
%It needs the analysis going back to the beginning. ]]]] [[[[[[
%The risk premium of straddle is zero. ]]]]]}

\noindent \textbf{Proof.} The proof of this result is tedious and presented
next in Section~\ref{appsec:cccv1}.
$\blacksquare$ \vspace{-4mm}

%\subsection{{\color{blue}Proof of equation
%(\ref{eq:InequalityInBlackScholesTypeFormula3})}}

\subsection{Proof that equation (\ref{eq:InequalityInBlackScholesTypeFormula3}) holds}
%$\mathbb{E}_{t}^{\mathbb{Q}}( \mathcal{R}_{T_O}^{\tiny \mbox{span~diffusive}} \max( G_{T_O} - k, 0 ) ) -
%\mathbb{E}_{t}^{\mathbb{Q}}( \max( G_{T_O} - k, 0 ) )> 0$}
\label{appsec:cccv1}

%{\color{red} [[[[ The task is to prove equation (\ref{eq:InequalityInBlackScholesTypeFormula3}), specifically
%that $\mathbb{E}_{t}^{\mathbb{Q}}( \mathcal{R}_{T_O}^{\tiny \mbox{span~diffusive}} \, \max( G_{T_O} - k, 0 ) ) -
%\mathbb{E}_{t}^{\mathbb{Q}}( \max( G_{T_O} - k, 0 ) )$ is positive. ]]]]}

By conditioning
on the jump component of the equity futures and its variance,
and exploiting independence from the diffusive
components,
one can see that, for the purpose of determining
the \emph{sign} of $\mathbb{E}_{t}^{\mathbb{Q}}( \mathcal{R}_{T_O}^{\tiny \mbox{span~diffusive}}  \max( G_{T_O} - k, 0 ) ) -
\mathbb{E}_{t}^{\mathbb{Q}}( \max( G_{T_O} - k, 0 ) )$, one can reduce the problem
to computing this quantity when there are \emph{no jumps}.

In other words, with no loss of generality, we are justified in working
with the following $\mathbb{Q}$ dynamics:
\begin{eqnarray}
\frac{d G_{t}}{G_{t}} & = & \sqrt{\mathrm{v}_t} \,d z^{\mathbb{Q}}_t ~~ \mbox{ \, \, \, \, \, \, \, \quad \quad } ~~ ~ ~\mathrm{and}~ ~ ~~~~ \label{eqa:spp1a}
\\
d \mathrm{v}_t & = &  ( \phi_{\tiny \mbox{vol}}^{\mathbb{Q}} - \kappa_{\tiny \mbox{vol}}^{\mathbb{Q}} \,\mathrm{v}_t )\, dt ~ + ~
\sigma_{\tiny \mbox{vol}} \, \sqrt{\mathrm{v}_t} \,\rho_{\tiny \mbox{vol}} \,d z_t^{\mathbb{Q}}
~+~\sigma_{\tiny \mbox{vol}} \sqrt{\mathrm{v}_t} \, \sqrt{1-\rho^2_{\tiny \mbox{vol}}  }
\, du_t^{\mathbb{Q}}.
~ ~ \mbox{ \, \, } ~ \mbox{ \, \, \, \, \, \, }
\label{eqa:spp1}
\end{eqnarray}
\noindent \textbf{Step 1.} For the purpose of the proof, we recast the Brownian motions by introducing
independent Brownian
motions $w_t^{(\mathbb{Q},1)}$ and $w_t^{(\mathbb{Q},2)}$, under
$\mathbb{Q}$, as follows:
\begin{eqnarray}
w_t^{(\mathbb{Q},1)} \, = \, \sqrt{1-\rho^2_{\tiny \mbox{vol}}} \, z_t^{\mathbb{Q}} \, - \, \rho_{\tiny \mbox{vol}} \,  u_t^{\mathbb{Q}} ~ \mbox{ \, \, \quad and \, \, \, \, } ~
w_t^{(\mathbb{Q},2)} \, = \, \rho_{\tiny \mbox{vol}} \, z_t^{\mathbb{Q}} \, + \,
\sqrt{1-\rho^2_{\tiny \mbox{vol}}} \, u_t^{\mathbb{Q}}. ~ \mbox{ \, \, \, \, \, \, \, \, } \label{eq:CholeskyTypeDef}
\end{eqnarray}
Hence, we have
\begin{equation}
z_t^{\mathbb{Q}} \, = \, \sqrt{1-\rho^2_{\tiny \mbox{vol}}} \, w_t^{(\mathbb{Q},1)} \, + \, \rho_{\tiny \mbox{vol}} \,  w_t^{(\mathbb{Q},2)}.
\end{equation}

\noindent \textbf{Step 2.} The variance process $(\mathrm{v}_s)$
is driven only by $(w_s^{(\mathbb{Q},2)})$.
%In equation (\ref{eq:CholeskyTypeDef}) and henceforth, we drop the subscript ${\mathrm{vol}}$ and write
%$\rho_{\tiny \mbox{vol}} \, \equiv \, \rho$.
Next, we proceed as follows:
\begin{eqnarray}
G_{T_O} & = & \overbrace{G_t}^{= \, 1} \, \, e^{ \int_{t}^{{T}_O} \{ -\frac{1}{2} \mathrm{v}_s\, ds + \sqrt{\mathrm{v}_s} \, d z_s^{\mathbb{Q}} \}} ~ \, \nonumber \\
&=&  e^{ \int_{t}^{{T}_O} \{ -\frac{1}{2} \mathrm{v}_s \, ds + \sqrt{\mathrm{v}_s} \,
[\sqrt{1-\rho^2_{\tiny \mbox{vol}}} \, d w_s^{(\mathbb{Q},1)} \, + \, \rho_{\tiny \mbox{vol}} \, d w_s^{(\mathbb{Q},2)}] \}}
~ \mbox{ \, \, }\\
& = & {G}_{T_O}^{\perp} \, e^{ \int_{t}^{{T}_O} \{ -\frac{1}{2} (\sqrt{\mathrm{v}_s})^2 \, (1-\rho^2_{\tiny \mbox{vol}}) \, ds
+ (\sqrt{\mathrm{v}_s}) \, \sqrt{1-\rho^2_{\tiny \mbox{vol}}} \, d w_s^{(\mathbb{Q},1)} \}}.
\end{eqnarray}
Additionally,
\begin{eqnarray}
\mathcal{R}_{T_O}^{\tiny \mbox{span~diffusive}} & = &
e^{ \int_{t}^{{T}_O} \{ -\frac{1}{2} (\eta[s,\mathrm{v}_s])^2 \, ds - \eta[s,\mathrm{v}_s] \, d z_s^{\mathbb{Q}} \}} \mbox{ \, } \mbox{ \, \, \, }
\nonumber \\
& = &
\mathcal{R}_{T_O}^{\perp} \, e^{ \int_{t}^{{T}_O} \{ -\frac{1}{2} (\eta[s,\mathrm{v}_s])^2 (1-\rho^2_{\tiny \mbox{vol}}) \, ds + \eta[s,\mathrm{v}_s] \sqrt{1-\rho^2_{\tiny \mbox{vol}}}\, d w_s^{(\mathbb{Q},1)} \}}. ~
~ \mbox{ \, \, } ~~
\end{eqnarray}
Finally,
\begin{eqnarray}
\mathcal{R}_{T_O}^{\tiny \mbox{span~diffusive}} \, G_{T_O}
%& = & e^{ \int_{t}^{{T}_O} \{ -\frac{1}{2} (\eta[s,\mathrm{v}_s])^2 ds - \eta[s,\mathrm{v}_s] d z_s^{\mathbb{Q}} \}} \, G_t \, e^{ \int_{t}^{{T}_O} \{ -\frac{1}{2} \mathrm{v}_s ds + \sqrt{\mathrm{v}_s} d z_s^{\mathbb{Q}} \}} ~ \mbox{ \, \, \, \, } \\
& = & e^{ \int_{t}^{{T}_O} \{ -\frac{1}{2} (\eta[s,\mathrm{v}_s])^2 ds - \eta[s,\mathrm{v}_s] (\sqrt{1-\rho^2_{\tiny \mbox{vol}}} d w_s^{(\mathbb{Q},1)} \, + \, \rho_{\tiny \mbox{vol}} d w_s^{(\mathbb{Q},2)}) \}} ~ \mbox{ \, } \times \mbox{ \, \, \, } \nonumber \\
& & ~ \mbox{ \, \, \, \, } ~ ~ \overbrace{G_t}^{= \, 1} \, \, e^{ \int_{t}^{{T}_O} \{ -\frac{1}{2} \mathrm{v}_s ds + \sqrt{\mathrm{v}_s} (\sqrt{1-\rho^2_{\tiny \mbox{vol}}} d w_s^{(\mathbb{Q},1)} \, + \, \rho_{\tiny \mbox{vol}} d w_s^{(\mathbb{Q},2)}) \}} ~ \mbox{ \, } \mbox{ \, \, \, } \nonumber \\
& = & \mathcal{R}_{T_O}^{\perp}  \, {G}_{T_O}^{\perp} \,
{\cal V}_{T_{O}}^{\bullet}
 \, e^{ \int_{t}^{{T}_O} \{ -\frac{1}{2} (\sqrt{\mathrm{v}_s}-\eta[s,\mathrm{v}_s])^2 (1-\rho^2_{\tiny \mbox{vol}}) \,ds + (\sqrt{\mathrm{v}_s}-\eta[s,\mathrm{v}_s]) \sqrt{1-\rho^2_{\tiny \mbox{vol}}} \, d w_s^{(\mathbb{Q},1)} \}}. \nonumber
~ \mbox{ \, \, \, } ~
%\mbox{ \, \, \, }
\end{eqnarray}
We have defined, for compactness of presentation, the following quantities:
\begin{eqnarray}
\mathcal{R}_{T_O}^{\perp} & \equiv & e^{ \int_{t}^{{T}_O} \{ -\frac{1}{2} (-\eta[s,\mathrm{v}_s])^2 \, \rho^2_{\tiny \mbox{vol}} \, ds + (-\eta[s,\mathrm{v}_s]) \, \rho_{\tiny \mbox{vol}} \, d w_s^{(\mathbb{Q},2)} \}},
~ ~ ~ \mbox{ \, \, \, \, \, \, \, \, } ~ ~ ~ ~
\label{cvg.21} \\
{G}_{T_O}^{\perp} & \equiv & e^{ \int_{t}^{{T}_O} \{ -\frac{1}{2} (\sqrt{\mathrm{v}_s})^2 \rho^2_{\tiny \mbox{vol}} ds + (\sqrt{\mathrm{v}_s}) \, \rho_{\tiny \mbox{vol}}\, d w_s^{(\mathbb{Q},2)} \}}, ~~ \mbox{ \, \, \, \, \, \quad \quad \quad \quad } ~ \mbox{ \, and } ~ ~ ~
\label{cvg.22}\\
{\cal V}_{T_{O}} & \equiv & {\cal V}_{T_{O}}^{\bullet} \, \, {\cal V}_{T_{O}}^{\perp}, ~~ ~~~ \mbox{ \, \, \, \, \quad \quad \quad \quad } \mbox{ \, \, \, \, \quad \quad \quad \quad \quad \quad \, \, \, where } ~~~
\\
{\cal V}_{T_{O}}^{\bullet} & \equiv & e^{ \int_{t}^{{T}_O}
\{ (1-\rho^2_{\tiny \mbox{vol}}) \sqrt{\mathrm{v}_s} ( - \eta[s,\mathrm{v}_s] )\, ds \}}
~ \mbox{ \, \, \, \, \, and \, \, \, \,  } ~ {\cal V}_{T_{O}}^{\perp} ~ \, \equiv
~ \, e^{ \int_{t}^{{T}_O}
\{ \rho^2_{\tiny \mbox{vol}} \sqrt{\mathrm{v}_s} ( - \eta[s,\mathrm{v}_s] ) \, ds \}}. ~ \mbox{ \, \, \, \, \, \, } ~
~ \label{eq:DefinitionOfA-TO}
\end{eqnarray}
\noindent \textbf{Step 3.} With these substitutions,
we have decomposed
$G_{T_O}$,
$\mathcal{R}_{T_O}^{\tiny \mbox{span~diffusive}}$, and
$\mathcal{R}_{T_O}^{\tiny \mbox{span~diffusive}} \, G_{T_O}$
into the product of terms whose increments are (instantaneously)
perfectly correlated with $(w_s^{(\mathbb{Q},1)})$ and terms
(i.e., ${G}_{T_O}^{\perp}$ and $\mathcal{R}_{T_O}^{\perp}$),
whose
increments are independent of $(w_s^{(\mathbb{Q},1)})$,
as well as a term ${\cal V}_{T_{O}}^{\bullet}$ which is
informative about the sign of the equity premium. Furthermore,
\begin{equation}
\mbox{the variance process $(\mathrm{v}_s)$ is independent of the Brownian motion} ~ (w_s^{(\mathbb{Q},1)}). ~ ~ ~ ~ \mbox{ \, \, \, \, \, \, }
\end{equation}

Hence, the distribution of
(i) $\log( \frac{\mathcal{R}_{T_O}^{\tiny \mbox{span~diffusive}} \, G_{T_O}}{ \mathcal{R}_{T_O}^{\perp} \,{G}_{T_O}^{\perp} \, {\cal V}_{T_{O}}^{\bullet}
} )$,
(ii) $\log( \frac{G_{T_O}}{{G}_{T_O}^{\perp}} )$,
and (iii) $\log( \frac{\mathcal{R}_{T_O}^{\tiny \mbox{span~diffusive}}}{\mathcal{R}_{T_O}^{\perp}} )$, conditional on
the path of variance $\lbrace \mathrm{v}_s, t \leq s \leq T_{O} \rbrace$
and on $\mathcal{F}_t$, is jointly normal
with
\begin{eqnarray}
\log( \frac{\mathcal{R}_{T_O}^{\tiny \mbox{span~diffusive}} \, G_{T_O}}{\mathcal{R}_{T_O}^{\perp} \,{G}_{T_O}^{\perp} \, {\cal V}_{T_{O}}^{\bullet} } ) {\Big |} \lbrace \mathrm{v}_s, t \leq s \leq T_{O} \rbrace, \mathcal{F}_t ~ & \sim & {\cal N}( - \frac{1}{2}
\mathbbm{d}_{t,T_O}^2, \mathbbm{d}_{t,T_O}^2 ), ~ ~ \mbox{ \, \, \, \quad } ~ ~ ~ \mbox{ \, \, } \nonumber \\
\log( \frac{G_{T_O}}{{G}_{T_O}^{\perp}} ) {\Big |} \lbrace \mathrm{v}_s, t \leq s \leq T_{O} \rbrace, \mathcal{F}_t ~ & \sim & {\cal N}( - \frac{1}{2} \mathbbm{v}_{t,T_O}^2, \mathbbm{v}_{t,T_O}^2 ), ~ \mbox{ \, \, \, } ~ \mbox{ \, \, \, and \, \, \, }
\nonumber \\
\log( \frac{\mathcal{R}_{T_O}^{\tiny \mbox{span~diffusive}}}{\mathcal{R}_{T_O}^{\perp}} )
{\Big |} \lbrace \mathrm{v}_s, t \leq s \leq T_{O} \rbrace, \mathcal{F}_t ~ & \sim & {\cal N}( - \frac{1}{2} \mathbbm{e}_{t,T_O}^2, \mathbbm{e}_{t,T_O}^2 ), ~ ~ \mbox{ \, \, \, \quad } ~ ~ ~ \mbox{ \, \, }
\end{eqnarray}
where
\begin{eqnarray}
\mathbbm{d}_{t,T_O}^2 &\equiv&  \int_{t}^{{T}_O} (\sqrt{\mathrm{v}_s}-\eta[s,\mathrm{v}_s])^2 (1-\rho^2_{\tiny \mbox{vol}}) ds, \\
\mathbbm{v}_{t,T_O}^2 &\equiv& \int_{t}^{{T}_O} \mathrm{v}_s (1-\rho^2_{\tiny \mbox{vol}}) ds, ~ ~ ~ \mbox{ \, \, \, \, \, \, \, \, \, \, and \, \, \, \, \, \, \, \, \, \, } ~ ~ ~ \\
\mathbbm{e}_{t,T_O}^2  &\equiv& \int_{t}^{{T}_O} (\eta[s,\mathrm{v}_s])^2 (1-\rho^2_{\tiny \mbox{vol}}) \,ds.
\end{eqnarray}

\noindent \textbf{Step 4.} Using a technique
that
is a variant of \citet*{HullWhite:87},
%and \citet*[Section 4]{Romano_Touzi:MF1997},
we condition on the path of variance
$\lbrace \mathrm{v}_s, t \leq s \leq T_{O} \rbrace$ and on $\mathcal{F}_t$, to
derive as follows:
\small
\begin{eqnarray}
& & \mathbb{E}_{t}^{\mathbb{Q}}( \mathcal{R}_{T_O}^{\tiny \mbox{span~diffusive}} \,
\max( G_{T_O} - k, 0 ) )
\nonumber \\
%\label{eq:BlackScholesTypeFormula1} \\
& & ~ ~ \mbox{ \, \quad } = \,
\mathbb{E}_{t}^{\mathbb{Q}}( \mathcal{R}_{T_O}^{\perp} \, \{
{G}_{T_O}^{\perp} \,{\cal V}_{T_{O}}^{\bullet}  {\cal N}\big( \frac{\log(\frac{
{G}_{T_O}^{\perp} \, {\cal V}_{T_{O}}^{\bullet}   }{k})
+ \frac{1}{2} \mathbbm{v}_{t,T_O}^2 }{ \mathbbm{v}_{t,T_O}} \big)
 \, - \, k {\cal N}\big( \frac{\log(\frac{ {G}_{T_O}^{\perp}  \, {\cal V}_{T_{O}}^{\bullet}  }{k}) - \frac{1}{2} \mathbbm{v}_{t,T_O}^2 }{\mathbbm{v}_{t,T_O}} \big) \} \, ), ~ \mbox{ \, \, \, \quad \quad } ~
\end{eqnarray} \normalsize
where ${\cal N}(.)$ is the standard normal cumulative distribution function.
Similarly,
\small
\begin{eqnarray}
\mathbb{E}_{t}^{\mathbb{Q}}( \max( G_{T_O} - k, 0 ) ) &= &
 \mathbb{E}_{t}^{\mathbb{Q}}( \underbrace{ {G}_{T_O}^{\perp} {\cal N}\big( \frac{\log(\frac{ {G}_{T_O}^{\perp}}{k}) + \frac{1}{2} \mathbbm{v}_{t,T_O}^2
 }{ \mathbbm{v}_{t,T_O}} \big) \, - \, k {\cal N}\big( \frac{\log(\frac{ {G}_{T_O}^{\perp}}{k}) - \frac{1}{2} \mathbbm{v}_{t,T_O}^2 }{ \mathbbm{v}_{t,T_O}} \big)}_{\equiv ~ \mathrm{call}_t^{\tiny \mbox{BS}}[
{G}_{T_O}^{\perp}, k ]} ). \mbox{ \, \, \, \, \, \, \, \, } ~ \mbox{ \, \, \quad }  \label{eq:BlackScholesTypeFormula2}
\end{eqnarray} \normalsize
\noindent \textbf{Step 5.} We ask the following question:
\begin{eqnarray}
\mathrm{When~is}~\mathbb{E}_{t}^{\mathbb{Q}}( \mathcal{R}_{T_O}^{\tiny \mbox{span~diffusive}} \, \max( G_{T_O} - k, 0 ) ) ~ - ~
\mathbb{E}_{t}^{\mathbb{Q}}( \max( G_{T_O} - k, 0 ) ) ~ > ~ 0  \mbox{?} ~~~ \mbox{ \, } \mbox{ \, \, \, }
\end{eqnarray}
A few observations are in order. First, the equity premium is positive when $\alpha_{\tiny \mbox{vol}} > 0$ and $\lambda_{\tiny \mbox{vol}} > 0$. This
implies that $\eta[s,\mathrm{v}_s]= - \frac{1}{ \sqrt{\mathrm{v}_s}}(
\alpha_{\tiny \mbox{vol}} +\lambda_{\tiny \mbox{vol}} \, \mathrm{v}_s) < 0$.

Hence, by equation (\ref{eq:DefinitionOfA-TO}) and $\eta[s,\mathrm{v}_s]<0$, it holds that
\begin{equation}
{\cal V}_{T_{O}} \, > \, 1 ~~ \mbox{ \, \, \, } ~ \mathrm{and} ~~ \mbox{ \, \, \, } ~ {\cal V}_{T_{O}}^{\bullet} \, > \, 1. ~~~ \mbox{ \, \, \, \, \, \, }
\end{equation}
Since call option prices are monotonically increasing
in the price of the underlying and since
${G}_{T_O}^{\perp} \mathcal{V}_{T_{O}}^{\bullet} \, > \, {G}_{T_O}^{\perp}$, we have
\begin{equation}
\mathrm{call}_t^{\tiny \mbox{BS}}[ {G}_{T_O}^{\perp} \mathcal{V}_{T_{O}}^{\bullet} , k]
\, > \,
\mathrm{call}_t^{\tiny \mbox{BS}}[ {G}_{T_O}^{\perp}, k ]. ~ ~ ~ \label{eq:FirstIneq1}
\end{equation}
We note that the covariance between $\mathcal{R}_{T_O}^{\perp}$
and
${G}_{T_O}^{\perp}$ under $\mathbb{Q}$ is positive (by
%(\ref{eq:DefinitionOfA-TO})
(\ref{cvg.21})--(\ref{cvg.22})
and since $(-\eta[s,\mathrm{v}_s]) \sqrt{\mathrm{v}_s} >0$).
Furthermore, call options have a nonnegative delta. The upshot is
that $\mathcal{R}_{T_O}^{\perp}$ and $\mathrm{call}_t^{\tiny \mbox{BS}}[ {G}_{T_O}^{\perp}, k ]$ have a positive
covariance under $\mathbb{Q}$. With
$\mathbb{E}_{t}^{\mathbb{Q}}( \mathcal{R}_{T_O}^{\perp})=1$, it holds that
\begin{equation}
\mathbb{E}_{t}^{\mathbb{Q}}( \mathcal{R}_{T_O}^{\perp} \, \{
\mathrm{call}_t^{\tiny \mbox{BS}}[ {G}_{T_O}^{\perp}, k ] \} )
\, - \, \mathbb{E}_{t}^{\mathbb{Q}}( \mathrm{call}_t^{\tiny \mbox{BS}}[ {G}_{T_O}^{\perp}, k ] ) ~ = ~
\mathrm{cov}_t^{\mathbb{Q}}( \mathcal{R}_{T_O}^{\perp},  \mathrm{call}_t^{\tiny \mbox{BS}}[ {G}_{T_O}^{\perp}, k ]  ) \, > \, 0.
~ \mbox{ \, } ~ \label{eq:SecondIneq2}
\end{equation}
Therefore,
combining (\ref{eq:FirstIneq1})
and
(\ref{eq:SecondIneq2}), we have $\mathbb{E}_{t}^{\mathbb{Q}}( \mathcal{R}_{T_O}^{\perp} \, \{
\mathrm{call}_t^{\tiny \mbox{BS}}[ {G}_{T_O}^{\perp} \mathcal{V}_{T_{O}}^{\bullet}, k ] \} )
 >  \mathbb{E}_{t}^{\mathbb{Q}}( \mathrm{call}_t^{\tiny \mbox{BS}}[ {G}_{T_O}^{\perp}, k ] )$.
The consequence is that
\begin{eqnarray}
\mathbb{E}_{t}^{\mathbb{Q}}( \mathcal{R}_{T_O}^{\tiny \mbox{span~diffusive}} \, \max( G_{T_O} - k, 0 ) ) ~ - ~
\mathbb{E}_{t}^{\mathbb{Q}}( \max( G_{T_O} - k, 0 ) ) ~~ > ~ 0.
~ \mbox{ \, } \mbox{ \, \, \, }
\label{app:caaal1}
\end{eqnarray}
We have the proof. $\blacksquare$ \vspace{-4mm}



\subsection{No unspanned risks in the pricing kernel imply zero straddle risk premium}
\label{app:jumps_acrossPutsStraddles}

The statement to prove is the following: When there are no unspanned risks in the
pricing kernel, the straddle risk premium (corresponding to $k=1$) is zero.

%{\color{red}[[[ No. This is NOT the statement to prove. ]]]
%\begin{eqnarray}
%\mathbb{E}_{t}^{\mathbb{Q}}( \mathcal{R}_{T_O}^{\tiny \mbox{span~diffusive}}  \{
%\max( k - 1, 0 )
%+ \max( 1  - k, 0 ) \} ) -
%\mathbb{E}_{t}^{\mathbb{Q}}( \{
%\max( k - 1, 0 )
%+ \max( 1  - k, 0 ) \} ) =  0. ~~~
%%~ \mbox{ \, } \mbox{ \, \, \, }
%\end{eqnarray}
%]]]]]]}

For the proof, we
first state the following companion result corresponding to equation (\ref{app:caaal1}) for OTM puts (steps are similar and omitted):

\noindent \textbf{Result.} The following result is true:
\begin{eqnarray}
\mathbb{E}_{t}^{\mathbb{Q}}( \mathcal{R}_{T_O}^{\tiny \mbox{span~diffusive}} \, \max( k - G_{T_O}, 0 ) ) ~ - ~
\mathbb{E}_{t}^{\mathbb{Q}}( \max( k - G_{T_O}, 0 ) ) ~ < ~ 0.
\label{eq:InequalityInBlackScholesTypeFormula3Put}
\end{eqnarray}
Move next to our object of interest, specifically the straddle risk premium.

Recall from Appendix~\ref{appsec:jumppps} (part III) that
\begin{eqnarray*}
\mathbb{A}_t^{T_O}[1] ~ \equiv ~  \sum_{t < \ell \leq T_O} \underbrace{\{
 \mathbbm{1}_{\{G_{\ell \, -} < 1\}} \, \max(  G_{\ell}- 1, 0 ) \, + \,
 \mathbbm{1}_{\{G_{\ell  -} > 1\}} \, \max( 1 - G_{\ell}, 0 )\}}_{\tiny \mbox{jumps~crossing~the~strike~from~below~and~above,}~ k = 1}. ~ \mbox{ \, \, \, \, } ~ &
%\nonumber
\end{eqnarray*} %}
%{\color{blue}For brevity, define the risk premium}
%\begin{eqnarray}
%\mathbf{A}_t^{T_O}[1] ~ \, \equiv ~ \mathbb{E}_{t}^{\mathbb{P}}( \mathbb{A}_t^{T_O}[1] )
%\, - \, \mathbb{E}_{t}^{\mathbb{Q}}( \mathbb{A}_t^{T_O}[1] ). ~~
%\mbox{ \, \, \quad \quad }
%\end{eqnarray}
%{\color{blue}For brevity, define the risk premium}
%\begin{eqnarray}
%\mathbf{A}_t^{T_O}[1] ~ \, \equiv ~
%\{\mathbb{E}_{t}^{\mathbb{P}}( \mathbb{A}_t^{T_O}[1] ) - \mathbb{E}_{t}^{\mathbb{Q}}( \mathbb{A}_t^{T_O}[1] )\}  . ~~
%\mbox{ \, \, \quad \quad }
%\end{eqnarray}
{Using
%{\color{red}[[[ two of these
%equations have now dissappeared. ]]]]] [[[[[[
%equations (\ref{eq:ReexpressCallRiskPremium}),
%(\ref{eq:ReexpressPutRiskPremium}),
%(\ref{eq:BlackScholesTypeFormula2}),
%and (\ref{eq:BlackScholesTypeFormula2Put}), ]]]]}
equations (\ref{eq:ReexpressCallRiskPremium}) and
(\ref{eq:BlackScholesTypeFormula2}) (as well as the analogous
(but, for brevity, not presented)
equations for put options)
the sign of the risk premium on ATM straddles is the same as the sign of
\begin{eqnarray}
& & 2 \overbrace{\mathbb{E}_{t}^{\mathbb{Q}}( \mathcal{R}_{T_O}^{\tiny \mbox{span~diffusive}} \times
\mathrm{cov}_t^{\mathbb{Q}}(
\mathcal{R}_{T_O}^{\tiny \mbox{unspan~diffusive}}
\times \mathcal{R}_{T_O}^{\tiny \mbox{unspan~jump}},
\mathbb{L}_t^{{T}_O}[1] {\Big |}  \mathcal{I}_{T_O} ) )}^{\text{ \tiny (already signed by equations (\ref{eq:ConditCovarSpanning2Linesa}) and
(\ref{eq:JumpRelatedLT2})) }} \nonumber \\
&&~~
+2
\{\mathbb{E}_{t}^{\mathbb{P}}( \mathbb{A}_t^{T_O}[1] ) - \mathbb{E}_{t}^{\mathbb{Q}}( \mathbb{A}_t^{T_O}[1] )\} ~
%\mbox{ \, \, \, \,\,\,\, \,\,\,\,}
\nonumber \\
& & ~ ~ ~ + ~ \mathbb{E}_{t}^{\mathbb{P}}( \int_{t+}^{T_O} \mathbbm{1}_{\{G_{\ell  -} > 1\}} dG_{\ell} ) ~ \nonumber \\
&&- ~ \overbrace{\mathbb{E}_{t}^{\mathbb{P}}( \frac{1}{\mathcal{R}_{T_O}^{\tiny \mbox{unspan~diffusive}} \, \mathcal{R}_{T_O}^{\tiny \mbox{unspan~jump}}} \, \, \int_{t+}^{T_O} \mathbbm{1}_{\{G_{\ell  -} > 1\}} dG_{\ell} )}^{= ~ \mathbb{E}_{t}^{\mathbb{Q}}(  \mathcal{R}_{T_O}^{\tiny \mbox{span~diffusive}} \, \int_{t+}^{T_O} \mathbbm{1}_{\{G_{\ell  -} > 1\}} dG_{\ell} )} ~ ~  \mbox{ \, } ~
\nonumber \\
& & ~ ~ ~ + ~
\underbrace{ ~ \mathbb{E}_{t}^{\mathbb{Q}}( \mathcal{R}_{T_O}^{\tiny \mbox{span~diffusive}} \, \max( G_{T_O} - 1, 0 ) )}_{\text{ \tiny
$= \, \mathbb{E}_{t}^{\mathbb{Q}}( \mathcal{R}_{T_O}^{\perp} \,
\{ \mathrm{call}_t^{\tiny \mbox{BS}}[ {G}_{T_O}^{\perp} \mathcal{V}_{T_{O}}^{\bullet} , 1] \} )$ }} ~ - ~
\underbrace{ ~ \mathbb{E}_{t}^{\mathbb{Q}}( \max( G_{T_O} - 1, 0 ) )}_{\text{ \tiny
$= \, \mathbb{E}_{t}^{\mathbb{Q}}( \mathrm{call}_t^{\tiny \mbox{BS}}[ {G}_{T_O}^{\perp}, 1 ] )$ }} \nonumber \\
& & ~  - \mathbb{E}_{t}^{\mathbb{P}}( \int_{t+}^{T_O} \mathbbm{1}_{\{G_{\ell  -} < 1\}} dG_{\ell} ) \nonumber \\
&& + \overbrace{\mathbb{E}_{t}^{\mathbb{P}}( \frac{1}{\mathcal{R}_{T_O}^{\tiny \mbox{unspan~diffusive}} \, \mathcal{R}_{T_O}^{\tiny \mbox{unspan~jump}}} \,\int_{t+}^{T_O} \mathbbm{1}_{\{G_{\ell  -} < 1\}} dG_{\ell} )}^{= \mathbb{E}_{t}^{\mathbb{Q}}(  \mathcal{R}_{T_O}^{\tiny \mbox{span~diffusive}} \, \int_{t+}^{T_O} \mathbbm{1}_{\{G_{\ell  -} < 1\}} dG_{\ell} )}
 ~ ~  \mbox{ \, } ~
\nonumber \\
& & +
\underbrace{ ~ \mathbb{E}_{t}^{\mathbb{Q}}( \mathcal{R}_{T_O}^{\tiny \mbox{span~diffusive}} \, \max( 1 - G_{T_O}, 0 ) )}_{\text{ \tiny
$= \mathbb{E}_{t}^{\mathbb{Q}}( \mathcal{R}_{T_O}^{\perp} \{ \mathrm{put}_t^{\tiny \mbox{BS}}[ {G}_{T_O}^{\perp} \mathcal{V}_{T_{O}}^{\bullet} , 1] \} )$ }} -
\underbrace{ ~ \mathbb{E}_{t}^{\mathbb{Q}}( \max( 1 - G_{T_O}, 0 ) ).}_{\text{ \tiny
$=\mathbb{E}_{t}^{\mathbb{Q}}( \mathrm{put}_t^{\tiny \mbox{BS}}[ {G}_{T_O}^{\perp}, 1 ] )$ }}
\label{eq:ReexpressStraddleRiskPremium}
\end{eqnarray}
Simplifying equation
(\ref{eq:ReexpressStraddleRiskPremium}), the sign
of the risk premium on straddles is the same as the sign of
\begin{eqnarray}
& & 2 ~ \mathbb{E}_{t}^{\mathbb{Q}}( \mathcal{R}_{T_O}^{\tiny \mbox{span~diffusive}} \times
\mathrm{cov}_t^{\mathbb{Q}}(
\mathcal{R}_{T_O}^{\tiny \mbox{unspan~diffusive}}
\times \mathcal{R}_{T_O}^{\tiny \mbox{unspan~jump}},
\, \mathbb{L}_t^{{T}_O}[1] {\Big |} \, \mathcal{I}_{T_O} ) )
\nonumber \\
& &
~+~  2 \{\mathbb{E}_{t}^{\mathbb{P}}( \mathbb{A}_t^{T_O}[1] ) - \mathbb{E}_{t}^{\mathbb{Q}}( \mathbb{A}_t^{T_O}[1] )\} \nonumber \\
& & ~ + ~ \mathbb{E}_{t}^{\mathbb{P}}( \int_{t+}^{T_O} \mathbbm{1}_{\{G_{\ell  -} > 1\}} dG_{\ell} ) ~ - ~ \mathbb{E}_{t}^{\mathbb{P}}( \frac{1}{\mathcal{R}_{T_O}^{\tiny \mbox{unspan~diffusive}} \, \mathcal{R}_{T_O}^{\tiny \mbox{unspan~jump}}} \, \, \int_{t+}^{T_O} \mathbbm{1}_{\{G_{\ell  -} > 1\}} dG_{\ell} ) ~ ~  \mbox{ \, \, \, } ~
\nonumber \\
& & ~ - ~ \mathbb{E}_{t}^{\mathbb{P}}( \int_{t+}^{T_O} \mathbbm{1}_{\{G_{\ell  -} < 1\}} dG_{\ell} ) ~ + ~ \mathbb{E}_{t}^{\mathbb{P}}( \frac{1}{\mathcal{R}_{T_O}^{\tiny \mbox{unspan~diffusive}} \, \mathcal{R}_{T_O}^{\tiny \mbox{unspan~jump}}} \, \, \int_{t+}^{T_O} \mathbbm{1}_{\{G_{\ell  -} < 1\}} dG_{\ell} ) ~ ~  \mbox{ \, \, \, } ~
\nonumber \\
&&
~+~\mathbb{E}_{t}^{\mathbb{Q}}( \mathcal{R}_{T_O}^{\perp} \,
\{ \mathrm{straddle}_t^{\tiny \mbox{BS}}[ {G}_{T_O}^{\perp} \mathcal{V}_{T_{O}}^{\bullet} , 1] \} ) ~ - ~
\mathbb{E}_{t}^{\mathbb{Q}}( \mathrm{straddle}_t^{\tiny \mbox{BS}}[ {G}_{T_O}^{\perp}, 1 ] ),
\nonumber
%\label{eq:ReexpressStraddleRiskPremiumSimplified}
\end{eqnarray}
where
\begin{eqnarray}
\mathrm{straddle}_t^{\tiny \mbox{BS}}[ {G}_{T_O}^{\perp} \mathcal{V}_{T_{O}}^{\bullet} , 1]
&\equiv&
\mathrm{call}_t^{\tiny \mbox{BS}}[ {G}_{T_O}^{\perp} \mathcal{V}_{T_{O}}^{\bullet} , 1] \, + \, \mathrm{put}_t^{\tiny \mbox{BS}}[ {G}_{T_O}^{\perp} \mathcal{V}_{T_{O}}^{\bullet} , 1] ~~~ ~ ~ \mbox{ \, \, \, and \, \, \, } ~ ~
\\
\mathrm{straddle}_t^{\tiny \mbox{BS}}[ {G}_{T_O}^{\perp}, 1 ]
& \equiv &
\mathrm{call}_t^{\tiny \mbox{BS}}[ {G}_{T_O}^{\perp}, 1 ] \, + \,  \mathrm{put}_t^{\tiny \mbox{BS}}[ {G}_{T_O}^{\perp}, 1 ]. ~
\mbox{ \, }
\end{eqnarray}

Now, we assume that
\begin{eqnarray}
& & \mathrm{straddle}_t^{\tiny \mbox{BS}}[ {G}_{T_O}^{\perp} \mathcal{V}_{T_{O}}^{\bullet} , 1] ~ \, \approx ~
\mathrm{straddle}_t^{\tiny \mbox{BS}}[ {G}_{T_O}^{\perp}, 1 ], ~~
~ ~
\mbox{ \, \, \, \, \quad \quad and that \, \, \quad \, } ~ \label{eq:straddleFirstAssumption-a} \\
& & \mathbb{E}_{t}^{\mathbb{Q}}( \mathcal{R}_{T_O}^{\perp} \,
\{ \mathrm{straddle}_t^{\tiny \mbox{BS}}[ {G}_{T_O}^{\perp} , 1] \} )
~ \, \approx ~
\mathbb{E}_{t}^{\mathbb{Q}}( \mathrm{straddle}_t^{\tiny \mbox{BS}}[ {G}_{T_O}^{\perp} , 1] ).
\mbox{ \, \, \, \, \, \, \quad } ~~ \label{eq:straddleSecondAssumption-b}
\end{eqnarray}
The first condition in (\ref{eq:straddleFirstAssumption-a}) is consistent with straddles being approximately delta-neutral. Next,
$\mathcal{R}_{T_O}^{\perp}$ is a term which comes from the \emph{spanned} component of the pricing kernel
and so the correlation between this quantity and the (delta-neutral) straddle is (approximately)
zero, leading to (\ref{eq:straddleSecondAssumption-b}).

It follows that the sign of the risk premium on straddles is
that of
\begin{eqnarray}
& & 2 ~ \mathbb{E}_{t}^{\mathbb{Q}}( \mathcal{R}_{T_O}^{\tiny \mbox{span~diffusive}} \times
\mathrm{cov}_t^{\mathbb{Q}}(
\mathcal{R}_{T_O}^{\tiny \mbox{unspan~diffusive}}
\times \mathcal{R}_{T_O}^{\tiny \mbox{unspan~jump}},
\, \mathbb{L}_t^{{T}_O}[1] {\Big |} \, \mathcal{I}_{T_O} ) )  \nonumber \\
& & ~ ~ + ~ 2 \{\mathbb{E}_{t}^{\mathbb{P}}( \mathbb{A}_t^{T_O}[1] ) - \mathbb{E}_{t}^{\mathbb{Q}}( \mathbb{A}_t^{T_O}[1] )\} ~ \mbox{ \, \, \, } \nonumber \\
& & ~ ~ + ~ \mathbb{E}_{t}^{\mathbb{P}}( \int_{t+}^{T_O} \mathbbm{1}_{\{G_{\ell  -} > 1\}} dG_{\ell} ) ~ - ~ \mathbb{E}_{t}^{\mathbb{P}}( \frac{1}{\mathcal{R}_{T_O}^{\tiny \mbox{unspan~diffusive}} \, \mathcal{R}_{T_O}^{\tiny \mbox{unspan~jump}}} \, \, \int_{t+}^{T_O} \mathbbm{1}_{\{G_{\ell  -} > 1\}} dG_{\ell} ) ~ ~  \mbox{ \, \, \, } ~
\nonumber \\
& & ~ ~ - ~ \mathbb{E}_{t}^{\mathbb{P}}( \int_{t+}^{T_O} \mathbbm{1}_{\{G_{\ell  -} < 1\}} dG_{\ell} ) ~ + ~ \mathbb{E}_{t}^{\mathbb{P}}( \frac{1}{\mathcal{R}_{T_O}^{\tiny \mbox{unspan~diffusive}} \, \mathcal{R}_{T_O}^{\tiny \mbox{unspan~jump}}} \, \, \int_{t+}^{T_O} \mathbbm{1}_{\{G_{\ell  -} < 1\}} dG_{\ell} ). ~ ~  \mbox{ \, \, \, } ~
\label{eq:ReexpressStraddleRiskPremiumSimplifiedAgain}
\end{eqnarray}
In particular,
if there were no unspanned risks in the pricing kernel; that is, if
$\mathcal{R}_{T_O}^{\tiny \mbox{unspan~diffusive}} \equiv 1$ and $\mathcal{R}_{T_O}^{\tiny \mbox{unspan~jump}} \equiv 1$,
equation (\ref{eq:ReexpressStraddleRiskPremiumSimplifiedAgain}) would
evaluate to zero. Our rationale is:
\begin{itemize}

\item The
%This would be so because
covariance term would then be identically zero.
%{\color{magenta} (if there were
%would
%no jumps).}

\item Additionally, $\mathbb{E}_{t}^{\mathbb{P}}( \mathbb{A}_t^{T_O}[1] ) - \mathbb{E}_{t}^{\mathbb{Q}}( \mathbb{A}_t^{T_O}[1] ) =  0$ (if there were no jumps).
    \item Finally, the third and fourth lines would cancel.

\end{itemize}
Thus, to support our empirical findings
of a negative risk premium on straddles, there must be
unspanned risks in the pricing} kernel. $\blacksquare$ %$\mathbbm{R}$ ${\mathcal R}$
%%%%%%%%%%%%%%%%%%% END %%%%%%%%%%%%%%%%%%%%%%%%%%%%%%%%%%%%%%%%%%%%%%%%%%%%%%%%%%%%%%%%%%%%%%%%%%%%%%%%%%%%%%%%%%%%%%%%%%%%%%%%%%%%%%%%%%%%%%%%
%%%%%%%%%%%%%%%%%%% END %%%%%%%%%%%%%%%%%%%%%%%%%%%%%%%%%%%%%%%%%%%%%%%%%%%%%%%%%%%%%%%%%%%%%%%%%%%%%%%%%%%%%%%%%%%%%%%%%%%%%%%%%%%%%%%%%%%%%%%%

%%%%%%%%%%%%%%%%%%%%%%%%%%%%%%%%%%%%%%%%%%%%%%%%%%%%%%%%%%%%%%%%%%%%%%%%%%%%%%%%%%%%%%%%%%%%%%%%%%%%%%%%%%%%%%%%%%%%%%%%%%%%%%%%%%%%%%%%%%%
%\numberwithin{table}{section}
\setcounter{table}{0}
\renewcommand{\thetable}{IA-\arabic{table}}









%%%%%%%%%%%%%%%%% Table 2: 2 and 3 day day options %%%%%%%%%%%%%%%%%%%%%%%%%%%%%%%%%%%%%%%%%%%%%%%%%%%%%%%%
\newpage
\begin{table}[h!]
%\scriptsize
\footnotesize
\caption{\textbf{{
Risk premiums for holding weekly options over 2-day and 3-day windows}}}
\vspace{2mm}
\label{tab:2and3day}
The sample period is 01/13/2011 to 12/20/2018, with 415 weekly option expiration cycles. The weekly options data on S\&P 500 index is from the CBOE. We construct the excess return of OTM puts, OTM calls, and straddles (ATM and crash-neutral).  These calculations are done at the ask option price. The returns of a crash-neutral
straddle combines a long straddle position and a short 3\% OTM put position.
\begin{description}

\item[$-$] Panel A computes option returns from Wednesday to Friday (2-day to maturity (on average)).
\item[$-$] Panel B computes option returns from Tuesday to Friday (3-day to maturity (on average)).
\end{description}
We indicate statistical significance at 1\%, 5\%, and 10\% by the superscripts ***, **, and *, respectively,
where the $p$-values rely on the \citet*{NeweyWest:87} HAC estimator (with the lag selected automatically).
%according to \citet*{NeweyWest:1994RES}).
The reported
put (respectively, call) delta is $-{\cal N}(-d_1)$ (respectively, ${\cal N}(d_1)$), where $d_1=  \frac{1}{ \sigma \sqrt{T_O-t}} \{ - \log k + r (T_O-t) + \frac{1}{2} \sigma^2 (T_O-t)\}$. %(following \citet*[page 718]{BollenWhaley:2004}).
SD is the standard deviation, and $\mathbbm{1}_{\{ q_{t, {T}_O} >0 \}}$ is the proportion (in \%) of
option positions that generate positive returns.
\begin{center}
\setlength{\tabcolsep}{0.08in}
\begin{tabular}{lll ccc c ccc c cc} \hline
       &   & &           &           &           &           &           &           &  \\
&&&\multicolumn{10}{c}{\textbf{Panel A: \emph{2-day} holding period returns}} \\ \\
       &   & &\multicolumn{3}{c}{OTM puts on equity} &           &  \multicolumn{3}{c}{OTM calls on equity}& &\multicolumn{2}{c}{Straddle} \\
       &   & &\multicolumn{3}{c}{ $\log(k)\times 100$} &        & \multicolumn{3}{c}{$\log(k)\times 100$} &    &\multicolumn{2}{c}{on equity} \\
          \cline{4-6} \cline{8-10} \cline{12-13}
Moneyness (\%)   &  &      & -3         & -2         & -1         &          & 1         & 2         & 3 &    &  ATM  &     Crash-Neutral\\
Delta (\%)       &  & & -2         & -5         & -17         &           & 17         & 5         & 2 &    &   &     Neutral\\
%Open Interest ($\times 1,000$)    &  & & 10.2         & 9.3         & 7.4         &           & 9.1         & 7.9         & 6.9 &    &   &    \\
%Volume ($\times 1,000)$     &  & & 2.5        & 2.5         & 2.6         &           & 3.1         & 2.4        & 1.8 &    &   &    \\
 &         &       &    &           &           &           &                      &  &    &    &\\ \hline
           &           &           &           &           &           &           &           &           & &    &    & \\
\multicolumn{1}{l}{\textbf{Unconditional}} &           & \multicolumn{1}{l}{Average} &-82   & -44   & -31   &       & -17   & -43   & -79   &       & -12   & -1 \\
\multicolumn{1}{l}{\textbf{Estimates}} &           & \multicolumn{1}{l}{SD} & 170   & 367   & 296   &       & 281   & 402   & 310   &       & 94    & 5 \\
\multicolumn{1}{l}{ } &           & \multicolumn{1}{l}{$\mathbbm{1}_{\{ q_{t, {T}_O} >0 \}}$} & 2\%     & 5\%     & 10\%    &       & 13\%    & 4\%     & 1\%     &       & 35\%    & 36\% \\
          &           &           &           &           &           &           &           &           & &    &    & \\ \hline \\
&&&\multicolumn{10}{c}{\textbf{Panel B: \emph{3-day} holding period returns}} \\ \\
                &   & &\multicolumn{3}{c}{OTM puts on equity} &           &  \multicolumn{3}{c}{OTM calls on equity}& &\multicolumn{2}{c}{Straddle} \\
       &   & &\multicolumn{3}{c}{ $\log(k)\times 100$} &        & \multicolumn{3}{c}{$\log(k)\times 100$} &    &\multicolumn{2}{c}{on equity} \\
          \cline{4-6} \cline{8-10} \cline{12-13}
Moneyness (\%)   &  &      & -3         & -2         & -1         &          & 1         & 2         & 3 &    &  ATM  &     Crash-Neutral\\
Delta (\%)       &  & & -1         & -3         & -13         &           & 13         & 3         & 1 &    &   &     Neutral\\
%Open Interest ($\times 1,000$)    &  & & 10.2         & 9.3         & 7.4         &           & 9.1         & 7.9         & 6.9 &    &   &    \\
%Volume ($\times 1,000)$     &  & & 2.5        & 2.5         & 2.6         &           & 3.1         & 2.4        & 1.8 &    &   &    \\
 &         &       &    &           &           &           &                      &  &    &    &\\ \hline
           &           &           &           &           &           &           &           &           & &    &    & \\
\multicolumn{1}{l}{\textbf{Unconditional}} &           & \multicolumn{1}{l}{Average} & -72   & -47   & -27   &       & 18    & -28   & -57   &       & -6    & -0 \\
\multicolumn{1}{l}{\textbf{Estimates}} &           & \multicolumn{1}{l}{SD} & 205   & 252   & 227   &       & 546   & 471   & 443   &       & 87    & 6  \\
\multicolumn{1}{l}{ } &           & \multicolumn{1}{l}{$\mathbbm{1}_{\{ q_{t, {T}_O} >0 \}}$}
& 3\%     & 7\%     & 15\%    &
& 19\%    & 7\%     & 3\%     &
& 40\%    & 42\% \\
          &           &           &           &           &           &           &           &           & &    &    & \\ \hline
\end{tabular}%
\end{center}
\end{table}
%%%%%%%%%%%%%%%%%%%%%%%%%%%%%%%%%%%%%%%%%%%%%%%%%%%%%%%%%%%%%%%%%%%%%%%%%%%%%%



%%%%%%%%%%%%%%%%% Deep Weekly options %%%%%%%%%%%%%%%%%%%%%%%%%%%%%%%%%%%%%%%%%%%%%%%%%%%%%%%%
\newpage
\begin{table}[h!]
%\scriptsize
\footnotesize
\caption{\textbf{{
%Estimates of the
Risk premiums for \emph{weekly} OTM calls on the S\&P 500 index, deeper than 3\% OTM}}}
\vspace{2mm}
\label{tab:deep_weekly}
This table complements Table~\ref{tab:weekly} by presenting results on call option excess returns deeper than 3\% OTM. These calculations are done at the ask option price. The sample period is 01/13/2011 to 12/20/2018, with 415 weekly option expiration cycles (8 days to maturity (on average)).
%The options data on S\&P 500 futures (respectively, S\&P 500 index) is from the CME (respectively, CBOE).
The weekly options data on
the S\&P 500 index is from the CBOE.
%We construct the excess return of OTM puts, OTM calls, and straddles (ATM and crash-neutral) over weekly
%expiration cycles.
%The returns of a crash-neutral
%straddle combines a long straddle position and a short 3\% OTM put position.
%Presented are the results from the following regression specification (analogously for puts and straddles):
The following is the regression specification (analogously for puts and straddles):
\begin{align*}
&q_{t,{\tiny \mathrm{call}} }^{{T}_O}[k] =
\mu_{\{ {\cal F}_{t} \in \mathfrak{s}_{\tiny \mbox{bad}} \} } \mathbbm{1}_{\{ {\cal F}_{t} \in\mathfrak{s}_{\tiny \mbox{bad}} \}}
+ \mu_{\{ {\cal F}_{t} \in \mathfrak{s}_{\tiny \mbox{normal}} \} }
\mathbbm{1}_{\{ {\cal F}_{t} \in\mathfrak{s}_{\tiny \mbox{normal}} \}}
+ \mu_{\{ {\cal F}_{t} \in \mathfrak{s}_{\tiny \mbox{good}} \} }
\mathbbm{1}_{\{ {\cal F}_{t} \in\mathfrak{s}_{\tiny \mbox{good}} \}} ~+~ \underbrace{\epsilon_{T_{O}}.}_{\mathrm{error~term}}&
\end{align*}
We use proxies for the variable $\mathfrak{s}$, known at the beginning of the expiration cycle.
The variable construction for this weekly exercise is described
in the text.
For example, WEI is the weekly economic index. % (Federal Reserve Bank of New York).
\\ %\vspace{2mm}

We indicate statistical significance at 1\%, 5\%, and 10\% by the superscripts ***, **, and *, respectively,
where the $p$-values rely on the \citet*{NeweyWest:87} HAC estimator (with the lag selected automatically).
%according to \citet*{NeweyWest:1994RES}).
The reported
put (respectively, call) delta is $-{\cal N}(-d_1)$ (respectively, ${\cal N}(d_1)$), where $d_1=  \frac{1}{ \sigma \sqrt{T_O-t}} \{ - \log k + r (T_O-t) + \frac{1}{2} \sigma^2 (T_O-t)\}$. %(following \citet*[page 718]{BollenWhaley:2004}).
SD is the standard deviation, and $\mathbbm{1}_{\{ q_{t, {T}_O} >0 \}}$ is the proportion (in \%) of
option positions that generate positive returns.
We tabulate the average open interest and trading volume, all observed on the first day of the weekly option expiration cycle.
\begin{center}
\setlength{\tabcolsep}{0.14in}
\begin{tabular}{lll ccc} \hline
%      &   & &           &           &           &           &           &           &  \\
       &   & & \multicolumn{3}{c}{OTM calls on equity}\\
       &   & & \multicolumn{3}{c}{$\log(k)\times 100$} \\
          \cline{4-6}
Moneyness (\%)                    &  &      & 4         & 5         & 6 \\
Delta (\%)                        &  &      & 3         & 2         & 1 \\ \\
Open Interest ($\times 1,000$)    &  &      &  5.6   & 5.3   & 3.9    \\
Volume ($\times 1,000)$           &  &      &  0.9   & 0.9   & 0.6    \\ \hline
                   &  &      &          &          &  \\
\multicolumn{1}{l}{Change in WEI} & \multicolumn{1}{l}{L} & \multicolumn{1}{l}{$\mathfrak{s}_{\tiny \mbox{bad}}$} & -73*** & -92*** & -100*** \\
\multicolumn{1}{l}{ } & \multicolumn{1}{l}{M} & \multicolumn{1}{l}{$\mathfrak{s}_{\tiny \mbox{normal}}$} &  -91*** & -96*** & -97***  \\
\multicolumn{1}{l}{ } & \multicolumn{1}{l}{H} & \multicolumn{1}{l}{$\mathfrak{s}_{\tiny \mbox{good}}$} & -90*** & -99*** & -100*** \\ \hline
\multicolumn{1}{l}{Quadratic Variation} & \multicolumn{1}{l}{H} & \multicolumn{1}{l}{$\mathfrak{s}_{\tiny \mbox{bad}}$} & \-65*** & -87*** & -97*** \\
\multicolumn{1}{l}{ } & \multicolumn{1}{l}{M} & \multicolumn{1}{l}{$\mathfrak{s}_{\tiny \mbox{normal}}$} & -88*** & -100*** & -100*** \\
\multicolumn{1}{l}{ } & \multicolumn{1}{l}{L} & \multicolumn{1}{l}{$\mathfrak{s}_{\tiny \mbox{good}}$} & -100*** & -100*** & -100***\\
\hline
\multicolumn{1}{l}{Risk Reversal} & \multicolumn{1}{l}{H} & \multicolumn{1}{l}{$\mathfrak{s}_{\tiny \mbox{bad}}$} & -88*** & -94*** & -100*** \\
\multicolumn{1}{l}{ } & \multicolumn{1}{l}{M} & \multicolumn{1}{l}{$\mathfrak{s}_{\tiny \mbox{normal}}$} &  -88*** & -100*** & -100*** \\
\multicolumn{1}{l}{ } & \multicolumn{1}{l}{L} & \multicolumn{1}{l}{$\mathfrak{s}_{\tiny \mbox{good}}$} & -78*** & -93*** & -97***\\
\hline
\multicolumn{1}{l}{Change in Volatility} & \multicolumn{1}{l}{H} & \multicolumn{1}{l}{$\mathfrak{s}_{\tiny \mbox{bad}}$} & -79*** & -91*** & -100***  \\
\multicolumn{1}{l}{ } & \multicolumn{1}{l}{M} & \multicolumn{1}{l}{$\mathfrak{s}_{\tiny \mbox{normal}}$} & -88*** & -96*** & -97*** \\
\multicolumn{1}{l}{ } & \multicolumn{1}{l}{L} & \multicolumn{1}{l}{$\mathfrak{s}_{\tiny \mbox{good}}$} & -87*** & -100*** & -100***\\
\hline
\multicolumn{1}{l}{Recent Market} & \multicolumn{1}{l}{L} & \multicolumn{1}{l}{$\mathfrak{s}_{\tiny \mbox{bad}}$} & -76*** & -88*** & -97***  \\
\multicolumn{1}{l}{ } & \multicolumn{1}{l}{M} & \multicolumn{1}{l}{$\mathfrak{s}_{\tiny \mbox{normal}}$} & -81*** & -100*** & -100*** \\
\multicolumn{1}{l}{ } & \multicolumn{1}{l}{H} & \multicolumn{1}{l}{$\mathfrak{s}_{\tiny \mbox{good}}$} & -96*** & -98*** & -100***  \\
\hline
\\
\multicolumn{1}{l}{\textbf{Unconditional}} &           & \multicolumn{1}{l}{Average} & -85   & -96   & -99 \\
\multicolumn{1}{l}{\textbf{Estimates}} &           & \multicolumn{1}{l}{SD} & 131   & 52    & 21\\
\multicolumn{1}{l}{ } &           & \multicolumn{1}{l}{$\mathbbm{1}_{\{ q_{t, {T}_O} >0 \}}$} & 2\%     & 1\%     & 0\%\\

\\ \hline


\end{tabular}%
\end{center}
\end{table}
%%%%%%%%%%%%%%%%%%%%%%%%%%%%%%%%%%%%%%%%%%%%%%%%%%%%%%%%%%%%%%%%%%%%%%%%%%%%%%


%%%%%%%%%%%%%%%%% Bid-ask%%%%%%%%%%%%%%%%%%%%%%%%%%%%%%%%%%%%%%%%%%%%%%%%%%%%%%%%
\newpage
\begin{table}[h!]
%\scriptsize
\footnotesize
\caption{\textbf{Option risk premiums
based on the midpoint of bid and ask prices}}
\vspace{2mm}
\label{tab:bid-ask}
All option return calculations are done at the \emph{midpoint} of bid and ask option prices.
The sample period of this exercise for S\&P 500 index options is as follows:
\begin{description}
\item[$-$] Weekly options: 01/13/2011 to 12/20/2018, with 415 weekly expiration cycles (8 days to maturity (on average)).
\item[$-$] 28-day options: 01/22/1990 to 12/24/2018, with 348 expiration cycles (28 days to maturity (on average)).
\end{description}
We construct the excess return of OTM puts, OTM calls, and straddles (ATM and crash-neutral) over expiration cycles.
Presented are the results from the following regression specification (analogously for puts and straddles):
\begin{align*}
&q_{t,{\tiny \mathrm{call}} }^{{T}_O}[k] =
\mu_{\{ {\cal F}_{t} \in \mathfrak{s}_{\tiny \mbox{bad}} \} } \mathbbm{1}_{\{ {\cal F}_{t} \in\mathfrak{s}_{\tiny \mbox{bad}} \}}
+ \mu_{\{ {\cal F}_{t} \in \mathfrak{s}_{\tiny \mbox{normal}} \} }
\mathbbm{1}_{\{ {\cal F}_{t} \in\mathfrak{s}_{\tiny \mbox{normal}} \}}
+ \mu_{\{ {\cal F}_{t} \in \mathfrak{s}_{\tiny \mbox{good}} \} }
\mathbbm{1}_{\{ {\cal F}_{t} \in\mathfrak{s}_{\tiny \mbox{good}} \}} +
\epsilon_{T_{O}}.&
\end{align*}
We use proxies for the variable $\mathfrak{s}$, known at the beginning of the option expiration cycle.
The variable construction for the weekly and monthly exercise is described
in the text. We indicate statistical significance at 1\%, 5\%, and 10\% by the superscripts ***, **, and *, respectively,
where the $p$-values rely on the \citet*{NeweyWest:87} HAC estimator (with the lag selected automatically).
%The reported put (respectively, call) delta is $-{\cal N}(-d_1)$ (respectively, ${\cal N}(d_1)$), where $d_1=  \frac{1}{ \sigma \sqrt{T_O-t}} \{ - \log k + r (T_O-t) + \frac{1}{2} \sigma^2 (T_O-t)\}$.
SD is the standard deviation, and $\mathbbm{1}_{\{ q_{t, {T}_O} >0 \}}$ is the proportion (in \%) of
option positions that generate positive returns.
\begin{center}
\setlength{\tabcolsep}{0.05in}
\begin{tabular}{lll ccc c ccc c cc} \hline
       &   & &           &           &           &           &           &           &  \\
&&&\multicolumn{10}{c}{\textbf{Panel A: Weekly options}} \\ \\
&   & &\multicolumn{3}{c}{OTM puts on equity} &           &  \multicolumn{3}{c}{OTM calls on equity}& &\multicolumn{2}{c}{Straddle} \\
       &   & &\multicolumn{3}{c}{ $\log(k)\times 100$} &        & \multicolumn{3}{c}{$\log(k)\times 100$} &    &\multicolumn{2}{c}{on equity} \\
          \cline{4-6} \cline{8-10} \cline{12-13}
Moneyness (\%)   &  &      & -3         & -2         & -1         &          & 1         & 2         & 3 &    &  ATM  &     Crash-Neutral\\
%Delta (\%)       &  & & -6         & -12         & -26         &           & 27         & 12         & 6 &    &   &     Neutral\\ \\
%Open Interest ($\times 1,000$)    &  & & 10.2         & 9.3         & 7.4         &           & 9.1         & 7.9         & 6.9 &    &   &    \\
%Volume ($\times 1,000)$     &  & & 2.5        & 2.5         & 2.6         &           & 3.1         & 2.4        & 1.8 &    &   &    \\
 &         &       &    &           &           &           &                      &  &    &    &\\ \hline
 &         &       &    &           &           &           &                      &  &    &    &\\
\multicolumn{1}{l}{Risk Reversal} & \multicolumn{1}{l}{H} & \multicolumn{1}{l}{$\mathfrak{s}_{\tiny \mbox{bad}}$} & -68*** & -49*** & -32** &       & 32    & -46** & -92*** &       & -7    & 0  \\
\multicolumn{1}{l}{ } & \multicolumn{1}{l}{M} & \multicolumn{1}{l}{$\mathfrak{s}_{\tiny \mbox{normal}}$} & -93*** & -73*** & -53*** &       & 25    & 13    & -22   &       & -10*  & 0    \\
\multicolumn{1}{l}{ } & \multicolumn{1}{l}{L} & \multicolumn{1}{l}{$\mathfrak{s}_{\tiny \mbox{good}}$} & -14   & -5    & -4    &       & -1    & -13   & -43** &       & -3    & 0 \\
          &           &           &           &           &           &           &           &           &  &    &    &\\
\multicolumn{1}{l}{Change in Volatility} & \multicolumn{1}{l}{H} & \multicolumn{1}{l}{$\mathfrak{s}_{\tiny \mbox{bad}}$} & -39   & -35   & -35** &       & 7     & -1    & -50*** &       & -12*  & -1\\
\multicolumn{1}{l}{ } & \multicolumn{1}{l}{M} & \multicolumn{1}{l}{$\mathfrak{s}_{\tiny \mbox{normal}}$} & -69*** & -48*** & -27   &       & 60    & 10    & -36   &       & -1    & 0 \\
\multicolumn{1}{l}{ } & \multicolumn{1}{l}{L} & \multicolumn{1}{l}{$\mathfrak{s}_{\tiny \mbox{good}}$}  & -68*** & -44*** & -28*  &       & -9    & -54*** & -71*** &       & -6    & 0 \\
          &           &           &           &           &           &           &           &           &  &    &    & \\ \hline
          &           &           &           &           &           &           &           &           &  &    &    &\\
          \multicolumn{1}{l}{\textbf{Unconditional}} &           & \multicolumn{1}{l}{Average} &-59   & -43   & -30   &       & 19    & -15   & -52   &       & -7    & -0 \\
\multicolumn{1}{l}{\textbf{Estimates}} &           & \multicolumn{1}{l}{SD} & 250   & 226   & 189   &       & 294   & 292   & 250   &       & 80    & 7 \\
\multicolumn{1}{l}{ } &           & \multicolumn{1}{l}{$\mathbbm{1}_{\{ q_{t, {T}_O} >0 \}}$} & 7\%     & 10\%    & 17\%    &       & 27\%    & 12\%    & 5\%     &       & 42\%    & 45\%\\
          &           &           &           &           &           &           &           &           & &    &    & \\ \hline \\
&&&\multicolumn{10}{c}{\textbf{Panel B: 28-day options}} \\ \\

                          &   & &\multicolumn{3}{c}{OTM puts on equity} &           &  \multicolumn{3}{c}{OTM calls on equity}& &\multicolumn{2}{c}{Straddle} \\
       &   & &\multicolumn{3}{c}{ $\log(k)\times 100$} &        & \multicolumn{3}{c}{$\log(k)\times 100$} &    &\multicolumn{2}{c}{on equity} \\
          \cline{4-6} \cline{8-10} \cline{12-13}
Moneyness (\%)   &  &      & -5         & -3         & -1         &          & 1         & 3         & 5 &    &  ATM  &     Crash-Neutral\\
%Delta (\%)       &  & & -6         & -12         & -26         &           & 27         & 12         & 6 &    &   &     Neutral\\ \\
%Open Interest ($\times 1,000$)    &  & & 10.2         & 9.3         & 7.4         &           & 9.1         & 7.9         & 6.9 &    &   &    \\
%Volume ($\times 1,000)$     &  & & 2.5        & 2.5         & 2.6         &           & 3.1         & 2.4        & 1.8 &    &   &    \\
 &         &       &    &           &           &           &                      &  &    &    &\\ \hline
 &         &       &    &           &           &           &                      &  &    &    &\\
\multicolumn{1}{l}{Risk Reversal} & \multicolumn{1}{l}{H} & \multicolumn{1}{l}{$\mathfrak{s}_{\tiny \mbox{bad}}$} & -69*** & -47*** & -37*** &       & 25    & 5     & -25   &       & -11   & 0 \\
\multicolumn{1}{l}{ } & \multicolumn{1}{l}{M} & \multicolumn{1}{l}{$\mathfrak{s}_{\tiny \mbox{normal}}$} &  -83*** & -66*** & -58*** &       & 19    & 14    & -8    &       & -16*** & -1 \\
\multicolumn{1}{l}{ } & \multicolumn{1}{l}{L} & \multicolumn{1}{l}{$\mathfrak{s}_{\tiny \mbox{good}}$} & -42** & -32   & -23   &       & -23** & -40*** & -54*** &       & -19** & -3**  \\
          &           &           &           &           &           &           &           &           &  &    &    &\\
\multicolumn{1}{l}{Change in Volatility} & \multicolumn{1}{l}{H} & \multicolumn{1}{l}{$\mathfrak{s}_{\tiny \mbox{bad}}$} & -49*** & -29*  & -23   &       & 16    & 17    & 5     &       & -5    & 1 \\
\multicolumn{1}{l}{ } & \multicolumn{1}{l}{M} & \multicolumn{1}{l}{$\mathfrak{s}_{\tiny \mbox{normal}}$} &-81*** & -59*** & -51*** &       & 6     & -11   & -19   &       & -22*** & -3* \\
\multicolumn{1}{l}{ } & \multicolumn{1}{l}{L} & \multicolumn{1}{l}{$\mathfrak{s}_{\tiny \mbox{good}}$} & -63*** & -55*** & -43*** &       & -1    & -30   & -80*** &       & -19*** & -2** \\
          &           &           &           &           &           &           &           &           &  &    &    & \\ \hline
          &           &           &           &           &           &           &           &           &  &    &    &\\
\multicolumn{1}{l}{\textbf{Unconditional}} &           & \multicolumn{1}{l}{Average} & -65   & -49   & -40   &       & 8     & -4    & -22   &       & -16   & -2 \\
\multicolumn{1}{l}{\textbf{Estimates}} &           & \multicolumn{1}{l}{SD} & 161   & 167   & 152   &       & 157   & 273   & 420   &       & 74    & 14 \\
\multicolumn{1}{l}{ } &           & \multicolumn{1}{l}{$\mathbbm{1}_{\{ q_{t, {T}_O} >0 \}}$} & 6\%     & 11\%    & 16\%    &       & 38\%    & 20\%    & 8\%     &       & 32\%    & 41\%\\
          &           &           &           &           &           &           &           &           & &    &    & \\ \hline
\end{tabular}%
\end{center}
\end{table}
%%%%%%%%%%%%%%%%%%%%%%%%%%%%%%%%%%%%%%%%%%%%%%%%%%%%%%%%%%%%%%%%%%%%%%%%%%%%%%




\end{document}

\clearpage
\newpage
\section{Collecting from John emails}

\subsection{Dou para}

{\color{magenta} The latter two terms are referred to dark matter embedded in
the non-arbitrage model of option pricing, in the same spirit as Chen et al. (2021). More precisely,
the subtle joint dynamics of the local time, the jump crossing the strike, and the SDF are difficult
to identify directly using the returns of the underlying asset; instead, the evidence on these subtle
dynamics can only be indirectly inferred from option prices through the lens of the model-implied
restrictions.}

*****************Dou 2

{\color{magenta} Through Tanaka's formula, we are able to provides two "sufficient statistics" - local time and jumps crossing the strike - to
summarize model option returns and the observed option returns in the data can be reconciled, especially on the upside of returns.
Accordingly, we show that that volatility and jump risk premiums can help rationalize the observed negative expected excess
returns of the OTM calls and the ATM straddles. Through this channel, we investigate the dark matter property of option pricing
models.}


{\color{red} [[[[[[[[[[
% which is consistent with the findings in the literature (e.g., Jones, 2006). In
my view, the paper is very intriguing and goes after a first-order question of asset pricing (i.e., the
"dark matter" feature of asset pricing theories) in the setting of option pricing. I like the fact that
this paper tries to investigate the "dark matter" property of option pricing models, a class of asset
pricing models with tremendous impact in practice.]]]]]]]]}


*********************Dou 3

{\color{magenta} What are the consequences of identifying dark matter embedded in the option pricing theory? In this light,
we draw the connections to \citet*{Chen_Dou_Kogan:JF2020}, who show that asset pricing models rely upon
economic dark matter, the concept we also emphasize in the context of option pricing. The key central point
advertised in \citet*{Chen_Dou_Kogan:JF2020} is that an asset pricing model with a large amount of dark matter will be hard
to refute. These models may convey acceptable insample overfit but may, in fact, perform poorly upon external validation exercises.
raises at least two concerns regarding its robustness.
%First, it will be difficult to detect any potential misspecification in the dark matter elements
%of a model due to a lack of direct evidence to measure them based on the underlying asset
%returns. Consequently, economic dark matter raises a model�s effective degrees of freedom and
%leads to low refutability by the standard optimal test procedures. Second, the high effective
%degrees of freedom will likely cause the model to overfit the data in sample and lead to poor
%expected out-of-sample fit. It is a main contribution of this paper (probably the most impor-
tant contribution)
Our contribution is to show that theories of option pricing rely on the presence of
dark matter to explain the observed option returns. }


%However, the current draft stops before allowing the researchers who work on option pricing
%and trading to understand �so what� if option pricing models feature excessive economic dark
%matter. The authors should provide more discussions and guidance on the consequence of
%the so-called dark matter, and inspire researchers and practitioners to work out more robust
%option pricing models and robust evaluation procedures for the option pricing models. This
%explicit, strong message will highlight the contribution of this paper and add a lot of value.


%*******************Dou4
%Elaborating more on that, I believe the authors should give a more elaborate guidance to
%the readers on what to do next or how to deal with dark matter after learning that

{\color{magenta} The gist of our exercises is that theories of option pricing risk premiums have to depend on economic
dark matter to reconcile
%rationalize
the empirical counterparts.
%In fact, empirical evaluations of the option pricing models have been a central question in both academia and industry (e.g., Bates, 1996; Bakshi et al., 1997).
The core idea is to test whether option prices are consistent with the time series properties of the underlying asset price.
An option pricing model with large dark matter means that the option models relies upon some subtle dynamics of the underlying
asset returns, which are hard to identify by the underlying asset returns alone. This makes the tests of consistency between
options and time series of the underlying assets not very meaningful, thereby rendering the model irrefutable using the standard
optimal specification tests (see \citet*{Chen_Dou_Kogan:JF2020}). The useful message is that the "robust" model evaluation procedures developed
by \citet*{Cheng_Dou_Liao:ECMTA2021} can be modified to tackle the dark matter issue in the empirical tests of the option pricing
models. For example, how the procedure can be modified to include restrictions under $\mathbb{P}$ and $\mathbb{Q}$, while using the
joint time-series of index options and option prices along the double dimension of strikes and maturities. In this regard, we
focus on the dark matter property of option pricing models and provide guidance on linking our theory to the data.}



Both \citet*{Chen_Dou_Kogan:JF2020} and \citet*{Cheng_Dou_Liao:ECMTA2021} show that many macrofinance asset pricing
models rely upon economic dark matter, the concept that we emphasize for option pricing models. Importantly,
\citet*{Chen_Dou_Kogan:JF2020} show that an asset pricing model with a large amount of dark matter raises at least
two concerns regarding its robustness. First, it will be difficult to detect any potential misspecification in the
dark {\color{blue}
matter elements of a model due to a lack of direct evidence to measure them based on the underlying asset returns.
Consequently, economic dark matter raises a model's effective degrees of freedom and leads to low refutability by the
standard optimal test procedures. Second, the high effective degrees of freedom will likely cause the model to
overfit the data in-sample and lead to poor expected out-of-sample fit. It is a main contribution of this paper to show
that option pricing models often have to rely upon a significant amount of dark matter to explain
observed option returns.
We provide the context for the consequence of dark matter and guidance for working out models
and robust evaluation procedures for option pricing models.
\citet*{Cheng_Dou_Liao:ECMTA2021} introduce robust
evaluation procedures for estimating macro-finance
models.
They dichotomize between
``baseline" moments
and ``asset pricing" moments
in estimating model parameters
that are only weakly} identified by ``baseline" moments. OVERCOME

An option pricing model with large dark matter means that the model relies upon some subtle dynamics of the
underlying asset returns, which are hard to identify by the underlying asset returns alone. {\color{blue}This makes the
tests of consistency between options and time series ambivalent, thereby rendering the model irrefutable using the
standard optimal specification tests (see \citet*{Chen_Dou_Kogan:JF2020}).
I believe one important and useful message this paper should send to the readers is that the robust model evaluation
procedures (e.g., \citet*{Cheng_Dou_Liao:ECMTA2021}) may be necessary to tackle the dark matter issue in the empirical
tests of the option pricing models.}


%how the researchers and practitioners should act to use the findings of this paper.


%Perhaps because of the lack of a strong and consistent focus on the core issue of economic dark matter, at times the
%motivation is hard to follow while I was reading the paper, and the contributions are likely
%to be hard for the researchers and practitioners who are not familiar with the concept of
%economic dark matter. Just to be clear, I don�t think the authors should compute the dark
%matter measure or apply the test procedure of Cheng et al. (2021) in the revision; rather, I
%strongly suggest that the revised draft focuses on the dark matter property of option pricing
%models and provides more guidance on how the researchers and practitioners should act to
%use the findings of this paper.

%{\color{red} [[[[ This dark matter and its risk premium are missed by the majority of models as
%they have yet to fully integrate observed data features from equity option returns. ]]]]}
%del



%Our empirical work centers on the time-series excess returns to holding out-of-the-money (OTM)
%options (puts and calls) of different moneyness and to holding ATM straddles, on the S&P 500
%futures and the S&P 500 index, of maturities averaging 8 days (weeklies), 28 days or 88 days.
%We consider excess returns conditional on the value of some variable s, known at the start of the
%option holding period, which proxies for the prevailing economic condition. These proxy variables
%(we consider five of them) include changes in the Weekly Economic Index (WEI, published by the
%New York Fed), changes in implied volatility and recent momentum (prior changes in the S&P 500
%index).

%Previewing our main empirical findings, we find that excess returns on 1\% OTM calls are,
%on average, positive but are sometimes negative in favorable prevailing economic conditions. In
%contrast, excess returns on 3\% OTM calls are, on average, slightly negative while average excess
%returns on 5\% OTM calls are markedly negative. For example, for 88 day maturity options, 5\% OTM
%calls have average excess returns which are negative across all fifteen partitions of proxy variables
%(and statistically significantly negative at the 1\% confidence level in 11 out of 15 partitions). These
%empirical outcomes are a validation of our theory and demonstrate that dark matter is relevant
%and the risk premium on dark matter is negative. Turning to ATM straddles of an average of 8 days to maturity, we find that the partitioned
%average excess returns of straddles are negative (in 13 out of 15 estimates), and the unconditional
%average is -7% (per 8 days � not annualized). The evidence for negative partitioned average excess
%returns of straddles is even stronger for longer maturity options where we find that the partitioned
%average excess returns of straddles are negative in all 15 partitions, for both 28 day and 88 day
%2
%options. These empirical results, again, support the risk premium on dark matter being negative
%and economically important. While other authors have previously linked negative excess returns of ATM straddles to negative
%variance risk premiums, we show that our dark matter risk premium is economically distinct from
%(although partially related to) the volatility or variance risk premium.

%%%%%%%%%%%%%%%%%%%%%%%%%%%%%%%%%%%%%%%%%%%%%%%%%%%%%%%%%%%%%%%%%%%%%%%%%%%%%%%%%%%%%%%%%%%%%%%%%%%%%%%%%%%%%%


%%%%%%%%%%%%%%%%%%%%%%%%%%%%%%%%%%%%%%%%%%%%%%%%%%%%%%%%%%%%%%%%%%%%%%%%%%%%%%%%%%%%%%%%%%%%%%%%%%%%%%%%%%%%%%

%%%%%%%%%%%%%%%%%%%%%%%%%%%%%%%%%%%%%%%%%%%%%%%%%%%%%%%%%%%%%%%%%%%%%%%%%%%%%%%%%%%%%%%%%%%%%%%%%%%%%%%%%%%%%%




%One
%crucial
%attribute of our theory is the result that local time risk premiums are identically zero when the modeling
%setup is devoid of unspanned risks.

%{\color{red}[[Equity index returns and their volatility are perhaps the most studied economic variables, and
%our theoretical setup can be extended to individual equities.
%Building such a model
%involves introducing
%idiosyncratic risks and possibly unspanned risks in the dynamics of individual equity prices.\footnote{\citet*[Table 5]{SophieNi:2008} documents unconditional returns to
%calls on \emph{individual stocks}. Over the 1996 to 2005 sample, she finds that the grand average (across individual stocks)
%of call returns tend to decrease with higher moneyness.} Our theoretical framework can also be modified to address questions related to risk premiums and features of option returns on commodity futures and VIX futures.
%Such extensions are worthy of scrutiny
%but left to another study.]]}
%

%Among the various choices,
%the total individual return volatility can be ascribed to spanned risks, unspanned risks, and idiosyncratic
%risks. This depiction of individual equity volatility adds new layers of
%complexity, but may enable a better understanding of individual
%equities and their option returns.


%{\color{red}[[[[[[[[[[
%% This has lead to
%%%vibrant literature on
%%semimartingale models of equity prices (e.g., processes with or without discontinuities and
%%stochastic volatility)  that have energized  theory and practice of trading equity and their derivatives.
%%%and on understanding and depicting volatility.
%
%The question that we address in this paper is the following: What can be gained by introducing unspanned risks
%in equity and volatility dynamics?
%%and to quantifying their pricing impacts?
%If so, are unspanned risks economically and empirically pertinent, and how does one go about measuring and identifying their effects from asset prices?
%]]]]]]]]}



%{\color{blue} Then, for small $T_{O}$, we can specialize (\ref{eq:StraddleInterim1Jumpsemp})
%as follows
%\begin{eqnarray}
%\overbrace{1 + \mu^{T_{O}}_{t,{\tiny \mathrm{straddle}}} - e^{r (T_{O}-t)}}^{\mathrm{straddle~risk~premium}}&=& \frac{1}{\mathbb{E}_{t}^{\mathbb{Q}}( \mathbb{A}_t^{{T}_O}[1] )} ~
%\underbrace{(\mathbb{E}_{t}^{\mathbb{P}}( \mathbb{A}_t^{{T}_O}[1] ~-~\mathbb{E}_{t}^{\mathbb{Q}}( \mathbb{A}_t^{{T}_O}[1] )}_{\underset{\color{blue} \mathrm{strike~from~below~and~above}~k=1}{\tiny \mbox{risk~premium~for~jumps~crossing~the}}} \}.~~~~~\mathrm{for~small}~~T_{O}
%\label{eq:StraddleInterim1Jumpsemp}
%\end{eqnarray}
%}
%




%\footnote{----- {\color{green} move later} Our setup is different from
%the
%%continuous-time
%models of
%%{\color{red}[[[ \citet*{Merton:73}, ]]]}
%\citet*{HullWhite:87},
%\citet*{CarrJarrow:90},
%\citet*{Stein_Stein:RFS1991},
%\citet*{Heston:1993},
%%{\color{green}\citet*{AitSahaliaLo:2000},}
%%\citet*{BakshiCaoChen:97},
%\citet*{Bates:2000},
%\citet*{BrittenJonesNeuberger:2000JF},
%\citet*{DuffiePanSingleton:2000},
%\citet*{AndersenBollerslevDieboldLabys:ecma2003},
%\citet*{CarrWu:2009RFS_a},
%\citet*{Song_Xiu:JOE2016}, \citet*{Amengual_Xiu:JOE2018}, and \citet*{Aitsahalia_Karaman_Mancini:JOE2019}?}



%While we have focussed on equity index dynamics with no unspanned risks, our theoretical setting is versatile and
%can be extended to individual stocks, commodities, and VIX.  In this vein, consider some general asset with price $A_t$. Then,
%one way to extend our framework is to postulate the following system of stochastic differential equations (in addition to that for
%$M_t$):
%\begin{eqnarray}
%\frac{d A_{t}}{A_{t}} & =& \mu_S[t,\mathbf{X}]\, dt
%~+~ \mathbf{H}[t,\mathbf{X}]^{'}
%\underbrace{d \mathbf{z}^{\mathbb{P}}_t}_{\text{\tiny{spanned}}}
%~+~ \mathbf{N}[t,\mathbf{X}]^{'}
%\underbrace{d \mathbf{u}^{\mathbb{P}}_t}_{\text{\tiny{unspanned}}},
%~~~\mathrm{with} \label{eq:con1} \\
%d \mathbf{H}_t[t,\mathbf{X}] & = & {\bm \mu}_H[t,\mathbf{X}] \,dt
%~+~ {\bm \sigma}_{H,z}[t,\mathbf{X}]^{'} \underbrace{d \mathbf{z}^{\mathbb{P}}_t}_{\text{\tiny{spanned}}}
%~+~ {\bm \sigma}_{H,u}[t,\mathbf{X}]^{'}  \underbrace{d \mathbf{u}^{\mathbb{P}}_t}_{\text{\tiny{unspanned}}},
%\label{eq:con2} \\
%d \mathbf{N}_t[t,\mathbf{X}] & = & {\bm \mu}_N[t,\mathbf{X}] \,dt
%~+~ {\bm \sigma}_{N,z}[t,\mathbf{X}]^{'} \underbrace{d \mathbf{z}^{\mathbb{P}}_t}_{\text{\tiny{spanned}}}
%~+~ {\bm \sigma}_{N,u}[t,\mathbf{X}]^{'}  \underbrace{d \mathbf{u}^{\mathbb{P}}_t}_{\text{\tiny{unspanned}}}.
%\label{eq:con3}
%\end{eqnarray}
%This framework, while more complicated, may be a more gratifying way to address outstanding questions in the markets for individual stocks,
%commodities, and the VIX.

%While we have focussed on equity index dynamics with no unspanned risks, our theoretical setting is versatile and
%can be extended to individual stocks, commodities, and VIX.


\subsection{Collecting Dou}
{\color{blue}\citet*{Cheng_Dou_Liao:ECMTA2021}} introduce robust
evaluation procedures for estimating macro-finance
models.
They dichotomize between
``baseline" moments
and ``asset pricing" moments
in estimating model parameters
that are only weakly identified by ``baseline" moments.
Our investigation is different to
but complementary to theirs.
In particular, we show that matching
asset pricing moments (Euler equations) involving
the expected excess return of options allows
one to infer the presence of ``dark matter".
Specifically, the ``dark matter"
in our paper is unspanned risks in the SDF,
and without the use of option returns, estimating
this dark matter is futile.

\subsection{Collecting R1}

Thus, while our use of the ``dark matter" phraseology
is somewhat different to that
of {\color{blue}\citet*{Cheng_Dou_Liao:ECMTA2021}}
and
\citet*{Chen_Dou_Kogan:JF2020}. Our usage
is consistent with the notion from astronomy:
An object which must exist in abundance
but whose existence can only be indirectly inferred.
Our central tenet is that
the pricing kernel must contain
risks not spanned by (i.e., are independent of)
equity index returns but which affect
option risk premiums. These unspanned risks
are typically assumed away in extant macro-finance models.

\subsection{Bakshi: 10/17/2021 Dou para for introduction}

{\color{red} [[[{\color{blue}\citet*{Cheng_Dou_Liao:ECMTA2021}} introduce robust
evaluation procedures for estimating macro-finance
models.
They dichotomize between
``baseline" moments
and ``asset pricing" moments
in estimating model parameters
that are only weakly identified by ``baseline" moments.
Our investigation is different to
but complementary to theirs.
In particular, we show that matching
asset pricing moments (Euler equations) involving
the expected excess return of options allows
one to infer the presence of ``dark matter".
Specifically, the ``dark matter"
in our paper is unspanned risks in the SDF,
and without the use of option returns, estimating
this dark matter is futile.]]]]}




{\color{red} [[[[[[[[[[[[[[[[
\subsection{Collecting Dou first paper}


\citet*{Chen_Dou_Kogan:JF2020} introduce a measure of
\emph{dark matter} which captures the degree
of fragility for models that are potentially
misspecified and unstable: A large dark matter
measure signifies a model's lack of internal refutability (weak power of specification tests)
and high overfitting tendency.
Even in a simple illustrative Gordon dividend growth setting, they show that adding
the well-known model implications for the
price-dividend ratio results in an estimation
of the dividend growth rate which has 95\% confidence
levels so wide as to mean that the form
of the price-dividend ratio can almost never be rejected
by the data. They explain that
% (in a similar fashion to weak identification)
this is, in essence, a problem with the model rather than
a finite-sample problem.
\citet*{Chen_Dou_Kogan:JF2020} show that the same issue rears its head in the
rare disasters model. The research
question that we seek to answer has similarities
albeit we attack the problem from the opposite
direction. We introduce options. We show that
models of SDFs that do not incorporate a
role for unspanned risks will (and must) be
rejected by options data because they
cannot match the negative expected
excess returns of deep OTM call options observed
empirically.]]]]]]]]]]]]]]]]]]}




%%%%%%%%%%%%%%%%%%%%%%%
{\color{red}[[ beginning of new --- added 10th October 2021 ]]}
%%%%%%%%%%%%%%%%%%%%%%%
%%%%%%%%%%%%%%%%%%%%%%%

\subsection{Jones}

{\color{blue}
\citet*{Jonesc:2006}
considers
{\color{red}[[ referee asks about \citet*{Jonesc:2006} -- this is what I wrote.  ----  10th October 2021 ]]}
both linear and non-linear
factor models of index options returns, with the factors including the returns of the S\&P~500 index, changes in the VIX and changes in interest-rates.
However, as he himself emphasizes,
even multi-factor non-linear models
are still insufficient to explain the magnitudes of the
observed excess returns, particularly for short-term out-of-the-money puts. Seen through the prism
of our methodology, we make two comments about the
\citet*{Jonesc:2006} factor model. First, it
does not emphasize jumps in the equity index.
Second, to the extent that it incorporates
the returns of the S\&P~500 index as a factor,
it does not differentiate between
diffusive returns and returns generated
by large price discontinuities.

\subsection{Internet Appendix}

While our approach is non-parametric,
one can use popular parametric
models to sharpen intuition and to guide
future model design.
In the Internet Appendix, we consider such models
and highlight the potential sources
of risk premiums for local time and
jumps across the strikes.

\subsection{Merton and Kou}

Stochastic volatility models (such as \citet*{Heston:1993}), when combined with unspanned risks in the pricing kernel,
can capture the former.
Models with jumps (but constant
diffusive volatility) such as those of
\citet*{Merton:76} and
\citet*{Kou:2002} cannot capture the former but can capture the latter provided that
contemporaneous jumps appear
non-trivially in the pricing kernel
(i.e., the index return jumps are priced -- the Internet Appendix provides details).

\subsection{Bates**** Junk}

{\color{red} [[[[[[[[[[The %\citet*{Bates:96}
Bates  model combines
%\citet*{Heston:1993}
Heston stochastic volatility
and \citet*{Merton:76} jumps. Thus,
provided there are unspanned risks
in the pricing kernel, it can capture
a risk premium for both local time
and for jumps across the strike.
The ``double jump" model
of \citet*{DuffiePanSingleton:2000} additionally
allows for jumps in the equity return variance.
Our analysis (in the Internet Appendix) shows
that allowing for jumps in the variance will have little
or no additional ability (over and above
the Bates:96 model)
for explaining risk
premiums. Of course, jumps in the variance may allow
for better capturing the
dynamics of the implied volatilty skew.]]]]]]]]]]]]]}


%
%OR: Point 3 of R2: Can you sketch a coherent answer
%
%It would be very useful to numerically explain why some specic option pricing models with the volatility and jump risk premia can or cannot fully rationalize the expected returns of relevant option portfolios through the lens of the decomposition (i.e., the framework of local time and jumps crossing the strike) proposed by this paper. For example, Jones (2006) emphasizes that priced volatility and jump risk contribute to the expected excess returns of the OTM option portfolios and the ATM straddles, but they are insu cient to explain the magnitudes of the observed excess returns, particularly for short-term out-of-the-money puts. Can this paper help explicitly identify what are missing?

%----
%
%Date 31st August
%
%Possible removal:
%
%p 25: Section IV and Lemma 1 should be removed. It will save a small amount of space. Equation (A11) is not quite right anymore.
%
%
%Remove dispersion uncertainty. Will remove 2 1/8 pages in appendix and 1/4 page on page 10.
%
%p 14: Remove (this is already in the table)
%The superscripts ***, **, and * on estimates indicate statistical ..... with lag selected automatically.
%
%------
%Date August 27th 2021

\subsection{Dou again}

Our use of the ``dark matter" phraseology
is a little different from that
of {\color{blue}\citet*{Cheng_Dou_Liao:ECMTA2021}}
and
\citet*{Chen_Dou_Kogan:JF2020} -- but still consistent with the notion from astronomy:
An object which must exist in abundance
but whose existence can only be indirectly inferred.

\citet*{Chen_Dou_Kogan:JF2020} quantify
additional information that can be obtained about the dynamics of economic variables via the use of implied asset pricing restrictions.
We use the returns of equity index options
to identify the necessity of components
in the pricing kernel that would be missed
if option returns were to be ignored.
Our central tenet is that
the pricing kernel must contain
risks not spanned by (i.e., are independent of)
equity index returns but which affect
option risk premiums. These unspanned risks
are typically assumed away in extant macro-finance models.

Thus, our usage of the ``dark matter" phraseology
is complementary to that
of {\color{blue}\citet*{Cheng_Dou_Liao:ECMTA2021}}
and
\citet*{Chen_Dou_Kogan:JF2020}.


---------
Thus, while our use of the ``dark matter" phraseology
is somewhat different to that
of {\color{blue}\citet*{Cheng_Dou_Liao:ECMTA2021}}
and
\citet*{Chen_Dou_Kogan:JF2020}. Our usage
is consistent with the notion from astronomy:
An object which must exist in abundance
but whose existence can only be indirectly inferred.
Our central tenet is that
the pricing kernel must contain
risks not spanned by (i.e., are independent of)
equity index returns but which affect
option risk premiums. These unspanned risks
are typically assumed away in extant macro-finance models.



--------------

{\color{blue}\citet*{Cheng_Dou_Liao:ECMTA2021}} introduce robust
evaluation procedures for estimating macro-finance
models.
They dichotomize between
``baseline" moments
and ``asset pricing" moments
in estimating model parameters
that are only weakly identified by ``baseline" moments.
Our investigation is different to
but complementary to theirs.
In particular, we show that matching
asset pricing moments (Euler equations) involving
the expected excess return of options allows
one to infer the presence of ``dark matter".
Specifically, the ``dark matter"
in our paper is unspanned risks in the SDF,
and without the use of option returns, estimating
this dark matter is futile.

} % end of color blue

%
%\citet*{Chen_Dou_Kogan:JF2020} introduce a measure of
%\emph{dark matter} which captures the degree
%of fragility for models that are potentially
%misspecified and unstable: A large dark matter
%measure signifies a model's lack of internal refutability (weak power of specification tests)
%and high overfitting tendency.
%Even in a simple illustrative Gordon dividend growth setting, they show that adding
%the well-known model implications for the
%price-dividend ratio results in an estimation
%of the dividend growth rate which has 95\% confidence
%levels so wide as to mean that the form
%of the price-dividend ratio can almost never be rejected
%by the data. They explain that
%% (in a similar fashion to weak identification)
%this is, in essence, a problem with the model rather than
%a finite-sample problem.
%\citet*{Chen_Dou_Kogan:JF2020} show that the same issue rears its head in the
%rare disasters model (of, for example, \citet{Wachter:2013disaster_JF}).
%The research
%question that we seek to answer has similarities
%albeit we attack the problem from the opposite
%direction. We introduce options. We show that
%models of SDFs that do not incorporate a
%role for unspanned risks will (and must) be
%rejected by options data because they
%cannot match the negative expected
%excess returns of deep OTM call options observed
%empirically.



%%%%%%%%%%%%%%%%%%%%%%%
%%%%%%%%%%%%%%%%%%%%%%%
%%%%%%%%%%%%%%%%%%%%%%%
{\color{red}[[ end of new  ----  10th October 2021 ]]}
%%%%%%%%%%%%%%%%%%%%%%%
%%%%%%%%%%%%%%%%%%%%%%%
%%%%%%%%%%%%%%%%%%%%%%%


\subsection{Xiaohui notes 2021-09-21}

R1, point 3: The literature recognizes that deep OTM call options may also be affected by other factors, such as liquidity considerations, heterogeneous agent trading, etc. All these can impact the average option returns, and specifically, the finding of a negative call premium. The authors may want to provide a discussion of the relevance of these factors for their risk-based analysis.



Could liquidity considerations and heterogeneous agent trading impact the observed average option returns, particularly the deep OTM calls? We address this issue from two angles. First, we tabulate the open interest and trading volume of the OTM puts and calls, all observed on the first day of the expiration cycle. Overall, deep OTM calls do not appear to suffer from significantly lower open interest and thin trading volume.



Second, we compute the returns using the mid price instead of the ask price. We observe that all return patterns are retained. Both exercise mitigate the concern that liquidity plays a significant role in the observed option returns.



2021-09-19



Summary of work so far:

1)      R2 comment: add crash neutral straddle return (with 3\% put for weekly,
5\% put for monthly, and 13\% put for 3 month), tables in Excel ready.

2)      R1 comment: address the liquidity concern by

tabulate open interest (number of contracts, not dollar amount) and trading volume of otm options
redo all the exercise with mid price and show that the pattern retains
can cite papers on the liquidity issue
3)      Editor comment: bootstrap: we did the bootstrap for IID, stationary, circular, and wild. We need to figure out the table format to parsimonious present the bootstrap results.

4)      Editor comment: 2 day and 3 day returns of weekly options, returns are overall more negative, but the overall pattern retains.

\subsection{Crash-neutral straddles}

To construct the return of crash neutral straddles, we implement the following strategy
\begin{enumerate}

\item {\color{blue}Buy an ATM straddle.}
\item {\color{blue}Sell a 5\% OTM put.}
\end{enumerate}
The excess return has two components. First,
{\color{red} [[ Rejigged eqn -- changed $I_t$ to $S_t$]]]}
{\color{blue}
\begin{equation}
\mathrm{Denominator} ~ = ~ \mathrm{call}_t[S_t]  + \mathrm{put}_t[S_t] + \underbrace{\mathbb{C}_t.}_{{\mathrm{Collateral}}} ~~
\end{equation}
With $K = S_t e^{-0.05}$, the strike of the 5\% OTM put, {\color{blue}we have that the collateral term $\mathbb{C}_t[K]$ is
\begin{equation}
\mathbb{C}_t[K] ~ = ~ \mathrm{put}_t[K] \, + \, \max (0.1 \, K, 0.15 \, S_t - \max(S_t-K,0)). ~~~ \mbox{ \, \, }
\end{equation}
The numerator is {\color{red} [[ make risk-free rate notation consistent
with the main paper. ]]]}
\begin{equation}
\mathrm{Numerator} \, = ~ \, \overbrace{~ \lvert S_{T_{O}} - S_t \rvert ~}^{\text{\tiny ATM~straddle}} ~ - \, \overbrace{~ \max( K - S_{T_{O}},0 ) ~}^{\text{\tiny $5\%$~OTM~put}} \, \, ~ + \, \, e^{r (T_{O} - t)} \, ( \mathbb{C}_t[K] + \mathrm{put}_t[K]). ~ ~ ~ \mbox{ \, \, \, \, \, \, }
\end{equation}
The excess return of {\color{blue}the crash-neutral} {\color{red}[[ crash neutral ]]]} straddle is
{\color{red} [[ make return notation consistent
with the main paper. ]]]}
\begin{equation}
q_{t}^{T_{O} \, \text{\tiny crash-neutral~straddle}} ~ = ~ \,  \frac{\mathrm{Numerator}}{\mathrm{Denominator~}} \, \, - \, e^{r (T_{O} - t)}. ~ \mbox{ \, \, \, \, \, \, }
\end{equation}

{\color{blue}The results are reported in
{\color{green}Table xxxx.}
The expected excess returns of crash-neutral straddles
are typically close to zero. Crash-neutral straddles
are composed of a long ATM straddle and a short OTM put. We have already seen that ATM straddles and OTM puts
are each consistent with a negative dark matter risk premium (albeit at
different strikes). Thus, the finding that
the expected excess returns of crash-neutral straddles
are typically close to zero is consistent with the view that
the dark matter risk premium is consistently negative
across strikes.}


The results are reported in {\color{green}Table xxxx.}
The expected excess returns of crash-neutral straddles are close to zero and not statistically significant. Crash-neutral straddles are composed of a long ATM straddle and a short OTM put. We have already seen that ATM straddles and OTM puts are each consistent with a negative dark matter risk premium. Thus, the finding that the expected excess returns of crash-neutral straddles are close to zero is consistent with the view that the dark matter risk premium is negative across the relevant strikes and that jumps across the strike are a pertinent component of dark matter.





%%%%%%%%%%%%%%%%%%%%%%%%%%%%%%%%%%%%%% END %%%%%%%%%%%%%%%%%%%%%%%%%%%%%%%%%%%



\end{document}

%%%%%%%%%%%%%%%%%%%%%%%%%%%%%%%%%%%%%%%%%%%%%%%%%%%%%%%%%%%%%%%%%%%%%%%%%%%%%%

\end{document}





%%%%%%%%%%%%%%%%%%%%%%%%%%%%%%%%%%%%%%%%%%%%%%%%%%%%%%%%%%%%%%%%%%%%%%%%%%%%%%
%%%%%%%%%%%%%%%%%%%%%%%%%%%%%%%%%%%%%% END %%%%%%%%%%%%%%%%%%%%%%%%%%%%%%%%%%%

%%%%%%%%%%%%%%%%%%%%%%%%%%%%%%%%%%%%%% END %%%%%%%%%%%%%%%%%%%%%%%%%%%%%%%%%%%
%\end{document}
\newpage
%==*==*==*==*==*==*==*==*==*==*==*==*==*==*==*==*==*== Tables
 %\setcounter{table}{0}  % reset counter

%\renewcommand{\thetable}{\arabic{table}}

 %\setcounter{figure}{0}  % reset counter

%\renewcommand{\thefigure}{\arabic{figure}}

%  \renewcommand{\theequation}{\arabic{equation}}
%  \setcounter{equation}{0}  % reset counter
%\renewcommand{\thetheorem}{\arabic{theorem}}
%  \setcounter{theorem}{0}  % reset counter

\newpage
\thispagestyle{empty}
\clearpage



\begin{center}
{\Large{Dark Matter in (Volatility and) Equity Option Risk Premiums}} \\
\vspace{0.04in}
%Gurdip Bakshi~~~John Crosby~~~Xiaohui Gao \\
\textbf{\underline{Internet Appendix: Not for Publication}}
\end{center}
%\vspace{1mm}
\begin{center}
\textbf{Abstract}
\end{center}

\noindent {\color{blue} This Internet Appendix focuses on the continuous semimartingale theoretical environment (i.e.,
the jumps crossing the strike terms --- $a_t^{T_O}[k]$, $b_t^{T_O}[k]$, $c_t^{T_O}[k]$, and $d_t^{T_O}[k]$ --- are all zero).
Corollary~\ref{claimm:claim1call} is about option risk premiums when there are unspanned diffusive risks in the dynamics of the pricing kernel
and the volatility. Corollary~\ref{claimm:SV} shows that a suitably motivated stochastic volatility model (under the $\mathbb{P}$ measure)
can synthesize negative call risk premiums provided that certain restrictions are imposed on unspanned risks. Finally,
Lemma~\ref{eq:lemmon} shows that if unspanned risks  are irrelevant, then the local time risk premium is zero.}


%\newpage
\thispagestyle{empty}
%\clearpage

\thispagestyle{empty}
% *************** start of text ****************************************
\newpage
\setcounter{page}{1}
\renewcommand{\thefootnote}{\arabic{footnote}}
\setcounter{footnote}{0}

%\setcounter{equation}{0}
%\renewcommand{\theequation}{B\arabic{equation}}

\setcounter{section}{0}
\renewcommand{\thesection}{\Roman{section}}
%\renewcommand{\thesection}{\Roman{section}}
%\renewcommand{\thesubsection}{\thesection.\Roman{subsection}}
\renewcommand{\thesubsection}{\thesection.\arabic{subsection}}
%\renewcommand{\thesubsection}{\Roman{subsection}}

%\section{ \bf \large Internet Appendix}

                                                        \setcounter{equation}{0}
                                                        \renewcommand{\theequation}{IA-\arabic{equation}}
                                                        %\setcounter{section}{0}
%\numberwithin{equation}{section}
\numberwithin{table}{section}
\numberwithin{theorem}{section}
\numberwithin{figure}{section}
%                                            \numberwithin{theorem}{subsection}

%%%%%%%%%%%%%%%%%%%%%%%%%%%%%%%%%%%%%%%% Begin I %%%%%%%%%%%%%%%%%%%%%%%%%%%%%%%%%%%%%%%%%%%%%%%%%%%%%
%%%%%%%%%%%%%%%%%%%%%%%%%%%%%%%%%%%%%%%%%%%%%%%%%%%%%%%%%%%%%%%%
                                                        \setcounter{equation}{0}
                                                        \renewcommand{\theequation}{I\arabic{equation}}
%\begin{center}
%\textbf{Internet Appendix}
%\end{center}
%%%%%%%%%%%%%%%%%%%%%%%%%%%%%%%%%%%%%%%%%%%%%%%%%%%%%%%%%%%%%%%%%%%%%%%%%%%%%%%%%%%%%%%%%%%%%%%%%%%%%%%%%%%%
%%%%%%%%%%%%%%%%%%%%%%%%%%%%%%%%%%%%%%%%%%%%%%%%%%%%%%%%%%%%%%%%%%%%%%%%%%%%%%%%%%%%%%%%%%%%%%%%%%%%%%%%%%%%
\section{Unspanned risks in a continuous semimartingale setting}
\label{seimimartingales_continuous}

{\color{magenta} In a continuous semimartingale setting, the jumps crossing the strike terms, namely,
$a_t^{T_O}[k]$, $b_t^{T_O}[k]$, $c_t^{T_O}[k]$, and $d_t^{T_O}[k]$, vanish.} \vspace{-3mm}

%{\color{green} A continuous semimartingale theoretical environment can be revealing for three reasons.
%First, the jumps crossing the strike terms ---
%$a_t^{T_O}[k]$, $b_t^{T_O}[k]$, $c_t^{T_O}[k]$, and $d_t^{T_O}[k]$ --- \emph{vanish}.
%Second, one can delineate the distinction between spanned and unspanned \emph{diffusive} risks. Third, the risk premium adjustments that
%link $\mathbb{P}$ to $\mathbb{Q}$ are explicit through Girsanov's change of measure theorem.} \vspace{-3mm}

\subsection{Implications of local time being the only source of dark matter}
\label{gggs}

In what follows, we dichotomize between spanned and unspanned risks in the following manner. \vspace{-2mm}
\begin{align}
&\mathrm{Let}~\mathbf{z}^{\mathbb{P}}_t~\mbox{denote a vector of independent standard Brownian motions under}~\mathbb{P}.& \\
&\mathrm{Additionally,}~\mathbf{u}^{\mathbb{P}}_t~\mbox{is another vector of independent standard Brownian motions under}~\mathbb{P}. \mbox{ \, } &
\end{align}

\noindent \textbf{Model.} By assumption, $\mathbf{z}^{\mathbb{P}}_t$ is  spanned,
while $\mathbf{u}^{\mathbb{P}}_t$ \emph{cannot} be spanned by equity futures.
%{\color{blue} Risks not spanned by equity futures could be spanned by options written on the equity futures
%(or on the equity).}
With the vector of state variables denoted by $\mathbf{Y}_t$,
consider the system of stochastic differential equations (SDEs) for
the pricing kernel $M_t$ and for the equity index $S_{t}$, as follows:
\begin{eqnarray}
\frac{d M_t}{M_t} & = & -r\, dt
~+~{\bm\eta}[t,\mathbf{Y}_t]^{\top} \, \underbrace{d \mathbf{z}^{\mathbb{P}}_t}_{\text{\tiny{spanned~risks}}}
~+~{\bm\theta}[t,\mathbf{Y}_t]^{\top} \, \underbrace{d \mathbf{u}^{\mathbb{P}}_t,}_{\text{\tiny{unspanned~risks}}}
\mbox{ \, \, \, }
\label{eq:GeneralDynamics1} \\
%r&=&\mathrm{spot~interest~rate,~assumed~constant,}~ \\
\frac{d S_{t}}{S_{t}} & =&
(r ~-~ \underbrace{{\bm \eta}[t,\mathbf{Y}_t]^{\top} \mathbf{V}[t,\mathbf{Y}_t]}_{=~\mathrm{cov}^{\mathbb{P}}_t( \frac{dM_t}{M_t}, \frac{d S_t}{S_t})/dt})
\, dt ~+~ \mathbf{V}[t,\mathbf{Y}_t]^{\top}
\underbrace{d \mathbf{z}^{\mathbb{P}}_t,}_{\text{\tiny{spanned~risks}}} \label{eq:index3} \\
\underbrace{d \,\mathbf{V}[t,\mathbf{Y}_t]}_{\tiny \mathrm{volatility~dynamics}} & = & {\bm \mu}_V[t,\mathbf{Y}_t] \,dt
~+~ \underbrace{{\bm \sigma}_{V,z}[t,\mathbf{Y}_t]}_{\mathrm{matrix}} \underbrace{d \mathbf{z}^{\mathbb{P}}_t}_{\text{\tiny{spanned~risks}}}
~+~ \underbrace{{\bm \sigma}_{V,u}[t,\mathbf{Y}_t]}_{\mathrm{matrix}}  \underbrace{d \mathbf{u}^{\mathbb{P}}_t,}_{\text{\tiny{unspanned~risks}}}
%\mathrm{and}
%~~ ~
%~\mbox{and}
\label{eq:index3VolDy} \\
\underbrace{\frac{dF_{t}^{T_F}}{F_{t}^{T_F}}}_{\tiny \mbox{using (\ref{fuut})}} & = &
-{\bm \eta}[t,\mathbf{Y}_t]^{\top} \mathbf{V}[t,\mathbf{Y}_t]
 \,dt
~+~ \mathbf{V}[t,\mathbf{Y}_t]^{\top} \underbrace{d \mathbf{z}^{\mathbb{P}}_t,}_{\mathrm{spanned~risks}}
\label{eq:FutP}
\end{eqnarray}
where $\mathbf{V}[t,\mathbf{Y}_t]$ is a vector conformable
with $\mathbf{z}^{\mathbb{P}}_t$. The notation ${\top}$ represents transpose of a vector. The dynamics of $\frac{d G_\ell}{G_\ell}$ coincides with those of $\frac{d F_{\ell}^{T_F}}{F_{\ell}^{T_F}}$ for all $\ell$ satisfying $t \leq \ell \leq T_F$.

Our differentiating element
is that the standard Brownian motions $\mathbf{u}^{\mathbb{P}}_t$ are present
in the SDEs for $M_t$ and for $\mathbf{V}[t,\mathbf{Y}_t]$.
The introduction of $\mathbf{u}^{\mathbb{P}}_t$ is akin to a form of market incompleteness,
and our treatment of $\mathbf{u}^{\mathbb{P}}_t$, and its risk compensation, is pertinent to our theoretical analysis,
empirical identifications, and the basis of what we call ``dark matter."

By assumption, ${\bm\eta}[t,\mathbf{Y}_t]^{\top} d \mathbf{z}^{\mathbb{P}}_t$ is spanned by $\mathbf{V}[t,\mathbf{Y}_t]^{\top} d \mathbf{z}^{\mathbb{P}}_t$. Thus, it must hold that
\begin{eqnarray}
{\bm\eta}[t,\mathbf{Y}_t]^{\top} \, {\bm\eta}[t,\mathbf{Y}_t] & = & \mathfrak{g}_t \, \mathbf{V}[t,\mathbf{Y}_t]^{\top} \, \mathbf{V}[t,\mathbf{Y}_t], ~~ \mbox{ \, \, \, for all $t$, \, \, \, for some scalar variable $\mathfrak{g}_t$. \, \, \, \, } ~ ~ \label{eq:RestrictionOnEta}
\end{eqnarray}

The feature that the volatility
dynamics (i.e., those of $\mathbf{V}[t,\mathbf{Y}_t]$) contains
unspanned risks --- the randomness that cannot be removed
by trading in equity futures
--- is data-motivated and essential to
our developments. The economic effect of $\mathbf{u}^{\mathbb{P}}_t$
is notable, because in its absence, the risk premium on local time
would
be \emph{zero} in
this
continuous semimartingale setting
(soon to be formalized).\footnote{Since Brownian shocks are amenable to ``rotation," one may ask:
Could one have an alternative but equivalent representation, in which $\mathbf{u}^{\mathbb{P}}_t$ is a part of
$S_t$ and $M_t$ dynamics, but not of
volatility?
Our \emph{definition} of unspanned risks aligns with a notion that $\mathbf{u}^{\mathbb{P}}_t$
appear in the dynamics of the latent variables (i.e., in $M_t$ and in $\mathbf{V}[t,\mathbf{Y}_t]$).}
This potentially separates us from
other studies on equity volatility.

In equations (\ref{eq:GeneralDynamics1})--(\ref{eq:FutP}),
the drift and diffusion coefficients may
depend upon
$\mathbf{Y}_t$
and
are adapted to $\mathcal{F}_t$.
At this stage, we do not specify which
economic
variables enter
$\mathbf{Y}_t$.
In general, the dynamics of $\mathbf{Y}_t$
will
impact the drift (under $\mathbb{P}$) and diffusions of $\frac{d S_{t}}{S_{t}}$ and the form of risk compensation associated with
the spanned and unspanned components of
the pricing kernel.\footnote{Additionally, we assume that the drift and diffusion coefficients
are differentiable so that Ito's lemma can be applied, and
they are sufficiently regular so that the SDEs
have a unique solution. In particular,
the vector
${\bm \mu}_V[t,\mathbf{Y}_t]$, and conformable matrices
${\bm \sigma}_{V,z}[t,\mathbf{Y}_t]$ and ${\bm \sigma}_{V,u}[t,\mathbf{Y}_t]$
in (\ref{eq:index3VolDy}) must be such that elements of $\mathbf{V}[t,\mathbf{Y}_t]$ are nonnegative.
See \citet*[pages 364--366]{cir:85a} for the regularity conditions on the SDEs, including that
the covariance matrices be nonnegative definite.
Furthermore, we
preclude that
$\frac{dS_t}{S_t}$ is perfectly
correlated with increments to its variance.
{\color{green} The work of \citet*{Bates:2000} provides the context
for a two-factor model of return volatility.}}

In light of Girsanov's theorem,
$\mathbf{z}^{\mathbb{Q}}_t$ and $\mathbf{u}^{\mathbb{Q}}_t$ are
vectors of independent standard Brownian motions under the probability measure $\mathbb{Q}$,
linked to $\mathbf{z}^{\mathbb{P}}_t$ and $\mathbf{u}^{\mathbb{P}}_t$, by
\begin{align}
&d \mathbf{z}^{\mathbb{P}}_t~-~d \mathbf{z}^{\mathbb{Q}}_t = {\bm \eta}[t,\mathbf{Y}_t] \,dt&
&\mathrm{and}&
&d \mathbf{u}^{\mathbb{P}}_t~-~d \mathbf{u}^{\mathbb{Q}}_t =  {\bm \theta}[t,\mathbf{Y}_t] \, dt.&
\label{fg.1}
\end{align}
The dynamics of $\frac{d G_\ell}{G_\ell} = \frac{d F_{\ell}^{T_F}}{F_{\ell}^{T_F}}$ under $\mathbb{Q}$,
from (\ref{eq:FutP}), becomes $\frac{d G_\ell}{G_\ell} = \mathbf{V}[\ell,\mathbf{Y}_{\ell}]^{\top} d \mathbf{z}^{\mathbb{Q}}_{\ell}.$

\setcounter{theorem}{0}
\begin{corollary}[Continuous semimartingales]
\label{claimm:claim1call}
The following are true:
\begin{enumerate}
\item The OTM call risk premium %($k>1$)
\emph{can} be negative only if $\mathbb{E}^{\mathbb{P}}_{t}( \mathbb{L}^{T_O}_t[k] ) - \mathbb{E}^{\mathbb{Q}}_{t}( \mathbb{L}^{T_O}_t[k] ) <0$.
It is
positive if
$\mathbb{E}_{t}^{\mathbb{P}}( \int_{t}^{{T}_O} \mathbbm{1}_{\{G_{\ell} > k\}} \,dG_{\ell} ) >
- \{\mathbb{E}_{t}^{\mathbb{P}}( \mathbb{L}^{T_O}_t[k] )
-  \mathbb{E}_{t}^{\mathbb{Q}}( \mathbb{L}^{T_O}_t[k] ) \}$.

\item The {\color{magenta} straddle risk premium is zero (respectively, negative) if, and only if, the local time risk premium
(and hence the dark matter risk premium) for
$k=1$ is zero (respectively, negative).}

\end{enumerate}
\end{corollary}

\noindent {\bf Proof:} We specialize Theorem~\ref{claimm:claim1call_jump} to a continuous semimartingale setting (see Appendix~\ref{appsec:jumppps}). $\blacksquare$



\normalsize

Our results on call risk premium
are introduced
without parameterizing the diffusion or drift coefficients
of $M_t$ and $F^{T_F}_t$. In so doing, we highlight the mechanism of unspanned risks,
and the rationale that a negative local time (and, hence, dark matter) risk premium
could help to understand
puzzling data
features in the equity markets.
%{\color{red}[[[[ Our setup suggests the value of refining extant modeling frameworks to include unspanned
%risks in the equity  volatility dynamics. ]]]]}

%\begin{corollary}[Continuous semimartingales]%Expected excess returns of OTM calls ($k>1$)]
%\label{claimm:claim1call}
%The OTM call risk premium %($k>1$)
%%, $1 + \mu^{{T}_O}_{t,{\tiny \mathrm{call}}}[k] - e^{r (T_O-t)}$,
%\emph{can} be negative only if $\mathbb{E}^{\mathbb{P}}_{t}( \mathbb{L}^{T_O}_t[k] ) - \mathbb{E}^{\mathbb{Q}}_{t}( \mathbb{L}^{T_O}_t[k] ) <0$.
%It is
%positive if
%$\mathbb{E}_{t}^{\mathbb{P}}( \int_{t}^{{T}_O} \mathbbm{1}_{\{G_{\ell} > k\}} \,dG_{\ell} ) >
%- \{\mathbb{E}_{t}^{\mathbb{P}}( \mathbb{L}^{T_O}_t[k] )
%-  \mathbb{E}_{t}^{\mathbb{Q}}( \mathbb{L}^{T_O}_t[k] ) \}$. %\vspace{-2mm}
%\end{corollary}
%\noindent {\bf Proof:} We specialize Theorem~\ref{claimm:claim1call_jump} to a continuous semimartingale setting (see Appendix~\ref{appsec:jumppps}). $\blacksquare$


Corollary~\ref{claimm:claim1call}
can be traced
to nontrivial contributions of
${\bm\theta}[t,\mathbf{Y}_t]^{\top} \, d \mathbf{u}^{\mathbb{P}}_t$
(i.e., $M_t$
has unspanned risks) and
${\bm \sigma}_{V,u}[t,\mathbf{Y}_t] \,d \mathbf{u}^{\mathbb{P}}_t$
(i.e., volatility has unspanned risks).
Implicit
is an insight that the local time risk premium
for any
$k$ is zero only if
${\bm\theta}[t,\mathbf{Y}_t] = {\bf 0}$
(i.e., $M_t$
does not contain unspanned risks) or
${\bm \sigma}_{V,u}[t,\mathbf{Y}_t] = {\bf 0}$
(i.e.,
volatility does not contain unspanned risks).
Due to
its relevance to
the dark matter risk premium
we corroborate this statement as Lemma~\ref{eq:lemmon} ({\color{magenta} Internet}~Appendix~\ref{appsec:SV1}). %\vspace{-3mm}
%\begin{corollary}[Risk premium of a straddle
%%when $(F_{\ell}^{T_F})$ is a
%(continuous semimartingale setting)]
%\label{claimm:straddles}
%Assume that
%\begin{equation}
%\underbrace{\mathbb{E}_{t}^{\mathbb{P}}( \int_{t}^{{T}_O}  \mathbbm{1}_{\{G_{\ell} > 1\}} \,dG_{\ell} )}_{\mathrm{upside~risk~premium~for}~k=1} ~-~ \\
%\underbrace{\mathbb{E}_{t}^{\mathbb{P}}( \int_{t}^{{T}_O} \mathbbm{1}_{\{G_{\ell} < 1\}} \,dG_{\ell} )}_{\mathrm{downside~risk~premium~for}~k=1} ~\approx~  0.
%\end{equation}
%Then, the risk premium of a straddle is zero (respectively, negative) if,
%and only if, the local time risk premium
%(and hence also the dark matter risk premium)
%for
%$k=1$ is zero (respectively, negative). \vspace{-3mm}
% \end{corollary}
%\noindent {\bf Proof:}
%See Appendix~\ref{appsec:jumppps} (part III).
%%The proof is the continuous semimartingale version of Theorem~\ref{claimm:claim1call_jump}  (Appendix~\ref{appsec:jumppps} part III).
%$\blacksquare$
%%{\color{blue} (see also Internet Appendix (Section~\ref{appsec:straddle})).}
%

%The condition $\mathbb{E}_{t}^{\mathbb{P}}( \int_{t}^{{T}_O}  \mathbbm{1}_{\{G_{\ell} > 1\}} \,dG_{\ell} ) - \mathbb{E}_{t}^{\mathbb{P}}( \int_{t}^{{T}_O} \mathbbm{1}_{\{G_{\ell} < 1\}} \,dG_{\ell} ) \approx 0$
%is akin to the
%unforecastability of the combined long and short futures position to the upside or the downside pertaining to $k=1$.
%Corollary~\ref{claimm:straddles} reflects a
%testable prediction of our theory using excess straddle returns.
%%Depending upon term to expiration $T_O-t$,
%The nature of the
%local time risk premium at moneyness $k$ --- which associates with \emph{dark matter}
%in option risk premiums --- can be understood by evaluating the excess returns of straddles.

%\newpage

\noindent \textbf{\color{magenta} Summary and complementary big picture.} We show theoretically
that the presence of unspanned risks in the dynamics of
the pricing kernel
%dynamics
and
the volatility
is in the direction of
%crucial to
addressing certain
questions in the market for options on
equity index and futures.
Intuitively, the presence of unspanned risks in the volatility dynamics impacts the quadratic variation, which in turn
impacts local time. This feature, in conjunction with a nonzero contribution of the unspanned risks ${\bm\theta}[t,\mathbf{Y}_t]^{\top} \, d \mathbf{u}^{\mathbb{P}}_t$, gives rise to, in general, a nonzero local time risk premium (for all $k$).\footnote{
One may be tempted to cast the local time risk premium
as a gamma risk premium
(since $\mathbb{L}^{T_O}_t[k] = \frac{1}{2} \int_{t}^{T_O} \delta_{\{G_\ell ~-~ k\}} d [ G,G]_\ell$),
given that the Dirac delta function is the second-order derivative of $\max(G_{\ell}-k,0)$ with respect to $G_{\ell}$ (i.e., reflects the gamma). While the concept of a gamma risk premium is appealing in the continuous semimartingale context, the dynamics of the equity futures have a nonzero correlation with quadratic variation.}

The
possibility
that
local time (dark matter) risk premiums may be
dependent upon
$T_O-t$ is embedded
within our characterizations. \vspace{-3mm}


\subsection{The role of unspanned risks in a stochastic volatility model}
%Role of unspanned risks in a parameterized model
%that
%fosters
%is
%aimed
%towards
%economic
%}
\label{subsec:model_sv}

%The
Our purpose
%of this
%section
is threefold. First, we present
a parametric (continuous semimartingale) setting that
explicitly models (i) spanned and unspanned risks in the pricing kernel
and (ii) spanned and unspanned risks
 in equity return volatility. Second,
we show that our framework subsumes the baseline specification of no unspanned risks in the pricing kernel and no
unspanned risks in equity return volatility.
Third, we interpret
%digest
the economic restrictions under which a stochastic
volatility model, with unspanned and spanned risks, can be consistent with negative risk premiums for OTM calls.
Our
alternative specification
can be viewed as a stepping stone to understanding the distinction between the local time risk premium
(corresponding to $k$) and the variance risk premium.

%\noindent \textbf{Stochastic volatility model with unspanned risks.}
Consider the dynamics of
%We specialize the dynamics in equations (\ref{eq:GeneralDynamics1}) and (\ref{eq:index3}) for
$M_t$ and $S_t$, as follows: %$\mathrm{v}$
\begin{eqnarray}
\frac{dM_t}{M_t} & = & -r\, dt
~+~ \eta[t,\mathrm{v}_t] \underbrace{d z_t^{\mathbb{P}}}_{\mathrm{spanned~risks}}
~+~ \theta[t,\mathrm{v}_t]  \underbrace{du_t^{\mathbb{P}},}_{\mathrm{unspanned~risks}}~~~~\mbox{ \, \quad \, with\, \, }~~~~ \label{eq:v1}\\
& &
\eta[t,\mathrm{v}_t] \, = \, - \frac{1}{ \sqrt{\mathrm{v}_t}}(
\alpha_{\mathrm{vol}} +\lambda_{\mathrm{vol}} \, \mathrm{v}_t),
%~~\mbox{ \, }
~~~~~~~~
%~~\mathrm{and}~~~~
\theta[t,\mathrm{v}_t]  =  - \theta_{\mathrm{LT}}\, \sqrt{\mathrm{v}_t},
~\mbox{ \, }
~\mathrm{and}~~~\mbox{ \, \, }
\label{eq:v2}\\
%& &
%\theta[t,v_t]  =  - \theta_{\mathrm{LT}}\, \sqrt{v_t},
%~\mbox{ \, }~\mathrm{and}~~~\mbox{ \, \, } \label{eq:v2a}\\
\frac{d S_t}{S_t} & = &  ( r  ~-~ \eta[t,\mathrm{v}_t] \,\sqrt{\mathrm{v}_t} )  \,dt ~+~
\sqrt{\mathrm{v}_t} \underbrace{d z^{\mathbb{P}}_t,}_{\mathrm{spanned~risks}}
\label{eq:SimpleSquareRootVolStockDynmaics1}
\end{eqnarray}
where $\mathrm{v}_t$ denotes the instantaneous variance of the equity return, which also constitutes the
single economic state variable (i.e., $\mathbf{Y}_t= \mathrm{v}_t$).
We specify the dynamics for $\mathrm{v}_t$ under $\mathbb{P}$ in (\ref{eq:SimpleHestonDynmaics}) (below).
In (\ref{eq:v1}), $z_t^{\mathbb{P}}$ and $u_t^{\mathbb{P}}$ are each a one dimensional standard Brownian motion.


In our setup, $\alpha_{\mathrm{vol}}$, $\lambda_{\mathrm{vol}}$, and $\theta_{\mathrm{LT}}$ are constants.
Provided that
%\begin{align}
%&\alpha_{\mathrm{vol}} > 0&
%&\mathrm{and}&
%&\lambda_{\mathrm{vol}} > 0,&
%\end{align}
(i) $\alpha_{\mathrm{vol}} > 0$ and (ii) $\lambda_{\mathrm{vol}} > 0$, the
risk premium on
the equity (and its futures) is positive; that is,
$\alpha_{\mathrm{vol}}+\lambda_{\mathrm{vol}} \, \mathrm{v}_t  >  0$.

Crucial from our perspective, we will show
(see Corollary~\ref{claimm:SV} below) that the parameter $\theta_{\mathrm{LT}}$
affecting $\theta[t,\mathrm{v}_t]  =  - \theta_{\mathrm{LT}}\, \sqrt{\mathrm{v}_t}$ in (\ref{eq:v2})
determines the sign of the local time risk premium, and, consequently, the call risk premium.
It
holds that local time risk premium exerts \emph{influence} on, and yet (see
(\ref{eq:ConditionForNegVRPInExtendedHeston})) is \emph{economically distinct} from, the variance risk premium.


Next,  (\ref{eq:SimpleHestonDynmaics}) develops the view that random fluctuations
in equity return
variance $\mathrm{v}_t$ can
arise from
both spanned and unspanned diffusive risks, whereas (by definition) the source of randomness
in the equity futures price is solely due to spanned diffusive risks.


For tractability, we assume that the dynamics of return variance, $\mathrm{v}_t$, follow
\begin{eqnarray}
d\mathrm{v}_t  &=& ( \phi_{\mathrm{vol}}^{\mathbb{P}} - \kappa_{\mathrm{vol}}^{\mathbb{P}} \,\mathrm{v}_t )\, dt ~+~
  \sigma_{\mathrm{vol}} \, \sqrt{\mathrm{v}_t} \,\rho_{\mathrm{vol}} \, \underbrace{d z_t^{\mathbb{P}}}_{\mathrm{spanned~risks}}  ~+~ \sigma_{\mathrm{vol}} \sqrt{\mathrm{v}_t} \, \sqrt{1-\rho^2_{\mathrm{vol}}} \, \underbrace{du_t^{\mathbb{P}}.}_{\mathrm{unspanned~risks}} ~~ \mbox{ \, \,  }~ \label{eq:SimpleHestonDynmaics} \\
\mbox{Hence,} & & \frac{d G_t}{G_t} \, \, = \, \, \frac{d F_{t}^{T_F}}{F_{t}^{T_F}} \, = \, \overbrace{( \alpha_{\mathrm{vol}} + \lambda_{\mathrm{vol}} \, \mathrm{v}_t  )}^{\mathrm{futures~risk~premium}}  \, dt ~+~
\sqrt{\mathrm{v}_t} \, \overbrace{d z^{\mathbb{P}}_t.}^{\mathrm{spanned~risks}}~~\mbox{  \, \, }~~
~~\mbox{ \, \, }~~\label{eq:SimpleSquareRootVolStockDynmaics}
\end{eqnarray}

To shed light on option risk premiums and
compensation
for unspanned risks reflected in the risk-neutral drift of the volatility process,
(\ref{eq:SimpleHestonDynmaics}) draws
a distinction between the
one dimensional Brownian motions $z_t^{\mathbb{P}}$ and $u_t^{\mathbb{P}}$.
Additionally,
$\mathrm{cov}^{\mathbb{P}}_{t}(\frac{d F_{t}^{T_F}}{F_{t}^{T_F}}, d\mathrm{v}_t)/dt = \rho_{\mathrm{vol}} \,
\sigma_{\mathrm{vol}} \,\mathrm{v}_t$ and, hence, $\rho_{\mathrm{vol}}$ is
the instantaneous correlation between
$\frac{d F_{t}^{T_F}}{F_{t}^{T_F}}$ and $d\mathrm{v}_t$.

\noindent \textbf{Call risk premium.}
%Considering the nature of spanned and unspanned risks, under $\mathbb{P}$ and $\mathbb{Q}$,
It follows, from (\ref{fg.1}), that
\begin{align}
&
dz_t^{\mathbb{P}} ~-~ dz^{\mathbb{Q}}_t \, = \,
 -\frac{1}{ \sqrt{\mathrm{v}_t}}(
\alpha_{\mathrm{vol}} +\lambda_{\mathrm{vol}} \, \mathrm{v}_t) \, dt&
&\mathrm{and}&
& du_t^{\mathbb{P}} ~-~ du_t^{\mathbb{Q}} \, = \,  -\theta_{\mathrm{LT}} \,\sqrt{\mathrm{v}_t} \, dt.    &
\end{align}
The workings of this theory implies that
the $\mathbb{Q}$ dynamics of $G_t$ and of $\mathrm{v}_t$ are $\frac{dG_t}{G_t} = \sqrt{\mathrm{v}_t} \, dz_t^{\mathbb{Q}}$ and
\begin{eqnarray}
d\mathrm{v}_t & = & ( \phi_{\mathrm{vol}}^{\mathbb{Q}} - \kappa_{\mathrm{vol}}^{\mathbb{Q}} \, \mathrm{v}_t ) \, dt
+ \sigma_{\mathrm{vol}} \, \sqrt{\mathrm{v}_t} \, \rho_{\mathrm{vol}}\, dz_t^{\mathbb{Q}}
+ \sigma_{\mathrm{vol}} \, \sqrt{\mathrm{v}_t}  \, \sqrt{1-\rho^2_{\mathrm{vol}}} \, du_t^{\mathbb{Q}}, ~~ \mbox{ \, \, \, \, \, } \mathrm{with} ~~ \label{volqdy} \\
\kappa_{\mathrm{vol}}^{\mathbb{Q}} & = &\kappa_{\mathrm{vol}}^{\mathbb{P}}
+ \sigma_{\mathrm{vol}} \,\rho_{\mathrm{vol}}\, \lambda_{\mathrm{vol}}
+\underbrace{\theta_{\mathrm{LT}}\,\sigma_{\mathrm{vol}} \,\sqrt{1-\rho^2_{\mathrm{vol}}}}_{\tiny\mathrm{not~in~Heston~(1993)}},
~\mbox{ \, \, and \, \, }~\phi_{\mathrm{vol}}^{\mathbb{Q}} \, = \,  \phi_{\mathrm{vol}}^{\mathbb{P}} - \rho_{\mathrm{vol}} \, \alpha_{\mathrm{vol}} \, \sigma_{\mathrm{vol}}. ~\mbox{ \, \, \, \, }~~~ \label{eq:KappaPKappaQGammaPGammaQ}
\end{eqnarray}

At the heart of this
model are unspanned risks in the dynamics of $M_t$ and $\mathrm{v}_t$,
with theoretical effects conceptually
different from the baseline, as in \citet*[equations (4) and (8)]{Heston:1993}.

\setcounter{theorem}{1}
\begin{corollary}[OTM call risk premium
in a model with unspanned risks]
\label{claimm:SV}
Under the parameterizations
in
(\ref{eq:v1})--(\ref{eq:SimpleHestonDynmaics}),
the call risk premium
can be negative if
$\theta_{\mathrm{LT}} < 0$.
Absent a contribution of unspanned risks (when $\theta_{\mathrm{LT}} = 0$)
or when $\theta_{\mathrm{LT}} \geq 0$,
the call risk premium is positive.  %\vspace{-3mm}
\end{corollary}
\noindent {\bf Proof:} See Internet Appendix~\ref{appsec:SV1}. $\blacksquare$

The restriction $\theta_{\mathrm{LT}} < 0$ implies a negative local time risk
premium. If the latter is sufficiently negative, that is, large enough in magnitude to offset
a positive
equity futures risk premium to the upside,
the call risk premium
can be negative.

\noindent \textbf{Differentiating local time risk premium from variance risk premium.}
%{\color{red} [Emphasizing a departure,
%we now
%differentiate
%between the local time risk premium
%and the \emph{variance} risk premium.]}
We can derive
%The time $t$ conditional equity variance risk premium is
\begin{equation} \small
\underbrace{\mathbb{E}_t^{\mathbb{P}}( \int_{t}^{T_O} \mathrm{v}_{\ell} d \ell ) - \mathbb{E}_t^{\mathbb{Q}}( \int_{t}^{T_O} \mathrm{v}_{\ell} d \ell )}_{\mathrm{variance~risk~premium}}
=  \frac{ \phi_{\mathrm{vol}}^{\mathbb{P}} ( T_0 - t ) - \{ \mathbb{E}_t^{\mathbb{P}}( \mathrm{v}_{T_O} ) - \mathrm{v}_{t} \} }{\kappa_{\mathrm{vol}}^{\mathbb{P}}}
~- ~ \frac{ \phi_{\mathrm{vol}}^{\mathbb{Q}} ( T_0 - t ) - \{ \mathbb{E}_t^{\mathbb{Q}}( \mathrm{v}_{T_O} ) - \mathrm{v}_{t} \} }{\kappa_{\mathrm{vol}}^{\mathbb{Q}}}.~~\mbox{ \, }
\label{vrp.1}
\end{equation}
Furthermore, the \emph{unconditional} variance risk premium is $ \, $ $\frac{\phi_{\mathrm{vol}}^{\mathbb{P}} ( T_0 - t )}{\kappa_{\mathrm{vol}}^{\mathbb{P}}} - \frac{\phi_{\mathrm{vol}}^{\mathbb{Q}} ( T_0 - t )}{\kappa_{\mathrm{vol}}^{\mathbb{Q}}}$, $ \, $ which is negative if, and only,
$\frac{\phi_{\mathrm{vol}}^{\mathbb{P}}}{\kappa_{\mathrm{vol}}^{\mathbb{P}}} < \frac{\phi_{\mathrm{vol}}^{\mathbb{Q}}}{\kappa_{\mathrm{vol}}^{\mathbb{Q}}}$
or, rearranging using (\ref{eq:KappaPKappaQGammaPGammaQ}), if, and only if,
\begin{eqnarray}
\frac{\phi_{\mathrm{vol}}^{\mathbb{Q}} + \rho_{\mathrm{vol}}\, \alpha_{\mathrm{vol}} \, \sigma_{\mathrm{vol}}}{\kappa_{\mathrm{vol}}^{\mathbb{Q}} ~+~
\sigma_{\mathrm{vol}} \{-\theta_{\mathrm{LT}}\, \sqrt{1-\rho_{\mathrm{vol}}^2} ~-~ \rho_{\mathrm{vol}}\,\lambda_{\mathrm{vol}}\}} ~<~ \frac{\phi_{\mathrm{vol}}^{\mathbb{Q}}}{\kappa_{\mathrm{vol}}^{\mathbb{Q}}}. ~~~
\text{ \footnotesize
\, \, (for negative variance risk premium) \, \, } ~~ \label{eq:ConditionForNegVRPInExtendedHeston}
\end{eqnarray}

The parametrization $\rho_{\mathrm{vol}} \leq 0$
(and $\alpha_{\mathrm{vol}} > 0$) is instructive, whereby the variance risk premium is negative if the local time
risk premium is negative; that is, if $\theta_{\mathrm{LT}} < 0$.\footnote{The variance risk premium is connected to $\alpha_{\mathrm{vol}}$ and $\lambda_{\mathrm{vol}}$
through the terms
$\rho_{\mathrm{vol}} \, \sigma_{\mathrm{vol}}\, \lambda_{\mathrm{vol}}$ and $\rho_{\mathrm{vol}} \, \alpha_{\mathrm{vol}} \, \sigma_{\mathrm{vol}}$, implying that some of the variance risk premium is entangled with the
equity futures risk premium (when $\rho_{\mathrm{vol}} \neq 0$).
In contrast, the irrelevance
of unspanned risks (i.e., $\theta_{\mathrm{LT}} = 0$) implies zero local time risk premiums for all $k$.}


%{\color{red} [[[ I would have left this in? ]] [[[ While there is strong support in the extant empirical literature for
%$\rho_{\mathrm{vol}} \leq 0$, $\alpha_{\mathrm{vol}} > 0$ and for
%a negative variance risk premium,
%based on equation (\ref{eq:ConditionForNegVRPInExtendedHeston}),
%a negative variance risk premium \emph{could potentially} co-exist with $\theta_{\mathrm{LT}}$ being zero or positive.
%Our innovation is to show that empirical data on option returns
%supports the view that local time risk premiums must be negative and
%%in this model setup,
%the latter requires
%$\theta_{\mathrm{LT}} < 0$.]]]] -- in THE TAKEAWAY ]]]]}

\noindent \textbf{Takeaways.} A suitably
motivated stochastic volatility model
%(under the $\mathbb{P}$ measure)
 can
synthesize negative call risk premiums
provided that certain restrictions are imposed on unspanned risks.
The linchpin of our framework
is that
%equity
volatility embeds unspanned risks and shapes
%moulds
%agrees
%is aligned
%with
a negative
local time risk premium.\footnote{
If jumps in variance (as in \citet*{DuffiePanSingleton:2000} and \citet*{Amengual_Xiu:JOE2018})
were to be added while maintaining the
equity dynamics in (\ref{eq:SimpleSquareRootVolStockDynmaics1}), then Corollary~\ref{claimm:SV} remains valid.}
One may envision
$\theta_{\mathrm{LT}} \{ \sigma_{\mathrm{vol}} \sqrt{ 1 - \rho^2_{\mathrm{vol}}} \}< 0$
(in  (\ref{eq:KappaPKappaQGammaPGammaQ}))
as arising from the feature that unspanned volatility risks are disliked (and $\theta_{\mathrm{LT}}<0$ is needed to produce negative local time risk premiums). \vspace{-3mm}

%\subsection{\bf \small Appendix C: Statement and proof of Lemma~\ref{eq:lemmon} and proof of Corollary~\ref{claimm:SV}}
\subsection{Statement and proof of Lemma~\ref{eq:lemmon} and proof of Corollary~\ref{claimm:SV}}
\label{appsec:SV1}
We focus on the
continuous semimartingale setting of equations (\ref{eq:GeneralDynamics1})--(\ref{fg.1})
%{\color{red} [in Section~\ref{gggs}]}
--- with drift and diffusion coefficients unspecified ---  and accomplish
two tasks.
First, we show that if unspanned risks are
irrelevant (i.e., if ${\bm \theta}[t, \mathbf{Y}] =  \mathbf{0}$),
then the local time risk premium is zero. This is our Lemma~\ref{eq:lemmon}.
%{\color{red}[and our aim is to strengthen intuition.]}
Second, after specializing to the setting of
%equations
(\ref{eq:v1})--(\ref{eq:SimpleSquareRootVolStockDynmaics}),
%{\color{red}[in Section~\ref{subsec:model_sv}]},
we present the proof of Corollary~\ref{claimm:SV}.

%{\color{blue}We first develop some ``stepping stone" results in the general setting of equations (\ref{eq:GeneralDynamics1})-(\ref{fg.1}).}

Proceeding, for any generic claim with payoff $\complement_{T}$, at time $T$, it holds that
\begin{equation}
\mathbb{E}_{t}^{\mathbb{Q}}( \complement_{T} ) = e^{r(T- t)} \,\, \mathbb{E}_{t}^{\mathbb{P}}( \frac{M_{T}}{M_{t}} \,\complement_{T} ).~~~~~\mbox{ \quad } \label{eq:PtoQExpOfPayoffc}
\end{equation}
The risk premium is $\mathbb{E}_{t}^{\mathbb{P}}( \complement_{T} )-\mathbb{E}_{t}^{\mathbb{Q}}( \complement_{T} )$,
which we deduce next. This step is useful because we will set $\complement_{T}= \mathbb{L}^{T_O}_t[k]$, and, hence, we deduce
$\mathbb{E}_{t}^{\mathbb{P}}( \mathbb{L}^{T_O}_t[k] )-\mathbb{E}_{t}^{\mathbb{Q}}( \mathbb{L}^{T_O}_t[k] )$.

By  a property of covariances,
\begin{eqnarray}
\mathrm{cov}_t^{\mathbb{Q}}( \frac{M_{t}}{M_{T} e^{r (T- t)} }, \complement_{T} ) &=&
\mathbb{E}_{t}^{\mathbb{Q}}( \frac{M_{t}}{M_{T} e^{r (T- t)} } \, \complement_{T} ) ~-~
\overbrace{\mathbb{E}_{t}^{\mathbb{Q}}( \frac{M_{t}}{M_{T} e^{r (T- t)} })}^{=1} \,
\mathbb{E}_{t}^{\mathbb{Q}}( \complement_{T} )~~~~~\mbox{ \quad }  \label{c5.1} \\
&=& \mathbb{E}_{t}^{\mathbb{Q}}( \frac{M_{t}}{M_{T} e^{r (T- t)} } \complement_{T} ) ~-~
\mathbb{E}_{t}^{\mathbb{Q}}( \complement_{T} )~~~~~\mbox{ \quad } \label{c5.2} \\
&=& e^{r(T- t)} \mathbb{E}_{t}^{\mathbb{P}}( \frac{M_{T}}{M_{t}} \times \{ \frac{M_{t}}{M_{T} e^{r (T- t)} } \complement_{T} \} ) ~-~
\mathbb{E}_{t}^{\mathbb{Q}}( \complement_{T} ) \label{c5.3}\\
&=& \mathbb{E}_{t}^{\mathbb{P}}( \complement_{T} ) ~-~ \mathbb{E}_{t}^{\mathbb{Q}}( \complement_{T} ).~~~~~\mbox{ \quad }
\label{c5.4}
\end{eqnarray}
Specializing $\complement_{T}$ to the random variable $\mathbb{L}^{T_O}_t[k]$ and guided by (\ref{c5.4}), we obtain the
risk premium for local time with moneyness $k$ as follows:
\begin{equation}
\overbrace{
\mathbb{E}_{t}^{\mathbb{P}}( \mathbb{L}^{T_O}_t[k] ) ~ - ~
\mathbb{E}_{t}^{\mathbb{Q}}( \mathbb{L}^{T_O}_t[k] )}^{\text{Local~time~risk~premium}} ~=~\mathrm{cov}_t^{\mathbb{Q}}( \frac{M_{t}}{M_{{T}_O} e^{r ({T}_O - t)}} , \, \mathbb{L}^{T_O}_t[k] ). ~~\mbox{ \, \, \, \quad } ~
\label{eq:covqgsbstatement}
\end{equation}


%We now state.
\setcounter{theorem}{0}
\begin{lemma}
\label{eq:lemmon}
{\color{magenta} In the
%continuous semimartingale
setting of
%equations
(\ref{eq:GeneralDynamics1})--(\ref{fg.1}),} if unspanned risks  are irrelevant; that is,
\begin{equation}
\mathrm{if}~{\bm \theta}[t, \mathbf{Y}] \, = \, \mathbf{0}, ~~ \mbox{ \, \, } \mathrm{then} ~~ \mbox{ \, } \mathbb{E}_{t}^{\mathbb{P}}( \mathbb{L}^{T_O}_t[k] )  -
\mathbb{E}_{t}^{\mathbb{Q}}( \mathbb{L}^{T_O}_t[k] ) \, = \, 0. ~ \mbox{ \, \, \, \, } ~
%~~\mathrm{in~a~continuous~semimartingale~setting.}
\label{eq:covqgsbstatementxy}
\end{equation}
\end{lemma}

\noindent \textbf{Proof:} Using the
dynamics of the pricing kernel $M_t$
under the $\mathbb{P}$ measure in
(\ref{eq:GeneralDynamics1}), and then the change of measures in (\ref{fg.1}); that is,
\begin{align*}
&d \mathbf{z}^{\mathbb{P}}_t~-~d \mathbf{z}^{\mathbb{Q}}_t = {\bm \eta}[t,\mathbf{Y}_t] \,dt&
&\mathrm{and}&
&d \mathbf{u}^{\mathbb{P}}_t~-~d \mathbf{u}^{\mathbb{Q}}_t =  {\bm \theta}[t,\mathbf{Y}_t] \, dt,&
\end{align*}
%$d \mathbf{z}^{\mathbb{P}}_t-d \mathbf{z}^{\mathbb{Q}}_t = {\bm \eta}[t,\mathbf{Y}_t] dt$ and
%$d \mathbf{u}^{\mathbb{P}}_t-d \mathbf{u}^{\mathbb{Q}}_t =  {\bm \theta}[t,\mathbf{Y}_t] dt$,
it follows
that (hereon
suppressing the time subscript on $\mathbf{Y}_t$)
\begin{equation}
\frac{M_{t}}{M_{T_{O}}e^{r ({T}_O - t)}} = e^{
\int_{t}^{{T}_O} \{
-\frac{1}{2} {\bm \eta}[\ell,\mathbf{Y}]^{\top} {\bm \eta}[\ell,\mathbf{Y}] d\ell
- {\bm \eta}[\ell,\mathbf{Y}]^{\top}  d \mathbf{z}^{\mathbb{Q}}_\ell
- \frac{1}{2} {\bm \theta}[\ell,\mathbf{Y}]^{\top}
{\bm \theta}[\ell,\mathbf{Y}] d\ell
- {\bm\theta}[\ell,\mathbf{Y}]^{\top} d \mathbf{u}^{\mathbb{Q}}_\ell \} }.~\mbox{ \, }~\mbox{ \, }~
\label{eq:IntegratedRecipmGeneralDynamicsUnderQ}
\end{equation}
We note that $\mathbb{E}_{t}^{\mathbb{Q}}( \frac{M_{t}}{M_{T_{O}} e^{r ({T}_O - t)}} ) \, = \, 1$. Let
\begin{equation}
\mathcal{I}_s~\mathrm{be~the~sub\mbox{-}filtration~of}~\mathcal{F}_s~\mathrm{generated~by}~{\bm \eta}^{\mathbb{Q}}_s~\mathrm{and}~{\bm \eta}[s,\mathbf{Y}]^{\top} \,d \mathbf{z}^{\mathbb{Q}}_s. ~ \mbox{ \, \, } ~
\end{equation}

%We denote by $\mathcal{I}_s$ the sub-filtration of $\mathcal{F}_s$ generated by ${\bm \eta}^{\mathbb{Q}}_s$ and ${\bm \eta}[s,\mathbf{Y}]^{'} \,d %\mathbf{z}^{\mathbb{Q}}_s$.


Exploiting the law of total covariance,
\begin{eqnarray}
& & \mathrm{cov}_t^{\mathbb{Q}}( \overbrace{\frac{M_{t}}{M_{{T}_O} e^{r ({T}_O - t)}} }^{\mathrm{from~\tiny(\ref{eq:IntegratedRecipmGeneralDynamicsUnderQ}})}, \, \mathbb{L}_t^{{T}_O}[k] )
\nonumber \\ %\label{eq:NewConditionalCovariance1} \\
& & ~= ~ \mathbb{E}_{t}^{\mathbb{Q}}( \mathrm{cov}_t^{\mathbb{Q}}(
e^{ \int_{t}^{{T}_O} \{ -\frac{1}{2} {\bm \eta}[s,\mathbf{Y}]^{\top} {\bm  \eta}[s,\mathbf{Y}] ds - {\bm \eta}[s,\mathbf{Y}]^{\top} d \mathbf{z}^{\mathbb{Q}}_s
- \frac{1}{2} {\bm \theta}[s, \mathbf{Y}]^{\top} {\bm \theta}[s, \mathbf{Y}] ds - {\bm \theta}[s, \mathbf{Y}]^{\top}
d \mathbf{u}^{\mathbb{Q}}_s \}},
\, \mathbb{L}_t^{{T}_O}[k] {\Big |} \, \mathcal{I}_{T_O} ) )~~\mbox{ \, \, }~~~ \nonumber \\
& & ~\mbox{ \, \, }~\mbox{ \, \, } \mbox{ \quad \quad \quad } \, + \, \mathrm{cov}_t^{\mathbb{Q}}( \underbrace{ \mathbb{E}_{t}^{\mathbb{Q}}( \frac{M_{t}}{M_{{T}_O}} e^{-r ({T}_O - t)} \, \, {\Big |} \mathcal{I}_{T_O} ) }_{\, = \, \,  \mathrm{a~constant} }, \, \, \mathbb{E}_{t}^{\mathbb{Q}}( \mathbb{L}_t^{{T}_O}[k] \, {\Big |} \mathcal{I}_{T_O} ) ) \mbox{ \quad }~\mbox{ \, \, }~  \label{eq:NewConditionalCovariance1Eq} \\
& & ~= ~ \mathbb{E}_{t}^{\mathbb{Q}}( e^{ \int_{t}^{{T}_O} \{ -\frac{1}{2} {\bm \eta}[s,\mathbf{Y}]^{\top} {\bm  \eta}[s,\mathbf{Y}] ds - {\bm \eta}[s,\mathbf{Y}]^{\top} \, \, d \mathbf{z}^{\mathbb{Q}}_s \} } \, \, \times
\nonumber \\
& & ~\mbox{ \, \, }~\mbox{ \, \, } \mbox{ \quad \quad \, } \mathrm{cov}_t^{\mathbb{Q}}(
e^{ \int_{t}^{{T}_O} \{ - \frac{1}{2} {\bm \theta}[s, \mathbf{Y}]^{\top} {\bm \theta}[s, \mathbf{Y}] ds -
{\bm \theta}[s, \mathbf{Y}]^{\top} \,
d \mathbf{u}^{\mathbb{Q}}_s \}}, \, \mathbb{L}_t^{{T}_O}[k] \,{\Big |} \mathcal{I}_{T_O} ) ),
\label{eq:NewConditionalCovariance2Eq}
\end{eqnarray}
because the covariance of the two expectations in
(\ref{eq:NewConditionalCovariance1Eq}) vanishes since one term
(conditional on $\mathcal{I}_{T_O}$) is a constant.
If it were the case that ${\bm \theta}[t, \mathbf{Y}]$ is identically zero and, thus,
${\bm \theta}[s, \mathbf{Y}]^{\top} d \mathbf{u}^{\mathbb{Q}}_s=0$,
then, by
%equation
(\ref{eq:NewConditionalCovariance2Eq}), $\mathrm{cov}_t^{\mathbb{Q}}( \frac{M_{t}}{M_{{T}_O} e^{r ({T}_O - t)}} , \, \mathbb{L}^{T_O}_t[k] ) = 0$.

Then, by (\ref{eq:covqgsbstatement}), $\mathbb{E}_{t}^{\mathbb{P}}( \mathbb{L}^{T_O}_t[k] ) - \mathbb{E}_{t}^{\mathbb{Q}}( \mathbb{L}^{T_O}_t[k] ) = 0$. $\square$ \vspace{2mm}

Based on (\ref{eq:NewConditionalCovariance2Eq}), the sign of the
local time risk premium inherits the sign of the $\mathbb{Q}$ measure conditional covariance
$\mathrm{cov}_t^{\mathbb{Q}}(
e^{ \int_{t}^{{T}_O} \{ - \frac{1}{2} {\bm \theta}[s, \mathbf{Y}]^{\top} {\bm \theta}[s, \mathbf{Y}] ds -
{\bm \theta}[s, \mathbf{Y}]^{\top} \,
d \mathbf{u}^{\mathbb{Q}}_s \}}, \, \mathbb{L}_t^{{T}_O}[k] \,{\Big |} \mathcal{I}_{T_O} )$.
More specifically, (\ref{eq:covqgsbstatement}) and (\ref{eq:NewConditionalCovariance2Eq}) indicate
that the local time risk premium $\mathbb{E}_{t}^{\mathbb{P}}( \mathbb{L}^{T_O}_t[k] ) -
\mathbb{E}_{t}^{\mathbb{Q}}( \mathbb{L}^{T_O}_t[k] )$ is negative, if and only if,
%{\color{red}[[[ removed ``for all $t$." ]]]}
\begin{equation}
\mathrm{cov}_t^{\mathbb{Q}}(
e^{ \int_{t}^{{T}_O} \{ - \frac{1}{2} {\bm \theta}[s, \mathbf{Y}]^{\top} {\bm \theta}[s, \mathbf{Y}] ds -
{\bm \theta}[s, \mathbf{Y}]^{\top} \, \,
d \mathbf{u}^{\mathbb{Q}}_s \}}, \, \mathbb{L}_t^{{T}_O}[k] \,{\Big |} \mathcal{I}_{T_O} ) < 0.
%~~~~~~~~\mbox{ \, \, }
\label{c5.10}
\end{equation}
In this continuous semimartingale setting, it further holds that the
\begin{equation}
\mathrm{sign~of}~\mathbb{E}_{t}^{\mathbb{P}}( \mathbb{L}^{T_O}_t[k] ) -
\mathbb{E}_{t}^{\mathbb{Q}}( \mathbb{L}^{T_O}_t[k] )~\mathrm{is~the~sign~of}~
\mathrm{cov}_t^{\mathbb{Q}}( \int_{t}^{{T}_O} - {\bm \theta}[s, \mathbf{Y}]^{\top} \,
d \mathbf{u}^{\mathbb{Q}}_s, \, \mathbb{L}_t^{{T}_O}[k] \,{\Big |} \mathcal{I}_{T_O} ). ~
\label{c5.11}
\end{equation}
With $\sum_{t \leq h \leq \ell} (G_{h} - G_{h -})^2  =  0$
(no jumps for any $h$)
for continuous semimartingales, $\mathbb{L}^{T_O}_t[k] = \frac{1}{2} \int_{t}^{T_O}
\delta_{\{G_\ell ~-~ k\}} d [ G, G ]_{\ell}$, where
$\delta_{\{\bullet\}}$ is the Dirac delta function and $[ G, G ]_{\ell}$ is the quadratic variation.

\noindent \textbf{Proof of Corollary~\ref{claimm:SV}.}  Mindful of
equations (\ref{c5.10}) and (\ref{c5.11}),
we return to our model in equations (\ref{eq:v1})--(\ref{eq:SimpleHestonDynmaics}) and derive the form of
$\mathbb{L}_t^{{T}_O}[k]$. To do so, note that the evolution of variance satisfies
\begin{eqnarray}
\mathrm{v}_{\ell} & =  & \mathrm{v}_{t}\, e^{ \kappa_{\mathrm{vol}}^{\mathbb{Q}} ( t - \ell ) } + \int_{t}^{\ell} \phi_{\mathrm{vol}}^{\mathbb{Q}} e^{ \kappa_{\mathrm{vol}}^{\mathbb{Q}} ( s - \ell ) } ds \nonumber \\
&+&\sigma_{\mathrm{vol}}\,\rho_{\mathrm{vol}} \int_{t}^{\ell}  e^{ \kappa_{\mathrm{vol}}^{\mathbb{Q}} ( s - \ell ) } \sqrt{\mathrm{v}_{s}} \,  dz_s^{\mathbb{Q}}
+\sigma_{\mathrm{vol}} \, \sqrt{1-\rho^2_{\mathrm{vol}}} \int_{t}^{\ell}  e^{ \kappa_{\mathrm{vol}}^{\mathbb{Q}} ( s - \ell ) } \sqrt{\mathrm{v}_{s}} \,  du_s^{\mathbb{Q}},~~~\mbox{ for $\ell \geq t$.\, \, \, }~~
\label{cv.s}
\end{eqnarray}
It further holds that $dG_t  = \sqrt{\mathrm{v}_t} \, G_t \, dz_t^{\mathbb{Q}}$. Hence, the \emph{quadratic variation}
$[ G, G ]_{s}$ is
 \begin{equation}
[ G, G ]_{s}  = \, \int_{t}^s \{ \sqrt{\mathrm{v}_{\ell}} \, G_{\ell} \}^2 \, d\ell \, \, = \, \int_{t}^s \, \mathrm{v}_{\ell} \, G_\ell^2 \, d\ell. ~ \mbox{ \, } ~ ~~
\end{equation}
Using the differential form $d [ G, G ]_{\ell} =  \mathrm{v}_{\ell} \, G_\ell^2 \, d \ell$, we deduce, from (\ref{ltt.1}) in
conjunction with
$\sum_{t \leq h \leq \ell} (G_{h} - G_{h -})^2  =  0$ (no jumps for any $h$),
that
\begin{eqnarray}
\mathbb{L}^{T_O}_t[k] &= & \frac{1}{2} \int_{t}^{T_O} \delta_{\{G_\ell ~-~ k\}} d [ G, G ]_{\ell}
~=~  \frac{1}{2} \, \int_{t}^{{T}_O} \, \delta_{\{G_\ell ~-~ k\}} \, \mathrm{v}_{\ell} \, G_\ell^2 \, d\ell. ~ \mbox{ \, } ~ ~~
\end{eqnarray}

Specializing ${\bm \theta}[t, \mathbf{Y}]$ to
${\bm \theta}[t, \mathbf{Y}]  =  - \, \theta_{\mathrm{LT}} \sqrt{\mathrm{v}_t}$ as per our setup,
we obtain the sign of the $\mathbb{Q}$-measure conditional
covariance, in light of equations (\ref{c5.10}) and (\ref{c5.11}), as follows:
\begin{eqnarray}
& & \mathrm{cov}_t^{\mathbb{Q}}( \int_{t}^{{T}_O} - {\bm \theta}[s, \mathbf{Y}]^{'} \, \,
d \mathbf{u}^{\mathbb{Q}}_s, \, \mathbb{L}_t^{{T}_O}[k] \,{\Big |} \mathcal{I}_{T_O} ) \nonumber \\
& & =~ \mathrm{cov}_t^{\mathbb{Q}}( \int_{t}^{{T}_O} -
\{-\theta_{\mathrm{LT}} \, \sqrt{\mathrm{v}_s} \, \, d u^{\mathbb{Q}}_{s}\},
 \,
\frac{1}{2} \, \int_{t}^{{T}_O} \, \delta_{\{G_\ell ~-~ k\}} \, \mathrm{v}_{\ell} \, G_\ell^2 \, d\ell \, \,{\Big |} \mathcal{I}_{T_O} ) \nonumber \\
& & =~ \mathrm{cov}_t^{\mathbb{Q}}( \int_{t}^{{T}_O}
\theta_{\mathrm{LT}} \, \sqrt{\mathrm{v}_s} \, \, d u^{\mathbb{Q}}_{s},
\frac{1}{2} \int_{t}^{{T}_O} \int_{t}^{\ell} \, \sigma_{\mathrm{vol}} e^{\kappa_{\mathrm{vol}}^{\mathbb{Q}}( s - \ell )} \sqrt{\mathrm{v}_s} \, \sqrt{1-\rho_{\mathrm{vol}}^2} \,
du^{\mathbb{Q}}_{s} \, \delta_{\{G_\ell ~-~ k\}} \, G_\ell^2  d \ell  \,{\Big |} \mathcal{I}_{T_O} ) \mbox{ \, }~\mbox{ \, \, } \nonumber \\
& & =~ \mathrm{cov}_t^{\mathbb{Q}}( \int_{t}^{{T}_O} \theta_{\mathrm{LT}} \, \sqrt{\mathrm{v}_s} \, \, \, d u^{\mathbb{Q}}_{s},
\, \int_{t}^{{T}_O} \sqrt{\mathrm{v}_{s}} \, \{\ \int_{s}^{{T}_O} \frac{\sigma_{\mathrm{vol}}}{2} e^{\kappa_{\mathrm{vol}}^{\mathbb{Q}} ( s - \ell ) } \, \sqrt{1-\rho_{\mathrm{vol}}^2} \, \delta_{\{G_\ell ~-~ k\}} \, G_\ell^2 \, d\ell \} \, du^{\mathbb{Q}}_{s} \,{\Big |} \mathcal{I}_{T_O} )   \nonumber \\
& & =~ \mathbb{E}_{t}^{\mathbb{Q}}( \int_{t}^{{T}_O} \theta_{\mathrm{LT}} \, \sqrt{\mathrm{v}_s} \, \, \sqrt{\mathrm{v}_s} \,
\, \{\ \int_{s}^{{T}_O} \frac{\sigma_{\mathrm{vol}} }{2} e^{\kappa_{\mathrm{vol}}^{\mathbb{Q}} ( s - \ell ) } \, \sqrt{1-\rho_{\mathrm{vol}}^2} \, \delta_{\{G_\ell ~-~ k\}}
\, G_\ell^2 \, d\ell \} \,  ds \, {\Big |} \mathcal{I}_{T_O} )  \nonumber \\
& & =~ \theta_{\mathrm{LT}} \, \, \,
\underbrace{\mathbb{E}_{t}^{\mathbb{Q}}(  \int_{t}^{{T}_O} \, \mathrm{v}_s \, \, \{\ \int_{s}^{{T}_O} \frac{\sigma_{\mathrm{vol}}}{2}
e^{ \kappa_{\mathrm{vol}}^{\mathbb{Q}} ( s - \ell ) } \, \sqrt{1-\rho_{\mathrm{vol}}^2} \, \delta_{\{G_\ell ~-~ k\}}
\, G_\ell^2 \, d\ell \}\,  ds \,{\Big |} \mathcal{I}_{T_O} )}_{~\geq ~0}.
\label{eq:FinalCovarianceTerm3Heston}
\end{eqnarray}
Inspection of (\ref{eq:FinalCovarianceTerm3Heston}) shows
that the
\begin{align}
&\mathrm{local~time~risk~premium~is~negative,~if~and~only~if,}~\theta_{\mathrm{LT}} < 0.&
%&&
%& \mbox{(unspanned volatility risks are disliked)}&
\label{intuu}
\end{align}
When $\theta_{\mathrm{LT}} < 0$, by
Corollary~\ref{claimm:claim1call}, the  OTM call risk premium can be negative. The intuition behind
(\ref{intuu}) is that unspanned volatility risks are disliked.

%If the equity variance dynamics {\color{blue}were to} contain only spanned risks, it
%%{\color{blue}would correspond} to
%%{\color{blue}$\sqrt{1-\rho_{\mathrm{vol}}^2} = 0$ and
%%the economic effect would be as if $\theta_{\mathrm{LT}}= 0$ in the sense that
%%we find that the covariance in equation (\ref{eq:FinalCovarianceTerm3Heston}) would be zero,
%the local time risk premium would be zero and, hence, the risk premium of OTM calls would be positive.

%Elaborating on the workings of
Commenting on the steps in  (\ref{eq:FinalCovarianceTerm3Heston}), the second line of (\ref{eq:FinalCovarianceTerm3Heston}) recognizes that only the term $\sigma_{\mathrm{vol}} \, \sqrt{1-\rho^2_{\mathrm{vol}}} \int_{t}^{\ell}  e^{ \kappa_{\mathrm{vol}}^{\mathbb{Q}} ( s - \ell ) } \sqrt{\mathrm{v}_{s}} \, \, du_s^{\mathbb{Q}}$ in (\ref{cv.s}) is relevant for the covariance. Furthermore, the third line changes the order of integration. Finally, the fourth line uses Ito's isometry formula. $\blacksquare$
\vspace{-3mm}

\end{document}



%\end{document}
\newpage
%==*==*==*==*==*==*==*==*==*==*==*==*==*==*==*==*==*== Tables
 %\setcounter{table}{0}  % reset counter

%\renewcommand{\thetable}{\arabic{table}}

 %\setcounter{figure}{0}  % reset counter

%\renewcommand{\thefigure}{\arabic{figure}}

%  \renewcommand{\theequation}{\arabic{equation}}
%  \setcounter{equation}{0}  % reset counter
%\renewcommand{\thetheorem}{\arabic{theorem}}
%  \setcounter{theorem}{0}  % reset counter

\newpage
\thispagestyle{empty}
\clearpage



\begin{center}
{\Large{Dark Matter in (Volatility and) Equity Option Risk Premiums}} \\
\vspace{0.04in}
%Gurdip Bakshi~~~John Crosby~~~Xiaohui Gao \\
\textbf{\underline{Internet Appendix: Not for Publication}}
\end{center}
%\vspace{1mm}
\begin{center}
\textbf{Abstract}
\end{center}

\noindent {\color{magenta} We focus on the continuous semimartingale theoretical environment
with two features: (i) there are spanned and unspanned diffusive risks and (ii) the jumps crossing the strike
terms --- $a_t^{T_O}[k]$, $b_t^{T_O}[k]$, $c_t^{T_O}[k]$, and $d_t^{T_O}[k]$ --- are all zero.
Corollary~\ref{claimm:claim1call} depicts option risk premiums when there are unspanned diffusive risks in the dynamics of the pricing kernel
and the volatility. Corollary~\ref{claimm:SV} shows that a suitably motivated stochastic volatility model (under the $\mathbb{P}$ measure)
can generate
%synthesize
negative call risk premiums provided that certain restrictions are imposed on unspanned risks. Finally,
Lemma~\ref{eq:lemmon} shows that if unspanned risks  are irrelevant, then the local time risk premium is \emph{zero}.}


%\newpage
\thispagestyle{empty}
%\clearpage

\thispagestyle{empty}
% *************** start of text ****************************************
\newpage
\setcounter{page}{1}
\renewcommand{\thefootnote}{\arabic{footnote}}
\setcounter{footnote}{0}

%\setcounter{equation}{0}
%\renewcommand{\theequation}{B\arabic{equation}}

\setcounter{section}{0}
\renewcommand{\thesection}{\Roman{section}}
%\renewcommand{\thesection}{\Roman{section}}
%\renewcommand{\thesubsection}{\thesection.\Roman{subsection}}
\renewcommand{\thesubsection}{\thesection.\arabic{subsection}}
%\renewcommand{\thesubsection}{\Roman{subsection}}

%\section{ \bf \large Internet Appendix}

                                                        \setcounter{equation}{0}
                                                        \renewcommand{\theequation}{IA-\arabic{equation}}
                                                        %\setcounter{section}{0}
%\numberwithin{equation}{section}
\numberwithin{table}{section}
\numberwithin{theorem}{section}
\numberwithin{figure}{section}
%                                            \numberwithin{theorem}{subsection}

%%%%%%%%%%%%%%%%%%%%%%%%%%%%%%%%%%%%%%%% Begin I %%%%%%%%%%%%%%%%%%%%%%%%%%%%%%%%%%%%%%%%%%%%%%%%%%%%%
%%%%%%%%%%%%%%%%%%%%%%%%%%%%%%%%%%%%%%%%%%%%%%%%%%%%%%%%%%%%%%%%
                                                        \setcounter{equation}{0}
                                                        \renewcommand{\theequation}{I\arabic{equation}}
%\begin{center}
%\textbf{Internet Appendix}
%\end{center}
%%%%%%%%%%%%%%%%%%%%%%%%%%%%%%%%%%%%%%%%%%%%%%%%%%%%%%%%%%%%%%%%%%%%%%%%%%%%%%%%%%%%%%%%%%%%%%%%%%%%%%%%%%%%
%%%%%%%%%%%%%%%%%%%%%%%%%%%%%%%%%%%%%%%%%%%%%%%%%%%%%%%%%%%%%%%%%%%%%%%%%%%%%%%%%%%%%%%%%%%%%%%%%%%%%%%%%%%%
\section{Unspanned risks in a continuous semimartingale setting}
\label{seimimartingales_continuous}

In a continuous semimartingale setting, the following jump crossing the strike terms vanish:
\begin{align}
&{\color{magenta}a_t^{T_O}[k]=0,}&
&b_t^{T_O}[k]=0,&
&c_t^{T_O}[k]=0,~~ \mathrm{and}&
&d_t^{T_O}[k]=0.&
\end{align}
%, namely, $a_t^{T_O}[k]$, $b_t^{T_O}[k]$, $c_t^{T_O}[k]$, and $d_t^{T_O}[k]$, vanish.} \vspace{-3mm}
%{\color{green} A continuous semimartingale theoretical environment can be revealing for three reasons.
%First, the jumps crossing the strike terms ---
%$a_t^{T_O}[k]$, $b_t^{T_O}[k]$, $c_t^{T_O}[k]$, and $d_t^{T_O}[k]$ --- \emph{vanish}.
%Second, one can delineate the distinction between spanned and unspanned \emph{diffusive} risks. Third, the risk premium adjustments that
%link $\mathbb{P}$ to $\mathbb{Q}$ are explicit through Girsanov's change of measure theorem.} \vspace{-3mm}

\subsection{Implications of local time being the only source of dark matter}
\label{gggs}

In what follows, we dichotomize between spanned and unspanned risks in the following manner. \vspace{-2mm}
\begin{align}
&\mathrm{Let}~\mathbf{z}^{\mathbb{P}}_t~\mbox{denote a vector of independent standard Brownian motions under}~\mathbb{P}.& \\
&\mathrm{Additionally,}~\mathbf{u}^{\mathbb{P}}_t~\mbox{is another vector of independent standard Brownian motions under}~\mathbb{P}. \mbox{ \, } &
\end{align}

\noindent \textbf{Model.} By assumption, $\mathbf{z}^{\mathbb{P}}_t$ is  spanned,
while $\mathbf{u}^{\mathbb{P}}_t$ \emph{cannot} be spanned by equity futures.
%{\color{blue} Risks not spanned by equity futures could be spanned by options written on the equity futures
%(or on the equity).}
With the vector of state variables denoted by $\mathbf{Y}_t$,
consider the system of stochastic differential equations (SDEs) for
the pricing kernel $M_t$ and for the equity index $S_{t}$, as follows:
\begin{eqnarray}
\frac{d M_t}{M_t} & = & -r\, dt
~+~{\bm\eta}[t,\mathbf{Y}_t]^{\top} \, \underbrace{d \mathbf{z}^{\mathbb{P}}_t}_{\text{\tiny{spanned~risks}}}
~+~{\bm\theta}[t,\mathbf{Y}_t]^{\top} \, \underbrace{d \mathbf{u}^{\mathbb{P}}_t,}_{\text{\tiny{unspanned~risks}}}
\mbox{ \, \, \, }
\label{eq:GeneralDynamics1} \\
%r&=&\mathrm{spot~interest~rate,~assumed~constant,}~ \\
\frac{d S_{t}}{S_{t}} & =&
(r ~-~ \underbrace{{\bm \eta}[t,\mathbf{Y}_t]^{\top} \mathbf{V}[t,\mathbf{Y}_t]}_{=~\mathrm{cov}^{\mathbb{P}}_t( \frac{dM_t}{M_t}, \frac{d S_t}{S_t})/dt})
\, dt ~+~ \mathbf{V}[t,\mathbf{Y}_t]^{\top}
\underbrace{d \mathbf{z}^{\mathbb{P}}_t,}_{\text{\tiny{spanned~risks}}} \label{eq:index3} \\
\underbrace{d \,\mathbf{V}[t,\mathbf{Y}_t]}_{\tiny \mathrm{volatility~dynamics}} & = & {\bm \mu}_V[t,\mathbf{Y}_t] \,dt
~+~ \underbrace{{\bm \sigma}_{V,z}[t,\mathbf{Y}_t]}_{\mathrm{matrix}} \underbrace{d \mathbf{z}^{\mathbb{P}}_t}_{\text{\tiny{spanned~risks}}}
~+~ \underbrace{{\bm \sigma}_{V,u}[t,\mathbf{Y}_t]}_{\mathrm{matrix}}  \underbrace{d \mathbf{u}^{\mathbb{P}}_t,}_{\text{\tiny{unspanned~risks}}}
%\mathrm{and}
%~~ ~
%~\mbox{and}
\label{eq:index3VolDy} \\
\underbrace{\frac{dF_{t}^{T_F}}{F_{t}^{T_F}}}_{\tiny \mbox{using (\ref{fuut})}} & = &
-{\bm \eta}[t,\mathbf{Y}_t]^{\top} \mathbf{V}[t,\mathbf{Y}_t]
 \,dt
~+~ \mathbf{V}[t,\mathbf{Y}_t]^{\top} \underbrace{d \mathbf{z}^{\mathbb{P}}_t,}_{\mathrm{spanned~risks}}
\label{eq:FutP}
\end{eqnarray}
where $\mathbf{V}[t,\mathbf{Y}_t]$ is a vector conformable
with $\mathbf{z}^{\mathbb{P}}_t$. The notation ${\top}$ represents transpose of a vector. The dynamics of $\frac{d G_\ell}{G_\ell}$ coincides with those of $\frac{d F_{\ell}^{T_F}}{F_{\ell}^{T_F}}$ for all $\ell$ satisfying $t \leq \ell \leq T_F$.

Our differentiating element
is that the standard Brownian motions $\mathbf{u}^{\mathbb{P}}_t$ are present
in the SDEs for $M_t$ and for $\mathbf{V}[t,\mathbf{Y}_t]$.
The introduction of $\mathbf{u}^{\mathbb{P}}_t$ is akin to a form of market incompleteness,
and our treatment of $\mathbf{u}^{\mathbb{P}}_t$, and its risk compensation, is pertinent to our theoretical analysis,
empirical identifications, and the basis of what we call ``dark matter."

By assumption, ${\bm\eta}[t,\mathbf{Y}_t]^{\top} d \mathbf{z}^{\mathbb{P}}_t$ is spanned by $\mathbf{V}[t,\mathbf{Y}_t]^{\top} d \mathbf{z}^{\mathbb{P}}_t$. Thus, it must hold that
\begin{eqnarray}
{\bm\eta}[t,\mathbf{Y}_t]^{\top} \, {\bm\eta}[t,\mathbf{Y}_t] & = & \mathfrak{g}_t \, \mathbf{V}[t,\mathbf{Y}_t]^{\top} \, \mathbf{V}[t,\mathbf{Y}_t], ~~ \mbox{ \, \, \, for all $t$, \, \, \, for some scalar variable $\mathfrak{g}_t$. \, \, \, \, } ~ ~ \label{eq:RestrictionOnEta}
\end{eqnarray}

The feature that the volatility
dynamics (i.e., those of $\mathbf{V}[t,\mathbf{Y}_t]$) contains
unspanned risks --- the randomness that cannot be removed
by trading in equity futures
--- is data-motivated and essential to
our developments. The economic effect of $\mathbf{u}^{\mathbb{P}}_t$
is notable, because in its absence, the risk premium on local time
would
be \emph{zero} in
this
continuous semimartingale setting
(soon to be formalized {\color{magenta}as Lemma~\ref{eq:lemmon}}).\footnote{Since Brownian shocks are amenable to ``rotation," one may ask:
Could one have an alternative but equivalent representation, in which $\mathbf{u}^{\mathbb{P}}_t$ is a part of
$S_t$ and $M_t$ dynamics, but not of
volatility?
Our \emph{definition} of unspanned risks aligns with a notion that $\mathbf{u}^{\mathbb{P}}_t$
appear in the dynamics of the latent variables (i.e., in $M_t$ and in $\mathbf{V}[t,\mathbf{Y}_t]$).}
This potentially separates us from
other studies on equity volatility.

In equations (\ref{eq:GeneralDynamics1})--(\ref{eq:FutP}),
the drift and diffusion coefficients may
depend upon
$\mathbf{Y}_t$
and
are adapted to $\mathcal{F}_t$.
At this stage, we do not specify which
economic
variables enter
$\mathbf{Y}_t$.
In general, the dynamics of $\mathbf{Y}_t$
will
impact the drift (under $\mathbb{P}$) and diffusions of $\frac{d S_{t}}{S_{t}}$ and the form of risk compensation associated with
the spanned and unspanned components of
the pricing kernel.\footnote{Additionally, we assume that the drift and diffusion coefficients
are differentiable so that Ito's lemma can be applied, and
they are sufficiently regular so that the SDEs
have a unique solution. In particular,
the vector
${\bm \mu}_V[t,\mathbf{Y}_t]$, and conformable matrices
${\bm \sigma}_{V,z}[t,\mathbf{Y}_t]$ and ${\bm \sigma}_{V,u}[t,\mathbf{Y}_t]$
in (\ref{eq:index3VolDy}) must be such that elements of $\mathbf{V}[t,\mathbf{Y}_t]$ are nonnegative.
See \citet*[pages 364--366]{cir:85a} for the regularity conditions on the SDEs, including that
the covariance matrices be nonnegative definite.
Furthermore, we
preclude that
$\frac{dS_t}{S_t}$ is perfectly
correlated with increments to its variance.
{\color{red} [The work of \citet*{Bates:2000} provides the context
for a two-factor model of return volatility.]}}

In light of Girsanov's theorem,
$\mathbf{z}^{\mathbb{Q}}_t$ and $\mathbf{u}^{\mathbb{Q}}_t$ are
vectors of independent standard Brownian motions under the probability measure $\mathbb{Q}$,
linked to $\mathbf{z}^{\mathbb{P}}_t$ and $\mathbf{u}^{\mathbb{P}}_t$, by
\begin{align}
&d \mathbf{z}^{\mathbb{P}}_t~-~d \mathbf{z}^{\mathbb{Q}}_t = {\bm \eta}[t,\mathbf{Y}_t] \,dt&
&\mathrm{and}&
&d \mathbf{u}^{\mathbb{P}}_t~-~d \mathbf{u}^{\mathbb{Q}}_t =  {\bm \theta}[t,\mathbf{Y}_t] \, dt.&
\label{fg.1}
\end{align}
The dynamics of $\frac{d G_\ell}{G_\ell} = \frac{d F_{\ell}^{T_F}}{F_{\ell}^{T_F}}$ under $\mathbb{Q}$,
from (\ref{eq:FutP}), becomes $\frac{d G_\ell}{G_\ell} = \mathbf{V}[\ell,\mathbf{Y}_{\ell}]^{\top} d \mathbf{z}^{\mathbb{Q}}_{\ell}.$

\setcounter{theorem}{0}
\begin{corollary}[Continuous semimartingales]
\label{claimm:claim1call}
The following are true:
\begin{enumerate}
\item The OTM call risk premium %($k>1$)
\emph{can} be negative only if $\mathbb{E}^{\mathbb{P}}_{t}( \mathbb{L}^{T_O}_t[k] ) - \mathbb{E}^{\mathbb{Q}}_{t}( \mathbb{L}^{T_O}_t[k] ) <0$.
It is
positive if
$\mathbb{E}_{t}^{\mathbb{P}}( \int_{t}^{{T}_O} \mathbbm{1}_{\{G_{\ell} > k\}} \,dG_{\ell} ) >
- \{\mathbb{E}_{t}^{\mathbb{P}}( \mathbb{L}^{T_O}_t[k] )
-  \mathbb{E}_{t}^{\mathbb{Q}}( \mathbb{L}^{T_O}_t[k] ) \}$.

\item The {\color{magenta} straddle risk premium is zero (respectively, negative) if, and only if, the local time risk premium
(and hence the dark matter risk premium) for
$k=1$ is zero (respectively, negative).}

\end{enumerate}
\end{corollary}

\noindent {\bf Proof:} We specialize Theorem~\ref{claimm:claim1call_jump} to a continuous semimartingale setting (see Appendix~\ref{appsec:jumppps}). $\blacksquare$



\normalsize

Our results on call risk premium
are introduced
without parameterizing the diffusion or drift coefficients
of $M_t$ and $F^{T_F}_t$. In so doing, we highlight the mechanism of unspanned risks.
%the
{\color{magenta} Having unspanned risks is a part of our rationale} that {\color{red}[a]} negative local time (and, hence, dark matter) risk premium
could help to understand
puzzling data
features in the equity markets.
%{\color{red}[[[[ Our setup suggests the value of refining extant modeling frameworks to include unspanned
%risks in the equity  volatility dynamics. ]]]]}



Corollary~\ref{claimm:claim1call}
can be traced
to nontrivial contributions of
${\bm\theta}[t,\mathbf{Y}_t]^{\top} \, d \mathbf{u}^{\mathbb{P}}_t$
(i.e., $M_t$
has unspanned risks) and
${\bm \sigma}_{V,u}[t,\mathbf{Y}_t] \,d \mathbf{u}^{\mathbb{P}}_t$
(i.e., volatility has unspanned risks).
Implicit
is an insight that the local time risk premium
for any
$k$ is zero only if
${\bm\theta}[t,\mathbf{Y}_t] = {\bf 0}$
(i.e., $M_t$
does not contain unspanned risks) or
${\bm \sigma}_{V,u}[t,\mathbf{Y}_t] = {\bf 0}$
(i.e.,
volatility does not contain unspanned risks).
Due to
its relevance to
the dark matter risk premium
we corroborate this statement as Lemma~\ref{eq:lemmon} ({\color{magenta} Internet}~Appendix~(Section~\ref{appsec:SV1})). %\vspace{-3mm}
%\begin{corollary}[Risk premium of a straddle
%%when $(F_{\ell}^{T_F})$ is a
%(continuous semimartingale setting)]
%\label{claimm:straddles}
%Assume that
%\begin{equation}
%\underbrace{\mathbb{E}_{t}^{\mathbb{P}}( \int_{t}^{{T}_O}  \mathbbm{1}_{\{G_{\ell} > 1\}} \,dG_{\ell} )}_{\mathrm{upside~risk~premium~for}~k=1} ~-~ \\
%\underbrace{\mathbb{E}_{t}^{\mathbb{P}}( \int_{t}^{{T}_O} \mathbbm{1}_{\{G_{\ell} < 1\}} \,dG_{\ell} )}_{\mathrm{downside~risk~premium~for}~k=1} ~\approx~  0.
%\end{equation}
%Then, the risk premium of a straddle is zero (respectively, negative) if,
%and only if, the local time risk premium
%(and hence also the dark matter risk premium)
%for
%$k=1$ is zero (respectively, negative). \vspace{-3mm}
% \end{corollary}
%\noindent {\bf Proof:}
%See Appendix~\ref{appsec:jumppps} (part III).
%%The proof is the continuous semimartingale version of Theorem~\ref{claimm:claim1call_jump}  (Appendix~\ref{appsec:jumppps} part III).
%$\blacksquare$
%%{\color{blue} (see also Internet Appendix (Section~\ref{appsec:straddle})).}
%

%The condition $\mathbb{E}_{t}^{\mathbb{P}}( \int_{t}^{{T}_O}  \mathbbm{1}_{\{G_{\ell} > 1\}} \,dG_{\ell} ) - \mathbb{E}_{t}^{\mathbb{P}}( \int_{t}^{{T}_O} \mathbbm{1}_{\{G_{\ell} < 1\}} \,dG_{\ell} ) \approx 0$
%is akin to the
%unforecastability of the combined long and short futures position to the upside or the downside pertaining to $k=1$.
%Corollary~\ref{claimm:straddles} reflects a
%testable prediction of our theory using excess straddle returns.
%%Depending upon term to expiration $T_O-t$,
%The nature of the
%local time risk premium at moneyness $k$ --- which associates with \emph{dark matter}
%in option risk premiums --- can be understood by evaluating the excess returns of straddles.

%\newpage

\noindent \textbf{\color{magenta} Summary and complementary big picture.} We show theoretically
that the presence of unspanned risks in the dynamics of
the pricing kernel
%dynamics
and
the volatility
is in the direction of
%crucial to
addressing certain
questions in the market for options on
equity index and futures.\footnote{The
possibility
that
local time (dark matter) risk premiums may be
dependent upon
$T_O-t$ is embedded
within our characterizations.}
Intuitively, the presence of unspanned risks in the volatility dynamics impacts the quadratic variation, which in turn
impacts local time. This feature, in conjunction with a nonzero contribution of the unspanned risks ${\bm\theta}[t,\mathbf{Y}_t]^{\top} \, d \mathbf{u}^{\mathbb{P}}_t$, gives rise to, in general, a nonzero local time risk premium (for all $k$).\footnote{
One may be tempted to cast the local time risk premium
as a gamma risk premium
(since $\mathbb{L}^{T_O}_t[k] = \frac{1}{2} \int_{t}^{T_O} \delta_{\{G_\ell ~-~ k\}} d [ G,G]_\ell$),
given that the Dirac delta function is the second-order derivative of $\max(G_{\ell}-k,0)$ with respect to $G_{\ell}$ (i.e., reflects the gamma). While the concept of a gamma risk premium is appealing in the continuous semimartingale context, the dynamics of the equity futures have a nonzero correlation with quadratic variation.} \vspace{-3mm}


\subsection{The role of unspanned risks in a stochastic volatility model}
%Role of unspanned risks in a parameterized model
%that
%fosters
%is
%aimed
%towards
%economic
%}
\label{subsec:model_sv}

%The
Our purpose
%of this
%section
is threefold. First, we present
a parametric (continuous semimartingale) setting that
explicitly models (i) spanned and unspanned risks in the pricing kernel
and (ii) spanned and unspanned risks
 in equity return volatility. Second,
we show that our framework subsumes the baseline specification of no unspanned risks in the pricing kernel and no
unspanned risks in equity return volatility.
Third, we interpret
%digest
the economic restrictions under which a stochastic
volatility model, with unspanned and spanned risks, can be consistent with negative risk premiums for OTM calls.
Our
alternative specification
can be viewed as a stepping stone to understanding the distinction between the local time risk premium
(corresponding to $k$) and the variance risk premium.

%\noindent \textbf{Stochastic volatility model with unspanned risks.}
Consider the dynamics of
%We specialize the dynamics in equations (\ref{eq:GeneralDynamics1}) and (\ref{eq:index3}) for
$M_t$ and $S_t$, as follows: %$\mathrm{v}$
\begin{eqnarray}
\frac{dM_t}{M_t} & = & -r\, dt
~+~ \eta[t,\mathrm{v}_t] \underbrace{d z_t^{\mathbb{P}}}_{\mathrm{spanned~risks}}
~+~ \theta[t,\mathrm{v}_t]  \underbrace{du_t^{\mathbb{P}},}_{\mathrm{unspanned~risks}}~~~~\mbox{ \, \quad \, with\, \, }~~~~ \label{eq:v1}\\
& &
\eta[t,\mathrm{v}_t] \, = \, - \frac{1}{ \sqrt{\mathrm{v}_t}}(
\alpha_{\mathrm{vol}} +\lambda_{\mathrm{vol}} \, \mathrm{v}_t),
%~~\mbox{ \, }
~~~~~~~~
%~~\mathrm{and}~~~~
\theta[t,\mathrm{v}_t]  =  - \theta_{\mathrm{LT}}\, \sqrt{\mathrm{v}_t},
~\mbox{ \, }
~\mathrm{and}~~~\mbox{ \, \, }
\label{eq:v2}\\
%& &
%\theta[t,v_t]  =  - \theta_{\mathrm{LT}}\, \sqrt{v_t},
%~\mbox{ \, }~\mathrm{and}~~~\mbox{ \, \, } \label{eq:v2a}\\
\frac{d S_t}{S_t} & = &  ( r  ~-~ \eta[t,\mathrm{v}_t] \,\sqrt{\mathrm{v}_t} )  \,dt ~+~
\sqrt{\mathrm{v}_t} \underbrace{d z^{\mathbb{P}}_t,}_{\mathrm{spanned~risks}}
\label{eq:SimpleSquareRootVolStockDynmaics1}
\end{eqnarray}
where $\mathrm{v}_t$ denotes the instantaneous variance of the equity return, which also constitutes the
single economic state variable (i.e., $\mathbf{Y}_t= \mathrm{v}_t$).
We specify the dynamics for $\mathrm{v}_t$ under $\mathbb{P}$ in (\ref{eq:SimpleHestonDynmaics}) (below).
In (\ref{eq:v1}), $z_t^{\mathbb{P}}$ and $u_t^{\mathbb{P}}$ are each a one dimensional standard Brownian motion.


In our setup, $\alpha_{\mathrm{vol}}$, $\lambda_{\mathrm{vol}}$, and $\theta_{\mathrm{LT}}$ are constants.
Provided that
%\begin{align}
%&\alpha_{\mathrm{vol}} > 0&
%&\mathrm{and}&
%&\lambda_{\mathrm{vol}} > 0,&
%\end{align}
(i) $\alpha_{\mathrm{vol}} > 0$ and (ii) $\lambda_{\mathrm{vol}} > 0$, the
risk premium on
the equity (and its futures) is positive; that is,
$\alpha_{\mathrm{vol}}+\lambda_{\mathrm{vol}} \, \mathrm{v}_t  >  0$.

Crucial from our perspective, we will show
(see Corollary~\ref{claimm:SV} below) that the parameter $\theta_{\mathrm{LT}}$
affecting $\theta[t,\mathrm{v}_t]  =  - \theta_{\mathrm{LT}}\, \sqrt{\mathrm{v}_t}$ in (\ref{eq:v2})
determines the sign of the local time risk premium, and, consequently, the call risk premium.
It
holds that local time risk premium exerts \emph{influence} on, and yet (see
(\ref{eq:ConditionForNegVRPInExtendedHeston})) is \emph{economically distinct} from, the variance risk premium.


Next,  (\ref{eq:SimpleHestonDynmaics}) develops the view that random fluctuations
in equity return
variance $\mathrm{v}_t$ can
arise from
both spanned and unspanned diffusive risks, whereas (by definition) the source of randomness
in the equity futures price is solely due to spanned diffusive risks.


For tractability, we assume that the dynamics of return variance, $\mathrm{v}_t$, follow
\begin{eqnarray}
d\mathrm{v}_t  &=& ( \phi_{\mathrm{vol}}^{\mathbb{P}} - \kappa_{\mathrm{vol}}^{\mathbb{P}} \,\mathrm{v}_t )\, dt ~+~
  \sigma_{\mathrm{vol}} \, \sqrt{\mathrm{v}_t} \,\rho_{\mathrm{vol}} \, \underbrace{d z_t^{\mathbb{P}}}_{\mathrm{spanned~risks}}  ~+~ \sigma_{\mathrm{vol}} \sqrt{\mathrm{v}_t} \, \sqrt{1-\rho^2_{\mathrm{vol}}} \, \underbrace{du_t^{\mathbb{P}}.}_{\mathrm{unspanned~risks}} ~~ \mbox{ \, \,  }~ \label{eq:SimpleHestonDynmaics} \\
\mbox{Hence,} & & \frac{d G_t}{G_t} \, \, = \, \, \frac{d F_{t}^{T_F}}{F_{t}^{T_F}} \, = \, \overbrace{( \alpha_{\mathrm{vol}} + \lambda_{\mathrm{vol}} \, \mathrm{v}_t  )}^{\mathrm{futures~risk~premium}}  \, dt ~+~
\sqrt{\mathrm{v}_t} \, \overbrace{d z^{\mathbb{P}}_t.}^{\mathrm{spanned~risks}}~~\mbox{  \, \, }~~
~~\mbox{ \, \, }~~\label{eq:SimpleSquareRootVolStockDynmaics}
\end{eqnarray}

To shed light on option risk premiums and
compensation
for unspanned risks reflected in the risk-neutral drift of the volatility process,
(\ref{eq:SimpleHestonDynmaics}) draws
a distinction between the
one dimensional Brownian motions $z_t^{\mathbb{P}}$ and $u_t^{\mathbb{P}}$.
Additionally,
$\mathrm{cov}^{\mathbb{P}}_{t}(\frac{d F_{t}^{T_F}}{F_{t}^{T_F}}, d\mathrm{v}_t)/dt = \rho_{\mathrm{vol}} \,
\sigma_{\mathrm{vol}} \,\mathrm{v}_t$ and, hence, $\rho_{\mathrm{vol}}$ is
the instantaneous correlation between
$\frac{d F_{t}^{T_F}}{F_{t}^{T_F}}$ and $d\mathrm{v}_t$.

\noindent \textbf{Call risk premium.}
%Considering the nature of spanned and unspanned risks, under $\mathbb{P}$ and $\mathbb{Q}$,
It follows, from (\ref{fg.1}), that
\begin{align}
&
dz_t^{\mathbb{P}} ~-~ dz^{\mathbb{Q}}_t \, = \,
 -\frac{1}{ \sqrt{\mathrm{v}_t}}(
\alpha_{\mathrm{vol}} +\lambda_{\mathrm{vol}} \, \mathrm{v}_t) \, dt&
&\mathrm{and}&
& du_t^{\mathbb{P}} ~-~ du_t^{\mathbb{Q}} \, = \,  -\theta_{\mathrm{LT}} \,\sqrt{\mathrm{v}_t} \, dt.    &
\end{align}
The workings of this theory implies that
the $\mathbb{Q}$ dynamics of $G_t$ and of $\mathrm{v}_t$ are $\frac{dG_t}{G_t} = \sqrt{\mathrm{v}_t} \, dz_t^{\mathbb{Q}}$ and
\begin{eqnarray}
d\mathrm{v}_t & = & ( \phi_{\mathrm{vol}}^{\mathbb{Q}} - \kappa_{\mathrm{vol}}^{\mathbb{Q}} \, \mathrm{v}_t ) \, dt
+ \sigma_{\mathrm{vol}} \, \sqrt{\mathrm{v}_t} \, \rho_{\mathrm{vol}}\, dz_t^{\mathbb{Q}}
+ \sigma_{\mathrm{vol}} \, \sqrt{\mathrm{v}_t}  \, \sqrt{1-\rho^2_{\mathrm{vol}}} \, du_t^{\mathbb{Q}}, ~~ \mbox{ \, \, \, \, \, } \mathrm{with} ~~ \label{volqdy} \\
\kappa_{\mathrm{vol}}^{\mathbb{Q}} & = &\kappa_{\mathrm{vol}}^{\mathbb{P}}
+ \sigma_{\mathrm{vol}} \,\rho_{\mathrm{vol}}\, \lambda_{\mathrm{vol}}
+\underbrace{\theta_{\mathrm{LT}}\,\sigma_{\mathrm{vol}} \,\sqrt{1-\rho^2_{\mathrm{vol}}}}_{\tiny\mathrm{not~in~Heston~(1993)}},
~\mbox{ \, \, and \, \, }~\phi_{\mathrm{vol}}^{\mathbb{Q}} \, = \,  \phi_{\mathrm{vol}}^{\mathbb{P}} - \rho_{\mathrm{vol}} \, \alpha_{\mathrm{vol}} \, \sigma_{\mathrm{vol}}. ~\mbox{ \, \, \, \, }~~~ \label{eq:KappaPKappaQGammaPGammaQ}
\end{eqnarray}

At the heart of this
model are unspanned risks in the dynamics of $M_t$ and $\mathrm{v}_t$,
with theoretical effects conceptually
different from the baseline, as in \citet*[equations (4) and (8)]{Heston:1993}.

\setcounter{theorem}{1}
\begin{corollary}[OTM call risk premium
in a model with unspanned risks]
\label{claimm:SV}
Under the parameterizations
in
(\ref{eq:v1})--(\ref{eq:SimpleHestonDynmaics}),
the call risk premium
can be negative if
$\theta_{\mathrm{LT}} < 0$.
Absent a contribution of unspanned risks (when $\theta_{\mathrm{LT}} = 0$)
or when $\theta_{\mathrm{LT}} \geq 0$,
the call risk premium is positive.  %\vspace{-3mm}
\end{corollary}
\noindent {\bf Proof:} See Internet Appendix (Section~\ref{appsec:SV1}). $\blacksquare$

The restriction $\theta_{\mathrm{LT}} < 0$ implies a negative local time risk
premium. If the latter is sufficiently negative, that is, large enough in magnitude to offset
a positive
equity futures risk premium to the upside,
the call risk premium
can be negative.

\noindent \textbf{Differentiating local time risk premium from variance risk premium.}
We can derive
\begin{equation} \small
\underbrace{\mathbb{E}_t^{\mathbb{P}}( \int_{t}^{T_O} \mathrm{v}_{\ell} d \ell ) - \mathbb{E}_t^{\mathbb{Q}}( \int_{t}^{T_O} \mathrm{v}_{\ell} d \ell )}_{\mathrm{variance~risk~premium}}
~=~  \frac{ \phi_{\mathrm{vol}}^{\mathbb{P}} ( T_0 - t ) - \{ \mathbb{E}_t^{\mathbb{P}}( \mathrm{v}_{T_O} ) - \mathrm{v}_{t} \} }{\kappa_{\mathrm{vol}}^{\mathbb{P}}}
~-~  \frac{ \phi_{\mathrm{vol}}^{\mathbb{Q}} ( T_0 - t ) - \{ \mathbb{E}_t^{\mathbb{Q}}( \mathrm{v}_{T_O} ) - \mathrm{v}_{t} \} }{\kappa_{\mathrm{vol}}^{\mathbb{Q}}}.~~\mbox{ \, }
\label{vrp.1}
\end{equation}
Furthermore, the \emph{unconditional} variance risk premium is $ \, $ $\frac{\phi_{\mathrm{vol}}^{\mathbb{P}} ( T_0 - t )}{\kappa_{\mathrm{vol}}^{\mathbb{P}}} - \frac{\phi_{\mathrm{vol}}^{\mathbb{Q}} ( T_0 - t )}{\kappa_{\mathrm{vol}}^{\mathbb{Q}}}$, $ \, $ which is negative if, and only,
$\frac{\phi_{\mathrm{vol}}^{\mathbb{P}}}{\kappa_{\mathrm{vol}}^{\mathbb{P}}} < \frac{\phi_{\mathrm{vol}}^{\mathbb{Q}}}{\kappa_{\mathrm{vol}}^{\mathbb{Q}}}$
or, rearranging using (\ref{eq:KappaPKappaQGammaPGammaQ}), if, and only if,
\begin{eqnarray}
\frac{\phi_{\mathrm{vol}}^{\mathbb{Q}} + \rho_{\mathrm{vol}}\, \alpha_{\mathrm{vol}} \, \sigma_{\mathrm{vol}}}{\kappa_{\mathrm{vol}}^{\mathbb{Q}} ~+~
\sigma_{\mathrm{vol}} \{-\theta_{\mathrm{LT}}\, \sqrt{1-\rho_{\mathrm{vol}}^2} ~-~ \rho_{\mathrm{vol}}\,\lambda_{\mathrm{vol}}\}} ~<~ \frac{\phi_{\mathrm{vol}}^{\mathbb{Q}}}{\kappa_{\mathrm{vol}}^{\mathbb{Q}}}. ~~~
\text{ \footnotesize
\, \, (for negative variance risk premium) \, \, } ~~ \label{eq:ConditionForNegVRPInExtendedHeston}
\end{eqnarray}

The parametrization $\rho_{\mathrm{vol}} \leq 0$
(and $\alpha_{\mathrm{vol}} > 0$) is instructive, whereby the variance risk premium is negative if the local time
risk premium is negative; that is, if $\theta_{\mathrm{LT}} < 0$.\footnote{The variance risk premium is connected to $\alpha_{\mathrm{vol}}$ and $\lambda_{\mathrm{vol}}$
through the terms
$\rho_{\mathrm{vol}} \, \sigma_{\mathrm{vol}}\, \lambda_{\mathrm{vol}}$ and $\rho_{\mathrm{vol}} \, \alpha_{\mathrm{vol}} \, \sigma_{\mathrm{vol}}$, implying that some of the variance risk premium is entangled with the
equity futures risk premium (when $\rho_{\mathrm{vol}} \neq 0$).
In contrast, the irrelevance
of unspanned risks (i.e., $\theta_{\mathrm{LT}} = 0$) implies zero local time risk premiums for all $k$.}


%{\color{red} [[[ I would have left this in? ]] [[[ While there is strong support in the extant empirical literature for
%$\rho_{\mathrm{vol}} \leq 0$, $\alpha_{\mathrm{vol}} > 0$ and for
%a negative variance risk premium,
%based on equation (\ref{eq:ConditionForNegVRPInExtendedHeston}),
%a negative variance risk premium \emph{could potentially} co-exist with $\theta_{\mathrm{LT}}$ being zero or positive.
%Our innovation is to show that empirical data on option returns
%supports the view that local time risk premiums must be negative and
%%in this model setup,
%the latter requires
%$\theta_{\mathrm{LT}} < 0$.]]]] -- in THE TAKEAWAY ]]]]}

\noindent \textbf{Takeaways.} A suitably
motivated stochastic volatility model
%(under the $\mathbb{P}$ measure)
 can
synthesize negative call risk premiums
provided that certain restrictions are imposed on unspanned risks.
The linchpin of our framework
is that
%equity
volatility embeds unspanned risks and shapes
%moulds
%agrees
%is aligned
%with
a negative
local time risk premium.\footnote{
If jumps in variance (as in \citet*{DuffiePanSingleton:2000} and \citet*{Amengual_Xiu:JOE2018})
were to be added while maintaining the
equity dynamics in (\ref{eq:SimpleSquareRootVolStockDynmaics1}), then Corollary~\ref{claimm:SV} remains valid.}
One may envision
$\theta_{\mathrm{LT}} \{ \sigma_{\mathrm{vol}} \sqrt{ 1 - \rho^2_{\mathrm{vol}}} \}< 0$
(in  (\ref{eq:KappaPKappaQGammaPGammaQ}))
as arising from the feature that unspanned volatility risks are disliked (and $\theta_{\mathrm{LT}}<0$ is needed to produce negative local time risk premiums). \vspace{-3mm}

%\subsection{\bf \small Appendix C: Statement and proof of Lemma~\ref{eq:lemmon} and proof of Corollary~\ref{claimm:SV}}
\subsection{Statement and proof of Lemma~\ref{eq:lemmon} and proof of Corollary~\ref{claimm:SV}}
\label{appsec:SV1}
We focus on the
{\color{magenta} (continuous semimartingale)} setting of equations (\ref{eq:GeneralDynamics1})--(\ref{fg.1})
%{\color{red} [in Section~\ref{gggs}]}
--- with drift and diffusion coefficients unspecified ---  and accomplish
two tasks.
First, we show that if unspanned risks are
irrelevant (i.e., if ${\bm \theta}[t, \mathbf{Y}] =  \mathbf{0}$),
then the local time risk premium is zero. This is our Lemma~\ref{eq:lemmon}.
%{\color{red}[and our aim is to strengthen intuition.]}
Second, we present the proof of Corollary~\ref{claimm:SV} {\color{magenta} (specializing to the setting of
(\ref{eq:v1})--(\ref{eq:SimpleSquareRootVolStockDynmaics})).}

%{\color{blue}We first develop some ``stepping stone" results in the general setting of equations (\ref{eq:GeneralDynamics1})-(\ref{fg.1}).}

%Proceeding,
{\color{magenta} For any} generic claim with payoff $\complement_{T}$, at time $T$, it holds that
\begin{equation}
\mathbb{E}_{t}^{\mathbb{Q}}( \complement_{T} ) = e^{r(T- t)} \,\, \mathbb{E}_{t}^{\mathbb{P}}( \frac{M_{T}}{M_{t}} \,\complement_{T} ).~~~~~\mbox{ \quad } \label{eq:PtoQExpOfPayoffc}
\end{equation}
The risk premium is $\mathbb{E}_{t}^{\mathbb{P}}( \complement_{T} )-\mathbb{E}_{t}^{\mathbb{Q}}( \complement_{T} )$,
which we deduce next. This step is useful because we will set $\complement_{T}= \mathbb{L}^{T_O}_t[k]$, and, hence, we deduce
$\mathbb{E}_{t}^{\mathbb{P}}( \mathbb{L}^{T_O}_t[k] )-\mathbb{E}_{t}^{\mathbb{Q}}( \mathbb{L}^{T_O}_t[k] )$.

By  a property of covariances,
\begin{eqnarray}
\mathrm{cov}_t^{\mathbb{Q}}( \frac{M_{t}}{M_{T} e^{r (T- t)} }, \complement_{T} ) &=&
\mathbb{E}_{t}^{\mathbb{Q}}( \frac{M_{t}}{M_{T} e^{r (T- t)} } \, \complement_{T} ) ~-~
\overbrace{\mathbb{E}_{t}^{\mathbb{Q}}( \frac{M_{t}}{M_{T} e^{r (T- t)} })}^{=1} \,
\mathbb{E}_{t}^{\mathbb{Q}}( \complement_{T} )~~~~~\mbox{ \quad }  \label{c5.1} \\
&=& \mathbb{E}_{t}^{\mathbb{Q}}( \frac{M_{t}}{M_{T} e^{r (T- t)} } \complement_{T} ) ~-~
\mathbb{E}_{t}^{\mathbb{Q}}( \complement_{T} )~~~~~\mbox{ \quad } \label{c5.2} \\
&=& e^{r(T- t)} \mathbb{E}_{t}^{\mathbb{P}}( \frac{M_{T}}{M_{t}} \times \{ \frac{M_{t}}{M_{T} e^{r (T- t)} } \complement_{T} \} ) ~-~
\mathbb{E}_{t}^{\mathbb{Q}}( \complement_{T} ) \label{c5.3}\\
&=& \mathbb{E}_{t}^{\mathbb{P}}( \complement_{T} ) ~-~ \mathbb{E}_{t}^{\mathbb{Q}}( \complement_{T} ).~~~~~\mbox{ \quad }
\label{c5.4}
\end{eqnarray}
Specializing $\complement_{T}$ to the random variable $\mathbb{L}^{T_O}_t[k]$ and guided by (\ref{c5.4}), we obtain the
risk premium for local time with moneyness $k$ as follows:
\begin{equation}
\overbrace{
\mathbb{E}_{t}^{\mathbb{P}}( \mathbb{L}^{T_O}_t[k] ) ~ - ~
\mathbb{E}_{t}^{\mathbb{Q}}( \mathbb{L}^{T_O}_t[k] )}^{\text{Local~time~risk~premium}} ~=~\mathrm{cov}_t^{\mathbb{Q}}( \frac{M_{t}}{M_{{T}_O} e^{r ({T}_O - t)}} , \, \mathbb{L}^{T_O}_t[k] ). ~~\mbox{ \, \, \, \quad } ~
\label{eq:covqgsbstatement}
\end{equation}


%We now state.
\setcounter{theorem}{0}
\begin{lemma}
\label{eq:lemmon}
{\color{magenta} In the
%continuous semimartingale
setting of
%equations
(\ref{eq:GeneralDynamics1})--(\ref{fg.1}),} if unspanned risks  are irrelevant; that is,
\begin{equation}
\mathrm{if}~{\bm \theta}[t, \mathbf{Y}] \, = \, \mathbf{0}, ~~ \mbox{ \, \, } \mathrm{then} ~~ \mbox{ \, } \mathbb{E}_{t}^{\mathbb{P}}( \mathbb{L}^{T_O}_t[k] )  -
\mathbb{E}_{t}^{\mathbb{Q}}( \mathbb{L}^{T_O}_t[k] ) \, = \, 0. ~ \mbox{ \, \, \, \, } ~
%~~\mathrm{in~a~continuous~semimartingale~setting.}
\label{eq:covqgsbstatementxy}
\end{equation}
\end{lemma}

\noindent \textbf{Proof:} Using the
dynamics of the pricing kernel $M_t$
under the $\mathbb{P}$ measure in
(\ref{eq:GeneralDynamics1}), and then the change of measures in (\ref{fg.1}); that is,
\begin{align*}
&d \mathbf{z}^{\mathbb{P}}_t~-~d \mathbf{z}^{\mathbb{Q}}_t = {\bm \eta}[t,\mathbf{Y}_t] \,dt&
&\mathrm{and}&
&d \mathbf{u}^{\mathbb{P}}_t~-~d \mathbf{u}^{\mathbb{Q}}_t =  {\bm \theta}[t,\mathbf{Y}_t] \, dt,&
\end{align*}
%$d \mathbf{z}^{\mathbb{P}}_t-d \mathbf{z}^{\mathbb{Q}}_t = {\bm \eta}[t,\mathbf{Y}_t] dt$ and
%$d \mathbf{u}^{\mathbb{P}}_t-d \mathbf{u}^{\mathbb{Q}}_t =  {\bm \theta}[t,\mathbf{Y}_t] dt$,
it follows
that (hereon
suppressing the time subscript on $\mathbf{Y}_t$)
\begin{equation}
\frac{M_{t}}{M_{T_{O}}e^{r ({T}_O - t)}} = e^{
\int_{t}^{{T}_O} \{
-\frac{1}{2} {\bm \eta}[\ell,\mathbf{Y}]^{\top} {\bm \eta}[\ell,\mathbf{Y}] d\ell
- {\bm \eta}[\ell,\mathbf{Y}]^{\top}  d \mathbf{z}^{\mathbb{Q}}_\ell
- \frac{1}{2} {\bm \theta}[\ell,\mathbf{Y}]^{\top}
{\bm \theta}[\ell,\mathbf{Y}] d\ell
- {\bm\theta}[\ell,\mathbf{Y}]^{\top} d \mathbf{u}^{\mathbb{Q}}_\ell \} }.~\mbox{ \, }~\mbox{ \, }~
\label{eq:IntegratedRecipmGeneralDynamicsUnderQ}
\end{equation}
We note that $\mathbb{E}_{t}^{\mathbb{Q}}( \frac{M_{t}}{M_{T_{O}} e^{r ({T}_O - t)}} ) \, = \, 1$. Let
\begin{equation}
\mathcal{I}_s~\mathrm{be~the~sub\mbox{-}filtration~of}~\mathcal{F}_s~\mathrm{generated~by}~{\bm \eta}^{\mathbb{Q}}_s~\mathrm{and}~{\bm \eta}[s,\mathbf{Y}]^{\top} \,d \mathbf{z}^{\mathbb{Q}}_s. ~ \mbox{ \, \, } ~
\end{equation}

%We denote by $\mathcal{I}_s$ the sub-filtration of $\mathcal{F}_s$ generated by ${\bm \eta}^{\mathbb{Q}}_s$ and ${\bm \eta}[s,\mathbf{Y}]^{'} \,d %\mathbf{z}^{\mathbb{Q}}_s$.


Exploiting the law of total covariance,
\begin{eqnarray}
& & \mathrm{cov}_t^{\mathbb{Q}}( \overbrace{\frac{M_{t}}{M_{{T}_O} e^{r ({T}_O - t)}} }^{\mathrm{from~\tiny(\ref{eq:IntegratedRecipmGeneralDynamicsUnderQ}})}, \, \mathbb{L}_t^{{T}_O}[k] )
\nonumber \\ %\label{eq:NewConditionalCovariance1} \\
& & ~= ~ \mathbb{E}_{t}^{\mathbb{Q}}( \mathrm{cov}_t^{\mathbb{Q}}(
e^{ \int_{t}^{{T}_O} \{ -\frac{1}{2} {\bm \eta}[s,\mathbf{Y}]^{\top} {\bm  \eta}[s,\mathbf{Y}] ds - {\bm \eta}[s,\mathbf{Y}]^{\top} d \mathbf{z}^{\mathbb{Q}}_s
- \frac{1}{2} {\bm \theta}[s, \mathbf{Y}]^{\top} {\bm \theta}[s, \mathbf{Y}] ds - {\bm \theta}[s, \mathbf{Y}]^{\top}
d \mathbf{u}^{\mathbb{Q}}_s \}},
\, \mathbb{L}_t^{{T}_O}[k] {\Big |} \, \mathcal{I}_{T_O} ) )~~\mbox{ \, \, }~~~ \nonumber \\
& & ~\mbox{ \, \, }~\mbox{ \, \, } \mbox{ \quad \quad \quad } \, + \, \mathrm{cov}_t^{\mathbb{Q}}( \underbrace{ \mathbb{E}_{t}^{\mathbb{Q}}( \frac{M_{t}}{M_{{T}_O}} e^{-r ({T}_O - t)} \, \, {\Big |} \mathcal{I}_{T_O} ) }_{\, = \, \,  \mathrm{a~constant} }, \, \, \mathbb{E}_{t}^{\mathbb{Q}}( \mathbb{L}_t^{{T}_O}[k] \, {\Big |} \mathcal{I}_{T_O} ) ) \mbox{ \quad }~\mbox{ \, \, }~  \label{eq:NewConditionalCovariance1Eq} \\
& & ~= ~ \mathbb{E}_{t}^{\mathbb{Q}}( e^{ \int_{t}^{{T}_O} \{ -\frac{1}{2} {\bm \eta}[s,\mathbf{Y}]^{\top} {\bm  \eta}[s,\mathbf{Y}] ds - {\bm \eta}[s,\mathbf{Y}]^{\top} \, \, d \mathbf{z}^{\mathbb{Q}}_s \} } \, \, \times
\nonumber \\
& & ~\mbox{ \, \, }~\mbox{ \, \, } \mbox{ \quad \quad \, } \mathrm{cov}_t^{\mathbb{Q}}(
e^{ \int_{t}^{{T}_O} \{ - \frac{1}{2} {\bm \theta}[s, \mathbf{Y}]^{\top} {\bm \theta}[s, \mathbf{Y}] ds -
{\bm \theta}[s, \mathbf{Y}]^{\top} \,
d \mathbf{u}^{\mathbb{Q}}_s \}}, \, \mathbb{L}_t^{{T}_O}[k] \,{\Big |} \mathcal{I}_{T_O} ) ),
\label{eq:NewConditionalCovariance2Eq}
\end{eqnarray}
because the covariance of the two expectations in
(\ref{eq:NewConditionalCovariance1Eq}) vanishes since one term
(conditional on $\mathcal{I}_{T_O}$) is a constant.
If it were the case that ${\bm \theta}[t, \mathbf{Y}]$ is identically zero and, thus,
${\bm \theta}[s, \mathbf{Y}]^{\top} d \mathbf{u}^{\mathbb{Q}}_s=0$,
then, by
%equation
(\ref{eq:NewConditionalCovariance2Eq}), $\mathrm{cov}_t^{\mathbb{Q}}( \frac{M_{t}}{M_{{T}_O} e^{r ({T}_O - t)}} , \, \mathbb{L}^{T_O}_t[k] ) = 0$.

Then, by (\ref{eq:covqgsbstatement}), $\mathbb{E}_{t}^{\mathbb{P}}( \mathbb{L}^{T_O}_t[k] ) - \mathbb{E}_{t}^{\mathbb{Q}}( \mathbb{L}^{T_O}_t[k] ) = 0$. $\square$ \vspace{2mm}

Based on (\ref{eq:NewConditionalCovariance2Eq}), the sign of the
local time risk premium inherits the sign of the $\mathbb{Q}$ measure conditional covariance
$\mathrm{cov}_t^{\mathbb{Q}}(
e^{ \int_{t}^{{T}_O} \{ - \frac{1}{2} {\bm \theta}[s, \mathbf{Y}]^{\top} {\bm \theta}[s, \mathbf{Y}] ds -
{\bm \theta}[s, \mathbf{Y}]^{\top} \,
d \mathbf{u}^{\mathbb{Q}}_s \}}, \, \mathbb{L}_t^{{T}_O}[k] \,{\Big |} \mathcal{I}_{T_O} )$.
More specifically, (\ref{eq:covqgsbstatement}) and (\ref{eq:NewConditionalCovariance2Eq}) indicate
that the local time risk premium $\mathbb{E}_{t}^{\mathbb{P}}( \mathbb{L}^{T_O}_t[k] ) -
\mathbb{E}_{t}^{\mathbb{Q}}( \mathbb{L}^{T_O}_t[k] )$ is negative, if and only if,
%{\color{red}[[[ removed ``for all $t$." ]]]}
\begin{equation}
\mathrm{cov}_t^{\mathbb{Q}}(
e^{ \int_{t}^{{T}_O} \{ - \frac{1}{2} {\bm \theta}[s, \mathbf{Y}]^{\top} {\bm \theta}[s, \mathbf{Y}] ds -
{\bm \theta}[s, \mathbf{Y}]^{\top} \, \,
d \mathbf{u}^{\mathbb{Q}}_s \}}, \, \mathbb{L}_t^{{T}_O}[k] \,{\Big |} \mathcal{I}_{T_O} ) < 0.
%~~~~~~~~\mbox{ \, \, }
\label{c5.10}
\end{equation}
In this continuous semimartingale setting, it further holds that the
\begin{equation}
\mathrm{sign~of}~\mathbb{E}_{t}^{\mathbb{P}}( \mathbb{L}^{T_O}_t[k] ) -
\mathbb{E}_{t}^{\mathbb{Q}}( \mathbb{L}^{T_O}_t[k] )~\mathrm{is~the~sign~of}~
\mathrm{cov}_t^{\mathbb{Q}}( \int_{t}^{{T}_O} - {\bm \theta}[s, \mathbf{Y}]^{\top} \,
d \mathbf{u}^{\mathbb{Q}}_s, \, \mathbb{L}_t^{{T}_O}[k] \,{\Big |} \mathcal{I}_{T_O} ). ~
\label{c5.11}
\end{equation}
With $\sum_{t \leq h \leq \ell} (G_{h} - G_{h -})^2  =  0$
(no jumps for any $h$)
for continuous semimartingales, $\mathbb{L}^{T_O}_t[k] = \frac{1}{2} \int_{t}^{T_O}
\delta_{\{G_\ell ~-~ k\}} d [ G, G ]_{\ell}$, where
$\delta_{\{\bullet\}}$ is the Dirac delta function and $[ G, G ]_{\ell}$ is the quadratic variation.

\noindent \textbf{Proof of Corollary~\ref{claimm:SV}.}  Mindful of
%equations
(\ref{c5.10}) and (\ref{c5.11}),
we return to our model in
equations (\ref{eq:v1})--(\ref{eq:SimpleHestonDynmaics}) and derive the form of
$\mathbb{L}_t^{{T}_O}[k]$. To do so, note that the evolution of variance satisfies
\begin{eqnarray}
\mathrm{v}_{\ell} & =  & \mathrm{v}_{t}\, e^{ \kappa_{\mathrm{vol}}^{\mathbb{Q}} ( t - \ell ) } + \int_{t}^{\ell} \phi_{\mathrm{vol}}^{\mathbb{Q}} e^{ \kappa_{\mathrm{vol}}^{\mathbb{Q}} ( s - \ell ) } ds \nonumber \\
&+&\sigma_{\mathrm{vol}}\,\rho_{\mathrm{vol}} \int_{t}^{\ell}  e^{ \kappa_{\mathrm{vol}}^{\mathbb{Q}} ( s - \ell ) } \sqrt{\mathrm{v}_{s}} \,  dz_s^{\mathbb{Q}}
+\sigma_{\mathrm{vol}} \, \sqrt{1-\rho^2_{\mathrm{vol}}} \int_{t}^{\ell}  e^{ \kappa_{\mathrm{vol}}^{\mathbb{Q}} ( s - \ell ) } \sqrt{\mathrm{v}_{s}} \,  du_s^{\mathbb{Q}},~~~\mbox{ for $\ell \geq t$.\, \, \, }~~
\label{cv.s}
\end{eqnarray}
It further holds that $dG_t  = \sqrt{\mathrm{v}_t} \, G_t \, dz_t^{\mathbb{Q}}$. Hence, the \emph{quadratic variation}
$[ G, G ]_{s}$ is
 \begin{equation}
[ G, G ]_{s}  = \, \int_{t}^s \{ \sqrt{\mathrm{v}_{\ell}} \, G_{\ell} \}^2 \, d\ell \, \, = \, \int_{t}^s \, \mathrm{v}_{\ell} \, G_\ell^2 \, d\ell. ~ \mbox{ \, } ~ ~~
\end{equation}
Using the differential form $d [ G, G ]_{\ell} =  \mathrm{v}_{\ell} \, G_\ell^2 \, d \ell$, we deduce, from (\ref{ltt.1}) in
conjunction with
$\sum_{t \leq h \leq \ell} (G_{h} - G_{h -})^2  =  0$ (no jumps for any $h$),
that
\begin{eqnarray}
\mathbb{L}^{T_O}_t[k] &= & \frac{1}{2} \int_{t}^{T_O} \delta_{\{G_\ell ~-~ k\}} d [ G, G ]_{\ell}
~=~  \frac{1}{2} \, \int_{t}^{{T}_O} \, \delta_{\{G_\ell ~-~ k\}} \, \mathrm{v}_{\ell} \, G_\ell^2 \, d\ell. ~ \mbox{ \, } ~ ~~
\end{eqnarray}

Specializing ${\bm \theta}[t, \mathbf{Y}]$ to
${\bm \theta}[t, \mathbf{Y}]  =  - \, \theta_{\mathrm{LT}} \sqrt{\mathrm{v}_t}$ as per our setup,
we obtain the sign of the $\mathbb{Q}$-measure conditional
covariance, in light of equations (\ref{c5.10}) and (\ref{c5.11}), as follows:
\begin{eqnarray}
& & \mathrm{cov}_t^{\mathbb{Q}}( \int_{t}^{{T}_O} - {\bm \theta}[s, \mathbf{Y}]^{'} \, \,
d \mathbf{u}^{\mathbb{Q}}_s, \, \mathbb{L}_t^{{T}_O}[k] \,{\Big |} \mathcal{I}_{T_O} ) \nonumber \\
& & =~ \mathrm{cov}_t^{\mathbb{Q}}( \int_{t}^{{T}_O} -
\{-\theta_{\mathrm{LT}} \, \sqrt{\mathrm{v}_s} \, \, d u^{\mathbb{Q}}_{s}\},
 \,
\frac{1}{2} \, \int_{t}^{{T}_O} \, \delta_{\{G_\ell ~-~ k\}} \, \mathrm{v}_{\ell} \, G_\ell^2 \, d\ell \, \,{\Big |} \mathcal{I}_{T_O} ) \nonumber \\
& & =~ \mathrm{cov}_t^{\mathbb{Q}}( \int_{t}^{{T}_O}
\theta_{\mathrm{LT}} \, \sqrt{\mathrm{v}_s} \, \, d u^{\mathbb{Q}}_{s},
\frac{1}{2} \int_{t}^{{T}_O} \int_{t}^{\ell} \, \sigma_{\mathrm{vol}} e^{\kappa_{\mathrm{vol}}^{\mathbb{Q}}( s - \ell )} \sqrt{\mathrm{v}_s} \, \sqrt{1-\rho_{\mathrm{vol}}^2} \,
du^{\mathbb{Q}}_{s} \, \delta_{\{G_\ell ~-~ k\}} \, G_\ell^2  d \ell  \,{\Big |} \mathcal{I}_{T_O} ) \mbox{ \, }~\mbox{ \, \, } \nonumber \\
& & =~ \mathrm{cov}_t^{\mathbb{Q}}( \int_{t}^{{T}_O} \theta_{\mathrm{LT}} \, \sqrt{\mathrm{v}_s} \, \, \, d u^{\mathbb{Q}}_{s},
\, \int_{t}^{{T}_O} \sqrt{\mathrm{v}_{s}} \, \{\ \int_{s}^{{T}_O} \frac{\sigma_{\mathrm{vol}}}{2} e^{\kappa_{\mathrm{vol}}^{\mathbb{Q}} ( s - \ell ) } \, \sqrt{1-\rho_{\mathrm{vol}}^2} \, \delta_{\{G_\ell ~-~ k\}} \, G_\ell^2 \, d\ell \} \, du^{\mathbb{Q}}_{s} \,{\Big |} \mathcal{I}_{T_O} )   \nonumber \\
& & =~ \mathbb{E}_{t}^{\mathbb{Q}}( \int_{t}^{{T}_O} \theta_{\mathrm{LT}} \, \sqrt{\mathrm{v}_s} \, \, \sqrt{\mathrm{v}_s} \,
\, \{\ \int_{s}^{{T}_O} \frac{\sigma_{\mathrm{vol}} }{2} e^{\kappa_{\mathrm{vol}}^{\mathbb{Q}} ( s - \ell ) } \, \sqrt{1-\rho_{\mathrm{vol}}^2} \, \delta_{\{G_\ell ~-~ k\}}
\, G_\ell^2 \, d\ell \} \,  ds \, {\Big |} \mathcal{I}_{T_O} )  \nonumber \\
& & =~ \theta_{\mathrm{LT}} \, \, \,
\underbrace{\mathbb{E}_{t}^{\mathbb{Q}}(  \int_{t}^{{T}_O} \, \mathrm{v}_s \, \, \{\ \int_{s}^{{T}_O} \frac{\sigma_{\mathrm{vol}}}{2}
e^{ \kappa_{\mathrm{vol}}^{\mathbb{Q}} ( s - \ell ) } \, \sqrt{1-\rho_{\mathrm{vol}}^2} \, \delta_{\{G_\ell ~-~ k\}}
\, G_\ell^2 \, d\ell \}\,  ds \,{\Big |} \mathcal{I}_{T_O} )}_{~\geq ~0}.
\label{eq:FinalCovarianceTerm3Heston}
\end{eqnarray}
Inspection of (\ref{eq:FinalCovarianceTerm3Heston}) shows
that the
\begin{align}
&\mathrm{local~time~risk~premium~is~negative,~if~and~only~if,}~\theta_{\mathrm{LT}} < 0.&
%&&
%& \mbox{(unspanned volatility risks are disliked)}&
\label{intuu}
\end{align}
When $\theta_{\mathrm{LT}} < 0$, by
Corollary~\ref{claimm:claim1call}, the  OTM call risk premium can be negative. The intuition behind
(\ref{intuu}) is that unspanned volatility risks are disliked.

%If the equity variance dynamics {\color{blue}were to} contain only spanned risks, it
%%{\color{blue}would correspond} to
%%{\color{blue}$\sqrt{1-\rho_{\mathrm{vol}}^2} = 0$ and
%%the economic effect would be as if $\theta_{\mathrm{LT}}= 0$ in the sense that
%%we find that the covariance in equation (\ref{eq:FinalCovarianceTerm3Heston}) would be zero,
%the local time risk premium would be zero and, hence, the risk premium of OTM calls would be positive.

%Elaborating on the workings of
Commenting on the steps in  (\ref{eq:FinalCovarianceTerm3Heston}), the second line of (\ref{eq:FinalCovarianceTerm3Heston}) recognizes that only the term $\sigma_{\mathrm{vol}} \, \sqrt{1-\rho^2_{\mathrm{vol}}} \int_{t}^{\ell}  e^{ \kappa_{\mathrm{vol}}^{\mathbb{Q}} ( s - \ell ) } \sqrt{\mathrm{v}_{s}} \, \, du_s^{\mathbb{Q}}$ in (\ref{cv.s}) is relevant for the covariance. Furthermore, the third line changes the order of integration. Finally, the fourth line uses Ito's isometry formula. $\blacksquare$
\vspace{-3mm}

\section{JUNK FROM Appendix A}

{\color{red}[[[[ Lemma \ref{eq:lemmon} will need significant editing. It is \textbf{{\color{magenta}\underline{\large{NOT}}}} very relevant for Theorem 1 anyway so I have placed it after
Theorem~\ref{claimm:claim1call_jump} --- not inside it.

{\color{magenta}Here is what we say in the text about Lemma \ref{eq:lemmon}.

``\textit{
We will establish that the source of the local time risk premiums is also unspanned risks. Specifically, if there are no
unspanned risks, then the local time risk premium \emph{is zero} for every $k$.
This is our Lemma~\ref{eq:lemmon}, and we prove this result as a part of the proof of Theorem \ref{claimm:claim1call_jump} (in Appendix~\ref{appsec:jumppps})."}}

We know this isn't true.
What is true is: {\color{magenta}``Specifically, if there are no
unspanned risks, then the risk premium for call options \emph{is positive} for every $k$ (provided only that the risk premium on the equity index is positive)."}

]]]] [[[[[[
\noindent \textbf{IV.
Local time risk premium is zero if unspanned risks are absent or not relevant.} Recall that $\mathbb{L}^{T_O}_t[k]=\frac{1}{2} \int_{t}^{T_O} \delta_{\{G_\ell ~-~ k\}}\,d [ G^\mathrm{c}, G^\mathrm{c} ]_{\ell}$, where $\delta_{\{\bullet\}}$ is the Dirac delta function and $[ G^\mathrm{c}, G^\mathrm{c} ]_{\ell}=
[ G, G ]_{\ell} - \sum_{t \leq h \leq \ell} (G_{h} - G_{h -})^2$ is the path-by-path
continuous part of the quadratic variation. Thus, local time is unaffected by jumps in the evolution of $F_{t}^{T_F}$.
\setcounter{theorem}{0}
\begin{lemma}
\label{eq:lemmon}
Suppose $F_{t}^{T_F}$ is a continuous semimartingale. Additionally, suppose that return volatility and/or the pricing kernel
is \emph{absent} of unspanned risks, or that unspanned volatility risks are not priced. It holds that
\begin{equation}
%\mathrm{if}~{\bm \theta}[t, \mathbf{Y}] \, = \, \mathbf{0}, ~~ \mbox{ \, \, } \mathrm{then} ~~
\mbox{ \, }
\underbrace{\mathbb{E}_{t}^{\mathbb{P}}( \mathbb{L}^{T_O}_t[k] )  ~-~
\mathbb{E}_{t}^{\mathbb{Q}}( \mathbb{L}^{T_O}_t[k] )}_{\mathrm{local~time~risk~premium~for~moneyness}~k} \, = \, \, 0. ~ \mbox{ \, \, \, \, } ~
%~~\mathrm{in~a~continuous~semimartingale~setting.}
\label{eq:covqgsbstatementxy}
\end{equation}
\end{lemma}

\noindent \textbf{Proof:} The proof is tedious and requires the setup of pricing kernel dynamics with spanned and unspanned (diffusive)
risks. We present the proof in the Internet Appendix~(Section~\ref{appsec:SV1}). $\square$

We have the proof of Theorem~\ref{claimm:claim1call_jump}.
%{\color{red} [Corollary~\ref{claimm:claim1call}, and Corollary~\ref{claimm:straddles}.]}
$\blacksquare$ \vspace{-3mm}

]]]]]]]]}

%%%%%%%%%%%%%%%%%%%%%%%%%
%%%%%%%%%%%%%%%%%%%%%%%%%

{\color{blue}
\setcounter{theorem}{0}
\begin{lemma}
\label{eq:lemmon}
We maintain that the risk premium on the equity index is positive.

Suppose $F_{t}^{T_F}$ is a continuous semimartingale.

$\bullet$ Additionally, suppose that return volatility and/or the pricing kernel
is \emph{absent} of unspanned risks, or that unspanned volatility risks are not priced. It holds that: The risk premium for call options \emph{is positive} for every $k$.

$\bullet$ Suppose, that the risk premium for call options is negative for some $k$. Then: (a) There must be unspanned risks in the pricing kernel; and (b) the local time risk premium (for that value of $k$) must be negative.

\end{lemma}
\noindent \textbf{Proof:} The proof is tedious and requires the setup of pricing kernel dynamics with spanned and unspanned (diffusive)
risks. We present the proof in the Internet Appendix~(Section~\ref{appsec:SV1}). $\square$}




\end{document}


\end{document}

%%%%%%%%%%%%%%%%%%%%%%%%%%%%%%%%%%%%%%%%%%%%%%%%%%%%%%%%%%%%%%%%%%%%%%%%%%%%%%
%%%%%%%%%%%%%%%%%%%%%%%%%%%%%%%%%%%%%% END %%%%%%%%%%%%%%%%%%%%%%%%%%%%%%%%%%%

%%%%%%%%%%%%%%%%%%%%%%%%%%%%%%%%%%%%%% END %%%%%%%%%%%%%%%%%%%%%%%%%%%%%%%%%%%
%\end{document}
\newpage
%==*==*==*==*==*==*==*==*==*==*==*==*==*==*==*==*==*== Tables
 %\setcounter{table}{0}  % reset counter

%\renewcommand{\thetable}{\arabic{table}}

 %\setcounter{figure}{0}  % reset counter

%\renewcommand{\thefigure}{\arabic{figure}}

%  \renewcommand{\theequation}{\arabic{equation}}
%  \setcounter{equation}{0}  % reset counter
%\renewcommand{\thetheorem}{\arabic{theorem}}
%  \setcounter{theorem}{0}  % reset counter

\newpage
\thispagestyle{empty}
\clearpage



\begin{center}
{\Large{Dark Matter in (Volatility and) Equity Option Risk Premiums}} \\
\vspace{0.04in}
%Gurdip Bakshi~~~John Crosby~~~Xiaohui Gao \\
\textbf{\underline{Internet Appendix: Not for Publication}}
\end{center}
%\vspace{1mm}
\begin{center}
\textbf{Abstract}
\end{center}

\noindent {\color{magenta} This Internet Appendix focuses on the continuous semimartingale theoretical environment (i.e.,
the jumps crossing the strike terms --- $a_t^{T_O}[k]$, $b_t^{T_O}[k]$, $c_t^{T_O}[k]$, and $d_t^{T_O}[k]$ --- are all zero).
Corollary~\ref{claimm:claim1call} is about option risk premiums when there are unspanned diffusive risks in the dynamics of the pricing kernel
and the volatility. Corollary~\ref{claimm:SV} shows that a suitably motivated stochastic volatility model (under the $\mathbb{P}$ measure)
can synthesize negative call risk premiums provided that certain restrictions are imposed on unspanned risks. Finally,
Lemma~\ref{eq:lemmon} shows that if unspanned risks  are irrelevant, then the local time risk premium is zero.}


%\newpage
\thispagestyle{empty}
%\clearpage

\thispagestyle{empty}
% *************** start of text ****************************************
\newpage
\setcounter{page}{1}
\renewcommand{\thefootnote}{\arabic{footnote}}
\setcounter{footnote}{0}

%\setcounter{equation}{0}
%\renewcommand{\theequation}{B\arabic{equation}}

\setcounter{section}{0}
\renewcommand{\thesection}{\Roman{section}}
%\renewcommand{\thesection}{\Roman{section}}
%\renewcommand{\thesubsection}{\thesection.\Roman{subsection}}
\renewcommand{\thesubsection}{\thesection.\arabic{subsection}}
%\renewcommand{\thesubsection}{\Roman{subsection}}

%\section{ \bf \large Internet Appendix}

                                                        \setcounter{equation}{0}
                                                        \renewcommand{\theequation}{IA-\arabic{equation}}
                                                        %\setcounter{section}{0}
%\numberwithin{equation}{section}
\numberwithin{table}{section}
\numberwithin{theorem}{section}
\numberwithin{figure}{section}
%                                            \numberwithin{theorem}{subsection}

%%%%%%%%%%%%%%%%%%%%%%%%%%%%%%%%%%%%%%%% Begin I %%%%%%%%%%%%%%%%%%%%%%%%%%%%%%%%%%%%%%%%%%%%%%%%%%%%%
%%%%%%%%%%%%%%%%%%%%%%%%%%%%%%%%%%%%%%%%%%%%%%%%%%%%%%%%%%%%%%%%
                                                        \setcounter{equation}{0}
                                                        \renewcommand{\theequation}{I\arabic{equation}}
%\begin{center}
%\textbf{Internet Appendix}
%\end{center}
%%%%%%%%%%%%%%%%%%%%%%%%%%%%%%%%%%%%%%%%%%%%%%%%%%%%%%%%%%%%%%%%%%%%%%%%%%%%%%%%%%%%%%%%%%%%%%%%%%%%%%%%%%%%
%%%%%%%%%%%%%%%%%%%%%%%%%%%%%%%%%%%%%%%%%%%%%%%%%%%%%%%%%%%%%%%%%%%%%%%%%%%%%%%%%%%%%%%%%%%%%%%%%%%%%%%%%%%%
\section{Unspanned risks in a continuous semimartingale setting}
\label{seimimartingales_continuous}

{\color{magenta} In a continuous semimartingale setting, the jumps crossing the strike terms, namely,
$a_t^{T_O}[k]$, $b_t^{T_O}[k]$, $c_t^{T_O}[k]$, and $d_t^{T_O}[k]$, vanish.} \vspace{-3mm}

%{\color{green} A continuous semimartingale theoretical environment can be revealing for three reasons.
%First, the jumps crossing the strike terms ---
%$a_t^{T_O}[k]$, $b_t^{T_O}[k]$, $c_t^{T_O}[k]$, and $d_t^{T_O}[k]$ --- \emph{vanish}.
%Second, one can delineate the distinction between spanned and unspanned \emph{diffusive} risks. Third, the risk premium adjustments that
%link $\mathbb{P}$ to $\mathbb{Q}$ are explicit through Girsanov's change of measure theorem.} \vspace{-3mm}

\subsection{Implications of local time being the only source of dark matter}
\label{gggs}

In what follows, we dichotomize between spanned and unspanned risks in the following manner. \vspace{-2mm}
\begin{align}
&\mathrm{Let}~\mathbf{z}^{\mathbb{P}}_t~\mbox{denote a vector of independent standard Brownian motions under}~\mathbb{P}.& \\
&\mathrm{Additionally,}~\mathbf{u}^{\mathbb{P}}_t~\mbox{is another vector of independent standard Brownian motions under}~\mathbb{P}. \mbox{ \, } &
\end{align}

\noindent \textbf{Model.} By assumption, $\mathbf{z}^{\mathbb{P}}_t$ is  spanned,
while $\mathbf{u}^{\mathbb{P}}_t$ \emph{cannot} be spanned by equity futures.
%{\color{blue} Risks not spanned by equity futures could be spanned by options written on the equity futures
%(or on the equity).}
With the vector of state variables denoted by $\mathbf{Y}_t$,
consider the system of stochastic differential equations (SDEs) for
the pricing kernel $M_t$ and for the equity index $S_{t}$, as follows:
\begin{eqnarray}
\frac{d M_t}{M_t} & = & -r\, dt
~+~{\bm\eta}[t,\mathbf{Y}_t]^{\top} \, \underbrace{d \mathbf{z}^{\mathbb{P}}_t}_{\text{\tiny{spanned~risks}}}
~+~{\bm\theta}[t,\mathbf{Y}_t]^{\top} \, \underbrace{d \mathbf{u}^{\mathbb{P}}_t,}_{\text{\tiny{unspanned~risks}}}
\mbox{ \, \, \, }
\label{eq:GeneralDynamics1} \\
%r&=&\mathrm{spot~interest~rate,~assumed~constant,}~ \\
\frac{d S_{t}}{S_{t}} & =&
(r ~-~ \underbrace{{\bm \eta}[t,\mathbf{Y}_t]^{\top} \mathbf{V}[t,\mathbf{Y}_t]}_{=~\mathrm{cov}^{\mathbb{P}}_t( \frac{dM_t}{M_t}, \frac{d S_t}{S_t})/dt})
\, dt ~+~ \mathbf{V}[t,\mathbf{Y}_t]^{\top}
\underbrace{d \mathbf{z}^{\mathbb{P}}_t,}_{\text{\tiny{spanned~risks}}} \label{eq:index3} \\
\underbrace{d \,\mathbf{V}[t,\mathbf{Y}_t]}_{\tiny \mathrm{volatility~dynamics}} & = & {\bm \mu}_V[t,\mathbf{Y}_t] \,dt
~+~ \underbrace{{\bm \sigma}_{V,z}[t,\mathbf{Y}_t]}_{\mathrm{matrix}} \underbrace{d \mathbf{z}^{\mathbb{P}}_t}_{\text{\tiny{spanned~risks}}}
~+~ \underbrace{{\bm \sigma}_{V,u}[t,\mathbf{Y}_t]}_{\mathrm{matrix}}  \underbrace{d \mathbf{u}^{\mathbb{P}}_t,}_{\text{\tiny{unspanned~risks}}}
%\mathrm{and}
%~~ ~
%~\mbox{and}
\label{eq:index3VolDy} \\
\underbrace{\frac{dF_{t}^{T_F}}{F_{t}^{T_F}}}_{\tiny \mbox{using (\ref{fuut})}} & = &
-{\bm \eta}[t,\mathbf{Y}_t]^{\top} \mathbf{V}[t,\mathbf{Y}_t]
 \,dt
~+~ \mathbf{V}[t,\mathbf{Y}_t]^{\top} \underbrace{d \mathbf{z}^{\mathbb{P}}_t,}_{\mathrm{spanned~risks}}
\label{eq:FutP}
\end{eqnarray}
where $\mathbf{V}[t,\mathbf{Y}_t]$ is a vector conformable
with $\mathbf{z}^{\mathbb{P}}_t$. The notation ${\top}$ represents transpose of a vector. The dynamics of $\frac{d G_\ell}{G_\ell}$ coincides with those of $\frac{d F_{\ell}^{T_F}}{F_{\ell}^{T_F}}$ for all $\ell$ satisfying $t \leq \ell \leq T_F$.

Our differentiating element
is that the standard Brownian motions $\mathbf{u}^{\mathbb{P}}_t$ are present
in the SDEs for $M_t$ and for $\mathbf{V}[t,\mathbf{Y}_t]$.
The introduction of $\mathbf{u}^{\mathbb{P}}_t$ is akin to a form of market incompleteness,
and our treatment of $\mathbf{u}^{\mathbb{P}}_t$, and its risk compensation, is pertinent to our theoretical analysis,
empirical identifications, and the basis of what we call ``dark matter."

By assumption, ${\bm\eta}[t,\mathbf{Y}_t]^{\top} d \mathbf{z}^{\mathbb{P}}_t$ is spanned by $\mathbf{V}[t,\mathbf{Y}_t]^{\top} d \mathbf{z}^{\mathbb{P}}_t$. Thus, it must hold that
\begin{eqnarray}
{\bm\eta}[t,\mathbf{Y}_t]^{\top} \, {\bm\eta}[t,\mathbf{Y}_t] & = & \mathfrak{g}_t \, \mathbf{V}[t,\mathbf{Y}_t]^{\top} \, \mathbf{V}[t,\mathbf{Y}_t], ~~ \mbox{ \, \, \, for all $t$, \, \, \, for some scalar variable $\mathfrak{g}_t$. \, \, \, \, } ~ ~ \label{eq:RestrictionOnEta}
\end{eqnarray}

The feature that the volatility
dynamics (i.e., those of $\mathbf{V}[t,\mathbf{Y}_t]$) contains
unspanned risks --- the randomness that cannot be removed
by trading in equity futures
--- is data-motivated and essential to
our developments. The economic effect of $\mathbf{u}^{\mathbb{P}}_t$
is notable, because in its absence, the risk premium on local time
would
be \emph{zero} in
this
continuous semimartingale setting
(soon to be formalized).\footnote{Since Brownian shocks are amenable to ``rotation," one may ask:
Could one have an alternative but equivalent representation, in which $\mathbf{u}^{\mathbb{P}}_t$ is a part of
$S_t$ and $M_t$ dynamics, but not of
volatility?
Our \emph{definition} of unspanned risks aligns with a notion that $\mathbf{u}^{\mathbb{P}}_t$
appear in the dynamics of the latent variables (i.e., in $M_t$ and in $\mathbf{V}[t,\mathbf{Y}_t]$).}
This potentially separates us from
other studies on equity volatility.

In equations (\ref{eq:GeneralDynamics1})--(\ref{eq:FutP}),
the drift and diffusion coefficients may
depend upon
$\mathbf{Y}_t$
and
are adapted to $\mathcal{F}_t$.
At this stage, we do not specify which
economic
variables enter
$\mathbf{Y}_t$.
In general, the dynamics of $\mathbf{Y}_t$
will
impact the drift (under $\mathbb{P}$) and diffusions of $\frac{d S_{t}}{S_{t}}$ and the form of risk compensation associated with
the spanned and unspanned components of
the pricing kernel.\footnote{Additionally, we assume that the drift and diffusion coefficients
are differentiable so that Ito's lemma can be applied, and
they are sufficiently regular so that the SDEs
have a unique solution. In particular,
the vector
${\bm \mu}_V[t,\mathbf{Y}_t]$, and conformable matrices
${\bm \sigma}_{V,z}[t,\mathbf{Y}_t]$ and ${\bm \sigma}_{V,u}[t,\mathbf{Y}_t]$
in (\ref{eq:index3VolDy}) must be such that elements of $\mathbf{V}[t,\mathbf{Y}_t]$ are nonnegative.
See \citet*[pages 364--366]{cir:85a} for the regularity conditions on the SDEs, including that
the covariance matrices be nonnegative definite.
Furthermore, we
preclude that
$\frac{dS_t}{S_t}$ is perfectly
correlated with increments to its variance.
{\color{green} The work of \citet*{Bates:2000} provides the context
for a two-factor model of return volatility.}}

In light of Girsanov's theorem,
$\mathbf{z}^{\mathbb{Q}}_t$ and $\mathbf{u}^{\mathbb{Q}}_t$ are
vectors of independent standard Brownian motions under the probability measure $\mathbb{Q}$,
linked to $\mathbf{z}^{\mathbb{P}}_t$ and $\mathbf{u}^{\mathbb{P}}_t$, by
\begin{align}
&d \mathbf{z}^{\mathbb{P}}_t~-~d \mathbf{z}^{\mathbb{Q}}_t = {\bm \eta}[t,\mathbf{Y}_t] \,dt&
&\mathrm{and}&
&d \mathbf{u}^{\mathbb{P}}_t~-~d \mathbf{u}^{\mathbb{Q}}_t =  {\bm \theta}[t,\mathbf{Y}_t] \, dt.&
\label{fg.1}
\end{align}
The dynamics of $\frac{d G_\ell}{G_\ell} = \frac{d F_{\ell}^{T_F}}{F_{\ell}^{T_F}}$ under $\mathbb{Q}$,
from (\ref{eq:FutP}), becomes $\frac{d G_\ell}{G_\ell} = \mathbf{V}[\ell,\mathbf{Y}_{\ell}]^{\top} d \mathbf{z}^{\mathbb{Q}}_{\ell}.$

\setcounter{theorem}{0}
\begin{corollary}[Continuous semimartingales]
\label{claimm:claim1call}
The following are true:
\begin{enumerate}
\item The OTM call risk premium %($k>1$)
\emph{can} be negative only if $\mathbb{E}^{\mathbb{P}}_{t}( \mathbb{L}^{T_O}_t[k] ) - \mathbb{E}^{\mathbb{Q}}_{t}( \mathbb{L}^{T_O}_t[k] ) <0$.
It is
positive if
$\mathbb{E}_{t}^{\mathbb{P}}( \int_{t}^{{T}_O} \mathbbm{1}_{\{G_{\ell} > k\}} \,dG_{\ell} ) >
- \{\mathbb{E}_{t}^{\mathbb{P}}( \mathbb{L}^{T_O}_t[k] )
-  \mathbb{E}_{t}^{\mathbb{Q}}( \mathbb{L}^{T_O}_t[k] ) \}$.

\item The {\color{magenta} straddle risk premium is zero (respectively, negative) if, and only if, the local time risk premium
(and hence the dark matter risk premium) for
$k=1$ is zero (respectively, negative).}

\end{enumerate}
\end{corollary}

\noindent {\bf Proof:} We specialize Theorem~\ref{claimm:claim1call_jump} to a continuous semimartingale setting (see Appendix~\ref{appsec:jumppps}). $\blacksquare$



\normalsize

Our results on call risk premium
are introduced
without parameterizing the diffusion or drift coefficients
of $M_t$ and $F^{T_F}_t$. In so doing, we highlight the mechanism of unspanned risks,
and the rationale that a negative local time (and, hence, dark matter) risk premium
could help to understand
puzzling data
features in the equity markets.
%{\color{red}[[[[ Our setup suggests the value of refining extant modeling frameworks to include unspanned
%risks in the equity  volatility dynamics. ]]]]}

%\begin{corollary}[Continuous semimartingales]%Expected excess returns of OTM calls ($k>1$)]
%\label{claimm:claim1call}
%The OTM call risk premium %($k>1$)
%%, $1 + \mu^{{T}_O}_{t,{\tiny \mathrm{call}}}[k] - e^{r (T_O-t)}$,
%\emph{can} be negative only if $\mathbb{E}^{\mathbb{P}}_{t}( \mathbb{L}^{T_O}_t[k] ) - \mathbb{E}^{\mathbb{Q}}_{t}( \mathbb{L}^{T_O}_t[k] ) <0$.
%It is
%positive if
%$\mathbb{E}_{t}^{\mathbb{P}}( \int_{t}^{{T}_O} \mathbbm{1}_{\{G_{\ell} > k\}} \,dG_{\ell} ) >
%- \{\mathbb{E}_{t}^{\mathbb{P}}( \mathbb{L}^{T_O}_t[k] )
%-  \mathbb{E}_{t}^{\mathbb{Q}}( \mathbb{L}^{T_O}_t[k] ) \}$. %\vspace{-2mm}
%\end{corollary}
%\noindent {\bf Proof:} We specialize Theorem~\ref{claimm:claim1call_jump} to a continuous semimartingale setting (see Appendix~\ref{appsec:jumppps}). $\blacksquare$


Corollary~\ref{claimm:claim1call}
can be traced
to nontrivial contributions of
${\bm\theta}[t,\mathbf{Y}_t]^{\top} \, d \mathbf{u}^{\mathbb{P}}_t$
(i.e., $M_t$
has unspanned risks) and
${\bm \sigma}_{V,u}[t,\mathbf{Y}_t] \,d \mathbf{u}^{\mathbb{P}}_t$
(i.e., volatility has unspanned risks).
Implicit
is an insight that the local time risk premium
for any
$k$ is zero only if
${\bm\theta}[t,\mathbf{Y}_t] = {\bf 0}$
(i.e., $M_t$
does not contain unspanned risks) or
${\bm \sigma}_{V,u}[t,\mathbf{Y}_t] = {\bf 0}$
(i.e.,
volatility does not contain unspanned risks).
Due to
its relevance to
the dark matter risk premium
we corroborate this statement as Lemma~\ref{eq:lemmon} ({\color{magenta} Internet}~Appendix~\ref{appsec:SV1}). %\vspace{-3mm}
%\begin{corollary}[Risk premium of a straddle
%%when $(F_{\ell}^{T_F})$ is a
%(continuous semimartingale setting)]
%\label{claimm:straddles}
%Assume that
%\begin{equation}
%\underbrace{\mathbb{E}_{t}^{\mathbb{P}}( \int_{t}^{{T}_O}  \mathbbm{1}_{\{G_{\ell} > 1\}} \,dG_{\ell} )}_{\mathrm{upside~risk~premium~for}~k=1} ~-~ \\
%\underbrace{\mathbb{E}_{t}^{\mathbb{P}}( \int_{t}^{{T}_O} \mathbbm{1}_{\{G_{\ell} < 1\}} \,dG_{\ell} )}_{\mathrm{downside~risk~premium~for}~k=1} ~\approx~  0.
%\end{equation}
%Then, the risk premium of a straddle is zero (respectively, negative) if,
%and only if, the local time risk premium
%(and hence also the dark matter risk premium)
%for
%$k=1$ is zero (respectively, negative). \vspace{-3mm}
% \end{corollary}
%\noindent {\bf Proof:}
%See Appendix~\ref{appsec:jumppps} (part III).
%%The proof is the continuous semimartingale version of Theorem~\ref{claimm:claim1call_jump}  (Appendix~\ref{appsec:jumppps} part III).
%$\blacksquare$
%%{\color{blue} (see also Internet Appendix (Section~\ref{appsec:straddle})).}
%

%The condition $\mathbb{E}_{t}^{\mathbb{P}}( \int_{t}^{{T}_O}  \mathbbm{1}_{\{G_{\ell} > 1\}} \,dG_{\ell} ) - \mathbb{E}_{t}^{\mathbb{P}}( \int_{t}^{{T}_O} \mathbbm{1}_{\{G_{\ell} < 1\}} \,dG_{\ell} ) \approx 0$
%is akin to the
%unforecastability of the combined long and short futures position to the upside or the downside pertaining to $k=1$.
%Corollary~\ref{claimm:straddles} reflects a
%testable prediction of our theory using excess straddle returns.
%%Depending upon term to expiration $T_O-t$,
%The nature of the
%local time risk premium at moneyness $k$ --- which associates with \emph{dark matter}
%in option risk premiums --- can be understood by evaluating the excess returns of straddles.

%\newpage

\noindent \textbf{\color{magenta} Summary and complementary big picture.} We show theoretically
that the presence of unspanned risks in the dynamics of
the pricing kernel
%dynamics
and
the volatility
is in the direction of
%crucial to
addressing certain
questions in the market for options on
equity index and futures.
Intuitively, the presence of unspanned risks in the volatility dynamics impacts the quadratic variation, which in turn
impacts local time. This feature, in conjunction with a nonzero contribution of the unspanned risks ${\bm\theta}[t,\mathbf{Y}_t]^{\top} \, d \mathbf{u}^{\mathbb{P}}_t$, gives rise to, in general, a nonzero local time risk premium (for all $k$).\footnote{
One may be tempted to cast the local time risk premium
as a gamma risk premium
(since $\mathbb{L}^{T_O}_t[k] = \frac{1}{2} \int_{t}^{T_O} \delta_{\{G_\ell ~-~ k\}} d [ G,G]_\ell$),
given that the Dirac delta function is the second-order derivative of $\max(G_{\ell}-k,0)$ with respect to $G_{\ell}$ (i.e., reflects the gamma). While the concept of a gamma risk premium is appealing in the continuous semimartingale context, the dynamics of the equity futures have a nonzero correlation with quadratic variation.}

The
possibility
that
local time (dark matter) risk premiums may be
dependent upon
$T_O-t$ is embedded
within our characterizations. \vspace{-3mm}


\subsection{The role of unspanned risks in a stochastic volatility model}
%Role of unspanned risks in a parameterized model
%that
%fosters
%is
%aimed
%towards
%economic
%}
\label{subsec:model_sv}

%The
Our purpose
%of this
%section
is threefold. First, we present
a parametric (continuous semimartingale) setting that
explicitly models (i) spanned and unspanned risks in the pricing kernel
and (ii) spanned and unspanned risks
 in equity return volatility. Second,
we show that our framework subsumes the baseline specification of no unspanned risks in the pricing kernel and no
unspanned risks in equity return volatility.
Third, we interpret
%digest
the economic restrictions under which a stochastic
volatility model, with unspanned and spanned risks, can be consistent with negative risk premiums for OTM calls.
Our
alternative specification
can be viewed as a stepping stone to understanding the distinction between the local time risk premium
(corresponding to $k$) and the variance risk premium.

%\noindent \textbf{Stochastic volatility model with unspanned risks.}
Consider the dynamics of
%We specialize the dynamics in equations (\ref{eq:GeneralDynamics1}) and (\ref{eq:index3}) for
$M_t$ and $S_t$, as follows: %$\mathrm{v}$
\begin{eqnarray}
\frac{dM_t}{M_t} & = & -r\, dt
~+~ \eta[t,\mathrm{v}_t] \underbrace{d z_t^{\mathbb{P}}}_{\mathrm{spanned~risks}}
~+~ \theta[t,\mathrm{v}_t]  \underbrace{du_t^{\mathbb{P}},}_{\mathrm{unspanned~risks}}~~~~\mbox{ \, \quad \, with\, \, }~~~~ \label{eq:v1}\\
& &
\eta[t,\mathrm{v}_t] \, = \, - \frac{1}{ \sqrt{\mathrm{v}_t}}(
\alpha_{\mathrm{vol}} +\lambda_{\mathrm{vol}} \, \mathrm{v}_t),
%~~\mbox{ \, }
~~~~~~~~
%~~\mathrm{and}~~~~
\theta[t,\mathrm{v}_t]  =  - \theta_{\mathrm{LT}}\, \sqrt{\mathrm{v}_t},
~\mbox{ \, }
~\mathrm{and}~~~\mbox{ \, \, }
\label{eq:v2}\\
%& &
%\theta[t,v_t]  =  - \theta_{\mathrm{LT}}\, \sqrt{v_t},
%~\mbox{ \, }~\mathrm{and}~~~\mbox{ \, \, } \label{eq:v2a}\\
\frac{d S_t}{S_t} & = &  ( r  ~-~ \eta[t,\mathrm{v}_t] \,\sqrt{\mathrm{v}_t} )  \,dt ~+~
\sqrt{\mathrm{v}_t} \underbrace{d z^{\mathbb{P}}_t,}_{\mathrm{spanned~risks}}
\label{eq:SimpleSquareRootVolStockDynmaics1}
\end{eqnarray}
where $\mathrm{v}_t$ denotes the instantaneous variance of the equity return, which also constitutes the
single economic state variable (i.e., $\mathbf{Y}_t= \mathrm{v}_t$).
We specify the dynamics for $\mathrm{v}_t$ under $\mathbb{P}$ in (\ref{eq:SimpleHestonDynmaics}) (below).
In (\ref{eq:v1}), $z_t^{\mathbb{P}}$ and $u_t^{\mathbb{P}}$ are each a one dimensional standard Brownian motion.


In our setup, $\alpha_{\mathrm{vol}}$, $\lambda_{\mathrm{vol}}$, and $\theta_{\mathrm{LT}}$ are constants.
Provided that
%\begin{align}
%&\alpha_{\mathrm{vol}} > 0&
%&\mathrm{and}&
%&\lambda_{\mathrm{vol}} > 0,&
%\end{align}
(i) $\alpha_{\mathrm{vol}} > 0$ and (ii) $\lambda_{\mathrm{vol}} > 0$, the
risk premium on
the equity (and its futures) is positive; that is,
$\alpha_{\mathrm{vol}}+\lambda_{\mathrm{vol}} \, \mathrm{v}_t  >  0$.

Crucial from our perspective, we will show
(see Corollary~\ref{claimm:SV} below) that the parameter $\theta_{\mathrm{LT}}$
affecting $\theta[t,\mathrm{v}_t]  =  - \theta_{\mathrm{LT}}\, \sqrt{\mathrm{v}_t}$ in (\ref{eq:v2})
determines the sign of the local time risk premium, and, consequently, the call risk premium.
It
holds that local time risk premium exerts \emph{influence} on, and yet (see
(\ref{eq:ConditionForNegVRPInExtendedHeston})) is \emph{economically distinct} from, the variance risk premium.


Next,  (\ref{eq:SimpleHestonDynmaics}) develops the view that random fluctuations
in equity return
variance $\mathrm{v}_t$ can
arise from
both spanned and unspanned diffusive risks, whereas (by definition) the source of randomness
in the equity futures price is solely due to spanned diffusive risks.


For tractability, we assume that the dynamics of return variance, $\mathrm{v}_t$, follow
\begin{eqnarray}
d\mathrm{v}_t  &=& ( \phi_{\mathrm{vol}}^{\mathbb{P}} - \kappa_{\mathrm{vol}}^{\mathbb{P}} \,\mathrm{v}_t )\, dt ~+~
  \sigma_{\mathrm{vol}} \, \sqrt{\mathrm{v}_t} \,\rho_{\mathrm{vol}} \, \underbrace{d z_t^{\mathbb{P}}}_{\mathrm{spanned~risks}}  ~+~ \sigma_{\mathrm{vol}} \sqrt{\mathrm{v}_t} \, \sqrt{1-\rho^2_{\mathrm{vol}}} \, \underbrace{du_t^{\mathbb{P}}.}_{\mathrm{unspanned~risks}} ~~ \mbox{ \, \,  }~ \label{eq:SimpleHestonDynmaics} \\
\mbox{Hence,} & & \frac{d G_t}{G_t} \, \, = \, \, \frac{d F_{t}^{T_F}}{F_{t}^{T_F}} \, = \, \overbrace{( \alpha_{\mathrm{vol}} + \lambda_{\mathrm{vol}} \, \mathrm{v}_t  )}^{\mathrm{futures~risk~premium}}  \, dt ~+~
\sqrt{\mathrm{v}_t} \, \overbrace{d z^{\mathbb{P}}_t.}^{\mathrm{spanned~risks}}~~\mbox{  \, \, }~~
~~\mbox{ \, \, }~~\label{eq:SimpleSquareRootVolStockDynmaics}
\end{eqnarray}

To shed light on option risk premiums and
compensation
for unspanned risks reflected in the risk-neutral drift of the volatility process,
(\ref{eq:SimpleHestonDynmaics}) draws
a distinction between the
one dimensional Brownian motions $z_t^{\mathbb{P}}$ and $u_t^{\mathbb{P}}$.
Additionally,
$\mathrm{cov}^{\mathbb{P}}_{t}(\frac{d F_{t}^{T_F}}{F_{t}^{T_F}}, d\mathrm{v}_t)/dt = \rho_{\mathrm{vol}} \,
\sigma_{\mathrm{vol}} \,\mathrm{v}_t$ and, hence, $\rho_{\mathrm{vol}}$ is
the instantaneous correlation between
$\frac{d F_{t}^{T_F}}{F_{t}^{T_F}}$ and $d\mathrm{v}_t$.

\noindent \textbf{Call risk premium.}
%Considering the nature of spanned and unspanned risks, under $\mathbb{P}$ and $\mathbb{Q}$,
It follows, from (\ref{fg.1}), that
\begin{align}
&
dz_t^{\mathbb{P}} ~-~ dz^{\mathbb{Q}}_t \, = \,
 -\frac{1}{ \sqrt{\mathrm{v}_t}}(
\alpha_{\mathrm{vol}} +\lambda_{\mathrm{vol}} \, \mathrm{v}_t) \, dt&
&\mathrm{and}&
& du_t^{\mathbb{P}} ~-~ du_t^{\mathbb{Q}} \, = \,  -\theta_{\mathrm{LT}} \,\sqrt{\mathrm{v}_t} \, dt.    &
\end{align}
The workings of this theory implies that
the $\mathbb{Q}$ dynamics of $G_t$ and of $\mathrm{v}_t$ are $\frac{dG_t}{G_t} = \sqrt{\mathrm{v}_t} \, dz_t^{\mathbb{Q}}$ and
\begin{eqnarray}
d\mathrm{v}_t & = & ( \phi_{\mathrm{vol}}^{\mathbb{Q}} - \kappa_{\mathrm{vol}}^{\mathbb{Q}} \, \mathrm{v}_t ) \, dt
+ \sigma_{\mathrm{vol}} \, \sqrt{\mathrm{v}_t} \, \rho_{\mathrm{vol}}\, dz_t^{\mathbb{Q}}
+ \sigma_{\mathrm{vol}} \, \sqrt{\mathrm{v}_t}  \, \sqrt{1-\rho^2_{\mathrm{vol}}} \, du_t^{\mathbb{Q}}, ~~ \mbox{ \, \, \, \, \, } \mathrm{with} ~~ \label{volqdy} \\
\kappa_{\mathrm{vol}}^{\mathbb{Q}} & = &\kappa_{\mathrm{vol}}^{\mathbb{P}}
+ \sigma_{\mathrm{vol}} \,\rho_{\mathrm{vol}}\, \lambda_{\mathrm{vol}}
+\underbrace{\theta_{\mathrm{LT}}\,\sigma_{\mathrm{vol}} \,\sqrt{1-\rho^2_{\mathrm{vol}}}}_{\tiny\mathrm{not~in~Heston~(1993)}},
~\mbox{ \, \, and \, \, }~\phi_{\mathrm{vol}}^{\mathbb{Q}} \, = \,  \phi_{\mathrm{vol}}^{\mathbb{P}} - \rho_{\mathrm{vol}} \, \alpha_{\mathrm{vol}} \, \sigma_{\mathrm{vol}}. ~\mbox{ \, \, \, \, }~~~ \label{eq:KappaPKappaQGammaPGammaQ}
\end{eqnarray}

At the heart of this
model are unspanned risks in the dynamics of $M_t$ and $\mathrm{v}_t$,
with theoretical effects conceptually
different from the baseline, as in \citet*[equations (4) and (8)]{Heston:1993}.

\setcounter{theorem}{1}
\begin{corollary}[OTM call risk premium
in a model with unspanned risks]
\label{claimm:SV}
Under the parameterizations
in
(\ref{eq:v1})--(\ref{eq:SimpleHestonDynmaics}),
the call risk premium
can be negative if
$\theta_{\mathrm{LT}} < 0$.
Absent a contribution of unspanned risks (when $\theta_{\mathrm{LT}} = 0$)
or when $\theta_{\mathrm{LT}} \geq 0$,
the call risk premium is positive.  %\vspace{-3mm}
\end{corollary}
\noindent {\bf Proof:} See Internet Appendix~\ref{appsec:SV1}. $\blacksquare$

The restriction $\theta_{\mathrm{LT}} < 0$ implies a negative local time risk
premium. If the latter is sufficiently negative, that is, large enough in magnitude to offset
a positive
equity futures risk premium to the upside,
the call risk premium
can be negative.

\noindent \textbf{Differentiating local time risk premium from variance risk premium.}
%{\color{red} [Emphasizing a departure,
%we now
%differentiate
%between the local time risk premium
%and the \emph{variance} risk premium.]}
We can derive
%The time $t$ conditional equity variance risk premium is
\begin{equation} \small
\underbrace{\mathbb{E}_t^{\mathbb{P}}( \int_{t}^{T_O} \mathrm{v}_{\ell} d \ell ) - \mathbb{E}_t^{\mathbb{Q}}( \int_{t}^{T_O} \mathrm{v}_{\ell} d \ell )}_{\mathrm{variance~risk~premium}}
=  \frac{ \phi_{\mathrm{vol}}^{\mathbb{P}} ( T_0 - t ) - \{ \mathbb{E}_t^{\mathbb{P}}( \mathrm{v}_{T_O} ) - \mathrm{v}_{t} \} }{\kappa_{\mathrm{vol}}^{\mathbb{P}}}
~- ~ \frac{ \phi_{\mathrm{vol}}^{\mathbb{Q}} ( T_0 - t ) - \{ \mathbb{E}_t^{\mathbb{Q}}( \mathrm{v}_{T_O} ) - \mathrm{v}_{t} \} }{\kappa_{\mathrm{vol}}^{\mathbb{Q}}}.~~\mbox{ \, }
\label{vrp.1}
\end{equation}
Furthermore, the \emph{unconditional} variance risk premium is $ \, $ $\frac{\phi_{\mathrm{vol}}^{\mathbb{P}} ( T_0 - t )}{\kappa_{\mathrm{vol}}^{\mathbb{P}}} - \frac{\phi_{\mathrm{vol}}^{\mathbb{Q}} ( T_0 - t )}{\kappa_{\mathrm{vol}}^{\mathbb{Q}}}$, $ \, $ which is negative if, and only,
$\frac{\phi_{\mathrm{vol}}^{\mathbb{P}}}{\kappa_{\mathrm{vol}}^{\mathbb{P}}} < \frac{\phi_{\mathrm{vol}}^{\mathbb{Q}}}{\kappa_{\mathrm{vol}}^{\mathbb{Q}}}$
or, rearranging using (\ref{eq:KappaPKappaQGammaPGammaQ}), if, and only if,
\begin{eqnarray}
\frac{\phi_{\mathrm{vol}}^{\mathbb{Q}} + \rho_{\mathrm{vol}}\, \alpha_{\mathrm{vol}} \, \sigma_{\mathrm{vol}}}{\kappa_{\mathrm{vol}}^{\mathbb{Q}} ~+~
\sigma_{\mathrm{vol}} \{-\theta_{\mathrm{LT}}\, \sqrt{1-\rho_{\mathrm{vol}}^2} ~-~ \rho_{\mathrm{vol}}\,\lambda_{\mathrm{vol}}\}} ~<~ \frac{\phi_{\mathrm{vol}}^{\mathbb{Q}}}{\kappa_{\mathrm{vol}}^{\mathbb{Q}}}. ~~~
\text{ \footnotesize
\, \, (for negative variance risk premium) \, \, } ~~ \label{eq:ConditionForNegVRPInExtendedHeston}
\end{eqnarray}

The parametrization $\rho_{\mathrm{vol}} \leq 0$
(and $\alpha_{\mathrm{vol}} > 0$) is instructive, whereby the variance risk premium is negative if the local time
risk premium is negative; that is, if $\theta_{\mathrm{LT}} < 0$.\footnote{The variance risk premium is connected to $\alpha_{\mathrm{vol}}$ and $\lambda_{\mathrm{vol}}$
through the terms
$\rho_{\mathrm{vol}} \, \sigma_{\mathrm{vol}}\, \lambda_{\mathrm{vol}}$ and $\rho_{\mathrm{vol}} \, \alpha_{\mathrm{vol}} \, \sigma_{\mathrm{vol}}$, implying that some of the variance risk premium is entangled with the
equity futures risk premium (when $\rho_{\mathrm{vol}} \neq 0$).
In contrast, the irrelevance
of unspanned risks (i.e., $\theta_{\mathrm{LT}} = 0$) implies zero local time risk premiums for all $k$.}


%{\color{red} [[[ I would have left this in? ]] [[[ While there is strong support in the extant empirical literature for
%$\rho_{\mathrm{vol}} \leq 0$, $\alpha_{\mathrm{vol}} > 0$ and for
%a negative variance risk premium,
%based on equation (\ref{eq:ConditionForNegVRPInExtendedHeston}),
%a negative variance risk premium \emph{could potentially} co-exist with $\theta_{\mathrm{LT}}$ being zero or positive.
%Our innovation is to show that empirical data on option returns
%supports the view that local time risk premiums must be negative and
%%in this model setup,
%the latter requires
%$\theta_{\mathrm{LT}} < 0$.]]]] -- in THE TAKEAWAY ]]]]}

\noindent \textbf{Takeaways.} A suitably
motivated stochastic volatility model
%(under the $\mathbb{P}$ measure)
 can
synthesize negative call risk premiums
provided that certain restrictions are imposed on unspanned risks.
The linchpin of our framework
is that
%equity
volatility embeds unspanned risks and shapes
%moulds
%agrees
%is aligned
%with
a negative
local time risk premium.\footnote{
If jumps in variance (as in \citet*{DuffiePanSingleton:2000} and \citet*{Amengual_Xiu:JOE2018})
were to be added while maintaining the
equity dynamics in (\ref{eq:SimpleSquareRootVolStockDynmaics1}), then Corollary~\ref{claimm:SV} remains valid.}
One may envision
$\theta_{\mathrm{LT}} \{ \sigma_{\mathrm{vol}} \sqrt{ 1 - \rho^2_{\mathrm{vol}}} \}< 0$
(in  (\ref{eq:KappaPKappaQGammaPGammaQ}))
as arising from the feature that unspanned volatility risks are disliked (and $\theta_{\mathrm{LT}}<0$ is needed to produce negative local time risk premiums). \vspace{-3mm}

%\subsection{\bf \small Appendix C: Statement and proof of Lemma~\ref{eq:lemmon} and proof of Corollary~\ref{claimm:SV}}
\subsection{Statement and proof of Lemma~\ref{eq:lemmon} and proof of Corollary~\ref{claimm:SV}}
\label{appsec:SV1}
We focus on the
continuous semimartingale setting of equations (\ref{eq:GeneralDynamics1})--(\ref{fg.1})
%{\color{red} [in Section~\ref{gggs}]}
--- with drift and diffusion coefficients unspecified ---  and accomplish
two tasks.
First, we show that if unspanned risks are
irrelevant (i.e., if ${\bm \theta}[t, \mathbf{Y}] =  \mathbf{0}$),
then the local time risk premium is zero. This is our Lemma~\ref{eq:lemmon}.
%{\color{red}[and our aim is to strengthen intuition.]}
Second, after specializing to the setting of
%equations
(\ref{eq:v1})--(\ref{eq:SimpleSquareRootVolStockDynmaics}),
%{\color{red}[in Section~\ref{subsec:model_sv}]},
we present the proof of Corollary~\ref{claimm:SV}.

%{\color{blue}We first develop some ``stepping stone" results in the general setting of equations (\ref{eq:GeneralDynamics1})-(\ref{fg.1}).}

Proceeding, for any generic claim with payoff $\complement_{T}$, at time $T$, it holds that
\begin{equation}
\mathbb{E}_{t}^{\mathbb{Q}}( \complement_{T} ) = e^{r(T- t)} \,\, \mathbb{E}_{t}^{\mathbb{P}}( \frac{M_{T}}{M_{t}} \,\complement_{T} ).~~~~~\mbox{ \quad } \label{eq:PtoQExpOfPayoffc}
\end{equation}
The risk premium is $\mathbb{E}_{t}^{\mathbb{P}}( \complement_{T} )-\mathbb{E}_{t}^{\mathbb{Q}}( \complement_{T} )$,
which we deduce next. This step is useful because we will set $\complement_{T}= \mathbb{L}^{T_O}_t[k]$, and, hence, we deduce
$\mathbb{E}_{t}^{\mathbb{P}}( \mathbb{L}^{T_O}_t[k] )-\mathbb{E}_{t}^{\mathbb{Q}}( \mathbb{L}^{T_O}_t[k] )$.

By  a property of covariances,
\begin{eqnarray}
\mathrm{cov}_t^{\mathbb{Q}}( \frac{M_{t}}{M_{T} e^{r (T- t)} }, \complement_{T} ) &=&
\mathbb{E}_{t}^{\mathbb{Q}}( \frac{M_{t}}{M_{T} e^{r (T- t)} } \, \complement_{T} ) ~-~
\overbrace{\mathbb{E}_{t}^{\mathbb{Q}}( \frac{M_{t}}{M_{T} e^{r (T- t)} })}^{=1} \,
\mathbb{E}_{t}^{\mathbb{Q}}( \complement_{T} )~~~~~\mbox{ \quad }  \label{c5.1} \\
&=& \mathbb{E}_{t}^{\mathbb{Q}}( \frac{M_{t}}{M_{T} e^{r (T- t)} } \complement_{T} ) ~-~
\mathbb{E}_{t}^{\mathbb{Q}}( \complement_{T} )~~~~~\mbox{ \quad } \label{c5.2} \\
&=& e^{r(T- t)} \mathbb{E}_{t}^{\mathbb{P}}( \frac{M_{T}}{M_{t}} \times \{ \frac{M_{t}}{M_{T} e^{r (T- t)} } \complement_{T} \} ) ~-~
\mathbb{E}_{t}^{\mathbb{Q}}( \complement_{T} ) \label{c5.3}\\
&=& \mathbb{E}_{t}^{\mathbb{P}}( \complement_{T} ) ~-~ \mathbb{E}_{t}^{\mathbb{Q}}( \complement_{T} ).~~~~~\mbox{ \quad }
\label{c5.4}
\end{eqnarray}
Specializing $\complement_{T}$ to the random variable $\mathbb{L}^{T_O}_t[k]$ and guided by (\ref{c5.4}), we obtain the
risk premium for local time with moneyness $k$ as follows:
\begin{equation}
\overbrace{
\mathbb{E}_{t}^{\mathbb{P}}( \mathbb{L}^{T_O}_t[k] ) ~ - ~
\mathbb{E}_{t}^{\mathbb{Q}}( \mathbb{L}^{T_O}_t[k] )}^{\text{Local~time~risk~premium}} ~=~\mathrm{cov}_t^{\mathbb{Q}}( \frac{M_{t}}{M_{{T}_O} e^{r ({T}_O - t)}} , \, \mathbb{L}^{T_O}_t[k] ). ~~\mbox{ \, \, \, \quad } ~
\label{eq:covqgsbstatement}
\end{equation}


%We now state.
\setcounter{theorem}{0}
\begin{lemma}
\label{eq:lemmon}
{\color{magenta} In the
%continuous semimartingale
setting of
%equations
(\ref{eq:GeneralDynamics1})--(\ref{fg.1}),} if unspanned risks  are irrelevant; that is,
\begin{equation}
\mathrm{if}~{\bm \theta}[t, \mathbf{Y}] \, = \, \mathbf{0}, ~~ \mbox{ \, \, } \mathrm{then} ~~ \mbox{ \, } \mathbb{E}_{t}^{\mathbb{P}}( \mathbb{L}^{T_O}_t[k] )  -
\mathbb{E}_{t}^{\mathbb{Q}}( \mathbb{L}^{T_O}_t[k] ) \, = \, 0. ~ \mbox{ \, \, \, \, } ~
%~~\mathrm{in~a~continuous~semimartingale~setting.}
\label{eq:covqgsbstatementxy}
\end{equation}
\end{lemma}

\noindent \textbf{Proof:} Using the
dynamics of the pricing kernel $M_t$
under the $\mathbb{P}$ measure in
(\ref{eq:GeneralDynamics1}), and then the change of measures in (\ref{fg.1}); that is,
\begin{align*}
&d \mathbf{z}^{\mathbb{P}}_t~-~d \mathbf{z}^{\mathbb{Q}}_t = {\bm \eta}[t,\mathbf{Y}_t] \,dt&
&\mathrm{and}&
&d \mathbf{u}^{\mathbb{P}}_t~-~d \mathbf{u}^{\mathbb{Q}}_t =  {\bm \theta}[t,\mathbf{Y}_t] \, dt,&
\end{align*}
%$d \mathbf{z}^{\mathbb{P}}_t-d \mathbf{z}^{\mathbb{Q}}_t = {\bm \eta}[t,\mathbf{Y}_t] dt$ and
%$d \mathbf{u}^{\mathbb{P}}_t-d \mathbf{u}^{\mathbb{Q}}_t =  {\bm \theta}[t,\mathbf{Y}_t] dt$,
it follows
that (hereon
suppressing the time subscript on $\mathbf{Y}_t$)
\begin{equation}
\frac{M_{t}}{M_{T_{O}}e^{r ({T}_O - t)}} = e^{
\int_{t}^{{T}_O} \{
-\frac{1}{2} {\bm \eta}[\ell,\mathbf{Y}]^{\top} {\bm \eta}[\ell,\mathbf{Y}] d\ell
- {\bm \eta}[\ell,\mathbf{Y}]^{\top}  d \mathbf{z}^{\mathbb{Q}}_\ell
- \frac{1}{2} {\bm \theta}[\ell,\mathbf{Y}]^{\top}
{\bm \theta}[\ell,\mathbf{Y}] d\ell
- {\bm\theta}[\ell,\mathbf{Y}]^{\top} d \mathbf{u}^{\mathbb{Q}}_\ell \} }.~\mbox{ \, }~\mbox{ \, }~
\label{eq:IntegratedRecipmGeneralDynamicsUnderQ}
\end{equation}
We note that $\mathbb{E}_{t}^{\mathbb{Q}}( \frac{M_{t}}{M_{T_{O}} e^{r ({T}_O - t)}} ) \, = \, 1$. Let
\begin{equation}
\mathcal{I}_s~\mathrm{be~the~sub\mbox{-}filtration~of}~\mathcal{F}_s~\mathrm{generated~by}~{\bm \eta}^{\mathbb{Q}}_s~\mathrm{and}~{\bm \eta}[s,\mathbf{Y}]^{\top} \,d \mathbf{z}^{\mathbb{Q}}_s. ~ \mbox{ \, \, } ~
\end{equation}

%We denote by $\mathcal{I}_s$ the sub-filtration of $\mathcal{F}_s$ generated by ${\bm \eta}^{\mathbb{Q}}_s$ and ${\bm \eta}[s,\mathbf{Y}]^{'} \,d %\mathbf{z}^{\mathbb{Q}}_s$.


Exploiting the law of total covariance,
\begin{eqnarray}
& & \mathrm{cov}_t^{\mathbb{Q}}( \overbrace{\frac{M_{t}}{M_{{T}_O} e^{r ({T}_O - t)}} }^{\mathrm{from~\tiny(\ref{eq:IntegratedRecipmGeneralDynamicsUnderQ}})}, \, \mathbb{L}_t^{{T}_O}[k] )
\nonumber \\ %\label{eq:NewConditionalCovariance1} \\
& & ~= ~ \mathbb{E}_{t}^{\mathbb{Q}}( \mathrm{cov}_t^{\mathbb{Q}}(
e^{ \int_{t}^{{T}_O} \{ -\frac{1}{2} {\bm \eta}[s,\mathbf{Y}]^{\top} {\bm  \eta}[s,\mathbf{Y}] ds - {\bm \eta}[s,\mathbf{Y}]^{\top} d \mathbf{z}^{\mathbb{Q}}_s
- \frac{1}{2} {\bm \theta}[s, \mathbf{Y}]^{\top} {\bm \theta}[s, \mathbf{Y}] ds - {\bm \theta}[s, \mathbf{Y}]^{\top}
d \mathbf{u}^{\mathbb{Q}}_s \}},
\, \mathbb{L}_t^{{T}_O}[k] {\Big |} \, \mathcal{I}_{T_O} ) )~~\mbox{ \, \, }~~~ \nonumber \\
& & ~\mbox{ \, \, }~\mbox{ \, \, } \mbox{ \quad \quad \quad } \, + \, \mathrm{cov}_t^{\mathbb{Q}}( \underbrace{ \mathbb{E}_{t}^{\mathbb{Q}}( \frac{M_{t}}{M_{{T}_O}} e^{-r ({T}_O - t)} \, \, {\Big |} \mathcal{I}_{T_O} ) }_{\, = \, \,  \mathrm{a~constant} }, \, \, \mathbb{E}_{t}^{\mathbb{Q}}( \mathbb{L}_t^{{T}_O}[k] \, {\Big |} \mathcal{I}_{T_O} ) ) \mbox{ \quad }~\mbox{ \, \, }~  \label{eq:NewConditionalCovariance1Eq} \\
& & ~= ~ \mathbb{E}_{t}^{\mathbb{Q}}( e^{ \int_{t}^{{T}_O} \{ -\frac{1}{2} {\bm \eta}[s,\mathbf{Y}]^{\top} {\bm  \eta}[s,\mathbf{Y}] ds - {\bm \eta}[s,\mathbf{Y}]^{\top} \, \, d \mathbf{z}^{\mathbb{Q}}_s \} } \, \, \times
\nonumber \\
& & ~\mbox{ \, \, }~\mbox{ \, \, } \mbox{ \quad \quad \, } \mathrm{cov}_t^{\mathbb{Q}}(
e^{ \int_{t}^{{T}_O} \{ - \frac{1}{2} {\bm \theta}[s, \mathbf{Y}]^{\top} {\bm \theta}[s, \mathbf{Y}] ds -
{\bm \theta}[s, \mathbf{Y}]^{\top} \,
d \mathbf{u}^{\mathbb{Q}}_s \}}, \, \mathbb{L}_t^{{T}_O}[k] \,{\Big |} \mathcal{I}_{T_O} ) ),
\label{eq:NewConditionalCovariance2Eq}
\end{eqnarray}
because the covariance of the two expectations in
(\ref{eq:NewConditionalCovariance1Eq}) vanishes since one term
(conditional on $\mathcal{I}_{T_O}$) is a constant.
If it were the case that ${\bm \theta}[t, \mathbf{Y}]$ is identically zero and, thus,
${\bm \theta}[s, \mathbf{Y}]^{\top} d \mathbf{u}^{\mathbb{Q}}_s=0$,
then, by
%equation
(\ref{eq:NewConditionalCovariance2Eq}), $\mathrm{cov}_t^{\mathbb{Q}}( \frac{M_{t}}{M_{{T}_O} e^{r ({T}_O - t)}} , \, \mathbb{L}^{T_O}_t[k] ) = 0$.

Then, by (\ref{eq:covqgsbstatement}), $\mathbb{E}_{t}^{\mathbb{P}}( \mathbb{L}^{T_O}_t[k] ) - \mathbb{E}_{t}^{\mathbb{Q}}( \mathbb{L}^{T_O}_t[k] ) = 0$. $\square$ \vspace{2mm}

Based on (\ref{eq:NewConditionalCovariance2Eq}), the sign of the
local time risk premium inherits the sign of the $\mathbb{Q}$ measure conditional covariance
$\mathrm{cov}_t^{\mathbb{Q}}(
e^{ \int_{t}^{{T}_O} \{ - \frac{1}{2} {\bm \theta}[s, \mathbf{Y}]^{\top} {\bm \theta}[s, \mathbf{Y}] ds -
{\bm \theta}[s, \mathbf{Y}]^{\top} \,
d \mathbf{u}^{\mathbb{Q}}_s \}}, \, \mathbb{L}_t^{{T}_O}[k] \,{\Big |} \mathcal{I}_{T_O} )$.
More specifically, (\ref{eq:covqgsbstatement}) and (\ref{eq:NewConditionalCovariance2Eq}) indicate
that the local time risk premium $\mathbb{E}_{t}^{\mathbb{P}}( \mathbb{L}^{T_O}_t[k] ) -
\mathbb{E}_{t}^{\mathbb{Q}}( \mathbb{L}^{T_O}_t[k] )$ is negative, if and only if,
%{\color{red}[[[ removed ``for all $t$." ]]]}
\begin{equation}
\mathrm{cov}_t^{\mathbb{Q}}(
e^{ \int_{t}^{{T}_O} \{ - \frac{1}{2} {\bm \theta}[s, \mathbf{Y}]^{\top} {\bm \theta}[s, \mathbf{Y}] ds -
{\bm \theta}[s, \mathbf{Y}]^{\top} \, \,
d \mathbf{u}^{\mathbb{Q}}_s \}}, \, \mathbb{L}_t^{{T}_O}[k] \,{\Big |} \mathcal{I}_{T_O} ) < 0.
%~~~~~~~~\mbox{ \, \, }
\label{c5.10}
\end{equation}
In this continuous semimartingale setting, it further holds that the
\begin{equation}
\mathrm{sign~of}~\mathbb{E}_{t}^{\mathbb{P}}( \mathbb{L}^{T_O}_t[k] ) -
\mathbb{E}_{t}^{\mathbb{Q}}( \mathbb{L}^{T_O}_t[k] )~\mathrm{is~the~sign~of}~
\mathrm{cov}_t^{\mathbb{Q}}( \int_{t}^{{T}_O} - {\bm \theta}[s, \mathbf{Y}]^{\top} \,
d \mathbf{u}^{\mathbb{Q}}_s, \, \mathbb{L}_t^{{T}_O}[k] \,{\Big |} \mathcal{I}_{T_O} ). ~
\label{c5.11}
\end{equation}
With $\sum_{t \leq h \leq \ell} (G_{h} - G_{h -})^2  =  0$
(no jumps for any $h$)
for continuous semimartingales, $\mathbb{L}^{T_O}_t[k] = \frac{1}{2} \int_{t}^{T_O}
\delta_{\{G_\ell ~-~ k\}} d [ G, G ]_{\ell}$, where
$\delta_{\{\bullet\}}$ is the Dirac delta function and $[ G, G ]_{\ell}$ is the quadratic variation.

\noindent \textbf{Proof of Corollary~\ref{claimm:SV}.}  Mindful of
equations (\ref{c5.10}) and (\ref{c5.11}),
we return to our model in equations (\ref{eq:v1})--(\ref{eq:SimpleHestonDynmaics}) and derive the form of
$\mathbb{L}_t^{{T}_O}[k]$. To do so, note that the evolution of variance satisfies
\begin{eqnarray}
\mathrm{v}_{\ell} & =  & \mathrm{v}_{t}\, e^{ \kappa_{\mathrm{vol}}^{\mathbb{Q}} ( t - \ell ) } + \int_{t}^{\ell} \phi_{\mathrm{vol}}^{\mathbb{Q}} e^{ \kappa_{\mathrm{vol}}^{\mathbb{Q}} ( s - \ell ) } ds \nonumber \\
&+&\sigma_{\mathrm{vol}}\,\rho_{\mathrm{vol}} \int_{t}^{\ell}  e^{ \kappa_{\mathrm{vol}}^{\mathbb{Q}} ( s - \ell ) } \sqrt{\mathrm{v}_{s}} \,  dz_s^{\mathbb{Q}}
+\sigma_{\mathrm{vol}} \, \sqrt{1-\rho^2_{\mathrm{vol}}} \int_{t}^{\ell}  e^{ \kappa_{\mathrm{vol}}^{\mathbb{Q}} ( s - \ell ) } \sqrt{\mathrm{v}_{s}} \,  du_s^{\mathbb{Q}},~~~\mbox{ for $\ell \geq t$.\, \, \, }~~
\label{cv.s}
\end{eqnarray}
It further holds that $dG_t  = \sqrt{\mathrm{v}_t} \, G_t \, dz_t^{\mathbb{Q}}$. Hence, the \emph{quadratic variation}
$[ G, G ]_{s}$ is
 \begin{equation}
[ G, G ]_{s}  = \, \int_{t}^s \{ \sqrt{\mathrm{v}_{\ell}} \, G_{\ell} \}^2 \, d\ell \, \, = \, \int_{t}^s \, \mathrm{v}_{\ell} \, G_\ell^2 \, d\ell. ~ \mbox{ \, } ~ ~~
\end{equation}
Using the differential form $d [ G, G ]_{\ell} =  \mathrm{v}_{\ell} \, G_\ell^2 \, d \ell$, we deduce, from (\ref{ltt.1}) in
conjunction with
$\sum_{t \leq h \leq \ell} (G_{h} - G_{h -})^2  =  0$ (no jumps for any $h$),
that
\begin{eqnarray}
\mathbb{L}^{T_O}_t[k] &= & \frac{1}{2} \int_{t}^{T_O} \delta_{\{G_\ell ~-~ k\}} d [ G, G ]_{\ell}
~=~  \frac{1}{2} \, \int_{t}^{{T}_O} \, \delta_{\{G_\ell ~-~ k\}} \, \mathrm{v}_{\ell} \, G_\ell^2 \, d\ell. ~ \mbox{ \, } ~ ~~
\end{eqnarray}

Specializing ${\bm \theta}[t, \mathbf{Y}]$ to
${\bm \theta}[t, \mathbf{Y}]  =  - \, \theta_{\mathrm{LT}} \sqrt{\mathrm{v}_t}$ as per our setup,
we obtain the sign of the $\mathbb{Q}$-measure conditional
covariance, in light of equations (\ref{c5.10}) and (\ref{c5.11}), as follows:
\begin{eqnarray}
& & \mathrm{cov}_t^{\mathbb{Q}}( \int_{t}^{{T}_O} - {\bm \theta}[s, \mathbf{Y}]^{'} \, \,
d \mathbf{u}^{\mathbb{Q}}_s, \, \mathbb{L}_t^{{T}_O}[k] \,{\Big |} \mathcal{I}_{T_O} ) \nonumber \\
& & =~ \mathrm{cov}_t^{\mathbb{Q}}( \int_{t}^{{T}_O} -
\{-\theta_{\mathrm{LT}} \, \sqrt{\mathrm{v}_s} \, \, d u^{\mathbb{Q}}_{s}\},
 \,
\frac{1}{2} \, \int_{t}^{{T}_O} \, \delta_{\{G_\ell ~-~ k\}} \, \mathrm{v}_{\ell} \, G_\ell^2 \, d\ell \, \,{\Big |} \mathcal{I}_{T_O} ) \nonumber \\
& & =~ \mathrm{cov}_t^{\mathbb{Q}}( \int_{t}^{{T}_O}
\theta_{\mathrm{LT}} \, \sqrt{\mathrm{v}_s} \, \, d u^{\mathbb{Q}}_{s},
\frac{1}{2} \int_{t}^{{T}_O} \int_{t}^{\ell} \, \sigma_{\mathrm{vol}} e^{\kappa_{\mathrm{vol}}^{\mathbb{Q}}( s - \ell )} \sqrt{\mathrm{v}_s} \, \sqrt{1-\rho_{\mathrm{vol}}^2} \,
du^{\mathbb{Q}}_{s} \, \delta_{\{G_\ell ~-~ k\}} \, G_\ell^2  d \ell  \,{\Big |} \mathcal{I}_{T_O} ) \mbox{ \, }~\mbox{ \, \, } \nonumber \\
& & =~ \mathrm{cov}_t^{\mathbb{Q}}( \int_{t}^{{T}_O} \theta_{\mathrm{LT}} \, \sqrt{\mathrm{v}_s} \, \, \, d u^{\mathbb{Q}}_{s},
\, \int_{t}^{{T}_O} \sqrt{\mathrm{v}_{s}} \, \{\ \int_{s}^{{T}_O} \frac{\sigma_{\mathrm{vol}}}{2} e^{\kappa_{\mathrm{vol}}^{\mathbb{Q}} ( s - \ell ) } \, \sqrt{1-\rho_{\mathrm{vol}}^2} \, \delta_{\{G_\ell ~-~ k\}} \, G_\ell^2 \, d\ell \} \, du^{\mathbb{Q}}_{s} \,{\Big |} \mathcal{I}_{T_O} )   \nonumber \\
& & =~ \mathbb{E}_{t}^{\mathbb{Q}}( \int_{t}^{{T}_O} \theta_{\mathrm{LT}} \, \sqrt{\mathrm{v}_s} \, \, \sqrt{\mathrm{v}_s} \,
\, \{\ \int_{s}^{{T}_O} \frac{\sigma_{\mathrm{vol}} }{2} e^{\kappa_{\mathrm{vol}}^{\mathbb{Q}} ( s - \ell ) } \, \sqrt{1-\rho_{\mathrm{vol}}^2} \, \delta_{\{G_\ell ~-~ k\}}
\, G_\ell^2 \, d\ell \} \,  ds \, {\Big |} \mathcal{I}_{T_O} )  \nonumber \\
& & =~ \theta_{\mathrm{LT}} \, \, \,
\underbrace{\mathbb{E}_{t}^{\mathbb{Q}}(  \int_{t}^{{T}_O} \, \mathrm{v}_s \, \, \{\ \int_{s}^{{T}_O} \frac{\sigma_{\mathrm{vol}}}{2}
e^{ \kappa_{\mathrm{vol}}^{\mathbb{Q}} ( s - \ell ) } \, \sqrt{1-\rho_{\mathrm{vol}}^2} \, \delta_{\{G_\ell ~-~ k\}}
\, G_\ell^2 \, d\ell \}\,  ds \,{\Big |} \mathcal{I}_{T_O} )}_{~\geq ~0}.
\label{eq:FinalCovarianceTerm3Heston}
\end{eqnarray}
Inspection of (\ref{eq:FinalCovarianceTerm3Heston}) shows
that the
\begin{align}
&\mathrm{local~time~risk~premium~is~negative,~if~and~only~if,}~\theta_{\mathrm{LT}} < 0.&
%&&
%& \mbox{(unspanned volatility risks are disliked)}&
\label{intuu}
\end{align}
When $\theta_{\mathrm{LT}} < 0$, by
Corollary~\ref{claimm:claim1call}, the  OTM call risk premium can be negative. The intuition behind
(\ref{intuu}) is that unspanned volatility risks are disliked.

%If the equity variance dynamics {\color{blue}were to} contain only spanned risks, it
%%{\color{blue}would correspond} to
%%{\color{blue}$\sqrt{1-\rho_{\mathrm{vol}}^2} = 0$ and
%%the economic effect would be as if $\theta_{\mathrm{LT}}= 0$ in the sense that
%%we find that the covariance in equation (\ref{eq:FinalCovarianceTerm3Heston}) would be zero,
%the local time risk premium would be zero and, hence, the risk premium of OTM calls would be positive.

%Elaborating on the workings of
Commenting on the steps in  (\ref{eq:FinalCovarianceTerm3Heston}), the second line of (\ref{eq:FinalCovarianceTerm3Heston}) recognizes that only the term $\sigma_{\mathrm{vol}} \, \sqrt{1-\rho^2_{\mathrm{vol}}} \int_{t}^{\ell}  e^{ \kappa_{\mathrm{vol}}^{\mathbb{Q}} ( s - \ell ) } \sqrt{\mathrm{v}_{s}} \, \, du_s^{\mathbb{Q}}$ in (\ref{cv.s}) is relevant for the covariance. Furthermore, the third line changes the order of integration. Finally, the fourth line uses Ito's isometry formula. $\blacksquare$
\vspace{-3mm}

\end{document}

