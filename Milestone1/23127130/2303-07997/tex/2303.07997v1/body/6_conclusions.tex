\section{CONCLUSIONS}

In this paper, we present an approach to localize and reconstruct an unknown shape through vision and tactile feedback.
A factor graph-based framework is proposed to jointly optimize all measurement units, and we close the loop by actively checking for loop closures.
Such a system has the advantage of utilizing both global features and local textures of an object for pose tracking and reconstruction.
Quantitative and qualitative evaluations show that our proposed method is able to achieve high tracking accuracy and produce high-fidelity reconstructed shapes on novel objects.

In the future, this approach can be potentially improved by utilizing force vectors at the contact surface, which encapsulate unique higher order information of the movement.
Furthermore, following works on neural probabilistic inference \cite{zhao2019towards}, instead of using fixed covariance matrices for each factor, having the networks predict both the most likely pose and its variance can potentially help the factor graph make better decisions.
% Another interesting direction would be to take advantage of categorical object mesh prior knowledge, which could provide useful information to reduce accumulated error. \todo{maybe remove the last direction}