\definecolor{graytext}{RGB}{130,130,130}

\begin{table}[t]
\setlength{\tabcolsep}{3.0pt}
 \def\arraystretch{1.1}
\centering
\resizebox{0.7\linewidth}{!}{
\begin{tabular}{llccc}
\toprule
   & Method  & $\text{CLIP}_{Sim}\uparrow$ & $\text{CLIP}_{Dir} \uparrow$  \\ 
    \midrule 
    \multirow{3}{*}{\rotatebox[origin=c]{90}{Local}} & Text2Mesh & \color{graytext}{0.34*} & \color{graytext}{0.10}* \\
    & SketchShape & 0.29 & 0.01  \\
    & Ours &\textbf{0.36} & \textbf{0.08} \\
    \midrule 
    \multirow{3}{*}{\rotatebox[origin=c]{90}{Global}} & Text2Mesh & \color{graytext}{0.32*} & \color{graytext}{0.03*}  \\
    & SketchShape & 0.27 & 0.01\\
    & Ours & \textbf{0.35} & \textbf{0.02} \\
\bottomrule
\end{tabular}
}
\caption{\textbf{Quantitative Evaluation.} We compare against the 3D object editing techniques Text2Mesh~\cite{michel2022text2mesh} and SketchShape~\cite{metzer2022latent} over local (top) and global (bottom) edits.  *Note that Text2Mesh explicitly trains to minimize a clip loss, and thus directly comparing them with SkechShape and ours is uninformative.
}
\label{tab:baseline-stats}
\end{table}
