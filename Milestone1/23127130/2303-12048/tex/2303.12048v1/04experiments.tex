
\section{Results}
\label{results}

\begin{figure*}[ht]
    \centering
    \includegraphics[scale=0.15,trim={0 2.5cm 0 5cm},clip]{images/aoi-single_burst}
    \caption{The time average peak Age of Information with burst and \gls{soa} loss values against the dynamic reliability logic for different network topologies.}
    \label{fig:aoi_burst}\vspace{-0.4cm}
\end{figure*}


This paper focuses on both transport layer and application layer metrics to determine the feasibility of dynamic reliability. For this, we have selected the session packet volume, as transmitted, retransmitted, lost and backlogged packets as \glspl{kpi} for the transport layer; while focusing on the \gls{aoi} for the application layer. The \gls{aoi} was chosen as a crucial indicator for the freshness of packets in real-time applications. More specifically, this work adopts the time average peak \gls{aoi} equation \cite{aoi_equation} depicted in Eq. \ref{aoi}, where $\Delta(r_{i+1})$ is the $i$th update at the time it was received at the server, for a session time period of $\tau$.

\begin{equation}
    \label{aoi}
    \gls{aoi}_\tau = \frac{1}{n-1}\sum_{i=1}^{n-1} \Delta(r_{i+1})
\end{equation}

We include a comparison between the vanilla QUIC implementation which does not enjoy the dynamic reliability extension, with a number of dynamic reliability policies. The tests were run a number of times for statistical significance, with the mean value of vanilla implementation used as a baseline for comparison. The topology utilised both random loss and bursty loss to explore the bounds of dynamic reliability. The \gls{soa} loss in the figures correspond to the loss values presented in Table. \ref{tab:path_char}, for ease of comparison between bursty and random loss scenarios.

\subsection{Transport-Layer KPIs}

To analyse the performance gain at the transport layer due to dynamic reliability, the volume of transmitted and backlogged packets is examined. The figures are in the form of boxplots, which take the vanilla implementation as a benchmark, depicted as the red dashed line.

As seen in Fig. \ref{fig:sent_burst}, the loss plays a crucial role in the performance of the reliability policies. The policies under random loss did incredibly well for the networks with a larger capacity, namely \gls{mmwave} and Sub-6~GHz, whereas for burst loss, the lower network capacities had a larger packet reduction. With the increase in burst loss, the behaviour of the set split reliable policies became unpredictable, if a reliable assignment happened to coincide with a burst loss, the number of transmitted packets increases, and vice versa. On the other hand, in smarter policies, such as Loss-Aware, the performance lightly matched the vanilla baseline, as the reliable assignment dominated the session to compensate for a higher burst loss. Not only that but, the burst loss also impacted the variance of the transmitted packets for the policies.

Unsurprisingly, the unreliable focused policy, 80-20 split, outperformed other policies for all topologies in random and bursty loss scenarios, with an approximate reduction of 80\%. That being said, the majority of the policies reduced the transmitted packets on the link by approximately 70\% for random loss, while the reduction started at $\approx 15\%$ and decreased as the loss increased for the burst loss scenario.

The retransmitted and lost packets, not shown due to space limitations, followed the same trend as the transmitted packets for the random loss scenarios. However, for the burst loss scenarios, the larger capacity networks had a lower reduction in the retransmitted and lost packets. This can be seen as a favorable outcome since the lower capacity networks are scarce on resources. It is important to note that the Loss-Aware policy mimicked the vanilla approach as the burst loss increased, signifying the overwhelming appointment of reliable packets in adapting to the harsh burst loss conditions.
 
Alternatively, Fig. \ref{fig:backlog_burst} clearly shows a stark comparison between the policies and loss scenario in the reduction of the backlogged packets. The Loss-Aware policy for random loss scenario reduced the backlogged packets by up to 50\%, beating all other policies by approximately 30\%. Furthermore, it is clear that the unreliability focused policies resulted in the lowest backlog for the session. In comparison, we notice that the burst loss and the backlogged frequency have a positive correlation, where the maximum reduction of the backlogged packets for the policies is at most 20\%. Much like the transmitted packets, the probability of a burst loss occurrence plays a vital role in the number of retransmissions sent and by extension the number of backlogged packets. Thus, we can conclude that the stress placed on the buffer is a result of the reliable packets which is tightly coupled with the congestion on the session. Whereas, unreliable focused policies did not encounter such a phenomenon regardless if it was experiencing a burst loss.


\subsection{Application-Layer KPIs}

The feasibility of dynamic reliability for real-time applications can be determined by the \gls{aoi}, with comparison across different topologies and policies. If we take a strict approach and consider anything below $10$~ms is real-time \cite{real-time}, then all the reliability policies passed that requirement, which is attractive for real-time applications, as shown in Fig. \ref{fig:aoi_burst}. Utilising the median as an estimate of the runs, the policies in the WLAN and Sub-6~GHz topology with random loss floated around $4-5$~ms with negligible difference, while the \gls{aoi} for \gls{mmwave} was $\approx 2-3$~ms. It is clear that the \gls{aoi} and the network capacity have a negative correlation, as the network capacity decreases, the \gls{aoi} increases. The same correlation is extended to the bursty loss scenarios, where \gls{mmwave} dominated the other topologies. That being said, it is crucial to note that the \gls{aoi} for the reliability policies is often slightly better than or equal to the \gls{aoi} of the vanilla implementation, proving that dynamic reliability reduces the congestion of the session at no cost to the \gls{aoi}.

\begin{figure} %
\centering
\rotatebox{90}{\whitetxt{xxx}Input} \hfill
\jsubfig{\includegraphics[height=1.9cm]{images/real_scenes/flowers/flowers_input.png}}{} \hfill
\jsubfig{\includegraphics[height=1.9cm]{images/real_scenes/pineapple/pineapple_input.png}}{} \hfill
\jsubfig{\includegraphics[height=1.9cm]{images/real_scenes/pinecone/pinecone_input.png}}{} \vspace{1.5pt} 
\\ 
\rotatebox{90}{\whitetxt{xxx}Initial} \hspace{0.5pt}
\jsubfig{\includegraphics[height=1.9cm]{images/real_scenes/flowers/flowers_initial_grid.png}}{} \hfill
\jsubfig{\includegraphics[height=1.9cm]{images/real_scenes/pineapple/pineapple_initial_grid.png}}{} \hfill
\jsubfig{\includegraphics[height=1.9cm]{images/real_scenes/pinecone/pinecone_initial_grid.png}}{} \vspace{1.5pt} 
\\ 
\rotatebox{90}{\whitetxt{xxx}Edited} \hfill
\jsubfig{\includegraphics[height=1.9cm]{images/real_scenes/flowers/flowers_ours.png}}{\footnotesize {``A vase with sunflowers"}}\hfill
\jsubfig{\includegraphics[height=1.9cm]{images/real_scenes/pineapple/pineapple_ours.png}}{\footnotesize {``A pineapple on the ground"}}\hfill
\jsubfig{\includegraphics[height=1.9cm]{images/real_scenes/pinecone/pinecone_ours.png}}{\footnotesize {``A pinecone floating in a pond"}}
\vspace{1.5pt}
\caption{Qualitative results of our method on the publicly available $360^\circ$ \emph{Real Scenes} \cite{mildenhall2021nerf}. We show images rendered from the initial grid in the middle row and from the our edited grid in the bottom row. As illustrated above, our method can generate convincing edits on these real scenes without explicitly modeling the foreground and background regions. 
 }
\label{fig:realscenes}
\end{figure}


\section{Experiments}
\label{sec:results}
We show qualitative editing results over diverse 3D objects and various edits in Figures \ref{fig:teaser}, \ref{fig:results}, \ref{fig:realscenes}, \ref{fig:comparison2d}, \ref{fig:comparisons}, \ref{fig:t23dablation}. 
Please refer to the supplementary material for many fly-through visualizations demonstrating that our results are indeed consistent across different views.
In Figure \ref{fig:realscenes}, we demonstrate that our method also succeeds in modeling and editing real scenes using the  $360^\circ$ \emph{Real Scenes} available by Mildenhall et al.~\cite{mildenhall2021nerf}. As illustrated in the figure, our approach can locally edit the foreground (e.g., turning the flowers into sunflowers) or the background (e.g. turning the ground into a pond).

To assess the quality of our object editing approach, we conduct several sets of experiments, evaluating both the quality of the images rendered from our volumetric representation as well as the extent of which these rendered images conform to the target text prompt.  We compare to prior 3D object editing methods in Section \ref{sec:comp3D}. In Section \ref{sec:comp2D}, we compare with state-of-the-art image editing techniques. We show ablations in Section \ref{sec:ablations}. Finally, we discuss limitations in Section \ref{sec:limitations}.

\medskip \noindent \textbf{Synthetic Object Dataset.} 
We assembled a dataset using five freely available meshes found on the internet. %
Each mesh was rendered from 100 views in Blender. 
For a quantitative evaluation, we paired each object in our dataset with a number of both local and global edit prompts including:
\begin{itemize}
    \item ``A $\left<object\right>$ wearing sunglasses''.
    \item ``A $\left<object\right>$ wearing a wizard's hat''. 
    \item ``A $\left<object\right>$ wearing a Christmas sweater''.
    \item ``A  yarn doll of a $\left<object\right>$''.
    \item ``A wood carving of a $\left<object\right>$''.
\end{itemize}
We separately evaluate local and global edits. For instance, the first three prompts above are considered local edits and the last two as edits that should produce global edits. We provide additional details in the supplementary material.


\subsection{Metrics}
\medskip \noindent \textbf{Edit Fidelity.} We evaluate how well the generated results capture the target text prompt using two metrics:  
 \smallskip \newline \emph{CLIP Similarity} ($\text{CLIP}_{Sim}$) measures the semantic similarity between the output objects and the target text prompts. We encode both the prompt and images rendered from our 3D outputs using CLIP's text and image encoders, respectively, and measure the cosine-distance between these encodings. 
 \smallskip \newline \emph{CLIP Direction Similarity} ($\text{CLIP}_{Dir}$) evaluates the quality of the edit in regards to the input by measuring the directional CLIP similarity first introduced by Gal et al.~\cite{gal2021stylegannada}. This metric measures the cosine distance between the direction of the change from the input and output rendered images and the direction of the change from an input prompt (\emph{i.e.} ``a dog") to the one describing the edit (\emph{i.e.} ``a dog wearing a hat").


\medskip \noindent \textbf{Edit Magnitude.}  For ablating components in our model, we use the Frech\'et Inception Distance (FID) \cite{Heusel2017GANsTB,Seitzer2020FID} to measure the difference in visual appearance between: (i) the output and input images ($\text{FID}_{Input}$) and (ii) the output and images generated by the initial reconstruction grid ($\text{FID}_{Rec}$). We show both to demonstrate to what extent the appearance is affected by the edit versus the expressive power of our framework.





\begin{figure} %
\centering
\rotatebox{90}{Latent-Paint~}
\jsubfig{\includegraphics[height=1.91cm]{images/comparisons/latent_paint/paint_wood.png}}{}\hfill
\jsubfig{\includegraphics[height=1.91cm]{images/comparisons/latent_paint/paint_santa.png}}{} \hfill
\jsubfig{\includegraphics[height=1.91cm]{images/comparisons/latent_paint/paint_donkey.png}}{}  \hfill
\jsubfig{\includegraphics[height=1.91cm]{images/comparisons/latent_paint/paint_carousel.png}}{} \vspace{1.5pt} 
\\ 
\rotatebox{90}{SketchShape}
\jsubfig{\includegraphics[height=1.91cm]{images/comparisons/sketch_shape/sketch_shape_wood.png}}{} \hfill
\jsubfig{\includegraphics[height=1.91cm]{images/comparisons/sketch_shape/sketch_shape_santa.png}}{} \hfill
\jsubfig{\includegraphics[height=1.91cm]{images/comparisons/sketch_shape/sketch_shape_donkey.png}}{} \hfill
\jsubfig{\includegraphics[height=1.91cm]{images/comparisons/sketch_shape/sketch_shape_carousel.png}}{} \vspace{1.5pt} 
\\
\rotatebox{90}{\whitetxt{x}Text2Mesh}
\jsubfig{\includegraphics[height=1.91cm]{images/comparisons/text_2_mesh/text_2_mesh_wood.png}}{}\hfill
\jsubfig{\includegraphics[height=1.91cm]{images/comparisons/text_2_mesh/text_2_mesh_santa.png}}{} \hfill
\jsubfig{\includegraphics[height=1.91cm]{images/comparisons/text_2_mesh/text_2_mesh_donkey.png}}{} \hfill
\jsubfig{\includegraphics[height=1.91cm]{images/comparisons/text_2_mesh/text_2_mesh_carousel.png}}{} \vspace{1.5pt} 
\\ 
\rotatebox{90}{\whitetxt{xxx}Ours}
\jsubfig{\includegraphics[height=1.91cm]{images/comparisons/ours/ours_wood.png}}{\footnotesize {``A wood carving of a horse"}}\hfill
\jsubfig{\includegraphics[height=1.91cm]{images/comparisons/ours/ours_santa.png}}{\footnotesize {``Horse wearing a santa hat"}} \hfill
\jsubfig{\includegraphics[height=1.91cm]{images/comparisons/ours/ours_donkey.png}}{\footnotesize {``A donkey"}}\hfill
\jsubfig{\includegraphics[height=1.91cm]{images/comparisons/ours/ours_carousel.png}}{\footnotesize {``A carousel horse"}}
\vspace{5pt} 
\caption{\textbf{Comparison to other 3D Object editing techniques}. We show qualitative results obtained using  Text2Mesh~\cite{michel2022text2mesh} and two applications of Latent-NeRF ~\cite{metzer2022latent} (Latent-Paint and SketchShape) and compare to our method. To accommodate their problem setting, all methods are provided with uncolored meshes. Note that the input meshes are visible on the top row (as Latent-Paint does not edit the object's geometry). As illustrated above, prior methods struggle at achieving semantic localized edits. Our method succeeds, while maintaining high fidelity to input object. 
}
\label{fig:comparisons}
\end{figure}

 \section{Ablations}
\label{sec:ab}
In this section, we show a more detailed ablation study which evaluates the effect of our volumetric regularization loss (Section \ref{sec:ablation-reg}) and an additional experiment, demonstrating the effect of using high order spherical harmonics coefficients (Section \ref{sec:sh}).

\subsection{Alternative Regularization Objectives}
\label{sec:ablation-reg}
Table \ref{tab:grid_losses} shows a quantitative comparison over different image-space and volumetric regularizations. Only the image-space L1 loss also appears in the main paper. % (Table 2). However, unfortunately we only noticed after the paper deadline that we incorrectly reported that specific experiment---the actual performance of this ablation is significantly lower, as illustrated in this corrected table. 
Below we provide additional details on these ablations.

\begin{figure}[h!] %
\centering %17v,20h
  \jsubfig{\includegraphics[height=1.9cm]{images/comparisons/image_editing/dog_white.png}}{}
\jsubfig{\includegraphics[height=1.9cm]{images/ablations/fcl/duck_nofcl.png}}{} 
\jsubfig{\includegraphics[height=1.9cm] {images/comparisons/image_editing/dog_white.png}}{} %\hspace{0.005pt}
\jsubfig{\includegraphics[height=1.9cm]{images/ablations/fcl/dog_nofcl_cropped.png}}{}
\rotatebox[origin=tc]{-90}{$\mathcal{L}_{reg3D}$ \whitetxt{xxxxx}}
\vspace{-10pt}
\\ %20v, 20h
\jsubfig{\includegraphics[height=1.93cm]{images/ablations/fcl/duck_ref_small.png}}{\footnotesize {Input}}
\jsubfig{\includegraphics[height=1.9cm]{images/ablations/fcl/duck_fcl.png}}{\footnotesize {''A wood carving of a duck"}} %\hspace{0.005pt} % [trim={left bottom right top},clip]
\jsubfig{\includegraphics[height=1.9cm] {images/ablations/fcl/dog_ref_cropped3.png}}{\footnotesize {Input}} %\hspace{0.005pt}
\jsubfig{\includegraphics[height=1.9cm]{images/ablations/fcl/dog_fcl_cropped.png}}{\footnotesize {''A dog wearing a christmas sweater"}}
\rotatebox[origin=tc]{-90}{$\mathcal{L}_{reg3D++}$ \whitetxt{xxxxxx}}%\hspace{0.005pt}
\vspace{3.2pt}
%\jsubfig{\includegraphics[height=1.9cm]{images/comparisons/image_editing/dog_postpix2pix_000062b.png}}{}

%\vspace{-12pt} 
\caption{\textbf{Regularizing RGB colors in addition to volumetric densities}. We show results obtained when using our default regularization objective $\mathcal{L}_{reg3D}$ (top-row) compared against results obtained when using $\mathcal{L}_{reg3D++}$- an alternative version of $\mathcal{L}_{reg3D}$ (bottom-row) in which we penalize the miscorrelation between both density and color features. These results show that regularizing both density and RGB can be limiting, especially when the edit requires a drastic change in color, such as changing the white fur of the dog into a vibrant christmas sweater. }

\label{fig:supp_fcl}
\end{figure}


\paragraph{Image-space Regularization}
%We ablate our choice to use our Volumetric Regularization loss $\mathcal{L}_\text{reg3D}$ by setting its weight to zero and replacing it with image space losses. 
In this setting we render images from our editing grid $G_e$ in the poses corresponding to the input images during each iteration of the optimization stage. Rather than using a volumetric regularization, we incur a loss between the images rendered from $G_e$ and the corresponding input image while using the same weight used to balance $\mathcal{L}_\text{reg3D}$ with the annealed SDS loss (this weight is set to $200$, as detailed in Section \ref{sec:imp}). We evaluate this ablation using $L_1$ and $L_2$ image space loss functions. %For each function, we generate all scenes in the test set and compare them against outputs generated by our unaltered method.

\begin{table}[t]
\centering
%\addtolength{\tabcolsep}{2pt}
\resizebox{\linewidth}{!}{
\begin{tabular}{llcccc}
\toprule
 &Loss Function & $\text{CLIP}_{Sim}\uparrow$ & $\text{CLIP}_{Dir}\uparrow$ & $\text{FID}_{Rec}\downarrow$  & $\text{FID}_{Input}\downarrow$ \\
\midrule
\multirow{2}{*}{2D}&$L_1$ & 0.26 & 0.02 & 415.96 & 437.09  \\ 
&$L_2$ & 0.25 & 0.02 & 437.68 & 467.14 \\
\midrule
\multirow{3}{*}{3D}&$L_1$ & 0.36 & 0.05 & 222.91 & 284.86  \\ 
&$L_2$ & 0.35 & 0.05 & 240.50 & 284.83 \\
&$\mathcal{L}_{reg3D++}$ & 0.34 & 0.02 & \textbf{210.46} & \textbf{242.73} 
\\
&$\mathcal{L}_\text{reg3D}$ & \textbf{0.36} & \textbf{0.06} & 223.89 & 272.73 \\
\bottomrule
\end{tabular}
}
\\

\caption{\textbf{Detailed ablation study}, evaluating the effect of different  regularization objectives. We compare the performance using $\mathcal{L}_\text{reg3D}$, with image-space (top rows) and volumetric (bottom rows) $L_1$ and $L_2$ losses, as well as  $\mathcal{L}_\text{reg3D++}$, which also penalizes miscorrelations between color features.}
%loss while learning decade models ($\mathcal{L}_\text{id}^{\text{(I)}}$) and during \methodshort{} ($\mathcal{L}_\text{id}^{\text{(II)}}$), using a blended image with layer swapping (LS), \methodshort{}, and masking the images during \methodshort{} (Mask).  \hec{this is a placeholder} %$\mathcal{L}_\text{id}$: with/without ID loss, IVT: with/without IVT refinement,  Mask: with/without segmentation mask during refinement. 
%LS with/without layer swapping for the ID loss,
\label{tab:grid_losses}
\end{table}

\paragraph{Alternative Volumetric Regularization Functions}
In this setting we replace our correlation-based regularization with
%We ablate our choice of implementation for the Volumetric Regularization loss by replacing our density feature correlation encouraging implementation $\mathcal{L}_\text{reg3D}$ with 
other functions that encourage similarity between the density features of the grids using the same balancing weight. % (\emph{i.e.}, $200$). 
Namely we compare against $L1$ and $L2$ volumetric loss functions, both penalizing the distance between the density features of $G_i$ and those of $G_e$. We additionally compare against an alternative version of $\mathcal{L}_\text{reg3D}$ in which we penalize the miscorrelation between both density and color features, formally:

\begin{equation}
    \mathcal{L}_{reg3D++}  = \mathcal{L}_{reg3D} + (
1 - \frac{Cov(f^{rgb}_i, f^{rgb}_e)}
{\sqrt{Var(f^{rgb}_i)Var(f^{rgb}_e)}}
)
\end{equation}

We find that using this loss yields better reconstruction scores, at the expense of significantly lower CLIP-based scores (e.g., $\text{CLIP}_{Dir}$ scores drop from 0.08 to 0.02). Qualitatively, constraining RGB values as well as density features appears too limiting for our purposes. This can be seen in Figure \ref{fig:supp_fcl}, where we compare results obtained when using $\mathcal{L}_{reg3D++}$ against results obtained when using $\mathcal{L}_{reg3D}$. When observing these results, we can see that the edit integrity is reduced at the expense of the preservation of the origin object's color. This is evident in the duck, for instance, where the brown wooden color of the body is only clearly visible in the $\mathcal{L}_{reg3D}$ example. Furthermore, the colors of the sweater on the dog are significantly faded when regularized with $\mathcal{L}_{reg3D++}$ as the colors of a standard christmas sweater are typically much more vibrant than the white fur of the dog.

%We again generate all scenes in the test set and compare them against outputs generated by our unaltered method for each function.

\ignorethis{
\subsubsection*{Refinement}
We ablate the refinement stage of our pipeline by comparing our system's outputs with and without refining the outputs of the editing stage. Note that this ablation only effects local prompts.
\subsubsection*{Image-space Regularization}
We ablate our choice to use our Volumetric Regularization loss $\mathcal{L}_\text{reg3D}$ by setting its weight to zero and replacing it with image space losses. In this setting we render images from our editing grid $G_e$ in the poses corresponding to the input images at each iteration of the Text-Guided Object Editing stage. We incur a loss between the images rendered from $G_e$ and the corresponding input image while using the same weight used to balance $\mathcal{L}_\text{reg3D}$ with the annealed SDS loss - $200$. We evaluate this ablation using two different image space loss functions - $L_1$ and $L_2$. For each function we generate all scenes in the test set and compare them against outputs generated by our unaltered method.

\subsubsection*{Alternative Volumetric Regularization Functions}
We ablate our choice of implementation for the Volumetric Regularization loss by replacing our density feature correlation encouraging implementation $\mathcal{L}_\text{reg3D}$ with other functions that encourage similarity between the density features of $G_e$ and $G_i$ and using the same balancing weight - $200$. Namely we test two different loss functions - $L1$ and $L2$, both penalizing the distance between the density features of $G_i$ and those of $G_e$. We again generate all scenes in the test set and compare them against outputs generated by our unaltered method for each function.

\subsubsection*{Refinement}
We ablate the refinement stage of our pipeline by comparing our system's outputs with and without refining the outputs of the editing stage. Note that this ablation only effects local prompts.
}

\begin{figure*} %
\centering
\rotatebox{90}{2D cross-attention}\hfill% \hspace{0.001pt}
\jsubfig{\includegraphics[height=2.72cm]{images/supp_timestamps/kangaroo/2d/ts_1_view_120.png}
\includegraphics[height=2.72cm]{images/supp_timestamps/kangaroo/2d/ts_200_view_120.png}
\includegraphics[height=2.72cm]{images/supp_timestamps/kangaroo/2d/ts_400_view_120.png}
\includegraphics[height=2.72cm]{images/supp_timestamps/kangaroo/2d/ts_600_view_120.png}
\includegraphics[height=2.72cm]{images/supp_timestamps/kangaroo/2d/ts_800_view_120.png}
\includegraphics[height=2.72cm]{images/supp_timestamps/kangaroo/2d/ts_999_view_120.png}}{}% \hspace{0.001pt}
\\
\rotatebox{90}{\whitetxt{xx}3D grid ($A_e$)} \hfill 
\jsubfig{\includegraphics[height=2.72cm]{images/supp_timestamps/kangaroo/3d/1.png}
\includegraphics[height=2.72cm]{images/supp_timestamps/kangaroo/3d/200.png}
\includegraphics[height=2.72cm]{images/supp_timestamps/kangaroo/3d/400.png}
\includegraphics[height=2.72cm]{images/supp_timestamps/kangaroo/3d/600.png}
\includegraphics[height=2.72cm]{images/supp_timestamps/kangaroo/3d/800.png}
\includegraphics[height=2.72cm]{images/supp_timestamps/kangaroo/3d/999.png}}{}
%\hspace{0.001pt}
\\
\rotatebox{90}{2D cross-attention}\hfill%\hspace{0.001pt}
\jsubfig{\includegraphics[height=2.72cm]{images/supp_timestamps/duck/2d/ts_1_view_120.png}
\includegraphics[height=2.72cm]{images/supp_timestamps/duck/2d/ts_200_view_120.png}
\includegraphics[height=2.72cm]{images/supp_timestamps/duck/2d/ts_400_view_120.png}
\includegraphics[height=2.72cm]{images/supp_timestamps/duck/2d/ts_600_view_120.png}
\includegraphics[height=2.72cm]{images/supp_timestamps/duck/2d/ts_800_view_120.png}
\includegraphics[height=2.72cm]{images/supp_timestamps/duck/2d/ts_999_view_120.png}}{}%\hspace{0.001pt}
\\
\rotatebox{90}{\whitetxt{xx}3D grid ($A_e$)}\hfill \hspace{8pt}
\jsubfig{\includegraphics[height=2.72cm]{images/supp_timestamps/duck/3d/1.png}}{\footnotesize {t=1}} 
\jsubfig{\includegraphics[height=2.72cm]{images/supp_timestamps/duck/3d/200.png}}{\footnotesize {t=200}} 
\jsubfig{\includegraphics[height=2.72cm]{images/supp_timestamps/duck/3d/400.png}}{\footnotesize {t=400}} 
\jsubfig{\includegraphics[height=2.72cm]{images/supp_timestamps/duck/3d/600.png}}{\footnotesize {t=600}}
\jsubfig{\includegraphics[height=2.72cm]{images/supp_timestamps/duck/3d/800.png}}{\footnotesize {t=800}}
\jsubfig{\includegraphics[height=2.72cm]{images/supp_timestamps/duck/3d/999.png}}{\footnotesize {t=999}}
\vspace{5pt} 
\caption{\textbf{Visualizing 2D cross-attention maps and 3d cross-attention grids over different diffusion timestamps}. We visualize the trained 3d cross-attention grids and the corresponding 2D cross-attention maps used as supervision across different diffusion timestamps.  We show them for the edit region corresponding to the token associated with the word ``rollerskates" (top two rows) and ``hat" (bottom two rows).}
\label{fig:supp_timestamps}
\end{figure*}


\ignorethis{
\begin{figure} %
\centering %[trim={left bottom right top},clip]
\jsubfig{\includegraphics[height=1.860cm,trim={1.2cm 0.8cm 1.2cm 1.6cm},clip] {images/t23dablation/input.png}}{\footnotesize {Input}}\hfill
\jsubfig{\includegraphics[height=1.860cm]{images/t23dablation/latent_nerf_skates_.png}\includegraphics[height=1.860cm,trim={1.2cm 0.8cm 1.2cm 1.6cm},clip]{images/t23dablation/ours_skates.png}}{\footnotesize {``Kangaroo on rollerskates"}} \hfill
\jsubfig{\includegraphics[height=1.860cm]{images/t23dablation/latent_nerf_skis_.png}\includegraphics[height=1.860cm,trim={0cm 0cm 0cm 1.75cm},clip]{images/t23dablation/ours_skis.png}}{\footnotesize {``Kangaroo on skis"}}
\vspace{-1pt} 
\caption{\textbf{Comparison to unconditional text-to-3D generation}. We compare to unconditional text-to-3D methods by comparing to Latent-NeRF ~\cite{metzer2022latent}, providing it with the two target prompts displayed above. We display these alongside our results (LatentNeRF on the left, ours on the right). As illustrated above, unconditional methods cannot easily match an input object, and are also not guaranteed to generate a consistent object over different prompts.  }
\label{fig:t23dablation}
\end{figure}

}




\subsection{Ablating the Color Representation}
\label{sec:sh}
As mentioned in Section 3.1 of the main paper, we  do not model view dependent effects using higher order spherical harmonics as that leads to undesirable effects. We demonstrate this by observing these effects in examples rendered with 1st and 2nd order spherical harmonic coefficients as color features. These results can be seen in videos available on our project page. % (please refer to  \url{index.html} for these videos). 

When observing these results we can clearly see how view-dependent colors yield undesirable effects such as the feet of the ``yarn kangaroo" varying from green to yellow across views or the head of the dog becoming a birthday party hat when it faces away from the camera. We additionally see the colors become over-saturated, especially when using second-order spherical harmonic coefficients. It is also evident that the added expressive capabilities of the model allow it to over-fit more easily to specific views, creating unrealistic results such as the ``cat wearing glasses" in the first and second order coefficient models, where glasses are scattered along various parts of its body. We note that while this expressive power currently produces undesirable effects it does potentially enable higher quality and more realistic renders, and therefore, we believe that constraining this power is an interesting topic for future research.

\subsection{Cross-attention Grid Supervision}
%\paragraph{Visualizing cross-attention maps from different timestamps}.
As explained in Section \ref{sec:imp}, we use a constant time-stamp of 0.2 when extracting attention maps for training our attention grids $A_e$ and $A_{obj}$. This value was chosen empirically as we found that higher time-steps tend to be noisier and less focused, while lower time-steps varied largely from pose to pose producing inferior attention grids. This can be seen qualitatively in Figure \ref{fig:supp_timestamps}. As illustrated in the figure, the attention values for the edit region get gradually more smeared and unfocused as the time-steps increase. This is evident, for instance, in warmer regions around the kangaroo's tail or the head of the duck. While perhaps less visually distinct, we can also observe that in lower timestamps the warm regions denoting high attention values cover a smaller area of the region which should be edited. We empirically find that this makes it more challenging for separating the object and edit regions.

%While not completely incorrect we find that this %makes separating the edit from the object harder in our implementation.\hec{Explain that we observe this and demonstrate it in Figure...}\ref{fig:supp_timestamps}.

 \begin{figure} %
\centering %[trim={left bottom right top},clip]
\jsubfig{\includegraphics[height=1.860cm,trim={1.2cm 0.8cm 1.2cm 1.6cm},clip] {images/t23dablation/input.png}}{\footnotesize {Input}}\hfill
\jsubfig{\includegraphics[height=1.860cm]{images/t23dablation/latent_nerf_skates_.png}\includegraphics[height=1.860cm,trim={1.2cm 0.8cm 1.2cm 1.6cm},clip]{images/t23dablation/ours_skates.png}}{\footnotesize {``Kangaroo on rollerskates"}} \hfill
\jsubfig{\includegraphics[height=1.860cm]{images/t23dablation/latent_nerf_skis_.png}\includegraphics[height=1.860cm,trim={0cm 0cm 0cm 1.75cm},clip]{images/t23dablation/ours_skis.png}}{\footnotesize {``Kangaroo on skis"}}
\vspace{-1pt} 
\caption{\textbf{Comparison to unconditional text-to-3D generation}. We compare to unconditional text-to-3D methods by comparing to Latent-NeRF ~\cite{metzer2022latent}, providing it with the two target prompts displayed above. We display these alongside our results (LatentNeRF on the left, ours on the right). As illustrated above, unconditional methods cannot easily match an input object, and are also not guaranteed to generate a consistent object over different prompts.  }
\label{fig:t23dablation}
\end{figure}

\subsection{3D Object Editing Comparisons}
\label{sec:comp3D}

To the best of our knowledge, there are no existing methods that can directly perform our task of text-guided localized edits for 3D objects given a set of posed input images. 
Thus, we consider Text2Mesh~\cite{michel2022text2mesh} and Latent-Nerf~\cite{metzer2022latent} which can be applied in a similar setting to ours. These experiments highlight the differences between prior works and our proposed object editing technique.

Text2Mesh~\cite{michel2022text2mesh} aims at editing the style of a given input mesh to conform to a target prompt with a style transfer network that predicts color and a displacement along the normal direction. As it only predicts displacements along the normal direction, the geometric edits enabled by Text2Mesh are limited mostly to small changes.
Latent-Paint and SketchShape are two applications introduced in Latent-Nerf~\cite{metzer2022latent} which operate on input meshes. 
SketchShape generates shape and appearance from coarse input geometry, while Latent-Paint only edits the appearance of an existing mesh.
Note that these methods are designed for slightly more constrained inputs than our approach. While our focus is on editing 3D models with arbitrary textures (as depicted from associated imagery), they only operate on uncolored meshes. 

We show a qualitative comparison in Figure \ref{fig:comparisons} over an uncolored mesh (its geometry can be observed on the top row as Latent-Paint keeps the input geometry fixed). As illustrated in the figure, Text2Mesh cannot produce significant geometric edits (\emph{e.g.}, adding a Santa hat to the horse or turning the horse into a donkey). Even SketchShape, which is designed to allow geometric edits, cannot achieve significant localized edits. Furthermore, it fails to preserve the geometry of the input---although, we again note that this method is not intended to preserve the input geometry. 
Our method, on the other hand, succeeds in conforming to the target text prompt, while preserving the input geometry, allowing for semantically meaningful changes to both geometry and appearance.

We perform a quantitative evaluation in Table \ref{tab:baseline-stats} on our dataset. To perform a fair comparison where all methods operate within their training domain, we use meshes without texture maps as input for all baseline methods. As illustrated in the table, our method outperforms all baselines over both local and global edits in terms of CLIP similarity, but Text2Mesh yields slightly higher CLIP direction similarity. We note that Text2Mesh is advantaged in terms of the CLIP metrics as it explicitly optimizes on CLIP similarities and thus its scores are not entirely indicative. 

In Figure \ref{fig:t23dablation} we compare to the unconditional text-to-3D model proposed in Latent-NeRF, to show that such unconditional models are also not guaranteed to generate a consistent object over different prompts. We also note that this result (as well as our edits) would certainly look better if fueled with a proprietary big diffusion model \cite{saharia2022photorealistic}, but nonetheless, these models cannot preserve identity.

\begin{figure} %
\centering %17v,20h
 \jsubfig{\includegraphics[height=1.9cm]{images/comparisons/image_editing/dog_white.png}}{}
 \rotatebox{90}{\whitetxt{x}SDEdit\whitetxt{x}\includegraphics[height=0.24cm]{images/bg.png}}
 \hfill
\jsubfig{\includegraphics[height=1.9cm]{images/comparisons/image_editing/sde_dog_1.jpg}}{} 
\jsubfig{\includegraphics[height=1.9cm] {images/comparisons/image_editing/sde_dog_2.jpg}}{} %\hspace{0.005pt}
\jsubfig{\includegraphics[height=1.9cm]{images/comparisons/image_editing/sde_dog_3.jpg}}{} %\hspace{0.005pt}
%\jsubfig{\includegraphics[height=1.9cm]{images/comparisons/image_editing/sde_dog_4.jpg}}{}
\\ %20v, 20h
\jsubfig{\includegraphics[height=1.9cm]{images/comparisons/image_editing/dogpix2pixin_000004b.png}}{}
\rotatebox{90}{\whitetxt{x}IPix2Pix\whitetxt{x}\includegraphics[height=0.24cm]{images/bg.png}}
\hfill 
\hfill 
\jsubfig{\includegraphics[height=1.9cm]{images/comparisons/image_editing/dog_postpix2pix_000004b.png}}{} %\hspace{0.005pt} % [trim={left bottom right top},clip]
\jsubfig{\includegraphics[height=1.9cm] {images/comparisons/image_editing/dog_postpix2pix_000087b.png}}{} %\hspace{0.005pt}
\jsubfig{\includegraphics[height=1.9cm]{images/comparisons/image_editing/dog_postpix2pix_000058b.png}}{} %\hspace{0.005pt}
%\jsubfig{\includegraphics[height=1.9cm]{images/comparisons/image_editing/dog_postpix2pix_000062b.png}}{}
\\
\jsubfig{\includegraphics[height=1.9cm] {images/comparisons/image_editing/dog_white.png}}{}
\rotatebox{90}{\whitetxt{xxx}SDEdit}
\hfill
\jsubfig{\includegraphics[height=1.9cm]{images/comparisons/image_editing/sde_dog_1_nobg.jpg}}{} %\hfill % [trim={left bottom right top},clip]
\jsubfig{\includegraphics[height=1.9cm] {images/comparisons/image_editing/sde_dog_2_nobg.jpg}}{} %\hfill
\jsubfig{\includegraphics[height=1.9cm]{images/comparisons/image_editing/sde_dog_3_nobg.jpg}}{} %\hfill
%\jsubfig{\includegraphics[height=1.9cm]{images/comparisons/image_editing/sde_dog_4_nobg.jpg}}{}
\\ %20v, 20h
\jsubfig{\includegraphics[height=1.9cm] {images/comparisons/image_editing/dog_white.png}}{}
\rotatebox{90}{\whitetxt{xx}IPix2Pix}
\hfill
\jsubfig{\includegraphics[height=1.9cm]{images/comparisons/image_editing/postpix2pix_050.png}}{} %\hfill % [trim={left bottom right top},clip]
\jsubfig{\includegraphics[height=1.9cm] {images/comparisons/image_editing/postpix2pix_000.png}}{} %\hfill
\jsubfig{\includegraphics[height=1.9cm]{images/comparisons/image_editing/postpix2pix_090.png}}{} %\hfill
%\jsubfig{\includegraphics[height=1.9cm]{images/comparisons/image_editing/dog_postpix2pix_000062.png}}{}
\\
\jsubfig{\includegraphics[height=1.9cm]{images/comparisons/image_editing/dog_pix2pixin_000004.png}}{}\rotatebox{90}{\whitetxt{xxxxx}Ours}
\hfill
\jsubfig{\includegraphics[height=1.9cm]{images/comparisons/image_editing/color_50_dog.png}}{} %\hfill % [trim={left bottom right top},clip]
\jsubfig{\includegraphics[height=1.9cm] {images/comparisons/image_editing/color_0_dog.png}}{} %\hfill
\jsubfig{\includegraphics[height=1.9cm] {images/comparisons/image_editing/color_90_dog.png}}{} %\hfill
%\jsubfig{\includegraphics[height=1.9cm]{images/comparisons/image_editing/color_150_dog.png}}{}
%\\
\vspace{-12pt} 
\caption{\textbf{Comparison to 2D image editing techniques}. We compare to the text-guided image editing techniques InstructPix2Pix (IPix2Pix)~\cite{brooks2022instructpix2pix} and SDEdit~\cite{meng2022sdedit} by providing it with images from different viewpoints and a target instruction text prompt (``put sunglasses on the dog" for IPix2Pix and ``a dog with sunglasses" for SDEdit and our method). We show one input image on the left, and three outputs on the right (side, front and back views), where the leftmost output corresponds to the input viewpoint. We show two variants, one with added backgrounds (top rows), as we observe that it allows for better preserving the object's appearance. As illustrated above, 2D techniques cannot easily achieve 3D-consistent edit results (illustrated, for instance, by the sunglasses added on the dog's back). %\todo{Replace second example} 
}
\label{fig:comparison2d}
\end{figure}



%\rotatebox{90}%{SDEdit\whitetxt{x}\includegraphics[height=0.24cm]%{images/bg.png}}
%\jsubfig{\includegraphics[height=1.9cm]%{images/comparisons/image_editing/pix2pixin_50b.png}}%{}\hfill 
%\jsubfig{\includegraphics[height=1.9cm]%{images/comparisons/image_editing/postpix2pix_50b.png}}%{} %\hspace{0.005pt} % [trim={left bottom right %top},clip]
%\jsubfig{\includegraphics[height=1.9cm] %{images/comparisons/image_editing/postpix2pix_0b.png}}%{} %\hspace{0.005pt}
%\jsubfig{\includegraphics[height=1.9cm]%{images/comparisons/image_editing/postpix2pix_90b.png}}%{} %\hspace{0.005pt}
%\jsubfig{\includegraphics[height=1.9cm]%{images/comparisons/image_editing/postpix2pix_150b.png}%}{}
%\\ %20v, 20h
%\rotatebox{90}{\whitetxt{x}IPix2Pix}
%\jsubfig{\includegraphics[height=1.9cm] %{images/comparisons/image_editing/pix2pixin_50.png}}%{}\hfill
%\jsubfig{\includegraphics[height=1.9cm]%{images/comparisons/image_editing/postpix2pix_50.png}}%{} %\hfill % [trim={left bottom right top},clip]
%\jsubfig{\includegraphics[height=1.9cm] %{images/comparisons/image_editing/postpix2pix_0.png}}{}% %\hfill
%\jsubfig{\includegraphics[height=1.9cm]%{images/comparisons/image_editing/postpix2pix_90.png}}%{} %\hfill
%\jsubfig{\includegraphics[height=1.9cm]%{images/comparisons/image_editing/postpix2pix_150.png}}%{}
%\\
%\rotatebox{90}{IPix2Pix\includegraphics[height=0.24cm]%{images/bg.png}}
%\jsubfig{\includegraphics[height=1.9cm]%{images/comparisons/image_editing/pix2pixin_50b.png}}%{}\hfill 
%\jsubfig{\includegraphics[height=1.9cm]%{images/comparisons/image_editing/postpix2pix_50b.png}}%{} %\hspace{0.005pt} % [trim={left bottom right %top},clip]
%\jsubfig{\includegraphics[height=1.9cm] %{images/comparisons/image_editing/postpix2pix_0b.png}}%{} %\hspace{0.005pt}
%\jsubfig{\includegraphics[height=1.9cm]%{images/comparisons/image_editing/postpix2pix_90b.png}}%{} %\hspace{0.005pt}
%\jsubfig{\includegraphics[height=1.9cm]%{images/comparisons/image_editing/postpix2pix_150b.png}%}{}
%\\ %20v, 20h
%\rotatebox{90}{\whitetxt{x}IPix2Pix}
%\jsubfig{\includegraphics[height=1.9cm] %{images/comparisons/image_editing/pix2pixin_50.png}}%{}\hfill
%\jsubfig{\includegraphics[height=1.9cm]%{images/comparisons/image_editing/postpix2pix_50.png}}%{} %\hfill % [trim={left bottom right top},clip]
%\jsubfig{\includegraphics[height=1.9cm] %{images/comparisons/image_editing/postpix2pix_0.png}}{}% %\hfill
%\jsubfig{\includegraphics[height=1.9cm]%{images/comparisons/image_editing/postpix2pix_90.png}}%{} %\hfill
%\jsubfig{\includegraphics[height=1.9cm]%{images/comparisons/image_editing/postpix2pix_150.png}}%{}
%\\
%\rotatebox{90}{\whitetxt{xxx}Ours}
%\jsubfig{\includegraphics[height=1.9cm,clip]%{images/comparisons/image_editing/pix2pixin_50.png}}%{}\hfill
%\jsubfig{\includegraphics[height=1.9cm]%{images/comparisons/image_editing/color_50.png}}{} %%\hfill % [trim={left bottom right top},clip]
%\jsubfig{\includegraphics[height=1.9cm] %{images/comparisons/image_editing/color_0.png}}{} %%\hfill
%\jsubfig{\includegraphics[height=1.9cm] %{images/comparisons/image_editing/color_90.png}}{} %%\hfill
%\jsubfig{\includegraphics[height=1.9cm]{images/comparisons/image_editing/color_150.png}}{}


\subsection{2D Image Editing Comparisons}

An underlying assumption in our work is that editing 3D geometry cannot easily be done by reconstructing edited 2D images depicting the scene. To test this hypothesis, we modified images rendered from various viewpoints using the diffusion-based image editing methods InstructPix2Pix~\cite{brooks2022instructpix2pix} and SDEdit~\cite{meng2022sdedit}. %
As illustrated in Figure~\ref{fig:comparison2d}, 2D methods often struggle to produce meaningful results from less \emph{canonical} views (\emph{e.g.}, adding sunglasses on the dog's back) and also produce highly view-inconsistent results. %

\label{sec:comp2D}


\subsection{Ablations}
\label{sec:ablations}
\begin{table}[h!]\scriptsize	
    \centering
    \begin{tabular}{c|c|c|c|c|c}
        \multicolumn{2}{c|}{} & Baseline & w/o normals & w/o viscosity & w/o coarea \\ \hline
        \multirow{4}{*}{Anchor}
            & $d_C$ & \textbf{0.21} & 0.61 & 0.55 & 0.72 \\
            & $d_H$ & \textbf{3.00} & 7.82 & 10.83 & 10.24 \\
            & $d_C^\too$ & 0.15 & 0.37 & 0.27 & 0.36 \\
            & $d_H^\too$ & 1.07 & 7.84 & 1.44 & 9.68 \\ \hline
        \multirow{4}{*}{Daratech}
            & $d_C$ & 0.26 & 0.24 & 0.24 & \textbf{0.23} \\
            & $d_H$ & 4.06 & 4.2 & 4.3 & \textbf{2.19} \\
            & $d_C^\too$ & 0.14 & 0.13 & 0.12 & 0.13 \\
            & $d_H^\too$ & 1.76 & 2.69 & 1.77 & 1.77 \\ \hline
        \multirow{4}{*}{DC}
            & $d_C$ & \textbf{0.15} & \textbf{0.15} & \textbf{0.15} & 0.34 \\
            & $d_H$ & \textbf{2.22} & 2.24 & 2.24 & 6.58 \\
            & $d_C^\too$ & 0.09 & 0.08 & 0.08 & 0.16 \\
            & $d_H^\too$ & 2.76 & 2.76 & 2.79 & 2.82 \\ \hline
        \multirow{4}{*}{Gargoyle}
            & $d_C$ & \textbf{0.17} & 0.58 & 0.47 & 0.59 \\
            & $d_H$ & \textbf{4.40} & 6.32 & 10.38 & 6.35 \\
            & $d_C^\too$ & 0.11 & 0.07 & 0.26 & 0.38 \\
            & $d_H^\too$ & 0.96 & 2.39 & 1.34 & 1.25 \\ \hline
        \multirow{4}{*}{Lord Quas}
            & $d_C$ & \textbf{0.12} & 0.12 & 0.12 & 0.58 \\
            & $d_H$ & 1.06 & 1.38 & \textbf{1.04} & 6.05 \\
            & $d_C^\too$ & 0.07 & 0.37 & 0.06 & 0.32 \\
            & $d_H^\too$ & 0.64 & 0.69 & 0.64 & 3.73 \\ \hline %
            
    \end{tabular} \vspace{5pt}
    \caption{Ablations study. We show the contribution of each component of VisCo Grids. Baseline is the full method. The remaining columns correspond to optimizing without normal loss, viscosity loss and coarea loss, respectively. We show results for each mesh of the benchmark \cite{williams2019deep}. The results justify the use of the different components in VisCo Grids.}
    \label{tab:ablations}
\end{table}
\begin{table}[h!]\scriptsize	
    \centering
    \begin{tabular}{c|c|c|c|c|c}
        \multicolumn{2}{c|}{} & Baseline & w/o normals & w/o viscosity & w/o coarea \\ \hline
        \multirow{4}{*}{Anchor}
            & $d_C$ & \textbf{0.21} & 0.61 & 0.55 & 0.72 \\
            & $d_H$ & \textbf{3.00} & 7.82 & 10.83 & 10.24 \\
            & $d_C^\too$ & 0.15 & 0.37 & 0.27 & 0.36 \\
            & $d_H^\too$ & 1.07 & 7.84 & 1.44 & 9.68 \\ \hline
        \multirow{4}{*}{Daratech}
            & $d_C$ & 0.26 & 0.24 & 0.24 & \textbf{0.23} \\
            & $d_H$ & 4.06 & 4.2 & 4.3 & \textbf{2.19} \\
            & $d_C^\too$ & 0.14 & 0.13 & 0.12 & 0.13 \\
            & $d_H^\too$ & 1.76 & 2.69 & 1.77 & 1.77 \\ \hline
        \multirow{4}{*}{DC}
            & $d_C$ & \textbf{0.15} & \textbf{0.15} & \textbf{0.15} & 0.34 \\
            & $d_H$ & \textbf{2.22} & 2.24 & 2.24 & 6.58 \\
            & $d_C^\too$ & 0.09 & 0.08 & 0.08 & 0.16 \\
            & $d_H^\too$ & 2.76 & 2.76 & 2.79 & 2.82 \\ \hline
        \multirow{4}{*}{Gargoyle}
            & $d_C$ & \textbf{0.17} & 0.58 & 0.47 & 0.59 \\
            & $d_H$ & \textbf{4.40} & 6.32 & 10.38 & 6.35 \\
            & $d_C^\too$ & 0.11 & 0.07 & 0.26 & 0.38 \\
            & $d_H^\too$ & 0.96 & 2.39 & 1.34 & 1.25 \\ \hline
        \multirow{4}{*}{Lord Quas}
            & $d_C$ & \textbf{0.12} & 0.12 & 0.12 & 0.58 \\
            & $d_H$ & 1.06 & 1.38 & \textbf{1.04} & 6.05 \\
            & $d_C^\too$ & 0.07 & 0.37 & 0.06 & 0.32 \\
            & $d_H^\too$ & 0.64 & 0.69 & 0.64 & 3.73 \\ \hline %
            
    \end{tabular} \vspace{5pt}
    \caption{Ablations study. We show the contribution of each component of VisCo Grids. Baseline is the full method. The remaining columns correspond to optimizing without normal loss, viscosity loss and coarea loss, respectively. We show results for each mesh of the benchmark \cite{williams2019deep}. The results justify the use of the different components in VisCo Grids.}
    \label{tab:ablations}
\end{table}
We provide an ablation study in Table~\ref{tab:ablations} and Figure~\ref{fig:ablations}. Specifically, we ablate our volumetric regularization ($\mathcal{L}_{reg3D}$) and our 3D cross-attention-based spatial refinement module (SR). When ablating our volumetric regularization, we use a single volumetric grid and regularize the SDS objective with an image-based L2 regularization loss. More details and additional ablations are provided in the supplementary material.

The benefit of using our volumetric regularization is further illustrated in Figure~\ref{fig:ablations}, which shows that image-space regularization leads to very noisy results, and often complete failures (see, for instance, the cat result, where the output is not at all correlated with the input object). Quantitatively, we can also observe that images rendered from these models are of significantly different appearance (as measured using the FID metrics). 

Regarding the SR module, as expected, it increases  similarity to the inputs (reflected in lower FID scores). This is also clearly visible in Figure \ref{fig:ablations}---for example, geometric differences are apparent by looking at the animals' paws. The output textures after refinement also are more similar to the input textures. However, we also see that this module slightly hinders CLIP similarity to the edit and text prompt. This is also somewhat expected as we are further constraining the output to stay similar to the input, sometimes at the expense of the editing signal. 


\subsection{Limitations}
\label{sec:limitations}
\section{Limitations and Future Work}

We summarize the limitations we have identified for our method and propose
future research directions.

\textbf{Parallel implementation:} 
With a focus on accuracy and algorithms, our implementation for this work is
serial. Some of the most time-consuming routines in our method can easily
benefit from a parallel implementation, while the same is not obvious for the
SAP solver and the Schur complement computation. Leveraging the power of
parallelization on modern hardware for these computations is an interesting area
for future investigation.

\textbf{Rotational invariance:} 
As with all other linear constitutive models, our linearized model with lagged
rotational component is not rotationally invariant. Thus it is not suitable for
simulation of extreme deformations using large time steps. For those scenarios,
we fall back to traditional nonlinear models with Hessian positive definite
corrections proposed in \cite{bib:teran2005robust}.

\textbf{Self-contact:} 
We do not consider self-contact at the moment due to the lack of support by our
geometry engine. Self-contact can be incorporated into our method by updating the
geometry engine to augment the set of contacts reported.

\textbf{Tunneling at high speeds:} Though our method has a lower computational
cost, it could benefit from continuous collision detection strategies
\cite{bib:li2020ipc} to provide constraints before contact is established. This
would allow to mitigate issues such as objects tunneling past each other at high
speeds. Efficient solution to mitigate this issue is a topic of active research
for the authors.

\textbf{Redundant constraints:} Our geometry engine often introduces a large
number of constraints to resolve contact. Similarly, welding a large number of
deformable mesh vertices to a rigid body (as done in Section
\ref{sec:bubble_gripper}) introduces many constraints. Even though our SAP
solver \cite{bib:castro2022unconstrained} provides existence and uniqueness
guarantees, a large number of constraints hurts performance as can be observed
in the \emph{Soft-bubble} example. We are currently investigating strategies to
significantly reduce the number of constraints without sacrificing accuracy.


Our method applies a wide range of edits with high fidelity to 3D objects, however, there are several limitations to consider. As shown in Figure \ref{fig:limitations}, since we optimize over different views, our method attempts to edit the same object in differing spatial locations, thus failing on certain prompts. Moreover, the figure shows that some of our edits fail due to incorrect attribute binding, where the model binds attributes to the wrong subjects, which is a common challenge in large-scale diffusion-based models \cite{chefer2023attend}. 
Finally, we inherit the limitations of our volumetric representation. Thus, the quality of real scenes, for instance, could be significantly improved by borrowing ideas from works such as \cite{barron2022mip} (e.g. scene contraction to model the background).


\ignorethis{
\subsection{Object Editing Alternatives}
As no prior work directly address our task of performing such text-guided edits of 3D objects, we compare against several related works, adapting them to our problem setting and performing compherensive comparisons over various metrics.

\medskip \noindent \textbf{SOTA Text-guided Image Editing.} 

We begin by comparing output frames from our pipeline with SOTA Text-Guided Image editing methods that were given our input frames and text prompts as input. \esc{no need to run relu field on these}
\begin{itemize}
    \item Instructpix2pix + ReLU fields
    \item SDEdit + ReLU fields ?
    \item TEXT2LIVE + ReLU fields ?
    \item Best one + NeRF?
\end{itemize}

\medskip \noindent \textbf{Text-to-3D.} 
Bottom line message: look if we remove the original model guidance we get this noisy thing, of course this result and our edits would look better if fueled with a proprietary big diffusion model, but the point still stands that identity is not preserved.
\begin{itemize}
    \item DreamFusion
    \item LatentNeRF
\end{itemize}

\medskip \noindent \textbf{Neural Field Editing.} 
\begin{itemize}
    \item ?
\end{itemize}

\medskip \noindent \textbf{3D Object Editing} 
Bottom line message: look if we remove the original model guidance we get this noisy thing, of course this result and our edits would look better if fueled with a proprietary big diffusion model, but the point still stands that identity is not preserved.
\begin{itemize}
    \item DreamFusion
    \item LatentNeRF
\end{itemize}




\subsection{Ablations / Baselines} User study in addition to metrics?
\begin{itemize}
    \item no annealing
    \item single ReLU
    \item no attention grid
\end{itemize}
}



