\begin{figure}[h!] %
\centering %17v,20h
  \jsubfig{\includegraphics[height=1.9cm]{images/comparisons/image_editing/dog_white.png}}{}
\jsubfig{\includegraphics[height=1.9cm]{images/ablations/fcl/duck_nofcl.png}}{} 
\jsubfig{\includegraphics[height=1.9cm] {images/comparisons/image_editing/dog_white.png}}{} %\hspace{0.005pt}
\jsubfig{\includegraphics[height=1.9cm]{images/ablations/fcl/dog_nofcl_cropped.png}}{}
\rotatebox[origin=tc]{-90}{$\mathcal{L}_{reg3D}$ \whitetxt{xxxxx}}
\vspace{-10pt}
\\ %20v, 20h
\jsubfig{\includegraphics[height=1.93cm]{images/ablations/fcl/duck_ref_small.png}}{\footnotesize {Input}}
\jsubfig{\includegraphics[height=1.9cm]{images/ablations/fcl/duck_fcl.png}}{\footnotesize {''A wood carving of a duck"}} %\hspace{0.005pt} % [trim={left bottom right top},clip]
\jsubfig{\includegraphics[height=1.9cm] {images/ablations/fcl/dog_ref_cropped3.png}}{\footnotesize {Input}} %\hspace{0.005pt}
\jsubfig{\includegraphics[height=1.9cm]{images/ablations/fcl/dog_fcl_cropped.png}}{\footnotesize {''A dog wearing a christmas sweater"}}
\rotatebox[origin=tc]{-90}{$\mathcal{L}_{reg3D++}$ \whitetxt{xxxxxx}}%\hspace{0.005pt}
\vspace{3.2pt}
%\jsubfig{\includegraphics[height=1.9cm]{images/comparisons/image_editing/dog_postpix2pix_000062b.png}}{}

%\vspace{-12pt} 
\caption{\textbf{Regularizing RGB colors in addition to volumetric densities}. We show results obtained when using our default regularization objective $\mathcal{L}_{reg3D}$ (top-row) compared against results obtained when using $\mathcal{L}_{reg3D++}$- an alternative version of $\mathcal{L}_{reg3D}$ (bottom-row) in which we penalize the miscorrelation between both density and color features. These results show that regularizing both density and RGB can be limiting, especially when the edit requires a drastic change in color, such as changing the white fur of the dog into a vibrant christmas sweater. }

\label{fig:supp_fcl}
\end{figure}
