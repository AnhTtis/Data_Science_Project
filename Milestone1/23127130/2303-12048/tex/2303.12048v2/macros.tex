\newcommand{\jbox}[2]{
  \fbox{%
  	\begin{minipage}{#1}%
  		\hfill\vspace{#2}%
  	\end{minipage}%
  }}
\newbox\jsavebox
\newcommand{\jsubfig}[2]{%
	\sbox\jsavebox{#1}%
	\parbox[t]{\wd\jsavebox}{\centering\usebox\jsavebox\\#2}%
	}

%Our comments:
\newcommand{\esc}[1]{{\color{red}[\textbf{ES:} #1]}}
\newcommand{\gfc}[1]{{\color{blue}[\textbf{GF:} #1]}}
\newcommand{\hec}[1]{{\color{teal}[\textbf{HE:} #1]}}
\newcommand{\phc}[1]{{\color{violet}[\textbf{PH:} #1]}}
\newcommand{\todo}[1]{{\color{red}[\textbf{TODO:} #1]}}

\newcommand{\rev}[1]{{\color{red}#1}}
%Noticable new adds:
\newcommand{\es}[1]{{\color{red}#1}}
\newcommand{\gf}[1]{{\color{blue}#1}}
\newcommand{\he}[1]{{\color{teal}#1}}
\newcommand{\ph}[1]{{\color{violet}#1}}

\long\def\ignorethis#1{}

\usepackage[T1]{fontenc} 
\usepackage{iftex} % to access the \ifLuaTeX command

% NOTE: the following test is required for reasons explained
% in section 3 of The LuaTeX Reference Manual, available
% from CTAN: http://mirrors.ctan.org/systems/doc/luatex/luatex.pdf

\ifLuaTeX
\protected\def\pdfmapline {\pdfextension mapline }
\fi

% Use \pdfmapline to update pdfteX's font map---telling it
% about our custom font Delphine. The syntax used in \pdfmapline
% is described in "The pdfTeX user manual" which is available on
% CTAN: http://mirrors.ctan.org/systems/doc/pdftex/manual/pdftex-a.pdf

\newcommand{\whitetxt}[1]{{\color{white}#1}\normalfont}

% \pdfmapline{+delphine < Delphine.ttf <T1-WGL4.enc}
\pdfmapline{+delphine < Delphine.ttf <T1-WGL4.enc}

% Define a new command to use Delphine
\newcommand\delphinefont[1]{{\usefont{T1}{delphine}{m}{n} #1 }}
