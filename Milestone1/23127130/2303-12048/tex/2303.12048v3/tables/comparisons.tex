\definecolor{graytext}{RGB}{130,130,130}

\begin{table}[t]
\setlength{\tabcolsep}{3.0pt}
% \def\arraystretch{1.05}
 \def\arraystretch{1.1}
%  \footnotesize
\centering
\resizebox{0.75\linewidth}{!}{
\begin{tabular}{llccc}
\toprule
   & Method  & $\text{CLIP}_{Sim}\uparrow$ & $\text{CLIP}_{Dir} \uparrow$  \\ 
    \midrule 
    \multirow{4}{*}{\rotatebox[origin=c]{90}{Local}} & DFF+CN & \color{graytext}{0.34*} & \color{graytext}{0.05*} \\
    & Text2Mesh & \color{graytext}{0.36*} & \color{graytext}{0.08}* \\
    & Latent-NeRF (Sketch / Paint) & 0.32 / 0.31 & 0.01 / 0.01  \\
    & Ours &\textbf{0.36} & \textbf{0.07} \\
    \midrule 
    \multirow{4}{*}{\rotatebox[origin=c]{90}{Global}} & DFF+CN & \color{graytext}{0.32*} & \color{graytext}{0.01*}\\
    & Text2Mesh & \color{graytext}{0.34*} & \color{graytext}{0.03*}  \\
    & Latent-NeRF (Sketch / Paint) & 0.30 / 0.31 & 0.01 / 0.01 \\
    & Ours & \textbf{0.34} & \textbf{0.02} \\
\bottomrule
\end{tabular}
}
\caption{\textbf{Quantitative Evaluation.} We compare against the 3D object editing techniques Text2Mesh~\cite{michel2022text2mesh}, two variants of Latent-NeRF~\cite{metzer2022latent}: SketchShape (Sketch) and Latent-Paint (Paint) and DFF+CN~\cite{kobayashi2022decomposing,wang2022clip}, over local (top) and global (bottom) edits.  *Note that Text2Mesh and DFF+CN explicitly train to minimize a CLIP loss, and thus directly comparing them is uninformative over these metrics.
}
\label{tab:baseline-stats}
\end{table}
