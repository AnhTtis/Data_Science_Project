\begin{figure} %
\ignorethis{
\centering % [trim={left bottom right top},clip]
\jsubfig{\includegraphics[height=2.2cm,trim={0.0cm 3.0cm 1.0cm 3.0cm},clip]{images/limitations/pig.png}}{\footnotesize {``A horse with a pig tail"}}\hfill
\jsubfig{\includegraphics[height=2.2cm,trim={2.0cm 3.0cm 1.0cm 3.0cm},clip]{images/limitations/unicorn2.png}}{\footnotesize {``A pink unicorn"}}\hfill
\jsubfig{\includegraphics[height=2.2cm,trim={2.0cm 3.0cm 1.0cm 3.0cm},clip]{images/limitations/carpet2.png}}{\footnotesize {``A horse riding on a magic carpet"}}
}
\centering % [trim={left bottom right top},clip]
\jsubfig{\includegraphics[height=2.2cm,trim={1.6cm 2.4cm 0.8cm 2.4cm},clip]{images/limitations/pig.png}}{\footnotesize {``A horse with a pig tail"}}\hfill
\jsubfig{\includegraphics[height=2.2cm,trim={1.6cm 2.4cm 0.8cm 2.4cm},clip]{images/limitations/unicorn2.png}}{\footnotesize {``A pink unicorn"}}\hfill
\jsubfig{\includegraphics[height=2.2cm,trim={1.6cm 2.4cm 0.8cm 2.4cm},clip]{images/limitations/carpet2.png}}{\footnotesize {``A horse riding on a magic carpet"}}

\vspace{1.5pt}
\caption{\textbf{Limitations}. Above, we present several failure cases (when provided with rendered images of the uncolored mesh displayed in Figure \ref{fig:comparisons}, top row). These likely result from incorrect attribute binding (the horse's nose turning into a pig's nose), inconsistencies across views (two horns on the unicorn) or excessive regularization to the input object (carpet on the horse, not below).
}
\label{fig:limitations}
\end{figure}
