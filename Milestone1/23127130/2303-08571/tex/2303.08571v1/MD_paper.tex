\documentclass[aps,pra,twocolumn]{revtex4-2}
\usepackage{amsfonts}
\usepackage{amsmath}
\usepackage{mathrsfs}
\usepackage{amssymb}
\usepackage{graphicx}
%\usepackage{dcolumn}
\usepackage{bm}
\usepackage{color}
\usepackage{CJK}
\usepackage{dcolumn}
%\usepackage[braket]{qcircuit}
\usepackage{verbatim}
\usepackage{float}
\usepackage{booktabs}   %导入三线表需要的宏包
%\usepackage{graphics}   %导入表宽设置的宏包
\usepackage{array}    %导入设置每列宽度的宏包
\usepackage{multirow}  %导入合并单元格的宏包
\usepackage{threeparttable}  %导入表格注释的宏包
\usepackage[colorlinks=true,linkcolor=blue,anchorcolor=blue,citecolor=blue,urlcolor=blue]{hyperref}

\begin{document}
\preprint{APS/123-QED}

\title{Simulation of chemical reaction dynamics based on quantum computing}

\author{Qiankun Gong,$^1$ Qingmin Man,$^1$ Ye Li,$^1$ Menghan Dou,$^1$ Qingchun Wang,$^{2,3,*}$ Yu-Chun Wu,$^{2,3}$ Guo-Ping Guo$^{1,2,3,*}$}

\affiliation{$^1$Origin Quantum Computing Company Limited, Hefei 230026, People’s Republic of China\\
	$^2$Key Laboratory of Quantum Information, Chinese Academy of Sciences, School of Physics, University of Science and Technology of China, Hefei, Anhui, 230026, People’s Republic of China\\
	$^3$CAS Center For Excellence in Quantum Information and Quantum Physics, University of Science and Technology of China, Hefei, Anhui, 230026, People’s Republic of China }

\begin{abstract}
In recent years, the molecular energies of many chemical systems have been successfully simulated on quantum computers, which is regarded as the killer application of quantum computing. Compared to molecular energy, however, reaction dynamics play a more fundamental role in practical application, especially in catalytic activity, material design. Limited the capabilities of the noisy intermediate scale quantum (NISQ) devices, directly simulating the reaction dynamics and determining reaction pathway remain a challenge. Here, we employ the ab initio molecular dynamics based on the variational quantum eigensolver (VQE) algorithm to simulate reaction dynamics by extending correlated sampling approach. Moreover, we also use this approach to calculate Hessian matrix and evaluate computation resources. We numerically test the approach by simulating hydrogen exchange reaction and bimolecular nucleophilic substitubion S$\rm_N$2 reaction. The test results suggest that it is reliable to characterize the molecular structues, properties, and reactivities, which is another important expansion of the application of quantum computing.

\end{abstract}

\maketitle

\section{Introduction}
Reactive collision, which involves the chemical transformation, is regarded as the central phenomenon in physical chemistry\cite{guo2012quantum,zhang2016recent}. The reaction dynamics usually occurs on the femtosecond scale, involving the breakage of old bonds and the formation of new bonds. Therefore, the quantum mechanics is necessary to comprehensively describe and character the reactive event.  The H + H$_2$ reaction dynamics had been simulated by the full quantum mechanical calculation about forty years ago\cite{kuppermann1975quantum,karplus2014development}. And many chemical reactions have been studied using quantum mechanics methods with the increase of computer power. However, the reaction dynamics is still limited to small molecular systems due to the exponential increment of quantum mechanics in computational complexity\cite{guo2012quantum,fu2018ab}. Quantum computer, a new computing pattern utilizing the superposition and entanglement of qubits, has powerful processing speed, and is expected to bring new solutions to the electronic structure problems of traditional chemistry calculations\cite{mcardle2020quantum,cao2019quantum,barkoutsos2021quantum,ma2020quantum}. Therefore, some quantum algorithms have been developed to simulate the molecular propertories, such geometry optimization, response properties, vibrational structure and so on\cite{kassal2009quantum,delgado2021variational,o2019calculating,mitarai2020theory,ollitrault2020hardware,sawaya2021near,lotstedt2021calculation,cai2020quantum,christensen2019operators,oftelie2022computing,di2022quantum}.


According to quantum computer framework, some dynamics approaches based on quantum algorithms have been proposed to investigate the chemical process\cite{kassal2008polynomial,ollitrault2020nonadiabatic,ollitrault2021molecular,macdonell2021analog,langkabel2022quantum,fedorov2021ab,sokolov2021microcanonical}. For exanmple,  Kassal, Ollitrault and MacDonell simulated the real-time evolution of nonadiabatic chemical processes on real quantum devices or digital simulators respectively\cite{kassal2008polynomial,ollitrault2021molecular,macdonell2021analog}. However, the circuit depth and coherence time required by their approaches, as well as the subsequent error correction on quantum hardware, aren’t achievable in present noisy intermediate-scale quantum (NISQ) era. Consequently, Sokolov and Fedorov proposed the correlated sampling approach by combining ab initio molecular dynamics (AIMD) and variational quantum eigensolver (VQE) algorithm, and tested the framework by simulating the simplest vibraion of H$_2$ molecule\cite{fedorov2021ab}. In AIMD simulation, Born–Oppenheimer (or adiabatic) approximation is adopted, which assumes that the wave function of the whole system can be regarded as a simple product of the nuclear and electronic wave functions in molecular systems, and the crossing of potential energy surfaces (PES) is not considered, thus one can deal with the motion of electrons and nuclei independently\cite{karplus2014development,karplus1965exchange} Compared to full quantum molecular dynamics and classical molecular dynamics, the AIMD achieve a good balance between speed and accuracy. 

In this study, we first extend the correlated sampling approach to general cases, then simulate the H$^+$ + H$_2$ → H$_2$  + H$^+$  reaction dynamics by AIMD simulation based on VQE algorithm. In the extended approach, the measurement expectation can be reused if the Pauli string operator of force is the same as the ground-state energy, otherwise one need measure the Pauli string operator based on the same ground-state wave function, which minimums the quantum calculation resources in AIMD simulaiton. To test the approach in more complex system, the bimolecular nucleophilic substitubion S$\rm_N$2 (Cl$^-$ + CH$_3$Cl → ClCH$_3$ + Cl$^-$) reaction is also simulated. Moreover, we investigate the Hessian matrix based on the correlated sampling approach. The results indicate that the computation complexity of Hessian matrix is an approximate constant compared with the classical numerical methods. Final, we locate the transition structure of H$^+$ + H$_2$ reaction and S$\rm_N$2 reaction based on the Hessian matrix.

This paper is organized as follows. In the “Methods” section, we describe the calculated approaches of the ground-state energy, force, and transition state. In the “Result and discussion” section, we display the simulated results for the hydrogen exchange reaction dynamics and S$\rm_N$2 reaction.

\section{Methods}
\label{sec2}
 The VQE algorithm is a hybrid quantum–classical approach\cite{peruzzo2014variational,o2016scalable,kandala2017hardware,grimsley2019adaptive,google2020hartree}, in which the preparation and measurement of a parametrized ansatz are carried out on the quantum computer, and these parameters are iteratively optimized on a classical digital computer according to the Rayleigh–Ritz variational principle
\begin{equation}
	\label{eq1}
	E \leq\left\langle\Psi(\boldsymbol{\theta})\left|\hat{\mathrm{H}}_{e l}(\boldsymbol{R})\right| \Psi(\boldsymbol{\theta})\right\rangle,
\end{equation}
assuming the trial state is normalized, $\boldsymbol{\theta}$ is the variational parameters, $E$ is the ground-state energy, $\boldsymbol{R}$ is the coordinates of molecule. The second-quantized electronic Hamiltonian is
\begin{equation}
	\label{eq2}
	\hat{H}_{\mathrm{e} l}=\sum_{\mathrm{pq}} h_{p q} a_p^{\dagger} a_q+\frac{1}{2} \sum_{\mathrm{pqrs}} h_{p q r s} a_p^{\dagger} a_q^{\dagger} a_r a_s,
\end{equation}	
 where $h_{p q}$ and $h_{p q r s}$ is the singles- and doubles- electron integrals computed by classical computer respectively. $a^{\dagger}$ and $a$ are the creation and annihilation operators for corresponding spin orbitals.
 
 
 To simulate the second-quantized Hamiltonian, the creation and annihilation operators need to be encoded into qubits. The Jordan-Wigner (JW) transformation, parity transformation and Bravyi-Kitaev (BK) transformation\cite{bravyi2002fermionic} are the most popular encoding approaches now. Among them, JW transformation is a basic and widely used encoding, and the orbital occupation status corresponds to the qubit states. In this study, the JW transformation is used to produce the Pauli string operator formation of electronic Hamiltonian
 \begin{equation}
 	\label{eq3}
    \hat{H}_{\mathrm{el}}=\sum h_\alpha\left(\prod_k^N \hat{p}_k\right)_\alpha,
\end{equation}
 $h_\alpha$ relies on the $h_{p q}$ and $h_{p q r s}$, $\hat{p}_k$ is the Pauli operator $\{X, Y, Z, I\}$, $N$ is the number of qubit. Then the ground-state energy after optimizer
\begin{equation}
	\label{eq4}
	\mathrm{E}=\sum h_{\boldsymbol{\alpha}}\left\langle\Psi(\boldsymbol{\theta})\left|\left(\prod_k^N \hat{p}_k\right)_{\boldsymbol{\alpha}}\right| \Psi(\boldsymbol{\theta})\right\rangle=\sum h_{\boldsymbol{\alpha}} P_{\boldsymbol{\alpha}},
\end{equation}
$P_{\boldsymbol{\alpha}}$ is the measurement expectation of Pauli string operator.


 The ansatz method is also a significant concept in VQE algorithm. A good ansatz could decrease the depth of the circuit and improve the calculation precision. The commond ansatz methods include unitary coupled cluster (UCC) ansatz\cite{yaris1964linked,yaris1965cluster,taube2006new,peruzzo2014variational}, symmetry-preserved (SP)\cite{gard2020efficient} ansatz and hardware-efficient ansatz\cite{kandala2017hardware}. Benefit from the success of classical coupled cluster method and its variants, the UCC ansatz is the well-known ansatz in quantum computing field. The advantages of symmetry-preserved ansatz is using 2-qubit block as basis element and suitable to particular molecular system. The hardware-efficient ansatz includes many layers and is suitable to run on NISQ hardware. Here we make use of the unitary coupled cluster with singles and doubles (UCCSD) ansatz method
 \begin{equation}
 	\label{eq5}
 	|\Psi(\boldsymbol{\theta})\rangle=e^{\hat{T}(\theta)-\hat{T}(\theta)^{\dagger}}|\Psi\rangle_{H F},
\end{equation}
$\hat{T}(\theta)^{\dagger}$ is the excitation operator and the initial state is Hartree-Fock state.


The ab initio molecular dynamics simulations (AIMD) can be performed to directly simulate the reaction dynamics. In AIMD simulations, the Born-Oppenheimer approximation is used to separate the motions of the nucleus and electron. To reduce error for integrating the motion equations, the fairly small time steps is necessary and the total energy conservation is often applied to evaluate the simulation quality. Since the velocity-Verlet lgorithm can keep the total energy conservation over long time simulation\cite{verlet1967computer,swope1982computer}, the positions and velocities of nuclei are updated at per step by 
\begin{equation}
	\label{eq6}
r(t+\Delta t)=r(t)+v(t) \Delta t+\frac{1}{2 m} F(t) \Delta t^2,
\end{equation}
\begin{equation}
	\label{eq7}
	v(t+\Delta t)=v(t)+\frac{1}{2 m}(F(t)+F(t+\Delta t)) \Delta t,
\end{equation}
$r$, $v$, $t$, $\Delta t$, $m$, $F$ are coordinate, velocity, time, time step, mass and force respectively. The initial velocities and coordinates can be obtained according to actual condition. Therefore, the calculation of forces is the key of AIMD simulations.

In the Hellmann-Feynman theorem, the force of atomic nucleus along with the $j$ direction equals to
\begin{equation}
	\label{eq8}
F_j=-\left\langle\Psi(\boldsymbol{\theta})\left|\frac{\partial \hat{H}_{e l}(\boldsymbol{R})}{\partial \boldsymbol{R}_j}\right| \Psi(\boldsymbol{\theta})\right\rangle.
\end{equation}
Here we employ the central finite difference approach to approximate the force
\begin{equation}
	\label{eq9}
	F_j=-\frac{\left\langle\Psi(\boldsymbol{\theta}) \left| \Delta H, .\right|\Psi(\boldsymbol{\theta})\right\rangle}{2 \times \Delta d}
\end{equation}
\begin{equation}
	\label{eq10}
	\Delta H=\sum_{\gamma} h_\gamma\left(\prod_k^N \hat{p}_k\right)_\gamma-\sum_\beta h_\beta\left(\prod_k^N \hat{p}_k\right)_\beta
\end{equation}

%%%%%%%%%%%%%%%%%%%%%%%%%%%%%%%%%%%%%%%%%%%%%%%%%%%%%%%%%%%%%%%
\begin{figure*}[]
	\centering
	\includegraphics[width=13cm,height=8cm]{Fig1.png}
	\caption{The process of H$^+$ closed to the fixed H$_2$ molecule from 5.0 Å to 0.45 Å. (a) ground state energies vs distances, (b) forces vs distances, (c) energy deviations $\left|E_{VQE} - E_{CCSD}\right|$ vs distances, (d) force deviations $\left|F_{VQE} - F_{CCSD}\right|$ vs distances.}
	\label{fig1}
\end{figure*}
%%%%%%%%%%%%%%%%%%%%%%%%%%%%%%%%%%%%%%%%%%%%%%%%%%%%%%%%%%%%%%%

If $\left(\prod_k^N \hat{p}_k\right)_\gamma$ or $\left(\prod_k^N \hat{p}_k\right)_\beta$ is the same as the $\left(\prod_k^N \hat{p}_k\right)_\alpha$ , the measure probability $P_{\boldsymbol{\alpha}}$ can be reused. Otherwise, additional measurements are necessary based on the same ground-state wave function. In extended correlated sampling approach, the force can be obtained 
\begin{equation}
	\label{eq11}
     F_j=-\frac{\Delta E_j^{\text {reuse }}+\Delta E_j^{\text {extra }}}{2 \times \Delta d},
\end{equation}
\begin{equation}
	\label{eq12}	
    \Delta E_j^{\text {reuse }}=\sum_{\gamma, \alpha} h_\gamma P_\alpha-\sum_{\beta, \alpha} h_\beta P_\alpha,
\end{equation}
\begin{equation}
	\label{eq13}
	\Delta E_j^{\text {extra }}= \left\langle\Psi(\boldsymbol{\theta}) \left| \hat{O} \right| \Psi(\boldsymbol{\theta})\right\rangle.
\end{equation}

\begin{equation}
	\label{eq14}
	\hat{O}= \sum_{\gamma \notin \alpha} h_\gamma\left(\prod_k^N \hat{p}_k\right)_\gamma - \sum_{\beta \notin \alpha} h_\beta\left(\prod_k^N \hat{p}_k\right)_\beta\
\end{equation}

 Transition state is the maximum energy point along the reaction pathway that connects the two minimum points. It is a first-order saddle points on potential energy surface, and there is one and only one image frequency. Here, the normal mode analysis is used to evaluate the vibration frequency. In normal mode analysis, the harmonic approximation is used to characterize the potential energy surface, and the coupled and collective motions of atoms can be sperated into individually motions.
 
In present work, searching transition state is based on Newton-Raphson method
\begin{equation}
	\label{eq15}
    \boldsymbol{R}_{k+1}=\boldsymbol{R}_{k}-{H}_{k}^{-1} \nabla {E}_k,
\end{equation}
${H}_{k}^{-1}$, ${E}_k$ and $\boldsymbol{R}_{k}$ are the inverse of the Hessian matrix, gradient, coordinates at $k$th iteration respectively. In general, the Hessian matrix must have only one negative eigenvalue at each iteration to ensure the TS optimization toward the desired direction. As a result, controlling step size is necessary. Here, we use the simplest way to control the step size, namely, scaling the step size by factor 0.5 when the only one negative eigenvalue is vanished. 

The Hessian matrix involves the second order energy derivative
\begin{equation}
	\label{eq16}
	H_{j,i}=\frac{\partial^2 {E}(\boldsymbol{R})}{\partial \boldsymbol{R}_{\mathrm{j}} \partial \boldsymbol{R}_{i}}.
\end{equation}
Using central finite difference approach
\begin{equation}
	\label{eq17}
    S_{j, i}=\frac{E_1+E_2-E_3-E_4}{4 \times \Delta d^2},
\end{equation}
where $E_1$$=$$E\left(\boldsymbol{R}+\Delta d \boldsymbol{e}_{i}, \boldsymbol{R}+\Delta d \boldsymbol{e}_{j}\right)$, $E_2$$=$$E\left(\boldsymbol{R}-\Delta d \boldsymbol{e}_{i}, \boldsymbol{R}-\Delta d \boldsymbol{e}_{j}\right)$, $E_3=E\left(\boldsymbol{R}-\Delta d \boldsymbol{e}_{i}, \boldsymbol{R}+\Delta d \boldsymbol{e}_{j}\right)$, $E_4=E\left(\boldsymbol{R}+\Delta d \boldsymbol{e}_{i}, \boldsymbol{R}-\Delta d \boldsymbol{e}_{j}\right)$. When $i$ is equal to $j$
\begin{equation}
	\label{eq18}
	{H}_{{j}, {i}}=\frac{{E}\left(\boldsymbol{R}+2 \Delta {d} \boldsymbol{e}_{i}\right) - E\left(\boldsymbol{R}-2 \Delta {d} \boldsymbol{e}_{i}\right)}{4 \times \Delta {d}^2}.
\end{equation}


Compared to ground-state energy calculation, the Hessian matrix need optimizer $18N^2$ ground-state ware functions for $N$ atoms system, which is unrealistic for NISQ devices. Inspired by the calculation of forces, the second order energy derivative can be approximated
\begin{equation}
	\label{eq19}
    {H}_{{j}, {i}}=\left\langle\Psi(\boldsymbol{R})\left|\frac{\partial^2 \widehat{{H}}(\boldsymbol{R})}{\partial \boldsymbol{R}_{{j}} \partial \boldsymbol{R}_{{i}}}\right| \Psi(\boldsymbol{R})\right\rangle.
\end{equation}
It ignores the wave function derivatives with nuclear coordinates. According to the extended correlated sampling approach, the calculation of Hessian matrix can apply the same ground-state ware function and reutilize measure expectation as done in force calculation.

In this study, all calculations are performed using the ChemiQ software developed by our group\cite{wang2021chemiq}. To evaluate the precision, the classical CCSD method is also performed to obtain ground state energy, force and vibration frequency by PySCF software\cite{sun2018pyscf,sun2020recent}. The STO-3G basis and Jordan-Wigner transformation are used for all calculations. The optimization processes of VQE algorithm are done by SLSQP optimizer. The S$\rm_N$2 reaction adopt the HOMO/LUMO active space (2, 2) with UCCD ansatz to generate fewer qubits and lower circuit. The AIMD time step is 0.2 fs and the difference step is 1.0×10$^{-3}$ Å. 

\section{Results and Discussion}
\label{sec3}
Compared to molecular energy, chemical reactions play a more fundamental role in practical application, which refers to the molecular structures, properties and reactivities. The H$^+$ + H$_2$ → H$_2$ + H$^+$ reaction and S$\rm_N$2 (Cl$^-$ + CH$_3$Cl → ClCH$_3$ + Cl$^-$) reaction are the most basic reactions in computational chemistry, which are often used as a model to test new method. Though the hydrogen exchange reaction only includes three atoms, but it envolves the breakage of old bond and the formation of new bond. Moreover, it is significant for astrophysicist to understand the thermodynamical evolution of early universe. Up to now, there are still some research groups investigating the exchange reaction. For instance, Tomas et al. simulated theoretically the H$^+$ + H$_2$  exchange reaction by different approaches\cite{gonzalez2006detailed}. They found that the reaction dynamics is governed by an insertion mechanism. Recently, the rate constants of H$^+$ + H$_2$  reaction was also studied using a statistical quantum method between T = 5 K and 3000 K\cite{gonzalez2021rate}, which is related to the galaxy formation and evolution.

Before simulating the reaction dyanmics of hydrogen exchange reaction, we performed an accurate calculation for H$_3 ^+$ system by CCSD method to evalute the quality of ground-state energies and forces described in Section \ref{sec2}. The potential energy surfaces of ground-state energies and energy deviations obtained by VQE algorithm and CCSD method for H$^+$ closed to H$_2$ molecule are shown in Fig. 1(a) and (c). We can see that the  potential energy surface discribled by VQE algorithm is excellent agreement with the CCSD method, and the maximal deviation occurred at reaction region is about 0.001 mHa far less than the chemical accuracy 1.6 mHa. Fig. 1(b) and (d) displays the forces of H$^+$ and corresponded absolute deviations. Even though the force deviation curve becomes rough and fluctuation probably due to the numerical difference, the maximal error is only about 0.025 mHa/a.u which does not affect the AIMD accuray. Therefore, these results provide the basis for directly simulating the reaction dynamics of hydrogen exchange reaction.

%%%%%%%%%%%%%%%%%%%%%%%%%%%%%%%%%%%%%%%%%%%%%%%%%%%%%%%%%%%%%%%
\begin{figure}[h]
	\centering
	\includegraphics[width=8.5cm,height=7.5cm]{Fig2.png}
	\caption{H$^+$ + H$_2$ colinear reactive collision in the gas phase. (a) Schematic diagram of reaction process, (b) The reaction trajectory obtained by AIMD simulations within 60 fs.}
	\label{fig2}
\end{figure}
%%%%%%%%%%%%%%%%%%%%%%%%%%%%%%%%%%%%%%%%%%%%%%%%%%%%%%%%%%%%%%%
\begin{table*}[] %[!ht]表格在文本中放置的位置参数(努力放在当前位置,实在放不下,将放在下一页的顶部)
	\centering  %表格整体居中
	\begin{threeparttable}
		\caption{The optimized structures and corresponded frequencies}  %表格标题
		\begin{tabular} {p{1.7cm}<{\centering} p{1.7cm}<{\centering} p{1.7cm}<{\centering} p{1.7cm}<{\centering} p{2.1cm}<{\centering} p{2.1cm}<{\centering} p{2.1cm}<{\centering}}
			%{ccccccc}% 其中,tabular是表格内容的环境;c表示centering,即文本格式居中;c的个数代表列的个数
			\toprule %[2pt]设置线宽     
			system   				&${l_0}$(Å)$^a$ & $l$(Å)$^{b1}$ & $l$(Å)$^{b2}$ & Freq(cm$^{-1}$)$^{c1}$ &Freq(cm$^{-1}$)$^{c2}$ &Freq(cm$^{-1}$)$^{c3}$ \\%换行
			\midrule %[2pt]  
			H$_2$   			 		& 1.0      & 0.735 & 0.735   & 5001.9   &  5000.2    &  5201.3  \\
			\midrule
			LiH      			 		& 1.15     & 1.547 & 1.548   & 1680.7   &  1683.3    &  1730.5  \\
			\midrule
			\multirow{3}{*}{H$_3^+$}  	& 1.208    & 0.986 & 0.986   & 3445.9   &  3447.3    &  3526.2  \\
			\multirow{3}{*}{}           & 1.603    & 0.986 & 0.986   & 2116.3   &  2122.3    &  2166.7  \\
			\multirow{3}{*}{}           & 2.566    & 0.986 & 0.986   & 2116.3   &  2115.9    &  2159.2   \\
			\bottomrule %[2pt]     
		\end{tabular}
		$^a$The initial bond length.
		$^{b1}$The optimized bond length by CCSD.
		$^{b2}$The optimized bond length by VQE.
		$^{c1}$The frequency calculated by CCSD.
		$^{c2}$The frequency calculated by VQE.
		$^{c3}$The frequency calculated by VQE with approximated Hessian matrix.
	\end{threeparttable}
\end{table*}

%%%%%%%%%%%%%%%%%%%%%%%%%%%%%%%%%%%%%%%%%%%%%%%%%%%%%%%%%%%%%%%
\begin{figure*}[]
	\centering
	\includegraphics[width=13cm,height=8cm]{Fig3.png}
	\caption{The process of Cl$^-$ closed to CH$_3$Cl molecule from 6.5 Å to 1.25 Å. (a) ground-state energies vs distances, (b) forces vs distances, (c) energy deviations $\left|E_{VQE} - E_{CCSD}\right|$ vs distances, (d) force deviations $\left|F_{VQE} - F_{CCSD}\right|$ vs distances.}
	\label{fig3}
\end{figure*}
%%%%%%%%%%%%%%%%%%%%%%%%%%%%%%%%%%%%%%%%%%%%%%%%%%%%%%%%%%%%%%%

Since the exchange reaction only includes three particles, it can be described by three relative distances. Fig. 2 shows the H$^+$ + H$_2$ colinear reactive collision in the gas phase by AIMD simulations based on VQE algorithm within 60 fs. The reactive process refers to the H$_A ^+$ ion colliding the hydrogen molecule and causes that a new H$_2$ molecule forms and the H$_C ^+$ ion escapes. The initial distance between H$_A ^+$ and H$_2$ molecule is 5.0 Å and corresponded velocity is 0.125 Å/fs. In the starting stage, the hydrogen molecule does the periodic vibration around the equilibrum bond length and the distance between the H$_A ^+$ ion and H$_B ^+$ atom decreases uniformly. Next, the three particles come into being a complex molecule structure in the chemical reaction region. Final, a new molceule H$_A$-H$_B$ forms by exchange the energy and momentum. 

Usually, the fairly small time step is necessary for AIMD simulation to reduce the integrating error and the total energy conservation can be used to assess the MD trajectory quality. Here the maximal total energy drifty is observed about 0.92 mHa which happends around 30 fs. Our results demonstate that the reaction dynamics based on the quantum computing can be simulated accurately. Though the total energy drifty is much less than the chemical accuracy, it suggests that the chemical reaction region especially around the trasition state puts forward higher requirements for quantum computing accuracy compared to the equilibrum structure. 


Vibration frequency is an another important molecular propertory, which can be used to calculate the vibrational entropy, determine the transition structure. Usually, the local extreme point of potential energy surface can be approximated by harmonic potential funciton, and the atomic coupled motions can be sperated into individual motions by normal mode analysis. The vibration spectrum involves the calculation of force constant matrix (mass-weithted Hessian matrix), which is often complication and difficult. Here, we numerically evaluate the vibration frequencies of H$_2$, LiH and H$_3^{+}$ based on VQE algorithm and normal mode analysis. Firstly, the molecules are optimized to the stationary point from random strating structures by gradient descent method. Next, the vibration frequencies are performed by Hessian matrix or approximated Hessian matrix. Moreover, the CCSD method is applied to obtain the optimized structures and exact frequencies. 


%%%%%%%%%%%%%%%%%%%%%%%%%%%%%%%%%%%%%%%%%%%%%%%%%%%%%%%%%%%%%%%
\begin{figure*}[]
	\centering
	\includegraphics[width=13cm,height=8cm]{Fig4.png}
	\caption{View of the S$_N$2 reaction process from MD trajectory.}
	\label{fig4}
\end{figure*}
%%%%%%%%%%%%%%%%%%%%%%%%%%%%%%%%%%%%%%%%%%%%%%%%%%%%%%%%%%%%%%%

Table 1 gives the optimized bond lengeths and corresponded vibration frequencies. The quilibrum bond lengths obtained by VQE is quite consistent with the classical CCSD results, which is the basis for accurately calculating the frequencies.  Since H$_2$ and LiH are both diatomic molecules, there is only one frequency. The frequencies based on  Hessian matrix by VQE algorithm of H$_2$ and LiH molecules are 5000.2 cm$^{-1}$ and 1683.3 cm$^{-1}$ respectively, which is close to the exact reasults 5001.9 cm$^{-1}$ and 1680.7 cm$^{-1}$. For H$_3^{+}$ molecule, there are three vibration frequencies and the average difference between VQE and CCSD is about 2.6 cm$^{-1}$, it approximates the spectrum precision. Even though the frequency can be high accurately calculated by this way, the Hessian matrix need optimize 18N$^2$ ground-state wave functions for N atomic system, which is infeasible for current hardwares. In this perspective, we also investigate the precision of frequency based on the same ground-state wave function and reused the measure probability inspired by the calculation of force. Although there are obvious errors by approximated Hessian matrix, the relative precision can reach 96\% - 98\% for the three systems. Therefore, it can be applied to research the vibration frequency and search the transition state.


Transition state (TS) is a first saddle point on potential energy surface and corresponds the maximum energy along the reaction coordinates\cite{schlegel2003exploring,schlegel2011geometry}. Moreover, TS is the basis for calculating the activation energy and determining the reaction path. Up to now, some endeavors have been done to expand the VQE algorithm or its variants to determine the reaction path\cite{kanno2020quantum,sarkar2022modular,azad2022quantum,lim2022quantum,li2022toward}. But almost all studies perform the potential energy surface scanning along the predefined paths obtained from classical chemistry software. It is obvious that the transition structure is unknow in advance universally. Therefore, how to search the TS and determine the reaction path based on quantum algorithm is still an open problem.

To find the transition state of H$^+$ + H$_2$ → H$_2$ + H$^+$ reaction, the starting sturcture should be closed to the transition structure or within quadratic region. Here, we uniformly fetch five points in reaction region to generate the starting structure from 25fs to 33fs. We observe that there is only one image frequency -4316 cm$^{-1}$ and -4724 cm$^{-1}$ at 29 fs and 31 fs. Consequently, the corresponded structures can be adopt as the initial structure to search the transition state. Though searching from different structures, they converge to the same point with with image frequency -974 cm$^{-1}$ and the bond length R$_{AB}$, R$_{BC}$, R$_{AC}$ are 0.876 Å, 0.876 Å and 1.752 Å respectively. To verify our results, the energy gradient and frequency of the H$_3^+$ transition state are also computed by CCSD method. The CCSD data illustrates that the maximal gradient is only 0.014 mHa/a.u with one image frequency -993 cm$^{-1}$. It suggests that our approach successfully find the transition state of H$^+$ + H$_2$ raection.


 The S$\rm_N$2 reaction is a fundamentail and significant chemical reaction, has been widely applied in field of drug discovery and organic synthesis\cite{mikosch2008imaging}. Thus, studing the reaction kinetics in the atomic level is very impornt for understanding the S$\rm_N$2 microscopic processes. Here, the HOMO/LUMO active space is adopted to generate fewer qubits and lower circuit depths. Fig. 3 gives the variation of ground-state energies and forces when Cl$^-$ closes to CH$_3$Cl molecule from 6.5 Å to 1.25 Å. We observe that the potential energy surface and forces are good consistent with the CCSD data with low errors 0.17 mHa and 0.33 mHa/a.u respectively. Similarly, the errors will increase when Cl$^-$ is near the CH$_3$Cl at chemical reaction region. In summary, it is safe to directly simulate the process of nucleophile chloride ion (Cl$^-$) attacking electrophile chloromethane (CH$_3$Cl) by AIMD simulation.

Fig. 4 displays a typical reaction process for the S$_N$2 reaction with a initial velocity 0.04 Å/fs for Cl$^-$ along the x axis. We can observe the famous Walden inversion mechanism from the MD trajectory. The nucleophilic reagent Cl$^-$ attacks the center carbon atom from the back, then the configuration between carbon atom and three hydrogen atoms changes from umbrella to plane at 109.8 fs, the bond between nucleophilic reagent and carton atom begin to form and leaving group escapes in the end. To proceed the initial guess structure, we first calculate the frequencies at 109.8 fs , and then uniformly change the distances between Cl and C by fixing CH$_3$. Finally, the transition state is found with an image frequency -179 cm$^{-1}$. We adopt the CCSD method to verify the transiton structure. The result shows that it is only one image frequency -301 cm$^{-1}$ and the maximal energy gradient is about 0.0017 Ha/a.u. It suggests that our approach find the approximate transition state of Cl$^-$ + CH$_3$Cl → ClCH$_3$ + Cl$^-$ reaction.



\section{Conclusion}
\label{sec4}
Quantum computer is regarded as a promising tool to solve the electronic structure problem. However limited the capabilities of the NISQ devices, directly simulating the reaction progress remains a challenge. Here we extend the correlated sampling approach based on VQE algorithm to describle the reaction dynamics by AIMD simulation.

 Before simulating the chemical reactions H$^+$ + H$_2$ → H$_2$ + H$^+$ and Cl$^-$ + CH$_3$Cl → ClCH$_3$ + Cl$^-$, we first calculate the ground-state energies and forces. The results demonstrate that our approach can accurately describle the potential energy surface and forces in chemical reaction region. Next, we view the hydrogen exchange process for H$^+$ + H$_2$ reaction and Walden inversion mechanism for Cl$^-$ + CH$_3$Cl reaction from AIMD trajectories. Besides, we also investigate the Hessian matrix based on correlated sampling approach to calculate the vibration frequency and search the transition structure. The test results suggest that our approach can accurately evalate the molecular properties,  which is another important expansion of the application of quantum algorithm.

\bibliographystyle{elsarticle-num-names}
\bibliography{Reference.bib}

\end{document}
