\documentclass[english,aps, pre, superscriptaddress, citeautoscript, reprint, onecolumn, 10pt, tightenlines,longbibliography]{revtex4-1}
\usepackage{amsmath, amssymb, graphicx, bm}
\usepackage{soul}
\usepackage{gensymb}
\usepackage{float}
\usepackage{xr}


\makeatletter
\usepackage[colorlinks=true,linkcolor=MidnightBlue,urlcolor=black,citecolor=MidnightBlue,anchorcolor=MidnightBlue]{hyperref}\usepackage[dvipsnames]{xcolor}
\newcommand*{\addFileDependency}[1]{
  \typeout{(#1)}
  \@addtofilelist{#1}
  \IfFileExists{#1}{}{\typeout{No file #1.}}
}
\makeatother

\newcommand*{\myexternaldocument}[1]{
    \externaldocument{#1}
    \addFileDependency{#1.tex}
    \addFileDependency{#1.aux}
}
%%% END HELPER CODE

% put all the external documents here!
% \myexternaldocument{SI}
\begin{document}

\title{Emergent dynamics due to chemo-hydrodynamic self-interactions in active polymers}

\author{Manoj Kumar}
\email{manojk@ncbs.res.in}

\affiliation{Simons Centre for the Study of Living Machines, National Centre for
Biological Sciences, Tata Institute of Fundamental Research, Bangalore,
India}
\author{Aniruddh Murali}
\affiliation{Simons Centre for the Study of Living Machines, National Centre for
Biological Sciences, Tata Institute of Fundamental Research, Bangalore,
India}

\author{Arvin Gopal Subramaniam}
\affiliation{Department of Physics, Indian Institute of Technology, Chennai, India}

\author{Rajesh Singh}
\email{rsingh@physics.iitm.ac.in}
\affiliation{Department of Physics, Indian Institute of Technology, Chennai, India}

\author{Shashi Thutupalli}
\email{shashi@ncbs.res.in}
\affiliation{Simons Centre for the Study of Living Machines, National Centre for
Biological Sciences, Tata Institute of Fundamental Research, Bangalore,
India}
\affiliation{International Centre for Theoretical Sciences, Tata Institute of Fundamental
Research, Bangalore, India}

\begin{abstract}
We create freely-jointed active polymers using self-propelled droplets as monomeric units, and show that self-shaping chemo-hydrodynamic interactions within the polymer result in novel dissipative structures and steady-states. Our experiments reveal that the interactions between the monomeric droplets give rise to ballistic propulsion of the active polymers and is associated with rigidity and stereotypical shapes of the polymers. These traits, quantified by the curvatures and speeds of the active polymers, vary systematically with the chain length. Using simulations of a minimal model, we establish that the emergent propulsion and rigidity are a generic consequence of quasi two-dimensional confinement and auto-repulsive chemical interactions between the freely jointed active droplets. Finally, we tune the interplay between the chemical and hydrodynamic fields to experimentally demonstrate oscillatory dynamics of the rigid polymer propulsion. Altogether, our work highlights the possible first steps towards synthetic self-morphic active matter.
\end{abstract}

\maketitle

Understanding and tuning the orchestration between mechanics and other guiding fields, especially self-generated ones such as chemistry, is a fundamental challenge in the field of active matter~\cite{ramaswamy2010mechanics, marchetti2013hydrodynamics, cates2015motility, bechinger2016active, saha2014clusters}, both in the context of living systems~\cite{howard2001mechanics,berg1975chemotaxis,gross2017active} and synthetic non-living emulations~\cite{howse2007self,paxton2004catalytic,thutupalli2011,hokmabad2022chemotactic,meredith2020predator,ren2020programmed}. While there has been an increasing focus of experimental work on the emergent consequences of chemo-mechanical self-coupling in synthetic active matter systems comprised of point-like unit active particles~\cite{aubret2018targeted,hokmabad2022chemotactic, meredith2020predator}, the experimental realisation and study of such dynamics in extended active objects such as molecules~\cite{lowen2018active}, polymers~\cite{jayaraman2012autonomous,laskar2015brownian, winkler2017active, prathyusha2022emergent} and sheets~\cite{manna2022harnessing} remains elusive. \par 

Several approaches have attempted the construction of active assemblies -- especially linear chains -- using monomer units such as colloids, to serve as idealised experimental realizations of multi-body ``polymeric'' systems. It must be noted that the assemblies have mostly been achieved and kept out of equilibrium using \textit{external} electric fields, magnetic fields, or light energy~\cite{zhang2016natural,vutukuri2017rational,nishiguchi2018flagellar,snezhko2011magnetic,dreyfus2005microscopic,stenhammar2016light}. While these systems are out-of-equilibrium, the dynamics is predominantly due to external fields, in contrast to active and living systems in which drive is at the scale of the individual units. This difference in the external versus internal driving is crucial -- since the control fields are imposed externally, the interactions between the monomers are not self-responsive, leading to a fundamental difference in the emergent dynamics~\cite{ramaswamy2010mechanics,marchetti2013hydrodynamics,gross2017active}. Chemically linked catalyst-coated colloids, that are actuated locally, have also been used to construct linear polymers~\cite{biswas2017linking,biswas2021rigidity} --- however, the ``monomeric'' units are neither self-propelled nor are orientationally free. Emulsion droplets, due to their internal fluidity offer a promising monomeric unit --- in the past, passive linear assemblies that are flexible or freely jointed have been made using emulsion droplets using sticky DNA linkers~\cite{zhang2017sequential,mcmullen2018freely,mcmullen2022self,zhang2018multivalent}. 

In this study, we construct polymeric chains of freely jointed self-propelling active droplets. The propulsion mechanism of the monomers, \emph{i.e.} droplets, creates external chemical and hydrodynamic fields, causing self-shaping interactions within the polymer. We measure these chemical and hydrodynamic fields and then quantify, as a function of the polymer lengths, the emergent rigidity, shapes and ballistic self-propulsion of the active polymers in quasi two-dimensional confinement. We show that a model which includes only chemical interactions between the monomers captures all these activity-driven conformational features quantitatively, thereby establishing the minimal requirements for the emergent self-organization: (i) the monomer droplets of the chain propel due to (gradients of) their self-generated chemical field and (ii) the structures are confined to move in quasi two-dimensions. Finally, by tuning the coupling between the chemical and hydrodynamic fields, we demonstrate oscillatory gaits of the polymers. Altogether, we envision these as first steps towards a kind of synthetic active matter capable of emergent self-morphic dynamics.\par

\subsection{Freely jointed active polymers of self-propelling emulsion droplets}  

The self-propelled droplets that we use as monomers are comprised of oil, slowly dissolving into an external supramicellar aqueous solution of ionic surfactants~\cite{peddireddy2012,peddireddy2014liquid}. The dissolution of the droplets results in the spontaneous development of self-sustaining gradients of surfactant coverage around the droplets. These gradients give rise to Marangoni stresses causing the droplets to propel~\cite{herminghaus2014interfacial} (Fig.~\ref{fig1}\textbf{A}). As such, these self-propelled droplets may be viewed as a physico-chemical realization of the so-called squirmer model for microswimmers, a sphere with a prescribed surface slip velocity that exchanges momentum with the fluid in which it is immersed~\cite{blake1971spherical,ehlers1996synechococcus,ishikawa2006hydrodynamic}. The qualitative and quantitative nature of this slip velocity governs the near- and far-field hydrodynamic flow perturbation around the squirmer and eventually its self-propulsion. \par

The hydrodynamic flow fields around the oil droplet microswimmer (Fig.~\ref{fig1}\textbf{B}, Supplementary Video~SV1) depend in tunable ways on geometric and, as we also show later, chemical conditions~\cite{peddireddy2014liquid,hokmabad2021emergence,dwivedi2021solute,ramesh2022interfacial}. In addition to the hydrodynamic flows, these microswimmers leave a trail of chemical fields~\cite{hokmabad2022chemotactic} comprised of oil-filled surfactant micelles formed by the transfer of oil molecules into empty surfactant micelles, which can be visualised using an oil-soluble fluorescent dye (Fig.~\ref{fig1}\textbf{C}, Supplementary Video~SV1). The trail persists because the oil-filled micelles take longer to diffuse than the surrounding oil-free surfactant micelles and causes a remodeling of the environment around the droplets, leading to time delayed negative chemotatic self-interactions~\cite{hokmabad2022chemotactic}. \par

To create tethered assemblages, we use biotin-streptavidin chemistry to form freely jointed chains of the active droplets (Fig.~\ref{fig1}\textbf{D},~\ref{Materials and Methods}). Briefly, we use a hybrid (surfactant-lipid) monolayer of surfactant and biotinylated lipids that coat the droplets spontaneously (SI Fig.~S1\textbf{A, B}). Using these biotinylated lipids in conjunction with the well known specific interaction of biotin with streptavidin, we ``polymerise'' the droplets to form linear chains via controlled assembly and incubation (details in~\ref{Materials and Methods} and Supplementary Information, SI Fig.~S1\textbf{B}). This assembly process robustly results in linear assemblies of 5CB oil emulsion droplets such as dimers, trimers, and so on up to decamers (Fig.~\ref{fig1}\textbf{E}). Our protocol rarely yields chains longer than $N=10$ without branching, and longest linear chains we obtain consist of $N=13$ droplets (Fig.~\ref{fig1}\textbf{E} and SI Fig.~S1\textbf{D}). The droplets comprising the polymers are freely-jointed, as seen from the distribution of the internal bond angles (Fig.~\ref{fig1}\textbf{F}), only restricted by the steric interaction between the droplets. Furthermore, since these are liquid droplets, there are also convective flows that are generated inside these self-propelled oil droplets~\cite{PNAS_Thutupalli} which can cause dynamic reorientation of the propulsion direction (this is in strong contrast with asymmetric colloids such as Janus colloids, whose orientation is fixed by design). We activate the chains by immersing them in 25~wt\% SDS surfactant solution. This causes the monomers within the droplet to propel (Fig.~\ref{fig1}\textbf{G}) and thereby the entire polymer is set into motion -- this propulsion is fully three-dimensional and for ease of visualisation we only focused on confined settings (using Hele-Shaw like geometries) in this paper (SI Fig.~S1\textbf{E}). Furthermore, since these are liquid droplets, there are also convective flows that are generated inside these self-propelled oil droplets~\cite{PNAS_Thutupalli} which can cause dynamic reorientation of the propulsion direction; these fully flexible active polymer dynamics are evident in the partially three-dimensional random motion and orientation of the polymer chains (Fig.~\ref{fig1}\textbf{G} and Supplementary Video~SV2). \par

\begin{figure*}[h!]
    \centering
    \includegraphics[width=0.95 \textwidth]{Figure_1.png}
    \caption{\textbf{Self-propelled droplets are used as monomers to assemble freely-jointed active polymers.} \textbf{(A)} A monomeric liquid crystal droplet in a surfactant micellar solution, swims due to a spontaneously generated surfactant gradient at its interface. \textbf{(B)} The propulsion of the nematic liquid crystal 5CB droplets results in characteristic hydrodynamic flow fields (\textit{left half:} experimental measurements and \textit{right half:} simulations of a squirmer model) and \textbf{(C)} chemical fields visualised using oil-soluble fluorescent (Nile red) dye mixed with 5CB. Inset: the symmetry breaking associated with the self-propulsion and the defect structure within the nematic droplet is apparent in the bright field and cross-polarised images of the self-propelling droplets. \textbf{(D)} Schema for assembling linear polymers of the active droplets using biotin--streptavidin chemistry. \textbf{(E)} Linear, chemically linked, inactive N-meric chains of 5CB emulsion droplets, comprised of increasing numbers of monomer units. \textbf{(F)} The bond angles of the polymer exhibit a uniform distribution i.e. the polymers are freely-jointed. \textbf{(G)} When activated, i.e. immersed into surfactant solutions of high enough concentration, the polymers exhibit active motion. The image shows three time points of the polymer, overlaid with the center of mass trajectory, moving in a weakly confined setting. (All scale bars represent $50~\mu m$).
    }
    \label{fig1}
\end{figure*}

\subsection{Emergent rigidity and ballistic propulsion of the polymers in quasi two-dimensional confinement}

When the active polymers are strongly confined to a quasi two-dimensional setting \emph{i.e.} a Hele-Shaw cell, the chains no longer exhibit flexible motion but rather become rigid and adopt stereotypic C-like shapes (Fig.~\ref{fig2}\textbf{A, B}, Supplementary Videos~SV3). We quantify the curvature of the steady state chain, via the Monge representation~\cite{helfrich1973elastic}, which reveals a systematic dependence, \textit{i.e.} a straightening of the polymer, with increase in the polymer length, suggesting stiffening of the polymers (Fig~\ref{fig2}\textbf{C}). To test this further, we compute the ``positional-'' and ''orientational rigidity'' of the chains. The positional rigidity is quantified by the distribution of the positions of the monomers from center of mass of the chain along both parallel and perpendicular directions (Fig.~\ref{fig2}\textbf{C} inset and Fig.~\ref{fig2}\textbf{D}, SI Fig.~S1\textbf{F}). The orientational rigidity is computer by computing the correlation between the orientations of the monomers with respect direction orthogonal to the long axis of the polymer (Fig.~\ref{fig2}\textbf{C} \textit{inset} and Fig.~\ref{fig2}\textbf{E}). In weak confinements ($h/2b = 1.4$), the histograms for the positional (Fig.~\ref{fig2}\textbf{D}) and orientational (Fig.~\ref{fig2}\textbf{E}) rigidities exhibit a broad distribution for droplet positions indicating a freely-jointed or flexible nature of the chain. However, in strong confinement $(h/2b = 1)$, we see distinct peaks for the positions and orientations of the monomers within the chains, clearly indicating an emergence of rigidity of the C-shape configurations in strong confinement (Fig.~\ref{fig2}\textbf{D--E}). This emergent rigidity of the active chains is associated with ballistic propulsion of the polymers in a direction orthogonal to the length of the polymer. As with the curvature, the propulsion is sensitive to the number of monomeric units comprising the polymer (Fig.~\ref{fig2}\textbf{F--G}, Supplementary Videos~SV4) with longer polymers moving faster than shorter ones --- it is noteworthy that while the speeds of the polymers less than that of the monomeric units, they asymptote to the instantaneous speed of a monomeric droplet with increasing polymer length (Fig.~\ref{fig2}\textbf{H}). Altogether, we find that in confined settings, the active polymers exhibit emergent rigidity, straighten with increasing polymer lengths and propel ballistically in a direction orthogonal to the long axis of the polymers(Supplementary Videos~SV4).\par

\begin{figure*}[h!]
    \centering
    \includegraphics[width=0.95 \textwidth]{Figure_2.png}
    \caption{\textbf{Emergent ballistic propulsion and rigidity of the polymer in strong quasi two-dimensional confinement.} \textbf{A} and \textbf{B} The active chains are flexible in weak confinement ($\textit{h/2b} = 1$) and become rigid in strong confinement ($\textit{h/2b} = 1.4 $). Panels \textbf{A} and \textbf{B} are the time-dependent configurations (\textit{left} to \textit{right}) of the active chains in their weak and strong 2D confinements (scale bar: $50~\mu m$). \textbf{(C)} Dimensionless mean curvature (obtained by multiplying curvature with the droplet radius) of different polymer chain lengths decreases with chain length ($N$) (Supplementary Experimental Section, SI Fig.~S2 and Fig.~S3) . \textbf{D} and \textbf{E} are the distributions of the positional and orientational rigidity of the active polymer chain measured in weak confinement ($\textit{h/2b} = 1$) and in strong confinement ($\textit{h/2b} = 1.4 $). The positional rigidity measured from the chain centre of the mass in a direction parallel and orthogonal to the direction of motion of the centre of mass. The orientational rigidity of the chain measured as $e_{i}$*$n_{CM}$ where $e_{i}$ is the orientation of the unit vector $n_{i}$ and $n_{CM}$ is the unit vector of the center of mass. \textbf{(F)} Trajectories of different polymer chain lengths are shown in different colors, follow the \textbf{(G)} for the color representations (Supplementary Experimental Section, SI Fig.~S2 and Fig.~S3). \textbf{(G)} Mean squared displacement (MSD) profiles of experimental trajectories (scale bar: 1000~$\mu m$) for different length chains ($N=1$ to $N=7$), represented in dimensionless form by dividing the MSD with $b^2$ and time with $1~s$ where $b$ is the radius of the droplet (Supplementary Video~SV4). Dashed line indicates the ballistic $\sim t^{2}$ limit. \textbf{(H)} Normalised speed of different polymer chain lengths ($N$) which increases with the chain length. The speed of a polymer chain is normalised with respect to the monomer speed, $v_{s}$ (Supplementary Experimental Section, SI Fig.~S3).}
    \label{fig2}
\end{figure*}

\subsection{Self-shaping chemo-hydrodynamic interactions within the polymer}

\begin{figure*}[t]
    \centering
    \includegraphics[width=0.90\textwidth]{Figure_3.png}
    \caption{\textbf{Self-shaping hydrodynamic and chemical fields of  active polymer chains.} \textbf{(A)} Experimentally measured hydrodynamic flow fields (Supplementary Section~PIV, FlowTrace), SI Fig.~S(4\textbf{C, D, E}), from \textit{left to right}: dimer ($N = 2$), trimer ($N = 3$), and octamer ($N = 8$). \textbf{(B)} Experimentally measured chemical fields (Supplementary Section~CF, SI Fig.~S4 \textbf{A, B}), from \textit{left to right}: dimer, trimer, and octamer (scale bar: $50~\mu m$). All error bars indicate standard deviations.}
    \label{fig3}
\end{figure*}

To guide our understanding of these emergent traits, we start with quantitatively mapping the self-shaping hydrodynamic and chemical interaction fields within the polymer (exemplary hydrodynamic and chemical fields are shown in Figs.~\ref{fig3}\textbf{A}, \textbf{B} and associated Supplementary Videos~SV5--SV10). The flow fields were measured using standard microPIV techniques with fluorescent tracer particles of diameter 0.5~$\mu m$ pre-mixed in the surrounding continuous medium. On the other hand, the chemical fields were visualized using an oil-soluble fluorescent dye (Supplementary Experimental Section). When the oil droplets dissolve via micellar solubilization, micelles filled with dye-doped oil are formed, enabling the visualisation of the field of filled-oil-micelles (``chemical field'') around the monomer droplets in the chain (Fig.~\ref{fig3}\textbf{B}). The hydrodynamic flow fields around the chains are not obtained from a simple superposition of the field around a monomer (Fig.~\ref{fig1}\textbf{B}) underscoring the non-linearity of the chemo-hydrodynamic coupling between the monomers within the chains (Supplementary Information). Further, there is a qualitative difference in the fields of a dimer from that of the other N-mers: the dimer attains a metastable symmetric configuration during which it does not propel while the steady-state hydrodynamic and chemical fields of all other N-mers are asymmetric (see SI Fig.~S6 \textbf{A--H}, Supplementary Videos~SV5--SV10). The symmetries of the chemical and hydrodynamic flow fields follow each other and it is remarkable to note that steady state chemical fields are also established in front of the chains in the direction of propulsion --- the polymers propel even through the negatively chemotactic fields that they self-generate -- this together with the force-balance resulting from the orientations of the swimmers could account for the reduction in the speed of the polymers. Despite the advection-diffusion coupled chemo-hydrodynamic effects, we next ask if chemical interactions between the monomeric droplets (in a freely jointed chain) alone can rationalize the experimental observations. Indeed, such a rationalization might be anticipated \textit{a priori}, given that such approaches successfully describe the dynamics of individual droplets~\cite{hokmabad2022chemotactic} and also that in confinement (two dimensions), chemical fields decay slower than hydrodynamic fields~\cite{kanso2019phoretic}. \par

\subsection {Chemical interactions alone account for the emergent rigidity and ballistic propulsion}
We develop a minimal model for the active polymers in which the chemical field around a monomer droplet is modeled by considering the $i$th monomer as a point source of the chemical field centered at $\bm R_i$, which self-propels with speed $v_s$ along the direction $\bm e_i$. The velocity and direction of the monomer can change due to chemical interactions. The position $\bm R_i$ and orientation, given by the unit vector $\bm e_i$, of the $i$th monomer is updated using the following kinematic equations:
\begin{align}
\frac{d \bm R_i}{dt} = {\bm V}_i,\qquad \frac{d \bm e_i}{dt} = {\bm \Omega}_i \times \bm e_i. 
\label{eq:dyn}
\end{align}
Here, the translational velocity $\bm V_i$ and angular velocity $\mathbf \Omega_i$ of the $i$th monomer are given as:
 \begin{align}
     {\bm V}_i = v_s \bm e_i + \chi_t \,\bm {\mathcal J}_i + \mu \bm F_i
   %+ \sqrt{2D_t}\,\bm\xi_t
    ,\qquad
    \bm \Omega_i = \chi_r
    \left(\bm e_i\times\bm {\mathcal J}_i 
    \right)
  % + \sqrt{2D_r}\,\bm\xi_r
  .
    \label{le}
\end{align}
%%------------------------
\begin{figure*} 
    \centering
    \includegraphics[width=.99\textwidth]{Figure_4.png}
    \caption{\textbf{Simulations of a minimal model including only chemical interactions capture the active polymer dynamics}. \textbf{(A)} Snapshots of active chains from simulations are shown in four columns, for chain sizes of $N=2$, $N=3$, $N=4$ and $N=8$, respectively (from left to right). Black arrows indicate the orientation $\bm e_i$ of the particles. We show snapshots for the initial-state ($t=0$), transient-state ($t=9s$), and steady states  ($t=38s$ and $93$s). The pseudo-color plot of the chemical field has also been overlaid on the plots for configuration at $t=93$s.  
(\textbf{B}) Shows the saturating angles of the chain in the steady state, with $\beta_{in}$ and $\beta_{out}$ retaining their definitions as in Fig. \ref{fig3}.
(\textbf{C}) shows the MSD for different chain lengths, along with the ballistic limit of $\sim t^{2}$ as reference. 
(\textbf{D}) contains the speed of polymers %$v^A$ 
in steady state, normalised by the speed of a single particle $v_s$ in simulations, as a function of the number of monomers in the polymer ($N$). 
(\textbf{E}) shows the dimensionless curvature of the active polymer in simulations as a function of the $N$. 
For Figures \textbf{F}, \textbf{H}, and \textbf{I}, we display the positional and angular dynamics for the $N=8$ chain. (\textbf{F}) is the distribution of the angle between monomer inside the chain ($\beta_{in}$) and on the edge of the chain ($\beta_{out}$) for the trajectory shown in panel (\textbf{A}).
(\textbf{G}) is a schematic of distance from the centre of mass in directions parallel $r^{\parallel}-r_{CM}^{\parallel}$ and perpendicular $r^{\perp}-r_{CM}^{\perp}$ to the motion of centre of mass, with the $r_{CM}$ coordinates denoted by dotted red lines. The distribution of these distances are shown in (\textbf{F}) and (\textbf{I}) respectively for the dynamics shown in (\textbf{A}).
}
    \label{fig4}
\end{figure*}
%%------------------------
In the above equation, $\mu$ is the mobility of a monomer, while the chemical interactions between the monomers are contained in the vector $\bm {\mathcal J}_i =-\left(\bm\nabla c\right)_{\bm r=\bm R_i}$, where $c$ is the concentration of filled micelles. The concentration field of the filled micelles is obtained by considering each emulsion droplet as a point source of the chemical field (explicit expressions of $c$ and $\bm {\mathcal J}_i$ are given in the Supplementary Information). The constants $\chi_t$ and $\chi_r$ take positive values. $\chi_t>0$ implies a repulsive chemical interaction between the monomers while they are held together by the attractive spring potential described below. The constant $\chi_r>0$ implies that the monomers rotate away from each other in the freely joined chain --- such reorientation is consistent with the negative auto-chemotactic behavior of the monomer droplets~\cite{hokmabad2021emergence}, unlike the chemical interactions of purely phoretic colloids. In the experiments, the droplets are freely joined using biotin-streptavidin chemistry, which is captured in the model by an attractive spring potential. The resulting spring force on the $i$th monomer is given as: 
$\bm F_i = -\left(\bm\nabla  U\right)_{\bm r=\bm R_i}$, 
where $U=\sum_{i=1}^{N-1} U^C (\mathbf R_i,\mathbf R_{i+1})$ and $U^C=k\left(r_{ij}-r_0\right)^2$ is spring potential of stiffness $k$ and natural length $r_0$ which holds the chain together. Here, $r_{ij}=|\bm R_i - \bm R_j|$ and the spring's natural length is equal to the diameter of the monomer used in the experiment, $r_0=2b$. 
In our experiment, the typical active force [$\mathcal {O}(6\pi\eta b v_s) \sim 10^{-11} N$], with $\eta$ being the viscosity of the solvent is dominant to the typical Brownian forces [$\mathcal O(k_BT/b)\sim 10^{-16}N$], and is thus, ignored in our model. We simulate the model by integrating the above equations numerically (simulation details and a complete list of parameters are given in the Supplementary Information).\par 

Snapshots from simulations of the above model are shown in Fig.~\ref{fig4}\textbf{A}, where the polymer structures are confined to move in two dimensions. This setting corresponds to the strong confinement of our experiments. We find from our simulations that the active polymer chain exhibits an emergent rigidity and propels ballistically in C-configurations in the steady state, as is the case with experiments as shown in Fig.~\ref{fig2} (Supplementary Video SV11). We also measure the saturating angles between the monomers in the steady state; see Fig.~\ref{fig4}\textbf{B}. The MSD (mean-squared displacement) is also shown in Fig.~\ref{fig4}\textbf{C} scales as $\varpropto t^2$, which is in agreement with the experimentally measured MSD (see Fig.~\ref{fig2}\textbf{G}). It should be noted that a dimer stops propelling in the steady state once the monomers comprising it point away from each other (i.e. $\bm{e}_1 \cdot \bm{e}_2 = -1$). 
We find that the speed of the chain increases with the chain length $N$ (which is the number of monomers in the chain), as shown in Fig.~\ref{fig4}\textbf{D}. Finally, we also show that the curvature decreases with the chain length $N$; see Fig.~\ref{fig4}\textbf{E} (see SI for details on computation of the curvature). Thus, we show that the experimental observations in Fig.~\ref{fig2} are reproduced from simulations (see Fig.~\ref{fig4}) of our minimal model. 
We can now focus on a particular chain, namely the $N=8$ case (i.e the dynamics shown in Fig.~\ref{fig3}\textbf{A}), \textit{right}), and further quantify  the positional and angular rigidity of our system across its dynamical evolution; these are shown in Figures ~\ref{fig4}\textbf{F},\textbf{G}, \textbf{H} and \textbf{I}. Fig.~\ref{fig4}\textbf{F} shows the angle between monomer inside the chain ($\beta_{in}$) and on the edge of the chain ($\beta_{out}$) become fixed as the polymer reaches a steady state. Here, peaks of the histogram correspond to the steady state $\beta$s as in Fig.~\ref{fig4}\textbf{B}, whereas the tail of the distribution corresponds to the initial transient $\beta$s in the simulation before the chain acquires rigidity.  Fig.~\ref{fig4}\textbf{G} shows the schematic of distance  from the centre of mass in directions parallel $r^{\parallel}-r_{CM}^{\parallel}$ and perpendicular $r^{\perp}-r_{CM}^{\perp}$ to the motion of centre of mass, whose distributions are shown in Fig.~\ref{fig4}\textbf{H} and Fig.~\ref{fig4}\textbf{I}, respectively. These distributions are for the entire dynamical evolution of the chain, see SI for corresponding distributions in the steady-state - which correspond to the peaks of these figures. Overall, we see that the positions and hence internal angles of the chain acquire fixed values as the system reaches its steady state.
\par
It is to be noted that the chemical fields generated in our model are not an exact match to those as seen in the experiments -- for instance, the dipolar field for $N=2$ is not reproduced (compare Fig.~\ref{fig4}\textbf{A}, \textit{top row} with Fig.~\ref{fig3}\textbf{B}). Additionally, the dependence of the speed and curvature of the active chain on the number of monomers is only a semi-quantitative match in that the qualitative behaviour is reproduced but the exact numbers differ (compare the y-axis of Fig.~\ref{fig2}\textbf{H} with Fig.~\ref{fig4}\textbf{D} and Fig.~\ref{fig2}\textbf{C} with Fig.~\ref{fig4}\textbf{E}). These differences arise from (presumably) neglecting hydrodynamic flow fields in the simulations, which advect the chemicals. Nevertheless, our model captures the essential features of the emergent dynamical traits seen in the experiments, indicating that chemical interactions between the monomers are responsible for the emergent rigidity of the active polymers with hydrodynamics potentially giving rise only to higher order quantitative corrections.
% In a follow-up paper~\cite{Subramaniam2023}, we study the full phase diagram of chemical interaction of both repulsive and attractive kinds, along with non-reciprocal interactions between monomers.
\par 

\subsection*{Metastable configurations exhibiting spontaneous polymer rotations}

To emphasise the versatility of the model, we now show that the chemical interactions can also capture the dynamics of other metastable shapes of the active polymers. As discussed earlier, the chains adopt a stable C-shape self-propelling configuration. These shapes can be destabilised in two ways: (i) by increasing the height of confinement such that the droplets forming the chain can sample the full three dimensional space (e.g. Fig.~\ref{fig1}\textbf{F}, Fig.~\ref{fig2}\textbf{A} and Supplementary Video~SV2); and (ii) by direct-collision or interactions with surrounding assemblies (e.g.  Supplementary Video~SV12) . In the experiments, we often see such destabilisation (in quasi-2D settings due to the interactions of the chain with surrounding assemblies) and this causes the the chains to transition from the C-configurations to a metastable S-shaped symmetric configuration (Fig.~\ref{fig5} along with Supplementary Videos SV13 and SV14).\par

\begin{figure*}[h!]
    \centering
    \includegraphics[width=0.95 \textwidth]{Figure_5.png}
    \caption{\textbf{Persistent rotation in a metastable S-shaped configuration and its transition to a stable C-shaped configuration.} Longer active polymers transiently adopt S-shaped configurations with associated characteristic \textbf{(A)} hydrodynamic flow fields and \textbf{(B)} symmetry broken chiral chemical fields (Supplementary Videos~SV13, SV14). \textbf{(C)} In this metastable configuration the polymers undergo persistent rotation shown by the snapshots and transition spontaneously into the stable C-shaped configuration. \textbf{Top:} The top panel (\textit{left to right}) shows time snapshots of cross-polarised microscopy images of active polymer chain configurations. The instantaneous propulsion directions of droplets in the chain can be visualised with the help of a nematic director pointing in the propulsion direction. The initial time ($t = 0s$ and $40s$) frames show that active chain rotation in transient S-configuration (indicated with curved and red color arrows). At $t \sim 168s$, the active chain completely transitions to a stable C-configuration and the arrow (white) pointing in the chain translational direction (self-propulsion). Note that the transition time (from S to C-configuration) can be different for different chain lengths. \textbf{Bottom:} The above phenomena are reproduced in our simulations. We see the rotation (curved red arrows) in the S-configuration ($t=0s$ and $t=40s$), 
    %(ii) transition to C ($t=98s$), and (iii) after the transition, 
    and a purely linearly translating (long black arrow) rigid chain a $t=168s$. }
    \label{fig5}
\end{figure*}

In the S-configuration, the active chains exhibit a broken rotation symmetry in the hydrodynamic and chemical fields and undergo spontaneous rotational motion (Fig.~\ref{fig5}\textbf{A}, \textbf{B}). In these configurations, the chains cannot escape long term interaction with the self-generated chemical fields. As a consequence, the S-configuration is only transiently stable, switching back to the C-configuration via a series of monomer reorganizations (Fig.~\ref{fig5}\textbf{C}, \textit{top panel}, Supplementary Videos~SV15 and SV16). Simulations of our model, match this dynamical behaviour (both the polymer rotation and spontaneous transition to the C-shape) and show that, indeed, the dynamics of the S-configuration is due to interactions with the self-generated chemical field (Fig.~\ref{fig5}\textbf{C}, \textit{bottom panel}).\par

Our results so far suggest that the emergent dynamics of the flexible active polymers --- rigidity, stereoscopic shapes and propulsion --- can be qualitatively (and to some extent quantitatively) captured by considering the effects of confinement and chemical interactions between the droplets. However, the effects of the coupling between the chemical and hydrodynamic fields are apparent in our system due to their mutual shaping of each other. Such coupling can lead to feedback and time-delay effects which in some scenarios can give rise to time-dependent steady-states such as oscillations beyond what we have considered so far. Hints of such behavior can already be seen from the dynamics of monomers and dimers (Fig.~\ref{fig2}\textbf{F}). We now experimentally explore such regimes of chemo-hydrodynamic coupling via tuning of the monomer propulsion.\par

\subsection*{Tuning the chemo-hydrodynamic fields of the monomer self-propelled droplet}

\begin{figure*}[h!]
    \centering
    \includegraphics[width=0.95 \textwidth]{Figure_6.png}
    \caption{\textbf{Tuning the chemo-hydrodynamic fields of a monomer.} The schematics showing swimming monomer droplets without additional oil-filled micelles ($\Phi = 0$) \textbf{(A)} and with additional oil-filled micelles to the swimming droplet environment ($\Phi = 0.7$) \textbf{(B)}. \textbf{(C)} Shows the speed of the monomer droplet decreasing with increasing concentration of oil-filled micelles. \textbf{(D)} Experimental trajectories of monomer droplets at a different concentration of oil-filled micelles added to its environment. \textbf{(E)} The contractile (quadrupolar) nature of the steady-state hydrodynamic field of monomer in a chemically tuned environment with oil-filled micellar solution ($\Phi = 0.7$) and \textbf{(F)} is the corresponding modified steady-state chemical field. \textbf{(G)} Shows the normalised chemical field intensity \textit{behind the swimmer,} along the line normal to the direction of motion of swimming monomer droplets.} 
    \label{fig6}.
\end{figure*}

The propulsion of the monomer can be tuned by modulating the transport of fresh surfactant micelles to the droplet interface~\cite{dwivedi2021solute,ramesh2022interfacial,ramesh2023arrested}. Specifically, such modulation affects the distribution of the surfactant gradient, and thereby the slip velocity, on the surface of the droplet --- altogether, this qualitatively and quantitatively changes the chemical and hydrodynamic flow fields around a self-propelling monomer. We achieved such tuning by controlling the ratios between fresh (``unfilled'') and ``filled'' surfactant micelles in the aqueous medium in which the swimmers are embedded (Fig.~\ref{fig6}\textbf{A}). The filled micelles are prepared by pre-dissolving 5CB oil droplets in a 25~wt\% SDS solution to a saturation. We then adjust the ratio $\Phi$, the total fraction of the filled micelles in the solution, using which we obtain a quantitative tuning of the monomer speed (Fig.~\ref{fig6}\textbf{C}) and its dynamics (Fig.~\ref{fig6}\textbf{D}). It can be seen that with increasing $\Phi$, the speed of the monomer reduces sharply. This reduction in speed is associated with qualitative changes in the flow and chemical fields around the monomer (Fig.~\ref{fig6}\textbf{E--F} and SI Fig.~S9). Since our focus is to demonstrate non-trivial effects due to the coupling between the chemical and hydrodynamic fields, we focus on the case with $\Phi = 0.7$ which results in a flow field that is reminiscent of a ``puller'' type of squirmer (Fig.~\ref{fig6}\textbf{E}). Our rationale is as follows --- due to the reduced speed of the droplet in this configuration, the associated chemical field is able to effectively diffuse further, and form a wider trail behind the droplet (Fig.~\ref{fig6}\textbf{F--G}). In addition, due to the nature of the flow field, the filled micelles are transported to the front of the droplet advectively thus changing time-scales associated with the self-interaction of the monomers with their own chemical trails. \par

\subsection*{Time varying chemo-hydrodynamic fields and the emergence of polymer oscillations}

We now turn our attention to the dynamics of the chains in this condition \textit{i.e.} $\Phi = 0.7$. There are now striking differences in the chemical and hydrodynamic flow-fields around the polymers (Fig.~\ref{fig7}, in comparison with Fig.~\ref{fig3}). Specifically, clear asymmetries in the fields appear in a time-snapshot and these represent time-dependent dynamics (Supplementary Video~SV17--SV18) -- these asymmetries are also reflected in the orientations of the monomers within the chains. Generally, these represent complex dynamics of active polymers and here, we specifically focus only on time-periodic motion (oscillations). For this, we illustratively focus on the case of trimers --- oscillatory gaits of the trimers can readily be seen from the trajectories (Fig.~\ref{fig8}\textbf{A}) Supplementary Videos~SV19). By measuring the time-varying chemical fields around the trimer, we rationalise that these oscillations are caused due to the repulsive autochemotactic nature of the droplets. It can be seen  that the interplay between the flow and chemical fields leads to an asymmetric build-up of the chemical field in front of the chain (Fig.~\ref{fig8}\textbf{A}, \textit{lower row})  due to which the monomer self-propulsion direction undergoes a re-orientation. The periodicity of these oscillations can be seen clearly from the quantification of the intensity and phase angle measurements $\phi$, which periodically change with time on either side of the trimer (Fig.~\ref{fig8}\textbf{B}) --- the sharp transitions are suggestive of threshold concentrations of the chemical field that cause sharp re-orientations of the monomers.

\begin{figure*}[h!]
    \centering
    \includegraphics[width=0.95 \textwidth]{Figure_7.png}
    \caption{\textbf{Self-shaping hydrodynamic and chemical fields of  active polymer chains in a chemically tuned environment.} \textbf{(A)} Experimentally measured hydrodynamic flow fields (Supplementary Section~PIV, FlowTrace), SI Fig.~S4(\textbf{C, D, E}), from \textit{left to right}: dimer ($N = 2$), trimer ($N = 3$), and nonamer ($N = 9$). \textbf{(B)} Experimentally measured chemical fields (Supplementary Section~CF, SI Fig.~S4\textbf{A, B}), from \textit{left to right}: dimer, trimer, and nonamer (scale bar: $50~\mu m$).} 
    \label{fig7}.
\end{figure*}

\section*{Discussion}

Using chemo-hydrodynamically active monomer droplets, we constructed freely jointed polymers; not only are the joints between the monomers fully flexible but the self-propulsion direction of each monomer can also freely evolve. The self-interactions within the polymer give rise to emergent self-organization, particularly self-propulsion --- stable translation and metastable persistent rotations --- of the polymers in rigid configurations. We quantitatively mapped these interactions and using a simple model based on chemical interactions alone, identified minimal, generic conditions in which such novel self-organization can arise. It must be noted that the active chains in our experiments self-propel in a direction normal to their body axis, which is the direction of maximum drag for a slender body; this is in contrast to natural objects such as microbes and polymers which typically propel along their major body axis \emph{i.e.} the direction of minimum drag. 

The ``dry'' chemical active matter model we have considered here is in line with similar recent successful descriptions for the scattering dynamics of monomer droplets due to their repulsive auto-chemotaxis~\cite{hokmabad2022chemotactic,kranz2016effective}. However, the effects of the hydrodynamic coupling are apparent in our system due to their shaping of the chemical fields and vice-versa. While such coupling does not qualitatively effect the emergent rigidity and self-propulsion that we have reported here, feedback and time-delay effects from such coupling can give rise to steady-states such as oscillations. We have experimentally explored that such states may indeed be achieved via tuning of the monomer propulsion~\cite{ramesh2022interfacial,dwivedi2021solute} --- developing the full theoretical framework for such dynamics involving the coupling of the hydrodynamics with the chemical fields remains a future challenge.\par

\begin{figure*}[h!]
    \centering
    \includegraphics[width=0.95 \textwidth]{Figure_8.png}
    \caption{\textbf{Oscillatory dynamics of active chain in a chemically tuned environment.} \textbf{(A)} Oscillatory trajectories of a trimer and the time -dependent asymmetry in the corresponding chemical field in a strong two-dimensional confinement and chemically tuned environment. \textbf{(B)} Time-dependent periodic changes are observed in chemical field intensities measured along the droplet orientations. The periodic changes in the droplet orientations are measured for the droplet at the edge in reference to the orientations to the central droplet.} 
    \label{fig8}.
\end{figure*}

% There are several examples of flexible linear chains comprised of microscopic droplets or colloids that have been previously reported for the validation of polymer theories on colloidal length scales, where equilibrium structures change conformation due to their diffusive nature~\cite{verweij2021conformations, mcmullen2018freely}. Although in theoretical studies using active Brownian particle filaments, hydrodynamic interactions have been shown to modify the dynamics and conformational properties of the chain, but fluid-mediated interactions due to active force were neglected and arise solely due to intramolecular forces~\cite{martin2019active}. To understand the mechanical properties of active and living systems, it is crucial to understand their autonomous mechanical behavior, for example, the experimentally reported self-oscillatory behavior in biological and synthetic systems~\cite{berg1975chemotaxis,machin1958wave}.
% In various theoretical and macroscopic experimental settings, mechanical behavior or self-oscillations have been shown to emerge from the interplay between the activity and elasticity~\cite{zheng2023self,camalet1999self}. Additionally, there are examples of microscopic flexible colloidal chains that show oscillatory dynamics or transitions to different dynamic states under externally applied fields or catalytic activity and demonstrate the interplay between chain elasticity and active force~\cite{yan2016reconfiguring}. However, we report a flexible active chain of emulsion droplets where the active force on a strongly confined chain in 2D and chemical interactions make the flexible chain conformationally stable and stiffer. An active flexible chain becomes rigid upon interactions with the self-generated chemical field, and when the activity of the chain is decreased and the chemical interaction is further increased, the active chain exhibits self-oscillatory dynamics. Here, for the first time we show that the interplay of activity and chemical interactions can also give rise to oscillatory dynamics in flexible chains. There are examples of active chains reported previously~\cite{yan2016reconfiguring}, those chains are not connected through any type of bond, except that they are arranged by the external field, which can give more flexibility, but the chains break and form at the same time, which does not display the characteristic dynamics of the individual chain. 

% There are several examples of flexible linear chains comprised of microscopic droplets or colloids that have been previously reported for the validation of polymer theories on colloidal length scales, where equilibrium structures change conformation due to their diffusive nature~\cite{verweij2021conformations, mcmullen2018freely}. Although in theoretical studies using active Brownian particle filaments, hydrodynamic interactions have been shown to modify the dynamics and conformational properties of the chain, but fluid-mediated interactions due to active force were neglected and arise solely due to intramolecular forces~\cite{martin2019active}. To understand the mechanical properties of active and living systems, it is crucial to understand their autonomous mechanical behavior, for example, the experimentally reported self-oscillatory behavior in biological and synthetic systems~\cite{berg1975chemotaxis,machin1958wave}.
% In various theoretical and macroscopic experimental settings, mechanical behavior or self-oscillations have been shown to emerge from the interplay between the activity and elasticity~\cite{zheng2023self,camalet1999self}. Additionally, there are examples of microscopic flexible colloidal chains that show oscillatory dynamics or transitions to different dynamic states under externally applied fields or catalytic activity and demonstrate the interplay between chain elasticity and active force~\cite{yan2016reconfiguring}. However, we report a flexible active chain of emulsion droplets where the active force on a strongly confined chain in 2D and chemical interactions make the flexible chain conformationally stable and stiffer. An active flexible chain becomes rigid upon interactions with the self-generated chemical field, and when the activity of the chain is decreased and the chemical interaction is further increased, the active chain exhibits self-oscillatory dynamics. Here, for the first time we show that the interplay of activity and chemical interactions can also give rise to oscillatory dynamics in flexible chains. There are examples of active chains reported previously~\cite{yan2016reconfiguring}, those chains are not connected through any type of bond, except that they are arranged by the external field, which can give more flexibility, but the chains break and form at the same time, which does not display the characteristic dynamics of the individual chain. 

The emergent dynamics we have reported are in quasi two-dimensional settings. As we have seen, relaxation of this criterion results in a reduction of the rigidity and affects the stability of the stereotypical polymer configurations. For example, even when the height of the Hele-Shaw cell $h>2b$, there is room for extra conformational degrees of freedom for the freely jointed chain. The fully three-dimensional dynamics of active polymers with chemical and hydrodynamic interactions should hold rich possibilities for future exploration. Further, while we have only explored short linear assemblies here, the extension to longer polymers and higher dimensional assemblies such as sheets~\cite{manna2022harnessing} with more tunable bonds (``multiflavored assemblies'')~\cite{zhang2018multivalent}, complex shapes and thereby a richer dynamical repertoire is possible. Such multicomponent monomers with attractive, repulsive and even non-reciprocal interactions~\cite{meredith2020predator} may be used to create functional self-morphing assemblies.  Finally, emulsion droplets, as such, offer many possible modifications in both controlling the internal chemistry of the droplets to tune their activity and in controlling the flexibility of the assemblies because of the mobile surface of the droplets in comparison to colloidal systems. Therefore, these systems could also be step in the direction of designing the multicomponent and multistimuli active delivery systems~\cite{downs2020multi}. 

\section*{Materials and Methods} \label{Materials and Methods}

\subsection{Materials}

We used 5CB (4-cyno-4’-pentylbiphenyl) liquid crystal procured from Frinton Laboratories, Inc. and SDS (sodium dodecyl sulphate) surfactant procured from Sigma Aldrich. DOPC (1,2-dioleoyl-sn-glcero-3-phosphocoline), Liss-Rhod-PE (1,2-dioleoyl-sn-glycero-phosphoethanolamine-N-(lissamine B sulfonyl) (ammonium salt) and biotinyl-cap-PE (1,2-dioleoyl-sn-glycero-phosphoethanolamine-N-(cap biotinyl) (sodium salt) were purchased from Avanti Polar. Alexa Fluor\textregistered 488 conjugated streptavidin was procured from Sigma-Aldrich. FluoSpheres™ carboxylate-modified Microspheres, yellow-green fluorescent (505/515~$nm$) and Nile red dye were purchased from Invitrogen (by ThermoFisher Scientific). All chemicals were used as received.

\subsection{Preparation of (Lipid (DOPC) - surfactant (SDS)) micellar solution} \label{micellar solution}

All lipid stocks were prepared in chloroform: DOPC at 25~$mg/mL$, Biotinyl-cap-PE at 10~$\mu g/mL$, and Liss-Rhod-PE at 10~$\mu g/mL$. We aliquoted 25~$\mu L$ of DOPC from the stock, 2~$\mu L$ of  Biot-Cap-PE, and 10~$\mu L$ of Liss-Rhod-PE in a 2~$mL$ effendorf tube to prepare the aqueous phase. The lipids were vacuum dried overnight and then nitrogen dried before being mixed with the aqueous phase. To the dried lipids containing eppendorf, we added 1~$m L$ of 0.125 (w/v)\%~SDS solution prepared with Milli-Q water. The eppendorf was then left at room temperature (T = 25$\degree$C) for lipid hydration. A probe sonicator (VC 750 (750W), stepped microtip diameter: 3~$mm$) with an on/off pulse of 1.5~s/1.5~s and 32\% of the maximum sonicator amplitude was used to mix the components. The tube was placed in an ice bath during the sonication steps (sonication for 30~s with a 1~min gap) to avoid overheating of the sample. 

\subsection{Droplet production} \label{DP}

We used an oil-in-water (O/W) emulsion system to make the droplets. 5CB liquid crystal droplets (oil phase) were stabilised using micellar solution (surfactant and lipids) in the aqueous phase. When 5CB LC (the oil phase) is injected through a microfluidic channel and encounters the aqueous phase (an aqueous solution of lipids and surfactant, SI Fig.~S1\textbf{A}), adsorption of free lipid-surfactant results in stable 5CB droplets. The monodisperse droplets generated by the microfluidic device were collected and stored in 0.25\%~SDS solution, where they remained stable for months (SI Fig.~S1\textbf{C}, \textit{left panel}). The presence of lipids in the monolayer of lipids and surfactant stabilising the 5CB emulsion droplets was confirmed using a red channel of the fluorescence microscopy (SI Fig.~S1\textbf{C}, \textit{middle panel}), streptavidin functionalization of biotinylated 5CB droplets was confirmed using a green channel of the fluorescence microscopy (SI Fig.~S1\textbf{C}, \textit{right panel}) . Details of the fabrication of microfluidic devices can be found in the Supplementary Information, SI Fig.~S1\textbf{A}, and the experimental section (DF). 

\subsection{Preparation of droplet assemblies} \label{Droplet assemblies}
We used a simple method to prepare the 5CB droplet assemblies: we split the sample into two populations. Only biotinylated 5CB droplets were used in population "I" (SI Fig.~S1\textbf{B, a}) and streptavidin functionalised biotinylated 5CB droplets in population "II" (SI Fig.~S1\textbf{B, b, c}, \textit{right panel}). To remove free biotinylated lipids and streptavidin from the bulk, both populations (I and II) were washed 2-3 times with 0.25\%~SDS solution. The population "I" is then centrifuged at 6000~rpm for 30~s to settle down all the droplets at the bottom of the tube and remove the solvent from the top. On top of the settled fraction of the droplets, we gently added 200~$\mu L$ of fresh 0.25\%~SDS solution. The tube was then filled with 5~$\mu L$ of a 50~$m M$ NaCl salt solution prepared in 0.25\%~SDS solution, and it was centrifuged for 30~s (at 6000~rpm). We then gently added a 20~$\mu L$ volume of 5CB biotinylated droplets functionalized with streptavidin from population "II" to population "I" at the bottom of the tube (SI Fig.~S1\textbf{B}). The tube is then centrifuged at 12000~rpm for 35~min in an eppendorf centrifuge (Eppendorf centrifuge model 5418) at T = 15$\degree$C. The assemblies were studied inside a quasi two-dimensional flow cell (SI Fig.~S1\textbf{E)}; the details of the flow cell fabrication are discussed in the experimental section of the supporting information~(FlowCell).

\subsection{Imaging (Bright Field and Fluorescence Microscopy)} \label{microscopy}

We used both multi-channel epifluorescence and brightfield microscopy to visualize 5CB emulsion droplets and their linear assemblies. The assemblies were imaged on an Olympus IX81 microscope with a 4x, 10x, and 20x UPLSAPO objective, and images were captured with a Photometrics Prime camera. For the fluorescence images, we used a CoolLED PE-4000 lamp. A 16-bit multichannel image of 2048 x 2048 pixels was taken, consisting of a green channel (excitation: 460~$nm$, dichroic: Sem-rock’s Quad Band), a red channel (excitation: 550~$nm$, dichroic: Sem-rock’s Quad Band), and a bright field channel. We recorded the images and videos using Olympus cellSens software.
We used a Leica bright field microscope (model M205FA) equipped with Leica-DFC9000GT camera to record longer time videos of the linear assemblies. Videos are recorded with a microscope zoom of 2, an exposure time of 0.02~s, and a 1~fps frame rate. We captured a 16-bit image of 2048 x 2048 pixels using a Leica-DFC9000GT-VSC06748 camera.

\section*{Data Availability Statement}

No additional data was used besides the results described in the text. Additional summary statistics of the data plotted may be available upon reasonable request to the corresponding authors.

\section*{Code Availability Statement}

All code used to produce these results will be made available upon reasonable request to the corresponding authors.

%merlin.mbs apsrev4-1.bst 2010-07-25 4.21a (PWD, AO, DPC) hacked
%Control: key (0)
%Control: author (0) dotless jnrlst
%Control: editor formatted (1) identically to author
%Control: production of article title (0) allowed
%Control: page (1) range
%Control: year (0) verbatim
%Control: production of eprint (0) enabled
\begin{thebibliography}{48}%
\makeatletter
\providecommand \@ifxundefined [1]{%
 \@ifx{#1\undefined}
}%
\providecommand \@ifnum [1]{%
 \ifnum #1\expandafter \@firstoftwo
 \else \expandafter \@secondoftwo
 \fi
}%
\providecommand \@ifx [1]{%
 \ifx #1\expandafter \@firstoftwo
 \else \expandafter \@secondoftwo
 \fi
}%
\providecommand \natexlab [1]{#1}%
\providecommand \enquote  [1]{``#1''}%
\providecommand \bibnamefont  [1]{#1}%
\providecommand \bibfnamefont [1]{#1}%
\providecommand \citenamefont [1]{#1}%
\providecommand \href@noop [0]{\@secondoftwo}%
\providecommand \href [0]{\begingroup \@sanitize@url \@href}%
\providecommand \@href[1]{\@@startlink{#1}\@@href}%
\providecommand \@@href[1]{\endgroup#1\@@endlink}%
\providecommand \@sanitize@url [0]{\catcode `\\12\catcode `\$12\catcode
  `\&12\catcode `\#12\catcode `\^12\catcode `\_12\catcode `\%12\relax}%
\providecommand \@@startlink[1]{}%
\providecommand \@@endlink[0]{}%
\providecommand \url  [0]{\begingroup\@sanitize@url \@url }%
\providecommand \@url [1]{\endgroup\@href {#1}{\urlprefix }}%
\providecommand \urlprefix  [0]{URL }%
\providecommand \Eprint [0]{\href }%
\providecommand \doibase [0]{http://dx.doi.org/}%
\providecommand \selectlanguage [0]{\@gobble}%
\providecommand \bibinfo  [0]{\@secondoftwo}%
\providecommand \bibfield  [0]{\@secondoftwo}%
\providecommand \translation [1]{[#1]}%
\providecommand \BibitemOpen [0]{}%
\providecommand \bibitemStop [0]{}%
\providecommand \bibitemNoStop [0]{.\EOS\space}%
\providecommand \EOS [0]{\spacefactor3000\relax}%
\providecommand \BibitemShut  [1]{\csname bibitem#1\endcsname}%
\let\auto@bib@innerbib\@empty
%</preamble>
\bibitem [{\citenamefont {Ramaswamy}(2010)}]{ramaswamy2010mechanics}%
  \BibitemOpen
  \bibfield  {author} {\bibinfo {author} {\bibfnamefont {Sriram}\ \bibnamefont
  {Ramaswamy}},\ }\bibfield  {title} {\enquote {\bibinfo {title} {The mechanics
  and statistics of active matter},}\ }\href@noop {} {\bibfield  {journal}
  {\bibinfo  {journal} {Annual Reviews of Condensed Matter Physics}\ }\textbf
  {\bibinfo {volume} {1}},\ \bibinfo {pages} {323--345} (\bibinfo {year}
  {2010})}\BibitemShut {NoStop}%
\bibitem [{\citenamefont {Marchetti}\ \emph {et~al.}(2013)\citenamefont
  {Marchetti}, \citenamefont {Joanny}, \citenamefont {Ramaswamy}, \citenamefont
  {Liverpool}, \citenamefont {Prost}, \citenamefont {Rao},\ and\ \citenamefont
  {Simha}}]{marchetti2013hydrodynamics}%
  \BibitemOpen
  \bibfield  {author} {\bibinfo {author} {\bibfnamefont {M~Cristina}\
  \bibnamefont {Marchetti}}, \bibinfo {author} {\bibfnamefont
  {Jean-Fran{\c{c}}ois}\ \bibnamefont {Joanny}}, \bibinfo {author}
  {\bibfnamefont {Sriram}\ \bibnamefont {Ramaswamy}}, \bibinfo {author}
  {\bibfnamefont {Tanniemola~B}\ \bibnamefont {Liverpool}}, \bibinfo {author}
  {\bibfnamefont {Jacques}\ \bibnamefont {Prost}}, \bibinfo {author}
  {\bibfnamefont {Madan}\ \bibnamefont {Rao}}, \ and\ \bibinfo {author}
  {\bibfnamefont {R~Aditi}\ \bibnamefont {Simha}},\ }\bibfield  {title}
  {\enquote {\bibinfo {title} {Hydrodynamics of soft active matter},}\
  }\href@noop {} {\bibfield  {journal} {\bibinfo  {journal} {Reviews of Modern
  Physics}\ }\textbf {\bibinfo {volume} {85}},\ \bibinfo {pages} {1143}
  (\bibinfo {year} {2013})}\BibitemShut {NoStop}%
\bibitem [{\citenamefont {Cates}\ and\ \citenamefont
  {Tailleur}(2015)}]{cates2015motility}%
  \BibitemOpen
  \bibfield  {author} {\bibinfo {author} {\bibfnamefont {Michael~E}\
  \bibnamefont {Cates}}\ and\ \bibinfo {author} {\bibfnamefont {Julien}\
  \bibnamefont {Tailleur}},\ }\bibfield  {title} {\enquote {\bibinfo {title}
  {Motility-induced phase separation},}\ }\href@noop {} {\bibfield  {journal}
  {\bibinfo  {journal} {Annual Reviews of Condensed Matter Physics}\ }\textbf
  {\bibinfo {volume} {6}},\ \bibinfo {pages} {219--244} (\bibinfo {year}
  {2015})}\BibitemShut {NoStop}%
\bibitem [{\citenamefont {Bechinger}\ \emph {et~al.}(2016)\citenamefont
  {Bechinger}, \citenamefont {Di~Leonardo}, \citenamefont {L{\"o}wen},
  \citenamefont {Reichhardt}, \citenamefont {Volpe},\ and\ \citenamefont
  {Volpe}}]{bechinger2016active}%
  \BibitemOpen
  \bibfield  {author} {\bibinfo {author} {\bibfnamefont {Clemens}\ \bibnamefont
  {Bechinger}}, \bibinfo {author} {\bibfnamefont {Roberto}\ \bibnamefont
  {Di~Leonardo}}, \bibinfo {author} {\bibfnamefont {Hartmut}\ \bibnamefont
  {L{\"o}wen}}, \bibinfo {author} {\bibfnamefont {Charles}\ \bibnamefont
  {Reichhardt}}, \bibinfo {author} {\bibfnamefont {Giorgio}\ \bibnamefont
  {Volpe}}, \ and\ \bibinfo {author} {\bibfnamefont {Giovanni}\ \bibnamefont
  {Volpe}},\ }\bibfield  {title} {\enquote {\bibinfo {title} {Active particles
  in complex and crowded environments},}\ }\href@noop {} {\bibfield  {journal}
  {\bibinfo  {journal} {Reviews of Modern Physics}\ }\textbf {\bibinfo {volume}
  {88}},\ \bibinfo {pages} {045006} (\bibinfo {year} {2016})}\BibitemShut
  {NoStop}%
\bibitem [{\citenamefont {Saha}\ \emph {et~al.}(2014)\citenamefont {Saha},
  \citenamefont {Golestanian},\ and\ \citenamefont
  {Ramaswamy}}]{saha2014clusters}%
  \BibitemOpen
  \bibfield  {author} {\bibinfo {author} {\bibfnamefont {Suropriya}\
  \bibnamefont {Saha}}, \bibinfo {author} {\bibfnamefont {Ramin}\ \bibnamefont
  {Golestanian}}, \ and\ \bibinfo {author} {\bibfnamefont {Sriram}\
  \bibnamefont {Ramaswamy}},\ }\bibfield  {title} {\enquote {\bibinfo {title}
  {Clusters, asters, and collective oscillations in chemotactic colloids},}\
  }\href@noop {} {\bibfield  {journal} {\bibinfo  {journal} {Physical Review
  E}\ }\textbf {\bibinfo {volume} {89}},\ \bibinfo {pages} {062316} (\bibinfo
  {year} {2014})}\BibitemShut {NoStop}%
\bibitem [{\citenamefont {Howard}\ \emph {et~al.}(2001)\citenamefont {Howard}
  \emph {et~al.}}]{howard2001mechanics}%
  \BibitemOpen
  \bibfield  {author} {\bibinfo {author} {\bibfnamefont {Jonathon}\
  \bibnamefont {Howard}} \emph {et~al.},\ }\href
  {https://global.oup.com/ushe/product/mechanics-of-motor-proteins-and-the-cytoskeleton-9780878933334?cc=in&lang=en&}
  {\emph {\bibinfo {title} {Mechanics of motor proteins and the
  cytoskeleton}}},\ Vol.\ \bibinfo {volume} {743}\ (\bibinfo {year}
  {2001})\BibitemShut {NoStop}%
\bibitem [{\citenamefont {Berg}(1975)}]{berg1975chemotaxis}%
  \BibitemOpen
  \bibfield  {author} {\bibinfo {author} {\bibfnamefont {Howard~C}\
  \bibnamefont {Berg}},\ }\bibfield  {title} {\enquote {\bibinfo {title}
  {Chemotaxis in bacteria},}\ }\href@noop {} {\bibfield  {journal} {\bibinfo
  {journal} {Annual review of biophysics and bioengineering}\ }\textbf
  {\bibinfo {volume} {4}},\ \bibinfo {pages} {119--136} (\bibinfo {year}
  {1975})}\BibitemShut {NoStop}%
\bibitem [{\citenamefont {Gross}\ \emph {et~al.}(2017)\citenamefont {Gross},
  \citenamefont {Kumar},\ and\ \citenamefont {Grill}}]{gross2017active}%
  \BibitemOpen
  \bibfield  {author} {\bibinfo {author} {\bibfnamefont {Peter}\ \bibnamefont
  {Gross}}, \bibinfo {author} {\bibfnamefont {K~Vijay}\ \bibnamefont {Kumar}},
  \ and\ \bibinfo {author} {\bibfnamefont {Stephan~W}\ \bibnamefont {Grill}},\
  }\bibfield  {title} {\enquote {\bibinfo {title} {How active mechanics and
  regulatory biochemistry combine to form patterns in development},}\
  }\href@noop {} {\bibfield  {journal} {\bibinfo  {journal} {Annual review of
  biophysics}\ }\textbf {\bibinfo {volume} {46}},\ \bibinfo {pages} {337--356}
  (\bibinfo {year} {2017})}\BibitemShut {NoStop}%
\bibitem [{\citenamefont {Howse}\ \emph {et~al.}(2007)\citenamefont {Howse},
  \citenamefont {Jones}, \citenamefont {Ryan}, \citenamefont {Gough},
  \citenamefont {Vafabakhsh},\ and\ \citenamefont
  {Golestanian}}]{howse2007self}%
  \BibitemOpen
  \bibfield  {author} {\bibinfo {author} {\bibfnamefont {Jonathan~R}\
  \bibnamefont {Howse}}, \bibinfo {author} {\bibfnamefont {Richard~AL}\
  \bibnamefont {Jones}}, \bibinfo {author} {\bibfnamefont {Anthony~J}\
  \bibnamefont {Ryan}}, \bibinfo {author} {\bibfnamefont {Tim}\ \bibnamefont
  {Gough}}, \bibinfo {author} {\bibfnamefont {Reza}\ \bibnamefont
  {Vafabakhsh}}, \ and\ \bibinfo {author} {\bibfnamefont {Ramin}\ \bibnamefont
  {Golestanian}},\ }\bibfield  {title} {\enquote {\bibinfo {title} {Self-motile
  colloidal particles: from directed propulsion to random walk},}\ }\href@noop
  {} {\bibfield  {journal} {\bibinfo  {journal} {Physical review letters}\
  }\textbf {\bibinfo {volume} {99}},\ \bibinfo {pages} {048102} (\bibinfo
  {year} {2007})}\BibitemShut {NoStop}%
\bibitem [{\citenamefont {Paxton}\ \emph {et~al.}(2004)\citenamefont {Paxton},
  \citenamefont {Kistler}, \citenamefont {Olmeda}, \citenamefont {Sen},
  \citenamefont {St.~Angelo}, \citenamefont {Cao}, \citenamefont {Mallouk},
  \citenamefont {Lammert},\ and\ \citenamefont {Crespi}}]{paxton2004catalytic}%
  \BibitemOpen
  \bibfield  {author} {\bibinfo {author} {\bibfnamefont {Walter~F}\
  \bibnamefont {Paxton}}, \bibinfo {author} {\bibfnamefont {Kevin~C}\
  \bibnamefont {Kistler}}, \bibinfo {author} {\bibfnamefont {Christine~C}\
  \bibnamefont {Olmeda}}, \bibinfo {author} {\bibfnamefont {Ayusman}\
  \bibnamefont {Sen}}, \bibinfo {author} {\bibfnamefont {Sarah~K}\ \bibnamefont
  {St.~Angelo}}, \bibinfo {author} {\bibfnamefont {Yanyan}\ \bibnamefont
  {Cao}}, \bibinfo {author} {\bibfnamefont {Thomas~E}\ \bibnamefont {Mallouk}},
  \bibinfo {author} {\bibfnamefont {Paul~E}\ \bibnamefont {Lammert}}, \ and\
  \bibinfo {author} {\bibfnamefont {Vincent~H}\ \bibnamefont {Crespi}},\
  }\bibfield  {title} {\enquote {\bibinfo {title} {Catalytic nanomotors:
  autonomous movement of striped nanorods},}\ }\href@noop {} {\bibfield
  {journal} {\bibinfo  {journal} {Journal of the American Chemical Society}\
  }\textbf {\bibinfo {volume} {126}},\ \bibinfo {pages} {13424--13431}
  (\bibinfo {year} {2004})}\BibitemShut {NoStop}%
\bibitem [{\citenamefont {Thutupalli}\ \emph {et~al.}(2011)\citenamefont
  {Thutupalli}, \citenamefont {Seemann},\ and\ \citenamefont
  {Herminghaus}}]{thutupalli2011}%
  \BibitemOpen
  \bibfield  {author} {\bibinfo {author} {\bibfnamefont {Shashi}\ \bibnamefont
  {Thutupalli}}, \bibinfo {author} {\bibfnamefont {Ralf}\ \bibnamefont
  {Seemann}}, \ and\ \bibinfo {author} {\bibfnamefont {Stephan}\ \bibnamefont
  {Herminghaus}},\ }\bibfield  {title} {\enquote {\bibinfo {title} {Swarming
  behavior of simple model squirmers},}\ }\href@noop {} {\bibfield  {journal}
  {\bibinfo  {journal} {New Journal of Physics}\ }\textbf {\bibinfo {volume}
  {13}},\ \bibinfo {pages} {073021} (\bibinfo {year} {2011})}\BibitemShut
  {NoStop}%
\bibitem [{\citenamefont {Hokmabad}\ \emph {et~al.}(2022)\citenamefont
  {Hokmabad}, \citenamefont {Agudo-Canalejo}, \citenamefont {Saha},
  \citenamefont {Golestanian},\ and\ \citenamefont
  {Maass}}]{hokmabad2022chemotactic}%
  \BibitemOpen
  \bibfield  {author} {\bibinfo {author} {\bibfnamefont {Babak~Vajdi}\
  \bibnamefont {Hokmabad}}, \bibinfo {author} {\bibfnamefont {Jaime}\
  \bibnamefont {Agudo-Canalejo}}, \bibinfo {author} {\bibfnamefont {Suropriya}\
  \bibnamefont {Saha}}, \bibinfo {author} {\bibfnamefont {Ramin}\ \bibnamefont
  {Golestanian}}, \ and\ \bibinfo {author} {\bibfnamefont {Corinna~C}\
  \bibnamefont {Maass}},\ }\bibfield  {title} {\enquote {\bibinfo {title}
  {Chemotactic self-caging in active emulsions},}\ }\href@noop {} {\bibfield
  {journal} {\bibinfo  {journal} {Proceedings of the National Academy of
  Sciences}\ }\textbf {\bibinfo {volume} {119}},\ \bibinfo {pages}
  {e2122269119} (\bibinfo {year} {2022})}\BibitemShut {NoStop}%
\bibitem [{\citenamefont {Meredith}\ \emph {et~al.}(2020)\citenamefont
  {Meredith}, \citenamefont {Moerman}, \citenamefont {Groenewold},
  \citenamefont {Chiu}, \citenamefont {Kegel}, \citenamefont {van Blaaderen},\
  and\ \citenamefont {Zarzar}}]{meredith2020predator}%
  \BibitemOpen
  \bibfield  {author} {\bibinfo {author} {\bibfnamefont {Caleb~H}\ \bibnamefont
  {Meredith}}, \bibinfo {author} {\bibfnamefont {Pepijn~G}\ \bibnamefont
  {Moerman}}, \bibinfo {author} {\bibfnamefont {Jan}\ \bibnamefont
  {Groenewold}}, \bibinfo {author} {\bibfnamefont {Yu-Jen}\ \bibnamefont
  {Chiu}}, \bibinfo {author} {\bibfnamefont {Willem~K}\ \bibnamefont {Kegel}},
  \bibinfo {author} {\bibfnamefont {Alfons}\ \bibnamefont {van Blaaderen}}, \
  and\ \bibinfo {author} {\bibfnamefont {Lauren~D}\ \bibnamefont {Zarzar}},\
  }\bibfield  {title} {\enquote {\bibinfo {title} {Predator--prey interactions
  between droplets driven by non-reciprocal oil exchange},}\ }\href@noop {}
  {\bibfield  {journal} {\bibinfo  {journal} {Nature Chemistry}\ }\textbf
  {\bibinfo {volume} {12}},\ \bibinfo {pages} {1136--1142} (\bibinfo {year}
  {2020})}\BibitemShut {NoStop}%
\bibitem [{\citenamefont {Ren}\ \emph {et~al.}(2020)\citenamefont {Ren},
  \citenamefont {Wang}, \citenamefont {Gao}, \citenamefont {Teng},
  \citenamefont {Xu}, \citenamefont {Wang}, \citenamefont {Pan},\ and\
  \citenamefont {Epstein}}]{ren2020programmed}%
  \BibitemOpen
  \bibfield  {author} {\bibinfo {author} {\bibfnamefont {Lin}\ \bibnamefont
  {Ren}}, \bibinfo {author} {\bibfnamefont {Liyuan}\ \bibnamefont {Wang}},
  \bibinfo {author} {\bibfnamefont {Qingyu}\ \bibnamefont {Gao}}, \bibinfo
  {author} {\bibfnamefont {Rui}\ \bibnamefont {Teng}}, \bibinfo {author}
  {\bibfnamefont {Ziyang}\ \bibnamefont {Xu}}, \bibinfo {author} {\bibfnamefont
  {Jing}\ \bibnamefont {Wang}}, \bibinfo {author} {\bibfnamefont {Changwei}\
  \bibnamefont {Pan}}, \ and\ \bibinfo {author} {\bibfnamefont {Irving~R}\
  \bibnamefont {Epstein}},\ }\bibfield  {title} {\enquote {\bibinfo {title}
  {Programmed locomotion of an active gel driven by spiral waves},}\
  }\href@noop {} {\bibfield  {journal} {\bibinfo  {journal} {Angewandte
  Chemie}\ }\textbf {\bibinfo {volume} {132}},\ \bibinfo {pages} {7172--7178}
  (\bibinfo {year} {2020})}\BibitemShut {NoStop}%
\bibitem [{\citenamefont {Aubret}\ \emph {et~al.}(2018)\citenamefont {Aubret},
  \citenamefont {Youssef}, \citenamefont {Sacanna},\ and\ \citenamefont
  {Palacci}}]{aubret2018targeted}%
  \BibitemOpen
  \bibfield  {author} {\bibinfo {author} {\bibfnamefont {Antoine}\ \bibnamefont
  {Aubret}}, \bibinfo {author} {\bibfnamefont {Mena}\ \bibnamefont {Youssef}},
  \bibinfo {author} {\bibfnamefont {Stefano}\ \bibnamefont {Sacanna}}, \ and\
  \bibinfo {author} {\bibfnamefont {J{\'e}r{\'e}mie}\ \bibnamefont {Palacci}},\
  }\bibfield  {title} {\enquote {\bibinfo {title} {Targeted assembly and
  synchronization of self-spinning microgears},}\ }\href@noop {} {\bibfield
  {journal} {\bibinfo  {journal} {Nature Physics}\ }\textbf {\bibinfo {volume}
  {14}},\ \bibinfo {pages} {1114--1118} (\bibinfo {year} {2018})}\BibitemShut
  {NoStop}%
\bibitem [{\citenamefont {L{\"o}wen}(2018)}]{lowen2018active}%
  \BibitemOpen
  \bibfield  {author} {\bibinfo {author} {\bibfnamefont {Hartmut}\ \bibnamefont
  {L{\"o}wen}},\ }\bibfield  {title} {\enquote {\bibinfo {title} {Active
  colloidal molecules},}\ }\href@noop {} {\bibfield  {journal} {\bibinfo
  {journal} {Europhysics Letters}\ }\textbf {\bibinfo {volume} {121}},\
  \bibinfo {pages} {58001} (\bibinfo {year} {2018})}\BibitemShut {NoStop}%
\bibitem [{\citenamefont {Jayaraman}\ \emph {et~al.}(2012)\citenamefont
  {Jayaraman}, \citenamefont {Ramachandran}, \citenamefont {Ghose},
  \citenamefont {Laskar}, \citenamefont {Bhamla}, \citenamefont {Kumar},\ and\
  \citenamefont {Adhikari}}]{jayaraman2012autonomous}%
  \BibitemOpen
  \bibfield  {author} {\bibinfo {author} {\bibfnamefont {Gayathri}\
  \bibnamefont {Jayaraman}}, \bibinfo {author} {\bibfnamefont {Sanoop}\
  \bibnamefont {Ramachandran}}, \bibinfo {author} {\bibfnamefont {Somdeb}\
  \bibnamefont {Ghose}}, \bibinfo {author} {\bibfnamefont {Abhrajit}\
  \bibnamefont {Laskar}}, \bibinfo {author} {\bibfnamefont {M~Saad}\
  \bibnamefont {Bhamla}}, \bibinfo {author} {\bibfnamefont {PB~Sunil}\
  \bibnamefont {Kumar}}, \ and\ \bibinfo {author} {\bibfnamefont
  {R}~\bibnamefont {Adhikari}},\ }\bibfield  {title} {\enquote {\bibinfo
  {title} {Autonomous motility of active filaments due to spontaneous
  flow-symmetry breaking},}\ }\href@noop {} {\bibfield  {journal} {\bibinfo
  {journal} {Physical review letters}\ }\textbf {\bibinfo {volume} {109}},\
  \bibinfo {pages} {158302} (\bibinfo {year} {2012})}\BibitemShut {NoStop}%
\bibitem [{\citenamefont {Laskar}\ and\ \citenamefont
  {Adhikari}(2015)}]{laskar2015brownian}%
  \BibitemOpen
  \bibfield  {author} {\bibinfo {author} {\bibfnamefont {Abhrajit}\
  \bibnamefont {Laskar}}\ and\ \bibinfo {author} {\bibfnamefont
  {R}~\bibnamefont {Adhikari}},\ }\bibfield  {title} {\enquote {\bibinfo
  {title} {Brownian microhydrodynamics of active filaments},}\ }\href@noop {}
  {\bibfield  {journal} {\bibinfo  {journal} {Soft Matter}\ }\textbf {\bibinfo
  {volume} {11}},\ \bibinfo {pages} {9073--9085} (\bibinfo {year}
  {2015})}\BibitemShut {NoStop}%
\bibitem [{\citenamefont {Winkler}\ \emph {et~al.}(2017)\citenamefont
  {Winkler}, \citenamefont {Elgeti},\ and\ \citenamefont
  {Gompper}}]{winkler2017active}%
  \BibitemOpen
  \bibfield  {author} {\bibinfo {author} {\bibfnamefont {Roland~G}\
  \bibnamefont {Winkler}}, \bibinfo {author} {\bibfnamefont {Jens}\
  \bibnamefont {Elgeti}}, \ and\ \bibinfo {author} {\bibfnamefont {Gerhard}\
  \bibnamefont {Gompper}},\ }\bibfield  {title} {\enquote {\bibinfo {title}
  {Active polymers—emergent conformational and dynamical properties: A brief
  review},}\ }\href@noop {} {\bibfield  {journal} {\bibinfo  {journal} {Journal
  of the Physical Society of Japan}\ }\textbf {\bibinfo {volume} {86}},\
  \bibinfo {pages} {101014} (\bibinfo {year} {2017})}\BibitemShut {NoStop}%
\bibitem [{\citenamefont {Prathyusha}\ \emph {et~al.}(2022)\citenamefont
  {Prathyusha}, \citenamefont {Ziebert},\ and\ \citenamefont
  {Golestanian}}]{prathyusha2022emergent}%
  \BibitemOpen
  \bibfield  {author} {\bibinfo {author} {\bibfnamefont {KR}~\bibnamefont
  {Prathyusha}}, \bibinfo {author} {\bibfnamefont {Falko}\ \bibnamefont
  {Ziebert}}, \ and\ \bibinfo {author} {\bibfnamefont {Ramin}\ \bibnamefont
  {Golestanian}},\ }\bibfield  {title} {\enquote {\bibinfo {title} {Emergent
  conformational properties of end-tailored transversely propelling
  polymers},}\ }\href@noop {} {\bibfield  {journal} {\bibinfo  {journal} {Soft
  Matter}\ }\textbf {\bibinfo {volume} {18}},\ \bibinfo {pages} {2928--2935}
  (\bibinfo {year} {2022})}\BibitemShut {NoStop}%
\bibitem [{\citenamefont {Manna}\ \emph {et~al.}(2022)\citenamefont {Manna},
  \citenamefont {Laskar}, \citenamefont {Shklyaev},\ and\ \citenamefont
  {Balazs}}]{manna2022harnessing}%
  \BibitemOpen
  \bibfield  {author} {\bibinfo {author} {\bibfnamefont {Raj~Kumar}\
  \bibnamefont {Manna}}, \bibinfo {author} {\bibfnamefont {Abhrajit}\
  \bibnamefont {Laskar}}, \bibinfo {author} {\bibfnamefont {Oleg~E}\
  \bibnamefont {Shklyaev}}, \ and\ \bibinfo {author} {\bibfnamefont {Anna~C}\
  \bibnamefont {Balazs}},\ }\bibfield  {title} {\enquote {\bibinfo {title}
  {Harnessing the power of chemically active sheets in solution},}\ }\href@noop
  {} {\bibfield  {journal} {\bibinfo  {journal} {Nature Reviews Physics}\
  }\textbf {\bibinfo {volume} {4}},\ \bibinfo {pages} {125--137} (\bibinfo
  {year} {2022})}\BibitemShut {NoStop}%
\bibitem [{\citenamefont {Zhang}\ and\ \citenamefont
  {Granick}(2016)}]{zhang2016natural}%
  \BibitemOpen
  \bibfield  {author} {\bibinfo {author} {\bibfnamefont {Jie}\ \bibnamefont
  {Zhang}}\ and\ \bibinfo {author} {\bibfnamefont {Steve}\ \bibnamefont
  {Granick}},\ }\bibfield  {title} {\enquote {\bibinfo {title} {Natural
  selection in the colloid world: active chiral spirals},}\ }\href@noop {}
  {\bibfield  {journal} {\bibinfo  {journal} {Faraday Discussions}\ }\textbf
  {\bibinfo {volume} {191}},\ \bibinfo {pages} {35--46} (\bibinfo {year}
  {2016})}\BibitemShut {NoStop}%
\bibitem [{\citenamefont {Vutukuri}\ \emph {et~al.}(2017)\citenamefont
  {Vutukuri}, \citenamefont {Bet}, \citenamefont {Van~Roij}, \citenamefont
  {Dijkstra},\ and\ \citenamefont {Huck}}]{vutukuri2017rational}%
  \BibitemOpen
  \bibfield  {author} {\bibinfo {author} {\bibfnamefont {Hanumantha~Rao}\
  \bibnamefont {Vutukuri}}, \bibinfo {author} {\bibfnamefont {Bram}\
  \bibnamefont {Bet}}, \bibinfo {author} {\bibfnamefont {Ren{\'e}}\
  \bibnamefont {Van~Roij}}, \bibinfo {author} {\bibfnamefont {Marjolein}\
  \bibnamefont {Dijkstra}}, \ and\ \bibinfo {author} {\bibfnamefont
  {Wilhelm~TS}\ \bibnamefont {Huck}},\ }\bibfield  {title} {\enquote {\bibinfo
  {title} {Rational design and dynamics of self-propelled colloidal bead
  chains: from rotators to flagella},}\ }\href@noop {} {\bibfield  {journal}
  {\bibinfo  {journal} {Scientific reports}\ }\textbf {\bibinfo {volume} {7}},\
  \bibinfo {pages} {16758} (\bibinfo {year} {2017})}\BibitemShut {NoStop}%
\bibitem [{\citenamefont {Nishiguchi}\ \emph {et~al.}(2018)\citenamefont
  {Nishiguchi}, \citenamefont {Iwasawa}, \citenamefont {Jiang},\ and\
  \citenamefont {Sano}}]{nishiguchi2018flagellar}%
  \BibitemOpen
  \bibfield  {author} {\bibinfo {author} {\bibfnamefont {Daiki}\ \bibnamefont
  {Nishiguchi}}, \bibinfo {author} {\bibfnamefont {Junichiro}\ \bibnamefont
  {Iwasawa}}, \bibinfo {author} {\bibfnamefont {Hong-Ren}\ \bibnamefont
  {Jiang}}, \ and\ \bibinfo {author} {\bibfnamefont {Masaki}\ \bibnamefont
  {Sano}},\ }\bibfield  {title} {\enquote {\bibinfo {title} {Flagellar dynamics
  of chains of active janus particles fueled by an ac electric field},}\
  }\href@noop {} {\bibfield  {journal} {\bibinfo  {journal} {New Journal of
  Physics}\ }\textbf {\bibinfo {volume} {20}},\ \bibinfo {pages} {015002}
  (\bibinfo {year} {2018})}\BibitemShut {NoStop}%
\bibitem [{\citenamefont {Snezhko}\ and\ \citenamefont
  {Aranson}(2011)}]{snezhko2011magnetic}%
  \BibitemOpen
  \bibfield  {author} {\bibinfo {author} {\bibfnamefont {Alexey}\ \bibnamefont
  {Snezhko}}\ and\ \bibinfo {author} {\bibfnamefont {Igor~S}\ \bibnamefont
  {Aranson}},\ }\bibfield  {title} {\enquote {\bibinfo {title} {Magnetic
  manipulation of self-assembled colloidal asters},}\ }\href@noop {} {\bibfield
   {journal} {\bibinfo  {journal} {Nature Materials}\ }\textbf {\bibinfo
  {volume} {10}},\ \bibinfo {pages} {698--703} (\bibinfo {year}
  {2011})}\BibitemShut {NoStop}%
\bibitem [{\citenamefont {Dreyfus}\ \emph {et~al.}(2005)\citenamefont
  {Dreyfus}, \citenamefont {Baudry}, \citenamefont {Roper}, \citenamefont
  {Fermigier}, \citenamefont {Stone},\ and\ \citenamefont
  {Bibette}}]{dreyfus2005microscopic}%
  \BibitemOpen
  \bibfield  {author} {\bibinfo {author} {\bibfnamefont {R{\'e}mi}\
  \bibnamefont {Dreyfus}}, \bibinfo {author} {\bibfnamefont {Jean}\
  \bibnamefont {Baudry}}, \bibinfo {author} {\bibfnamefont {Marcus~L}\
  \bibnamefont {Roper}}, \bibinfo {author} {\bibfnamefont {Marc}\ \bibnamefont
  {Fermigier}}, \bibinfo {author} {\bibfnamefont {Howard~A}\ \bibnamefont
  {Stone}}, \ and\ \bibinfo {author} {\bibfnamefont {J{\'e}r{\^o}me}\
  \bibnamefont {Bibette}},\ }\bibfield  {title} {\enquote {\bibinfo {title}
  {Microscopic artificial swimmers},}\ }\href@noop {} {\bibfield  {journal}
  {\bibinfo  {journal} {Nature}\ }\textbf {\bibinfo {volume} {437}},\ \bibinfo
  {pages} {862--865} (\bibinfo {year} {2005})}\BibitemShut {NoStop}%
\bibitem [{\citenamefont {Stenhammar}\ \emph {et~al.}(2016)\citenamefont
  {Stenhammar}, \citenamefont {Wittkowski}, \citenamefont {Marenduzzo},\ and\
  \citenamefont {Cates}}]{stenhammar2016light}%
  \BibitemOpen
  \bibfield  {author} {\bibinfo {author} {\bibfnamefont {Joakim}\ \bibnamefont
  {Stenhammar}}, \bibinfo {author} {\bibfnamefont {Raphael}\ \bibnamefont
  {Wittkowski}}, \bibinfo {author} {\bibfnamefont {Davide}\ \bibnamefont
  {Marenduzzo}}, \ and\ \bibinfo {author} {\bibfnamefont {Michael~E}\
  \bibnamefont {Cates}},\ }\bibfield  {title} {\enquote {\bibinfo {title}
  {Light-induced self-assembly of active rectification devices},}\ }\href@noop
  {} {\bibfield  {journal} {\bibinfo  {journal} {Science Advances}\ }\textbf
  {\bibinfo {volume} {2}},\ \bibinfo {pages} {e1501850} (\bibinfo {year}
  {2016})}\BibitemShut {NoStop}%
\bibitem [{\citenamefont {Biswas}\ \emph {et~al.}(2017)\citenamefont {Biswas},
  \citenamefont {Manna}, \citenamefont {Laskar}, \citenamefont {Kumar},
  \citenamefont {Adhikari},\ and\ \citenamefont
  {Kumaraswamy}}]{biswas2017linking}%
  \BibitemOpen
  \bibfield  {author} {\bibinfo {author} {\bibfnamefont {Bipul}\ \bibnamefont
  {Biswas}}, \bibinfo {author} {\bibfnamefont {Raj~Kumar}\ \bibnamefont
  {Manna}}, \bibinfo {author} {\bibfnamefont {Abhrajit}\ \bibnamefont
  {Laskar}}, \bibinfo {author} {\bibfnamefont {PB~Sunil}\ \bibnamefont
  {Kumar}}, \bibinfo {author} {\bibfnamefont {Ronojoy}\ \bibnamefont
  {Adhikari}}, \ and\ \bibinfo {author} {\bibfnamefont {Guruswamy}\
  \bibnamefont {Kumaraswamy}},\ }\bibfield  {title} {\enquote {\bibinfo {title}
  {Linking catalyst-coated isotropic colloids into “active” flexible chains
  enhances their diffusivity},}\ }\href@noop {} {\bibfield  {journal} {\bibinfo
   {journal} {ACS Nano}\ }\textbf {\bibinfo {volume} {11}},\ \bibinfo {pages}
  {10025--10031} (\bibinfo {year} {2017})}\BibitemShut {NoStop}%
\bibitem [{\citenamefont {Biswas}\ \emph {et~al.}(2021)\citenamefont {Biswas},
  \citenamefont {Mitra}, \citenamefont {Kp}, \citenamefont {Bhat},
  \citenamefont {Chatterji},\ and\ \citenamefont
  {Kumaraswamy}}]{biswas2021rigidity}%
  \BibitemOpen
  \bibfield  {author} {\bibinfo {author} {\bibfnamefont {Bipul}\ \bibnamefont
  {Biswas}}, \bibinfo {author} {\bibfnamefont {Debarshi}\ \bibnamefont
  {Mitra}}, \bibinfo {author} {\bibfnamefont {Fayis}\ \bibnamefont {Kp}},
  \bibinfo {author} {\bibfnamefont {Suresh}\ \bibnamefont {Bhat}}, \bibinfo
  {author} {\bibfnamefont {Apratim}\ \bibnamefont {Chatterji}}, \ and\ \bibinfo
  {author} {\bibfnamefont {Guruswamy}\ \bibnamefont {Kumaraswamy}},\ }\bibfield
   {title} {\enquote {\bibinfo {title} {Rigidity dictates spontaneous helix
  formation of thermoresponsive colloidal chains in poor solvent},}\
  }\href@noop {} {\bibfield  {journal} {\bibinfo  {journal} {ACS Nano}\
  }\textbf {\bibinfo {volume} {15}},\ \bibinfo {pages} {19702--19711} (\bibinfo
  {year} {2021})}\BibitemShut {NoStop}%
\bibitem [{\citenamefont {Zhang}\ \emph {et~al.}(2017)\citenamefont {Zhang},
  \citenamefont {McMullen}, \citenamefont {Pontani}, \citenamefont {He},
  \citenamefont {Sha}, \citenamefont {Seeman}, \citenamefont {Brujic},\ and\
  \citenamefont {Chaikin}}]{zhang2017sequential}%
  \BibitemOpen
  \bibfield  {author} {\bibinfo {author} {\bibfnamefont {Yin}\ \bibnamefont
  {Zhang}}, \bibinfo {author} {\bibfnamefont {Angus}\ \bibnamefont {McMullen}},
  \bibinfo {author} {\bibfnamefont {Lea-Laetitia}\ \bibnamefont {Pontani}},
  \bibinfo {author} {\bibfnamefont {Xiaojin}\ \bibnamefont {He}}, \bibinfo
  {author} {\bibfnamefont {Ruojie}\ \bibnamefont {Sha}}, \bibinfo {author}
  {\bibfnamefont {Nadrian~C}\ \bibnamefont {Seeman}}, \bibinfo {author}
  {\bibfnamefont {Jasna}\ \bibnamefont {Brujic}}, \ and\ \bibinfo {author}
  {\bibfnamefont {Paul~M}\ \bibnamefont {Chaikin}},\ }\bibfield  {title}
  {\enquote {\bibinfo {title} {Sequential self-assembly of dna functionalized
  droplets},}\ }\href@noop {} {\bibfield  {journal} {\bibinfo  {journal}
  {Nature Communications}\ }\textbf {\bibinfo {volume} {8}},\ \bibinfo {pages}
  {21} (\bibinfo {year} {2017})}\BibitemShut {NoStop}%
\bibitem [{\citenamefont {McMullen}\ \emph {et~al.}(2018)\citenamefont
  {McMullen}, \citenamefont {Holmes-Cerfon}, \citenamefont {Sciortino},
  \citenamefont {Grosberg},\ and\ \citenamefont {Brujic}}]{mcmullen2018freely}%
  \BibitemOpen
  \bibfield  {author} {\bibinfo {author} {\bibfnamefont {Angus}\ \bibnamefont
  {McMullen}}, \bibinfo {author} {\bibfnamefont {Miranda}\ \bibnamefont
  {Holmes-Cerfon}}, \bibinfo {author} {\bibfnamefont {Francesco}\ \bibnamefont
  {Sciortino}}, \bibinfo {author} {\bibfnamefont {Alexander~Y}\ \bibnamefont
  {Grosberg}}, \ and\ \bibinfo {author} {\bibfnamefont {Jasna}\ \bibnamefont
  {Brujic}},\ }\bibfield  {title} {\enquote {\bibinfo {title} {Freely jointed
  polymers made of droplets},}\ }\href@noop {} {\bibfield  {journal} {\bibinfo
  {journal} {Physical Review Letters}\ }\textbf {\bibinfo {volume} {121}},\
  \bibinfo {pages} {138002} (\bibinfo {year} {2018})}\BibitemShut {NoStop}%
\bibitem [{\citenamefont {McMullen}\ \emph {et~al.}(2022)\citenamefont
  {McMullen}, \citenamefont {Mu{\~n}oz~Basagoiti}, \citenamefont {Zeravcic},\
  and\ \citenamefont {Brujic}}]{mcmullen2022self}%
  \BibitemOpen
  \bibfield  {author} {\bibinfo {author} {\bibfnamefont {Angus}\ \bibnamefont
  {McMullen}}, \bibinfo {author} {\bibfnamefont {Maitane}\ \bibnamefont
  {Mu{\~n}oz~Basagoiti}}, \bibinfo {author} {\bibfnamefont {Zorana}\
  \bibnamefont {Zeravcic}}, \ and\ \bibinfo {author} {\bibfnamefont {Jasna}\
  \bibnamefont {Brujic}},\ }\bibfield  {title} {\enquote {\bibinfo {title}
  {Self-assembly of emulsion droplets through programmable folding},}\
  }\href@noop {} {\bibfield  {journal} {\bibinfo  {journal} {Nature}\ }\textbf
  {\bibinfo {volume} {610}},\ \bibinfo {pages} {502--506} (\bibinfo {year}
  {2022})}\BibitemShut {NoStop}%
\bibitem [{\citenamefont {Zhang}\ \emph {et~al.}(2018)\citenamefont {Zhang},
  \citenamefont {He}, \citenamefont {Zhuo}, \citenamefont {Sha}, \citenamefont
  {Brujic}, \citenamefont {Seeman},\ and\ \citenamefont
  {Chaikin}}]{zhang2018multivalent}%
  \BibitemOpen
  \bibfield  {author} {\bibinfo {author} {\bibfnamefont {Yin}\ \bibnamefont
  {Zhang}}, \bibinfo {author} {\bibfnamefont {Xiaojin}\ \bibnamefont {He}},
  \bibinfo {author} {\bibfnamefont {Rebecca}\ \bibnamefont {Zhuo}}, \bibinfo
  {author} {\bibfnamefont {Ruojie}\ \bibnamefont {Sha}}, \bibinfo {author}
  {\bibfnamefont {Jasna}\ \bibnamefont {Brujic}}, \bibinfo {author}
  {\bibfnamefont {Nadrian~C}\ \bibnamefont {Seeman}}, \ and\ \bibinfo {author}
  {\bibfnamefont {Paul~M}\ \bibnamefont {Chaikin}},\ }\bibfield  {title}
  {\enquote {\bibinfo {title} {Multivalent, multiflavored droplets by
  design},}\ }\href@noop {} {\bibfield  {journal} {\bibinfo  {journal}
  {Proceedings of the National Academy of Sciences}\ }\textbf {\bibinfo
  {volume} {115}},\ \bibinfo {pages} {9086--9091} (\bibinfo {year}
  {2018})}\BibitemShut {NoStop}%
\bibitem [{\citenamefont {Peddireddy}\ \emph {et~al.}(2012)\citenamefont
  {Peddireddy}, \citenamefont {Kumar}, \citenamefont {Thutupalli},
  \citenamefont {Herminghaus},\ and\ \citenamefont {Bahr}}]{peddireddy2012}%
  \BibitemOpen
  \bibfield  {author} {\bibinfo {author} {\bibfnamefont {Karthik}\ \bibnamefont
  {Peddireddy}}, \bibinfo {author} {\bibfnamefont {Pramoda}\ \bibnamefont
  {Kumar}}, \bibinfo {author} {\bibfnamefont {Shashi}\ \bibnamefont
  {Thutupalli}}, \bibinfo {author} {\bibfnamefont {Stephan}\ \bibnamefont
  {Herminghaus}}, \ and\ \bibinfo {author} {\bibfnamefont {Christian}\
  \bibnamefont {Bahr}},\ }\bibfield  {title} {\enquote {\bibinfo {title}
  {Solubilization of thermotropic liquid crystal compounds in aqueous
  surfactant solutions},}\ }\href@noop {} {\bibfield  {journal} {\bibinfo
  {journal} {Langmuir}\ }\textbf {\bibinfo {volume} {28}},\ \bibinfo {pages}
  {12426--12431} (\bibinfo {year} {2012})}\BibitemShut {NoStop}%
\bibitem [{\citenamefont {Peddireddy}(2014)}]{peddireddy2014liquid}%
  \BibitemOpen
  \bibfield  {author} {\bibinfo {author} {\bibfnamefont {Karthik~Reddy}\
  \bibnamefont {Peddireddy}},\ }\bibfield  {title} {\enquote {\bibinfo {title}
  {Liquid crystals in aqueous ionic surfactant solutions: Interfacial
  instabilities \& optical applications},}\ }\href@noop {} {\bibfield
  {journal} {\bibinfo  {journal} {{PhD} Dissertation, G\"ottingen University}\
  } (\bibinfo {year} {2014})}\BibitemShut {NoStop}%
\bibitem [{\citenamefont {Herminghaus}\ \emph {et~al.}(2014)\citenamefont
  {Herminghaus}, \citenamefont {Maass}, \citenamefont {Kr{\"u}ger},
  \citenamefont {Thutupalli}, \citenamefont {Goehring},\ and\ \citenamefont
  {Bahr}}]{herminghaus2014interfacial}%
  \BibitemOpen
  \bibfield  {author} {\bibinfo {author} {\bibfnamefont {Stephan}\ \bibnamefont
  {Herminghaus}}, \bibinfo {author} {\bibfnamefont {Corinna~C}\ \bibnamefont
  {Maass}}, \bibinfo {author} {\bibfnamefont {Carsten}\ \bibnamefont
  {Kr{\"u}ger}}, \bibinfo {author} {\bibfnamefont {Shashi}\ \bibnamefont
  {Thutupalli}}, \bibinfo {author} {\bibfnamefont {Lucas}\ \bibnamefont
  {Goehring}}, \ and\ \bibinfo {author} {\bibfnamefont {Christian}\
  \bibnamefont {Bahr}},\ }\bibfield  {title} {\enquote {\bibinfo {title}
  {Interfacial mechanisms in active emulsions},}\ }\href@noop {} {\bibfield
  {journal} {\bibinfo  {journal} {Soft matter}\ }\textbf {\bibinfo {volume}
  {10}},\ \bibinfo {pages} {7008--7022} (\bibinfo {year} {2014})}\BibitemShut
  {NoStop}%
\bibitem [{\citenamefont {Blake}(1971)}]{blake1971spherical}%
  \BibitemOpen
  \bibfield  {author} {\bibinfo {author} {\bibfnamefont {John~R}\ \bibnamefont
  {Blake}},\ }\bibfield  {title} {\enquote {\bibinfo {title} {A spherical
  envelope approach to ciliary propulsion},}\ }\href@noop {} {\bibfield
  {journal} {\bibinfo  {journal} {Journal of Fluid Mechanics}\ }\textbf
  {\bibinfo {volume} {46}},\ \bibinfo {pages} {199--208} (\bibinfo {year}
  {1971})}\BibitemShut {NoStop}%
\bibitem [{\citenamefont {Ehlers}\ \emph {et~al.}(1996)\citenamefont {Ehlers},
  \citenamefont {Berg},\ and\ \citenamefont
  {Montgomery}}]{ehlers1996synechococcus}%
  \BibitemOpen
  \bibfield  {author} {\bibinfo {author} {\bibfnamefont {KM}~\bibnamefont
  {Ehlers}}, \bibinfo {author} {\bibfnamefont {HC}~\bibnamefont {Berg}}, \ and\
  \bibinfo {author} {\bibfnamefont {R}~\bibnamefont {Montgomery}},\ }\bibfield
  {title} {\enquote {\bibinfo {title} {Do synechococcus swim using traveling
  surface waves},}\ }\href@noop {} {\bibfield  {journal} {\bibinfo  {journal}
  {Proceedings of the National Academy of Sciences}\ }\textbf {\bibinfo
  {volume} {93}},\ \bibinfo {pages} {8340--8344} (\bibinfo {year}
  {1996})}\BibitemShut {NoStop}%
\bibitem [{\citenamefont {Ishikawa}\ \emph {et~al.}(2006)\citenamefont
  {Ishikawa}, \citenamefont {Simmonds},\ and\ \citenamefont
  {Pedley}}]{ishikawa2006hydrodynamic}%
  \BibitemOpen
  \bibfield  {author} {\bibinfo {author} {\bibfnamefont {Takuji}\ \bibnamefont
  {Ishikawa}}, \bibinfo {author} {\bibfnamefont {MP}~\bibnamefont {Simmonds}},
  \ and\ \bibinfo {author} {\bibfnamefont {Timothy~J}\ \bibnamefont {Pedley}},\
  }\bibfield  {title} {\enquote {\bibinfo {title} {Hydrodynamic interaction of
  two swimming model micro-organisms},}\ }\href@noop {} {\bibfield  {journal}
  {\bibinfo  {journal} {Journal of Fluid Mechanics}\ }\textbf {\bibinfo
  {volume} {568}},\ \bibinfo {pages} {119--160} (\bibinfo {year}
  {2006})}\BibitemShut {NoStop}%
\bibitem [{\citenamefont {Hokmabad}\ \emph {et~al.}(2021)\citenamefont
  {Hokmabad}, \citenamefont {Dey}, \citenamefont {Jalaal}, \citenamefont
  {Mohanty}, \citenamefont {Almukambetova}, \citenamefont {Baldwin},
  \citenamefont {Lohse},\ and\ \citenamefont {Maass}}]{hokmabad2021emergence}%
  \BibitemOpen
  \bibfield  {author} {\bibinfo {author} {\bibfnamefont {Babak~Vajdi}\
  \bibnamefont {Hokmabad}}, \bibinfo {author} {\bibfnamefont {Ranabir}\
  \bibnamefont {Dey}}, \bibinfo {author} {\bibfnamefont {Maziyar}\ \bibnamefont
  {Jalaal}}, \bibinfo {author} {\bibfnamefont {Devaditya}\ \bibnamefont
  {Mohanty}}, \bibinfo {author} {\bibfnamefont {Madina}\ \bibnamefont
  {Almukambetova}}, \bibinfo {author} {\bibfnamefont {Kyle~A}\ \bibnamefont
  {Baldwin}}, \bibinfo {author} {\bibfnamefont {Detlef}\ \bibnamefont {Lohse}},
  \ and\ \bibinfo {author} {\bibfnamefont {Corinna~C}\ \bibnamefont {Maass}},\
  }\bibfield  {title} {\enquote {\bibinfo {title} {Emergence of bimodal
  motility in active droplets},}\ }\href@noop {} {\bibfield  {journal}
  {\bibinfo  {journal} {Physical Review X}\ }\textbf {\bibinfo {volume} {11}},\
  \bibinfo {pages} {011043} (\bibinfo {year} {2021})}\BibitemShut {NoStop}%
\bibitem [{\citenamefont {Dwivedi}\ \emph {et~al.}(2021)\citenamefont
  {Dwivedi}, \citenamefont {Si}, \citenamefont {Pillai},\ and\ \citenamefont
  {Mangal}}]{dwivedi2021solute}%
  \BibitemOpen
  \bibfield  {author} {\bibinfo {author} {\bibfnamefont {Prateek}\ \bibnamefont
  {Dwivedi}}, \bibinfo {author} {\bibfnamefont {Bishwa~Ranjan}\ \bibnamefont
  {Si}}, \bibinfo {author} {\bibfnamefont {Dipin}\ \bibnamefont {Pillai}}, \
  and\ \bibinfo {author} {\bibfnamefont {Rahul}\ \bibnamefont {Mangal}},\
  }\bibfield  {title} {\enquote {\bibinfo {title} {Solute induced jittery
  motion of self-propelled droplets},}\ }\href@noop {} {\bibfield  {journal}
  {\bibinfo  {journal} {Physics of Fluids}\ }\textbf {\bibinfo {volume} {33}},\
  \bibinfo {pages} {022103} (\bibinfo {year} {2021})}\BibitemShut {NoStop}%
\bibitem [{\citenamefont {Ramesh}\ \emph {et~al.}(2022)\citenamefont {Ramesh},
  \citenamefont {Hokmabad}, \citenamefont {Rahalia}, \citenamefont
  {Mathijssen}, \citenamefont {Pushkin},\ and\ \citenamefont
  {Maass}}]{ramesh2022interfacial}%
  \BibitemOpen
  \bibfield  {author} {\bibinfo {author} {\bibfnamefont {Prashanth}\
  \bibnamefont {Ramesh}}, \bibinfo {author} {\bibfnamefont {Babak~Vajdi}\
  \bibnamefont {Hokmabad}}, \bibinfo {author} {\bibfnamefont {Myriam}\
  \bibnamefont {Rahalia}}, \bibinfo {author} {\bibfnamefont {Arnold~JTM}\
  \bibnamefont {Mathijssen}}, \bibinfo {author} {\bibfnamefont {Dmitri~O}\
  \bibnamefont {Pushkin}}, \ and\ \bibinfo {author} {\bibfnamefont {Corinna~C}\
  \bibnamefont {Maass}},\ }\bibfield  {title} {\enquote {\bibinfo {title}
  {Interfacial activity dynamics of confined active droplets},}\ }\href@noop {}
  {\bibfield  {journal} {\bibinfo  {journal} {arXiv preprint arXiv:2202.08630}\
  } (\bibinfo {year} {2022})}\BibitemShut {NoStop}%
\bibitem [{\citenamefont {Thutupalli}\ \emph {et~al.}(2018)\citenamefont
  {Thutupalli}, \citenamefont {Geyer}, \citenamefont {Singh}, \citenamefont
  {Adhikari},\ and\ \citenamefont {Stone}}]{PNAS_Thutupalli}%
  \BibitemOpen
  \bibfield  {author} {\bibinfo {author} {\bibfnamefont {Shashi}\ \bibnamefont
  {Thutupalli}}, \bibinfo {author} {\bibfnamefont {Delphine}\ \bibnamefont
  {Geyer}}, \bibinfo {author} {\bibfnamefont {Rajesh}\ \bibnamefont {Singh}},
  \bibinfo {author} {\bibfnamefont {Ronojoy}\ \bibnamefont {Adhikari}}, \ and\
  \bibinfo {author} {\bibfnamefont {Howard~A.}\ \bibnamefont {Stone}},\
  }\bibfield  {title} {\enquote {\bibinfo {title} {Flow-induced phase
  separation of active particles is controlled by boundary conditions},}\
  }\href {\doibase 10.1073/pnas.1718807115} {\bibfield  {journal} {\bibinfo
  {journal} {Proceedings of the National Academy of Sciences}\ }\textbf
  {\bibinfo {volume} {115}},\ \bibinfo {pages} {5403--5408} (\bibinfo {year}
  {2018})},\ \Eprint
  {http://arxiv.org/abs/https://www.pnas.org/content/115/21/5403.full.pdf}
  {https://www.pnas.org/content/115/21/5403.full.pdf} \BibitemShut {NoStop}%
\bibitem [{\citenamefont {Helfrich}(1973)}]{helfrich1973elastic}%
  \BibitemOpen
  \bibfield  {author} {\bibinfo {author} {\bibfnamefont {Wolfgang}\
  \bibnamefont {Helfrich}},\ }\bibfield  {title} {\enquote {\bibinfo {title}
  {Elastic properties of lipid bilayers: theory and possible experiments},}\
  }\href@noop {} {\bibfield  {journal} {\bibinfo  {journal} {Zeitschrift
  f{\"u}r Naturforschung c}\ }\textbf {\bibinfo {volume} {28}},\ \bibinfo
  {pages} {693--703} (\bibinfo {year} {1973})}\BibitemShut {NoStop}%
\bibitem [{\citenamefont {Kanso}\ and\ \citenamefont
  {Michelin}(2019)}]{kanso2019phoretic}%
  \BibitemOpen
  \bibfield  {author} {\bibinfo {author} {\bibfnamefont {Eva}\ \bibnamefont
  {Kanso}}\ and\ \bibinfo {author} {\bibfnamefont {S{\'e}bastien}\ \bibnamefont
  {Michelin}},\ }\bibfield  {title} {\enquote {\bibinfo {title} {Phoretic and
  hydrodynamic interactions of weakly confined autophoretic particles},}\
  }\href@noop {} {\bibfield  {journal} {\bibinfo  {journal} {The Journal of
  Chemical Physics}\ }\textbf {\bibinfo {volume} {150}},\ \bibinfo {pages}
  {044902} (\bibinfo {year} {2019})}\BibitemShut {NoStop}%
\bibitem [{\citenamefont {Ramesh}\ \emph {et~al.}(2023)\citenamefont {Ramesh},
  \citenamefont {Chen}, \citenamefont {R{\"a}der}, \citenamefont {Morsbach},
  \citenamefont {Jalaal},\ and\ \citenamefont {Maass}}]{ramesh2023arrested}%
  \BibitemOpen
  \bibfield  {author} {\bibinfo {author} {\bibfnamefont {Prashanth}\
  \bibnamefont {Ramesh}}, \bibinfo {author} {\bibfnamefont {Yibo}\ \bibnamefont
  {Chen}}, \bibinfo {author} {\bibfnamefont {Petra}\ \bibnamefont {R{\"a}der}},
  \bibinfo {author} {\bibfnamefont {Svenja}\ \bibnamefont {Morsbach}}, \bibinfo
  {author} {\bibfnamefont {Maziyar}\ \bibnamefont {Jalaal}}, \ and\ \bibinfo
  {author} {\bibfnamefont {Corinna~C}\ \bibnamefont {Maass}},\ }\bibfield
  {title} {\enquote {\bibinfo {title} {Arrested on heating: controlling the
  motility of active droplets by temperature},}\ }\href@noop {} {\bibfield
  {journal} {\bibinfo  {journal} {arXiv preprint arXiv:2303.13442}\ } (\bibinfo
  {year} {2023})}\BibitemShut {NoStop}%
\bibitem [{\citenamefont {Kranz}\ \emph {et~al.}(2016)\citenamefont {Kranz},
  \citenamefont {Gelimson}, \citenamefont {Zhao}, \citenamefont {Wong},\ and\
  \citenamefont {Golestanian}}]{kranz2016effective}%
  \BibitemOpen
  \bibfield  {author} {\bibinfo {author} {\bibfnamefont {W~Till}\ \bibnamefont
  {Kranz}}, \bibinfo {author} {\bibfnamefont {Anatolij}\ \bibnamefont
  {Gelimson}}, \bibinfo {author} {\bibfnamefont {Kun}\ \bibnamefont {Zhao}},
  \bibinfo {author} {\bibfnamefont {Gerard~CL}\ \bibnamefont {Wong}}, \ and\
  \bibinfo {author} {\bibfnamefont {Ramin}\ \bibnamefont {Golestanian}},\
  }\bibfield  {title} {\enquote {\bibinfo {title} {Effective dynamics of
  microorganisms that interact with their own trail},}\ }\href@noop {}
  {\bibfield  {journal} {\bibinfo  {journal} {Physical Review Letters}\
  }\textbf {\bibinfo {volume} {117}},\ \bibinfo {pages} {038101} (\bibinfo
  {year} {2016})}\BibitemShut {NoStop}%
\bibitem [{\citenamefont {Downs}\ \emph {et~al.}(2020)\citenamefont {Downs},
  \citenamefont {Lunn}, \citenamefont {Booth}, \citenamefont {Sauer},
  \citenamefont {Ramsay}, \citenamefont {Klemperer}, \citenamefont {Hawker},\
  and\ \citenamefont {Bayley}}]{downs2020multi}%
  \BibitemOpen
  \bibfield  {author} {\bibinfo {author} {\bibfnamefont {Florence~G}\
  \bibnamefont {Downs}}, \bibinfo {author} {\bibfnamefont {David~J}\
  \bibnamefont {Lunn}}, \bibinfo {author} {\bibfnamefont {Michael~J}\
  \bibnamefont {Booth}}, \bibinfo {author} {\bibfnamefont {Joshua~B}\
  \bibnamefont {Sauer}}, \bibinfo {author} {\bibfnamefont {William~J}\
  \bibnamefont {Ramsay}}, \bibinfo {author} {\bibfnamefont {R~George}\
  \bibnamefont {Klemperer}}, \bibinfo {author} {\bibfnamefont {Craig~J}\
  \bibnamefont {Hawker}}, \ and\ \bibinfo {author} {\bibfnamefont {Hagan}\
  \bibnamefont {Bayley}},\ }\bibfield  {title} {\enquote {\bibinfo {title}
  {Multi-responsive hydrogel structures from patterned droplet networks},}\
  }\href@noop {} {\bibfield  {journal} {\bibinfo  {journal} {Nature chemistry}\
  }\textbf {\bibinfo {volume} {12}},\ \bibinfo {pages} {363--371} (\bibinfo
  {year} {2020})}\BibitemShut {NoStop}%
\end{thebibliography}%

\section*{Acknowledgements}
We acknowledge discussions with Sriram Ramaswamy, Ignacio Pagonabarraga, and Ronojoy Adhikari; and thank Abhrajit Laskar for his help with the numerical computation of the flow fields. We acknowledge support from the Department of Atomic Energy (India), under project no.\,RTI4006, the Simons Foundation (Grant No.\,287975), the Human Frontier Science Program and the Max Planck Society through a Max-Planck-Partner-Group. Work funded in part by the Indian Institute of Technology, Madras, India through their seed and initiation grants and  Start-up Research Grant, SERB, India, to RS. 


\section*{Author Contributions}

ST, RS, and MK conceived and designed the study. MK and AM performed the experiments and data analysis with help from ST. RS and AGS designed the theoretical model and carried out the simulations. MK, RS, AGS, and ST wrote the paper.

\section*{Competing Interests Statement}

The authors declare no competing interests.

\end{document}