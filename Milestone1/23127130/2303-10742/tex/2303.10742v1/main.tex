\documentclass[english,aps, prl, superscriptaddress, onecolumn, longbibliography]{revtex4-1}
\usepackage{amsmath, amssymb, graphicx, bm}
\usepackage{soul}
\usepackage{gensymb}
\usepackage{float}
\usepackage{xr}


\makeatletter
\usepackage[colorlinks=true,linkcolor=MidnightBlue,urlcolor=black,citecolor=MidnightBlue,anchorcolor=MidnightBlue]{hyperref}\usepackage[dvipsnames]{xcolor}
\newcommand*{\addFileDependency}[1]{
  \typeout{(#1)}
  \@addtofilelist{#1}
  \IfFileExists{#1}{}{\typeout{No file #1.}}
}
\makeatother

\newcommand*{\myexternaldocument}[1]{
    \externaldocument{#1}
    \addFileDependency{#1.tex}
    \addFileDependency{#1.aux}
}
%%% END HELPER CODE

% put all the external documents here!
% \myexternaldocument{SI}
\begin{document}

\title{Emergent rigidity in chemically self-interacting flexible active polymers}

\author{Manoj Kumar}
\email{manojk@ncbs.res.in}

\affiliation{Simons Centre for the Study of Living Machines, National Centre for
Biological Sciences, Tata Institute of Fundamental Research, Bangalore,
India}
\author{Aniruddh Murali}
\affiliation{Simons Centre for the Study of Living Machines, National Centre for
Biological Sciences, Tata Institute of Fundamental Research, Bangalore,
India}

\author{Arvin Gopal Subramaniam}
\affiliation{Department of Physics, Indian Institute of Technology, Chennai, India}

\author{Rajesh Singh}
\email{rsingh@physics.iitm.ac.in}
\affiliation{Department of Physics, Indian Institute of Technology, Chennai, India}


\author{Shashi Thutupalli}
\email{shashi@ncbs.res.in}
\affiliation{Simons Centre for the Study of Living Machines, National Centre for
Biological Sciences, Tata Institute of Fundamental Research, Bangalore,
India}
\affiliation{International Centre for Theoretical Sciences, Tata Institute of Fundamental
Research, Bangalore, India}

\begin{abstract}
Active, point-like, self-propelling particles -- microorganisms, autophoretic colloids, and active droplets -- are known to self-organise into collective states via inter-agent interactions. Here, we create conformationally free extended active objects --- polymers --- and show that self-shaping interactions within the polymer result in novel dissipative structures. Using experiments, simulations and theory, we show that chemical interactions between the jointed active droplets result in emergent rigidity, stereotypical shapes and ballistic propulsion of the active polymers. These traits, quantified by the curvatures and speeds of the active polymers, vary systematically with the number of monomers comprising the chain. Using simulations of a minimal model, we establish that the emergent properties are a generic consequence of quasi two-dimensional confinement and chemical interactions between the freely jointed active droplets. Our work provides insights into the emergent organization of extended active objects due to self-shaping interaction landscapes.
\end{abstract}

\maketitle

Understanding the orchestration between chemistry and mechanics is a fundamental challenge in the field of active matter~\cite{ramaswamy2010mechanics, marchetti2013hydrodynamics, cates2015motility, bechinger2016active, saha2014clusters} --- both in living systems~\cite{howard2001mechanics,berg1975chemotaxis,gross2017active} and also in some remarkable non-living emulations~\cite{howse2007self,paxton2004catalytic,thutupalli2011,hokmabad2022chemotactic,meredith2020predator,ren2020programmed}. While there has been an increasing focus of experimental work on the emergent consequences of such chemo-mechanical coupling in synthetic active matter systems comprised of of point-like active particles~\cite{hokmabad2022chemotactic, meredith2020predator}, the experimental realisation and study of such dynamics in extended active objects such as molecules~\cite{lowen2018active}, polymers~\cite{winkler2017active} and sheets~\cite{manna2022harnessing} remains elusive. \par 

Several approaches have attempted the construction of active linear assemblies to serve as idealised experimental realizations of multi-body molecular and polymeric systems. The assemblies have mostly been achieved using external electric fields, magnetic fields, or light energy~\cite{zhang2016natural,nishiguchi2018flagellar,snezhko2011magnetic,dreyfus2005microscopic,stenhammar2016light}. While these systems are out-of-equilibrium, they are not active in the sense of being autonomously/internally driven and the dynamics is predominantly controlled by external fields (electric, or magnetic, or light). On the other hand, chemically linking catalyst-coated colloids to construct linear polymers~\cite{biswas2017linking,biswas2021rigidity} result in polymeric structures which are semi-flexible but the monomeric units are neither self-propelled nor orientationally free. Emulsion droplets, due to their internal fluidity offer a promising monomeric unit --- flexible or freely jointed linear passive assemblies have been made using emulsion droplets using sticky DNA linkers to both the monomeric droplets~\cite{zhang2017sequential,mcmullen2018freely,mcmullen2022self,zhang2018multivalent}.\par

In this study, we construct polymeric chains of freely jointed self-propelling active ``monomer'' droplets. The propulsion mechanism of the monomer droplets creates external chemical and hydrodynamic fields, causing self-shaping interactions within the polymer. We measure these chemical and hydrodynamic fields and then quantify, as a function of the polymer lengths, the emergent rigidity, shapes and ballistic self-propulsion of the active polymers in quasi two-dimensional confinement. We show that a model which includes only chemical interactions between the monomers captures all these activity-driven conformational features quantitatively, thereby establishing the minimal requirements for the emergent self-organization: (i) the monomer droplets of the chain self-propel due to (gradients of) their self-generated chemical field and (ii) the structures are confined to move in quasi two-dimensions. Finally we show that other metastable configurations and transient dynamics such as spontaneous rotations of the polymers are possible due to the internal chemo-hydrodynamic interactions. Altogether, we envision these as first steps towards a kind of active matter with emergent self-morphic dynamics.\par

\subsection{Freely jointed active polymers of self-propelling emulsion droplets}

The monomer self-propelled droplets that we use are comprised of oil, slowly dissolving into an external supramicellar aqueous solution of ionic surfactants~\cite{peddireddy2012}. The dissolution of the droplets results in the spontaneous development of self-sustaining gradients of surfactant coverage around the droplets. These gradients give rise to Marangoni stresses causing the droplets to propel~\cite{herminghaus2014interfacial}. As such, these self-propelled droplets may be viewed as a physico-chemical realization of the so-called squirmer model for microswimmers, a sphere with a prescribed surface slip velocity that exchanges momentum with the fluid in which it is immersed~\cite{blake1971spherical,ehlers1996synechococcus,ishikawa2006hydrodynamic}. The qualitative and quantitative nature of this slip velocity governs the near- and far-field hydrodynamic flow perturbation around the squirmer and eventually its self-propulsion. \par

The hydrodynamic flow fields around the oil droplet microswimmer (Fig.~\ref{fig1}\textbf{A}, Supplementary Video~SV1) depend in tunable ways on geometric and chemical conditions~\cite{PNAS_Thutupalli,hokmabad2021emergence,ramesh2022interfacial}. In addition to the hydrodynamic flows, these microswimmers leave a trail of chemical fields~\cite{hokmabad2022chemotactic} comprised of oil-filled surfactant micelles formed by the transfer of oil molecules into empty surfactant micelles (Fig.~\ref{fig1}\textbf{B}, Supplementary Video~SV1). The trail persists because the oil-filled micelles take longer to diffuse than the surrounding oil-free surfactant micelles and causes a remodeling of the environment around the droplets, leading to time delayed negative chemotatic self-interactions~\cite{hokmabad2022chemotactic}. \par 
% By creating extended objects comprised of tethered assemblages of these droplets, here we show that these chemotactic interactions can be self-shaping and give rise to novel emergent dynamical traits. 

\begin{figure*}[h!]
    \centering
    \includegraphics[width=0.95 \textwidth]{Figure_1.png}
    \caption{\textbf{Active polymers comprised of freely-jointed active droplets.} The monomer units are self-propelled emulsion droplets of the nematic liquid crystal 5CB. These droplets are propelled due to a chemo-hydrodynamic mechanism and exhibit characteristic \textbf{(A)} hydrodynamic flow fields (\textit{left half:} experimental measurements and \textit{right half:} simulations of a squirmer model) and \textbf{(B)} chemical fields visualised using oil-soluble fluorescent (Nile red) dye mixed with 5CB (scale bar: $50~\mu m$). Inset: the symmetry breaking associated with the self-propulsion is also apparent in the bright field and cross-polarised images of the self-propelling droplets. Scale bars: $50~\mu m$. \textbf{(C)} Schema for assembling linear polymers of the active droplets using biotin - streptavidin chemistry. \textbf{(D)} Linear, inactive N-meric chains of 5CB emulsion droplets, comprised of increasing numbers of monomer units (scale bar: $50~\mu m$). \textbf{(E)} The joint/``bond'' between the droplets is fully flexible, quantified by the spread of bond-angles, $\theta$, for one illustrative bond angle. \textbf{(F)} Time snapshots of a polymer exhibiting self-propelled motion, when activated. The motion is fully flexible and three-dimensional (evident from the different focal planes the droplets occupy in the images) configurations. Arrows indicate the instantaneous propulsion direction of the individual droplets (scale bar: $50~\mu m$).}
    \label{fig1}
\end{figure*}

We use biotin-streptavidin chemistry in order to form freely jointed chains of the active droplets (Fig.~\ref{fig1}\textbf{C}). Briefly, we use a hybrid (surfactant-lipid) monolayer of surfactant and biotinylated lipids that coat the droplets spontaneously (SI Fig.~S1\textbf{A, C}). Using these biotinylated lipids in conjunction with the well known specific interaction of biotin with streptavidin, we ``polymerise'' the droplets to form linear chains via controlled assembly and incubation (details in Methods and Supplementary Information, SI Fig.~S1\textbf{B}). This assembly process robustly results in linear assemblies of 5CB oil emulsion droplets such as dimers, trimers, and  so on to decamers (see SI Fig.~S1\textbf{D}). Our protocol rarely yields chains longer than $N=10$ without branching, and longest linear chains we obtain consist of $N=13$ droplets.\par

When these linear assemblies are inactive, they adopt random configurations, suggesting a freely jointed nature of the bonds between the droplets (Fig.~\ref{fig1} \textbf{D}). This is further confirmed when the chains are activated and the droplets move freely; the bonds remain intact when active and the distribution of the bond angles (Fig.~\ref{fig1}\textbf{E}, Supplementary Video~SV2) measured from their temporal evolution for a randomly moving active polymer, indeed indicates the freely-jointed nature of the bond. Further, since these are liquid droplets, there are also convective flows that are generated inside these self-propelled oil droplets~\cite{PNAS_Thutupalli} which can cause dynamic re-orientation of the propulsion direction, in contrast to chains of active colloids~\cite{biswas2017linking} which are not free to reorient within the chains -- these flexible active polymer dynamics are evident in the fully three-dimensional random motion and orientation of the polymer chains (Fig.~\ref{fig1}\textbf{F} and Supplementary Video~SV3).\par

\subsection{Emergent dynamics of freely jointed active chains in two-dimensional confinement}
 
In a quasi two-dimensional setting \emph{i.e.} a Hele-Shaw cell, the chains exhibit in-plane self-propulsion (Fig.~\ref{fig2} \textbf{A}). The propulsion is associated with characteristic shapes (like the alphabet C) of the polymers with characteristic bond angles (Fig.~\ref{fig2} \textbf{A}) that depend systematically on the polymer length --- with increasing polymer lengths, the polymer contours become straighter \emph{i.e.} the steady-state bond angles tend closer to $\pi$ (SI Fig.~S5). The propulsion of the polymers is ballistic and becomes increasingly so with the polymer lengths (as seen from the mean-squared-displacements of the centers of mass of the chains, Fig.~\ref{fig2}\textbf{B} (Supplementary Videos~SV4)). The propulsion speeds of the polymers asymptote to the instantaneous speed of a monomeric droplet (Fig.~\ref{fig2}\textbf{C}) --- in contrast to the random walks themselves (Fig.~\ref{fig2}\textbf{B} and inset) --- with increasing rigidity of the polymers (Fig.~\ref{fig2}\textbf{D}). Altogether, in confined settings we find that the active polymers increase in rigidity associated with a characteristic shape leading to ballistic self-propulsion.\par

\begin{figure*}[t]
    \centering
    \includegraphics[width=0.90\textwidth]{Figure_2.png}
    \caption{\textbf{Emergent dynamics of active polymer chains in quasi two-dimensional confinements.} \textbf{(A)} Self-propulsion of active polymer chains in strong two-dimensional confinement. Exemplary snapshots of a 3-mer and 8-mer are shown, arrows indicating the instantaneous propulsion direction of the monomers. The polymers exhibit stereotypical C-shapes with characteristic steady-state bond angles dependent on the chain length. \textbf{(B)} Mean squared displacement (MSD) profiles of experimental trajectories (inset, scale bar: 1000~$\mu m$) for different length chains ($N=1$ to $N=7$), represented in dimensionless form by dividing the MSD with $b^2$ and time with $1~s$ where $b$ is the radius of the droplet (Supplementary Video~SV4). Dashed line indicates the ballistic $\sim t^{2}$ limit. \textbf{(C)} Normalised speed of different polymer chain lengths ($N$) which increases with the chain length. The speed of a polymer chain is normalised with respect to the monomer speed, $v_{s}$. \textbf{(D)} Dimensionless mean curvature (obtained by multiplying curvature with the droplet radius) of different polymer chain lengths decreases with chain length ($N$). \textbf{(E)} Experimentally measured hydrodynamic flow fields (Supplementary Section~PIV, FlowTrace), SI Fig.~S4 \textbf{C, D,  E}), from \textit{left to right}: dimer ($N = 2$), trimer ($N = 3$), and octamer ($N = 8$). \textbf{(F)} Experimentally measured chemical fields (Supplementary Section~CF, SI Fig.~S4 \textbf{A, B}), from \textit{left to right}: dimer, trimer, and octamer (scale bar: $50~\mu m$)}.
    \label{fig2}
\end{figure*}

To guide our understanding of these emergent traits, we start with quantitatively mapping the self-shaping hydrodynamic and chemical interaction fields within the polymer (exemplary hydrodynamic and chemical fields are shown in Figs.~\ref{fig2}\textbf{E}, \textbf{F} and associated Supplementary Videos~SV5--SV7). These fields are not obtained from a simple superposition of the monomer fields underscoring the self-shaping nature of these intra-polymer interactions (Supplementary Information). There is a qualitative difference in the fields of a dimer from that of the other N-mers: the dimer attains a metastable symmetric configuration during which it does not propel while the steady-state hydrodynamic and chemical fields of all other N-mers are asymmetric (see SI Fig.~S6, Supplementary Videos~SV5--SV7). The symmetries of the chemical and hydrodynamic flow fields follow each other and it is remarkable to note that steady state chemical fields are also established in front of the chains in the direction of propulsion --- the polymers propel even through the negatively chemotactic fields that they self-generate. Despite the advection-diffusion coupled chemo-hydrodynamic effects, we next ask if chemical interactions between the monomeric droplets (in a freely jointed chain) alone can rationalize the experimental observations. Indeed, such a rationalization might be anticipated \textit{a priori}, given that such approaches successfully describe the dynamics of individual droplets~\cite{hokmabad2022chemotactic} and also that in confinement (two dimensions), chemical fields decay slower than hydrodynamic fields~\cite{kanso2019phoretic}. \par

\subsection{A minimal model with chemical interactions accounts for the emergent traits}
We develop a minimal model for the active polymers in which the chemical field around a monomer droplet is modeled by considering the $i$th monomer as a point source of the chemical field centered at $\bm R_i$, which self-propels with speed $v_s$ along the direction $\bm e_i$. The velocity and direction of the monomer can change due to chemical interactions. The position $\bm R_i$ and orientation, given by the unit vector $\bm e_i$, of the $i$th particle is updated using the following kinematic equations:
\begin{align}
\frac{d \bm R_i}{dt} = {\bm V}_i,\qquad \frac{d \bm e_i}{dt} = {\bm \Omega}_i \times \bm e_i. 
\label{eq:dyn}
\end{align}
Here, the translational velocity $\bm V_i$ and angular velocity $\mathbf \Omega_i$ of the $i$th particle are given as:
 \begin{align}
     {\bm V}_i = v_s \bm e_i + \chi_t \,\bm {\mathcal J}_i + \mu \bm F_i
   %+ \sqrt{2D_t}\,\bm\xi_t
    ,\qquad
    \bm \Omega_i = \chi_r
    \left(\bm e_i\times\bm {\mathcal J}_i 
    \right)
  % + \sqrt{2D_r}\,\bm\xi_r
  .
    \label{le}
\end{align}
%%------------------------
\begin{figure*} 
    \centering
    \includegraphics[width=.99\textwidth]{Figure_3.png}
    \caption{\textbf{Simulations of a minimal model including only chemical interactions capture the active polymer dynamics}. \textbf{(A)} Snapshots of active chains from simulations are shown in four columns, for chain sizes of $N=2$, $N=3$, $N=4$ and $N=8$ respectively (from left to right). Black arrows indicate the orientation $\bm e_i$ of the particles. We show snapshots for the initial-state ($t=0$), transient-state ($t=9s$), and steady states  ($t=38s$ and $93$s). The pseudo-color plot of the chemical field has also been overlaid on the plots for configuration at $t=93$s.  
(\textbf{B}) Shows the saturating angles of the chain in the steady state, with $\beta_{in}$ and $\beta_{out}$ retaining their definitions as in Fig. \ref{fig2}. Panel (\textbf{C}) shows the MSD for different chain lengths, along with the ballistic limit of $\sim t^{2}$ as reference. Panel (\textbf{D}) contains the speed of polymers $v^A$ in steady state, normalised by the speed of a single particle $v_s$ in simulations, as a function of the number of monomers in the polymer $N$. The panel (\textbf{E}) shows the dimensionless curvature of the active polymer in simulations as a function of the $N$. }
    \label{fig3}
\end{figure*}
%%------------------------
In the above equation, $\mu$ is the mobility of the particle, while the chemical interactions between the particles are contained in the vector $\bm {\mathcal J}_i =-\left(\bm\nabla c\right)_{\bm r=\bm R_i}$, where $c$ is the concentration of filled micelles. The concentration field of the filled micelles is obtained by considering each emulsion droplet as a point source of the chemical field (explicit expressions of $c$ and $\bm {\mathcal J}_i$ are given in the Supplementary Information).  The constants $\chi_t$ and $\chi_r$ take positive values. $\chi_t>0$ implies a repulsive chemical interaction between the particles while they are held together by the attractive spring potential described below. The constant $\chi_r>0$ implies that the particles rotate away from each other in the freely joined chain. In the experiments, the droplets are freely joined using biotin-streptavidin chemistry. This connectivity is captured by an attractive spring potential. The resulting spring force on the $i$th particle is given as: 
$\bm F_i = -\left(\bm\nabla  U\right)_{\bm r=\bm R_i}$, 
where $U=k\left(r_{ij}-r_0\right)^2$ is spring potential of stiffness $k$ and natural length $r_0$ which holds the chain together. Here, $r_{ij}=|\bm R_i - \bm R_j|$ and the spring's natural length is equal to the diameter of the particle used in the experiment, $r_0=2b$. 
In our experiment, the typical active force [$\mathcal {O}(6\pi\eta b v_s) \sim 10^{-11} N$), with $\eta$ being the viscosity of the solvent] is dominant to the typical Brownian forces [$\mathcal O(k_BT/b)\sim 10^{-16}N)$], and is thus, ignored in our model. We simulate the model by integrating the above equations numerically (simulation details and a complete list of parameters are given in the Supplementary Information).\par 

Snapshots from simulations of the above model are shown in Fig.~\ref{fig3} \textbf{A}, where the particles are confined to move in two dimensions. This setting corresponds to the strong confinement of our experiments. We find from our simulations that the active polymer chain shows an emergent rigidity and propels ballistically in C-configurations in the steady state, as is the case with experiments as shown in Fig.~\ref{fig2} (Supplementary Video SV8). We also measure the saturating angles between the monomers in the steady state; see Fig.~\ref{fig3} \textbf{B}. The MSD (mean-squared displacement) is also shown in Fig.~\ref{fig3} \textbf{C} scales as $\varpropto t^2$, which is in agreement with the experimentally measured MSD; see Fig.~\ref{fig2} \textbf{B}. It should be noted that a chain of 2 particles stops propelling in the steady state once they are pointing away from each other (i.e. $\bm{e}_1 \cdot \bm{e}_2 = -1$). 
We find that the speed of the chain increases with the chain length $N$ (which is the number of monomers in the chain), as shown in Fig.~\ref{fig3} \textbf{D}. Finally, we also show that the curvature decreases with the chain length $N$; see Fig.~\ref{fig3} \textbf{E}. Thus, we show that the experimental observations in Fig.~\ref{fig2} are reproduced from simulations (see Fig.~\ref{fig3}) of our minimal model.\par

It is to be noted that the chemical fields generated in our point particle model are not an exact match to those as seen in the experiments -- for instance, the dipolar field for $N=2$ is not reproduced (compare Fig.~\ref{fig3} \textbf{A}, \textit{top row} with Fig.~\ref{fig2} \textbf{F}). Additionally, the dependence of the speed and curvature of the active chain on the number of monomers is only a semi-quantitative match - in that the qualitative behaviour is reproduced but the exact numbers differ (compare the y-axis of Fig.~\ref{fig2} \textbf{C} with Fig.~\ref{fig3} \textbf{D} and Fig.~\ref{fig2} \textbf{D} with Fig.~\ref{fig3} \textbf{E}).
These differences arise from (presumably) neglecting hydrodynamic flow fields in the simulations, which advect the chemicals around the particle. Nevertheless, our model captures the essential features of the emergent dynamical traits seen in the experiments, indicating that chemical interactions between the monomers are responsible for the emergent rigidity of the active polymers with hydrodynamics potentially giving rise only to higher order quantitative corrections.\par 

\subsection*{Metastable configurations exhibiting spontaneous polymer rotations}

We now turn to the stability of the shapes of the active polymers. As discussed earlier, the chains adopt a stable C-shape self-propelling configuration that can be destabilised in two ways: (i) by direct-collision or interactions with surrounding assemblies; and (ii) by increasing the height of confinement such that the droplets forming the chain can sample the full three dimensional space (e.g. Fig~\ref{fig1}\textbf{E}). Often, such a destabilisation results in the chains transitioning from a C-configuration to a metastable S-shaped symmetric configuration (Fig.~\ref{fig4}).\par

\begin{figure*}[h!]
    \centering
    \includegraphics[width=0.95 \textwidth]{Figure_4.png}
    \caption{\textbf{Persistent rotation in a metastable S-shaped configuration and its transition to a stable C-shaped configuration.} Longer active polymers transiently adopt S-shaped configurations with associated characteristic \textbf{(A)} hydrodynamic flow fields and \textbf{(B)} symmetry broken chiral chemical fields. \textbf{(C)} In this metastable configuration the polymers undergo persistent rotation shown by the snapshots and transition spontaneously into the stable C-shaped configuration. The top panel (\textit{left to right}) shows time snapshots of cross-polarised microscopy images of active polymer chain configurations. The instantaneous propulsion directions of droplets in the chain can be visualised with the help of a nematic director pointing in the propulsion direction. The initial time ($t = 0s$ and $40s$) frames show that active chain rotation in transient S-configuration (indicated with curved and red color arrows). At $t \sim 168s$, the active chain completely transitions to a stable C-configuration and the arrow (white) pointing in the chain translational direction (self-propulsion). Note that the transition time (from S to C-configuration) can be different for different chain lengths. \textbf{Bottom:} The above phenomena are reproduced in our simulations. We see the rotation (curved red arrows) in the S-configuration ($t=0s$ and $t=40s$), 
    %(ii) transition to C ($t=98s$), and (iii) after the transition, 
    and a purely linearly translating (long black arrow) rigid chain a $t=168s$. }
    \label{fig4}
\end{figure*}

In the S-configuration, the active chains exhibit a broken rotation symmetry in the hydrodynamic and chemical fields and undergo spontaneous rotational motion (Fig.~\ref{fig4}\textbf{A}, \textbf{B}). In these configurations, the chains cannot escape long term interaction with the self-generated chemical fields. As a consequence, the S-configuration is only transiently stable, quickly switching back to the C-configuration via a series of monomer reorganizations (Fig.~\ref{fig4}\textbf{C}, \textit{top panel}, Supplementary Videos SV9 and SV10). Simulations of our model, match this behaviour and show that, indeed, the S-configuration is unstable due to its interactions with the self-generated chemical field (Fig.~\ref{fig4}\textbf{C}, \textit{bottom panel}).\par

\section{Discussion}

Using chemo-hydrodynamically active monomer droplets, we constructed freely jointed polymers; not only are the joints between the monomers fully flexible but the self-propulsion direction of each monomer can also freely evolve. The self-interactions within the polymer give rise to emergent self-organization, particularly self-propulsion --- stable translation and metastable persistent rotations --- of the polymers in rigid configurations. We quantitatively mapped these interactions and using a simple model based on chemical interactions alone, identified minimal, generic conditions in which such novel self-organization can arise. The active chains in our experiments self-propel in a direction normal to their body axis, which is the direction of maximum drag for a slender body. This is in contrast to natural objects such as microbes and polymers which typically propel along their major body axis \emph{i.e.} the direction of minimum drag. 

The ``dry'' chemical active matter model we have considered here is in line with similar recent successful descriptions for the scattering dynamics of monomer droplets due to their repulsive auto-chemotaxis~\cite{hokmabad2022chemotactic,kranz2016effective}. However, it is clear that the effects of the hydrodynamic coupling are apparent in our system due to their shaping of the chemical fields and vice-versa. While such coupling does not qualitatively effect the emergent rigidity and self-propulsion that we have reported here, feedback and time-delay effects from such coupling can give rise to time varying steady-states, such as oscillations and self-folding. Indications of such behavior can already been seen from the trajectories of the motion of dimers (Fig.~\ref{fig2}\textbf{B}, inset and SI Fig.~3\textbf{A}). Experimentally, such states may be accessed via tuning of the monomer propulsion~\cite{ramesh2022interfacial,meredith2020predator} and developing the full theoretical framework for such dynamics is a future challenge.

The emergent dynamics we have reported are in quasi two-dimensional settings. As we have seen, relaxation of this criterion results in a reduction of the rigidity and affects the stability of the stereotypical polymer configurations. For example, even when the height of the Hele-Shaw cell $h>2b$, there is room for extra conformational degrees of freedom for the freely jointed chain. However the fully three-dimensional dynamics of active polymers with chemical and hydrodynamic interactions should hold rich possibilities for future exploration. Further, while we have only explored short linear assemblies here, the extension to longer polymers and higher dimensional assemblies~\cite{manna2022harnessing} with more tunable bonds~\cite{zhang2018multivalent}, complex shapes and thereby emergent dynamics is possible. Finally, multicomponent monomers with attractive, repulsive and even non-reciprocal interactions~\cite{meredith2020predator} may be used to create active self-morphing assemblies.

\section*{Materials and Methods} \label{Materials and Methods}

\subsection{Materials}

We used 5CB (4-cyno-4’-pentylbiphenyl) liquid crystal procured from Frinton Laboratories, Inc. and SDS (sodium dodecyl sulphate) surfactant procured from Sigma Aldrich. DOPC (1,2-dioleoyl-sn-glcero -3-phosphocoline), Liss-Rhod-PE (1,2-dioleoyl-sn-glycero-phosphoethanolamine-N-(lissamine B sulfonyl) (ammonium salt) and biotinyl-cap-PE (1,2-dioleoyl-sn-glycero-phosphoethanolamine-N-(cap biotinyl) (sodium salt) were purchased from Avanti Polar. Alexa Fluor\textregistered 488 conjugated streptavidin was procured from Sigma-Aldrich. FluoSpheres™ carboxylate-modified Microspheres, yellow-green fluorescent (505/515~$nm$) and nile red dye were purchased from Invitrogen (by ThermoFisher Scientific). All the chemicals were used as received.

\subsection{Preparation of (Lipid (DOPC) - surfactant (SDS)) micellar solution} \label{micellar solution}

All lipid stocks were prepared in chloroform: DOPC at 25~$mg/mL$, Biotinyl-cap-PE at 10~$\mu g/mL$, and Liss-Rhod-PE at 10~$\mu g/mL$. We aliquoted 25~$\mu L$ of DOPC from the stock, 2~$\mu L$ of  Biot-Cap-PE, and 10~$\mu L$ of Liss-Rhod-PE in a 2~$mL$ effendorf tube to prepare the aqueous phase. The lipids were vacuum dried overnight and then nitrogen dried before being mixed with the aqueous phase. To the dried lipids containing eppendorf, we added 1~$m L$ of 0.125 (w/v)\%~SDS solution prepared with Milli-Q water. The eppendorf was then left at room temperature (T = 25$\degree$C) for lipid hydration. A probe sonicator (VC 750 (750W), stepped microtip diameter: 3~$mm$) with an on/off pulse of 1.5~s/1.5~s and 32\% of the maximum sonicator amplitude was used to mix the components. The tube was placed in an ice bath during the sonication steps (sonication for 30~s with a 1~min gap) to avoid overheating of the sample. 

\subsection{Droplet production} \label{DP}

We used an oil-in-water (O/W) emulsion system to make the droplets. 5CB liquid crystal droplets (oil phase) were stabilised using micellar solution (surfactant and lipids) in the aqueous phase. When 5CB LC (the oil phase) is injected through a microfluidic channel and encounters the aqueous phase (a micellar solution of lipids and surfactant, SI Fig.~S1 \textbf{A}), adsorption of free lipid-surfactant or micelles results in stable 5CB droplets. The monodisperse droplets generated by the microfluidic device were collected and stored in 0.25\%~SDS solution, where they remained stable for months (SI Fig.~S1 \textbf{C}, \textit{left panel}. The presence of lipids in the monolayer of lipids and surfactant stabilising the 5CB emulsion droplets was confirmed using a red channel of the fluorescence microscopy (SI Fig.~S1 \textbf{C}, \textit{middle panel}). Details of the fabrication of microfluidic devices can be found in the Supplementary Information, SI Fig.~S1 \textbf{A}, and the experimental section (DF). 

\subsection{Preparation of droplet assemblies} \label{Droplet assemblies}
We used a simple method to prepare the 5CB droplet assemblies: we split the sample into two populations. Only biotinylated 5CB droplets were used in population "I" (SI Fig.~S1 \textbf{B, a}) and streptavidin functionalised biotinylated 5CB droplets in population "II" (SI Fig.~S1 \textbf{B, b, C} , \textit{right panel}). To remove free biotinylated lipids and streptavidin from the bulk, both populations (I and II) were washed 2-3 times with 0.25\%~SDS solution. The population "I" is then centrifuged at 6000~rpm for 30~s to settle down all the droplets at the bottom of the tube and remove the solvent from the top. On top of the settled fraction of the droplets, we gently added 200~$\mu L$ of fresh 0.25\%~SDS solution. The tube was then filled with 5~$\mu L$ of a 50~$m M$ NaCl salt solution prepared in 0.25\%~SDS solution, and it was centrifuged for 30~s (at 6000~rpm). We then gently added a 20~$\mu L$ volume of 5CB biotinylated droplets functionalized with streptavidin from population "II" to population "I" at the bottom of the tube (SI Fig.~S1 \textbf{B}). The tube is then centrifuged at 12000~rpm for 35~min in an eppendorf centrifuge (Eppendorf centrifuge model 5418) at T = 15$\degree$C. The assemblies were studied inside a quasi two-dimensional flow cell (SI Fig.~1 \textbf{D, E)}; the details of the flow cell fabrication are discussed in the experimental section of the supporting information~(FlowCell).

\subsection{Imaging (Bright Field and Fluorescence Microscopy)} \label{microscopy}

We used both multi-channel epifluorescence and brightfield microscopy to visualize 5CB droplets and their linear assemblies. The assemblies were imaged on an Olympus IX81 microscope with a 4x, 10x, and 20x UPLSAPO objective, and images were captured using a Photometrics Prime camera. For fluorescence images, we used a CoolLED PE-4000 lamp. A multi-channel 16-bit image of 2048 x 2048 pixels was taken, consisting of a green channel (excitation: 460~$nm$, dichroic: Sem-rock’s Quad Band), a red channel (excitation: 550~$nm$, dichroic: Sem-rock’s Quad Band), and a bright field channel. We recorded the images and videos using Olympus cellSens software.
We used a Leica bright field microscope (model M205FA) equipped with Leica-DFC9000GT camera to record longer time videos of the linear assemblies. Videos are recorded with a microscope zoom of 2, an exposure time of 0.02~s, and a 1~fps frame rate. We captured a 16-bit image of 2048 x 2048 pixels using a Leica-DFC9000GT-VSC06748 camera.

 
\subsection{Acknowledgements}

We acknowledge Abhrajit Laskar for his help with the numerical computation of the flow fields. We acknowledge support from the Department of Atomic Energy (India), under project no.\,RTI4006, the Simons Foundation (Grant No.\,287975), the Human Frontier Science Program and the Max Planck Society through a Max-Planck-Partner-Group. 

\bibliography{Reference}

\end{document}