In this section, we construct a family of graphs and subsequently prove a lower bound on the running time of our algorithm.\\ \vspace{-10pt}

\begin{figure}
    \begin{center}
        \includegraphics[height=1.5in,width=4.5in]{lower_bound.jpg}\vspace{-10pt}
        \caption{Illustration of a worst-case instance for our algorithm.}\label{fig:lowerbound}
    \end{center}
\end{figure}\vspace{-10pt}

\noindent We run our algorithm $\KFProb$ on the graph $D$ (Figure \ref{fig:lowerbound} in the appendix), where $V(D)=\{a_i,b_i,c_i\mid 1\leq i\leq \frac{n}{3}\}$ and $E(D)=\{(a_i,b_i),(b_i,c_i),(c_i,a_i)\mid 1\leq i\leq \frac{n}{3}\}$. We claim that in the worst case, our algorithm takes $\mathcal{O}^*(3^{n/3})$ time to solve \KFProb~on $D$ which we prove via adversarial arguments. Initially since $\phi(v) = 1$ for every vertex of $D$, we have $\psi(a_i)=3$, $\psi(b_i)=3$, $\psi(c_i)=3$. Since $\psi(v)<3.75$ and $\mathcal{S}_{v}\neq\emptyset$ for every vertex, Subroutine 1 and 2 are not applicable. The adversary chooses the vertex $a_i$ to branch on. Since $b_i\in\mathcal{S}_{a_i}$ and $b_i,c_i\in R^+(b_i)$ we get one branch where $a_i$ becomes a sink with potential drop 3. Another where $a_i$ is not a sink, but $b_i$ is, with drop 3. Finally, one where $a_i,b_i$ are not sinks, but $c_i$ is with drop 3. In all of the above branches, $\{a_i,b_i,c_i\}$ is removed, giving a recurrence relation $T(n)=3T(n-3)$ which implies $T(n)$ is $3^{n/3}\geq 1.4422^n$ for this instance. Which gives us the following theorem.\\

\begin{theorem}
    Algorithm $\KFProb$ runs in time $\Omega(1.4422^n)$.
\end{theorem}