In Subroutine $3$ we branch on undecided vertices with a non-empty surviving set and at least one undecided neighbour. Let us define $D_x = \{u\ |\ \exists\ v\in R^+(y),\ v\neq u,\ N^+(v)\subseteq N^+(u)\}$ and $B_x$ to be $R^+(y)\setminus D_x$. We also fix an arbitrary ordering of vertices of $B_x$ and define $NS_i = \{s_{j}\in B_x\ |\ j<i\}$\\

\begin{lemma}\label{lem:subroutine3}
    Let $(D,\phi)$ be an instance such that $\psi(v) < 3.75$, for every vertex in $\mathcal{U}$, let $x$ be a vertex maximizing $\mathcal{U}\cap N(x)$ and $y\in\mathcal{S}_x$. We claim $$\KFProb(D,\phi) = min\{\KFProb(D_x,\phi_x)+|N^+(x)|,\ min_{s_i}\{\KFProb(D_i,\phi_i)+|N^+(s_i)|\}\}$$ where $(D_x,\phi_x) = update(D,\phi,x,-)$ and $(D_i,\phi_i) = update(D,\phi,s_i,NS_i)$ for all $s_i\in B_x$.
\end{lemma}
\begin{proof}
    We claim that $y\in\mathcal{S}_x$ survives in every minimal solution where $x$ is not a sink. We have $y\in S_{min}$ if and only if some in-neighbour of $y$ is a sink in $D-S_{min}$, but $y\in\mathcal{S}_x$ implies  $N^-(y)\cap\mathcal{U} = \{x\}$. Hence, if $x\notin Z_{min}$, then $y\notin S_{min}$ and we can assume that if $x$ is not a sink then $y$ survives and must reach some sink. This sink by definition must be in $R^+(y)$. But if some distinct $u,v$ in $R^+(y)$ satisfy $N^+(v)\subseteq N^+(u)$, then in every minimal solution where $u$ is a sink, $v$ is also a sink and $\KFProb(update(D,\phi,v,-))\leq\KFProb(update(D,\phi,u,-))$. Thus, every possible minimal solution contains at least one element from the set $\{x\}\cup (R^+(y)\setminus D_x) = \{x\}\cup B_x$ in its sink set. Further, we can fix an arbitrary ordering for elements of $B_x$ and branch on the first vertex of this ordering which becomes a sink. In the branch where the $i^{th}$ vertex becomes a sink, we can assume that all the vertices before it are non-sinks. Hence by induction, assuming that $\KFProb(D',\phi')$ returns the optimal solution for every instance smaller than $(D,\phi)$, proves the claim.\qed
\end{proof}

\noindent Since $\psi(x)<3.75$, $N(x)$ has at most two undecided vertices whenever $x$ is undecided. We will analyze the branching vector for this subroutine in two steps. For undecided vertices with two undecided neighbours and for undecided vertices with one undecided neighbour. If $R^+(y)$ is empty, then $x$ must be a sink and no branching is involved. Also, Lemma \ref{lem:neighbourhood} implies $R^+(y)$ can have cardinality at most 2, which leads to the following possibilities. 

\subsection*{$\boldsymbol{B_x = \{s\}}$}\label{1_branch}
    Observe that if $x$ is a sink, then since $x$ has at least three vertices in its closed neighbourhood, at least two of which are undecided, we have $\psi(x)\geq 2.25$. But, if $x$ is not a sink and $s$ is the sink $y$ reaches, then since $s$ has at least two neighbours, and $x\in N^+(s)$ or $x\in R^-(s)$, we have $\psi(s)\geq\phi(N[s]\cup  \{x\})\geq 2.25$. Refer Figure \ref{S3_C1} for an illustration of this case. Hence the worst case branching vector is $(2.25,2.25)$.\\
    \begin{figure}
        \centering
        \includegraphics[height=1.25in,width=1.25in]{S3_C1.png}
        \caption{Case 1: $B_x = \{s\}$}
        \label{S3_C1}
    \end{figure}
    
\subsection*{$\boldsymbol{B_x = \{s_1,s_2\}}$}\label{2_branch}
    In this case, depending on the number of undecided neighbours $x$ has, we have the following cases.\\[-15pt]
    \paragraph*{\textbf{Case 1:} Branching on a vertex with two undecided neighbours}\quad\\[3pt]
    Here $|N(x)\cap\mathcal{U}| = 2$ implies $\phi(N[x]) \geq 3$ and hence $B_x\setminus N(x) = \emptyset$. Otherwise, the vertex in $B_x\setminus N(x)$ would contribute $0.75$ to $\psi(x)$ making it at least $3.75$ which contradicts our assumption that Subroutine 1 is no longer applicable. Now, assuming $y\in\mathcal{S}_x$ is the chosen vertex, we have the following possibilities.
    
    \begin{itemize}
        \item \textbf{Subcase 1.1:} $\boldsymbol{s_2\in R^-(s_1)}$ or $\boldsymbol{s_1\in R^-(s_2)}$\\[3pt]
        The arguments for both cases are identical, hence assume that  $\boldsymbol{s_2\in R^-(s_1)}$.
        \begin{itemize}
            \item[$\bullet$] If $x$ is a sink, then since $x$ has at least three undecided vertices in its closed neighbourhood each contributing 1 giving $\psi(x)\geq 3$.
            \item[$\bullet$] If $x$ is not a sink and $s_1$ is a sink which  reaches in the final solution. Then we have $x\in N(s_1)$, $s_2\in R^-(s_1)$ and hence $\psi(s_1)\geq 3$
            \item[$\bullet$] If $x$ is not a sink, $s_1$ is not a sink in the final solution but $s_2$ is. Then we have $x\in N(s_2)$, $y\in R^-(s_2)$ and $s_1\in R^+(s_2)$. Now if $s_1\in R^-(s_2)$ then $x,s_1$ and $s_2$ each contribute 1 to $\psi(s_2)$. Else, we can assume $y\neq s_1$ and $x,y,s_2$ contributes a total of 2.25 and $s_1$ contributes 0.75 giving $\psi(s_2)\geq 3$. 
        \end{itemize}
        \begin{figure}
            \centering 
            \includegraphics[height=1.25in,width=1.25in]{S3_C2_1_1.png}
            \caption{Case 2.1.1: $s_2\in R^-(s_1)$}
            \label{S3_C2_1_1}
        \end{figure}
        \noindent Hence we have a worst case branching vector $(3,3,3)$. Further, in the rest of the sub-cases, we can assume $s_1\notin R^-(s_2)$ and vice-versa which implies $y\notin \{s_1, s_2\}$. Now consider the case where $s_1\in N^+(x),\ s_2\in N^-(x)$, since $x\notin N^+(s_1)$, either $s_2\in N^+(s_1)$ or $s_2\in R^-(s_1)$. But $s_2\notin R^-(s_1)$ is and $s_2\in N^+(s_1)$ implies $s_1\in R^-(s_2)$, which cannot be the case. Thus this possibility is already considered, which leaves us with the following configurations.\\
        
        \item \textbf{Subcase 1.2:} $\boldsymbol{s_1, s_2\in N^-(x)}$\\[-5pt]
        \begin{itemize}
            \item[$\bullet$] If $x$ is a sink, since $N[x]$ has at least four vertices out of which three are undecided, we have $\psi(x)\geq 3.25$.
            \item[$\bullet$] If $x$ is not a sink and $s_1$ is a sink which $y$ reaches in the final solution. Then we have $x\in N^+(s_1)$, $y\in R^-(s_1)$. Also, since $N^+(s_1)\not\subseteq N^+(s_2)$ and $x\in N^+(s_2)$, $N^+(s_1)$ has at least one more vertex other than $x$, giving $\psi(s_1)\geq 2.5$.
            \item[$\bullet$] If $x$ is not a sink, $s_1$ is not a sink in the final solution, but $s_2$ is. Then we have $x\in N^+(s_2)$, $y\in R^-(s_2)$. Also, since $N^+(s_2)\not\subseteq N^+(s_1)$ and $x\in N^+(s_1)$, $N^+(s_2)$ has at least one more vertex, giving $\phi(R(s_2))\geq 2.5$. Further, since we also have the added information that $s_1$ is not a sink which gives a drop of 0.75, the total drop in potential is at least 3.25. Refer Figure \ref{S3_C2_1_2} for an illustration of this case.
        \end{itemize}
        \begin{figure}
            \centering
            \includegraphics[height=1.25in,width=1.5in]{S3_C2_1_2.png}
            \caption{Case 2.1.2: $s_1, s_2\in N^-(x)$}
            \label{S3_C2_1_2}
        \end{figure}
        \noindent Here we have a worst case branching vector $(2.5,3.25,3.25)$\\
        
        \item \textbf{Subcase 1.3:} $\boldsymbol{s_1, s_2\in N^+(x)}$\\[-5pt]
        \begin{itemize}
            \item[$\bullet$] If $x$ is a sink, since $N[x]$ has at least four vertices out of which three are undecided, we have $\psi(x)\geq 3.25$.
            \item[$\bullet$] If $x$ is not a sink and $s_1$ is a sink which $y$ reaches in the final solution. Then we have $x\in N^-(s_1)$, $y\in R^-(s_1)$ and at least one other vertex in $N^+(s_1)$ giving $\psi(s_1)\geq 2.5$.
            \item[$\bullet$] If $x$ is not a sink, $s_1$ is not a sink in the final solution, but $s_2$ is. Then we have, $x\in N^-(s_2)$, $y\in R^-(s_2)$ and at least one other vertex in $N^+(s_2)$ giving $\psi(s_2)\geq 2.5$. Further, we also have the added information that $s_1$ is not a sink which gives a drop of 0.75. Hence, the total drop in potential is at least 3.25. 
        \end{itemize}
        \begin{figure}
            \centering
            \includegraphics[height=1.2in,width=2in]{S3_C2_1_3.png}
            \caption{Case 2.1.3: $s_1,s_2\in N^+(x)$}
            \label{S3_C2_1_3}
        \end{figure}
        \noindent Here we have a worst case branching vector $(3.25,2.5,3.25)$.\\
    \end{itemize}
    \paragraph*{\textbf{Case 2:} Branching on a vertex with one undecided neighbour}\quad\\[3pt]
    Here, $|N(x)\cap\mathcal{U}| = 1$ implies $\phi(N[x]) \geq 2.25$ and hence, $|B_x\setminus N(x)| \leq 1$. Otherwise, the vertices in $B_x\setminus N(x)$ would contribute $0.75$ to $\psi(x)$ making it at least $3.75$ which contradicts our assumption that Subroutine 1 is no longer applicable. Now, assuming $s_1\in N(x)$ and $y\in\mathcal{S}_x$ is the chosen vertex, we have the following possibilities.
    
    \begin{itemize}
        \item \textbf{Subcase 2.1:} $\boldsymbol{s_1\in R^-(s_2)}$ or $\boldsymbol{s_2\in R^-(s_1)}$\\[3pt]
        The arguments for both the case are identical, hence we assume $s_1\in R^-(s_2)$. \\[-8pt]
        \begin{itemize}
            \item[$\bullet$] If $x$ is a sink, then since $x$ has at least three vertices in its closed neighbourhood exactly two of which are undecided and $s_2\in R^+(x)$, we get $\psi(x)\geq 3$.
            \item[$\bullet$] If $x$ is not a sink and $s_1$ is a sink which $y$ reaches in the final solution. Then we have at least three vertices in $N[s_1]$ out of which only $x,\ s_1$ are undecided. This and $s_2\in R^+(s_1)$ give $\psi(s_1)\geq 3$.
            \item[$\bullet$] If $x$ is not a sink, $s_1$ is not a sink in the final solution but $s_2$ is. Then we have $x\in R^-(s_2)$ and $s_1\in R^-(s_2)$. Now, $s_2$ has an out-neighbour which cannot be $x$ since $x$ has only one undecided neighbour and $s_1$ since $s_2\in R^+(s_1)$. This out-neighbour contributes at least $0.25$ making the total drop at least $3.25$. Refer Figure \ref{S3_C2_2_1_a} and \ref{S3_C2_2_1_b} for examples.\\
        \end{itemize}
        Hence the worst case branching vector is $(3,3,3.25)$.\\
        \begin{figure}
            \centering
            \begin{minipage}{.5\textwidth}
                \centering
                \includegraphics[height=1in,width=2in]{S3_C2_2_1_a.png}
                \caption{Case 2.2.1: $s_1\in R^-(s_2)$}
                \label{S3_C2_2_1_a}
            \end{minipage}%
            \begin{minipage}{.5\textwidth}
                \centering
                \includegraphics[height=1.25in,width=2in]{S3_C2_2_1_b.png}
                \caption{Case 2.2.1: $s_2\in R^-(s_1)$}
                \label{S3_C2_2_1_b}
            \end{minipage}
        \end{figure}
        
        \item \textbf{Subcase 2.2:} $\boldsymbol{s_2\notin R^-(s_1)\ \&\ s_1\notin R^-(s_2)}$\\[5pt]
        Here, $s_1\notin N^-(x)$, since in that case $s_1\in R^-(s_2)$ or $s_1\in N^+(s_2)$ which implies $s_2\in R^-(s_1)$. Also, $y\neq s_1$ since otherwise $s_1\in R^-(s_2)$.\\[-8pt]
        \begin{itemize}
            \item[$\bullet$] If $x$ is a sink, then since $x$ has at least two out-neighbours $y,s_1$ and some in-neighbour $z$, along with $s_2\in R^+(x)$ we have $\psi(x)\geq 3.25$. 
            \item[$\bullet$] If $x$ is not a sink and $s_1$ is a sink which $y$ reaches. Then we have $x,y$ in $R^-(s_1)$ and since $s_1$ has some other out-neighbour, we get $\psi(s_1)\geq 2.5$.
            \item[$\bullet$] If $x$ is not a sink, $s_1$ is not a sink in the final solution but $s_2$ is. Then we have, $x,y,z$ in $R^-(s_2)$ which gives $\psi(s_2)\geq 2.5$. Further, since we also have the added information that $s_1$ is not a sink which gives a drop of 0.75, the total drop in potential is at least 3.25. 
        \end{itemize}
        \begin{figure}
            \centering
            \includegraphics[height=1.25in,width=2in]{S3_C2_2_2.png}
            \caption{Case 2.2.2: $s_1,s_2\in N^+(x)$}
            \label{S3_C2_2_2}
        \end{figure}
        \noindent Hence the worst case branching vector is $(3.25,2.5,3.25)$.
    \end{itemize}