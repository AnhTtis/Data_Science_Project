In Subroutine $1$ we branch on an undecided vertex $x\in V(D)$, which has $\psi(x) \geq 3.75$. The following lemma proves the correctness of the subroutine. 

\begin{lemma}\label{lem:subroutine1}
    If $\ \exists\ x\in V(D)$ such that $x\in\mathcal{U}$ and $\psi(x)\geq 3.75$, then $\KFProb(D,\phi)=\min\{\KFProb(D_1,\phi_1)+|N^+(x)|,\text{ }\KFProb(D_2,\phi_2)\}$ where, $(D_1,\phi_1) = update(D,\phi,x,-)$ and $(D_2,\phi_2) = update(D,\phi,-,x)$.
\end{lemma}
\begin{proof}
    We prove the lemma using an inductive argument on the potential of the instance $(\phi(D))$. Observe that if we consider the base case $\phi(D) = 1$, then there is only one undecided vertex in the input instance and the recurrence holds true. Now given an instance $D$, assume that the algorithm computes the correct solution for all smaller instances. Let the solutions for $\KFProb(D_1,\phi_1)$ and $\KFProb(D_2,\phi_2)$ be $S_1$ and $S_2$, respectively. Assuming $S_{opt}$ is an optimal solution for $\KFProb(D,\phi)$, we evaluate the two possibilities: 
    \begin{itemize}
        \item \textbf{Case 1:} $\boldsymbol{x\in Z_{opt}}$.
           We show that $S_1\cup N^+(x)$ is an optimal solution for  ${\KFProb(D,\phi)}$ if and only if  $S_1$ is an optimal solution for ${\KFProb(D_1,\phi_1)}$. The arguments are exactly the same as that in Corollary \ref{cor2}. We also claim that $\KFProb(D_2,\phi_2) \geq \KFProb(D_1,\phi_1)+|N^+(x)| $. For contradiction, suppose that is not the case, then any optimal solution $S_2$ for ($D_2,\phi_2$) is also a feasible solution for $\KFProb(D,\phi)$. But $S_2$ has size strictly smaller than $S_{opt}$, which contradicts our assumption that it is optimal. Hence, $\KFProb(D,\phi)=$ $\min\{\KFProb$ $(D_1,\phi)+|N^+(x)|,\text{ }\KFProb(D,\phi_2)\}$.\vspace*{2mm}
        \item \textbf{Case 2:} $\boldsymbol{x\notin Z_{opt}}$.
            In this case, $\KFProb (D_2,\phi_2)=\KFProb(D,\phi)$, by definition. Also $\KFProb(D_2,\phi_2) \leq \KFProb(D_1,\phi_1)+|N^+(x)| $, otherwise we have $S'=S_1\cup N^+(x)$ as a solution with size strictly smaller than $S_{opt}$ which contradicts our assumption that it is optimal. Hence, $\KFProb(D,\phi)=$ $\min\{\KFProb(D_1,\phi)+|N^+(x)|,\text{ } \KFProb(D,\phi_2)\}$.
    \end{itemize}\qed
\end{proof}

In Subroutine $1$, we get a branching vector $(3.75,0.75)$. After exhaustively running Subroutine $1$, every remaining undecided vertex satisfies $\psi(x)\leq 3.75$. An important implication of this bound is the restricted number of undecided vertices in $N(x)$ as well as $\mathcal{C}_x$, which we prove in the 
 following lemma.

\begin{lemma}\label{lem:neighbourhood}
    If $(D,\phi)$ is an instance on which Subroutine 1 is no longer applicable, then every $x\in\mathcal{U}$ satisfies, $|\mathcal{C}_x|\leq 2$.
\end{lemma}
\begin{proof}
    If Subroutine 1 is no longer applicable to $(D,\phi)$, then every undecided vertex has at most 2 undecided neighbours, since $N(x)\subseteq R(x)$ and $\phi(R(x))$ is counted towards $\psi(x)$.\\
    To begin with, consider the possibility that, $|R^+(x)\setminus R(x)| \geq 3$. Observe that while computing $\psi(x)$, vertices of $|R^+(x)\setminus R(x)|$ contribute a total of $2.25$, the potential of $x$ contributes $1$ and the in-neighbour and out-neighbour of $x$ has to contribute atleast $0.5$. This adds up to a total of $3.75$ which is not allowed since Subroutine 1 is no longer applicable. Hence $|R^+(x)\setminus R(x)| \leq 2$. Also, recall that $\mathcal{C}_x\subseteq N^-(x)\cup R^+(x)\subseteq R(x)\cup R^+(x)$. Now we can consider the following possibilities.
    \begin{itemize}
        \item \textbf{Case 1:} $\boldsymbol{|R^+(x)\setminus R(x)| = 0}$. Here $\mathcal{C}_x\subseteq R(x)$. Also $\phi(R(x))$ is less than $3.75$, out of which $x$ contributes 1. Hence $R(x)$ can have at most 2 more undecided vertices and consequently, $|\mathcal{C}_x| \leq 2$.\vspace{-5pt}\\
        \item \textbf{Case 2:} $\boldsymbol{|R^+(x)\setminus R(x)| = 1}$. In this case, since $R^+(x)\setminus R(x)$ contributes $0.75$, the contribution of $R(x)$ to  $\psi(x)$ has to be less than $3$. Since $x$ itself contributes 1, we can have at most one other undecided vertex in $R(x)$. Hence $|\mathcal{C}_x| \leq 2$.\vspace{-5pt}\\
        \item \textbf{Case 3:} $\boldsymbol{|R^+(x)\setminus R(x)| = 2}$. Here, vertices in $R^+(x)\setminus R(x)$ contribute a total of $1.5$ to $\psi(x)$ which gives $\phi(R(x)) < 2.25$. Hence if $|\mathcal{U}\cap R(x)|\geq 2$ then we get $|R(x)| = 2$ which is not possible since $d^+(x), d^-(x)\geq 1$. Hence $\mathcal{U}\cap R(x)=\{x\}$ and $|\mathcal{C}_x| \leq 2$.
    \end{itemize}\qed
\end{proof}