%!TEX root = main.tex

In this section, we provide an exact exponential algorithm (Algorithm \ref{alg_1}) to compute the minimum knot-free vertex deletion set. We begin by initialising the potential of all vertices to $1$ and as soon as we decide a vertex to be non-sink we reduce its potential to $0.25$. In our algorithm, whenever we encounter a sink or a source, we remove them using Reduction Rules \ref{rr2} and \ref{rr3}. If all the vertices have potential $0.25$, then we apply Reduction Rule \ref{rr1} to solve the instance in polynomial time. 

At any point, if there exists an undecided vertex $x$ of potential $\psi(x)\geq 3.75$ then we branch on the possibility of it being a sink or non-sink in the optimal solution. Here we benefit from the high potential drop in the branch where $x$ becomes a sink which gives us a branching factor of $(3.75,0.75)$. Once such vertices are exhausted, we choose an undecided vertex $x$ with the \emph{maximum} number of undecided neighbours to branch on. We show that if $x$ is not a sink in the optimal solution then some other vertex $s$ from $\mathcal{C}_{x}$ has to be. Further, the bound on $\psi(x)$ helps us limit the cardinality of $\mathcal{C}_{x}$ and consequently the number of branches. In this case, we get a set of vertices from which atleast one has to be a sink in the final solution. Since, each branch has some vertex becoming a sink, the potential drop will be high enough to give a \emph{good} running time for our algorithm.