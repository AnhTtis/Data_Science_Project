\documentclass[runningheads]{llncs}
\usepackage[lined,boxed,commentsnumbered,ruled,vlined,noend,linesnumbered,boxed]{algorithm2e}
\usepackage{float}
\usepackage{amssymb}
\usepackage{amsmath}
\usepackage{mathtools}
\usepackage{relsize}
\usepackage{todonotes}
\usepackage{hyperref}
\DeclarePairedDelimiter{\ceil}{\lceil}{\rceil}
\DeclarePairedDelimiter{\floor}{\lfloor}{\rfloor}
\usepackage{framed}
\usepackage{url}
\usepackage{cite}
\usepackage{comment}
\usepackage{tcolorbox}
\tcbuselibrary{skins}
\usepackage{xcolor}
\colorlet{mix}{red!50!black}
\usepackage{hyperref}
\hypersetup{colorlinks={true},linkcolor={blue},citecolor=blue}
\newtheorem{rr}{Reducion Rule}
\usepackage{enumitem}
\usepackage{comment}
\usepackage{tikz}
\usepackage{epsfig}
\usepackage{wrapfig}
\usepackage{xspace}
\usetikzlibrary{shapes,backgrounds, patterns}
\usepackage{lipsum}
\usepackage[symbol]{footmisc}
\renewcommand{\thefootnote}{\fnsymbol{footnote}}
\usepackage{comment}

\newcommand{\KFProb}[0]{{\sf KFVD}\xspace}

\bibliographystyle{plain}

\title{An Improved  Exact Algorithm for Knot-Free Vertex Deletion}
\titlerunning{Knot-Free Vertex Deletion}

\author{Ajaykrishnan E S\inst{1}
%\thanks{} 
\and Soumen Maity\inst{1}
%\thanks{}
\and Abhishek Sahu\inst{2}
%\thanks{}
\and  \\Saket Saurabh\inst{3,4} 
%\thanks{}
}

\authorrunning{E S.\,Ajaykrishnan et al.}
% First names are abbreviated in the running head.
% If there are more than two authors, 'et al.' is used.

\institute{Indian Institute of Science Education and Research, Pune, India \and National Institute of Science Education and Research, An OCC of Homi Bhabha National Institute, Bhubaneswar, India \and The Institute of Mathematical Sciences, Chennai, India \and University of Bergen, Bergen, Norway\\
\email{\texttt{ajaykrishnan.es@students.iiserpune.ac.in}};
\email{\texttt{soumen@iiserpune.ac.in}};
\email{\texttt{abhisheksahu@niser.ac.in}};
\email{\texttt{saket@imsc.res.in}}\\
}

%\ccsdesc{} 
%\Copyright{}
%\relatedversion{} 

\colorlet{bscolor}{blue}

\newcommand{\bscomment}[1]{\textcolor{bscolor}{BS:#1}}


\DeclareUnicodeCharacter{B0}{\textless}

\begin{document}
\maketitle

\begin{abstract}
    

Over the past few years, there has been a significant amount of research focused on studying the ReLU activation function, with the aim of achieving neural network convergence through over-parametrization. However, recent developments in the field of Large Language Models (LLMs) have sparked interest in the use of exponential activation functions, specifically in the attention mechanism.

Mathematically, we define the neural function $F: \R^{d \times m} \times  \mathbb{R}^d \rightarrow \mathbb{R}$ using an exponential activation function. Given a set of data points with labels $\{(x_1, y_1), (x_2, y_2), \dots, (x_n, y_n)\} \subset \mathbb{R}^d \times \mathbb{R}$ where $n$ denotes the number of the data. Here $F(W(t),x)$ can be expressed as $F(W(t),x) := \sum_{r=1}^m a_r \exp(\langle w_r, x \rangle)$, where $m$ represents the number of neurons, and $w_r(t)$ are weights at time $t$. It's standard in literature that $a_r$ are the fixed weights and it's never changed during the training. We initialize the weights $W(0) \in \mathbb{R}^{d \times m}$ with random Gaussian distributions, such that $w_r(0) \sim \mathcal{N}(0, I_d)$ and initialize $a_r$ from random sign distribution for each $r \in [m]$.

Using the gradient descent algorithm, we can find a weight $W(T)$ such that $\| F(W(T), X) - y \|_2 \leq \epsilon$ holds with probability $1-\delta$, where $\epsilon \in (0,0.1)$ and $m = \Omega(n^{2+o(1)}\log(n/\delta))$. To optimize the over-parametrization bound $m$, we employ several tight analysis techniques from previous studies [Song and Yang arXiv 2019, Munteanu, Omlor, Song and Woodruff ICML 2022]. 

 

    \keywords{exact algorithm, knot-free graphs, branching algorithm, measure and conquer}
\end{abstract}

\section{Introduction}
\section{Introduction}
\label{sec:introduction}
% \begin{itemize}
%     % Diffusion of FL
%     \item {\st{Diffusion of FL}}
%     % Security threats to FL
%     \item {\st{Security threats to FL with particular focus on model poisoning}}
%     % Limitations of existing countermeasures
%     \item {\st{Current countermeasures (e.g., KRUM) and their limitations}}
%     % Proposed method and its advantages
%     \item {\st{Intuitive description of the proposed method and its difference (i.e., advantages) w.r.t. state of the art}}
%     % Main contributions
%     \item {\st{Summary of the main contributions of this work}}
%     % Paper's structure and organization
%     \item {\st{Paper's structure and organization}}
% \end{itemize}

% Diffusion of FL
Recently, {\em federated learning} (FL) has emerged as the leading paradigm for training distributed, large-scale, and privacy-preserving machine learning (ML) systems~\cite{mcmahan2017googleai,mcmahan2017aistats}. 
The core idea of FL is to allow multiple edge clients to collaboratively train a shared, global model without disclosing their local private training data.
%Specifically, an FL system consists of a central server and many edge clients; 
A typical FL round involves the following steps: {\em(i)} the server randomly picks some clients and sends them the current, global model; {\em(ii)} each selected client locally trains its model with its own private data; then, it sends the resulting local model to the server;\footnote{Whenever we refer to global/local model, we mean global/local model {\em parameters}.} {\em(iii)} the server updates the global model by computing an \emph{aggregation function}, usually the average (FedAvg), on the local models received from clients.
% \begin{enumerate}
%     \item[{\em(i)}] the server sends the current, global model to the clients and appoints some of them for training;
%     \item[{\em(ii)}] each selected client locally trains its copy of the global model with its own private data; then, it sends the resulting local model back to the server;\footnote{Whenever we refer to global/local model, we mean global/local model {\em parameters}.}
%     \item[{\em(iii)}] the server updates the global model by computing an \emph{aggregation function} on the local models received from clients (by default, the average, also referred to as FedAvg~\cite{mcmahan2017aistats}).
% \end{enumerate}
This process goes on until the global model converges. %(e.g., after a certain number of rounds or other similar stopping criteria).
%\\
% The advantages of FL over the traditional, centralized learning paradigm are undoubtedly clear in terms of flexibility/scalability (clients can join/disconnect from the FL network dynamically), network communications (only model weights\footnote{We will use \textit{parameters} and \textit{weights} interchangeably.} are exchanged between clients and server), and privacy (each client's private training data is kept local at the client's end and not uploaded to the server).
\\
% Security threats to FL
%However, the growing adoption of FL also raises security concerns~\cite{costa2022covert}, particularly about its confidentiality, integrity, and availability.
Although its advantages over standard ML, FL also raises security concerns~\cite{costa2022covert}. %, particularly about its confidentiality, integrity, and availability~\cite{costa2022covert}.
% OLD, LONG VERSION
% Indeed, some work deals with privacy leakage that may expose the local data of some clients~\cite{melis2019sp}. 
% A large body of work, instead, investigates attacks that usually aim to detriment the predictive accuracy of the learned global model. For instance, \emph{data poisoning} attacks achieve this goal by letting an adversary pollute the training set of some corrupt FL clients with maliciously crafted examples~\cite{jagielski2018sp}.
% Similarly, in \emph{model poisoning} the attacker attempts to tweak the global model weights~\cite{bhagoji2019pmlr} by directly perturbing the local model's weights of some infected FL clients before these are sent to the central server for aggregation, usually via so-called Byzantine attacks. 
% It turns out that Byzantine model poisoning attacks severely impact standard FedAvg; therefore, more robust aggregation functions must be designed to make FL systems secure.
Here, we focus on \emph{untargeted model poisoning} attacks~\cite{bhagoji2019pmlr}, where an adversary attempts to tweak the global model weights %\footnote{We will use the terms \textit{parameters} and \textit{weights} interchangeably.} 
by directly perturbing the local model's parameters of some infected clients before these are sent to the central server for aggregation.
In doing so, the adversary aims to jeopardize the global model \textit{indiscriminately} at inference time.
Such model poisoning attacks severely impact standard FedAvg; therefore, more robust aggregation functions must be designed to secure FL systems.
\\
% In this paper, we focus on designing a novel robust aggregation scheme at the server's end to contrast the effect of Byzantine model poisoning attacks.
%
% Current countermeasures and their limitations
%Several countermeasures have been proposed in the literature to combat model poisoning attacks on FL systems.
% Some methods use simple statistics more robust than plain average to smooth the impact of malicious updates (e.g., Trimmed Mean and FedMedian~\cite{yin2018icml}). 
% Other defenses implement outlier detection techniques to discard malicious updates from the aggregation performed at the server's end. Those are either based on heuristics (e.g., Krum/Multi-Krum~\cite{blanchard2017nips} and Bulyan~\cite{mhamdi2018pmlr}) or data-driven approaches (e.g., K-means clustering~\cite{shen2016acm} or DnC via spectral analysis~\cite{shejwalkar2021ndss}). 
% Finally, some strategies rely on a centralized ``source of trust'' to spot potential malicious updates (e.g., FLTrust~\cite{cao2020fltrust}).
% Several countermeasures have been proposed in the literature to combat model poisoning attacks on FL systems, i.e., to discard possible malicious local updates from the aggregation performed at the server's end. 
% These techniques range from simple statistics more robust than plain average (e.g., Trimmed Mean and FedMedian~\cite{yin2018icml}) to outlier detection heuristics (e.g., Krum/Multi-Krum~\cite{blanchard2017nips} and Bulyan~\cite{mhamdi2018pmlr}) or data-driven approaches (e.g., spectral analysis via K-means clustering~\cite{shen2016acm} or spectral analysis), or methods based on ``source of trust'' (e.g., FLTrust~\cite{cao2020fltrust}).
% OLD, LONG VERSION
%Several countermeasures have been proposed in the literature to combat Byzantine model poisoning attacks on FL systems.
% Descriptive statistics
% For example, Trimmed Mean and FedMedian aggregate local model updates using more robust statistics than standard average~\cite{yin2018icml}.
%
% % Heuristics for outlier detection
% Many existing Byzantine-resilient strategies implement some outlier detection heuristics to discard the model updates sent by potentially malicious clients from the input of the aggregation function.
% One of the most popular heuristics is Krum~\cite{blanchard2017nips}.
% This strategy tries to mitigate the impact of Byzantine attacks by selecting as a global model the local model with the smallest sum of Euclidean distances to {\em all} the other local models.
% Although powerful, Krum requires the server to know (or, at least, estimate) the number of malicious FL clients upfront, which is generally impossible in a realistic attack scenario. %
% Moreover, Krum may become ineffective for complex, high-dimensional model parameter spaces due to the curse of dimensionality.
% Bulyan~\cite{mhamdi2018pmlr} tries to overcome this issue by combining Krum with a variant of Trimmed Mean.
% % Data-driven outlier detection
% Other strategies use data-driven outlier detection techniques -- e.g., via K-means clustering~\cite{shen2016acm} -- to spot potential malicious local model updates. 
% %For instance, Shen et al. propose to cluster local model updates with K-means and thus identify outliers.
%
% % Other techniques
% As far as the server is concerned, any local model received can be from a potential malicious client. 
% FLTrust~\cite{cao2020fltrust} assumes the server acts as a client, i.e., trains a local model on an additional {\em trustworthy} dataset at the server's end and compares it against all the local models from other clients. 
% This way, the server can rely on some ``source of trust'' when discarding potentially malicious clients.
%\\
% Limitations of existing Byzantine-resilient strategies
Unfortunately, existing defense mechanisms either rely on simple heuristics (e.g., Trimmed Mean and FedMedian by~\cite{yin2018icml}) or need strong and unrealistic assumptions to work effectively (e.g., foreknowledge or estimation of the number of malicious clients in the FL system, as for Krum/Multi-Krum~\cite{blanchard2017nips} and Bulyan~\cite{mhamdi2018pmlr}, which, however, cannot exceed a fixed threshold).
Furthermore, outlier detection methods using K-means clustering~\cite{shen2016acm} or spectral analysis like DnC~\cite{shejwalkar2021ndss} do not directly consider the temporal evolution of local model updates received.
Finally, strategies like FLTrust~\cite{cao2020fltrust} require the server to collect its own dataset and act as a proper client, thereby altering the standard FL protocol.
\\
% OLD, LONG VERSION
% Overall, existing Byzantine-resilient strategies are either simple heuristics (e.g., FedMedian) or, if they are more complex, they rely on strong and unrealistic assumptions to work effectively (e.g., knowing the number of malicious clients in the FL system in advance, as for Krum and alike).
% Furthermore, data-driven outlier detection methods do not consider the temporary evolution of local model updates received (e.g., K-means clustering). 
% Finally, strategies like FLTrust requires the server to collect its own dataset and act as a proper client, thereby altering the standard FL protocol.
%
% Description of the proposed method
This work introduces a novel pre-aggregation \textit{filter} robust to untargeted model poisoning attacks. Notably, this filter $(i)$ operates without requiring prior knowledge or constraints on the number of malicious clients and $(ii)$ inherently integrates temporal dependencies. 
The FL server can employ this filter as a preprocessing step before applying \textit{any} aggregation function, be it standard like FedAvg or robust like Krum or Bulyan.
Specifically, we formulate the problem of identifying corrupted updates as a multidimensional (i.e., matrix-valued) time series anomaly detection task. 
The key idea is that legitimate local updates, resulting from well-calibrated iterative procedures like stochastic gradient descent (SGD) with an appropriate learning rate, show \textit{higher predictability} compared to malicious updates. This hypothesis stems from the fact that the sequence of gradients (thus, model parameters) observed during legitimate training exhibit regular patterns, as validated in Section~\ref{subsec:intuition}. %until convergence. 
%This regularity may be more pronounced for smooth convex loss functions, but it can still be captured within an appropriate time window, even for more complex and convoluted loss surfaces. 
%We provide evidence of this claim in Appendix~B, where we show that the average mutual information (i.e., ``predictability''), calculated over pairs of legitimate model updates sent at different FL rounds, is significantly higher than the corresponding computation for a malicious client.
\\
Inspired by the matrix autoregressive (MAR) framework for multidimensional time series forecasting~\cite{chen2021je}, we propose the FLANDERS ({\em \textbf{F}ederated \textbf{L}earning meets \textbf{AN}omaly \textbf{DE}tection for a \textbf{R}obust and \textbf{S}ecure}) filter.
The main advantages of FLANDERS over existing strategies like FLDetector~\cite{zhao2020multivariate} are its resilience to large-scale attacks, where $50\%$ or more FL participants are hostile, and the capability of working under realistic non-iid scenarios.
We attribute such a capability to two key factors: $(i)$ FLANDERS works without knowing a priori the ratio of corrupted clients, and $(ii)$ it embodies temporal dependencies between intra- and inter-client updates, quickly recognizing local model drifts caused by evil players. Below, we summarize our main contributions:

\begin{itemize}
\item[{\em(i)}]
We provide empirical evidence that the sequence of models sent by legitimate clients is more predictable than those of malicious participants performing untargeted model poisoning attacks.
\\
\item[{\em(ii)}] 
We introduce FLANDERS, the first pre-aggregation filter for FL robust to untargeted model poisoning based on multidimensional time series anomaly detection.
\\
\item[{\em(iii)}] 
We integrate FLANDERS into Flower,\footnote{\scriptsize{\url{https://flower.dev/}}} a popular FL simulation framework for reproducibility.
\\
\item[{\em(iv)}] 
We show that FLANDERS improves the robustness of the existing aggregation methods under multiple settings: different datasets, client's data distribution (non-iid), models, and attack scenarios.
\\
\item[{\em(v)}] 
We publicly release all the implementation code of FLANDERS along with our experiments.\footnote{\scriptsize{\url{https://anonymous.4open.science/r/flanders_exp-7EEB}}}
\end{itemize}

% Paper's structure and organization
The remainder of the paper is structured as follows. %some related work and the current state-of-the-art solutions to security issues that FL entails. 
Section~\ref{sec:background} covers background and preliminaries. 
In Section~\ref{sec:related}, we discuss related work.
Section~\ref{sec:problem} and Section~\ref{sec:method} describe the problem formulation and the method proposed. % to tackle it. 
Section~\ref{sec:experiments} gathers experimental results. %, and Section~\ref{sec:limitations} discusses some limitations of this work.
Finally, we conclude in Section~\ref{sec:conclusion}.
 %discusses the limitations of this work and draws future research directions.
%reports conclusions and draws perspectives for future research directions.

%%%%%%% OLD %%%%%%%
%to overcome the resilience of Byzantine failures in distributed Stochastic Gradient Descent computations. 
% The strength of Krum is its time complexity, which is linear in the gradient dimension. 
% However, the robustness of the approach is guaranteed for gradient-based learning applications only when the majority of the clients are not compromised. 
% Besides, the aggregation mechanism of Krum, as well as that of similar methods, is robust from a coarse-grained perspective and does not provide solutions to errors and perturbations that may occur at inference time.
%A related approach to~\cite{blanchard2017nips} is the work of Su et al.~\cite{su2016dc}. Here, the authors propose an iterated approximate agreement to tackle a multi-layer scenario attacked by Byzantine agents. 
%However, the method works efficiently on the sole discrete context and it is inapplicable to continuous state environments.
%\gabri{Maybe, we should just talk about the main limitations of existing countermeasures without digging into their details (or, we can just mention Krum as this is the most popular one). I will move the description of all these methods to the Related Work section.}
\section{Preliminaries  and Auxiliary Results}
In this section we state some commonly used definitions, notations and useful auxiliary results. We also formalize the potential function which is integral to our algorithm and finally state and prove a few reduction rules which are used throughout the paper.\\

\noindent 
{\bf Notation:}  
For a set $S \subseteq V(D)$, $G[S]$ denotes the subgraph of $D$ induced on $S$ and $G[D-S]$ denotes the subgraph induced on $V(D)\setminus S$. A {\em path} $P_{(u,w)}$ from $u$ to $w$ of length $\ell$ is a sequence of distinct vertices $v_1,v_2,\ldots,v_{\ell}$ such that $(v_i,v_{i+1})$ is an arc, for each $i,\ i\in [\ell-1]$ and $v_1=u$, $v_\ell=w$. We define the \emph{in-reachability set} of a vertex $v$ denoted by $R^-(v)$, as the set of vertices that can reach $v$ via some directed path in $D - N^+(v)$. Notice that $v\in R^-(v)$. 
We define $R(v)=N^+(v)\cup R^-(v)$. For graph-theoretic terms and definitions not stated explicitly here, we refer to \cite{diestel-book}. 

\subsection{Auxiliary Results}
In this subsection, we first state some of the known reduction rules and some new ones that we use in our branching algorithm. 

\begin{proposition}{\rm\cite{DBLP:conf/iwpec/BessyBCPS19}\label{prop:1}}
    A digraph $D$ is knot-free if and only if for every vertex $v$ of $D$, $v$ has a path to a sink.
\end{proposition}

\begin{corollary}{\rm\cite{DBLP:conf/iwpec/BessyBCPS19}\label{cor1}}
    For any minimal solution $S\subseteq V(D)$ with the set of sink vertices $Z$ in $D-S$, we have $N^+(Z) = S$. 
\end{corollary}
  
Proposition \ref{prop:1} and Corollary \ref{cor1} imply that given a digraph $D$, the problem of finding a set of sink vertices $Z$ such that every vertex in $V(D)-N^+(Z)$ has a directed path to a vertex in $Z$ and  $|N^+(Z)|$ is minimum; this is equivalent to the \textsc{Knot-Free Vertex Deletion} ({\KFProb}) problem. Therefore, our algorithm aims to find the set of sink vertices $Z$ corresponding to an optimal solution, while minimizing $|N^+(Z)|$ instead of directly finding the deletion set. \\

\noindent
{\bf Strategy of our Algorithm.}  The algorithm expands on the ideas used in \cite{ramanujan2022exact} where the algorithm branches on the possibility that a vertex $v\in V(D)$ is either a sink or a non-sink vertex in some optimal solution. In the branch where we conclude $v$ to be a non-sink vertex there are two possibilities, $v$ is either in the deletion set or not. To track this additional information that $v$ is non-sink, we use a potential function $\phi$ for $V(D)$ defined as follows.
  
\begin{definition}[Potential function]
    Given a digraph $D=(V,E)$, we define a \emph{potential function} on $V(D)$, $\phi:V(D)\rightarrow \{0.25,1\}$ such that $\phi(v)=1$, if $v$ is a potential vertex to become a sink in an optimal solution and $\phi(v)=0.25$, if $v$ is a non-sink vertex. For any subset $V'\subseteq V(D)$, $\phi(V')=\sum_{x\in V'} \phi(x)$. We call a vertex $v$ as an \emph{undecided} vertex if $\phi(v)=1$ and a \emph{semi-decided} vertex if $\phi(v)=0.25$. 
\end{definition}

\begin{definition}[Feasible solution]
    A set $S\subseteq V(D)$ is called a feasible solution for $(D,\phi)$ if $D-S$ is knot-free and for any sink vertex $s$ in $D-S$, $\phi(s)=1$. ${\KFProb}(D,\phi)$ denotes the size of an optimal solution for $(D,\phi)$. 
\end{definition}

\noindent To solve the \textsc{Knot-Free Vertex Deletion} problem on a digraph $D$, we initialize the potential values of all vertices to $1$. As soon as we decide a vertex to be a non-sink vertex, we drop its potential by $0.75$. Any vertex whose potential is $0.25$ cannot become a sink in the final knot-free graph corresponding to an optimal feasible vertex deletion set of $(D,\phi)$.

\noindent Now we are ready to define the \emph{out-reachability set} of $v$ which is the set of \emph{undecided} vertices which are reachable from $v$ even after deleting their out-neighbours, denoted as $R^+(v)$. Note that, $u \in R^+(v)$ if and only if $v\in R^-(u)$. We shall use $S_{min}$ to denote a minimal solution for the given instance $(D,\phi)$ and $Z_{min}$ for the set of sinks in $D-S_{min}$. Similarly, we use $S_{opt}$ and $ Z_{opt}$ to refer to an optimal solution and its sink set respectively. We further define an update function to make our algorithm description concise.\\
\vspace{-20pt}
\begin{algorithm}[ht!]
	\SetAlgoLined
	\SetKwData{I}{I}\SetKwData{size}{size}\SetKwData{clrr}{cLRR}\SetKwData{Stop}{Stop}\SetKwData{sizec}{size(cLRR)}
	\small 
	\vspace{4pt}
    \KwIn{A directed graph $D$, a potential function $\phi$, a sink vertex $s$ (optional) and a set  of non-sink vertices $NS$ (optional)}
    \KwOut{ An updated digraph $D'$ and an updated potential function $\phi'$}\vspace*{2mm}
    \hrule
    \label{helper1}\vspace*{2mm}
    \caption{$update(D,\phi,s,NS)$}
    $D' = D$, $\phi' = \phi$\\
    \If {\text{input} s \text{is provided}} {
        $D'= D' - R(s)$\\
        \For{v $\in R^+(s)$ }{
            $\phi'(v) = 0.25$}}
    \If {input NS is provided} { 
        \For{v $\in$ NS}{
            $\phi'(v) = 0.25$}}
    \Return{$D', \phi'$}\vspace*{2mm}
\end{algorithm}\\[-37pt]

\begin{rr}{\rm\cite{ramanujan2022exact}}\label{rr1}
    If all the vertices in $D$ are semi-decided and $D$ has no source or sink vertices, then $V(D)$ is contained inside any feasible solution for $(D,\phi)$. 
\end{rr}

\begin{rr}{\rm\cite{ramanujan2022exact}}\label{rr2} 
    Let $v\in D$ be such that $N^-(v)=\emptyset$, $S$ is an optimal solution for $D$ iff $S$ is an optimal solution for $D' = D-v$.
\end{rr}

\begin{rr}{\rm\cite{ramanujan2022exact}}\label{rr3} 
    Let $v\in D$ be such that $N^+(v)=\emptyset$, $S$ is an optimal solution for $D$ iff $S$ is an optimal solution for $D' = D-R(v)$.
\end{rr}
 
\begin{proposition}\label{claim:1} 
    If  $x \in Z_{min}$, then $N^+(x) \subseteq S_{min}$, $S_{min}\cap R^-(x)=\emptyset$ and $Z_{min} \cap R^+(x) = \emptyset$.
\end{proposition}
\begin{proof}
    By the definition of a sink vertex, if $x \in Z_{min}$ then $N^+(x)$ is in $S_{min}$. Let $Y= S_{min}\cap R^-(x)$. We claim $S'=S_{min}\setminus Y$ is also a solution which will contradict the fact that $S_{min}$ is a minimal solution. Suppose $S'$ is not a solution, then there exists a vertex $v$ in $D-S'$, that does not reach a sink in $D-S'$ but $v$ reaches some sink $s$ in $D-S_{min}$. Since every vertex in $Y$ reaches sink $x$ in $D-S'$, $v\notin Y$. If $s$ is not a sink vertex in $D-S'$, then some vertex $y\in Y$ is an out-neighbor of $s$. But then $v$ can reach the sink $x$ in $D-S'$ via $y$. Hence, $S'$ is a solution of size strictly smaller than $S_{min}$, which is a contradiction. Since $S_{min}\cap R^-(x)=\emptyset$, we also have $Z_{min}\cap R^-(x)=\emptyset$. Now, let $s\in R^+(x)$. If $s\in Z_{min}$ then by definition of $R^+(x)$, $x\in Z_{min}\cap R^-(s)$ which is a contradiction. \qed
\end{proof}

\begin{proposition}\label{prop:2}
    If  $x \in Z_{min}$, then $S_{min}=N^+(x)\cup S'$, where $S'$ is a minimal feasible solution for $(D',\phi') = update(D,\phi,x,-)$. Also, if $S_{min}'$ is a minimal solution for $(D',\phi')$ then $S=N^+(x)\cup S_{min}'$ is a solution for $(D,\phi)$.
\end{proposition}
\begin{proof}
    First, assume that $S'_{min}$ is a minimal solution for $(D',\phi')$. We claim that $S=S'_{min}\cup N^+(x)$ is a solution for $(D,\phi)$. Suppose not, then there exists a vertex $v$ that does not reach a sink in $D - S$. Note that $v\notin R^-(x)$ as all vertices in $R^-(x)$ can reach sink $x$ in $D - S$. Then $v\notin R(x)\cup S$ and hence it is also in $D'-S_{min}$, where it reaches a sink $s$. Since this path is disjoint from $S'_{min}\cup R^{-}(x)$, $v$ still can reach $s$ via the same path in $D - S$. Note that $s$ is not a sink in $D - S$ only if it has an out-neighbor in $R^-(x)\setminus N^+(x)$. But then $s$ and $v$ are in $R^-(x)$ which is a contradiction. Hence, $v$ can still reach the same sink $s$ and $S=S'_{min}\cup N^+(x)$ is a solution for $D$. 
    
    Now, we prove that $S'=S_{min}\setminus  N^+(x)$ is a solution for $(D',\phi')$. 
    Since, $S_{min}$ is optimal, $S'\cap R^+(x)=\emptyset$ via Claim \ref{claim:1} and thus, $S'$ is feasible. If $S'$ is not a solution, then there has to be a vertex $v$ in $D'-S'$ that does not reach any sink. Since $S_{min}\subseteq R(x)\cup S'$, $v\notin S_{min}$. But $S_{min}$ is a solution for $(D,\phi)$ and $v$ can reach some sink $s \in Z_{min}$ in $D-S_{min}$. 
    Note that $s\neq x$, since $v$ is not in $R^-(x)$ and cannot reach $x$ in $D-S_{opt}$. Then $v$ has a path to $s$ which is disjoint from $S_{opt}\supseteq N^+(x)$. Moreover this path is also disjoint from the set $R(x)\setminus N^+(x)$, since $v\notin R(x)$. 
    Hence this path is disjoint from $R(x)$ and $S_{opt}$. Therefore, $v$ can reach $s$ via the same path in $D'-S'$. If $s$ is a sink in $D-S_{opt}$, then $s$ is also a sink in $D'-S'$. Hence, $v$ has a path to a sink in $D'-S'$, which is a contradiction. It implies that  $S'=S_{min}\setminus  N^+(x)$ is a solution for $(D',\phi')$.
    
    Finally, assume $S'=S_{min}\setminus  N^+(x)$ is not a minimal solution for $(D',\phi')$. Then there exists $S''\subset S'$ which is also a solution for $(D',\phi')$, but then $S''\cup N^+(x)$ is also a solution for $(D,\phi)$ contradicting the minimality of $S_{min}$.\qed
\end{proof}

\begin{corollary}\label{cor2}
    If  $x \in Z_{opt}$, then $|S_{opt}|=  |N^+(x)|+{\KFProb}(D',\phi')$, where $(D',\phi') = update(D,\phi,x,-)$.
\end{corollary}
\begin{proof}
    Let us assume $S_{opt}'$ to be an optimal solution to $(D',\phi')$. By Proposition \ref{prop:2}, $S = N^+(x)\cup S_{opt}'$ is a solution to $(D,\phi)$. Now, if $(D,\phi)$ has a minimal solution $\Tilde{S}$ smaller than $S$ then via Proposition \ref{prop:2}, $\Tilde{S}\setminus N^+(x)$ is a solution to $(D',\phi')$ that is smaller than $S_{opt}'$. This is a contradiction and hence the claim is true.\qed
\end{proof}

Now, we define a \emph{drop function} for every undecided vertex, denoted by $\psi(x)$ which takes the value $\psi (x) = \phi(R(x)) + 0.75*|R^+(x)\setminus R(x)|$. This function keeps track of the drop in potential when $x$ becomes a sink in some branch. Further, we define the \emph{surviving set} of vertex $x$, denoted by $\mathcal{S}_x = \{u\in N^+(x) | N^-(u)\cap \mathcal{U} = \{x\}\}$. Observe that for a given instance $(D,\phi)$, if $x\in \mathcal{U}$ has a nonempty survivor set, then the vertices of $\mathcal{S}_x$ is in a minimal solution $S_{min}$ if and only if $x$ is in $Z_{min}$ since Corollary \ref{cor1} requires $S_{min} = N^+(Z_{min})$ and $x$ is the only in-neighbour of elements of $\mathcal{S}_x$ which can be in $Z_{min}$. Finally, we have \emph{candidate sink set} of vertex $x$, denoted by $\mathcal{C}_x = \{u\in R^+(y)\ |\ y\in N^+(x)\}$. Observe that for any given instance $(D,\phi)$ if $x\notin Z_{min}$ then some out-neighbour $y$ of $x$ must satisfy $y\notin S_{min}$. This $y$ must reach some sink $s$ in the final solution, and by definition, $s$ must belong to $R^+(y)$. Thus $\mathcal{C}_x$ is the set of sinks the out neighbours of $x$ can possibly reach if $x$ is not a sink in the final solution. Also, note $\mathcal{C}_x \subseteq N^-(x)\cup R^+(x)$, since given $y\in N^+(x)$ and $s\in R^+(y)$, either $x\in N^+(s)$ or $x$ has a path to $s$ via $y$ in $D-N^+(s)$.

\section{An algorithm to compute minimum knot-free vertex deletion set}
%!TEX root = main.tex

In this section, we provide an exact exponential algorithm (Algorithm \ref{alg_1}) to compute the minimum knot-free vertex deletion set. We begin by initialising the potential of all vertices to $1$ and as soon as we decide a vertex to be non-sink we reduce its potential to $0.25$. In our algorithm, whenever we encounter a sink or a source, we remove them using Reduction Rules \ref{rr2} and \ref{rr3}. If all the vertices have potential $0.25$, then we apply Reduction Rule \ref{rr1} to solve the instance in polynomial time. 

At any point, if there exists an undecided vertex $x$ of potential $\psi(x)\geq 3.75$ then we branch on the possibility of it being a sink or non-sink in the optimal solution. Here we benefit from the high potential drop in the branch where $x$ becomes a sink which gives us a branching factor of $(3.75,0.75)$. Once such vertices are exhausted, we choose an undecided vertex $x$ with the \emph{maximum} number of undecided neighbours to branch on. We show that if $x$ is not a sink in the optimal solution then some other vertex $s$ from $\mathcal{C}_{x}$ has to be. Further, the bound on $\psi(x)$ helps us limit the cardinality of $\mathcal{C}_{x}$ and consequently the number of branches. In this case, we get a set of vertices from which atleast one has to be a sink in the final solution. Since, each branch has some vertex becoming a sink, the potential drop will be high enough to give a \emph{good} running time for our algorithm.
\begin{algorithm}[ht!]	\label{alg_1}
    \SetAlgoLined
    \SetKwData{I}{I}
    \SetKwData{size}{size}
    \SetKwData{clrr}{cLRR}
    \SetKwData{Stop}{Stop}
    \SetKwData{sizec}{size(cLRR)}
    \small 
    \vspace{6pt}
    \KwIn{A directed graph $D$ and a potential function $\phi$ }
    \KwOut{ The size of a minimum knot-free vertex deletion set}\vspace*{3mm}
    \hrule
    \label{algo1}\vspace*{3mm}
    \caption{\KFProb($D,\phi$)}
    
    \If {$\exists\ x$ such that $N^-(x)=\emptyset$} {
        \Return{$\KFProb(D-\{x\},\phi)$;}}\vspace*{2mm}
    \If {$\exists\ x$ such that $N^+(x)=\emptyset$} { 
        \Return{ $\KFProb(D-R(x),\phi)$;}}\vspace*{2mm}
    \If {$V(D) \subseteq\ \Bar{\mathcal{U}}$} {
        \Return{ $|V(D)|$;}}\vspace*{2mm}
    \If {$\exists\ x\in \mathcal{U}$ such that $\psi(x)\geq 3.75$}{
        $D_1, \phi_1 = update(D,\phi,x,-)$;\\
        $D_2, \phi_2 = update(D,\phi,-,x)$;\\
        \Return{ $\min\{\KFProb(D_1,\phi_1)+|N^+(x)|,\ \KFProb(D_2,\phi_2)\}$;}}\vspace*{2mm}

    pick $x\in \mathcal{U}$ maximizing $\lvert N(x)\cap\mathcal{U}\rvert$\vspace*{2mm}
            
    \If {$\mathcal{S}_x = \emptyset$}{  
        \For{$s_i \in\mathcal{C}_x$}{
            $D_i,\phi_i = update(D,\phi,s_i,-)$}
        \Return{ $\min\{\KFProb(D_x,\phi_x)+|N^+(x)|,\ \min_{s_i}\{\KFProb(D_i,\phi_i)+|N^+(s_i)|\}\}$;}}
        
    \Else {
        \If {$\lvert N(x)\cap\mathcal{U}\rvert \geq 1$} {
            $D_x,\phi_x = update(D,\phi,x,-)$;\\
            pick $y \in\mathcal{S}_x$\\
            set $B_x = R^+(y)$\\
            \For{$s_i \in  B_x$}{
                \If{$N^+(s_i)\subseteq N^+(s_j)$ for some $s_j\in B_x$}{
                    $B_x = B_x-s_j$}}
            \For{$s_i \in  B_x$}{
                $NS_i = \{s_j : j<i\}$\\
                $D_i,\phi_i = update(D,\phi,s_i,NS_i)$}  
            \Return{ $\min\{\KFProb(D_x,\phi_x)+|N^+(x)|,\ \min_{s_i}\{\KFProb(D_i,\phi_i)+|N^+(s_i)|\}\}$;}}
        \Else {
            $D_x,\phi_x = update(D,\phi,x,-)$;\\
            pick $y \in\mathcal{S}_x$, $s \in R^+(y)$\\
            $D_s,\phi_s = update(D,\phi,s,-)$\\
            \Return{ $\min\{\KFProb(D_x,\phi_x)+|N^+(x)|,\ \KFProb(D_s,\phi_s)+|N^+(s)|\}\}$;}}}\vspace*{6pt}
\end{algorithm}

\section{Correctness of the Algorithm}
In this section we prove that  Algorithm KFVD returns an optimal knot-free vertex deletion set for any input instance. Let \emph{Subroutine $0$} represent the lines $1-6$ of the algorithm. The correctness of lines $5-6$, $1-2$ and $3-4$ follow from Reduction Rules \ref{rr1}, \ref{rr2} and \ref{rr3} respectively. We use the terms \emph{Subroutine $1$}, \emph{Subroutine $2$}, \emph{Subroutine $3$} and \emph{Subroutine $4$} to refer to lines $7$-$10$, $12$-$15$, $17$-$27$ and $28$-$32$ of Algorithm \KFProb~respectively. The correctness of the subroutines are verified in the following sections.
    \subsection {Correctness of Subroutine 1} \label{sub1}
    In Subroutine $1$ we branch on an undecided vertex $x\in V(D)$, which has $\psi(x) \geq 3.75$. The following lemma proves the correctness of the subroutine. 

\begin{lemma}\label{lem:subroutine1}
    If $\ \exists\ x\in V(D)$ such that $x\in\mathcal{U}$ and $\psi(x)\geq 3.75$, then $\KFProb(D,\phi)=\min\{\KFProb(D_1,\phi_1)+|N^+(x)|,\text{ }\KFProb(D_2,\phi_2)\}$ where, $(D_1,\phi_1) = update(D,\phi,x,-)$ and $(D_2,\phi_2) = update(D,\phi,-,x)$.
\end{lemma}
\begin{proof}
    We prove the lemma using an inductive argument on the potential of the instance $(\phi(D))$. Observe that if we consider the base case $\phi(D) = 1$, then there is only one undecided vertex in the input instance and the recurrence holds true. Now given an instance $D$, assume that the algorithm computes the correct solution for all smaller instances. Let the solutions for $\KFProb(D_1,\phi_1)$ and $\KFProb(D_2,\phi_2)$ be $S_1$ and $S_2$, respectively. Assuming $S_{opt}$ is an optimal solution for $\KFProb(D,\phi)$, we evaluate the two possibilities: 
    \begin{itemize}
        \item \textbf{Case 1:} $\boldsymbol{x\in Z_{opt}}$.
           We show that $S_1\cup N^+(x)$ is an optimal solution for  ${\KFProb(D,\phi)}$ if and only if  $S_1$ is an optimal solution for ${\KFProb(D_1,\phi_1)}$. The arguments are exactly the same as that in Corollary \ref{cor2}. We also claim that $\KFProb(D_2,\phi_2) \geq \KFProb(D_1,\phi_1)+|N^+(x)| $. For contradiction, suppose that is not the case, then any optimal solution $S_2$ for ($D_2,\phi_2$) is also a feasible solution for $\KFProb(D,\phi)$. But $S_2$ has size strictly smaller than $S_{opt}$, which contradicts our assumption that it is optimal. Hence, $\KFProb(D,\phi)=$ $\min\{\KFProb$ $(D_1,\phi)+|N^+(x)|,\text{ }\KFProb(D,\phi_2)\}$.\vspace*{2mm}
        \item \textbf{Case 2:} $\boldsymbol{x\notin Z_{opt}}$.
            In this case, $\KFProb (D_2,\phi_2)=\KFProb(D,\phi)$, by definition. Also $\KFProb(D_2,\phi_2) \leq \KFProb(D_1,\phi_1)+|N^+(x)| $, otherwise we have $S'=S_1\cup N^+(x)$ as a solution with size strictly smaller than $S_{opt}$ which contradicts our assumption that it is optimal. Hence, $\KFProb(D,\phi)=$ $\min\{\KFProb(D_1,\phi)+|N^+(x)|,\text{ } \KFProb(D,\phi_2)\}$.
    \end{itemize}\qed
\end{proof}

In Subroutine $1$, we get a branching vector $(3.75,0.75)$. After exhaustively running Subroutine $1$, every remaining undecided vertex satisfies $\psi(x)\leq 3.75$. An important implication of this bound is the restricted number of undecided vertices in $N(x)$ as well as $\mathcal{C}_x$, which we prove in the 
 following lemma.

\begin{lemma}\label{lem:neighbourhood}
    If $(D,\phi)$ is an instance on which Subroutine 1 is no longer applicable, then every $x\in\mathcal{U}$ satisfies, $|\mathcal{C}_x|\leq 2$.
\end{lemma}
\begin{proof}
    If Subroutine 1 is no longer applicable to $(D,\phi)$, then every undecided vertex has at most 2 undecided neighbours, since $N(x)\subseteq R(x)$ and $\phi(R(x))$ is counted towards $\psi(x)$.\\
    To begin with, consider the possibility that, $|R^+(x)\setminus R(x)| \geq 3$. Observe that while computing $\psi(x)$, vertices of $|R^+(x)\setminus R(x)|$ contribute a total of $2.25$, the potential of $x$ contributes $1$ and the in-neighbour and out-neighbour of $x$ has to contribute atleast $0.5$. This adds up to a total of $3.75$ which is not allowed since Subroutine 1 is no longer applicable. Hence $|R^+(x)\setminus R(x)| \leq 2$. Also, recall that $\mathcal{C}_x\subseteq N^-(x)\cup R^+(x)\subseteq R(x)\cup R^+(x)$. Now we can consider the following possibilities.
    \begin{itemize}
        \item \textbf{Case 1:} $\boldsymbol{|R^+(x)\setminus R(x)| = 0}$. Here $\mathcal{C}_x\subseteq R(x)$. Also $\phi(R(x))$ is less than $3.75$, out of which $x$ contributes 1. Hence $R(x)$ can have at most 2 more undecided vertices and consequently, $|\mathcal{C}_x| \leq 2$.\vspace{-5pt}\\
        \item \textbf{Case 2:} $\boldsymbol{|R^+(x)\setminus R(x)| = 1}$. In this case, since $R^+(x)\setminus R(x)$ contributes $0.75$, the contribution of $R(x)$ to  $\psi(x)$ has to be less than $3$. Since $x$ itself contributes 1, we can have at most one other undecided vertex in $R(x)$. Hence $|\mathcal{C}_x| \leq 2$.\vspace{-5pt}\\
        \item \textbf{Case 3:} $\boldsymbol{|R^+(x)\setminus R(x)| = 2}$. Here, vertices in $R^+(x)\setminus R(x)$ contribute a total of $1.5$ to $\psi(x)$ which gives $\phi(R(x)) < 2.25$. Hence if $|\mathcal{U}\cap R(x)|\geq 2$ then we get $|R(x)| = 2$ which is not possible since $d^+(x), d^-(x)\geq 1$. Hence $\mathcal{U}\cap R(x)=\{x\}$ and $|\mathcal{C}_x| \leq 2$.
    \end{itemize}\qed
\end{proof}
    \subsection{Correctness of Subroutine 2} \label{sec4:subsec2}
    In Subroutine 2, we deal with undecided vertices which has an empty surviving set corresponding to them. The fact that every out-neighbour of such a vertex has some other undecided in-neighbour helps us establish a high potential for elements of $\mathcal{C}_x$. 

\begin{lemma}\label{lem:subroutine2}
    Let $(D,\phi)$ be an instance such that $\psi(v) < 3.75$, for every vertex in $\mathcal{U}$. let $x$ be a vertex maximizing $\mathcal{U}\cap N(x)$ and $\mathcal{S}_x = \emptyset$. We claim, $$\KFProb(D,\phi) = min\{\KFProb(D_x,\phi_x)+|N^+(x)|,\ min_{s_i}\{\KFProb(D_i,\phi_i)+|N^+(s_i)|\}\}$$ where $(D_x,\phi_x) = update(D,\phi,x,-)$ and $(D_i,\phi_i) = update(D,\phi,s_i,-)$ for every $s_i\in\mathcal{C}_x$.
\end{lemma}
\begin{proof}
    In any given minimal solution, either $x$ is a sink or at least one of its out-neighbours must survive. The out-neighbour which survives, say $y$, must be able to reach a sink in the final solution. This sink by definition, belongs to $R^+(y)$. Hence, if $x$ is not a sink in the final solution, then at least one vertex of $\mathcal{C}_x$ has to be. Thus, every possible minimal solution contains at least one element from the set $\{x\}\cup\mathcal{C}_x$ in its sink set. By induction, assuming that $\KFProb(D',\phi')$ returns the optimal solution for every instance smaller than $(D,\phi)$ along with Corollary \ref{cor2}, proves the lemma.\qed
\end{proof}

\noindent Observe that, since $|\mathcal{C}_x|\leq 2$, we have at most 3 branches when running Subroutine 2. Further we claim that whenever we branch on the case where $s_i\in\mathcal{C}_x$ becomes a sink, the potential drop is at least $3$. We have, either $x\in N^+(s_i)$ or $x\in R^-(s_i)$. Further, $s_i\in R^+(y)$ for some $y\in N^+(x)$ and since $\mathcal{S}_x = \emptyset$, $y$ has some undecided in-neighbour $z$ different from $x$. Note that, $z\neq s_i$ since $s_i\in R^+(y)$. Similar to $x$, $z\in N^+(s_i)$ or $z\in R^-(s_i)$. Hence $\psi(s_i)\geq\phi(\{s_i,x,z\})=3$. Now let us look at the possible branches which could arise and their corresponding worst case branching factor. Note that if $|\mathcal{C}_x| = 0$, no branching is involved and the subroutine is executed as a reduction rule. 
\begin{itemize}
    \item \textbf{Case 1:} $\boldsymbol{\mathcal{C}_x = \{s\}}$. Here, we have the following possibilities:\\
    \begin{itemize}
        \item $|N(x)\cap\mathcal{C}_x| = 0$, in which case, $N[x]$ has potential at least $1.5$, since $x$ has atleast two neighbours
        and $\phi(x) = 1$. Further, $s\in\mathcal{C}_x$ contributes $0.75$, giving $\psi(x)\geq 2.25$.
        \begin{figure}
            \centering
            \includegraphics[height=1.25in,width=2.25in]{S2_C1_1.png}
            \caption{Case 1.1: $N[x]\cap\mathcal{C}_{x}=\emptyset$}
            \label{S2_C1_1}
        \end{figure}\footnote{White vertices are undecided, yellow ones are semi-decided. Dotted lines indicate possibility of in and out-neighbours, dashed lines denote directed paths and thick lines denote edges.}
        \item $|N(x)\cap\mathcal{C}_x| = 1$, in which case, $N[x]$ has potential at least $2.25$, since $\phi(x)=\phi(s)=1$, and $x$ has at least one more neighbour, giving $\psi(x)\geq 2.25$.\\
        \begin{figure}
            \centering
            \includegraphics[height=1.3in,width=2.25in]{S2_C1_2.png}
            \caption{Case 1.2: $s\in N[x]\cap\mathcal{C}_{x}$}
            \label{S2_C1_2}
        \end{figure}
    \end{itemize}
    \newpage
    \item \textbf{Case 2:} $\boldsymbol{\mathcal{C}_x = \{s_1, s_2\}}$. Here the possibilities are as follows.\\
    \begin{itemize}
        \item $|N(x)\cap\mathcal{C}_x| = 0$, in which case, $N[x]$ has potential at least $1.5$, since $x$ has at least two neighbours and $\phi(x) = 1$. Further $s_1, s_2\in\mathcal{C}_x$ contribute $1.5$, giving $\psi(x)\geq 3$.
        \begin{figure}
            \centering 
            \includegraphics[height=1.20in,width=1.85in]{S2_C2_3.png}
            \caption{Case 2.1: $s_1,s_2\in N[x]\cap\mathcal{C}_{x}$}
            \label{S2_C2_3}
        \end{figure}
        \item $N(x)\cap\mathcal{C}_x = \{s_1\}$, in which case, $N[x]$ has potential at least $2.25$, since $x$ has at least two neighbours, including $s_1$. Further $s_2\in\mathcal{C}_x\setminus N(x)$ contributes $0.75$, giving $\psi(x)\geq 3$.
        \begin{figure}
            \centering
            \includegraphics[height=1.20in,width=1.85in]{S2_C2_2.png}
            \caption{Case 2.2: $s_1\in N[x]\cap\mathcal{C}_{x}$}
            \label{S2_C2_4}
        \end{figure}
        \item $|N(x)\cap\mathcal{C}_x| = 2$, in which case, $N[x]$ has potential at least $3$, due to $\phi(x)=\phi(s_1)=\phi(s_2)=1$ and hence $\psi(x)\geq 3$.
        \begin{figure}
            \centering
            \includegraphics[height=1.20in,width=2.25in]{S2_C2_1.png}
            \caption{Case 2.3: $N[x]\cap\mathcal{C}_{x}=\emptyset$}
            \label{S2_C2_5}\vspace{-20pt}
        \end{figure}
    \end{itemize}
\end{itemize}
\noindent Hence the worst case branching vectors from Subroutine 2 are $(3,2.25)$ and $(3,3,3)$. 
    \subsection{Correctness of Subroutine 3} \label{sec4:subsec3}
    In Subroutine $3$ we branch on undecided vertices with a non-empty surviving set and at least one undecided neighbour. Let us define $D_x = \{u\ |\ \exists\ v\in R^+(y),\ v\neq u,\ N^+(v)\subseteq N^+(u)\}$ and $B_x$ to be $R^+(y)\setminus D_x$. We also fix an arbitrary ordering of vertices of $B_x$ and define $NS_i = \{s_{j}\in B_x\ |\ j<i\}$\\

\begin{lemma}\label{lem:subroutine3}
    Let $(D,\phi)$ be an instance such that $\psi(v) < 3.75$, for every vertex in $\mathcal{U}$, let $x$ be a vertex maximizing $\mathcal{U}\cap N(x)$ and $y\in\mathcal{S}_x$. We claim $$\KFProb(D,\phi) = min\{\KFProb(D_x,\phi_x)+|N^+(x)|,\ min_{s_i}\{\KFProb(D_i,\phi_i)+|N^+(s_i)|\}\}$$ where $(D_x,\phi_x) = update(D,\phi,x,-)$ and $(D_i,\phi_i) = update(D,\phi,s_i,NS_i)$ for all $s_i\in B_x$.
\end{lemma}
\begin{proof}
    We claim that $y\in\mathcal{S}_x$ survives in every minimal solution where $x$ is not a sink. We have $y\in S_{min}$ if and only if some in-neighbour of $y$ is a sink in $D-S_{min}$, but $y\in\mathcal{S}_x$ implies  $N^-(y)\cap\mathcal{U} = \{x\}$. Hence, if $x\notin Z_{min}$, then $y\notin S_{min}$ and we can assume that if $x$ is not a sink then $y$ survives and must reach some sink. This sink by definition must be in $R^+(y)$. But if some distinct $u,v$ in $R^+(y)$ satisfy $N^+(v)\subseteq N^+(u)$, then in every minimal solution where $u$ is a sink, $v$ is also a sink and $\KFProb(update(D,\phi,v,-))\leq\KFProb(update(D,\phi,u,-))$. Thus, every possible minimal solution contains at least one element from the set $\{x\}\cup (R^+(y)\setminus D_x) = \{x\}\cup B_x$ in its sink set. Further, we can fix an arbitrary ordering for elements of $B_x$ and branch on the first vertex of this ordering which becomes a sink. In the branch where the $i^{th}$ vertex becomes a sink, we can assume that all the vertices before it are non-sinks. Hence by induction, assuming that $\KFProb(D',\phi')$ returns the optimal solution for every instance smaller than $(D,\phi)$, proves the claim.\qed
\end{proof}

\noindent Since $\psi(x)<3.75$, $N(x)$ has at most two undecided vertices whenever $x$ is undecided. We will analyze the branching vector for this subroutine in two steps. For undecided vertices with two undecided neighbours and for undecided vertices with one undecided neighbour. If $R^+(y)$ is empty, then $x$ must be a sink and no branching is involved. Also, Lemma \ref{lem:neighbourhood} implies $R^+(y)$ can have cardinality at most 2, which leads to the following possibilities. 

\subsection*{$\boldsymbol{B_x = \{s\}}$}\label{1_branch}
    Observe that if $x$ is a sink, then since $x$ has at least three vertices in its closed neighbourhood, at least two of which are undecided, we have $\psi(x)\geq 2.25$. But, if $x$ is not a sink and $s$ is the sink $y$ reaches, then since $s$ has at least two neighbours, and $x\in N^+(s)$ or $x\in R^-(s)$, we have $\psi(s)\geq\phi(N[s]\cup  \{x\})\geq 2.25$. Refer Figure \ref{S3_C1} for an illustration of this case. Hence the worst case branching vector is $(2.25,2.25)$.\\
    \begin{figure}
        \centering
        \includegraphics[height=1.25in,width=1.25in]{S3_C1.png}
        \caption{Case 1: $B_x = \{s\}$}
        \label{S3_C1}
    \end{figure}
    
\subsection*{$\boldsymbol{B_x = \{s_1,s_2\}}$}\label{2_branch}
    In this case, depending on the number of undecided neighbours $x$ has, we have the following cases.\\[-15pt]
    \paragraph*{\textbf{Case 1:} Branching on a vertex with two undecided neighbours}\quad\\[3pt]
    Here $|N(x)\cap\mathcal{U}| = 2$ implies $\phi(N[x]) \geq 3$ and hence $B_x\setminus N(x) = \emptyset$. Otherwise, the vertex in $B_x\setminus N(x)$ would contribute $0.75$ to $\psi(x)$ making it at least $3.75$ which contradicts our assumption that Subroutine 1 is no longer applicable. Now, assuming $y\in\mathcal{S}_x$ is the chosen vertex, we have the following possibilities.
    
    \begin{itemize}
        \item \textbf{Subcase 1.1:} $\boldsymbol{s_2\in R^-(s_1)}$ or $\boldsymbol{s_1\in R^-(s_2)}$\\[3pt]
        The arguments for both cases are identical, hence assume that  $\boldsymbol{s_2\in R^-(s_1)}$.
        \begin{itemize}
            \item[$\bullet$] If $x$ is a sink, then since $x$ has at least three undecided vertices in its closed neighbourhood each contributing 1 giving $\psi(x)\geq 3$.
            \item[$\bullet$] If $x$ is not a sink and $s_1$ is a sink which  reaches in the final solution. Then we have $x\in N(s_1)$, $s_2\in R^-(s_1)$ and hence $\psi(s_1)\geq 3$
            \item[$\bullet$] If $x$ is not a sink, $s_1$ is not a sink in the final solution but $s_2$ is. Then we have $x\in N(s_2)$, $y\in R^-(s_2)$ and $s_1\in R^+(s_2)$. Now if $s_1\in R^-(s_2)$ then $x,s_1$ and $s_2$ each contribute 1 to $\psi(s_2)$. Else, we can assume $y\neq s_1$ and $x,y,s_2$ contributes a total of 2.25 and $s_1$ contributes 0.75 giving $\psi(s_2)\geq 3$. 
        \end{itemize}
        \begin{figure}
            \centering 
            \includegraphics[height=1.25in,width=1.25in]{S3_C2_1_1.png}
            \caption{Case 2.1.1: $s_2\in R^-(s_1)$}
            \label{S3_C2_1_1}
        \end{figure}
        \noindent Hence we have a worst case branching vector $(3,3,3)$. Further, in the rest of the sub-cases, we can assume $s_1\notin R^-(s_2)$ and vice-versa which implies $y\notin \{s_1, s_2\}$. Now consider the case where $s_1\in N^+(x),\ s_2\in N^-(x)$, since $x\notin N^+(s_1)$, either $s_2\in N^+(s_1)$ or $s_2\in R^-(s_1)$. But $s_2\notin R^-(s_1)$ is and $s_2\in N^+(s_1)$ implies $s_1\in R^-(s_2)$, which cannot be the case. Thus this possibility is already considered, which leaves us with the following configurations.\\
        
        \item \textbf{Subcase 1.2:} $\boldsymbol{s_1, s_2\in N^-(x)}$\\[-5pt]
        \begin{itemize}
            \item[$\bullet$] If $x$ is a sink, since $N[x]$ has at least four vertices out of which three are undecided, we have $\psi(x)\geq 3.25$.
            \item[$\bullet$] If $x$ is not a sink and $s_1$ is a sink which $y$ reaches in the final solution. Then we have $x\in N^+(s_1)$, $y\in R^-(s_1)$. Also, since $N^+(s_1)\not\subseteq N^+(s_2)$ and $x\in N^+(s_2)$, $N^+(s_1)$ has at least one more vertex other than $x$, giving $\psi(s_1)\geq 2.5$.
            \item[$\bullet$] If $x$ is not a sink, $s_1$ is not a sink in the final solution, but $s_2$ is. Then we have $x\in N^+(s_2)$, $y\in R^-(s_2)$. Also, since $N^+(s_2)\not\subseteq N^+(s_1)$ and $x\in N^+(s_1)$, $N^+(s_2)$ has at least one more vertex, giving $\phi(R(s_2))\geq 2.5$. Further, since we also have the added information that $s_1$ is not a sink which gives a drop of 0.75, the total drop in potential is at least 3.25. Refer Figure \ref{S3_C2_1_2} for an illustration of this case.
        \end{itemize}
        \begin{figure}
            \centering
            \includegraphics[height=1.25in,width=1.5in]{S3_C2_1_2.png}
            \caption{Case 2.1.2: $s_1, s_2\in N^-(x)$}
            \label{S3_C2_1_2}
        \end{figure}
        \noindent Here we have a worst case branching vector $(2.5,3.25,3.25)$\\
        
        \item \textbf{Subcase 1.3:} $\boldsymbol{s_1, s_2\in N^+(x)}$\\[-5pt]
        \begin{itemize}
            \item[$\bullet$] If $x$ is a sink, since $N[x]$ has at least four vertices out of which three are undecided, we have $\psi(x)\geq 3.25$.
            \item[$\bullet$] If $x$ is not a sink and $s_1$ is a sink which $y$ reaches in the final solution. Then we have $x\in N^-(s_1)$, $y\in R^-(s_1)$ and at least one other vertex in $N^+(s_1)$ giving $\psi(s_1)\geq 2.5$.
            \item[$\bullet$] If $x$ is not a sink, $s_1$ is not a sink in the final solution, but $s_2$ is. Then we have, $x\in N^-(s_2)$, $y\in R^-(s_2)$ and at least one other vertex in $N^+(s_2)$ giving $\psi(s_2)\geq 2.5$. Further, we also have the added information that $s_1$ is not a sink which gives a drop of 0.75. Hence, the total drop in potential is at least 3.25. 
        \end{itemize}
        \begin{figure}
            \centering
            \includegraphics[height=1.2in,width=2in]{S3_C2_1_3.png}
            \caption{Case 2.1.3: $s_1,s_2\in N^+(x)$}
            \label{S3_C2_1_3}
        \end{figure}
        \noindent Here we have a worst case branching vector $(3.25,2.5,3.25)$.\\
    \end{itemize}
    \paragraph*{\textbf{Case 2:} Branching on a vertex with one undecided neighbour}\quad\\[3pt]
    Here, $|N(x)\cap\mathcal{U}| = 1$ implies $\phi(N[x]) \geq 2.25$ and hence, $|B_x\setminus N(x)| \leq 1$. Otherwise, the vertices in $B_x\setminus N(x)$ would contribute $0.75$ to $\psi(x)$ making it at least $3.75$ which contradicts our assumption that Subroutine 1 is no longer applicable. Now, assuming $s_1\in N(x)$ and $y\in\mathcal{S}_x$ is the chosen vertex, we have the following possibilities.
    
    \begin{itemize}
        \item \textbf{Subcase 2.1:} $\boldsymbol{s_1\in R^-(s_2)}$ or $\boldsymbol{s_2\in R^-(s_1)}$\\[3pt]
        The arguments for both the case are identical, hence we assume $s_1\in R^-(s_2)$. \\[-8pt]
        \begin{itemize}
            \item[$\bullet$] If $x$ is a sink, then since $x$ has at least three vertices in its closed neighbourhood exactly two of which are undecided and $s_2\in R^+(x)$, we get $\psi(x)\geq 3$.
            \item[$\bullet$] If $x$ is not a sink and $s_1$ is a sink which $y$ reaches in the final solution. Then we have at least three vertices in $N[s_1]$ out of which only $x,\ s_1$ are undecided. This and $s_2\in R^+(s_1)$ give $\psi(s_1)\geq 3$.
            \item[$\bullet$] If $x$ is not a sink, $s_1$ is not a sink in the final solution but $s_2$ is. Then we have $x\in R^-(s_2)$ and $s_1\in R^-(s_2)$. Now, $s_2$ has an out-neighbour which cannot be $x$ since $x$ has only one undecided neighbour and $s_1$ since $s_2\in R^+(s_1)$. This out-neighbour contributes at least $0.25$ making the total drop at least $3.25$. Refer Figure \ref{S3_C2_2_1_a} and \ref{S3_C2_2_1_b} for examples.\\
        \end{itemize}
        Hence the worst case branching vector is $(3,3,3.25)$.\\
        \begin{figure}
            \centering
            \begin{minipage}{.5\textwidth}
                \centering
                \includegraphics[height=1in,width=2in]{S3_C2_2_1_a.png}
                \caption{Case 2.2.1: $s_1\in R^-(s_2)$}
                \label{S3_C2_2_1_a}
            \end{minipage}%
            \begin{minipage}{.5\textwidth}
                \centering
                \includegraphics[height=1.25in,width=2in]{S3_C2_2_1_b.png}
                \caption{Case 2.2.1: $s_2\in R^-(s_1)$}
                \label{S3_C2_2_1_b}
            \end{minipage}
        \end{figure}
        
        \item \textbf{Subcase 2.2:} $\boldsymbol{s_2\notin R^-(s_1)\ \&\ s_1\notin R^-(s_2)}$\\[5pt]
        Here, $s_1\notin N^-(x)$, since in that case $s_1\in R^-(s_2)$ or $s_1\in N^+(s_2)$ which implies $s_2\in R^-(s_1)$. Also, $y\neq s_1$ since otherwise $s_1\in R^-(s_2)$.\\[-8pt]
        \begin{itemize}
            \item[$\bullet$] If $x$ is a sink, then since $x$ has at least two out-neighbours $y,s_1$ and some in-neighbour $z$, along with $s_2\in R^+(x)$ we have $\psi(x)\geq 3.25$. 
            \item[$\bullet$] If $x$ is not a sink and $s_1$ is a sink which $y$ reaches. Then we have $x,y$ in $R^-(s_1)$ and since $s_1$ has some other out-neighbour, we get $\psi(s_1)\geq 2.5$.
            \item[$\bullet$] If $x$ is not a sink, $s_1$ is not a sink in the final solution but $s_2$ is. Then we have, $x,y,z$ in $R^-(s_2)$ which gives $\psi(s_2)\geq 2.5$. Further, since we also have the added information that $s_1$ is not a sink which gives a drop of 0.75, the total drop in potential is at least 3.25. 
        \end{itemize}
        \begin{figure}
            \centering
            \includegraphics[height=1.25in,width=2in]{S3_C2_2_2.png}
            \caption{Case 2.2.2: $s_1,s_2\in N^+(x)$}
            \label{S3_C2_2_2}
        \end{figure}
        \noindent Hence the worst case branching vector is $(3.25,2.5,3.25)$.
    \end{itemize}
    \subsection{Correctness of Subroutine 4} \label{sec4:subsec4}
    In Subroutine $4$, we branch on undecided vertices with a non-empty surviving set and zero undecided neighbour. It is executed only when Subroutines $1,2$ and $3$ is no longer applicable. Hence, we can assume all the vertices no longer satisfy the requirements of Subroutine $1,2$ and $3$. In particular, note that every undecided vertex has zero undecided neighbours.

\begin{lemma}\label{lem:subroutine4}
    Let $(D,\phi)$ be an instance such that $\psi(v) < 3.75$ and $\mathcal{U}\cap N(v) = \emptyset$ for every vertex in $\mathcal{U}$. Let $x$ be an undecided vertex, $y\in\mathcal{S}_x$ and $s\in R^+(y)$. We claim $$\KFProb(D,\phi) = min\{\KFProb(D_x,\phi_x)+|N^+(x)|,\ \KFProb(D_s,\phi_s)+|N^+(s)|\}$$ where $(D_x,\phi_x) = update(D,\phi,x,-)$ and $(D_s,\phi_s) = update(D,\phi,s,-)$.
\end{lemma}
\begin{proof}
    We claim that $y\in\mathcal{S}_x$ survives in every minimal solution where $x$ is not a sink. We have, $y\in S_{min}$ if and only if some in-neighbour of $y$ is a sink in $D-S_{min}$, but $y\in\mathcal{S}_x$ implies  $N^-(y)\cap\mathcal{U} = \{x\}$. Hence, if $x\notin Z_{min}$, then $y\notin S_{min}$. Now, assuming $x$ is not a sink, $y$ survives, and it must reach some sink, which by definition must belong to $R^+(y)$. Now assume that $s\in R^+(y)$ is not a sink in the final solution where $x$ is not a sink and $y$ survives. Since $s$ is not a sink, it must either belong to the solution set or reach a sink vertex. But $s$ cannot belong to any minimal feasible solution to $(D,\phi)$, since $N^-(s)\cap\mathcal{U} = \emptyset$. Hence we can assume $s$ reaches some sink $z$. Now there are two possibilities  either $N^+(z)$ intersects a vertex of $P_{(y,s)}$ or not.
    See Figure \ref{S4_C1} and Figure \ref{S4_C2} in appendix for an illustration of the following cases:    
    \begin{itemize}
        \item \textbf{Case 1:} $\boldsymbol{w_1\in N^+(z)\cap P_{(y,s)}}$. Here, $z$ has a path to $s$ in $D-N^+(s)$ via $P_{(w_1,s)}$. Also, $x,z\notin N(s)$ since $s$ is undecided and cannot have undecided neighbours and hence $x\in R^-(s)$ since it has a path to $s$ via $y$. Now, $w_1\neq y$, since $z$ is a sink that $s$ can reach in a solution where $y$ survives and hence $y,w_1\in R^-(s)$. Thus, $R^-(s)$ contains $x,y,w_1,z$ and $s$. Now, out-neighbour of $s$ in $P_{(s,z)}$ cannot be $y$ or $w_1$ since $N^+(s)$ does not intersect $P_{(y,s)}$ and $y,w_1\in P_{(y,s)}$. Thus $N^+(s)$ contains at least one semidecided vertex $w_2$ different from $y,w_1$ which implies that $x,y,w_1,z,w_2$ and $s$ belongs to $R(s)$. This leads to a contradiction since this implies $\phi(R(s))\geq 3.75$.\\
        \begin{figure}
            \centering
            \includegraphics[height=1.3in,width=3.25in]{S4_C1.png}
            \caption{Case 1: $w_1\in N^+(z)\cap P_{(y,s)}$}
            \label{S4_C1}
        \end{figure}
        
        \item \textbf{Case 2:} $\boldsymbol{N^+(z)\cap P_{(y,s)}=\emptyset}$. Here, $y$ can reach $z$ in $D-N^+(z)$ via the path $P_{(y,s)}$ followed by $P_{(s,z)}$. Observe that $x\notin N^+(z)$ as both are undecided, and $y\in R^-(z)$ which implies that $x\in R^-(z)$. Now, out-neighbour of $s$ in $P_{(s,z)}$ cannot be $y$ since $N^+(s)$ does not intersect $P_{(y,s)}$ and $y\in P_{(y,s)}$. Thus $P_{(s,z)}$ contains at least one semidecided vertex $w_1$ different from $y$ and $R^-(z)$ contains $x,y,w_1,z$ and $s$. Now, out-neighbour of $z$ cannot be $y$, since $z$ is a sink that $s$ can reach in a solution where $y$ survives and it cannot be $w_1$ since $N^+(z)$ does not intersect $P_{(s,z)}$ and $w_1\in P_{(s,z)}$. Thus $N^+(z)$ contains at least one semidecided vertex $w_2$ different from $y,w_1$ which implies that $x,y,w_1,z,w_2$ and $s$ belong to $R(z)$. This leads to a contradiction since, $\phi(R(z))\geq 3.75$.\\
        \begin{figure}
            \centering
            \includegraphics[height=1.3in,width=3.25in]{S4_C2.png}
            \caption{Case 2: $N^+(z)\cap P_{(y,s)}=\emptyset$}
            \label{S4_C2}
        \end{figure}
    \end{itemize}
    
    \noindent Observe that both possibilities lead to a contradiction and hence our assumption that $s$ reaches some sink $z$ in a minimal solution where $x$ is not a sink and $y$ survives, has to be wrong. Thus, for the given instance where Subroutines $1,2$ and $3$ are no longer applicable, in any minimal solution where $x$ is not a sink, $y\in\mathcal{S}_x$ and $s\in R^+(y)$, $s$ has to be a sink. By induction, assuming that $\KFProb(D',\phi')$ returns the optimal solution for every instance smaller than $(D,\phi)$, proves the claim.\qed
\end{proof}

\noindent If $R^+(y)=\emptyset$ then $x$ has to be a sink and the subroutine is executed as a reduction rule without branching. Now, since $x$ has at least two neighbours and as $s\in R^+(x)$ we get $\psi(x)\geq 2.25$. Similarly, since $s$ has at least two neighbours and $x\in R^-(s)$ we get $\psi(s)\geq 2.5$. Hence the worst case branching factor is $(2.25,2.5)$.
    
\section{Running time analysis}
Observe that the reduction Rules $1$, $2$ and $3$ can be applied on any input instance in polynomial time. The branching vector obtained in Subroutine $1$ was $(3.75,0.75)$, which gives the recurrence $f(\mu)\leq f(\mu-3.75)+f(\mu-0.75)$. This solves to $f(\mu)=\mathcal{O}(1.4549^\mu)$. For Subroutine $2$ we observe a worst case drop in measure of $(2.25,3)$ and $(3,3,3)$, amongst which $(3,3,3)$ has the higher running time $f(\mu)=\mathcal{O}(1.4422^\mu)$. Similarly, for Subroutine $3$, we have branching vectors $(2.25,2.25),\ (3.25,2.5,3.25)$, and $(3,3,3)$, out of which $(3.25,2.5,3.25)$ gives the worst running time $f(\mu)=\mathcal{O}(1.4465^\mu)$. Finally Subroutine $4$ has a branching vector $(2.25,2.5)$ which gives $f(\mu)=\mathcal{O}(1.3393^\mu)$.\\

\noindent For an input instance $D$ of {\sc Knot-Free Vertex Deletion}, we initialise the potential of each vertex to $1$ and run KFVD($D,\phi$). Observe that, $\phi(V(D)) = \lvert V(D)\rvert = n$ and during the recursive calls the potential of the instance never increases or drops below 0. Hence we can bound the running time of the algorithm by the run time corresponding to its worst case subroutine (here, Subroutine 1), which gives us the following theorem.\\

\begin{theorem}
 Algorithm ${\KFProb}$ solves {\sc Knot-Free Vertex Deletion}  in $\mathcal{O}^*(1.4549^n)$.
\end{theorem}

\section{A lower bound on running time}
\begin{proof}[Proof of Theorem \ref{thm: lower_bd}]
For sake of contradiction, assume there is an online algorithm  $A$ that is a $c$-approximation for $c\leq \frac{k-1}{2} -\epsilon$ and any $\epsilon >0$. The algorithm is given a budget $\textsf{OPT}$.  The sequence of points that arrives will lie in $\mathbb{R}^k$. First there are $10k^2$ points $P_0$ that arrive at location $ (0,0,\dots, 0)$.  We may assume that $A$ assigns each of these the same label.  

There are $k-1$ remaining phases for how points are released. Index these phases as $1,2, \ldots k-1$. At the end of the $i$th phase, $A$ will have accumulated cost at least $\frac{i}{2} \textsf{OPT}$.  In the end, $A$'s cost will be at least $\frac{k-1}{2}\textsf{OPT}$, contradicting the assumption that $A$ is a $c$-approximation. Additionally, by the end of the $i$th phase, the algorithm will have used at least $i+1$ labels.  All points in the $i$th phase will arrive at location $(0, 0, \ldots \textsf{OPT}, \ldots 0)$ where the $i$th dimension is non-zero.  The number of points that arrive will be $10 k^2$ at this location.  The majority of these points will have to be labelled with a single label by the algorithm. 

Consider phase $i$.  The adversary releases one point $q_i$ at location $(0, 0, \ldots \textsf{OPT}, \ldots 0)$.  First \emph{assume} the algorithm the algorithm does not give this  point a new label. We will revisit this assumption at the end of the proof. Then regardless of what label the algorithm gives this point, it is distance at least $\textsf{OPT}$ from all other points that have arrived.  Thus, this point will add an additional $\textsf{OPT}/2$ cost to the algorithm.  Inductively, the total algorithm cost is at least $\frac{i}{2} \textsf{OPT}$.  

Next $10k^2 -1$ points arrive at this same location.  If the algorithm does not label half of these points with a new label, then these points will be grouped with points that arrived previously.  Since prior points are distances at least $\textsf{OPT}$ away, this will make the algorithm's total cost larger than $\Omega(k^2 \textsf{OPT})$. Thus, the algorithm must label half of these points with a new label.  Now the inductive invariants are satisfied.

At the end, the optimal solution labels the points that arrive in each phases a different label. The total optimal cost is $0$ and the algorithm has cost $\Omega(k \textsf{OPT})$.

Now we revisit the assumption that $q_i$ must not be given a new label by $A$.  Indeed, say it is given a new label.  Then the above procedure terminates at this time.  Instead, we know the algorithm has used $i+1$ labels.  There is a clustering of the points that have arrived whose cost is below \textsf{OPT} and only uses $i$ labels.  Intuitively, this is a large mistake. 

Indeed, consider the following. There will be $k-i$ points $P^*$ that will arrive.  The $\ell$th point arrives at location $(0, 0, \ldots L \cdot \textsf{OPT}, \ldots 0)$ where $L$ is a parameter. Here $\ell \in [k-i]$. The non-zero dimension is the $(i+\ell)th$ dimension. 

The optimal solution labels $q_i$ the same label as the points in $P_0$ and this cluster has cost \textsf{OPT}.  All other points get a unique label depending on which dimension is non-zero.  $k-1$ dimensions are used, combined with $P_0$ this is $k$ labels. The cost of these points are $0$ and the only cost is $q_i$ paying cost $\textsf{OPT}$.  The optimal cost is bounded. 

On the other hand, $A$  only has $k-i-1$ unused labels when points in $P^*$ begin arriving.  One of them must be given a label as some other point whose distance is at least $L\textsf{OPT}$ away.  The cost of this cluster is at least $\frac{L}{2} \textsf{OPT}$.  Setting $L$ large, contradicts the bounded approximation ratio of $A$. 






\end{proof}


\section{Upper \& lower bounds for the number of minimal knot-free vertex deletion sets}
We claim that if we run our algorithm on any given directed graph, and create a decision tree then for every minimal knot-free vertex deletion set there exists a leaf node of the decision tree which corresponds to it. Note that any algorithm which finds all the minimal solutions needs at least unit time to find each solution and hence the number of minimal solutions for any graph of size $n$ cannot exceed the complexity of the algorithm. This observation gives us the following theorem.
 
\begin{theorem}
    The number of inclusion-wise minimal knot-free vertex deletion sets is $\mathcal{O}^*(1.4549^n)$.
\end{theorem}

\begin{proof}
    Let $S_{min}$ be a minimal solution and $Z_{min}$ be the set of sinks in $D-S_{min}$. Beginning with the root of decision tree use the following set of rules to find the corresponding leaf node. If the node corresponds to an execution of subroutine 1 on a vertex $v$, then if $v\in Z_{min}$ choose the branch of the tree where $v$ is added to the sink set, else choose the branch where $v$ is labelled as a non-sink vertex.  If the node corresponds to any other subroutine, then by correctness of the algorithm proven in the earlier section, at that node it branches on a set of vertices, which intersects the sink set of any minimal solution. Choose a branch corresponding to a vertex $s$ such that $s\in Z_{min}$. Follow this procedure to get to a leaf node which corresponds to a knot-free vertex deletion set $S_{leaf}$ and sink set $Z_{leaf}$. Note that due to our choice of leaf node, we have $Z_{leaf}\subseteq Z_{min}$ and consequently $S_{leaf}\subseteq S_{min}$. This along with minimality of $S_{min}$ gives $S_{leaf}=S_{min}$. Hence the number of minimal knot-free vertex deletion sets of $D$ cannot exceed the number of leaf nodes of the decision tree corresponding to any run of the algorithm on $D$. Hence maximum number of minimal knot-free vertex deletion sets is $\mathcal{O}^*(1.4549^n)$.\qed
\end{proof}

\begin{theorem}
    There exists an infinite family of graphs with $\Omega(1.4422^n)$ many inclusion-wise minimal knot-free vertex deletion sets.
\end{theorem}

\begin{proof}
    Consider the graph in Figure \ref{fig:lowerbound}. Each strongly connected component $\{a_i,b_i,c_i\}$ can be made knot-free by deleting a single vertex. Hence every set $S$ which contains only one element from $\{a_i,b_i,c_i\}$ is a knot-free vertex deletion set. Observe that there are $3^{\frac{n}{3}}$ many of them since we can choose the element for each $i$ in 3 ways and $i$ ranges from $1$ to $\frac{n}{3}$. Further, any proper subset $S'$ of such a set will not intersect $\{a_i,b_i,c_i\}$ for some $i$, leaving $D-S'$ with at least one knot. Hence the graph in Figure \ref{fig:lowerbound} has at least $3^{\frac{n}{3}}\geq 1.4422^n$ many minimal knot-free vertex deletion sets. Now, by taking graphs which are disjoint union of triangles, we obtain an infinite family of graphs such that each element of that family has at least $1.4422^n$ many minimal knot-free vertex deletion sets.\qed
\end{proof}

\section{Conclusion}
\section{Conclusion}\label{sec:conclusion}
In this work, we focus on addressing the fundamental challenge of OOD detection tasks, which is how to fully understand the semantic discrepancy between the ID/OOD samples. We reveal that the key to success in the realistic SCOOD task is to allocate as many ID samples in the unlabeled set correctly as possible. To this end, we propose a novel uncertainty-aware optimal transport scheme that introduces class-specific energy scores as guidance for effective label assignment. Experimental results show that our method achieves better performance than previous state-of-the-art methods on SCOOD benchmarks.

\textbf{Limitations.} In addition to temperature scaling, other techniques such as feature clipping applied in ReAct~\cite{sun2021react} also enhance the performance of energy score, so how to obtain an OOD score that best fits the SCOOD task can be further explored. Moreover, a setting highly related to SCOOD has been proposed in \cite{katz2022training} and formulated as a constrained optimization problem. We will also theoretically analyze these practical OOD settings in our feature work.

% \section*{Acknowledgments}
\textbf{Acknowledgments.} 
This work is supported by National Key R\&D Program of China under Grant 2020AAA0105701, National Natural Science Foundation of China (NSFC) under Grants 61872327, Major Special Science and Technology Project of Anhui, National Natural Science Foundation of China (62033012) and Ant Group through Ant Research Intern Program.


\section*{Acknowledgements} 
\chapter*{Acknowledgement}
\addcontentsline{toc}{chapter}{Acknowledgement}
The authors thank Andrzej Kupsc, Sergey Barsuk, Olivier Callot and Wolfgang K{\"u}hn for their contribution on the CDR draft.
%The authors thank the international review committee XXX for their great effort in reading the CDR draft and providing valuable suggestions. 
The STCF working group thanks all 
the colleagues in the world-wide community for many profitable discussions
and expresses gratitude to the Hefei Comprehensive National Science Center for their strong support.  This work is supported by: international 
partnership program of the Chinese Academy of Sciences Grant No. 211134KYSB20200057.
\bibliography{my}

\end{document}