In Subroutine 2, we deal with undecided vertices which has an empty surviving set corresponding to them. The fact that every out-neighbour of such a vertex has some other undecided in-neighbour helps us establish a high potential for elements of $\mathcal{C}_x$. 

\begin{lemma}\label{lem:subroutine2}
    Let $(D,\phi)$ be an instance such that $\psi(v) < 3.75$, for every vertex in $\mathcal{U}$. let $x$ be a vertex maximizing $\mathcal{U}\cap N(x)$ and $\mathcal{S}_x = \emptyset$. We claim, $$\KFProb(D,\phi) = min\{\KFProb(D_x,\phi_x)+|N^+(x)|,\ min_{s_i}\{\KFProb(D_i,\phi_i)+|N^+(s_i)|\}\}$$ where $(D_x,\phi_x) = update(D,\phi,x,-)$ and $(D_i,\phi_i) = update(D,\phi,s_i,-)$ for every $s_i\in\mathcal{C}_x$.
\end{lemma}
\begin{proof}
    In any given minimal solution, either $x$ is a sink or at least one of its out-neighbours must survive. The out-neighbour which survives, say $y$, must be able to reach a sink in the final solution. This sink by definition, belongs to $R^+(y)$. Hence, if $x$ is not a sink in the final solution, then at least one vertex of $\mathcal{C}_x$ has to be. Thus, every possible minimal solution contains at least one element from the set $\{x\}\cup\mathcal{C}_x$ in its sink set. By induction, assuming that $\KFProb(D',\phi')$ returns the optimal solution for every instance smaller than $(D,\phi)$ along with Corollary \ref{cor2}, proves the lemma.\qed
\end{proof}

\noindent Observe that, since $|\mathcal{C}_x|\leq 2$, we have at most 3 branches when running Subroutine 2. Further we claim that whenever we branch on the case where $s_i\in\mathcal{C}_x$ becomes a sink, the potential drop is at least $3$. We have, either $x\in N^+(s_i)$ or $x\in R^-(s_i)$. Further, $s_i\in R^+(y)$ for some $y\in N^+(x)$ and since $\mathcal{S}_x = \emptyset$, $y$ has some undecided in-neighbour $z$ different from $x$. Note that, $z\neq s_i$ since $s_i\in R^+(y)$. Similar to $x$, $z\in N^+(s_i)$ or $z\in R^-(s_i)$. Hence $\psi(s_i)\geq\phi(\{s_i,x,z\})=3$. Now let us look at the possible branches which could arise and their corresponding worst case branching factor. Note that if $|\mathcal{C}_x| = 0$, no branching is involved and the subroutine is executed as a reduction rule. 
\begin{itemize}
    \item \textbf{Case 1:} $\boldsymbol{\mathcal{C}_x = \{s\}}$. Here, we have the following possibilities:\\
    \begin{itemize}
        \item $|N(x)\cap\mathcal{C}_x| = 0$, in which case, $N[x]$ has potential at least $1.5$, since $x$ has atleast two neighbours
        and $\phi(x) = 1$. Further, $s\in\mathcal{C}_x$ contributes $0.75$, giving $\psi(x)\geq 2.25$.
        \begin{figure}
            \centering
            \includegraphics[height=1.25in,width=2.25in]{S2_C1_1.png}
            \caption{Case 1.1: $N[x]\cap\mathcal{C}_{x}=\emptyset$}
            \label{S2_C1_1}
        \end{figure}\footnote{White vertices are undecided, yellow ones are semi-decided. Dotted lines indicate possibility of in and out-neighbours, dashed lines denote directed paths and thick lines denote edges.}
        \item $|N(x)\cap\mathcal{C}_x| = 1$, in which case, $N[x]$ has potential at least $2.25$, since $\phi(x)=\phi(s)=1$, and $x$ has at least one more neighbour, giving $\psi(x)\geq 2.25$.\\
        \begin{figure}
            \centering
            \includegraphics[height=1.3in,width=2.25in]{S2_C1_2.png}
            \caption{Case 1.2: $s\in N[x]\cap\mathcal{C}_{x}$}
            \label{S2_C1_2}
        \end{figure}
    \end{itemize}
    \newpage
    \item \textbf{Case 2:} $\boldsymbol{\mathcal{C}_x = \{s_1, s_2\}}$. Here the possibilities are as follows.\\
    \begin{itemize}
        \item $|N(x)\cap\mathcal{C}_x| = 0$, in which case, $N[x]$ has potential at least $1.5$, since $x$ has at least two neighbours and $\phi(x) = 1$. Further $s_1, s_2\in\mathcal{C}_x$ contribute $1.5$, giving $\psi(x)\geq 3$.
        \begin{figure}
            \centering 
            \includegraphics[height=1.20in,width=1.85in]{S2_C2_3.png}
            \caption{Case 2.1: $s_1,s_2\in N[x]\cap\mathcal{C}_{x}$}
            \label{S2_C2_3}
        \end{figure}
        \item $N(x)\cap\mathcal{C}_x = \{s_1\}$, in which case, $N[x]$ has potential at least $2.25$, since $x$ has at least two neighbours, including $s_1$. Further $s_2\in\mathcal{C}_x\setminus N(x)$ contributes $0.75$, giving $\psi(x)\geq 3$.
        \begin{figure}
            \centering
            \includegraphics[height=1.20in,width=1.85in]{S2_C2_2.png}
            \caption{Case 2.2: $s_1\in N[x]\cap\mathcal{C}_{x}$}
            \label{S2_C2_4}
        \end{figure}
        \item $|N(x)\cap\mathcal{C}_x| = 2$, in which case, $N[x]$ has potential at least $3$, due to $\phi(x)=\phi(s_1)=\phi(s_2)=1$ and hence $\psi(x)\geq 3$.
        \begin{figure}
            \centering
            \includegraphics[height=1.20in,width=2.25in]{S2_C2_1.png}
            \caption{Case 2.3: $N[x]\cap\mathcal{C}_{x}=\emptyset$}
            \label{S2_C2_5}\vspace{-20pt}
        \end{figure}
    \end{itemize}
\end{itemize}
\noindent Hence the worst case branching vectors from Subroutine 2 are $(3,2.25)$ and $(3,3,3)$. 