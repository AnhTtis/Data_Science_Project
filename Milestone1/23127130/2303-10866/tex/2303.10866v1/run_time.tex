Observe that the reduction Rules $1$, $2$ and $3$ can be applied on any input instance in polynomial time. The branching vector obtained in Subroutine $1$ was $(3.75,0.75)$, which gives the recurrence $f(\mu)\leq f(\mu-3.75)+f(\mu-0.75)$. This solves to $f(\mu)=\mathcal{O}(1.4549^\mu)$. For Subroutine $2$ we observe a worst case drop in measure of $(2.25,3)$ and $(3,3,3)$, amongst which $(3,3,3)$ has the higher running time $f(\mu)=\mathcal{O}(1.4422^\mu)$. Similarly, for Subroutine $3$, we have branching vectors $(2.25,2.25),\ (3.25,2.5,3.25)$, and $(3,3,3)$, out of which $(3.25,2.5,3.25)$ gives the worst running time $f(\mu)=\mathcal{O}(1.4465^\mu)$. Finally Subroutine $4$ has a branching vector $(2.25,2.5)$ which gives $f(\mu)=\mathcal{O}(1.3393^\mu)$.\\

\noindent For an input instance $D$ of {\sc Knot-Free Vertex Deletion}, we initialise the potential of each vertex to $1$ and run KFVD($D,\phi$). Observe that, $\phi(V(D)) = \lvert V(D)\rvert = n$ and during the recursive calls the potential of the instance never increases or drops below 0. Hence we can bound the running time of the algorithm by the run time corresponding to its worst case subroutine (here, Subroutine 1), which gives us the following theorem.\\

\begin{theorem}
 Algorithm ${\KFProb}$ solves {\sc Knot-Free Vertex Deletion}  in $\mathcal{O}^*(1.4549^n)$.
\end{theorem}