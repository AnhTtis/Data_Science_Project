In Subroutine $4$, we branch on undecided vertices with a non-empty surviving set and zero undecided neighbour. It is executed only when Subroutines $1,2$ and $3$ is no longer applicable. Hence, we can assume all the vertices no longer satisfy the requirements of Subroutine $1,2$ and $3$. In particular, note that every undecided vertex has zero undecided neighbours.

\begin{lemma}\label{lem:subroutine4}
    Let $(D,\phi)$ be an instance such that $\psi(v) < 3.75$ and $\mathcal{U}\cap N(v) = \emptyset$ for every vertex in $\mathcal{U}$. Let $x$ be an undecided vertex, $y\in\mathcal{S}_x$ and $s\in R^+(y)$. We claim $$\KFProb(D,\phi) = min\{\KFProb(D_x,\phi_x)+|N^+(x)|,\ \KFProb(D_s,\phi_s)+|N^+(s)|\}$$ where $(D_x,\phi_x) = update(D,\phi,x,-)$ and $(D_s,\phi_s) = update(D,\phi,s,-)$.
\end{lemma}
\begin{proof}
    We claim that $y\in\mathcal{S}_x$ survives in every minimal solution where $x$ is not a sink. We have, $y\in S_{min}$ if and only if some in-neighbour of $y$ is a sink in $D-S_{min}$, but $y\in\mathcal{S}_x$ implies  $N^-(y)\cap\mathcal{U} = \{x\}$. Hence, if $x\notin Z_{min}$, then $y\notin S_{min}$. Now, assuming $x$ is not a sink, $y$ survives, and it must reach some sink, which by definition must belong to $R^+(y)$. Now assume that $s\in R^+(y)$ is not a sink in the final solution where $x$ is not a sink and $y$ survives. Since $s$ is not a sink, it must either belong to the solution set or reach a sink vertex. But $s$ cannot belong to any minimal feasible solution to $(D,\phi)$, since $N^-(s)\cap\mathcal{U} = \emptyset$. Hence we can assume $s$ reaches some sink $z$. Now there are two possibilities  either $N^+(z)$ intersects a vertex of $P_{(y,s)}$ or not.
    See Figure \ref{S4_C1} and Figure \ref{S4_C2} in appendix for an illustration of the following cases:    
    \begin{itemize}
        \item \textbf{Case 1:} $\boldsymbol{w_1\in N^+(z)\cap P_{(y,s)}}$. Here, $z$ has a path to $s$ in $D-N^+(s)$ via $P_{(w_1,s)}$. Also, $x,z\notin N(s)$ since $s$ is undecided and cannot have undecided neighbours and hence $x\in R^-(s)$ since it has a path to $s$ via $y$. Now, $w_1\neq y$, since $z$ is a sink that $s$ can reach in a solution where $y$ survives and hence $y,w_1\in R^-(s)$. Thus, $R^-(s)$ contains $x,y,w_1,z$ and $s$. Now, out-neighbour of $s$ in $P_{(s,z)}$ cannot be $y$ or $w_1$ since $N^+(s)$ does not intersect $P_{(y,s)}$ and $y,w_1\in P_{(y,s)}$. Thus $N^+(s)$ contains at least one semidecided vertex $w_2$ different from $y,w_1$ which implies that $x,y,w_1,z,w_2$ and $s$ belongs to $R(s)$. This leads to a contradiction since this implies $\phi(R(s))\geq 3.75$.\\
        \begin{figure}
            \centering
            \includegraphics[height=1.3in,width=3.25in]{S4_C1.png}
            \caption{Case 1: $w_1\in N^+(z)\cap P_{(y,s)}$}
            \label{S4_C1}
        \end{figure}
        
        \item \textbf{Case 2:} $\boldsymbol{N^+(z)\cap P_{(y,s)}=\emptyset}$. Here, $y$ can reach $z$ in $D-N^+(z)$ via the path $P_{(y,s)}$ followed by $P_{(s,z)}$. Observe that $x\notin N^+(z)$ as both are undecided, and $y\in R^-(z)$ which implies that $x\in R^-(z)$. Now, out-neighbour of $s$ in $P_{(s,z)}$ cannot be $y$ since $N^+(s)$ does not intersect $P_{(y,s)}$ and $y\in P_{(y,s)}$. Thus $P_{(s,z)}$ contains at least one semidecided vertex $w_1$ different from $y$ and $R^-(z)$ contains $x,y,w_1,z$ and $s$. Now, out-neighbour of $z$ cannot be $y$, since $z$ is a sink that $s$ can reach in a solution where $y$ survives and it cannot be $w_1$ since $N^+(z)$ does not intersect $P_{(s,z)}$ and $w_1\in P_{(s,z)}$. Thus $N^+(z)$ contains at least one semidecided vertex $w_2$ different from $y,w_1$ which implies that $x,y,w_1,z,w_2$ and $s$ belong to $R(z)$. This leads to a contradiction since, $\phi(R(z))\geq 3.75$.\\
        \begin{figure}
            \centering
            \includegraphics[height=1.3in,width=3.25in]{S4_C2.png}
            \caption{Case 2: $N^+(z)\cap P_{(y,s)}=\emptyset$}
            \label{S4_C2}
        \end{figure}
    \end{itemize}
    
    \noindent Observe that both possibilities lead to a contradiction and hence our assumption that $s$ reaches some sink $z$ in a minimal solution where $x$ is not a sink and $y$ survives, has to be wrong. Thus, for the given instance where Subroutines $1,2$ and $3$ are no longer applicable, in any minimal solution where $x$ is not a sink, $y\in\mathcal{S}_x$ and $s\in R^+(y)$, $s$ has to be a sink. By induction, assuming that $\KFProb(D',\phi')$ returns the optimal solution for every instance smaller than $(D,\phi)$, proves the claim.\qed
\end{proof}

\noindent If $R^+(y)=\emptyset$ then $x$ has to be a sink and the subroutine is executed as a reduction rule without branching. Now, since $x$ has at least two neighbours and as $s\in R^+(x)$ we get $\psi(x)\geq 2.25$. Similarly, since $s$ has at least two neighbours and $x\in R^-(s)$ we get $\psi(s)\geq 2.5$. Hence the worst case branching factor is $(2.25,2.5)$.