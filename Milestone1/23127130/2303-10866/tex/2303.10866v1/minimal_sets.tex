We claim that if we run our algorithm on any given directed graph, and create a decision tree then for every minimal knot-free vertex deletion set there exists a leaf node of the decision tree which corresponds to it. Note that any algorithm which finds all the minimal solutions needs at least unit time to find each solution and hence the number of minimal solutions for any graph of size $n$ cannot exceed the complexity of the algorithm. This observation gives us the following theorem.
 
\begin{theorem}
    The number of inclusion-wise minimal knot-free vertex deletion sets is $\mathcal{O}^*(1.4549^n)$.
\end{theorem}

\begin{proof}
    Let $S_{min}$ be a minimal solution and $Z_{min}$ be the set of sinks in $D-S_{min}$. Beginning with the root of decision tree use the following set of rules to find the corresponding leaf node. If the node corresponds to an execution of subroutine 1 on a vertex $v$, then if $v\in Z_{min}$ choose the branch of the tree where $v$ is added to the sink set, else choose the branch where $v$ is labelled as a non-sink vertex.  If the node corresponds to any other subroutine, then by correctness of the algorithm proven in the earlier section, at that node it branches on a set of vertices, which intersects the sink set of any minimal solution. Choose a branch corresponding to a vertex $s$ such that $s\in Z_{min}$. Follow this procedure to get to a leaf node which corresponds to a knot-free vertex deletion set $S_{leaf}$ and sink set $Z_{leaf}$. Note that due to our choice of leaf node, we have $Z_{leaf}\subseteq Z_{min}$ and consequently $S_{leaf}\subseteq S_{min}$. This along with minimality of $S_{min}$ gives $S_{leaf}=S_{min}$. Hence the number of minimal knot-free vertex deletion sets of $D$ cannot exceed the number of leaf nodes of the decision tree corresponding to any run of the algorithm on $D$. Hence maximum number of minimal knot-free vertex deletion sets is $\mathcal{O}^*(1.4549^n)$.\qed
\end{proof}

\begin{theorem}
    There exists an infinite family of graphs with $\Omega(1.4422^n)$ many inclusion-wise minimal knot-free vertex deletion sets.
\end{theorem}

\begin{proof}
    Consider the graph in Figure \ref{fig:lowerbound}. Each strongly connected component $\{a_i,b_i,c_i\}$ can be made knot-free by deleting a single vertex. Hence every set $S$ which contains only one element from $\{a_i,b_i,c_i\}$ is a knot-free vertex deletion set. Observe that there are $3^{\frac{n}{3}}$ many of them since we can choose the element for each $i$ in 3 ways and $i$ ranges from $1$ to $\frac{n}{3}$. Further, any proper subset $S'$ of such a set will not intersect $\{a_i,b_i,c_i\}$ for some $i$, leaving $D-S'$ with at least one knot. Hence the graph in Figure \ref{fig:lowerbound} has at least $3^{\frac{n}{3}}\geq 1.4422^n$ many minimal knot-free vertex deletion sets. Now, by taking graphs which are disjoint union of triangles, we obtain an infinite family of graphs such that each element of that family has at least $1.4422^n$ many minimal knot-free vertex deletion sets.\qed
\end{proof}