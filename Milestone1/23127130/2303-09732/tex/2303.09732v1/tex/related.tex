\section{Related Work}
% Move the discussion about black/white-box watermarking to related work
\noindent\textbf{DNN Watermarking Schemes.}
Two categories of DNN watermarking methods, i.e., black-box and white-box algorithms, have been proposed to support model ownership verification. The black-box watermark schemes \cite{adi2018turning, jia2021entangled} mostly embed the identity information into the input-output patterns of the target model on a secret trigger set (similar to backdoor attacks \cite{gu2017badnets}).  As reported in \cite{adi2018turning}, the trade-off is sometimes evident between successfully embedding a black-box watermark and preserving the correct predictions on normal inputs. Moreover, recent progress on backdoor defenses also exposes a new attack surface on these black-box watermark schemes \cite{liu2019abs, liu2018finepruning, wang2019neuralcleanse}. % Specifically, the adversary can either remove the secret watermark via performing specific transforming to the trigger data or throw doubt on the security of the watermarked model by successfully detecting some trojaned neurons. 
White-box watermarking requires access to the parameters or the neuron activation of the protected model to extract the watermark. According to the location of the embedded message, the white-box watermarking methods can be classified into three groups: weight-based \cite{uchida2017embedding, wang2021riga, ong2021iprgan, liu2021greedyresiduals, chen2021lottery}, activation-based\cite{darvish2019deepsigns, lim2022ipcaption}, and passport-based\cite{fan2021deepip, zhang2020passportaware}. Recent watermarking schemes always show strong robustness against existing removal attacks including fine-tuning, pruning and overwriting \cite{wang2019overwrite}. Very recently, a concurrent work by Yan et al. \cite{yan2022cracking} reveals the overly dependence of existing white-box watermarks on the local neuron features which are fragile under neuron permutation and rescaling, while our revealed vulnerability is rooted in their common dependence on the structural identity of the target and the suspect model.

As the black-box and white-box watermarking schemes do not conflict with each other, some recent works combine them to provide more robust protection to the model copyright \cite{chen2021lottery,fan2021deepip,zhang2020passportaware,ong2021iprgan,darvish2019deepsigns}. During their watermark verification, these hybrid watermark algorithms first collect sufficient evidence via remote queries to the suspect models. Then, the owner further attains full access to the model with law enforcement to detect the identity information in the model internals, which yields a strong copyright statement. 
%As our attack framework breaks most of the existing verification procedures in white-box watermark algorithms, these hybrid watermark schemes could hardly survive under our attack. 

\noindent\textbf{Program Obfuscation.}
To prevent data structures and control flow of source code from being exposed through reverse engineering attacks, program obfuscation transforms a computer program that is semantic-equivalent to the original one but is harder to be analyzed for protecting the confidentiality of the program internals \cite{cifuentes1995decompilation, tip1994surveyanalyze}. This prevents an attacker or an analyst from reverse-engineering or debugging a proprietary software program \cite{xu2017ProgramObfuscationSurvey, schrittwieser2016canitobfuscation} via layout transformation, control-flow transformation, or data obfuscations \cite{collberg1997taxonomyobfuscating,majumdar2006surveyobfuscation,drape2010intellectualobfuscation}. 

Our proposed neural structural obfuscation is designed for a similar goal as program obfuscation, i.e., to prevent the copyright verification algorithm from successfully validating the watermark existence. Technically different from program obfuscation, neural structural obfuscation is done via different structurally invariant transforms on a neural network protected by the white-box watermarking. The obfuscated neural network is functionally equivalent to the original one, while the existing verification procedures can no longer recognize the original watermark from the model.