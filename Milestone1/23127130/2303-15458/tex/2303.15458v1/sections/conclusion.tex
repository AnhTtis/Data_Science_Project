\section{Conclusion} \label{sec:conclusion}

In this paper we propose a novel stochastic method for solving time fractional partial differential equations. These equations are already being used for modeling natural phenomenon subject to memory effects, and microscopically are typically described by non-Markovian processes. Fractional equations are capable of capturing such effects due to the inherent non-locality of the operator. As a consequence, the classical numerical schemes often based on time-stepping suffer from heavy memory storage requirements since the numerical solution of the fPDEs at current time depends on all preceding time instances. This can be even worse when dealing with high-dimensional problems, degrading significantly the performance of the corresponding numerical algorithms. 

The main advantage of the proposed method rests on the fact that it allows for the computation of the solution at single point of a given domain, or equivalently a single entry of the corresponding vector solution. Moreover, the numerical algorithm is not based on any time-stepping scheme, and therefore  the solution is obtained without the need of storing previous results. Rather, the solution is computed through an expected value of a functional of random processes which resembles the non-Markovian process found in the microscopic description of the phenomenon, and hence exploits somehow naturally the non-locality of the fractional operators.  

Furthermore, since the method is based on Monte Carlo it inherits all the known advantages from a computational point of view, such as the comparative ease of implementation in parallel, fault-tolerance, and in general of being well suited for heterogeneous architectures. In fact, our parallel implementation was able to solve large-scale problems efficiently in both shared-memory and distributed-memory systems, demonstrating the versatility and scalability of the probabilistic method compared to the classical numerical schemes.

