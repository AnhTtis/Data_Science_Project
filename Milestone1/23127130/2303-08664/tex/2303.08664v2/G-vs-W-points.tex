
\documentclass[a4paper]{amsart}
\usepackage{amsthm,amssymb}
\usepackage[T1]{fontenc}
\usepackage[utf8]{inputenc}
\usepackage{enumitem}
\usepackage{mathrsfs}
\usepackage{dsfont}
\usepackage{lmodern}
\usepackage{latexsym}
\usepackage{scalerel,stackengine}
\usepackage{nicefrac}
\usepackage{natbib}



\usepackage{graphicx}
\usepackage[colorlinks,citecolor=blue,urlcolor=blue]{hyperref}

\usepackage{tikz}


\usepackage{thmtools}

\declaretheorem[style=definition,name=Definition,qed=$\dashv$]{definition}
\declaretheorem[style=remark,name=Remark,qed=$\dashv$]{remark}
\declaretheorem[style=remark,name=Example,qed=$\dashv$]{example}


\numberwithin{equation}{section}


\newtheorem{theorem}{Theorem}[section]
\newtheorem{lemma}[theorem]{Lemma}
\newtheorem{proposition}[theorem]{Proposition}
\newtheorem{corollary}[theorem]{Corollary}


\theoremstyle{remark}
\newtheorem*{convention}{Convention}%[section]
%\newtheorem{definition}{Definition}[section]
\newtheorem{problem}{Problem}
%\newtheorem{example}{Example}[section]
%\newtheorem{remark}{Remark}[section]
%%%%%%%%%%%%%%%%%%%%%%%%%%%%%%%%%%%%%%%%%%%%%%%%%%%%%%%%%%%%%%%%%%%%%%%%%%%%%%%%%%%%%%


\newcommand\boldd{\boldsymbol{d}} %a distinguished element in a CA
\newcommand{\cond}{\con_{\boldd}} %contact via distinguished region
\newcommand{\notcond}{\notcon_{\boldd}} %not-contact via distinguished region

\newcommand\cl[1]{\overline{#1}} %the closure bar

\newcommand\lld{\ll_{\boldd}} %non-tangential inclusion via distiguished region
\newcommand{\Basis}{\mathscr{B}} %basis

\newcommand\frP{\mathfrak{P}}


\DeclareMathOperator{\Fin}{Fin}
\newcommand{\pofin}{\power(\omega)/_{\Fin}}

\title{Grzegorczyk and Whitehead points: the story continues}

\author{Rafa\l\ Gruszczy\'nski, Santiago Jockwich Martinez}

\date{}

\address{Rafa\l\ Gruszczy\'nski\\
Department of Logic\\
Nicolaus Copernicus University in Toru\'n\\
Poland}

\email{gruszka@umk.pl}

\urladdr{www.umk.pl/\textasciitilde gruszka}

\address{Santiago Jockwich Martinez\\
Department of Philosophy, Classics, History of Art and Ideas\\
University of Oslo\\
Norway}

\email{s.j.martinez@ifikk.uio.no}

%\urladdr{www.umk.pl/\textasciitilde gruszka}

%%%%%%%%%%%%%%%%%%%%%%%%%%%%%%%%%%%%%%%%%%%%%%%%%%%%%%%%%%

\newcommand\frB{\mathfrak{B}}
\newcommand\frR{\mathfrak{R}}
\newcommand\frM{\mathfrak{M}}
\newcommand\frN{\mathfrak{N}}

\newcommand\calE{\mathcal{E}}
\newcommand\calV{\mathcal{V}}

\newcommand{\isep}{\mbox{\textup{\tiny\texttt{)\!(}}}}

\newcommand{\scrP}{\mathscr{P}}
%%%%%%%%%%%%%%%%%% txfonts AMSa %%%%%%%%%%%%%%%%%%%%%%%%%
%\DeclareMathAlphabet{\txmathbcal}{OMS}{txsy}{bx}{n}
%\newcommand{\F}{\txmathbcal{F}}
\newcommand{\Ko}{\forall}
\newcommand{\Ks}{\exists}
\newcommand\Pociaga{\;\Longrightarrow\;}
\newcommand{\IMP}{\mathrel{\Longrightarrow}}
   \let\Imp=\Rightarrow
%%%%%%%%%%%%%%%%%%%%%%%%%%%%%%%%%%%%%%%%%%%%%%%%%%%%%%%
%\DeclareSymbolFont{AMSa}{U}{txsya}{m}{n}
\DeclareMathSymbol{\ppartof}{\mathord}{AMSa}{64}
\DeclareMathSymbol{\partof}{\mathrel}{AMSa}{64}
%%%%%%%%%%%%%%%%%%%%%%%%%%%%
%%%%%%%%%%%%%%% txfonts symbolsC %%%%%%%%%%%%%%%%%%
\DeclareSymbolFont{symbolsC}{U}{txsyc}{m}{n}
\DeclareMathSymbol{\poverl}{\mathord}{symbolsC}{7}
\DeclareMathSymbol{\overl}{\mathrel}{symbolsC}{7}
\DeclareMathSymbol{\pext}{\mathord}{symbolsC}{78}
%\DeclareMathSymbol{\ext}{\mathrel}{symbolsC}{78}
\DeclareMathSymbol{\npartof}{\mathrel}{symbolsC}{97}
\DeclareMathSymbol{\ningr}{\mathrel}{symbolsC}{64}
\DeclareMathSymbol{\nll}{\mathrel}{symbolsC}{51}

\newcommand\ext{\mathrel{\bot}}
%%%%%%%%%%%%%%%%%%%%%%%%%%%%%%%%%%%%%%%%%%%%%
%%%%%%%%% txfonts varmathbb %%%%%%%%%%%%%%%
%\DeclareMathSymbol{\vbbC}{\mathord}{lettersA}{131}
%\DeclareMathSymbol{\vbbN}{\mathord}{lettersA}{142}
%\DeclareMathSymbol{\vbbR}{\mathord}{lettersA}{146}


%%%%% newcommands for topology
\DeclareMathOperator{\Int}{Int}      %topol. operation of interior
\DeclareMathOperator{\Cl}{Cl}        %topol. operation of closure
\DeclareMathOperator{\Fr}{Fr}        %topol. Fr operation

\newcommand{\czz}{\ingr^{\mathit{0}}}
\newcommand{\Tz}{\mathrm{T}_0}
\newcommand{\Ti}{\mathrm{T}_1}
\newcommand{\Tii}{\mathrm{T}_2}
\newcommand{\Tiih}{\mathrm{T}_{2\frac{1}{2}}}
\newcommand{\Tiii}{\mathrm{T}_3}
\newcommand{\Tiiih}{\mathrm{T}_{3\frac{1}{2}}}
\newcommand{\Tiv}{\mathrm{T}_4}


%% newcommands for logic and set theory
\newcommand\iffdef{\;\mathrel{\mathord{:}\mathord{\longleftrightarrow}}\;}
\let\dfiff\iffdef
\newcommand\iffslim{\longleftrightarrow}
%\newcommand{\dfiff}{\;\mathrel{\mathord{:}\mathord{\Longleftrightarrow}}\;}
\newcommand\defeq{\coloneqq}
\newcommand\eqdef{\eqqcolon}
\newcommand\rarrow{\longrightarrow}
\newcommand\larrow{\longleftarrow}
\newcommand\Rarrow{\Longrightarrow}
\newcommand\Larrow{\Longleftarrow}
\newcommand\fun{\ensuremath{\rightarrow}} %--function arrow
\newcommand{\power}{\mathcal{P}}
\newcommand{\powerne}{\power^{\scriptscriptstyle +}}
\newcommand{\powerfin}{\power_{\!\scriptscriptstyle\textup{fin}}}
\newcommand{\powernefin}{\power^{\scriptscriptstyle  +}_{\!\scriptscriptstyle\textup{fin}}}

\let\la=\langle
\let\ra=\rangle

%% newcommands for mereology
\newcommand{\zero}{\mathsf{0}}
\newcommand{\one}{\mathsf{1}}
\newcommand{\suma}{\mathrel{\mathsf{sum}}}
\newcommand{\sumop}{\bigsqcup}
\newcommand{\Atom}{\mathrm{At}} %the set of atoms
\newcommand{\cAtom}{\mathrm{cAt}} %the set of co-atoms
\newcommand{\AtM}{\Atom_{\mathfrak{M}}}
\newcommand{\AtR}{\Atom_{\mathfrak{R}}}
\newcommand{\AtB}{\Atom_{\mathfrak{B}}}
\newcommand{\AtA}{\Atom_{\mathfrak{A}}}
\newcommand{\cAtM}{\cAtom_{\mathfrak{M}}}
\newcommand{\cAtR}{\cAtom_{\mathfrak{R}}}
\newcommand{\cAtB}{\cAtom_{\mathfrak{B}}}
\newcommand{\cAtA}{\cAtom_{\mathfrak{A}}}
%%%%%%%%%%%%%%%%%%%
\newcommand{\En}{\mathcal{E}^n}
\newcommand{\Ei}{\mathcal{E}^1}
%%%%%%%%%%%%%
\newcommand{\qsepS}{\textup{\footnotesize\texttt{qsep}}}
\newcommand{\qsT}{\qsepS\Topo}
\newcommand{\qsTsr}{\qsepS\Toposr}
%\DeclareMathOperator{\GcS}{\textup{\footnotesize\texttt{G}}_{\!\textup{\tiny\texttt{c}}}}
\newcommand{\GcS}{\textup{\footnotesize\texttt{G}}_{\!\textup{\tiny\texttt{c}}}}
%\DeclareMathOperator{\sGcS}{\textup{\tiny\texttt{G}}_{\!\textup{\tiny\texttt{c}}}}
\newcommand{\sGcS}{\textup{\tiny\texttt{G}}_{\!\textup{\tiny\texttt{c}}}}
\newcommand{\GcT}{\GcS\Topo}
\newcommand{\GcTych}{\GcS T} %G-structure for Punctured Tychonoff Plank
\newcommand{\sGcT}{\sGcS\Topo}
\newcommand{\GcEn}{\GcS\En}
\newcommand{\sGcEn}{\sGcS\En}
\newcommand{\GcEi}{\GcS\Ei}
\newcommand{\sGcEi}{\sGcS\Ei}
\newcommand{\sepS}{\textup{\footnotesize\texttt{sep}}}
\newcommand{\sT}{\sepS\Topo}
\newcommand{\sTsr}{\sepS\Toposr}
\newcommand{\sEn}{\sepS\En}
\newcommand{\sEi}{\sepS\Ei}
\newcommand{\ssEn}{\mathfrak{E}^n}
\newcommand{\ssEi}{\mathfrak{E}^1}
%%%%%%%%%%%%%%%%%%%%%%%
\newcommand{\gT}{\mathtt{g}\Topo}
\newcommand{\gTsr}{\mathtt{g}\Toposr}
%%%%%%%%%%%%%%%%%%%%%
\newcommand{\tsR}{\mathtt{ts}\mathfrak{R}}
\newcommand{\tsOmega}{\mathtt{ts}\Omega}
\newcommand{\tsOmegac}{\mathtt{ts}\overline{\Omega}}
%\newcommand{\Gts}{\mathtt{Gts}}
\DeclareMathOperator{\Gts}{\mathtt{Gts}}
\newcommand{\GtsR}{\Gts\,\mathfrak{R}}
\newcommand{\GtsEn}{\Gts(\GcEn)}
\newcommand{\GtsT}{\Gts\,\sT}
\newcommand{\Stone}{\mathtt{Ult}}
\newcommand{\StoneR}{\Stone\mathfrak{R}}
\newcommand{\StoneB}{\Stone\mathfrak{B}}
\newcommand{\StoneA}{\Stone\mathfrak{A}}

%% newcommands for BA
\newcommand\bzero{\mit{0}}      %zero w algebrze Boole
\newcommand\bone{\mit{1}}         %jedynka w algebrze Boole

%\DeclareMathOperator{\bplus}{\boldsymbol{+}}
%\DeclareMathOperator{\bmulti}{\boldsymbol{\cdot}}
\DeclareMathOperator{\bminus}{\textup{\textbf{\textsf{--}}}}
\DeclareMathOperator{\bleq}{\leq}

%%%%%%%%%%%%%%%%%%%%%%-separacja-etc-%%%%%%%%%%%%%%%%%%%%%%%%%%%%%%%%
\newcommand{\msep}{\mathrel{\mbox{\textup{\footnotesize\texttt{)\!(}}}}}
\let\separ=\msep
\newcommand{\fmsep}{\mathrel{\mbox{\textup{\scriptsize\texttt{)\!(}}}}}
\newcommand{\tmsep}{\mathrel{\mbox{\textup{\tiny\texttt{)\!(}}}}}
\let\ssepar=\tmsep
%\newcommand{\separi}{\mathrel{\mathbin{\raisebox{1pt}{{\normalfont\scriptsize\textsf{][}}}}}}
\newcommand{\separi}{\mathrel{\mbox{\textup{\texttt{]\![}}}}}
\newcommand{\separibfsf}{\mathrel{\mbox{\textup{\textbf{\textsf{)(}}}}}}
\newcommand{\separibf}{\mathrel{\mbox{\textup{\textbf{)(}}}}}
\newcommand{\separisf}{\mathrel{\mbox{\textup{\textsf{)(}}}}}
\let\separii=\separisf

%\DeclareMathOperator{\ET}{\mathsf{ET}} %relacja zewnêtrznej stycznoœci
\newcommand{\con}{\mathrel{\mathsf{C}}} %relacja styczności
\DeclareMathOperator{\cont}{\mathsf{C}_{\mathrm{T}}} %relacja stycznoœci topologiczna
\DeclareMathOperator{\confull}{\mathsf{C}_{\mathrm{L}}} %relacja stycznoœci pe³na
\DeclareMathOperator{\cono}{\mathsf{C}^{\circ}} %relacja stycznoœci w RO(R)^\circ


\newcommand\mathbackslash{\raisebox{.4pt}{\texttt{/}}}
\def\notcon{% the separation relation
  \renewcommand\stacktype{L}\mathrel{\ensurestackMath{%
  \ThisStyle{\stackon[0pt]{\SavedStyle\con}{\SavedStyle\mathbackslash}}}}%
}

\let\separ=\notcon


\newcommand{\rel}{\mathrel{R}}

\def\notrel{% the complement of R
  \renewcommand\stacktype{L}\mathrel{\ensurestackMath{%
  \ThisStyle{\stackon[0pt]{\SavedStyle\rel}{\SavedStyle\mathbackslash}}}}%
}

\DeclareMathOperator{\Con}{\boldsymbol{\mathsf{C}}} %topologiczna relacja stycznoœci


%\def\notcon{%
%  \renewcommand\stacktype{L}\mathrel{\ensurestackMath{%
%  \ThisStyle{\stackon[0pt]{\SavedStyle\con}{\SavedStyle\mathbackslash}}}}%
%}


%\DeclareMathOperator{\conf}{\mathrel{\infty}} %connection relation for filters in Roeper
\newcommand{\conf}{\mathbin{\dwakola}}

%\def\nconf{%  \renewcommand\stacktype{L}\mathrel{\ensurestackMath{%
%  \ThisStyle{\stackon[0pt]{\SavedStyle\infty}{\SavedStyle/}}}}%

\def\nconf{%
\renewcommand\stacktype{L}\mathbin{\ensurestackMath{%
  \ThisStyle{\stackon[0pt]{\SavedStyle\dwakola}{\SavedStyle/}}}}}

%\def\nconf{\mathrel{\not\dwakola}}

\def\dwakola{% \renewcommand\stacktype{S}\mathrel{\ensurestackMath{%
%  \ThisStyle{\stackon[0pt]{\SavedStyle\medcirc}{\SavedStyle\medcirc}}}}}
%\mathrel{\medcirc\hspace{-1pt}\medcirc}}
\mathord{\circ}\!\mathord{\circ}}

\newcommand{\GX}{\mathbf{Q}_{\mathrm{G}}}
\def\QG #1#2#3{\mathrm{G}^{{#1}}_{{#2},{#3}}}
\newcommand{\prePt}{\mathbf{Q}_G}
\newcommand{\prePtW}{\mathbf{Q}_W}
\newcommand{\Abs}{\mathbf{A}}


\newcommand{\wprePt}{\mathbf{Q}^{\boldsymbol{\omega}}}
\newcommand{\prePtR}{\prePt_{\mathfrak R}}
\newcommand{\wprePtR}{\wprePt_{\mathfrak R}}
\newcommand{\Pt}{\mathbf{Grz}}
\let\Grz=\Pt
\newcommand\Eq{\mathbf{Eq}}
\newcommand\Wthd{\mathbf{W}}
\newcommand{\PtR}{\mathbf{Pt}_{\mathfrak R}} %the set of all points
\newcommand{\PtS}{\mathbf{Pt}_{\mathfrak S}} %the set of all points
%\newcommand{\Pt}{\Pi} %the set of all points
\newcommand{\Pts}{\Pt^{\star}}
%%%%%%%%%%%%%
%\DeclareMathAlphabet{\mathb}{OT1}{cmr}{b}{n}
\DeclareMathOperator{\Irl}{\textup{\bfseries Irl}}
\newcommand{\Adh}{\textup{\bfseries A}}
\newcommand{\Is}{\textup{\bfseries Iso}}

\newcommand{\rAdh}{\mathbin{\Adh}}
\newcommand{\Open}{\boldsymbol{\mathsf{O}}}
\newcommand{\OpenEn}{\Open_{\sGcEn}}
\newcommand{\OpenR}{\Open_{\mathfrak R}}
\newcommand{\OpenRp}{\OpenR^{\scriptscriptstyle+}}
\newcommand{\rOpenR}{\mathrm{r}\OpenR}
\newcommand{\rOpenRp}{\rOpenR^{\scriptscriptstyle+}}
%\newcommand{\Is}{\mathsf{Is}}
\newcommand{\Closed}{\boldsymbol{\mathsf{Cl}}}
\newcommand{\ClosedR}{\Closed_{\mathfrak R}}
\newcommand{\Clop}{\boldsymbol{\mathsf{Clop}}}
\newcommand{\ClopR}{\Clop_{\mathfrak R}}
\newcommand{\Baza}{\boldsymbol{\mathsf{B}}}
\newcommand{\BazaR}{\Baza_{\mathfrak R}}
\newcommand{\BazaEn}{\Baza_{\sGcEn}}
\newcommand{\podBaza}{\textup{\textit{\bfseries{\textsf{B}\/}}}}
\newcommand{\Topo}{\mathsf{T}}
\newcommand{\Toposr}{\Topo_{\!\!\mathrm{sr}}}
\newcommand{\topo}{\mathscr{O}}
\newcommand{\toposr}{\topo_{\!\mathrm{sr}}}
\newcommand{\topop}{\topo^{\scriptscriptstyle+}}
\newcommand{\topoE}{\mathscr{E}}
\newcommand{\topoEp}{\topoE_{\!\scriptscriptstyle+}}
\newcommand{\rtopo}{\mathrm{r}\topo} %zbiory regularnie otwarte, nie u¿ywam w tym artykule
\DeclareMathOperator{\RO}{RO} %zbiory regularnie otwarte, u¿ywam w tym artykule
\newcommand{\rtopoE}{\mathrm{r}\topoE}
\newcommand{\rtopoEp}{\mathrm{r}\topoEp}
\newcommand{\rtoposr}{\mathrm{r}\toposr}
\newcommand{\rtopop}{\rtopo^{\scriptscriptstyle+}}
%\newcommand{\rtoporp}{\rtopo_{\mathrm{r}}^{\scriptscriptstyle+}}
\newcommand{\baza}{\mathscr{B}}
\newcommand{\bazaQ}{\mathscr{Q}}
\newcommand{\rbaza}{\mathrm{r}\baza}
\newcommand{\Dtopo}{\mathscr{C}}
\newcommand{\ClOp}{\mathscr{C\!O}}

\newcommand\ccc{\mathrm{(ccc)}}

%\newcommand{\fil}{\mathcal{F}}
%\newcommand{\fil}{\nabla}
\newcommand{\fil}{\mathscr{F}}
\newcommand{\ult}{\mathscr{U}}
\newcommand{\rfil}{\mathrm{F}}
\newcommand{\End}{\mathbf{End}}
\newcommand{\Fil}{\mathbf{Fil}}
\newcommand{\Rnd}{\mathbf{Rnd}}
\newcommand{\ultf}{F}
\newcommand{\Ult}{\mathbf{Ult}}
\newcommand{\UltM}{\mathop{\Ult(\mathfrak M)}}
\newcommand{\UltR}{\mathop{\Ult(\mathfrak R)}}
\newcommand{\UltB}{\mathop{\Ult(\mathfrak B)}}

\newcommand\Sat{\mathbf{Sat}} %saturated family of filters

\DeclareMathOperator\upop{\uparrow} %upward closure operation
\DeclareMathOperator\downop{\downarrow} %downward closure operation

\newcommand\twodownop{\raisebox{-1pt}{$\mathop{\rotatebox{90}{$\twoheadleftarrow$}}$}} %downward << closure operation
\newcommand\twoupop{\raisebox{-1pt}{$\mathop{\rotatebox{90}{$\twoheadrightarrow$}}$}} %upward << closure operation

\newcommand{\CFil}{\mathrm{CFil}}
\newcommand{\MCFil}{\mathrm{MCFil}}
\newcommand{\CFilR}{\mathop{\CFil}(\mathfrak R)}
\newcommand{\MCFilR}{\mathop{\MCFil}(\mathfrak R)}
%\newcommand{\ult}{\mathop\mathrm{ult}}
%\newcommand{\ult}{\mathop\mathrm{s}}
%%%%%%%%%%%%%%%%%%%%%%%%%%Zbiory-etc.%%%%%%%%%%%%%%%%%%%%%%%%%%%%%%%%%%%%%

\newcommand\R[1]{{\mathds{R}}^{{#1}}} %the set R3
%\newcommand\Nat{\text{\textomega}} %{\mathds{N}} %the set of natural numbers
\newfont{\eurxi}{eurm10 at 10.95pt}
\newfont{\eurviii}{eurm7 at 8pt}
\newfont{\eurvii}{eurm7}
\newcommand{\eriota}{\mbox{\eurxi\symbol{19}}}
\newcommand{\fNat}{\mbox{\eurviii\symbol{33}}} %the set of natural numbers for footnotes
\newcommand{\sNat}{\mbox{\eurvii\symbol{33}}} %the set of natural numbers for indexes
\newcommand{\Nat}{\mbox{\eurxi\symbol{33}}} %the set of natural numbers
\newcommand \Natp{\Nat^{\scriptscriptstyle\!+}}  %the set of positive natural numbers
\newcommand \sNatp{\sNat^{\scriptscriptstyle\!+}}%the set of positive natural numbers for indexes
\newcommand\Real{\mathds{R}} %the set of real numbers
\newcommand\Rat{\mathds{Q}} %the set of rational numbers
\newcommand\Irr{\mathds{I}} %the set of irrational numbers
\newcommand\C[1]{{\mathds{C}}_{#1}} %Cartesian space

\newcommand{\rO}{\mathrm{r}\topo}
\newcommand{\rOp}{\rO^{\!\scriptscriptstyle+}}
\newcommand{\ROp}{\RO^{\!\scriptscriptstyle+}}
\newcommand\rOCi{\rO_{\C{1}}}
\newcommand\rOCiii{\rO^{\,\C{3}}}
\newcommand\rOCiiip{\rOp^{\,\C{3}}}


\newcommand{\NEG}{\mathop{\neg}}
\newcommand{\ROW}{\;\Leftrightarrow\;}
\newcommand{\KS}{\exists}
\newcommand{\KO}{\forall}


\newcommand{\bi}{\mathrm{b}}
\newcommand{\Bb}{\mathrm{B}}

\def\dywiz{\kern0sp\discretionary{-}{}{-}\penalty10000\hskip0sp\relax}

%%%%%%%%%%%%%%%%Poni¿ej s¹ ze starego artyku³u

\newcommand{\MS}{\boldsymbol{\mathsf{MS}}}
\newcommand{\MF}{\boldsymbol{\mathsf{MF}}}
\newcommand{\MFone}{\boldsymbol{\mathsf{MF1}}}
\newcommand{\qSep}{\boldsymbol{\mathsf{qSep}}} %quasi-separation structures
\newcommand{\qSepone}{\boldsymbol{\mathsf{qSep1}}} %quasi-separation structures with one
\newcommand{\qSepc}{\boldsymbol{\mathsf{qSep}_{\mathsf{c}}}} %complete quasi-separation %structures
\newcommand{\Sep}{\boldsymbol{\mathsf{Sep}}} %separation structures
\newcommand{\Sepc}{\boldsymbol{\mathsf{Sep}_{\mathsf{c}}}} %separation structures
\newcommand{\Sepone}{\boldsymbol{\mathsf{Sep1}}} %separation structures
\newcommand{\G}{\boldsymbol{\mathsf{G}}} %G-structures
\newcommand{\Gc}{\boldsymbol{\mathsf{G}_{\mathsf{c}}}} %complete G-structures
\newcommand{\Gcc}{\boldsymbol{\mathsf{G}^{\aleph_0}_{\mathsf{c}}}} %complete and countable G-structures
\newcommand{\Gone}{\boldsymbol{\mathsf{G1}}} %G-structures
\newcommand{\Gonec}{\boldsymbol{\mathsf{G1}_{\mathsf{c}}}} %G-structures
\newcommand{\GS}{\boldsymbol{\mathsf{G}^{\star}}} %G-structures

%%%%%%%%%%%%%%%%%%%%%%%%%%%%%%%%%%%%%%%%%%%%%%%%%%%%%%%%%%%%%%%
\newcommand\Klass{\boldsymbol{\mathsf{K}}} %abstract class of structures
\newcommand\BA{\boldsymbol{\mathsf{BA}}} %the class of Boolean algebras
\newcommand\BCA{\boldsymbol{\mathsf{BCA}}} %the class of Boolean contact algebras
\newcommand\TBCA{\boldsymbol{\mathsf{TBCA}}} %the class of topological Boolean contact algebras
\newcommand\BPCA{\boldsymbol{\mathsf{BPCA}}} %the class of Boolean contact algebras
\newcommand\GCA{\boldsymbol{\mathsf{GCA}}} %the class of Grzegorczyk contact algebras
\newcommand\BWCA{\boldsymbol{\mathsf{BWCA}}} %the class of Boolean weak contact algebras
\newcommand\Conc{\boldsymbol{\mathsf{Conc}}} %the class of concentric spaces
\newcommand\Lob{\boldsymbol{\mathsf{Lob}}} %the class of lob-spaces, i.e. the spaces which have local basis ordered linearly by reversed subset relation


%\newcommand\Even{\varmathbb{E}} %even ordinals
\newcommand\Even{\mathds{E}}
%\newcommand\Odd{\varmathbb{O}} %odd ordinals
\newcommand\Odd{\mathds{O}}

\newcommand\dsA{\mathds{A}}
\newcommand\dsB{\mathds{B}}
\newcommand\dsP{\mathds{P}}

%points

\newcommand\frp{\mathfrak{p}}
\newcommand\frq{\mathfrak{q}}
\newcommand\frr{\mathfrak{r}}
\newcommand\frs{\mathfrak{s}}

\newcommand\frg{\mathfrak{g}}
\newcommand\frh{\mathfrak{h}}

%
\newcommand{\Inf}{\mathrel{\SF{inf}}}
\newcommand{\bsuma}{\sqcup}
\let\bprod=\sqcap
\DeclareSymbolFont{stx}{OMS}{txsy}{m}{n}
\DeclareMathSymbol{\Ing}{\mathrel}{stx}{118}
\DeclareMathSymbol{\ll}{\mathrel}{stx}{28}
\newcommand{\llt}{\ll_{\mathrm{T}}}
\newcommand{\llfull}{\ll_{\mathrm{F}}}
\DeclareMathSymbol{\medcirc}{\mathord}{symbolsC}{7}
\DeclareMathSymbol{\Ext}{\mathrel}{symbolsC}{78}
%\DeclareMathSymbol{\Ing}{\mathrel}{symbols}{118}

\newcommand{\covers}{\trianglerighteq} %covering relation
\newcommand{\ncovers}{\ntrianglerighteq} %complement covering relation
\newcommand\covered{\trianglelefteq} %being covered relation
\newcommand\ncovered{\ntrianglelefteq} %complement of being covered relation

\newcommand{\llast}{\mathrel{\underline{\mathord{\ll}}}}

\DeclareMathSymbol{\nIng}{\mathrel}{symbolsC}{64}
\DeclareMathSymbol{\Ov}{\mathrel}{symbolsC}{7}
\newcommand{\sep}{\mathbin{\raisebox{1pt}{{\normalfont\scriptsize\textsf{)(}}}}}
\let\SF=\mathsf
\newcommand{\opI}{\SF{I}}
\let\ingr=\Ing
\let\ing=\Ing

\newcommand{\FC}{\mathrm{FC}}
\newcommand{\FCN}{\FC(\Nat)}
\newcommand{\FCNp}{\FC^{+}(\Nat)}
\newcommand{\FCS}{\FC(S)}
\newcommand{\FCSp}{\FC^{+}(S)}

\newcommand{\eqqcolon}{\mathrel{\mathord{=}\mathord{:}}}
\newcommand{\coloneqq}{\mathrel{\mathord{:}\mathord{=}}}
\let\coloneq=\coloneqq


\newcommand\sqcupo{\sqcup^\circ}
\newcommand\sqcapo{\sqcap^\circ}
\newcommand\compo{-^\circ}

\newcommand\iso[1]{#1^{\mathrm{i}}} %isolated points of a given set


\endinput 

\newsavebox{\putwodownop}
\savebox{\putwodownop}{$\exists\mathord{\twodownop}$}

%%%%%%%%%%%%%%%%%%%%%%%%%%%%%%%%%%%%%%%%

\begin{document}



\begin{abstract}
One of the main goals of region-based theories of space is to formulate a geometrically appealing definition of \emph{points}. The paper is devoted to the analysis of two such seminal definitions: Alfred N. Whitehead's (\citeyear{Whitehead-PR}) and Andrzej Grzegorczyk's (\citeyear{Grzegorczyk-AGWP}). Relying on the work of Loredana Biacino's and Ginagiacomo Gerla's (\citeyear{Biacino-Gerla-CSGWDP}), we improve their results, solve some open problems concerning the mutual relationship between Whitehead and Grzegorczyk points, and put forward open problems for future investigation.

\medskip

\noindent MSC: 00A30, 03G05, 06E25.

\medskip

\noindent Keywords: Boolean contact algebras; region-based theories of space; point-free theories of space; points; spatial reasoning; Grzegorczyk; Whitehead
\end{abstract}

\maketitle

\section*{Introduction}

Alfred N. Whitehead (\citeyear{Whitehead-PR}) was one of the first thinkers who---following ideas to be found in the seminal paper of \citet{Laguna-PLSSS}---proposed a~geometrically appealing  definition of \emph{point} in terms of \emph{regions of space} and the \emph{contact} relation. His construction was inventive and elegant yet lacked mathematical rigor. In the 1960s, the Polish logician Andrzej Grzegorczyk put forward one of the first mathematically satisfactory systems of region-based topology, in which he formulated a different, yet also geometrically motivated, construction of points. The comparison of the two approaches was carried out by Loredana Biacino and Giangiacomo Gerla (\citeyear{Biacino-Gerla-CSGWDP}), who---under some reasonable assumptions---demonstrated that the two notions of \emph{point} coincide.

The seminal paper by \citet{Biacino-Gerla-CSGWDP} is the foundation for our work.  The two main results of the paper were Theorems 5.1 and 5.3. The former establishes that every Grzegorczyk representative of a point is a Whitehead representative; the latter shows that the reverse inclusion holds for those Whitehead representatives that can be represented by countable families of regions.

To prove the first inclusion Biacino and Gerla work with the second-order theory of Grzegorczyk's (\citeyear{Grzegorczyk-AGWP}). We show that the specific axioms can be eliminated in favour of the standard first-order mereotopological postulates. Moreover, we prove that the second-order monadic statement `every Grzegorczyk representative is a Whitehead representative' is equivalent (in the subclass of Boolean weak contact algebras in the sense of \citet{Duntsch-Winter-WCS} in which every region has a non-tangential part)\marginpar{\tiny !!!} to the first-order statement `there are no atoms'. For the completeness of presentation we show that no part of this equivalence holds in the (general) class of Boolean weak contact algebras.

As for the second inclusion, we identify a gap in the proof of Theorem 5.3, and we show that it cannot be carried out without assuming an additional axiom postulating coherence, a mereotopological counterpart of the connectedness property. We also improve the original result by addressing an open problem from \citep{Biacino-Gerla-CSGWDP}. That is, we show that the countability assumption about Whitehead representatives can be eliminated, if we assume a stronger second-order version of the standard mereotopological interpolation axiom.

Moreover, we prove that in complete structures, purely mereological notions are too weak to guarantee the existence of Whitehead representatives of points. The English logician himself envisaged this, but no general proof of this fact exists in the literature so far.\footnote{We elaborate on this further on p.\,\pageref{page:purely-mereological}.}


We also provide various examples of Whitehead points within algebraic structures. This provides evidence for the claim that Whitehead points are mathematically tractable.



More or less from the beginning of the 21st century, Boolean contact algebras (see e.g.,~\citealp{Stell-BCAANATRCC,Bennett-Duntsch-AAT}) have provided the standard mathematical framework for doing region-based topology. This is a~comfortable situation that allows for the unification and comparison of different approaches to point-free theories of space. For this reason, in this paper, we also use the aforementioned algebras. This approach is different from the original approaches of Whitehead and Grzegorczyk, as the former used a \emph{contact} relation as the only primitive, and the latter worked in mereology (the theory of the \emph{part of} relation) extended with contact. From a technical point of view, these differences are irrelevant. At the same time, the unified well-established environment of Boolean contact algebras allows for a precise and clear presentation of both approaches to region-based theories.





The paper is organized as follows: In Section~\ref{sec:WCA}, we review some preliminaries and  introduce the main objects of study, viz., Boolean (weak) contact algebras. In Section~\ref{sec:G-points}, we present two formal accounts of Grzegorczyk points, i.e.,  Grzegorczyk points defined in terms of equivalences classes and Grzegorczyk points understood as filters. Moreover,  we show that these two definitions are  equivalent in the context of Boolean weak contact algebras. Sections \ref{sec:W-points} addresses a formal account of Whitehead points. We study some of their properties and provide examples of such points within regular open algebras. This section witnesses as well a proof of the insufficiency of purely mereological notions for the existence of Whitehead points.  Section \ref{sec:G-points-are-W-points} studies the minimal constraints that a Boolean Weak Contact Algebras has to satisfy to guarantee that every Grzegorczyk representative of a point is a  Whitehead representative. In particular, in this section we strengthen Theorem 5.1 of \cite{Biacino-Gerla-CSGWDP}. In Section~\ref{sec:W-points-are-G-points-countable} we fill the mentioned gap in the original proof of Theorem 5.3 of   \cite{Biacino-Gerla-CSGWDP} and study the logical status of the second-order condition `every Whitehead representative is a Grzegorczyk representative'. In Section~\ref{sec:W-points-are-G-points-general}, we generalize Theorem 5.3  to  Whitehead points of any size.

\section{Weak-contact and contact algebras}\label{sec:WCA}

As usual,  $\neg$, $\wedge$, $\vee$, $\rarrow$, $\iffslim$, $\forall$ and $\exists$  denote the standard logical constants of negation, conjunction, disjunction, material implication, material equivalence, universal and the existential quantifier. We use `$\nexists$' as an abbreviation for `$\neg\exists$'. Moreover, $\iffdef$ means \emph{equivalent by definition}, and $\defeq$ means \emph{equal by definition}. We use $\omega$ to denote the set of natural numbers understood as von Neumann ordinals. For a fixed space $X$ and $x\subseteq X$, $\complement x\defeq X\setminus x$ is the set-theoretical complement of~$x$ in~$X$. $|X|$ is the~cardinal number of a~set $X$, and $\power(X)$ is its power set.

Moreover, let:
\[
\frB=\langle \boldsymbol{B},\mathord{\cdot},\mathord{+},-,\zero,\one\rangle
\]
be a~Boolean algebra (BA for short) with the operations of, respectively, meet, join, and boolean complement; and with the two distinguished elements: the minimum $\zero$ and the maximum $\one$. Elements of the domain will be called \emph{regions}. The class of all Boolean algebras will be denoted by `$\BA$'. We will often refer to the domain of BA via its name `$\frB$'. Notice that this convention will not lead to any ambiguities.

In $\frB$ we define two standard order relations:
\begin{align}
x\leq y&{}\iffdef x\cdot y=x\,,\tag{$\mathrm{df}\,\mathord{\leq}$}\\
x<y&{}\iffdef x\leq y\wedge x\neq y\,.\tag{$\mathrm{df}\,\mathord{<}$}
\end{align}
In the former case we say that $x$ is \emph{part} of~$y$ or that $x$ is \emph{below}~$y$, in the latter that $x$ is \emph{proper part} of~$y$ or that $x$ is \emph{strictly below}~$y$.

Any Boolean algebra $\frB$ is turned into a~\emph{Boolean contact algebra} (BCA for short) by extending it to a~structure $\langle\boldsymbol{B},\mathord{\cdot},\mathord{+},-,\zero,\one,\con\rangle$ where $\mathord{\con}\subseteq\boldsymbol{B}^2$ is a~\emph{contact} relation which satisfies the following five axioms:

\begin{gather}
\neg(\zero\mathrel{\mathsf{C}} x),\label{C0}\tag{C0}\\
x\leq y\wedge x\neq\zero\rarrow x\mathrel{\mathsf{C}} y,\label{C1}\tag{C1}\\
x\mathrel{\mathsf{C}} y\rarrow y\mathrel{\mathsf{C}} x,\label{C2}\tag{C2}\\
x\leq y\rarrow\forall_{z\in B}(z\mathrel{\mathsf{C}} x\rarrow z\mathrel{\mathsf{C}} y)\,, \label{C3}\tag{C3}\\
x \con (y+ z) \rarrow x \con y\vee x\con z\,.\label{C4}\tag{C4}
\end{gather}
The complement of $\con$ will be denoted by `$\separ$', and in the case $x\separ y$ we say that $x$ is \emph{separated from} $y$.  The class of all Boolean contact algebras will be denoted by `$\BCA$'. If $\mathord{\con}$ satisfies \eqref{C0}--\eqref{C3}, it is called---after \citet{Duntsch-Winter-WCS}---a~\emph{weak contact} relation and the corresponding structure bears the name of a Boolean \emph{weak contact} algebra (BWCA for short). The class of all weak contact algebras will be denoted by `$\BWCA$'.

We introduce the convention according to which given a~class $\Klass$ of structures and some conditions $\varphi_1,\ldots,\varphi_n$ put upon elements of $\Klass$, $\Klass+\varphi_1+\ldots+\varphi_n$ (or $\Klass+\{\varphi_1,\ldots,\varphi_n\}$) is the subclass of $\Klass$ in which every structure satisfies all $\varphi_1,\ldots,\varphi_n$, e.g.,
\[
\BCA=\BWCA+\eqref{C4}\,.
\]




In $\frB\in\BWCA$ we define an auxiliary relation of \emph{non-tangential} inclusion (or \emph{way below}, \emph{well-inside}) relation:
\begin{equation}\label{df:ll}
x\ll y\iffdef x\separ -y\tag{$\mathrm{df}\,\mathord{\ll}$}\,.
\end{equation}
We also define $x\overl y$ to mean that $x\cdot y\neq\zero$, and take $\mathord{\ext}\subseteq\frB\times\frB$ to be the set-theoretical complement of $\overl$. In the former case we say that $x$ \emph{overlaps} $y$, in the latter, that $x$ \emph{is disjoint from} $y$ or $x$ \emph{is incompatible with} $y$. A structure $\langle \boldsymbol{B},\mathord{\cdot},\mathord{+},-,\zero,\one,\mathord{\overl}\rangle$ is a~standard example of a~BCA.\footnote{The overlap relation is actually the smallest contact relation on a~BCA, see \citep{Duntsch-Winter-CBCA}.} The most well-known interpretation of contact is the \emph{topological} one. For a~fixed space $\langle X,\topo\rangle$ we take the underlying algebra to be either the complete algebra $\RO(X)$ of all regular open subsets of $X$, or its subalgebra $B$.  The Boolean operations\footnote{ $\Int$ and $\Cl$ are the standard topological \emph{interior} and \emph{closure} operators.} are:
\begin{align*}
  x\cdot y &{}\defeq x\cap y \\
  x+ y & {}\defeq\Int\Cl(x\cup y)\\
  -x & {}\defeq\Int\complement x
\end{align*}
and the contact relation is given by:
\begin{equation*}%\tag{$\mathord{\con}_{\mathrm{T}}$}\label{df:CT}
x\cont y\iffdef\Cl x\cap\Cl y\neq\emptyset\,.
\end{equation*}
Moreover, we have:
\[
x\llt y\iffslim \Cl x\subseteq y\,.
\]
The relation $\cont$ satisfies axioms \eqref{C0}--\eqref{C4}, so any topological contact algebra is in the class $\BCA$.

We may use a~similar interpretation on the whole power set algebra of $X$, i.e., $\langle\power(X),\mathord{\cont}\rangle$ is a Boolean contact algebra (provided $X$ is equipped with a~topology, of course). Observe that despite the algebra being atomic, the contact relation does not collapse to the overlap relation. For example, in the case of $\Real$ with the standard topology, the open intervals $(0,1)$ and $(1,2)$ are disjoint, yet they are in contact since their closures share an atom. However, we may look upon $\cont$ as a~form of an overlap relation since in this special case of the power set algebra, we have:
\[
x\cont y\iffslim\Cl x\overl\Cl y\,,
\]
which usually is not true when we take into account regular open algebras. From this, it follows that closed sets are in contact only if they overlap, and thus the contact between atoms reduces to identity, if the underlying topology is $T_1$.

The following facts are standardly proven to hold in $\BWCA$:
\begin{gather}
x\ll y\rarrow x\leq y\,, \label{eq:llcIngr}\\
%\forall_{x,y\in B}(\Ko_{u\in B}\; u\Ing y\IMP x\ll y), \label{new-property-1-bez1} \\
%\forall_{x,y\in B}\bigl(x\ll y\IMP\Ko_{u\in B}\; u\Ing y\:\vee\:\Ks_{z\in B}(z\ext y \:\wedge\: z\separ x)\bigr), \label{new-property}\\
x\ll y\wedge y\ll x \rarrow x=y\,,\label{antis-ll}\\
x\ll y \wedge y\leq z \rarrow x\ll z\,, \label{eq:ll-ingr->ll}\\
x\leq y \wedge y\ll z \rarrow x\ll z\,,\label{eq:ingr-ll-ll}\\
x\ll y \wedge y\ll z \rarrow x\ll z\,, \label{eq:ll-ll->ll}\\
x\ll y\iffslim -y\ll -x\,.\label{eq:ll-contraposition}
\end{gather}
%Conditions \eqref{new-property-1-bez1}  and \eqref{new-property} entail that:\vspace{-3pt}
%\begin{gather}
%\forall_{x\in B}\; x\ll \one\,, \label{ll-one}\tag{\ref{new-property-1-bez1}$'$}\\
%\forall_{x,y\in B}\bigl(x\ll y\IMP y=\one\,\vee\,\Ks_{z\in B}(z\ext y \:\wedge\: z\separ x)\bigr)\,. \label{new-property'}\tag{\ref{new-property}$'$}
%\end{gather}

\begin{definition}
  An \emph{atom} of a Boolean (contact) algebra is a non-zero region $x$ that is minimal with respect to $\leq$ among non-zero regions. A BWCA is \emph{atomic} iff its underlying BA is atomic iff every non-zero region contains an atom. A BWCA is \emph{atomless} iff it does not have any atoms, i.e., satisfies the following condition:
  \begin{equation}\label{eq:no-atoms}
      (\forall x\in\frB\setminus\{\zero\})(\exists y\in \frB\setminus\{\zero\})\,y<x\,.\tag{$\nexists\mathrm{At}$}
  \end{equation}
\end{definition}

\section{Grzegorczyk points}\label{sec:G-points}
A~\emph{Grzegorczyk representative of a point} (for short: \emph{G-representative})\footnote{Both the term and its abbreviation are adopted from \citep{Biacino-Gerla-CSGWDP}.} in $\frB\in\BWCA$ is a~non-empty set $Q$ of regions such that:
\begin{gather}
\zero\notin Q\,,\tag{r0}\label{r0}\\
(\forall u,v\in Q)(u=v\vee u\ll v \vee v\ll u)\, , \tag{r1}\label{r1}\\
(\forall u\in Q)(\exists v\in Q)\; v\ll u\, , \tag{r2}\label{r2}\\
(\forall x,y\in\frB)\bigl((\forall u\in Q)(u\overl x\wedge u\overl y) \rarrow x\mathrel{\mathsf{C}} y \bigl)\, . \tag{r3}\label{r3}
\end{gather}
Let $\prePt$ be the set of all G-representatives of $\mathfrak B$.
The purpose of the definition is to formally grasp the intuition of a point as a~system of diminishing regions determining a~unique location in space. We call it a~\emph{representative}, since if we understand a~point as a perfect representation of a~location in space, then two different sets of regions may represent the same location (see Figure~\ref{fig:the-same-point} for a geometrical intuition on the Cartesian plane). Further, we will identify such G-representatives to be one point. Although the definition has a strong geometrical flavor, G-representatives may be somewhat strange entities in BCAs that have little to do with spatial intuitions. We will look at some indicative examples. But first, let us go through an example of a~G-representative in a~well-known setting: the reals.



\begin{figure}[t!]
\begin{center}\footnotesize
\begin{tikzpicture}
\draw[line width=1pt,dotted] (-0.25,-0.25) rectangle (3.25,3.25);
\draw[line width=1pt,dotted] (0,0) rectangle (3,3);
\draw[line width=1pt,dotted] (2.75,0.25) rectangle (0.25,2.75);
\draw[line width=1pt,dotted] (2.5,0.5) rectangle (0.5,2.5);
\draw[line width=1pt,dotted] (2.25,0.75) rectangle (0.75,2.25);
\draw[line width=1pt,dotted] (2,1) rectangle (1,2);
\draw[line width=1pt,dotted] (1.75,1.25) rectangle (1.25,1.75);

\draw (1.5,1.5) circle (2cm)
circle (1.75cm)
circle (1.5cm)
circle (1.25cm)
circle (1cm)
circle (0.75cm)
circle (0.5cm)
circle (0.25cm);

\node (0) at (-0.75,1.3) {$Q_1$};
\node (0) at (-0.75,-0.2) {$Q_2$};
\end{tikzpicture}
\end{center}
\vspace*{-12pt}
\caption{$Q_1$ and $Q_2$ representing the same location in two-dimensional Euclidean space}\label{fig:the-same-point}
\end{figure}

\begin{example}\label{ex:RO(R)}
Take the real line $\Real$ with the Euclidean topology. It is a standard result that the pair $\langle\RO(\Real),\mathord{\cont}\rangle$, where $\RO(\Real)$ is the complete algebra of regular open subsets of $\Real$ and $\mathord{\cont}$ is the standard topological interpretation of contact (as defined above) is a Boolean contact algebra.

Take $0\in\Real$. Obviously, the set:
\[
\textstyle\left\{\left(-\nicefrac{1}{n},\nicefrac{1}{n}\right)\,\middle\vert\,n\in \omega\setminus\{0\}\right\}
\]
is a~G-representative. But also:
\[
\textstyle\left\{\left(-\nicefrac{1}{n},\nicefrac{1}{n}\right)\,\middle\vert\,n\in\Odd\right\}
\]
where $\Odd\subseteq\omega$ is the set of odd numbers, and
\[
\textstyle\left\{\left(-\nicefrac{1}{r},\nicefrac{1}{r}\right)\,\middle\vert\,r\  \text{is a~positive irrational}\right\}
\]
are G-representatives standing for the same location in the one-dimensional space, i.e., number~0. Moreover, one can easily see that there are uncountably many such G-representatives.\footnote{The reader interested in philosophical issues related to Grzegorczyk points is asked to consult~\citep{Gruszczynski-Pietruszczak-SPM}.}
\end{example}



\begin{definition}
If $X,Y$ are subsets of a~BWCA, then $Y$ \emph{covers} $X$ (or $X$ is \emph{covered by} $Y$) iff for every $y\in Y$ there is $x\in X$ such that $x\leq y$. We write `$X\covers Y$' meaning $X$ covers $Y$, and `$X\trianglelefteq Y$' meaning $X$ is covered by $Y$. Let $\ncovered$ be the set-theoretical complement of $\covered$. %\footnote{This is a~modern terminology for the \emph{covering} relation of Whitehead's. To be more precise, as defined in \cite[p. 298]{Whitehead-PR}, $X$ \emph{covers} $Y$ iff for every $x\in X$ there is $y\in Y$ such that $y\leq x$. So $X$ covers $Y$ iff $Y$ is coinitial with $X$.}

For a~region $x$ of a~BWCA, let $\downop x\defeq\{y\in B\mid y\leq x\}$, i.e., $\downop x$ is the set of all parts of~$x$.
\end{definition}

The general fact that different G-representatives can represent the same location in space follows also from:

\begin{lemma}\emph{\cite[Lemma 5.6]{Gruszczynski-et-al-ASGPFT1}}
If $Q$ is a~G-representative in $\frB\in\BWCA$, then every subset of $Q$ covered by~$Q$ is also a~G-representative. In particular, for any region $x$, $Q\cap\downop x$ is a~G-representative, provided $Q\cap\downop x\neq\emptyset$.
\end{lemma}

\smallskip

In light of the above, to speak about points we need to be able to identify different G-representatives which stand for the same locus.


\subsection{Grzegorczyk points as quotients}\label{sec:G-points-as-quotients}
Let us begin with the definition:
\begin{equation}\tag{$\mathrm{df}\,\mathord{\twodownop}$}
\twodownop x\defeq\{y\in\frB\mid y\ll x\}
\end{equation}
and two lemmas.

\begin{lemma}\label{lem:coi-separ-ext}
If $Q_1$ and $Q_2$ are G-representatives in $\frB\in\BWCA$, then\/\textup{:}
\[
(\forall x\in Q_1)(\forall y\in Q_2)\,x\con y\qquad\text{iff}\qquad Q_2\covered Q_1.
\]
% \begin{enumerate}[label=(\arabic*)]
% \item[(i)] If $Q_2\ncovered Q_1$, then there are $x\in Q_1$ and $y\in Q_2$ such that $x\separ y$.
% \item[(ii)] If there are $x\in Q_1$ and $y\in Q_2$ such that $x\ext y$, then $Q_2\ncovered Q_1$.
% \end{enumerate}
\end{lemma}
\begin{proof}
(i) Assume that $Q_2$ is not covered by $Q_1$, i.e., there is $x_1\in Q_1$ such that for every $y\in Q_2$, $y-x_1\neq\zero$.\footnote{`$x-y$' abbreviates `$x\cdot-y$'.}  By \eqref{r2}, there is $x_0\in Q_1\cap\twodownop x_1$ (i.e., by definition of $\ll$ we have $x_0\separ -x_1$). Observe that for every $z,y\in Q_2$, $z\overl y-x_1$. Indeed, if $z,y\in Q_2$, we have that either (a) $z\leq y$ or (b) $y\leq z$. If (a) holds, $z-x_1\leq z$ and $z-x_1\leq y-x_1$. If (b) holds, $y-x_1\leq z$. If $(\forall z\in Q_2)\,z\overl x_0$, then by \eqref{r3} we obtain that $x_0\con y-x_1$, a~contradiction. So there is $z_0\in Q_2$ such that $z_0\ext x_0$. By \eqref{r2} again there is $z_1\in Q_2\cap\twodownop z_0$, so $z_1\separ x_0$.

\smallskip

(ii) Suppose there are $x\in Q_1$ and $y\in Q_2$ such that $x\notcon y$, but $Q_2\covered Q_1$. Take $z\in Q_2\cap\downop x$. If $z\leq y$, then $y\overl x$, and if $y\leq z$, then $y \leq x$, a~contradiction, as $x\ext y$ by the fact that $\mathord{\overl}\subseteq\mathord{\con}$, which is easily verified.
\end{proof}

In consequence we have:


\begin{corollary}\label{cor:coi-equivalence}
Let $\frB\in\BWCA$. If $Q_1$ and $Q_2$ are G-representatives and $Q_1\covered Q_2$, then $Q_2\covered Q_1$.
\end{corollary}
% \begin{proof}
% Let $Q_1\covered Q_2$ but $Q_2\ncovered Q_1$. By Lemma~\ref{lem:coi-separ-ext} applied twice, we have that for all $x\in Q_1$ and $y\in Q_2$: $x\overl y$, and there are $x_0\in Q_1$ and $y_0\in Q_2$ such that $x_0\ext y_0$, a~contradiction.
% \end{proof}

\begin{theorem}
If $\frB\in\BWCA$, then $\covered$ is an equivalence relation on the set of G-representatives.
\end{theorem}
\begin{proof}
The symmetry of $\covered$ holds by  Corollary \ref{cor:coi-equivalence}. The reflexivity and transitivity of $\covered$ follow from the reflexivity and transitivity of~$\leq$.
\end{proof}

We are now in a position to say precisely that G-representatives $Q_1$ and $Q_2$ \emph{represent the same location} if and only if $Q_1$ is covered by $Q_2$ and $Q_2$ is covered by $Q_1$.
Therefore, it is reasonable to define \emph{points} as equivalence classes of $\covered$ on the set of all G-representatives~$\prePt$ (to emphasize the fact that $Q_1$ and $Q_2$ represent the same location, i.e., are mutually covered by each another, we will write `$Q_1\sim Q_2$'):
\[
\Eq\defeq\prePt/_{\mathord{\sim}}\,.
\]

For any two sets of regions $X$ and~$Y$ such that $X$ covers $Y$ and $Y$ covers $X$, we will say that $X$ and $Y$ are \emph{coinitial}.



\subsection{Grzegorczyk points as filters}\label{sec:G-points-as-filters}
The second (chronologically the first) idea---used by \citet{Grzegorczyk-AGWP}---is to define points as filters that are generated by G-representatives:
\begin{equation}\notag
\fil\ \text{is a point iff}\ (\exists Q\in\prePt)\,\fil=\{x\in\frB\mid(\exists q\in Q)\,q\leq x\}\,.
\end{equation}
By `$\fil_Q$', we will denote a~point generated by the G-representative~$Q$. These filters will be called \emph{G-points}, and the set of all G-points will be denoted by `$\Grz$', while its elements by small fraktur letters `$\frp$', `$\frq$' and `$\frr$', indexed if necessary. For every G-point $\fil_Q$ we have:
\begin{equation}\label{eq:conditions-for-in-FQ}
  x\in\fil_Q\iffslim(\exists y\in Q)\,y\ll x\iffslim(\exists y\in Q)\,y\leq x\,.
\end{equation}

Observe that for $Q_1,Q_2\in\prePt$:
\begin{equation}
Q_1\sim Q_2\iffslim(\exists \frp\in\Grz)\,Q_1\cup Q_2\subseteq\frp\,.
\end{equation}
\begin{proof}
($\rarrow$) This follows from Corollary~\ref{cor:coi-equivalence}, the definition of G-points and~\eqref{eq:conditions-for-in-FQ}.

\smallskip

($\leftarrow$) Let $X=\fil_Q$. Since both $Q_1$ and $Q_2$ are subsets of $\fil_Q$, they must be coinitial with $Q$, and so $Q_1\sim Q_2$.
\end{proof}
Thus, as we see, by considering points as filters we can recover the equivalence relation between G-representatives. The reverse transition---from equivalence classes to filters---is obvious, since for a~given class $[Q]_{\sim}$ it is enough to take $\fil_Q$.

Let us conclude this section with an observation that there is a~1--1 correspondence between G-points as equivalence classes and G-points as filters:
\begin{lemma}
Let $\frB\in\BWCA$. The function $f\colon\Eq\fun\Grz$ such that $f([Q]_{\sim})\defeq\fil_Q$ is a~bijection.
\end{lemma}
\begin{proof}
If $[Q_1]\neq [Q_2]$, then $Q_1\ncovered Q_2$. Therefore, by Lemma~\ref{lem:coi-separ-ext}(i) there are $x\in Q_1$ and $y\in Q_2$ such that $x\ext y$. So $\fil_{Q_1}\neq\fil_{Q_2}$, since otherwise both $x$ and $y$ are in the same filter. Surjectivity is obvious, since every G-point is a filter $\fil_Q$, for some G-representative~$Q$.\end{proof}





\section{Whitehead points}\label{sec:W-points}

In this section, we present a~mathematical analysis of (a representative of) a point as formulated by \citet{Whitehead-PR}, and then used by \citet{Biacino-Gerla-CSGWDP}. Roughly, the idea is that Whitehead points are minimal elements of a poset of abstractive sets of a Boolean weak contact algebra.\footnote{More about the philosophy of and motivations for Whitehead points can be found in two excellent papers by \citet{Gerla-PFC} and \citet{Varzi-PHOC}.}

We begin with the crucial definition:

\begin{definition}\label{df:abstractive-set}
  A set of regions $A$ of a BWCA is an \emph{abstractive set} iff it satisfies \eqref{r0}, \eqref{r1} and:
  \begin{equation}\tag{A}\label{A}
    (\nexists x\in\frB)(\forall y\in A)\,x\leq y\,.
  \end{equation}
The class of all abstractive sets of a~given BWCA is denoted by `$\Abs$'. Since by \eqref{r1} and \eqref{eq:llcIngr} every abstractive set is a~chain w.r.t. $\leq$, it must be the case that for every $x\in A$ there is $y\in A$ such that $y<x$. So, by the Axiom of Dependent Choices, every abstractive set is infinite.
\end{definition}




The idea behind the definition is that we can abstract geometrical objects---like lines, segments, and points---from other entities. However, unlike representatives of Grzegorczyk's, these entities do not have to represent points  but might be planes, straight lines, line segments, triangles, and so on. To use a~simple example, we take the algebra $\RO(\Real^2)$ and the family of regular open sets of the form:
\[
\left\{\langle x,y\rangle\,\middle\vert\, y\in \left(-\nicefrac{1}{n},\nicefrac{1}{n}\right)\right\}\quad\text{for $n\in\omega\setminus\{0\}$}\,,
\]
which is an abstractive set that represents the straight line $y=0$ (i.e., some object from beyond the domain $\RO(\Real^2)$). Of course, we easily see that it is not a~G-representative, since regions:
\[
\{\langle x,y\rangle\mid x\geqslant 1\wedge y\in (-1,1)\}\quad\text{and}\quad \{\langle x,y\rangle\mid x\leqslant -1\wedge y\in (-1,1)\}
\]
overlap all regions from the abstractive set, but are not in~contact (in the sense of $\cont$ for the algebra $\RO(\Real^2)$). So the set violates~\eqref{r3}.

We use the same terminology and symbols for the \emph{covering} relation between abstractive sets that has been introduced for Grzegorczyk representatives. In particular, recall that $A$ and $B$ are coinitial in the case where $A$ is covered by $B$ and $B$ is covered by $A$.

% If this situation holds between abstractive sets then:

% \begin{proposition}\label{prop:inf-abs-coincide} Let $\frB\in\BWCA$. If $A$ and $B$ are coinitial abstractive sets, then their infima coincide.
% \end{proposition}
% \begin{proof}
% Let $x\leq\bigwedge A$. So for every $a\in A$, $x\leq a$. Pick a $b$ from~$B$. As $B$ covers $A$, there is $a\in A$ such that $a\leq b$. So $x\leq b$, and arbitrariness of $b$ entails that $x\leq\bigwedge B$. The other direction is analogous.
% \end{proof}

Unlike in the case of covering relation on G-representatives, covering on abstractive sets does not have to be an equivalence relation since it is not---in general---symmetric. However, it is reflexive and transitive, so the coinitiality on abstractive sets is an equivalence relation. Following Whitehead, we will call every element of $\Abs/_{\mathord{\sim}}$ a \emph{geometrical element}. Given $A\in\Abs$ its equivalence class w.r.t. $\sim$ will be denoted by `$[A]$'. If $A_1,A_2\in\Abs$, define a binary relation on $\Abs/_{\mathord{\sim}}$:
\[
[A_1]\preceq[A_2]\iffdef A_1\covered A_2\,.
\]
The relation $\preceq$ is clearly a~partial order. Moreover,  $A_1\sim A_2$ and $B_1\sim B_2$ together entail that: $A_1\covered B_1$ iff $A_2\covered B_2$, thus $\preceq$ is well-defined.
\begin{definition}
For $A\in\Abs$, $[A]$ is a~\emph{Whitehead point} (\emph{W-point}) iff $[A]$ is minimal in $\langle\Abs/_{\mathord{\sim}},\mathord{\preceq}\rangle$. The set of all Whitehead points will be denoted by `$\Wthd$'. $A\in\Abs$ is a \emph{W-representative} of a~point iff $[A]\in\Wthd$. Let $\prePtW$ be the set of all W-representatives of a~given BWCA.
\end{definition}

Alternatively, for an abstractive set $A$ we have:
\begin{equation}\label{eq:alternative-W-representative}
  A\in\prePtW\iffslim\forall_{B\in\Abs}\,(B\covered A\rarrow A\covered B)\,.
\end{equation}

Recall that a BWCA is \emph{atomic} iff its underlying BA is atomic iff every non-zero region contains an atom (i.e., an element that is minimal w.r.t. the standard Boolean order). As an immediate consequence of the definition of an abstractive set we get the following:
\begin{corollary}
    If $\frB\in\BWCA$ is atomic, then $\frB$ does not have any abstractive sets, more so it does not have W-representatives.
\end{corollary}


\subsection{W-representatives in regular open algebras}

If $\RO(X)$ is a~regular open algebra and $A$ is its W-representative, then of course $\bigwedge A=\zero$. However, this does not exclude the possibility in which $\bigcap A\neq\emptyset$, as $\bigwedge A=\Int\bigcap A$. Thus we may ask about set-theoretical intersections of abstractive sets.
\begin{lemma}
If $X$ is a topological space, and $\langle\RO(X),\mathord{\cont}\rangle$ is its topological contact algebra, then for every abstractive set $A\subseteq\RO(X)$, $\bigcap A$ is closed.  Therefore if $\bigcap A\neq\emptyset$ and $\bigcap A\in\RO(X)$, then the space $X$ is disconnected.
\end{lemma}
\begin{proof}
Fix an abstractive set $A$ whose elements are from a~$\langle\RO(X),\mathord{\cont}\rangle$. According to the characterization of $\llt$, $A$ and $\Cl[A]=\{\Cl a\mid a\in A\}$ are coinitial. Indeed, if $a\in A$, then by \eqref{r2} there is $b\in A$ such that $b\llt a$, i.e., $\Cl b\subseteq a$. So $A$ covers $\Cl[A]$. The other direction is obvious since $a\subseteq\Cl a$. Thus,  $\bigcap A=\bigcap\Cl[A]$, and in consequence $\bigcap A$ is closed in~$X$.

Since the infima in $\RO(X)$ are given by the interiors of the intersections, if $X$ is connected, $\bigcap A$ is never an element of the algebra if non-empty.
\end{proof}


This lemma gives rise to a~philosophical interpretation of abstractive sets. If the underlying regular algebra is composed of sets that are models of objects from the physical space (spatial bodies), it usually is a sub-algebra of $\RO(\Real^n)$, where $\Real^n$ is given the standard topology. Various choices are possible\footnote{See for example \citep{DelPiero-CFRUSCM,DelPiero-NCFR}, \citep{Lando-et-al-ACORRBMAT}.}, yet irrespective of these for no abstractive set $A\subseteq\RO(\Real^n)$, $\bigcap A\neq\emptyset$. In this sense abstractive sets represent objects from beyond the universe of models of spatial bodies, i.e., serve as abstraction processes to introduce objects that may be called \emph{geometrical}, \emph{ideal} or, precisely, \emph{abstract}. These objects are, of course, elements of the power set algebra of $\Real^n$, but the idea is that there are <<too many>> objects in $\power(\Real^n)$ from the perspective of the physical space, yet some of the elements of $\power(\Real^n)$ can be treated as approximations made via elements of subalgebras of the regular open algebra of~$\Real^n$.

The definition of a W-representative from the point of view of Euclidean spaces seems to be neat and grasp a certain way in which we may abstract points as higher-order objects. However, in the sequel, we will point to <<strange>> examples. But first, we prove that there are contact algebras that have W-points. Their existence stems from the following:
\begin{theorem}\label{th:local-basis-is-W}
    Let $\langle X,\topo\rangle$ be a topological space. If $A$ is an abstractive set in $\langle\RO(X),\mathord{\cont}\rangle$ that is at the same time a local basis at a~point $p\in X$, then $A$ is a W-representative.
\end{theorem}
    \begin{proof} Suppose $A$ is an abstractive set  in $\langle\RO(X),\mathord{\cont}\rangle$ and a local basis at~$p$. Let $B \subseteq \RO(X)$ be an abstractive set such that $B\covered A$. We show that $p\in\bigcap B$. Suppose otherwise, i.e., let $b_0 \in B$ be such that $p \notin b_0$. Since $B$ is an abstractive set it follows that there exists a $b_1 \in B$ such that $b_1 \llt b_0 $ i.e., $\Cl b_1\subseteq b_0$ and thus $p \notin \Cl  b_1$. Therefore, we have $p \in  X \setminus  \Cl  b_1$, where $X \setminus  \Cl  b_1$ is an open set in $\topo$. It follows that there exists an $a \in A$ such that $a \leq X\setminus  \Cl  b_1 $ and  $p \in a$. Hence,  $a \cdot b_1 = \zero$. In consequence, there exists no $b \in B$ such that $b \leq a$, which contradicts our initial assumption that $A$ covers $B$.

    Since  $p \in\bigcap B$ and $A$ is a~local basis at~$p$, we know that for every $b \in B $ there exists an $a \in A$ such that $a \subseteq b$, so $A$ must be covered by~$B$. Thus,  $A$ is a~W-representative.
\end{proof}
In consequence we have:
\begin{corollary}
    The real line with the standard topology has a W-representative at every point of the space.
\end{corollary}

\begin{definition}[\citealp{Davis-SLOLB}]
    A \emph{lob-space} is a~topological space that at every of its point has a~local basis linearly ordered by the subset relation.
\end{definition}
\begin{definition}[\citealp{Gruszczynski-NTP}]
    A topological space~$X$ is \emph{concentric} iff it is $T_1$ and at every $p\in X$ there is a local basis $\Basis^p$ such that:
    \begin{equation}\tag{R1}\label{eq:R1}
        (\forall U,V\in\Basis^p)\,(U=V\vee \Cl U\subseteq V\vee\Cl V\subseteq U)\,.\qedhere
    \end{equation}
\end{definition}
Thus, concentric spaces are those $T_1$-spaces whose all points have local bases that satisfy the topological version of \eqref{r1} condition for G-representatives. The theorem below demonstrates that these are a subclass of Davies's lob-spaces.
\begin{theorem}[\citealp{Gruszczynski-et-al-GPFBCA}]
    A topological space $X$ is concentric iff it is a~regular lob-space.
\end{theorem}

\begin{theorem}\label{th:concentric-W-representatives}
    If $X$ is a concentric space whose regular open algebra is atomless, then at every point there is a local basis that is a W-representative.
\end{theorem}
\begin{proof}
Since $X$ is regular, it is also semi-regular, so $\RO(X)$ is its basis, which is atomless by assumption. Therefore, the local basis at any point $p$ that satisfies \eqref{eq:R1} must be an abstractive set. So by Theorem~\ref{th:local-basis-is-W} the basis must be a~W-representative.
\end{proof}

% \begin{corollary}
%     If $X$ is a metrizable space in which the closure of an open ball $B$ with the radius $r$ is precisely the closed ball with the same radius, then at every point there is a local basis that is a W-representative.
% \end{corollary}
% \begin{proof}
% If $d$ is a metric inducing the topology on $X$, then connectedness of $X$ entails that for every point $p$ of $X$, $B_d(p,\nicefrac{1}{n+1})\subsetneq B_d(p,\nicefrac{1}{n})$, for $n\in\omega\setminus\{0\}$.
% \end{proof}

% \begin{problem}\marginpar{\tiny New problem 30.11.2022}
%   In which topologies induced by a metric, $\{B_d(x,\nicefrac{1}{n})\mid n\in\omega\setminus\{0\}\}$ is a W-representative? Firstly, in which metric spaces this set is an abstractive set? For example, it not an abstractive set in any discrete topology as it must be finite, as it contains $\{x\}$ and the whole space only.
% \end{problem}

Moreover, we have the following result regarding W-representatives, which shows that in the case of topological interpretation, they represent <<small>> chunks of the underlying space.

\begin{lemma}
If $X$ is a regular topological space, and $\langle\RO(X),\mathord{\cont}\rangle$ is its topological contact algebra, then for every W-representative  $A\subseteq\RO(X)$, $\bigcap A$ is a nowhere dense subset of $X$.\footnote{The observation and the proof that $\bigcap A$ is nowhere dense is due to \citet{Hart-APONOIOCCOROS}.}
\end{lemma}
\begin{proof}
Assume  that $x\defeq\bigcap A$ has a~non-empty interior. Therefore, there is a~non-empty regular open set $y$ such that $\Cl y\subseteq\Int x$. This in particular means that $y\llt a$, for all $a\in A$. Since the space is regular and the algebra atomless, we can construct a~sequence such that $y_0\defeq y$ and $y_{n+1}\llt y_n$. In consequence for $Y\defeq\{y_n\mid n\in\omega\}$ we have that $A$ covers $A\cup Y$ but not vice versa, since no element of $Y$ contains an element of $A$. So $A$ is not a W-representative.
\end{proof}

What is common for all W-representatives whose existence follows from the above result is that although they do not have infima as subsets for regular open algebras, they do have a non-empty intersection that is precisely the point of the space that they represent as a basis. This raises the question  whether there is a~regular open algebra that has a W-representative whose set-theoretical intersection is non-empty and that may be interpreted as a~<<new>> point, i.e., something similar to a free ultrafilter being treated as a point of a topological space. The answer is positive, and the example is due to Klaas Pieter Hart.

\begin{example}[\citealp{Hart-APONOIOCCOROS}]\label{ex:Hart}
Consider the ordinal space $X\defeq[0,\omega_1)$, where $\omega_1$ is the first uncountable ordinal. Recall that if $x$ and $y$ are closed and unbounded subsets of $X$, then $x\cap y\neq\emptyset$. Due to this, for any open subsets $x$ and $y$ of $X$, if $\Cl x\subseteq y$, then either $\Cl x$ is compact or $y$ contains an interval $[\alpha+1,\omega_1)$, for some $\alpha<\omega_1$. For if $\Cl x$ is not compact, then it must be unbounded, and since $\Cl x\cap\complement y=\emptyset$,  $\complement y$ is bounded, i.e., there is $\alpha<\omega_1$ such that $\complement y\subseteq[0,\alpha]$. Therefore $[\alpha+1,\omega_1)\subseteq y$. The following set $A\defeq\{[\alpha+1,\omega_1)\mid\alpha<\omega_1\}$ consists of clopen---and the more so regular open---subsets of $X$, and is an abstractive set. If $B$ is also an abstractive set such that $A$ covers $B$, then every element $b\in B$ must be unbounded. The more so the closure of every element of $B$ is unbounded, and since for every $b\in B$ there is $b_0$ in $B$ such that $\Cl b_0\subseteq b$, $b$ must contain an interval $[\alpha+1,\omega)$. Thus $B$ covers $A$, and so $A$ is a W-representative in $\RO(X)$. Of course, $\bigcap A=\emptyset$, and the W-point $[A]$ represents the ordinal $\omega_1$ that is absent from $X$.
\end{example}

This example is quite important from the point of view of the hidden assumptions behind Whitehead points.  \citet[p.\,30]{Bostock-WAROP} writes that:
\begin{quote}
[\ldots] Whitehead's construction [\ldots] does actually have the idea of boundedness built into it: only a bounded nest\footnote{A nest is the counterpart of an abstractive set, it is bounded if it contains only bounded regions (actually it is enough that it contains one such region to be considered bounded). } can satisfy Whitehead's definition of a point-nest. (But I do not suppose that Whitehead recognised this.)
\end{quote}
What the example shows then is that boundedness is not built into the idea of Whitehead points. It only is as far as <<natural>> spaces---like Euclidean spaces---are considered, which follows from further results and properties of Grzegorczyk points proven in \citep{Gruszczynski-NTP,Gruszczynski-et-al-ASGPFT1,Gruszczynski-et-al-ASGPFT2}. In an abstract setting, relevant for this paper, the notion of \emph{boundedness} does not have to be considered, as \citeauthor{Hart-APONOIOCCOROS}'s example shows. Also, this example shows that the \emph{connectedness} of regions that constitute W-representatives is not built into the general idea of points in the sense of Whitehead, as every region in the W-representative from the example is topologically disconnected.\footnote{The issue of connectedness of regions in Whitehead's theory is elaborately discussed in \citep{Bostock-WAROP}.}



Let us make another philosophical remark at this point. It may also be the case that we do not know whether a particular subset of a regular open algebra is a W-representative  due to our current state of knowledge. Consider the following example.

\begin{example} Let $\RO(\Real)$ be the complete algebra of  regular open subsets of $\Real$ and $\langle\RO(\Real),\mathord{\cont}\rangle$ its topological contact algebra. Then, consider the following set.
\[
A\defeq\{(-\nicefrac{1}{p}, \nicefrac{1}{p}) \mid  p\defeq \mbox{ max}\{s,t\} \mbox{ where } s,t \mbox{ are  twin primes} \}\,
\]
The twin prime conjecture, i.e., the claim that there exist infinitely many twin primes, is still an unsolved problem within number theory. So we do not know whether $A$ is finite or infinite. This implies that we also do not know  whether $A$ is an abstractive set and thus  a W-representative.\end{example}


Observe that there are BCAs without any W-representatives and, therefore, without any Whitehead points.

\begin{definition}
  An ordinal number $\alpha$ is \emph{even} iff there is an ordinal number $\beta$ such that $\alpha=2\cdot\beta$ (where $\cdot$ is the standard ordinal multiplication). Otherwise, it is \emph{odd}. Let $\Even_\kappa$ and $\Odd_\kappa$ be, respectively, the set of all even and odd ordinals smaller than~$\kappa$.
\end{definition}

\begin{lemma}\label{lem:no-W-points-atomless-C=O} No complete $\frB \in \BCA$ in which $\mathord{\con}=\mathord{\overl}$ has W-representatives.\footnote{If, additionally, $\frB$ is atomless, then it does not have any G-representative either, which is entailed by Theorem~\ref{th:Q-and-A-is-W} proven further.}
\end{lemma}

\begin{proof}
If $\frB$ is finite, then it cannot have any abstractive sets. The more so it cannot have W-representatives.

So suppose $\frB$ is infinite, and let $\langle x_\alpha\mid\alpha<\kappa\rangle$ be an abstractive set, for some limit cardinal $\kappa$. Since we consider the case in which contact is overlap, we have that:
\[
x_0 > x_1>\ldots>x_n>x_{n+1}>\ldots>x_{\beta}>x_{\beta+1}>\ldots
\]
is an abstractive set. For any $\alpha<\kappa$ define: $y_\alpha\defeq x_\alpha-x_{\alpha+1}$ and consider the antichain $\langle y_\alpha\mid \alpha<\kappa\rangle$. Let $\Odd_\kappa$ and $\Even_\kappa$ be, respectively, all odd and all even ordinals smaller than $\kappa$. Divide the sequence into two sub-sequences:
\[
\langle u_\alpha\mid\alpha\in\Odd_\kappa\rangle\quad\text{and}\quad\langle v_\alpha\mid\alpha\in\Even_\kappa\rangle\,,
\]
take the following suprema:
\[
a_\beta\defeq\bigvee\{u_\alpha\mid\alpha\in\Odd_\kappa\setminus (\Odd_\kappa\cap(2\cdot\beta+1))\}\,,
    % b_\beta&{}\defeq\bigvee\{u_\alpha\mid\alpha\in\Even_\kappa\setminus (\Even_\kappa\cap(2\cdot\beta))\}\,.
\]
and gather them into $A\defeq\{a_\beta\mid\beta<\kappa\}$. $A$ is an abstractive set covered by $\langle x_\alpha\mid\alpha<\kappa\rangle$, but does not cover this sequence. Therefore the sequence is not a W-representative. In consequence, no abstractive set is a~Whitehead representative.
\end{proof}


Let us round off this section with the following remarks.\label{page:purely-mereological} Lemma~\ref{lem:no-W-points-atomless-C=O} is a mathematical embodiment of what \cite{Whitehead-CN} discovered himself: the purely mereological notion of \emph{parthood} is too weak to represent his concept of \emph{point} as a~collection of regions. Some arguments for this can be found in Whitehead's book, \citep{Bostock-WAROP} and \citep{Varzi-PHOC}. However, their common weaknesses are that (a) they refer to particular kind of regions that invoke the notions of \emph{dimension} and of \emph{shape} (either explicitly or implicitly) and (b) they do not single out precise assumptions. These are arguments, not proofs in a strict mathematical sense. We present a fully-fledged proof, which is general in the sense that we consider regions as abstract elements of any Boolean algebra. What remains to be eliminated is the assumption of completeness. Thus, we put forward the following open problem:
\begin{problem}
Is there an incomplete BCA in which both $\mathord{\con}=\mathord{\overl}$ and there exists a W-representative?
\end{problem}
Observe that Lemma~\ref{lem:no-W-points-atomless-C=O} does not exclude such algebras, as the property of being a W-representative does not have to be preserved for completions of BAs. That is, if $\frB$ is an incomplete BA that has a W-representative~$A$, and $\cl{\frB}$ is the completion of $\frB$, then the structure of~$A$ is preserved by the canonical embedding $e\colon\frB\to\cl{\frB}$. In consequence, we can repeat the reasoning from Lemma~\ref{lem:no-W-points-atomless-C=O} and show that $e[A]$ is not a W-representative.

\section{G-representatives are W-representatives (under additional assumptions)}\label{sec:G-points-are-W-points}

In this section, we are occupied with two problems: (a) what are the minimal conditions for BWCAs that guarantee that every G-representative is a W-representative, and (b) what is the content of the second order monadic statement about the dependency between the two sets of representatives. Theorem \ref{th:Q-and-A-is-W} below is a stronger version of \cite[Theorem 5.1]{Biacino-Gerla-CSGWDP}. \citeauthor{Biacino-Gerla-CSGWDP}'s proof to establish that every G-representative is a W-representative uses the second-order constraints that postulate the existence of Grzegorczyk points. These are their axioms $\mathrm{G}_4$ and $\mathrm{G}_5$, closely related to the original Grzegorczyk axiom from his paper.\footnote{See \citep{Gruszczynski-et-al-ACOTSOPFT} for a comparison of the original axiomatization of \citeauthor{Grzegorczyk-AGWP}'s system with the system of \citeauthor{Biacino-Gerla-CSGWDP}'s.} We prove that the original result of the Italian mathematicians can be substantially improved, as we only assume the axioms for the weak contact relation plus:
\begin{gather*}
(\forall x\neq\zero)(\exists y\neq\zero)\,y\ll x\,,\tag{C5}\label{C5}
\end{gather*}
and the atomlessness of the underlying Boolean algebra. \eqref{C5}  is known as the \emph{non-tangential part} axiom, and  it is equivalent---by \eqref{df:ll}---to:
\begin{equation*}
 %\tag{C5n}\label{C5n}
(\forall x\neq\one)(\exists y\neq\zero)\,x\notcon y\,,
\end{equation*}
the so-called \emph{disconnection} axiom.
In the class $\BWCA$ both these axioms are equivalent to the \emph{extensionality} axiom:
\begin{equation*}%\tag{C5e}\label{eq:C-extensionality}
    (\forall z\in B)\,(z\con x \iffslim z\con y)\rarrow x=y\,.
\end{equation*}







Moreover, we show that in the class $\BWCA+\eqref{C5}$ the second-order monadic statement `every G-representative is a W-representative' is equivalent to the first order condition `there are no atoms'. Additionally, in Theorem~\ref{th:independence-no-atoms-G-are-W} we demonstrate that both the implications fail if we omit the axiom~\eqref{C5}. Thus, via the two theorems, we provide answers to both (a) and (b) above.


The first requirement that G-representatives must meet to be W-representatives is that they are abstractive sets. In general, it does not have to be true: there are contact algebras with G-representatives that are not abstractive sets since the former do not have to satisfy~\eqref{A}. \citeauthor{Biacino-Gerla-CSGWDP} do not have to take this into account since their definition of G-representative contains the requirement that it is a set of regions without the minimal element. This, however, is the assumption that is absent from the definition introduced in the original paper by \citeauthor{Grzegorczyk-AGWP}.

In connection with this we have:
\begin{proposition}[\citealp{Gruszczynski-Pietruszczak-GSPFT}]\label{prop:atom-in-Q}
If $\frB\in\BWCA+\eqref{C5}$ and $\frB$ has an atom~$a$, then $\{a\}\in\prePt$.
\end{proposition}
\begin{proof}
Fix an atom~$a$. By \eqref{C5} there exists a~non-zero $b\in B$ such that $b\ll a$. So $b\leq a$ by \eqref{eq:llcIngr}, and thus $b=a$. From this, we can see that the conditions \eqref{r0}--\eqref{r2} are satisfied. For \eqref{r3}, if $x\overl a$ and $y\overl a$, then $a\leq x$ and $a\leq y$, so $x\overl y$.
\end{proof}
So, if a~BWCA has atoms and satisfies \eqref{C5} there are G-representatives, which are not abstractive sets. Thus, in general, it is not the case that $\prePt\subseteq\Abs$, and the natural thought is to eliminate the existence of atoms, especially due to the fact that G-points generated by atoms are---in a way---not very interesting, similarly as are not principal ultrafilters.

Before we do this, we prove a~propoisition that will help us establish the main results of this section.


\begin{proposition}\label{prop:A-r2s}
Let $\frB\in\BWCA$.
\begin{enumerate}[label=(\arabic*)]
\item[(i)] If $A\in\Abs$, then $A$ satisfies the strong version of~\eqref{r2}\/\textup{:}
  \begin{equation}\tag{r2${}^{\mathrm{s}}$}\label{r2s}
  (\forall x\in A)(\exists y\in A)\,(y\ll x\wedge y\neq x)\,.
  \end{equation}
\item[(ii)] If $X$ is a set of regions such that $X\covered Q$ for some G-representative $Q$, then $X$ satisfies \eqref{r3}.
\item[(iii)] \label{cor:A-coi-Q-is-G-rep} If $A\in\Abs$, $Q\in\prePt$ and $A\covered Q$, then $A\in\prePt$ and $[A]_{\mathord{\sim}}=[Q]_{\mathord{\sim}}$.
\end{enumerate}
\end{proposition}
\begin{proof}
(i) For every $x\in A$ there is a $y\in A$ such that $y<x$. But, by \eqref{r1}, either $x\ll y$ or $y\ll x$. Since the former cannot hold in light of \eqref{eq:llcIngr}, we have the latter.

\smallskip

(ii) Assume that for all $x\in X$, $x\con u$ and $x\con v$. If $q\in Q$, then by covering of $X$ by $Q$~and by \eqref{C3} we have that $q\con u$ and $q\con v$. Therefore $u\con v$, by \eqref{r3} for~$Q$.

\smallskip

(iii)  Follows from the previous items (i), (ii) and Corollary~\ref{cor:coi-equivalence}.
\end{proof}

\begin{proposition}\label{prop:separated-parts} If $\frB\in\BWCA+\eqref{C5}+\eqref{eq:no-atoms}$, then every non-zero region of $B$ has two proper parts that are separated from each other.
\end{proposition}
\begin{proof}
Let $x$ be a non-zero element of $\frB$. Since the algebra has no atoms, there is a non-zero $y$ that is a proper part of $x$. So, by the Boolean axioms, there is another non-zero $z<x$ that is incompatible with $y$. But by \eqref{C5}, $z$ must have a~non-tangential part~$z_0$. So $z_0\notcon y$, and both regions are parts of~$x$.
\end{proof}


\begin{theorem}\label{th:Q-and-A-is-W}
If $\frB\in\BWCA+\eqref{C5}$, then $\frB$ is atomless iff in $\frB$ every G-representative is a W-representative.
\end{theorem}
\begin{proof}
Suppose $\frB\in\BWCA+\eqref{C5}+\eqref{eq:no-atoms}$. Observe that if $Q$ is a G-representative of an algebra $\frB$ from the class, then $Q$ is an abstractive set by Proposition~\ref{prop:separated-parts}. Further, $[Q]$ is a~geometrical element. Suppose $A\in\Abs$ is such that $[A]_{\mathord{\sim}}\preceq [Q]$, i.e., $A\covered Q$. Therefore by Proposition~\ref{prop:A-r2s} we have that $A\in[Q]$, which means that $[A]=[Q]$, as required.

On the other hand, if $\frB \in \BWCA+\eqref{C5}$  and $\frB$ has an atom $a$, then by Proposition~\ref{prop:atom-in-Q}, $\{a\}\in\prePt$. Thus $\prePt\nsubseteq\prePtW$.
\end{proof}

The axiom \eqref{C5} cannot be dropped, even in the class of Boolean contact algebras, that is:
\begin{theorem}\label{th:independence-no-atoms-G-are-W}
    There is a $\frB\in\BCA+\neg\eqref{C5}$ in which there are no atoms, and in which $\prePt\nsubseteq\prePtW$; and there is also an algebra from the same class that has atoms and in which $\prePt\subseteq\prePtW$.
\end{theorem}


To\label{page:d-contact} prove the first part of the theorem we provide a general method for constructing contact algebras. Given a $\frB\in\BA$, let $\boldd$ be its non-zero element that we call \emph{distinguished}. By means of it, we define the following relation:
\[
x\cond y\iffdef x\overl y\vee(x\overl\boldd\wedge y\overl\boldd)\,.
\]
It is routine to verify that $\cond$ is a contact relation, i.e., satisfies axioms \eqref{C0}--\eqref{C4}. Observe that the largest contact relation on $\frB$, i.e., $\frB^+\times \frB^+$, is a special case of $\cond$ in which $\boldd=\one$, or more generally, where $\boldd$ is a~\emph{dense} region in $\frB$ (i.e., such that every non-zero $x$ overlaps it).

We have that:
\begin{align*}
x\lld y&{}\iffslim x\leq y\wedge (x\leq-\boldd \vee \boldd\leq y)\\
&{}\iffslim x\leq y-\boldd \vee x+\boldd\leq y\,.
\end{align*}
from which it follows immediately that:
\begin{equation}\label{eq:d-ll-d}
\boldd\lld\boldd\,.
\end{equation}
We also have that:
\begin{corollary}\label{prop:C5-fails-for-d}
    If $x<\boldd$ and $x\neq\zero$, then $x$ does not have any non-tangential part. In consequence any Boolean contact algebra $\langle\frB,\mathord{\cond}\rangle$ fails to satisfy \eqref{C5}.
\end{corollary}



\begin{lemma}\label{lem:d-G-representative}
$\{\boldd\}$ is a G-representative of $\langle B,\mathord{\cond}\rangle$, and $\uparrow\{\boldd\}$ is its only G-point.
\end{lemma}
\begin{proof}
\eqref{r0} holds by the definition of $\boldd$, \eqref{r1} and \eqref{r2} by \eqref{eq:d-ll-d}, and \eqref{r3} by the definition of $\cond$. In consequence, $\uparrow\{\boldd\}$ is a G-point.

Suppose there is a G-point $\frp\neq{}\uparrow\{\boldd\}$. By \citep[Fact 6.28]{Gruszczynski-NTP} there are $x\in\frp$ and $y\geq\boldd$ such that $x\notcond y$, i.e., $y\lld -x$. Therefore either $y\leq-\boldd$ or $\boldd\leq-x$. The first possibility is exlcuded by the fact that $\boldd\neq\zero$ and $\boldd$ is below $y$. Therefore the second one holds, and thus $x\leq-\boldd$.
\end{proof}

\begin{proof}[Proof of the first part of Theorem~\ref{th:independence-no-atoms-G-are-W}]
Take any atomless Boolean algebra $\frB$, fix its distinguished element $\boldd$, and expand it to the Boolean contact algebra $\langle\frB,\cond\rangle$. By Lemma~\ref{lem:d-G-representative}, the singleton $\{\boldd\}$ is a G-representative that is finite and therefore cannot be a~W-representative. By Proposition~\ref{prop:C5-fails-for-d}, the algebra has regions without non-tangential parts, so it fails to satisfy~\eqref{C5}.
\end{proof}

\begin{proof}[Proof of the second part of Theorem~\ref{th:independence-no-atoms-G-are-W}] Consider the following two contact algebras, $\langle\RO(\Real),\mathord{\cont}\rangle$ and the four element Boolean algebra $\frB_4\defeq\{\zero,a,b,\one\}$ with the full contact relation $\con_\one$ (i.e., $\boldd\defeq\one$). Consider their product\label{page:P-algebra} $\frP\defeq \RO(\Real)\times\frB_4$ as Boolean algebras (i.e., all the operations are defined coordinate-wise) but with the contact defined as:
\[
\langle x,u\rangle\con\langle y,w\rangle\iffdef x\cont y\vee u\con_1 w\,.
\]
It is routine to verify that $\con$ satisfies \eqref{C0}--\eqref{C4}.\footnote{This is not, then, the product of the two algebras as the \emph{contact} algebras. The relation $\langle x,u\rangle\mathrel{R}\langle y,w\rangle\iffdef x\cont y\wedge u\con_1 w$ is not contact, as the reader may easily convince themself.} We have that:
\begin{align*}
\langle x,u\rangle\ll\langle y,w\rangle&{}\iffslim x\llt y\wedge u\ll_\one w\\
&{}\iffslim x\llt y\wedge (u=\zero\vee w=\one)\,.
\end{align*}
The algebra has two atoms: $\langle\zero,a\rangle$ and $\langle\zero,b\rangle$. $\frP$ does not satisfy \eqref{C5}, as none of the two atoms is its own non-tangential part. In consequence, neither the singleton of the former nor the singleton of the latter is a G-representative.

Observe that the set of G-representatives of $\frP$ contains all sets of the form $Q\times\{\zero\}$, where $Q$ is a~G-representative of $\langle\RO(\Real),\mathord{\cont}\rangle$. It is quite obvious that \eqref{r0}--\eqref{r2} are satisfied by $Q\times\{\zero\}$. As for \eqref{r3}, if we have pairs $\langle x,u\rangle$ and $\langle y,w\rangle$ that overlap every element of $Q\times\{\zero\}$, then for all $z\in Q$, $x\overl z$ and $y\overl z$, so $x\cont y$, which is enough to conclude that $\langle x,u\rangle\con\langle y,w\rangle$.


The only products that are G-representatives in the algebra are sets of the form $Q\times\{\zero\}$, where $Q$ is a G-representative in $\RO(\Real)$. Firstly, neither $Q\times\{a\}$ nor $Q\times\{b\}$ can be G-representatives, as no element of any of the two sets has a non-tangential part. Secondly, any set of the form $Q\times\{\one\}$, where $Q$ is a G-representative in $\RO(\Real)$, fails to satisfy \eqref{r3}. To see this, take any regular open set $x$ that overlaps every element of $Q$ and consider pairs $\langle x,0\rangle$ and $\langle 0,\one\rangle$. We see that for any $\langle y,\one\rangle\in Q\times\{\one\}$, $\langle x,0\rangle\overl \langle y,\one\rangle$ and $\langle 0,\one\rangle\overl \langle y,\one\rangle$, yet $\langle x,0\rangle\notcon\langle 0,\one\rangle$. Thirdly, any product $Q\times M$, where $M$ is an at least two element subset of $\frB_4$, fails to be a chain and so cannot be a G-representative. Fourthly, if we have a set $M\times\{\zero\}$ where $M$ is not a G-representative, i.e., it fails to meet one of the conditions \eqref{r0}--\eqref{r3}, then since $\zero\ll_\one\zero$, $M\times\{\zero\}$ also fails to meet one of the four conditions for~$\ll$.

Of course, every $Q\times\{\zero\}$ is an abstractive set, and in the case where it covers an abstractive set $A\times\{0\}$, $Q$ must cover $A$ in $\RO(\Real)$. So  by Proposition~\ref{prop:A-r2s} (iii) $A$ is a W-representative in $\RO(\Real)$, and so $A\times\{\zero\}$ is a W-representative in $\frP$.
\end{proof}

\section{W-representatives with countable coinitiality are G-representatives }\label{sec:W-points-are-G-points-countable}

In this section, we reconstruct \citeauthor{Biacino-Gerla-CSGWDP}'s proof according to which every W-representative that can be represented by an $\omega$-sequence (in the sense explained below) is also a G-representative. We aim to show that with respect to the original proof from \citep{Biacino-Gerla-CSGWDP} we need to assume the so-called \emph{coherence} axiom to assure that the machinery works properly.

To be more precise, \citeauthor{Biacino-Gerla-CSGWDP} in Theorem 5.3 prove that if we extend the standard axiomatization for contact with the \emph{interpolation} axiom\footnote{They call it the \emph{normality} axiom.}:
\begin{equation}\tag{IA}\label{IA}
  x\ll y\rarrow(\exists z\in\frB)\,x\ll z\ll y\
\end{equation}
we can prove that every Whitehead representative that can be represented as an $\omega$-sequence is a Grzegorczyk representative. However, to show that a certain sequence of regions is an abstractive set they make a~transition that cannot be justified without an application of the so-called \emph{coherence} axiom, that we introduce below. Thus, in the premises of Theorem~\ref{th:BG-improved} we explicitly assume coherence in the form of \eqref{C6} below. Coherence is a~mereotopological counterpart of topological connectedness.\footnote{See, e.g., \citep{Bennett-Duntsch-AAT} for details.}


\begin{definition}
For a~given chain $C$ let the \emph{coinitiality} of $C$ be the smallest cardinal number~$\kappa$ such that there exists an antitone function $f\colon\kappa\fun C$ with $f[\kappa]$ coinitial with~$C$.
\end{definition}

\begin{definition}
  For a~given $\frB\in\BWCA$, let $\prePtW^\omega$ be the set of all Whitehead representatives whose coinitiality is~$\omega$.
\end{definition}

Observe that the set~$A$ from Example~\ref{ex:Hart} is an instance of a~W-representative whose coinitiality is $\omega_1$. On the other hand, a~local basis of any point $r\in\Real$ (with the standard topology) that satisfies \eqref{eq:R1} is a~W-representative that is an element of $\prePtW^\omega$ in $\RO(\Real)$.

\begin{definition}\label{df:coherent}
% A non-zero region $x$ of a $\frB\in\BWCA$ is \emph{coherent} iff for any its non-zero parts $y$ and $z$ such that $x=y+z$, $y\con z$, i.e. $x$ cannot be represented as the sum of two of its separated non-zero parts.
A Boolean weak contact algebra is \emph{coherent}\footnote{The term is taken from \citep{Roeper-RBT}.} iff its unity is coherent, iff it satisfies the following \emph{coherence} axiom:
\begin{equation}\tag{C6}\label{C6}
  x\notin\{\zero,\one\}\rarrow x\con -x\,.\qedhere
\end{equation}
\end{definition}


\begin{proposition}\label{prop:C6-ll-in-part} In the class $\BWCA$, \eqref{C6} is equivalent to\/\textup{:}
  \begin{equation*}%\tag{C6n}\label{C6n}
  x\notin\{\zero,\one\}\wedge x\ll y\rarrow x<y\,.
  \end{equation*}
\end{proposition}
\begin{proof}
($\rarrow$) If $x$ is neither $\zero$ nor $\one$, then $x\con-x$. Assume that $x\ll y$, i.e., $x\notcon-y$. If $y=\one$, then $x<y$. So suppose $y\neq\one$. Since $x\leq y$, also $y\neq\zero$ and $y\con-y$. Therefore $x\neq y$, which means that $x<y$, as required.

\smallskip

($\larrow$) If $x\notin\{\zero,\one\}$ and $x\notcon-x$, then $x\ll x$, and $x<x$ by the assumption. A~contradiction.
\end{proof}


Recall that the following condition holds in every BCA:
\begin{equation}\label{eq:ll-sum-prod}
x\ll u\wedge y\ll v\rarrow x\cdot y\ll u\cdot v\,.
\end{equation}

\begin{proposition}\label{prop:for-key-theorem}
    In every $\frB\in\BCA$, \eqref{C6} is equivalent to the following condition\/\textup{:}
    \[
    x\neq u\wedge x\ll u\wedge y\ll v\rarrow x\cdot y\ll u\cdot v\wedge x\cdot y\neq u\cdot v\,.
    \]
\end{proposition}
\begin{proof}
($\rarrow$) Assume \eqref{C6}. If $x\neq u\wedge x\ll u\wedge y\ll v$, then by \eqref{eq:ll-sum-prod} we have that $x\cdot y\ll u\cdot v$, so by \eqref{C6} we obtain: $x\cdot y\neq u\cdot v$,

\smallskip

($\larrow$) Suppose there is $x\notin\{\zero,\one\}$ such that $x\ll x$. Since $x\ll\one$ and $x\neq\one$, by the hypothesis we have that $x\cdot x\neq x\cdot\one$, a contradiction.
\end{proof}

\begin{theorem}[after \citealp{Biacino-Gerla-CSGWDP}]\label{th:BG-improved}
  If $\frB\in\BCA+\eqref{IA}+\eqref{C6}$, then $\prePtW^\omega\subseteq\prePt$.
\end{theorem}
\begin{proof}
Let $A$ be an~abstractive set, and let $(x_i)_{i\in\omega}$ be its coinitial subset. Assume it is not  a~G-representative, i.e., that it fails to satisfy~\eqref{r3}. Let then~$u$ and~$v$ be such that for every $i\in\omega$, $u\overl x_i\overl v$, but $u\ll-v$. Observe that $u\neq\zero\neq v$. By~\eqref{IA} (and by the Axiom of Dependent Choices), there is a~sequence $(u_i)_{i\in\omega}$ such that:
\[
u\ll\ldots u_2\ll u_1\ll u_0=-v\,.
\]
We have that for every $i,j\in\omega$, (a) $u_i\cdot x_i\neq\zero$ and (b) $x_j\nleq u_i\cdot x_i$ (since $u_i\cdot x_i\ll -v$ and $x_j\overl v$, i.e. $x_j\nleq-v$). Observe that $(u_i\cdot x_i)_{i\in\omega}$ is an abstractive set. \eqref{r0} is a~consequence of~(a), \eqref{r1} holds for the sequence because of \eqref{eq:ll-sum-prod}: $u_{i+1}\cdot x_{i+1}\ll u_i\cdot x_i$. But $u_i\cdot x_i\neq\one$, so thanks to Proposition~\ref{prop:C6-ll-in-part} we have that $(u_i\cdot x_i)_{i\in\omega}$ satisfies~\eqref{A}. The sequence is covered by  $(x_i)_{i\in\omega}$, but $(x_i)_{i\in\omega}$ is not covered by $(u_i\cdot x_i)_{i\in\omega}$ by (b), so the former cannot be a~W-representative. So $A$ is not a~W-representative, either.
\end{proof}

As it can be seen in the proof above, there are two ways two justify the conclusion that $(u_i\cdot x_i)_{i\in\omega}$ meets the condition~\eqref{A}: either by an application of Proposition~\ref{prop:C6-ll-in-part} or by a~reference to Proposition~\ref{prop:for-key-theorem}. Both these conditions are equivalent to \eqref{C6}, so there is no way to escape the coherence axiom in the construction. This, of course, does not show that `$\prePtW^\omega\subseteq\prePt$' is independent from $\BCA+\eqref{IA}$, as it only means that \emph{the proof} itself requires \eqref{C6}. Nor does it undermine the construction from the proof, as the remedy is relatively simple and only calls for the explicit assumption of the axiom. However, it would be desirable to fully know the status of the coherence axiom with respect to the sentence `$\prePtW^\omega\subseteq\prePt$'. As we have not been  able to settle it, we put forward the following:
\begin{problem}
    Show that `$\prePtW^\omega\subseteq\prePt$' cannot be deduced from the axioms for Boolean contact algebras extended with the interpolation axiom. That is, find an algebra $\frB\in\BCA+\eqref{IA}+\eqref{eq:no-atoms}$ that has a~W-representative which is not a~G-representative. Similarly, show that `$\prePtW^\omega\subseteq\prePt$' is not true in some $\frB\in\BCA+\eqref{C6}+\eqref{eq:no-atoms}$. We add the assumption about the non-existence of atoms, since Biaciono and Gerla have it among their postulates.
\end{problem}

It was proven in \citep{Duntsch-et-al-ARAATTRCC} that those Boolean contact algebras that satisfy \eqref{C5} and \eqref{C6} must be atomless.  In light of this and Theorems~\ref{th:Q-and-A-is-W} and \ref{th:BG-improved} we have:
\begin{theorem}\label{th:conclusion}
    If $\frB\in\BCA+\eqref{IA}+\eqref{C5}+\eqref{C6}$, then $\prePtW^\omega\subseteq\prePt\subseteq\prePtW$. If additionally $\frB$ satisfies the countable chain condition, then $\prePtW=\prePt$.
\end{theorem}


Recall that a~topological space $X$ is \emph{semi-regular} iff it has a~basis that consist of regularly open subsets of~$X$. It is \emph{weakly regular} iff for every non-empty open set $M$ there exists a non-empty open set $K$ such that $\Cl K\subseteq M$. $X$ is \emph{$\kappa$-normal}\footnote{These spaces were introduced and studied by \citet{Shchepin-RFNNS}.} (or \emph{weakly normal}) iff any pair of disjoint regular closed sets can be separated by disjoint open sets.


By \citep[Proposition 3.7]{Duntsch-et-al-RTBCA} we have that for a space $X$ and a dense subalgebra $\frB$ of $\langle\RO(X),\mathord{\cont}\rangle$ the following correspondences hold:
\begin{enumerate}[label=(\roman*)]
    \item $\cont$ satisfies \eqref{C5} iff $X$ is weakly regular,
    \item $\cont$ satisfies \eqref{C6} iff $X$ is connected,
    \item $\cont$ satisfies \eqref{IA} iff $X$ is $\kappa$-normal.
\end{enumerate}
Let $\TBCA$ be the class of all \emph{topological} contact algebras, that is those of the form $\langle \frB,\mathord{\cont}\rangle$, where $\frB$ is a subalgebra of a~regular open algebra $\RO(X)$ of a~topological space~$X$.

\begin{lemma}[{\citealp[Lemma 3.56]{Bennett-Duntsch-AAT}}]
    $\frB\in\TBCA+\eqref{IA}+\eqref{C5}+\eqref{C6}$ iff $\frB$ is a~dense subalgebra of $\langle\RO(X),\mathord{\cont}\rangle$, where $X$ is a~$\kappa$-normal, connected $T_1$-space.
\end{lemma}

In consequence, by this and by Theorem~\ref{th:conclusion} we obtain:
\begin{corollary}
    If $X$ is a~$\kappa$-normal, connected $T_1$-space, and $\frB$ is a~dense subalgebra of $\langle\RO(X),\mathord{\cont}\rangle$, then in $\frB$\/\textup{:} $\prePtW^\omega\subseteq\prePt\subseteq\prePtW$. If additionally $X$ as a~topological space satisfies the countable chain condition, then $\prePtW=\prePt$.
\end{corollary}




Theorem~\ref{th:Q-and-A-is-W} shows that the second-order statement `$\prePt\subseteq\prePtW$' corresponds to a~first-order property of atomlessness of Boolean algebras. It is then natural to ask if the statements `$\prePtW^\omega\subseteq\prePt$' and `$\prePtW\subseteq\prePt$' correspond to any familiar, not necessarily first-order, properties of BCAs, or topological BCAs. Let us focus on this, and let us prove some negative results.

Observe that the coherence axiom cannot be deduced by means of `$\prePtW\subseteq\prePt$' (more so, by means of `$\prePtW^\omega\subseteq\prePt$'), in the following sense:
\begin{proposition}\label{prop:C6-cannot-be-deduced}
    \eqref{C6} is not true in $\BCA+\eqref{IA}+\eqref{C5}+\eqref{eq:no-atoms}+\prePtW\subseteq\prePt$.
\end{proposition}
\begin{proof} Take the contact algebra $\langle\RO(\Real),\mathord{\overl}\rangle$ and apply Lemma~\ref{lem:no-W-points-atomless-C=O}.
\end{proof}


If the reader finds the reference to Lemma~\ref{lem:no-W-points-atomless-C=O} somewhat sneaky (as there are no W-representatives in BCAs that satisfy the premises), we can construct an example of a~BCA that has W-representatives and meets the conditions of the proposition but not \eqref{C6} by taking the topological space $X\defeq[0,1]\cup[2,3]$ as the subspace of the reals with the standard topology and considering $\langle\RO(X),\cont\rangle$. $X$ is normal (more so, $\kappa$-normal), $T_1$, satisfies the countable chain condition. Thus $\prePtW\subseteq\prePt$. $\RO(X)$ is obviously atomless.  Moreover, $X$ is metrizable, so it's a~concentric space and thus has W-representatives by Theorem~\ref{th:concentric-W-representatives}. But $X$ has two non-trivial components, and thus $\cont$ does not satisfy \eqref{C6}. Does adding the assumption $\prePtW\neq\emptyset$ does not improve the situation.

Making a suitable modification to $X$, e.g., taking $Y\defeq[0,1]\cup\{2\}$ we see that:
\begin{proposition}
    \eqref{eq:no-atoms} is not a~consequence of $\BCA+\eqref{IA}+\eqref{C5}+\prePtW\subseteq\prePt$.
\end{proposition}
\noindent Of course, by Theorem~\ref{th:Q-and-A-is-W}, in $\RO(Y)$ we have $\prePt\nsubseteq\prePtW$, specifically with $\{\{2\}\}$ being the culprit.

As for the relation of \eqref{C5} to `$\prePtW\subseteq\prePt$', observe that since the axiom is not a~consequence of $\{\text{\eqref{C0}--\eqref{C4}},\eqref{IA},\eqref{C6},\eqref{eq:no-atoms}\}$, it cannot be a~consequence of $\{\text{\eqref{C0}--\eqref{C4}},\text{`$\prePtW^\omega\subseteq\prePt$'}\}$ by Theorem~\ref{th:BG-improved}. Thus there may be no deeper connection between the two statements.

The above are selective remarks that show what cannot be proven by or about the inclusion `$\prePtW\subseteq\prePt$'. So, we put forward other open problems:
\begin{problem}
    Find $\frB\in\BCA+\eqref{C6}+\prePtW\subseteq\prePt$ in which the interpolation axiom fails.
\end{problem}
\begin{problem}
    Characterize the class $\BCA+\prePtW\subseteq\prePt$, with possibly additional constraints.
\end{problem}
We also ask:
\begin{problem}
    Is there any topological property of a space $X$ that corresponds to the statement $\prePtW\subseteq\prePt$, as a~second order statement formulated about a~subalgebra of $\langle\RO(X),\mathord{\cont}\rangle$?
\end{problem}









\section{W-representatives are G-representatives (the general case)}\label{sec:W-points-are-G-points-general}

Having proven Theorem 5.3 (of whose counterpart is---by a fortunate coincidence---our Theorem~\ref{th:BG-improved}) \citeauthor{Biacino-Gerla-CSGWDP} write:
\begin{quotation}
  [\ldots] the just proven theorem holds for W-representatives that are expressible by [countable] sequences. We do not know if this result holds in any case.
\end{quotation}
In this section, we prove that the we can extend Theorem~\ref{th:BG-improved} to W-representatives of an arbitrary coinitiality provided we assume the \emph{generalized} version of the interpolation axiom.





Observe that by \eqref{eq:ll-sum-prod} the interpolation axiom is equivalent to the following:
\[
x\ll x_1\wedge\ldots\wedge x\ll x_n\rarrow (\exists z\in\frB)\,(x
\ll z\wedge z\ll x_1\cdot\ldots\cdot x_n)\,.
\]
Thus its natural extension to infinite cases is the following second-order constraint:
\begin{equation}\tag{GIA}\label{GIA}
    (\forall Y\in\power(\frB))\,((\forall y\in Y)\,x\ll y\rarrow(\exists z\in\frB)\,(x\ll z\wedge(\forall y\in Y)\,z\ll y))\,.
\end{equation}

Observe that \eqref{GIA} axiom puts a serious constraint upon the existence of W-representatives in the class of complete contact algebras:
\begin{theorem}\label{th:GIA+C5-no-W}
If $\frB\in\BCA+\eqref{C5}+\eqref{GIA}$ is complete, then there are no Whitehead representatives in $\frB$.
\end{theorem}



\begin{proof}%[Proof of Theorem~\ref{th:GIA+C5-no-W}]
It is easy to show that in complete Boolean contact algebras, \eqref{GIA} is equivalent to the following second-order constraint:
\[
(\forall y\in J)\,x\ll y\rarrow x\ll \bigwedge J
\]
which is equivalent to the following generalized version of \eqref{C4}:
\[
x\con\bigvee J\rarrow(\exists y\in J)\,x\con y\,.
\]
As it has been shown in \citep{Gruszczynski-Menchon-FCRTMOAB}, in presence of \eqref{C5} the latter axiom entails that $\mathrm{\con}=\mathrm{\overl}$. Thus by Lemma~\ref{lem:no-W-points-atomless-C=O}, $\frB$ does not have any W-representatives.
\end{proof}
It follows that if we want to have BCAs that satisfy \eqref{GIA} and have Whitehead representatives, we must drop either \eqref{C5} or completeness, or both. Fortunately, we can demonstrate that such algebras exist.
\begin{theorem}\label{th:GIA-consistency}
Any $\langle\frB,\mathord{\cond}\rangle$ satisfies    \eqref{GIA}.
\end{theorem}
\begin{proof}
Take as the distinguished element of a $\frB\in\BA$ any $\boldd\neq\zero$ and consider $\cond$. Suppose that $A\subseteq\frB$, and let $x$ be such that for all $a\in A$, $x\lld a$, which means that either $x\leq a-\boldd$ or $x+\boldd\leq a$. In the former case, $x\leq-\boldd$, so $x\lld x$, and we are done. In the latter, $x+\boldd\lld a$, as $-a$ must be disjoint from $x+\boldd$. Since it is always true that $x\lld x+\boldd$, we have that $x+\boldd$ is the region that is strongly between $x$ and $a$.
\end{proof}
In light of the above theorem, any Boolean algebra can be turned into a Boolean contact algebra that meets the generalized interpolation axiom. In particular, there will be such algebras that are either complete or incomplete, with or without any atoms, and of arbitrary cardinality. However, in light of Corollary~\ref{prop:C5-fails-for-d}, none of these algebras will satisfy \eqref{C5}.


Having shown the consistency of \eqref{GIA} with the standard axioms for contact, we go on to prove the following:
\begin{theorem}\label{th:BG-general}
  If $\frB\in\BCA+\eqref{GIA}+\eqref{C6}$, then $\prePtW\subseteq\prePt$.
\end{theorem}
\begin{proof}
Suppose $|B|=\kappa$. Fix an abstractive set $A$. Since it is linearly ordered by $\leq$, it must have a~coinitial sequence $\langle x_\alpha\mid\alpha<\lambda\rangle$ for a~limit ordinal $\lambda\leqslant\kappa$. Again, suppose the sequence is not a~G-representative, i.e. it fails to satisfy \eqref{r3}. Let $u$ and $v$ be regions such that each of them overlaps every $x_\alpha$ from the sequence, yet they are separated, i.e., $u\ll-v$. We construct another $\lambda$-sequence repeating the technique from the proof of Theorem~\ref{th:BG-improved}, but applying \eqref{GIA}.

If $\lambda=\omega$, then it is enough to observe that \eqref{IA} is just a special case of \eqref{GIA} where $Y\defeq\{-v\}$. So assume that $\lambda>\omega$.
Suppose $\alpha<\lambda$ is a~limit ordinal and for every $\delta<\beta<\alpha$ we defined $u_\beta$ and $u_\delta$ such that:
\[
u\ll u_\beta\ll u_\delta\ll-v\,.
\]
Consequently, we have that $u$ is a non-tangential part of every element of the sequence $\langle u_\beta\mid\beta<\alpha\rangle$. Thus, by \eqref{GIA} we may choose $u_\alpha$ to be a~region $z$ such that $(\forall \beta<\alpha)\,z\ll u_\beta$ and $u\ll z$. Following this procedure we can construct the $\lambda$-sequence $\langle u_\alpha\mid\alpha < \lambda\rangle$. We go on to show that $\langle x_{_\alpha} \cdot u_{\alpha} \mid \alpha < \lambda \rangle $  is an abstractive set.  \eqref{r0} holds given that we have $u_\alpha  \cdot x_\alpha \neq \zero$ for any  $\alpha < \lambda$.  \eqref{r1} holds due to \eqref{eq:ll-sum-prod}: $u_{\delta}\cdot x_{\delta}\ll u_{\beta}\cdot x_{\beta}$ for any $\beta,  \delta \in \lambda$ such that $\beta < \delta$. Additionally,we have that $u_{\alpha+1}\cdot x_{\alpha +1 }\ll u_{\alpha}\cdot x_{\alpha}$ and $u_{\alpha}\cdot x_{\alpha} \neq\one$  for any  $\alpha < \lambda$. Therefore, by  Proposition~\ref{prop:C6-ll-in-part} the sequence $\langle x_{_\alpha} \cdot u_{\alpha} \mid \alpha < \lambda \rangle $  satisfies~\eqref{A}.

Notice that the sequence $\langle x_{_\alpha} \cdot u_{\alpha} \mid \alpha < \lambda \rangle $  is covered by  $\langle x_\alpha\mid\alpha<\lambda\rangle$. However, we also have that $x_\beta\nleq u_\delta \cdot x_\delta $ for any $\beta,  \delta \in \lambda$ since $u_\delta \cdot x_\delta \ll -v$ and  $x_\beta \nleq-v$. Therefore, $\langle x_\alpha\mid\alpha<\lambda\rangle$
 is not covered by $\langle x_{_\alpha} \cdot u_{\alpha} \mid \alpha < \lambda \rangle $. Thus, $\langle x_\alpha\mid\alpha<\lambda\rangle$ is not a~W-representative and we can conclude that also  $A$ is not a~W-representative.
\end{proof}

\begin{problem}
    In Theorem~\ref{th:GIA-consistency} we have shown that the generalized interpolation axiom is consistent with the standard axioms for BCAs. However, what we do not know is whether there are contact algebras in which both the axioms \eqref{GIA}, \eqref{C6} hold, and there is at least one W-representative. Thus we ask: \emph{are there such BCAs?}
\end{problem}

%\section{Concluding remarks}\label{sec:conclusion}





\section*{Acknowledgements}

This research was funded by the National Science Center (Poland), grant number 2020/39/B/HS1/00216, ``Logico-philosophical foundations of geometry and topology''.

For the purpose of Open Access, the authors have applied a CC-BY public copyright license to any Author Accepted Manuscript (AAM) version arising from this submission.




\appendix

% \section{Contact relations via ultrafilters}

% In Proposition 4.4 of \citep{Duntsch-Winter-CBCA} the authors introduce a~method of defining extensions of contact relation, which themselves are contact relations. That is, given a~Boolean contact algebra $\frB$ and (proper) ultrafilters $\fil$ and $\ult$, the relation:
% \[
% {\con}\cup(\fil\times\ult)\cup(\ult\times\fil)
% \]
% is a~contact relation, i.e. satisfies \eqref{C0}--\eqref{C4}. In particular, since $\overl$ is a~contact relation, then so is:
% \[
% R\defeq\mathord{\overl}\cup(\fil\times\ult)\cup(\ult\times\fil)\,.
% \]
% Call $\frB\in\BCA$ \emph{dense} iff it is dense as a~boolean algebra, i.e. for any $x<y$, there is~$z$ such that $x<z<y$.

% \begin{lemma}
% If $\frB\in\BA$ is dense, then $\rel$ is a~contact relation that satisfies the interpolation axiom.
% \end{lemma}
% \begin{proof}
% Suppose $x\notrel -y$, which according to the definition means that:
% \[
% \text{(a)}\ x\ext-y\qquad\text{(b)}\ \langle x,-y\rangle\notin\fil\times\ult\qquad\text{(c)}\ \langle x,-y\rangle\notin\ult\times\fil\,.
% \]
% If either $x\notrel-x$ or $y\notrel-y$, we are done. So assume that none of the two holds, i.e.
% \[
% \text{(i)}\ \langle x,-x\rangle\in (\fil\times\ult)\cup(\ult\times\fil)\qquad \text{(ii)}\ \langle y,-y\rangle\in (\fil\times\ult)\cup(\ult\times\fil)\,.
% \]
% So $x<y$, and by density, there is $z$ such that $x<z<y$. Assume towards contradiction that $x\rel-z$, which results in two cases two consider:
% \[
% \text{either}\quad \langle x,-z\rangle\in\fil\times\ult\quad\text{or}\quad\langle x,-z\rangle\in\ult\times\fil\,.
% \]
% In the first possibility, $x\in\fil$, so by (b) we get that $y\in\ult$. By (a), also $y\in\fil$, so:
% \[
% \langle y,y\rangle\in (\fil\times\ult)\cap(\ult\times\fil)\,,
% \]
% which contradicts~(ii). In the second possibility, we have that $x\in\ult$ and so $y\in\fil$, by~(c). But by (a) $y$ is also in $\ult$ and we get a~contradiction again.

% Now, suppose that $z\rel-y$, which gives us two possibilities anew:
% \[
% \text{either}\quad \langle z,-y\rangle\in\fil\times\ult\quad\text{or}\quad\langle z,-y\rangle\in\ult\times\fil\,.
% \]
% In the first one, $-y\in\ult$, so by (b) $-x\in\fil$. By (a) $-x\in\ult$, so:
% \[
% \langle-x,-x\rangle\in(\fil\times\ult)\cap(\ult\times\fil)\,,
% \]
% which contradicts~(i). In the second one, $-y\in\fil$, so by (c) $-x\in\ult$. But by (a) $-x\in\fil$, and we end up in the situation analogous to the previous case.
% \end{proof}

%\bibliography{mat-eng}
\bibliographystyle{apalike}


% Created 3/2/2011
% Last modified: 3/7/09
%
\documentclass[journal, twocolumn]{IEEEtran}
%\pdfoutput=1
\IEEEoverridecommandlockouts



%\pdfoutput=1
\IEEEoverridecommandlockouts
\usepackage[dvips]{graphicx}
\usepackage[cmex10]{amsmath}
\usepackage{amsfonts,amssymb,bm}
%\usepackage[caption=false]{caption}
\usepackage[tight,footnotesize]{subfigure}
%\usepackage[caption=false,font=footnotesize]{subfig}
\usepackage{fixltx2e}
\usepackage{dblfloatfix}
%\usepackage{stfloats}
%\usepackage{enumerate}
\usepackage{rotating}
\usepackage{multirow}
\usepackage{url}
\usepackage{cite}
%\usepackage{enumitem}
%\usepackage{algorithmic,algorithm}
\usepackage[linesnumbered,lined, ruled]{algorithm2e}

%\usepackage{slashbox}
\usepackage{cite}
\usepackage{setspace}
\usepackage{footnote}
\usepackage[T1]{fontenc}
\usepackage{ae,aecompl}
\usepackage{epsfig}
\usepackage{multicol}
\usepackage{multirow}
%-------------
\usepackage{subfigure}
\usepackage{times}
%======================colored fonts
\usepackage{color}
\usepackage{comment}
\usepackage{amsthm}
\usepackage{caption}

\usepackage{colortbl}
\usepackage[official]{eurosym}

\newcommand\pro[1]{{\texttt{#1}}}
\newcommand\proalg[1]{{{#1}}}
\newcommand\mycommfont[1]{\footnotesize
\textcolor{blue}{#1}}
\SetCommentSty{mycommfont}
\newlength{\commentWidth}
\setlength{\commentWidth}{3.75cm}
\newcommand{\atcp}[1]{\tcp*[r]{\makebox[\commentWidth]{#1\hfill}}}

%\newcommand{\myitem}{\noindent$\bullet$}

\newcommand{\closebracket}{)}
\SetKwRepeat{Repeat}{repeat}{until}
\SetKwRepeat{Forever}{repeat}{forever}
\SetKwRepeat{On}{on}{end}
\SetNlSty{bfseries}{\color{black}}{}


%\usepackage{subfig}
%------------
%\usepackage{flushend}
%
%\newcommand\T{\rule{0pt}{2.6ex}}
%\newcommand\B{\rule[-1.2ex]{0pt}{0pt}}
%\DeclareMathOperator*{\argmax}{arg\,max}
%\DeclareMathOperator*{\arglexmax}{arg\,lex\,max}
%\long\def\symbolfootnote[#1]#2{\begingroup%2
%\def\thefootnote{\fnsymbol{footnote}}\footnote[#1]{#2}\endgroup}
%
\pagestyle{empty}
%\algsetup{linenodelimiter=.}
%\newtheorem{theorem}{Theorem}[section]
%\newtheorem{claim}{Claim}[section]
%\newtheorem{corollary}{Corollary}[section]
%\newtheorem{proposition}{Proposition}
%\newtheorem{proof}{Proof}
\newtheorem{remark}{Remark}
%\newtheorem{lemma}{Lemma}[section]
\newtheorem{definition}{Definition}

\newcommand{\ie}[0]{\textit{i.e.},~}
\newcommand{\eg}[0]{\textit{e.g.},~}
\newcommand{\etal}[0]{\textit{et al.}}
\newcommand{\etc}[0]{\textit{etc.}}
\newcommand{\vs}[0]{\textit{vs.}~}

\begin{document}

\vspace{-10pt}
\title{Fair Energy Allocation in Risk-aware Energy Communities}

%\vspace{-5pt}
%\author{\IEEEauthorblockN{authors}\\ \vspace{-12pt}
%\IEEEauthorblockA{Department of Informatics and Telecommunications,\\
%National and Kapodistrian University of Athens\\
%%Ilissia, 157 84 Athens, Greece\\
%Email: \{authors\}@di.uoa.gr}
%%\thanks{This work has been supported by EINS, the Network of Excellence in Internet Science (FP7-ICT-288021) and the Univ. of Athens (ELKE-10812).}
%}
%
%`
\author{
\IEEEauthorblockN{Eleni Stai,~\IEEEmembership{Member,~IEEE},  Lesia Mitridati, \IEEEmembership{Member,~IEEE}, Ioannis Stavrakakis,~\IEEEmembership{Fellow,~IEEE}, Evangelia Kokolaki, Petros Tatoulis, Gabriela Hug,~\IEEEmembership{Senior Member,~IEEE}} 


\thanks{E. Stai, P. Tatoulis and G. Hug are with the EEH - Power Systems Laboratory, ETH Z\"urich, Physikstrasse 3, 8092 Z\"urich, Switzerland.  L. Mitridati is with the Department of Wind \& Energy Systems, DTU. I. Stavrakakis is with the Department of Informatics and Telecommunications, National \& Kapodistrian University of Athens, 15784 Athens, Greece. E. Kokolaki is with the Hellenic Ministry of Environment and Energy, Mesogeion 119, 11526 Athens, Greece. E. Kokolaki’s work was carried out while she was with the National \& Kapodistrian University of Athens. Emails: elstai@ethz.ch, lemitri@dtu.dk, ioannis@di.uoa.gr, e.kokolaki@prv.ypeka.gr, petrost@student.ethz.ch, ghug@ethz.ch.}
%\thanks{Acknowledgement:}
}


\maketitle

\begin{abstract} 
This work introduces a decentralized mechanism for the fair and efficient allocation of limited renewable energy sources among consumers in an energy community. In the proposed non-cooperative game, the self-interested community members independently decide whether to compete or not for access to RESs during peak hours and shift their loads analogously. In the peak hours, a proportional allocation (PA) policy is used to allocate the limited RESs among the competitors. The existence of a Nash equilibrium (NE) or dominant strategies in this non-cooperative game is shown, and closed-form expressions of the renewable energy demand and social cost are derived. Moreover, a decentralized algorithm for choosing consumers' strategies that lie on NE states is designed. The work shows that the risk attitude of the consumers can have a significant impact on the deviation of the induced social cost from the optimal. Besides, the proposed decentralized mechanism with the PA policy is shown to attain a much lower social cost than one using the naive equal sharing policy.
\end{abstract}

\begin{IEEEkeywords}
energy communities; renewable energy sources; game theory; risk; demand side management;
\end{IEEEkeywords}
\section{Introduction}
\label{sec:introduction}
% \begin{itemize}
%     % Diffusion of FL
%     \item {\st{Diffusion of FL}}
%     % Security threats to FL
%     \item {\st{Security threats to FL with particular focus on model poisoning}}
%     % Limitations of existing countermeasures
%     \item {\st{Current countermeasures (e.g., KRUM) and their limitations}}
%     % Proposed method and its advantages
%     \item {\st{Intuitive description of the proposed method and its difference (i.e., advantages) w.r.t. state of the art}}
%     % Main contributions
%     \item {\st{Summary of the main contributions of this work}}
%     % Paper's structure and organization
%     \item {\st{Paper's structure and organization}}
% \end{itemize}

% Diffusion of FL
Recently, {\em federated learning} (FL) has emerged as the leading paradigm for training distributed, large-scale, and privacy-preserving machine learning (ML) systems~\cite{mcmahan2017googleai,mcmahan2017aistats}. 
The core idea of FL is to allow multiple edge clients to collaboratively train a shared, global model without disclosing their local private training data.
%Specifically, an FL system consists of a central server and many edge clients; 
A typical FL round involves the following steps: {\em(i)} the server randomly picks some clients and sends them the current, global model; {\em(ii)} each selected client locally trains its model with its own private data; then, it sends the resulting local model to the server;\footnote{Whenever we refer to global/local model, we mean global/local model {\em parameters}.} {\em(iii)} the server updates the global model by computing an \emph{aggregation function}, usually the average (FedAvg), on the local models received from clients.
% \begin{enumerate}
%     \item[{\em(i)}] the server sends the current, global model to the clients and appoints some of them for training;
%     \item[{\em(ii)}] each selected client locally trains its copy of the global model with its own private data; then, it sends the resulting local model back to the server;\footnote{Whenever we refer to global/local model, we mean global/local model {\em parameters}.}
%     \item[{\em(iii)}] the server updates the global model by computing an \emph{aggregation function} on the local models received from clients (by default, the average, also referred to as FedAvg~\cite{mcmahan2017aistats}).
% \end{enumerate}
This process goes on until the global model converges. %(e.g., after a certain number of rounds or other similar stopping criteria).
%\\
% The advantages of FL over the traditional, centralized learning paradigm are undoubtedly clear in terms of flexibility/scalability (clients can join/disconnect from the FL network dynamically), network communications (only model weights\footnote{We will use \textit{parameters} and \textit{weights} interchangeably.} are exchanged between clients and server), and privacy (each client's private training data is kept local at the client's end and not uploaded to the server).
\\
% Security threats to FL
%However, the growing adoption of FL also raises security concerns~\cite{costa2022covert}, particularly about its confidentiality, integrity, and availability.
Although its advantages over standard ML, FL also raises security concerns~\cite{costa2022covert}. %, particularly about its confidentiality, integrity, and availability~\cite{costa2022covert}.
% OLD, LONG VERSION
% Indeed, some work deals with privacy leakage that may expose the local data of some clients~\cite{melis2019sp}. 
% A large body of work, instead, investigates attacks that usually aim to detriment the predictive accuracy of the learned global model. For instance, \emph{data poisoning} attacks achieve this goal by letting an adversary pollute the training set of some corrupt FL clients with maliciously crafted examples~\cite{jagielski2018sp}.
% Similarly, in \emph{model poisoning} the attacker attempts to tweak the global model weights~\cite{bhagoji2019pmlr} by directly perturbing the local model's weights of some infected FL clients before these are sent to the central server for aggregation, usually via so-called Byzantine attacks. 
% It turns out that Byzantine model poisoning attacks severely impact standard FedAvg; therefore, more robust aggregation functions must be designed to make FL systems secure.
Here, we focus on \emph{untargeted model poisoning} attacks~\cite{bhagoji2019pmlr}, where an adversary attempts to tweak the global model weights %\footnote{We will use the terms \textit{parameters} and \textit{weights} interchangeably.} 
by directly perturbing the local model's parameters of some infected clients before these are sent to the central server for aggregation.
In doing so, the adversary aims to jeopardize the global model \textit{indiscriminately} at inference time.
Such model poisoning attacks severely impact standard FedAvg; therefore, more robust aggregation functions must be designed to secure FL systems.
\\
% In this paper, we focus on designing a novel robust aggregation scheme at the server's end to contrast the effect of Byzantine model poisoning attacks.
%
% Current countermeasures and their limitations
%Several countermeasures have been proposed in the literature to combat model poisoning attacks on FL systems.
% Some methods use simple statistics more robust than plain average to smooth the impact of malicious updates (e.g., Trimmed Mean and FedMedian~\cite{yin2018icml}). 
% Other defenses implement outlier detection techniques to discard malicious updates from the aggregation performed at the server's end. Those are either based on heuristics (e.g., Krum/Multi-Krum~\cite{blanchard2017nips} and Bulyan~\cite{mhamdi2018pmlr}) or data-driven approaches (e.g., K-means clustering~\cite{shen2016acm} or DnC via spectral analysis~\cite{shejwalkar2021ndss}). 
% Finally, some strategies rely on a centralized ``source of trust'' to spot potential malicious updates (e.g., FLTrust~\cite{cao2020fltrust}).
% Several countermeasures have been proposed in the literature to combat model poisoning attacks on FL systems, i.e., to discard possible malicious local updates from the aggregation performed at the server's end. 
% These techniques range from simple statistics more robust than plain average (e.g., Trimmed Mean and FedMedian~\cite{yin2018icml}) to outlier detection heuristics (e.g., Krum/Multi-Krum~\cite{blanchard2017nips} and Bulyan~\cite{mhamdi2018pmlr}) or data-driven approaches (e.g., spectral analysis via K-means clustering~\cite{shen2016acm} or spectral analysis), or methods based on ``source of trust'' (e.g., FLTrust~\cite{cao2020fltrust}).
% OLD, LONG VERSION
%Several countermeasures have been proposed in the literature to combat Byzantine model poisoning attacks on FL systems.
% Descriptive statistics
% For example, Trimmed Mean and FedMedian aggregate local model updates using more robust statistics than standard average~\cite{yin2018icml}.
%
% % Heuristics for outlier detection
% Many existing Byzantine-resilient strategies implement some outlier detection heuristics to discard the model updates sent by potentially malicious clients from the input of the aggregation function.
% One of the most popular heuristics is Krum~\cite{blanchard2017nips}.
% This strategy tries to mitigate the impact of Byzantine attacks by selecting as a global model the local model with the smallest sum of Euclidean distances to {\em all} the other local models.
% Although powerful, Krum requires the server to know (or, at least, estimate) the number of malicious FL clients upfront, which is generally impossible in a realistic attack scenario. %
% Moreover, Krum may become ineffective for complex, high-dimensional model parameter spaces due to the curse of dimensionality.
% Bulyan~\cite{mhamdi2018pmlr} tries to overcome this issue by combining Krum with a variant of Trimmed Mean.
% % Data-driven outlier detection
% Other strategies use data-driven outlier detection techniques -- e.g., via K-means clustering~\cite{shen2016acm} -- to spot potential malicious local model updates. 
% %For instance, Shen et al. propose to cluster local model updates with K-means and thus identify outliers.
%
% % Other techniques
% As far as the server is concerned, any local model received can be from a potential malicious client. 
% FLTrust~\cite{cao2020fltrust} assumes the server acts as a client, i.e., trains a local model on an additional {\em trustworthy} dataset at the server's end and compares it against all the local models from other clients. 
% This way, the server can rely on some ``source of trust'' when discarding potentially malicious clients.
%\\
% Limitations of existing Byzantine-resilient strategies
Unfortunately, existing defense mechanisms either rely on simple heuristics (e.g., Trimmed Mean and FedMedian by~\cite{yin2018icml}) or need strong and unrealistic assumptions to work effectively (e.g., foreknowledge or estimation of the number of malicious clients in the FL system, as for Krum/Multi-Krum~\cite{blanchard2017nips} and Bulyan~\cite{mhamdi2018pmlr}, which, however, cannot exceed a fixed threshold).
Furthermore, outlier detection methods using K-means clustering~\cite{shen2016acm} or spectral analysis like DnC~\cite{shejwalkar2021ndss} do not directly consider the temporal evolution of local model updates received.
Finally, strategies like FLTrust~\cite{cao2020fltrust} require the server to collect its own dataset and act as a proper client, thereby altering the standard FL protocol.
\\
% OLD, LONG VERSION
% Overall, existing Byzantine-resilient strategies are either simple heuristics (e.g., FedMedian) or, if they are more complex, they rely on strong and unrealistic assumptions to work effectively (e.g., knowing the number of malicious clients in the FL system in advance, as for Krum and alike).
% Furthermore, data-driven outlier detection methods do not consider the temporary evolution of local model updates received (e.g., K-means clustering). 
% Finally, strategies like FLTrust requires the server to collect its own dataset and act as a proper client, thereby altering the standard FL protocol.
%
% Description of the proposed method
This work introduces a novel pre-aggregation \textit{filter} robust to untargeted model poisoning attacks. Notably, this filter $(i)$ operates without requiring prior knowledge or constraints on the number of malicious clients and $(ii)$ inherently integrates temporal dependencies. 
The FL server can employ this filter as a preprocessing step before applying \textit{any} aggregation function, be it standard like FedAvg or robust like Krum or Bulyan.
Specifically, we formulate the problem of identifying corrupted updates as a multidimensional (i.e., matrix-valued) time series anomaly detection task. 
The key idea is that legitimate local updates, resulting from well-calibrated iterative procedures like stochastic gradient descent (SGD) with an appropriate learning rate, show \textit{higher predictability} compared to malicious updates. This hypothesis stems from the fact that the sequence of gradients (thus, model parameters) observed during legitimate training exhibit regular patterns, as validated in Section~\ref{subsec:intuition}. %until convergence. 
%This regularity may be more pronounced for smooth convex loss functions, but it can still be captured within an appropriate time window, even for more complex and convoluted loss surfaces. 
%We provide evidence of this claim in Appendix~B, where we show that the average mutual information (i.e., ``predictability''), calculated over pairs of legitimate model updates sent at different FL rounds, is significantly higher than the corresponding computation for a malicious client.
\\
Inspired by the matrix autoregressive (MAR) framework for multidimensional time series forecasting~\cite{chen2021je}, we propose the FLANDERS ({\em \textbf{F}ederated \textbf{L}earning meets \textbf{AN}omaly \textbf{DE}tection for a \textbf{R}obust and \textbf{S}ecure}) filter.
The main advantages of FLANDERS over existing strategies like FLDetector~\cite{zhao2020multivariate} are its resilience to large-scale attacks, where $50\%$ or more FL participants are hostile, and the capability of working under realistic non-iid scenarios.
We attribute such a capability to two key factors: $(i)$ FLANDERS works without knowing a priori the ratio of corrupted clients, and $(ii)$ it embodies temporal dependencies between intra- and inter-client updates, quickly recognizing local model drifts caused by evil players. Below, we summarize our main contributions:

\begin{itemize}
\item[{\em(i)}]
We provide empirical evidence that the sequence of models sent by legitimate clients is more predictable than those of malicious participants performing untargeted model poisoning attacks.
\\
\item[{\em(ii)}] 
We introduce FLANDERS, the first pre-aggregation filter for FL robust to untargeted model poisoning based on multidimensional time series anomaly detection.
\\
\item[{\em(iii)}] 
We integrate FLANDERS into Flower,\footnote{\scriptsize{\url{https://flower.dev/}}} a popular FL simulation framework for reproducibility.
\\
\item[{\em(iv)}] 
We show that FLANDERS improves the robustness of the existing aggregation methods under multiple settings: different datasets, client's data distribution (non-iid), models, and attack scenarios.
\\
\item[{\em(v)}] 
We publicly release all the implementation code of FLANDERS along with our experiments.\footnote{\scriptsize{\url{https://anonymous.4open.science/r/flanders_exp-7EEB}}}
\end{itemize}

% Paper's structure and organization
The remainder of the paper is structured as follows. %some related work and the current state-of-the-art solutions to security issues that FL entails. 
Section~\ref{sec:background} covers background and preliminaries. 
In Section~\ref{sec:related}, we discuss related work.
Section~\ref{sec:problem} and Section~\ref{sec:method} describe the problem formulation and the method proposed. % to tackle it. 
Section~\ref{sec:experiments} gathers experimental results. %, and Section~\ref{sec:limitations} discusses some limitations of this work.
Finally, we conclude in Section~\ref{sec:conclusion}.
 %discusses the limitations of this work and draws future research directions.
%reports conclusions and draws perspectives for future research directions.

%%%%%%% OLD %%%%%%%
%to overcome the resilience of Byzantine failures in distributed Stochastic Gradient Descent computations. 
% The strength of Krum is its time complexity, which is linear in the gradient dimension. 
% However, the robustness of the approach is guaranteed for gradient-based learning applications only when the majority of the clients are not compromised. 
% Besides, the aggregation mechanism of Krum, as well as that of similar methods, is robust from a coarse-grained perspective and does not provide solutions to errors and perturbations that may occur at inference time.
%A related approach to~\cite{blanchard2017nips} is the work of Su et al.~\cite{su2016dc}. Here, the authors propose an iterated approximate agreement to tackle a multi-layer scenario attacked by Byzantine agents. 
%However, the method works efficiently on the sole discrete context and it is inapplicable to continuous state environments.
%\gabri{Maybe, we should just talk about the main limitations of existing countermeasures without digging into their details (or, we can just mention Krum as this is the most popular one). I will move the description of all these methods to the Related Work section.}
\vspace{-0.1in}
\section{Decentralized Energy Sharing Mechanism} \label{sec:esc}


\subsection{Energy Sharing Community} \label{sec:community}

The energy community consists of $N$ consumers, indexed by $i \in \mathcal{N}= \{1,...,N\}$, who have access to multiple energy sources in order to cover their flexible loads.

\subsubsection{Energy Sources}\label{sec:energy_sources}

We consider that the energy community has access to two distinct types of energy sources, namely \textit{local production} from community-owned RESs, and \textit{imports} from the distribution grid. We consider that the local RESs production is available only during daytime (e.g., PV panels), with a limited capacity $\mathcal{RE}>0$, whereas the community's imports from the grid are unlimited. Therefore, during nighttime the community's aggregate load is fully covered by imports from the grid, and, during daytime if the community's aggregate load exceeds the available RES capacity, the remainder is covered by imports from the grid.

%%
Production from the community-owned RESs is priced by the community manager at a constant low tariff $c^{RES}$ (in units per energy), whereas imports from the grid are priced by an energy retailer using TOU tariffs, typically for daytime and nighttime consumption. We define the daytime and nighttime tariffs with respect to the RESs tariff, as $c^{grid,d}= \gamma c^{RES}$ and $c^{grid,n} = \beta c^{RES}$, respectively, with $\gamma>\beta>1$.
%%
These TOU tariffs reflect the sum of energy prices and grid tariffs and are designed to incentivize consumers to shift their flexible loads from daytime to nighttime to reduce energy production costs and congestion during peak hours. In addition, the low cost of the local RESs production promotes self-consumption within the community and reduction of grid imports. We assume that the energy source-related parameters $\Omega = \{\mathcal{RE}, c^{RES}, \beta, \gamma\}$ are perfectly known by all consumers in the community at the beginning of the day.

\subsubsection{Consumers Preferences}

The consumers have a broad range of flexible loads, namely, (i) shiftable appliances (e.g. washing machines) that do not need to be scheduled every day, (ii) batteries or electric vehicles (EVs) with flexible state-of-charge requirements at the end of the day, and (iii) thermostatically controlled loads (e.g., water heater, heat pumps) with flexible set-points. The level of consumption and the time-schedule of these loads are flexible. For instance, an EV owner has a daily inflexible load required to cover her daytime transportation needs, and a daily flexible load, representing the additional energy to achieve a desired state-of-charge by the end of the day.
%At the beginning of the day, the daily flexible loads of all consumers in the community can be scheduled during daytime or nighttime. The level and time interval at which they schedule their daily flexible loads depend on their consumption preferences, as well as the available energy sources. 
However, once scheduled, these loads cannot be interrupted or shifted to another time interval.
%The available energy sources available at a given TOU interval will then be shared among the consumers whose daily flexible loads are scheduled during the same TOU interval. 
As a result, consumers whose daily flexible loads are scheduled during daytime incur the risk of paying for high-priced imports from the grid if the community's aggregate daytime energy demand exceeds the available local RES production. When scheduling their daily flexible loads across different time intervals, consumers wish to achieve a trade-off between their desired daily energy consumption, and the financial risks incurred. And, risk-averse consumers may choose to reduce their daily energy consumption if they are scheduled during daytime, to mitigate the financial risks incurred. For instance, if scheduled during nighttime, a risk-averse EV owner may prefer to consume enough energy to fully charge her EV by the end of the day, whereas, if scheduled during daytime, she may prefer to consume a smaller amount of energy in order to charge her EV at e.g., $75\%$ by the end of the day.

The risk attitude and daily energy consumption preferences of each consumer $i \in \mathcal{N}$ in the community can be represented by her type $\vartheta_i \in \Theta = \{1,...M\}$. The type accounts for consumer's (i) daily flexible load $U_{\vartheta_i}>0$ (in energy unit); and (ii) risk-aversion degree $\mu_{\vartheta_i} \in [0,1]$, representing the share of her daily flexible load that she is willing to consume if scheduled during daytime. 

%%
With this parametric representation of the consumers' flexibility preferences, if the daily flexible load of a consumer $i$ of type $\vartheta_i$ is scheduled during daytime, her daytime energy demand is $E_{\vartheta_i} = \mu_{\vartheta_i} U_{\vartheta_i}$ (and the remainder of her daily flexible load $(1-\mu_{\vartheta_i})U_{\vartheta_i}$ is deferred to the following day), whereas, if her daily flexible load is scheduled during nighttime, her nighttime demand is $U_{\vartheta_i}$. Therefore, $\mu_{\vartheta_i}=1$ represents a risk-seeking consumer, and $\mu_{\vartheta_i} < 1$ a risk-conservative consumer. 

At the beginning of each day, each consumer knows her own flexibility preferences and type, but this information is considered private. We assume that the community manager and consumers in the community only know the probability distribution $\bm{r}=[r_1,...,r_{M}]^T$ over the consumers types $\Theta$, where $0 \leq r_{\vartheta} \leq 1$ is the probability that a consumer in the community is of type $\vartheta \in \Theta$. Furthermore, the consumers' preferences, and therefore their types, can vary from day to day. Since this paper studies a single scheduling day, the daily time indexes are omitted. 

Following the law of large numbers, the number of consumers of type $\vartheta \in \Theta$ can be approximated as $r_\vartheta \cdot N$. Thus, based on the above, the maximum daytime energy demand of the community, i.e., if the daily flexible loads of all consumers are scheduled during daytime is 
\vspace{-0.08in}
\begin{align}
 D^{Total} = N  \sum_{\vartheta \in \Theta} r_{\vartheta}~E_{\vartheta}.  
\end{align}
\vspace{-0.02in}
For notational simplicity, in the remainder of the paper, we introduce $\varepsilon_{\vartheta_i}=\frac{1}{\mu_{\vartheta_i}}$, such that $U_{\vartheta_i}=\varepsilon_{\vartheta_i} \cdot E_{\vartheta_i}$. Thus, $\varepsilon_{\vartheta_i}=1$ represents a risk-seeking consumer $i$, and $\varepsilon_{\vartheta_i}>1$ a risk-conservative consumer. Finally, we assume without loss of generality that $E_1\leq E_2 \leq ...\leq E_{M}$.

%Therefore, the expected maximum daytime schedule of the community, i.e. if all consumers' flexible loads are scheduled during daytime can be expressed as $D^{d,Total} = N \sum_{\vartheta \in \Theta} r_{\vartheta} E_{\vartheta}$. 
\vspace{-0.15in}
\subsection{Decentralized Energy Sharing Mechanism (D-ESM)}

The problem faced by the energy sharing community is to schedule the daily flexible loads of all consumers across the different TOU intervals and to allocate the different energy sources among them within each TOU interval. The role of the community manager is to design a mechanism that optimally coordinates the interactions among consumers in the community towards desirable outcomes, namely: (i) minimizing the social cost for the community as a whole, and (ii) sharing the community-owned assets among the consumers fairly. We introduce below the proposed decentralized energy sharing mechanism (D-ESM) for this energy sharing community.

\subsubsection{Load Scheduling}

In the proposed D-ESM, each consumer independently schedules her own daily flexible loads across the different TOU intervals, at the beginning of the day, in order to maximize her own utility under the set energy source allocation and payment policies. In contrast, in a Centralized ESM (C-ESM), the community manager would schedule the daily flexible loads of all consumers across the different TOU intervals in order to minimize the social cost of the community as a whole under the set energy source allocation and payment policies (see Section \ref{sec:coordinated}). As implementing this centralized approach would require for the community manager to have information on each consumer's preferences, it can only be considered as an ideal benchmark against which to compare the efficiency of the proposed D-ESM.

In this paper, we study \textit{mixed strategies} of consumer types. A \textit{mixed strategy} is a probability distribution $\mathbf{p}_{\vartheta}=[p_{\vartheta}^d, p_{\vartheta}^n]^T$, with $p_{\vartheta}^d \in [0,1]$ denoting the probability that a consumer of type $\vartheta \in \Theta$ schedules her daily flexible load during daytime, and $p_{\vartheta}^n \in [0,1]$ during nighttime. At the beginning of the day a consumer $i$ determines her mixed strategy based on her type $\vartheta_i \in \Theta$, $\mathbf{p}_{\vartheta_i}$. Then she schedules her daily flexible loads either in daytime or in nighttime with probabilities $p_{\vartheta}^d$, $p_{\vartheta}^n$, correspondingly. Let also $\bm{p}$ be the collection of mixed strategies of all consumers, i.e., $\bm{p}= \{\bm{p}_{\vartheta_i}\}_{i \in \mathcal{N}}$. 

%As the self-interested consumers internalize the energy source allocation and payment policies set by the community manager to schedule they daily flexible loads, 
%It is essential to design adequate energy source allocation policies that will incentivize consumers to act in a beneficial way for the community as a whole and in the following we introduce the proposed PA policy that satisfies desirable notions of fairness.
 
\subsubsection{Energy Source Allocation and Payment Policies} \label{sec:policies}
 
Once the daily flexible loads of all consumers have been scheduled, the community manager must allocate the available energy sources at each TOU interval (daytime or nighttime) among them. During nighttime, all scheduled loads are covered by grid imports since this is the sole available energy source for this TOU interval. During daytime, the community manager allocates in priority the local RESs production to cover the scheduled daytime loads, in order to maximize local consumption from the community and reduce energy costs. However, if the expected aggregate daytime energy demand exceeds the available local RESs production, the community manager must share this limited resource among those consumers with loads scheduled during daytime. This raises the challenging issue of allocating fairly a limited resource among users with equal claims to it.

%%
In order to ensure a notion of fairness among community members, the community manager allocates to each consumer $i$ of type $\vartheta_i$ a share of the local RESs production proportional to her daytime load schedule. As a result, under this PA policy, the local RESs production allocated to a consumer whose daily flexible load is scheduled during daytime is

\vspace{-0.1in}
\begin{small}
\begin{eqnarray}
res^{PA}_{\vartheta_i}(\mathbf{p}) &=& \frac{E_{\vartheta_i}}{\max(\mathcal{RE} , D^d(\mathbf{p}))}\mathcal{RE},
\label{eq:prop_alloc_energy}
\end{eqnarray}
\end{small}
\vspace{-0.1in}

\noindent where $D^d(\mathbf{p})$ denotes the expected aggregate daytime demand of the community.

Each consumer $i$ of type $\vartheta_i$ must then pay for the different energy sources covering her scheduled load at each TOU interval, ensuring budget balance of the proposed mechanism.

%%% assumptions
\vspace{-0.1in}
\subsection{Non-cooperative Game Formulation} \label{sec:game_def}
%\vspace{-0.05in}

Based on the proposed D-ESM framework, if a consumer schedules her daily flexible load during daytime, she competes with other consumers to use the limited local RESs production and incurs a financial risk. This competition among the consumers participating in the proposed D-ESM (for one single day) can be modeled as an Energy Sharing Game (ESG), as defined bellow.

%%When scheduling their daily flexible loads, each consumer $i$ has perfect knowledge about the energy sources parameters $\Omega = \{\mathcal{RE}, c^{RES}, c^{grid,d}, c^{grid,n}\}$ and its own flexibility preferences ($E_{\vartheta_i}$ and $\mu_{\vartheta_i}$), represented by its type $\vartheta_i$, and imperfect knowledge about other players' preferences, represented by the distribution $\mathbf{r}$ over consumers' types $\Theta = \{\vartheta_i\}_{i \in \mathcal{N}}$.

\vspace{-0.05in}
\begin{definition}\label{def:energy_source_game}
An \emph{Energy Sharing Game (ESG)} is a single-shot noncooperative game, defined by the tuple
\\$\Gamma=(\mathcal{N}, \{\mathcal{P}_{\vartheta_i}\}_{i\in\mathcal{N}}, \{\upsilon_{\vartheta_i}\}_{i\in \mathcal{N}})$, where:
\begin{itemize}
    \item $\mathcal{N}=\{1,...,N\}$ is the set of players, i.e., the consumers in the energy sharing community.
    \item $\mathcal{P}_{\vartheta_i} = \{ \mathbf{p}_{\vartheta_i} | \mathbf{p}_{\vartheta_i} : A_i \in \mathcal{A} \rightarrow p^{A_i}_{\vartheta_i} \in \mathbb{R}^+ , \text{ with } \sum_{A_i \in \mathcal{A}} p^{A_i}_{\vartheta_i} = 1 \}$ is the set of mixed strategies of player $i$ of type $\vartheta_i$ over the set of pure strategies $\mathcal{A}=\{d,n\}$, consisting of the choices to schedule her daily flexible load during daytime ($A_i = d$) or during nighttime ($A_i = n$). Therefore, each consumer $i$ of type $\vartheta_i$ with a mixed strategy $\mathbf{p}_{\vartheta_i}$, plays this game by randomly selecting an action $A_i \in \mathcal{A}$ with probability $p^{A_i}_{\vartheta_i}$\footnote{Note that a \textit{pure strategy} is a special case of a mixed strategy where one action has a probability equal to 1 (and the remaining have 0).}.
    \item $\upsilon_{\vartheta_i} : A_i \in \mathcal{A} \rightarrow  \upsilon^{A_i}_{\vartheta_i}$ is the payoff function of a consumer $i$ of type $\vartheta_i$ over the set of pure strategies $\mathcal{A}$. The cost of a consumer $i$ of type $\vartheta_i$ who plays the pure strategy $A_i = d$, is 
    \begin{small}
    \begin{align}
    \upsilon^{d}_{\vartheta_i} = c^{RES} res_{\vartheta_i}^{PA}(\mathbf{p}) + c^{grid,d} (E_{\vartheta_i}-res_{\vartheta_i}^{PA}(\mathbf{p})),
    \label{eq:RES_cost}
    \end{align}
    \end{small}
    \hspace{-0.1in} and depends on the strategy profile $\bm{p}$ of all consumers via the community's expected aggregate daytime energy demand $D^d(\mathbf{p})$. The cost of a consumer who plays the pure strategy $A_i = n$ is
    \begin{small}
    \begin{align}\upsilon^{n}_{\vartheta_i} = U_{\vartheta_i}  c^{grid,n},
    \label{eq:nonRES_cost}
    \end{align}
    \end{small}
    \hspace{-0.1in} and is independent on other consumers' mixed strategies.
    Before making their decisions all players have perfect knowledge of the energy sources parameters in the set $\Omega$ and their own preferences and type, and have prior knowledge on the probability distribution $\bm{r}$ over the other consumers' types.
\end{itemize}
\end{definition}


A consumer of type $\vartheta \in \Theta$ repeatedly playing the mixed strategy $\bm{p}_{\vartheta}$ over multiple instances of the ESG %(or equivalently, a large number of consumers of type $\vartheta \in \Theta$ playing the mixed strategy $\bm{p}_{\vartheta}$ over a single instance of the ESG), 
would have an \textit{expected} daytime and nighttime energy demand equal to $D_{\vartheta}^{d} = p_{\vartheta}^{d} E_{\vartheta}$ and $D_{\vartheta}^{n} = p_{\vartheta}^{n} U_{\vartheta}$, respectively. Therefore, the mixed strategy of a consumer $i$ of type $\vartheta_i$ can alternatively be interpreted as splitting her daily flexible loads between daytime and nighttime, such that her daytime load schedule is equal to $D_{\vartheta_i}^{d}$, and their nighttime load schedule is equal to $D_{\vartheta_i}^{n}$. %Then, each consumer $i$ tries to maximize its expected profit by scheduling a load equal to $E_{\vartheta_i}$ during the day with probability $p^{d}_{\vartheta_i}$, and a load equal to $U_{\vartheta_i}$ during the night with probability $p^{d}_{\vartheta_i}$.
With these notations, the \textit{expected} aggregate daytime and nighttime energy demands of the community are respectively expressed as

\vspace{-0.1in}
\begin{small}
\begin{subequations}
\begin{align}
    & D^d (\mathbf{p}) = N\sum_{\vartheta\in \Theta} r_{\vartheta }~p^d_{\vartheta}~E_{\vartheta,}\label{eq:demand} \\
     & D^n (\mathbf{p}) = N\sum_{\vartheta\in \Theta} r_{\vartheta }~p^n_{\vartheta}~U_{\vartheta}\label{eq:demand_n}.
 \end{align}
\end{subequations}
\end{small}
\vspace{-0.15in}



\vspace{-0.1in}\section{Analysis of the Decentralized Energy Sharing Mechanism}\label{sec:gameanalysis}

%This behavior represents consumers with a broad range of flexible loads, including shiftable appliances, such as washing machines that do not need to run every day, as well as EVs, water heaters, and batteries that do not need to be fully charged at the end of a given day. For instance EV owners would compute their minimum (inflexible) daily energy load, representing the energy needed to cover their transportation needs for the day, as well as their flexible energy load, representing the additional energy needed to fully charge their EV. Then, if they decide to compete for RESs, they may be willing to engage only part of their daily flexible load to mitigate the risk of paying for the high-priced peak-load production. At the beginning of the following day that they play this game, they would update their daily inflexible and flexible loads and their risk attitude based on their new state-of-charge and transportation needs.

In this section, we study analytically the uncoordinated decisions of the self-interested consumers participating in the proposed D-ESM. In the following, we study the conditions on the parameter values for the existence of dominant strategies and mixed-strategy NE under the proposed PA and payment policies, and provide closed-form formulations of these equilibrium states, i.e., ranges on the values of the vector of mixed strategies at NE, denoted as $\mathbf{p^{NE}}$ and an analytical expression on the value of the expected aggregate daytime energy demand. The proofs of the theoretical results presented below are available in the {Appendix \ref{sec:proofsESSG}} of \cite{arxiv_version}.

First, we recall that, for a mixed-strategy NE to exist, the expected costs of each consumer for all pure strategies in the support of the mixed-strategy NE must be equal. Using the expressions of the costs in \eqref{eq:RES_cost} and \eqref{eq:nonRES_cost}, we obtain that the amount of RESs allocated to a consumer type $\vartheta \in \Theta$ at a NE must satisfy:

 \vspace{-0.1in} 
\small
\begin{equation}\label{eq:conditionEQ_extra_demand}
res^{NE}_{\vartheta}(\mathbf{p^{NE}}) = \frac{\gamma-\varepsilon_{\vartheta}\beta}{\gamma-1}E_{\vartheta},~ \forall \vartheta \in \Theta.
\end{equation}
\normalsize 
\vspace{-0.1in}

\noindent Thus, in the ESG, a mixed-strategy NE exists under the condition:

 \vspace{-0.1in} 
 \small
\begin{equation}\label{eq:condition_PA_NE}
res_{\vartheta}^{PA}(\mathbf{p}^{NE}) = res_{\vartheta}^{NE}(\mathbf{p}^{NE}), ~\forall \vartheta \in \Theta, \end{equation}
\normalsize 
\vspace{-0.15in}

\noindent where $res_{\vartheta}^{PA}(\mathbf{p}^{NE})$ is defined in \eqref{eq:prop_alloc_energy}. In the following analysis, we obtain the mixed-strategy NE competing probabilities $\mathbf{p}^{NE}$ by solving Equation \eqref{eq:condition_PA_NE}. We further distinguish cases with respect to the available RESs production, TOU tariffs, and consumers' types.


\subsubsection*{\textbf{Case $1$: $\bm{\mathcal{RE}}$ exceeds $\bm{D^{Total}}$}}

As consumers have knowledge of $\mathcal{RE}$ and $D^{Total}$, it is straightforward to show that the dominant-strategy for all consumers is to schedule their daily flexible loads during daytime. As a result, the competing probabilities that lead to equilibrium states are equal to $p_{\vartheta}^{d,NE} = 1$ for all consumer types $\vartheta \in \Theta$.


\subsubsection*{\textbf{Case $2$: $\bm{\mathcal{RE}}$ is lower than $\bm{D^{Total}}$}} 

In this case, the strategies of the consumers depend on their respective risk aversion degrees and the TOU tariffs. We define two complementary subsets of consumer types, depending on their risk aversion degrees: $\Sigma_1 = \Bigl\{\vartheta \in \Theta : \varepsilon_{\vartheta} \geq \gamma/\beta \Bigr\} \subset \Theta$, and $\Sigma_2 = \Bigl\{ \vartheta \in \Theta :1\leq \varepsilon_{\vartheta} < \gamma/\beta \Bigr\} \subset \Theta$.


Firstly, the dominant strategy for all consumers $i$ whose type $\vartheta_i$ is in the set $\Sigma_1$ is to schedule their daily flexible loads during daytime, i.e., to play the pure strategy $A_i = d$ with probability $p_{\vartheta_i}^{d,NE} = 1$. 
%Their expected aggregate daytime load schedule at NE is $D^{Total}_{\Sigma_1£\mathcal{S}$=N\sum_{\theta \in \Sigma_1}r_{\theta} E_{\theta}$.

Secondly, the strategies of the consumers $i$ whose type $\vartheta_i$ is in the set $\Sigma_2$ depend on their daily flexible loads and risk-aversion degrees. We define two distinct subsets of consumer types in $\Sigma_2$: $\Sigma_{2,1} = \left\{ \vartheta \in \Sigma_2 : E_{\vartheta} > \mathcal{RE}\frac{(\gamma-1)}{(\gamma-\varepsilon_{\vartheta}\beta)} \right\}$ and $\Sigma_{2,2} = \left\{ \vartheta \in \Sigma_2 : E_{\vartheta} \leq \mathcal{RE}\frac{(\gamma-1)}{(\gamma-\varepsilon_{\vartheta}\beta)}\right\}$.

%$\Sigma_{2,1} = \left\{ \vartheta \in \Sigma_2 : E_{\vartheta} > \left(\mathcal{RE}-D^{Total}_{\Sigma_1}\right)\frac{(\gamma-1)}{(\gamma-\varepsilon_{\vartheta}\beta)} \right\}$ and $\Sigma_{2,2} = \left\{ \vartheta \in \Sigma_2 : E_{\vartheta} \leq \left(\mathcal{RE}-D^{Total}_{\Sigma_1}\right)\frac{(\gamma-1)}{(\gamma-\varepsilon_{\vartheta}\beta)}\right\}$

For consumers $i$ whose type $\vartheta_i$ is in the set $\Sigma_{2,1}$, the dominant strategy is to schedule their daily flexible loads during nighttime, i.e., to play the pure strategy $A_i=n$ with probability $p^{n,NE}_{\vartheta_i}=1$ and $A_i=d$ with probability $p^{d,NE}_{\vartheta_i}=0$.

For consumers $i$ whose type $\vartheta_i$ is in the set $\Sigma_{2,2}$, a mixed-strategy NE under the PA policy exists if and only if the following condition holds:

 \vspace{-0.1in} 
 \small
\begin{equation}\label{eq:relation_E_0_E_1_pa_ne_extra_demand}
\begin{split}
& \mathcal{RE}\frac{(\gamma-1)}{(\gamma-\varepsilon_{\vartheta}\beta)}-E_{\vartheta}  = \mathcal{RE}\frac{(\gamma-1)}{(\gamma-\varepsilon_{\tilde{\vartheta} }\beta)}-E_{\tilde{\vartheta}} , \ \forall \vartheta , \tilde{\vartheta} \in \Sigma_{2,2}.
\end{split}
 \end{equation}
\normalsize 
\vspace{-0.1in}

\noindent Assuming that all consumers of the same type play the same mixed strategy, the competing probabilities that lead to NE states lie in the range $p^{min}_{\vartheta} \leq p^{d,NE}_{\vartheta} \leq p^{max}_{\vartheta}$ for all consumer types $\vartheta \in \Sigma_{2,2}$ with:

 \vspace{-0.1in} 
 \footnotesize
   \begin{align}
& p^{max}_{\vartheta} = \min \left\{1,\frac{\frac{\mathcal{RE}(\gamma-1)}{(\gamma-\varepsilon_{\vartheta}\beta)}-D^{Total}_{\Sigma_1}}{N~ r_{\vartheta}~ E_{\vartheta}}\right\},
    \label{eq:plmaxbound} \\
&  p^{min}_{\vartheta}= \max \left\{0,\frac{\frac{\mathcal{RE}(\gamma-1)}{(\gamma-\varepsilon_{\vartheta}\beta)}-D^{Total}_{\Sigma_1 \bigcup \Sigma_{2,2}\setminus \{\vartheta\}}}{N~ r_{\vartheta}~E_{\vartheta}} \right\},\label{eq:plminbound}
  \end{align}
\normalsize  
where for any subset of consumer types $\mathcal{S} \subset \Theta$, $D^{Total}_{\mathcal{S}}$ represents the maximum aggregate daytime demand of consumers whose type is in $\mathcal{S}$, e.g., $D^{Total}_{\Sigma_1}=N\sum_{\theta \in \Sigma_1}r_{\theta} E_{\theta}$.


 As a result, the expected aggregate daytime demand, $D^{d,NE}$ at NE is

\vspace{-0.1in} 
\footnotesize
\begin{align}
& D^{d,NE} = D^{Total}_{\Sigma_1} \nonumber \\
& + \min \left\{ D^{Total}_{\Sigma_{2,2}} , \max \left\{\frac{N\left(\frac{\mathcal{RE}(\gamma-1)}{(\gamma-\varepsilon_{\vartheta}\beta)}-E_{\vartheta}-D^{Total}_{\Sigma_1}\right)}{(N-1)},0 \right\} \right\}. \label{eq:demand1_2c}
\end{align}
\normalsize 
%\vspace{-0.1in}

\begin{remark} \label{rem:risk_degrees_relation}
Note that condition \eqref{eq:relation_E_0_E_1_pa_ne_extra_demand} can hold, and therefore a NE can exist, only if for any pair $\vartheta , \tilde{\vartheta} \in \Sigma_{2,2}$ such that $\vartheta \leq \tilde{\vartheta}$, it holds that $\varepsilon_{\vartheta} \leq \varepsilon_{\tilde{\vartheta}}$. Since by assumption, $E_{\vartheta} \leq E_{\tilde{\vartheta}}$, this means that consumers with lower daytime energy demand levels should be more risk-seeking than those with higher ones.
\end{remark}
%\vspace{-0.15in}
\begin{remark}\label{rem:risk_seeking}In particular, if all consumers whose type is in $\Sigma_{2,2}$ are risk-seeking (i.e., $\varepsilon_{\vartheta}= 1, \forall \vartheta \in \Sigma_{2,2}$), a NE can only exist if  $E_{\vartheta}=E_{\tilde{\vartheta}} , \ \forall \vartheta , \tilde{\vartheta} \in \Sigma_{2,2}$.
\end{remark}

%Finally, the social cost is given by:
%\vspace{-0.15in} 

%\begin{small}
	%\begin{align}
	%C(\mathbf{p}^{NE}) &=  \min\{\mathcal{RE}, D^d(\mathbf{p^{NE}})\} c^{RES} \nonumber \\&+  \max\{0,D^d(\mathbf{p^{NE}})-\mathcal{RE}\}c^{grid,d} \nonumber \\
	%&+N \left[ \sum_{\vartheta \in \Theta} r_{\vartheta }~ p^{n,NE}_{\vartheta}~ \varepsilon_{\vartheta }~E_{\vartheta}\right] c^{grid,n}.
	%\label{eq:social_cost_pa_extra_demand}
	%\end{align}
%\end{small}



\section{Centralized Energy Sharing Mechanism}\label{sec:coordinated}

In this section we study an ideal centralized scheduling problem, in which an energy community manager with perfect knowledge of the available energy sources and types of the consumers in the community, centrally schedules their daily flexible loads.

\subsection{Problem Formulation}

%Therefore, the aggregate amount of daytime loads is equal to $D^{Total}$. 

Based on the available information, the community manager aims at finding the optimal load schedule of each consumer type, which minimize the social cost of the community under the chosen PA and payment policy. The community's social cost $C^{PA}(\bm{p})$ can be expressed as a function of the \textit{expected} aggregate daytime energy demand ($D^d(\bm{p})$) and nighttime energy demand ($D^n(\bm{p})$) of the community (as defined in Section \ref{sec:game_def}), such that:

\vspace{-0.15in} 
\begin{small}
	\begin{align}
	C^{PA}(\bm{p}) &=  \min\{\mathcal{RE}, D^d(\bm{p})\} \cdot c^{RES} \nonumber \\
    &+ \max\{0,D^d(\bm{p})-\mathcal{RE}\} \cdot c^{grid,d} + D^n(\bm{p}) \cdot c^{grid,n},
\label{eq:social_cost_pa_extra_demand}
	\end{align}
\end{small}

\noindent where the probabilities $p^d_{\vartheta}$ and $p^n_{\vartheta}$ (as defined in Section \ref{sec:game_def}) can be interpreted as the proportion of consumers of type $\vartheta$ that the community manager schedules during daytime and nighttime, respectively.
Although this objective cost is non-convex, we observe that during daytime, for any expected aggregate load schedule, the community manager minimizes the cost from grid imports. Therefore, by introducing the optimization variable $D^{grid}$ representing the expected aggregate grid imports during daytime, we can write the community manager's optimal load scheduling problem under the PA policy as a linear optimization problem, as follows:

 \vspace{-0.1in} 
 \begin{small}
 \begin{subequations} \label{eq:social_cost_x_opt}
\begin{alignat}{2}
& \min_{\mathbf{p},D^{grid}} \ && c^{grid,d}  D^{grid} + c^{RES}  \left(N\sum_{\vartheta \in \Theta} r_{\vartheta} p_{\vartheta}^d E_{\vartheta} - D^{grid}\right) \nonumber \\
& \quad && + c^{grid,n}  N \sum_{\vartheta \in \Theta} r_{\vartheta} {p}_{\vartheta}^{n}U_{\vartheta} \label{eq:opt_1} \\
 & \text{s.t. } &&  p^{d}_{\vartheta} + p^{n}_{\vartheta} = 1 , \ \forall \vartheta \in \Theta, \label{eq:opt_2.1} \\
  & \quad && 0 \leq p^{d}_{\vartheta},  p^{n}_{\vartheta},  \ \forall \vartheta \in \Theta, \label{eq:opt_2.2} \\
 & \quad && D^{grid} \geq \mathcal{ER} - N \sum_{\vartheta \in \Theta} r_{\vartheta} {p}_{\vartheta}^{d} E_{\vartheta}, \label{eq:opt_3.1} \\
 & \quad && D^{grid} \geq 0. \label{eq:opt_3.2}
 \end{alignat}
 \end{subequations}
\end{small} \vspace{-0.1in}  

\noindent This optimization problem minimizes the social cost of the community \eqref{eq:opt_1}, subject to constraints on the daytime and nighttime probabilities \eqref{eq:opt_2.1}-\eqref{eq:opt_2.2}  as well as to lower bounds on the expected aggregate grid imports during daytime \eqref{eq:opt_3.1}-\eqref{eq:opt_3.2}.

\subsection{Solution Analysis} \label{sec:centralsol}

In the following, we provide insights and analytical formulations of the optimal solutions $\mathbf{p^{*}}$ of this centralized mechanism in different cases. The proofs are available in the {Appendix \ref{appendix:dual}} of \cite{arxiv_version}.
 
\subsubsection*{\textbf{Case $1$: $\bm{\mathcal{RE}}$ exceeds $\bm{D^{Total}}$}}

In this trivial case, the optimal solutions to the C-ESM is to schedule all consumers' daily flexible loads during daytime, such that $p^{d,*}_{\vartheta}=1$, $\forall \vartheta \in \Theta$, and the expected grid imports $D^{grid,*} =0$.

\subsubsection*{\textbf{Case $2$: $\bm{\mathcal{RE}}$ is lower than $\bm{D^{Total}}$}}

In this case, it is optimal for the centralized ESM to schedule loads during the day so that the total RES capacity is fully utilized. To perform the analysis, we use the two complementary subsets of consumer types, $\Sigma_1$ and $\Sigma_2$, as those are defined in Section \ref{sec:gameanalysis}. 

For all consumers whose type $\vartheta \in \Sigma_1$, it is optimal for the community to schedule them during daytime, such that $p^{d,*}_{\vartheta}=1 $. For the optimal load schedule of the remaining consumers whose type $\vartheta \in \Sigma_2$, we observe that the consumer types are scheduled during daytime in order of increasing risk aversion (i.e., decreasing $\varepsilon_\vartheta$), until the local RESs production is fully utilized. Therefore, the optimal competing probabilities for the consumers whose types are in $\Sigma_2=\{\tilde{\vartheta}^1, \tilde{\vartheta}^2, \dots, \tilde{\vartheta}^K \}$, can be expressed as:

 \vspace{-0.1in} 
 \footnotesize
\begin{align}
 & p^{d,*}_{\tilde{\vartheta}^k} = \max \Bigg\{ \min \Bigg\{ 1, \dfrac{\left( \mathcal{RE} - D^{Total}_{\Sigma_1} - N \sum_{i=1}^{k-1}r_{{\tilde{\vartheta}^i}} E_{{\tilde{\vartheta}^i}} p^{d,*}_{\tilde{\vartheta}^i}  \right)}{N r_{{\tilde{\vartheta}^k}} E_{{\tilde{\vartheta}^k} }}\Bigg\} , 0 \Bigg\} , \nonumber \\
&  \forall k \in \{1,...,K\},
\end{align}
\normalsize 
\vspace{-0.1in}  

\noindent where the consumer types in $\Sigma_2$ are ordered such that $\varepsilon_{\tilde{\vartheta}^1} \geq \varepsilon_{\tilde{\vartheta}^2} \geq ... \geq \varepsilon_{\tilde{\vartheta}^K}$.

%If $N\sum_{{\vartheta} \in \Sigma_1}r_{{\vartheta}} E_{{\vartheta}} \geq \mathcal{RE}$, i.e., the daytime consumption of the consumers whose type $\vartheta \in \Sigma_1$ fully utilizes the RES capacity, then the solution to this optimization problem is trivial, and for all consumers whose type $\vartheta \in \Sigma_2$, $p^{}_{RES,\vartheta}=0$.


\section{Efficiency Loss of D-ESM vs. C-ESM}\label{sec:efficiency}
The (in)efficiency of equilibrium strategies in the D-ESM compared to the optimal C-ESM solution is quantified by the Price of Anarchy (PoA) metric \cite{Koutsoupias09}, representing the ratio of the worst case social cost among all mixed strategy NE, denoted as $C^{PA,NE}_{WC}$, over the optimal minimum social cost of the C-ESM, such that:

%\vspace{-5pt}
 \vspace{-0.1in} 
 \small
\begin{align}
 \hspace{-5pt} \textit{PoA}  = \frac{C^{PA,NE}_{WC}}{C^{PA}(\mathbf{p^{^*}})}.
\label{eq:poa_pa}
\end{align}
\normalsize 
\vspace{-0.1in}  

First observe that $C^{PA}(\mathbf{p^{^*}})$ is uniquely determined for each particular case (Section \ref{sec:coordinated}). Now, in order to obtain $C^{PA,NE}_{WC}$ when there exist multiple possible NE, we can maximize the social cost $C^{PA}(\mathbf{p^{NE}})$ (Eq. \eqref{eq:social_cost_pa_extra_demand}) with respect to $\mathbf{p^{NE}}$. 


%Note that $C^{PA,NE}(\mathbf{p^{NE}})$ takes its optimal value (i.e, minimum value) when the night-time cost is minimized. This solution coincides with the optimal centralized solution and thus in this case PoA takes the optimal (unity) value. 


 %The last observation is the fact that the demand $D^{PA,NE}$ is constant with respect to $\mathbf{p^{PA,NE}}$. Also, in the special case of the risk-seeking consumers, from Eq. \eqref{eq:social_cost_pa_sc}, the social cost is constant with respect to the probabilities $\mathbf{p^{PA,NE}}$ for each case of energy profile values. Thus, $C^{PA,NE}_w$ is given by Eq. \eqref{eq:social_cost_pa_sc}.

%$2$. If having risk-conservative consumers, there exist multiple possible equilibria in the sub-cases 2(a) and 2(c) as well as in case 4. Of course, the multiple combinations of probabilities can lead to NE with possibly different social cost values. Based on the Remark \ref{rem:risk_degrees_relation}, the existence of NE is possible if consumers with lower energy demands have lower risk aversion degrees. Thus, the night cost is minimized if the optimal probabilities for RES take lower values for consumers with lower energy demands. 

\begin{algorithm}
\footnotesize
    \caption{$k$-SALSA}
    \label{alg:overall}
    \textbf{Input:} Private dataset $X=(x_1,\dots,x_n)$, auxiliary dataset $X_0$ for GAN model training, integer $k>1$ (assume $n = mk$ for integer $m$ without loss of generality), number of iterations $T$, loss ratio parameter $\lambda$ \\
    \textbf{Output:} Synthetic dataset $\tilde{X}$ of size $m$ with $k$-anonymity

    \begin{algorithmic}[1]
        \State Train a GAN generator $G$ and a GAN inversion encoder $E$ on $X_0$
        \State Obtain latent code $w_i = E(x_i)$ for each $i\in [n]$ and let $W=\{w_i\}_{i=1}^n$
        \State $(C_1,\dots,C_m)= \textsf{SameSizeClustering}(W, k)$  \Comment{$C_j\subset W$, $|C_j|=k$, $|C_j\cap C_{j'\neq j}|=0$, $\forall j$}
      
        \State Initialize $\tilde{X} = \emptyset$
        \For {each cluster $j\in [m]$}
        \State Let $C_j = (w_1',\dots,w_k')$, and $x_i'$ the original image of $w_i'$ for each $i$
        \State Compute $w_0 = \frac{1}{k} \sum_{i=1}^{k} w'_i$ and generate $x_0 = G(w_0)$
        \State Initialize $w_\text{avg}^{(0)} = w_0$
        \For {each iteration $t \in [T]$}
            \State Generate $x_\text{avg}^{(t-1)} = G(w_\text{avg}^{(t-1)})$
            \State Compute content loss $\mathcal{L}_\text{content}(x_0, x_\text{avg}^{(t-1)})$ using Eq.~\ref{eq:loss-content} 
            \State Compute local style alignment loss $\mathcal{L}_\text{style}((x'_1,\dots,x'_k), x_\text{avg}^{(t-1)})$ using Eq.~\ref{eq:loss-style}
            \State Compute total loss $\mathcal{L}_\text{total}=\lambda \mathcal{L}_\text{content}+(1-\lambda)\mathcal{L}_\text{style}$
            \State Update $w_{\text{avg}}^{(t)}$ using $w_{\text{avg}}^{(t-1)}$ and the gradient $\nabla_{w_\text{avg}^{(t-1)}} \mathcal{L}_\text{total}$
        \EndFor
        \State Add $G(w_{\text{avg}}^{(T)})$ to $\tilde{X}$
        \EndFor
    \State    \Return $\tilde{X}$

    \end{algorithmic}
\end{algorithm}

 \section{Benchmarks and Evaluation}
\label{sec:eval}

We evaluate \krakenSpace to answer the following set of questions:
\begin{itemize}
\item How much improvement does partial evaluation and our implemented compiler optimizations give \kraken? %(\S \ref{sec:eval2})
\item How much faster is our purely functional f-expr language, \krakenSpace, compared to other implementations of fexprs? %(\S \ref{sec:eval1} - \ref{sec:eval2})
\item How does \kraken's performance, with its fexprs, compare to macros? %(\S \ref{sec:eval1}, \S \ref{sec:eval3})
\item How do the different partial evaluation mechanisms/optimizations in \krakenSpace contribute towards reduction in overall runtime?
%\item What does \krakenSpace do internally when we create a data structure and evaluate it for some function? (\S \ref{sec:casestudy})
\end{itemize}

\textbf{Experimental Setup}: 
We ran these experiments in a reproducible Nix environment on a NixOS install \cite{10.1145/1411203.1411255} (Kernel 6.0.0) on a laptop with 8 cores / 16 threads and 64 GB of RAM.
Our code contains the scripts and Nix Flakes needed to reproduce the exact set of dependencies to run our tests.
%The code can be found at \url{https://github.com/limvot/kraken}.

The Kraken benchmarks were run using both the Wasmtime and WAVM WebAssembly engines for most benchmarks.
The Wasmtime WebAssembly engine is one of the most popular, developed by the Bytecode Alliance itself, and uses the CraneLift code generation backend.
The WAVM WebAssembly engine is interesting for its use of LLVM, and it often produces the fastest code on benchmarks but has a higher startup time.
We eliminated the Cfold Wasmtime benchmark due to problems running out of stack space (a known property of the Cfold benchmark).

\textbf{Benchmarks}: 
To showcase the capability of Kraken, we created benchmarks that are commonly implemented in functional languages and have been used as benchmarks in other papers \cite{reinking2021perceus, 10.1145/3547646}.
The benchmarks are
\begin{itemize}
\item Fib - Calculating the nth Fibonacci number
\item RB-Tree - Inserting n items into a red-black tree, then traversing the tree to sum its values
\item Deriv - Computing a symbolic derivative of a large expression
\item Cfold - Constant-folding a large expression
\item NQueens - Placing n number of queens on the board such that no two queens are diagonal, vertical, or horizontal from each other
\end{itemize}
All benchmarks besides Fibonacci use the fexpr version of match for pattern matching in \kraken, which is equivalent to the macro version in NewLisp. We also RB-Tree using NewLisp's~\cite{mueller2018newlisp} version of fexpr match. We modified the sizes of the problems presented to the benchmark to account for the longer running times of some of the less-optimized implementations.
The code for Kraken and NewLisp is very similar, and we should note that it is very unidiomatic NewLisp.
Our goal was not to compare Kraken and NewLisp as implementation languages for Red-Black Trees, but to stress test a single reasonably complex fexpr/macro, namely pattern matching.
% \textbf{Comparison with other languages}: We evaluated \krakenSpace against a language that contains f-exprs, as well as against itself with various optimizations disabled. The only other language we could find which contains a real f-expr mechanism is NewLisp~\cite{mueller2018newlisp} and so we ported \kraken's benchmark implementation to NewLisp.

%The six state-of-the-art languages are Java 17.0.1, Swift 5.4.2, Koka 2.3.2, C++, Haskell 8.10.7, and OCaml 4.12.
%The language choices were taken directly from Perceus reference-counting paper \cite{reinking2021perceus}.
%The Fibonacci benchmark additionally tests Python 3.9.11 and Chez Scheme 9.5.4.
%Koka, Ocaml and Haskell are good comparison points as statically-typed, compiled, functional programming languages, while Chez Scheme is a good comparison point as a mature and industrial strength dynamically-typed Scheme implementation known for its performance. 
%\subsection{Basic Level Comparison}
\subsection{The Effect of Partial Evaluation on Eval Calls}

\begin{table}[h]
\caption{Number of eval calls with no partial evaluation for Fexprs}
	\begin{tabular}{||c | c c c c c ||} 
		\hline
		&Evals & Eval w1 Calls & Eval w0 Calls & Comp Dyn & Comp Dyn\\ 
        & & & & w1 Calls & w0 Calls\\ [0.5ex] 
		\hline\hline
		Cfold 5 & 10897376 & 2784275 & 879066  & 1 & 0 \\ 
		\hline
		  Deriv 2  & 11708558 & 2990090 & 946500 & 1 & 0 \\ 
        \hline
		  NQueens 7 & 13530241 & 3429161 & 1108393 & 1 & 0 \\ 
    \hline
		  Fib 30 & 119107888 & 30450112 & 10770217 & 1 & 0 \\ 
    \hline
		  RB-Tree 10 & 5032297 & 1291489 & 398104 & 1 & 0 \\ 
		\hline
	\end{tabular}
    \label{npe:calls}
 \end{table}

As mentioned before, using fexprs without partial evaluation will prelude optimization and cause a massive amount of repeated work. Table \ref{npe:calls} and Table \ref{pe:calls} show the number of calls to the \krakenSpace runtime's eval function, the number of times the runtime's eval function executed a call to an applicative with wrap\_level=1, the number of times the runtime's eval function executed a call to an operative with wrap\_level=0, the number of compiled dynamic calls to applicatives with wrap\_level=1, and the number of compiled dynamic calls to operatives with wrap\_level=0.
These are shown for \krakenSpace test cases with partial evaluation turned off and turned on. 
\begin{table}[h]
\caption{Number of eval calls in Partially Evaluated Fexprs}
	\begin{tabular}{||c | c c c c c ||} 
		\hline
		&Evals & Eval w1 Calls & Eval w0 Calls & Comp Dyn & Comp Dyn\\ 
        & & & & w1 Calls & w0 Calls\\ [0.5ex] 
		\hline\hline
		Cfold 5 & 0 & 0 & 0  & 0 & 0 \\ 
		\hline
		  Deriv 2  & 0 & 0 & 0 & 2 & 0 \\ 
        \hline
		  NQueens 7 & 0 & 0 & 0 & 0 & 0 \\ 
    \hline
		  Fib 30 & 0 & 0 & 0 & 0 & 0 \\ 
    \hline
		  RB-Tree 10 & 0 & 0 & 0 & 10 & 0 \\ 
		\hline
	\end{tabular}
    \label{pe:calls}
 \end{table}

\begin{table}[h]
\caption{Number of calls to the runtime's eval function for RB-Tree. The table shows the non-partial evaluation numbers -> partial evaluation numbers.}
	\begin{tabular}{||c | c c c c c ||} 
		\hline
		&Evals & Eval w1 Calls & Eval w0 Calls & Comp Dyn & Comp Dyn\\ 
        & & & & w1 Calls & w0 Calls\\ [0.5ex] 
		\hline\hline
		  RB-Tree 7 & 2952848 -> 0 & 757932 -> 0 & 233513 -> 0 & 1 -> 7 & 0 -> 0\\ 
        \hline
		  RB-Tree 8 & 3532131 -> 0 & 906548 -> 0 & 279379 -> 0 & 1 -> 8 & 0 -> 0\\ 
        \hline
		  RB-Tree 9 & 4278001 -> 0 & 1097965 -> 0 & 3383831 -> 0 & 1 -> 9 & 0 -> 0\\ 
		\hline
	\end{tabular}
    \label{pe:rb}
    \vspace{-4mm}
 \end{table}

Without partial evaluation, no compilation can be done because it is impossible to tell if arguments to calls will be evaluated. In all benchmarks, partial evaluation removed all calls to the runtime's eval function, resulting in a completely compiled program. Looking at RB-Tree, there are over a million calls to combiners with wrap level 1 (normal functions), and 398,000 calls to combiners with wrap level 0 (operatives replacing macros). This massive blowup in the number of calls is due to the repeated and exponential re-execution of macro-like-combiners in the definition of other macro-like-combiners, as discussed in the Introduction.

The non-partially-evaluated benchmarks show 1 compiled dynamic call to an applicative (its the first call into eval) and 0 compiled dynamic calls to operatives, because there is no compilation at all. For the partially evaluated benchmarks, there are a few compiled dynamic calls to applicatives due to higher-order function use in the benchmarks, and there are no compiled dynamic calls to operatives, as all operative use has been eliminated.
We also varied the inputs for RB-Tree shown in Table \ref{pe:rb} to give a sense for how the number scale with respect to input size.

The incredible slowdown implied by these tables comes to full fruition in our RB-Tree test in Fig.~\ref{fig:kraken_nqueens_rbtree}.
We kept this run shorter because Kraken's non-partial-evaluating interpreter takes an incredibly long time even for 100 insertions (40 minutes).
The compounding layers of repeated macro-like operative calls in the non-partially-evaluated Kraken version cause a ~70,000x slowdown relative to the partial evaluated, optimized, and compiled version.
For the remaining benchmarks, we remove the naive interpreted \krakenSpace version, as in each case its performance is so bad as to blow out the graph and make it impossible to do any comparison.
In our optimized Kraken, our partial evaluation algorithm is able to fully collapse these levels of inefficiency, evaluate and inline the results, and give the backend more specialized code to optimize, emitting a compiled version that handily beats not only the NewLisp-fexpr implementation but even the NewLisp-macro implementation, as can be seen in Fig.~\ref{fig:kraken_vs_world_fib}.
We kept the benchmark sizes small in this test because the stack limits of NewLisp prevent sizes larger then ~880, while the Tail Call Elimination performed by the \krakenSpace compiler allows us to run much larger benchmarks, including the run of 4,800,000 inserts to the RB-Tree.
This result shows the dramatic effect of partial evaluation and compiler optimizations on runtime for \kraken. Our technique takes the performance of a fully fexpr based language from being completely infeasible to being faster than a macro-based dynamic scripting language currently in use.
% \begin{center}
% \begin{table}[ht]
% \caption{Number of call to the runtime's eval function for Fib. The table shows the non-partial evaluation numbers -> partial evaluation numbers}
% 	\begin{tabular}{||c | c c c c c ||} 
% 		\hline
% 		&Evals & Eval w1 Calls & Eval w0 Calls & Comp Dyn w1 Calls & Comp Dyn w0 Calls\\ [0.5ex] 
% 		\hline\hline
% 		Fib 10 & 8468 -> 0 & 2167 -> 0  & 777 -> 0 & 1 -> 0 & 0 -> 0 \\ 
% 		\hline
% 		  Fib 15  & 87916 -> 0 & 22478 -> 0 & 7961 -> 0 & 1 -> 0 & 0 -> 0 \\ 
%         \hline
% 		  Fib 20 & 969010 -> 0 & 247731 -> 0 & 87633 -> 0 & 1 -> 0 & 0 -> 0 \\ 
%     \hline
% 		  Fib 25 & 10740492 -> 0 & 2745825 -> 0  & 971209 -> 0 & 1 -> 0 & 0 -> 0 \\ 
% 		\hline
% 	\end{tabular}
%     \label{pe:fib}
%  \end{table}
% \end{center}

\begin{figure}[h]
\caption{Constant Fold and Deriv}
\includegraphics[width=0.45\textwidth]{cfold_table.csv_}
\includegraphics[width=0.45\textwidth]{deriv_table.csv_}
\label{fig:kraken_const_deriv}
\vspace{-6mm}
\end{figure}
\subsection{Comparison between Kraken Versions}
Beyond the massive speedup from partial-evaluation, Fig. \ref{fig:kraken_const_deriv} and \ref{fig:kraken_nqueens_rbtree} show the effect of the various compiler optimizations we described by disabling them one by one.
 Our main four optimizations have a strong positive effect on runtime, with the exception of lazy environment instantiation. Lazy environment instantiation helps massively on fib, and some on Deriv, but generally hurts the rest slightly.


\begin{figure}[h]
\caption{N-Queens}
\includegraphics[width=0.45\textwidth]{nqueens_table.csv_}
\includegraphics[width=0.45\textwidth]{slow_rbtree_table.csv_}
\label{fig:kraken_nqueens_rbtree}
\vspace{-4mm}
\end{figure}


\subsection{Comparison against Others}


To give a general idea of our current performance, we also show a Fibonacci benchmark that mostly exercises pure function-call speed and inlining as seen in Fig. ~\ref{fig:kraken_vs_world_fib}.
We include Python and Chez Scheme to give a general idea for where an exemplar slow and an exemplar fast dynamic language would fall.
With the benefit of our partial evaluation, compilation, and leaning upon mature WebAssembly implementations, we beat both, but this should be taken with a grain of salt, as this is a very limited micro-benchmark only meant to give a general sense of the order of magnitude of our performance.



\label{sec:eval1}
\begin{figure}[h]
\caption{Kraken vs. Others. Ordered by fastest to slowest}
\includegraphics[width=0.45\textwidth]{fib_table.csv_}
\includegraphics[width=0.45\textwidth]{rbtree_table.csv_}
\label{fig:kraken_vs_world_fib}
\end{figure}

%\label{sec:eval_nqueens}
%\begin{figure}[h]
%\caption{N-Queens}
%\includegraphics[width=0.45\textwidth]{nqueens_table.csv_}
%\includegraphics[width=0.45\textwidth]{slow_nqueens_table.csv_}
%\label{fig:kraken_nqueens}
%\end{figure}

%\label{sec:eval_nqueens}
%\begin{figure}[h]
%\caption{Kraken, N-Queens, absolute value and log-scale}
%\includegraphics[width=0.45\textwidth]{nqueens_table.csv_}
%\includegraphics[width=0.45\textwidth]{nqueens_table.csv_log}
%\label{fig:kraken_nqueens}
%\end{figure}
%\label{sec:eval_nqueensp}
%\begin{figure}[h]
%\caption{Kraken, N-Queens, absolute value and log-scale}
%\includegraphics[width=0.45\textwidth]{slow_nqueens_table.csv_}
%\includegraphics[width=0.45\textwidth]{slow_nqueens_table.csv_log}
%\label{fig:kraken_nqueensp}
%\end{figure}

%\label{sec:eval_cfold}
%\begin{figure}[h]
%\caption{C-Fold}
%\includegraphics[width=0.45\textwidth]{cfold_table.csv_}
%\includegraphics[width=0.45\textwidth]{slow_cfold_table.csv_}
%\label{fig:kraken_cfold}
%\end{figure}
%\label{sec:eval_cfold}
%\begin{figure}[h]
%\caption{Kraken, C-Fold, absolute value and log-scale}
%\includegraphics[width=0.45\textwidth]{cfold_table.csv_}
%\includegraphics[width=0.45\textwidth]{cfold_table.csv_log}
%\label{fig:kraken_cfold}
%\end{figure}
%\label{sec:eval_cfoldp}
%\begin{figure}[h]
%\caption{Kraken, C-Fold, absolute value and log-scale}
%\includegraphics[width=0.45\textwidth]{slow_cfold_table.csv_}
%\includegraphics[width=0.45\textwidth]{slow_cfold_table.csv_log}
%\label{fig:kraken_cfoldp}
%\end{figure}

%\label{sec:eval_deriv}
%\begin{figure}[h]
%\caption{Deriv}
%\includegraphics[width=0.45\textwidth]{deriv_table.csv_}
%\includegraphics[width=0.45\textwidth]{slow_deriv_table.csv_}
%\label{fig:kraken_deriv}
%\end{figure}
%\label{sec:eval_deriv}
%\begin{figure}[h]
%\caption{Kraken, Deriv, absolute value and log-scale}
%\includegraphics[width=0.45\textwidth]{deriv_table.csv_}
%\includegraphics[width=0.45\textwidth]{deriv_table.csv_log}
%\label{fig:kraken_deriv}
%\end{figure}
%\label{sec:eval_derivp}
%\begin{figure}[h]
%\caption{Kraken, Deriv, absolute value and log-scale}
%\includegraphics[width=0.45\textwidth]{slow_deriv_table.csv_}
%\includegraphics[width=0.45\textwidth]{slow_deriv_table.csv_log}
%\label{fig:kraken_derivp}
%\end{figure}

%\subsection{Comparison against state-of-the-art languages}
%\label{sec:eval3}

%\begin{figure}[h]
%\caption{Kraken vs. S.o.t.A.}
%\includegraphics[width=0.45\textwidth]{cfold_table.csv_}
%\includegraphics[width=0.45\textwidth]{rbtree_table.csv_}
%\label{fig:kraken_vs_world1}
%\end{figure}

%\begin{figure}[h]
%\caption{Kraken vs. S.o.t.A.}
%\includegraphics[width=0.45\textwidth]{deriv_table.csv_}
%\includegraphics[width=0.45\textwidth]{nqueens_table.csv_}
%\label{fig:kraken_vs_world2}
%\end{figure}

% \begin{figure}[h]
% \caption{Kraken vs. S.o.t.A. (Log)}
% \includegraphics[width=0.45\textwidth]{cfold_table.csv_log}
% \includegraphics[width=0.45\textwidth]{rbtree_table.csv_log}
% \label{fig:kraken_vs_world_log_1}
% \end{figure}
% \begin{figure}[h]
% \caption{Kraken vs. S.o.t.A. (Log)}
% \includegraphics[width=0.45\textwidth]{deriv_table.csv_log}
% \includegraphics[width=0.45\textwidth]{nqueens_table.csv_log}
% \label{fig:kraken_vs_world_log_2}
% \end{figure}

%As we noted before with the Fib(30) microbenchmark in Section \ref{sec:eval1}, we remain significantly slower than state-of-the-art compiled languages.
%This is particularly true for memory-intensive benchmarks due to our naive reference-counting and malloc/free implementations.
%However, our results are of a similar order of magnitude to the difference between the state-of-the-art compiled languages and dynamic scripting languages, like Python's results in the Fib(30) microbenchmark.
%We assert that is not a fundamental limitation because the classic f-expr slowness is being eliminated, as shown by Fig. \ref{fig:kraken_vs_newlisp1} and Fig. \ref{fig:kraken_vs_newlisp2}.
%In future work, we plan to expand our compile-time analysis and optimization to implement a modified, dynamic-language version of Perceus reference counting.
%With this change, we belive \krakenSpace can be competitive with these state-of-the-art languages.

%\subsection{Case Study: Red-Black Tree}
%\label{sec:casestudy}

%\begin{figure}[h]
%\caption{Kraken vs. S.o.t.A. - RB-Tree Focus}
%\includegraphics[width=0.4\textwidth]{rbtree_table.csv_}
%\includegraphics[width=0.4\textwidth]{rbtree_table.csv_log}
%\label{fig:kraken_vs_world_rbtree}
%\end{figure}


%To evaluate our partial evaluation algorithm and compiler, we extracted the benchmarks used by the Koka language project from their code repository and added Kraken versions, as well as implementing a naive Fibonacci microbenchmark ourselves to evaluate pure function call speed.\\
%With partial evaluation and the compiler optimizations listed above, we get fairly strong performance on purely numerical computations, such as the naive Fibonacci microbenchmark.
%Unfortunately, the overhead of our unsophisticated reference counting, dynamic type checking, and bounds checking causes poor performance on benchmarks involving data structures relative to mainstream programming language implementations.
%This is not a fundamental limitation, and will be addressed in future work, as recounted in the next section.
%It should be noted, however, that while the performance relative to established language implementations is very poor for the memory-intensive benchmarks (600-900x slower), we still realize a massive speedup compared to an unoptimized and non-partial-evaluated f-expr implementation (100,000x faster)!

\section{Conclusion}\label{sec:conclusion}
In this work, we focus on addressing the fundamental challenge of OOD detection tasks, which is how to fully understand the semantic discrepancy between the ID/OOD samples. We reveal that the key to success in the realistic SCOOD task is to allocate as many ID samples in the unlabeled set correctly as possible. To this end, we propose a novel uncertainty-aware optimal transport scheme that introduces class-specific energy scores as guidance for effective label assignment. Experimental results show that our method achieves better performance than previous state-of-the-art methods on SCOOD benchmarks.

\textbf{Limitations.} In addition to temperature scaling, other techniques such as feature clipping applied in ReAct~\cite{sun2021react} also enhance the performance of energy score, so how to obtain an OOD score that best fits the SCOOD task can be further explored. Moreover, a setting highly related to SCOOD has been proposed in \cite{katz2022training} and formulated as a constrained optimization problem. We will also theoretically analyze these practical OOD settings in our feature work.

% \section*{Acknowledgments}
\textbf{Acknowledgments.} 
This work is supported by National Key R\&D Program of China under Grant 2020AAA0105701, National Natural Science Foundation of China (NSFC) under Grants 61872327, Major Special Science and Technology Project of Anhui, National Natural Science Foundation of China (62033012) and Ant Group through Ant Research Intern Program.


%% !TeX spellcheck = en_US
%\section*{Code Availability Statement}
%A MATLAB implementation of the methods and simulations presented in this paper are openly available in an open-source repository available at {\small\texttt{\url{https://fish-tue.github.io/single-origin-destination-routing}}}.


\section*{Acknowledgment}
We thank Dr.\ I.\ New and F.\ Paparella for proofreading the~paper.



%\section*{Acknowledgements}
%
\bibliographystyle{IEEEtran}
\vspace{-10pt}
\bibliography{output.bib}
\appendices
\newpage

\section{Proofs for Case $2$ of the D-ESM}
\label{sec:proofsESSG}
\vspace{-0.05in}
For the consumers in $\Sigma_1$, we need to show that  $\upsilon^d_{\vartheta}(\mathbf{p})<\upsilon^n_{ \vartheta}(\mathbf{p})$, $\forall \mathbf{p}$ and $\forall \vartheta \in \Sigma_1$. Assume a consumer type $\vartheta\in \Sigma_{1}$ and that her allocated RES energy is $E'$. Then, we have that $\upsilon^d_{ \vartheta}(\mathbf{p})= E'  \cdot c^{RES}+(E_{\vartheta}-E')\cdot \gamma \cdot  c^{RES}$ and 
$\upsilon^n_{ \vartheta}(\mathbf{p})= \varepsilon_\vartheta \cdot E_{\vartheta} \cdot \beta \cdot c^{RES}$. The inequality $\upsilon^d_{\vartheta}(\mathbf{p})<\upsilon^n_{ \vartheta}(\mathbf{p})$ is then equivalent to $ E'  (1-\gamma) \cdot c^{RES} <  E_\vartheta \cdot (\varepsilon_\vartheta \cdot \beta -\gamma) \cdot c^{RES}$, which is true by assumption, since $(1-\gamma)<0$ and $(\varepsilon_\vartheta \cdot \beta -\gamma)>0$.

                                     
  Next, for the consumers in $\Sigma_{2,1}$, we need to show that $\upsilon^d_{ \vartheta}(\mathbf{p})>\upsilon^n_{ \vartheta}(\mathbf{p})$, $\forall \mathbf{p}$ and $\forall \vartheta\in \Sigma_{2,1}$. Assume a consumer type $\vartheta \in \Sigma_{2,1}$ and that her allocated RES energy is $E'$. Then, the inequality $\upsilon^d_{\vartheta}(\mathbf{p})>\upsilon^n_{ \vartheta}(\mathbf{p})$ is equivalent to the inequality $E_\vartheta >E' \frac{(\gamma-1)}{(\gamma-\varepsilon_\vartheta\beta)}$, which is true by assumption, since $E'<\mathcal{RE}$.

Now, we prove the condition of existence of a mixed strategies NE for the consumers in $\Sigma_{2,2}$. Recall that in the ESG under the PA policy, a mixed strategy NE, $\mathbf{p^{NE}}$, among consumers in $\Sigma_{2,2}$ exists under the condition
\vspace{-0.05in}

\small
\begin{equation}\label{eq:condition_PA_NE_2}
res_{\vartheta}^{PA}(\mathbf{p}^{NE}) =res_{\vartheta}^{NE}(\mathbf{p}^{NE}), \forall \vartheta \in \Sigma_{2,2}. \end{equation}
\normalsize
%Next we give the conditions such that either \eqref{eq:condition_PA_NE} holds and mixed NE exist or there exist dominant strategies. For this study, we distinguish cases with respect to the RES capacity, the risk aversion degree values and the daytime energy demand levels. 

To derive condition \eqref{eq:relation_E_0_E_1_pa_ne_extra_demand} we re-write \eqref{eq:condition_PA_NE} first with assuming that a consumer $i$ of type $\vartheta_i \in \Sigma_{2,2}$ plays the pure strategy $A_i=d$ (in \eqref{eq:probrelation1}) and second with assuming that a consumer $j$ with type $\vartheta_j \in \Sigma_{2,2} \setminus \{\vartheta_i\}$ plays the pure strategy $A_j=d$ (in \eqref{eq:probrelation2}):

\vspace{-0.1in}
\begin{small}
\begin{align}
&  \mathcal{RE}\frac{(\gamma-1)}{(\gamma-\varepsilon_{\vartheta_i}\beta)}-E_{\vartheta_i}= D^{Total}_{\Sigma_1}+\sum_{ {\vartheta'}\in \Sigma_{2,2}} r_{\vartheta'}~ (N-1)~E_{\vartheta'}~p^{d,NE}_{\vartheta'},
    \label{eq:probrelation1}\\
  &  \mathcal{RE}\frac{(\gamma-1)}{(\gamma-\varepsilon_{\vartheta_j}\beta)}-E_{\vartheta_j}=D^{Total}_{\Sigma_1}+  \sum_{ {\vartheta'}\in \Sigma_{2,2}} r_{\vartheta'} ~(N-1)~E_{\vartheta'}~ p^{d,NE}_{\vartheta'}.
    \label{eq:probrelation2}  
\end{align}
\end{small}

%Eq. \eqref{eq:condition_PA_NE} can be re-written in a similar way for any other type $\vartheta_j \in \Theta$. 
Note that to derive (\ref{eq:probrelation1}) we consider that if a consumer $i$ in $\Sigma_{2,2}$ of type $\vartheta_i$ plays  the pure strategy $A_i=d$, then, the aggregate expected daytime energy of the consumers in $\Sigma_{2,2}$, $D_{\Sigma_{2,2}}(\mathbf{p^{NE}})$ can be expressed as $E_{\vartheta_i}+ \sum_{ {\vartheta'}\in \Sigma_{2,2}} r_{\vartheta'}~ (N-1)~E_{\vartheta'}~p^{d,NE}_{\vartheta'}$ for a large number of consumers and similarly also for (\ref{eq:probrelation2}). Then, since the right-hand sides of \eqref{eq:probrelation1}-\eqref{eq:probrelation2} are equal, the left-hand sides will be also equal and \eqref{eq:relation_E_0_E_1_pa_ne_extra_demand} derives. %\eqref{eq:probrelation1}-\eqref{eq:probrelation2} 

To derive the probability bounds, we re-write \eqref{eq:condition_PA_NE} assuming that all consumers of the same type play the same mixed strategy, i.e., 

\vspace{-0.1in}
\begin{small}
\begin{align}
&  \mathcal{RE}\frac{(\gamma-1)}{(\gamma-\varepsilon_{\vartheta_i}\beta)}= D^{Total}_{\Sigma_1}+N\sum_{ {\vartheta'}\in \Sigma_{2,2}} r_{\vartheta'}~ E_{\vartheta'}~p^{d,NE}_{\vartheta'}.
    \label{eq:probrelation3}  
\end{align}
\end{small}

The minimum bound on the probability for competing for RESs, $p_{\vartheta}^{\min}$, derives by setting in (\ref{eq:probrelation3}) $p^{d,NE}_{\tilde{\vartheta}}=1$, $\forall \tilde{\vartheta}\in \Sigma_{2,2}$ with $\tilde{\vartheta}\neq \vartheta=\vartheta_i$. Similarly, the maximum bound on the probability for competing for RESs, $p_{\vartheta}^{\max}$, derives by setting in (\ref{eq:probrelation3}) $p^{d,NE}_{\tilde{\vartheta}}=0$, $\forall \tilde{\vartheta}\in \Sigma_{2,2}$ with $\tilde{\vartheta}\neq \vartheta=\vartheta_i$. 

Finally, the expression for the aggregate expected daytime energy demand given in \eqref{eq:demand1_2c} is constructed as follows. First we can write that 
\begin{align}
 D^{d,NE} = D^{Total}_{\Sigma_1} +N\sum_{ {\vartheta'}\in \Sigma_{2,2}} r_{\vartheta'}~E_{\vartheta'}~p^{d,NE}_{\vartheta'}. \label{eq:totdemand}
 \end{align}

Second, by multiplying \eqref{eq:probrelation1} with $\frac{N}{N-1}$, we obtain:


\begin{small}
\begin{align}
&  N\sum_{ {\vartheta'}\in \Sigma_{2,2}} r_{\vartheta'}~E_{\vartheta'}~p^{d,NE}_{\vartheta'}=\frac{N}{N-1}\left[\frac{\mathcal{RE}(\gamma-1)}{(\gamma-\varepsilon_{\vartheta_i}\beta)}-E_{\vartheta_i}-D^{Total}_{\Sigma_1}\right].
    \label{eq:probrelation3}
\end{align}
\end{small}

Third, by replacing \eqref{eq:probrelation3} in \eqref{eq:totdemand} we obtain \eqref{eq:demand1_2c}, where the $\min\{.\}, ~\max\{.\}$ operators account for the case that the initially obtained probability values by  \eqref{eq:probrelation1} do not lie in the range $[0,1]$ and should be set to the values $1$ or $0$, correspondingly. 

\section{Proofs for Case $2$ of the C-ESM}
\label{appendix:dual}

In this case, it is optimal for the C-ESM to schedule loads during the day so that the total RES capacity is fully utilized, i.e., the expected aggregate daytime energy demand is greater than or equal to the RES capacity:

 \vspace{-0.1in} 
 \small
\begin{align}
N \sum_{{\vartheta} \in \Theta}r_{{\vartheta}} ~E_{{\vartheta}}~p^{d}_{{\vartheta}} \geq \mathcal{RE}.
\label{eq:optimal}
\end{align}
\normalsize 
\vspace{-0.1in}


\noindent Therefore, the social cost reduces to:

 \vspace{-0.1in} 
 \begin{small}
\begin{align}
C(\mathbf{p}) &=  \mathcal{RE} \cdot c^{RES}   + \left[N \sum_{{\vartheta} \in \Theta} r_{\vartheta} ~p_{{\vartheta}}^{d}~  E_{\vartheta} - \mathcal{RE}\right]  \gamma \cdot c^{RES} \nonumber \\
& + N \left[ \sum_{{\vartheta} \in \Theta} r_{\vartheta } \left(1-p^{d}_{{\vartheta}}\right)\varepsilon_{{\vartheta} }~E_{{\vartheta}}\right] \beta \cdot c^{RES},
 \label{eq:social_cost_pa_extra_demand_2}
 \end{align}
\end{small} \vspace{-0.1in}

\noindent and the C-ESM optimization problem \eqref{eq:social_cost_x_opt} is equivalent to minimizing $ N \sum_{\vartheta \in \Theta} \left[ r_{\vartheta} E_{\vartheta} \left(\gamma - \varepsilon_{\vartheta} \beta \right) p^{d}_{\vartheta} \right] c^{RES}$, subject to constraints \eqref{eq:opt_2.1}-\eqref{eq:opt_3.2} and \eqref{eq:optimal}. Below, we derive closed-form expressions of the solutions of this linear optimization problem.

We define two complementary subsets of consumer types, depending on their risk aversion degrees: $\Sigma_1 = \Bigl\{\vartheta \in \Theta : \varepsilon_{\vartheta} \geq \gamma/\beta \Bigr\} \subset \Theta$, and $\Sigma_2 = \Bigl\{ \vartheta \in \Theta :1\leq \varepsilon_{\vartheta} < \gamma/\beta \Bigr\} \subset \Theta$.

For all consumers whose type $\vartheta \in \Sigma_1$, it is optimal for the C-ESM to schedule them during daytime, such that $p^{d,*}_{\vartheta}=1 $. Therefore, the optimal schedule for the remaining consumers whose type $\vartheta \in \Sigma_2$ can be found by solving the following linear optimization problem: 


%\vspace{-0.1in} 
 \begin{small}
 \begin{subequations} \label{eq:social_cost_x_opt_2}
\begin{alignat}{2}
& \min_{\mathbf{p}} \ &&  N \sum_{\vartheta \in \Sigma_2} \left[ r_{\vartheta} ~E_{\vartheta} \left(\gamma - \varepsilon_{\vartheta} \beta \right) p^{d}_{\vartheta} \right] c^{RES} \label{eq:opt_S2_1} \\
 & \text{s.t. } && \eqref{eq:opt_2.1}-\eqref{eq:opt_3.2} \label{eq:opt_S2_2}\\
 & \quad && N \sum_{{\vartheta} \in \Sigma_2}r_{{\vartheta}} ~E_{{\vartheta}}~p^{d}_{{\vartheta}} \geq \left( \mathcal{RE} - N \sum_{{\vartheta} \in \Sigma_1}r_{{\vartheta}} E_{{\vartheta}}\right). \label{eq:opt_S2_3} 
 \end{alignat}
 \end{subequations}
\end{small} %\vspace{-0.1in}

\noindent And the dual function of this optimization problem is 

\begin{footnotesize}
 \begin{align} \label{eq:social_cost_x_opt_2_dual}
 \max_{\lambda \geq 0}\min_{\mathbf{p}} \quad & N \sum_{\vartheta \in \Sigma_2} \left[ r_{\vartheta} E_{\vartheta} \left(\gamma - \varepsilon_{\vartheta} \beta \right) p^{d}_{\vartheta} \right] c^{RES} \nonumber \\&-\lambda\left( N \sum_{{\vartheta} \in \Sigma_2}r_{{\vartheta}} E_{{\vartheta}}p^{d}_{{\vartheta}} - \left( \mathcal{RE} - N \sum_{{\vartheta} \in \Sigma_1}r_{{\vartheta}} E_{{\vartheta}}\right)\right),
 \end{align}
\end{footnotesize} \vspace{-0.1in}

\hspace{-0.2in} subject to \eqref{eq:opt_S2_2}, where $\lambda$ represents the dual variable associated with \eqref{eq:opt_S2_3} and let $\lambda^*$ represent its optimal value.

It results that:\\
$\bullet$ for all $\vartheta \in \Sigma_{2}$ where $1 \leq \varepsilon_\vartheta < \dfrac{\gamma c^{RES} - \lambda^*}{\beta c^{RES}}$, $p^{d,*}_{\vartheta}=0$,\\
$\bullet$ for all $\vartheta \in \Sigma_2$ where $ \varepsilon_\vartheta = \dfrac{\gamma c^{RES} - \lambda^*}{\beta c^{RES}}$, $0<p^{d,*}_{\vartheta}<1$,\\
$\bullet$  for all $\vartheta \in \Sigma_2$ where $ \dfrac{\gamma c^{RES} - \lambda^*}{\beta c^{RES}} < \varepsilon_\vartheta <\dfrac{\gamma}{\beta}$, $p^{d,*}_{\vartheta}=1$.

This means that the consumer types are fully dispatched during the day in the order of increasing risk aversion degree (or decreasing $\varepsilon_\vartheta$), until constraint \eqref{eq:opt_S2_3} is satisfied. 


\section{Analysis For the ES Allocation Policy}
\subsection{Decentralized Energy Sharing Mechanism Under ES}
The analysis and proofs of this section follow similar lines as the analysis and proofs for the PA policy. Most proofs are however omitted for brevity.

In the ESG with the ES policy, a mixed-strategy NE exists under the condition:


 %\vspace{-0.1in} 
 \small
\begin{equation}\label{eq:condition_ES_NE}
rse^{ES}_{\vartheta_i}(\mathbf{p^{NE}}) =res_{\vartheta}^{NE}(\mathbf{p}^{NE}), ~\forall \vartheta \in \Theta. \end{equation}
\normalsize 
\vspace{-0.1in}  

\noindent 
%Therefore, for all cases, any existing mixed-strategy NE competing probabilities, $\mathbf{p}^{NE}$, are obtained by resolving condition \eqref{eq:condition_ES_NE}.
Let us distinguish the following cases:

\subsubsection*{\textbf{Case $1$: $\bm{\mathcal{RE}}$ exceeds $\bm{D^{Total}}$}}

As consumers have knowledge of $\mathcal{RE}$ and $D^{Total}$, it is straightforward to show that the dominant-strategy for all consumers is to schedule their daily flexible loads during daytime. As a result, the competing probabilities that lead to equilibrium states are equal to $p_{\vartheta}^{d,NE} = 1$ for all consumer types $\vartheta \in \Theta$.


\subsubsection*{\textbf{Case $2$: $\bm{\mathcal{RE}}$ is lower than $\bm{D^{Total}}$}} 

In this case, the strategies of the consumers depend on their respective risk aversion degrees and the TOU tariffs. We define two complementary subsets of consumer types, depending on their risk aversion degrees: $\Sigma_1 = \Bigl\{\vartheta \in \Theta : \varepsilon_{\vartheta} \geq \gamma/\beta \Bigr\} \subset \Theta$, and $\Sigma_2 = \Bigl\{ \vartheta \in \Theta :1\leq \varepsilon_{\vartheta} < \gamma/\beta \Bigr\} \subset \Theta$.


Firstly, the dominant strategy for all consumers $i$ whose type $\vartheta_i$ is in the set $\Sigma_1$ is to schedule their daily flexible loads during daytime, i.e., to play the pure strategy $A_i = d$ with probability $p_{\vartheta_i}^{d,NE} = 1$. Their expected aggregate daytime energy demand is then $D^{Total}_{\Sigma_1}=N\sum_{\theta \in \Sigma_1}r_{\theta} E_{\theta}$.

Secondly, the strategies of the consumers $i$ whose type $\vartheta_i$ is in the set $\Sigma_2$ depends on their daily flexible loads and risk-aversion degrees. Therefore, we define two distinct subsets of consumer types in $\Sigma_2$: $\Sigma_{2,1} = \left\{ \vartheta \in \Sigma_2 : E_{\vartheta} > \mathcal{RE}\frac{(\gamma-1)}{(\gamma-\varepsilon_{\vartheta}\beta)} \right\}$ and $\Sigma_{2,2} = \left\{ \vartheta \in \Sigma_2 : E_{\vartheta} \leq \mathcal{RE}\frac{(\gamma-1)}{(\gamma-\varepsilon_{\vartheta}\beta)}\right\}$.

%$\Sigma_{2,1} = \left\{ \vartheta \in \Sigma_2 : E_{\vartheta} > \left(\mathcal{RE}-D^{Total}_{\Sigma_1}\right)\frac{(\gamma-1)}{(\gamma-\varepsilon_{\vartheta}\beta)} \right\}$ and $\Sigma_{2,2} = \left\{ \vartheta \in \Sigma_2 : E_{\vartheta} \leq \left(\mathcal{RE}-D^{Total}_{\Sigma_1}\right)\frac{(\gamma-1)}{(\gamma-\varepsilon_{\vartheta}\beta)}\right\}$

For consumers $i$ whose type $\vartheta_i$ is in the set $\Sigma_{2,1}$, the dominant strategy is to schedule their daily flexible loads during nighttime, i.e., to play the pure strategy $A_i=n$ with probability $p^{n,NE}_{\vartheta_i}=1$, and $A_i=d$ with probability $p^{d,NE}_{\vartheta_i}=0$.

For consumers whose types are in the set $\Sigma_{2,2}$, a mixed-strategy NE with the ES policy exists if and only if the following condition holds:

 \vspace{-0.1in} 
 \small
\begin{equation}\label{eq:relation_E_0_E_1_es_ne_extra_demand}
(\gamma-\varepsilon_{\vartheta}\beta)\cdot E_{\vartheta} = (\gamma-\varepsilon_{\tilde{\vartheta} }\beta)\cdot E_{\tilde{\vartheta}} , \ \forall \vartheta , \tilde{\vartheta} \in \Sigma_{2,2}.
\end{equation}
\normalsize 


To derive condition \eqref{eq:relation_E_0_E_1_es_ne_extra_demand} we re-write \eqref{eq:condition_ES_NE} first with assuming that a consumer $i$ of type $\vartheta_i \in \Sigma_{2,2}$ plays the strategy $A_i=d$ with probability $p^{d,NE}_{\vartheta_i}=1$ (in \eqref{eq:probrelation1es}) and second with assuming that a consumer $j$ with type $\vartheta_j \in \Sigma_{2,2} \setminus \{\vartheta_i\}$ plays the strategy $A_j=d$ with probability $p^{d,NE}_{\vartheta_j}=1$ (in \eqref{eq:probrelation2es}).

\vspace{-0.1in}
\begin{small}
\begin{align} \label{eq:probrelation1es}
  D^{Total}_{\Sigma_1}+1+\sum_{ {\vartheta'}\in \Sigma_{2,2}} r_{\vartheta'}~ (N-1)~p^{d,NE}_{\vartheta'}=\frac{\mathcal{RE}(\gamma-1)}{E_{\vartheta_i}(\gamma-\varepsilon_{\vartheta_i}\beta)},
\end{align}
\end{small}
\vspace{-0.1in}

\begin{small}
\begin{align} \label{eq:probrelation2es}
  D^{Total}_{\Sigma_1}+1+\sum_{ {\vartheta'}\in \Sigma_{2,2}} r_{\vartheta'}~ (N-1)~p^{d,NE}_{\vartheta'}=\frac{\mathcal{RE}(\gamma-1)}{E_{\vartheta_j}(\gamma-\varepsilon_{\vartheta_j}\beta)}.
\end{align}
\end{small}
Then, since the right-hand sides of \eqref{eq:probrelation1es}-\eqref{eq:probrelation2es} are equal, the left-hand sides will be also equal and \eqref{eq:relation_E_0_E_1_es_ne_extra_demand} derives.


Additionally, for the consumers of type $\vartheta \in \Sigma_{2,2}$, the competing probabilities that lead to NE states lie in the range $p^{min}_{\vartheta} \leq p^{d,NE}_{{\vartheta}} \leq p^{max}_{\vartheta}$, where:

 \vspace{-0.1in} 
 \footnotesize
\begin{align}
&  p^{min}_{\vartheta}=\nonumber\\&\max \left\{0,\frac{\frac{\mathcal{RE}(\gamma-1)}{E_{\vartheta}(\gamma-\varepsilon_{\vartheta}\beta)}-
  \sum\limits_{\tilde{\vartheta} \in \Sigma_{2,2} \cup \Sigma_1 \setminus \{\vartheta\}}N r_{\tilde{\vartheta}} }{N r_{\vartheta}} \right\}, \label{eq:plminbound_appendix}\\
& p^{max}_{\vartheta} = \min \left\{1,\frac{\frac{\mathcal{RE}(\gamma-1)}{E_{\vartheta}(\gamma-\varepsilon_{\vartheta}\beta)}-\sum\limits_{\tilde{\vartheta} \in  \Sigma_1 }N r_{\tilde{\vartheta}} }{N r_{\vartheta}}\right\}.
    \label{eq:plmaxbound_appendix}
\end{align}
\normalsize  
\vspace{-0.1in}

To derive the probability bounds, we re-write \eqref{eq:condition_ES_NE} assuming that all consumers of the same type play the same mixed strategy, i.e., 

\vspace{-0.1in}
\begin{small}
\begin{align} \label{eq:probrelation3es}
  D^{Total}_{\Sigma_1}+\sum_{ {\vartheta'}\in \Sigma_{2,2}} N~r_{\vartheta'}~p^{d,NE}_{\vartheta'}=\frac{\mathcal{RE}(\gamma-1)}{E_{\vartheta_i}(\gamma-\varepsilon_{\vartheta_i}\beta)}.
\end{align}
\end{small}

The minimum bound on the probability for playing RES, $p_{\vartheta}^{\min}$, derives by setting in (\ref{eq:probrelation3es}) $p^{d,NE}_{\tilde{\vartheta}}=1$, $\forall \tilde{\vartheta}\in \Sigma_{2,2}$ with $\tilde{\vartheta}\neq \vartheta=\vartheta_i$. Similarly, the maximum bound on the probability for playing RES, $p_{\vartheta}^{\max}$, derives by setting in (\ref{eq:probrelation3es}) $p^{d,NE}_{\tilde{\vartheta}}=0$, $\forall \tilde{\vartheta}\in \Sigma_{2,2}$ with $\tilde{\vartheta}\neq \vartheta=\vartheta_i$. 

The Remarks 3 and 4, which are stated for the PA allocation policy in Section \ref{sec:gameanalysis}, also hold in case of the ES allocation policy. 

The social cost under the ES policy can be expressed as 

\footnotesize
\begin{align}
&C^{ES}(\mathbf{p^{NE}}) =  N \sum_{\vartheta \in \Theta} r_{\vartheta}~ \min\{sh(\mathbf{p^{NE}}), E_{\vartheta}\} ~p_{\vartheta}^{d,NE}~ c^{RES} \nonumber\\ + &\left[D(\mathbf{p^{NE}}) -N \sum_{\vartheta \in \Theta} r_{\vartheta} ~\min\{sh(\mathbf{p^{NE}}), E_{\vartheta}\} ~p_{ \vartheta}^{d,NE}\right]
 ~c^{grid,d}\nonumber\\ + &
 N \left[ \sum_{\vartheta \in \Theta} r_{\vartheta }~ p^{n,NE}_{\vartheta}~ \varepsilon_{\vartheta }~E_{\vartheta}\right] c^{grid,n}.
 \label{eq:social_cost_es_sc}
 \end{align}
\normalsize

\subsection{Centralized Energy Sharing Mechanism Under ES Policy}
Similar to C-ESM under the PA policy (Section \ref{sec:coordinated}), the C-ESM under the ES policy is modeled as an optimization problem, defined as:

 \vspace{-0.1in} 
 \begin{small}
 \begin{subequations} \label{eq:social_cost_x_opt_es}
\begin{alignat}{2}
& \min_{\mathbf{p}} \ && C^{ES}(\mathbf{p}) \label{eq:opt_1_es} \\
 & \text{s.t. } &&p^{d}_{\vartheta},~ p^{n}_{\vartheta}\geq 0  ,  \ \forall \vartheta \in \Theta \label{eq:opt_2_es} \\
 & \quad && p^{d}_{\vartheta} + p^{n}_{\vartheta} = 1,  \ \forall \vartheta \in \Theta. \label{eq:opt_4_es}
 \end{alignat}
 \end{subequations}
\end{small} 

The problem \eqref{eq:social_cost_x_opt_es} is non-convex due to its objective function and the form of the equal share $sh(\mathbf{p^{NE}})$ (Eq. \eqref{eq:fairshare}). In our simulations in Section \ref{sec:comptoES}, we solve it with genetic algorithms using the Global Optimization Toolbox of MATLAB. 

\clearpage


%APPENDICES are optional
%\balancecolumns
%\appendices
%Appendix A
%% \section{Analogies and disanalogies to nuclear}
% \copied{1. main thing to highlight earlier when linking is that it’s debated by scholars why nuclear verification succeeded, and the similarities vs. differences with AI training will determine whether AI will succeed for a similar reason. We give arguments why it may be easier, and why it may be harder.
% 2. Analogies
%     1. Both involve using a flow (centrifuges/chips) to  aggregate a stock (total training time)
%     2. Small amounts are fine, large amounts are bad
%     3. Positive economic use-cases that everyone should benefit from, negative misuse use-cases that we should limit to the extent possible.
% 3. Disanalogies
%     1. With uranium, after the enrichment occurs, you can still track the physically-produced uranium. (I.e. violations are reversible.) With compute, once a model has been trained, it can be copied at will.
%     2. With nuclear, most countries that have pursued their own nuclear program have eventually been able to discretely build their own centrifuges (e.g. with design-support from AQ Khan). With advanced compute, this seems very unlikely - fabs cost many billions, and are rarely spun up for a single purpose. (Even the US military failed at this.) Thus, compute supply chain is much more concentrated - less possible to “build an AI project in a bunker somewhere” detached from existing supply chains.
%     3. With HEU, inspecting the end-product is sufficient to know how enrichment was done. With compute, it’s hard to determine “how much an NN was trained” without being provided additional information on the process by which we arrived at those weights.
%     4. With HEU, we know ahead of time how much enrichment is sufficient to be dangerous (as there are physical requirements to causing a supercritical fission  reaction). However with AI, algorithmic progress means that the threshold for any particular dangerous use-case decreases over time. That said, “more compute” will always be riskier than “less compute”, and thus a useful heuristic for regulation. Even if relevant thresholds need to change over time, so long as the requirements do not shrink to the point where detection would be impossible, it is still just as important to have a governing framework for compute.
%     5. With HEU, enrichment must happen (mostly) at the same location. With AI training, enrichment can be parallelized over the internet, although there is some critical concentration required per location.
%     6. AI is much more civilian-economically valuable than uranium, and implemented with more parts of the supply chain. Many analysts believe that part of the success of the NPT is that states do not have a strong incentive to cheat as they’re part of security alliances; it is not clear whether this will be the same for AI. It may be that similar alliances are required.
% 4. Inspection mechanisms from nuclear we’d like to copy
%     1. Tracking centrifuge production
%     2. Centrifuge flow monitoring
%     3. Accounting for total usage of centrifuges (i.e. being able to show results)
% }

% \section{Case study in a way this could break: student-teacher}
% \ot{One way to break this scheme: rather than one single long training run, to evade detection, the model can be repeatedly self-distilled into a new model, and then that new model trained for an additional period. This increases a compute-overhead, but so long as the distilling time does not scale linearly with the training time, this can make it possible to hide longer runs as a series of shorter runs. However, the total compute required (and the total time that the in-RAM model is of sufficiently-low-loss to trigger an audit) is still large, meaning that a randomly-sampled chip would still need to attest to being part of a dangerous training run.
% Note also that this behavior would look kind of weird, because each self-distillation run would be using a very large number of chips *in parallel* for a short time. This is because such self-distillation-chains must occur sequentially across time, and at each timestep all the resources go into a particular snapshot.}

\newpage

\section{Discussion on future training requirements}\label{app.howmuch}

\subsection{Will the most capable ML models require large-scale training?}
This paper's proposed framework is premised on the assumption that large-scale training is and continues to be a necessary requirement for the most advanced (and thus most dangerous) ML models.
There is intense disagreement within the field about how important large-scale training is, and how long that will remain the case.

Many of the recent breakthroughs in machine learning model capabilities, across every domain, have come from increasing the model size or quantity of training data, each of which corresponds to a greater usage of compute \cite{kaplan2020scaling, hoffman2022training, zhai2021scaling}.
Indeed, some capabilities, such as chain-of-thought reasoning, appear to only emerge at the largest training scales \cite{wei2022chain}.
At the same time, any one narrow capability can often be achieved with a much smaller compute budget \cite{magister2022teaching, madani2023large}.
Nonetheless, Sutton's ``Bitter Lesson'' \cite{sutton2019bitter} that ``general methods that leverage computation are ultimately the most effective'' is a frequent diagnosis of the likely future of deep learning.
Though algorithmic progress \cite{erdil2022algorithmic} and the continued progress of Moore's Law will continue to reduce the number of chips required for any specific capability, we may compensate by gradually increasing enforcement parameters to work for smaller quantities of specialized compute.
At the same time, the increasing investment in compute by frontier AI firms \cite{lardinois_2022, wiggers_2022} suggests that industry insiders continue to believe that the most capable frontier models --- likeliest to yield new capabilities and surface new risks to public safety --- are expected to require ever more compute.

\subsection{Will large-scale training continue to require specialized datacenter chips?}

Nearly all large-scale training runs are executed on high-end datacenter accelerators \cite{chowdhery2022palm, kaplan2020scaling, zeng2022glm}.
The main difference between these chips and their consumer-oriented counterparts is their much higher inter-chip communication bandwidth (e.g., 900GB/s for the NVIDIA H100 SXM vs. 64GB/s for the NVIDIA GeForce RTX 4090 \cite{nvidiah100, nvidia4090}).
This extra bandwidth is today crucial for parallelizing NN training, especially tensor parallelism and data parallelism, which require frequent transfers of large matrices between many chips \cite{smith2022computation}.
Organizations doing large-scale training also favor these datacenter chips for other reasons: they are generally more energy efficient, and license requirements often prevent organizations from placing consumer-oriented chips in datacenters\cite{moss_2023}.

Still, recent work has suggested it may be \emph{possible} to do large-scale training on consumer chips with low interconnect, though with substantial cost and speed penalties\cite{binhang2022distributed, ryabinin2023distributed}.
If such methods become feasible for bad actors, then we may need to adjust to a different regulatory model for detecting training activity.
Possibilities include focusing on spotting and monitoring datacenters (similar to the IAEA's work to detect undeclared nuclear facilities \cite{harry1996iaea}), or regulating the high-capacity switches that could be necessary to enable fast networking between low-interconnect chips.
So long as they can be detected, it may be possible to retrofit consumer chips (e.g. with a permanently-mated host CPU, see Section \ref{s.onchip}) to enable similar monitoring capabilities.

It is important to note that the current framework \emph{does} apply in the setting where clusters of chips are split across several datacenters (e.g. multiple cloud providers), so long as these high-end chips are used at each datacenter.

%\section{Headings in Appendices}
%\balancecolumns
% That's all folks!

\end{document}




\end{document}



