

\newtheorem{theorem}{Theorem}[section]
\newtheorem{lemma}[theorem]{Lemma}
\newtheorem{proposition}[theorem]{Proposition}
\newtheorem{corollary}[theorem]{Corollary}

\theoremstyle{remark}
\newtheorem*{convention}{Convention}%[section]
\newtheorem{definition}{Definition}[section]
\newtheorem{problem}{Problem}
\newtheorem{example}{Example}[section]
\newtheorem{remark}{Remark}[section]
\newtheorem{exercise}{Exercise}[section]

%---------------------------------Arrows-------------------------------

\newcommand\Rarrow{\Rightarrow}
\newcommand\Larrow{\Leftarrow}
\newcommand\Iff{\Leftrightarrow}
\newcommand\rarrow{\rightarrow}
\newcommand\larrow{\leftarrow}
\renewcommand\iff{\longleftrightarrow}
\newcommand\fun{\ensuremath{\longrightarrow}}

\newcommand\inj{\stackrel{1-1}{\longrightarrow}}
\newcommand\surj{\stackrel{\text{na}}{\longrightarrow}}
\newcommand\bij{\rightleftarrows}

%---------------------------------Sets of numbers-------------------------------

\newcommand\Real{\varmathbb{R}} %requires txfonts or an external file
\newcommand\Nat{\varmathbb{N}}
\newcommand\Rat{\varmathbb{Q}}
\newcommand\Irr{\varmathbb{I}}
\newcommand\Intgr{\varmathbb{Z}}

\DeclareMathOperator{\Fin}{Fin}


\newcommand{\Even}{\varmathbb{E}} %the set of even numbers
\newcommand{\Odd}{\varmathbb{O}} %the set of odd numbers
\DeclareMathOperator{\lcm}{lcm}  %the least common factor

%---------------------------------Power set-------------------------------

\newcommand{\power}{\mathcal{P}} %power set
\newcommand{\powerne}{\power_{\!+}} %power set without the empty set


%---------------------------------Definitions-------------------------------

%\newcommand\iffdef{\stackrel{\rm df}{\longleftrightarrow}}
\newcommand\iffdef{\;\mathrel{\mathord{:}\mathord{\longleftrightarrow}}\;}
\newcommand\defeq{\coloneqq} %requires mathtools package
\newcommand\eqdef{\eqqcolon}

%---------------------------------Algebras-------------------------------

\newcommand{\zero}{\mathbf{0}} %zero of an algebra
\newcommand{\one}{\mathbf{1}} %unity of an algebra

\newcommand{\ingr}{\sqsubseteq} %part of relation
\newcommand{\ningr}{\not\sqsubseteq} %the complement of part of
\newcommand{\partof}{\sqsubset} %proper part
\newcommand{\overl}{\mathrel{\bigcirc}} %overlap relation
\newcommand{\ext}{\mathrel{\bot}} %incompatibility relation

\newcommand{\Fus}{\mathrel{\mathsf{Fus}}} %mereological fusion relation
\newcommand{\Sum}{\mathrel{\mathsf{Sum}}} %mereological sum relation

\newcommand{\fil}{\mathscr{F}} % a filter
\newcommand{\ult}{\mathscr{U}} % an ultrafilter
\newcommand{\ide}{\mathscr{I}} %an ideal

\DeclareMathOperator{\Ult}{Ult} %all ultrafilters of the algebra
\DeclareMathOperator{\Fil}{Fil} %all filters of the algebra
\DeclareMathOperator{\PFil}{PFil} %all principal filters of the algebra
\DeclareMathOperator{\Gr}{Gr} %all ideals of the algebra
\DeclareMathOperator{\Ide}{Id} %all ideals of the algebra
%\newcommand\Rel{\mathrel{R}} %binary relation
\DeclareMathOperator{\Hom}{Hom} %the collection of all morphisms

\DeclareMathOperator{\CO}{CO} %clopen algebra
\DeclareMathOperator{\RO}{RO} %regular open
\DeclareMathOperator{\RC}{RC} %regular closed

\DeclareMathOperator{\Rego}{R_o} %open regularization
\DeclareMathOperator{\Regc}{R_c} %closed regularization

\newcommand\BA{\boldsymbol{\mathsf{BA}}} %the class of Boolean algebras
\newcommand\BAc{\boldsymbol{\mathsf{BA^c}}} %the class of Boolean contact algebras
\newcommand\BCA{\boldsymbol{\mathsf{BCA}}} %the class of Boolean contact algebras
\newcommand\BCAc{\boldsymbol{\mathsf{BCA^c}}} %the class of complete Boolean contact algebras
\newcommand\BWCA{\boldsymbol{\mathsf{BWCA}}} %the class of Boolean contact algebras
\newcommand\RCA{\boldsymbol{\mathsf{RCA}}} %the class of resolution contact algebras
\newcommand\RCAc{\boldsymbol{\mathsf{RCA^c}}} %the class of complete resolution contact algebras
\newcommand\KTB{\boldsymbol{\mathsf{KTB}}} %the class of KTB modal logics
\newcommand\KTBc{\boldsymbol{\mathsf{KTB^c}}} %the class of complete KTB modal algebras
\newcommand\Cat{\boldsymbol{\mathsf{C}}} %a category

\DeclareMathOperator{\con}{\mathsf{C}} %relacja stycznoœci
\DeclareMathOperator{\conk}{\mathsf{K}}
\DeclareMathOperator{\cond}{\mathsf{D}}
\DeclareMathOperator{\Overl}{\mathsf{O}} %the set of all regions that overlap given object


%\usepackage{scalerel,stackengine} is needed for the commands below

%connection relation for filters
\def\twocircs{\mathord{\circ}\!\mathord{\circ}}
\newcommand{\conf}{\mathbin{\twocircs}}
\def\nconf{%
\renewcommand\stacktype{L}\mathbin{\ensurestackMath{%
  \ThisStyle{\stackon[0pt]{\SavedStyle\dwakola}{\SavedStyle/}}}}}
\newcommand{\grill}{\mathscr{G}}

\newcommand\mathbackslash{\raisebox{.4pt}{\texttt{/}}}
\def\notcon{% the separation relation
  \renewcommand\stacktype{L}\mathrel{\ensurestackMath{%
  \ThisStyle{\stackon[0pt]{\SavedStyle\con}{\SavedStyle\mathbackslash}}}}%
}
\let\separ=\notcon

\def\notconk{% the separation relation
  \renewcommand\stacktype{L}\mathrel{\ensurestackMath{%
  \ThisStyle{\stackon[0pt]{\SavedStyle\conk}{\SavedStyle\mathbackslash}}}}%
}

\def\notcond{% the separation relation
  \renewcommand\stacktype{L}\mathrel{\ensurestackMath{%
  \ThisStyle{\stackon[0pt]{\SavedStyle\cond}{\SavedStyle\mathbackslash}}}}%
}

\DeclareMathOperator{\upop}{\uparrow} %upper arrow operator
\DeclareMathOperator{\downop}{\downarrow} %lower arrow operator
\newcommand\twodownop{\raisebox{-1pt}{$\mathop{\rotatebox{90}{$\twoheadleftarrow$}}$}} %downward << closure operation
\newcommand\twoupop{\raisebox{-1pt}{$\mathop{\rotatebox{90}{$\twoheadrightarrow$}}$}} %upward << closure operation

%%%%%%%%%%%%%%%%%%%%%%%%%%%%%%%%%%%%%%%%%%%%%


%---------------------------------Theories-------------------------------
\DeclareMathOperator{\Th}{Th} %theory of a model
\newcommand\PA{\mathsf{PA}} %first order Peano Arithmetic
\newcommand{\num}{\overline} %the numeral for a number
\newcommand{\godel}[1]{\ulcorner #1\urcorner} %godel number of an expression
\newcommand{\forces}{\Vdash} %forcing relation
\newcommand{\nforces}{\nVdash} %complement of forcing relation


%---------------------------------Topology-------------------------------

\DeclareMathOperator{\Cl}{Cl}
\DeclareMathOperator{\Int}{Int}
\newcommand\cl[1]{\overline{#1}}

\newcommand{\Basis}{\mathscr{B}} %basis for topology
\newcommand{\topo}{\mathscr{O}} %family of open sets
\newcommand{\topone}{\mathscr{O}^+} %family of non-empty open sets
\newcommand{\topoc}{\mathscr{C}} %family of closed sets

%---------------------------------Fonts---------------------------------------
\newcommand\calA{\mathcal{A}}
\newcommand\calB{\mathcal{B}}
\newcommand\calC{\mathcal{C}}
\newcommand\calD{\mathcal{D}}
\newcommand\calE{\mathcal{E}}
\newcommand\calF{\mathcal{F}}
\newcommand\calG{\mathcal{G}}
\newcommand\calH{\mathcal{H}}
\newcommand\calK{\mathcal{K}}
\newcommand\calL{\mathcal{L}}
\newcommand\calM{\mathcal{M}}
\newcommand\calN{\mathcal{N}}
\newcommand\calR{\mathcal{R}}
\newcommand\calS{\mathcal{S}}
\newcommand\calT{\mathcal{T}}
\newcommand\calU{\mathcal{U}}
\newcommand\calX{\mathcal{X}}
\newcommand\calY{\mathcal{Y}}

\newcommand\mfr{\mathfrak}
\newcommand\frA{\mfr{A}}
\newcommand\frB{\mfr{B}}
\newcommand\frD{\mfr{D}}
\newcommand\frF{\mfr{F}}
\newcommand\frH{\mfr{H}}
\newcommand\frI{\mfr{I}}
\newcommand\frK{\mfr{K}}
\newcommand\frL{\mfr{L}}
\newcommand\frM{\mfr{M}}
\newcommand\frN{\mfr{N}}
\newcommand\frO{\mfr{O}}
\newcommand\frP{\mfr{P}}
\newcommand\frQ{\mfr{Q}}
\newcommand\frR{\mfr{R}}
\newcommand\frU{\mfr{U}}


\newcommand{\frg}{\mfr{g}}
\newcommand{\frh}{\mfr{h}}
\newcommand{\frp}{\mfr{p}}
%\newcommand{\frq}{\mfr{q}}
\newcommand{\frr}{\mfr{r}}
\newcommand{\frs}{\mfr{s}}

\newcommand\mbf{\mathbf}
\newcommand\bfB{\mbf{B}}
\newcommand\bfC{\mbf{C}}
\newcommand\bfF{\mbf{F}}
\newcommand\bfG{\mbf{G}}
\newcommand\bfK{\mbf{K}}

\newcommand\mrm{\mathrm}

\newcommand\rmp{\mrm{p}}
\newcommand\rmq{\mrm{q}}

\newcommand\rmI{\mrm{I}}
\newcommand\rmM{\mrm{M}}
\newcommand\rmU{\mrm{U}}
\newcommand\rmT{\mrm{T}}

\newcommand{\scrA}{\mathscr{A}}
\newcommand{\scrB}{\mathscr{B}}
\newcommand{\scrC}{\mathscr{C}}
\newcommand{\scrD}{\mathscr{D}}
\newcommand{\scrE}{\mathscr{E}}
\newcommand{\scrF}{\mathscr{F}}
\newcommand{\scrG}{\mathscr{G}}
\newcommand{\scrH}{\mathscr{H}}
\newcommand{\scrI}{\mathscr{I}}
\newcommand{\scrJ}{\mathscr{J}}
\newcommand{\scrK}{\mathscr{K}}
\newcommand{\scrL}{\mathscr{L}}
\newcommand{\scrM}{\mathscr{M}}
\newcommand{\scrN}{\mathscr{N}}
\newcommand{\scrR}{\mathscr{R}}
\newcommand{\scrS}{\mathscr{S}}

%---------------------------------Relations----------------------------------------

\newcommand\mrel{\mathrel}
\newcommand\Rel{\mathrel{R}}
\newcommand\leqs{\leqslant}
\newcommand\geqs{\geqslant}

%---------------------------------Other----------------------------------------

\newcommand\seq[1]{\langle #1_n \rangle_{n \in \Nat}}
\newcommand\dftt{\mathtt{df}\,}
\newcommand{\dom}{\mathrm{dom}} %the domain of a structure, function etc.
\newcommand{\Tmax}{T_{\mathrm{max}}} %the set of maximal extension of a theory T


\newcommand\suchthat{\,\middle\vert\,} %abstraction operator

%rules of inference macro

\newcommand*{\inference}[3][t]{%
   \begingroup
   \def\and{\\}%
   \begin{tabular}[#1]{@{\enspace}c@{\enspace}}
   #2 \\
   \hline
   #3
   \end{tabular}%
   \endgroup
}





