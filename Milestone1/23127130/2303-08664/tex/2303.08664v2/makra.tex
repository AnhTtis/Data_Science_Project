%%%%%%%%%%%%%%%%%%%%%%%%%%%%%%%%%%%%%%%%%%%%%%%%%%%%%%%%%%

\newcommand\frB{\mathfrak{B}}
\newcommand\frR{\mathfrak{R}}
\newcommand\frM{\mathfrak{M}}
\newcommand\frN{\mathfrak{N}}

\newcommand\calE{\mathcal{E}}
\newcommand\calV{\mathcal{V}}

\newcommand{\isep}{\mbox{\textup{\tiny\texttt{)\!(}}}}

\newcommand{\scrP}{\mathscr{P}}
%%%%%%%%%%%%%%%%%% txfonts AMSa %%%%%%%%%%%%%%%%%%%%%%%%%
%\DeclareMathAlphabet{\txmathbcal}{OMS}{txsy}{bx}{n}
%\newcommand{\F}{\txmathbcal{F}}
\newcommand{\Ko}{\forall}
\newcommand{\Ks}{\exists}
\newcommand\Pociaga{\;\Longrightarrow\;}
\newcommand{\IMP}{\mathrel{\Longrightarrow}}
   \let\Imp=\Rightarrow
%%%%%%%%%%%%%%%%%%%%%%%%%%%%%%%%%%%%%%%%%%%%%%%%%%%%%%%
%\DeclareSymbolFont{AMSa}{U}{txsya}{m}{n}
\DeclareMathSymbol{\ppartof}{\mathord}{AMSa}{64}
\DeclareMathSymbol{\partof}{\mathrel}{AMSa}{64}
%%%%%%%%%%%%%%%%%%%%%%%%%%%%
%%%%%%%%%%%%%%% txfonts symbolsC %%%%%%%%%%%%%%%%%%
\DeclareSymbolFont{symbolsC}{U}{txsyc}{m}{n}
\DeclareMathSymbol{\poverl}{\mathord}{symbolsC}{7}
\DeclareMathSymbol{\overl}{\mathrel}{symbolsC}{7}
\DeclareMathSymbol{\pext}{\mathord}{symbolsC}{78}
%\DeclareMathSymbol{\ext}{\mathrel}{symbolsC}{78}
\DeclareMathSymbol{\npartof}{\mathrel}{symbolsC}{97}
\DeclareMathSymbol{\ningr}{\mathrel}{symbolsC}{64}
\DeclareMathSymbol{\nll}{\mathrel}{symbolsC}{51}

\newcommand\ext{\mathrel{\bot}}
%%%%%%%%%%%%%%%%%%%%%%%%%%%%%%%%%%%%%%%%%%%%%
%%%%%%%%% txfonts varmathbb %%%%%%%%%%%%%%%
%\DeclareMathSymbol{\vbbC}{\mathord}{lettersA}{131}
%\DeclareMathSymbol{\vbbN}{\mathord}{lettersA}{142}
%\DeclareMathSymbol{\vbbR}{\mathord}{lettersA}{146}


%%%%% newcommands for topology
\DeclareMathOperator{\Int}{Int}      %topol. operation of interior
\DeclareMathOperator{\Cl}{Cl}        %topol. operation of closure
\DeclareMathOperator{\Fr}{Fr}        %topol. Fr operation

\newcommand{\czz}{\ingr^{\mathit{0}}}
\newcommand{\Tz}{\mathrm{T}_0}
\newcommand{\Ti}{\mathrm{T}_1}
\newcommand{\Tii}{\mathrm{T}_2}
\newcommand{\Tiih}{\mathrm{T}_{2\frac{1}{2}}}
\newcommand{\Tiii}{\mathrm{T}_3}
\newcommand{\Tiiih}{\mathrm{T}_{3\frac{1}{2}}}
\newcommand{\Tiv}{\mathrm{T}_4}


%% newcommands for logic and set theory
\newcommand\iffdef{\;\mathrel{\mathord{:}\mathord{\longleftrightarrow}}\;}
\let\dfiff\iffdef
\newcommand\iffslim{\longleftrightarrow}
%\newcommand{\dfiff}{\;\mathrel{\mathord{:}\mathord{\Longleftrightarrow}}\;}
\newcommand\defeq{\coloneqq}
\newcommand\eqdef{\eqqcolon}
\newcommand\rarrow{\longrightarrow}
\newcommand\larrow{\longleftarrow}
\newcommand\Rarrow{\Longrightarrow}
\newcommand\Larrow{\Longleftarrow}
\newcommand\fun{\ensuremath{\rightarrow}} %--function arrow
\newcommand{\power}{\mathcal{P}}
\newcommand{\powerne}{\power^{\scriptscriptstyle +}}
\newcommand{\powerfin}{\power_{\!\scriptscriptstyle\textup{fin}}}
\newcommand{\powernefin}{\power^{\scriptscriptstyle  +}_{\!\scriptscriptstyle\textup{fin}}}

\let\la=\langle
\let\ra=\rangle

%% newcommands for mereology
\newcommand{\zero}{\mathsf{0}}
\newcommand{\one}{\mathsf{1}}
\newcommand{\suma}{\mathrel{\mathsf{sum}}}
\newcommand{\sumop}{\bigsqcup}
\newcommand{\Atom}{\mathrm{At}} %the set of atoms
\newcommand{\cAtom}{\mathrm{cAt}} %the set of co-atoms
\newcommand{\AtM}{\Atom_{\mathfrak{M}}}
\newcommand{\AtR}{\Atom_{\mathfrak{R}}}
\newcommand{\AtB}{\Atom_{\mathfrak{B}}}
\newcommand{\AtA}{\Atom_{\mathfrak{A}}}
\newcommand{\cAtM}{\cAtom_{\mathfrak{M}}}
\newcommand{\cAtR}{\cAtom_{\mathfrak{R}}}
\newcommand{\cAtB}{\cAtom_{\mathfrak{B}}}
\newcommand{\cAtA}{\cAtom_{\mathfrak{A}}}
%%%%%%%%%%%%%%%%%%%
\newcommand{\En}{\mathcal{E}^n}
\newcommand{\Ei}{\mathcal{E}^1}
%%%%%%%%%%%%%
\newcommand{\qsepS}{\textup{\footnotesize\texttt{qsep}}}
\newcommand{\qsT}{\qsepS\Topo}
\newcommand{\qsTsr}{\qsepS\Toposr}
%\DeclareMathOperator{\GcS}{\textup{\footnotesize\texttt{G}}_{\!\textup{\tiny\texttt{c}}}}
\newcommand{\GcS}{\textup{\footnotesize\texttt{G}}_{\!\textup{\tiny\texttt{c}}}}
%\DeclareMathOperator{\sGcS}{\textup{\tiny\texttt{G}}_{\!\textup{\tiny\texttt{c}}}}
\newcommand{\sGcS}{\textup{\tiny\texttt{G}}_{\!\textup{\tiny\texttt{c}}}}
\newcommand{\GcT}{\GcS\Topo}
\newcommand{\GcTych}{\GcS T} %G-structure for Punctured Tychonoff Plank
\newcommand{\sGcT}{\sGcS\Topo}
\newcommand{\GcEn}{\GcS\En}
\newcommand{\sGcEn}{\sGcS\En}
\newcommand{\GcEi}{\GcS\Ei}
\newcommand{\sGcEi}{\sGcS\Ei}
\newcommand{\sepS}{\textup{\footnotesize\texttt{sep}}}
\newcommand{\sT}{\sepS\Topo}
\newcommand{\sTsr}{\sepS\Toposr}
\newcommand{\sEn}{\sepS\En}
\newcommand{\sEi}{\sepS\Ei}
\newcommand{\ssEn}{\mathfrak{E}^n}
\newcommand{\ssEi}{\mathfrak{E}^1}
%%%%%%%%%%%%%%%%%%%%%%%
\newcommand{\gT}{\mathtt{g}\Topo}
\newcommand{\gTsr}{\mathtt{g}\Toposr}
%%%%%%%%%%%%%%%%%%%%%
\newcommand{\tsR}{\mathtt{ts}\mathfrak{R}}
\newcommand{\tsOmega}{\mathtt{ts}\Omega}
\newcommand{\tsOmegac}{\mathtt{ts}\overline{\Omega}}
%\newcommand{\Gts}{\mathtt{Gts}}
\DeclareMathOperator{\Gts}{\mathtt{Gts}}
\newcommand{\GtsR}{\Gts\,\mathfrak{R}}
\newcommand{\GtsEn}{\Gts(\GcEn)}
\newcommand{\GtsT}{\Gts\,\sT}
\newcommand{\Stone}{\mathtt{Ult}}
\newcommand{\StoneR}{\Stone\mathfrak{R}}
\newcommand{\StoneB}{\Stone\mathfrak{B}}
\newcommand{\StoneA}{\Stone\mathfrak{A}}

%% newcommands for BA
\newcommand\bzero{\mit{0}}      %zero w algebrze Boole
\newcommand\bone{\mit{1}}         %jedynka w algebrze Boole

%\DeclareMathOperator{\bplus}{\boldsymbol{+}}
%\DeclareMathOperator{\bmulti}{\boldsymbol{\cdot}}
\DeclareMathOperator{\bminus}{\textup{\textbf{\textsf{--}}}}
\DeclareMathOperator{\bleq}{\leq}

%%%%%%%%%%%%%%%%%%%%%%-separacja-etc-%%%%%%%%%%%%%%%%%%%%%%%%%%%%%%%%
\newcommand{\msep}{\mathrel{\mbox{\textup{\footnotesize\texttt{)\!(}}}}}
\let\separ=\msep
\newcommand{\fmsep}{\mathrel{\mbox{\textup{\scriptsize\texttt{)\!(}}}}}
\newcommand{\tmsep}{\mathrel{\mbox{\textup{\tiny\texttt{)\!(}}}}}
\let\ssepar=\tmsep
%\newcommand{\separi}{\mathrel{\mathbin{\raisebox{1pt}{{\normalfont\scriptsize\textsf{][}}}}}}
\newcommand{\separi}{\mathrel{\mbox{\textup{\texttt{]\![}}}}}
\newcommand{\separibfsf}{\mathrel{\mbox{\textup{\textbf{\textsf{)(}}}}}}
\newcommand{\separibf}{\mathrel{\mbox{\textup{\textbf{)(}}}}}
\newcommand{\separisf}{\mathrel{\mbox{\textup{\textsf{)(}}}}}
\let\separii=\separisf

%\DeclareMathOperator{\ET}{\mathsf{ET}} %relacja zewnêtrznej stycznoœci
\newcommand{\con}{\mathrel{\mathsf{C}}} %relacja styczności
\DeclareMathOperator{\cont}{\mathsf{C}_{\mathrm{T}}} %relacja stycznoœci topologiczna
\DeclareMathOperator{\confull}{\mathsf{C}_{\mathrm{L}}} %relacja stycznoœci pe³na
\DeclareMathOperator{\cono}{\mathsf{C}^{\circ}} %relacja stycznoœci w RO(R)^\circ


\newcommand\mathbackslash{\raisebox{.4pt}{\texttt{/}}}
\def\notcon{% the separation relation
  \renewcommand\stacktype{L}\mathrel{\ensurestackMath{%
  \ThisStyle{\stackon[0pt]{\SavedStyle\con}{\SavedStyle\mathbackslash}}}}%
}

\let\separ=\notcon


\newcommand{\rel}{\mathrel{R}}

\def\notrel{% the complement of R
  \renewcommand\stacktype{L}\mathrel{\ensurestackMath{%
  \ThisStyle{\stackon[0pt]{\SavedStyle\rel}{\SavedStyle\mathbackslash}}}}%
}

\DeclareMathOperator{\Con}{\boldsymbol{\mathsf{C}}} %topologiczna relacja stycznoœci


%\def\notcon{%
%  \renewcommand\stacktype{L}\mathrel{\ensurestackMath{%
%  \ThisStyle{\stackon[0pt]{\SavedStyle\con}{\SavedStyle\mathbackslash}}}}%
%}


%\DeclareMathOperator{\conf}{\mathrel{\infty}} %connection relation for filters in Roeper
\newcommand{\conf}{\mathbin{\dwakola}}

%\def\nconf{%  \renewcommand\stacktype{L}\mathrel{\ensurestackMath{%
%  \ThisStyle{\stackon[0pt]{\SavedStyle\infty}{\SavedStyle/}}}}%

\def\nconf{%
\renewcommand\stacktype{L}\mathbin{\ensurestackMath{%
  \ThisStyle{\stackon[0pt]{\SavedStyle\dwakola}{\SavedStyle/}}}}}

%\def\nconf{\mathrel{\not\dwakola}}

\def\dwakola{% \renewcommand\stacktype{S}\mathrel{\ensurestackMath{%
%  \ThisStyle{\stackon[0pt]{\SavedStyle\medcirc}{\SavedStyle\medcirc}}}}}
%\mathrel{\medcirc\hspace{-1pt}\medcirc}}
\mathord{\circ}\!\mathord{\circ}}

\newcommand{\GX}{\mathbf{Q}_{\mathrm{G}}}
\def\QG #1#2#3{\mathrm{G}^{{#1}}_{{#2},{#3}}}
\newcommand{\prePt}{\mathbf{Q}_G}
\newcommand{\prePtW}{\mathbf{Q}_W}
\newcommand{\Abs}{\mathbf{A}}


\newcommand{\wprePt}{\mathbf{Q}^{\boldsymbol{\omega}}}
\newcommand{\prePtR}{\prePt_{\mathfrak R}}
\newcommand{\wprePtR}{\wprePt_{\mathfrak R}}
\newcommand{\Pt}{\mathbf{Grz}}
\let\Grz=\Pt
\newcommand\Eq{\mathbf{Eq}}
\newcommand\Wthd{\mathbf{W}}
\newcommand{\PtR}{\mathbf{Pt}_{\mathfrak R}} %the set of all points
\newcommand{\PtS}{\mathbf{Pt}_{\mathfrak S}} %the set of all points
%\newcommand{\Pt}{\Pi} %the set of all points
\newcommand{\Pts}{\Pt^{\star}}
%%%%%%%%%%%%%
%\DeclareMathAlphabet{\mathb}{OT1}{cmr}{b}{n}
\DeclareMathOperator{\Irl}{\textup{\bfseries Irl}}
\newcommand{\Adh}{\textup{\bfseries A}}
\newcommand{\Is}{\textup{\bfseries Iso}}

\newcommand{\rAdh}{\mathbin{\Adh}}
\newcommand{\Open}{\boldsymbol{\mathsf{O}}}
\newcommand{\OpenEn}{\Open_{\sGcEn}}
\newcommand{\OpenR}{\Open_{\mathfrak R}}
\newcommand{\OpenRp}{\OpenR^{\scriptscriptstyle+}}
\newcommand{\rOpenR}{\mathrm{r}\OpenR}
\newcommand{\rOpenRp}{\rOpenR^{\scriptscriptstyle+}}
%\newcommand{\Is}{\mathsf{Is}}
\newcommand{\Closed}{\boldsymbol{\mathsf{Cl}}}
\newcommand{\ClosedR}{\Closed_{\mathfrak R}}
\newcommand{\Clop}{\boldsymbol{\mathsf{Clop}}}
\newcommand{\ClopR}{\Clop_{\mathfrak R}}
\newcommand{\Baza}{\boldsymbol{\mathsf{B}}}
\newcommand{\BazaR}{\Baza_{\mathfrak R}}
\newcommand{\BazaEn}{\Baza_{\sGcEn}}
\newcommand{\podBaza}{\textup{\textit{\bfseries{\textsf{B}\/}}}}
\newcommand{\Topo}{\mathsf{T}}
\newcommand{\Toposr}{\Topo_{\!\!\mathrm{sr}}}
\newcommand{\topo}{\mathscr{O}}
\newcommand{\toposr}{\topo_{\!\mathrm{sr}}}
\newcommand{\topop}{\topo^{\scriptscriptstyle+}}
\newcommand{\topoE}{\mathscr{E}}
\newcommand{\topoEp}{\topoE_{\!\scriptscriptstyle+}}
\newcommand{\rtopo}{\mathrm{r}\topo} %zbiory regularnie otwarte, nie u¿ywam w tym artykule
\DeclareMathOperator{\RO}{RO} %zbiory regularnie otwarte, u¿ywam w tym artykule
\newcommand{\rtopoE}{\mathrm{r}\topoE}
\newcommand{\rtopoEp}{\mathrm{r}\topoEp}
\newcommand{\rtoposr}{\mathrm{r}\toposr}
\newcommand{\rtopop}{\rtopo^{\scriptscriptstyle+}}
%\newcommand{\rtoporp}{\rtopo_{\mathrm{r}}^{\scriptscriptstyle+}}
\newcommand{\baza}{\mathscr{B}}
\newcommand{\bazaQ}{\mathscr{Q}}
\newcommand{\rbaza}{\mathrm{r}\baza}
\newcommand{\Dtopo}{\mathscr{C}}
\newcommand{\ClOp}{\mathscr{C\!O}}

\newcommand\ccc{\mathrm{(ccc)}}

%\newcommand{\fil}{\mathcal{F}}
%\newcommand{\fil}{\nabla}
\newcommand{\fil}{\mathscr{F}}
\newcommand{\ult}{\mathscr{U}}
\newcommand{\rfil}{\mathrm{F}}
\newcommand{\End}{\mathbf{End}}
\newcommand{\Fil}{\mathbf{Fil}}
\newcommand{\Rnd}{\mathbf{Rnd}}
\newcommand{\ultf}{F}
\newcommand{\Ult}{\mathbf{Ult}}
\newcommand{\UltM}{\mathop{\Ult(\mathfrak M)}}
\newcommand{\UltR}{\mathop{\Ult(\mathfrak R)}}
\newcommand{\UltB}{\mathop{\Ult(\mathfrak B)}}

\newcommand\Sat{\mathbf{Sat}} %saturated family of filters

\DeclareMathOperator\upop{\uparrow} %upward closure operation
\DeclareMathOperator\downop{\downarrow} %downward closure operation

\newcommand\twodownop{\raisebox{-1pt}{$\mathop{\rotatebox{90}{$\twoheadleftarrow$}}$}} %downward << closure operation
\newcommand\twoupop{\raisebox{-1pt}{$\mathop{\rotatebox{90}{$\twoheadrightarrow$}}$}} %upward << closure operation

\newcommand{\CFil}{\mathrm{CFil}}
\newcommand{\MCFil}{\mathrm{MCFil}}
\newcommand{\CFilR}{\mathop{\CFil}(\mathfrak R)}
\newcommand{\MCFilR}{\mathop{\MCFil}(\mathfrak R)}
%\newcommand{\ult}{\mathop\mathrm{ult}}
%\newcommand{\ult}{\mathop\mathrm{s}}
%%%%%%%%%%%%%%%%%%%%%%%%%%Zbiory-etc.%%%%%%%%%%%%%%%%%%%%%%%%%%%%%%%%%%%%%

\newcommand\R[1]{{\mathds{R}}^{{#1}}} %the set R3
%\newcommand\Nat{\text{\textomega}} %{\mathds{N}} %the set of natural numbers
\newfont{\eurxi}{eurm10 at 10.95pt}
\newfont{\eurviii}{eurm7 at 8pt}
\newfont{\eurvii}{eurm7}
\newcommand{\eriota}{\mbox{\eurxi\symbol{19}}}
\newcommand{\fNat}{\mbox{\eurviii\symbol{33}}} %the set of natural numbers for footnotes
\newcommand{\sNat}{\mbox{\eurvii\symbol{33}}} %the set of natural numbers for indexes
\newcommand{\Nat}{\mbox{\eurxi\symbol{33}}} %the set of natural numbers
\newcommand \Natp{\Nat^{\scriptscriptstyle\!+}}  %the set of positive natural numbers
\newcommand \sNatp{\sNat^{\scriptscriptstyle\!+}}%the set of positive natural numbers for indexes
\newcommand\Real{\mathds{R}} %the set of real numbers
\newcommand\Rat{\mathds{Q}} %the set of rational numbers
\newcommand\Irr{\mathds{I}} %the set of irrational numbers
\newcommand\C[1]{{\mathds{C}}_{#1}} %Cartesian space

\newcommand{\rO}{\mathrm{r}\topo}
\newcommand{\rOp}{\rO^{\!\scriptscriptstyle+}}
\newcommand{\ROp}{\RO^{\!\scriptscriptstyle+}}
\newcommand\rOCi{\rO_{\C{1}}}
\newcommand\rOCiii{\rO^{\,\C{3}}}
\newcommand\rOCiiip{\rOp^{\,\C{3}}}


\newcommand{\NEG}{\mathop{\neg}}
\newcommand{\ROW}{\;\Leftrightarrow\;}
\newcommand{\KS}{\exists}
\newcommand{\KO}{\forall}


\newcommand{\bi}{\mathrm{b}}
\newcommand{\Bb}{\mathrm{B}}

\def\dywiz{\kern0sp\discretionary{-}{}{-}\penalty10000\hskip0sp\relax}

%%%%%%%%%%%%%%%%Poni¿ej s¹ ze starego artyku³u

\newcommand{\MS}{\boldsymbol{\mathsf{MS}}}
\newcommand{\MF}{\boldsymbol{\mathsf{MF}}}
\newcommand{\MFone}{\boldsymbol{\mathsf{MF1}}}
\newcommand{\qSep}{\boldsymbol{\mathsf{qSep}}} %quasi-separation structures
\newcommand{\qSepone}{\boldsymbol{\mathsf{qSep1}}} %quasi-separation structures with one
\newcommand{\qSepc}{\boldsymbol{\mathsf{qSep}_{\mathsf{c}}}} %complete quasi-separation %structures
\newcommand{\Sep}{\boldsymbol{\mathsf{Sep}}} %separation structures
\newcommand{\Sepc}{\boldsymbol{\mathsf{Sep}_{\mathsf{c}}}} %separation structures
\newcommand{\Sepone}{\boldsymbol{\mathsf{Sep1}}} %separation structures
\newcommand{\G}{\boldsymbol{\mathsf{G}}} %G-structures
\newcommand{\Gc}{\boldsymbol{\mathsf{G}_{\mathsf{c}}}} %complete G-structures
\newcommand{\Gcc}{\boldsymbol{\mathsf{G}^{\aleph_0}_{\mathsf{c}}}} %complete and countable G-structures
\newcommand{\Gone}{\boldsymbol{\mathsf{G1}}} %G-structures
\newcommand{\Gonec}{\boldsymbol{\mathsf{G1}_{\mathsf{c}}}} %G-structures
\newcommand{\GS}{\boldsymbol{\mathsf{G}^{\star}}} %G-structures

%%%%%%%%%%%%%%%%%%%%%%%%%%%%%%%%%%%%%%%%%%%%%%%%%%%%%%%%%%%%%%%
\newcommand\Klass{\boldsymbol{\mathsf{K}}} %abstract class of structures
\newcommand\BA{\boldsymbol{\mathsf{BA}}} %the class of Boolean algebras
\newcommand\BCA{\boldsymbol{\mathsf{BCA}}} %the class of Boolean contact algebras
\newcommand\TBCA{\boldsymbol{\mathsf{TBCA}}} %the class of topological Boolean contact algebras
\newcommand\BPCA{\boldsymbol{\mathsf{BPCA}}} %the class of Boolean contact algebras
\newcommand\GCA{\boldsymbol{\mathsf{GCA}}} %the class of Grzegorczyk contact algebras
\newcommand\BWCA{\boldsymbol{\mathsf{BWCA}}} %the class of Boolean weak contact algebras
\newcommand\Conc{\boldsymbol{\mathsf{Conc}}} %the class of concentric spaces
\newcommand\Lob{\boldsymbol{\mathsf{Lob}}} %the class of lob-spaces, i.e. the spaces which have local basis ordered linearly by reversed subset relation


%\newcommand\Even{\varmathbb{E}} %even ordinals
\newcommand\Even{\mathds{E}}
%\newcommand\Odd{\varmathbb{O}} %odd ordinals
\newcommand\Odd{\mathds{O}}

\newcommand\dsA{\mathds{A}}
\newcommand\dsB{\mathds{B}}
\newcommand\dsP{\mathds{P}}

%points

\newcommand\frp{\mathfrak{p}}
\newcommand\frq{\mathfrak{q}}
\newcommand\frr{\mathfrak{r}}
\newcommand\frs{\mathfrak{s}}

\newcommand\frg{\mathfrak{g}}
\newcommand\frh{\mathfrak{h}}

%
\newcommand{\Inf}{\mathrel{\SF{inf}}}
\newcommand{\bsuma}{\sqcup}
\let\bprod=\sqcap
\DeclareSymbolFont{stx}{OMS}{txsy}{m}{n}
\DeclareMathSymbol{\Ing}{\mathrel}{stx}{118}
\DeclareMathSymbol{\ll}{\mathrel}{stx}{28}
\newcommand{\llt}{\ll_{\mathrm{T}}}
\newcommand{\llfull}{\ll_{\mathrm{F}}}
\DeclareMathSymbol{\medcirc}{\mathord}{symbolsC}{7}
\DeclareMathSymbol{\Ext}{\mathrel}{symbolsC}{78}
%\DeclareMathSymbol{\Ing}{\mathrel}{symbols}{118}

\newcommand{\covers}{\trianglerighteq} %covering relation
\newcommand{\ncovers}{\ntrianglerighteq} %complement covering relation
\newcommand\covered{\trianglelefteq} %being covered relation
\newcommand\ncovered{\ntrianglelefteq} %complement of being covered relation

\newcommand{\llast}{\mathrel{\underline{\mathord{\ll}}}}

\DeclareMathSymbol{\nIng}{\mathrel}{symbolsC}{64}
\DeclareMathSymbol{\Ov}{\mathrel}{symbolsC}{7}
\newcommand{\sep}{\mathbin{\raisebox{1pt}{{\normalfont\scriptsize\textsf{)(}}}}}
\let\SF=\mathsf
\newcommand{\opI}{\SF{I}}
\let\ingr=\Ing
\let\ing=\Ing

\newcommand{\FC}{\mathrm{FC}}
\newcommand{\FCN}{\FC(\Nat)}
\newcommand{\FCNp}{\FC^{+}(\Nat)}
\newcommand{\FCS}{\FC(S)}
\newcommand{\FCSp}{\FC^{+}(S)}

\newcommand{\eqqcolon}{\mathrel{\mathord{=}\mathord{:}}}
\newcommand{\coloneqq}{\mathrel{\mathord{:}\mathord{=}}}
\let\coloneq=\coloneqq


\newcommand\sqcupo{\sqcup^\circ}
\newcommand\sqcapo{\sqcap^\circ}
\newcommand\compo{-^\circ}

\newcommand\iso[1]{#1^{\mathrm{i}}} %isolated points of a given set


\endinput 