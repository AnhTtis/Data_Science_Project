\begin{appendices} 
	\section{Solvability of the linear problem}	\label{Appendix:Semi-discrete}
	We present here the proof of existence of a unique solution of \eqref{Ph_nl} under the assumptions of Proposition~\ref{Prop:LinStability}. Given a basis $\{\phi_i\}_{i=1}^{n=n(h)}$ of $V_h$, we have
	\begin{equation}
	\begin{aligned}
	\uh(x,t) =\sum_{i=1}^{n=n(h)} \xi_i(t)\phi_i(x).
	\end{aligned}
\end{equation}
Denoting $\bfxi= [\xi_1 \ldots \xi_n]^T$, the semi-discrete problem can be rewritten in matrix form
	\begin{equation} \label{matrix_eq}
		\begin{aligned}
			\mathbb{M}_{1+k\alphah}(t)\boldsymbol{\xi}_{tt}+ c^2 \mathbb{K} \boldsymbol{\xi}+\eps \mathbb{K}\, \frakK*\boldsymbol{\xi}_t+	\mathbb{M}_{\alphaht}(t)\boldsymbol{\xi}_{t}= \boldsymbol{b}(t)
		\end{aligned}
	\end{equation}
	where the entries of the weighted mass matrices $\mathbb{M}_{1+k\alphah}(t)=[\mathbb{M}_{1+k\alphah, ij}]$, $\mathbb{M}_{\alphaht}(t)=[\mathbb{M}_{\alphaht, ij}]$,  and  the stiffness matrix $\mathbb{K}=[\mathbb{K}_{ij}]$ are given by
	%  and the vector $\boldsymbol{f}(t)=[f_i(t)]$ 
	\begin{equation} \label{matrices}
		\begin{aligned}
			 &\mathbb{M}_{1+k\alphah, ij}(t)= ((1+k \alphah(t))  \phi_i, \phi_j)_{L^2}, \quad  \mathbb{M}_{\alphaht, ij}(t)= (k \alphaht(t)  \phi_i, \phi_j)_{L^2},  \\[1mm]
			  &\mathbb{K}_{ij}= (\nabla \phi_i, \nabla \phi_j)_{L^2}.
		\end{aligned}
	\end{equation}
The entries of the right-hand side vector $\boldsymbol{b}=[\boldsymbol{b}_j]$ are
\[
\boldsymbol{b}_j(t)=(f(t), \phi_j)_{L^2}.
\]
Note that since $f \in W^{1,1}(0,T; \Ltwo) \hookrightarrow C([0,T]; \Ltwo)$, we have $\boldsymbol{b} \in C[0,T]$.	If	we  introduce the vectors of coordinates of the approximate initial data $(u_{0h}, u_{1h})$ in the basis: 
	\[
	\boldsymbol{\xi}_0=[\xi_{0,1} \ \xi_{0,2} \ \ldots \ \xi_{0,n}]^T, \quad \boldsymbol{\xi}_1=[\xi_{1,1} \ \xi_{1,2} \ \ldots \ \xi_{1,n}]^T,
	\]
	then by setting $\bfmu=\boldsymbol{\xi}_{tt}$, we have
	\begin{equation} \label{eq_xi}
		\bfxi_t(t)=1\Lconv\bfmu+\bfxi_1, \quad	\boldsymbol{\xi}(t)=\boldsymbol{\xi}_0 +t \boldsymbol{\xi}_1+1*1*\bfmu.
	\end{equation}
	Therefore, the semi-discrete problem can be rewritten as
	\begin{equation}
		\begin{aligned}
			\begin{multlined}[t]
				\mathbb{M}_{1+k\alphah}(t)\bfmu+ c^2 \mathbb{K}(\boldsymbol{\xi}_0 +t \boldsymbol{\xi}_1+1*1*\bfmu)+\eps \mathbb{K}\, \frakK*(1\Lconv\bfmu+\bfxi_1)\\ \hspace*{4cm}
				+\mathbb{M}_{\alphaht}(t)(1\Lconv\bfmu+\bfxi_1)=\ \boldsymbol{b}(t). \end{multlined}
		\end{aligned}
	\end{equation}
	Since $\mathbb{M}_{1+k\alphah}$ is positive definite due to the assumptions on $\alphah$, the semi-discrete problem can be seen as a system of Volterra integral equations 
	\begin{equation}
		\begin{aligned}
			\bfmu+\boldsymbol{K}*\bfmu(s)\ds = \boldsymbol{{f}}(t)
		\end{aligned}
	\end{equation}
	with the kernel
	\begin{equation}
	\begin{aligned}
		\boldsymbol{K}= \left\{\mathbb{M}^{-1}_{1+k\alphah}(t)\right\}^{-1}\left\{c^2 \mathbb{K} 1*1+\eps \mathbb{K}\, \frakK*1+\mathbb{M}_{\alphaht}(t)\right\}
	\end{aligned}	
		\end{equation}
	and the right-hand side
	\begin{equation}
	\begin{aligned}
	\boldsymbol{{f}}(t)=&\, \begin{multlined}[t]\left\{\mathbb{M}_{1+k\alphah}(t)\right\}^{-1}\left\{\boldsymbol{b}(t)- c^2 \mathbb{K}(\boldsymbol{\xi}_0 +t \boldsymbol{\xi}_1)-\eps \mathbb{K}\, \frakK*\bfxi_1-\mathbb{M}_{\alphaht}(t)\bfxi_1 \right\}.
	\end{multlined}
	\end{aligned}	
\end{equation}
Since $\alphah \in W^{2,1}(0,T; V_h) \hookrightarrow C^1([0,T]; V_h)$ and $\boldsymbol{b} \in C[0,T]$, we have
	\[
	\boldsymbol{K} \in C[0,T], \quad  \boldsymbol{{f}} \in C[0,T].
	\]
The existence theory for systems of Volterra integral equations of the second kind~\cite[Ch.\ 2, Theorem 4.2]{gripenberg1990volterra} yields a unique $\bfmu \in L^\infty(0,T)$. Using $\bfxi_{tt}=\bfmu$ supplemented by the initial data, we conclude that there exists a unique $\bfxi \in W^{2, \infty}(0,T)$ and thus $\uh \in W^{2, \infty}(0,T; V_n)$. Owing to the higher regularity of the coefficient $\alphah$ and source term $f$ in time, an additional bootstrap argument shows that $\uh \in W^{3,1}(0,T; V_h)$.
\end{appendices}