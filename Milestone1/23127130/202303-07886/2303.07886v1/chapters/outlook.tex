\section{Conclusion and Outlook}
\label{sec:outl}
In this paper, we outlined a map-based risk warning device for urban driving. RNS aligns the GNSS position in a relational local dynamic map. Upcoming traffic objects are then queried along the navigation route (i.e., detected other cars from lidar, sharp turns and crosswalks). Here, we derive road geometries and object positions with augmented OpenStreetMap data and utilize direct data links in a graph representation. The situation can be evaluated with TTCE for car-to-car collision risk, maximal curvature for curve risk and the distance to full stop for regulatory risks. 

A detailed driving context is now visualized using a 2D/3D renderer. We chose OpenGL augmented by elements that convey the risk sources with spatio-temporal vehicle interactions. Each of them have the same underlying color code from green as safe to red as dangerous. With an application of RNS on real intersections, we eventually showed the versatile warning functionality. The user is supported to perceive the relevant obstacles in long range, safe behaviors are recommended in medium terms and risky encounters are highlighted for short times.    %high urgency respectively

Instead of pure GNSS usage, the RNS could benefit from adding vision-based localization components (as mentioned before).  % should be used for inexpensive and reliable applications. 
The visualizations of RNS depend on the quality of the map and also on the reference within the map. Matching detected lanes on images with stored street polylines is promising \cite{flade2018}. This potentially also reduce the errors from the shared map. Intersections are still to be investigated because of typically mixed curvy and straight segments. %In addition, as the position is only available every 1 second, need to fuse the velocity. 

The risk prediction in RNS draws upon time metrics and does not consider uncertainties in driving. In previous research, we developed the survival analysis \cite{puphal2019} that consists of Gaussian and Poisson distributions for the modeling of states and critical events. It showed to detect risks early and with less false positives. Hence, we may be able to make RNS more effective. A combination of intuitive with uncertainty-aware risks for warning is possible. 

Furthermore, augmented reality is deployable besides the 2D screen display. A 3D environment could be simulated in first-person-view and overlayed on camera images. Here, the pitch angle vibration from bumps in the road has to be compensated by changing the camera angle of the simulator. We could thus circumvent the higher effort of processing 3D compared to 2D information.

Finally, the paper evaluated and discussed technological aspects of a Risk Navigation System and showed its potential. What remains to be done as a next clear step is a user study that investigates the ergonomics, psychological workload and societal acceptance.  
