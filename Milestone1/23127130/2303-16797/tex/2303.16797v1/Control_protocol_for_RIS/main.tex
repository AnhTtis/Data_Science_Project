\documentclass[12pt, draftclsnofoot, onecolumn]{IEEEtran}
% \documentclass[comsoc]{IEEEtran}

% Graphics
\usepackage[svgnames]{xcolor}
\usepackage{graphicx}
\usepackage[font=small, skip=5pt, belowskip=0pt]{caption}
\usepackage[font=footnotesize]{subcaption}
% Tables
\usepackage{multirow}
\usepackage{array}
\usepackage{booktabs}
\usepackage{pdflscape}
\usepackage[ruled,vlined,linesnumbered]{algorithm2e}
% Text
\usepackage[T1]{fontenc}
\usepackage[utf8]{inputenc}
\usepackage{cite}
\usepackage{url}
\usepackage{comment}
% Spacing & Tabulation
\usepackage{mleftright,xparse}
% Math
\usepackage{amssymb,amsmath,amsfonts,amsthm,bm,mathtools,cuted,bbold}
\usepackage{siunitx}
\sisetup{output-exponent-marker=\ensuremath{\mathrm{e}}}
% Plotting
\usepackage{tikz}
\usetikzlibrary{math, shapes}
\usepackage{pgfplots}
\usepgfplotslibrary{groupplots}
\pgfplotsset{
compat=1.13,
legend style={font=\footnotesize, fill opacity=0.7,  draw opacity=1, text opacity=1, draw=white!15!black, legend cell align=left, align=left}, 
width=6cm, 
height=6cm,
yminorticks=false,
xminorticks=false,
label style={font=\small},
title style={font=\small},
tick style={color=black},
tick align=outside,
tick pos=left,
tick label style={font=\footnotesize},
grid style={line width=.1pt, draw=gray!20},
major grid style={line width=.1pt,draw=gray!20},}

\usepackage[normalem]{ulem}

% Acronyms
\usepackage[acronym]{glossaries}
\begin{acronym}
    \acro{LDA}{\emph{Latent Dirichlet Allocation}}
    \acro{CMT}{\emph{Conference Management Toolkit}}
    \acro{TPMS}{\emph{The Toronto Paper Matching System}}
    \acro{MCMF}{\emph{MinCost-MaxFlow}}
\end{acronym}

% ============ %
% NEW COMMANDS %
% ============ %
\newcommand{\T}{^{\mathsf{T}}}     % transpose
\renewcommand{\H}{^{\mathsf{H}}}   % hermitian
\newcommand{\E}[1]{\mathbb{E}\left[ #1 \right]} % expectation
\newcommand{\mc}[1]{\mathcal{#1}}   % mathcal abbreviation
\newcommand{\mb}[1]{\mathbf{#1}}    % mathbold abbreviation
\newcommand{\dpar}[2]{\mathop{}\!\frac{\partial #1}{\partial #2}}   % partial derivative
\newcommand*\diff{\mathop{}\!\mathrm{d}}    % differential in the integral
\DeclareMathOperator*{\argmax}{arg\,max}    % argmax
\DeclareMathOperator*{\argmin}{arg\,min}    % argmin
% Labels
\newcommand{\oce}{_\mathrm{CE}}
\newcommand{\bsw}{_\mathrm{CB}} 
\newcommand{\alg}{_\mathrm{alg}} 
\newcommand{\set}{_\mathrm{set}} 
\newcommand{\ack}{_\mathrm{ack}} 
\newcommand{\pay}{_\mathrm{pay}} 


% Color pool
\definecolor{amaranth}{rgb}{0.9, 0.17, 0.31}
\definecolor{cadmiumgreen}{rgb}{0.0, 0.42, 0.24}

% Comments

\newcommand{\TODO}[1]{\textcolor{orange}{TODO: {#1}}}
\newcommand{\colr}[1]{\textcolor{red}{#1}}
\newcommand{\colb}[1]{\textcolor{blue}{#1}}

\newcommand{\RK}[1]{\textcolor{orange}{#1}}

\newcommand{\pp}[1]{\textcolor{BlueViolet}{#1}}

\newcommand{\ga}[1]{\textcolor{CadetBlue}{#1}}
\newcommand{\GA}[1]{\textcolor{CadetBlue}{\footnotesize \textsf{[George: #1]}}}

\newcommand{\fs}[1]{\textcolor{amaranth}{#1}} % Fabio variations
\newcommand{\FS}[1]{\textcolor{amaranth}{\footnotesize \textsf{[Fabio: #1]}}} % Fabio comments

\newcommand{\vc}[1]{\textcolor{cadmiumgreen}{#1}}

\newcommand{\ks}[1]{\textcolor{cyan}{#1}}
\newcommand{\KS}[1]{\textcolor{cyan}{ \textsf{[KS: #1]}}}

% Metadata
\title{A Framework for Control Channels Applied to Reconfigurable Intelligent Surfaces}
%\title{Control Channels for Communications Aided by Reconfigurable Intelligent Surfaces} 

\author{
    Fabio Saggese,~\IEEEmembership{Member,~IEEE}, Victor Croisfelt,~\IEEEmembership{Student~Member,~IEEE},  Rados\l{}aw Kotaba, Kyriakos Stylianopoulos,~\IEEEmembership{Student~Member,~IEEE}, George C. Alexandropoulos,~\IEEEmembership{Senior~Member,~IEEE}, and Petar Popovski,~\IEEEmembership{Fellow,~IEEE}%, 
    \thanks{This work was partly supported by the Villum Investigator grant ``WATER'' from the Villum Foundation, Denmark, and by the EU H2020 RISE-6G project under grant number 101017011. F. Saggese, V. Croisfelt, R. Kotaba, and P. Popovski are with the Connectivity Section of the Department of Electronic Systems, Aalborg University, Aalborg, Denmark (e-mails: \{fasa, vcr, rak, petarp\}@es.aau.dk). K. Stylianopoulos and G. C. Alexandropoulos are with the Department of Informatics and Telecommunications, National and Kapodistrian University of Athens, Panepistimiopolis Ilissia, 15784 Athens, Greece (e-mails: \{kstylianop, alexandg\}@di.uoa.gr).}
}


\begin{document}

\maketitle

\begin{abstract}
    The research on Reconfigurable Intelligent Surfaces (RISs) has dominantly been focused on physical-layer aspects and analyses of the achievable adaptation of the propagation environment. Compared to that, the questions related to link/MAC protocol and system-level integration of RISs have received much less attention. This paper addresses the problem of designing and analyzing control/signaling procedures, which are necessary for the integration of RISs as a new type of network element within the overall wireless infrastructure. We build a general model for designing control channels along two dimensions: \emph{i}) allocated bandwidth (in-band and out-of band) and \emph{ii}) rate selection (multiplexing or diversity). Specifically, the second dimension results in two transmission schemes, one based on channel estimation and the subsequent adapted RIS configuration, while the other is based on sweeping through predefined RIS phase profiles. The paper analyzes the performance of the control channel in multiple communication setups, obtained as combinations of the aforementioned dimensions. While necessarily simplified, our analysis reveals the basic trade-offs in designing control channels and the associated communication algorithms. Perhaps the main value of this work is to serve as a framework for subsequent design and analysis of various system-level aspects related to the RIS technology.
    % This letter provides a comparison between the communication based on beam sweeping and the channel estimation techniques taking into account the impact of the RIS control channel.   
    % \textbf{Research question}: how reliable should the control channel be? Normally, it is assumed in the literature that the control channel always works, but what happens if it is lost (fail)?
    % \KS{While Reconfigurable Intelligent Surfaces (RISs) have the potential to dynamically create favorable radio propagation environments, thus boosting the performance of wireless systems with minimal deployment and energy costs, their deployment entails increased overheads in terms of protocol implementation, channel setup, and control messages.
    % Nevertheless, such aspects have largely been neglected by the research community, and the primary focus has been on the the development of algorithmic methods for optimizing the RIS configurations. 
    % To truly assess the benefits of RIS-empowered configurations, a comprehensive view of the system is required notwithstanding.
    % Toward that end, this paper is centered around the practical design of control channels for RIS-aided communications and provides a systematic performance analysis.
    % Particularly, a simplified, yet generic, protocol framework is developed that is in large algorithmic-agnostic. The structure of control messages between the RIS and the system controller is investigated with regard to the in-band / out-of-band cases and events of erroneous transmissions, while a general throughput metric is derived that captures the true performance of the system.
    % Two main algorithmic paradigms of channel estimation \& iterative optimization and codebook-based beam sweeping are further analyzed in terms of their protocol requirements and achievable performance.
    % The results of the numerical analysis demonstrate that ...}
\end{abstract}
\begin{IEEEkeywords}
Reconfigurable intelligent surfaces, control channel, protocol design, performance analysis.
\end{IEEEkeywords}

\section{Introduction} \label{sec:intro}
\section{Introduction}
\label{sec:introduction}
% \begin{itemize}
%     % Diffusion of FL
%     \item {\st{Diffusion of FL}}
%     % Security threats to FL
%     \item {\st{Security threats to FL with particular focus on model poisoning}}
%     % Limitations of existing countermeasures
%     \item {\st{Current countermeasures (e.g., KRUM) and their limitations}}
%     % Proposed method and its advantages
%     \item {\st{Intuitive description of the proposed method and its difference (i.e., advantages) w.r.t. state of the art}}
%     % Main contributions
%     \item {\st{Summary of the main contributions of this work}}
%     % Paper's structure and organization
%     \item {\st{Paper's structure and organization}}
% \end{itemize}

% Diffusion of FL
Recently, {\em federated learning} (FL) has emerged as the leading paradigm for training distributed, large-scale, and privacy-preserving machine learning (ML) systems~\cite{mcmahan2017googleai,mcmahan2017aistats}. 
The core idea of FL is to allow multiple edge clients to collaboratively train a shared, global model without disclosing their local private training data.
%Specifically, an FL system consists of a central server and many edge clients; 
A typical FL round involves the following steps: {\em(i)} the server randomly picks some clients and sends them the current, global model; {\em(ii)} each selected client locally trains its model with its own private data; then, it sends the resulting local model to the server;\footnote{Whenever we refer to global/local model, we mean global/local model {\em parameters}.} {\em(iii)} the server updates the global model by computing an \emph{aggregation function}, usually the average (FedAvg), on the local models received from clients.
% \begin{enumerate}
%     \item[{\em(i)}] the server sends the current, global model to the clients and appoints some of them for training;
%     \item[{\em(ii)}] each selected client locally trains its copy of the global model with its own private data; then, it sends the resulting local model back to the server;\footnote{Whenever we refer to global/local model, we mean global/local model {\em parameters}.}
%     \item[{\em(iii)}] the server updates the global model by computing an \emph{aggregation function} on the local models received from clients (by default, the average, also referred to as FedAvg~\cite{mcmahan2017aistats}).
% \end{enumerate}
This process goes on until the global model converges. %(e.g., after a certain number of rounds or other similar stopping criteria).
%\\
% The advantages of FL over the traditional, centralized learning paradigm are undoubtedly clear in terms of flexibility/scalability (clients can join/disconnect from the FL network dynamically), network communications (only model weights\footnote{We will use \textit{parameters} and \textit{weights} interchangeably.} are exchanged between clients and server), and privacy (each client's private training data is kept local at the client's end and not uploaded to the server).
\\
% Security threats to FL
%However, the growing adoption of FL also raises security concerns~\cite{costa2022covert}, particularly about its confidentiality, integrity, and availability.
Although its advantages over standard ML, FL also raises security concerns~\cite{costa2022covert}. %, particularly about its confidentiality, integrity, and availability~\cite{costa2022covert}.
% OLD, LONG VERSION
% Indeed, some work deals with privacy leakage that may expose the local data of some clients~\cite{melis2019sp}. 
% A large body of work, instead, investigates attacks that usually aim to detriment the predictive accuracy of the learned global model. For instance, \emph{data poisoning} attacks achieve this goal by letting an adversary pollute the training set of some corrupt FL clients with maliciously crafted examples~\cite{jagielski2018sp}.
% Similarly, in \emph{model poisoning} the attacker attempts to tweak the global model weights~\cite{bhagoji2019pmlr} by directly perturbing the local model's weights of some infected FL clients before these are sent to the central server for aggregation, usually via so-called Byzantine attacks. 
% It turns out that Byzantine model poisoning attacks severely impact standard FedAvg; therefore, more robust aggregation functions must be designed to make FL systems secure.
Here, we focus on \emph{untargeted model poisoning} attacks~\cite{bhagoji2019pmlr}, where an adversary attempts to tweak the global model weights %\footnote{We will use the terms \textit{parameters} and \textit{weights} interchangeably.} 
by directly perturbing the local model's parameters of some infected clients before these are sent to the central server for aggregation.
In doing so, the adversary aims to jeopardize the global model \textit{indiscriminately} at inference time.
Such model poisoning attacks severely impact standard FedAvg; therefore, more robust aggregation functions must be designed to secure FL systems.
\\
% In this paper, we focus on designing a novel robust aggregation scheme at the server's end to contrast the effect of Byzantine model poisoning attacks.
%
% Current countermeasures and their limitations
%Several countermeasures have been proposed in the literature to combat model poisoning attacks on FL systems.
% Some methods use simple statistics more robust than plain average to smooth the impact of malicious updates (e.g., Trimmed Mean and FedMedian~\cite{yin2018icml}). 
% Other defenses implement outlier detection techniques to discard malicious updates from the aggregation performed at the server's end. Those are either based on heuristics (e.g., Krum/Multi-Krum~\cite{blanchard2017nips} and Bulyan~\cite{mhamdi2018pmlr}) or data-driven approaches (e.g., K-means clustering~\cite{shen2016acm} or DnC via spectral analysis~\cite{shejwalkar2021ndss}). 
% Finally, some strategies rely on a centralized ``source of trust'' to spot potential malicious updates (e.g., FLTrust~\cite{cao2020fltrust}).
% Several countermeasures have been proposed in the literature to combat model poisoning attacks on FL systems, i.e., to discard possible malicious local updates from the aggregation performed at the server's end. 
% These techniques range from simple statistics more robust than plain average (e.g., Trimmed Mean and FedMedian~\cite{yin2018icml}) to outlier detection heuristics (e.g., Krum/Multi-Krum~\cite{blanchard2017nips} and Bulyan~\cite{mhamdi2018pmlr}) or data-driven approaches (e.g., spectral analysis via K-means clustering~\cite{shen2016acm} or spectral analysis), or methods based on ``source of trust'' (e.g., FLTrust~\cite{cao2020fltrust}).
% OLD, LONG VERSION
%Several countermeasures have been proposed in the literature to combat Byzantine model poisoning attacks on FL systems.
% Descriptive statistics
% For example, Trimmed Mean and FedMedian aggregate local model updates using more robust statistics than standard average~\cite{yin2018icml}.
%
% % Heuristics for outlier detection
% Many existing Byzantine-resilient strategies implement some outlier detection heuristics to discard the model updates sent by potentially malicious clients from the input of the aggregation function.
% One of the most popular heuristics is Krum~\cite{blanchard2017nips}.
% This strategy tries to mitigate the impact of Byzantine attacks by selecting as a global model the local model with the smallest sum of Euclidean distances to {\em all} the other local models.
% Although powerful, Krum requires the server to know (or, at least, estimate) the number of malicious FL clients upfront, which is generally impossible in a realistic attack scenario. %
% Moreover, Krum may become ineffective for complex, high-dimensional model parameter spaces due to the curse of dimensionality.
% Bulyan~\cite{mhamdi2018pmlr} tries to overcome this issue by combining Krum with a variant of Trimmed Mean.
% % Data-driven outlier detection
% Other strategies use data-driven outlier detection techniques -- e.g., via K-means clustering~\cite{shen2016acm} -- to spot potential malicious local model updates. 
% %For instance, Shen et al. propose to cluster local model updates with K-means and thus identify outliers.
%
% % Other techniques
% As far as the server is concerned, any local model received can be from a potential malicious client. 
% FLTrust~\cite{cao2020fltrust} assumes the server acts as a client, i.e., trains a local model on an additional {\em trustworthy} dataset at the server's end and compares it against all the local models from other clients. 
% This way, the server can rely on some ``source of trust'' when discarding potentially malicious clients.
%\\
% Limitations of existing Byzantine-resilient strategies
Unfortunately, existing defense mechanisms either rely on simple heuristics (e.g., Trimmed Mean and FedMedian by~\cite{yin2018icml}) or need strong and unrealistic assumptions to work effectively (e.g., foreknowledge or estimation of the number of malicious clients in the FL system, as for Krum/Multi-Krum~\cite{blanchard2017nips} and Bulyan~\cite{mhamdi2018pmlr}, which, however, cannot exceed a fixed threshold).
Furthermore, outlier detection methods using K-means clustering~\cite{shen2016acm} or spectral analysis like DnC~\cite{shejwalkar2021ndss} do not directly consider the temporal evolution of local model updates received.
Finally, strategies like FLTrust~\cite{cao2020fltrust} require the server to collect its own dataset and act as a proper client, thereby altering the standard FL protocol.
\\
% OLD, LONG VERSION
% Overall, existing Byzantine-resilient strategies are either simple heuristics (e.g., FedMedian) or, if they are more complex, they rely on strong and unrealistic assumptions to work effectively (e.g., knowing the number of malicious clients in the FL system in advance, as for Krum and alike).
% Furthermore, data-driven outlier detection methods do not consider the temporary evolution of local model updates received (e.g., K-means clustering). 
% Finally, strategies like FLTrust requires the server to collect its own dataset and act as a proper client, thereby altering the standard FL protocol.
%
% Description of the proposed method
This work introduces a novel pre-aggregation \textit{filter} robust to untargeted model poisoning attacks. Notably, this filter $(i)$ operates without requiring prior knowledge or constraints on the number of malicious clients and $(ii)$ inherently integrates temporal dependencies. 
The FL server can employ this filter as a preprocessing step before applying \textit{any} aggregation function, be it standard like FedAvg or robust like Krum or Bulyan.
Specifically, we formulate the problem of identifying corrupted updates as a multidimensional (i.e., matrix-valued) time series anomaly detection task. 
The key idea is that legitimate local updates, resulting from well-calibrated iterative procedures like stochastic gradient descent (SGD) with an appropriate learning rate, show \textit{higher predictability} compared to malicious updates. This hypothesis stems from the fact that the sequence of gradients (thus, model parameters) observed during legitimate training exhibit regular patterns, as validated in Section~\ref{subsec:intuition}. %until convergence. 
%This regularity may be more pronounced for smooth convex loss functions, but it can still be captured within an appropriate time window, even for more complex and convoluted loss surfaces. 
%We provide evidence of this claim in Appendix~B, where we show that the average mutual information (i.e., ``predictability''), calculated over pairs of legitimate model updates sent at different FL rounds, is significantly higher than the corresponding computation for a malicious client.
\\
Inspired by the matrix autoregressive (MAR) framework for multidimensional time series forecasting~\cite{chen2021je}, we propose the FLANDERS ({\em \textbf{F}ederated \textbf{L}earning meets \textbf{AN}omaly \textbf{DE}tection for a \textbf{R}obust and \textbf{S}ecure}) filter.
The main advantages of FLANDERS over existing strategies like FLDetector~\cite{zhao2020multivariate} are its resilience to large-scale attacks, where $50\%$ or more FL participants are hostile, and the capability of working under realistic non-iid scenarios.
We attribute such a capability to two key factors: $(i)$ FLANDERS works without knowing a priori the ratio of corrupted clients, and $(ii)$ it embodies temporal dependencies between intra- and inter-client updates, quickly recognizing local model drifts caused by evil players. Below, we summarize our main contributions:

\begin{itemize}
\item[{\em(i)}]
We provide empirical evidence that the sequence of models sent by legitimate clients is more predictable than those of malicious participants performing untargeted model poisoning attacks.
\\
\item[{\em(ii)}] 
We introduce FLANDERS, the first pre-aggregation filter for FL robust to untargeted model poisoning based on multidimensional time series anomaly detection.
\\
\item[{\em(iii)}] 
We integrate FLANDERS into Flower,\footnote{\scriptsize{\url{https://flower.dev/}}} a popular FL simulation framework for reproducibility.
\\
\item[{\em(iv)}] 
We show that FLANDERS improves the robustness of the existing aggregation methods under multiple settings: different datasets, client's data distribution (non-iid), models, and attack scenarios.
\\
\item[{\em(v)}] 
We publicly release all the implementation code of FLANDERS along with our experiments.\footnote{\scriptsize{\url{https://anonymous.4open.science/r/flanders_exp-7EEB}}}
\end{itemize}

% Paper's structure and organization
The remainder of the paper is structured as follows. %some related work and the current state-of-the-art solutions to security issues that FL entails. 
Section~\ref{sec:background} covers background and preliminaries. 
In Section~\ref{sec:related}, we discuss related work.
Section~\ref{sec:problem} and Section~\ref{sec:method} describe the problem formulation and the method proposed. % to tackle it. 
Section~\ref{sec:experiments} gathers experimental results. %, and Section~\ref{sec:limitations} discusses some limitations of this work.
Finally, we conclude in Section~\ref{sec:conclusion}.
 %discusses the limitations of this work and draws future research directions.
%reports conclusions and draws perspectives for future research directions.

%%%%%%% OLD %%%%%%%
%to overcome the resilience of Byzantine failures in distributed Stochastic Gradient Descent computations. 
% The strength of Krum is its time complexity, which is linear in the gradient dimension. 
% However, the robustness of the approach is guaranteed for gradient-based learning applications only when the majority of the clients are not compromised. 
% Besides, the aggregation mechanism of Krum, as well as that of similar methods, is robust from a coarse-grained perspective and does not provide solutions to errors and perturbations that may occur at inference time.
%A related approach to~\cite{blanchard2017nips} is the work of Su et al.~\cite{su2016dc}. Here, the authors propose an iterated approximate agreement to tackle a multi-layer scenario attacked by Byzantine agents. 
%However, the method works efficiently on the sole discrete context and it is inapplicable to continuous state environments.
%\gabri{Maybe, we should just talk about the main limitations of existing countermeasures without digging into their details (or, we can just mention Krum as this is the most popular one). I will move the description of all these methods to the Related Work section.}

\section{System Model} \label{sec:model}
\change{
This section introduces a simplified communication model involving an \gls{ris}. We focus on the \gls{ul} scenario depicted in Fig.~\ref{fig:system-model:scenario}, which comprises a single-antenna \gls{bs}, a single-antenna \gls{ue}, and a solely reflective \gls{ris}~\cite{Tsinghua_RIS_Tutorial}. The primary aim of this system is to facilitate efficient data communication between the \gls{ue} and the \gls{bs} with the support of the \gls{ris}. To accomplish this, the \gls{bs} establishes control signaling links with both the \gls{ris} and the \gls{ue}. The details and the key assumptions are given below. % To achieve this, the \gls{bs} establishes control signaling connections with both the \gls{ris} and the \gls{ue}, which are further elaborated below where important assumptions are emphasized..
}

\begin{figure*}
    \centering
    \begin{subfigure}{0.4\textwidth}
        \centering
        \includegraphics[width=5cm]{figs/ris-control-scenario.pdf}    
        \caption{Control flow.}
    \end{subfigure}  
    \begin{subfigure}{0.4\textwidth}
        \centering
        \includegraphics[width=5cm]{figs/ris-data-scenario.pdf}    
        \caption{Data flow.}
    \end{subfigure}
    \caption{Setup of interest: an \gls{ris} extends the coverage of the \gls{bs} which has a blocked link to the \gls{ue}. During control signaling, the \gls{risc} loads a \glsfirst{ctrl} configuration to the \gls{ris} to deliver low-rate control packets to the \gls{ue}. During data transmission, the \gls{bs} controls and communicates with the \gls{risc} to load a certain configuration to achieve a desired communication performance.}    
    \label{fig:system-model:scenario}
    \vspace{-0.5cm}
\end{figure*}

\paragraph{RIS operation model}
\change{
The internal operations of the \gls{ris} are divided into the \gls{ris} panel and the \gls{risc}. The panel comprises $N$ elements equally spaced on a planar surface. The surface is solely reflective, implying that each element only controls the reflection properties of the incoming waves and that the \gls{ris} cannot process any of them. In particular, we focus on configuring the phases of the elements to change the reflection angle of an incoming wave, where their phase shifts are denoted as $\varphi_n \in [0,2\pi]$, $\forall n\in\mc{N}= \{1,\dots,N\}$\footnote{For the sake of simplicity and following the standard practice in literature, we consider an ideal \gls{ris} to show the theoretical performance achievable by the system at hand. We expect that more realistic models addressing attenuation, mutual coupling, quantization, and non-linear effects would reduce the overall performance~\cite{Tsinghua_RIS_Tutorial}.}. We denote as $\bm{\phi}= [e^{j\varphi_1}, \dots, e^{j\varphi_N}]\T\in \mathbb{C}^{N}$ the vector representing a particular \emph{configuration}, \emph{i.e.}, the set of phase shifts configured at the metasurface's elements at a given time. The \gls{risc} is in charge of loading different configurations to the \gls{ris} surface. Without loss of generality, we indicate as $\tau_s\in\mathbb{R}_+$ the time needed to switch to a new configuration. Moreover, the \gls{risc} is equipped with a communication module, which is used to exchange control signals with the \gls{bs}. We refer to the link connecting the \gls{risc} to the \gls{bs} as the \gls{ris}-\gls{cc}. We consider that the \gls{risc} stores a look-up table containing a set of predefined configurations, namely the \emph{common codebook} of configurations, $\mc{C}=\{\bm{\phi}_1,\dots,\bm{\phi}_C\}$, $|\mc{C}| = C$, that can be designed according to the task at hand; a copy of $\mc{C}$ is stored in the \gls{bs} as well. The control signals between the \gls{bs} and the \gls{risc} can then be based on indexing elements of $\mathcal{C}$ or by explicitly sending a configuration $\bm{\phi}$, if the \gls{bs} wants to load a configuration not present in the common codebook, \emph{i.e.}, $\bm{\phi}\notin\mc{C}$. In the next sections, $\mc{C}$ is going to be defined more explicitly according to \gls{oce} and \gls{bsw} paradigms.
}

\paragraph{UE communication model}\label{sec:system-model:ue-model}
\change{
To analyze the communication performance, we assume a frame-based fixed time system of $\tau\in\mathbb{R}_{+}$ seconds . The frame is further divided into phases that organize the signals exchanged at a given time. Within a frame, control and data information are exchanged in different phases. In each frame, the \gls{ue} communicate with the \gls{bs} through the \gls{ris}: it transmits payload data using the \gls{ue}-\gls{dc}, while the exchange of control messages uses the \gls{ue}-\gls{cc}. 
% The primary objective of the \gls{ul} setup illustrated in Fig.~\ref{fig:system-model:scenario} is to ensure effective communication between the \gls{ue} and the \gls{bs} with the assistance of the \gls{ris}, where control information is transmitted between the \gls{bs} and the \gls{ue} via the \gls{ue}-\gls{cc}, while the \gls{ue} sends payload information through the \gls{ue}-\gls{dc}, both transverses through the \gls{ris}. 
To ensure mathematical tractability, ideal timing and frequency synchronization among devices at the \gls{phy} layer is assumed. Perfect synchronization is also considered at the frame level across these devices. Also, we consider that the \gls{ris} can be configured at any time within a frame, and we make the following assumption about the behavior of the \gls{ris} when the \gls{bs} and the \gls{ue} exchange control information.
}

\begin{assumption}[\gls{ris} ctrl configuration]
\label{assu:ctrl}
   \change{To allow control signaling exchange between the \gls{bs} and the \gls{ue}, we consider that the \gls{ris} loads a \emph{\gls{ctrl} configuration} that ensures that control messages traveling through the \gls{ue}-\gls{cc} reach the destination\footnote{The design of the \gls{ctrl} configuration is out of the scope of this paper; potential candidates for \gls{ctrl} configurations are wide-width beams and hierarchical radiation patterns, which generally offer good reliability and coverage when low data rates are needed~\cite{RISBroadCoverage,alexandropoulos2022hierarchical}.}. Without loss of generality, we assume that the \gls{risc} loads the \gls{ctrl} configuration when in idle, \emph{i.e.}, anytime it does not receive any explicit command from the \gls{bs}.}
\end{assumption}
%\noindent Further details about the frame design are given in Section~\ref{sec:paradigms}.

%Herein, we do not focus on designing the \gls{ctrl} configuration, whose design can be based on other works (\emph{e.g.}, see the configuration design proposed in. 

\paragraph{Channel models}
\change{We adopt the block fading model at a frame level, implying that the channel conditions remain constant throughout the frame duration. Remark that this binds the frame duration $\tau$ to the channel coherence time: the higher the coherence time, the longer the frame duration. The following assumption is made on the channels, further described in the sequel.}

\begin{assumption}[Narrowband channels]
\label{assu:narrowband}
   \change{To allow a simple analysis of both control and data channels, the three channels under consideration -- \emph{1}) the \gls{ue}-\gls{dc}, \emph{2}) the \gls{ue}-\gls{cc}, and \emph{3}) the \gls{ris}-\gls{cc} -- are considered to be narrowband.\footnote{The analysis can be straightforwardly extended to a wideband transmission, as discussed in Section~\ref{sec:extension}.}}
\end{assumption}
% We model the three channels under consideration -- \emph{1}) the \gls{ue}-\gls{dc}, \emph{2}) the \gls{ue}-\gls{cc}, and \emph{3}) the \gls{ris}-\gls{cc} -- based on the narrowband assumption\footnote{The narrowband assumption is considered to simplify the analysis done throughout the paper, allowing us to investigate to successfully perform a \gls{ris}-aided wireless transmission, as specified in Sects.~\ref{sec:paradigms} and~\ref{sec:ris-control}. Nevertheless, the analysis can be straightforwardly extended to a wideband or \gls{ofdm} cases.}. %\vc{In particular, they are modeled as follows when considering a given frame.}

%To study the impact of the control signals on communication, we focus on characterizing three narrowband\footnote{The narrowband assumption of the channel is considered to simplify the analysis done throughout the paper so that we can analyze the control and data channels to successfully perform a \gls{ris}-aided wireless transmission, as specified in Sects.~\ref{sec:paradigms} and~\ref{sec:ris-control}. Nevertheless, the system aspects of the \gls{cc} can be easily extended to a wideband or \gls{ofdm} case.} wireless channels: \emph{a}) the \gls{ue}-\gls{dc}, where the \gls{ue} sends payload data to the \gls{bs}, \emph{b}) the \gls{ue}-\gls{cc}, in which the \gls{bs} and the \gls{ue} can share control messages to coordinate their communication, and \emph{c}) the \gls{ris}-\gls{cc} that connects the \gls{bs} to the \gls{risc} so that the former can control the operation of the latter. Fig.~\ref{fig:system-model:scenario} illustrates the channels further detailed below.

\subsubsection{UE-DC} 
This channel operates at a central frequency $f_d$ with a bandwidth of $B_d$. The \gls{ul} \gls{snr} can be calculated as: 
\begin{equation} \label{eq:snr:uedc}
    \gamma = \frac{\rho_u}{\sigma_b^2} |\bm{\phi}\T  (\mb{h}_d \odot \mb{g}_d)|^2 = \frac{\rho_u}{\sigma_b^2} |\bm{\phi}\T \mb{z}_d|^2,
\end{equation}
where $\mb{h}_d\in\mathbb{C}^N$ and $\mb{g}_d\in\mathbb{C}^N$ are the gains of \gls{ue}-\gls{ris} and \gls{ris}-\gls{bs} links, respectively. We further define the equivalent \gls{dc} as $\mb{z}_d = \mb{h}_d \odot \mb{g}_d \in \mathbb{C}^{N}$. The \gls{ue} transmit power is $\rho_u$, and $\sigma_b^2$ is the noise power at the \gls{bs} \gls{rf} chain\footnote{In the remainder of the paper, we assume that the \gls{bs} knows the transmit and noise powers denoted through this section, \emph{i.e.}, $\rho_u$, $\rho_b$, $\sigma_r^2$, $\sigma_u^2$ and $\sigma_b^2$: the transmit powers are usually determined by the protocol or set by the \gls{bs} itself; the noise powers can be considered static for a long time horizon and hence estimated previously through standard estimation techniques, \emph{e.g.},~\cite{Yucek2006noise}.}. \change{The configurations $\bm{\phi}$ supporting the \gls{ue}-\gls{dc} are subject to design and will be further specified in the following sections.}

\subsubsection{UE-CC} 
This channel operates at central frequency $f_u$ with a bandwidth of $B_u$. %As specified in Subsection \ref{sec:system-model:ue-model}, we assume that the control and payload information are exchanged in different times. Therefore, 
The \gls{ue}-\gls{cc} channel is defined as
\begin{equation} \label{eq:channel:uecc}
    h_{cu} = \bm{\phi}\T_\mathrm{ctrl} (\mb{h}_c \odot \mb{g}_c) = \bm{\phi}\T_\mathrm{ctrl} \mb{z}_c,
\end{equation}
where $\bm{\phi}_\mathrm{ctrl}$ is the \gls{ctrl} configuration (see Assumption~\ref{assu:ctrl}) and $\mb{h}_c\in\mathbb{C}^N$ and $\mb{g}_c\in\mathbb{C}^N$ are the gains of the \gls{ue}-\gls{ris} and \gls{ris}-\gls{bs} links, respectively. The equivalent end-to-end channel is $\mb{z}_c = \mb{h}_c \odot \mb{g}_c \in \mathbb{C}^N$. We consider a worst-case scenario where the channel~\eqref{eq:channel:uecc} has no \gls{los} component, \emph{i.e.}, it is distributed as $h_{cu} \sim \mc{CN}(0, \tilde{\lambda}_u)$; $\tilde{\lambda}_u$ is a term accounting for the large-scale fading -- known by the \gls{bs} -- which is dependent on the \gls{ctrl} configuration design. Hence, the instantaneous \gls{snr} measured at the \gls{ue} is
\begin{equation} \label{eq:snr:uecc}
    \Gamma_{u} = \frac{\rho_b}{\sigma_u^2} |h_{cu}|^2 \sim \text{Exp}\left(\frac{1}{\lambda_u}\right), 
\end{equation}
where $\lambda_u = \frac{\rho_b \tilde{\lambda}_u}{\sigma^2_u}$ denotes the average \gls{snr} at the \gls{ue}, being $\rho_b$ the \gls{bs} transmit power and $\sigma^2_u$ the \gls{ue}'s \gls{rf} chain noise power. 
\change{We make the following assumption on this channel.}
\begin{assumption} [\gls{ue}-\gls{cc} design]
    \label{assu:ue-cc}
    \change{
    We assume that the \gls{ue}-\gls{cc} operates as an \gls{ibcc}, meaning that the frequency $f_u$ matches the frequency $f_d$, and the physical resources allocated for the \gls{ue}-\gls{cc} coincide with those utilized for the \gls{ue}-\gls{dc}. This assumption is based on the premise that any \gls{ue}-\gls{cc} signal must travel through the \gls{ris} and the \gls{ris} operates at the same data frequency and bandwidth.
    }
\end{assumption}

%\colr{\textbf{WHY do we make this assumption?}}

\subsubsection{RIS-CC} 
This narrowband channel operates on central frequency $f_r$ with bandwidth $B_r$. Let $h_{cr}\in\mathbb{C}$ denote the channel coefficient of the \gls{ris}-\gls{cc}. To obtain simple analytical results, we assume that $h_{cr} \sim \mc{CN}(0, \tilde{\lambda}_r)$, where $\tilde{\lambda}_r$ accounts for the large-scale fading, assumed known by the \gls{bs}. Hence, the instantaneous \gls{snr} measured at the \gls{risc} is
\begin{equation} 
    \Gamma_{r} = \frac{\rho_b}{\sigma_r^2} |h_{cr}|^2 \sim \text{Exp}\left(\frac{1}{\lambda_r}\right),
    \label{eq:snr:riscc}
\end{equation}
where $\lambda_r = \frac{\rho_b \tilde{\lambda}_r}{\sigma_r^2}$ denotes the average \gls{snr} with $\sigma^2_r$ being the noise power at the \gls{risc} \gls{rf} chain. 
\change{We make the following assumption on the \gls{ris}-\gls{cc}.}
\begin{assumption}[\gls{ris}-\gls{cc} design]
    \label{assu:ris-cc}
    \change{The \gls{ris}-\gls{cc} can either be: $i$) \gls{ibcc}, implying that the physical resources employed by this channel are overlapping with the one used by the \gls{ue}-\gls{dc}, \emph{i.e.}, $f_r = f_d$;  or $ii$) \gls{obcc}, where the physical resources are orthogonal, \emph{e.g.}, simulating a wired connection between the \gls{bs} and the \gls{risc}. In the case of \gls{obcc}, we further assume that the \gls{ris}-\gls{cc} is an error-free channel, \emph{i.e.}, $\lambda_r\rightarrow\infty$, with feedback capabilities since the system designer can easily make the \gls{ris}-\gls{cc} as reliable as possible.}     
\end{assumption}

%\colr{\textbf{WHY do we make this assumption?}}

% while the \gls{bs} waits for control messages from the \gls{ue} it commands the \gls{ris} to load a wide beam-width configuration, namely the \emph{control configuration}, (\emph{e.g.}, by using the design proposed in~\cite{alexandropoulos2022hierarchical}).
% Being interested in studying the impact of the \gls{ris}-\gls{cc}, we assume that the \gls{ue}-\gls{cc} is error-free as long as the control configuration is loaded correctly, based on the fact that \gls{ue} control messages employs a very low communication rate.\footnote{The impact of the UE-CC errors will be studied in a future work.}

% \FS{With Kyriakos, we discuss about this and realize that at least for the first packet is very difficult to make the previous assumption work. Therefore, we may define a control configuration $\bm{\theta}$ that provide an overall channel
% \begin{equation}
%     h_{cu} = \bm{\theta}\T \mb{z}_c
% \end{equation}
% where $\mb{z}_c = \mb{h}_c \circ \mb{g}_c$ is the equivalent end-to-end channel.  Then we can even assume that $h_{cu} \sim \mc{CN}(0, \lambda_u(\bm{\theta}))$ where $\lambda_u(\bm{\theta})$ is a term that account for the large scale fading + the impact of the control configuration. 
% }






\section{RIS-Aided Communication Paradigms} \label{sec:paradigms}
\change{In this section, we first describe the structure and building blocks of a generic \gls{ris}-enabled communication paradigm. We use this to instantiate two particular transmission strategies: the \gls{oce} and the \gls{bsw}, representing the multiplexing and diversity paradigms discussed in Section~\ref{sec:intro:contributions}}.
\change{We analyze their performance in terms of the expected \gls{snr} and \gls{se}, while specifying the errors eventually occurring during their operations.}

\subsection{Generic structure}
\change{In a system without \glspl{ris}, a generic structure for a typical transmission strategy can be divided into three key phases occurring in every frame: ``\emph{Signaling},'' ``\emph{Algorithmic},'' and ``\emph{Payload}.'' The Signaling phase encompasses the actions conducted on the \glspl{cc} required for network node control. This phase relies on the quality of the \glspl{cc} and the information content within the control messages. The Algorithmic phase involves operations aiming at optimizing transmission parameters, such as selecting the transmit \gls{se}. The specifics of this phase are contingent on the chosen communication paradigm. The Payload phase handles the actual data transmission.}

\change{In an \gls{ris}-aided system, we identify that a generic structure would have two Signaling phases, namely ``\emph{Initialization}'' and ``\emph{Setup},'' in conjunction with the ``\emph{Algorithmic}'' and ``\emph{Payload}'' phases. Sequentially, we have: Initialization, Algorithmic, Setup and Payload}\footnote{We note that there could be communication strategies in which some of these phases may not be present, \emph{e.g.}, access procedures; however, the mentioned four phases set a basis for a sufficiently general framework for \gls{ris}-aided communication that can be used, in principle, to design other schemes where some of the steps are merged or omitted, as discussed in Section~\ref{sec:extension}.}. \change{The time of a frame is thus divided as $\tau=\tau\ini+\tau\alg+\tau\set+\tau\pay$, where $\tau_i < \tau$, with $i\in\{{\rm ini},{\rm alg},{\rm set},{\rm pay}\}$, is the time of the corresponding phase. The duration of each phase can vary depending on the paradigm and the kind of \glspl{cc}.} The phases are further elaborated as follows.
% Within each frame, we divide a generic \gls{ris}-aided communication paradigm into four (sequential) phases, namely , \emph{``Algorithmic''}, \emph{``Acknowledgement''}, and \emph{``Payload''}. Initialization and Setup are \emph{signaling} phases, \emph{i.e.}, phases having the main purpose of exchanging control information, and hence mainly dependent on the available \glspl{cc}.   % \vc{These phases occur in sequential order as given and are further elaborated below}

% Before proceeding in describing the two possible implementations of the protocol, we illustrate a simple mechanism for the detection of the end of the channel coherence frame. As stated before, we assume that the channel changes in a negligibly during the frame duration. Therefore, the RIS configuration and the user's transmission power are the only factors that affect the received \gls{snr} at the \gls{bs}. In the payload phase, the \gls{ris} configuration does not change and the \gls{bs} is assumed to know the average user's transmit power. Therefore, a running average of the received \gls{snr} can be kept by the \gls{bs}. When the running average is observed to differ substantially from the average \gls{snr} at the beginning of the payload phase, it is safe to assume that the channel coherence frame is at its end. At this point, the \gls{bs} re-sends the control information initializing the Initialization phase of the next frame.

\paragraph{Initialization} 
The \gls{bs} notifies the \gls{ris} and \gls{ue} about the beginning of a frame over the \glspl{cc}. It is assumed that the \gls{risc} loads the \gls{ctrl} configuration at the start of this phase. %The duration of this phase is denoted with $\tau\set < \tau$ and depends on the type of \glspl{cc}. 
\change{Although not considered here, the Initialization phase can also incorporate a random access procedure (see, \emph{e.g.},~\cite{Croisfelt2023}) as an intermediate step where newly connected \glspl{ue} are scheduled.}

\paragraph{Algorithmic} 
\change{
This phase encompasses all the processes and computations needed to optimize the subsequent \change{Payload phase}, where the actual data transmission takes place. % The algorithmic phase has a $\tau\alg < \tau$ duration that depends on the choice of the employed communication paradigm. 
Objectives of this phase encompass the selection/optimization of the appropriate \gls{ris} configuration(s), while others, such as determining the transmission parameters for the \gls{ue} and/or the \gls{bs}, could be included. To tackle these objectives, some form of wireless environment sensing from the end nodes is required, typically enabled by the transmission of pilot sequences, whose specifications -- their number, waveform, \gls{ul} or \gls{dl} transmission, etc -- are defined by the transmission strategy. The \gls{bs} can then use the collected pilot signals and invoke pre-defined algorithms to fulfil the objectives. The outcome of these algorithms might be affected by different types of \emph{algorithmic errors} that may prevent the system from performing as expected, and thus, should be considered when analyzing the overall performance.
}

\paragraph{Setup}
\change{
%The Setup phase starts once the Algorithmic phase ends; 
During this phase, the \gls{ris} configuration chosen during the Algorithmic phase needs to be communicated to the \gls{risc}, which in turn commands the \gls{ris} to load it. Additionally, further control signaling may occur between the \gls{bs} and the \gls{ue} as a final check before data transmission to, e.g., set the \gls{mcs}.} 
%It is implied that the \gls{risc} loads the \gls{ctrl} configuration at the beginning of this phase. 
%The acknowledgment phase duration $\tau\set < \tau$, depends on the type of \gls{cc} used.

\paragraph{Payload}
\change{Here, the actual data transmission takes place while the \gls{ris} loads the configuration specified before. This phase may or may not include feedback of the data sent by the \gls{ue} at the end; this aspect is not considered in this paper. The communication performance of the considered \gls{ris}-enabled communication system is measured during this phase.}

\change{In the following subsections, we describe two state-of-the-art paradigms using the generic structure defined above. After a general description, we investigate their Algorithmic phases, analyzing their performance and possible errors.} 
The first paradigm is the \emph{\gls{oce}}, which follows a multiplexing transmission: the \gls{ue}'s \gls{csi} is evaluated at the \gls{bs} and then exploited to compute the \gls{ris}' optimal configuration and the corresponding achievable data rate. \change{Then, the transmission is made using the optimized configuration and \gls{mcs}.} %\gls{risc} loads the optimal configuration to the \gls{ris}, and the \gls{ue} transmit the data using the chosen \gls{mcs}. 
The second approach is the \emph{\gls{bsw}}, defined as a communication paradigm in~\cite{An2022}, but already used in previous works (\emph{e.g.},~\cite{Jamali2022, alexandropoulos2022hierarchical, Croisfelt2022randomaccess}). \change{This paradigm resembles the concept of diversity transmission: the \gls{bs} selects a target \gls{kpi} \emph{a priori}; then, it instructs the \gls{risc} to sweep through a set of predefined configurations, expecting that at least one will satisfy the target \gls{kpi}; then, the transmission is made using the chosen configuration.}
%the \gls{bs} does not spend time figuring out the optimal configuration; 
% to improve the quality of the \gls{ue}-\gls{dc} and does not tune the transmission rate; 
%it instead applies a best-effort strategy.} 
% Specifically, the \gls{bs} instructs the \gls{risc} to sweep through a set of predefined configurations, expecting that at least one will satisfy a target \gls{kpi} specified \emph{a priori}. %for the transmission (\emph{e.g.}, a minimum \gls{snr} to support a predefined rate). 
Fig.~\ref{fig:RIS-frames} shows the data exchange diagrams of the two paradigms, comprised of \gls{cc} messages, configuration loading, processing operations, and data transmission. % Based on these, 
% A detailed description of the two paradigms is given in the following, while the impact of the control signaling is investigated in Sect.~\ref{sec:ris-control}.

\begin{figure*}[!t]
    \centering
    \begin{subfigure}[t]{0.45\linewidth}
        \centering
        \includegraphics[width=0.9\textwidth]{figs/frame-oce.pdf}
        \caption{\gls{oce} paradigm.}
        \label{fig:RIS-oce}
    \end{subfigure}
    \hfill
    \begin{subfigure}[t]{0.45\linewidth}
        \centering
        \includegraphics[width=0.9\textwidth]{figs/frame-bsw.pdf}
        \caption{\gls{bsw} paradigm.}
        \label{fig:RIS-bsw}
    \end{subfigure} 
    \caption{Data exchange diagram of the two \gls{ris}-enabled communication paradigms. Signals traveling through \gls{ris}-\gls{cc}, \gls{ue}-\gls{cc}, and \gls{ue}-\gls{dc} are represented by solid \textcolor{red}{red}, solid \textcolor{blue}{blue}, and solid black lines, respectively. \gls{risc} to \gls{ris} commands are indicated with dashed black lines. \gls{bs} operations are in \texttt{monospaced font}.}
    \label{fig:RIS-frames}
\end{figure*}

%%%%%%%%%%%%%%%%%%%%%%%%%%%%%%
\subsection{Optimization based on Channel Estimation (OPT-CE)}\label{sec:communication-paradigms:oce}
%%%%%%%%%%%%%%%%%%%%%%%%%%%%%%
In \gls{oce}, the \gls{bs} needs to obtain the \gls{csi} for the \gls{ue} to optimize the \gls{ris} configuration. The necessary measurements can be collected by transmitting pilot sequences from the \gls{ue}. During the Initialization phase, the \gls{bs} informs the other entities that the procedure is starting. First, the \gls{ue} is informed through the \gls{ue}-\gls{cc} to prepare to send pilots. Second, to solve the indeterminacy of the $N$-path \gls{ce} due to the \gls{ris} presence~\cite{Wang2020}, the \gls{ris} is instructed to sweep through a common codebook of configurations during the Algorithmic phase, called \emph{\gls{ce} codebook} and denoted as $\mc{C}\oce \subseteq \mc{C}$. To change between the configurations in the \gls{ce} codebook, we consider that the \gls{bs} needs to send only a single control message to the \gls{risc} since the \gls{ris} sweeps following the order stipulated by the codebook. During the Algorithmic phase, the \gls{ue} sends replicas of its pilot sequence, subject to different \gls{ris} configurations, to let each of them experience a different propagation environment. After a sufficient number of samples is received, the \gls{bs} estimates the \gls{csi} and compute the optimal \gls{ris}'s configuration~\cite{Tsinghua_RIS_Tutorial}. Then, the Setup phase starts, in which the \gls{bs} uses the the ctrl configuration\footnote{We remark that the \gls{ctrl} configuration is automatically loaded after the Algorithmic phase ends due to the idle state of the \gls{ris}. Another approach is loading the optimal configuration evaluated in the Algorithmic phase also to send the control information toward the \gls{ue}; nevertheless, the \gls{risc} needs to be informed previously by a specific control message by the \gls{bs}. We do not consider this approach to keep the frame structure of the two paradigms similar, thereby simplifying the analysis and the presentation in Section~\ref{sec:ris-control}.} to implement the \gls{ue}-\gls{cc} and instruct the \gls{ue} to start sending data. Subsequently, the \gls{bs} informs the \gls{risc}, over the \gls{ris}-\gls{cc}, to load the optimal configuration. Finally, the Payload phase takes place.


%As we are striving for a general \gls{ce} framework, we consider the strategy proposed by~\cite{Wang2020}, which is simple enough to yield an analytical formulation of the estimation error. The procedure occurs as follows: First, the \gls{ris} elements are turned off, so that the \gls{bs} can get an estimate of the direct channel; second, the \gls{bs} carries out the \gls{ce} for a single, typical \gls{ue}; finally, the other \glspl{ue} transmit pilots and the \gls{bs} estimates their channels by exploiting the fact that they are scaled versions of the typical \gls{ue}'s channel. Due to the aim of the paper, the scenario at hand, and the channel model, we focus only on the second phase, where the \gls{bs} estimates the channel coefficients of a single \gls{ue}. \RK{I think this paragraph can be removed completely. We are introducing something just to say we won't be doing it anyway. If the analytical method for estimating error based on~\cite{Wang2020} is used, we can just reference the paper later}%More specifically, the \gls{bs} aims to estimate $N$ channel coefficients associated with the equivalent channel observed at the \gls{bs} after reflection.
%The replicas of the \gls{ue} pilot sequences are designed to be proportional to the number of \gls{ris} elements $N$ for an accurate \gls{ce}. Also,
%\vc{To the \gls{ce} be able to capture the additional spatial dimension from the \gls{ris}, a}
%Accordingly, the 
%set of configurations $\mc{C}\oce$, also named as \emph{codebook}, must be designed, where %to let this estimation be successful~\cite{He2020}. Note that 
%the knowledge of such a codebook is shared by \gls{risc} and the \gls{bs}, assuming that a setup phase has taken place to deploy the \gls{ris} in the network.

\subsubsection{Performance Analysis}
We now present the \gls{ce} procedure and analyze its performance in connection to the cardinality of the employed codebook $C\oce = |\mc{C}\oce|$. The considered method employed is a simplified version proposed in~\cite{Wang2020}. Let us start with the pilot sequence transmission and its processing. %Every pilot sequence is made up of $p$ complex symbols. Hence, 
We denote a single pilot sequence as $\bm{\psi} \in \mathbb{C}^{p}$, spanning $p$ symbols and having $\lVert \bm{\psi} \rVert^2 = p$. Every time a configuration from the codebook is loaded at the \gls{ris}, the \gls{ue} sends a replica of the sequence $\bm{\psi}$ towards the \gls{bs}. When the configuration $c\in\mc{C}\oce $ is loaded, the following signal is received at the \gls{bs}
\begin{equation} \label{eq:oce:replica}
    \mb{y}_c\T  = \sqrt{\rho_u} \bm{\phi}_c\T \mb{z}_d  \, \bm{\psi}\T + \tilde{\mb{w}}_c\T \in\mathbb{C}^{1 \times p},
\end{equation}
where $\bm{\phi}_c$ denotes the phase profile vector of the configuration $c\in\mc{C}\oce$, $\rho_u$ is the transmit power, and $\tilde{\mb{w}}_c\sim \mc{CN}(0, \sigma_b^2 \mb{I}_p)$ is the \gls{awgn}. The received symbol is then correlated with the pilot sequence and normalized by $\sqrt{\rho_u} p$, yielding
\begin{equation} \label{eq:oce:pilotprocess}
    y_c  = \frac{1}{\sqrt{\rho_u} p}\mb{y}_c\T \bm{\psi}^* = \bm{\phi}_c\T \, \mb{z}_d + w_c \in\mathbb{C},
\end{equation}
where $w_c\sim\mc{CN}(0, \frac{\sigma_b^2}{p \rho_u})$ is the resulting \gls{awgn}.\footnote{The consideration of dividing the pilot transmission over configurations over small blocks of $p$ symbols serves three purposes: $i$) from the hardware point of view, it might be difficult to change the phase-shift profile of an \gls{ris} within the symbol time, $ii$) to reduce the impact of the noise, and $iii$) to have the possibility of separating up to $p$ \gls{ue}'s data streams, if the pilots are designed to be orthogonal to each other~\cite{massivemimobook}.}
% \fs{The noise is reduced with this assumption
% \begin{equation}
% \begin{aligned}
%     \E{w} &= \frac{1}{\sqrt{\rho_u} p} \sum_{i=1}^p \psi_c^* \E{w_c} = 0 \\
%     \E{|w|^2} &= \frac{1}{\rho_u p^2} \E{\sum_{i=1}^p \psi_c^* w_c w_c^* \psi_c} + \frac{1}{\rho_u p} \E{\sum_{i=1}^p \sum_{j \neq i}^p \psi_c^* w_c w_j^* \psi_j} \\
%     & =  \frac{1}{\rho_u p^2} \sum_{i=1}^p \psi_c^* \E{w_c w_c^*} \psi_c + \frac{1}{\rho_u p^2} \sum_{i=1}^p \sum_{j \neq i}^p \psi_c^* \underbrace{\E{w_c w_j^*}}_{\E{w_c}\E{w_j^*} = 0} \psi_j \\
%     &= \frac{\sigma_b^2}{\rho_u p^2} \sum_{i=1}^p \psi_c^* \psi_c =  \frac{\sigma_b^2}{\rho_u p^2} \underbrace{\bm{\psi}\H \bm{\psi}}_{\lVert\bm{\psi}\rVert^2 = p} = \frac{\sigma_b^2}{\rho_u p}.
% \end{aligned}
% \end{equation}
% }
The pilot transmissions and the processing in~\eqref{eq:oce:pilotprocess} are repeated for all \gls{ris} configurations, \emph{i.e.}, $\forall c\in\mc{C}\oce$. The resulting signal $\mb{y} = [y_1, y_2, \dots, y_{C\oce}]\T\in\mathbb{C}^{C\oce}$ can be then compactly written in the following form:
\begin{equation} \label{eq:oce:signal}
    \mb{y}= \mb{\Theta}\T \mb{z}_d + \mb{w},
\end{equation}
where $\bm{\Theta} = [\bm{\phi}_1, \bm{\phi}_2, \dots, \bm{\phi}_{C\oce}]\in\mathbb{C}^{N\times C\oce }$ is the matrix containing all the configurations used and $\mb{w} = [w_1, \dots, w_{C\oce}]\T \in\mathbb{C}^{C\oce}$ is the \gls{awgn} term. For the sake of generality, we will assume that there is no prior information about the channel distribution at the \gls{bs}. Therefore, we cannot estimate separately $\mb{h}_d$ and $\mb{g}_d$, but only the cascade channel $\mb{z}_d$. 
%Indeed, we can rewrite the eq.~\eqref{eq:oce:signal} as
%each component of~\eqref{eq:oce:signal} as
% \begin{equation}
%     y_c = \sum_{n=1}^{N} \phi_{n,c} \underbrace{h^{*}_{n} g_{n}}_{z_{n}} + w = \sum_{n=1}^{N} \phi_{n,c} {z_{n}} + w
% \end{equation}
% and the overall signal is
% \begin{equation} \label{eq:oce:signal2}
    % \mb{y} = \bm{\Theta}\T \mb{z} + \mb{w}
% \end{equation}
% where $\mb{z}=[z_{1},z_{2},\dots,z_{N}]\T = \mb{h} \odot \mb{g}\in\mathbb{C}^{N \times{1}}$ is the equivalent channel. 
It is possible to show that a necessary (but not sufficient) condition to perfectly estimate the channel coefficients is that $C\oce\geq{N}$~\cite{Wang2020}. Indeed, we want a linearly independent set of equations, which can be obtained by constructing the configuration codebook for \gls{ce} to be at least of rank $N$. As an example, we can use the \gls{dft} matrix, \emph{i.e.}, $[\bm{\Theta}]_{n,c} = e^{-j2\pi \frac{(n-1) (c-1)}{C\oce}}$, with $n=\mc{N}$ and $c\in\mc{C}\oce$,
% \begin{equation}
%     \bm{\Theta} =
%     \begin{bmatrix}
%         1 & 1 & 1 & \cdots & 1 \\
%         1 & \omega & \omega^2 & \cdots & \omega^{(C_{\mathrm{OCE}}-1)} \\
%         1 & \omega^2 & \omega^2 & \cdots & \omega^{2(C_{\mathrm{OCE}}-1)} \\
%         \vdots & \vdots & \vdots & \ddots & \vdots \\
%         1 & \omega^{N-1} & \omega^{2(N-1)} & \cdots &\omega^{(N-1)(C_{\mathrm{OCE}}-1)},
%     \end{bmatrix}
% \end{equation}
% where $\omega=e^{-j 2\pi / C_{\mathrm{OCE}}}$ 
with $\bm{\Theta}^* \bm{\Theta}\T = C\oce \mb{I}_{N}$. Considering that the parameter vector of interest is deterministic, the least-squares estimate yields the estimation~\cite{Kay1997}
\begin{equation}
    \hat{\mb{z}}_d=\dfrac{1}{C\oce} \bm{\Theta}^* \mb{y} = \mb{z}_d + \mb{n},
\end{equation}
where $\mb{n}\sim\mc{CN}(0, \frac{\sigma_b^2}{p \rho_u C\oce } \mb{I}_N)$, and whose performance is proportional to $1/C\oce$.
%$ Remark: the higher the cardinality of the codebook, the better the \gls{ce} performance.
% \fs{In fact:
% \begin{equation}
%     \begin{aligned}
%     \E{\frac{1}{C\oce} \bm{\Theta}^* \mb{w}} &= 0 \\
%     \E{(\frac{1}{C\oce} \bm{\Theta}^* \mb{w})( \frac{1}{C\oce} \bm{\Theta}^* \mb{w})\H} &= \frac{1}{C\oce^2} \bm{\Theta}^* \E{\mb{w} \mb{w}\H} \bm{\Theta}\T  = \frac{\sigma_b^2}{\rho_u p C\oce} \mb{I}_N
%     \end{aligned}
% \end{equation}}
%
Based on the estimated equivalent channel, the \gls{bs} can obtain the configuration $\bm{\phi}_\star$ that maximizes the instantaneous \gls{snr} of the typical \gls{ue}, as follows: 
\begin{equation}
    \begin{aligned}
        \bm{\phi}_\star &= \max_{\phi}\left \{ \lvert \bm{\phi}\T \hat{\mb{z}}_d \rvert^2 \, \big| \, \lVert\bm{\phi}\rVert^2 = N \right\},
    \end{aligned}
\end{equation}
which turns out to provide the intuitive setting $(\bm{\phi}_\star)_n=e^{-j\angle{(\hat{\mb{z}}_d)_n}}$, $\forall n \in \mc{N}$. The \gls{ul} estimated \gls{snr} at the \gls{bs} is:
\begin{equation}
    \hat{\gamma}\oce = \frac{\rho_u}{\sigma_b^2} |\bm{\phi}_\star\T \hat{\mb{z}}_d|^2.
\end{equation}
Based on the estimated \gls{snr}, the \gls{se} of the data communication can be adapted to be the maximum achievable, \emph{i.e.},
\begin{equation} \label{eq:oce:se}
    \eta\oce = \log_2(1 + \hat{\gamma}\oce).
\end{equation}

\subsubsection{Algorithmic errors}
For the \gls{oce} paradigm, a communication outage occurs in the case of an \emph{overestimation error}, \emph{i.e.}, if the selected \gls{se} $\eta\oce$ is higher than the actual channel capacity, leading to an unachievable communication rate~\cite{Shannon1949}. The probability of this event is
\begin{equation} \label{eq:oce:ae1}
    p_\mathrm{ae} = \mc{P} \left[\eta\oce = \log(1 + \hat{\gamma}\oce) \ge \log_2\left(1+ \gamma\oce \right) \right],
\end{equation}
where $\gamma\oce = \frac{\rho_u}{\sigma_b^2} |\bm{\phi}_\star\T \mb{z}_d|^2$ is the actual \gls{snr} at the \gls{bs}. Eq.~\eqref{eq:oce:ae1} translates to the condition
\begin{equation} \label{eq:oce:ae2}
    p_\mathrm{ae} = \mc{P}\left[\hat{\gamma}\oce \ge \gamma\oce\right] = \mc{P}\left[ |\bm{\phi}_\star\T \mb{z}_d + \bm{\phi}_\star\T \mb{n}|^2 \ge  |\bm{\phi}_\star\T \mb{z}_d|^2 \right].
\end{equation}
A detailed analysis of~\eqref{eq:oce:ae2} relies on the channel model of $\mb{z}_d$, and thereby on a prior assumption about the scenario (\emph{e.g.},~\cite{RIS_Nakagami}); we therefore numerically evaluate  the impact of the \gls{oce} algorithmic error.

%Nevertheless, we found experimentally that the impact of the noise on the \gls{snr} estimation is generally negligible for this paradigm, as shown in Fig.~\ref{fig:snrcdf} where the \gls{cdf} of the estimated $\hat{\gamma}\oce$ and actual \gls{snr} ${\gamma}\oce$ are presented.  This finding is justified by the fact that the noise power is proportional to $1/C\oce$ where $C\oce \ge N$ as described in Section~\ref{sec:communication-paradigms:oce}. %Therefore, considering the generally high number of \gls{ris} elements deployed, the impact of the noise results is negligible. Because of that, we assume that the algorithmic error for the \gls{oce} paradigm is $p_\mathrm{ae} \approx 0$ in the remainder of the paper.

\subsection{Codebook-Based Beam Sweeping (CB-BSW)}\label{sec:communication-paradigms:bsw}
In \gls{bsw}, the \gls{bs} now does not require explicit \gls{csi} of the \gls{ue}. In the Initialization phase, the \gls{bs} commands the start of a new frame by signaling to the \gls{ris} and \gls{ue}. \change{A \emph{\gls{BSW} process}, \emph{i.e.}, an \gls{ris} configuration selection, is performed during the Algorithmic phase. This process comprises the \gls{ue} sending reference signals, while the \gls{bs} commands the \gls{ris} to change its configuration at regular periods accordingly to a set of predefined configurations, labeled as the \emph{\gls{BSW} codebook} denoted by $\mc{C}\bsw \subseteq \mc{C}$.} The \gls{bs} receives the reference signals that are used to measure \gls{ue}'s \gls{kpi}. The \gls{BSW} process is triggered when a single \gls{bs} control message is received by the \gls{risc}. At its end, the \gls{bs} selects a configuration satisfying the target \gls{kpi}. During the Setup phase, the \gls{bs} informs the \gls{ue} over the \gls{ue}-\gls{cc} to prepare to send data, while the \gls{ris} uses the \gls{ctrl} configuration, and informs the \gls{risc} through the \gls{ris}-\gls{cc} to load the selected configuration. Finally, the Payload phase takes place. 

\begin{remark}[Fixed vs Flexible frames]
    \change{The \gls{BSW} process during the Algorithmic phase may make use of i) a \emph{fixed} or ii) a \emph{flexible} frame structure. The fixed frame ends after a fixed number of \gls{BSW} codebook configurations have been loaded. The flexible frame structure allows stopping the \gls{BSW} as soon as a \gls{kpi} value measured is above the target one. Flexible frame requires on-the-fly \gls{kpi} measurements at the the \gls{bs}, while \gls{ue}-\gls{cc} needs to be reserved to promptly inform the \gls{ue} about the frame termination when the target \gls{kpi} is met, thus modifying the overall frame (see Section~\ref{sec:ris-control}).}
    \label{remark:cb-bsw:fixed-vs-flexible}
\end{remark}

\subsubsection{Performance analysis}
For \gls{bsw} it is necessary to assume that the target \gls{kpi} is a target \gls{snr} $\gamma_0$ measured at the \gls{bs} via the average \gls{rss} metric. In this case, a fixed \gls{se} is considered \emph{a priori}, which is given by
\begin{equation} \label{eq:bsw:se}
    \eta\bsw = \log_2(1 + \gamma_0),
\end{equation}
and the goal is find a configuration from the \gls{ris} codebook that supports it. We analyze the system performance starting from the pilot sequence transmission and processing. As before, every pilot sequence consists of $p$ symbols\footnote{The pilot sequences for \gls{oce} and \gls{bsw} can be different and have different lengths. In practice, they should be designed and optimized for each of those approaches, which is beyond the scope of this paper. We use the same pilot sequence length notation in both paradigms for simplicity.}. Once again, we denote a single sequence as $\bm{\psi} \in \mathbb{C}^{p}$ having $\lVert\bm{\psi} \rVert^2 = p$. After the \gls{ris} loads configuration $c\in\mc{C}\bsw$, the \gls{ue} sends a replica of $\bm{\psi}$ and, similar to~\eqref{eq:oce:replica}, the \gls{bs} receives the signal: 
\begin{equation} \label{eq:bsw:replica}
    \mb{y}_c\T  = \sqrt{\rho_u} \bm{\phi}_c\T \mb{z}_d \bm{\psi}\T + \tilde{\mb{w}}_c\T \in\mathbb{C}^{1 \times p},
\end{equation}
where $\bm{\phi}_c$ denotes the configuration $c \in \mc{C}\bsw$. The received signal is then correlated with $\bm{\psi}$ and normalized by $p$:
\begin{equation} \label{eq:bsw:pilotprocess}
    y_c  = \frac{1}{p}\mb{y}_c\T \bm{\psi}^* = \sqrt{\rho_u} \bm{\phi}_c\T \mb{z}_d + w_c \in\mathbb{C},
\end{equation}
where $w_c\sim\mc{CN}(0, \frac{\sigma_b^2}{p})$ is the resulting \gls{awgn}. 
The \gls{snr} provided by the configuration can be estimated as follows: %by taking the absolute square of the sample and dividing it by the (known) noise variance as
\begin{equation} \label{eq:bsw:gammahat}
    \hat{\gamma}_c = \frac{|y_c|^2}{\sigma_b^2} \hspace{-0.3mm} = \hspace{-0.3mm} \underbrace{\frac{\rho_u}{\sigma_b^2} |\bm{\phi}_c\T \mb{z}_d|^2 }_{\gamma_c} \hspace{-0.1mm}+\hspace{-0.2mm} 2 \Re\hspace{-0.5mm}\left\{ \frac{\sqrt{\rho_u}}{\sigma_b^2} \bm{\phi}_c\T \mb{z}_d\, w_c\right\} \hspace{-0.6mm}+\hspace{-0.5mm} \frac{|w_c|^2}{\sigma_b^2},
\end{equation}
where $|w_c|^2 \sigma_b^{-2} \sim \mathrm{Exp}(p)$. 
It is worth noting that the estimated \gls{snr} is affected by the exponential error generated by the noise, but also by the error of the mixed product between the signal and the noise, whose \gls{pdf} depends on the \gls{pdf} of $\mb{z}_d$. Based on~\eqref{eq:bsw:gammahat}, we can select the best configuration $c^\star\in\mc{C\bsw}$ providing the target \gls{kpi}. According to Remark~\ref{remark:cb-bsw:fixed-vs-flexible}, we next discuss the selection of the configuration for the two different frame structures.

\paragraph{Fixed Frame} 
When the frame has a fixed structure, the \gls{BSW} procedure ends after the \gls{ris} sweeps through the whole codebook. In this case, we can measure the \glspl{kpi} for all available configurations. The configuration selected for the payload phase is the one achieving the highest estimated \gls{snr} among the ones satisfying the target \gls{kpi} $\gamma_0$, as
\begin{equation} \label{eq:bsw:cstar:fixed}
    c^\star = \argmax_{c\in\mc{C}\bsw} \{\hat{\gamma}_c \,|\, \hat{\gamma}_c \ge \gamma_0\}.
\end{equation}
If no configuration achieves the target \gls{kpi}, the communication is not feasible, and we run into an outage event.

\paragraph{Flexible Frame} 
When the frame has a flexible structure, the end of the \gls{BSW} process is triggered by the \gls{bs} when the measured \gls{kpi} reaches the target value. A simple on-the-fly selection method involves testing if the estimated \gls{snr} is greater than the target $\gamma_0$; \emph{i.e.}, after eq.~\eqref{eq:bsw:gammahat} is obtained for configuration $c\in\mc{C}\bsw$, we set
\begin{equation} \label{eq:bsw:cstar:flexible}
    c^\star = c \iff \hat{\gamma}_c \ge \gamma_0.
\end{equation}
As soon as $c^\star$ is found, the \gls{bs} communicates to both \gls{ris} and \gls{ue} that the Payload phase can start; otherwise, the \gls{BSW} process continues until a configuration is selected. If no configuration of the codebook $C\bsw$ satisfies the condition~\eqref{eq:bsw:cstar:flexible}, then communication is not feasible and outage occurs.

\subsubsection{Algorithmic errors}
For the \gls{bsw} paradigm, an outage event occurs when no configuration in the \gls{BSW} codebook satisfies the target \gls{kpi}, and when the selected configuration provides an \gls{snr} lower than the target one, although the estimated \gls{snr} was higher; we denote the latter as the overestimation event. These two events are mutually exclusive, and hence, their probability is
\begin{equation} \label{eq:bsw:ae}
\begin{aligned}
    p_\mathrm{ae} &= \mc{P} \left[ \gamma_{c^\star} \le \gamma_0 | \hat{\gamma}_{c^\star} > \gamma_0 \right] + \mc{P}\left[ \hat{\gamma}_{c} \le \gamma_0, \, \forall c\in\mc{C}\bsw \right]  \\
    &= \mc{P} \left[ \hat{\gamma}_{c^\star} - \gamma_0 \le  \frac{|w_{c^\star}|^2}{\sigma_b^2} + 2 \Re\left\{ \frac{\sqrt{\rho_u}}{\sigma_b^2} \bm{\phi}_{c^\star}\T \mb{z}_d\, w_c\right\} \right] \\ &\quad+ \mc{P}\left[ \hat{\gamma}_{1} \le \gamma_0, \dots, \hat{\gamma}_{C\bsw} \le \gamma_0  \right],
\end{aligned}    
\end{equation}
where $\gamma_{c^\star} = \frac{\rho_u}{\sigma_b^2} | \bm{\phi}_{c^\star}\T \mb{z}_d|^2$ is the actual \gls{snr} and $\hat{\gamma}_{c^\star} - \gamma_0  > 0$.
%
By applying Chebychev inequality, the overestimation probability (first term) can be upper bounded by
    \begin{equation} \label{eq:bsw:oebound}
     \mc{P} \left[ \hat{\gamma}_{c^\star} \hspace{-1mm} - \hspace{-0.8mm} \gamma_0 \hspace{-.5mm}\le \hspace{-0.8mm} \frac{|w_{c^\star}|^2}{\sigma_b^2} \hspace{-.5mm}+\hspace{-.5mm} 2 \Re\hspace{-.5mm}\left\{ \frac{\sqrt{\rho_u}}{\sigma_b^2} \bm{\phi}_{c^\star}\T \mb{z}_d\, w_c\hspace{-.5mm}\right\} \right] % \le \frac{\E{\frac{|w_{c^\star}|^2}{\sigma_b^2} + 2 \Re\left\{ \frac{\sqrt{\rho_u}}{\sigma_b^2} \bm{\phi}_{c^\star}\T \mb{z}_d\, w_c\right\}}}{\hat{\gamma}_{c^\star} - \gamma_0} = 
     \hspace{-1mm}\le \hspace{-1mm}\frac{p^{-1}}{\hat{\gamma}_{c^\star} \hspace{-1mm}-\hspace{-.8mm} \gamma_0},
\end{equation}
%From eq.~\eqref{eq:bsw:oebound}, 
from which we infer that the higher the gap between $\hat{\gamma}_{c^\star}$ and $\gamma_0$, the lower the probability of error. The \gls{bsw} employing the fixed structure generally has a higher value of ${\hat{\gamma}_{c^\star} - \gamma_0}$ than the one with the flexible structure due to the use of the $\argmax$ operator to select the configuration $c^\star$. Therefore, the fixed structure is generally more robust to overestimation errors. 
%From eqs.~\eqref{eq:bsw:ae} and~\eqref{eq:bsw:oebound}, we note that the fixed structure is more robust to overestimation errors. Indeed, the higher the gap between $\hat{\gamma}_{c^\star}$ and $\gamma_0$, the lower the probability of error. 
On the other hand, the evaluation of the probability that no configuration in the beam sweeping codebook satisfies the target \gls{kpi} requires the knowledge of the \gls{cdf} of the estimated \gls{snr}, whose analytical expression is channel-model dependent and generally hard to obtain.
\begin{comment}
By assuming \gls{iid} measures of the \gls{snr} values, the probability that no configuration in the beam sweeping codebook satisfies the target \gls{kpi} is
\begin{equation} \label{eq:bsw:outageiid}
     \mc{P}\left[ \hat{\gamma}_{1} \le \gamma_0, \dots, \hat{\gamma}_{C\bsw} \le \gamma_0  \right] = \prod_{c\in\mc{C}\bsw} \mc{P}\left[ \hat{\gamma}_{c} \le \gamma_0\right] = \left[F_{\hat{\gamma}_1}(\gamma_0)\right]^{C\bsw},
\end{equation}
where $F_{\hat{\gamma}_c}(\gamma_0)$ is the \gls{cdf} of $\hat{\gamma}_c$.
Remark that the assumption of \gls{iid} $\hat{\gamma}_c$ measurements is an oversimplification, considering that the propagation environment is in general spatially correlated. %Hence, the configuration codebook should be specifically designed to have \gls{iid} $\hat{\gamma}_c$ measurements, which, in turn, depends on the channel model of $\mb{z}_d$. On the other hand, 
On the other hand, when considering completely correlated measurements, we obtain
\begin{equation} \label{eq:bsw:outagecorr}
     \mc{P}\left[ \hat{\gamma}_{1} \le \gamma_0, \dots, \hat{\gamma}_{C\bsw} \le \gamma_0  \right] = \mc{P}\left[ \hat{\gamma}_{1} \le \gamma_0 \right] = F_{\hat{\gamma}_1}(\gamma_0).
\end{equation}
In a real environment, the actual outage probability will be in the range given by eqs.~\eqref{eq:bsw:outageiid} and~\eqref{eq:bsw:outagecorr}. In any case, both eqs.~\eqref{eq:bsw:outageiid} and~\eqref{eq:bsw:outagecorr} depend on the \gls{cdf} of the estimated \gls{snr}, whose analytical expression is channel model dependent and generally hard to obtain. From the equations, we can infer that the more spatially correlated is the channel in the environment, the lower the outage probability. \fs{To say more than this is complicated.}
\end{comment}
Here, we also resort to numerical evaluation of the impact of the \gls{bsw} algorithmic errors.

\subsection{Trade-offs in different transmission paradigms}
\label{sec:paradigms:comment}
The two aforedescribed \gls{ris}-aided transmission paradigms can be seen as a generalization of the \emph{fixed rate} (multiplexing) and \emph{adaptive rate} (diversity) transmission approaches. Essentially, the \gls{se} of the \gls{oce} is adapted to the achievable rate under the optimal configuration (see~\eqref{eq:oce:se}), while the \gls{se} of the \gls{bsw} is set \emph{a priori} according to the target \gls{kpi} (see~\eqref{eq:bsw:se}). Comparing~\eqref{eq:oce:se} and~\eqref{eq:bsw:se} under the same environmental conditions, we have that 
\begin{equation}
    \eta\bsw \le \eta\oce,
\end{equation}
where the price to pay for the higher \gls{se} of the \gls{oce} paradigm is the increased overhead. \gls{oce} needs an accurate \gls{csi} for reliable rate adaptation, which translates into a higher number of sequences to be transmitted by the \gls{ue} compared to \gls{bsw}. Furthermore, additional time and processing are required to determine the optimal configuration of the \gls{ris}. Consequently, the \gls{se} of data transmission alone cannot be considered a fair comparison metric, as it does not consider the overheads generated by the communication paradigms. % In the next section, we will introduce the impact of the \glspl{cc} giving the main metric of the comparison.


\section{Impact of the Control Channels} \label{sec:ris-control}
In this section, we define a performance metric that simultaneously measures the communication performance and the impact of control signaling. We then characterize the terms of this metric regarding the overhead and the reliability of the signaling for the presented paradigms. %of Sect.~\ref{sec:paradigms}.

\subsection{Utility function definition}
To measure the communication performance, we define a utility function that takes into account \emph{a}) the overhead and the error of the communication paradigms and \emph{b}) the reliability of the \glspl{cc}. Regarding the former, \change{we define the \emph{goodput} $R$ as a discrete random variable whose value depends on the communication paradigm and its algorithmic errors:}
\begin{equation} \label{eq:netthroughput}
    R(\tau\pay, \eta) = 
    \begin{cases}
        \frac{\tau\pay}{\tau} B_d \, \eta, \text{ with prob. } 1 - p_\mathrm{ae},\\
        0, \text{ with prob. } p_\mathrm{ae},
    \end{cases}
    %(1 - p_\mathrm{ae}) \frac{\tau\pay}{\tau} B_d \, \eta,
\end{equation}
In this expression, $\eta = \eta\oce$ in~\eqref{eq:oce:se} or $\eta = \eta\bsw$ in~\eqref{eq:bsw:se} if \gls{oce} or \gls{bsw} is respectively employed, $\tau\pay$ is the duration of the payload phase, and $\tau$ is the overall frame duration. The overall overhead time is the sum of the time to carry out the Initialization, Algorithmic, and Setup phases, denoted as $\tau\ini$, $\tau\alg$, and $\tau\set$, respectively\footnote{\change{We remark that the overhead time directly impacts on the latency experienced by the \gls{ue}: given a fixed frame duration, a higher overhead translates into a lower time opportunity for the Payload phase, reducing the transmitted data in each slot, and hence, increasing the overall latency.}}. Accordingly, the payload time can be written as $\tau\pay = \tau - \tau\ini - \tau\alg - \tau\set$.
%\begin{equation}
%    \tau\pay = \tau - \tau\ini - \tau\alg - \tau\set.
%\end{equation}
While the overall frame length is fixed, the overhead time depends on the transmission paradigm, being a function of: the duration of a pilot, $\tau_p$, and the number of replicas transmitted; the optimization time, $\tau_A$; and the time to control the \gls{ris}, composed of the time employed for the transmission of the control packets to the \gls{ue} (\gls{risc}), $\tau_{i}^{(u)}$ ($\tau_{i}^{(r)}$), and the time needed by the \gls{ris} to switch configuration, $\tau_s$.

Regarding the reliability of the \glspl{cc}, we denote as $P = P_u + P_r$ the total number of control packets needed to let a communication paradigm work, where $P_u$ and $P_r$ are the numbers of control packets intended for the \gls{ue} and the \gls{risc}, respectively. Whenever one of such packets is erroneously decoded or lost, an event of \emph{erroneous control} occurs. We assume that these events are independent of each other (and of the algorithmic errors). We denote the probability of erroneous control on the packet $i$ toward entity $k\in\{u,r\}$ as $p_i^{(k)}$, with $i\in\{1, \dots, P_k\}$ and $k \in\{u,r\}$. Erroneous controls may influence the overhead time and the communication performance: \change{the \gls{ris} configuration might change unpredictably}\footnote{\change{In our scenario, if the control packet is not received, the \gls{risc} will load the \gls{ctrl} configuration, \emph{i.e.}, a predictable configuration change. However, if the \gls{risc} receives, but incorrectly decodes, a control packet, the \gls{bs} cannot know which configuration, if any, will be loaded.}}, leading to a degradation of the performance, or worse, letting the data transmission fail. While losing a single control packet may be tolerable depending on its content, we assume all control packets must be received correctly to make the communication successful. In other words, no erroneous control event is allowed. Consequently, the probability of correct control is
\begin{equation} \label{eq:pcc}
    p_\mathrm{cc} = \prod_{k\in\{u,r\}} \prod_{i=1}^{P_k} \left(1 - p_i^{(k)}\right).
\end{equation}

\change{We can include the control reliability in the communication performance, taking into account the probability of correct control in the goodput metric in~\eqref{eq:netthroughput}. By assuming that the control and algorithmic errors are independent, the goodput is re-expressed as follows:}
\begin{equation} \label{eq:netthroughput:2}
    R(\tau\pay, \eta) = 
    \begin{cases}
        \frac{\tau\pay}{\tau} B_d \, \eta, \text{ with prob. } p_\mathrm{cc}(1 - p_\mathrm{ae}),\\
        0, \text{ with prob. } 1 - p_\mathrm{cc} (1 - p_\mathrm{ae}),
    \end{cases}
    \end{equation}
\change{Hence, the performance of the considered \gls{ris}-enabled communication system can be described by averaging $R$ w.r.t. the control, obtaining the following \emph{utility function}:}
\begin{equation} \label{eq:utility}
    U(\tau\pay, \eta) \hspace{-0.5mm}%\mathbb{E}\left[  R(\tau\pay, \eta) \right] 
    = \hspace{-0.5mm} p_\mathrm{cc} (1 - p_\mathrm{ae}) \hspace{-1mm}\left( 1 - \frac{\tau\ini + \tau\alg + \tau\set}{\tau}\right) B_d  \eta.
\end{equation}


\subsection{Overhead evaluation} \label{sec:overhead}

\begin{figure}[tbh]
    \centering
    \begin{subfigure}{\columnwidth}
        \centering
        \includegraphics[width=\textwidth]{figs/data-oce.pdf}        
        \caption{\gls{oce}}
    \end{subfigure}
    \begin{subfigure}{\columnwidth}
        \centering
        \includegraphics[width=\textwidth]{figs/data-bsw-fixed.pdf}
        \caption{\gls{bsw}: fixed frame structure}
    \end{subfigure}
    \begin{subfigure}{\columnwidth}
        \centering
        \includegraphics[width=\textwidth]{figs/data-bsw-flexi.pdf}
        \caption{\gls{bsw}: flexible frame structure}
    \end{subfigure}       
    \caption{Frame structure for the communication paradigms under study. Packets colored in \textcolor{blue}{blue} and in \textcolor{yellow}{yellow} have \gls{dl} and \gls{ul} directions, respectively. Remark that INI-R (SET-R) packet and its feedback (fb) are sent at the same time as the INI-U (SET-U), but on different resources, if \gls{obcc} is present (dashed lines).}
    \label{fig:data-frames}
\end{figure}

Following the description of Section~\ref{sec:paradigms}, we present in Fig.~\ref{fig:data-frames} the frame structures of the two considered communication paradigms used to evaluate the induced overhead, where the rows represent the time horizon of the packets traveling on the different channels (first three rows) and the configuration loading time at the \gls{risc} (last row).
The time horizon is obtained assuming that all the operations span multiple numbers of \glspl{tti}, each of duration of $T$ seconds with $\lceil\tau / T\rceil \in \mathbb{N}$ being the total number of \glspl{tti} in a frame. At the beginning of each \gls{tti}, if the \gls{risc} loads a new configuration, the first $\tau_s$ seconds of data might be lost due to the unpredictable response of the channel during this switching period. When needed, we consider a guard period of $\tau_s$ seconds in the overhead evaluation to avoid data disruption. Remember that the \gls{risc} loads the \gls{ctrl} configuration any time it is in an idle state, \emph{i.e.}, at the beginning of the Initialization and Setup phases.

In Fig.~\ref{fig:data-frames}, we note that the overhead generated by the Initialization and Setup phases is \emph{transmission paradigm independent}\footnote{The reliability is still dependent on the paradigm (see Section~\ref{sec:reliability}).}, while it is \emph{\gls{cc} dependent}.  Both paradigms make use of $P = 4$ control packets, $P_u = 2$ control packets sent on the \gls{ue}-\gls{cc} and $P_r = 2$ on \gls{ris}-\gls{cc}. Nevertheless, \change{employing an OB-\gls{ris}-\gls{cc} can reduce the overhead by transmitting the \gls{ris} control packets through orthogonal resources.} On the other hand, the Algorithmic phase overhead is \emph{\gls{cc} independent} and \emph{transmission paradigm dependent}, being designed to achieve the goal of the specific paradigm regardless of the \gls{cc}. In the following, the overhead is evaluated.

\subsubsection{Initialization phase}
This phase starts with the initialization control packet sent on the \gls{ue}-\gls{cc} (INI-U) informing the \gls{ue} that the \gls{oce} procedure has started. In the \gls{ibcc} case, this is followed by the transmission of the INI-R packet to the \gls{risc} to notify the beginning of the procedure. A consequent \gls{tti} for feedback is reserved to notify back to the \gls{bs} if the INI-R packet has been received. If an \gls{obcc} is employed, no \gls{tti} needs to be reserved because the INI-R and its feedback are scheduled simultaneously since the INI-U packet relies on different resources (see Assumption~\ref{assu:ris-cc}).
The phase duration is $\tau\ini = T$ or $\tau\ini=3T$ with OB- or IB-\gls{ris}-\gls{cc}, respectively.
% \begin{equation} 
%     \tau\ini = 
%     \begin{cases}
%         T, \quad \text{\gls{obcc}}, \\
%         % 2 T, \quad \text{\gls{ibno}}, \\
%         3 T, \quad \text{\gls{ibcc}}.
%     \end{cases}
% \end{equation}

\subsubsection{Setup phase}
 %The time needed to set up the \gls{ue} and the \gls{risc} follows the Initialization phase. 
 After the optimization has run, a setup (SET-U) packet spanning one \gls{tti} is sent to the \gls{ue} notifying it to prepare to send the data; then, with an \gls{ibcc}, a \gls{tti} is used to send the SET-R packet containing the information of which configuration to load during the Payload phase; a further \gls{tti} is reserved for feedback. Again, if an \gls{obcc} is present, the SET-R and its feedback are scheduled at the same time as the SET-U packet but on different resources; therefore, no \glspl{tti} needs to be reserved for the SET-R and its feedback. Remark that the $\tau_s$ guard period must be considered by the \gls{ue} when transmitting the data to avoid being disrupted during the load of the configuration employed in the Payload. For simplicity of evaluation, we account for this guard period in the Setup phase duration, resulting in $\tau\set = \tau\ini + \tau_s$.
% \begin{equation} \label{eq:oce:ack-time}
%     \tau\set = \tau\ini + \tau_s
%     % \begin{cases}
%     %     T + \tau_s, \quad \text{\gls{obcc}}, \\
%     %     2T + \tau_s, \quad \text{\gls{ibno}}, \\
%     %     3T + \tau_s, \quad \text{\gls{ibwf}}.
%     % \end{cases}
% \end{equation}

\subsubsection{Algorithmic phase}
This phase comprises the process of sending pilot sequences and the consequent evaluation of the configuration for the transmission. Regardless of the paradigm, we assume each pilot sequence spans an entire \gls{tti}, but the configuration's switching time must be considered a guard period. Therefore, the actual time occupied by a pilot sequence is $\tau_p \le T - \tau_s$ and the number of samples $p$ of every pilot sequence is given by p = $\left\lfloor \frac{T - \tau_s}{T_n} \right\rfloor$,
% \begin{equation}
%     p = \left\lfloor \frac{T - \tau_s}{T_n} \right\rfloor,
% \end{equation}
where $T_n$ is the symbol period in seconds. Assuming that the \gls{tti} duration and the symbol period are fixed, the \gls{ue} can compute the pilot length if it is informed about the guard period. The overall duration of the Algorithmic phase depends on the paradigm employed.

\paragraph{\gls{oce}} In this case, the Algorithmic phase starts with $C\oce$ \glspl{tti}; at the beginning of each of them, the \gls{risc} loads a different configuration, while the \gls{ue} transmits replicas of the pilot sequence. After all the sequences are received, the \gls{ce} process at the \gls{bs} starts, followed by the configuration optimization. The time needed to perform the \gls{ce} and optimization processes depends on the algorithm and the available hardware. To consider a generic case, we denote this time as $\tau_A = A T$.

\paragraph{\gls{bsw} fixed frame structure}
Similarly to the previous case, the Algorithmic phase starts with $C\bsw$ \glspl{tti}, at the beginning of which the \gls{risc} loads a different configuration, and the \gls{ue} transmits replicas of the pilot sequence. After receiving all sequences, the \gls{bs} selects the configuration as described in Section~\ref{sec:communication-paradigms:bsw}. The time needed to select the configuration is considered negligible. Thus, the Setup phase may start in the \gls{tti} after the last pilot sequence is sent.

\paragraph{\gls{bsw} flexible frame structure}
In this case, the number of \glspl{tti} used for the beam sweeping process is not known \emph{a priori} and depends on the measured \gls{snr}. However, to allow the system to react if the desired threshold is reached, a \gls{tti} is reserved for transmitting an acknowledgment (ACK-U) packet after each \gls{tti} used for pilot transmission. Hence, the number of \glspl{tti} needed is $2 c^\star - 1$, where $0< c^\star \le C\bsw$ is a random variable.

Accordingly, the Algorithmic phase duration is
\begin{equation} \label{eq:algorithmic-time}
    \tau\alg = 
    \begin{cases}
        (C\oce + A) T, \quad &\text{\gls{oce}}, \\
        C\bsw T, \quad &\text{\gls{bsw} fixed frame}, \\
        (2 c^\star - 1) T, \quad &\text{\gls{bsw} flexible frame}.
    \end{cases}
\end{equation}

\subsection{Reliability evaluation}
\label{sec:reliability}

\change{The reliability of the control packets depends on their informative content, the time reserved for their transmission, and the bandwidth of the \gls{cc}. With equal transmission time and bandwidth, transmitting a high informative packet is less reliable than a low informative packet; similarly, increasing the time reserved leads to higher reliability.
We account for this behavior via the outage probability of the $i$-the control packet intended to entity $k\in\{u,r\}$, which is given by}
\begin{equation} \label{eq:outagepe}
    p_i^{(k)} = \Pr\left\{ \log \left(1 + \Gamma_k \right) \le \frac{b_i^{(k)}}{\tau_{i}^{(k)} B_{k}} \right\}, \quad i =\{1,2\},
\end{equation}
where $i =1,2$ refers to the INI or SET packet, respectively; \change{$b_i^{(k)}$ is the amount of informative bits,} $\tau_{i}^{(k)}$ is the reserved time for transmission, and $B_k$ is the \gls{cc} bandwidth. Following the channel model in Section~\ref{sec:model},~\eqref{eq:outagepe} can be rewritten as
\begin{equation} \label{eq:outagepe2}
    p_i^{(k)} = 1 - \exp\left[- \frac{1}{\lambda_k} \left(2^{b_i^{(k)} / \tau_{i}^{(k)} / B_{k}} - 1 \right) \right].
\end{equation}
Plugging~\eqref{eq:outagepe2} into~\eqref{eq:pcc}, the correct control probability is
\begin{equation} \label{eq:pcc2}
\begin{aligned}
p_\mathrm{cc} =& \exp\left[ \frac{1}{\lambda_u} \left( 2 - \sum_{i=1}^2 2^{b_{i}^{(u)} / \tau_{i}^{(u)} / B_u}\right)\right] \times \\ &\exp\left[ \frac{1}{\lambda_r} \left( 2 - \sum_{i=1}^2 2^{b_{i}^{(r)} / \tau_{i}^{(r)} / B_r} \right) \right].    
\end{aligned}
\end{equation}
\change{Remark that the informative content $b_i^{(k)}$ and the reserved time $\tau_i^{(k)}$ depend on the control packet and the communication paradigm employed due to the need to communicate different control information. Hence, different paradigms require different values of the average \gls{snr} of the \gls{ue}-\gls{cc} $\lambda_u$ and the \gls{ris}-\gls{cc} $\lambda_r$ to obtain the same value of $p_\mathrm{cc}$. In practice, the minimum $\lambda_u$ and $\lambda_r$ required to obtain the target correct control probability give a measure of the complexity of the decoding process.}
In the following, we compute $\tau_i^{(k)}$ and $b_i^{(k)}$ for the cases under investigation.

\subsubsection{Reserved time for control packets}\label{sec:usefultime}
Each control packet spans an entire \gls{tti} following the data frame. However, the actual transmission time $\tau_{i}^{(k)}$, \emph{i.e.}, the time in which informative bits can be sent without risk of being disrupted, depends on the \gls{ris} switching time. As discussed in Section~\ref{sec:overhead}, a guard period $\tau_s$ must be considered if the \gls{risc} loads a new configuration in that \gls{tti}. Following the frame structure of Fig.~\ref{fig:data-frames}, INI-R and SET-R packets can use the whole \gls{tti}, while INI-U packets need the guard period. The SET-U control packet does not employ the guard period under the \gls{oce}, as long as $A \ge 1$. The guard period is needed for the \gls{bsw} paradigm. Hence, the transmission time of the control packets intended for the \gls{ue} is $\tau_1^{(u)} = T - \tau_s$ for all paradigms, and $\tau_2^{(u)} = T - \tau_s$ for \gls{bsw} and $\tau_2^{(u)}$ for \gls{oce},
% \begin{equation}
%     \begin{aligned}
%         \tau_{1}^{(u)} &= T - \tau_s, \qquad
%         \tau_2^{(u)} &= 
%         \begin{cases}
%             T- \tau_s, \quad &\text{\gls{bsw}}, \\
%             T, \quad &\text{\gls{oce}},
%         \end{cases}
%     \end{aligned}
% \end{equation}
while the time reserved for the control packets intended for the \gls{risc} is $\tau_{1}^{(r)} = \tau_{2}^{(r)} = T$.
%\begin{equation}
%    \tau_{1}^{(r)} = \tau_{2}^{(r)} = T.
%\end{equation}

\subsubsection{Control packet content}
\label{sec:bits}
\begin{figure}
    \centering
    \includegraphics[height=1.3cm]{figs/control-packet.pdf}
    \caption{General control packet structure.}% comprising a preamble and a payload part.}
    \label{fig:packet-structure}
    \vspace{-0.5cm}
\end{figure}

%We evaluate each control packet's amount of informative bits $b_i^{(k)}$ in this part. 
Without loss of generality, we can assume a common structure for all the control packets, comprising a control preamble and a control payload as depicted in Fig.~\ref{fig:packet-structure}. The preamble comprises $b^\mathrm{ID}$ bits representing the \emph{unique identifier (ID)} of the destination entity in the network and a single bit flag specifying if the packet is a INI or a SET one. From the preamble, the entity can understand if the control packet is meant to be decoded and how to interpret the control payload, whose informative bits depend on the kind of control packet and on the communication paradigm considered.

\paragraph{\gls{oce}}
To initialize the overall procedure at the \gls{ue}, the payload of the INI-U packet must contain the length of the frame $\tau$, the cardinality of the set $C\oce$, and the guard time $\tau_s$. To simplify the data transmission, the frame duration can be notified through an (unsigned) integer $b^\mathrm{frame}$ containing the number of total \glspl{tti} $\lceil \tau / T \rceil$ set for the frame. Similarly, we can translate the guard time into an unsigned integer representing the number of guard symbols $\lceil \tau_s / T_n \rceil$ to send $b^\mathrm{guard}$ bits. Finally, another integer of $b^\mathrm{conf}$ bits can be used to represent the cardinality $C\oce$ and to notify it to the \gls{ue}. Its minimum value is $b^\mathrm{conf} = \lfloor \log_2(C) \rfloor$, where $C$ is the total number of configurations stored in the common codebook.
Similarly, the payload of the INI-R packets needs to contain the information of the length of the frame $\tau$, and the \emph{set} of configuration  $\mc{C}\oce$ to switch through.
The former uses the same $b^\mathrm{frame}$ bits of the INI-U packet. To encode the latter, $b^\mathrm{conf}$ bits are used to identify a single configuration in the common codebook, and thus, $C\oce b^\mathrm{conf}$ needs to be transmitted to the \gls{risc}, one per desired configuration.
Regarding the Setup phase, the payload of the SET-U contains only the chosen \gls{se} of the communication $\eta\oce$. This can be encoded similarly to the \gls{mcs} in the 5G standard~\cite{3gpp:rel15}: a table of predefined values indexed by $b^\mathrm{SE}$ bits. The payload of the SET-R must contain the optimal configuration $\bm{\phi}_\star$, that is, a phase-shift value for each element. Without loss of generality, we denote by $b^\mathrm{quant}$ the number of bits used to control each element, \emph{i.e.}, the level of quantization of the \gls{ris}~\cite{EURASIP_RIS}. Hence, the overall number of informative bits is %equals the number of elements to control times the quantization level, \emph{i.e.}, 
$N b^\mathrm{quant}$.
To summarize:
\begin{equation} \label{eq:bits:oce}
    b_{i}^{(k)}\hspace{-0.2mm} = \hspace{-0.2mm} b^\mathrm{ID} \hspace{-1mm}+\hspace{-0.5mm} 1 \hspace{-0.5mm}+\hspace{-1mm}
    \begin{cases}
        b^\mathrm{frame} + b^\mathrm{guard} + b^\mathrm{conf}, \quad &k=u, \, i= 1, \\
        b^\mathrm{frame} + C\oce b^\mathrm{conf}, \quad &k= r, \, i=1, \\ 
        b^\mathrm{SE}, \quad &k= u, \, i=2,\\
        N b^\mathrm{quant}, \quad &k= r, \, i=2.
    \end{cases}
\end{equation}

\paragraph{\gls{bsw}}
The payload of the Initialization packets follows the same scheme used for the \gls{oce} paradigm. The INI-U packet contains the length of the frame $\tau$, the cardinality of the set $C\bsw$, and the guard time $\tau_s$ in the (unsigned) integers $b^\mathrm{frame}$, $b^\mathrm{guard}$, and $b^\mathrm{conf}$, respectively.
The payload of the INI-R packets contains the information of the length of the frame $\tau$, and the \emph{set} of configuration  $\mc{C}\bsw$ to switch through, encoded in the (unsigned) integers $b^\mathrm{frame}$ and $C\bsw b^\mathrm{conf}$, respectively.
Instead, the Setup contains different information. In particular, the payload of the SET-U is empty, according to the fixed rate transmission used by this paradigm. The payload of the SET-R contains the configuration $c^\star$ encoded by the same $b^\mathrm{conf}$ bits, representing an index in the common codebook. %\footnote{\change{Note that the \gls{oce} SET-R packet requires a higher informative content due to the need of transmitting the phase shift for each element of the \gls{ris}.}}.
To summarize, the packet length is:
\begin{equation} \label{eq:bits:bsw}
    b_{i}^{(k)}\hspace{-0.2mm} = \hspace{-0.2mm} b^\mathrm{ID} \hspace{-1mm}+\hspace{-0.5mm} 1 \hspace{-0.5mm}+\hspace{-1mm}
    \begin{cases}
        b^\mathrm{frame} + b^\mathrm{guard} + b^\mathrm{conf}, \quad &k=u, \, i= 1,\\
        b^\mathrm{frame} + C\bsw b^\mathrm{conf}, \quad &k= r, \, i=1, \\ 
        0, \quad &k= u, \, i=2, \\
        b^\mathrm{conf}, \quad &k= r, \, i=2.
    \end{cases}
\end{equation}
\change{Remark that the informative content of the \gls{bsw} packets is lower or equal to the one of \gls{oce}, leading the former to be more reliable than the latter.}

\section{Numerical Results} \label{sec:results}
\section{Results}
\label{results}

\begin{figure*}[ht]
    \centering
    \includegraphics[scale=0.15,trim={0 2.5cm 0 5cm},clip]{images/aoi-single_burst}
    \caption{The time average peak Age of Information with burst and \gls{soa} loss values against the dynamic reliability logic for different network topologies.}
    \label{fig:aoi_burst}\vspace{-0.4cm}
\end{figure*}


This paper focuses on both transport layer and application layer metrics to determine the feasibility of dynamic reliability. For this, we have selected the session packet volume, as transmitted, retransmitted, lost and backlogged packets as \glspl{kpi} for the transport layer; while focusing on the \gls{aoi} for the application layer. The \gls{aoi} was chosen as a crucial indicator for the freshness of packets in real-time applications. More specifically, this work adopts the time average peak \gls{aoi} equation \cite{aoi_equation} depicted in Eq. \ref{aoi}, where $\Delta(r_{i+1})$ is the $i$th update at the time it was received at the server, for a session time period of $\tau$.

\begin{equation}
    \label{aoi}
    \gls{aoi}_\tau = \frac{1}{n-1}\sum_{i=1}^{n-1} \Delta(r_{i+1})
\end{equation}

We include a comparison between the vanilla QUIC implementation which does not enjoy the dynamic reliability extension, with a number of dynamic reliability policies. The tests were run a number of times for statistical significance, with the mean value of vanilla implementation used as a baseline for comparison. The topology utilised both random loss and bursty loss to explore the bounds of dynamic reliability. The \gls{soa} loss in the figures correspond to the loss values presented in Table. \ref{tab:path_char}, for ease of comparison between bursty and random loss scenarios.

\subsection{Transport-Layer KPIs}

To analyse the performance gain at the transport layer due to dynamic reliability, the volume of transmitted and backlogged packets is examined. The figures are in the form of boxplots, which take the vanilla implementation as a benchmark, depicted as the red dashed line.

As seen in Fig. \ref{fig:sent_burst}, the loss plays a crucial role in the performance of the reliability policies. The policies under random loss did incredibly well for the networks with a larger capacity, namely \gls{mmwave} and Sub-6~GHz, whereas for burst loss, the lower network capacities had a larger packet reduction. With the increase in burst loss, the behaviour of the set split reliable policies became unpredictable, if a reliable assignment happened to coincide with a burst loss, the number of transmitted packets increases, and vice versa. On the other hand, in smarter policies, such as Loss-Aware, the performance lightly matched the vanilla baseline, as the reliable assignment dominated the session to compensate for a higher burst loss. Not only that but, the burst loss also impacted the variance of the transmitted packets for the policies.

Unsurprisingly, the unreliable focused policy, 80-20 split, outperformed other policies for all topologies in random and bursty loss scenarios, with an approximate reduction of 80\%. That being said, the majority of the policies reduced the transmitted packets on the link by approximately 70\% for random loss, while the reduction started at $\approx 15\%$ and decreased as the loss increased for the burst loss scenario.

The retransmitted and lost packets, not shown due to space limitations, followed the same trend as the transmitted packets for the random loss scenarios. However, for the burst loss scenarios, the larger capacity networks had a lower reduction in the retransmitted and lost packets. This can be seen as a favorable outcome since the lower capacity networks are scarce on resources. It is important to note that the Loss-Aware policy mimicked the vanilla approach as the burst loss increased, signifying the overwhelming appointment of reliable packets in adapting to the harsh burst loss conditions.
 
Alternatively, Fig. \ref{fig:backlog_burst} clearly shows a stark comparison between the policies and loss scenario in the reduction of the backlogged packets. The Loss-Aware policy for random loss scenario reduced the backlogged packets by up to 50\%, beating all other policies by approximately 30\%. Furthermore, it is clear that the unreliability focused policies resulted in the lowest backlog for the session. In comparison, we notice that the burst loss and the backlogged frequency have a positive correlation, where the maximum reduction of the backlogged packets for the policies is at most 20\%. Much like the transmitted packets, the probability of a burst loss occurrence plays a vital role in the number of retransmissions sent and by extension the number of backlogged packets. Thus, we can conclude that the stress placed on the buffer is a result of the reliable packets which is tightly coupled with the congestion on the session. Whereas, unreliable focused policies did not encounter such a phenomenon regardless if it was experiencing a burst loss.


\subsection{Application-Layer KPIs}

The feasibility of dynamic reliability for real-time applications can be determined by the \gls{aoi}, with comparison across different topologies and policies. If we take a strict approach and consider anything below $10$~ms is real-time \cite{real-time}, then all the reliability policies passed that requirement, which is attractive for real-time applications, as shown in Fig. \ref{fig:aoi_burst}. Utilising the median as an estimate of the runs, the policies in the WLAN and Sub-6~GHz topology with random loss floated around $4-5$~ms with negligible difference, while the \gls{aoi} for \gls{mmwave} was $\approx 2-3$~ms. It is clear that the \gls{aoi} and the network capacity have a negative correlation, as the network capacity decreases, the \gls{aoi} increases. The same correlation is extended to the bursty loss scenarios, where \gls{mmwave} dominated the other topologies. That being said, it is crucial to note that the \gls{aoi} for the reliability policies is often slightly better than or equal to the \gls{aoi} of the vanilla implementation, proving that dynamic reliability reduces the congestion of the session at no cost to the \gls{aoi}.


% \section{Discussion and future analysis} \label{sec:discussion}
% We provide some comments on the growth conditions which constituted the majority of our analysis in sections \ref{sec:Hmixing} and \ref{sec:Hsigma}. In the simplest cases of Lemma \ref{lemma:unstableGrowth}, growth was established in an analogous fashion to the old one-step expansion condition (\ref{eq:oldOneStepExpansion}), finding the relevant Jacobians $M_j$ and checking that their expansion factors $K(M_j)$ satisfy
\begin{equation}
    \label{eq:discussionOneStep}
    \sum_j \frac{1}{K(M_j)} <1.
\end{equation}
For the more complicated cases, the inductive method used to establish growth near the accumulation points in Lemma \ref{lemma:unstableGrowth} and the weakened one-step expansion condition (\ref{eq:oneStep}) both address the same fundamental issue: the splitting of unstable curves by singularities into an unbounded number of small components. They circumvent this obstacle in rather different ways, however. While (\ref{eq:oneStep}) generalises (\ref{eq:discussionOneStep}) to ensure an growth of unstable curves `on average' (see \cite{chernov_statistical_2009} for a precise statement), our inductive method is a more direct adaptation of (\ref{eq:discussionOneStep}), using it to generate contradictory geometric conditions which a hypothetical non-growing unstable curve must satisfy. It may be possible to prove Theorem \ref{sec:Hmixing} using (\ref{eq:oneStep}) as the basis for growth. Since we required (\ref{eq:oneStep}) anyway for proving Theorem \ref{thm:HsigmaExp}, this could potentially condense our analysis, but only to a minor extent. A convenience of the method used in section \ref{sec:Hmixing} is that, by way of the `simple intersection' property, it naturally gives geometric information on the images of manifolds, useful for proving the property \textbf{(M)} of Theorem \ref{thm:katok-strelcyn}.

We expect that essentially analogous analysis can be applied to establish mixing properties in a wide class of piecewise linear non-uniformly hyperbolic maps, including those (like the OTM) which sit on the boundary of ergodicity and beyond. While we have relied on the precise partition structure of $H_\sigma$, its fundamental feature (self-similar sequences of elements $A^k$, sharing boundaries with its neighbours $A^{k-1},A^{k+1}$ and accumulating onto some point $p$) is quite typical to return map systems. See, for example, those of various stadium billiards \cite{chernov_chaotic_2006,chernov_improved_2008,chernov_statistical_2009} and LTMs \cite{springham_polynomial_2014}. Indeed, the same method can be used to prove the Bernoulli property for non-monotonic LTMs \cite{myers_hill_mixing_2022}, where monotonicity of the manifold images cannot be assumed and the classical argument \cite{sturman_mathematical_2006} fails. The OTM is the pointwise limit of these maps as the boundary shrinks to null measure. It further has utility in proving growth conditions for maps which are uniformly hyperbolic but possess regions $A_j$ where the hyperbolicity is very weak, signified by $K(M_j) \approx 1$, so that (\ref{eq:discussionOneStep}) fails. Typically this leads to suboptimal bounds on mixing windows, see e.g. \cite{wojtkowski_model_1981,przytycki_ergodicity_1983,myers_hill_family_2022}. The map $H_{(\eta,\eta)}$ for $\eta \approx 1/2$ is another example, possessing weak hyperbolicity over $A_2, A_3$. Letting $\varepsilon = |\eta-1/2|>0$, there is an upper bound $N = N(\varepsilon)$ on escape times from the intersections $A_2\cap \sigma, A_3 \cap \sigma$. The growth lemma then follows by applying the inductive step roughly $N$ times and can be established for arbitrarily small $\varepsilon$, opening the door to establishing optimal mixing windows.

The above gives two examples of piecewise linear perturbations to $H$ where mixing with respect to Lebesgue is preserved and our methods can be applied. Nonlinear perturbations to the shear profiles complicate the analysis in several ways. Firstly as the map's Jacobians takes on a broader range of values, cone invariance becomes an increasingly harder condition to establish. Cones must be widened, giving looser bounds on expansion factors, which may already be weak due to new regions of weaker stretching. This, together with the change from polygonal to curvilinear return time partition elements and nonlinear local manifolds, adds some complexity to showing growth conditions. This does not rule out certain (small) nonlinear perturbations however. There is some leeway in the inequalities which govern cone invariance and growth of local manifolds, the latter of which is not too dissimilar from the piecewise linear setting (see Lemmas \ref{lemma:piecewiseApprox}, \ref{lemma:componentLength}). Certain small perturbations would not alter the \emph{topological} structure of the return time partition, i.e. which elements share boundaries, the key information needed for setting up the induction. Finally while the partition elements would no longer be polygonal, only coarse geometric information is required for verifying each inductive step. Following the above, a potential perturbation could be to replace the linear portions of each shear by a cubic, perturbing the tent profile
\[  f(t) = \begin{cases} 2t & 0 \leq t \leq 1/2, \\ 2(1-t) & 1/2 \leq t \leq 1 ,\end{cases} \]
of the OTM shears to
\[  f_a(t) = \begin{cases} \frac{1}{8} t \left(16 - a + 6at - 8at^{2} \right) & 0 \leq t \leq 1/2, \\ \frac{1}{8}\left(1-t\right)\left( 16 - a + 6a\left(1-t\right) - 8a\left(1-t\right)^{2}\right)  & 1/2 \leq t \leq 1, \end{cases}   \]
for $a>0$. For small enough $a$ the gradient range $f'(t)$ is restricted to small neighbourhoods of $\{ 2, -2\}$ and the escape time partition retains a similar structure. We illustrate this in Figure \ref{fig:perturbations}, showing escapes from the square $S_3$ under the map $G \circ F$, equivalent to escapes from the perturbed $A_3$ under the $G \circ F$, but with a cleaner geometry for comparison. When $a$ is too large the analogy to the OTM breaks down. At $a=16$ the map is twice differentiable everywhere and features a new source of slowed mixing, the Jacobian is the identity at the corner points $x,y \in \{  0, 1/2 \}$ giving locally parabolic behaviour (visible in the escape time partition). 

\begin{figure}
    \centering
    \includegraphics[width=0.24 \linewidth]{0.png}
    \includegraphics[width=0.24 \linewidth]{4.png}
    \includegraphics[width=0.24 \linewidth]{8.png}
    \includegraphics[width=0.24 \linewidth]{16.png}
    \caption{Partition of escape times from $S_3$ under the mapping $F \circ G$ for $a= 0,4,8,16$. }
    \label{fig:perturbations}
\end{figure}

\section{Conclusions} \label{sec:conclusions}
\section{Conclusions}
We consider the phase-extraction problem, and we showed that, given a unitary $U = e^{i\pi H}$ and its inverse $U^{\dag}$, we could implement a block-encoding of $\phi(H)$ for some smooth function $\phi(x)$. The word `smooth' here means existence and continuity of the derivatives: the higher the number of continuous derivatives that a function has, the faster its Fourier sum (and thus the Laurent polynomial on the eigenphases) uniformly converges to that function. We are confident this can have many more applications beyond what is shown in this work. It is also worth remarking that Jackson showed that the convergence rate of a Fourier series is almost-optimal, in the sense that no trigonometric (or, equivalently, complex exponential) series can approximate the desired function faster, up to that $\log d$ factor~\cite[p.\ 21]{jacksonTheoryApproximation1930a}. Also remember that `smoothing' a function, i.e., replacing its derivative with a continuous function, does not give faster convergence for free in general, as its derivative will become steep in the points where we smooth out discontinuities, and this translates to a high Lipschitz constant: a~clear example is given by Eq.~\ref{eq:lipschitz-constant-recurrence-solution}, but in that case, fortunately, nothing depends on the size of the input $N$, and thus does not influence the asymptotic query complexity of Algorithm~\ref{alg:prop-sampling-qsp}, although the constant factor can become large even for $p = 20$. From a theoretical point of view, this work shows that, for any $\eta > 0$, there is an algorithm with query complexity 
$$\Tilde{\bigO}\left(\frac{1}{\bar{c}^{\frac{1}{2} + \eta}} \frac{1}{\epsilon^\eta} \right)$$
solving the proportional-sampling problem. This statement seems to suggest there exists an algorithm which directly solves the problem with $\eta = 0$, and an open question would be to find such algorithm.


It is also interesting to remark that Theorems~\ref{thm:haah-construction},~\ref{thm:haah-completion} indeed allow the construction for any $\phi$, even complex-valued, provided that its absolute value is reciprocal.

One could think that, in Section~\ref{sec:prop-sampling}, instead of using the linear function in the phase-extraction subroutine, we could approximate the square root and then apply the transformation directly on $e^{i \pi c(x)}$. However, in the case of proportional sampling this would be inconvenient, as the derivative of the square root function has a discontinuity with an infinite jump around 0, and we could not choose a constant $\delta$ if we had values of the oracle that are too close to $0$.




% references section
\bibliographystyle{IEEEtranNoURL}
\bibliography{IEEEabbr,bib}

\end{document}