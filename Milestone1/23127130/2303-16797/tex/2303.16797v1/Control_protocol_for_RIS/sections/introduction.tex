% \pp{Outline for the introduction
% \begin{itemize}
%     \item Start by stating the obvious stuff about the increased research interest in RIS. However, most of the advances have been at the PHY layer ane much less at the link/MAC protocol. 
%     \item The operation of the control channel is one of the most important differentiator of a RIS-aided system and a relay (refer to the SPM paper I have with Emil, Luca, etc.). Nevertheless, a systematic analysis of the control channel options is missing and this paper aims to fill this important gap, relevant for both theory and practice of RIS-aided comms. 
%     \item Related works on protocols, here you can put our random access channels
%     \item Here you describe what we do in this paper. you can reuse the text below to a large extent, where we speak about many options for the control channel, but we limit ourselves to the generic options. 
%     \item Our framework considers the generic options and identifies several important performance indicators: algorithmic errors, control errors, overhead, etc. 
%     \item The paper is organized as follows.
% \end{itemize}
% }

\Glspl{ris} constitute a promising technology that in recent years has received significant attention within the wireless communication research community~\cite{Huang2019}. 
The main underlying idea is to electronically tune the reflective properties of an \gls{ris} in order to manipulate the phase, amplitude, and polarization of the incident electromagnetic waves~\cite{risTUTORIAL2020}. This results in the effect of creating a propagation environment that is, at least partially, controlled~\cite{RISE6G_COMMAG}. \glspl{ris} can be fabricated with classical antenna elements controlled through switching elements or, more advanced, can be based on matematerials with tunable electromagnetic properties~\cite{EURASIP_RIS}. 
In the context of 6G wireless systems, the \gls{ris} technology has been identified as one of the cost-effective solutions to address the increasing demand for higher data rates, reduced latency, and better coverage.
In particular, an \gls{ris} can improve the received signal strength and reduce interference by directing signals to intended receivers and away from non-intended ones; this leads to applications aiming increased communication security~\cite{Chen_2019} and/or reduced electromagnetic field exposure~\cite{NA2021}. \glspl{ris} can also extend the coverage of wireless communication systems by redirecting the signals to areas that are difficult to reach using conventional means.

In terms of challenges related to the \gls{ris} technology, the dominant part of the literature concerning \gls{ris}-aided communication systems deals with physical-layer (PHY) aspects. Recent studies have explored physics-based derivation of channel models, extending plane wave expansions beyond the far-field approximation \cite{Model_ProcIEEE}. While many papers have investigated the potential benefits of \gls{ris}-assisted systems in terms of spectral and energy efficiency (see, \emph{e.g.},~\cite{Tsinghua_RIS_Tutorial}), others have concentrated on optimizing the \gls{ris} configuration alone or jointly with the beamforming at the \gls{bs}. On the other hand, several works have focused on designing and evaluating \gls{ce} methods in the presence of \glspl{ris}, either relying on the cascaded end-to-end channel when dealing with reflective \glspl{ris}~\cite{CascadeCE_ProcIEEE}, or focusing on the individual links using simultaneous reflecting and sensing \glspl{ris}~\cite{CE_HRIS_2023}. The latter design belongs to the attempts to minimize the \gls{ris} reconfiguration overhead, which can be considerably large due to the expected high numbers of \gls{ris} elements \cite{popov2021experimental} or hardware-induced non linearities~\cite{Tsinghua_RIS_Tutorial}. A different research direction bypasses explicit channel estimation and relies on beam sweeping methods~\cite{singh2021fast}. Accordingly, within each coherent channel block, the \gls{ris} is scheduled to progressively realize phase configurations from a predefined codebook, for the end-to-end system to discover the most suitable reflective beamforming pattern~\cite{RISsweeping_2020, Jamali2022, alexandropoulos2022hierarchical, An2022}. The beams are practically optimized for different purposes~\cite{rahal_RISbeams}, possibly comprising hierarchical structures~\cite{HierarhicalCodebook, alexandropoulos2022hierarchical}.

Within the existing research literature, the questions related to link/MAC protocol and system-level integration of \glspl{ris} have received much less attention as compared to PHY topics. Specifically, the aspects related to control/signaling procedures have been largely neglected, despite the fact that those procedures are central to the integration of \glspl{ris} as a new type of network element within existing wireless infrastructure. In this regard, it is important to study the \gls{ris} control from two angles. 
\emph{First,} as an enabler of the new features that come with the \gls{ris} technology and a component that ensures its proper operation in general. Assessing this aspect requires looking into the required performance of the control channel in terms of, for example, rate, latency, and, not the least, reliability.
\emph{Second,} control procedures introduce an overhead in the system, such that it is important to characterize the trade-off between spending more time and resources on the auxiliary procedures (such as more robust channel estimation or optimization of the \gls{ris} reflection pattern) versus data transmission. This paper aims to fill this important knowledge gap, relevant to both theory and practice of \gls{ris}-aided communications, by providing a systematic analysis of the control architecture options and the associated protocols. 

% While the \gls{ris}-aided communication gets significant attention within the research community, there are also skeptical voices, pointing out that the \gls{ris} concept is identical to the relays. Hence, any \gls{ris}-induced advantage should be compared to a setup based on a classical relay. Nevertheless, without having the ambition to address this \gls{ris}-relay debate in its entirety, we observe that one important aspect in which \gls{ris}-aided communication differs from the relay is in the design and usage of the \gls{cc}. Indeed, any system that integrates \gls{ris} is a distributed system that consists of \gls{bs}, \gls{ris}, and \gls{ue} and a proper design of the \gls{cc} is instrumental in facilitating real-time operation in a controllable propagation environment.
% \KS{Highlight: (a) works don't stress those aspects. (b) different methods come with different requirements, and the impact of the CC is different.}

%%%%%
\subsection{Related literature}
%%%%%
One of the first works focusing on fast \gls{ris} programmability~\cite{RISsweeping_2020} presented a multi-stage configuration sweeping protocol. By tasking the \gls{ris} to dynamically illuminate the area where a \gls{ue} is located, a downlink transmission protocol, including sub-blocks of UE localization, \gls{ris} configuration,
and pilot-assisted end-to-end channel estimation, was introduced in~\cite{Jamali2022}. In~\cite{alexandropoulos2022hierarchical}, a fast near-field alignment scheme for the \gls{ris} phase shifts and
the transceiver beamformers, relying on a variable-width hierarchical \gls{ris} phase configuration codebook, was proposed. Very recently, in~\cite{An2022}, the overhead and challenges brought by the \gls{ris} network integration were discussed. It was argued that the reduced overhead offered by codebook-based \gls{ris} configuration schemes, is beneficial to the overall system performance. Nevertheless, the required control information that need to be exchanged for those schemes was not investigated. In~\cite{Croisfelt2022randomaccess} and~\cite{Croisfelt2023}, a detailed  protocol for RIS-aided communication system was presented tackling the initial access problem. It was showcased that, despite the configuration control overhead, the \gls{ris} brings notable performance benefits allowing more \glspl{ue} to access the network on average. However, it was assumed that the \gls{ris} control is perfect, which can be hardly true in practice. The effect of retransmission protocols in \gls{ris}-aided systems for cases of erroneous transmissions was studied in~\cite{PHY_Retransmission}, although the presented methodology assumed perfect control signals.


%\vc{
%In~\cite{Croisfelt2022randomaccess} and~\cite{Croisfelt2023}, we have introduced the first protocol that carefully incorporates the \gls{ris} to tackle the initial access problem. Despite the overhead introduced by the \gls{ris} control, we show that the \gls{ris} brings notable performance benefits, allowing more \glspl{ue} to access the network on average. However, we have assumed that the control of the \gls{ris} was perfect, whose assumption is hardly true in practice and is going to be challenged in this paper.
%}

% \RK{\textit{Don't have many ideas what to put there, The paper on random access for sure, but maybe the comparison with relay could actually go here.}}

There has been a significant discussion regarding the comparison of \glspl{ris} and conventional amplify-and-forward relays~\cite{Huang2019}. While the distinction between them can sometimes be blurred~\cite{Larsson_2021}, one way to make a clear distinction is the use of the flow of control and data through the communication layers~\cite[Fig.~2]{bjornson2022reconfigurable}. Those considerations set the basis for the definition of the control channel options in this paper.


%%%%%
\subsection{Contributions}
%%%%%
The main objective of this work is to develop a framework for designing and analyzing the \gls{cc} in \gls{ris}-aided communication systems. The number of actual \gls{cc} designs is subject to a combinatorial explosion, due to the large number of configurable parameters in the system, such as frame size or feedback design. Clearly, we cannot address all these designs in a single work, but what we are striving for is to get a simple, yet generic, model for analyzing the impact of \glspl{cc} that captures the essential design trade-offs and can be used as a framework to analyze other, more elaborate, \gls{cc} designs.

We build generic \gls{cc} models along two dimensions. The \emph{first dimension} is related to how the \gls{cc} interacts with the bandwidth used for data communication. An \emph{\gls{obcc}} uses communication resources that are orthogonal to the ones used for data communication. More precisely, \gls{obcc} exerts control over the propagation environment, but is not affected by this control. Contrary to this, an \emph{\gls{ibcc}} uses the same communication resources as data communication. This implies that the \gls{ibcc} decreases the number of degrees of freedom for transmission of useful data, thereby decreasing the \gls{se} of the overall system. Furthermore, the successful transmission of the control messages toward the RIS is dependent on its phase profile. For instance, an unfavorable \gls{ris} configuration may cause blockage of the \gls{ibcc} and transmission of further control messages, impacting the overall system performance. 

The \emph{second dimension} is built along the traditional diversity-multiplexing trade-off in wireless communication systems. In a \emph{diversity transmission}, the data rate is predefined and the sender hopes that the propagation environment is going to support that rate. If this is not the case, then, an outage occurs. To reflect this paradigm in an \gls{ris} setup, we consider a transmission setup in which the \gls{ris} sweeps through different configurations and the BS tries to select the one that is likely to support the predefined data rate. In a \emph{multiplexing transmission}, the data rate is adapted to the actual channel conditions; however, this incurs more signaling for channel estimation. In an \gls{ris}-aided setup, the multiplexing transmission corresponds to a case in which the \gls{ris} configuration is purposefully configured to maximize the link \gls{snr} and the data rate is chosen accordingly.

This paper analyzes the \gls{cc} performance and impact in several communication setups, obtained as combinations of the aforementioned dimensions. In doing so, we have necessarily made simplifying assumptions, such as the use of a frame of a fixed length in which the communication takes place. This is especially important when analyzing a \gls{cc} performance since any flexibility will affect the design of the \gls{cc}. For instance, if a frame has a flexible length that is dependent on the current communication conditions, then, this flexibility can only be enabled through specific signaling over the \gls{cc}, including encoding of control information and feedback\footnote{The simulation code for the paper is available at \url{https://github.com/lostinafro/ris-control}}.

\paragraph*{Paper Outline} In Section~\ref{sec:model}, the system model is described, while Section~\ref{sec:paradigms} describes the paradigms of communication, focusing on the general description of the methods and the description of the obtained \gls{snr}, \gls{se}, and (eventually) outage generated by the method itself \emph{without accounting for potential control error}. Section~\ref{sec:ris-control} describes firstly how to take into account the errors in the \glspl{cc}, which generate further outages/reduction of throughput performance. Then, the methods presented in the previous section are analyzed from the control perspective. In Section~\ref{sec:results}, the performance of the studied communication paradigms is evaluated, %taking into account the impact of the CCs. 
while Section~\ref{sec:conclusions} concludes the paper.



\paragraph*{Notation}
Lower and upper case boldface letters denote vectors and matrices, respectively; the Euclidean norm of $\mathbf{x}$ is $\lVert\mathbf{x}\rVert$; and $\odot$ denotes the element-wise product. $\mc{P}(e)$ is the probability that event $e$ occurs; $\mc{CN}(\bm{\mu},\mb{R})$ is the complex Gaussian distribution with mean $\bm{\mu}$ and covariance matrix $\mb{R}$, $\mathrm{Exp}(\lambda)$ is the exponential distribution with mean value $1/\lambda$. $\mathbb{E}[\cdot]$ is the expected value, $\lfloor a \rfloor$ is the nearest lower integer of $a$, and $j\triangleq \sqrt{-1}$.

