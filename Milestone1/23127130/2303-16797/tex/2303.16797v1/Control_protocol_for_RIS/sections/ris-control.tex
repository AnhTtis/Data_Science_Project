In this section, we define a performance metric that simultaneously measures the performance of a \gls{ris}-aided communication scheme and the impact of the \glspl{cc} over it. We then further characterize the terms regarding the overhead and the reliability of the \glspl{cc} for the particular paradigms discussed in Section~\ref{sec:paradigms}.

%we evaluate the impact of the \glspl{cc} on the performance of the communication paradigms described in Section~\ref{sec:paradigms}. To do so,

\subsection{Performance evaluation: Utility function}
We start by defining a utility function measuring the communication performance by taking into account \emph{a}) the overhead and error of the communication paradigms and \emph{b}) the reliability of the \glspl{cc}.
%
Regarding overhead and errors of the paradigms, we consider the \emph{goodput} metric defined as
\begin{equation} \label{eq:netthroughput}
    R(\tau\pay, \eta) = (1 - p_\mathrm{ae}) \frac{\tau\pay}{\tau} B_d \, \eta,
\end{equation}
where $1- p_\mathrm{ae}$ represents the probability that no algorithmic error occurs; $\eta = \eta\oce$ in~\eqref{eq:oce:se} or $\eta = \eta\bsw$ in~\eqref{eq:bsw:se} if \gls{oce} or \gls{bsw} is employed, respectively; $\tau\pay$ is the duration of the payload phase, and $\tau$ is the overall duration of a frame. The overhead time is the time employed by the Setup, Algorithmic and Acknowledgement phases, being denoted as $\tau\set$, $\tau\alg$, and $\tau\ack$, respectively. Accordingly, the payload time can be written as
\begin{equation}
    \tau\pay = \tau - \tau\set - \tau\alg - \tau\ack.
\end{equation}
While the overall frame length is fixed, the overhead time depends on the paradigm of communication, being a function of: the duration of a pilot, $\tau_p$, and the number of replicas transmitted; the optimization time, $\tau_A$; the time to control the \gls{ris}, composed of the time employed for the transmission of the control packets to the \gls{ue} (\gls{risc}), $\tau_{i}^{(u)}$ ($\tau_{i}^{(r)}$), and the time needed by the \gls{ris} to switch configuration, $\tau_s$. %The analysis of the previous quantities is carried out later in this section.

Regarding the reliability of the \glspl{cc}, we denote as $P = P_u + P_r$ the total number of control packets needed to let a communication paradigm work, where $P_u$ and $P_r$ are the numbers of control packets intended for the \gls{ue} and the \gls{risc}, respectively. Whenever one of such packets is erroneously decoded or lost, an event of \emph{erroneous control} occurs. We assume that these events are independent of each other (and of the algorithmic errors), and we denote the probability of erroneous control on the packet $i$ toward entity $k\in\{u,r\}$ as $p_i^{(k)}$, with $i\in\{1, \dots, P_k\}$ and $k \in\{u,r\}$. Erroneous controls may influence the overhead time and the communication performance: the \gls{ris} phase-shift profile may change in an unpredictable way leading to a degradation of the performance, or worse, letting the data transmission fails. While losing a single control packet may be tolerable depending on its content, here, we assume that all the control packets need to be received correctly in order to let the communication be successful. In other words, no erroneous control event is allowed. Consequently, the probability of correct control results
\begin{equation} \label{eq:pcc}
    p_\mathrm{cc} = \prod_{k\in\{u,r\}} \prod_{i=1}^{P_k} (1 - p_i^{(k)}).
\end{equation}
% Another source of unreliability is given by the algorithmic error which may occur when setting the configuration used in the Payload phase, as discussed in Sect.~\ref{sec:paradigms:comment}. The algorithmic error is independent of the erroneous control events and its probability is denoted as $p_\mathrm{ae}$.

We are interested to show the average performance of the analyzed communication paradigms. According to the considerations made so far, the goodput is a discrete random variable having value given by~\eqref{eq:netthroughput} if correct control occurs, while it is 0 otherwise. Therefore, the average performance can be described by the following \emph{utility function}:
\begin{equation} \label{eq:utility}
    U(\tau\pay, r) = \mathbb{E}_{k,i}\left[  R(\tau\pay, \gamma) \right] = p_\mathrm{cc} (1 - p_\mathrm{ae}) \left( 1 - \frac{\tau\set + \tau\alg + \tau\ack}{\tau}\right) B_d \,  \eta. %\log\left(1 + \gamma \right).
\end{equation}
In the following subsections, we analyze the terms involved in eqs.~\eqref{eq:utility} describing the difference between the transmission paradigms and particularizing the analysis for the different \glspl{cc}.

\subsection{Overhead evaluation}
\label{sec:overhead}

\begin{figure*}[tbh]
\centering
    \begin{subfigure}{\textwidth}
        \centering
        \includegraphics[width=\textwidth]{figs/data-oce.pdf}        
        \caption{\gls{oce}}
    \end{subfigure}
    \begin{subfigure}{\textwidth}
        \centering
        \includegraphics[width=\textwidth]{figs/data-bsw-fixed.pdf}
        \caption{\gls{bsw}: fixed frame structure}
    \end{subfigure}
    \begin{subfigure}{\textwidth}
        \centering
        \includegraphics[width=\textwidth]{figs/data-bsw-flexi.pdf}
        \caption{\gls{bsw}: flexible frame structure}
    \end{subfigure}       
    \caption{Frame structure for the communication paradigms under study. Packets colored in \textcolor{blue}{blue} and in \textcolor{yellow}{yellow} have \gls{dl} and \gls{ul} directions, respectively. Remark that SET-R (ACK-R) packet and its feedback are sent at the same time as the SET-U (ACK-U) but on different resources if \gls{obcc} is present.}
    \label{fig:data-frames}
\end{figure*}

Following the description given in Sect.~\ref{sec:paradigms}, we show the frame structures of the communication paradigms under study in Fig.~\ref{fig:data-frames}, where the rows represent the time horizon of the packets travelling on the different channels (first three rows) and the configuration loading time at the \gls{risc} (last row). The time horizon is obtained assuming that all the operations span multiple numbers of \glspl{tti}, each of duration of $T$ seconds, where $\lceil\tau / T\rceil \in \mathbb{N}$ is the total number of \glspl{tti} in a frame. 
At the beginning of each \gls{tti}, if the \gls{risc} loads a new configuration, the first $\tau_s$ seconds of data might be lost, due to the unpredictable response of the channel during this switching time. When needed, we will consider a guard period of $\tau_s$ seconds in the overhead evaluation to avoid data disruption. Remember that the \gls{risc} loads the widebeam crtl configuration any time it is in an idle state, \emph{i.e.}, at the beginning of the Setup and Acknowledgement phases (see the ``RISC'' row of Fig.~\ref{fig:data-frames}).

From Fig.~\ref{fig:data-frames}, we can note that the overhead generated by Setup and Acknowledgement phases is \emph{communication paradigm independent}\footnote{The reliability of the control packets exchanged is still dependent on the communication paradigm, see Section~\ref{sec:reliability}.}, while it is \emph{\gls{cc} dependent}. 
Indeed, all the paradigms make use of $P = P_u + P_r = 4$ control packets, $P_u = 2$ control packets sent on the \gls{ue}-\gls{cc} and $P_r = 2$ on \gls{ris}-\gls{cc}. Nevertheless, the kind of \gls{ris}-\gls{cc} employed can reduce the time employed for the communication of those packets. 
On the other hand, the Algorithmic phase is \emph{\gls{cc} independent} and \emph{communication paradigm dependent} being designed to achieve the goal of the communication paradigm itself.
In the following, the overhead evaluation is performed for the various cases of interest.

\subsubsection{Setup phase}
This phase starts with the SET-U control packet sent on the \gls{ue}-\gls{cc}, informing the \gls{ue} that the \gls{oce} procedure has started. 
If an \gls{ibcc} is employed, this is followed by the transmission of the SET-R packet to the \gls{risc} notifying the start of the procedure, and a consequent \gls{tti} for feedback is reserved to notify back to the \gls{bs} if the SET-R packet has been received. %, while no feedback \gls{tti} is reserved in case of \gls{ibno}. 
If an \gls{obcc} is employed, no \gls{tti} needs to be reserved because the SET-R and its feedback are scheduled at the same time as the SET-U packet but on different resources. This is compliant with the assumption of error-free \gls{cc} made on the definition of \gls{ris}-\gls{obcc} in Sect.~\ref{sec:model}.
Accordingly, the Setup phase duration is
\begin{equation} \label{eq:setup-time}
    \tau\set = 
    \begin{cases}
        T, \quad \text{\gls{obcc}}, \\
        % 2 T, \quad \text{\gls{ibno}}, \\
        3 T, \quad \text{\gls{ibcc}}.
    \end{cases}
\end{equation}

\subsubsection{Acknowledgement phase}
 The time needed to acknowledge the \gls{ue} and the \gls{risc} follows the setup phase: after the optimization has run, an acknowledgment (ACK-U) packet spanning one \gls{tti} is sent to the \gls{ue} notifying it to prepare to send the data; then, if a \gls{ibcc} is present, a \gls{tti} is used to send an \gls{ris} acknowledgment (ACK-R) packet containing the information of which configuration to load during the Payload phase; a further \gls{tti} is reserved for feedback. In the Setup phase, if a \gls{obcc} is present, no \gls{tti} needs to be reserved because the ACK-R and its feedback are scheduled at the same time as the SET-U packet but on different resources. Remark that the $\tau_s$ guard period must be considered by the \gls{ue} when transmitting the data, to avoid data being disrupted during the load of the configuration employed in the Payload. For simplicity of evaluation, we insert the guard period in the overall Acknowledgement phase duration, resulting in
\begin{equation} \label{eq:oce:ack-time}
    \tau_\mathrm{ack} = \tau\set + \tau_s
    % \begin{cases}
    %     T + \tau_s, \quad \text{\gls{obcc}}, \\
    %     2T + \tau_s, \quad \text{\gls{ibno}}, \\
    %     3T + \tau_s, \quad \text{\gls{ibwf}}.
    % \end{cases}
\end{equation}

\subsubsection{Algorithmic phase}
This phase comprises the process of sending pilot sequences and the consequent evaluation of the configuration for the transmission. 
Regardless of the paradigm, each pilot sequence spans an entire \gls{tti}, but the switching time of the configuration must be taken into account as a guard period. Therefore, the actual time occupied by a pilot sequence is $\tau_p \le T - \tau_s$ and the number of samples $p$ of every pilot sequence results
\begin{equation}
    p = \left\lfloor \frac{T - \tau_s}{T_n} \right\rfloor,
\end{equation}
where $T_n$ is the symbol period in seconds. Assuming that the \gls{tti} and the symbol period are fixed, the \gls{ue} is able to compute the pilot length if it is informed about the guard period.
On the other hand, the overall duration of the Algorithmic phase depends on the paradigm employed.

\paragraph{\gls{oce}} In this case, the Algorithmic phase starts with $C\oce$ \glspl{tti}; at the beginning of each of these \glspl{tti}, the \gls{risc} loads a different configuration, while the \gls{ue} transmits replicas of the pilot sequence. After all the sequences are received, the \gls{ce} process at the \gls{bs} starts, followed by the configuration optimization. The time needed to perform the \gls{ce} and optimization processes depends on the algorithm, as well as the available hardware. To consider a generic case, we denote this time as $\tau_A = A T$.

\paragraph{\gls{bsw} fixed frame structure}
Similarly to the previous case, the Algorithmic phase starts with $C\bsw$ \glspl{tti}, at the beginning of which the \gls{risc} loads a different configuration, and the \gls{ue} transmits replicas of the pilot sequence. After all the sequences are received, the \gls{bs} selects the configuration as described in Sect.~\ref{sec:communication-paradigms:bsw}. The time needed to select the configuration is considered negligible, and hence the Acknowledgement phase may start in the \gls{tti} after the last pilot sequence is sent.

\paragraph{\gls{bsw} flexible frame structure}
In this case, the number of \glspl{tti} used for the beam sweeping process is not known \emph{a priori} and it depends on the measured \gls{snr}. 
However, to allow the system to react in case the desired threshold is reached, a \gls{tti} is reserved for the transmission of the ACK-U after each \gls{tti} used for pilot transmission. Hence, the number of \glspl{tti} needed is $2 c^\star - 1$, where $0< c^\star \le C\bsw$ is a random variable.

According to the previous discussion, the Algorithmic phase duration is
\begin{equation} \label{eq:algorithmic-time}
    \tau\alg = 
    \begin{cases}
        (C\oce + A) T, \quad &\text{\gls{oce}}, \\
        C\bsw T, \quad &\text{\gls{bsw} fixed frame structure}, \\
        (2 c^\star - 1) T, \quad &\text{\gls{bsw} flexible frame structure}.
    \end{cases}
\end{equation}

\subsection{Reliability evaluation}
\label{sec:reliability}

In this section, we evaluate the reliability of the control packets. The content of each control packet depends on the kind of control packet considered and on the communication paradigms employed, as we will describe throughout the section. Without loss of generality, we assume that the $i$-th control packet toward entity $k$ comprises a total of $b_i^{(k)}$ informative bits. Accordingly, we can express the probability of error of a single packet by means of an outage event, obtaining
\begin{equation} \label{eq:outagepe}
    p_i^{(k)} = \Pr\left\{ \log \left(1 + \Gamma_k \right) \le \frac{b_i^{(k)}}{\tau_{i}^{(k)} B_{c}} \right\}, \quad k\in\{u,r\}, \quad i =\{1,2\}
\end{equation}
where $\tau_{i}^{(k)}$ is the time employed by the transmission of the $i$-th control packet intended for entity $k\in\{u,r\}$, and $B_c$ is the \gls{cc} transmission bandwidth. According to the assumption on the channel distribution, eq.~\eqref{eq:outagepe} can be rewritten as
\begin{equation} \label{eq:outagepe2}
    p_i^{(k)} = 1 - \exp\left\{- \frac{1}{\lambda_k} \left(2^{b_i^{(k)} / \tau_{i}^{(k)} / B_{c}} - 1 \right) \right\}.
\end{equation}
Plugging eq.~\eqref{eq:outagepe2} into~\eqref{eq:pcc}, the correct control probability for the paradigms under tests results
\begin{equation} \label{eq:pcc2}
p_\mathrm{cc} = \exp\left\{ \frac{1}{\lambda_u} \left( 2 - \sum_{i=1}^2 2^{b_{i}^{(u)} / \tau_{i}^{(u)} / B_c}\right)\right\} \exp\left\{ \frac{1}{\lambda_r} \left( 2 - \sum_{i=1}^2 2^{b_{i}^{(r)} / \tau_{i}^{(r)} / B_c} \right) \right\}.
\end{equation}
In the following, we evaluate the time occupied by the transmission of the control packets $\tau_i^{(k)}$ and the number of informative bits contained in the control packets $b_i^{(k)}$.

\subsubsection{Useful time for control packets}
\label{sec:usefultime}
Following the data frame, each control packet spans an entire \gls{tti}. However, the \emph{useful time} $\tau_{i}^{(k)}$, \emph{i.e.}, the time in which informative bits can be sent without risk to be disrupted, depends on the \gls{ris} switching time. As already discussed in Sect.~\ref{sec:overhead}, a guard period of $\tau_s$ seconds must be considered if the \gls{risc} loads a new configuration in that \gls{tti}. Following the frame structure of Fig.~\ref{fig:data-frames}, the SET-R and ACK-R packets can use the whole \gls{tti}, while the SET-U packets need the guard period. Note that the ACK-U control packet does not employ the guard period under the \gls{oce}, as long as the time employed by the optimization process is at least a \gls{tti}, \emph{i.e.}, $A \ge 1$. For the \gls{bsw} paradigm, the guard period is needed.
As a consequence, the useful time of the packets intended for the \gls{ue} results
\begin{equation}
    \begin{aligned}
        \tau_{1}^{(u)} &= T - \tau_s, \qquad
        \tau_2^{(u)} &= 
        \begin{cases}
            T- \tau_s, \quad &\text{\gls{bsw}}, \\
            T, \quad &\text{\gls{oce}},
        \end{cases}
    \end{aligned}
\end{equation}
while the useful time of the packets intended for the \gls{risc} results
\begin{equation}
    \tau_{1}^{(r)} = \tau_{2}^{(r)} = T.
\end{equation}

\subsubsection{Control packet content}
\label{sec:bits}

\begin{figure}
    \centering
    \includegraphics[height=1.7cm]{figs/control-packet.pdf}
    \caption{General control packet structure comprising a preamble and a payload part.}
    \label{fig:packet-structure}
\end{figure}

In this part, we evaluate the minimum number of informative bits $b_i^{(k)}$ of each control packet.
Without loss of generality, we can assume a common structure for all the control packets, comprising a control preamble and control payload parts as depicted in Fig.~\ref{fig:packet-structure}. 
The preamble comprises $b^\mathrm{ID}$ bits representing the \emph{unique identifier (ID)} of the destination entity in the network, and a single bit flag specifying if the packet is a SET or an ACK one. From the preamble, the \gls{ue} (\gls{risc}) can understand if the control packet is meant to be decoded and how to interpret the control payload. Accordingly, the remaining number of  bits, $b_i^{(k)} - 1 - b^\mathrm{ID}$, depends on the control payload, which, in turn, depends on the kind of control packet considered and on the communication paradigms employed.

\paragraph{\gls{oce}}
To initialize the overall procedure at the \gls{ue}, the payload of the SET-U packet must contain the length of the frame $\tau$, the cardinality of the set $C\oce$, and the guard time $\tau_s$. To simplify the data transmission, the frame duration can be notified through an (unsigned) integer of $b^\mathrm{frame}$ containing the number of total \glspl{tti} $\lceil \tau / T \rceil$ set for the frame. 
In the same manner, we can translate the guard time into an unsigned integer representing the number of guard symbols $\lceil \tau_s / T_n \rceil$ so as to send $b^\mathrm{guard}$ bits.
Finally, another integer of $b^\mathrm{conf}$ bits can be used to represent the cardinality $C\oce$ and to notify it to the \gls{ue}. Remark that the minimum $b^\mathrm{conf} = \lfloor \log_2(C) \rfloor$, where $C$ is the total number of configurations stored in the common codebook.
Similarly, the payload of the SET-R packets needs to contain the information of the length of the frame $\tau$, and the \emph{set} of configuration  $\mc{C}\oce$ to switch through.
Also, in this case, encoding the data as integers may reduce the number of informative bits to transmit. For the frame length, the same $b^\mathrm{frame}$ bits of the SET-U packet are used. To encode the information of the set to be employed, $b^\mathrm{conf}$ bits are used to identify a single configuration in the common codebook, and thus $C\oce b^\mathrm{conf}$ needs to be transmitted to the \gls{risc}, one per wanted configuration.
Regarding the Acknowledgement phase, we can assume that the payload of the ACK-U contains only the chosen \gls{se} of the communication $r\oce$. This can be encoded in similar manner the \gls{mcs} is encoded for the 5G standard~\cite{3gpp:rel15}: a table of predefined values indexed by $b^\mathrm{SE}$ bits.
On the other hand, the payload of the ACK-R must contain the optimal phase-shift profile $\bm{\phi}_\star$, that is, a value of the phase-shift for each element. Without loss of generality, we can denote as $b^\mathrm{quant}$ the number of bits used to control each element, \emph{i.e.}, the level of quantization of the \gls{ris}~\cite{EURASIP_RIS}. Hence, the overall number of informative bits is the number of elements to control times the level of quantization, \emph{i.e.}, $N b^\mathrm{quant}$.
To summarize, the packets length results
\begin{equation}
    b_{i}^{(k)} = b^\mathrm{ID} + 1 +
    \begin{cases}
        b^\mathrm{frame} + b^\mathrm{guard} + b^\mathrm{conf}, \quad &k=u, \, i= 1, \text{ (SET-U)},\\
        b^\mathrm{frame} + C\oce b^\mathrm{conf}, \quad &k= r, \, i=1,  \text{ (SET-R)}, \\ 
        b^\mathrm{SE}, \quad &k= u, \, i=2,  \text{ (ACK-U)}, \\
        N b^\mathrm{quant}, \quad &k= r, \, i=2,  \text{ (ACK-R)}.
    \end{cases}
\end{equation}


\paragraph{\gls{bsw}}
The payload of the Setup packets follows the same scheme used for the \gls{oce} paradigm. 
The SET-U packet contains the length of the frame $\tau$, the cardinality of the set $C\bsw$, and the guard time $\tau_s$ translated to (unsigned) integer of $b^\mathrm{frame}$, $b^\mathrm{guard}$ and $b^\mathrm{conf}$ bits, respectively.
The payload of the SET-R packets contains the information of the length of the frame $\tau$, and the \emph{set} of configuration  $\mc{C}\bsw$ to switch through, encoded in (unsigned) integers of $b^\mathrm{frame}$ and $C\bsw b^\mathrm{conf}$ bits, respectively.
Instead, the Acknowledgement contains different information. In particular, the payload of the ACK-U is empty, according to the fixed rate transmission used by this paradigm. The payload of the ACK-R contains the configuration $c^\star$ chosen, encoded by the same $b^\mathrm{conf}$ bits representing an index in the common codebook.
To summarize, the packets length results
\begin{equation}
    b_{i}^{(k)} = b^\mathrm{ID} + 1 +
    \begin{cases}
        b^\mathrm{frame} + b^\mathrm{guard} + b^\mathrm{conf}, \quad &k=u, \, i= 1, \text{ (SET-U)},\\
        b^\mathrm{frame} + C\bsw b^\mathrm{conf}, \quad &k= r, \, i=1,  \text{ (SET-R)}, \\ 
        0, \quad &k= u, \, i=2,  \text{ (ACK-U)}, \\
        b^\mathrm{conf}, \quad &k= r, \, i=2,  \text{ (ACK-R)}.
    \end{cases}
\end{equation}