Let us consider the simple \gls{ul} scenario depicted in Fig.~\ref{fig:system-model:scenario}, which consists of a single-antenna \gls{bs}, a single-antenna \gls{ue}, and a fully-reflective \gls{ris}. The \gls{ris} has $N$ elements equally spaced on a planar surface. Each \gls{ris} element is able to change the phase shift of an incoming wave by $\varphi_n \in [0,2\pi]$, $\forall n\in\mc{N}$\footnote{For the sake of simplicity and following the standard practice in literature, we consider an ideal \gls{ris} to show the theoretical performance achievable by the system at hand. We expect that more realistic models addressing attenuation, mutual coupling, and non-linear effects would reduce the overall performance~\cite{Tsinghua_RIS_Tutorial}.}. We denote as $\bm{\phi}= [e^{j\varphi_1}, \dots,  e^{j\varphi_N}]\T\in \mathbb{C}^{N}$ the vector representing a particular \emph{configuration} of the phase shifts of the elements at a given time. A \gls{risc} is in charge of loading different configurations to the \gls{ris} surface. The \gls{risc} is equipped with a look-up table containing a set of predefined configurations $\mc{C}$, $|\mc{C}| = C$; a copy of this look-up table is stored in the \gls{bs}, % in order to have shared knowledge of the predefined configurations. In this way, 
so that the \gls{bs} can send control signals to instruct the \gls{risc} to load a configuration already stored in $\mc{C}$. Therefore, the set $\mc{C}$ is the so-called \emph{common codebook} of configurations. Remark that the \gls{bs} can also issue a command to load a configuration not present in the common codebook by sending the explicit phase-shift values for each \gls{ris} element.

To study the impact of the control signals on communication, we focus on characterizing three narrowband\footnote{The narrowband assumption of the channel is considered to simplify the analysis done throughout the paper in order to focus on the timing of the operations needed in the control and data channels to successfully perform a \gls{ris}-aided wireless transmission, as specified in Sects.~\ref{sec:paradigms} and~\ref{sec:ris-control}. Nevertheless, the system aspects of the control channel can be carried over to a wideband or \gls{ofdm} case.} wireless channels: \emph{a}) the \gls{ue}-\gls{dc}, where the \gls{ue} sends payload data to the \gls{bs}, \emph{b}) the \gls{ue}-\gls{cc}, in which the \gls{bs} and the \gls{ue} can share control messages to coordinate their communication, and \emph{c}) the \gls{ris}-\gls{cc} that connects the \gls{bs} to the \gls{risc} so that the former can control the operation of the latter. Fig.~\ref{fig:system-model:scenario} illustrates the channels further detailed below.

\begin{figure}
    \centering
    \begin{subfigure}{0.45\textwidth}
        \centering
        \includegraphics[width=7cm]{figs/ris-data-scenario.pdf}    
        \caption{Data flow.}
    \end{subfigure}
    ~
    \begin{subfigure}{0.45\textwidth}
        \centering
        \includegraphics[width=7cm]{figs/ris-control-scenario.pdf}    
        \caption{Control flow.}
    \end{subfigure}    
    \caption{Scenario of interest: \gls{ris} extends the coverage of the \gls{bs}, which has a blocked link to the \gls{ue}. During data transmission, the \gls{risc} loads a configuration aiming to achieve a certain communication performance; during control signaling, the \gls{risc} loads a wide beamwidth configuration to deliver low-rate control packets to the \gls{ue}.}    
    \label{fig:system-model:scenario}
\end{figure}

\paragraph{UE-DC} 
This narrowband channel operates with frequency $f_d$ and bandwidth $B_d$.
The \gls{ul} \gls{snr} over the data channel is
\begin{equation} \label{eq:snr:uedc}
    \gamma = \frac{\rho_u}{\sigma_b^2} |\bm{\phi}\T  (\mb{h}_d \odot \mb{g}_d)|^2 = \frac{\rho_u}{\sigma_b^2} |\bm{\phi}\T \mb{z}_d|^2,
\end{equation}
where $\mb{h}_d\in\mathbb{C}^N$ is the data channel from the \gls{ue} to the \gls{ris}, while $\mb{g}_d\in\mathbb{C}^N$ defines the one from the \gls{ris} to the \gls{bs}. For simplicity of notation, we define the equivalent channel as $\mb{z}_d = (\mb{h}_d \odot \mb{g}_d) \in \mathbb{C}^{N}$. The \gls{ue} transmit power is $\rho_u$ and $\sigma_b^2$ is the noise power at the \gls{bs} \gls{rf} chain. In the remainder of the paper, we assume that the \gls{bs} knows the values of $\rho_u$ and $\sigma_b^2$: the transmit power is usually determined by the protocol or set by the \gls{bs} itself; the noise power can be considered static for a time horizon longer than the coherence time and hence estimated previously through standard estimation techniques, \emph{e.g.},~\cite{Yucek2006noise}. 

\paragraph{UE-CC} 
This narrowband channel operates on central frequency $f_u$ and bandwidth $B_u$ and is assumed to be a wireless \gls{ibcc}, meaning that the the physical resources employed for the \gls{ue}-\gls{cc} overlaps the one used for the \gls{ue}-\gls{dc}, \emph{i.e.}, $f_u = f_d$, and $B_u \ge B_d$. To ensure that the \gls{ue} control messages reflected by the \gls{ris} reach the \gls{bs} and vice-versa, we consider that the \gls{ris} makes use of a wide beamwidth configuration, termed \emph{\gls{ctrl} configuration}. Wide beamwidth radiation patterns generally offer increased robustness in terms of outage probabilities when low data rates are needed~\cite{RISBroadCoverage}, which makes them an ideal choice for \gls{ctrl} configurations. Without loss of generality, we assume that the \gls{risc} loads the \gls{ctrl} configuration anytime it is in an idle state. In other words, if the \gls{risc} has not been triggered to load other configurations, the \gls{ctrl} configuration is loaded.
In those cases, the \gls{ue}-\gls{cc} channel is described by
\begin{equation} \label{eq:channel:uecc}
    h_{cu} = \bm{\phi}\T_\mathrm{ctrl} (\mb{h}_c \odot \mb{g}_c) = \bm{\phi}\T_\mathrm{ctrl} \mb{z}_c,
\end{equation}
where $\bm{\phi}_\mathrm{ctrl}$ is the \gls{ctrl} configuration and $\mb{h}_c\in\mathbb{C}^N$ and $\mb{g}_c\in\mathbb{C}^N$ are the \gls{ue}-\gls{ris} and \gls{ris}-\gls{ue} \glspl{cc}, respectively. The equivalent end-to-end channel is $\mb{z}_c = \mb{h}_c \odot \mb{g}_c \in \mathbb{C}^N$. Herein, we do not focus on designing the \gls{ctrl} configuration, whose design can be based on other works (\emph{e.g.}, see the configuration design proposed in~\cite{alexandropoulos2022hierarchical}). Instead, we assume that the above channel in~\eqref{eq:channel:uecc} is Gaussian distributed as $h_{cu} \sim \mc{CN}(0, \tilde{\lambda}_u)$ with $\tilde{\lambda}_u$ being a term accounting for the (known) large-scale fading dependent on the \gls{ctrl} configuration. Hence, the \gls{snr} measured at the \gls{ue} is
\begin{equation} \label{eq:snr:uecc}
    \Gamma_{u} = \frac{\rho_b}{\sigma_u^2} |h_{cu}|^2 \sim \text{Exp}\left(\frac{1}{\lambda_u}\right), 
\end{equation}
where $\lambda_u = \frac{\rho_b \tilde{\lambda}_u}{\sigma^2_u}$ denotes the average \gls{snr} at the \gls{ue}, being $\rho_b$ the \gls{bs} transmit power and $\sigma^2_u$ the \gls{ue}'s \gls{rf} chain noise power. 

\paragraph{RIS-CC} 
We assume that the \gls{ris}-\gls{cc} is narrowband having central frequency $f_r$, bandwidth $B_r$, and channel coefficient denoted as $h_{cr}\in\mathbb{C}$. To obtain simple analytic results, we assume that the \gls{bs}-\gls{risc} coefficient can be modeled as $h_{cr} \sim \mc{CN}(0, \tilde{\lambda}_r)$ where $\tilde{\lambda}_r$ accounts for the large-scale fading, assumed known. Therefore, the \gls{snr} measured at the \gls{risc} results
\begin{equation} \label{eq:snr:riscc}
    \Gamma_{r} = \frac{\rho_b}{\sigma_r^2} |h_{cr}|^2 \sim \text{Exp}\left(\frac{1}{\lambda_r}\right),
\end{equation}
where $\lambda_r = \frac{\rho_b \tilde{\lambda}_r}{\sigma_r^2}$ denotes the average \gls{snr} at the \gls{risc} receiver, being $\sigma^2_r$ the noise power at its \gls{rf} chain. Remark that this channel can be either: $i$) \gls{ibcc} meaning that the physical resources employed by the \gls{ue}-\gls{dc} are overlapped by the one used by the \gls{ris}-\gls{cc}, \emph{i.e.}, $f_r = f_d$, $B_r \ge B_d$;  or $ii$) \gls{obcc}, where the physical resources of the control channel are not consuming degrees of freedom from the data-transmission resources, thereby simulating a cabled connection between the decision maker and the \gls{risc}. %In particular, we analyze the \gls{ibcc} with and without feedback capabilities, which are referred to as \gls{ibwf} and \gls{ibno}, respectively. 
In the case of \gls{obcc}, we further assume that the \gls{ris}-\gls{cc} is an error-free channel with feedback capabilities, \emph{i.e.}, $\lambda_r\rightarrow\infty$. This is reasonable to assume, as, once it is decided to use \gls{obcc}, the system designer has a large pool of reliable options.

% while the \gls{bs} waits for control messages from the \gls{ue} it commands the \gls{ris} to load a wide beam-width configuration, namely the \emph{control configuration}, (\emph{e.g.}, by using the design proposed in~\cite{alexandropoulos2022hierarchical}).
% Being interested in studying the impact of the \gls{ris}-\gls{cc}, we assume that the \gls{ue}-\gls{cc} is error-free as long as the control configuration is loaded correctly, based on the fact that \gls{ue} control messages employs a very low communication rate.\footnote{The impact of the UE-CC errors will be studied in a future work.}

% \FS{With Kyriakos, we discuss about this and realize that at least for the first packet is very difficult to make the previous assumption work. Therefore, we may define a control configuration $\bm{\theta}$ that provide an overall channel
% \begin{equation}
%     h_{cu} = \bm{\theta}\T \mb{z}_c
% \end{equation}
% where $\mb{z}_c = \mb{h}_c \circ \mb{g}_c$ is the equivalent end-to-end channel.  Then we can even assume that $h_{cu} \sim \mc{CN}(0, \lambda_u(\bm{\theta}))$ where $\lambda_u(\bm{\theta})$ is a term that account for the large scale fading + the impact of the control configuration. 
% }




