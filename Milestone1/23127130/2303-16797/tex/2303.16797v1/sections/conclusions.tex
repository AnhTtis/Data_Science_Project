In this paper, we proposed a general framework of four phases -- Setup, Algorithmic, Acknowledgement, Payload -- to evaluate \gls{ris}-aided communication performance, addressing the impact of control and signaling procedures in a generic scenario. Employing this framework, we detailed the data exchange and the frame structure diagram for two different communication paradigms employed in \gls{ris}-aided communications, namely \gls{oce} and \gls{bsw}. We analyzed the performance considering a utility function that takes into account the overhead generated by the various phases of the paradigms, the possible errors coming from the Algorithmic phase, and the impact of losing control packets needed for signalling purposes. Moreover, we particularized the performance evaluation for two kinds of \glspl{cc} connecting the decision maker and the \gls{risc} --  \gls{ibcc} and \gls{obcc} --, showcasing the minimum performance needed to obtain the desired target control reliability.
While some oversimplification has necessarily been introduced, we believe that the proposed framework can be used to include the control operations into the communication performance evaluation for various scenarios of interest. For example, the framework can be applied to multi-user wideband/\gls{ofdm} communications by accounting for the subcarrier allocation of the different control and payload messages. Differently from the cases studied in this paper, the Algorithmic phase should also consider the resource allocation process, whose output should be signaled to the \glspl{ue} through a specific design of the Acknowledgement phase. Other potential control and algorithmic designs can be addressed by using the proposed framework, taking care of omitting, merging, or repeating some of the general phases to meet the design requirements.
% - Include a comment about how to generalize to a wideband/OFDM case
% - Include a comment how other potential designs of control channel can use this framework by e.g. omitting or merging some phases.