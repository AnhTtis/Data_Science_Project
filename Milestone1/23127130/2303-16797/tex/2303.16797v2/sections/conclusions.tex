In this paper, we proposed a general framework of four phases (Initialization, Algorithmic, Setup, and Payload) to evaluate \gls{ris}-enabled communication performance while addressing the impact of control and signaling procedures. The data exchange and the frame structure for two different communication paradigms, namely \gls{oce} and \gls{bsw}, were detailed. The performance of the paradigms was analyzed employing a utility function that considers the overhead generated by its various phases, the possible errors coming from the Algorithmic phase, and the impact of losing control packets needed for signaling purposes. Moreover, we particularized the performance evaluation for two kinds of \glspl{cc} connecting the decision maker and the \gls{risc} --  \gls{ibcc} and \gls{obcc} --, showcasing the minimum performance needed to obtain the desired control reliability. \change{Possible extensions of the proposed framework %to include control signaling in communication performance evaluation of 
for more sophisticated scenarios of interest were discussed. Together with those extensions, in the future, we intend to study the impact of synchronization errors on the frame level and in the PHY-layer resources.}

%While some oversimplification has necessarily been introduced, we discussed how the proposed framework can be used to include the control operations in the communication performance evaluation for various scenarios of interest. \change{}

%For example, the framework can be applied to multi-user wideband/\gls{ofdm} communications by accounting for the subcarrier allocation of the different control and payload messages. Differently from the cases studied in this paper, the Algorithmic phase should also consider the resource allocation process, whose output should be signaled to the \glspl{ue} through a specific design of the Acknowledgement phase. Other potential control and algorithmic designs can be addressed by using the proposed framework, taking care of omitting, merging, or repeating some of the general phases to meet the design requirements.

%\vc{
%Future works: What happens if synchronization is not perfect?
%}