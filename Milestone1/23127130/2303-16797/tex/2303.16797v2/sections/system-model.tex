\change{
This section introduces a simplified communication model involving an \gls{ris}. We focus on the \gls{ul} scenario depicted in Fig.~\ref{fig:system-model:scenario}, which comprises a single-antenna \gls{bs}, a single-antenna \gls{ue}, and a solely reflective \gls{ris}~\cite{Tsinghua_RIS_Tutorial}. The primary aim of this system is to facilitate efficient data communication between the \gls{ue} and the \gls{bs} with the support of the \gls{ris}. To accomplish this, the \gls{bs} establishes control signaling links with both the \gls{ris} and the \gls{ue}. The details and the key assumptions are given below. % To achieve this, the \gls{bs} establishes control signaling connections with both the \gls{ris} and the \gls{ue}, which are further elaborated below where important assumptions are emphasized..
}

\begin{figure*}
    \centering
    \begin{subfigure}{0.4\textwidth}
        \centering
        \includegraphics[width=5cm]{figs/ris-control-scenario.pdf}    
        \caption{Control flow.}
    \end{subfigure}  
    \begin{subfigure}{0.4\textwidth}
        \centering
        \includegraphics[width=5cm]{figs/ris-data-scenario.pdf}    
        \caption{Data flow.}
    \end{subfigure}
    \caption{Setup of interest: an \gls{ris} extends the coverage of the \gls{bs} which has a blocked link to the \gls{ue}. During control signaling, the \gls{risc} loads a \glsfirst{ctrl} configuration to the \gls{ris} to deliver low-rate control packets to the \gls{ue}. During data transmission, the \gls{bs} controls and communicates with the \gls{risc} to load a certain configuration to achieve a desired communication performance.}    
    \label{fig:system-model:scenario}
    \vspace{-0.5cm}
\end{figure*}

\paragraph{RIS operation model}
\change{
The internal operations of the \gls{ris} are divided into the \gls{ris} panel and the \gls{risc}. The panel comprises $N$ elements equally spaced on a planar surface. The surface is solely reflective, implying that each element only controls the reflection properties of the incoming waves and that the \gls{ris} cannot process any of them. In particular, we focus on configuring the phases of the elements to change the reflection angle of an incoming wave, where their phase shifts are denoted as $\varphi_n \in [0,2\pi]$, $\forall n\in\mc{N}= \{1,\dots,N\}$\footnote{For the sake of simplicity and following the standard practice in literature, we consider an ideal \gls{ris} to show the theoretical performance achievable by the system at hand. We expect that more realistic models addressing attenuation, mutual coupling, quantization, and non-linear effects would reduce the overall performance~\cite{Tsinghua_RIS_Tutorial}.}. We denote as $\bm{\phi}= [e^{j\varphi_1}, \dots, e^{j\varphi_N}]\T\in \mathbb{C}^{N}$ the vector representing a particular \emph{configuration}, \emph{i.e.}, the set of phase shifts configured at the metasurface's elements at a given time. The \gls{risc} is in charge of loading different configurations to the \gls{ris} surface. Without loss of generality, we indicate as $\tau_s\in\mathbb{R}_+$ the time needed to switch to a new configuration. Moreover, the \gls{risc} is equipped with a communication module, which is used to exchange control signals with the \gls{bs}. We refer to the link connecting the \gls{risc} to the \gls{bs} as the \gls{ris}-\gls{cc}. We consider that the \gls{risc} stores a look-up table containing a set of predefined configurations, namely the \emph{common codebook} of configurations, $\mc{C}=\{\bm{\phi}_1,\dots,\bm{\phi}_C\}$, $|\mc{C}| = C$, that can be designed according to the task at hand; a copy of $\mc{C}$ is stored in the \gls{bs} as well. The control signals between the \gls{bs} and the \gls{risc} can then be based on indexing elements of $\mathcal{C}$ or by explicitly sending a configuration $\bm{\phi}$, if the \gls{bs} wants to load a configuration not present in the common codebook, \emph{i.e.}, $\bm{\phi}\notin\mc{C}$. In the next sections, $\mc{C}$ is going to be defined more explicitly according to \gls{oce} and \gls{bsw} paradigms.
}

\paragraph{UE communication model}\label{sec:system-model:ue-model}
\change{
To analyze the communication performance, we assume a frame-based fixed time system of $\tau\in\mathbb{R}_{+}$ seconds . The frame is further divided into phases that organize the signals exchanged at a given time. Within a frame, control and data information are exchanged in different phases. In each frame, the \gls{ue} communicate with the \gls{bs} through the \gls{ris}: it transmits payload data using the \gls{ue}-\gls{dc}, while the exchange of control messages uses the \gls{ue}-\gls{cc}. 
% The primary objective of the \gls{ul} setup illustrated in Fig.~\ref{fig:system-model:scenario} is to ensure effective communication between the \gls{ue} and the \gls{bs} with the assistance of the \gls{ris}, where control information is transmitted between the \gls{bs} and the \gls{ue} via the \gls{ue}-\gls{cc}, while the \gls{ue} sends payload information through the \gls{ue}-\gls{dc}, both transverses through the \gls{ris}. 
To ensure mathematical tractability, ideal timing and frequency synchronization among devices at the \gls{phy} layer is assumed. Perfect synchronization is also considered at the frame level across these devices. Also, we consider that the \gls{ris} can be configured at any time within a frame, and we make the following assumption about the behavior of the \gls{ris} when the \gls{bs} and the \gls{ue} exchange control information.
}

\begin{assumption}[\gls{ris} ctrl configuration]
\label{assu:ctrl}
   \change{To allow control signaling exchange between the \gls{bs} and the \gls{ue}, we consider that the \gls{ris} loads a \emph{\gls{ctrl} configuration} that ensures that control messages traveling through the \gls{ue}-\gls{cc} reach the destination\footnote{The design of the \gls{ctrl} configuration is out of the scope of this paper; potential candidates for \gls{ctrl} configurations are wide-width beams and hierarchical radiation patterns, which generally offer good reliability and coverage when low data rates are needed~\cite{RISBroadCoverage,alexandropoulos2022hierarchical}.}. Without loss of generality, we assume that the \gls{risc} loads the \gls{ctrl} configuration when in idle, \emph{i.e.}, anytime it does not receive any explicit command from the \gls{bs}.}
\end{assumption}
%\noindent Further details about the frame design are given in Section~\ref{sec:paradigms}.

%Herein, we do not focus on designing the \gls{ctrl} configuration, whose design can be based on other works (\emph{e.g.}, see the configuration design proposed in. 

\paragraph{Channel models}
\change{We adopt the block fading model at a frame level, implying that the channel conditions remain constant throughout the frame duration. Remark that this binds the frame duration $\tau$ to the channel coherence time: the higher the coherence time, the longer the frame duration. The following assumption is made on the channels, further described in the sequel.}

\begin{assumption}[Narrowband channels]
\label{assu:narrowband}
   \change{To allow a simple analysis of both control and data channels, the three channels under consideration -- \emph{1}) the \gls{ue}-\gls{dc}, \emph{2}) the \gls{ue}-\gls{cc}, and \emph{3}) the \gls{ris}-\gls{cc} -- are considered to be narrowband.\footnote{The analysis can be straightforwardly extended to a wideband transmission, as discussed in Section~\ref{sec:extension}.}}
\end{assumption}
% We model the three channels under consideration -- \emph{1}) the \gls{ue}-\gls{dc}, \emph{2}) the \gls{ue}-\gls{cc}, and \emph{3}) the \gls{ris}-\gls{cc} -- based on the narrowband assumption\footnote{The narrowband assumption is considered to simplify the analysis done throughout the paper, allowing us to investigate to successfully perform a \gls{ris}-aided wireless transmission, as specified in Sects.~\ref{sec:paradigms} and~\ref{sec:ris-control}. Nevertheless, the analysis can be straightforwardly extended to a wideband or \gls{ofdm} cases.}. %\vc{In particular, they are modeled as follows when considering a given frame.}

%To study the impact of the control signals on communication, we focus on characterizing three narrowband\footnote{The narrowband assumption of the channel is considered to simplify the analysis done throughout the paper so that we can analyze the control and data channels to successfully perform a \gls{ris}-aided wireless transmission, as specified in Sects.~\ref{sec:paradigms} and~\ref{sec:ris-control}. Nevertheless, the system aspects of the \gls{cc} can be easily extended to a wideband or \gls{ofdm} case.} wireless channels: \emph{a}) the \gls{ue}-\gls{dc}, where the \gls{ue} sends payload data to the \gls{bs}, \emph{b}) the \gls{ue}-\gls{cc}, in which the \gls{bs} and the \gls{ue} can share control messages to coordinate their communication, and \emph{c}) the \gls{ris}-\gls{cc} that connects the \gls{bs} to the \gls{risc} so that the former can control the operation of the latter. Fig.~\ref{fig:system-model:scenario} illustrates the channels further detailed below.

\subsubsection{UE-DC} 
This channel operates at a central frequency $f_d$ with a bandwidth of $B_d$. The \gls{ul} \gls{snr} can be calculated as: 
\begin{equation} \label{eq:snr:uedc}
    \gamma = \frac{\rho_u}{\sigma_b^2} |\bm{\phi}\T  (\mb{h}_d \odot \mb{g}_d)|^2 = \frac{\rho_u}{\sigma_b^2} |\bm{\phi}\T \mb{z}_d|^2,
\end{equation}
where $\mb{h}_d\in\mathbb{C}^N$ and $\mb{g}_d\in\mathbb{C}^N$ are the gains of \gls{ue}-\gls{ris} and \gls{ris}-\gls{bs} links, respectively. We further define the equivalent \gls{dc} as $\mb{z}_d = \mb{h}_d \odot \mb{g}_d \in \mathbb{C}^{N}$. The \gls{ue} transmit power is $\rho_u$, and $\sigma_b^2$ is the noise power at the \gls{bs} \gls{rf} chain\footnote{In the remainder of the paper, we assume that the \gls{bs} knows the transmit and noise powers denoted through this section, \emph{i.e.}, $\rho_u$, $\rho_b$, $\sigma_r^2$, $\sigma_u^2$ and $\sigma_b^2$: the transmit powers are usually determined by the protocol or set by the \gls{bs} itself; the noise powers can be considered static for a long time horizon and hence estimated previously through standard estimation techniques, \emph{e.g.},~\cite{Yucek2006noise}.}. \change{The configurations $\bm{\phi}$ supporting the \gls{ue}-\gls{dc} are subject to design and will be further specified in the following sections.}

\subsubsection{UE-CC} 
This channel operates at central frequency $f_u$ with a bandwidth of $B_u$. %As specified in Subsection \ref{sec:system-model:ue-model}, we assume that the control and payload information are exchanged in different times. Therefore, 
The \gls{ue}-\gls{cc} channel is defined as
\begin{equation} \label{eq:channel:uecc}
    h_{cu} = \bm{\phi}\T_\mathrm{ctrl} (\mb{h}_c \odot \mb{g}_c) = \bm{\phi}\T_\mathrm{ctrl} \mb{z}_c,
\end{equation}
where $\bm{\phi}_\mathrm{ctrl}$ is the \gls{ctrl} configuration (see Assumption~\ref{assu:ctrl}) and $\mb{h}_c\in\mathbb{C}^N$ and $\mb{g}_c\in\mathbb{C}^N$ are the gains of the \gls{ue}-\gls{ris} and \gls{ris}-\gls{bs} links, respectively. The equivalent end-to-end channel is $\mb{z}_c = \mb{h}_c \odot \mb{g}_c \in \mathbb{C}^N$. We consider a worst-case scenario where the channel~\eqref{eq:channel:uecc} has no \gls{los} component, \emph{i.e.}, it is distributed as $h_{cu} \sim \mc{CN}(0, \tilde{\lambda}_u)$; $\tilde{\lambda}_u$ is a term accounting for the large-scale fading -- known by the \gls{bs} -- which is dependent on the \gls{ctrl} configuration design. Hence, the instantaneous \gls{snr} measured at the \gls{ue} is
\begin{equation} \label{eq:snr:uecc}
    \Gamma_{u} = \frac{\rho_b}{\sigma_u^2} |h_{cu}|^2 \sim \text{Exp}\left(\frac{1}{\lambda_u}\right), 
\end{equation}
where $\lambda_u = \frac{\rho_b \tilde{\lambda}_u}{\sigma^2_u}$ denotes the average \gls{snr} at the \gls{ue}, being $\rho_b$ the \gls{bs} transmit power and $\sigma^2_u$ the \gls{ue}'s \gls{rf} chain noise power. 
\change{We make the following assumption on this channel.}
\begin{assumption} [\gls{ue}-\gls{cc} design]
    \label{assu:ue-cc}
    \change{
    We assume that the \gls{ue}-\gls{cc} operates as an \gls{ibcc}, meaning that the frequency $f_u$ matches the frequency $f_d$, and the physical resources allocated for the \gls{ue}-\gls{cc} coincide with those utilized for the \gls{ue}-\gls{dc}. This assumption is based on the premise that any \gls{ue}-\gls{cc} signal must travel through the \gls{ris} and the \gls{ris} operates at the same data frequency and bandwidth.
    }
\end{assumption}

%\colr{\textbf{WHY do we make this assumption?}}

\subsubsection{RIS-CC} 
This narrowband channel operates on central frequency $f_r$ with bandwidth $B_r$. Let $h_{cr}\in\mathbb{C}$ denote the channel coefficient of the \gls{ris}-\gls{cc}. To obtain simple analytical results, we assume that $h_{cr} \sim \mc{CN}(0, \tilde{\lambda}_r)$, where $\tilde{\lambda}_r$ accounts for the large-scale fading, assumed known by the \gls{bs}. Hence, the instantaneous \gls{snr} measured at the \gls{risc} is
\begin{equation} 
    \Gamma_{r} = \frac{\rho_b}{\sigma_r^2} |h_{cr}|^2 \sim \text{Exp}\left(\frac{1}{\lambda_r}\right),
    \label{eq:snr:riscc}
\end{equation}
where $\lambda_r = \frac{\rho_b \tilde{\lambda}_r}{\sigma_r^2}$ denotes the average \gls{snr} with $\sigma^2_r$ being the noise power at the \gls{risc} \gls{rf} chain. 
\change{We make the following assumption on the \gls{ris}-\gls{cc}.}
\begin{assumption}[\gls{ris}-\gls{cc} design]
    \label{assu:ris-cc}
    \change{The \gls{ris}-\gls{cc} can either be: $i$) \gls{ibcc}, implying that the physical resources employed by this channel are overlapping with the one used by the \gls{ue}-\gls{dc}, \emph{i.e.}, $f_r = f_d$;  or $ii$) \gls{obcc}, where the physical resources are orthogonal, \emph{e.g.}, simulating a wired connection between the \gls{bs} and the \gls{risc}. In the case of \gls{obcc}, we further assume that the \gls{ris}-\gls{cc} is an error-free channel, \emph{i.e.}, $\lambda_r\rightarrow\infty$, with feedback capabilities since the system designer can easily make the \gls{ris}-\gls{cc} as reliable as possible.}     
\end{assumption}

%\colr{\textbf{WHY do we make this assumption?}}

% while the \gls{bs} waits for control messages from the \gls{ue} it commands the \gls{ris} to load a wide beam-width configuration, namely the \emph{control configuration}, (\emph{e.g.}, by using the design proposed in~\cite{alexandropoulos2022hierarchical}).
% Being interested in studying the impact of the \gls{ris}-\gls{cc}, we assume that the \gls{ue}-\gls{cc} is error-free as long as the control configuration is loaded correctly, based on the fact that \gls{ue} control messages employs a very low communication rate.\footnote{The impact of the UE-CC errors will be studied in a future work.}

% \FS{With Kyriakos, we discuss about this and realize that at least for the first packet is very difficult to make the previous assumption work. Therefore, we may define a control configuration $\bm{\theta}$ that provide an overall channel
% \begin{equation}
%     h_{cu} = \bm{\theta}\T \mb{z}_c
% \end{equation}
% where $\mb{z}_c = \mb{h}_c \circ \mb{g}_c$ is the equivalent end-to-end channel.  Then we can even assume that $h_{cu} \sim \mc{CN}(0, \lambda_u(\bm{\theta}))$ where $\lambda_u(\bm{\theta})$ is a term that account for the large scale fading + the impact of the control configuration. 
% }




