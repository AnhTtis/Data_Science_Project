\Glspl{ris} constitute a promising technology for sixth generation (6G) wireless networks, which has received significant attention within the relevant research community in recent years~\cite{Huang2019}. The main underlying idea is to electronically tune the reflective properties of an \gls{ris} to manipulate the phase, amplitude, and polarization of the incident electromagnetic waves~\cite{risTUTORIAL2020}. This creates a propagation environment that is, at least, partially controllable~\cite{RISE6G_COMMAG}. \Glspl{ris} can be fabricated with classical antenna elements controlled through switching elements or, more advanced, with metamaterials having tunable electromagnetic properties~\cite{EURASIP_RIS}. In the 6G context, the \gls{ris} technology has been identified as one of the cost-effective solutions to address the increasing demand for higher data rates, reduced latency, and increased coverage. In particular, an \gls{ris} can improve the received signal strength and minimize interference by reflecting signals to intended receivers and away from non-intended ones; this leads to applications aiming for increased communication security~\cite{RIS_security_mal} and/or reduced electromagnetic field exposure~\cite{NA2021}. \glspl{ris} can also extend the coverage of wireless communication systems by reflecting the signals to areas that are difficult to reach using conventional means~\cite{Croisfelt2023}.

The dominant part of the literature concerning \gls{ris}-aided communication systems deals with \gls{phy} challenges. Recent studies have explored physics-based derivation of \gls{ris}-parametrized end-to-end channel models, incorporating causality, frequency selectivity, as well as any arising mutual coupling effects~\cite{PhysFad}. 
Many other papers have investigated the potential benefits of \gls{ris}-aided systems in terms of spectral and energy efficiencies by optimizing the parameters of the \gls{ris}' elements alone or jointly with the operations of the \gls{bs} (see, \emph{e.g.},~\cite{Tsinghua_RIS_Tutorial, massive_RIS}). \change{To enable such strategies, several works have focused on designing and evaluating \emph{\gls{ce} methods in the presence of \glspl{ris}, either relying on the observed equivalent end-to-end channel from the \gls{bs} to the \gls{ue}}, when dealing with solely reflective \glspl{ris}~\cite{CascadeCE_ProcIEEE, Mo2023ce}, or directly estimating individual channels -- \gls{bs}-to-\gls{ris} and \gls{ris}-to-\gls{ue} -- by using simultaneous reflecting and sensing \glspl{ris}~\cite{CE_HRIS_2023}. The latter \gls{ris} design belongs to the attempts to minimize the \gls{ce} overhead~\cite{HRIS_VTM}, which can be considerably large due to the expected high numbers of \gls{ris} elements~\cite{popov2021experimental} or hardware-induced non-linearities~\cite{Tsinghua_RIS_Tutorial}.} A different research direction bypasses explicit \gls{ce} and relies on \emph{\gls{bsw} methods}~\cite{singh2021fast}. Accordingly, the \gls{ris} is scheduled to progressively change among reflecting configurations from a predefined codebook so that the end-to-end system can discover the most suitable configuration~\cite{RISsweeping_2020, Jamali2022, alexandropoulos2022hierarchical, An2022}. \change{The predefined codebook can be practically optimized for different purposes~\cite{rahal_RISbeams}: a suitable approach is to use hierarchical codebook structures~\cite{HierarhicalCodebook, alexandropoulos2022hierarchical}.}

Within the existing research literature, the questions related to link/\gls{mac} protocol and system-level integration of \glspl{ris} have received much less attention than \gls{phy} topics. \change{\emph{Specifically, the literature lacks a systematic treatment of the control signaling aspects, required to remotely manage the behavior of the \glspl{ris} and the \glspl{ue}. Control signaling is often exerted by the \gls{bs} and relies upon well-defined \glspl{cc}.} The study of those procedures is central to integrating \glspl{ris} as a new type of network element within the existing wireless infrastructure. In this regard, we need to understand how the system performance is impacted by the design of the \glspl{cc} in terms of its rate, reliability as well as the overhead/trade-offs incurred by the control signaling procedures. To the best of our knowledge, this paper marks the initial attempt to systematically investigate the influence of control signaling on the performance of \gls{ris}-aided wireless systems.}

%%%%%
\subsection{Related literature}
%%%%%
\change{Most of the existing literature assumes that the control over an \gls{ris}-aided communication system is error-free and instantaneous. For example, in~\cite{Croisfelt2022randomaccess} and later~\cite{Croisfelt2023}, the authors presented a detailed protocol for random access aided by an \gls{ris}. It was showcased that, despite the physical overhead of switching its configurations, the \gls{ris} brings notable performance benefits, allowing more \glspl{ue} to access the \gls{bs} on average. However, these works ignored the impact of control signaling.} Similarly, in~\cite{PHY_Retransmission}, the effect of re-transmission protocols in \gls{ris}-aided systems in the case of erroneous transmission was studied, but the control impact was ignored.

\change{Other works did not even specify the required control information to be exchanged between the communication entities. For instance, one of the first works focusing on fast \gls{ris} programmability~\cite{RISsweeping_2020} presented a multi-stage \gls{bsw} protocol. By tasking the \gls{ris} to dynamically illuminate the area where a \gls{ue} is located, the authors of~\cite{Jamali2022} introduced a \gls{dl} transmission protocol, including \gls{ue} localization, \gls{ris} configuration, and pilot-aided end-to-end \gls{ce}.} In~\cite{alexandropoulos2022hierarchical}, a fast near-field alignment scheme was proposed for the \gls{ris} phase-shifts and the transceiver beamformer, relying on a variable-width hierarchical \gls{ris} phase configuration codebook. Recently,~\cite{An2022} discussed the overhead and challenges of integrating the \gls{ris} into the network,  arguing that the reduced overhead offered by \gls{bsw} schemes benefits the overall system performance. Nevertheless, the control signaling that needs to be exchanged for those schemes was not investigated.

%%%%%
\subsection{Contributions}\label{sec:intro:contributions}
%%%%%
\change{This paper aims to introduce a model quantifying the impact of control signaling on the performance of \gls{ris}-enabled wireless communications.}
The number of actual control options is subject to a combinatorial explosion due to the system's large number of configurable parameters, such as frame size or feedback design. Obviously, we cannot address all these designs in a single work, but what we are striving for is to get a simple\change{, yet generic, model for analyzing the impact of control that captures the essential design trade-offs and can be expanded to analyze other, more elaborate designs.} \change{For this reason, in this paper, we keep the complexity of the \gls{ris}-aided system model at the minimum, focusing on analyzing and evaluating the interaction between the control and the data planes. Considering a single-antenna \gls{bs} and narrowband communications, we analyze the communication performance when control signaling occurs over the \gls{ris}-\gls{cc} linking the \gls{bs} to the \gls{ris}, and the \gls{ue}-\gls{cc} that links the \gls{bs} to the \gls{ue} through the \gls{ris}. To generalize the proposed model, we further discuss, at the end of the paper, how to relax some of the assumptions made.} We build our generic control model along the two following dimensions that capture relevant trade-offs to be studied theoretically and which can be met in practice when deploying an \gls{ris}-aided communication system.

The \emph{first dimension} is shaped around evaluating the distinctions between traditional communication paradigms of multiplexing and diversity~\cite{Popovski2020}. In a \emph{multiplexing-oriented transmission}, \change{a \gls{kpi} of interest -- \emph{e.g.}, data rate --} is adapted to the actual channel conditions based on the \gls{csi} obtained via \gls{ce}. However, \gls{ce} \change{generally needs complex control signaling, implying a high control overhead.} \change{In an \gls{ris}-aided system, for example, a multiplexing transmission corresponds to a case in which the \gls{ris} configuration is purposefully configured to maximize the \gls{snr} of the cascaded end-to-end channel, based on acquired \gls{csi}, and the data rate is chosen accordingly. In a \emph{diversity transmission} scenario, the \gls{kpi} is pre-established, and the \gls{ue} relies on the expectation that the propagation environment will accommodate it. However, if the propagation environment does not meet these expectations, it can lead to an outage event, resulting in transmission failure.} \change{In an \gls{ris}-aided system, diversity transmission can be realized by using a \gls{bsw} procedure: while the \gls{ue} transmits, the \gls{bs} commands the \gls{ris} to successively load configurations, hoping that one of them will likely support the predefined \gls{kpi}.} \change{We refer to the paradigm of \gls{ris}-aided multiplexing transmission as \emph{\gls{oce}}, while \emph{\gls{bsw}} refers to the diversity paradigm.}

\change{The \emph{second dimension} regards how the resources used by the \glspl{cc} physically relate with the ones employed for the \gls{dc} used for data communication between the \gls{bs} and \glspl{ue}. We consider studying the following two options~\cite{bjornson2022reconfigurable}. First, an \emph{\gls{obcc}} uses communication resources that are orthogonal to the ones used by the \gls{dc}. More precisely, an \gls{obcc} exerts control over the propagation environment but is not affected by this control. Second, an \emph{\gls{ibcc}} uses the same communication resources as the \gls{dc}. This implies that the \gls{ibcc} reduces the available resources to transmit data, likely decreasing the performance of the overall system.} Furthermore, the successful transmission of the control messages toward the \gls{ris} depends on the current behavior of the \gls{ris} elements, meaning that the performance of the \gls{cc} is now dependent on how the \gls{ris} is configured.

\change{Our analysis suggests that employing either \gls{oce} or \gls{bsw} depends on the specific service requirements and the coherence time of the \gls{dc}.
If the scenario exhibits high channel coherence time, the \gls{oce} paradigm can potentially deliver superior data rate performance. However, its feasibility diminishes when the channel experiences rapid changes due to the increased control overhead. In contrast, the \gls{bsw} paradigm generally provides a lower data rate but proves suitable for scenarios with low coherence time due to its reduced control overhead. 
Additionally, our findings indicate that the reliability of control signaling is minimally affected by the paradigm used when the \gls{ris}-\gls{cc} is an \gls{obcc}. However, when an \gls{ibcc} serves as \gls{ris}-\gls{cc}, the results reveal that \gls{oce} control is less reliable than \gls{bsw} due to the increased complexity associated with the former in managing the \gls{ris}.}

The paper is organized as follows. Section~\ref{sec:model} presents the system model. 
Section~\ref{sec:paradigms} describes the transmission paradigms and their performance \emph{assuming error-free \glspl{cc}}. Section~\ref{sec:ris-control} shows how to analyze the impact of \glspl{cc} design in the communication performance, whose results are shown in Section~\ref{sec:results}.
Section~\ref{sec:extension} discusses how to relax some of the simplification assumptions made in this paper, while Section~\ref{sec:conclusions} concludes the paper.

\paragraph*{Notation}
Lower and upper case boldface letters denote vectors and matrices, respectively. \change{Calligraphic letters denote sets, whose cardinality is $|\cdot|$}. The Euclidean norm of $\mathbf{x}$ is $\lVert\mathbf{x}\rVert$; $\odot$ denotes the element-wise product. $\mc{P}(\cdot)$ is the probability of an event, $\mathbb{E}[\cdot]$ is the expected value; $\mc{CN}(\bm{\mu},\mb{R})$ is the complex Gaussian distribution with mean $\bm{\mu}$ and covariance matrix $\mb{R}$, $\mathrm{Exp}(\lambda)$ is the exponential distribution with mean value $1/\lambda$. $\lfloor a \rfloor$ is the nearest lower integer of $a$; $\mathbb{N}$ and $\mathbb{C}$ are the set of natural and complex numbers; $\Re\left\{\cdot\right\}$ returns the real part of a complex number, and $j\triangleq \sqrt{-1}$.
 
