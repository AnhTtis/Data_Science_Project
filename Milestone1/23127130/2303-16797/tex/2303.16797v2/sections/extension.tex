\change{The presented performance analysis framework paves the way for the control signaling design and quantification of more sophisticated \gls{ris}-empowered wireless systems. It can be applied, for example, to multi-user wideband/\gls{ofdm} communications~\cite{Huang2019,risTUTORIAL2020}, by accounting for the subcarrier allocation of the different control and payload messages. For this system setup, the Algorithmic phase needs to also consider the resource allocation process, whose output should be signaled to the \glspl{ue} through a specific design of the Setup phase. In addition, the current framework, by omitting, merging, or repeating some of its general phases, can set the basis for the control design in \gls{ris}-assisted networks with a multi-antenna \gls{bs} and multi-antenna \glspl{ue}, and smart wireless environments with multiple, possibly machine learning orchestrated, \glspl{ris}~\cite{RIS_pervasive_ML}, as well as shared \glspl{ris} among multiple communications pairs~\cite{EURASIP_RIS}. Of late interest are also multi-functional \glspl{ris}~\cite{Tsinghua_RIS_Tutorial}, and especially those possessing sensing capabilities~\cite{HRIS_VTM}, which may provide higher flexibility for efficient control signaling designs~\cite{croisfelt2023orchestration}.}

\change{We next elaborate in more detail on the case where the \gls{bs} is equipped with multiple antennas and there exists the possibility of a weak direct link between itself and the \gls{ue}. For the \gls{oce} communication paradigm, the \gls{ris} configuration and the \gls{bs} combiner can be jointly optimized~\cite{risTUTORIAL2020}, at the cost, of course, of a larger \gls{ce} overhead and complexity, as well as larger algorithmic complexity. It is noted, however, that the increased beamforming gain from the multiple \gls{bs} antennas might lead to cases where the \gls{bs}-\gls{ue} link is satisfactory, implying that the \gls{ris} deployment can be avoided, reducing the control overhead. For this mode to be realized, the operation protocol needs to enable, for example, the separate estimation of the \gls{ue}-\gls{ris}-\gls{bs} and \gls{ue}-\gls{bs} channels, via activation/deactivation of the \gls{ris} panel, as well as a relevant action during the Initialization phase.}   
\change{There exist various modes of operation when the \gls{bsw} paradigm is adopted. One is to perform \gls{BSW} at the \gls{bs} during the Initialization phase, together with \gls{BSW} at the \gls{ris}, again at a cost of a larger overhead for both the fixed and flexible frame structures. Alternatively, to reduce the control signaling overhead, the \gls{bs} combiner can be designed to solely match its channel with the \gls{ris}, or that with the \gls{ue} if the \gls{ris} can be avoided, as discussed in the \gls{oce} paradigm. One way to achieve the former is to capitalize on the common assumption that the \gls{ris} is placed such that there exists a strong line-of-sight with the BS~\cite{RIS_security_mal}. In this way, the \gls{bs} may adopt the reception configuration closest to maximal ratio combining. When the \gls{ris} can be avoided, the \gls{bs} combiner can be similarly designed, now for the channel towards the \gls{ue} - this operational mode can be decided similarly to the respective \gls{oce} case.}

\change{It is finally noted that the two presented communication paradigms, namely \gls{oce} and \gls{bsw}, can be combined to devise additional signaling schemes. One example is presented in~\cite{Jamali2022}, where the \gls{bsw} is performed to set the \gls{ris} configuration when the link budget falls below a certain threshold, and then \gls{oce} follows to set the BS combiner and/or the \gls{ue} beamformer, treating the configured \gls{ris} as an unknown scatterer.}



