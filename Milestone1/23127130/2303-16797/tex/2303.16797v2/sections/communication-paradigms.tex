\change{In this section, we first describe the structure and building blocks of a generic \gls{ris}-enabled communication paradigm. We use this to instantiate two particular transmission strategies: the \gls{oce} and the \gls{bsw}, representing the multiplexing and diversity paradigms discussed in Section~\ref{sec:intro:contributions}}.
\change{We analyze their performance in terms of the expected \gls{snr} and \gls{se}, while specifying the errors eventually occurring during their operations.}

\subsection{Generic structure}
\change{In a system without \glspl{ris}, a generic structure for a typical transmission strategy can be divided into three key phases occurring in every frame: ``\emph{Signaling},'' ``\emph{Algorithmic},'' and ``\emph{Payload}.'' The Signaling phase encompasses the actions conducted on the \glspl{cc} required for network node control. This phase relies on the quality of the \glspl{cc} and the information content within the control messages. The Algorithmic phase involves operations aiming at optimizing transmission parameters, such as selecting the transmit \gls{se}. The specifics of this phase are contingent on the chosen communication paradigm. The Payload phase handles the actual data transmission.}

\change{In an \gls{ris}-aided system, we identify that a generic structure would have two Signaling phases, namely ``\emph{Initialization}'' and ``\emph{Setup},'' in conjunction with the ``\emph{Algorithmic}'' and ``\emph{Payload}'' phases. Sequentially, we have: Initialization, Algorithmic, Setup and Payload}\footnote{We note that there could be communication strategies in which some of these phases may not be present, \emph{e.g.}, access procedures; however, the mentioned four phases set a basis for a sufficiently general framework for \gls{ris}-aided communication that can be used, in principle, to design other schemes where some of the steps are merged or omitted, as discussed in Section~\ref{sec:extension}.}. \change{The time of a frame is thus divided as $\tau=\tau\ini+\tau\alg+\tau\set+\tau\pay$, where $\tau_i < \tau$, with $i\in\{{\rm ini},{\rm alg},{\rm set},{\rm pay}\}$, is the time of the corresponding phase. The duration of each phase can vary depending on the paradigm and the kind of \glspl{cc}.} The phases are further elaborated as follows.
% Within each frame, we divide a generic \gls{ris}-aided communication paradigm into four (sequential) phases, namely , \emph{``Algorithmic''}, \emph{``Acknowledgement''}, and \emph{``Payload''}. Initialization and Setup are \emph{signaling} phases, \emph{i.e.}, phases having the main purpose of exchanging control information, and hence mainly dependent on the available \glspl{cc}.   % \vc{These phases occur in sequential order as given and are further elaborated below}

% Before proceeding in describing the two possible implementations of the protocol, we illustrate a simple mechanism for the detection of the end of the channel coherence frame. As stated before, we assume that the channel changes in a negligibly during the frame duration. Therefore, the RIS configuration and the user's transmission power are the only factors that affect the received \gls{snr} at the \gls{bs}. In the payload phase, the \gls{ris} configuration does not change and the \gls{bs} is assumed to know the average user's transmit power. Therefore, a running average of the received \gls{snr} can be kept by the \gls{bs}. When the running average is observed to differ substantially from the average \gls{snr} at the beginning of the payload phase, it is safe to assume that the channel coherence frame is at its end. At this point, the \gls{bs} re-sends the control information initializing the Initialization phase of the next frame.

\paragraph{Initialization} 
The \gls{bs} notifies the \gls{ris} and \gls{ue} about the beginning of a frame over the \glspl{cc}. It is assumed that the \gls{risc} loads the \gls{ctrl} configuration at the start of this phase. %The duration of this phase is denoted with $\tau\set < \tau$ and depends on the type of \glspl{cc}. 
\change{Although not considered here, the Initialization phase can also incorporate a random access procedure (see, \emph{e.g.},~\cite{Croisfelt2023}) as an intermediate step where newly connected \glspl{ue} are scheduled.}

\paragraph{Algorithmic} 
\change{
This phase encompasses all the processes and computations needed to optimize the subsequent \change{Payload phase}, where the actual data transmission takes place. % The algorithmic phase has a $\tau\alg < \tau$ duration that depends on the choice of the employed communication paradigm. 
Objectives of this phase encompass the selection/optimization of the appropriate \gls{ris} configuration(s), while others, such as determining the transmission parameters for the \gls{ue} and/or the \gls{bs}, could be included. To tackle these objectives, some form of wireless environment sensing from the end nodes is required, typically enabled by the transmission of pilot sequences, whose specifications -- their number, waveform, \gls{ul} or \gls{dl} transmission, etc -- are defined by the transmission strategy. The \gls{bs} can then use the collected pilot signals and invoke pre-defined algorithms to fulfil the objectives. The outcome of these algorithms might be affected by different types of \emph{algorithmic errors} that may prevent the system from performing as expected, and thus, should be considered when analyzing the overall performance.
}

\paragraph{Setup}
\change{
%The Setup phase starts once the Algorithmic phase ends; 
During this phase, the \gls{ris} configuration chosen during the Algorithmic phase needs to be communicated to the \gls{risc}, which in turn commands the \gls{ris} to load it. Additionally, further control signaling may occur between the \gls{bs} and the \gls{ue} as a final check before data transmission to, e.g., set the \gls{mcs}.} 
%It is implied that the \gls{risc} loads the \gls{ctrl} configuration at the beginning of this phase. 
%The acknowledgment phase duration $\tau\set < \tau$, depends on the type of \gls{cc} used.

\paragraph{Payload}
\change{Here, the actual data transmission takes place while the \gls{ris} loads the configuration specified before. This phase may or may not include feedback of the data sent by the \gls{ue} at the end; this aspect is not considered in this paper. The communication performance of the considered \gls{ris}-enabled communication system is measured during this phase.}

\change{In the following subsections, we describe two state-of-the-art paradigms using the generic structure defined above. After a general description, we investigate their Algorithmic phases, analyzing their performance and possible errors.} 
The first paradigm is the \emph{\gls{oce}}, which follows a multiplexing transmission: the \gls{ue}'s \gls{csi} is evaluated at the \gls{bs} and then exploited to compute the \gls{ris}' optimal configuration and the corresponding achievable data rate. \change{Then, the transmission is made using the optimized configuration and \gls{mcs}.} %\gls{risc} loads the optimal configuration to the \gls{ris}, and the \gls{ue} transmit the data using the chosen \gls{mcs}. 
The second approach is the \emph{\gls{bsw}}, defined as a communication paradigm in~\cite{An2022}, but already used in previous works (\emph{e.g.},~\cite{Jamali2022, alexandropoulos2022hierarchical, Croisfelt2022randomaccess}). \change{This paradigm resembles the concept of diversity transmission: the \gls{bs} selects a target \gls{kpi} \emph{a priori}; then, it instructs the \gls{risc} to sweep through a set of predefined configurations, expecting that at least one will satisfy the target \gls{kpi}; then, the transmission is made using the chosen configuration.}
%the \gls{bs} does not spend time figuring out the optimal configuration; 
% to improve the quality of the \gls{ue}-\gls{dc} and does not tune the transmission rate; 
%it instead applies a best-effort strategy.} 
% Specifically, the \gls{bs} instructs the \gls{risc} to sweep through a set of predefined configurations, expecting that at least one will satisfy a target \gls{kpi} specified \emph{a priori}. %for the transmission (\emph{e.g.}, a minimum \gls{snr} to support a predefined rate). 
Fig.~\ref{fig:RIS-frames} shows the data exchange diagrams of the two paradigms, comprised of \gls{cc} messages, configuration loading, processing operations, and data transmission. % Based on these, 
% A detailed description of the two paradigms is given in the following, while the impact of the control signaling is investigated in Sect.~\ref{sec:ris-control}.

\begin{figure*}[!t]
    \centering
    \begin{subfigure}[t]{0.45\linewidth}
        \centering
        \includegraphics[width=0.9\textwidth]{figs/frame-oce.pdf}
        \caption{\gls{oce} paradigm.}
        \label{fig:RIS-oce}
    \end{subfigure}
    \hfill
    \begin{subfigure}[t]{0.45\linewidth}
        \centering
        \includegraphics[width=0.9\textwidth]{figs/frame-bsw.pdf}
        \caption{\gls{bsw} paradigm.}
        \label{fig:RIS-bsw}
    \end{subfigure} 
    \caption{Data exchange diagram of the two \gls{ris}-enabled communication paradigms. Signals traveling through \gls{ris}-\gls{cc}, \gls{ue}-\gls{cc}, and \gls{ue}-\gls{dc} are represented by solid \textcolor{red}{red}, solid \textcolor{blue}{blue}, and solid black lines, respectively. \gls{risc} to \gls{ris} commands are indicated with dashed black lines. \gls{bs} operations are in \texttt{monospaced font}.}
    \label{fig:RIS-frames}
\end{figure*}

%%%%%%%%%%%%%%%%%%%%%%%%%%%%%%
\subsection{Optimization based on Channel Estimation (OPT-CE)}\label{sec:communication-paradigms:oce}
%%%%%%%%%%%%%%%%%%%%%%%%%%%%%%
In \gls{oce}, the \gls{bs} needs to obtain the \gls{csi} for the \gls{ue} to optimize the \gls{ris} configuration. The necessary measurements can be collected by transmitting pilot sequences from the \gls{ue}. During the Initialization phase, the \gls{bs} informs the other entities that the procedure is starting. First, the \gls{ue} is informed through the \gls{ue}-\gls{cc} to prepare to send pilots. Second, to solve the indeterminacy of the $N$-path \gls{ce} due to the \gls{ris} presence~\cite{Wang2020}, the \gls{ris} is instructed to sweep through a common codebook of configurations during the Algorithmic phase, called \emph{\gls{ce} codebook} and denoted as $\mc{C}\oce \subseteq \mc{C}$. To change between the configurations in the \gls{ce} codebook, we consider that the \gls{bs} needs to send only a single control message to the \gls{risc} since the \gls{ris} sweeps following the order stipulated by the codebook. During the Algorithmic phase, the \gls{ue} sends replicas of its pilot sequence, subject to different \gls{ris} configurations, to let each of them experience a different propagation environment. After a sufficient number of samples is received, the \gls{bs} estimates the \gls{csi} and compute the optimal \gls{ris}'s configuration~\cite{Tsinghua_RIS_Tutorial}. Then, the Setup phase starts, in which the \gls{bs} uses the the ctrl configuration\footnote{We remark that the \gls{ctrl} configuration is automatically loaded after the Algorithmic phase ends due to the idle state of the \gls{ris}. Another approach is loading the optimal configuration evaluated in the Algorithmic phase also to send the control information toward the \gls{ue}; nevertheless, the \gls{risc} needs to be informed previously by a specific control message by the \gls{bs}. We do not consider this approach to keep the frame structure of the two paradigms similar, thereby simplifying the analysis and the presentation in Section~\ref{sec:ris-control}.} to implement the \gls{ue}-\gls{cc} and instruct the \gls{ue} to start sending data. Subsequently, the \gls{bs} informs the \gls{risc}, over the \gls{ris}-\gls{cc}, to load the optimal configuration. Finally, the Payload phase takes place.


%As we are striving for a general \gls{ce} framework, we consider the strategy proposed by~\cite{Wang2020}, which is simple enough to yield an analytical formulation of the estimation error. The procedure occurs as follows: First, the \gls{ris} elements are turned off, so that the \gls{bs} can get an estimate of the direct channel; second, the \gls{bs} carries out the \gls{ce} for a single, typical \gls{ue}; finally, the other \glspl{ue} transmit pilots and the \gls{bs} estimates their channels by exploiting the fact that they are scaled versions of the typical \gls{ue}'s channel. Due to the aim of the paper, the scenario at hand, and the channel model, we focus only on the second phase, where the \gls{bs} estimates the channel coefficients of a single \gls{ue}. \RK{I think this paragraph can be removed completely. We are introducing something just to say we won't be doing it anyway. If the analytical method for estimating error based on~\cite{Wang2020} is used, we can just reference the paper later}%More specifically, the \gls{bs} aims to estimate $N$ channel coefficients associated with the equivalent channel observed at the \gls{bs} after reflection.
%The replicas of the \gls{ue} pilot sequences are designed to be proportional to the number of \gls{ris} elements $N$ for an accurate \gls{ce}. Also,
%\vc{To the \gls{ce} be able to capture the additional spatial dimension from the \gls{ris}, a}
%Accordingly, the 
%set of configurations $\mc{C}\oce$, also named as \emph{codebook}, must be designed, where %to let this estimation be successful~\cite{He2020}. Note that 
%the knowledge of such a codebook is shared by \gls{risc} and the \gls{bs}, assuming that a setup phase has taken place to deploy the \gls{ris} in the network.

\subsubsection{Performance Analysis}
We now present the \gls{ce} procedure and analyze its performance in connection to the cardinality of the employed codebook $C\oce = |\mc{C}\oce|$. The considered method employed is a simplified version proposed in~\cite{Wang2020}. Let us start with the pilot sequence transmission and its processing. %Every pilot sequence is made up of $p$ complex symbols. Hence, 
We denote a single pilot sequence as $\bm{\psi} \in \mathbb{C}^{p}$, spanning $p$ symbols and having $\lVert \bm{\psi} \rVert^2 = p$. Every time a configuration from the codebook is loaded at the \gls{ris}, the \gls{ue} sends a replica of the sequence $\bm{\psi}$ towards the \gls{bs}. When the configuration $c\in\mc{C}\oce $ is loaded, the following signal is received at the \gls{bs}
\begin{equation} \label{eq:oce:replica}
    \mb{y}_c\T  = \sqrt{\rho_u} \bm{\phi}_c\T \mb{z}_d  \, \bm{\psi}\T + \tilde{\mb{w}}_c\T \in\mathbb{C}^{1 \times p},
\end{equation}
where $\bm{\phi}_c$ denotes the phase profile vector of the configuration $c\in\mc{C}\oce$, $\rho_u$ is the transmit power, and $\tilde{\mb{w}}_c\sim \mc{CN}(0, \sigma_b^2 \mb{I}_p)$ is the \gls{awgn}. The received symbol is then correlated with the pilot sequence and normalized by $\sqrt{\rho_u} p$, yielding
\begin{equation} \label{eq:oce:pilotprocess}
    y_c  = \frac{1}{\sqrt{\rho_u} p}\mb{y}_c\T \bm{\psi}^* = \bm{\phi}_c\T \, \mb{z}_d + w_c \in\mathbb{C},
\end{equation}
where $w_c\sim\mc{CN}(0, \frac{\sigma_b^2}{p \rho_u})$ is the resulting \gls{awgn}.\footnote{The consideration of dividing the pilot transmission over configurations over small blocks of $p$ symbols serves three purposes: $i$) from the hardware point of view, it might be difficult to change the phase-shift profile of an \gls{ris} within the symbol time, $ii$) to reduce the impact of the noise, and $iii$) to have the possibility of separating up to $p$ \gls{ue}'s data streams, if the pilots are designed to be orthogonal to each other~\cite{massivemimobook}.}
% \fs{The noise is reduced with this assumption
% \begin{equation}
% \begin{aligned}
%     \E{w} &= \frac{1}{\sqrt{\rho_u} p} \sum_{i=1}^p \psi_c^* \E{w_c} = 0 \\
%     \E{|w|^2} &= \frac{1}{\rho_u p^2} \E{\sum_{i=1}^p \psi_c^* w_c w_c^* \psi_c} + \frac{1}{\rho_u p} \E{\sum_{i=1}^p \sum_{j \neq i}^p \psi_c^* w_c w_j^* \psi_j} \\
%     & =  \frac{1}{\rho_u p^2} \sum_{i=1}^p \psi_c^* \E{w_c w_c^*} \psi_c + \frac{1}{\rho_u p^2} \sum_{i=1}^p \sum_{j \neq i}^p \psi_c^* \underbrace{\E{w_c w_j^*}}_{\E{w_c}\E{w_j^*} = 0} \psi_j \\
%     &= \frac{\sigma_b^2}{\rho_u p^2} \sum_{i=1}^p \psi_c^* \psi_c =  \frac{\sigma_b^2}{\rho_u p^2} \underbrace{\bm{\psi}\H \bm{\psi}}_{\lVert\bm{\psi}\rVert^2 = p} = \frac{\sigma_b^2}{\rho_u p}.
% \end{aligned}
% \end{equation}
% }
The pilot transmissions and the processing in~\eqref{eq:oce:pilotprocess} are repeated for all \gls{ris} configurations, \emph{i.e.}, $\forall c\in\mc{C}\oce$. The resulting signal $\mb{y} = [y_1, y_2, \dots, y_{C\oce}]\T\in\mathbb{C}^{C\oce}$ can be then compactly written in the following form:
\begin{equation} \label{eq:oce:signal}
    \mb{y}= \mb{\Theta}\T \mb{z}_d + \mb{w},
\end{equation}
where $\bm{\Theta} = [\bm{\phi}_1, \bm{\phi}_2, \dots, \bm{\phi}_{C\oce}]\in\mathbb{C}^{N\times C\oce }$ is the matrix containing all the configurations used and $\mb{w} = [w_1, \dots, w_{C\oce}]\T \in\mathbb{C}^{C\oce}$ is the \gls{awgn} term. For the sake of generality, we will assume that there is no prior information about the channel distribution at the \gls{bs}. Therefore, we cannot estimate separately $\mb{h}_d$ and $\mb{g}_d$, but only the cascade channel $\mb{z}_d$. 
%Indeed, we can rewrite the eq.~\eqref{eq:oce:signal} as
%each component of~\eqref{eq:oce:signal} as
% \begin{equation}
%     y_c = \sum_{n=1}^{N} \phi_{n,c} \underbrace{h^{*}_{n} g_{n}}_{z_{n}} + w = \sum_{n=1}^{N} \phi_{n,c} {z_{n}} + w
% \end{equation}
% and the overall signal is
% \begin{equation} \label{eq:oce:signal2}
    % \mb{y} = \bm{\Theta}\T \mb{z} + \mb{w}
% \end{equation}
% where $\mb{z}=[z_{1},z_{2},\dots,z_{N}]\T = \mb{h} \odot \mb{g}\in\mathbb{C}^{N \times{1}}$ is the equivalent channel. 
It is possible to show that a necessary (but not sufficient) condition to perfectly estimate the channel coefficients is that $C\oce\geq{N}$~\cite{Wang2020}. Indeed, we want a linearly independent set of equations, which can be obtained by constructing the configuration codebook for \gls{ce} to be at least of rank $N$. As an example, we can use the \gls{dft} matrix, \emph{i.e.}, $[\bm{\Theta}]_{n,c} = e^{-j2\pi \frac{(n-1) (c-1)}{C\oce}}$, with $n=\mc{N}$ and $c\in\mc{C}\oce$,
% \begin{equation}
%     \bm{\Theta} =
%     \begin{bmatrix}
%         1 & 1 & 1 & \cdots & 1 \\
%         1 & \omega & \omega^2 & \cdots & \omega^{(C_{\mathrm{OCE}}-1)} \\
%         1 & \omega^2 & \omega^2 & \cdots & \omega^{2(C_{\mathrm{OCE}}-1)} \\
%         \vdots & \vdots & \vdots & \ddots & \vdots \\
%         1 & \omega^{N-1} & \omega^{2(N-1)} & \cdots &\omega^{(N-1)(C_{\mathrm{OCE}}-1)},
%     \end{bmatrix}
% \end{equation}
% where $\omega=e^{-j 2\pi / C_{\mathrm{OCE}}}$ 
with $\bm{\Theta}^* \bm{\Theta}\T = C\oce \mb{I}_{N}$. Considering that the parameter vector of interest is deterministic, the least-squares estimate yields the estimation~\cite{Kay1997}
\begin{equation}
    \hat{\mb{z}}_d=\dfrac{1}{C\oce} \bm{\Theta}^* \mb{y} = \mb{z}_d + \mb{n},
\end{equation}
where $\mb{n}\sim\mc{CN}(0, \frac{\sigma_b^2}{p \rho_u C\oce } \mb{I}_N)$, and whose performance is proportional to $1/C\oce$.
%$ Remark: the higher the cardinality of the codebook, the better the \gls{ce} performance.
% \fs{In fact:
% \begin{equation}
%     \begin{aligned}
%     \E{\frac{1}{C\oce} \bm{\Theta}^* \mb{w}} &= 0 \\
%     \E{(\frac{1}{C\oce} \bm{\Theta}^* \mb{w})( \frac{1}{C\oce} \bm{\Theta}^* \mb{w})\H} &= \frac{1}{C\oce^2} \bm{\Theta}^* \E{\mb{w} \mb{w}\H} \bm{\Theta}\T  = \frac{\sigma_b^2}{\rho_u p C\oce} \mb{I}_N
%     \end{aligned}
% \end{equation}}
%
Based on the estimated equivalent channel, the \gls{bs} can obtain the configuration $\bm{\phi}_\star$ that maximizes the instantaneous \gls{snr} of the typical \gls{ue}, as follows: 
\begin{equation}
    \begin{aligned}
        \bm{\phi}_\star &= \max_{\phi}\left \{ \lvert \bm{\phi}\T \hat{\mb{z}}_d \rvert^2 \, \big| \, \lVert\bm{\phi}\rVert^2 = N \right\},
    \end{aligned}
\end{equation}
which turns out to provide the intuitive setting $(\bm{\phi}_\star)_n=e^{-j\angle{(\hat{\mb{z}}_d)_n}}$, $\forall n \in \mc{N}$. The \gls{ul} estimated \gls{snr} at the \gls{bs} is:
\begin{equation}
    \hat{\gamma}\oce = \frac{\rho_u}{\sigma_b^2} |\bm{\phi}_\star\T \hat{\mb{z}}_d|^2.
\end{equation}
Based on the estimated \gls{snr}, the \gls{se} of the data communication can be adapted to be the maximum achievable, \emph{i.e.},
\begin{equation} \label{eq:oce:se}
    \eta\oce = \log_2(1 + \hat{\gamma}\oce).
\end{equation}

\subsubsection{Algorithmic errors}
For the \gls{oce} paradigm, a communication outage occurs in the case of an \emph{overestimation error}, \emph{i.e.}, if the selected \gls{se} $\eta\oce$ is higher than the actual channel capacity, leading to an unachievable communication rate~\cite{Shannon1949}. The probability of this event is
\begin{equation} \label{eq:oce:ae1}
    p_\mathrm{ae} = \mc{P} \left[\eta\oce = \log(1 + \hat{\gamma}\oce) \ge \log_2\left(1+ \gamma\oce \right) \right],
\end{equation}
where $\gamma\oce = \frac{\rho_u}{\sigma_b^2} |\bm{\phi}_\star\T \mb{z}_d|^2$ is the actual \gls{snr} at the \gls{bs}. Eq.~\eqref{eq:oce:ae1} translates to the condition
\begin{equation} \label{eq:oce:ae2}
    p_\mathrm{ae} = \mc{P}\left[\hat{\gamma}\oce \ge \gamma\oce\right] = \mc{P}\left[ |\bm{\phi}_\star\T \mb{z}_d + \bm{\phi}_\star\T \mb{n}|^2 \ge  |\bm{\phi}_\star\T \mb{z}_d|^2 \right].
\end{equation}
A detailed analysis of~\eqref{eq:oce:ae2} relies on the channel model of $\mb{z}_d$, and thereby on a prior assumption about the scenario (\emph{e.g.},~\cite{RIS_Nakagami}); we therefore numerically evaluate  the impact of the \gls{oce} algorithmic error.

%Nevertheless, we found experimentally that the impact of the noise on the \gls{snr} estimation is generally negligible for this paradigm, as shown in Fig.~\ref{fig:snrcdf} where the \gls{cdf} of the estimated $\hat{\gamma}\oce$ and actual \gls{snr} ${\gamma}\oce$ are presented.  This finding is justified by the fact that the noise power is proportional to $1/C\oce$ where $C\oce \ge N$ as described in Section~\ref{sec:communication-paradigms:oce}. %Therefore, considering the generally high number of \gls{ris} elements deployed, the impact of the noise results is negligible. Because of that, we assume that the algorithmic error for the \gls{oce} paradigm is $p_\mathrm{ae} \approx 0$ in the remainder of the paper.

\subsection{Codebook-Based Beam Sweeping (CB-BSW)}\label{sec:communication-paradigms:bsw}
In \gls{bsw}, the \gls{bs} now does not require explicit \gls{csi} of the \gls{ue}. In the Initialization phase, the \gls{bs} commands the start of a new frame by signaling to the \gls{ris} and \gls{ue}. \change{A \emph{\gls{BSW} process}, \emph{i.e.}, an \gls{ris} configuration selection, is performed during the Algorithmic phase. This process comprises the \gls{ue} sending reference signals, while the \gls{bs} commands the \gls{ris} to change its configuration at regular periods accordingly to a set of predefined configurations, labeled as the \emph{\gls{BSW} codebook} denoted by $\mc{C}\bsw \subseteq \mc{C}$.} The \gls{bs} receives the reference signals that are used to measure \gls{ue}'s \gls{kpi}. The \gls{BSW} process is triggered when a single \gls{bs} control message is received by the \gls{risc}. At its end, the \gls{bs} selects a configuration satisfying the target \gls{kpi}. During the Setup phase, the \gls{bs} informs the \gls{ue} over the \gls{ue}-\gls{cc} to prepare to send data, while the \gls{ris} uses the \gls{ctrl} configuration, and informs the \gls{risc} through the \gls{ris}-\gls{cc} to load the selected configuration. Finally, the Payload phase takes place. 

\begin{remark}[Fixed vs Flexible frames]
    \change{The \gls{BSW} process during the Algorithmic phase may make use of i) a \emph{fixed} or ii) a \emph{flexible} frame structure. The fixed frame ends after a fixed number of \gls{BSW} codebook configurations have been loaded. The flexible frame structure allows stopping the \gls{BSW} as soon as a \gls{kpi} value measured is above the target one. Flexible frame requires on-the-fly \gls{kpi} measurements at the the \gls{bs}, while \gls{ue}-\gls{cc} needs to be reserved to promptly inform the \gls{ue} about the frame termination when the target \gls{kpi} is met, thus modifying the overall frame (see Section~\ref{sec:ris-control}).}
    \label{remark:cb-bsw:fixed-vs-flexible}
\end{remark}

\subsubsection{Performance analysis}
For \gls{bsw} it is necessary to assume that the target \gls{kpi} is a target \gls{snr} $\gamma_0$ measured at the \gls{bs} via the average \gls{rss} metric. In this case, a fixed \gls{se} is considered \emph{a priori}, which is given by
\begin{equation} \label{eq:bsw:se}
    \eta\bsw = \log_2(1 + \gamma_0),
\end{equation}
and the goal is find a configuration from the \gls{ris} codebook that supports it. We analyze the system performance starting from the pilot sequence transmission and processing. As before, every pilot sequence consists of $p$ symbols\footnote{The pilot sequences for \gls{oce} and \gls{bsw} can be different and have different lengths. In practice, they should be designed and optimized for each of those approaches, which is beyond the scope of this paper. We use the same pilot sequence length notation in both paradigms for simplicity.}. Once again, we denote a single sequence as $\bm{\psi} \in \mathbb{C}^{p}$ having $\lVert\bm{\psi} \rVert^2 = p$. After the \gls{ris} loads configuration $c\in\mc{C}\bsw$, the \gls{ue} sends a replica of $\bm{\psi}$ and, similar to~\eqref{eq:oce:replica}, the \gls{bs} receives the signal: 
\begin{equation} \label{eq:bsw:replica}
    \mb{y}_c\T  = \sqrt{\rho_u} \bm{\phi}_c\T \mb{z}_d \bm{\psi}\T + \tilde{\mb{w}}_c\T \in\mathbb{C}^{1 \times p},
\end{equation}
where $\bm{\phi}_c$ denotes the configuration $c \in \mc{C}\bsw$. The received signal is then correlated with $\bm{\psi}$ and normalized by $p$:
\begin{equation} \label{eq:bsw:pilotprocess}
    y_c  = \frac{1}{p}\mb{y}_c\T \bm{\psi}^* = \sqrt{\rho_u} \bm{\phi}_c\T \mb{z}_d + w_c \in\mathbb{C},
\end{equation}
where $w_c\sim\mc{CN}(0, \frac{\sigma_b^2}{p})$ is the resulting \gls{awgn}. 
The \gls{snr} provided by the configuration can be estimated as follows: %by taking the absolute square of the sample and dividing it by the (known) noise variance as
\begin{equation} \label{eq:bsw:gammahat}
    \hat{\gamma}_c = \frac{|y_c|^2}{\sigma_b^2} \hspace{-0.3mm} = \hspace{-0.3mm} \underbrace{\frac{\rho_u}{\sigma_b^2} |\bm{\phi}_c\T \mb{z}_d|^2 }_{\gamma_c} \hspace{-0.1mm}+\hspace{-0.2mm} 2 \Re\hspace{-0.5mm}\left\{ \frac{\sqrt{\rho_u}}{\sigma_b^2} \bm{\phi}_c\T \mb{z}_d\, w_c\right\} \hspace{-0.6mm}+\hspace{-0.5mm} \frac{|w_c|^2}{\sigma_b^2},
\end{equation}
where $|w_c|^2 \sigma_b^{-2} \sim \mathrm{Exp}(p)$. 
It is worth noting that the estimated \gls{snr} is affected by the exponential error generated by the noise, but also by the error of the mixed product between the signal and the noise, whose \gls{pdf} depends on the \gls{pdf} of $\mb{z}_d$. Based on~\eqref{eq:bsw:gammahat}, we can select the best configuration $c^\star\in\mc{C\bsw}$ providing the target \gls{kpi}. According to Remark~\ref{remark:cb-bsw:fixed-vs-flexible}, we next discuss the selection of the configuration for the two different frame structures.

\paragraph{Fixed Frame} 
When the frame has a fixed structure, the \gls{BSW} procedure ends after the \gls{ris} sweeps through the whole codebook. In this case, we can measure the \glspl{kpi} for all available configurations. The configuration selected for the payload phase is the one achieving the highest estimated \gls{snr} among the ones satisfying the target \gls{kpi} $\gamma_0$, as
\begin{equation} \label{eq:bsw:cstar:fixed}
    c^\star = \argmax_{c\in\mc{C}\bsw} \{\hat{\gamma}_c \,|\, \hat{\gamma}_c \ge \gamma_0\}.
\end{equation}
If no configuration achieves the target \gls{kpi}, the communication is not feasible, and we run into an outage event.

\paragraph{Flexible Frame} 
When the frame has a flexible structure, the end of the \gls{BSW} process is triggered by the \gls{bs} when the measured \gls{kpi} reaches the target value. A simple on-the-fly selection method involves testing if the estimated \gls{snr} is greater than the target $\gamma_0$; \emph{i.e.}, after eq.~\eqref{eq:bsw:gammahat} is obtained for configuration $c\in\mc{C}\bsw$, we set
\begin{equation} \label{eq:bsw:cstar:flexible}
    c^\star = c \iff \hat{\gamma}_c \ge \gamma_0.
\end{equation}
As soon as $c^\star$ is found, the \gls{bs} communicates to both \gls{ris} and \gls{ue} that the Payload phase can start; otherwise, the \gls{BSW} process continues until a configuration is selected. If no configuration of the codebook $C\bsw$ satisfies the condition~\eqref{eq:bsw:cstar:flexible}, then communication is not feasible and outage occurs.

\subsubsection{Algorithmic errors}
For the \gls{bsw} paradigm, an outage event occurs when no configuration in the \gls{BSW} codebook satisfies the target \gls{kpi}, and when the selected configuration provides an \gls{snr} lower than the target one, although the estimated \gls{snr} was higher; we denote the latter as the overestimation event. These two events are mutually exclusive, and hence, their probability is
\begin{equation} \label{eq:bsw:ae}
\begin{aligned}
    p_\mathrm{ae} &= \mc{P} \left[ \gamma_{c^\star} \le \gamma_0 | \hat{\gamma}_{c^\star} > \gamma_0 \right] + \mc{P}\left[ \hat{\gamma}_{c} \le \gamma_0, \, \forall c\in\mc{C}\bsw \right]  \\
    &= \mc{P} \left[ \hat{\gamma}_{c^\star} - \gamma_0 \le  \frac{|w_{c^\star}|^2}{\sigma_b^2} + 2 \Re\left\{ \frac{\sqrt{\rho_u}}{\sigma_b^2} \bm{\phi}_{c^\star}\T \mb{z}_d\, w_c\right\} \right] \\ &\quad+ \mc{P}\left[ \hat{\gamma}_{1} \le \gamma_0, \dots, \hat{\gamma}_{C\bsw} \le \gamma_0  \right],
\end{aligned}    
\end{equation}
where $\gamma_{c^\star} = \frac{\rho_u}{\sigma_b^2} | \bm{\phi}_{c^\star}\T \mb{z}_d|^2$ is the actual \gls{snr} and $\hat{\gamma}_{c^\star} - \gamma_0  > 0$.
%
By applying Chebychev inequality, the overestimation probability (first term) can be upper bounded by
    \begin{equation} \label{eq:bsw:oebound}
     \mc{P} \left[ \hat{\gamma}_{c^\star} \hspace{-1mm} - \hspace{-0.8mm} \gamma_0 \hspace{-.5mm}\le \hspace{-0.8mm} \frac{|w_{c^\star}|^2}{\sigma_b^2} \hspace{-.5mm}+\hspace{-.5mm} 2 \Re\hspace{-.5mm}\left\{ \frac{\sqrt{\rho_u}}{\sigma_b^2} \bm{\phi}_{c^\star}\T \mb{z}_d\, w_c\hspace{-.5mm}\right\} \right] % \le \frac{\E{\frac{|w_{c^\star}|^2}{\sigma_b^2} + 2 \Re\left\{ \frac{\sqrt{\rho_u}}{\sigma_b^2} \bm{\phi}_{c^\star}\T \mb{z}_d\, w_c\right\}}}{\hat{\gamma}_{c^\star} - \gamma_0} = 
     \hspace{-1mm}\le \hspace{-1mm}\frac{p^{-1}}{\hat{\gamma}_{c^\star} \hspace{-1mm}-\hspace{-.8mm} \gamma_0},
\end{equation}
%From eq.~\eqref{eq:bsw:oebound}, 
from which we infer that the higher the gap between $\hat{\gamma}_{c^\star}$ and $\gamma_0$, the lower the probability of error. The \gls{bsw} employing the fixed structure generally has a higher value of ${\hat{\gamma}_{c^\star} - \gamma_0}$ than the one with the flexible structure due to the use of the $\argmax$ operator to select the configuration $c^\star$. Therefore, the fixed structure is generally more robust to overestimation errors. 
%From eqs.~\eqref{eq:bsw:ae} and~\eqref{eq:bsw:oebound}, we note that the fixed structure is more robust to overestimation errors. Indeed, the higher the gap between $\hat{\gamma}_{c^\star}$ and $\gamma_0$, the lower the probability of error. 
On the other hand, the evaluation of the probability that no configuration in the beam sweeping codebook satisfies the target \gls{kpi} requires the knowledge of the \gls{cdf} of the estimated \gls{snr}, whose analytical expression is channel-model dependent and generally hard to obtain.
\begin{comment}
By assuming \gls{iid} measures of the \gls{snr} values, the probability that no configuration in the beam sweeping codebook satisfies the target \gls{kpi} is
\begin{equation} \label{eq:bsw:outageiid}
     \mc{P}\left[ \hat{\gamma}_{1} \le \gamma_0, \dots, \hat{\gamma}_{C\bsw} \le \gamma_0  \right] = \prod_{c\in\mc{C}\bsw} \mc{P}\left[ \hat{\gamma}_{c} \le \gamma_0\right] = \left[F_{\hat{\gamma}_1}(\gamma_0)\right]^{C\bsw},
\end{equation}
where $F_{\hat{\gamma}_c}(\gamma_0)$ is the \gls{cdf} of $\hat{\gamma}_c$.
Remark that the assumption of \gls{iid} $\hat{\gamma}_c$ measurements is an oversimplification, considering that the propagation environment is in general spatially correlated. %Hence, the configuration codebook should be specifically designed to have \gls{iid} $\hat{\gamma}_c$ measurements, which, in turn, depends on the channel model of $\mb{z}_d$. On the other hand, 
On the other hand, when considering completely correlated measurements, we obtain
\begin{equation} \label{eq:bsw:outagecorr}
     \mc{P}\left[ \hat{\gamma}_{1} \le \gamma_0, \dots, \hat{\gamma}_{C\bsw} \le \gamma_0  \right] = \mc{P}\left[ \hat{\gamma}_{1} \le \gamma_0 \right] = F_{\hat{\gamma}_1}(\gamma_0).
\end{equation}
In a real environment, the actual outage probability will be in the range given by eqs.~\eqref{eq:bsw:outageiid} and~\eqref{eq:bsw:outagecorr}. In any case, both eqs.~\eqref{eq:bsw:outageiid} and~\eqref{eq:bsw:outagecorr} depend on the \gls{cdf} of the estimated \gls{snr}, whose analytical expression is channel model dependent and generally hard to obtain. From the equations, we can infer that the more spatially correlated is the channel in the environment, the lower the outage probability. \fs{To say more than this is complicated.}
\end{comment}
Here, we also resort to numerical evaluation of the impact of the \gls{bsw} algorithmic errors.

\subsection{Trade-offs in different transmission paradigms}
\label{sec:paradigms:comment}
The two aforedescribed \gls{ris}-aided transmission paradigms can be seen as a generalization of the \emph{fixed rate} (multiplexing) and \emph{adaptive rate} (diversity) transmission approaches. Essentially, the \gls{se} of the \gls{oce} is adapted to the achievable rate under the optimal configuration (see~\eqref{eq:oce:se}), while the \gls{se} of the \gls{bsw} is set \emph{a priori} according to the target \gls{kpi} (see~\eqref{eq:bsw:se}). Comparing~\eqref{eq:oce:se} and~\eqref{eq:bsw:se} under the same environmental conditions, we have that 
\begin{equation}
    \eta\bsw \le \eta\oce,
\end{equation}
where the price to pay for the higher \gls{se} of the \gls{oce} paradigm is the increased overhead. \gls{oce} needs an accurate \gls{csi} for reliable rate adaptation, which translates into a higher number of sequences to be transmitted by the \gls{ue} compared to \gls{bsw}. Furthermore, additional time and processing are required to determine the optimal configuration of the \gls{ris}. Consequently, the \gls{se} of data transmission alone cannot be considered a fair comparison metric, as it does not consider the overheads generated by the communication paradigms. % In the next section, we will introduce the impact of the \glspl{cc} giving the main metric of the comparison.
