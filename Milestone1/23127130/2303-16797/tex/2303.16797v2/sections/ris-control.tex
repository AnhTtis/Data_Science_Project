In this section, we define a performance metric that simultaneously measures the communication performance and the impact of control signaling. We then characterize the terms of this metric regarding the overhead and the reliability of the signaling for the presented paradigms. %of Sect.~\ref{sec:paradigms}.

\subsection{Utility function definition}
To measure the communication performance, we define a utility function that takes into account \emph{a}) the overhead and the error of the communication paradigms and \emph{b}) the reliability of the \glspl{cc}. Regarding the former, \change{we define the \emph{goodput} $R$ as a discrete random variable whose value depends on the communication paradigm and its algorithmic errors:}
\begin{equation} \label{eq:netthroughput}
    R(\tau\pay, \eta) = 
    \begin{cases}
        \frac{\tau\pay}{\tau} B_d \, \eta, \text{ with prob. } 1 - p_\mathrm{ae},\\
        0, \text{ with prob. } p_\mathrm{ae},
    \end{cases}
    %(1 - p_\mathrm{ae}) \frac{\tau\pay}{\tau} B_d \, \eta,
\end{equation}
In this expression, $\eta = \eta\oce$ in~\eqref{eq:oce:se} or $\eta = \eta\bsw$ in~\eqref{eq:bsw:se} if \gls{oce} or \gls{bsw} is respectively employed, $\tau\pay$ is the duration of the payload phase, and $\tau$ is the overall frame duration. The overall overhead time is the sum of the time to carry out the Initialization, Algorithmic, and Setup phases, denoted as $\tau\ini$, $\tau\alg$, and $\tau\set$, respectively\footnote{\change{We remark that the overhead time directly impacts on the latency experienced by the \gls{ue}: given a fixed frame duration, a higher overhead translates into a lower time opportunity for the Payload phase, reducing the transmitted data in each slot, and hence, increasing the overall latency.}}. Accordingly, the payload time can be written as $\tau\pay = \tau - \tau\ini - \tau\alg - \tau\set$.
%\begin{equation}
%    \tau\pay = \tau - \tau\ini - \tau\alg - \tau\set.
%\end{equation}
While the overall frame length is fixed, the overhead time depends on the transmission paradigm, being a function of: the duration of a pilot, $\tau_p$, and the number of replicas transmitted; the optimization time, $\tau_A$; and the time to control the \gls{ris}, composed of the time employed for the transmission of the control packets to the \gls{ue} (\gls{risc}), $\tau_{i}^{(u)}$ ($\tau_{i}^{(r)}$), and the time needed by the \gls{ris} to switch configuration, $\tau_s$.

Regarding the reliability of the \glspl{cc}, we denote as $P = P_u + P_r$ the total number of control packets needed to let a communication paradigm work, where $P_u$ and $P_r$ are the numbers of control packets intended for the \gls{ue} and the \gls{risc}, respectively. Whenever one of such packets is erroneously decoded or lost, an event of \emph{erroneous control} occurs. We assume that these events are independent of each other (and of the algorithmic errors). We denote the probability of erroneous control on the packet $i$ toward entity $k\in\{u,r\}$ as $p_i^{(k)}$, with $i\in\{1, \dots, P_k\}$ and $k \in\{u,r\}$. Erroneous controls may influence the overhead time and the communication performance: \change{the \gls{ris} configuration might change unpredictably}\footnote{\change{In our scenario, if the control packet is not received, the \gls{risc} will load the \gls{ctrl} configuration, \emph{i.e.}, a predictable configuration change. However, if the \gls{risc} receives, but incorrectly decodes, a control packet, the \gls{bs} cannot know which configuration, if any, will be loaded.}}, leading to a degradation of the performance, or worse, letting the data transmission fail. While losing a single control packet may be tolerable depending on its content, we assume all control packets must be received correctly to make the communication successful. In other words, no erroneous control event is allowed. Consequently, the probability of correct control is
\begin{equation} \label{eq:pcc}
    p_\mathrm{cc} = \prod_{k\in\{u,r\}} \prod_{i=1}^{P_k} \left(1 - p_i^{(k)}\right).
\end{equation}

\change{We can include the control reliability in the communication performance, taking into account the probability of correct control in the goodput metric in~\eqref{eq:netthroughput}. By assuming that the control and algorithmic errors are independent, the goodput is re-expressed as follows:}
\begin{equation} \label{eq:netthroughput:2}
    R(\tau\pay, \eta) = 
    \begin{cases}
        \frac{\tau\pay}{\tau} B_d \, \eta, \text{ with prob. } p_\mathrm{cc}(1 - p_\mathrm{ae}),\\
        0, \text{ with prob. } 1 - p_\mathrm{cc} (1 - p_\mathrm{ae}),
    \end{cases}
    \end{equation}
\change{Hence, the performance of the considered \gls{ris}-enabled communication system can be described by averaging $R$ w.r.t. the control, obtaining the following \emph{utility function}:}
\begin{equation} \label{eq:utility}
    U(\tau\pay, \eta) \hspace{-0.5mm}%\mathbb{E}\left[  R(\tau\pay, \eta) \right] 
    = \hspace{-0.5mm} p_\mathrm{cc} (1 - p_\mathrm{ae}) \hspace{-1mm}\left( 1 - \frac{\tau\ini + \tau\alg + \tau\set}{\tau}\right) B_d  \eta.
\end{equation}


\subsection{Overhead evaluation} \label{sec:overhead}

\begin{figure}[tbh]
    \centering
    \begin{subfigure}{\columnwidth}
        \centering
        \includegraphics[width=\textwidth]{figs/data-oce.pdf}        
        \caption{\gls{oce}}
    \end{subfigure}
    \begin{subfigure}{\columnwidth}
        \centering
        \includegraphics[width=\textwidth]{figs/data-bsw-fixed.pdf}
        \caption{\gls{bsw}: fixed frame structure}
    \end{subfigure}
    \begin{subfigure}{\columnwidth}
        \centering
        \includegraphics[width=\textwidth]{figs/data-bsw-flexi.pdf}
        \caption{\gls{bsw}: flexible frame structure}
    \end{subfigure}       
    \caption{Frame structure for the communication paradigms under study. Packets colored in \textcolor{blue}{blue} and in \textcolor{yellow}{yellow} have \gls{dl} and \gls{ul} directions, respectively. Remark that INI-R (SET-R) packet and its feedback (fb) are sent at the same time as the INI-U (SET-U), but on different resources, if \gls{obcc} is present (dashed lines).}
    \label{fig:data-frames}
\end{figure}

Following the description of Section~\ref{sec:paradigms}, we present in Fig.~\ref{fig:data-frames} the frame structures of the two considered communication paradigms used to evaluate the induced overhead, where the rows represent the time horizon of the packets traveling on the different channels (first three rows) and the configuration loading time at the \gls{risc} (last row).
The time horizon is obtained assuming that all the operations span multiple numbers of \glspl{tti}, each of duration of $T$ seconds with $\lceil\tau / T\rceil \in \mathbb{N}$ being the total number of \glspl{tti} in a frame. At the beginning of each \gls{tti}, if the \gls{risc} loads a new configuration, the first $\tau_s$ seconds of data might be lost due to the unpredictable response of the channel during this switching period. When needed, we consider a guard period of $\tau_s$ seconds in the overhead evaluation to avoid data disruption. Remember that the \gls{risc} loads the \gls{ctrl} configuration any time it is in an idle state, \emph{i.e.}, at the beginning of the Initialization and Setup phases.

In Fig.~\ref{fig:data-frames}, we note that the overhead generated by the Initialization and Setup phases is \emph{transmission paradigm independent}\footnote{The reliability is still dependent on the paradigm (see Section~\ref{sec:reliability}).}, while it is \emph{\gls{cc} dependent}.  Both paradigms make use of $P = 4$ control packets, $P_u = 2$ control packets sent on the \gls{ue}-\gls{cc} and $P_r = 2$ on \gls{ris}-\gls{cc}. Nevertheless, \change{employing an OB-\gls{ris}-\gls{cc} can reduce the overhead by transmitting the \gls{ris} control packets through orthogonal resources.} On the other hand, the Algorithmic phase overhead is \emph{\gls{cc} independent} and \emph{transmission paradigm dependent}, being designed to achieve the goal of the specific paradigm regardless of the \gls{cc}. In the following, the overhead is evaluated.

\subsubsection{Initialization phase}
This phase starts with the initialization control packet sent on the \gls{ue}-\gls{cc} (INI-U) informing the \gls{ue} that the \gls{oce} procedure has started. In the \gls{ibcc} case, this is followed by the transmission of the INI-R packet to the \gls{risc} to notify the beginning of the procedure. A consequent \gls{tti} for feedback is reserved to notify back to the \gls{bs} if the INI-R packet has been received. If an \gls{obcc} is employed, no \gls{tti} needs to be reserved because the INI-R and its feedback are scheduled simultaneously since the INI-U packet relies on different resources (see Assumption~\ref{assu:ris-cc}).
The phase duration is $\tau\ini = T$ or $\tau\ini=3T$ with OB- or IB-\gls{ris}-\gls{cc}, respectively.
% \begin{equation} 
%     \tau\ini = 
%     \begin{cases}
%         T, \quad \text{\gls{obcc}}, \\
%         % 2 T, \quad \text{\gls{ibno}}, \\
%         3 T, \quad \text{\gls{ibcc}}.
%     \end{cases}
% \end{equation}

\subsubsection{Setup phase}
 %The time needed to set up the \gls{ue} and the \gls{risc} follows the Initialization phase. 
 After the optimization has run, a setup (SET-U) packet spanning one \gls{tti} is sent to the \gls{ue} notifying it to prepare to send the data; then, with an \gls{ibcc}, a \gls{tti} is used to send the SET-R packet containing the information of which configuration to load during the Payload phase; a further \gls{tti} is reserved for feedback. Again, if an \gls{obcc} is present, the SET-R and its feedback are scheduled at the same time as the SET-U packet but on different resources; therefore, no \glspl{tti} needs to be reserved for the SET-R and its feedback. Remark that the $\tau_s$ guard period must be considered by the \gls{ue} when transmitting the data to avoid being disrupted during the load of the configuration employed in the Payload. For simplicity of evaluation, we account for this guard period in the Setup phase duration, resulting in $\tau\set = \tau\ini + \tau_s$.
% \begin{equation} \label{eq:oce:ack-time}
%     \tau\set = \tau\ini + \tau_s
%     % \begin{cases}
%     %     T + \tau_s, \quad \text{\gls{obcc}}, \\
%     %     2T + \tau_s, \quad \text{\gls{ibno}}, \\
%     %     3T + \tau_s, \quad \text{\gls{ibwf}}.
%     % \end{cases}
% \end{equation}

\subsubsection{Algorithmic phase}
This phase comprises the process of sending pilot sequences and the consequent evaluation of the configuration for the transmission. Regardless of the paradigm, we assume each pilot sequence spans an entire \gls{tti}, but the configuration's switching time must be considered a guard period. Therefore, the actual time occupied by a pilot sequence is $\tau_p \le T - \tau_s$ and the number of samples $p$ of every pilot sequence is given by p = $\left\lfloor \frac{T - \tau_s}{T_n} \right\rfloor$,
% \begin{equation}
%     p = \left\lfloor \frac{T - \tau_s}{T_n} \right\rfloor,
% \end{equation}
where $T_n$ is the symbol period in seconds. Assuming that the \gls{tti} duration and the symbol period are fixed, the \gls{ue} can compute the pilot length if it is informed about the guard period. The overall duration of the Algorithmic phase depends on the paradigm employed.

\paragraph{\gls{oce}} In this case, the Algorithmic phase starts with $C\oce$ \glspl{tti}; at the beginning of each of them, the \gls{risc} loads a different configuration, while the \gls{ue} transmits replicas of the pilot sequence. After all the sequences are received, the \gls{ce} process at the \gls{bs} starts, followed by the configuration optimization. The time needed to perform the \gls{ce} and optimization processes depends on the algorithm and the available hardware. To consider a generic case, we denote this time as $\tau_A = A T$.

\paragraph{\gls{bsw} fixed frame structure}
Similarly to the previous case, the Algorithmic phase starts with $C\bsw$ \glspl{tti}, at the beginning of which the \gls{risc} loads a different configuration, and the \gls{ue} transmits replicas of the pilot sequence. After receiving all sequences, the \gls{bs} selects the configuration as described in Section~\ref{sec:communication-paradigms:bsw}. The time needed to select the configuration is considered negligible. Thus, the Setup phase may start in the \gls{tti} after the last pilot sequence is sent.

\paragraph{\gls{bsw} flexible frame structure}
In this case, the number of \glspl{tti} used for the beam sweeping process is not known \emph{a priori} and depends on the measured \gls{snr}. However, to allow the system to react if the desired threshold is reached, a \gls{tti} is reserved for transmitting an acknowledgment (ACK-U) packet after each \gls{tti} used for pilot transmission. Hence, the number of \glspl{tti} needed is $2 c^\star - 1$, where $0< c^\star \le C\bsw$ is a random variable.

Accordingly, the Algorithmic phase duration is
\begin{equation} \label{eq:algorithmic-time}
    \tau\alg = 
    \begin{cases}
        (C\oce + A) T, \quad &\text{\gls{oce}}, \\
        C\bsw T, \quad &\text{\gls{bsw} fixed frame}, \\
        (2 c^\star - 1) T, \quad &\text{\gls{bsw} flexible frame}.
    \end{cases}
\end{equation}

\subsection{Reliability evaluation}
\label{sec:reliability}

\change{The reliability of the control packets depends on their informative content, the time reserved for their transmission, and the bandwidth of the \gls{cc}. With equal transmission time and bandwidth, transmitting a high informative packet is less reliable than a low informative packet; similarly, increasing the time reserved leads to higher reliability.
We account for this behavior via the outage probability of the $i$-the control packet intended to entity $k\in\{u,r\}$, which is given by}
\begin{equation} \label{eq:outagepe}
    p_i^{(k)} = \Pr\left\{ \log \left(1 + \Gamma_k \right) \le \frac{b_i^{(k)}}{\tau_{i}^{(k)} B_{k}} \right\}, \quad i =\{1,2\},
\end{equation}
where $i =1,2$ refers to the INI or SET packet, respectively; \change{$b_i^{(k)}$ is the amount of informative bits,} $\tau_{i}^{(k)}$ is the reserved time for transmission, and $B_k$ is the \gls{cc} bandwidth. Following the channel model in Section~\ref{sec:model},~\eqref{eq:outagepe} can be rewritten as
\begin{equation} \label{eq:outagepe2}
    p_i^{(k)} = 1 - \exp\left[- \frac{1}{\lambda_k} \left(2^{b_i^{(k)} / \tau_{i}^{(k)} / B_{k}} - 1 \right) \right].
\end{equation}
Plugging~\eqref{eq:outagepe2} into~\eqref{eq:pcc}, the correct control probability is
\begin{equation} \label{eq:pcc2}
\begin{aligned}
p_\mathrm{cc} =& \exp\left[ \frac{1}{\lambda_u} \left( 2 - \sum_{i=1}^2 2^{b_{i}^{(u)} / \tau_{i}^{(u)} / B_u}\right)\right] \times \\ &\exp\left[ \frac{1}{\lambda_r} \left( 2 - \sum_{i=1}^2 2^{b_{i}^{(r)} / \tau_{i}^{(r)} / B_r} \right) \right].    
\end{aligned}
\end{equation}
\change{Remark that the informative content $b_i^{(k)}$ and the reserved time $\tau_i^{(k)}$ depend on the control packet and the communication paradigm employed due to the need to communicate different control information. Hence, different paradigms require different values of the average \gls{snr} of the \gls{ue}-\gls{cc} $\lambda_u$ and the \gls{ris}-\gls{cc} $\lambda_r$ to obtain the same value of $p_\mathrm{cc}$. In practice, the minimum $\lambda_u$ and $\lambda_r$ required to obtain the target correct control probability give a measure of the complexity of the decoding process.}
In the following, we compute $\tau_i^{(k)}$ and $b_i^{(k)}$ for the cases under investigation.

\subsubsection{Reserved time for control packets}\label{sec:usefultime}
Each control packet spans an entire \gls{tti} following the data frame. However, the actual transmission time $\tau_{i}^{(k)}$, \emph{i.e.}, the time in which informative bits can be sent without risk of being disrupted, depends on the \gls{ris} switching time. As discussed in Section~\ref{sec:overhead}, a guard period $\tau_s$ must be considered if the \gls{risc} loads a new configuration in that \gls{tti}. Following the frame structure of Fig.~\ref{fig:data-frames}, INI-R and SET-R packets can use the whole \gls{tti}, while INI-U packets need the guard period. The SET-U control packet does not employ the guard period under the \gls{oce}, as long as $A \ge 1$. The guard period is needed for the \gls{bsw} paradigm. Hence, the transmission time of the control packets intended for the \gls{ue} is $\tau_1^{(u)} = T - \tau_s$ for all paradigms, and $\tau_2^{(u)} = T - \tau_s$ for \gls{bsw} and $\tau_2^{(u)}$ for \gls{oce},
% \begin{equation}
%     \begin{aligned}
%         \tau_{1}^{(u)} &= T - \tau_s, \qquad
%         \tau_2^{(u)} &= 
%         \begin{cases}
%             T- \tau_s, \quad &\text{\gls{bsw}}, \\
%             T, \quad &\text{\gls{oce}},
%         \end{cases}
%     \end{aligned}
% \end{equation}
while the time reserved for the control packets intended for the \gls{risc} is $\tau_{1}^{(r)} = \tau_{2}^{(r)} = T$.
%\begin{equation}
%    \tau_{1}^{(r)} = \tau_{2}^{(r)} = T.
%\end{equation}

\subsubsection{Control packet content}
\label{sec:bits}
\begin{figure}
    \centering
    \includegraphics[height=1.3cm]{figs/control-packet.pdf}
    \caption{General control packet structure.}% comprising a preamble and a payload part.}
    \label{fig:packet-structure}
    \vspace{-0.5cm}
\end{figure}

%We evaluate each control packet's amount of informative bits $b_i^{(k)}$ in this part. 
Without loss of generality, we can assume a common structure for all the control packets, comprising a control preamble and a control payload as depicted in Fig.~\ref{fig:packet-structure}. The preamble comprises $b^\mathrm{ID}$ bits representing the \emph{unique identifier (ID)} of the destination entity in the network and a single bit flag specifying if the packet is a INI or a SET one. From the preamble, the entity can understand if the control packet is meant to be decoded and how to interpret the control payload, whose informative bits depend on the kind of control packet and on the communication paradigm considered.

\paragraph{\gls{oce}}
To initialize the overall procedure at the \gls{ue}, the payload of the INI-U packet must contain the length of the frame $\tau$, the cardinality of the set $C\oce$, and the guard time $\tau_s$. To simplify the data transmission, the frame duration can be notified through an (unsigned) integer $b^\mathrm{frame}$ containing the number of total \glspl{tti} $\lceil \tau / T \rceil$ set for the frame. Similarly, we can translate the guard time into an unsigned integer representing the number of guard symbols $\lceil \tau_s / T_n \rceil$ to send $b^\mathrm{guard}$ bits. Finally, another integer of $b^\mathrm{conf}$ bits can be used to represent the cardinality $C\oce$ and to notify it to the \gls{ue}. Its minimum value is $b^\mathrm{conf} = \lfloor \log_2(C) \rfloor$, where $C$ is the total number of configurations stored in the common codebook.
Similarly, the payload of the INI-R packets needs to contain the information of the length of the frame $\tau$, and the \emph{set} of configuration  $\mc{C}\oce$ to switch through.
The former uses the same $b^\mathrm{frame}$ bits of the INI-U packet. To encode the latter, $b^\mathrm{conf}$ bits are used to identify a single configuration in the common codebook, and thus, $C\oce b^\mathrm{conf}$ needs to be transmitted to the \gls{risc}, one per desired configuration.
Regarding the Setup phase, the payload of the SET-U contains only the chosen \gls{se} of the communication $\eta\oce$. This can be encoded similarly to the \gls{mcs} in the 5G standard~\cite{3gpp:rel15}: a table of predefined values indexed by $b^\mathrm{SE}$ bits. The payload of the SET-R must contain the optimal configuration $\bm{\phi}_\star$, that is, a phase-shift value for each element. Without loss of generality, we denote by $b^\mathrm{quant}$ the number of bits used to control each element, \emph{i.e.}, the level of quantization of the \gls{ris}~\cite{EURASIP_RIS}. Hence, the overall number of informative bits is %equals the number of elements to control times the quantization level, \emph{i.e.}, 
$N b^\mathrm{quant}$.
To summarize:
\begin{equation} \label{eq:bits:oce}
    b_{i}^{(k)}\hspace{-0.2mm} = \hspace{-0.2mm} b^\mathrm{ID} \hspace{-1mm}+\hspace{-0.5mm} 1 \hspace{-0.5mm}+\hspace{-1mm}
    \begin{cases}
        b^\mathrm{frame} + b^\mathrm{guard} + b^\mathrm{conf}, \quad &k=u, \, i= 1, \\
        b^\mathrm{frame} + C\oce b^\mathrm{conf}, \quad &k= r, \, i=1, \\ 
        b^\mathrm{SE}, \quad &k= u, \, i=2,\\
        N b^\mathrm{quant}, \quad &k= r, \, i=2.
    \end{cases}
\end{equation}

\paragraph{\gls{bsw}}
The payload of the Initialization packets follows the same scheme used for the \gls{oce} paradigm. The INI-U packet contains the length of the frame $\tau$, the cardinality of the set $C\bsw$, and the guard time $\tau_s$ in the (unsigned) integers $b^\mathrm{frame}$, $b^\mathrm{guard}$, and $b^\mathrm{conf}$, respectively.
The payload of the INI-R packets contains the information of the length of the frame $\tau$, and the \emph{set} of configuration  $\mc{C}\bsw$ to switch through, encoded in the (unsigned) integers $b^\mathrm{frame}$ and $C\bsw b^\mathrm{conf}$, respectively.
Instead, the Setup contains different information. In particular, the payload of the SET-U is empty, according to the fixed rate transmission used by this paradigm. The payload of the SET-R contains the configuration $c^\star$ encoded by the same $b^\mathrm{conf}$ bits, representing an index in the common codebook. %\footnote{\change{Note that the \gls{oce} SET-R packet requires a higher informative content due to the need of transmitting the phase shift for each element of the \gls{ris}.}}.
To summarize, the packet length is:
\begin{equation} \label{eq:bits:bsw}
    b_{i}^{(k)}\hspace{-0.2mm} = \hspace{-0.2mm} b^\mathrm{ID} \hspace{-1mm}+\hspace{-0.5mm} 1 \hspace{-0.5mm}+\hspace{-1mm}
    \begin{cases}
        b^\mathrm{frame} + b^\mathrm{guard} + b^\mathrm{conf}, \quad &k=u, \, i= 1,\\
        b^\mathrm{frame} + C\bsw b^\mathrm{conf}, \quad &k= r, \, i=1, \\ 
        0, \quad &k= u, \, i=2, \\
        b^\mathrm{conf}, \quad &k= r, \, i=2.
    \end{cases}
\end{equation}
\change{Remark that the informative content of the \gls{bsw} packets is lower or equal to the one of \gls{oce}, leading the former to be more reliable than the latter.}