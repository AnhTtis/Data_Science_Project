\begin{footnotesize}
\begin{table}[t!]
\begin{footnotesize}
    
  \centering
  \caption{Configurations for workloads and architecture} 
  %\Rav{There is space for remarks too if needed}}
  %\TK{@Raveesh - by x do you mean you can do either s or t for that dimension? So does that mean each row is actually multiple datapoints? Thats confusing.}\Rav{It means that the datapoint that we choose for evaluation of that dataflow can have varaible tile sizes for dimension marked by X. S and T on the other hand mean that its NECESSARILY spatial or temporal}}
  \label{table:config}
  \begin{centering}
  \begin{tabular}{|l|l|}
    \hline
    \textbf{Parameter} & \textbf{Value}  \\
    \hline
    %Seq & $V\times F$ & $t_{AGG}+t_{CMB}$\\
    %\hline
    Bytes per word/element & 4B (32 bits)\\\hline
     Iterations of CG loop & 10\\\hline
    N rank in CG & Sweep: 1, 8, 16\\\hline
    Number of PEs per cluster & 1024\\ \hline
    Number of clusters & 16 \\\hline
    Total SRAM capacity & Sweep: 1MB, 4MB, 16MB\\ \hline
%    Off-chip bandwidth & \\\hline
 %   On-chip bandwidth & \\\hline
  %  Inter-cluster bandwidth & \\\hline
    Register file capacity & 0.5KB per PE\\\hline
    %Total L2 capacity & 32MB\\ \hline
    %Total L3 capacity & 256MB\\ \hline
  \end{tabular}
%\vspace{-3mm}
\end{centering}
\end{footnotesize}

\end{table}
\end{footnotesize}