\begin{footnotesize}

\begin{table}[t!]
\begin{footnotesize}
    %\vspace{-3mm}
  \centering
  \caption{Dataflow Configurations} 
  %\Rav{There is space for remarks too if needed}}
  %\TK{@Raveesh - by x do you mean you can do either s or t for that dimension? So does that mean each row is actually multiple datapoints? Thats confusing.}\Rav{It means that the datapoint that we choose for evaluation of that dataflow can have varaible tile sizes for dimension marked by X. S and T on the other hand mean that its NECESSARILY spatial or temporal}}
  \label{table:dataflow_config}
  \begin{tabular}{|l|l|}
    \hline
    \textbf{Config} & \textbf{Description}  \\
    \hline
    %Seq & $V\times F$ & $t_{AGG}+t_{CMB}$\\
    %\hline
%    SEQ-Inner & Op-by-op Dataflow with Inner Product\\ \hline
    SEQ-Flex & Op-by-op with best possible mapping per layer.\\
    & Operands begin and end in DRAM.\\\hline
    SEQ-Overflow & SEQ-Flex but only the portion of the operands\\  
    & overflowing get written to the DRAM. \\\hline
    GOGETA-df & GOGETA dataflow with inter-op pattern~\autoref{alg:inter-op} \\\hline
  %  & and loop orders~\autoref{alg:looporder}.\\\hline
    GOGETA-map & GOGETA-df with tiling strategy in~\autoref{sec:tiling}\\\hline
    Ideal & Infinite SRAM and prefect inter-operation reuse\\\hline
%    GOGETA+org & GOGETA-map with optimal tensor allocation ~\autoref{sec:tornado}\\\hline
    
    %Total L2 capacity & 32MB\\ \hline
    %Total L3 capacity & 256MB\\ \hline
  \end{tabular}
\vspace{-2mm}
\end{footnotesize}

\end{table}
\end{footnotesize}