\begin{table*}[]
\begin{tabular}{lllllll}
\textbf{Sequential}          & \{Vx, Fx, Nx\} & \{Vx, Gx, Fx\} & \begin{tabular}[c]{@{}l@{}}Buffer \& Data \\ Scheduler\end{tabular}                                                                            & \begin{tabular}[c]{@{}l@{}}NOC \& PEs requirement\\  depends on accelerator\end{tabular}                                                         & \begin{tabular}[c]{@{}l@{}}TPU, MAERI\\ (any direction)\end{tabular}                              & \begin{tabular}[c]{@{}l@{}}Similar to running one full or tiled layer at a time. Outputs of \\ one layer gets stored in S2 scratchpad / buffer and rescheduled\\ back into the PEs.\end{tabular}                                                                                                                                                                                                \\
\textbf{Sequential Pipeline} & \{Vs, Fs, Nt\} & \{Vs, Gs, Ft\} & \begin{tabular}[c]{@{}l@{}}None, output of one unit \\ is stationary in the PEs, \\ and can be used as input\\ for the next layer\end{tabular} & \begin{tabular}[c]{@{}l@{}}Requires in place\\ accumulation (so output \\ can be buffered inside PE\\ to use as inputs)\end{tabular}             & \begin{tabular}[c]{@{}l@{}}EnGN (comb --\textgreater agg)\\ (agg -\textgreater comb)\end{tabular} & \begin{tabular}[c]{@{}l@{}}Agg. uses Fs if it is agg --\textgreater comb, and Gs if it is comb --\textgreater agg.\\ Interleaved is possible because the output of combination (V,G) matrix is stationary, \\ and the input to agg is also (V, G), which is also stationary!\\ for agg-\textgreater{}comb, may need circular buffer to stream in property bank (might be temporal)\end{tabular} \\
                             & \{Vx, Fx, Nx\} & \{Vx, Gx, Fx\} & Buffer \& Data Scheduler                                                                                                                       & \begin{tabular}[c]{@{}l@{}}NoC must be able to transfer\\ data to the corresponding\\ PEs (need 2 compute engine\\ or allocate PEs)\end{tabular} &                                                                                                   & will be a buffer/ setup in between aggregation and combination                                                                                                                                                                                                                                                                                                                                  \\
\textbf{Pipeline parallel}   & Vt, \{Fs, Nt\} & \{Vs, Gs, Ft\} & \begin{tabular}[c]{@{}l@{}}Buffer \& Data \\ Scheduler\end{tabular}                                                                            & \begin{tabular}[c]{@{}l@{}}Agg \& Comb both require\\ in place accumulation. Comb\\ Engine requires Store \& \\ Fwd links\end{tabular}           & HyGCN (agg -\textgreater comb)                                                                    & \begin{tabular}[c]{@{}l@{}}Has 'aggregation buffer and coordinator' between aggregation \& combination unit. \\ Fixed number of PEs per unit may lead to stalls between units.\\ Ex: combination engine idle, waiting for aggregation to finish\end{tabular}                                                                                                                                    \\
                             & \{Vs, Ft, Ns\} & \{Vs, Gt, Fs\} & Buffer \& Data Scheduler                                                                                                                       & \begin{tabular}[c]{@{}l@{}}Multicast NoC required\\ between PEs.\end{tabular}                                                                    & AWB-GCN (both ways)                                                                               & \begin{tabular}[c]{@{}l@{}}Authors looked into intra and Inter Layer pipelining. There is also flexible allocation\\ of PEs for different layers to match production and consumption rates.\end{tabular}                                                                                                                                                                                        \\
                             & \{Vx, Fx, Nx\} & \{Vx, Gx, Fx\} & Buffer \& Data Scheduler                                                                                                                       & \begin{tabular}[c]{@{}l@{}}NoC must be able to transfer\\ data to the corresponding\\ PEs (need 2 compute engine\\ or allocate PEs)\end{tabular} &                                                                                                   & Different dataflow and mappings --\textgreater different production and consumption rates                                                                                                                                                                                                                                                                                                      
\end{tabular}
\label{tables:interleavebk}
\end{table*}