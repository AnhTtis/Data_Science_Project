% Figure command
\newcommand{\insertFigure}[2]{
    \begin{figure}[t]
%\setlength{\abovecaptionskip}{0pt}
%\setlength{\belowcaptionskip}{0pt}
        \centering
        \includegraphics[width=\linewidth]{figures/#1.pdf}
	\vspace{-6mm}
        \caption{\small #2}
	\vspace{0mm}
        \label{fig:#1}
    \end{figure}
}

\newcommand{\insertFigurePart}[2]{
    \begin{figure}[t]
%\setlength{\abovecaptionskip}{0pt}
%\setlength{\belowcaptionskip}{0pt}
        \centering
        \includegraphics[width=0.7\linewidth]{figures/#1.pdf}
	\vspace{-2mm}
        \caption{\small #2}
	\vspace{0mm}
        \label{fig:#1}
    \end{figure}
}

\newcommand{\insertFigurePartnn}[2]{
    \begin{figure}[t]
%\setlength{\abovecaptionskip}{0pt}
%\setlength{\belowcaptionskip}{0pt}
        \centering
        \includegraphics[width=0.85\linewidth]{figures/#1.pdf}
	\vspace{-2mm}
        \caption{\small #2}
	\vspace{0mm}
        \label{fig:#1}
    \end{figure}
}
\begin{comment}
\newcommand{\DataflowName}[0]{\texttt{GOD}~}
\newcommand{\DataflowNameTitle}[0]{\textit{GOD}~}
\newcommand{\TitleExpansion}[0]{\underline{G}eneralized Inter-\underline{O}peration \underline{Dataflow}
~}
\end{comment}
\begin{comment}

\newcommand{\DataflowName}[0]{\texttt{DAGGER}~}
\newcommand{\DataflowNameTitle}[0]{\textit{DAGGER}~}
\newcommand{\TitleExpansion}[0]{\underline{DAG}-level \underline{G}eneralized Dataflow for \underline{E}insum \underline{R}euse
~}
\end{comment}

%\begin{comment}

\newcommand{\DataflowName}[0]{\textsc{Gogeta}~}
\newcommand{\DataflowNameTitle}[0]{\textit{GOGETA}~}
\newcommand{\TitleExpansion}[0]{\underline{G}eneralized Inter-\underline{O}peration \underline{G}raph-Level \underline{E}insum \underline{T}iling and D\underline{A}taflow}
%\end{comment}

\newcommand{\insertWideFigure}[2]{

    \begin{figure*}[ht!]
%\setlength{\abovecaptionskip}{0pt}
%\setlength{\belowcaptionskip}{0pt}
        \centering
        \includegraphics[width=\textwidth]{figures/#1.pdf}
	\vspace{-6mm}
        \caption{\small #2}
	\vspace{0mm}
        \label{fig:#1}
    \end{figure*}

}

\newcommand{\reviewme}[1]{{{#1}}}

%%%% for tighter bullets
\newcommand{\squishlist}{
 \begin{list}{$\bullet$}
  { \setlength{\itemsep}{0pt}
     \setlength{\parsep}{3pt}
     \setlength{\topsep}{3pt}
     \setlength{\partopsep}{0pt}
     \setlength{\leftmargin}{1.5em}
     \setlength{\labelwidth}{1em}
     \setlength{\labelsep}{0.5em} } }

\newcommand{\squishlisttwo}{
 \begin{list}{$\bullet$}
  { \setlength{\itemsep}{0pt}
     \setlength{\parsep}{0pt}
    \setlength{\topsep}{0pt}
    \setlength{\partopsep}{0pt}
    \setlength{\leftmargin}{2em}
    \setlength{\labelwidth}{1.5em}
    \setlength{\labelsep}{0.5em} } }

\newcommand{\squishend}{
  \end{list}  }


\newcommand{\betterparagraph}[1]{\textbf{#1.}}


% Sections
\renewcommand*\sectionautorefname{\Snospace}
\def\sectionautorefname{Sec.}
\def\subsectionautorefname{Sec.}
\def\subsubsectionautorefname{Sec.}

% Figures
\def\figureautorefname{Fig.}

\def\algorithmautorefname{Alg.}


\newcommand{\TODO}[1]{\textcolor{blue}{TODO::: #1}}
\newcommand{\RG}[1]{{\color{orange}\bfseries [Raveesh::: #1]}}
\newcommand{\MP}[1]{{\color{olive}\bfseries [Michael::: #1]}}
\newcommand{\SR}[1]{{\color{purple}\bfseries [Siva::: #1]}}
\newcommand{\TK}[1]{{\color{teal}\bfseries [Tushar::: #1]}}
\newcommand{\fixme}[1]{{{\color{red} #1}}}
\newcommand{\GEMM}[0]{skewed GEMMs~}

\newcommand{\cmark}{\ding{51}}%
\newcommand{\xmark}{\ding{55}}%
%%%%%%%%%%%%%%%% TO TURN OFF THE COMMENTS TEMPORARILY TO GET THE IDEA OF PAGES %%%%%%%%%%%%%

%\begin{comment}
\renewcommand{\RG}[1]{\ignorespaces}
\renewcommand{\MP}[1]{\ignorespaces}
\renewcommand{\TK}[1]{\ignorespaces}
\renewcommand{\SR}[1]{\ignorespaces}
\renewcommand{\TODO}[1]{\ignorespaces}
{\ignorespaces}
%\end{comment}
