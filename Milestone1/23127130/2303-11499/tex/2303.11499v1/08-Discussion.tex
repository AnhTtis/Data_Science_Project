
%\vspace{-1mm}
\section{Discussion and Future Work}
\label{sec:discussion}
\vspace{-1mm}
HPC/scientific applications are interesting because of the extreme demands they put on our hardware systems. Ideally, those applications would be made up of kernels that can be independently analyzed and optimized, then composed together into dependency graphs. In this paper we demonstrate that this individual operation view is too limited in scope: there is major opportunity to improve optimization by using an inter-operation viewpoint to increase arithmetic intensity. 

%Infact such reuse patterns are a generalization of popular emerging algorithms. Conjugate Gradient has interesting inter-operation reuse patterns which also captures reuse patterns in other algorithms like Graph Convolution Networks which pipeline two operations. Infact operations 1 and 2 in Conjugate Gradient~\autoref{alg:cg_einsum} are similar to SpMM and Dense phases in a layer of GCN making it a wider generalization.

We have demonstrated that \textsc{Gogeta} can apply in scenarios where traditional loop fusion and unrolling are not sufficient. We demonstrate a geomean 6.7x reduction in memory accesses across a range of data sets and buffer sizes.  Today, existing mapping tools such as Timeloop \cite{timeloop} focus on single Einsum optimization. We hope that integrating the \textsc{Gogeta} algorithm directly into their toolflow will allow a broad audience to exploit inter-operation reuse.

