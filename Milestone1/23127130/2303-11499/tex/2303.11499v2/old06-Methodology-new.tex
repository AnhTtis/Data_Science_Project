%   \reviewme{\subsection{Architectural Implications}


\section{Evaluations}

\subsection{Experimental Methodology}
\label{sec:expt}

We describe our experimental methodology for demonstrating the efficiency of \DataflowName across baseline dataflows.
%\vspace{-2mm}


%\indent \textbf{Analytical Framework:~}
%\label{sec:simulation}
%We modify the OMEGA~\cite{garg2021understanding} framework to model the DAG of tensor operations for applications like Conjugate Gradient, Graph Convolution Networks, ResNet. OMEGA is a cost model for modelling performance and energy for various mappings with two-operation pipelining.
%Our framework builds upon STONNE~\cite{STONNE21} which is a cycle accurate simulator that models one DNN layer at a time and models flexible interconnects~\cite{kwon2018maeri} with ability to support multiple intra-operation dataflows. STONNE has been extensively validated against MAERI~\cite{kwon2018maeri} and SIGMA~\cite{sigma} RTLs. STONNE can model SpMM, DenseGEMMs/CONVs on the accelerator with reconfigurable interconnects. Prior work OMEGA~\cite{garg2021understanding} builds a cost model which uses the STONNE output and timestamps to model pipelining between two operations. We modify the cost model that OMEGA uses to model a DAG of tensors in applications like Conjugate Gradient. We use that to simulate the baselines and \DataflowName which are described in~\autoref{sec:baseline}.

\reviewme{\textbf{Modeling Framework.}
%\label{sec:simulation}
We developed an analytical framework to model the DAG of tensor operations for applications like Conjugate Gradient, GNNs, DNNs (e.g., ResNets) and schedule the operators in an op-by-op manner or via all the pipelined schemes introduced in this work. For each operator, we determine the best intra-operator dataflows and tile sizes (using individual operation tiling strategies in~\autoref{sec:tiling}) to maximize compute utilization.
%This dataflow search is similar to the one employed by mappers in MAESTRO~\cite{chatarasi2020marvel} and Timeloop~\cite{timeloop}.
%, assuming a flexible dataflow accelerator. 
Next, we leverage \DataflowName to determine the inter-operator mapping.
%We then schedule the operators over an accelerator simulated using STONNE~\cite{STONNE21}, which is a cycle accurate simulator for flexible dense and sparse dataflow accelerators, including our baseline Flexagon~\cite{flexagon}\footnote{\reviewme{The reason to pick STONNE, rather than an analytical cost model~\cite{timeloop} is due to its support for sparse inputs.}}.
Our framework captures the memory acceseses through the hierarchy and communication over the NoC across all the mapping strategies. DRAM accesses for individual operations are validated using the STONNE cycle-accurate simulator~\cite{STONNE21}. Our framework also computes the arithmetic intensities from the DRAM accesses and the throughput using the equations in~\autoref{sec:roofline}, which we use to plot gaps from ideal in our results. 
%We plot the throughput of all the data points and expand the last one into a roofline showing both throughput and arithmetic intensity. We also plot reduction in DRAM accesses and sensitivity studies on inter-cluster NoC communication and change in memory bandwidth (\autoref{fig:comm}).
}


% We developed an in-house analytical model to capture the memory accesses and communication across an accelerator's memory hierarchy. We validated per-layer statistics against STONNE~\cite{STONNE21}, which is a cycle accurate simulator that models one operation (it can model GEMM, SpMM, SpMSpM, CONV) at a time. 
% STONNE also models our baseline flexible accelerator Flexagon~\cite{flexagon}.

% We build an analytical model to capture the memory accesses and communication across clusters for the baselines and \DataflowName to compare the achieved arithmetic intensity of the dataflows independent of the accelerator implementation. We validate the DRAM statistics for a single layer by using STONNE~\cite{STONNE21}, which is a cycle accurate simulator that models one operation (it can model GEMM, SpMM, SpMSpM, CONV) at a time. STONNE can be supplied with various dataflows and tile sizes as inputs. STONNE also models the flexible accelerator Flexagon~\cite{flexagon}. We model SpMM (the sparse matrix is in CSR format) and GEMM the microarchitecture of Flexagon since it is Flexible enough to support any mapping to get a strong op-by-op baseline. %We validated the data movement for a pair of two pipelined tensor operations using OMEGA~\cite{garg2021understanding,omega} which is a wrapper around STONNE to model pipelining between two operations. 

% to use roman numerals for table references
% only required if using the endfloat package
% \renewcommand{\theposttbl}{\Roman{posttbl}}
% same for figures
% \renewcommand{\thepostfig}{\Roman{postfig}}

% change font size for title
% \Huge for 10pt is 24.88 (first number), with baselineskip = 1.2 * fontsize, 29.856
% no number for 9pt, use 24.88
% font sizes https://tex.stackexchange.com/questions/24599/what-point-pt-font-size-are-large-etc
\def\titlefontsize{\fontsize{24.88}{29.856}\selectfont}
% to change font size for bibliography
% when using biblatex
% default is \normalsize
% font list:
% \tiny, \scriptsize, \footnotesize, \small, \normalsize, \large, \Large, \LARGE, \huge and \Huge
\def\bibliographyfontsize{\fontsize{7.3}{8.3pt}\selectfont}

\label{table:dataset}

\begin{table}[t!]
\begin{footnotesize}

  \begin{centering}
  \caption{Workloads and Datasets evaluated. Protein is a graph classification dataset and we classify a batch of 64 graphs similar to OMEGA. 'nnz' stands for number of non zeros in $M\times M$ matrix. The ranks M,N,O are the ones described in the Einsums in~\autoref{sec:apps}. For Conjugate Gradients, we sweep N as a parameter as discussed in~\autoref{sec:params}.}
\begin{tabular}{lll}
\toprule
\textbf{Workload}   & \textbf{Dataset} & \textbf{Shapes and Sparsity} \\
\toprule

& aft02 & M=8184, nnz=127762\\
Conjugate & ecology1 & M=1000000, nnz=4996000\\
Gradient & Barth5 & M=15606, nnz=61484\\
& Nasa4704 & M=4704, nnz=104756\\
%& Barth5 & M=15606, nnz=61484\\

\hline

GCN & cora & M=2708, nnz=9464, N=1433, O=7\\
Layer & protein* & M=3786, nnz=14456, N=29, O=2\\

%Resnet & ImageNet & Conv\_3x residual block 1~\cite{resnet}\\

\bottomrule
\end{tabular}
\vspace{-2mm}
\end{centering}
\end{footnotesize}

\end{table}

\begin{comment}
Mutag% (MU)
& 188                & 17.93                                                              & 19.79                                                              & 28*                  \\
Proteins% (PR)
& 1113               & 39.06                                                              & 72.82                                                              & %1(reg)/
29(full)                   \\
Imdb-bin% (IB)
& 1000               & 19.77                                                              & 96.53                                                              & 136*                 \\
Reddit-bin% (RB)
& 2000               & 429.63                                                             & 497.75                                                             & 3782*                \\
Collab% (CL)
& 5000               & 74.49                                                              & 2457.78                                                            & 492*                 \\
\hline
%Pubmed (PB)     & 1                  & 19,717                                                             & 88,676                                                             & 500                  \\
Citeseer% (CS)
& 1                  & 3327                                                               & 9464                                                               & 3703                 \\
Cora% (CR)
& 1                  & 2708                                                               & 10858                                                              & 1433 
\end{comment}

\begin{scriptsize}
\begin{table*}[h]
    
\begin{scriptsize}
\begin{center}

    %\vspace{-3mm}
  \center
  \caption{Different schedules and buffers configurations evaluated with the corresponding SOTA accelerator works.}%\TK{please use the accelerator names in Fig 14 and Fig 15. In fact I would suggest naming the accelerators as A, B, C, in those figures so that the x axis label reduces and this will save significant spacre as we are short on space. And A, B, C would have been defined in this table}}%\TK{We can change this to accelerators evaluated, and include another column at the left with the accelerators. First one will be Flexagon-like, second and third you can give your own names Flex-LRU, Flex-BRRIP say, and then fourth should maybe list name of SOTA pipleined acceleratr and fifth can list our name. I think having the accel names will also help reviewers who feel we didnt compare against SOTA. Now hopefully the SOTA pipelined accel will show up by name}} 
  \label{table:dataflow_config}
    
\begin{tabular}{|p{0.11\linewidth}|p{0.10\linewidth}|p{.15\linewidth}|p{0.53\linewidth}|}
    \hline
    \textbf{Schedule} & \textbf{Buffer-hierarchy} & \textbf{Combined Configuration} & \textbf{Description of the Combination of Schedule and the Buffer-hierarchy}  \\
    \hline
    %Seq & $V\times F$ & $t_{AGG}+t_{CMB}$\\
    %\hline
%    SEQ-Inner & Op-by-op Dataflow with Inner Product\\ \hline
    Best Intra-layer   
 & Explicit  & Flexagon-like (Flexagon)~\cite{flexagon} & Flexagon's flexible architecture can optimally map a single (Op-by-op) operation with any shape and sparsity. This combination achieves  the best possible single operation reuse. All ops begin and end in DRAM. Its the oracle operation-by-operation dataflow.\\
    \hline
{Best Intra-layer} 
 & LRU Cache  & Flexagon with LRU (Flex+LRU) & All accesses go through the LRU cache without any explicit management.\\\hline

{Best Intra-layer}  
 & BRRIP Cache  &  Flexagon with BRRIP (Flex+BRRIP) & All accesses go through the BRRIP cache without any explicit management.\\\hline

    Pipelining  & {Explicit} & FLAT-like (FLAT)~\cite{flat} & We model the FLAT R-gran dataflow for pipelining between two operations when it is possible to apply (instances with delayed downstream consumers are not considered as pipeline just consumes the tensor without writeback). Also note, that this work assumes Parallel Pipeline (PP) throughout, and this baseline captures the dependency of intra-operation dataflows according to FLAT R-Gran, even though, the actual hardware implementation of FLAT uses Sequenatial Pipeline (SP) dataflow~\cite{garg2021understanding}. However, this does not impact the DRAM accesses. \\\hline
  %  GOGETA-df & GOGETA dataflow with inter-op pattern~\autoref{alg:inter-op} \\\hline
  %  & and loop orders~\autoref{alg:looporder}.\\\hline

%\DataflowNamenospace & Explicit RF, Explicit pipeline buffer and LRU for downstream tensors (\textit{LRU-hybrid}) & We use the schedule generated by~\DataflowNamenospace. Here, we use the LRU for large tensors that are reused within multiple operations. We still use register files for small operands and p-sum reduction, pipeline buffer for inter-operation pipelining. LRU is only used for downstream tensors (we just the~\SpadName buffer by cache).\\\hline
    
    %\DataflowNamenospace & \textit{BRRIP-hybrid} & Similar to~\DataflowName with \textit{LRU-hybrid}, except we use the BRRIP policy instead of the LRU.
   % \\\hline

    \DataflowNamenospace & \SpadNamenospace & \AccelNamenospace\ &(\textbf{This work}) We use~\SpadName buffer for reuse of large tensors that have downstream consumers along with the explicit pipeline buffers,  and  RFs.  It uses both~\PolicyA and~\PolicyB policies.
    \\\hline
    \hline

    Best Intra-layer & \PolicyAnospace-only & \PolicyAnospace-only & \textbf{(Additional study in \autoref{sec:sens})} We turn off all other optimizations, and model the effect of an SRAM with \PolicyA as the only policy.\\\hline

     Pipelining + Delayed Hold & Explicit & SET-like (SET)~\cite{isca-pip} & \textbf{(Additional study in \autoref{sec:sens})} We add another baseline SET which supports the delayed hold, for the ResNet evaluation since that is the only place the dependency shows up. \\\hline
%    GOGETA+org & GOGETA-map with optimal tensor allocation ~\autoref{sec:tornado}\\\hline
    
    %Total L2 capacity & 32MB\\ \hline
    %Total L3 capacity & 256MB\\ \hline
    
  \end{tabular}\end{center}

  %*{While Flexagon proposes an accelerator for SpMSpM, we model Flexagon’s flexible NoC and microarchitecture which can run any single operation dataflow and model SpMM and \GEMM on it.}

%\vspace{-3mm}





 % \vspace{1mm}
%\vspace{-4mm}
\end{scriptsize}

\end{table*}

\end{scriptsize}
\label{sec:dataset}

\reviewme{\textbf{Architecture Parameters}
\label{sec:arch} 
We consider a clustered architecture, each cluster of 1024 PEs connected using a NoC, backed by its own SRAM slice, as shown in \autoref{fig:spacc}. We sweep different SRAM capacities and memory bandwidths as shown in \autoref{table:config} . Together with diverse tensor sizes across workloads and datasets, this allows us to study the efficacy of \DataflowName across diverse tensor shapes and sizes.} 

\indent \textbf{Workloads and Datasets:~} 
\autoref{table:dataset} shows the workloads and the datasets.
We obtain the sparse matrices of the CG datasets from Suitesparse~\cite{suitesparse} for scientific problems like structural problems, circuit simulation, and so on. For CG, we also sweep the $N$ rank that corresponds to number of simultaneous initial guesses, as shown in \autoref{table:config}. 
We obtain GNN graphs from OMEGA~\cite{omega}.
We run workloads of different sizes and nnz's. We also evaluate ResNet and SSD ResNet.%{\color{red} @Raveesh -- the table also shows params for Attention layer now. Might be good to simply point to the table sayin we vary different params across workloads to save space}




% \indent \textbf{Workload and Architecture Parameters:~}
% \autoref{table:config} describes the architecture configuration we use for the evaluation and the parameters we sweep. We run workloads of different sizes and nnz's and we also sweep the $N$ rank that corresponds to number of simultaneous initial guesses and the SRAM size to capture different scenarios with varying ratios of tensor and SRAM sizes. In some cases, tensors are either too small or too large for the SRAM and in some cases, a varying proportions of tensors can fit.
% We consider a clustered architecture, each cluster of 1024 PEs backed by its own SRAM slice.

\label{sec:baseline}



\indent \textbf{Dataflow Configurations:~}
%\TK{Note to self -- we should be able to shorten this .. is repeating what table IV says}
We compare multiple dataflow configurations described in~\autoref{table:dataflow_config}. 
We evaluate all schemes over the same HW substrate (\autoref{sec:arch_implications}) to isolate the effects due to the mapping instead of microarchitectural effects.

\begin{comment}
Flexagon-like dataflow models an oracle operation-by-operation dataflow, running SpMM and~\GEMM on the flexible microarchitecture.
%It is able to achieve the lowest possible DRAM accesses for individual operations, but it requires the output tensor of each operation ending up in the DRAM and the input coming from the DRAM.}
%
%Flexagon-like dataflow is modelled based on the microarchitecture of Flexagon~\cite{flexagon}. We model SpMM and~\GEMM on the flexible microarchitecture. It is able to achieve the lowest possible DRAM accesses for individual operations but it considers the output tensor of each operation ending up in the DRAM and the input coming from the DRAM. Its the oracle operation-by-operation dataflow. 
%
Adjacent pipeline baseline consists of situations where only adjacent pipeline is possible and any kind of dependency that results in writeback or hold results in sequential execution. We demonstrate that in CG, adjacent pipeline cannot be applied while it can be applied in GNN layer and only part of ResNet layer.
\DataflowNamenospace-\textsc{Qpad} does not consider pipelining, however, it writes the portion of the tensor that fits in the scratchpad in FIFO order proposed in~\autoref{sec:tornado} but does not consider distance/frequency based replacement. This ensures that parts of intermediate tensors that could fit, are still reused inside the SRAM. \DataflowNamenospace-\textsc{Qpad} by itself does not minimize swizzle penalty, it still writes the data in FIFO manner for tensors that are not swizzled later.
%
\DataflowNamenospace-\textsc{all} uses all the contributions proposed.
%It uses the inter-op patterns proposed in~\autoref{sec:dataflows} and finds the individual loop orders which enable pipelining and avoid swizzle\_penalty proposed in~\autoref{sec:loop} and also uses the scratchpad management in~\autoref{sec:tornado}. It also uses scalable tiling strategy that reduces inter-cluster communication overhead proposed in~\autoref{sec:tiling}.

Ideal, represents the arithmetic intensity with perfect reuse and infinite SRAM capacity.
\end{comment}

\label{sec:params}

\insertFigure{AI}{Throughput for CG workload for first 10 iterations of the CG loop. Memory BW = 1000 GB/s. We sweep the SRAM capacity - \{1MB, 4MB, 16MB\}. N=\{1,16\} and is mentioned in parenthesis as (1)/(16) along with the dataflow configuration. \reviewme{At the bottom, we show the dataset-wise geomean throughput gain of \DataflowName \textsc{all} over Flexagon-like. We also show the last chart on a roofline.}\vspace{-3mm}}

\insertFigure{GNN}{Throughput for GNNs and ResNets\vspace{-1mm}}

\insertWideFigure{comm}{(a) Normalized DRAM access reduction for each dataflow configuration geomeaned across workload and architecture configurations.
(b) Inter-cluster link traversals in KBs for N=8. (c) Effect of Memory BW on throughput}
%\insertFigure{Energy}{Normalized DRAM accesses for each dataflow configuration geomeaned across all workload and architecture configurations. This directly impacts the energy.\vspace{-1mm}}
%\insertFigureRevision{bert}{Roofline plots for attention layers.}

% \indent \textbf{Workload and Architecture Parameters:~}
%\autoref{table:config} describes the architecture configuration we use for the evaluation and the parameters we sweep. We run workloads of different sizes and nnz's and we also sweep the $N$ rank that corresponds to number of simultaneous initial guesses and the SRAM size to capture different scenarios with varying ratios of tensor and SRAM sizes. In some cases, tensors are either too small or too large for the SRAM and in some cases, a varying proportions of tensors can fit.
%We consider a clustered architecture, each cluster of 1024 PEs backed by its own SRAM slice.
