
\begin{scriptsize}
    
\begin{table}[t!]
\begin{scriptsize}

\begin{center}

    
    %\vspace{-2mm}
  \center
  \caption{Performance of CG compared to Linpack (HPL) on Top5 supercomputers. Adapted from HPCG~\cite{hpcg2021}.
} 
    \label{tables:hpcg}
  %\Rav{There is space for remarks too if needed}}
  %\TK{@Raveesh - by x do you mean you can do either s or t for that dimension? So does that mean each row is actually multiple datapoints? Thats confusing.}\Rav{It means that the datapoint that we choose for evaluation of that dataflow can have varaible tile sizes for dimension marked by X. S and T on the other hand mean that its NECESSARILY spatial or temporal}}
  \begin{tabular}{|l|l|l|l|l|}
    \hline
    \textbf{Supercomputer} & \textbf{HPL} &\textbf{HPCG} & \textbf{HPCG flops} &\textbf{HPCG:}  \\
     & \textbf{Pflops/s} &\textbf{Pflops/s} & \textbf{as \% of HPL } &\textbf{\% of peak}  \\
    \hline
1. Frontier & 1206 & 14.05 & 1.16\% & 0.8\% \\ \hline
2. Aurora & 1012 & 5.61 & 0.55 & 0.3\\\hline
3. Eagle & 561.2 & \multicolumn{3}{l|}{\revision{Not available}} \\\hline
4. Fugaku & 442.01 & 16 & 3.62\% & 3\% \\ \hline
5. Lumi & 379.7 & 4.587 & 1.2\% & 0.87\% \\ \hline
%6. Leornardo & 238.7 & 3.114 & 1.3\% & 1\% \\ \hline
%7. Summit & 148.6 & 2.93 & 2\% & 1.5\% \\ \hline
  \end{tabular}
%\vspace{-5mm}
\end{center}
\end{scriptsize}

\end{table}
\end{scriptsize}

