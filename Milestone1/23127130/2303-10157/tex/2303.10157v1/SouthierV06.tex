\documentclass[aps,prl,twocolumn,superscriptaddress,floatfix]{revtex4-2}
\usepackage[dvips]{color}
\usepackage{graphicx}
\usepackage{amssymb}
\usepackage{amsmath}
\usepackage{amsfonts}
\usepackage{bbold}
\usepackage{soul}
\usepackage{xcolor}


\begin{document}

\title{Identifying primes from entanglement dynamics}

\author{A. L. M. Southier}
\affiliation{Departamento de F\'{\i}sica, Universidade Federal do Paran\'a, 81531-990, Curitiba, PR, Brazil}\author{L. F. Santos}
\affiliation{Department of Physics, University of Connecticut, Storrs, CT, USA}
\author{P. H. Souto Ribeiro}
\affiliation{Departamento de F\'{\i}sica, Universidade Federal de Santa Catarina, 88040-900, Florian\'{o}polis, SC, Brazil}
\author{A. D. Ribeiro}
\affiliation{Departamento de F\'{\i}sica, Universidade Federal do Paran\'a, 81531-990, Curitiba, PR, Brazil}

\begin{abstract}
Factorization is the most fundamental way to determine if a number $n$ is prime or composite. Yet, this approach becomes impracticable when considering large values of $n$, a difficulty that is exploited by cryptographic protocols. We propose an alternative method to decide the primality of a natural number, that is based on the analysis of the evolution of the linear entanglement entropy. Specifically, we show that a singular behavior in the amplitudes of the Fourier series of this entropy is associated with prime numbers. We also discuss how this idea could be experimentally implemented and examine possible connections between our results and the zeros of the Riemann zeta function. 
\end{abstract}


\maketitle

Prime numbers have captured the attention of researchers for centuries. In pure mathematics, given the prominent role of primes in factorizing a positive integer $n$, much effort has been made to unravel patterns in their distribution over the set of natural numbers $\mathbb{N}$~\cite{ribenboim}. Remarkable results in this direction involve the counting function $\pi(n)$, which gives the number of primes less than or equal to $n$. The complete knowledge of $\pi(n)$ would imply being able to determine the position of each prime number $p$, since jumps of this function expose their presence. However, achieving the counting function with satisfactory accuracy for large values of $n$ has been shown to be practically impossible. In applied mathematics, this lack of knowledge is exploited for cryptographic protocols~\cite{koblitz}, such as the RSA  (Rivest-Shamir-Adleman) algorithm. To break the RSA protocol, one needs to find the prime factors of a huge $n$, which would require the implementation of Shor's algorithm in a quantum computer.

Another fascinating result involving primes is their connection with the zeros of the Riemann zeta function~$\zeta(s) $, where $s$ is a complex variable, defined as~\cite{edwards}
%
\begin{equation}
\zeta(s) \equiv \sum_{n=1}^{\infty}\frac{1}{n^s} = 
\prod_{p} \frac{1}{1-p^{-s}}=
\frac{\Gamma(1-s)}{2\pi i}\int_\gamma 
\frac{(-x)^s}{\mbox{e}^x-1} \frac{\mbox{d}x}{x}.
\label{RS}
\end{equation}  
%
The summation and the product above converge when the real part of the variable $s$ satisfies $\Re[s]>1$. The first equality in Eq.~(\ref{RS}) is due to Euler and makes evident the relation between $p$ and $\zeta(s)$.  In 1859~\cite{riemann}, Riemann achieved the expression shown on the right side of Eq.~(\ref{RS}), which involves a line integral along a particular path $\gamma$~\cite{path} and the Gamma function $\Gamma(1-s)\equiv\int_0^\infty x^{-s} \mbox{e}^{-x}\mbox{d}x$. The expression is analytic for all values of $s$, except for a simple pole at $s=1$. In the region $\Re[s]>1$, the term with the integral recovers the other two expressions in Eq.~(\ref{RS}), so one considers it as their analytic continuation.  Using these ideas, Riemann~\cite{riemann} demonstrated how the complex zeros $s_0$ of $\zeta(s)$ encode the distribution of $p$. Specifically, he showed that a certain convergent series, running over all $s_0$, recovers the function $\pi(n)$. But for large values of $n$, the number of zeros required to accurately get the counting function is so large that the use of the series to identify $p$ becomes intractable. This sophisticated connection between $\zeta(s)$ and primes is but one of the  impressive results of Ref.~\cite{riemann}. Arguments in that work also gave origin to the Riemann hypothesis, which conjectures that $\Re[s_0]=\frac12$ for all nontrivial zeros of~$\zeta(s)$.

The almost mystical status of prime numbers has also reached the physics community, especially researchers working with quantum mechanics~\cite{hutchinson2011,wolf2020}. A particularly inspiring idea is the Hilbert-P\'olya conjecture that could lead to the proof of the Riemann hypothesis. The conjecture proposes that the nontrivial zeros of the Riemann zeta function, which supposedly fall on the critical line $\Re[s_0]=\frac12$, correspond to the eigenvalues of a Hermitian Hamiltonian operator. The first substantial evidence supporting this conjecture arose in the semiclassical studies carried out by Berry and Keating~\cite{keating99}. They compared the distribution of $s_0$ over the critical line with the equivalent function for the eigenvalues of a given Hamiltonian achieved through the Gutzwiller trace formula. In this way, they were able to identify a Hamiltonian operator that fulfills the conjecture. The operator is obtained by quantizing the classical Hamiltonian $H_{\mathrm cl} = xp$, where $x$ and $p$ are the particle position and momentum, respectively.

Inspired by the Hilbert-P\'olya conjecture and the Berry and Keating $xp$-model, several works~\cite{sierra2008,hutchinson2008,sierra2011,srednicki2011,sierra2012,wolf2014,muller2017} have aimed at interpreting, extending and circumventing technical difficulties of Ref.~\cite{keating99}. Other contributions looked for alternative physical systems, where the properties of $\zeta(s)$ could be identified. This has been done using quantum graph theory~\cite{richter2014}, many-body systems~\cite{mussardo2020}, wave-packet dynamics~\cite{schleich2010}, statistical mechanics of random energy landscapes~\cite{keating2012}, random matrix theory~\cite{keating1995,leboeuf2003}, quantum entanglement~\cite{schleich2013}, and there is also an experiment based on a periodically driven single qubit~\cite{guang2020}. In addition to these works, where the Riemann zeta function is the protagonist to link prime numbers and quantum physics, other quantum approaches deal directly with primes, studying, for instance, number factorization~\cite{schleich2018, martin2018} and the properties of prime states \cite{sierra2015,sierra2020}, which are superpositions of states corresponding to prime numbers.

In the present paper, we show that the dynamics of the linear entanglement entropy encodes prime and semiprime numbers.  Specifically, we demonstrate that the Fourier modes $c_n$ of this evolving function have amplitudes that present a singular behavior when $n$ corresponds to a prime, and we find curves $c_n^{f}$ for the location of families $f$ of semiprimes. This is shown for a system of two coupled harmonic oscillators and a system of two coupled spins. We discuss a possible experimental realization of this idea and speculate on how to link $c_n$ and the counting function~$\pi(n)$.



%%%%%%%%%%%%%%%%%%%%%%%%%%%%%%%%%%%%%%%%%%%%%
{\em Entanglement dynamics.---} We consider a system consisting of two interacting parts, $A$ and $B$, to which we assign distinct Hilbert spaces, $\mathcal{H}_A$ and $\mathcal{H}_B$, respectively.  The system is isolated and prepared in a pure state,  $\rho (0) = |\Psi(0)\rangle \langle \Psi(0) |$, which evolves according to the total Hamiltonian
\begin{equation}
H= H_A\otimes \mathbb{1}_B + \mathbb{1}_A \otimes H_B + \lambda H_{A} \otimes H_{B},
\label{Hoperator}
\end{equation}
where $\lambda$ is the coupling strength between the two subsystems. The linear entanglement entropy,
%
\begin{equation}
S_{L}(t) = 1 - \mathrm{Tr}
\left[ \rho_{A}^2(t) \right],
\label{Slin}
\end{equation}
%
where $\rho_{A} (t) \equiv\mathrm{Tr}_{B} [\rho (t)]$ is the reduced density matrix of system $A$, measures the bipartite entanglement in time between  $A$ and  $B$. When $\rho(t)$ is separable, $S_L (t)\!=\!0$, otherwise $0\!<\!S_{L}(t)\!<\!1$. 



%%%%%%%%%%%%%%%%%%%%%%%%%%%%%%%%%%%%%%%%%%%%%
{\em Coupled oscillators.---} The first Hamiltonian that we consider describes two coupled harmonic oscillators, where $H_{A}$ is the Hamiltonian for part~$A$ with eigenvalues $\hbar\omega_0(n_{A}+\frac12)$ and eigenstates $|n_{A}\rangle$, and equivalently for part $B$. The initial state is the product of two canonical coherent states~\cite{gazeau},
%
\begin{equation}
|\Psi (0)\rangle =
|\alpha_{A}\,\alpha_{B} \rangle \equiv \mbox{e}^{-u/2} \!\!\!\!\!\!
\sum_{n_{A},n_{B}=0}^\infty 
\frac{\alpha_{A}^{n_A}}{\sqrt{n_{A}!}} 
\frac{\alpha_{B}^{n_{B}}}{\sqrt{n_{B}!}}
|n_{A}\,n_{B}\rangle ,
\nonumber
\end{equation}
%
where we chose $|\alpha_{A}|^2=|\alpha_{B}|^2 = \frac{u}{2}$.  Defining $\omega \equiv \hbar \lambda \omega_0^2$,
%
\begin{equation}
S_L^{osc}(t) = 
1 - \mbox{e}^{-2u} \!\! 
\sum_{j,k,l,m} \!
\frac{(\frac{u}{2})^{j+k+l+m}}{j!\,k!\,l!\,m!} 
\mbox{e}^{-i\omega t(j-k)(l-m)}.
\label{SlinC}
\end{equation}
%
The sum above  runs from 0 to $\infty$ for all indexes. Since the entropy in Eq.~(\ref{SlinC}) is periodic in time, with period $T=2\pi/\omega$, we use the Fourier series and arrive at
%
\begin{equation}
S_L^{osc}(t) = c_0(u)
- \sum_{n=1}^\infty \;  c_n(u) \cos(n\omega t) ,
\label{FSC}
\end{equation}
%
where the coefficients $c_n(u)$ are given by
%
\begin{equation}
c_{n}(u) = 4 \, \mbox{e}^{-2u}
\sum_{k,m} \sum_{j>k} 
\sum_{l>m}\frac{(\frac{u}{2})^{j+k+l+m}}{j!\,k!\,l!\,m!} \delta_{(j-k)(l-m)}^n,
\label{Cn}
\end{equation}
% 
representing the amplitude of the mode with frequency~$n\omega$. In Eq.~(\ref{Cn}), the Kronecker delta function, $\delta^b_a$ for integers $a$ and $b$, prompts the analysis of how $n$ decomposes as a product of two integers. We introduce the set $\Lambda_n$ composed of all distinct positive divisors of $n$ and note that the non-null terms in Eq.~(\ref{Cn}) are those for which $(j-k)$ and $(l-m)$ belong to $\Lambda_n$ and their product equals $n$. Therefore,
%
\begin{equation}
c_n (u) = 4 \, \mbox{e}^{-2u} \,
\sum_{\mu\in\Lambda_n} \; I_\mu(u) \, I_{\frac{n}{\mu}}(u),
\label{genesis}
\end{equation}
%
where $I_{\chi}(w) \equiv \sum_{k=0}^\infty [k!(\chi+k)!]^{-1} (\frac{w}{2})^{2k+\chi}$ is the modified Bessel function of the first kind. As we show next,  $c_n(u)$ is very sensitive to the primality of $n$. 

%%%%%%%%%%%%%%%%%%%%%%%%%%%%%%%%%%%%%%%%%%%%%
{\em Identifying primes.}---To show that the entanglement dynamics quantified by $S_L^{osc}(t)$ is a prime identifier, we first assume that $n$ is prime, so that $\Lambda_n=\{1,n\}$ and Eq.~(\ref{genesis}) yields $c_n (u) = c_n^{\scriptscriptstyle{(p)}} (u) $, where
%
\begin{equation}
c_n^{\scriptscriptstyle{(p)}} (u) \equiv 
8 \,\mbox{e}^{-2u} I_1(u) \, I_{n}(u).
\label{prime}
\end{equation}
%
In contrast, when $n$ is as a composite number, $\Lambda_n$ necessarily consists of 1, $n$, and, at least, one more integer. By defining the set $\Lambda'_n = \Lambda_n - \{1,n\}$, Eq.~(\ref{genesis}) becomes 
%
\begin{equation}
c_n(u) =  c_n^{\scriptscriptstyle{(p)}} (u) + 4 \,\mbox{e}^{-2u}
\sum_{\mu\in\Lambda'_n} I_\mu(u) \, I_{\frac{n}{\mu}}(u) \ge
c_n^{\scriptscriptstyle{(p)}} (u),
\label{composite}
\end{equation}
%
which holds only for $n>1$.

Inequality~(\ref{composite}) is a main result of this work. It shows that the coefficients $c_n(u)$ coincides with $c_n^{\scriptscriptstyle{(p)}}(u)$ in Eq.~(\ref{prime}) if, and only if, $n$ is prime, while for composite numbers, the amplitudes are strictly lower bounded by $c_n^{\scriptscriptstyle{(p)}} (u)$. This means that if we have a way to determine the values of $c_n(u)$, other than through the construction of $\Lambda_n$,  the application of Eq.~(\ref{composite}) can reveal the primality of $n$. Indeed, $c_n(u)$ can be evaluated theoretically via Eq.~(\ref{SlinC}) and possibly experimentally.

In Fig.~\ref{fig1}, we mark the values of $c_n(u)$ with blue dots. For each $n$ identified as a prime, the dots are encircled with a black circle. All values of $c_n(u)$ corresponding to primes coincide with the red dotted line that represents the lower bound $c_n^{\scriptscriptstyle{(p)}} (u)$ and can, therefore, be clearly distinguished. In Fig.~\ref{fig1}~(a), we show results for $u=1$, but  $c_n(1)$ becomes too small when $n$ is large, so in Fig.~\ref{fig1}~(b), we use $u=10^3$ and larger values of $n$.
%
\begin{figure}
\vspace{-0.5cm}
\centerline{\includegraphics[width=8.5cm]{fig1a11p2X8V04.pdf}}
\vspace{-0.3cm}
\centerline{\includegraphics[width=8.5cm]{fig1b11p2X8V04.pdf}}
\vspace{-0.2cm}
\caption{Logarithm of the coefficients $c_n(u)$ as a function of $n$ for (a) $u=1$ and (b) $u=10^3$. The blue dots, connected with a blue solid line to guide the eye, represent $c_n(u)$, and the red points indicate the lower bound $c_n^{\scriptscriptstyle{(p)}}(u)$. For each prime $n$, the blue dots are encircled with black circles and touch the red dotted line. Green and cyan lines are for $c_n^{\scriptscriptstyle{(f_2)}}(u)$ and $c_n^{\scriptscriptstyle{(f_3)}}(u)$, respectively. Black squares [triangles] enclose the coefficients of the family $f_2$ [$f_3$]. Inset plots show a magnified region of the respective main graph. }
\label{fig1}
\end{figure}

%%%%%%%%%%%%%%%%%%%%%%%%%%%%%%%%%%%%%%%%%%%%%
{\em Semiprimes.}--- Figure~\ref{fig1} also reveals the square-free semiprimes, including the integer 2, which we denote by family $f_2$. For these numbers,  $\Lambda_n=\{1,2,P_n,n\}$, where $P_n=n/2\neq 2$ is a prime, and $c_n(u)=c_n^{\scriptscriptstyle{(f_2)}} (u)$, where
%
\begin{equation}
c_n^{\scriptscriptstyle{(f_2)}} (u) \equiv 
8 \,\mbox{e}^{-2u} \left[I_1(u) \, I_{n}(u) + I_2(u) \, I_{\frac{n}{2}}(u)\right].
\label{semi2}
\end{equation}
%
For a composite $n$ that presents the divisors: 1, 2, $n/2$, $n$, and at least one more integer, 
\begin{equation}
c_n^{\scriptscriptstyle{(2)}} (u) \equiv  c_n^{\scriptscriptstyle{(f_2)}} (u) +
4\,\mbox{e}^{-2u} 
{\sum_{\mu\in\Lambda^{'\scriptscriptstyle{(2)}}_n}} 
I_\mu(u) \, I_{\frac{n}{\mu}}(u) > c_n^{\scriptscriptstyle{(f_2)}} (u) ,
\nonumber
%\label{n2}
\end{equation}
%
where $\Lambda^{'\scriptscriptstyle{(2)}}_n = \Lambda_n - \{ 1, 2, n/2, n \}$. The semiprimes of family $f_2$ are identified with black squares in Fig.~\ref{fig1} and they all fall on the curve $c_n^{\scriptscriptstyle{(f_2)}} (u) $, indicated with a green line. This line consists of a lower bound for any integer composed by 2, except for $2$, $2^2$, and $2^3$.

It is straightforward to extend the analysis above to any other family of semiprimes. The particular case of the square-free semiprimes containing the number 3, denoted by $f_3$, is shown in Fig.~\ref{fig1}. The components of family $f_3$ are marked with black triangles and they are exactly located over the cyan curve $c_n^{\scriptscriptstyle{(f_3)}} (u)$~\cite{spf3}.


%%%%%%%%%%%%%%%%%%%%%%%%%%%%%%%%%%%%%%%%%%%%%
{\em Interacting spins.---} Analogously to the system of coupled harmonic oscillators, we now show that two interacting spins with a large quantum number $S$ is another physical system that can be used to identify primes. In the  Hamiltonian of Eq.~(\ref{Hoperator}), we assume that $H_A = \hbar \omega_0 S_A^{z}$, where $S_A^{z}$ is the $z$-component of the spin operator of part $A$ and $S_A^{z}|m_{A}\rangle=  m_{A} |m_{A}\rangle$, and equivalently for part $B$. The initial state is the product of spin coherent states~\cite{gazeau}, $|\Psi(0) \rangle =  |s_{A}\, s_{B} \rangle$, where
%
\begin{equation}
\nonumber
|s_{A}\, s_{B} \rangle 
\equiv N_s \!\!\!\!\!\!
\sum_{n_A,n_B=0}^{2S} \!\! \frac{s_{A}^{n_A} s_{B}^{n_B}}{n_{A}!n_{B}!} 
\sqrt{\textstyle\binom{2S}{n_{A}} \binom{2S}{n_{B}}}
|n_{A}-S,n_{B}-S\rangle,
\end{equation}
%
where $N_s=(1+u^2)^{-2S} $ and we chose $|s_{A}|^2= |s_{B}|^2 = u$.

The evolving linear entanglement entropy~(\ref{Slin}) becomes
%
\begin{equation}
S_L^{spin}(t) = 1 - \! \! \! \!
\! \sum_{j,k,l,m=0}^{2S} \! \! \! \! \xi_{j,k,l,m} \,
u^{j+k+l+m} \mbox{e}^{-i\omega t(j-k)(l-m)},
\label{Sspin}
\end{equation}
%
where $\xi_{j,k,l,m}\equiv \,
{\textstyle \binom{2S}{j} \binom{2S}{k} \binom{2S}{l} \binom{2S}{m}} (1+u)^{-8S}$. The Fourier series of Eq.~(\ref{Sspin})  has the same structure as Eq.~(\ref{FSC}), but with the coefficients
%
\begin{equation}
\nonumber
\bar{c}_{n}(u) =
4 \sum_{k,m=0}^{2S-1} \sum_{j>k}^{2S} \sum_{l>m}^{2S} 
\xi_{j,k,l,m} \, u^{j+k+l+m} \delta_{(j-k)(l-m)}^n.
\end{equation}
%
The nonvanishing terms in the equation above are again those for which $\mu\equiv(j-k)$ and $\nu\equiv(l-m)$ belong to the set $\Lambda_n$ of divisors of $n$. However, this system is bounded, so the upper limits of the above summations add new constraints to $\mu$ and $\nu$. Given a value for $k$ and $m$, we have that $\mu\in \mathcal{I}_{k}\equiv [1,2S-k]$ and $\nu\in \mathcal{I}_{m}\equiv  [1,2S-m]$. In addition, since  $\delta_{\mu\nu}^n$  implies that $\nu=n/\mu$, we also get that $\mu\in \mathcal{I}_{m,n}\equiv[\frac{n}{2S-m}, n]$. Therefore,
%
\begin{equation}
\bar{c}_{n}(u) = 4\sum_{k,m=0}^{2S-1} \, \sum_{\mu\in\bar{\Lambda}^{\scriptscriptstyle{(k,m)}}_n} 
\!\!\! \bar{\xi} \, u^{2k+2m+\mu+n/\mu} ,
\label{cns}
\end{equation}
%
where $\bar{\xi} \equiv  \xi_{k+\mu,k,m+\frac{n}{\mu},m}$ and $\bar{\Lambda}_n^{\scriptscriptstyle{(k,m)}}$ is the set of divisors of $n$ that satisfies all constraints commented above, 
%
\begin{equation}
\bar{\Lambda}_n^{\scriptscriptstyle{(k,m)}} \equiv \Lambda_n \cap \mathcal{I}_{k} \cap \mathcal{I}_{m,n}.
\label{Lmk}
\end{equation}

Contrary to $S_L^{osc}(t)$ in Eq.~(\ref{FSC}), there is a finite number of Fourier modes in $S_L^{spin}(t)$ due to the finite Hilbert space for the spin system. For $n>4S^2$, we have that $\mathcal{I}_{k} \cap \mathcal{I}_{m,n} =\emptyset$ for any value of $k$ and $m$, so $\bar{\Lambda}_n^{\scriptscriptstyle{(k,m)}}=\emptyset$ and $\bar{c}_n(u)=0$. In what follows, to assess the primality of $n$, we conveniently divide the interval $1<n\le 4S^2$ in two regions of interest, region I ($1 < n \le 2S$) and region II ($2S < n \le 4S$). In the complementary range $4S < n \le 4S^2$, one cannot find a behavior that distinguishes primes from composite numbers. There, even though all amplitudes of the Fourier modes for a prime $n$ vanish, the same can also happen to some composite numbers.

Region I ($1< n \le 2S$): A careful inspection of Eq.~(\ref{Lmk}) reveals that $1\in\bar{\Lambda}^{(k,m)}_{n}$ for all values of $k$ and $0\le m\le 2S-n$, while $n\in\bar{\Lambda}^{(k,m)}_{n}$ for all values of $m$ and $0\le k\le 2S-n$. Applying these results to Eq.~(\ref{cns}), we find that, for $n$ prime, $\bar{c}_n(u)=\bar{c}^{{\scriptscriptstyle (p)}}_{n,I}(u)$, where
%
\begin{equation}
\bar{c}_{n,I}^{{\scriptscriptstyle (p)}}(u) \equiv 8\,(1+u)^{-8S} G_1(u) G_n (u),  
\label{cnsp}
\end{equation}
%
with $G_\chi(w) \equiv \sum_{k=0}^{2S-\chi} \binom{2S}{k}\binom{2S}{k+\chi} w^{2k+\chi}$. On the other hand, for a composite number $n>1$ inside region I, we have that $\bar{c}_n(u) = \bar{c}_{n}^{{\scriptscriptstyle (p)}}(u)+\bar{d}_{n,I}(u)$, where
%
\begin{equation}
\bar{d}_{n,I}(u) \equiv 4
{\sum_{k,m=0}^{2S-1} \, \sum_{\mu\in{\bar{\Lambda}^{\prime\scriptscriptstyle{(k,m)}}_n}}}
\, \bar{\xi} \, u^{2k+2m+\mu+n/\mu}   
\label{cnsc}
\end{equation}
%
and ${\bar{\Lambda}^{\prime\scriptscriptstyle{(k,m)}}_n} \equiv \Lambda'_n \cap \mathcal{I}_{k} \cap \mathcal{I}_{m,n}$, where $\Lambda'_n$ was defined above Eq.~(\ref{composite}). One can show that at least for $k=m=0$, there is a positive non-null term in Eq.~(\ref{cnsc}). Therefore, for a composite number, we necessarily have $\bar{c}_n(u) > \bar{c}_{n,I}^{{\scriptscriptstyle (p)}}(u)$ and this inequality can be employed in the search for prime numbers in region I.

Region II ($2S<n\le 4S$): In this case, prime modes are not included in the Fourier series of Eq.~(\ref{Sspin}). Indeed, for a prime number $n$, we now have that $1\not\in \mathcal{I}_{m,n}$ and $n\not\in \mathcal{I}_{k} $, which implies that $\bar{\Lambda}_n^{\scriptscriptstyle{(k,m)}}=\emptyset$ and $\bar{c}_n(u)=0$.  For a composite $n$ in the same interval, similarly to region I, there is at least one integer in $\Lambda_n$, in addition to 1 and $n$, so that $\bar{c}_n(u) > 0$. So we can distinguish primes from composite numbers by checking if $\bar{c}_n(u)=0$ or $\bar{c}_n(u) \neq 0$.

%
\begin{figure}
    \centering
    \includegraphics[width=8.5cm]{setupV06.pdf}
    \caption{Setup for the identification of prime numbers. The symbols BS, HWP, PBS, M, KM, and PD refer to beam splitter, half-wave plate, polarizing beam splitter, mirror, Kerr medium, and photodiode, respectively. See text for details.}
    \label{fig_setup}
\end{figure}

%%%%%%%%%%%%%%%%%%%%%%%%%%%%%%%%%%%%%%%%%%%%%
{\em Experimental proposal.---} The Hamiltonian for the two coupled harmonic oscillators studied above also describes the interaction between two optical fields via a Kerr nonlinear medium~\cite{scully, yamamoto}. In Fig.~\ref{fig_setup}, we show the sketch of an experimental setup for implementing this optical system. A laser beam of frequency $\omega_0$ and linear polarization at $45^\circ$ is sent to a polarizing beam splitter (PBS). The vertically polarized~($\updownarrow$) component goes directly to the homodyne detection scheme~\cite{HomodyneDetect}, shown in the lower right corner of the figure, to act as the local field. The horizontally polarized ($\leftrightarrow$) component passes through a half waveplate (HWP), which rotates the polarization to $45^\circ$, and enters an unbalanced Mach-Zehnder interferometer, identified in Fig.~\ref{fig_setup} with dashed lines. In the interferometer, after the PBS, the $\updownarrow$~component beam propagates through the short path, and the $\leftrightarrow$~one goes through the long course. The path difference is longer than the coherence length, so that the recombined $\leftrightarrow$ and $\updownarrow$ beams at the interferometer output PBS are separable and no longer result in a pure mode with linear diagonal polarization. Next, these two beams are injected in the nonlinear Kerr medium (KM) of length $L$ and Kerr optical nonlinearity~$\chi^{(3)}$. During the propagation time inside the KM, each beam will experience a modified index of refraction due to the action of the other beam. This is how the coupling between the oscillators is physically implemented. By varying the length $L$, one changes the interaction time $t$. After passing KM, the $\leftrightarrow$ and $\updownarrow$ beams are split again. To identify the prime numbers, we measure one of the beams and ignore the other, which is equivalent to performing the trace over one of the interacting systems. We can analyze any of the two beams because of the interaction symmetry. In Fig.~\ref{fig_setup}, we choose the $\updownarrow$ beam, which goes to the homodyne detector, where quantum state tomography is performed to reconstruct the density matrix $\rho_A$. With the reduced density matrix, we calculate the linear entanglement entropy. After several experimental realizations spanning the needed values of~$t$, curves like those in Fig.~\ref{fig1} can be generated. 
 

%%%%%%%%%%%%%%%%%%%%%%%%%%%%%%%%%%%%%%%%%%%%%
{\em Conclusion.---} This paper shows that by analyzing the bipartite entanglement in time, one can identify prime and semiprime numbers in $\mathbb{N}$. The main ingredient is a Hamiltonian composed of two parts, $A$ and $B$, with equidistant energy levels for $H_A$ and $H_B$. We discussed how this idea could be implemented taking advantage of  an existing experimental setup.

Our work resonates with the Hilbert-P\'olya conjecture in the sense of proposing a physically measurable quantity to determine prime numbers. We speculate that there may be a way to connect our results with $\zeta(s)$. In fact, by defining $A(n,u) \equiv c_n(u) - c^{\scriptscriptstyle(p)}_n(u)$ for a fixed~$u$, the task of counting primes along $\mathbb{N}$ is equivalent to counting the zeros of $A(n,u)$. This alternative method to build $\pi(n)$ could be used to study the zeros of $\zeta(s)$ by means, for example, of a hypothetical inversion of the Riemann series that connects $s_0$ with $\pi(n)$, shedding new light on the Riemann hypothesis.
% https://empslocal.ex.ac.uk/people/staff/mrwatkin/zeta/encoding1.htm

Our last remark concerns the semiclassical approaches originated from the ideas of Berry and Keating~\cite{keating99}. The semiclassical version of the linear entanglement entropy was addressed in~\cite{arlans,matheus} for the same two physical systems treated here. In those papers, entanglement is reproduced by summing over sets of classical trajectories determined by the solutions of a given transcendental equation. The purely quantum formalism presented here encourages the re-examination of those works aiming at connecting those solutions with the distribution of primes.

%--------------------------------------------------------------------
%--------------------------------------------------------------------
\section*{Acknowledgements}

This study was financed by the Brazilian agencies: Funda\c{c}\~{a}o de Amparo \`{a} Pesquisa do Estado de Santa Catarina (FAPESC -- DOI 501100005667), Coordena\c{c}\~ao de Aperfei\c{c}oamento de Pessoal de N\'{i}vel Superior (CAPES -- DOI 501100002322), Conselho Nacional de Desenvolvimento Cient\'{\i}fico e Tecnol\'ogico (CNPq -- DOI 501100003593), and Instituto Nacional de Ci\^encia e Tecnologia de Informa\c{c}\~ao Qu\^antica (INCT-IQ 465469/2014-0). LFS was supported by the NSF CCI grant (Award Number 2124511). The authors also thank Gisele T. Paula and Jo\~ao V. P. Poletto for their valuable discussion at the very beginning of the work.

%--------------------------------------------------------------------
%--------------------------------------------------------------------
\begin{thebibliography}{99}

\bibitem{ribenboim}
P. Ribenboim,
{\it The book of prime number records}
(Springer-Verlag, New York, 1988)
  
\bibitem{koblitz}
N. Koblitz,
{\it A course in number theory and cryptography}
(Springer, New York, 1994)

\bibitem{edwards}
H. M. Edwards,
{\it Riemann's zeta function}
(Dover, New York, 2001).

\bibitem{riemann}
B. Riemann,
Uber die Anzahl der Primzahlen unter einer gegebenen Grösse, 
Monatsb. der Berliner Akad., 671 (1859).

\bibitem{path}
The path of integration $\gamma$, defined in the $x$-complex plane, starts at $x\to\infty+i\epsilon$, with $\epsilon\to0_+$, goes to the origin parallelly to the real axis, encircles it in the counter clock sense, and returns to the final point $x\to\infty-i\epsilon$, also parallelly to the real axis. 

\bibitem{hutchinson2011}
D. Schumayer and D. A. W. Hutchinson,
Physics of the Riemann hypothesis,
Rev. Mod. Phys. {\bf 83}, 307 (2011).

\bibitem{wolf2020}
M. Wolf,
Will a physicist prove the Riemann hypothesis?,
Rep. Prog. Phys. {\bf 83}, 036001 (2020).

\bibitem{keating99}
M. V. Berry and J. P. Keating,
The Riemann zeros and eigenvalues asymptotics,
SIAM {\bf 41}, 236 (1999).

\bibitem{sierra2008}
G. Sierra and P. K. Townsend,
Landau levels and Riemann zeros,
Phys. Rev. Lett. {\bf 101}, 110201~(2008).
%The number N(E) of complex zeros of the Riemann zeta function with positive imaginary part less than E is the sum of a “smooth” function ¯¯¯¯N(E) and a “fluctuation.” Berry and Keating have shown that the asymptotic expansion of ¯¯¯¯N(E) counts states of positive energy less than E in a “regularized” semiclassical model with classical Hamiltonian H=xp. For a different regularization, Connes has shown that it counts states “missing” from a continuum. Here we show how the “absorption spectrum” model of Connes emerges as the lowest Landau level limit of a specific quantum-mechanical model for a charged particle on a planar surface in an electric potential and uniform magnetic field. We suggest a role for the higher Landau levels in the fluctuation part of N(E).(--- They follow POLYA-HILBERT conjecture --- a physical interpretation for xp model --- suggestion of an experiment --- SEMICLASSICAL)

\bibitem{hutchinson2008}
D. Schumayer, B. P. van Zyl, and D. A. W. Hutchinson,
Quantum mechanical potentials related to the prime numbers and Riemann zeros,
Phys. Rev. E {\bf 78}, 056215 (2008).
%Prime numbers are the building blocks of our arithmetic; however, their distribution still poses fundamental questions. Riemann showed that the distribution of primes could be given explicitly if one knew the distribution of the nontrivial zeros of the Riemann ζ(s) function. According to the Hilbert-Pólya conjecture, there exists a Hermitian operator of which the eigenvalues coincide with the real parts of the nontrivial zeros of ζ(s). This idea has encouraged physicists to examine the properties of such possible operators, and they have found interesting connections between the distribution of zeros and the distribution of energy eigenvalues of quantum systems. We apply the Marčhenko approach to construct potentials with energy eigenvalues equal to the prime numbers and to the zeros of the ζ(s) function. We demonstrate the multifractal nature of these potentials by measuring the Rényi dimension of their graphs. Our results offer hope for further analytical progress. (-- PH-conjecture via Marčhenko’s method, one of the inverse scattering methods---  )

\bibitem{sierra2011}
G. Sierra and J. Rodr\'{\i}guez-Laguna,
$H=xp$ model revisited and the Riemann zeros,
Phys. Rev. Lett. {\bf 106}, 200201~(2011).
%Berry and Keating conjectured that the classical Hamiltonian H=xp is related to the Riemann zeros. A regularization of this model yields semiclassical energies that behave, on average, as the nontrivial zeros of the Riemann zeta function. However, the classical trajectories are not closed, rendering the model incomplete. In this Letter, we show that the Hamiltonian H=x(p+ℓ2p/p) contains closed periodic orbits, and that its spectrum coincides with the average Riemann zeros. This result is generalized to Dirichlet L functions using different self-adjoint extensions of H. We discuss the relation of our work to Polya’s fake zeta function and suggest an experimental realization in terms of the Landau model. (classical H whose quantum version would realize the POLYA-HILBERT conjecture --- SEMICLASSICAL)

\bibitem{srednicki2011}
M. Srednicki,
The Berry{\textendash}Keating Hamiltonian and the local Riemann hypothesis,
J. Phys. A: Math. Theor. {\bf 44}, 305202 (2011).
%The local Riemann hypothesis states that the zeros of the Mellin transform of a harmonic-oscillator eigenfunction (on a real or p-adic configuration space) have a real part 1/2. For the real case, we show that the imaginary parts of these zeros are the eigenvalues of the Berry–Keating Hamiltonian projected onto the subspace of oscillator eigenfunctions of a lower level. This gives a spectral proof of the local Riemann hypothesis for the reals, in the spirit of the Hilbert–Pólya conjecture. The p-adic case is also discussed.(--- exploring xp hamiltonian-PH conjecture)

\bibitem{sierra2012}
G. Sierra,	
General covariant $xp$ models and the Riemann zeros,
J. Phys. A: Math. Theor. {\bf 45}, 055209 (2012).
% We study a general class of models whose classical Hamiltonians are given by H = U(x)p + V(x)/p, where x and p are the position and momentum of a particle moving in one dimension, and U and V are positive functions. This class includes the Hamiltonians HI = x(p + 1/p) and HII = (x + 1/x)(p + 1/p), which have been recently discussed in connection with the nontrivial zeros of the Riemann zeta function. We show that all these models are covariant under general coordinate transformations. This remarkable property becomes explicit in the Lagrangian formulation which describes a relativistic particle moving in a (1+1)-dimensional spacetime whose metric is constructed from the functions U and V. General covariance is maintained by quantization and we find that the spectra are closely related to the geometry of the associated spacetimes. In particular, the Hamiltonian HI corresponds to a flat spacetime, whereas its spectrum approaches the Riemann zeros on average. The latter property also holds for the model HII, whose underlying spacetime is asymptotically flat. These results suggest the existence of a Hamiltonian whose underlying spacetime encodes the prime numbers, and whose spectrum provides the Riemann zeros.(--- exploring xp hamiltonian-PH conjecture)

\bibitem{wolf2014}
M. Wolf,
Nearest-neighbor-spacing distribution of prime numbers and quantum chaos,
Phys. Rev. E {\bf 2014}, 022922 (2014).
%We give heuristic arguments and computer results to support the hypothesis that, after appropriate rescaling, the statistics of spacings between adjacent prime numbers follows the Poisson distribution. The scaling transformation removes the oscillations in the nearest-neighbor-spacing distribution of primes. These oscillations have the very profound period of length six. We also calculate the spectral rigidity Δ3 for prime numbers by two methods. After suitable averaging one of these methods gives the Poisson dependence Δ3(L)=L/15.(--- PH-conjecture, Distribution of primes---)

\bibitem{muller2017}
C. M. Bender, D. J. Brody, and M. P. M\"uller,
Hamiltonian for the Zeros of the Riemann Zeta Function,
Phys. Rev. Lett. {\bf 118}, 130201 (2017).
% Hamiltonian operator ^H is constructed with the property that if the eigenfunctions obey a suitable boundary condition, then the associated eigenvalues correspond to the nontrivial zeros of the Riemann zeta function. The classical limit of ^H is 2xp, which is consistent with the Berry-Keating conjecture. While ^H is not Hermitian in the conventional sense, i^H is PT symmetric with a broken PT symmetry, thus allowing for the possibility that all eigenvalues of ^H are real. A heuristic analysis is presented for the construction of the metric operator to define an inner-product space, on which the Hamiltonian is Hermitian. If the analysis presented here can be made rigorous to show that ^H is manifestly self-adjoint, then this implies that the Riemann hypothesis holds true.(-- PH-conjecture --- RH is true!)



\bibitem{richter2014}
J. Kuipers, Q. Hummel, and K. Richter,
Quantum graphs whose spectra mimic the zeros of the Riemann zeta function,
Phys. Rev. Lett. {\bf 112}, 070406 (2014).
%One of the most famous problems in mathematics is the Riemann hypothesis: that the nontrivial zeros of the Riemann zeta function lie on a line in the complex plane. One way to prove the hypothesis would be to identify the zeros as eigenvalues of a Hermitian operator, many of whose properties can be derived through the analogy to quantum chaos. Using this, we construct a set of quantum graphs that have the same oscillating part of the density of states as the Riemann zeros, offering an explanation of the overall minus sign. The smooth part is completely different, and hence also the spectrum, but the graphs pick out the low-lying zeros.(--- exploring xp hamiltonian-PH conjecture, SEMICLASSICAL -- Using quantum GRAPHS--- "possible route to finding a quantum system that exactly mimics the Riemann zeros".)


\bibitem{mussardo2020}
G. Mussardo, A. Trombettoni, and Z. Zhang,
Prime suspects in a quantum ladder,
Phys. Rev. Lett. {\bf 125}, 240603~(2020).
%In this Letter we set up a suggestive number theory interpretation of a quantum ladder system made of N coupled chains of spin 1/2. Using the hard-core boson representation and a leg-Hamiltonian made of a magnetic field and a hopping term, we can associate to the spins σa the prime numbers pa so that the chains become quantum registers for square-free integers. The rung Hamiltonian involves permutation terms between next-neighbor chains and a coprime repulsive interaction. The system has various phases; in particular, there is one whose ground state is a coherent superposition of the first N prime numbers. We also discuss the realization of such a model in terms of an open quantum system with a dissipative Lindblad dynamics. (--- MANY BODY SYSTEM and number theory --- Use of primes to understand --- Ground state as a superposition of primes.)

\bibitem{schleich2010}
R. Mack, J. P. Dahl, H. Moya-Cessa, W. T. Strunz, R. Walser, and W. P. Schleich,
Riemann $\ensuremath{\zeta}$ function from wave-packet dynamics,
Phys. Rev. A {\bf 82}, 032119 (2010).
%We show that the time evolution of a thermal phase state of an anharmonic oscillator with logarithmic energy spectrum is intimately connected to the generalized Riemann ζ function ζ(s,a). Indeed, the autocorrelation function at a time t is determined by ζ(σ+iτ,a), where σ is governed by the temperature of the thermal phase state and τ is proportional to t. We use the JWKB method to solve the inverse spectral problem for a general logarithmic energy spectrum; that is, we determine a family of potentials giving rise to such a spectrum. For large distances, all potentials display a universal behavior; they take the shape of a logarithm. However, their form close to the origin depends on the value of the Hurwitz parameter a in ζ(s,a). In particular, we establish a connection between the value of the potential energy at its minimum, the Hurwitz parameter and the Maslov index of JWKB. We compare and contrast exact and approximate eigenvalues of purely logarithmic potentials. Moreover, we use a numerical method to find a potential which leads to exact logarithmic eigenvalues. We discuss possible realizations of Riemann ζ wave-packet dynamics using cold atoms in appropriately tailored light fields.

\bibitem{keating2012}
Y. V. Fyodorov, G. A. Hiary, and J. P. Keating,
Freezing Transition, Characteristic Polynomials of Random Matrices, and the Riemann Zeta Function,
Phys. Rev. Lett. {\bf 108}, 170601 (2012).
%We argue that the freezing transition scenario, previously explored in the statistical mechanics of 1/f—noise random energy models, also determines the value distribution of the maximum of the modulus of the characteristic polynomials of large N×N random unitary matrices. We postulate that our results extend to the extreme values taken by the Riemann zeta function ζ(s) over sections of the critical line s=1/2+it of constant length and present the results of numerical computations in support. Our main purpose is to draw attention to possible connections between the statistical mechanics of random energy landscapes, random-matrix theory, and the theory of the Riemann zeta function. (---Our focus here is on suggesting a new link with physics via the statistical mechanics of disordered landscapes ---)

\bibitem{keating1995}
E. B. Bogomolny and J. P. Keating,
Random matrix theory and the Riemann zeros. I. Three- and four-point correlations,
Nonlinearity {\bf 8}, 1115 (1995).
%The non-trivial zeros of the Riemann zeta-function have been conjectured to be pairwise distributed like the eigenvalues of matrices in the Gaussian Unitary Ensemble (GUE) of random matrix theory. They therefore behave like the energy levels of quantum systems whose classical limit is strongly chaotic and non-invariant with respect to time-reversal. We show that this analogy extends directly to higher-order statistics. Starting with an explicit formula relating the zeros to the prime numbers (the analogue of the Gutzwiller trace formula of quantum chaology), we demonstrate that the 3-point and 4-point zero correlation functions are asymptotically equivalent to the corresponding GUE results. Our method centres around a Hardy-Littlewood conjecture concerning the distribution of the primes. The calculation generalises a previous study of 2-point correlations and involves the introduction of several new techniques. These will form the basis of a demonstration, to be described in another paper, that the equivalence extends to the general n-point correlation function.

\bibitem{leboeuf2003}
P. Leboeuf and A. G. Monastra,
Quantum thermodynamic fluctuations of a chaotic Fermi-gas model,
Nucl. Phys. A {\bf 724}, 69 (2003).
%We investigate the thermodynamics of a Fermi gas whose single-particle energy levels are given by the complex zeros of the Riemann zeta function. This is a model for a gas, and in particular for an atomic nucleus, with an underlying fully chaotic classical dynamics. The probability distributions of the quantum fluctuations of the grand potential and entropy of the gas are computed as a function of temperature and compared, with good agreement, with general predictions obtained from random matrix theory and periodic orbit theory (based on prime numbers). In each case the universal and non-universal regimes are identified. (--- density of states will depende on the zeta --- writting it as functions of p implies all description em terms of p --- SEMICLASSICAL approach, comparision with GutzTF)

\bibitem{schleich2013}
C. Feiler and W. P. Schleich,
Entanglement and analytical continuation: an intimate relation told by the Riemann zeta function,
N. J. Phys. {\bf 15}, 063009 (2013).
%We propose measurements on a quantum system to realize the Riemann zeta function ζ. A single system, that is classical interference, suffices to create the Dirichlet representation of ζ. In contrast, we need measurements performed on two entangled quantum systems to extend ζ into the critical strip of complex space where the non-trivial zeros of ζ are located. As a consequence, we can view these zeros as a result of a Schrödinger cat which is by its very construction similar to, but in its details very different from, the superposition formed by two coherent states of identical amplitudes but opposite phases. This interpretation suggests that entanglement in quantum mechanics is the analogue of analytic continuation of complex analysis. (--- They mimic zeta function as quantum time-correlation function --- For Dirichlet series, one part is enough --- For Riemann formula, two entangled parties are needed --- Hamiltonian: 2 level atom with a convenient energy eigenvalues----)

\bibitem{guang2020}
Ran He, Ming-Zhong Ai, Jin-Ming Cui, Yun-Feng Huang, Yong-Jian Han, Chuan-Feng Li, Tao Tu, C. E. Creffield, G. Sierra, and Guang-Can Guo,
Identifying the Riemann zeros by periodically driving a single qubit,
Phys. Rev. A {\bf 101}, 043402 (2020).
%The Riemann hypothesis, one of the most important open problems in pure mathematics, implies the most profound secret of prime numbers. One of the most interesting approaches to solving this hypothesis is to connect the problem with the spectrum of the physical Hamiltonian of a quantum system. However, none of the proposed quantum Hamiltonians has been experimentally feasible. Here we report an experiment using a Floquet method to identify the first nontrivial zero of the Riemann ζ function and the first two zeros of Pólya's function. Through properly designed periodically driving functions, the zeros of these functions are characterized by the occurrence of crossings of quasienergies when the dynamics of the system is frozen. The experimentally obtained zeros are in good agreement with their exact values. Our study provides the experimental realization of the Riemann zeros in a quantum system, which may provide insights into the connection between the Riemann function and quantum physics.(---experiment---)

\bibitem{schleich2018}
F. Gleisberg, F. Di Pumpo, G. Wolff, and W. P. Schleich,
Prime factorization of arbitrary integers with a logarithmic energy spectrum,
J. Phys. B: Atom, Mol. and Opt. Phys. {\bf 51}, 035009 (2018)
%We propose an iterative scheme to factor numbers based on the quantum dynamics of an ensemble of interacting bosonic atoms stored in a trap where the single-particle energy spectrum depends logarithmically on the quantum number. When excited by a time-dependent interaction these atoms perform Rabi oscillations between the ground state and an energy state characteristic of the factors. The number to be factored is encoded into the frequency of the sinusoidally modulated interaction. We show that a measurement of the energy of the atoms at a time chosen at random yields the factors with probability one half. We conclude by discussing a protocol to obtain the desired prime factors employing a logarithmic energy spectrum which consists of prime numbers only.(--- Factorization using QM --- )

\bibitem{martin2018}
J. L. Rosales and V. Martin,
Quantum simulation of the integer factorization problem: Bell states in a Penning trap,
Phys. Rev. A {\bf 97}, 032325 (2018).

\bibitem{sierra2015}
J. I. Latorre and G. Sierra, 
There is entanglement in the primes, 
Quant. Info. Comput. {\bf 15}, 622 (2015).

\bibitem{sierra2020}
D. Garc{\'{i}}a-Mart{\'{i}}n, E. Ribas, S. Carrazza, J. I. Latorre, and G. Sierra,
The Prime state and its quantum relatives,
Quantum {\bf 4}, 371 (2020).

\bibitem{gazeau}
J. P. Gazeau, 
{\it Coherent states in quantum physics}
(Wiley, New York, 2009).

\bibitem{spf3}
All amplitudes corresponding to family $f_3$ satisfy $c_n (u) = c_n^{\scriptscriptstyle{(f_3)}} (u) \equiv 
8 \mbox{e}^{-2u}[I_1(u) \, I_{n}(u) + I_3(u) \, I_{\frac{n}{3}}(u)]$.

\bibitem{scully} 
M. O. Scully and M. S. Zubairy, {\em Quantum optics} (Cambridge University Press, U.K., 1997).

\bibitem{yamamoto}
N. Imoto, H. A. Haus, and Y. Yamamoto,
Quantum nondemolition measurement of the photon number via the optical Kerr effect,
Phys. Rev. A {\bf 32}, 2287 (1985).

\bibitem{HomodyneDetect}
H. P. Yuen and V. W. S. Chan,
Noise in homodyne and heterodyne detection,
Opt. Lett. {\bf 83}, 177 (1983).

\bibitem{arlans}
A. J. S. Lara and A. D. Ribeiro,
Entanglement via entangled-boundary-condition trajectories: Long-time accuracy,
Phys. Rev. A {\bf 100}, 042123 (2019)

\bibitem{matheus}
M. V. Scherer and A. D. Ribeiro,
Entanglement dynamics of spins using a few complex trajectories,
Phys. Rev. A {\bf 104}, 042222 (2021).

\end{thebibliography}


\end{document}
