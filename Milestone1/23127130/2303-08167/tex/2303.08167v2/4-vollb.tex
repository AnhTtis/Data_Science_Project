\section{Upper Bound on Hereditary Discrepancy}
% \color{red}
In this section we prove Theorem~\ref{thm:herdisc-ub}. To this end we introduce the volume lower bound on hereditary discrepancy, introduced by Lov\'{a}sz and Vesztergombi~\cite{LV89}, and, in a more general setting, by Banaszczyk~\cite{Bana-vollb}, and studied by Dadush, Nikolov, Talwar, and Tomczak-Jaegermann~\cite{dadush2018balancing}.

Let \(\mm{A}\) be an \(m\times n\) real matrix, and define the symmetric convex set $K_{\mm{A}} \coloneqq \{x \in \RR^n: \norm{\mm{Ax}}_{\infty}\leq 1\}$. Let us define the \emph{volume lower bound of $A$}, denoted $\mathrm{volLB}(A)$, by
\[\mathrm{volLB}(A) = \max_{k \in [n]}\max_{S \subseteq [n], |S| = k}\frac{1}{\mathrm{vol}_k\left(K_{\mm{A}} \cap W_S\right)^{1/k}},\]
where $W_S$ is the canonical subspace in the dimensions indexed by $S$ (i.e.\ $W_S = \mathrm{span}\left\{\mm{e}_i\mbox{, }i \in S\right\}$) and $\mathrm{vol}_k$ is the $k$-dimensional volume within $W_S$, i.e., the Lebesgue measure restricted to this subspace. We also define a dual volume lower bound by 
\[\mathrm{volLB}^*(A) = \max_{k \in [n]}\max_{S \subseteq [n], |S| = k}\frac{\mathrm{vol}_k\left(\mathrm{conv}(\pm \mm{\Pi}_S \mm{a}_1, \ldots \pm \mm{\Pi}_S \mm{a}_m)\right)^{1/k}}{c_k^{2/k}},\]
where $\mm{\Pi}_S$ is the orthogonal projection onto \(W_S\), $\mm{a_i}^\top$ is the \(i\)-th row of \(\mm{A}\), and \(c_k = \frac{\pi^{k/2}}{\Gamma(\frac{k}{2} + 1)}\) is the volume of the \(k\)-dimensional unit Euclidean ball. 

We also need the concept of a polar set of a set \(K \subseteq \RR^n\), defined as 
\[
K^\circ \coloneqq \{\mm{y} \in \RR^n: \mm{y}^\top \mm{x} \le 1 \ \forall \mm{x} \in K\}.
\]
It is a consequence of the hyperplane separator theorem that for any closed convex $K$ containing $0$, $K^{\circ\circ} = K$~\cite[Section 14]{Rockafellar}. Moreover, it is clear from the definition that $K\subseteq L$ implies $L^\circ \subseteq K^\circ$.

We have the following relationship between $\mathrm{volLB}(\mm{A})$ and $\mathrm{volLB}^*(\mm{A})$. 
\begin{claim}\label{claim:vollb-vs-vollb*}
    For any matrix $\mm{A} \in \RR^{m\times n}$, $\mathrm{volLB}(A) \cong \mathrm{volLB}^*(A)$.
\end{claim}
\begin{proof}
    Let $K_{\mm{A}}$ be defined as above, and let $L_{\mm{A}} \coloneqq \mathrm{conv}(\pm \mm{a}_1, \ldots \pm \mm{a}_m)$. We claim that, for any set \(S \subseteq [n]\), \((K_{\mm{A}} \cap W_S)^\circ = \mm{\Pi}_S L_{\mm{A}}\), where the polar \((K_{\mm{A}} \cap W_S)^\circ\) is taken within the subspace \(W_S\). It is sufficient to show this for $S = [n]$, as, otherwise, we can replace the matrix $\mm{A}$ by its submatrix consisting of the columns indexed by $S$. In the case $S = [n]$, we just need to show $K_{\mm{A}}^{\circ} = L_{\mm{A}}$. Notice that
    \begin{align*}
        K_{\mm{A}} &= \{\mm{x} \in \RR^{n}: \norm{\mm{A}\mm{x}}_{\infty} \leq 1\}\\
        &= \{\mm{x} \in \RR^{n}: \anglebrac{\mm{A}\mm{x}, \mm{y}} \leq 1 \mbox{ for all } \norm{\mm{y}}_1 \leq 1\}\\
        &= \{\mm{x} \in \RR^{n}: \anglebrac{\mm{x}, \mm{A}^{\top}\mm{y}} \leq 1 \mbox{ for all } \norm{\mm{y}}_1 \leq 1\}.
    \end{align*}
    By the definition of polar, we see that $K_{\mm{A}} = L_{\mm{A}}^{\circ}$ as 
    \[L_{\mm{A}} = \{\mm{A}^{\top}\mm{y}: \mm{y} \in \mathbb{R}^m\mbox{ where } \norm{\mm{y}}_1 \leq 1\}.\] 
    Thus $K_{\mm{A}}^{\circ} = L_{\mm{A}}^{\circ\circ} = L_{\mm{A}}$ as required. 
    
    Once we have established that \((K_{\mm{A}} \cap W_S)^\circ = \mm{\Pi}_S L_{\mm{A}}\), we have, by the Santal\'o-Blaschke and the reverse Santal\'o inequalities (see Chapters~1~and~8 of~\cite{ASGM15}),
    \[
    \mathrm{vol}_k(K_{\mm{A}} \cap W_S)^{1/k}\mathrm{vol}_k((K_{\mm{A}} \cap W_S)^\circ)^{1/k} \cong c_k^{2/k}.
    \]
    This completes the proof.
\end{proof}
\begin{comment}
    To show that \((K_{\mm{A}} \cap W_S)^\circ = \mm{\Pi}_S L_{\mm{A}}\), we first prove the containment \(\mm{\Pi}_S L_{\mm{A}} \subseteq (K_{\mm{A}} \cap W_S)^\circ\). Observe that \(L_{\mm{A}}\) can be written equivalently as
    \[
    L_{\mm{A}} = \left\{\sum_{i=1}^m \lambda_i \mm{a}_i: \mm{\lambda} \in \RR^m, \|\mm{\lambda}\|_1 \le 1\right\}.
    \]
    Then, for any \(\mm{x} \in K_{\mm{A}} \cap W_S\) and \(\mm{y} \in \mm{\Pi}_S L_{\mm{A}}\), written as \(\mm{y} = \sum_{i=1}^m \lambda_i \mm{\Pi}_S\mm{a}_i = \mm{\Pi}_S \mm{A}^\top \mm{\lambda}\) for \(\|\mm{\lambda}\|_1 \le 1\),
    \[
    \mm{y}^\top \mm{x} = (\mm{\Pi}_S\mm{A}^\top \mm{\lambda})^\top \mm{x}
    = \mm{\lambda}^\top (\mm{A}\mm{\Pi}_S\mm{x}) = \mm{\lambda}^\top (\mm{A}\mm{x})\le \|\mm{A}\mm{x}\|_\infty \le 1.
    \]
    The final equality above uses the fact that \(\mm{\Pi}_S \mm{x} = \mm{x}\) for all \(\mm{x} \in W_S\).
    The first inequality follows by H\"older's inequality and the assumption \(\|\mm{\lambda}\|_1 \le 1\), and the second inequality by the definition of \(K_{\mm{A}}\). Since \(\mm{x} \in K_{\mm{A}} \cap W_S\) and \(\mm{y} \in \mm{\Pi}_S L_{\mm{A}}\) were arbitrary, this shows that \(\mm{\Pi}_S L_{\mm{A}} \subseteq (K_{\mm{A}} \cap W_S)^\circ\).

    Next we prove \((\mm{\Pi}_S L_{\mm{A}})^\circ \subseteq K_{\mm{A}} \cap W_S\), which implies the reverse containment 
    \[
     (K_{\mm{A}} \cap W_S)^{\circ} \subseteq (\mm{\Pi}_S L_{\mm{A}})^{\circ\circ} = \mm{\Pi}_S L_{\mm{A}},
    \]
    since \(\mm{\Pi}_S L_{\mm{A}}\) is clearly closed, convex, and contains the origin. Let then \(\mm{x}\in (\mm{\Pi}_S L_{\mm{A}})^\circ\), with the polar taken inside \(W_S\) i.e., \(\mm{x} \in W_S\), and, for all \(\mm{y} \in \mm{\Pi}_S L_{\mm{A}}\),
    \(
    \mm{y}^\top \mm{x}  \le 1.
    \)
    Let \(i \in [m]\) be such that \(|\mm{a}_i^\top \mm{x}| = \|\mm{A}\mm{x}\|_\infty\), and let \(\sigma \in \{-1,+1\}\) have the same sign as \(\mm{a}_i^\top \mm{x}\). Then, \(\sigma \mm{a}_i \in L_{\mm{A}}\) and
    \[
    \|\mm{A}\mm{x}\|_\infty=
    |\mm{a}_i^\top \mm{x}|
    = \sigma \mm{a}_i^\top\mm{x}
    = \sigma \mm{a}_i^\top(\mm{\Pi}_S\mm{x})
    = (\sigma \mm{\Pi}_S \mm{a}_i)^\top\mm{x}
    \le 1.
    \]
    The penultimate equality follows because \(\mm{x} \in W_S\), so \(\mm{\Pi}_S \mm{x} = \mm{x}\). The inequality is implied by our assumption that \(\mm{x}\in (\mm{\Pi}_S L_{\mm{A}})^\circ\). This shows that \(\mm{x} \in K_{\mm{A}} \cap W_S\) as well, and finishes the proof.
\end{comment}

The next lemma shows a relationship between $\mathrm{volLB}(\mm{A})$ and \(\detlb(\mm{A})\) that, as far as we are aware, has not been observed before. 
\begin{lemma}\label{lem:vollb-vs-detlb}
    For any matrix \(\mm{A} \in \RR^{m \times n}\), \(\mathrm{volLB}^*(\mm{A}) \lesssim \sqrt{n}\cdot\detlb(\mm{A}). \)
\end{lemma}

In the proof of Lemma~\ref{lem:vollb-vs-detlb} we use the following result of Nikolov. Closely related results were shown earlier by Dvoretzky and Rogers~\cite[Theorem 5B]{DR50} and Ball~\cite[Proposition 7]{Ball89}.

\begin{lemma}\textup{(Theorem 10 in~\cite{nikolov2015randomized}).}\label{lem:cite-subdet}
    Let $m \geq n$ and \(E\subseteq \RR^n\) be a minimum volume ellipsoid containing the points \(\pm \mm{a}_1, \ldots, \pm \mm{a}_m \in \RR^n\). There exists a set \(T\subseteq [m]\) of size $n$ such that 
    \[
    |\det((\mm{a}_i)_{i \in T})|
    \ge 
    \sqrt{\frac{n!}{n^n}} \frac{\mathrm{vol}_n(E)}{c_n} 
    \cong n^{1/4} e^{-n/2} \frac{\mathrm{vol}_n(E)}{c_n},
    \]
    where $(\mm{a}_i)_{i \in T}$ is the matrix with columns $\mm{a}_i$ for \(i \in T\), and \(\mathrm{vol}_n\) is the \(n\)-dimensional Lebesgue measure.
\end{lemma}
Note that Theorem 10 in \cite{nikolov2015randomized} in fact shows that there is a distribution on random multisets $T$ for which $\EE|\det((\mm{a}_i)_{i \in T})|^2 = \frac{n!}{n^n} \frac{\mathrm{vol}_n(E)^2}{c_n^2}$. Since the determinant is zero unless $T$ is a set, this implies Lemma~\ref{lem:cite-subdet}.

\begin{proof}[Proof of Lemma~\ref{lem:vollb-vs-detlb}]
    Take some $S\subseteq [n]$ of size $k$ such that 
    \[\mathrm{volLB}^*(A) = \frac{\mathrm{vol}_k(\mm{\Pi}_SL_{\mm{A}})^{1/k}}{c_k^{2/k}},\]
where $\mm{a}_1^\top, \ldots, \mm{a}_m^\top$ are the rows of $\mm{A}$, and $L_{\mm{A}} \coloneqq \mathrm{conv}(\pm \mm{a}_1, \ldots \pm \mm{a}_m)$. Applying Lemma~\ref{lem:cite-subdet} to $\pm \mm{\Pi}_S \mm{a}_1, \ldots, \pm \mm{\Pi}_S \mm{a}_m$, we have that, taking $E \subseteq W_S$ to be the smallest volume ellipsoid containing $\pm \mm{\Pi}_S \mm{a}_1, \ldots, \pm \mm{\Pi}_S \mm{a}_m$,  there exists a set $T \subseteq [m]$ of size $k$ for which 
\[
|\det((\mm{\Pi}_S \mm{a}_i)_{i \in T})|
    \gtrsim
     k^{1/4} e^{-k/2} \frac{\mathrm{vol}_k(E)}{c_k}
    \ge k^{1/4} e^{-k/2} \frac{\mathrm{vol}_k(\mm{\Pi}_S L_{\mm{A}})}{c_k}.
\]
The last inequality follows because $L_{\mm{A}} \subseteq E$. Re-arranging and raising to the power $1/k$, this gives us that 
\[
\mathrm{volLB}^*(A) = \frac{\mathrm{vol}_k(\mm{\Pi}_SL_{\mm{A}})^{1/k}}{c_k^{2/k}}
\lesssim
\frac{|\det((\mm{\Pi}_S \mm{a}_i)_{i \in T})|^{1/k}}{c_k^{1/k}}
\lesssim
\sqrt{k}\cdot\detlb(A),
\]
where, in the final inequality, we used that \((\mm{\Pi}_S \mm{a}_i)_{i \in T}\) is the transpose of a $k$ by $k$ submatrix of $\mm{A}$, and we also used the estimate $c_k^{-1/k}\lesssim \sqrt{k}$, which follows from Stirling's approximation. Since $k\le n$, the result follows.
\end{proof}

We remark in passing that the trivial inequality $\mathrm{vol}_k(E) \ge \mathrm{vol}_k(\mm{\Pi}_S L_{\mm{A}})$ for a $k$-dimensional symmetric convex polytope with $2m$ vertices $\mm{\Pi}_S L_{\mm{A}}$ and an ellipsoid $E$ containing it can be improved to $\mathrm{vol}_k(E) \ge \sqrt{\frac{k}{\log(2m)}}\mathrm{vol}_k(\mm{\Pi}_S L_{\mm{A}})$ when $m$ is small, using, e.g., results of Gluskin~\cite{gluskin1989extremal}. Substituting this inequality in the proof of Lemma~\ref{lem:vollb-vs-detlb} gives the bound 
\(
\mathrm{volLB}^*(\mm{A}) \lesssim \sqrt{\log 2m} \cdot \detlb(\mm{A}).
\)

The final ingredient we need for the proof of Theorem~\ref{thm:herdisc-ub} is an upper bound on the hereditary discrepancy of partial colorings in terms of the volume lower bound, due to Dadush, Nikolov, Tomczak-Jaegermann, and Talwar.
\begin{lemma}\textup{(Lem.8 in~\cite{dadush2018balancing}).}\label{lem:cite-balancing}
There exist universal constants $c \geq 1$ and $\epsilon_0 \in (0, 1)$ such that the following holds. For any closed convex set  $K \subseteq \RR^n$ satisfying \(-K = K\) and 
\[
\min_{k=1}^n\min_{S \subseteq [n]: |S| = k} \mathrm{vol}_k(K \cap W_S) \ge 1,
\]
and any $\mm{y} \in (-1, 1)^n$, there exists an $\mm{x} \in [-1, 1]^n$ with $|\mathrm{fixed}(\mm{x})| \geq \ceil{\epsilon_0n}$ and $\mm{x} - \mm{y} \in cK$, where \(\mathrm{fixed}(\mm{x}) \coloneqq \{i \in [n]: |x_i| = 1\}\).
\end{lemma}

We are now ready to complete the proof. 
\begin{proof}[Proof of Theorem~\ref{thm:herdisc-ub}]
It suffices to show that \(\disc(\mm{A}) \lesssim \sqrt{n}\ \detlb(\mm{A})\), since this implies that for any submatrix \(\mm{B}\) of \(\mm{A}\) with \(k\) columns we also have \[\disc(\mm{B}) \lesssim \sqrt{k}\ \detlb(\mm{B})\le \sqrt{n} \ \detlb(\mm{A}).\]

Using Lemma~\ref{lem:cite-balancing}, we construct a sequence of partial colorings $\mm{x}_0 = \mathbf{0}, \ldots, \mm{x}_T \in \{-1,+1\}^n$, where \(T \lesssim 1+ \log_{1/(1-\epsilon_0)}(n)\), each \(\mm{x}_t \in [0,1]^n\), and 
\begin{equation}\label{eq:partial-ub}
    \|\mm{A}(\mm{x}_t - \mm{x}_{t-1})\|_\infty \lesssim \sqrt{n(1-\epsilon_0)^{t-1}}\ \detlb(\mm{A}).
\end{equation}
To construct \(\mm{x}_1\), we apply Lemma~\ref{lem:cite-balancing} to \(\mm{y} \coloneqq \mathbf{0}\), and \(K \coloneqq \mathrm{volLB}(A)\cdot K_{\mm{A}}\). By the definition of \(\mathrm{volLB}(A)\), this \(K\) satisfies the assumption of the lemma, and we let \(\mm{x}_1\) equal the \(\mm{x}\) guaranteed by the lemma. Since \(\mm{x}_1 \in c K = c\cdot \mathrm{volLB}(A)\cdot K_{\mm{A}}\), by the definition of \(K_{\mm{A}}\) we have that 
\[
\|\mm{A}\mm{x}_1\|_\infty \le c\cdot \mathrm{volLB}(A)
\cong \mathrm{volLB}^*(A) \lesssim \sqrt{n}\ \detlb(\mm{A}),
\]
where the last two inequalities follow, respectively, by Claim~\ref{claim:vollb-vs-vollb*} and by Lemma~\ref{lem:vollb-vs-detlb}. In general, to get the bound \eqref{eq:partial-ub} for \(\mm{x}_t - \mm{x}_{t-1}\) for \(t \ge 2\), we set \(S \coloneqq [n] \setminus \mathrm{fixed}(\mm{x}_{t-1})\), and apply Lemma~\ref{lem:cite-balancing} with \(\mm{y} \coloneqq \mm{\Pi}_S\mm{x}_1\), and \(K \coloneqq \mathrm{volLB}(\mm{A}_S)\cdot K_{\mm{A}_S}\), where \(\mm{A}_S\) is the submatrix of \(\mm{A}\) consisting of the columns indexed by \(S\). If \(\mm{x} \in [-1,+1]^S\) is the partial coloring guaranteed by the lemma, we define \(\mm{x}_t\) by setting its coordinates in \(S\) to equal the corresponding coordinates in \(\mm{x}\), and the remaining coordinates to equal the corresponding coordinates in \(\mm{x}_{t-1}\). It is straightforward to check that \(\mathrm{fixed}(\mm{x}_t) \ge (1-(1-\epsilon_0)^t)n\) and \eqref{eq:partial-ub} hold for all \(t\). Moreover, once \(t \ge T\ge 1 + \log_{1/(1-\epsilon_0)}(n)\), we must have \(\mm{x}_t \in \{-1,+1\}^n\).

Having constructed \(\mm{x}_1, \ldots, \mm{x}_T\), we observe that, by \eqref{eq:partial-ub} and the triangle inequality,
\[
\|\mm{A}\mm{x}\|_\infty 
\lesssim \sqrt{n}\cdot\detlb(\mm{B})\left(1 + \sqrt{(1-\epsilon_0)} + \sqrt{(1-\epsilon_0)^2} + \cdots \right)\cong \sqrt{n}\cdot \detlb(\mm{A}).\]
This completes the proof.
\end{proof}


\color{black}