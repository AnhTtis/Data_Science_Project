\section{Introduction}
Let $X$ be a finite universe of elements, $\mathcal{S}$ a collection of subsets of $X$ called a \emph{set system}, and $\chi: X \rightarrow \{\pm 1\}$ a $\pm 1$ coloring of the elements of $X$. The discrepancy of $\mathcal{S}$ with respect to $\chi$, denoted $\disc(\mathcal{S}, \chi)$, is defined as $\max_{S \in \mathcal{S}}||\chi^{-1}(1)\cap S| - |\chi^{-1}(-1)\cap S||$ i.e.\ the largest difference between the number of elements colored differently in any set $S \in \mathcal{S}$. The \emph{discrepancy of $\mathcal{S}$}, denoted $\disc(\mathcal{S})$, is then $\min_{\chi: X \rightarrow \{\pm 1\}}\disc(\mathcal{S}, \chi)$ i.e.\ among all $\pm 1$ colorings of the elements of $X$, the least unequal we can make the most unequal set of $\mathcal{S}$. If $|X| = n$ and $|\mathcal{S}| = m$, let $X = [n]$ and $\mathcal{S} = \{S_1, ..., S_m\}$. For each set system $\mathcal{S}$, we can construct an associated $m \times n$ incidence matrix $\mm{A}_{\mathcal{S}}$: the entry in row $i$ and column $j$ of $\mm{A}_{\mathcal{S}}$ is equal to one if element $j$ is in $S_i$ and zero otherwise. We define the \emph{discrepancy of a real-valued matrix \(\mm{A}\)} as 
\[\disc(\mm{A}) \coloneqq \min_{\mm{x} \in \{\pm 1\}^{n}} \norm{\mm{A}\mm{x}}_{\infty},\] 
where $\norm{\cdot}_{\infty}$ denotes the $L_\infty$-norm of a vector. Using this definition, we see that $\disc(\mm{A}_{\mathcal{S}}) = \disc(\mathcal{S})$. Throughout this exposition, we will use $\mathcal{S}$ and its indicator matrix $\mm{A}_{\mathcal{S}}$ interchangeably.

We would like discrepancy to be a robust quantity, but it can be sensitive to slight modifications to the incidence matrix e.g. the discrepancy of the matrix $[\mm{A}, \mm{A}]$ is always zero regardless of $\disc(\mm{A})$. \modified{This motivates the definition of the hereditary discrepancy of a matrix $\mm{A}$ as
\[\herdisc(\mm{A}) \coloneqq \max_{\mm{B} \in \mathcal{S}}\disc(\mm{B}),\] 
where $\mathcal{S}$ is the set of all sub-matrices of $\mm{A}$. Notice that adding rows to a matrix can never decrease its discrepancy so it suffices for $\mathcal{S}$ to consist of sub-matrices whose columns are a subset of the {columns} of $\mm{A}$.} In a sense, this definition generalizes total unimodularity. It is easy to show that totally unimodular matrices (TUM)\footnote{A matrix $\mm{A}$ is TUM if every square submatrix of $\mm{A}$ has determinant in $\{-1, 0, 1\}$. A linear systems of the form $\mm{A}\mm{x} \geq \mm{b}$ for TUM $\mm{A}$, integral $\mm{b}$, and $0 \le \mm{x}$ has an integral polyhedron as its feasible region.} have hereditary discrepancy at most one~\cite{schrijver1998theory}. Further, a result of Ghouila-Houri~\cite{ghouila1962caracterisation} states that the set of hereditary discrepancy one matrices with entries in $\{-1, 0, 1\}$ is exactly the set of TUM matrices. 

An important early work in discrepancy theory of Lov\'{a}sz, Spencer, and Vesztergombi~\cite{Lovasz1986discrepancy} showed that the determinant lower bound of $\mm{A}$, \modified{defined as
\[\detlb(\mm{A}) \coloneqq \max_{k}\max_{\mm{B}\in \mathcal{S}_k}\left|\mathrm{det}(\mm{B})\right|^{1/k},\]
where $\mathcal{S}_{k}$ denotes the set of all $k\times k$ sub-matrices of $\mm{A}$}, satisfies $2\herdisc(\mm{A}) \geq \detlb(\mm{A})$. This, again, generalizes what happens with totally unimodular matrices, for which both quantities are equal to one. The determinant lower bound has since become a powerful tool in proving nearly tight lower bounds on many natural and important set systems, e.g. axis-aligned boxes and point-line incidences \cite{chazellelvov2001discrepancy, matousek18factorization}. Given that the determinant lower bound often implies nearly tight discrepancy lower bounds, it is natural to ask how far it can be from the hereditary discrepancy. The first result in this direction is due to Matou\v{s}ek~\cite{matouvsek2013determinant} who showed that the ratio between hereditary discrepancy and the determinant lower bound is bounded from above as
\[\frac{\herdisc(\mm{A})}{\detlb(\mm{A})} \lesssim \log(2mn)\cdot\sqrt{\log(2n)}.\]
Here we used the notation $a \lesssim b$ to denote the the existence of a universal constant $c$ such that $a \leq c\cdot b$. Similarly, $a \gtrsim b$ denotes the existence of a universal constant $c > 0$ such that $a \geq c\cdot b$, and $a\cong b$ when $a\lesssim b$ and $a\gtrsim b$.

Matou\v{s}ek's bound was not believed to be tight as the largest known
value of \(\frac{\herdisc(\mm{A})}{\detlb(\mm{A})}\) is on the order
of \(\log n\). Both the large discrepancy three
permutations family of~Newman, Neiman, and
Nikolov~\cite{newman2012beck}  (see also \cite{franks2018simplified})
and a construction due to P{\'a}lv{\"o}lgyi~\cite{palvolgyi2010indecomposable} achieve this gap.
 Matou\v{s}ek's bound follows from a pair of inequalities
\begin{align}
    \herdisc(\mm{A}) \lesssim \log(2mn)\cdot\hervecdisc(\mm{A})\label{eq:matousek-eq1},\\
    \hervecdisc(\mm{A}) \lesssim \sqrt{\log 2n}\cdot\detlb(\mm{A}),\label{eq:matousek-eq2}
\end{align}
where the first inequality is implied by the seminal result of Bansal~\cite{bansal2010constructive}, and the second inequality is proved using duality. %This was a surprising and important result because it constructed a coloring achieving discrepancy nearly matching Spencer's upper bound when such an algorithm was not thought to exist. Bansal's approach considered the feasibility of a particular SDP, the dual of which Matou\v{s}ek uses to obtain Equation~\ref{eq:matousek-eq2}. 
Note that $\vecdisc$ is the \emph{vector discrepancy} of a set system, or matrix. This quantity is similar to discrepancy but the elements of the universe are ``colored'' by vectors rather than by $\pm 1$. In particular, for an ${m\times n}$ matrix $\mm{A}$, 
\[\vecdisc(\mm{A}) = \min_{\mm{v}_1, \ldots, \mm{v}_n \in S^{n-1}}\max_{j \in [m]}\left\|\sum_{i \in [n]} A_{j,i}\cdot \mm{v}_i\right\|_2,
\]
where $S^{n-1}$ is the unit sphere in $\RR^n$.
Note that vector discrepancy is a lower bound on discrepancy since a coloring $\chi: X \rightarrow \{\pm 1\}$ can be interpreted as a set of vectors where all vectors are parallel to each other.
The \emph{hereditary vector discrepancy of $\mm{A}$}, denoted $\hervecdisc$, which appears in \eqref{eq:matousek-eq1}, is the maximum vector discrepancy of any subset of the columns of $\mm{A}$. %This variant of discrepancy appears in the result of~\cite{bansal2010constructive} mentioned above. 

Recently Jiang and Reis~\cite{jiang2022tighter} were able to improve Matou\v{s}ek's result by showing that $\frac{\herdisc(\mm{A})}{\detlb(\mm{A})} \lesssim \sqrt{\log 2m\log 2n}$. Their work left open whether the \(\sqrt{\log m}\) term can be replaced by \(\sqrt{\log n}\) for large \(m\).
In the present work, we show that this is mostly not possible, and that the factor of $\sqrt{\log m}$ is necessary for all $m$ in the range $n \leq m \leq 2^{n^{1-\epsilon}}$ for any constant $\epsilon > 0$. Note that when $m > 2^n$, the matrix contains duplicated rows whose removal will not change the value of $\herdisc(\mm{A})$ nor $\detlb(\mm{A})$. Thus, our lower bound covers nearly the whole range of values for $m$.

Our main result is stated in the next theorem. Its proof appears at the end of Section~\ref{sec:haarbasis}.
\begin{theorem}\label{thm:main}
For any real number $\epsilon \in (0, 1)$, any integers $n \ge 2$ and $m \in \left[n, 2^{n^{1-\epsilon}}\right]$, there exists a matrix $\mm{A} \in \{0, 1\}^{m\times n}$ such that
\begin{equation}\label{eq:main}
    \frac{\herdisc(\mm{A})}{\detlb(\mm{A})} \gtrsim \sqrt{\log m \log n}.
\end{equation}
\end{theorem}
%The claim after ``further'' shows that the lower bound cannot be extended to $m = 2^{\omega(n/\log n)}$. %\modified{Different techniques extend the upper bound to real matrices $\mm{A} \in \RR^{m\times n}$. See Lemma~\ref{lem:herdiscub-general}.} 

\modified{Note that the lower bound in Theorem~\ref{thm:main} only holds for $m \le 2^{n^{1-\epsilon}}$ for an arbitrarily small but fixed constant $\varepsilon$. This leaves open whether such a lower bound holds all the way to $m = 2^n$. The next theorem gives a new upper bound on $\herdisc(\mm{A})$ in terms of $\detlb(\mm{A})$, which implies that Theorem~\ref{thm:main} cannot be extended to $m = 2^{\omega(n/\log n)}$.
\begin{theorem}\label{thm:herdisc-ub}
    For all positive integers $m$ and $n$, and all matrices $\mm{A} \in \RR^{m\times n}$, we have 
    \[
    \frac{\herdisc(\mm{A})}{\detlb(\mm{A})} \lesssim \sqrt{n}.
    \]
\end{theorem}
This upper bound is based on the relationship between the volume lower bound on discrepancy studied in~\cite{dadush2018balancing}, and the determinant lower bound. In particular, we show that the volume lower bound is bounded by a constant multiple of $\sqrt{n}\cdot \detlb(\mm{A})$, and use a result from~\cite{dadush2018balancing} characterizing the hereditary discrepancy of partial colorings in terms of the volume lower bound. We also give a simpler proof for the special case of \(\mm{A} \in \{0,1\}^{m\times n}\) in the Appendix, using the theory of VC dimension.
}

Crucial to the proof of Theorem~\ref{thm:main} is a recursively
defined $2^{k} \times 2^k$ matrix $\mm{A}_k$, based on the Haar
wavelet basis. We will also show that $\mm{A}_k$ is tight for Equation~\eqref{eq:matousek-eq2}. 
\begin{theorem}\label{thm:tight-matousek-example}
    With $n = 2^k$ for an integer $k\ge 1$, $\hervecdisc(\mm{A}_k) \gtrsim \sqrt{\log n}\cdot\detlb(\mm{A}_k)$.
\end{theorem}
Its proof appears in Section~\ref{sec:othernotableproperties}. It is
yet unknown whether Equation~\eqref{eq:matousek-eq1} is tight as Jiang
and Reis improved upon Matou\v{s}ek's bound by circumventing the
inequality altogether. The resolution of this problem via an efficient
algorithm would imply new and old constructive bounds, for example,
the constructive version of Banaszczyk's upper bound for the
Beck-Fiala problem~\cite{banaszczyk1998balancing,bansal2016algorithm},
and a constructive version of Nikolov's upper bound for Tu\'snady's
problem~\cite{nikolov17tusnady}.

Our use of the matrix $\mm{A}_k$ is inspired by work of
Kunisky~\cite{kunisky2023discrepancy}, \modified{who first used this matrix in the context of proving discrepancy lower bounds}. We also observe that this
matrix gives easier proofs of other known results in discrepancy
theory: see Section~\ref{sec:othernotableproperties}.

Before we begin the proof proper, we define a few variants of discrepancy which will appear throughout this exposition. Just as $\disc(\mm{A})$ was defined in terms of $L_{\infty}$, we can define discrepancy in terms of other norms. In particular, for $L_1$,
\[\disc_1(\mm{A}) \coloneqq \min_{\mm{x} \in \{\pm 1\}^n}\frac{\norm{\mm{A}\mm{x}}_1}{m}\]
and generally for $L_p$,
\[\disc_p(\mm{A}) \coloneqq \min_{\mm{x} \in \{\pm 1\}^n} \left(\frac{\norm{\mm{A}\mm{x}}_p^p}{m}\right)^{1/p}.\]
Note that $\disc_p(\mm{A}) \leq \disc_q(\mm{A})$ when $p \leq q$.
%%% Local Variables:
%%% mode: latex
%%% TeX-master: "main"
%%% End:
