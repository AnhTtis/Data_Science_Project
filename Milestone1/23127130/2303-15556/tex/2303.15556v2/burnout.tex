%%%%%%%%%%%%%%%%%%%%%%%%%%%%%%%%%%%%%%%%%%%%%%%%%%%%%%%%%
%%%%%%%%%%%%%%%%%%%%%%%%%%%%%%%%%%%%%%%%%%%%%%%%%%%%%%%%%
\section{Burnout}\label{sec:burn}
In this section, we show reconfiguration in $2$-burnout with species $(A, B, C)$ and reaction $A + B \rightarrow C + A$ is NP-complete in Theorem \ref{thm:hamPathRed}. Next, we show $1$-reconfiguration in $1$-burnout with $17$ species and $40$ reactions is NP-complete in Theorem \ref{thm:3satburn}.

Reconfiguration and 1-Reconfiguration for burnout sCRNs are in NP since there is the length of any reconfiguration is bounded. For space we do not include this proof but note this has been proved in other system such as Resource Bounded Cellular Automata \cite{dennunzio2017computational}, Freezing Cellular Automata \cite{goles2021complexity} and Freezing Tile Automata \cite{caballero2020verification}.


\subsection{2-Burnout Reconfiguration}
This is a simple reduction from Hamiltonian Path, specifically when we have a stated start and end vertex. %In the reduction species $B$ will represent unvisited vertices and species $A$ will represent the current head of the path. Each cel will participate in two reactions, changing from $B$ to $A$, adding a vertex to the path, and then changing from $A$ to $C$ the final target species.

\begin{figure}[t]
    \centering
    %\captionsetup{justification=centering}
    \includegraphics[width=0.85\textwidth]{images/hamPath.pdf}
    \caption{An example reduction from Hamiltonian Path. We are considering graphs on a grid, so any two adjacent locations are connected in the graph. Left: an initial board with the starting location in blue.  Middle: One step of the reaction. Right: The target configuration with the ending location in blue. Bottom: the single reaction rule.}
    \label{fig:hamPath}
\end{figure}

\begin{theorem}\label{thm:hamPathRed}
Reconfiguration in $2$-burnout sCRNs with species $(A, B, C)$ and reaction $A + B \rightarrow C + A$ is NP-complete even when the surface is a subset of the grid graph. It is also NP-complete with the same species and reactions without the $2$-burnout restriction. 
\label{thm:hamPath}
\end{theorem}
\begin{proof}
Let $\Gamma = \{(A, B, C), (A + B \rightarrow C + A)\}$. Given an instance of the Hamiltonian path problem on a grid graph $H$ with a specified start and target vertex $v_s$ and $v_t$, respectively, create a surface $G$ where each cell in $G$ is a node from $H$. Each cell contains the species $B$ except for the cell representing $v_s$ which contains species $A$. The target surface has species $C$ in every cell except for the final node containing $A$, $v_t$. An example can be seen in Figure \ref{fig:hamPath}.

The species $A$ can be thought of as an agent moving through the graph. The species $B$ represents a vertex that hasn't been visited yet, while the species $C$ represents one that has been. Each reaction moves the agent along the graph, marking the previous vertex as visited. 

($\Rightarrow$) If there exists a Hamiltonian path, then the target configuration is reachable. The sequence of edges in the path can be used as a reaction sequence moving the agent through the graph, changing each cell to species $C$ finishing at the cell representing $v_t$. 

($\Leftarrow$) If the target configuration is reachable, there exists a Hamiltonian path. The sequence of reactions can be used to construct the path that visits each of the vertices exactly once, ending at $v_t$. 

Note that we have not discussed the effect of Burnout on the reduction. However since each cell transitions through species in the following order: $B, A, C$ this reaction always results in a $2$-burnout sCRN so the reduction holds with and without the restriction. 

This means the CRN is $2$-burnout which bounds the max sequence length for reaching any reachable surface, putting the reconfiguration problem in NP. 
\end{proof}

\subsection{1-Burnout 1-Reconfiguration}
For $1$-burnout 1-reconfiguration, we show NP-completeness by reducing from 3SAT and utilizing the fact that once a cell has reacted it is burned out and can no longer participate in later reactions. 


\begin{figure}[ht]
    \centering
    %\captionsetup{justification=centering}
    \includegraphics[scale=0.8]{images/1B_vars_example.pdf}
    \caption{All the possible configurations of two variable gadgets.}
    \label{fig:settingVariables}
\end{figure}

%$\mathcal{O}(1)$

\begin{theorem} \label{thm:3satburn}
    % Schweller's Reduction
    $1$-reconfiguration in $1$-burnout sCRNs with $17$ species and $40$ reactions is NP-complete even when the surface is a subset of the grid graph. It is also NP-complete with the same species and reactions without the $1$-burnout restriction. 
\end{theorem}

\begin{proof}
We reduce from 3SAT. The idea is to have an `agent' species traverse the surface to assign variables and check that the clauses are satisfied by `walking' through each clause. If the agent can traverse the whole surface and mark the final vertex as  `satisfied', there is a variable assignment that satisfies the original 3SAT instance.

\emph{Variable Gadget.} The variable gadget is constructed to allow for a nondeterministic assignment of the variable via the agent walk. At each intersection, the agent `chooses' a path depending on the reaction that occurs. If the agent chooses `true' for a given variable, it will walk up then walk down to the center species. If the agent chooses `false', the agent will walk down then walk up to the center species. From the center species, the agent can only continue following the path it chose until it reaches the next variable gadget. Examples of the agent assigning variables can be seen in Figure \ref{fig:settingVariables}.

\begin{figure}[t]
    \centering
    %\captionsetup{justification=centering}
    \begin{subfigure}[b]{0.4\textwidth}
        \centering
        %\captionsetup{justification=centering}
        \includegraphics[scale=0.8]{images/var_turn.pdf}
        \caption{Successful navigation of an intersection.}
        \label{fig:var_turn}
    \end{subfigure}\hfil
    \begin{subfigure}[b]{0.5\textwidth}
        \centering
        %\captionsetup{justification=centering}
        \includegraphics[scale=0.8]{images/var_turn_block.pdf}
        \caption{Agent stuck due to not following the assignment.}
        \label{fig:var_turn_block}
    \end{subfigure}
    \caption{The assignment `locking' process.}
    \label{fig:var_blocking}
\end{figure}

Each variable assignment is `locked' by way of geometric blocking. When the agent encounters a variable gadget whose variable has already been assigned, the agent must follow that same assignment or it will get `stuck' trying to react with a burnt out vertex. This can be seen in Figure \ref{fig:var_blocking}.


\emph{Initial Configuration.} First, the configuration is constructed with variable gadgets connected in a row, one for each variable in the 3SAT instance. This row of variable gadgets is where the agent will nondeterministically assign values to the variables. Next, a row of variable gadgets, one row for each clause, is placed on top of the assignment row, connected with helper species to fill in the gaps.

For each clause, if a certain variable is present, the center species of the variable gadget reflects its literal value from the clause. For example, if the variable $x1$ in clause $c1$ should be true to satisfy the clause, the variable gadget representing $x1$ in $c1$'s row will contain a $T$ species in the center cell. Lastly, the agent species is placed in the bottom left of the configuration. An example configuration can be seen in Figure \ref{fig:burnoutInput}. 

\begin{figure}[t]
\centering
    %\captionsetup{justification=centering}
    \begin{subfigure}[b]{0.49\textwidth}
    %\captionsetup{justification=centering}
        \includegraphics[width=1.0\linewidth]{images/1B_17_input.pdf} 
        \caption{Example starting configuration.}
    \end{subfigure}\hfill
    \begin{subfigure}[b]{0.49\textwidth}
        %\captionsetup{justification=centering}
        \includegraphics[width=1.0\linewidth]{images/1B_C1.pdf}
        \caption{The surface after evaluating the first clause. }
    \end{subfigure}
    
    \caption{Reduction from 3SAT to 1-burnout 1-reconfiguration. (a) The starting configuration of the surface for the example formula $\varphi=(\lnot x_2 \lor x_3 \lor x_4)\land(\lnot x_1 \lor x_2 \lor x_4)\land(x_1 \lor \lnot x_2 \lor x_3)$. (b) The configuration after evaluating the first clause. A red outline represents the unsatisfied state, and a green outline represents the satisfied state.}
    \label{fig:burnoutInput}
\end{figure}

The agent begins walking and nondeterministically assigns a value to each variable. After assigning every variable, the agent walks right to left. If at an intersection, the agent chooses a different assignment than it did its first pass, the agent becomes `stuck' only being able to react with a burnt out vertex. 

After walking all the way to the left, the first clause can be checked. The agent starts in the unsatisfied state, walking through each variable in the row, left to right. If the current variable assignment at a variable gadget satisfies this clause, the agent changes to the satisfied state and continues walking. If the agent walks through all the variables without becoming satisfied, the computation ends. If the clause was satisfied, the agent continues by walking back, right to left, to begin evaluation of the next clause. If the agent walks all the way to the final vertex with a satisfied state, then the initial variable assignment satisfies all the clauses.

($\Rightarrow$) If there exists a variable assignment that satisfies the 3SAT instance, then the final vertex can be marked with the satisfied state $s$. The agent can only mark the final cell with the satisfied state $s$ if all clauses can be satisfied.

($\Leftarrow$) If the final vertex can be marked with satisfied state $s$, there exists a variable assignment that satisfies the 3SAT instance. The variable assignment that the agent nondeterministically chose can be read and used to satisfy the 3SAT instance.
\end{proof}

\begin{figure}[t]
    \centering
    \includegraphics[width=\textwidth]{images/Combined.pdf}
    \caption{Species identification and transition rules for 1-burnout 1-reconfiguration.}
    \label{fig:burnoutStatesTransitions}
\end{figure}

% \begin{figure}[t]
%     \centering

%     \includegraphics[scale=0.6]{images/1B_17_States.pdf}
%     \caption{Species identification.}
%     \label{fig:burnoutStates}

% \end{figure}
% \begin{figure}
%     \centering
%     \includegraphics[width=.8\textwidth]{images/transition_rules_1B.pdf}
%     \caption{Deterministic and Nondeterministic reaction rules for 1-burnout with 17 species}
%     \label{fig:my_label}
% \end{figure}

%\captionsetup{justification=centering}
    %\begin{subfigure}{.3\textwidth}
  %\end{subfigure}\hfill \captionsetup{justification=centering}
  
  
  %   \begin{subfigure}{.3\textwidth}
  %       \centering
  %       %\captionsetup{justification=centering}
  %       \includegraphics[scale=0.6]{images/1B_detTransitions.pdf}
  %       \caption{Deterministic reactions.}
  %       \label{fig:burnoutDetReactions}
  %   \end{subfigure}
  %   \caption{Species and reactions in the 1-burnout 1-reconfiguration reduction.}
  %   \label{fig:burnoutReactions}