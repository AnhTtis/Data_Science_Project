\section{Introduction}
The ability to engineer molecules to perform complex tasks is an essential goal of molecular programming.  A popular theoretical model for investigating molecular systems and distributed systems is Chemical Reaction Networks (CRNs) \cite{chen2014deterministic, soloveichik2008computation}. The model abstracts chemical reactions to independent rule-based interactions that creates a mathematical framework equivalent \cite{cook2009programmability} to other well-studied models such as Vector Addition Systems \cite{Karp:1969:JCSS} and Petri nets \cite{Petri1962PHD}. CRNs are also interesting for experimental molecular programmers, as examples have been built using DNA strand displacement (DSD) \cite{soloveichik2010dna}.



Abstract Surface Chemical Reaction Networks (sCRNs) were introduced in \cite{parScale} as a way to model chemical reactions that take place on a surface, where the geometry of the surface is used to assist with computation. In this work, the authors gave a possible implementation of the model similar to ideas of spatially organized DNA circuits \cite{muscat2013dna}. This strategy involves DNA strands being anchored to a DNA origami surface. These strands allow for ``species'' to be attached. Fuel complexes are pumped into the system, which perform the reactions. While these reactions are more complex than what has been implemented in current lab work, it shows a route to building these types of networks. 

\subsection{Motivation}
Feed-Forward circuits using DNA hairpins anchored to a DNA origami surface were implemented in \cite{chatterjee2017spatially}. This experiment used a single type of fuel strand. The copies of the fuel strand attached to the hairpins and were able to drive forward the computation. 

A similar model was proposed in \cite{dannenberg2015dna}, which modeled DNA walkers moving along tracks. These tracks have guards that can be opened or closed at the start of computation by including or omitting specific DNA species at the start. DNA walkers have provided interesting implementations such as robots that sort cargo on a surface \cite{thubagere2017cargo}.

A new variant of surface CRNs we introduce is the $k$-burnout model in which cells can switch states at most $k$ time before being stuck in their final state.  This models the practical scenario in which state changes expend some form of limited fuel to induce the state change.  Specific experimental examples of this type of limitation can be seen when species encode ``fire-once" DNA strand replacement reactions on the surface of DNA origami, as is done within the Signal Passing Tile Model~\cite{HSA:2012:PLS}.

% Reachability in CRNs
%(Reconfiguration relates to predicting secondary structure) 


\subsection{Previous Work}
The initial paper on sCRNs \cite{parScale} gave a 1D reversible Turing machine as an example of the computational power of the model. They also provided other interesting constructions such as building dynamic patterns, simulating continuously active Boolean logic circuits, and cellular automata. 
Later work in \cite{progSim} gave a simulator of the model, improved some results of \cite{parScale}, and gave many open problems- some of which we answer here.

 In \cite{swap}, the authors introduce the concept of swap reactions. These are reversible reactions that only ``swap'' the positions of the two species. The authors of \cite{swap} gave a way to build feed-forward circuits using only a constant number of species and reactions. These swap reactions may have a simpler implementation and also have the advantage of the reverse reaction being the same as the forward reaction, which makes it possible to reuse fuel species. 

A similar idea for swap reactions on a surface that has been studied theoretically are friends-and-strangers graphs \cite{defant2021friends}. This model was originally introduced to generalize problems such as the 15 Puzzle and Token Swapping. In the model, there is a location graph containing uniquely labeled tokens and a friends graph with a vertex for every token, and an edge if they are allowed to swap locations when adjacent in the location graph. The token swapping problem can be represented with a complete friends graph, and the 15 puzzle has a grid graph as the location graph and a star as the friends graph (the `empty square' can swap with any other square). Swap sCRNs can be described as multiplicities friends-and-strangers graph \cite{milojevic2022connectivity}, which relax the unique restriction, with the surface grid (in our case the square grid) as the location graph and the allowed reactions forming the edges of the friends graph. %\xxx{todo: this is perhaps too verbose.}


\subsection{Our Contributions}
In this work, we focus on two main problems related to sCRNs. The first is the reconfiguration problem, which asks given two configurations and a set of reactions, can the first configuration be transformed to the second using the set of reactions. The second is the $1$-reconfiguration problem, which asks whether a given cell can ever contain a given species.  Our results are summarized in Table~\ref{tab:results}. The first row of the table comes from the Turing machine simulation in \cite{parScale} although it is not explicitly stated. The size comes from the smallest known universal reversible Turing machine \cite{morita2007universal} (see \cite{woods2009complexity} for a survey on small universal Turing machines.)

We first investigate swap reactions in Section \ref{sec:swap}. We prove both problems are PSPACE-complete using only four species and three swap reactions. For reconfiguration, we show this complexity is tight by showing with three or less species and only swap reactions the problem is in P. 


In Section \ref{sec:burn}, we study a restriction on surface CRNs called $k$-burnout where each species is guaranteed to only transition $k$ times. This is similar to the freezing restriction from Cellular Automata \cite{goles2021complexity,goles2015introducing,theyssier2022freezing} and Tile Automata \cite{chalk2018freezing}. We start with a simple reduction showing reconfiguration is NP-complete in $2$-burnout. This is also of interest since the reduction only uses three species types and a reaction set of size one. For $1$-reconfiguration, we show the problem is also NP-complete in $1$-burnout sCRNs. This reduction uses a constant number of species. 

In Section \ref{sec:1react}, we analyze reconfiguration for all sCRNs that have a reaction set of size one. For the case of only two species, we show for every possible reaction, the problem is solvable in polynomial time. With three species or greater, we show that reconfiguration is NP-complete. The hardness comes from the reduction in burnout sCRNs. 

Finally, in Section \ref{sec:conc}, we conclude the paper by discussing the results as well as many open questions and other possible directions for future research related to surface CRNs.