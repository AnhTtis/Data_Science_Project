\section{Conclusion}\label{sec:conc}
In this paper, we explored the complexity of the configuration problem within natural variations of the surface CRN model.  While general reconfiguration is known to be PSPACE-complete, we showed that it is still PSPACE-complete even with several extreme constraints. We first considered the case where only swap reactions are allowed, and showed reconfiguration is PSPACE-complete with only four species and three distinct reaction types.  We further showed that this is the smallest possible number of species for which the problem is hard by providing a polynomial-time solution for three or fewer species when only using swap reactions.

We next considered surface CRNs with rules other than just swap reactions. First, we considered the burnout version of the reconfiguration problem, and then followed by the normal version with small species counts.  In the case of $2$-burnout, we showed reconfiguration is NP-complete for three species and one reaction type, and $1$-burnout is NP-complete for 17 species with 40 distinct reaction types.  Without burnout, we achieved, as a corollary, that three species, one reaction type is NP-complete while showing that dropping the species count down to two yields a polynomial-time solution.

\subsection{Computing Polynomial Space Functions}
An interpretation of Theorem \ref{thm:oneRecSwap} is that surface Chemical Reactions are capable of computing any function that can be computed in polynomial space. Perhaps the most important PSPACE-Complete is the acceptance problem for polynomial space Turing machines. While there may be a few reduction between these problems, we can may turn any polynomial space Turing machine into a surface CRN such that the robot species swaps with a wire species at a target location. In experiments one can imagine the target location as having a special type of wire species that acts as a reporting, emitting a signal when it reacts with the robot species. The size of the surface is polynomial in the space of the Turing machine since these are all polynomial time reductions. While we do not claim this experiment can be done with such a small number of species, but rather that theoretically more sequence efficient reaction systems which can compute should exists by taking advantage of the surface. 

Our polynomial time algorithms describe experiments with 1, 2, or 3 reactions on surfaces where well studied algorithms for problems such as matching and motion planning may be of use. 

\subsection{Open Problems}
This work introduced new concepts that leaves open a number of directions for future work. While we have fully characterized the complexity of reconfiguration for the swap-only version of the model, the complexity of reconfiguration with general rule types for three species systems remains open if the system uses more than one rule. All of hardness results also use a square grid graph, while our algorithms work on general surfaces. We would like to know if the threshold for hardness can be lowered on more general graphs. In the $1$-burnout variant of the model, we have shown 1-reconfiguration to be NP-complete, but the question of general reconfiguration remains a ``burning'' open question. 