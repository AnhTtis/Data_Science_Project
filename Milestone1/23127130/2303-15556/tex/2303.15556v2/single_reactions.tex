%%%%%%%%%%%%%%%%%%%%%%%%%%%%%%%%%%%%%%%%%%%%%%%%%%%%%%%%%
%%%%%%%%%%%%%%%%%%%%%%%%%%%%%%%%%%%%%%%%%%%%%%%%%%%%%%%%%
\section{Single Reaction}\label{sec:1react}
When limited to a single reaction, we show a complete characterization of the reconfiguration problem. There exists a reaction using $3$ species for which the problem is NP-complete. For all other cases of $1$ reaction, the problem is solvable in polynomial time. 

\subsection{2 Species}
We start with proving reconfiguration is in P when we only have $2$ species and a single reaction.  

\begin{lemma}
Reconfiguration with species $\{A, B\}$ and reaction $A + A \rightarrow A + B$ OR $A + B \rightarrow A + A$ is solvable in polynomial time on any surface. 
\label{lem:sameDiff}
\end{lemma}
\begin{proof}
The reaction $A + B \rightarrow A + A$ is the reverse of the first case. By flipping the target and initial configurations, we can reduce from reconfiguration with $A + B \rightarrow A + A$ to reconfiguration $A + A \rightarrow A + B$.

We now solve the case where we have the reaction $A + A \rightarrow A + B$. 

All cells that start and end with species $B$ can be ignored as they do not need to be changed, and can not participate in any reactions. If there is a cell that contains $B$ in the initial configuration but $A$ in the target, the instance is `no' as $B$ may never become $A$. 

Let any cell that starts in species $A$ but ends in species $B$ be called a \emph{flip} cell, and any species that starts in $A$ and stays in $A$ a \emph{catalyst} cell. 

An instance of reconfiguration with these reactions is solvable if and only if there exists a set of spanning trees, each rooted at a catalyst cell, that contain all the flip cells. Using these trees, we can construct a reaction sequence %as follows: for each non-root leaf of a spanning tree, react with its parent to change the leaf to $B$.
from post-order traversals of each spanning tree, where we have each non-root node react with its parent to change itself to a $B$.
In the other direction, given a reaction sequence, we can construct the spanning trees by pointing each flip cell to the neighbor it reacts with.
\end{proof}

\begin{lemma}
Reconfiguration with species $\{A, B\}$ and reaction $A + A \rightarrow B + B$ is solvable in polynomial time on any surface. 
\label{lem:2Sflip}
\end{lemma}
\begin{proof}
Reconfiguration in this case can be reduced to perfect matching. Create a graph $M$ including a node for each cell in $S$ containing the $A$ species initially and containing $B$ in the target, with edges between nodes of neighboring cells. If $M$ has a perfect matching, then each edge in the matching corresponds to a reaction that changes $A$ to $B$. If the target configuration is reachable, then the reactions form a perfect matching since they include each cell exactly once.
\end{proof}

%\xxx{This follows from the easy swaps lemma. Should we still include this lemma?}

%\begin{lemma}
%Reconfiguration with species $\{A, B\}$ and swap reaction $A + B \rightarrow B + A$ is solvable in polynomial time on any surface.
%\label{lem:1swap}
%\end{lemma}
%\begin{proof}
%The total number of each species may not change. However, every configuration with the same number of $A$s and $B$s is reachable. For each cell $c_i$ currently containing a $B$ that requires an $A$, draw a shortest path $p$ from $c_i$ to the closest cell $c_j$ that contains an $A$ but requires a $B$. We may move the $A$ from $c_j$ to $c_i$ as follows. If $p$ is all $B$ species, then can be done by just swapping the $A$ from $c_j$ along $p$ until it reaches $c_i$. If there exists any $A$ species along $p$, then they must already be in their target location. Let $c_k$ be the first cell after $c_i$ along $p$ containing an $A$. Swap the $A$ from $c_k$ to $c_i$, then repeat this process to move the $A$ from $c_j$ to $c_k$.
%\end{proof}

\begin{theorem}\label{thm:2s1r}
Reconfiguration with $2$ species and $1$ reaction is in P on any surface.
\end{theorem}
\begin{proof}
As we only have two species and a single reaction, we can analyze each of the four cases to show membership in P.
We divide into two cases:
%First, we will consider what happens when a species reacts with itself, then when two different species react. 

\textbf{\boldmath $A + A$:} When a species reacts with itself, it can either change both species, which is shown to be in P by Lemma~\ref{lem:2Sflip}; or it changes only one of the species, which is in P by Lemma~\ref{lem:sameDiff}. 

\textbf{\boldmath $A + B$:} When two different species react, they can either change to the same species, which is in P by Lemma~\ref{lem:sameDiff}; or they can both change, which is a swap and thus is in P by Theorem~\ref{thm:easySwaps}.
\end{proof}


\subsection{3 or more Species}
Moving up to $3$ species and $1$ reaction, we showed earlier that there exists a reaction for which reconfiguration is NP-complete in Theorem~\ref{thm:hamPathRed}.  
Here, we give reactions for which reconfiguration between $3$ species is in P, and in Corollary~\ref{cor:s3r1} we prove that all remaining reactions are isomorphic to one of the reactions we've analyzed.

\begin{lemma}\label{lem:s3Match}
Reconfiguration with species $(A, B, C)$ and reaction $A + B \rightarrow C + C$ is solvable in polynomial time on any surface.
\end{lemma}
\begin{proof}
    At a high level, we create a new graph of all the cells that must change to species $C$, and add an edge when the two cells can react with each other. Since a reaction changes both cells to $C$ we can think of the reaction as ``covering'' the two reacting cells. Finding a perfect matching in this new graph will give a set of edges along which to perform the reactions to reach the target configuration. 

    Consider a surface $G$ and a subgraph $G' \subseteq G$ where we include a vertex $v'$ in $G'$ for each cell that contain $A$ or $B$ in the initial configuration and $C$ in the target configuration. We include an edge $(u',v')$ between any vertices in $G'$ that contain different initial species, i.e. any pair of cell which one initially contains $A$ and the other initially $B$. 

    Reconfiguration is possible if and only if there is a perfect matching in $G'$. If there is a perfect matching then there exists a set of edges which cover each cell once. Since $G'$ represents the cells that must change states, and the edges between them are reactions, the covering can be used as a sequence of pairs of cells to react. If there is a sequence of reactions then there exists a perfect matching in $G'$: each cell only reacts once so the matching must be perfect, and the cells that react have edges between them in $G'$.
\end{proof}

\begin{lemma}\label{lem:s3r1catalyst}
Reconfiguration with species $(A, B, C)$ and reaction $A + B \rightarrow A + C$ is solvable in polynomial time on any surface.
\end{lemma}
\begin{proof}
    The instance of reconfiguration is solvable if and only if any cell that ends with species $C$ either contained $C$ in the initial configuration, or started with species $B$ and have an $A$ adjacent to perform the reaction. Additionally, since a reaction cannot cause a cell to change to $A$ or $B$, each cell with an $A$ or $B$ in the target configuration must contain the same species in the initial configuration.
\end{proof}

The final case we study is $4$ species $1$ reaction. Any sCRN with $5$ or more species and $1$ reaction has a species which is not included in the reaction. 

\begin{lemma}\label{lem:4s1r}
Reconfiguration with species $(A, B, C, D)$ and the reaction $A + B \rightarrow C + D$ is in P on any surface.
\end{lemma}
\begin{proof}
    We can reduce Reconfiguration with $A + B \rightarrow C + D$ to perfect matching similar to Lemma \ref{lem:s3Match}. Create a new graph with each vertex representing a cell in the surface that must change species. Add an edge between each pair of neighboring cells that can react (between one containing $A$ and the other $B$). A perfect matching then corresponds to a sequence of reactions that changes each of the species in each cell to $C$ or $D$. 
\end{proof}

\begin{corollary}\label{cor:s3r1}
Reconfiguration with $3$ or greater species and $1$ reaction is NP-complete on any surface. 
\end{corollary}
\begin{proof}
    First, from Theorem \ref{thm:hamPathRed} we see that there exists a case of reconfiguration with $3$ species that is NP-hard with or without the burnout restriction. 

    For membership in NP, we analyze each possible reaction. We note that we only need to consider two cases for the left hand side of the rule, $A + A$ and $A + B$. Any other reaction is isomorphic to one of this form as we can relabel the species. For example, rule $B + C \rightarrow A + A$ can be relabeled as $A + B \rightarrow C + C$. Also, we know that $C$ must appear somewhere in the right hand side of the rule. If it does not then the reaction only takes place between two species, which is always polynomial time as shown above, or it involves a species we can relabel as $C$.

    Here are the cases for $A + B$ and our analysis results:

    \begin{table}[htp]
    \centering
    \begin{tabular}{|c|c|}
    \hline
       $A + B \rightarrow A + C $  & P in Lemma \ref{lem:s3r1catalyst} \\ \hline
       $A + B \rightarrow C + B $  & P in Lemma \ref{lem:s3r1catalyst} under isomorphism \\ \hline
       $A + B \rightarrow C + A $  & NP in Theorem \ref{thm:hamPathRed} \\ \hline
       $A + B \rightarrow B + C $  & NP in Theorem \ref{thm:hamPathRed} under isomorphism \\ \hline
       $A + B \rightarrow C + C $  & P in Lemma \ref{lem:s3Match} \\ \hline
       $A + B \rightarrow C + D $  & P in Lemma \ref{lem:4s1r} \\ \hline
    \end{tabular}
    \end{table}

    When we have $A + A$ on the left side of the rule, the only case we must consider is $A + A \rightarrow B + C$ (since all $3$ species must be included in the rule). We have already solved this reaction: first swap the labels of $A$ and $C$ giving rule $C + C \rightarrow B + A$, then reverse the rule to $B + A \rightarrow C + C$ and swap the initial and target configuration. Finally since rules do not care about orientation this is equivalent to the rule $A + B \rightarrow C + C$ in Lemma \ref{lem:s3Match}.

    Finally, for $4$ species and greater, the only new case is $A + B \rightarrow C + D$, which is proven to be in P in Lemma \ref{lem:4s1r}. Any other case would have species that are not used since a rule can only have $4$ different species in it. 

    Thus, all cases are either in NP, or in P which is a subset of NP, therefore, the problem is in NP. Also, since our results for each case apply for any surface, the same is true in general.
\end{proof}

