

----------- Overall evaluation -----------
SCORE: 5 (accept)
----------- Review -----------
Complexity of Reconfiguration in Surface Chemical Reaction Networks

===================Summary of Manuscript===================
In this paper, the authors show that there are a variety of cases where it is computationally difficult to predict the behaviour of a given surface chemical reaction networks (sCRN).  sCRNs are inspired by the variety of experimental results where DNA strands are anchored on an origami surface.  These strands then allow for "species" strands to attach which can then interact with neighbouring anchored strands and their species.  This approach has been used to implement feed-forward circuits and DNA walkers. 

sCRNs abstractly model anchored strands as a node in a graph and there exists an edge between two nodes if they are close enough that their "species" strands can interact.  The configuration of "species" strands on an anchored strand is represented by the state of the node.  The nodes can then change state depending on their state and the states of their neighbours.

This manuscript concerns itself with the following two problems. 1) The reconfiguration problem: given two configurations and a set of reactions, can the first be transformed into the second with the given reactions and 2) 1-reconfiguration: will a given cell ever contain a given species.  The authors study these problems under a variety of scenarios where they vary the underlying graph geometry, the number of species, the number of rules, and under a variety of restrictions for the rules.  There are two types of rule restrictions the authors study: 1) restricting rules to be of the form A + B -> B + A and 2) restricting the number of state changes a node can undergo.  When an sCRN only contains rules which are of the form A + B -> B + A, they call it a swap sCRN.  When the number of state changes a node can undergo is restricted to be no more than k, it is called a k-burnout sCRN. 

Table 1 on page 4 shows a comprehensive table outlining the results of varying all these conditions.  Some variations are shown to be PSPACE-complete while others are shown to be NP-complete.  The authors also make some of these complexity results tight by showing that any reduction in the number of species or rules results in the problem being in P. 

For the PSPACE-complete results the authors use the framework of ref [11] which involves reducing to a one player game where the goal is to navigate a robot through a system of gadgets (with states) to reach a goal location.  The NP-complete results involve standard reductions to 3-SAT and the Hamiltonian path problem.

===================Evaluation of Manuscript================
Overall, this paper is well-written, and the results are of interest to the DNA computing community.  These results show that it is difficult to predict the final configuration of a DNA origami which contain DNA walkers or similar devices.  These results also hint at the computational power of these systems and suggest that, for most scenarios, they are computationally interesting.  The authors explicitly study scenarios which are relevant to the experimental community such as the case where an anchored strand can only change state (i.e. a new "species" strand attaches to the anchored strand) a fixed number of times (which is motivated by the fact these reactions consume some fuel), and they also study the experimentally interesting case where reactions must be reversible (thus requiring no fuel).  As these systems are becoming more popular experimentally, it would be very beneficial for the authors to present this work at the DNA conference so that experimentalist have the c
hance to see the elegant theoretical framework which exists for these systems.

The proofs are interesting especially the proofs regarding PSPACE-completeness.  The authors results are very comprehensive and show scenarios where subtracting a single species or rule moves the problem from being PSPACE-complete (intuitively very very hard to solve) to P (intuitively a problem which can be solved easily).  These proofs may be relevant for other similar models for DNA systems outside of sCRNs.



----------------------- REVIEW 2 --------------------

----------- Overall evaluation -----------
SCORE: 4 (weak accept)
----------- Review -----------
The authors analyze various reconfiguration problems related to surface CRNs. It seems interesting but I am not very familiar with this subfield. I have a few details to point out:

1. A couple of the figures (2 and 6) are not mentioned in the main text, which is not great for readability.

[TG - added reference. Needs updated text in figure. Maybe just remove? ]

2. In figure 5, the last two frames of the traversal are the same. This doesn’t seem correct, unless I’m not understanding something.

[TG - Fixed]

3. In figure 7, the T and F states should be explicitly defined as true and false in the legend, as well as the third `turn’ symbol.

[EG - labeled in later figure explicitly. Don't see how it helps here.]

----------- Overall evaluation -----------
SCORE: 4 (weak accept)
----------- Review -----------
This paper characterizes the computational complexity of two fundamental questions related to surface-based CRNs (sCRNs). First, the (2-)reconfiguration problem: Can one given configuration be reached from another, according to the reactions of the CRN? And second, the 1-reconfiguration problem: Can a particular cell on the surface contain a particular species? These questions are studied for three model variants that have been proposed in the literature, namely general sCRNs, sCRNs limited to swap reactions, and "k-burnout" sCRNs where the species at a cell may change only k times, for some small k. The questions are examined both on 2D surface grids and 1D grids. Further variants, where the number of CRN species or reactions are bounded, are also studied.

The authors show, for example, that the reconfiguration problem is PSPACE-complete, even for 1D grids, and also for swap CRns on 2D grids when there are only 4 species and 3 reactions. The complexity reduces to NP-complete for 2-burnout and 1-burnout sCRNs, and is in P if the number of species and/or reactions is sufficiently small.

The results are valuable in that they help us understand the computational power of the different model variants. The results use reductions from reachability problems in motion planning, and seem fairly straightforward, though it's nice that the authors obtain tight results, in that reducing the number of species/reactions further reduces the complexity. They are also fairly thorough in their coverage of model variants and species/reaction count restrictions.

I think that this work should be published, but is perhaps of less compelling interest to the DNA audience than other submissions.

Suggestions for improvement:

Can you discuss the implications of your results for sCRN implementations? The fact that the models remain powerful when the number of species and/or rules is small seems nice, but what is the trade-off with respect to the size of the grid needed?


Be clear in sections 1 (including the table) and 2 when you are assuming that all reactions in your models are reversible.

In Figure 1, there don't seem to be species assigned to the cells of the surface. Also, it's odd that the blue species doesn't have an associated letter, unlike the other species. The phrase "CRN system" is used in the caption. Perhaps "system" should be removed.

Are there two steps in Figure 2, rather than one step? Both the words "move" and "step" are used in the figure, are both words needed?
