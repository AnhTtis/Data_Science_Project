We would like to thank the reviewers and committee for their comments. We will make the corrections suggested. We appreciate the pointers to previous work and will add additional citations. 

Reviewer 1 notes that it may be possible to get hardness results for smaller species sets with a more complex graph. This is a great idea and should be considered in future work. We chose subsets of the square grid graph in order to stay close to models like CA and Tile Self-Assembly. Also square grids or planar graphs in general may be easier to implement when constructing a surface using techniques such as lithography or simple DNA origami. While 3D graphs might be constructed using more advanced DNA origami techniques implementation will be much more difficult. We plan to add of the motivation behind what graphs to select to the paper.  

Reviewer 2 notes Surface CRNs and Async. Cellular Automata are almost equivalent. This is true but we would like to point out two ways in which they differ, 
- Orientation: Each "cell" does not have a sense of direction so rules cannot be applied in a fixed way.
- Updating: In ACA only one cell may change state in a time step. While the models may be equivalent, the simulation is non-trivial and causes an increase in the number of reactions and species.
We will add more details of the differences between the models and add relevant citations on CA and Graph Grammars. 
