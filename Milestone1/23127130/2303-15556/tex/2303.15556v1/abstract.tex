\begin{abstract}

We analyze the computational complexity of basic reconfiguration problems for the recently introduced surface Chemical Reaction Networks (sCRNs), where ordered pairs of adjacent species nondeterministically transform into a different ordered pair of species according to a predefined set of allowed transition rules (chemical reactions).
In particular, two questions that are fundamental to the simulation of sCRNs are whether a given configuration of molecules can ever transform into another given configuration, and whether a given cell can ever contain a given species, given a set of transition rules. 
We show that these problems can be solved in polynomial time,
are NP-complete, or are PSPACE-complete in a variety of different settings, including when adjacent species just swap instead of arbitrary transformation (swap sCRNs), and when cells can change species a limited number of times ($k$-burnout). Most problems turn out to be at least NP-hard except with very few distinct species (2 or 3).
 

\end{abstract}


