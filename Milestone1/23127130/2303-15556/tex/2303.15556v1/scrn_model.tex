%%%%%%%%%%%%%%%%%%%%%%%%%%%%%%%%%%%%%%%%%%%%%%%%%%%%%%%
\section{Surface CRN model}

\textbf{Chemical Reaction Network.}
A \emph{chemical reaction network (CRN)} is a pair $\Gamma = (S, R)$ where $S$ is a set of species and $R$ is a set of reactions, each of the form $A_1 + \cdots + A_j \to B_1 + \cdots + B_k$ where $A_i,B_i \in S$.
(We do not define the dynamics of general CRNs, as we do not need them here.)

\textbf{Surface, Cell, and Species.}
A \emph{surface} for a CRN $\Gamma$ is an (infinite) undirected graph $G$.
The vertices of the surface are called \emph{cells}.
A \emph{configuration} is a mapping from each cell to a species from the set $S$. 
While our algorithmic results apply to general surfaces,
our hardness constructions assume the practical case where
$G$ is a grid graph, i.e., an induced subgraph of the infinite square grid
(where omitted vertices naturally correspond to cells without any species).
%Let $G(x, y)$ be the species at position $(x,y)$.
%
When $G$ is an infinite graph, we assume there is some periodic pattern of cells that is repeated on the edges of the surface. 
Figure \ref{fig:scrnModel} shown an example set of species and reactions and a configuration of a surface. 

\textbf{Reaction.}
A \emph{surface Chemical Reaction Network (sCRN)} consists of a surface and a CRN, where every
\emph{reaction} is of the form $A + B \rightarrow C + D$ denoting that, when $A$ and $B$ are in neighboring cells, they can be replaced with $C$ and $D$. $A$ is replaced with $C$ and $B$ with~$D$. 

\textbf{Reachable Surfaces.}
For two surfaces $I, T$, we write $I \rightarrow^1_\Gamma T$ if there exists a $r \in R$ such that performing reaction $r$ on $I$ yields the surface $T$. Let $I \rightarrow_\Gamma T$ be the transitive closure of $I \rightarrow^1_\Gamma T$, including loops from each surface to itself. Let $\Pi(\Gamma, I)$ be the set of all surfaces $T$ for which $I \rightarrow_\Gamma T$ is true. 


%\textbf{Sink Surfaces.}
%A surface $S$ is a sink if, for all pairs of neighboring cells, they cannot react---i.e., there are no reachable configurations from $I$. Let $\Psi(\Gamma, I)$ be the set of sink configurations reachable from $I$.

% \begin{figure}[h]
%     \centering
%     \captionsetup{justification=centering}
%     \begin{subfigure}{.45\textwidth}
%         \centering
%         %\captionsetup{justification=centering}
%         \includegraphics{images/scrn_model.pdf}
%         \caption{}
%         \label{fig:scrnModel}
%     \end{subfigure}
%     \begin{subfigure}{.45\textwidth}
%         \centering
%         %\captionsetup{justification=centering}
%         \includegraphics{images/reconfig_example.pdf}
%         \caption{}
%         \label{fig:signalPassed1B}
%     \end{subfigure}
%     \caption{(a) Example sCRN system. (b) A reconfiguration problem example. }
% \end{figure}

\begin{figure}[t]
    \centering
    \captionsetup{justification=centering}
    \includegraphics[width=0.6\textwidth]{images/scrn_model.pdf}
    \caption{Example sCRN system.}
    \label{fig:scrnModel}
\end{figure}

\begin{figure}
    \centering
    \includegraphics[width=0.7\textwidth]{images/target_mid_initial_config.pdf}
    \caption{An initial, single step, and target configurations}
    \label{fig:configs_example}
\end{figure}
\subsection{Restrictions}

\textbf{Reversible Reactions.}
A set of reactions $R$ is \emph{reversible} if, for every rule $A + B \rightarrow C + D$ in $R$, the reaction $C + D \rightarrow A + B$ is also in $R$. We may also denote this as a single reversible reaction $A + B \rightleftharpoons C + D$

%\textbf{Catalytic Reactions.}
%A catalytic reaction is of the form $A + B \rightarrow A + C$, i.e., only one species is allowed to change species.\footnote{In Tile Automata, this concept is referred to as ``single transitions.''}

\textbf{Swap Reactions.}
A reaction of the form $A + B \rightleftharpoons B + A$ is called a \emph{swap reaction}.

%\textbf{Catalytic-Swap Reactions.}
%A reaction of the form $A + B \rightarrow C + A$ is a catalytic swap since the species swap and change the non catalyst. 
%\xxx{This next line looks out of place, should it go under the definition of Sink Surface? Also, configuraiton is not defined, should we say `surface' instead?}  

\textbf{\boldmath $k$-Burnout.} 
In the $k$-burnout variant of the model, each vertex of the system's graph can only switch states at most $k$ times (before ``burning out" and being stuck in it's final state).

%Let the species graph of a set of reactions be the graph where each node is a species. There is an edge between two species $A$ and $B$ if there exists a reaction that transform $A$ to $B$. A set of species and reactions is $k$-burnout if the maximum path length in the species graph is $k$. 

%\textbf{Blank-Seed Configuration}
%A configuration is a Blank-Seed if each of the following is true, 

%\begin{itemize}
%    \item is a $m \times n$ grid graph.
%    \item there is a single cell containing a species $s$.
%    \item the rest of the cells contain the same species $b$.
%    \item there is no reaction of the form $b + b \rightarrow ? + ?$.
%\end{itemize}

%\subsection{Relaxations}

%\textbf{Orientation.} 
%We refer to \emph{partial orientation} reactions as having a direction $d \in \{v, h\}$ where $A + B$ must be vertical neighbors, or horizontal neighbors in a square grid graph. A \emph{full orientation} reaction allows the edges to be directed as well. We note that full orientation reactions are equivalent to transition rules from Tile Automata.  

%\textbf{Non-Uniform Surface.}
%We may generalize the model to have multiple cell types. Let $\Lambda$ be a set of cell types. A cell has a fixed cell type $\alpha$. A reaction when considering multiple cell types is of the form $(A, \alpha) + (B, \beta) \rightarrow C + D$. 

%\textbf{Edge-Oriented.}
%An edge-oriented sCRN allows the surface $G$ to be directed. Reactions are now ordered where $A + B$ can be replaced by $C + D$ only if the edge they share is directed from $A$ to $B$. 

\subsection{Problems}

\textbf{Reconfiguration Problem.}
Given a sCRN $\Gamma$ and two surfaces $I$ and $T$, is $T \in \Pi(\Gamma, S)$?

\textbf{\boldmath $1$-Reconfiguration Problem.} 
Given a sCRN $\Gamma$, a surface $I$, a vertex $v$, and a species $s$, does there exist a $T \in \Pi(\Gamma, S)$ such that $T$ has species $s$ at vertex $v$?

%\textbf{$q$-Reconfiguration Problem.}
%Given a sCRN $\Gamma$, a surface $I$, and a surface $t$ such that  $|t| = q$, does there exist a $T \in \Pi(\Gamma, S)$ such that $t \subset T$?

%\textbf{Unique Sink Problem.}
%Given a sCRN $\Gamma$ and two surfaces $I$ and $T$, is $T$ the only reachable sink configuration from $I$, and for all $J$ such that $I \rightarrow_\Gamma J$, $J \rightarrow T$. This problem asks for all reaction sequences of sufficient length is the surface $T$ always reached. 

%\subsection{Simulation}
%We define simulation for sCRNs on a square grid graph. 

%\textbf{Macrocell Mapping.}
%An $m\times n$ macrocell is a subset of a surface arranged in an $m \times n$ rectangle. For a set of species $S$ let $C^{m,n}_S$ be the set of all $m \times n$ macroblocks over the species in $S$. Let $\mu_{m, n}(c)\rightarrow s$ be a mapping from all $c \in C^{m,n}_S$ to a species from a second set $s \in S'$. 

%\textbf{Surface Mapping.}
%We say a surface $X$ is an image of surface $Y$ under a mapping function $\mu_{m,n}$ if for each cell at position $(x,y)$ in $Y$, the macroblock consisting of cells $(x', y')$ in $X$ for $xm \leq x' < (x+1)m$, $yn \leq y' < (y+1)n$. We denote this $X \approx_\mu Y$.


%\textbf{Strong Simulation.}

%A sCRN $\Gamma$ simulates a sCRN $\Gamma'$ if for all surfaces $X_\Gamma, Y_{\Gamma'}$ such that $X \approx_\mu Y$, 