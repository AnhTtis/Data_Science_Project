%%%%%%%%%%%%%%%%%%%%%%%%%%%%%%%%%%%%%%%%%%%%%%%%%%%%%%%%%
%%%%%%%%%%%%%%%%%%%%%%%%%%%%%%%%%%%%%%%%%%%%%%%%%%%%%%%%%
\section{Swap Reactions}\label{sec:swap}
In this section, we will show $1$-reconfiguration and reconfiguration with swap reactions is PSPACE-complete with only $4$ species and $3$ swaps in Theorems \ref{thm:oneRecSwap} and \ref{thm:swapReduc}. We continue by showing that this complexity is tight, that is, reconfiguration with $3$ species and swap reactions is tractable in Theorems \ref{thm:easySwaps} and \ref{thm:1recEzSwap}. 

\subsection{Reconfiguration is PSPACE-complete}
We prove PSPACE-completeness by reducing from the motion planning through gadgets framework introduced in \cite{demaine2018computational}. This is a one player game where the goal is to navigate a robot through a system of gadgets to reach a goal location. The problem of changing the state of the entire system to a desired state has been shown to be PSPACE-complete \cite{ani2022traversability}. This reduction treats the model as a game where the player must perform reactions moving a robot species through the surface.

\subsubsection*{The Gadgets Framework}

\textbf{Framework.} A gadget is a finite set of locations and a finite set of states. Each state is a directed graph on the locations of the gadgets, describing the \emph{traversals} of the gadget. An example can be seen in Figure~\ref{fig:L2T}. Each edge (traversal) describes a move the robot can take in the gadget and what state the gadget ends up in if the robot takes that traversal. A robot enters from the start of the edge and leaves at the exit. 

In a \emph{system} of gadgets there are multiple gadgets connected by their locations.  The \emph{configuration} of a system of gadgets is the state of all gadgets in the system. There is a single robot that starts at a specified location. The robot is allowed to move between connected locations and allowed to move along traversals within gadgets. The system of gadgets can also be restricted to be planar, in which case the cyclic order of the locations on the gadgets is fixed, and the gadgets along with their connections must be embeddable in the plane without crossings.

The \emph{1-player motion planning reachability problem} asks whether there exists a sequence of moves within a system of gadgets which takes the robot from its initial location to a target location. The \emph{1-player motion planning reconfiguration problem} asks whether there exists a sequence of moves which brings the configuration of a system of gadgets to some target configuration.

It may seem strange to reduce from a 1-player game when sCRNs are typically thought of as 0-player simulations; however, this is exactly the correct analogue. In the sCRN model there are many reactions that may occur and we are asking whether there exists a sequence of reactions which reaches some target configuration; in the same way 1-player motion planning asks if there exists a sequence of moves which takes the robot to the target location. The existential query of possible moves/swaps remains the same regardless of whether a player is making decisions vs them occurring by natural processes. The complexity of the gadgets used here are considered in the 0-player setting in \cite{demaine2022pspace}.

\begin{figure}[t]
    \centering
    %\begin{subfigure}{0.4\textwidth}
        \includegraphics[width=0.5\linewidth]{images/locking2toggle.pdf}
    %    \caption{Locking 2-toggle gadget states.}
    %    \label{fig:L2T}    
    %\end{subfigure}
    %\begin{subfigure}{0.3\textwidth}
    %    \centering
    %    \includegraphics[width=0.8\textwidth]{images/swapDiag.pdf}
    %    \caption{Swap rules and state names}
    %    \label{fig:swapDiag}
    %\end{subfigure}
    \caption{The Locking 2-Toggle (L2T) gadget and its states from the motion planning framework. The numbers above indicate the state and when a traversal happens across the arrows, the gadget changes to the indicated state.}
    \label{fig:L2T}    
\end{figure}

\textbf{Locking 2-Toggle.} The Locking $2$-toggle (L2T) is a $4$ location, $3$ state gadget. The states of the gadget are shown in Figure \ref{fig:L2T}. 

%\begin{figure}
%    \centering
%    \includegraphics[width=0.35\textwidth]{images/swapDiag.pdf}
%    \caption{Swap rules and state names}
%    \label{fig:swapDiag}
%\end{figure}



\subsubsection*{Constructing the L2T}

%\xxx{Make sure use of robot and agent are consistent}

We will show how to simulate the L2T in a swap sCRN system. Planar 1-player motion planning with the L2T was shown to be PSPACE-complete \cite{demaine2018computational}. We now describe this construction.

\textbf{Species.} We utilize $4$ species types in this reduction and we name each of them according to their role. First we have the \emph{wire}. The wire is used to create the connection graph between gadgets and can only swap with the robot species. The \emph{robot} species is what moves between gadgets by swapping with the wire and represents the robot in the framework. Each gadget initially contains $2$ robot species, and there is one species that starts at the initial location of the robot in the system. The robot can also swap with the key species. Each gadget has exactly $1$ \emph{key} species. The key species is what performs the traversal of the gadget by swapping with the lock species. The \emph{lock} species can only swap with the key. There are $4$ locks in each gadget. The locks ensure that only legal traversals are possible by the robot species. 



These species are arranged into gadgets consisting of two length-$5$ horizontal tunnels. The two tunnels are connected by a length-$3$ central vertical tunnel at their $3$rd cell. At the $4$th cell of both tunnels there is an additional degree $1$ cell connected we will call the holding cell. 

\begin{figure}
    \centering
    %\raisebox{0.35in}{\begin{subfigure}{0.25\textwidth}
    \begin{subfigure}[b]{0.31\textwidth}
        \centering
        \includegraphics[scale=0.6]{images/swapDiag.pdf}
        \caption{Swap rules/species}
        \label{fig:swapDiag}
    \end{subfigure}%}\hfill
    \begin{subfigure}[b]{0.2\textwidth}
        \includegraphics[scale=0.74]{images/L2Tstate2.pdf}
        \caption{State 1}
        \label{fig:gadgetS1}
    \end{subfigure}\hfill
    \begin{subfigure}[b]{0.2\textwidth}
        \includegraphics[scale=0.74]{images/L2Tstate1.pdf}
        \caption{State 2}
        \label{fig:gadgetS2}
    \end{subfigure}\hfill
    \begin{subfigure}[b]{0.2\textwidth}
        \includegraphics[scale=0.74]{images/L2Tstate3.pdf}
        \caption{State 3}
        \label{fig:gadgetS3}
    \end{subfigure}
    \caption{Locking 2-toggle implemented by swap rules. (a) The swap rules and species names. (b-d) The three states of the locking 2-toggle.}
    \label{fig:L2Tgadget}
\end{figure}

\textbf{States and Traversals.} The states of the gadget we build are represented by the location of the key species in each gadget. If the key is in the central tunnel of the gadget then we are in state $1$ as shown in Figure \ref{fig:gadgetS1}. Note that in this state the key may swap with the adjacent locks, however we consider these configurations to also be in state 1 and take advantage of this later. The horizontal tunnels of the gadget in this state contain a single lock with an adjacent robot species.

States $2$ and $3$ are reflections of each other (Figures \ref{fig:gadgetS2} and \ref{fig:gadgetS3}). This state has a robot in the central tunnel and the key in the respective holding cell. The gadget in this state can only be traversed from right to left in one of the tunnels. 

\begin{figure}
    \centering
%    \begin{subfigure}{0.08\textwidth}
%        \includegraphics[width=0.8\textwidth]{images/trav1.pdf}
%    \end{subfigure}
    \begin{subfigure}[b]{0.13\textwidth}
        \includegraphics[width=\linewidth]{images/trav2.pdf}
    \end{subfigure}\hfill
    \begin{subfigure}[b]{0.13\textwidth}
        \includegraphics[width=\linewidth]{images/trav3.pdf}
    \end{subfigure}\hfill
    \begin{subfigure}[b]{0.13\textwidth}
        \includegraphics[width=\linewidth]{images/trav4.pdf}
    \end{subfigure}\hfill
    \begin{subfigure}[b]{0.13\textwidth}
        \includegraphics[width=\linewidth]{images/trav5.pdf}
    \end{subfigure}\hfill
    \begin{subfigure}[b]{0.13\textwidth}
        \includegraphics[width=\linewidth]{images/trav6.pdf}
    \end{subfigure}\hfill
    \begin{subfigure}[b]{0.13\textwidth}
        \includegraphics[width=\linewidth]{images/trav7.pdf}
    \end{subfigure}\hfill
    \begin{subfigure}[b]{0.13\textwidth}
        \includegraphics[width=\linewidth]{images/trav8.pdf}
    \end{subfigure}
%    \begin{subfigure}{0.08\textwidth}
%        \includegraphics[width=0.85\textwidth]{images/trav9.pdf}
%    \end{subfigure}
    \caption{Traversal of the robot species.} %don't use the term traversal sequence since it is a technical term in the Gizmos framework.
    \label{fig:traversal}
\end{figure}

Figure \ref{fig:traversal} shows the process of a robot species traversing through the gadget. Notice when a robot species ``traverse'' a gadget, it actually traps itself to free another robot at the exit. We prove two lemmas to help verify the correctness of our construction. The lemmas prove the gadgets we design correctly implement the allowed traversals of a locking 2-toggle. 


\begin{lemma}\label{lem:rightward}
    A robot may perform a rightward traversal of a gadget through the north/south tunnel if and only if the key is moved from the central tunnel to the north/south holding cell. 
\end{lemma}
\begin{proof}
    The horizontal tunnels in state $1$ allow for a rightward traversal. The robot swaps with wires until it reaches the third cell where it is adjacent to two locks. However the key in the central tunnel may swap with the locks to reach the robot. The key and robot then swap. The key is then in the horizontal tunnel and can swap to the right with the lock there. It may then swap with the robot in the holding cell. This robot then may continue forward to the right and the key is stuck in the holding cell.

    Notice when entering from the left the robot will always reach  a cell adjacent to lock species. The robot may not swap with locks so it cannot traverse unless the key is in the central tunnel. 
\end{proof}



\begin{lemma}\label{lem:leftward}
    A robot may perform a leftward traversal of a gadget through the north/south tunnel if and only if the key is moved from the north/south holding cell to the central tunnel.
\end{lemma}
\begin{proof}
    In state $2$ the upper tunnel can be traversed and in state $3$ the lower tunnel can be traversed. The swap sequence for a leftward traversal is the reverse of the rightward traversal, meaning we are undoing the swaps to return to state $1$. The robot enters the gadget and swaps with the key, which swaps with the locks to move adjacent to the central tunnel. The key then returns to the central tunnel by swapping with the robot. The robot species can then leave the gadget to the left. 

    A robot entering from the right will not be able to swap to the position adjacent to the holding cell if it contains a lock. This is true in both tunnels in state $1$ and in the non-traversable tunnels in states $2$ and $3$. 
\end{proof}

We use these lemmas to first prove PSPACE-completeness of 1-reconfiguration. We reduce from the planar 1-player motion planning reachability problem.% which asks, given a configuration of a system of a gadgets and a target location, can the robot reach the target location? It is also important to point out that we reduce from this problem in planar systems of gadgets which is still hard \cite{demaine2018computational}.

\begin{theorem}\label{thm:oneRecSwap}
$1$-reconfiguration is PSPACE-complete with $4$ species and $3$ swap reactions or greater. 
\end{theorem}
\begin{proof}
    Given a system of gadgets create a surface encoding the connection graph between the locations. Each gadget is built as described above in a state representing the initial state of the system. Ports are connected using multiple cells containing wire species. When more than two ports are connected we use degree-3 cells with wire species. The target cell for $1$-reconfiguration is a cell containing a wire located at the target location in the system of gadgets.  

    If there exists a solution to the robot reachability problem then we can convert the sequence of gadget traversals to a sequence of swaps. The swaps relocate a robot species to the location as in the system of gadgets. 

    If there exists a swap sequence to place a robot species in the target cell there exists a solution to the robot reachability problem. Any swap sequence either moves an robot along a wire, or traverses it through a gadget. From Lemmas \ref{lem:rightward} and \ref{lem:leftward} we know the only way to traverse a gadget is to change its state (the location of its key) and a gadget can only be traversed in the correct state. 
\end{proof}

Now we show Reconfiguration in sCRNs is hard with the same set of swaps is PSPACE-complete as well. We do so by reducing from the Targeted Reconfiguration problem which asks, given an initial and target configuration of a system of gadgets, does there exist sequence of gadget traversals to change the state of the system from the initial to the target and has the robot reach a target location. Note prior work only shows reconfiguration (without specifying the robot location) is PSPACE-complete\cite{ani2022traversability} however a quick inspection of the proof shows the robot ends up at the initial location so requiring a target location does not change the computational complexity for the locking 2-toggle. One may also find it useful to note that the technique used in \cite{ani2022traversability} for gadgets and in \cite{hearn2009games} for Nondeterministic Constraint Logic can be applied to reversible deterministic systems more generally and could be used to give a reduction directly from 1-reconfiguration of swap sCRNs to reconfiguration of swap sCRNs.

\begin{theorem}\label{thm:swapReduc}
Reconfiguration is PSPACE-complete with $4$ species and $3$ swap reactions or greater. 
\end{theorem}
\begin{proof}
    Our initial and target configurations of the surface are built with the robot species at the robots location in the system of gadget, and each key is placed according to the starting configuration of the gadget. 

    Again as in the previous theorem we know from Lemmas \ref{lem:rightward} and \ref{lem:leftward} the robot species traversal corresponds to the traversals of the robot in the system of gadgets. The target surface can be reached if and only the target configuration in the system of gadgets is reachable. 
\end{proof}


\subsection{Polynomial-Time Algorithm}
%\xxx{Add 1-Reconfiguration Theorem/Cor}

%\xxx{Add introduction and perhaps remention friends-and-strangers relation. Seems neat that we connect sCRN, pebble games, and friends-and-strangers}

Here we show that the previous two hardness results are tight: when restricting to a smaller cases, both problems become solvable in polynomial time. We prove this by utilizing previously known algorithms for \emph{pebble games}, where labeled pebbles are placed on a subset of nodes of a graph (with at most one pebble per node). A \emph{move} consists of moving a pebble from its current node to an adjacent empty node. These pebble games are again a type of multiplicity friends-and-strangers graph. 

\begin{theorem}\label{thm:easySwaps}
Reconfiguration is in P with $3$ or fewer species and only swap reactions.
Reconfiguration is also in P with $2$ or fewer swap reactions and any number of species.
\end{theorem}
\begin{proof}

First we will cover the case of only two swap reactions. There are two possibilities: the two reactions share a common species or they do not. If they do not, we can partition the problem into two disjoint problems, one with only the species involved in the first reaction and the other with only the species from the second reaction. Each of these subproblems has only one reaction, and is solvable if and only if each connected component of the surface has the same number of each species in the initial and target configurations. 

The only other case is where we have three species, A, B, and C, where A and C can swap, B and C can swap, but A and B cannot swap. In this case, we can model it as a pebble motion problem on a graph. Consider the graph of the surface where we put a white pebble on each A species vertex, a black pebble on each B species vertex, and leave each C species vertex empty. A legal swap in the surface CRN corresponds to sliding a pebble to an adjacent empty vertex. Goraly et al.~\cite{goraly2010multi} gives a linear-time algorithm for determining whether there is a feasible solution to this pebble motion problem. Since the pebble motion problem is exactly equivalent to the surface CRN reconfiguration problem, the solution given by their algorithm directly says whether our surface CRN problem is feasible.
\end{proof}
\begin{theorem}\label{thm:1recEzSwap}
$1$-reconfiguration is in P with $3$ or fewer species and only swap reactions.
$1$-reconfiguration is also in P with $2$ or fewer swap reactions.
\end{theorem}
\begin{proof}
If there are only two swap reactions, we again have two cases depending on whether they share a common species. If they do not share a common species, then we only need to consider the rule involving the target species. The problem is solvable if and only if the connected component of the surface of species involved in this reaction containing the target cell also has at least one copy of the target species. Equivalently, if the target species is A, and A and B can swap, then there must either be A at the target location or a path of B species from the target location to the initial location of an A species.

The remaining case is when we again have three species, A, B, and C, where A and C can swap, B and C can swap, but A and B cannot swap. If C is the target species, then the problem is always solvable as long as there is any C in the initial configuration. Otherwise, suppose without loss of generality that the target species is A. Some initial A must reach the target location. For each initial A, consider the modified problem which has only that single A and replaces all of the other copies of A with B. A sequence of swaps in legal in this modified problem if and only if it was legal in the original problem. The original problem has a solution if and only if any of the modified ones do. We then convert each of these problems to a robot motion planning problem on a graph: place the robot at the vertex with a single copy of A, and place a moveable obstacle at each vertex with a B. A legal move is either sliding the robot to an adjacent empty vertex or sliding an obstacle to an adjacent empty vertex. Papadimitriou et al. \cite{papadimitriou1994motion} give a simple polynomial time algorithm for determining whether it is possible to get the robot to a given target location. By applying their algorithm to each of these modified problems (one for each cell that has an initial A), we can determine whether any of them have a solution in polynomial time (since there are only linearly many such problems), and thus determine whether the original 1-reconfiguration problem has a solution in polynomial time.

\end{proof}



%\subsection{Unique Sink}
%With reversible reactions the Unique Sink State problem is always no as there are no sink states. However we can use our previous reduction along with additional states to show this problem is PSPACE-complete in general. 

%\begin{theorem}
%    The Unique Sink State Problem is in P for reversible sCRNs.
%\end{theorem}

%\begin{theorem}
%    The Unique Sink State Problem is PSPACE-complete with $6$ species and $8$ %reactions or greater. 
%\end{theorem}