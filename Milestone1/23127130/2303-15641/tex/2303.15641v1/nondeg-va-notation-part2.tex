\begin{tabular}{lp{13cm}}
$V$ & a vertex algebra.\\
$U$ & a subspace of a weak $V$-module.\\
$\Omega_{V}(U)$&$
=\{\lu\in U\ \Big|\ 
a_{i}\lu=0\ \mbox{for all homogeneous }a\in V\mbox{and }i>\wt a-1\}$.\\
%$\mn$ & a non-zero complex number.\\
$\hei$ & a finite dimensional vector space equipped with a nondegenerate symmetric bilinear form
$\langle \mbox{ }, \mbox{ }\rangle$.\\
%$h$ & an element of $\fh$ with $\langle \wh,\wh\rangle=1$.\\
$h^{[1]},\ldots,h^{[\rankL]}$ & an orthonormal basis of $\fh$.\\
%$\alpha$ & an element of $\fh$ with $\langle\alpha,\alpha\rangle=\mn$.\\
$M(1)$ & the vertex operator algebra associated to the Heisenberg algebra.\\
$\lattice$ & a non-degenerate even lattice of finite rank.\\
$\rankL$ & the rank of $\lattice$.\\
$V_{\lattice}$ & the vertex algebra associated to $\lattice$.\\
$\theta$ & the automorphism of $V_{\lattice}$ induced from the $-1$ symmetry of $\lattice$.\\
$M(1)^{+}$ & the fixed point subalgbra of $M(1)$ under the action of $\theta$.\\
$V_{\lattice}^{+}$ & the fixed point subalgbra of $V_{\lattice}$ under the action of $\theta$.\\
%$\mK,\module,\mN,\mW$ & weak $M(1)^{+}$ (or $V_{\lattice}^{+}$)-modules.\\
$I(\mbox{ },x)$ & an intertwining operator for $M(1)^{+}$.\\
$\epsilon(\lu,\lv)$ & 
$\lu_{\epsilon(\lu,\lv)}\lv\neq 0\mbox{ and }\lu_{i}\lv=0\mbox{ for all }i>\epsilon(\lu,\lv)$
if $I(\lu,x)\lv\neq 0$ and $\epsilon(\lu,\lv)=-\infty$ if $I(\lu,x)\lv= 0$,
where $I : \module\times\mW\rightarrow \mN\db{x}$ is an intertwining operator and 
$\lu\in\module$, $\lv\in\mW$ (see \eqref{eqn:max-vanish}).\\
%$\langle\omega_i\rangle X$ & the space spanned by the elements $\omega_i^{j}\lu, j\in\Z_{\geq 0}, \lu\in X$. \\
$A(V)$ & the Zhu algebra of a vertex operator algebra $V$.\\
$A_{-}B$&$:=\Span_{\C}\{a_{-i}b\ |\ a\in A, b\in B,\mbox{ and }i\in\Z_{>0}\}$ (see \eqref{eqn:A-B:=Span}).\\
$\langle A_{-}\rangle B$ & see \eqref{eq:a(1)-i1cdotsa(n)-inb}.\\
$\omega$&$=(1/2)\sum_{i=1}^{\rankL}h^{[i]}(-1)^2\vac$.\nonumber\\
%$\Har$&$=(1/3)(h(-3)h(-1)\vac-h(-2)^2\vac)$.\\
%$J$&$=h(-1)^4\vac-2h(-3)h(-1)\vac+(3/2)h(-2)^2\vac=-9\Har+4\omega_{-1}^2\vac-3\omega_{-3}\vac$.\\
$\ExB(\alpha)$&$=e^{\alpha}+\theta(e^{\alpha})$ where $\alpha\in\fh$.\\
%$\lE$ & an integer such that $\lE\geq \epsilon(\ExB,\lu)$ or $\lE=\epsilon(\ExB,\lu)$ for a given non-zero element $\lu$.\\
$\omega^{[i]}$&$=(1/2)h^{[i]}(-1)^2$.\nonumber\\
$\Har^{[i]}$&$=(1/3)(h^{[i]}(-3)h^{[i]}(-1)\vac-h^{[i]}(-2)^2\vac)$.
%$\nS_{ij}(l,m)$&$=h^{[i]}(-l)h^{[j]}(-m)\vac$.\\
%$\epsilon(S)$& an integer such that $\epsilon(S)\geq \epsilon(S_{ij},\lu)$ or 
%$\epsilon(S)=\epsilon(S_{ij},\lu)$ for a given non-zero element $\lu$ and 
%$i,j\in\{1,\ldots,\rankL\}$.\\
%$B$ & a subspace of a weak $M(1)^{+}$-module $\mW$ \eqref{eq:B=SpanC(ajExB)}.\\
%$\zone$&$=\langle\alpha,h^{[1]}\rangle$.
\end{tabular}


%\label{eq:definition-omega-J-H}
