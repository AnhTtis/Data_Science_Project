\pdfoutput=1
\documentclass[12pt,letterpaper,floatfix]{article}
%% The amssymb package provides various useful mathematical symbols
\usepackage{amssymb}
\usepackage{hyperref}
\usepackage{cite}
\usepackage{amsmath,amssymb,amsthm,bbm,color}
\usepackage{graphicx}
\usepackage{epsfig}
\usepackage{euscript}
\usepackage{pslatex}
%\usepackage{xcolor}
\usepackage{fancybox}
\usepackage{enumerate}
\usepackage{widetext}
\usepackage{subfigure}
\usepackage{wasysym}
\usepackage[compat=1.0.0]{tikz-feynman}
%\usepackage{caption}
%\usepackage{cite}
\let\sstl=\scriptscriptstyle
\newcommand{\Meu}{\EuScript{M}}
\newcommand{\Mmf}{\mathfrak{M}}

\newcommand{\KK}{${\cal KK}$}
\def\Ordex#1{${\cal O}(#1)_{exp}$}
\def\Order#1{${\cal O}(#1)$}
\def\OrderLL#1{${\cal O}(#1)_{\rm LL}$}
\def\bbeta{\bar{\beta}}
\newcommand{\sfac}{\mathfrak{s}}
\def\hbeta{\hat{\beta}}
\newcommand{\sgn}{\mathop{\mathrm{sgn}}}
%?????????????
\def\st{\hbox{}} %% hbox is lower
\oddsidemargin 2mm
\topmargin -24mm  %CERN
\headheight 22mm
\headsep 0mm
\textheight 220mm
\textwidth 160mm
%\footheight 42mm
\footskip 30mm
\usepackage[figuresright]{rotating}

\begin{document}
\begin{titlepage}

\begin{center}
%\dochead{}
\begin{flushleft}
{\small \bf BU-HEPP-23-01, Jan. 2023}
\end{flushleft}
\vspace{18mm}
%% Use \dochead if there is an article header, e.g. \dochead{Short communication}

{\bf Collinearly Enhanced Realizations of the YFS MC Approach to Precision Resummation Theory}\\
\vspace{2mm}

{S. Jadach\footnote{Deceased.}$^{,\;a}$,~B.F.L. Ward$^b$,~ Z. Was$^a$}\\
{$^a$Institute of Nuclear Physics, Polish Academy of Sciences, Krakow, PL}\\
{$^b$Baylor University, Waco, TX, USA}\\
\end{center}
\centerline{\bf Abstract}
We extend the YFS IR resummation theory to include all of the attendant collinear contributions which exponentiate. This improves the original YFS formulation in which only a part of these contributions was exponentiated. 
We show that the new resummed contributions agree with known results from the collinear factorization approach and we argue that they improve the attendant precision tag for a given level of exactness in the respective YFS hard radiation residuals. 
%\end{abstract}
\end{titlepage}


%\section{Introduction: The YFS MC Approach to Precision Theory}
%\label{intro}
The exact amplitude-based  CEEX/EEX YFS MC approach to EW higher order corrections is given in Refs.~\cite{Jadach:1993yv,Jadach:1999vf,Jadach:2000ir,Jadach:2013aha}. Here, CEEX denotes the coherent exclusive exponentiation developed in Refs.~\cite{Jadach:1999vf,Jadach:2000ir,Jadach:2013aha} in which IR singularities are resummed at the level of the amplitude. EEX denotes exclusive exponentiation as originally formulated by Yennie, Frautschi and Suura (YFS) in Ref.~\cite{yfs:1961} and it is effected at the squared amplitude level. In the context of precision physics for $e^+e^-$ colliding beam devices, we and our collaborators have developed and implemented several MC event generators which realize the YFS MC approach for EW higher order corrections -- see Ref.~\cite{frix-lnen:2022} for a concise catalog of these programs. All of MC's in that catalog, except perhaps for KORALZ~\cite{Jadach:1993yv}, which has been superseded by KKMC-ee~\cite{Jadach:1999vf,dizet642,Jadach:2022vf}, are under consideration for the appropriate upgrades to meet the new precision requirements for the future $e^+e^-$ colliders. 

While the expectations for the future colliding beam devices are well-defined and formulated~\cite{Jadach:2019bye,frix-lnen:2022}, in what follows we present a collinear improvement of the original YFS algebra with an eye toward enhancing these expectations. We argue that this improvement yields even more precision for a given level of exactness in the respective hard photon residuals 
$\bar{\beta}_n \; (\hat{\beta}_n)$ in the EEX (CEEX) formulation respectively.
This work builds on ideas presented by one of us (SJ) already in Ref.~\cite{Jadach-unpublhd-2002}.

The discussion proceeds as follows. Since it is still not generally used, we first review CEEX/EEX realization of the higher order EW corrections to the Standard Theory (ST)\footnote{We follow Prof. D.J. Gross~\cite{djg-smat50} refer to the Standard Model as the Standard Theory of elementary particles.}~\cite{SM1,SM2,SM3,SM4} of elementary particles. Then we show how to extend the attendant YFS IR algebra to include the respective complete collinear contribution. We compare with known collinear leading log results. We close with some summary remarks. 

Specifically, we recall the master formula for the CEEX/EEX realization of the YFS resummation of the EW Standard theory.
For the prototypical process 
$e^+e^-\rightarrow f\bar{f}+n\gamma, \; f = \ell, q, \; \ell=e,\mu,\tau,\nu_e,\nu_\mu,\nu_\tau, \; q = u,d,s,c,b,t$, we have the cross section formula
\begin{equation}
\sigma =\frac{1}{\text{flux}}\sum_{n=0}^{\infty}\int d\text{LIPS}_{n+2}\; \rho_A^{(n)}(\{p\},\{k\}),
\label{eqn-yfsmth-1}
\end{equation}
where $\text{LIPS}_{n+2}$ denotes Lorentz-invariant phase-space for $n+2$ particles, $A=\text{CEEX},\;\text{EEX}$, the incoming and outgoing fermion momenta are abbreviated as $\{p\}$ and the $n$ photon momenta are denoted by $\{k\}$.
Thanks to the use of conformal symmetry, full $2+n$ body  phase space is covered without any
approximations. The respective MC algorithm's details are covered in Ref.~\cite{Jadach:1999vf}.
From Refs.~\cite{Jadach:2000ir,Jadach:1999vf} we have that 
\begin{equation}
\rho_{\text{CEEX}}^{(n)}(\{p\},\{k\})=\frac{1}{n!}e^{Y(\Omega;\{p\})}\bar{\Theta}(\Omega)\frac{1}{4}\sum_{\text{helicities}\;{\{\lambda\},\{\mu\}}}
\left|\Meu\left(\st^{\{p\}}_{\{\lambda\}}\st^{\{k\}}_{\{\mu\}}\right)\right|^2.
\label{eqn-yfsmth-2}
\end{equation}
(See Refs.~\cite{Jadach:2000ir,Jadach:1999vf} for the corresponding formula for the $A=\text{EEX}$ case.) Here,  $Y(\Omega;\{p\})$ is the YFS infrared exponent.
The respective infrared integration limits are specified by the region $\Omega$ and its characteristic function
$\Theta(\Omega,k)$ for a photon of energy $k$, with $\bar\Theta(\Omega;k)=1-\Theta(\Omega,k)$ and $$\bar\Theta(\Omega)=\prod_{i=1}^{n}\bar\Theta(\Omega,k_i).$$
By definition, $\Theta(\Omega,k)=1$  for $k\in\Omega$ 
and $\Theta(\Omega,k)=0$ for $k\not\in \Omega$.
As we will need it in what follows, we note that for  $\Omega$ defined with the condition $k^0<E_{\min},$ the YFS infrared exponent reads
%//////////////////////////////////////////////////
\begin{equation}
  \label{eq:YFS-ffactor}
  \begin{split}
   &Y(\Omega;p_a,...,p_d)
  =   Q_e^2   Y_\Omega(p_a,p_b)  +Q_f^2   Y_\Omega(p_c,p_d)\\
&\qquad\qquad
     +Q_e Q_f Y_\Omega(p_a,p_c)  +Q_e Q_f Y_\Omega(p_b,p_d) 
     -Q_e Q_f Y_\Omega(p_a,p_d)  -Q_e Q_f Y_\Omega(p_b,p_c).
  \end{split}
\end{equation}
where
%%
\begin{equation}
\label{eq:form-factor}
\begin{split}
Y_\Omega(p,q) 
     \equiv &  2 \alpha \tilde{B}(\Omega,p,q)   +2 \alpha \Re B(p,q) \\
     \equiv & -2 \alpha\;{ 1 \over 8\pi^2} \int {d^3k\over k^0} \Theta(\Omega;k)
                       \bigg({p\over kp} - {q\over kq} \bigg)^2 \\
            & +2 \alpha \Re \int {d^4k\over k^2} {i\over (2\pi)^3} 
                       \bigg( {2p-k \over 2kp-k^2} - {2q+k \over 2kq+k^2} \bigg)^2.
\end{split}
\end{equation}
See Refs.~\cite{Jadach:1999vf,Jadach:2000ir,Jadach:2013aha} for the definitions of the CEEX amplitudes~$\{\Meu\}$. 
\KK{MC} 4.22~\cite{Jadach:2013aha} featured the exact
${\cal O}(\alpha)$ EW corrections implemented using the  DIZET {6.21,...,6.42} EW libraries from the semi-analytical
program ZFITTER~\cite{zfitter1,zfitter6:1999,zfitter:2006,dizet642}.
Recent public versions of \KK{MC} include also the DIZET 6.45 version,
see \cite{Arbuzov:2020coe} and the recently released C++ version 5.0 of \KK{MC} now denoted as \KK{MC-ee}\footnote{This notation distinguishes it from the MC \KK{MC-hh}~\cite{kkmchh,kkmchh1,kkmchh2} which calculates CEEX ${\cal O}(\alpha^2 L^2, \alpha^2L)$ EW corrections to
the Drell-Yan single $Z/\gamma*$ production processes with decay to lepton pairs.}~\cite{Jadach:2022vf}.
As the respective implementation is described in Ref.~\cite{Jadach:2000ir}, we do not repeat it here.
In {\KK}MC-ee, the CEEX amplitudes $\{\Meu\}$ in
Eqs.~(\ref{eqn-yfsmth-1},\ref{eqn-yfsmth-2}) are exact 
in ${\cal O}(\alpha^2 L^2, \alpha^2L)$ in the sense that all terms in the respective cross section at orders ${\cal O}(\alpha^0),\;{\cal O}(\alpha),\; {\cal O}(\alpha L),\;{\cal O}(\alpha^2 L), \; \text{and}\; {\cal O}(\alpha^2 L^2)$ are all included in our result for that cross section.
Here the big log is $L=\ln\frac{Q^2}{m^2}$ where
$Q$ is the respective hard 4-momentum transfer and the charged lepton masses and the quark masses determine $m$, depending on the specific process under consideration.
We follow Ref.~\cite{mstw-mass} and use the current quark masses~\cite{PDG:2016} $m_u = 2.2 \text{MeV}.\; m_d = 4.7 \text{MeV}, \;  m_s = 0.150\text{GeV}, \; m_c = 1.2 \text{GeV},\;  m_b = 4.6\text{GeV} \text{and}\; m_t = 173.5 \text{GeV}$\footnote{See Ref.~\cite{kkmchh1} for a relevant discussion of the uncertainty of our results due to realistic uncertainties on our values of the current quark masses - we find a shift of our effects at the level of 10\% of the effects themselves due to the latter uncertainties.}. We note for completeness that in our MC's all real and soft virtual photonic corrections have
$\alpha=\alpha(0)=\frac{1}{137.035999...}$, since photons are massless.
For hard QED corrections, we use $\alpha=\alpha(Q),$ with the attendant hadronic vacuum polarization taken after Ref.~\cite{fjeger-fccwksp2019}. For completeness, we note that the EEX realization in {\KK}MC-ee includes as well the exact ${\cal O}(\alpha^3 L^3)$ corrections. The user always has the option to switch on this correction as needed.

The fundamental idea of the YFS resummation is to isolate and resum to all orders in $\alpha$ the infrared singularities so that these singularities are canceled to all such orders between real and virtual corrections. The question naturally arises as to what, if any, collinear singularities are also resummed in the YFS resummation algebra. We can see the answer to this question by examining the exact result for the YFS IR expononent $Y(\Omega;\{p\})$ in eq.(\ref{eqn-yfsmth-2}): focusing on the s-channel and s'-channel contributions for reasons of pedagogy, we have 
\begin{equation}
\begin{split}
  Y_e(\Omega_I;p_1,p_2)
 &=   \gamma_e \ln {2E_{min}\over \sqrt{2p_1p_2}}     +{1\over 4}\gamma_e
      +Q_e^2 {\alpha\over\pi} \bigg( -{1\over 2} +{\pi^2\over 3}\bigg),\\
  Y_f(\Omega_F;q_1,q_2)
 &=   \gamma_f \ln {2E_{min}\over \sqrt{2q_1q_2}}     +{1\over 4}\gamma_f
      +Q_f^2 {\alpha\over\pi} \bigg( -{1\over 2} +{\pi^2\over 3}\bigg),
\end{split}
\end{equation}
where
\begin{equation}
\gamma_e = 2 Q_e^2 {\alpha\over \pi} \bigg( \ln {2p_1p_2\over m_e^2} -1 \bigg),\quad
\gamma_f         = 2 Q_f^2 {\alpha\over \pi} \bigg( \ln {2q_1q_2\over m_f^2} -1 \bigg),
\end{equation}
using an obvious notation for the $f\bar{f}$ production process so that $p_a=p_1,\;p_2=p_b,\;p_c=q_1,
\;\text{and} p_d=q_2.$
We see that the YFS exponent has also resummed the collinear big log term $\frac{1}{2}Q^2{\alpha\over\pi} L$ to the infinite order in both the ISR and FSR contributions, where $Q=Q_e,\; Q_f,$ respectively. The question naturally arises as to whether or not the YFS algebra can be extended to resum further collinear big log contributions. Indeed, it is known~\cite{grgbv-lptv:1972} that from the QED form factor the term $\frac{3}{2}{\alpha\over\pi} L$ exponentiates. Does YFS algebra allow for this?\par

To investigate this point, we focus on the derivation of the YFS form factor as illustrated in Fig. ~\ref{fig1}. 
\begin{figure}[ht]
\begin{center}
\setlength{\unitlength}{1in}
\begin{picture}(6,2.4)(0,0)
%\put(1.0,0.2){\includegraphics[width=4in]{fig-csfrme-1.pdf}}
\put(1.0,0.2){\includegraphics[width=6in]{fig1-2207.pdf}}
%\put(3,0.2){\includegraphics[width=3in]{muon-pt-ratio-6-18-18.pdf}}
\end{picture}
\end{center}
\vspace{-10mm}
\caption{Virtual corrections which generate the YFS infrared function $B$.  Self-energy contributions are not shown.}
\label{fig1}
\end{figure}
This loop in which we have labeled the virtual photon momentum $k$ gives us the amplitude factor 
\begin{equation}
\begin{split}
{\cal M}_\mu &= \frac{\int d^4k}{(2\pi)^4}\frac{-i}{k^2+i\epsilon}\bar{v}(p_2)(-iQ_ee)\gamma^\alpha \frac{i}{-\!\not p_2-\!\not k-m+i\epsilon}(-ie)\gamma_\mu(v_A-a_A\gamma_5)\\
                      & \quad \quad \quad \quad \frac{i}{\!\not p_1-\!\not k-m+i\epsilon}(-iQ_ee)\gamma_\alpha u(p_1)
\end{split}
\label{yfsalg1}
\end{equation}
where $A=\gamma\;\text{or}\; Z$ according to whether we have the photon or Z exchange in the s-channel, respectively.
The couplings $v_A,\;a_A$ at the Standard Theory EW couplings in units of the positron charge $e$.
Scalarising the fermion propagator denominators gives us
\begin{equation}
{\cal M}_\mu = -ie\frac{\int d^4k(-i{Q_e}^2e^2)}{(2\pi)^4}\frac{1}{k^2+i\epsilon}\bar{v}(p_2)\gamma^\alpha \frac{-\!\not p_2-\!\not k+m}{k^2+2kp_2+i\epsilon}\gamma_\mu(v_A-a_A\gamma_5)\frac{\!\not p_1-\!\not k-m}{k^2-2kp_1+i\epsilon}\gamma_\alpha u(p_1).
\label{yfsalg2}
\end{equation}
The numerator factors for the fermion propagators in multiplication with the respective gamma matrices can be re-written, using the equations of motion, as
\begin{equation}
\begin{split}
(\!\not p_1-\!\not k-m)\gamma_\alpha u(p_1)&=\{(2p_1- k)_\alpha-\frac{1}{2}[\!\not k,\gamma_\alpha]\}u(p_1), \quad\qquad (a)\\
\bar{v}(p_2)\gamma^\alpha(-\!\not p_2-\!\not k+m)&=\bar{v}(p_2)\{-(2p_2+k)^\alpha+\frac{1}{2}[\!\not k,\gamma^\alpha]\}, \;\qquad (b).
\end{split}
\label{yfsalg3}
\end{equation}
This allows us to identify the contribution to $2Q_e^2\alpha B(p_1,p_2)$ corresponding to the cross-term in the virtual IR function on the RHS of eq.(\ref{eq:form-factor}):
\begin{equation}
2Q_e^2\alpha B(p_1,p_2)|_{\text{cross-term}}=\frac{\int d^4k(i{Q_e}^2e^2)}{8\pi^4}\frac{1}{k^2+i\epsilon}\frac{(2p_1-k)(2p_2 +k)}{(k^2-2kp_1+i\epsilon)(k^2+2kp_2+i\epsilon)}.
\label{cross-term}
\end{equation}
This term, together with the two squared terms in $2\alpha Q_e^2B(p_1,p_2),$ leads to the exponentiation of $\frac{1}{2}Q_e^2{\alpha\over\pi} L$ as we have indicated. 

If we look at the two commutator terms on the RHS of eq.(\ref{yfsalg3}), we see that, instead of dropping them from the YFS algebra as it is usually done, we can analyze them further for a possible IR finite collinearly enhanced improvement of the YFS virtual IR 
function $B$. For, if we isolate the collinear parts of $k$ via the change of variables
\begin{equation}
   k = c_1p_1 +c_2p_2+k_\perp
\label{yfsalg4}
\end{equation}
where $p_1k_\perp=0=p_2k_\perp$,
we have the relations
\begin{equation}
\begin{split}
c_1 &= \frac{p_1p_2}{(p_1p_2)^2-m^4}p_2k - \frac{m^2}{(p_1p_2)^2-m^4}p_1k \xrightarrow[CL]{}\frac{p_2k}{p_1p_2} \\
c_2 &= \frac{p_1p_2}{(p_1p_2)^2-m^4}p_1k - \frac{m^2}{(p_1p_2)^2-m^4}p_2k \xrightarrow[CL]{}\frac{p_1k}{p_1p_2},
\end{split}
\label{yfsalg5}
\end{equation}
where $CL$ denotes the collinear limit in which terms ${\cal O}(m^2/s)$ are dropped. Here $s$ is the cms squared energy. This means that the numerator term $(2p_1-k)^\alpha$ in eq.(\ref{yfsalg3}(a)) combines with the commutator term in the eq.(\ref{yfsalg3}(b)) to produce the
the numerator contribution 
\begin{equation}
\begin{split}
\bar{v}(p_2)\{(2p_1-k)_\alpha\frac{1}{2}[\!\not k,\gamma^\alpha]\}\gamma_\mu(v_A-a_A\gamma_5)u(p_1)&=\bar{v}(p_2)[\!\not k,\!\not p_1]\gamma_\mu(v_A-a_A\gamma_5)u(p_1)\\
                                                                                                 &\xrightarrow[CL]{}\bar{v}(p_2)[c_2\!\not p_2 ,\!\not p_1]\gamma_\mu(v_A-a_A\gamma_5)u(p_1)\\
                                                                                                 &\xrightarrow[CL]{}\bar{v}(p_2)(-2c_2p_1p_2)\gamma_\mu(v_A-a_A\gamma_5)u(p_1) \\
                                                                                                 &\xrightarrow[CL]{}\bar{v}(p_2)(-2p_1k)\gamma_\mu(v_A-a_A\gamma_5)u(p_1). \\
\end{split}
\label{yfsalg6}
\end{equation}
Similarly, the numerator term $-(2p_2+k)^\alpha$ in eq.(\ref{yfsalg3} (b)) combines with the commutator term in eq.(\ref{yfsalg3}(a)) to produce the numerator contribution
\begin{equation}
\begin{split}
\bar{v}(p_2)\gamma_\mu(v_A-a_A\gamma_5)\{-(2p_2+ k)^\alpha(-\frac{1}{2}[\!\not k,\gamma_\alpha])\}u(p_1)&=\bar{v}(p_2)\gamma_\mu(v_A-a_A\gamma_5)[\!\not k,\!\not p_2]u(p_1)\\
                                                                                       &\xrightarrow[CL]{}\bar{v}(p_2)\gamma_\mu(v_A-a_A\gamma_5)[c_1\!\not p_1,\!\not p_2]u(p_1)\\
                                                                                       &\xrightarrow[CL]{}\bar{v}(p_2)\gamma_\mu(v_A-a_A\gamma_5)(2c_1p_1p_2)u(p_1)\\
                                                                                       &\xrightarrow[CL]{}\bar{v}(p_2)\gamma_\mu(v_A-a_A\gamma_5)(2p_2k)u(p_1).\\
\end{split}
\label{yfsalg7}
\end{equation}
We therefore have the shift of the factor $(2p_1-k)(2p_2+k)$ on the RHS of eq.(\ref{cross-term}) as
\begin{equation}
(2p_1-k)(2p_2+k)\xrightarrow[CL]{} (2p_1-k)(2p_2+k) +2p_1k -2p_2k.
\label{yfsalg8}
\end{equation} 

The term in the numerator which is quadratic in the commutator ($C^2$) is superficially divergent in the UV so that we cannot drop
$k_\perp$ naively. Instead of doing that, we proceed directly: {\small
\begin{equation}
2Q_e^2\alpha B(p_1,p_2)|_{C^2}{\cal M}_{B\mu}\equiv \frac{\int d^4k(i{Q_e}^2e^2)}{8\pi^4}\frac{1}{k^2+i\epsilon}\frac{\frac{1}{4}\bar{v}(p_2)[\!\not k,\gamma^\alpha]\gamma_\mu[\!\not k,\gamma_\alpha](-ie)(v_A-a_A\gamma_5)u(p_1)}{(k^2-2kp_1+i\epsilon)(k^2+2kp_2+i\epsilon)}\bigg|_{CL},
\label{quadratic-term}
\end{equation} }
where we define
\begin{equation}
{\cal M}_{B\mu} = -ie\bar{v}(p_2)\gamma_\mu(v_A-a_A\gamma_5)u(p_1)
\end{equation}
in an obvious notation and we restrict further the definition of $CL$ here to include only contributions singular in the limit 
$m^2/s \rightarrow 0$. There are four terms in the numerator on the RHS of eq.(\ref{quadratic-term}) from the respective sum of gamma matrix products $\{\!\not k \gamma^\alpha \gamma_\mu \!\not k \gamma_\alpha -\!\not k \gamma^\alpha \gamma_\mu \gamma_\alpha \!\not k \\
- \gamma^\alpha \!\not k \gamma_\mu \!\not k \gamma_\alpha + 
\gamma^\alpha \!\not k \gamma_\mu \gamma_\alpha \!\not k\}=\{\gamma^\lambda \gamma^\alpha \gamma_\mu \gamma^{\lambda'}\gamma_\alpha -\gamma^\lambda \gamma^\alpha \gamma_\mu \gamma_\alpha \gamma^{\lambda'}\; - \gamma^\alpha \gamma^\lambda \gamma_\mu \gamma^{\lambda'} \gamma_\alpha + \gamma^\alpha \gamma^\lambda \gamma_\mu \gamma_\alpha \gamma^{\lambda'}\}k_\lambda k_{\lambda'}\equiv N^{\lambda\lambda'}_\mu k_\lambda k_{\lambda'},$ where this latter equivalence sign serves too define $N^{\lambda\lambda'}_\mu$. Using n-dimensional methods~\cite{BW1a}, we see
that, when we combine the denominators to get the standard Feynman parametrization, we need to evaluate {\small
\begin{equation}
I_\mu= 2\int_0^1d\alpha_1\int_0^{1-\alpha_1}d\alpha_2 \frac{\int d^nk'(i{Q_e}^2e^2)}{8\pi^4}\frac{\frac{1}{4}\bar{v}(p_2)N^{\lambda \lambda'}_\mu[\frac{{k'}^2}{n}g_{\lambda\lambda'}+\Delta_\lambda \Delta_{\lambda'}](-ie)(v_A-a_A\gamma_5)u(p_1)}{[{k'}^2-\Delta^2+i\epsilon]^3}\bigg|_{CL},
\label{quadratic-term2}
\end{equation} }
where $\Delta=\alpha_1p_1-\alpha_2p_2$. From the equations of motion we see that the terms involving $\Delta$ on the RHS of eq.(\ref{quadratic-term2}) do not make a collinearly enhanced contribution. Computation of the term contracted with $g_{\lambda \lambda'}$ on the RHS of eq.(\ref{quadratic-term2}) gives us 
\begin{equation}
I_\mu={\bigg\{} \frac{ -3Q_e^2\alpha}{ 4\pi}{\cal M}_{B\mu}{\bigg\}}{\bigg|}_{CL}\equiv 0
\end{equation}
so that there are no collinearly enhanced contributions from $I_\mu$. Eq.(\ref{yfsalg8}) gives the complete collinear enhancement of $B$.

This change in $B$ does not affect its IR behavior because the shift terms are IR finite. Thus, the entire YFS IR resummation is unaffected. But, the shifted terms can be seen to extend the YFS IR exponentiation to obtain the entire exponentiated $\frac{3}{2}Q_e^2\alpha L$.

Specifically, we have
\begin{equation}
\begin{split}
2\alpha Q_e^2\Delta B(p_1,p_2) &=\frac{\int d^4k(iQ_e^2e^2)}{8\pi^4}\frac{1}{k^2+i\epsilon}\frac{2p_1k - 2p_2k}{(k^2-2kp_1+i\epsilon)(k^2+2kp_2+i\epsilon)}\\
                                     &=2\int_{x_i\ge0,i=1,2,3} d^3x\delta(1-x_1-x_2-x_3)\frac{\int d^4k'(iQ_e^2e^2)}{8\pi^4}\frac{2(p_1 - p_2)p_x}{(k'^2-d+i\epsilon)^3}
\end{split}                                     
\label{yfsalg9}
\end{equation}
where $d=p_x^2$ with $p_x=x_1p_1-x_2p_2$. We get
\begin{equation}
2Q_e^2\alpha \Re \Delta B(p_1,p_2) = Q_e^2{\alpha\over\pi} L.
\label{yfsalg10}
\end{equation}
We see that indeed the entire term ${3\over2}Q_e^2{\alpha\over\pi} L$ is now exponentiated by our collinearly improved YFS virtual IR function $B_{CL}$ given by 
\begin{equation}
\begin{split}
B_{CL}&=B+\Delta B\\
          &= \int {d^4k\over k^2} {i\over (2\pi)^3} 
                       \bigg[\bigg( {2p-k \over 2kp-k^2} - {2q+k \over 2kq+k^2} \bigg)^2-\frac{4pk-4qk}{(2pk-k^2)(2qk+k^2)}\bigg].
\end{split}
\end{equation}
We stress again that the YFS IR resummation calculus is unaffected by the use of $B_{CL}$ instead of $B$\footnote{In Ref.~\cite{Jadach-unpublhd-2002} one of us (SJ) has identified the integrated form of $B_{CL}$ by matching with the respective Sudakov form factor.}.

In some applications, it can be of interest to improve collinearly the YFS real emission IR resummation algebra. In this connection we recall that the original YFS EEX formulation of the respective algebra leads to the
formula for the YFS IR function $\tilde{B}$ given above in eq.(\ref{eq:form-factor}). The situation is 
illustrated in Fig.~\ref{fig2}.
\begin{figure}[ht]
\begin{center}
\setlength{\unitlength}{1in}
\begin{picture}(6,2.4)(0,0)
%\put(1.0,0.2){\includegraphics[width=4in]{fig-csfrme-1.pdf}}
\put(0.1,0.2){\includegraphics[width=6in]{fig2-2207.pdf}}
%\put(3,0.2){\includegraphics[width=3in]{muon-pt-ratio-6-18-18.pdf}}
\end{picture}
\end{center}
\vspace{-10mm}
\caption{Real corrections which generate the YFS infrared function $\tilde{B}$.}
\label{fig2}
\end{figure}
Following the steps in the usual YFS algebra for real emission, we see that the corresponding contribution to $2\alpha \tilde{B}$ can be isolated from the respective squared matrix element integrated over the corresponding photon phase space as follows:
\begin{equation}
\begin{split}
2\alpha Q_e^2\tilde{B}{\cal M}^\dagger_{B\mu}{\cal M}_{B\mu'} &=\frac{\int d^3k (-1)e^2 Q_e^2}{2k_0(2\pi)^3}\bigg[\frac{\bar{u}(p_1)\large(2p_1^\lambda - k^\lambda + \frac{1}{2}[\!\not k,\gamma^\lambda]\large)\gamma_\mu(v_A-a_A\gamma_5)v(p_2)}{k^2-2kp_1}\\
                                              & + \frac{\bar{u}(p_1)\gamma_\mu(v_A-a_A\gamma_5)\large(-2p_2^\lambda + k^\lambda + \frac{1}{2}[\!\not k,\gamma^\lambda]\large)v(p_2)}{k^2-2kp_2}\bigg]\\
                                              & \bigg[\frac{\bar{v}(p_2)\gamma_{\mu'}(v_A-a_A\gamma_5)\large(2p_{1\lambda} - k_\lambda - \frac{1}{2}[\!\not k,\gamma_\lambda]\large)u(p_1)}{k^2-2kp_1}\\
                                             & + \frac{\bar{v}(p_2)\large(-2p_{2\lambda} + k_\lambda - \frac{1}{2}[\!\not k,\gamma_\lambda]\large)\gamma_{\mu'}(v_A-a_A\gamma_5)u(p_1)}{k^2-2kp_2}\bigg]\bigg|_{k^2=0} + K_{\mu\mu'}
\end{split}
\label{eq-real1}
\end{equation}
where $K_{\mu\mu'}$ is infrared finite. We can immediately see that if we drop the commutator terms on the RHS of eq.(\ref{eq-real1}) we recover the usual YFS formula for $2\alpha Q_e^2\tilde{B}$. We can again isolate collinearly enhanced contributions by using the representation in eq.(yfsalg4) for $k$. However, we have to respect the condition $k^2=0.$ This means
that we have to maintain $0=(c_1^2+c_2^2)m^2+ 2c_1c_2p_1p_2-|k_\perp|^2$. Upon isolating the collinearly enhanced contributions subject to this latter constraint we get the following collinear enhancement of $\tilde{B}$:
\begin{equation}
\begin{split}
2\alpha Q_e^2\tilde{B}_{CL} &=\frac{-\alpha Q_e^2}{4\pi^2}\int\frac{d^3k}{k_0}\bigg\{\large(\frac{p_1}{kp_1} - \frac{p_2}{kp_2}\large)^2 +\frac{1}{kp_1}\large(2 -\frac{kp_2}{p_1p_2}\large)\\
                                          &\qquad\qquad+\frac{1}{kp_2}\large(2 -\frac{kp_1}{p_1p_2}\large)\bigg\}.
\end{split}
\label{eq-real2}
\end{equation}
The extra non-IR divergent terms can be seen to reproduce the respective collinear big log of the exact result in Ref.~\ref{berends-neerver-burgers:1988} in the soft regime. 
This then leads us to consider the realization of eq.(\ref{eq-real2}) in the CEEX formalism. To this we now turn.\par
For the CEEX formalism, we revisit Fig.~\ref{fig2} with the use of amplitude-level isolation of real IR divergences.
We follow Ref.~\cite{ceex2:1999} and introduce the Kleiss-Stirling~\cite{kleiss-stirling:1985} photon polarization vectors 
so that the amplitude in Fig. 2 can be written as, for the photon polarization $\sigma$ and $e^-$  helicity $\lambda$,
\begin{equation}
{\cal M}_\mu = {\cal M}_{B\mu}\sfac_{CL,\sigma}(k),
\label{eq-real3}
\end{equation}
where we define the collinearly enhanced soft (eikonal) amplitude factor, an extension of the corresponding factor defined in Ref.~\cite{ceex2:1999}, via
\begin{equation}
\begin{split}
\sfac_{CL,\sigma}(k) = \sqrt{2}Q_ee\bigg[-\sqrt{\frac{p_1\zeta}{k\zeta}}\frac{<k\sigma|\hat{p}_1 -\sigma>}{2p_1k}
    + \delta_{\lambda\;-\sigma}\sqrt{\frac{k\zeta}{p_1\zeta}}\frac{<k\sigma|\hat{p}_1 \lambda>}{2p_1k}\\
    + \sqrt{\frac{p_2\zeta}{k\zeta}}\frac{<k\sigma|\hat{p}_2 -\sigma>}{2p_2k}+\delta_{\lambda \sigma}\sqrt{\frac{k\zeta}{p_2\zeta}}\frac{<\hat{p}_2 \lambda|k -\sigma>}{2p_2k}\bigg].
\end{split}
\label{eq-real4}
\end{equation}
We have introduced from Ref.~\cite{ceex2:1999} the notations
$\zeta\equiv (1,1,0,0)$ for our choice for the respective auxiliary vector in our Global Positioning of Spin (GPS)~\cite{gps:1998} spinor conventions with the consequent definition $\hat{p}= p - \zeta m^2/(2\zeta p)$
for any four vector $p$ with $p^2 = m^2.$ Again, upon taking the modulus squared of $\sfac_{CL,\sigma}(k)$ we can see that the extra non-IR divergent  contributions reproduce the known collinear big log contribution which is missed by the usual YFS algebra.\par

%\section{Summary}
In conclusion, we have extended the original YFS algebra to include collinear non-IR big logs that are known to be missed by the usual YFS soft functions. This implies that these new, collinealry enhanced soft functions will yield a higher level of accuracy for a given level of exactness in the IR-finite YFS hard photon residuals. They thus enhance the
set of the tools available to extend the CEEX YFS MC method to the other important processes in the future $e^+e^-$ colliders' precision physics programs in the effort to reach the new required precisions for these processes.

\vskip 2 mm
\centerline{\bf Acknowledgments}
\vskip 2 mm

The authors thank Prof. G. Giudice for the support 
and kind hospitality of the CERN TH Department. 
S.J. acknowledges funding from the European Union’s Horizon 2020 research and innovation programme under under grant agreement No 951754 and support of the National Science Centre, Poland, Grant No. 2019/34/E/ST2/00457.

\bibliography{Tauola_interface_design}{}
\bibliographystyle{utphys_spires}



\end{document}
