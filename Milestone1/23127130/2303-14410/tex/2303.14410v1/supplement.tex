\documentclass[twocolumn,reprint,prb,aps,amsmath,showpacs,floatfix,superscriptaddress]{revtex4-2}
%\documentclass[aip,reprint,amsmath,showpacs,floatfix]{revtex4-2}


\usepackage{graphicx}% Include figure files
\usepackage{dcolumn}% Align table columns on decimal point
\usepackage{mathtools}% include equation features
\usepackage{color}
\usepackage{braket}
\usepackage{hyperref}
\graphicspath{{figures/}}
\usepackage{soul}

\usepackage{algorithmicx}

\usepackage{xr}
\externaldocument{main}

%%%%%%%%%%%%%%%%%%%%%%%

% Defining symbols

\newcommand{\Ham}{H}
\newcommand{\fdag}{f^\dagger}
\renewcommand{\ddag}{d^\dagger}
\newcommand{\Hc}{\mathrm{H.c.}}
\newcommand{\eps}{\varepsilon}
\newcommand{\elr}{\varepsilon_{L/R}}
\newcommand{\n}{\hat{n}}
\newcommand{\ntot}{N_{tot}}
\newcommand{\ua}{\uparrow}
\newcommand{\da}{\downarrow}
\newcommand{\ulr}{U_{L/R}}
\newcommand{\der}{\mathrm{d}}


%%%%%%%%%%%%%%%%%%%%%%%

% commands for commenting

\usepackage[normalem]{ulem}
\newcommand{\comment}[1]{{\color{ForestGreen} \textsf{#1}}}
\newcommand{\add}[1]{{\color{blue} \textit{#1}}}
\newcommand{\remove}[1]{{\color{red} \sout{#1}}}
\newcommand{\emphasize}[1]{{\color{violet} {\bf #1}}}
\newcommand{\cn}{{\bf [cn]}}
			  
%%%%%%%%%%%%%%%%%%%%%%%			  
			  
% afilliations
\newcommand{\qutech}{QuTech and Kavli Institute of Nanoscience, Delft University of Technology, 2600 GA Delft, The Netherlands}

%%%%%%%%%%%%%%%%%%%%%%%

\begin{document}
\title{Supplementary simulations to ``Two Anderson impurities coupled through a superconducting island: charge stability diagrams and double impurity qubit''}

\author{Filip~K.~Malinowski}
\email{f.k.malinowski@tudelft.nl}
\affiliation{\qutech}

\date{\today}

\begin{abstract}
\end{abstract}

\maketitle

%\appendix
%\counterwithin{figure}{section}


This appendix includes a superset of simulations presented in the main text.

\begin{itemize}
	\item (Fig.~\ref{fig_no_tunneling_overview}) Charge stability diagrams without dot-island coupling (c.f.~Fig.~\ref{fig_no_tunneling})
	\item (Fig.~\ref{fig_symmetric_tunneling_overview}) Charge stability with $\alpha=0$, for symmetrically varied $v_{L/R}$ (c.f.~Fig.~\ref{fig_symmetric_tunneling})
	\item (Fig.~\ref{fig_asymmetric_tunneling_overview}) Charge stability with $\alpha=0$, for asymmetrically varied $v_{L/R}$
	\item (Fig.~\ref{fig_tunability_overview}) Coupling terms required to tune the DIQ to a desired frequency, for a larger set of qubit frequencies, as well as the corresponding $T_2^*$ for chemical-potential-only noise, as well for a combination of chemical-potential and coupling noise. $T_2^*$ vas estimated based on 100 qubit samples per pixel, compared to 500 samples in the main text, resulting in increased sampling noise. (c.f.~Fig.~\ref{fig_tunability})
\end{itemize}

\begin{figure*}[tb]
	\includegraphics[scale=1]{fig_no_tunneling_overview}
	\caption{Charge stability diagrams of the floating dot-island-dot, with a total charge fixed ($\ntot=0$). Columns illustrate changes of $U_S$, and rows illustrate simultaneous changes of $\ulr$. A panel in the top right corner presents a color coding of the charge stability regions. The figure illustrates a range of charging energies extended relative to Fig.~2..}
	\label{fig_no_tunneling_overview}
\end{figure*}

\begin{figure*}[tb]
	\includegraphics[scale=1]{fig_symmetric_tunneling_overview}\\
	\caption{Quantum capacitance on of the dot-island-dot with respect to dot plunger gate voltages in total even and odd occupancy, and in a range of symmetrically varied island-impurity tunnel couplings. Blue and orange color maps represent the quantum capacitance of the left and right impurity. The black color maps are composite of the two. Fourth row corresponds to the data presented in Fig.~3}
	\label{fig_symmetric_tunneling_overview}
\end{figure*}

\begin{figure*}[tb]
	\includegraphics[scale=1]{fig_asymmetric_tunneling_overview}\\
	\caption{Quantum capacitance on of the dot-island-dot with respect to dot plunger gate voltages in total even and odd occupancy, and in a range anti-symmetrically varied island-impurity tunnel couplings. Blue and orange color maps represent the quantum capacitance of the left and right impurity. The black color maps are composite of the two. Fourth row corresponds to the data presented in Fig.~3}
	\label{fig_asymmetric_tunneling_overview}
\end{figure*}

\begin{figure*}[tb]
	\includegraphics[scale=1]{fig_tunability_overview}
	\caption{Coupling between the impurities required to achieve the splitting between the qubit states, and the estimated inhomogenous dephasing times for the two nose models. The columns 1 and 4 present the same parameter values as Fig.~7, albeit for fewer realizations of the noise.}
	\label{fig_tunability_overview}
\end{figure*}

\bibliography{biblio}

\end{document}
