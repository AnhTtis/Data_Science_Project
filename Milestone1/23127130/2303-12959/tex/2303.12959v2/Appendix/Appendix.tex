

\section{Appendix}

\subsection{Architecture}\label{sec:arch}

\begin{table*}
\begin{center}
\begin{adjustbox}{width=\linewidth}


  \begin{tabular}{|l|l|l|l|}
    \hline
  
    Module & Input $\rightarrow$ Output & Layer  \\\Xhline{2.5\arrayrulewidth}
    \multirow{3}{*}{Target Attributes Encoder $E_{t}$} & \multirow{2}{*}{$I_{tgt}^{\downarrow}$ $\rightarrow$ $\mathbf{F}_{32 \times 32}$} & Conv(3,1,1) $\rightarrow$ LeakyReLU $\rightarrow$ Conv(3,1,1) $\rightarrow$ LeakyReLU $\rightarrow$ \\
                                     &                                                                  &$\{$ Conv(3,2,1) $\rightarrow$ LeakyReLU $\rightarrow$ Conv(3,1,1) $\rightarrow$ LeakyReLU $\}$ $\times$4 \\
                                     & $\mathbf{F}_{32 \times 32} \rightarrow \mathbf{I}_{32 \times 32}$& Conv(1,1,1) \\\hline
    Source Identity Encoder $E_{i}$  & $I_{src}^{\downarrow}$ $\rightarrow$ $\textbf{w}_{id}^{+}$       & pSp~\cite{psp} Encoder    \\\hline
    Source Shape Encoder $E_{s}$     & $I_{src}^{\downarrow}$ $\rightarrow$ $\alpha$                    & DECA~\cite{deca} Encoder    \\\hline  
    Mapper $M$                       & $\alpha$ $\rightarrow$ $\mathbf{w}_{shape}^{+}$                  & 5 EqualLinear Layers with LeakyReLU \\\hline  
    \multirow{2}{*}{$h \times w$ Block}               & $\mathbf{F}_{h/2 \times w/2}$, $\{\textbf{w}_{i}, \textbf{w}_{i+1}\}$ $\rightarrow$ $\mathbf{F}_{h \times w}$        & Upsample $\rightarrow$ StyleConv $\rightarrow$ NoiseInjection $\rightarrow$ StyleConv $\rightarrow$ NoiseInjection \\
                                     & $\mathbf{F}_{h \times w}$, $\mathbf{I}_{h/2 \times w/2}$, $\textbf{w}_{i+2}$ $\rightarrow$ $\mathbf{I}_{h \times w}$ & ToRGB \\ \hline
    
    \multirow{5}{*}{Generator $G$} 
    & $\mathbf{F}_{32 \times 32}$, $\mathbf{I}_{32 \times 32}$,     $\{\textbf{w}_{8}, \textbf{w}_{9}, \textbf{w}_{10}\}$ $\rightarrow$     $\mathbf{F}_{64 \times 64}$, $\mathbf{I}_{64 \times 64}$     & 64  $\times$ 64 Block \\
    & $\mathbf{F}_{64 \times 64}$, $\mathbf{I}_{64 \times 64}$,     $\{\textbf{w}_{10}, \textbf{w}_{11}, \textbf{w}_{12}\}$ $\rightarrow$   $\mathbf{F}_{128 \times 128}$, $\mathbf{I}_{128 \times 128}$ & 128 $\times$ 128 Block \\
    & $\mathbf{F}_{128 \times 128}$, $\mathbf{I}_{128 \times 128}$, $\{\textbf{w}_{12}, \textbf{w}_{13}, \textbf{w}_{14}\}$ $\rightarrow$   $\mathbf{F}_{256 \times 256}$, $\mathbf{I}_{256 \times 256}$ & 256 $\times$ 256 Block \\
    & $\mathbf{F}_{256 \times 256}$, $\mathbf{I}_{256 \times 256}$, $\{\textbf{w}_{14}, \textbf{w}_{15}, \textbf{w}_{16}\}$ $\rightarrow$   $\mathbf{F}_{512 \times 512}$, $\mathbf{I}_{512 \times 512}$ & 512 $\times$ 512 Block \\
    & $\mathbf{F}_{512 \times 512}$, $\mathbf{I}_{512 \times 512}$, $\{\textbf{w}_{16}, \textbf{w}_{17}, \textbf{w}_{18}\}$ $\rightarrow$   $\mathbf{F}_{1024 \times 1024}$, $\hat{I}$                   & 1024 $\times$ 1024 Block \\


    \hline
  \end{tabular}
\end{adjustbox}
\end{center}
\caption{Architecture details of \textbf{\ourmodel}. The target attributes encoder $E_{t}$ maps the spatial information of the down-sampled target image $I_{tgt}^{\downarrow} \in \mathbb{R}^{256 \times 256 \times 3}$ to the feature map $\mathbf{F}_{32 \times 32} \in \mathbb{R}^{32 \times 32 \times 512}$ and the low resolution image $\mathbf{I}_{32 \times 32}  \in \mathbb{R}^{32 \times 32 \times 3}$. Here, Conv(k,s,p) denotes a 2D Convolutional layer with kernel size k, stride size s, and padding size p. The source identity encoder $E_{i}$ and, the source shape encoder and mapper ($E_{s}$, $M$) map the identity information of the down-sampled source image $I_{src}^{\downarrow} \in \mathbb{R}^{256 \times 256 \times 3}$ to $\mathbf{w}$ vectors, $\mathbf{w}_{id}^{+}$ and $\mathbf{w}_{shape}^{+}$, respectively. In $h \times w$ Block, StyleConv, NoiseInjection, and ToRGB are exactly the same as in StyleGAN~\cite{sg2}. The generator $G$ consists of multiple $h \times w$ Blocks to generate the swapped image $\hat{I} \in \mathbb{R}^{1024 \times 1024 \times 3}$. } % todo : detail notation 설명 추가할것
\label{supp_table:architecture}

\end{table*}
% label은 캡션뒤로 하시오

% https://arxiv.org/pdf/2301.02379.pdf 참고
% Module multi row
We use symmetric convolutional networks for encoders and decoders as shown in Table~\ref{tab:architecture}. $c=1$ for dSprites, and $c=3$ for Shapes3D. All layers are activated by ReLU. The final layer of encoder generates 10 variables for \textit{mean} and 10 variables for the \textit{logvar}. 


\subsection{Disentanglement-invariant Representations}
\label{sec:proof}
In this section, we prove the proposed disentanglement-invariant transformation.
Consider that we have a new representation by multiplying a diagonal matrix: $\vect{z}' = \vect{w} \vect{z}$, $\vect{w}$.
We can calculate the Covariance between any two latent variables:
\begin{equation}
\begin{aligned}
\operatorname{Cov}(\vect{w}_i \vect{z}_i, \vect{w}_j \vect{z}_j) &=
\mathbb{E}[(\vect{w}_i \vect{z}_i-\mathbb{E}[\vect{w}_i \vect{z}_i])(\vect{w}_j \vect{z}_j-\mathbb{E}[\vect{w}_j \vect{z}_j])] \\
&=\vect{w}_i \vect{w}_j (\mathbb{E}[\vect{z}_j]-\mathbb{E}[ \vect{z}_i] \mathbb{E}[\vect{z}_j]) \\
&=\vect{w}_i \vect{w}_j \operatorname{Cov}(\vect{z}_i,\vect{z}_j),
\end{aligned}
\end{equation}
where the subscript denotes the index of latent variables.
Note that we use a different notion in this section to simplify the formula.

Then we can get the correlation coefficient by
\begin{equation}
\begin{aligned}
    \rho(\vect{w}_i \vect{z}_i, \vect{w}_j \vect{z}_j)&=\frac{\operatorname{Cov}(\vect{w}_i \vect{z}_i, \vect{w}_j \vect{z}_j)}{\sqrt{\operatorname{Var}[\vect{w}_i \vect{z}_i] \operatorname{Var}[\vect{w}_j \vect{z}_j]}}\\
    &= \rho( \vect{z}_i, \vect{z}_j).
\end{aligned}
\end{equation}

Therefore, the correlation matrix will not change by multiplying a diagonal matrix $w, w\neq0$.
The proposed transformation is disentanglement-invariant.

\subsection{ Estimation of $I(\vect{z}_j;\vect{c}_i)$}


Given an inference network $q(\vect{z}|\vect{x})$, we use the Markov chain Monte Carlo (MCMC) method to get samples from $q(\vect{z})$ by the formula $q(\vect{z}) = q(\vect{z|x})p(\vect{x})$.
We use 10, 000 points to estimate $q(\vect{z})$.
Then, we discretize these latent variables by a histogram with 20 bins. 
After discretizing one latent variable, we call a discrete mutual information estimation algorithm to calculate $I(\vect{w}_j \vect{z}_j;\vect{c}_i)$ by a 2D histogram.


\begin{figure}[t]
    \centering
    \includegraphics[width=0.95\textwidth]{pics/SamplingFromNoise.pdf}
    \caption{Reconstruction from noise.}
    \label{fig:sampling}
\end{figure}

\subsection{Visualization}\label{sec:visualization}


% 0 - floor color (10 different values)
% 1 - wall color (10 different values)
% 2 - object color (10 different values)
% 3 - object size (8 different values)
% 4 - object type (4 different values)
% 5 - azimuth (15 different values)
\begin{figure}[t]
    \centering
    \includegraphics[width=\linewidth]{pics/traversal_shapes3d.pdf}    
    \caption[]{Latent traversal on Shapes3D. ''back.`` denotes background color, ``floor'' denotes floor color, ``obj.'' denotes object, and ``wall'' denotes wall color.}
    \label{fig:traversal_shapes}
\end{figure}

\begin{figure}[t]
    \centering
    \includegraphics[width=\linewidth]{pics/traversal_dsprites.pdf}    
    \caption[]{Latent traversal on dSprites.}
    \label{fig:traversal_dsprites}
\end{figure}

\noindent{}\textbf{Latent Traversal.} \quad
We compare DeVAE to others with latent traversals on Shapes3D and dSprites.
Each column shows the generated images by traversing one latent variable from -2 to 2.
From Figure~\ref{fig:traversal_shapes} and Figure~\ref{fig:traversal_dsprites}, we can see that DeVAE has a lower entanglement level. Note that only DeVAE disentangles object size isolated on Shapes3D.


\noindent{}\textbf{Random Sampling.} \quad
We random sample noise from Guassian distribution $\mathcal{N}(0,1)$ and generate images from our disentanglement model trained on dSprites.
As shown in Figure~\ref{fig:sampling}, our model, generating heart, has a high reconstruction fidelity

% 
\begin{figure}
    \centering
    \includegraphics[width=\linewidth]{pics/quantitative.pdf}
    \caption{Box plots of quantitative benchmarks MIG, FactorVAE, Disentanglement, and reconstruction error on dSprites and Shapes3D.}\label{fig:quantitative}
\end{figure}

% \textbf{}



\subsection{CelebA}

We further conduct experiments on a real dataset CelebA~\cite{liu2015faceattributes}.
% Please add the following required packages to your document preamble:
% \usepackage{booktabs}
