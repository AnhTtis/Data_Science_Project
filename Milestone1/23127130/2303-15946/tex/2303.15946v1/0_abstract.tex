\begin{abstract}
%Graph Convolutional Networks (GCN) are state-of-the-art representation learning models that build rich node embeddings by exploiting the higher-order associations between nodes in a graph. 
Graph Convolutional Networks (GCN) have been recently employed as core component in the construction of recommender system algorithms, interpreting user-item interactions as the edges of a bipartite graph.
However, in the absence of \textit{side information}, the majority of existing models adopt an approach of randomly initialising the user embeddings and optimising them throughout the training process.
This strategy makes these algorithms inherently \textit{transductive}, curtailing their ability to generate predictions for users that were unseen at training time. 
%Addressing this user cold-start problem is vital to the success of real-world recommendation models. 
To address this issue, we propose a convolution-based algorithm, which is \textit{inductive} from the user perspective, while at the same time, depending only on implicit user-item interaction data.
We propose the construction of an item-item graph through a weighted projection of the bipartite interaction network and to employ convolution to inject higher order associations into item embeddings, while constructing user representations as weighted sums of the items with which they have interacted.
%Subsequently rather than training individual embeddings for each user, we project them into the item embedding space, as weighted sums of the items with which they have interacted.
Despite not training individual embeddings for each user our approach achieves state-of-the-art recommendation performance with respect to \textit{transductive} baselines on four real-world datasets, showing at the same time robust inductive performance. 
%We empirically demonstrate the effectiveness of our algorithm through extensive experiments on four real-world large-scale datasets. 
%Our ablation study highlights the unique contributions of the various components of the presented algorithm. 
%\footnote{\label{code_rep}\url{https://github.com/doubleblind148/IGCCF}}
\end{abstract}




%Thanks to their ability of exploiting higher-order  associations between nodes, Graph Convolutional Networks (GCN) have been recently employed as a core component in the construction of recommender system algorithms, interpreting user-item interactions as edges of a bipartite network.
%  %Recently Graph Convolutional Networks (GCN), thanks to their ability of exploiting higher-order associations between nodes in a graph, have been employed as a core component in the construction of recommender system algorithms, interpreting  user-item interactions as the edges of a bipartite network. 
% However, in the absence of \textit{side information}, the majority of existing models adopt an approach of randomly initialising the user embeddings and optimising them throughout the training process.
% This strategy makes these algorithms inherently \textit{transductive}, curtailing their ability to generate predictions for users that were unseen at training time. 
% %Addressing this user cold-start problem is vital to the success of real-world recommendation models. 
% To address this issue, we propose a convolution-based algorithm, which is \textit{inductive} from the user perspective, while at the same time, depending only on implicit user-item interaction data.
% We propose the construction of an item-item graph through a weighted projection of the bipartite interaction network and to employ convolution to inject higher order associations into item embeddings, while constructing user representations as weighted sums of the items with which they have interacted.
% %Subsequently rather than training individual embeddings for each user, we project them into the item embedding space, as weighted sums of the items with which they have interacted.
% Despite not training individual embeddings for each user our approach achieves state-of-the-art recommendation performance with respect to \textit{transductive} baselines on four large-scale datasets, showing at the same time robust inductive performance. 
% %We empirically demonstrate the effectiveness of our algorithm through extensive experiments on four real-world large-scale datasets. 
% %Our ablation study highlights the unique contributions of the various components of the presented algorithm. 
% The code used for this work is available on github.com\footnote{\url{add url}}
% \end{abstract}