\section{Introduction}
% The steady increase in the amount of information available on the online has created an information overload problem for end users of web-based applications. Personalised recommendations to assist customers in making better and faster decisions is now more than ever a vital requirement for many online applications ranging from \textit{e}-commerce to streaming platforms.
% The notion of \textit{collaborative filtering} is commonly used in recommender system models (CF). The core premise behind collaborative filtering is that observed interactions between users and objects are frequently strongly linked across different users and items. In more formal terms, if a user $a$ has the same opinion as a user $b$ on a product, $a$ is more likely to share $b$'s view on another product than a randomly picked user.
% These enable the utilisation of prior user and item data to learn about the users' future preferences.
% \textit{Model-based} CF algorithms operates by learning representations (a.k.a embeddings) for the users and items of the system in a low-dimensional space usually referred to as latent-space. The learnt embeddings are then used to estimate the user preferences towards items.
Recent years have witnessed the success of Graph Convolutional Networks based algorithm in many domains, such as social networks \cite{kipf2016semisupervised,chen2018fastgcn}, natural language processing \cite{yao2019graph} and computer vision \cite{wang2018zero}. The core component of Graph Convolutional Networks algorithms is the iterative process of aggregating information mined from node
neighborhoods, with the intent of capturing high-order associations between nodes in a graph. 
%This popularity is fuelled largely by GCN's success in achieving \textit{state-of-the-art} performance on various machine learning tasks, and this success has seen researchers apply GCN's to the recommender system problem as in He \etal \cite{He_2020}. The core idea behind GCNs is to aggregate node features from local graph neighborhoods to achieve richer node representations that can subsequently be used for several machine learning tasks.
GCNs have opened a new perspective for recommender systems in light of the fact that user-item interactions can be interpreted as the edges of a bipartite graph \cite{wang2019neural,Chen_2020,He_2020}. 
Real-world recommender system scenarios must contend with the issue that user-item graphs change dynamically over time. New users join the system on a daily basis, and existing users can produce additional knowledge by engaging with new products (introducing new edges in the user-item interaction graph). The capacity to accommodate new users to the system — those who were not present during training — and fast leverage novel user-item interactions is a highly desirable characteristic for recommender systems meant to used in real-world context. Delivering high quality recommendations 
under these circumstances poses a severe problem for many existing \textit{transductive} recommender system algorithms. Models such as \cite{wang2019neural,Chen_2020,He_2020} need to be completely re-trained to produce the embedding for a new user that joins the system post-training and the same happens when new user-item interactions must be considered; this limitation restricts their use in real-world circumstances. \cite{localfactormodels}.
% One solution present in literature is to resort to \emph{inductive} algorithms, capable of exploiting side information beyond the pure user-item interactions. Most often the objective is to learn a mapping function from user and item features to embeddings. While for item is very common to have rich side information such as photos, description or tags, extra information from the user-side is very hard to find. In some cases, the pre-existing user metadata (e.g. age, gender, demographic data) are used as features to seed the user profile , or explicit user signals such as ratings are also used\cite{zhang2019inductive,jain2013provable, pmlr-v80-hartford18a}. 



One solution present in literature, is to leverage side information (user and item metadata) beyond the pure user-item interactions in order to learn a mapping function from user and item features to embeddings \cite{volkovs2017dropoutnet,zhang2019inductive,jain2013provable,pmlr-v80-hartford18a}.  However, it can be difficult to obtain this additional side information in many real-world scenarios, as it may be hard to extract, unreliable, or simply unavailable. For example, when new users join a system, there may be very little or no information available about them, making it difficult or impossible to generate their embeddings. Even when it is possible to gather some information about these users, it may not be useful in inferring their preferences. Another way to account for new users and rapidly create embeddings which exploit new user-item interactions is to resort to \textit{item-based} models \cite{Cremonesi2010Performance,kabbur2013fism}. In this setting only the item representations are learnt and then exploited to build the user embeddings. Anyway these category of models do not directly exploit the extra source of information present in the user-item interaction graph, which have been shown to benefit the performance of the final model. Furthermore the application of standard Graph Convolution methods recently presented for the collaborative filtering problem have not been extended to work in a setting where only the item representations are learnt.
%Most of the existing \emph{inductive} methods tackle the problem of dynamic graphs by leveraging additional side information beyond the pure user-item interactions. In some cases, the pre-existing user metadata (e.g. age, gender, demographic data) are used as features to seed the user profile \cite{volkovs2017dropoutnet}, or explicit user signals such as ratings are also used\cite{zhang2019inductive,jain2013provable, pmlr-v80-hartford18a}. Unfortunately, in many real-world scenarios, such additional information is often hard to extract and can be unreliable, and in many cases it may be simply unavailable.

In this paper we propose a novel item-based model named Item Graph Convolutional Collaborative Filtering (IGCCF), capable of handling dynamic graphs while also leveraging the information contained in the user-item graph through graph convolution. It is designed to learn rich item embeddings capturing the higher-order relationships existing among them. To extract information from the user-item graph we propose the construction of an item-item graph through a weighted projection of the bipartite network associated to the user-item interactions with the intent of mining high-order associations between items. We then construct the user representations as a weighted combination of the item embeddings with which they have previously interacted, in this way we remove the necessity for the model to learn static one-hot embeddings for users, reducing the space complexity of previously introduced GCN-based models and, at the same time, unlocking the ability to handle dynamic graphs, making straightforward the creation of the embeddings for new users that join the system post training as well as the ability of updating them when new user-item interactions have been gathered, all of that without the need of an expensive retraining procedure.

% In summary, we make the following contributions:
% \begin{itemize}
% \item We introduce a novel approach to mine item relationships via graph convolution, based on the construction of an item-item graph from the user-item interaction network.
% \item We introduce an item-based recommendation model capable of handling dynamic graphs in the extreme setting where no side information is available.
% \item We show empirically that IGCCF achieves \textit{state-of-the-art} performance with respect to transductive convolutional collaborative filtering recommender systems without the need of directly parametrise the user representations.
% \end{itemize}

\begin{comment}
while at the same time being ca between items while at the same timelearn item embeddings through graph 

Of direct interest to us is to leverage the ability of GCN's to create richer node embeddings while at the same time, in the extreme setting where only implicit user-item interactions are available, achieve an inductive behaviour for new users without having a degradation of recommendation performance. This would be a crucial step to make GCN algorithms applicable in real-context where the only source of information are implicit user feedbacks. 
%We are interested in efficiently applying GCNs to address the user cold-start problem without compromising on recommendation quality.

In this paper we propose a novel procedure to learn item embeddings accounting for higher-order associations existing among them. Through a weighted projection of the user-item interaction graph we build an item-item graph on top of which we apply convolution to learn item embeddings. Instead of directly map every user-id to a fixed representations we propose to map them into the item space through a weighted combination of the items with which they have interacted. We furthermore introduce a regularisation with the aim of enhancing the performance on user with few interactions.
Instead of directly map every user to its own embedding we propose to compute

Thus, in this paper, we introduce a new framework called \textit{Item Graph Convolutional Collaborative Filtering} (\textit{IGCCF}), which achieves \textit{state-of-the-art} performance with respect to transductive graph convolutional models while at the same time showing robust user inductive performance with respect to the training data available.
To achieve that we introduce a novel


is inductive from the user side showing and demonstrates excellent performance in the most extreme situation where no side information is available for the users.
Compared to fully transductive models, \textit{UI-GCCF} addresses the user cold-start problem, showing robust results even when trained with a very small fraction of the training data  achieving \textit{state-of-the-art} performance on four real world datasets. In addition, \textit{UI-GCCF} is space-efficient ($O(|Items|)$) and maintains high recommendation performance on unseen users, even when trained with a very small fraction of the training data.

%\edoinline{Clearly state our contribution, rephrase under points in introduction}
%Summarizing our contribution:
In summary, we make the following contributions in this work:

\begin{itemize}[leftmargin=*]
\item We introduce a recommendation model with robust \textit{user-inductive} performance with respect to the available training data, in the extreme setting where there is no side information available.
\item We develop a model capable of a \textit{real-time} user-profile update, without any training process involved.
\item We introduce a \textit{weighting mechanism} over the user-item interactions, to allow fine-grained control over the importance of the interactions.
\item We show empirically that \textit{UI-GCCF} is achieving comparable performance to \textit{state-of-the-art} transductive convolutional collaborative filtering recommender systems.
\end{itemize}
\end{comment}






