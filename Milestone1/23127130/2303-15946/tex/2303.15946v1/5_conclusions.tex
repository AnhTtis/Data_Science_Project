\section{Conclusion and Future work}
In this work we presented IGCCF, an item-based model that employs graph convolution to learn refined item embeddings. We build upon the previously presented graph convolution models by removing the explicit parameterisation of users. The benefits of that are threefold: first, it reduces model complexity; second, it allows real-time user embeddings updates as soon as new interactions are gathered; and third, it enables inductive recommendations for new users who join the system post-training without the need for a new expensive training procedure. To do this, we devised a novel procedure that first constructs an item-item graph from the user-item bipartite network. A top-K pruning procedure is then employed to refine it, retaining only the most informative edges. Finally, during the representation learning phase, we mine item associations using graph convolution, building user embeddings as a weighted combination of items with which they have interacted. In the future, we will extend the provided methodology to operate in settings where item side-information are available.
%As future work we will explore the impact of weighting mechanisms on the user embedding module, as well as the different item embedding modules in the context in which item features are available, to compare the extension of the proposed methodology against models which are inductive for both users and items.
%TODO - what does this mean? 'fully inductive models'? user + item inductive?
%In this work we have shown that through the use of GCNs, we can achieve \textit{state-of-the-art} performance without the need for fixed user embeddings. This is beneficial to the space and time complexity of the algorithm, while also enabling the model to be inductive with respect to unseen users, which is a key feature for a recommender system algorithm.
%It is possible to model a user based on the weighted sum of the item embeddings with which he/she has interacted, and still obtain high-quality representations.
%This opens up different directions for future work, ranging from the exploration of different weighting mechanisms for the user interactions, such as applying weightings based on time, which are essential for session-based recommendation, as well as the possibility of creating multiple embeddings for the same user, using only a subset of his profile to enhance the diversity of generated recommendations. We have also shown empirically that the smoothing process derived from the convolution operation is more beneficial as the sparsity of the dataset increases. 