%%%%%%%%%%%%%%%%%%%%%%%%%%%%%%%%%%%%%%%%%%%%%%%%%%%%%%%%%%%%%%%%%%%%%%%%%%%%%%%%
%2345678901234567890123456789012345678901234567890123456789012345678901234567890
%        1         2         3         4         5         6         7         8

\documentclass[letterpaper, 10 pt, conference]{ieeeconf}  % Comment this line out if you need a4paper

%\documentclass[a4paper, 10pt, conference]{ieeeconf}      % Use this line for a4 paper

\IEEEoverridecommandlockouts                              % This command is only needed if 
                                                          % you want to use the \thanks command

\overrideIEEEmargins                                      % Needed to meet printer requirements.

%In case you encounter the following error:
%Error 1010 The PDF file may be corrupt (unable to open PDF file) OR
%Error 1000 An error occurred while parsing a contents stream. Unable to analyze the PDF file.
%This is a known problem with pdfLaTeX conversion filter. The file cannot be opened with acrobat reader
%Please use one of the alternatives below to circumvent this error by uncommenting one or the other
%\pdfobjcompresslevel=0
%\pdfminorversion=4

% See the \addtolength command later in the file to balance the column lengths
% on the last page of the document

% The following packages can be found on http:\\www.ctan.org
\usepackage[nolist]{acronym} % acronyms by the \ac{label} command
\usepackage{amsfonts,siunitx,booktabs,cite} % general useful stuff
\usepackage{graphics} % for pdf, bitmapped graphics files
\usepackage{epsfig} % for postscript graphics files
\usepackage{mathptmx} % assumes new font selection scheme installed
\usepackage{times} % assumes new font selection scheme installed
\usepackage{amsmath} % assumes amsmath package installed
\usepackage{amssymb}  % assumes amsmath package installed
\usepackage{stmaryrd}

\usepackage{graphicx}
\usepackage{caption}
\let\labelindent\relax
\usepackage{enumitem}
\usepackage{svg}
\usepackage{float}
\usepackage[normalem]{ulem}
\usepackage{hyperref}
% \usepackage[linesnumbered,ruled]{algorithm2e}
\usepackage{algorithm}
\usepackage{algpseudocode}
\newcommand{\ZH}[1]{\textbf{\color{red}#1}}
% \newcommand{\zh}[1]{{\color{red} #1}}
\usepackage{scalerel}
\usepackage{tikz,standalone,pgfplots}
%\usepackage{txfonts}
\newcommand\Tau{\scalerel*{\tau}{T}}
\usepackage{soul}
\begin{acronym}
    \acro{LDA}{\emph{Latent Dirichlet Allocation}}
    \acro{CMT}{\emph{Conference Management Toolkit}}
    \acro{TPMS}{\emph{The Toronto Paper Matching System}}
    \acro{MCMF}{\emph{MinCost-MaxFlow}}
\end{acronym}
\definecolor{purple}{rgb}{1, 0, 1}

\newcommand{\ie}{\emph{i.e.,}\xspace}
\newcommand{\eg}{\emph{e.g.,}\xspace}
\newcommand{\abr}{\emph{abbr.}\xspace}
\newcommand{\ea}{\emph{et al.}\xspace}
\newcommand{\gensync}{\emph{GenSync}\xspace}
\newcommand{\colosseum}{\emph{Colosseum}\xspace}
\newcommand{\srep}{\emph{SREP}\xspace} % Set Reconciliation Enhances
\newcommand{\srepsim}{\emph{SREPSim}\xspace}
% Propagation
\newcommand{\esrep}{\emph{E-SREP}\xspace}
\newcommand{\epsrep}{\emph{EP-SREP}\xspace}
\newcommand{\mesrep}{\emph{ME-SREP}\xspace}
\newcommand{\mempoolsync}{\emph{MempoolSync}}

\newcommand{\fref}[1]{Fig.~\ref{#1}}
\newcommand{\tref}[1]{Table~\ref{#1}}
\newcommand{\aref}[1]{Algorithm~\ref{#1}}
\newcommand{\procref}[1]{Procedure~\ref{#1}}
\newcommand{\sref}[1]{Section~\ref{#1}}
\newcommand{\lineref}[1]{line~\ref{#1}}
\newcommand{\appref}[1]{Appendix~\ref{#1}}

% Change \eqref
\LetLtxMacro{\originaleqref}{\eqref}
\renewcommand{\eqref}{Eq.~\originaleqref}

% Theorems and corollaries
\newcounter{theoremcount}
\setcounter{theoremcount}{0}
\DeclareRobustCommand{\theorem}[1]{%
  \refstepcounter{theoremcount}%
  \noindent\textit{\textbf{Theorem \thetheoremcount\label{theorem:#1}: }}%
}
\DeclareRobustCommand{\theoremref}[1]{Theorem~\ref{theorem:#1}}

\DeclareRobustCommand{\proof}{\emph{Proof:}\xspace}
\DeclareRobustCommand{\qqed}{\hfill$\blacksquare$}

\newcounter{corollcount}
\setcounter{corollcount}{0}
\DeclareRobustCommand{\coroll}[1]{%
  \refstepcounter{corollcount}%
  \noindent\textit{\textbf{Corollary \thecorollcount\label{coroll:#1}: }}%
}
\DeclareRobustCommand{\corollref}[1]{Corollary~\ref{coroll:#1}}

\newcounter{lemmacount}
\setcounter{lemmacount}{0}
\DeclareRobustCommand{\lemma}[1]{%
  \refstepcounter{lemmacount}%
  \noindent\textit{\textbf{Lemma \thelemmacount\label{lemma:#1}: }}%
}
\DeclareRobustCommand{\lemmaref}[1]{Lemma~\ref{lemma:#1}}

\newcounter{definitioncount}
\setcounter{definitioncount}{0}
\DeclareRobustCommand{\definition}[1]{%
  \refstepcounter{definitioncount}%
  \noindent\textit{\textbf{Definition \thedefinitioncount\label{definition:#1}: }}%
}
\DeclareRobustCommand{\defref}[1]{Definition~\ref{definition:#1}}

%notes of different authors
\newif\ifnotes
\notestrue
\notesfalse

\newif\ifdiff
\difftrue
\difffalse

\newcommand{\anote}[1]{\ifnotes $\ll$\textsf{\textcolor{purple}{Ari: {#1}}}$\gg$ \fi}
\newcommand{\nnote}[1]{\ifnotes $\ll$\textsf{\textcolor{orange}{Novak: {#1}}}$\gg$ \fi}
\newcommand{\diff}[1]{\ifdiff\textcolor{orange}{#1}\else#1\fi}

%%% Local Variables:
%%% mode: latex
%%% TeX-master: "main"
%%% End:



    
\title{\LARGE \bf
\methodname{}: An Approach-Constrained Generative Grasp Sampling Network
}

% Generative Sampling of Direction-Specific Grasps


\author{Zehang Weng, Haofei Lu, Jens Lundell, and Danica Kragic% <-this % stops a space
%\thanks{*This work was not supported by any organization}% <-this % stops a space
\thanks{All authors are with the division of Robotics, Perception, and Learning (RPL) at KTH, Stockholm, Sweden. 
        {\tt\small \{zehang,haofeil,jelundel,dani\}@kth.se}}%
}


% \begin{figure*}[htb]
%     \centering
%         \includegraphics[width=1\linewidth]{figures/shelf_procedure_with_grasps.png}
%     \caption{Real robot experiment of shelf object picking (with grasps).}
%     \label{fig:shelf_exp_with_grasps}
% \end{figure*}

\makeatletter
\let\@oldmaketitle\@maketitle% Store \@maketitle
\renewcommand{\@maketitle}{\@oldmaketitle% Update \@maketitle to insert...
  \setcounter{figure}{0}
  \vspace{1em}
  \centering
      \includegraphics[width=\linewidth]{figures/shelf_procedure_with_grasps.png}
      \captionof{figure}{The shelf-picking setup. Left: The green areas represent collision-free grasp directions. Middle: the grasps generated by our \methodname{} with the best grasp highlighted in red. Right: the robot successfully reaches the red grasp pose. \label{fig:shelf_exp_with_grasps}}
  }
\makeatother


\begin{document}
\maketitle
\thispagestyle{empty}
\pagestyle{empty}
%%%%%%%%%%%%%%%%%%%%%%%%%%%%%%%%%%%%%%%%%%%%%%%%%%%%%%%%%%%%%%%%%%%%%%%%%%%%%%%%
\begin{abstract}

%Grasping from confined spaces where the environment poses constraint on the possible grasps is challenging...
This work addresses the problem of learning approach-constrained data-driven grasp samplers. To this end, we propose \methodname{}: a generative grasp sampler that can constrain the grasp approach direction to a subset of $\mathrm{SO}(3)$. The key insight is to discretize $\mathrm{SO}(3)$ into a predefined number of bins and train \methodname{} to generate grasps whose approach directions are within those bins.  At run-time, the bin aligning with the second largest principal component of the observed \pc{} is selected. \methodname{} is benchmarked against \graspnet{}, a \sota{} unconstrained grasp sampler, in an unconfined grasping experiment in simulation and on an unconfined and confined grasping experiment in the real world. The results demonstrate that \methodname{} achieves higher success-over-coverage in simulation and a 12\%--18\% higher success rate in real-world table-picking and shelf-picking tasks than the baseline.


% Supplementary material is available at \url{https://sites.google.com/view/kp4dom}
\end{abstract}

% \begin{figure}[t]
%     % \begin{subfigure}{1\linewidth}
%     %   \centering
%     % %   \includegraphics[width=1\linewidth]{figs/fig_1_moti_textattn.pdf}  
%     % %   \includegraphics[width=1\linewidth]{figs/fig_1_moti_textattn_v2.pdf}  
%     %   \includegraphics[width=1\linewidth]{figs/fig_1_moti_textattn_v5.pdf}  
%     %   \vspace{-0.5cm}
%     %     \caption{Amount of attention added to each video clip from the source video and query text in the self-attention layers of Moment-DETR encoder.}
%     %     % \caption{Distribution of attention for source and query in Moment-DETR encoder}
%     %     % Visualization of video clip's self-attention score in Moment-DETR encoder.
%     %   \label{fig:fig1_text_attn_ex}
%     % \end{subfigure}%\hfill% or  or \hspace{0.3\textwidth}
%     \vspace{0.2cm}
%     % \begin{subfigure}{1\linewidth}
%       \centering
%     %   \includegraphics[width=1\linewidth]{figs/fig1_moti_negattn.pdf}  
%       \includegraphics[width=1\linewidth]{figs/fig1_moti_negattn_v3.pdf}  
%       \vspace{-0.4cm}
%     %   \caption{Correspondence of saliency scores on the relevance between video clips and the text query.}
%     % \caption{Predicted saliency scores against the video relevant positive query and video irrelevant negative query}
%       \label{fig:fig1_neg_attn_ex}
%     % \end{subfigure}%\hfill% or  or \hspace{0.3\textwidth}
%     \caption{
%     % 원준 원본
%     % (a) Comparison between attention scores of source and query for each video clip~(We sum the attention scores from video and text). 
%     % We observe that the attention scores are dominated by other clips in the source video. 
%     % Text queries do not account for much attention regardless of the relevance to the video clips.
%     % \textbf{(a)} Inspection of the query dependency in Moment-DETR encoder.
%     % % We visualize the attention score of video tokens in the transformer encoder and observe that text query accounts for only a low portion of attention.
%     % % This tendency occurs regardless of the relevance between the text query and video clips. 
%     % We visualize the attention score of video tokens in the transformer encoder and observe 1) text query only accounts for a low portion of attention, and 2) relevance between video-query pair does not affect the attention scores ratio of text.
%     \textbf{(b)} Comparison of highlight-ness when relevant and non-relevant queries are input.
%     As observed in , existing work only uses queries to play an insignificant role, thereby may not be capable of detecting false queries and considering the video-query relevance even when the problem in (a) is resolved. 
%     % \SE{} % 이 부분이 "not capable of" 란 용어가 세다는 피드백이 있는 듯 합니다. 이러한 능력이 없다는 것은 굉장히 강한 어조인거 같기는 하고, 이러한 경우들이 종종 있다거나 좀 약화시킬 필요가 있어보이긴 하네요.
%     On the other hand, our QD-DETR yields a query-dependent representation that the relevance between the source video and query text is updated in the saliency scores.
%     There is a large gap between positive and negative saliency scores, and scores are consistent since the clips are all highly correlated to others.
%     }
%     \label{fig:motivation_ex}
%     % \captionsetup{belowskip=13pt}
%     % \setlength{\belowcaptionskip}{-10pt}
% \end{figure}
\begin{figure}
    \centering
    \includegraphics[width=1\linewidth]{figs/fig1_moti_negattn_1111.pdf}
    % \includegraphics[width=1\linewidth]{figs/fig1_moti_negattn_1109.pdf}
    % \includegraphics[width=1\linewidth]{figs/fig1_moti_negattn_stat.pdf}
    \vspace{-0.6cm}
    \caption{
        % \SE{} % 수정 필요
        Comparison of highlight-ness~(saliency score) when relevant and non-relevant queries are given.
        We found that the existing work only uses queries to play an insignificant role, thereby may not be capable of detecting negative queries and video-query relevance; saliency scores for clips in ground-truth~(GT) moments are low and equivalent for positive and negative queries.
        % This also results in mispredicted moments when ground-truth~(GT) moment is dominated by clips unrelated to GT since their prediction is highly focused on the video.
        % \SE{} % 여기 한번 더 보면 좋을 듯 합니다. GT moment에 unrelated한 clip이 많으면? label이 틀렷을 경우를 말씀하시는건지?
        % As observed in saliency graph, existing work only uses queries to play an insignificant role, thereby may not be capable of detecting false queries and considering the video-query relevance.
        On the other hand, query-dependent representations of QD-DETR result in corresponding saliency scores to the video-query relevance and precisely localized moments.
        % On the other hand, our QD-DETR yields a query-dependent representation that the
        % saliency scores are in accordance with the relevance between the video and query.
        % text is in accordance with the saliency scores.
        % There is a large gap between positive and negative saliency scores, and scores are consistent since the clips are all highly correlated to others.
}
    \label{fig:motivation_ex}
\end{figure}


\section{Introduction}
% 원준 원본
% Along with the advance of digital devices and platforms, video is now one of the most desired data type for consumers. However, although the large information capacity of videos may be beneficial in many aspects, e.g., informative and entertaining, on the contrary perspective, videos are time-consuming, and hard to search for desirable moments. 
% This has led many creators to use extra manpower to crop and edit the video to generate highlight clips to gain the consumer’s attention.
Along with the advance of digital devices and platforms, video is now one of the most desired data types for consumers~\cite{apostolidis2021video,wu2017deep}.
% SE: Video aware deep learning application & survey papers?
Although the large information capacity of videos might be beneficial in many aspects, e.g., informative and entertaining, inspecting the videos is time-consuming, so that it is hard to capture the desired moments~\cite{anne2017localizing,apostolidis2021video}. 
% This has led many creators to use extra manpower to crop and edit the video to generate highlight clips to gain the consumer’s attention.


% On the other side, 
Indeed, the need to retrieve user-requested or highlight moments within videos is greatly raised.
Numerous research efforts were put into the search for the requested moments in the video~\cite{anne2017localizing, gao2017tall, liu2015multi, escorcia2019temporal} and summarizing the video highlights~\cite{zhang2016video, mahasseni2017unsupervised, badamdorj2022contrastive, wei2022learning}.
% Numerous research efforts were put into the search for the requested moments in the video~\cite{anne2017localizing, gao2017tall, liu2015multi, escorcia2019temporal}, summarizing the video to generate highlights was another popular topic~\cite{zhang2016video, mahasseni2017unsupervised, badamdorj2022contrastive, wei2022learning}.
Recently, Moment-DETR~\cite{momentdetr} further spotlighted the topic by proposing a QVHighlights dataset that enables the model to perform both tasks, retrieving the moments with their highlight-ness, simultaneously.

% 원준 원본
% To detect the desired moments, previous works employed transformer encoder-decoder architectural designs to fuse the text query into the video representations. Moment-DETR~\cite{mDETR} modified detection transformer to process capture the moment as a set, and UMT~\cite{umt} implemented transformer decoder as to output clip-wise saliency. 
% Yet to their outstanding breakthroughs in the literature of moment retrieval with the seminal architectures, their limitation is that the role of the given text query is insignificant in representing the query-conditioned video representation; the attention mechanism of moment DETR is not explicitly conditioned on the text query, and the text query is conditioned on multi-modal clips where the differences between the clips are smoothed after encoding process in UMT.



% \begin{figure}[t]
% \centering
%     \begin{subfigure}[l]{0.37\linewidth}
%       \centering
%       \vspace{0.20cm}
%     %   \includegraphics[width=1\linewidth]{figs/fig_1_moti_textattn.pdf}  
%     %   \includegraphics[width=1\linewidth]{figs/fig_1_moti_textattn_v2.pdf}  
%       \includegraphics[width=1\linewidth]{figs/fig1_moti_violin_a.pdf}  
%       \vspace{-0.60cm}
%     %   \caption{text attention}
%         \caption{Importance of queries in video representation}
%       \label{fig:fig1_text_attn}
%     \end{subfigure}%\hfill% or  or \hspace{0.3\textwidth}
%     \vspace{0.2cm}
%     \begin{subfigure}[r]{0.61\linewidth}
%       \centering
%     %   \includegraphics[width=1\linewidth]{figs/fig1_moti_negattn.pdf}  
%       \includegraphics[width=1\linewidth]{figs/fig1_moti_violin_b.pdf}  
%     %   \caption{neg attention}
%         % \caption{Relation between the highlight-ness and the relevance between videos and query texts.}
%         \caption{Highlight-ness~(saliency) histogram of positive and negative video-query pairs\SE{}}
%       \label{fig:fig1_neg_attn}
%     \end{subfigure}%\hfill% or  or \hspace{0.3\textwidth}
%     % \vspace{-0.2cm}
%     \caption{Overall statistics for attention scores in Fig.~\ref{fig:motivation_ex} in QVHighlights dataset. 
%     (a) For the attention scores that measure how much the text query is generally involved in video representation, we use violin plots to show the probability density. We plot the score for each layer in the encoder.
%     % (b) Using the histogram, we compare how the baseline and QD-DETR yield different salient scores given the positive and negative video-text pairs.
%     (b) Saliency histogram shows the distributional gap between positive and negative video-text query pairs of baseline~(Moment-DETR) and proposed QD-DETR.\SE{}
%     }
%     \label{fig:motivation}
%     % \captionsetup{belowskip=13pt}
%     % \setlength{\belowcaptionskip}{-10pt}
% \end{figure}

% \begin{figure}[t]
% \centering

%     \begin{subfigure}[r]{1\linewidth}
%       \centering
%       \hspace{-0.2cm}
%     %   \includegraphics[width=1\linewidth]{figs/fig1_moti_negattn.pdf}  
%       \includegraphics[width=1.1\linewidth]{figs/fig1_moti_violin_a_v2.pdf}  
%     %   \caption{neg attention}
%         % \caption{Relation between the highlight-ness and the relevance between videos and query texts.}
%         \vspace{-0.5cm}
%         % \caption{Saliency histogram of positive and negative video-query pairs}
%         \caption{We plot the histograms and its average value~(dotted line) to compare saliency scores when true and false text queries are given for each method. (left) Since the video representations do not include much textual information, both the true and false queries yield similar saliency scores. (Middle) Even when the video representation is enforced to be updated with the textual information, the issue is not much resolved. (Right) By extracting discriminative features in the text query, distributions are differentiated.
%         % \SE{} % R1@0.5 설명
%         Also, R1@0.5 indicates evaluation metric, Recall at 1 with IoU 0.5 threshold on QVhighlight \textit{val} set.
%         }
%       \label{fig:fig1_neg_attn}
%     \end{subfigure}%\hfill% or  or \hspace{0.3\textwidth}
%     \\
%     \begin{tabular}{cc}
%     \hspace{-0.2cm}
%         \begin{minipage}{.4\linewidth}
%             \begin{subfigure}[l]{1\linewidth}
%               \centering
%             %   \vspace{0.20cm}
%             %   \includegraphics[width=1\linewidth]{figs/fig_1_moti_textattn.pdf}  
%             %   \includegraphics[width=1\linewidth]{figs/fig_1_moti_textattn_v2.pdf}  
%               \includegraphics[width=1\linewidth]{figs/fig1_moti_violin_a.pdf}  
%               \vspace{-0.60cm}
%             %   \caption{text attention}
%                 \caption{Importance of queries in video representation}
%               \label{fig:fig1_text_attn}
%             \end{subfigure}%\hfill% or  or \hspace{0.3\textwidth}
%         \end{minipage}
        
%         \begin{minipage}{.6\linewidth}
%             \vspace{-0.2cm}
%             \caption{Overall statistics of Fig.~\ref{fig:motivation_ex} in QVHighlights dataset. 
%             (a) Saliency histogram shows the distributional gap between positive and negative video-text query pairs.
%             % (a) For the attention scores that measure how much the text query is generally involved in video representation, we use violin plots to show the probability density. We plot the score for each layer in the encoder.
%             % (b) Using the histogram, we compare how the baseline and QD-DETR yield different salient scores given the positive and negative video-text pairs.
%             % (b) Text ratio in self-attention layer to  of Moment-DETR
%             % (b) Ratio of text when representing video tokens in self-attention of Moment-DETR.
%             % (b) Magnitude of attention text query involved.
%             % (b) Attention score of video tokens
%             % (b) Magnitude of text query to refine the video tokens in self-attention layer of Moment-DETR.
%             (b) Probability density depicting the weight of the text query in attention score for video clips. Scores are from the self-attention layers in Moment-DETR encoder.
%             % (b) The text query ratio in attention score of video clips (Self-attention layer in Moment-DETR encoder). We use violin plots to show probability density.
%             % 텍스트 쿼리가, 비디오 피쳐에 얼만큼 attend 하는지
%             }
%         \end{minipage}
    
%     \end{tabular}
%     \vspace{-0.5cm}
%     \label{fig:moti}
%     % \captionsetup{belowskip=13pt}
%     % \setlength{\belowcaptionskip}{-10pt}
% \end{figure}


% \begin{figure}
%     \centering
%     % \includegraphics[width=1\linewidth]{figs/fig1_moti_negattn_1109.pdf}
%     \includegraphics[width=1\linewidth]{figs/fig1_moti_negattn_stat_v2.pdf}
%     \vspace{-0.8cm}
%     \caption{
%         Histogram of saliency when the positive and negative queries are given. We plot the histograms and its average value~(dotted line) to compare saliency scores when relevant~(positive) and irrelevant~(negative) text queries are given for each method. (Left) Since the video representations do not properly reflect textual information, both the positive and negative queries yield similar saliency scores. 
%         % (Middle) Even when the video representation is enforced to be updated with the textual information, the issue is not much resolved. 
%         (Right) By representing video clips in query-dependent manner, distributions are differentiated.
%     }
%     \vspace{-0.6cm}
%     \label{fig:motivation}
% \end{figure}


% One of the demanding task is moment retrieval task, which is detecting the desired moments from the given query, typically the text query.
When describing the moment, one of the most favored types of query is the natural language sentence~(text)\cite{anne2017localizing}. 
While early methods utilized convolution networks~\cite{zhang2020learning, gao2021fast, wang2020temporally}, recent approaches have shown that deploying the attention mechanism of transformer architecture is more effective to fuse the text query into the video representation.
% To handle these modalities, previous works simply employed the attention mechanism of transformer architecture to fuse the text query into the video representation.
For example, Moment-DETR~\cite{momentdetr} introduced the transformer architecture which processes both text and video tokens as input by modifying the detection transformer~(DETR), and UMT~\cite{umt} proposed transformer architectures to take multi-modal sources, e.g., video and audio. 
Also, they utilized the text queries in the transformer decoder.
Although they brought breakthroughs in the field of MR/HD with seminal architectures, they overlooked the role of the text query.
To validate our claim, we investigate the Moment-DETR~\cite{momentdetr} in terms of the impact of text query in MR/HD~(Fig.\ref{fig:motivation_ex}).
Given the video clips with a relevant positive query and an irrelevant negative query, we observe that the baseline often neglects the given text query when estimating the query-relevance scores, i.e., saliency scores, for each video clip.
% the output saliency score, i.e. query-relevance scores.
% Based on the observation, we traced the actual saliency prediction of the model against both the video-relevant query and the irrelevant dummy one where we find that the baseline often neglects the given text query when estimating the query-relevance scores of video clips.
% For example, in Fig.~\ref{fig:motivation_ex}, saliency scores are not affected even when the query is substituted with the dummy.
% % General statistics for Fig.~\ref{fig:motivation_ex} is shown in Fig.~\ref{fig:motivation}. 
% General statistics corresponding to Fig.~\ref{fig:motivation_ex} are also shown in Fig.~\ref{fig:motivation}.



% The limitation of the concrete baseline~\cite{momentdetr} is inspected in two different aspects; 1) Utilization of text-query in the encoding process and 2) the output saliency score, i.e. query-relevance scores.
% Firstly, we visualize the attention score when video clips are given as a query in self-attention. 
% We observe that the text queries have relatively small impacts compared to other video features, as shown in Fig.~\ref{fig:fig1_text_attn_ex}.
% That is, the text does not account for much in representing every video clip, although the goal of MR/HD is to detect query-relevant moments.
% Based on the observation, we traced the actual saliency prediction of the model against both the video-relevant query and the irrelevant dummy one where we find that the baseline often neglects the given text query when estimating the query-relevance scores of video clips.
% For example, in Fig.~\ref{fig:motivation_ex}, saliency scores are not affected even when the query is substituted with the dummy.
% % General statistics for Fig.~\ref{fig:motivation_ex} is shown in Fig.~\ref{fig:motivation}. 
% General statistics are also shown in Fig.~\ref{fig:motivation}.

% Consequently, in Fig.~\ref{fig:fig1_neg_attn_ex}~(b), we found that the baseline often neglects the given text query when estimating the query-relevance scores of video clips; 
% For example, 


% We validate the previous work sometimes neglects the given query when estimating the saliency of video clips.
% For example, there is an example that the saliency scores from positive and negative queries cannot be distinguishable, as shown in Fig.~\ref{fig:fig1_neg_attn_ex}.
% % 우리는 추가로 text attention을 추가도 해봤지만, 효과가 있긴 했으나, still 이슈가 있는 것을 확인하였다?
% % Still, we observe that assuring the high attendance of text queries does not resolve the overlap which motivates us to question the quality of the naive use of task-agnostic text representation~\cite{momentdetr, umt}.
% We found that introducing the text-attention for ensuring the high attendance of text queries relieve the overlap, but there still be a severe overlap.


% To validate their limitations, we inspect the impacts of text queries in the concrete baseline~\cite{momentdetr} with the two different aspects, 1) tendency of attention in self-attention layer and 2) saliency score, i.e. query-relevance scores. \SE{} % attention 이 갑자기 등장하는가?
% Firstly, we visualize the attention score when video clips are given as a query in self-attention. We observe the text queries have relatively low attention scores compared to the video features, as shown in Fig.~\ref{fig:fig1_text_attn_ex}.
% That is, the text does not account for much in representing every video clip, although the goal of MR/HD is to detect query-relevant moments.
% Based on this observation, we trace the actual saliency prediction of the model against both positive and negative text queries.
% We validate the previous work sometimes neglects the given query when estimating the saliency of video clips.
% For example, there is an example that the saliency scores from positive and negative queries cannot be distinguishable, as shown in Fig.~\ref{fig:fig1_neg_attn_ex}.
% % 우리는 추가로 text attention을 추가도 해봤지만, 효과가 있긴 했으나, still 이슈가 있는 것을 확인하였다?
% % Still, we observe that assuring the high attendance of text queries does not resolve the overlap which motivates us to question the quality of the naive use of task-agnostic text representation~\cite{momentdetr, umt}.
% We found that introducing the text-attention for ensuring the high attendance of text queries relieve the overlap, but there still be a severe overlap.



% Thus, we 
% query dependency를 높이기 위해 
% Cross-attention? text-attention? detailed explanation on text-attention should be needed?
% By handling these two issues, we find that more precise retrieval can be achieved.
% 
% 
%
% By projecting video-discriminative text features with high text attendance to source video, we f 
% We also find the need to improve the quality of query features since assuring high text attendance also results in...
% pairs are not finetuned to be discriminative that even the similarity within the pairs does not reflect the relevance between the query and the video clips.
% General statistics for Fig.~\ref{fig:motivation_ex} is shown in Fig.~\ref{fig:motivation}. 
% \SE{} % 이거 ??로 뜨는데, 위처럼 figure 그리면 label이 안되는걸까요
% \SE{}
% 형님 아래 사항 생각 좀 해보는게 좋을 거 같아요.
% fig 1. (a) 그림만 봤을 때 모든 clip에 대해 text attention이 일정이상 존재하긴 하니까, 뭔가 not assured to be conditioned가 와닿지 않는거 같아요.
% + 왜 text가 항상 attend 해야하나?
% not assured to be conditioned --> text shows relatively low affects compared to video 같이 실제 나타난 현상까지 같이 적으면 어떨까 싶어요.
% fig 1. (b) 덜 반영한다?

% \SU{}
% 일단 text가 attend 잘 되어야 한다는 것에 좀 궁금점이 생깁니다. 결국에는 text와 관련있는 frame들을 attend해서 higlight를 찾아야 하는게 아닐까요? 그리고, 현제 저희의 모델 구조상 text query가 Key와 Value로 거의 활용되고 있는데 그렇다면 결국에는 해당 모델은 text에 대한 attention이 전혀 없다고 봐도 무방하지 않을까요? 그런 면에서 text attention을 강조하는게 좀 걸리긴 합니다.

% Specifically, the text query is not assured to be explicitly conditioned on every clip of the video, and as the query texts are evenly treated, discriminative keywords may not be spotlighted.
% attention mechanism of Moment-DETR is not explicitly conditioned on the text query as shown in Fig~\ref{}(d), and in UMT, the text are only used for conditioning the queries while the video representation are refined itself by self-attention.

% \begin{figure}[t]
%     \begin{subfigure}{1\linewidth}
%       \centering
%     %   \includegraphics[width=1\linewidth]{figs/fig_1_moti_textattn.pdf}  
%     %   \includegraphics[width=1\linewidth]{figs/fig_1_moti_textattn_v2.pdf}  
%       \includegraphics[width=1\linewidth]{figs/fig_1_moti_textattn_v4.pdf}  
%       \vspace{-0.5cm}
%     %   \caption{text attention}
%         \caption{Distribution of attention scores in Moment-DETR encoder}
%       \label{fig:fig1_text_attn}
%     \end{subfigure}%\hfill% or  or \hspace{0.3\textwidth}
%     \vspace{0.2cm}
%     \begin{subfigure}{1\linewidth}
%       \centering
%     %   \includegraphics[width=1\linewidth]{figs/fig1_moti_negattn.pdf}  
%       \includegraphics[width=1\linewidth]{figs/fig1_moti_negattn_v2.pdf}  
%       \vspace{-0.5cm}
%     %   \caption{neg attention}
%         \caption{Saliency score against positive and negative text queries}
%       \label{fig:fig1_neg_attn}
%     \end{subfigure}%\hfill% or  or \hspace{0.3\textwidth}
%     \vspace{0.2cm}
%     \begin{subfigure}{1\linewidth}
%       \centering
%     %   \includegraphics[width=1\linewidth]{figs/fig1_moti_violin.pdf}  
%       \includegraphics[width=1\linewidth]{figs/fig1_moti_violin_v2.pdf}  
%       \vspace{-0.5cm}
%       \caption{violin}
%       \label{fig:fig1_violin}
%     \end{subfigure}%\hfill% or  or \hspace{0.3\textwidth}
%     \vspace{-0.2cm}
%     \caption{(a) 1. portion of text attention vs. video attention 2. relation with text query and content (e.g. fg, bg) of clip seems not to affect the attention score
%     (b) 1. high variability even though entire clips are highly correlated with the given text query 2. positive and negative query makes overlaps on saliency score distribution
%     (3) actual distribution on validation dataset.}
%     \label{fig:motivation}
%     % \captionsetup{belowskip=13pt}
%     % \setlength{\belowcaptionskip}{-10pt}
% \end{figure}

To this end, we propose Query-Dependent DETR~(QD-DETR) that produces query-dependent video representation.
% Our key focus is to ensure each clip in predicted moments is explicitly conditioned by the query, particularly on the video-descriptive portion of the text query.
% Our key focus is to ensure that query-relevant clips are predicted by enforcing each clip to be explicitly conditioned by the query.
%Our key focus is to ensure that the model prediction for each clip is highly relevant to the query.
Our key focus is to ensure that the model's prediction for each clip is highly dependent on the query.
% by enforcing each clip to be explicitly conditioned by the query. :)
% hmm...
% \SE {} % "query-relevant clips are predicted" 이 문장이 좀 애매한거 같습니다. relevant 클립을 놓지지 않고 찾는 것을 보장한다? 이런 느낌인지 아니면 높은 saliency 를 주는게 목적이다? model prediction이 query-relevance를 반영하는 것을 보장한다?
% Our key focus is to ensure that the model prediction reflects query-relevance of clips by enforcing each clip to be explicitly conditioned by the query.
First, to fully utilize the contextual information in the query, we revise the transformer encoder to be equipped with cross-attention layers at the very first layers.
% 상익's thought :  single video - query간의 관계만 고려 - 같은 word가 더 많이 쓰이는 것을 보고 
% 교수님's thought : neg pair 를 쓰면 쿼리를 보지 않고서는 video clip간만 고려하는 것이 사라짐. 왜냐면 0으로 내보내야 하기 때문. --> SE: relative difference 만 고려하다가, 
By inserting a video as the query and a text as the key and value of the cross-attention layers, our encoder enforces the engagement of the text query in extracting video representation.
% 원준 교수님 코멘트 반영해서 다시
Then, in order to not only inject a lot of textual information into the video feature but also make it fully exploited, we leverage the negative video-query pairs generated by mixing the original pairs.
Specifically, the model is learned to suppress the saliency scores of such  negative~(irrelevant) pairs.
Our expectation is the increased contribution of the text query in prediction since the videos will be sometimes required to yield high saliency scores and sometimes low ones depending on whether the text query is relevant or not.
% \SE{}
% learns to?
% By suppressing the saliency scores of the irrelevant video-query pairs, the model learns to spotlight only the video-specific discriminative words in the query.
% % \SE{} % ====================== 상익 수정 ========================
% However, this architectural design still lacks the capability of identifying the video-descriptive keywords in the query.
% % However, this architectural design still lacks in identifying proper query relevance.
% This is because the current training scheme only focuses on the interactions of video and clips within a single video while neglecting information shared throughout the entire video.
% % We argue the problem of the current training scheme that only focuses on distinguishing the clips in a single video while neglecting information shared throughout the entire video.
% Therefore, we leverage the negative video-query relationships to enhance the capability of identifying the contextual similarity of query and video clips.
% 
% 원준 원본 
% However, this architectural design heavily relies on the quality of the text query.
% Therefore, we leverage the negative video-query relationships to enable the model to emphasize key corresponding query features.
% By suppressing the saliency scores of the irrelevant video-query pairs, the model learns to spotlight only the video-specific discriminative words in the query.
% =========================================================
Lastly, to apply the dynamic criterion to mark highlights for each instance, we deploy a saliency token to represent the entire video and utilize it as an input-adaptive saliency criterion. 
With all components combined, our QD-DETR produces query-dependent video representation by integrating source and query modalities.
This further allows the use of positional queries~\cite{dabdetr} in the transformer decoder.
% Furthermore, we can exploit the advanced DETR decoder architectures using the positional information, e.g., DAB-DETR, since our encoded tokens consist of identical position representations from a single modality.
% \SE{} % ====================== 상익 수정 ========================
% Furthermore, we can exploit the advanced DETR decoder architectures using the positional information, e.g., DAB-DETR, since our video clip tokens consist of identical position representations from a single modality.
% 원준 원본
% It also enables the use of advanced DETR decoder architectures, e.g., DAB-DETR, for the first time, as these works exploit the position information within a single modality.
% =========================================================
Overall, our superior performances over the existing approaches validate the significance of the role of text query for MR/HD.
% Our extensive experiments on QVHighlights, TVSum, and Charades-STA datasets validate the significance of considering the role and the quality of text query.

% All components combined with dynamic anchor moments for the query of decoder, our FOQUE fosters the query-dependent video representation, thereby making the 
% All components combined, our modified transformer encoding process fosters the query-dependent video representation thereby achieving the state-of-the-art results on various benchmarks of moment-retrieval and highlight detection.
	
% -	Video Platform & Streamer & Consumer의 증가. 
% Video는 다른 데이터 타입보다 정보가 많아 유용하지만, 이는 다른 말로 해석하면 video를 보는 것은 time-consuming 하고, 원하는 것을 찾아보기에는 힘들 수 있음.
% 따라서, 많은 매체에서는 사람들의 더 많은 이목을 끌기 위해 highlight 비디오라는 것을 편집하여 공유도 함.
% 하지만, highlight video를 만들기 위해 사람의 노력이 필요한 현 시점에서, This spotlights the need to retrieve the user-requested / Highlight moments in the video.

% -	이전에도 이러한 문제를 해결하기 위해 (asdfasdf) for moment retrieval, (asdfasdf) for highlight detection 등이 제안 되었지만, 이들은 비디오의 특정 영역을 찾는다는 공통된 목적을 가지고 있으면서도, 데이터 셋의 한계로 인해 따로 연구되었음. 이를 문제 삼으며, 최근에는 두 task를 동시에 학습할 수 있는 dataset이 소개 되었는데, 컴퓨터비전에서 최근 각광을 받고 있는 Transformer 모델 도입과 함께 큰 발전을 거듭하고 있음.

% -	구체적으로, 이 두가지 task를 수행하기 위해서는 transformer를 두가지 방법으로 이용할 수 있는데, moment-DETR 처럼 moment 를 clip의 set 단위로 예측할 수 있고, UMT 처럼 clip-wise prediction을 할 수 있음. 하지만, 이들은 query를 condition이 아닌 video와 동등한 레벨로 취급하거나 [mDETR], 매 클립이 self-attention으로 mixing 된 후에 condition을 걸어주어 clip간의 차이를 확실하지 이용하지 못하였고, 또한, 확실하게 condition으로 주지 못하였고, video와 query 사이의 관계를 한정적으로만 이용하였다.

% -	따라서, we explore three different ways to fully exploit query information. First, we design one-way cross-attention layer to condition every clip with the query features. Then, we utilized the negative video-text pairs to better model the relationships between the video and the text embeddings. Lastly, we define the saliency token to be the video-query dependent saliency estimator.


















% ===================== neg pair 부분 ===========================
% Nevertheless, the current training scheme, only considering the given video-query pair, still disturbs the model from identifying proper query-relevance prediction.
% In detail, the model focus on learning the fine-grained discrepancy between video clips, while neglecting the information they share, which contains significant clues to understand the context of video.
% Therefore, we leverage the negative video-query relationships to enhance the capability of identifying the contextual similarity of query and video clips.
% Therefore, we leverage the negative video-query relationships by suppressing those pairs, so that enhance the capability of identifying the contextual similarity of query and video clips.
% We hypothsize the diversity in query-video pairs are insufficient to learn the general relationship between text query and video.
% Therefore, we leverage the negative video-query relationships by suppressing the saliency scores of the irrelevant video-query pairs.
% However, this architectural design still lacks in identifying proper query relevance.
% We argue that the current training scheme only focuses on learning the fine-grained discrepancy between clips in a single video, while neglecting the information they share, which contains significant clues to understand the context of the video.
% Therefore, we leverage the negative video-query relationships to enhance the capability of identifying the contextual similarity of query and video clips.
% However, this architectural design still lacks in identifying proper query relevance.
% We argue the problem of the current training scheme that only focuses on learning the fine-grained discrepancy between clips in a single video.
% That is, the current design neglects the information shared throughout the video, although it contains significant clues to understand the context of the video.

\section{Related Work} \label{sec:relatedwork}

The section introduces the research related to the paper, which can be divided into three parts: (1) high-utility pattern mining; (2) top-$k$ utility itemset mining; and (3) targeted pattern mining.

\subsection{High-utility pattern mining}

Frequent itemset mining (FIM) \cite{aggarwal2014frequent,agrawal1994fast,han2000mining} has been extensively studied for decades. However, relying only on frequency cannot bring enough benefits to users. Factors such as quantity and profit should also be considered. For this reason, Chen \textit{et al.} \cite{chan2003mining} put forward a new task called high-utility itemset mining (HUIM). Since then, utility mining research has developed rapidly \cite{gan2021survey,lin2016efficient,song2016high,wu2021haop}. For the convenience of discussion, high-utility itemset mining algorithms are grouped into the following three categories:

\textbf{Apriori-based algorithms}: Since Agrawal \textit{et al.} \cite{agrawal1993mining} proposed the Apriori property in 1994, lots of algorithms based on Apriori have been published. For example, Liu \textit{et al.} \cite{liu2005two} introduced the prominent Two-Phase algorithm to handle the difficulty that the utility, unlike frequency, is neither monotone nor anti-monotone. That algorithm uses an overestimation of the utility called \textit{TWU} (Transaction Weighted Utilization) to find candidate itemsets in a first phase. Thereafter, in a second phase, the database is searched again to determine the exact utility value of each candidate itemset. The IIDS algorithm \cite{li2008isolated} is an improved version of Two-Phase that discards isolated items to shrink the search space. However, the common disadvantage of Apriori-like algorithms is that plenty of candidate patterns are generated, resulting in considerable computational costs and memory consumption.

\textbf{Tree-based algorithms}: Tseng \textit{et al.} \cite{tseng2010up} designed the UP-tree structure, a utility-pattern tree, and introduced the UP-Growth algorithm inspired by FP-Growth. Subsequently, other versions of tree-based algorithms \cite{song2014mining,tseng2012efficient} have been presented. In general, utilizing the UP-tree can prevent many meaningless database scans. When working with large-scale databases, however, this structure grows increasingly complex and occupies a massive amount of memory.

\textbf{Other structure-based algorithms}: HUI-Miner \cite{liu2012mining} utilizes a novel data structure known as a utility-list, which avoids the difficulty of generating numerous candidates. Moreover, the FHM algorithm \cite{fournier2014fhm} reduces the cost of join operations by using a tighter upper bound, which results in outperforming HUI-Miner. However, the join operation on lists of these algorithms takes time and memory. Thus, Zida \textit{et al.} \cite{zida2015efim} proposed the EFIM algorithm with high-utility database projection (HDP) and high-utility transaction merging (HTM) techniques to lower the expensive cost of database passes. The utility-list-based CoUPM algorithm for correlated utility-based pattern mining \cite{gan2019correlated}. In summary, these algorithms integrate various strategies to discover HUIs as efficiently as possible.

\subsection{Top-$k$ utility itemset mining}

Although the above algorithms are effective in finding the desired set of itemsets, the efficiency of mining is strongly related to the selection of the minimum utility threshold. However, it is not easy to identify an appropriate threshold. Many top-$k$ pattern mining algorithms were thus designed to directly discover the set of top-$k$ HUIs, rather than asking users to specify a utility threshold. Top-$k$ HUIM algorithms mainly consist of two types: the first is the two-phase algorithms, and the other is the one-phase algorithms.

\textbf{Two-phase algorithms}: The task of discovering the top-$k$ HUIs was proposed by Wu \textit{et al.} \cite{wu2012mining} with the TKU algorithm, which outperformed HUIM algorithms in terms of speed. The TKU algorithm is a two-phase algorithm. In the first phase, a UP-Tree is built, and promising top-$k$ HUIs are generated. Then, in the second phase, the desired top-$k$ HUIs are selected among them. TKU applies several strategies to filter unpromising candidates during the search \cite{tseng2015efficient} and achieve higher efficiency. Subsequently, REPT \cite{ryang2015top} was introduced with optimizations to record and pre-calculate the utility of items to prune the search space effectively and raise the minimum utility threshold. REPT uses a tree structure and pre-evaluation matrixes as tools to store utility information. However, these two-phase algorithms still generate large sets of candidates, which causes unreasonably long runtimes and high memory usage.

\textbf{One-phase algorithms}: For top-$k$ HUIM, the one-phase TKO algorithm \cite{tseng2015efficient} was developed to solve the shortcomings of two-phase algorithms. TKO takes advantage of the utility-list structure of HUI-Miner, and outperforms the TKU and REPT algorithms according to experiments \cite{tseng2015efficient}. Similarly, another one-phase algorithm called KHMC \cite{duong2016efficient} also discovers the top-$k$ HUIs by using the utility-list structure. In KHMC, an estimated utility co-occurrence pruning (EUCP) technique is applied, which is based on precalculating the TWU of 2-itemsets. Moreover, the algorithm also adds another pruning strategy named early abandoning to avoid completely constructing the lists of unpromising itemsets. Three threshold-raising strategies are able to significantly shrink the search space and enhance the algorithm's efficiency. The THUI algorithm \cite{krishnamoorthy2019mining} has better performance thanks to introducing the concept of Leaf Itemset Utility (LIU), a triangular matrix, which can be implemented with only a small amount of memory to store utility information. Besides, the LIU-E and LIU-LB threshold raising strategies also accelerate the mining speed of the algorithm. THUI greatly outperforms TKO and KHMC, especially for dense or large datasets.

In addition, there are various other top-$k$ pattern mining problems and variations, such as mining top-$k$ sequential patterns \cite{zhang2021tkus}, mining top-$k$ HUIs in data streams \cite{cheng2021etkds}, discover top-$k$ high-utility sequential patterns \cite{zhang2021tkus}, and mining top-$k$ HUIs with negative utility values \cite{sun2021mining}.


\subsection{Targeted pattern mining}

Those algorithms listed above are designed to find all itemsets that meet a single predetermined criterion. Target-oriented query algorithms give an alternative solution to this problem by filtering out unnecessary information. Rather than searching for numerous but mostly insignificant items, the user can enter any target and then discover patterns containing the desired items. Several target-oriented query algorithms based on frequency have been developed in earlier studies. These interactive methods are capable of returning results containing a target. Kubat \textit{et al.} \cite{kubat2003itemset} were among the first to address the issue of processing target queries in a transactional database. They implemented target query processing algorithms for association mining by creating itemset trees that can be progressively updated. Fournier-Viger \textit{et al.} \cite{fournier2013meit} developed the Memory Efficient Itemset Tree (MEIT) to further reduce memory requirements. The tree is optimized to perform incremental modifications when new transactions are inserted, and it employs a node-compression method. For multi-objective mining of big data, the guided FP-growth (GFP-growth) algorithm based on FP-Growth was proposed by Shabtay \textit{et al.} \cite{shabtay2018guided}. In particular, many experiments have illustrated the excellent performance of the algorithm on imbalanced data. Target-oriented mining has also been studied and applied to discover sequential patterns. The targeted mining algorithm for sequential patterns proposed by Chueh \textit{et al.} \cite{chueh2010mining} speeds up the search for the target itemsets by using the reversion of the original sequence and comparing the reversed sequence with the related itemsets. Furthermore, clustering analysis is applied to automatically set time partition values for the task of time-interval sequential pattern mining. A novel target-oriented sequential pattern mining approach was presented by Chand \textit{et al.} \cite{chand2012target}, which uses RFM (recency, frequency, and monetary) constraints. As a result, fewer database projections are done, and the space complexity is reduced. To remove some useless or irrelevant patterns in high utility sequential pattern mining \cite{zhang2021shelf,gan2021explainable}, the TUSQ algorithm \cite{zhang2021tusq} first introduced the concept of utility into target sequence queries. The algorithm does not focus on frequency like previous algorithms, but rather on utility. Recently, the TargetUM algorithm \cite{miao2021targeted} has been proposed to fill the gap and perform target-oriented mining in HUIM.

In general, the TargetUM algorithm provides an integrated approach for high-utility mining with a target query, which serves as the foundation for this research. However, there are no studies combining top-$k$ high-utility methods with target pattern queries. This paper introduces the problem of targeted utility mining with the concept of top-$k$ patterns to prevent the generation of large sets of HUIs and to accurately and quickly process target queries.


\section{Problem Formulation}\label{sec:problem}

In this work, we address the problem of synthesizing successful (S=1) parallel-jaw grasps $\matr{G}$ on partially observed object point clouds $\matr{O} \in \mathbb{R}^{\text{N}\times 3}$ where the approach directions $\Vec{a} \in \mathbb{R}^3$ of the grasps are constrained to a subset $\text{C}\subset \mathrm{SO}(3)$ of the rotation group $\mathrm{SO}(3)$. Mathematically, the objective is to learn the joint distribution $\prob{(\matr{G}, \text{S}=1 | \matr{O}, \text{C})}$. 

If we assume that the success of a grasp $\text{S}$ is conditionally independent of $\text{C}$ given $\matr{G}$, we can factorize the joint distribution into $\prob{(\text{S}=1 | \matr{G}, \matr{O})}\prob{(\matr{G}|\matr{O}, \text{C})}$, where the first distribution $\prob{(\text{S}=1 | \matr{G}, \matr{O})}$ is a grasp discriminator and the second $\prob{(\matr{G}|\matr{O}, \text{C})}$ is a grasp generator. We approximate these distributions with separate \acp{dnn} $\mathcal{Q}_{\boldsymbol{\theta}}(\matr{G}|\matr{O}, \text{C})\approx \prob{(\matr{G}|\matr{O}, \text{C})}$ and $\mathcal{D}_{\boldsymbol{\psi}}(\text{S}=1 | \matr{G}, \matr{O}) \approx \prob{(\text{S}=1 | \matr{G}, \matr{O})}$, each with their own trainable parameters $\boldsymbol{\theta}$ and $\boldsymbol{\psi}$. 

%The goal of this work then becomes to design these networks to best fit to grasping data. 

Central to both networks are the 6-D grasp poses $\matr{G}$. In this work, we represent grasp poses $\matr{G}=[\matr{R}, \matr{T}]\in \mathbb{R}^7$ by a rotation $\matr{R}$ expressed as a 4-D quaternion and a translation $\matr{T}$ expressed as a 3-D vector. It is possible to  further decompose the rotation component $\matr{R}$ as visualized in \figref{fig:yaw-pitch-cam} into a unit length approach vector $\Vec{a}\in \mathbb{R}^3$ and a hand orientation angle $\alpha$ or, in terms of Euler angles, as a roll ($\alpha$), pitch ($\beta$) and yaw ($\gamma$) angle. Out of these components, the approach vector (pitch and yaw angles) is explicitly constrained by C, while the hand orientation (roll angle) is learned. 


\section{Method}\label{sec:method}
\begin{figure*}
    \centering
    \includegraphics[width=\linewidth,keepaspectratio]{figures/pipelines/pipeline_figure_full.pdf}
    %\vspace{-0.7cm}
    \caption[]{
    \OURS{} first generates a set of pseudo masks (top) to initiate self-training (bottom) for unsupervised 3D instance segmentation.
    We leverage features from 3D self-supervised pre-training in combination with 2D self-supervised features on an input mesh.
    These multi-modal features are then aggregated on geometric segment primitives, integrating low- and high-level signals for pseudo mask segmentation.
    These initial pseudo masks are then used as supervision for a 3D transformer-based model to produce updated instance masks that are integrated into the supervision of multiple self-training cycles.
    Finally, we obtain clean and dense instance segmentation without using any manual annotations.
    } 
    \label{fig:full_pipeline}
\end{figure*}

\paragraph{Problem definition}
We propose an unsupervised learning-based method for 3D instance segmentation. Formally, we assume a set of training 3D scenes $\{X_i\}_{i=1}^{n_t}$, represented as meshes, where each scene $X_i$ contains an unknown set of $n_i$ objects. We aim to train a model that can predict for a previously unseen input scene $X$, a set of 3D masks representing the different object instances in that scene. 

\paragraph{Method overview}
We employ a two-stage scheme for unsupervised 3D instance segmentation.
First, we group scene points into contiguous regions based on geometric primitives, and aggregate self-supervised features in these regions to generate an initial set of pseudo masks using Normalized Cut.  This grouping technique enables efficient handling of high-dimensional 3D data.
%
We then follow a series of self-training cycles to refine the pseudo annotations.
An overview of our approach is shown in Figure~\ref{fig:full_pipeline}.

\subsection{Geometric oversegmentation}\label{sec:oversegmentation}
Normalized Cut~\cite{shi2000normalized_cut} (NCut) with deep features has been successfully applied in the 2D domain on dense graphs generated from image patches \cite{wang2022tokencut,lis2022attentropy,wang2023cut}. However, adopting this directly to 3D would be computationally infeasible due to the cubic growth with dimensionality.

We thus propose a solution in the form of geometric oversegmentation through graph coarsening. We begin by creating a graph where each node represents a mesh vertex. Then, we aggregate nodes with similar normal direction and color values and cluster them into contiguous mesh segments using the efficient method proposed by \cite{felzenszwalb2004efficient}. This process reduces the graph size by multiple orders of magnitude. 
%
Our geometry aware segments, as opposed to voxels or points, additionally provide regularization to the subsequent stage of feature aggregation for generating pseudo masks, which we will describe in the following section.

\subsection{Initial pseudo mask generation}\label{sec:dataset_gen}

We first predict an initial set of pseudo masks.  Additional masks will be added during the self-training phase described in Section~\ref{sec:self_train}. 
Thus, we favor generating a reliable set of masks at the cost of restricting to a sparse initial set (i.e., missing potential instances rather than generating noisy masks for them).

\paragraph{Feature aggregation}

We aim to employ a strong set of features for pseudo mask generation and thus consider complementary geometric and color signals from RGB-D scan data.
We leverage geometric 3D self-supervised features from a state-of-the-art 3D pre-training approach, Contrastive Scene Contexts (CSC)~\cite{hou2021exploring}.
We additionally consider 2D self-supervised features from DINO~\cite{caron2021emerging_dino}, extracted from the RGB images and projected to 3D using the corresponding camera poses.
Both the 3D and 2D features are aggregated within each of our geometry-aware segments. 

\paragraph{Masked foreground separation}
We apply NCut to our aggregated features to extract foreground regions $M$ as our initial pseudo masks. 
%
Starting with an empty set $M^0=\{\}$, we iteratively compute the adjacency matrix and retrieve the masks. 
%
That is, we start from $N$ geometric segments with their corresponding $D$-dimensional features $\mathcal{F} \in \mathcal{R}^{N\times D}$, and construct the similarity matrix $A = sim(\mathcal{F})$, where $sim$ denotes cosine similarity. 
Additionally, for the multi-modal setup we calculate similarity matrices $A_{2D}$ and $A_{3D}$ independently and take their weighted average to obtain the final scores. 
Empirically, we found this to be more robust than direct feature fusion of the different modalities, due to their different statistical characteristics.
%
\indent We obtain $W_j$ for Equation~\ref{eq:general_eigenval} by thresholding $A$ at $\tau_{cut}$, where $j$ denotes the $j^{th}$ NCut iteration. 
Using $W_j$, we solve for the second eigenvector $v_j$ and threshold it to retrieve the partition $m_j$. 
We keep all separated foregrounds in $M^0$, where for each upcoming iteration, we mask out the row and column vectors from $W_i$, where $m_i \in M^0$ was already accepted as a foreground instance and $i$ being the segment ids. 
This allows greedy separation of instances in order of confidence  in every cut iteration.
Examples of our generated pseudo masks are visualized in Figures \ref{fig:freemask_vs_ncut} and \ref{fig:self_training_refinement}. \\
%
\indent As the adjacency graph is unaware of the mesh connectivity, NCut often results in masks that span  spatially separated scene regions. 
In 3D, we can leverage knowledge of physical distance to constrain masks to be contiguous in the coarsened scene connectivity graph. We thus filter masks that have separated components, keeping only the ones that contain the item with the maximum absolute value in  $v_j$. Separation is performed before saving $m_j$ into $M^0$, thus allowing for repeated separation of every component. 
We iterate until the maximum number of instances $M^0 = \{m_i\}_{i=1}^{N_m}$ are obtained, or there are no segments left in the scene. 

\subsection{Self-Training}\label{sec:self_train}

Our initial pseudo masks can provide a set of proposed instances $M^0$; however, these pseudo masks are quite sparse in the scenes and sometimes over- or under-split nearby instances.
We thus refine the pseudo mask data through an iterative self-training strategy, producing final instance segmentation predictions $M'$ with more dense and complete instance proposals.

We leverage a state-of-the-art 3D transformer-based backbone~\cite{Schult23mask3d} for our self-training from pseudo mask data as supervision. 
Through multiple training cycles we save the proposals of the $t^{th}$ iteration into $M^{t}$, from the self-trained model, and save these masks as an extension to the original pseudo dataset obtaining $M^t \supseteq M^0$. 
From the second training iteration, we can extract the most confident $K$ predictions and sample these new instance proposals as an addition to the pseudo annotations. 
Further, we only accept new instances if the added information value is larger than a minimum threshold, which we measure by simple segment IoU scores. This way, we can effectively densify the originally sparse annotations, but without limiting the quality of the originally clean pseudo masks. 
%
\paragraph{Loss} \label{par:losses}
We adapt DropLoss \cite{wang2023cut} for our self-training cycles, which is robust to sparse data and missing annotations. 
In particular, we use a weighted combination of cross-entropy and Dice \cite{sudre2017generalised_diceloss} losses for bipartite-matching with pseudo annotations.
We then drop losses for backpropagation which do not have at least $\tau_{drop}$ overlap with the annotations from the previous cycle.

\subsection{Implementation Details}\label{sec:implementation}
%
\paragraph{Backbones.} 
We use a Res16UNet34C sparse-voxel UNet implemented in the MinkowskiEngine~\cite{choy20194d} for 3D pre-trained feature extraction as well as for the 3D transformer during self-training. 

\paragraph{Self-training.} We employ the 3D transformer architecture of \cite{Schult23mask3d}, initialized from scratch. 
The first self-training cycle is trained for 600 epochs with a batch size of 8 until convergence, which takes $\approx 3$ days on a single NVIDIA RTX A6000 GPU. 
Further self-training cycles are all initialized from the previous state and finetuned for an additional 50 epochs in $\approx 4$ hours and for a total of 4 training cycles to produce the final set of instance predictions $S$. 
For the Hungarian assignment, we take the original weighted combination of dice and binary cross-entropy losses and only apply the DropLoss condition in the backpropagation phase.

\section{Dataset}\label{sec:dataset}

Training \methodname{} requires a dataset of successful grasps where the approach direction is labeled according to the discrete bins of $\mathrm{SO}(3)$. However, due to the discretization of $\mathrm{SO}(3)$ using the Euler angles, our dataset depends on which coordinate system we express the grasps in. Therefore, in this section, we first discuss the choice of the coordinate system and then introduce the dataset.    

\subsection{Camera-Centric Coordinate System}

Prior \ac{dl}-based grasping works has generally not discussed the coordinate system used for expressing their training data. The reason is that there is no need for such a discussion as the training data can seamlessly be transformed from one coordinate system to another without influencing the grasp labels. Unfortunately, this is not the case in our work, as the labels C are coordinate-specific.

Two commonly used coordinate systems for expressing the grasp poses are the global and camera-centric coordinate systems. We choose to express the grasp poses in the camera-centric coordinate system because of two benefits. Firstly, in a camera-centric coordinate system, the object \pc{} is always aligned with the camera axes. Secondly, the camera-centric coordinate system enables a human-like way of communicating the grasp approach direction. For instance, with the camera-centric coordinate system, we can express grasp approach directions similar to a human from the left, right, above, below, behind, or in front of the object. Neither of these benefits applies when using a global coordinate frame unless the global coordinate system coincides with the camera coordinate system, which is rarely the case. 

\subsection{The Approach-Constrained Grasping Dataset}

To train \methodname{}, we need a large-scale grasping dataset containing object \pcs{} $\matr{O}$ and successful grasps $\matr{G}$ where each grasp $\mathbf{g} \in \matr{G}$ has a specific constraint label C depending on its yaw and pitch angle. To date, no such grasping dataset exists. Still, instead of generating and labeling a completely new grasping dataset, we convert the grasps in the already established large-scale Acronym dataset~\cite{eppner2021acronym} that consists of 17.7 million simulated parallel-jaw grasps on 8872 objects from ShapeNet~\cite{chang2015shapenet} to fit our needs.

The steps for curating our dataset from the Acronym data are similar to the one presented in \cite{lundell2023constrained}, but with some key differences stemming from the need to discretize the space of rotations and label grasps accordingly. In detail, the steps we follow to create the dataset used in this work are:

\begin{enumerate}[label=(\roman*)]
    \item Discretize the space of rotations $\mathrm{SO}(3)$ into B bins using \eqref{eq:descritization}.
    \item For each object, randomize 100 camera poses pointing toward the object and render a \pc{} $\matr{O} \in \mathbb{R}^{\text{N}\times 3}$ from each pose.
    \item Transform 1000 random grasps $\matr{G}$ on each object from Acronym to the camera-centric coordinate system and then label them according to one of the B bins using \eqref{eq:vec_to_euler}.
\end{enumerate}
The above process was carried out on all the 8872 Acronym objects. From the final dataset, 112 random objects were held out for evaluation while the rest were used for training \methodname{}.



% !TEX root = main.tex

%%%%%%%%%%%%%%%%%%%%%%%%%%%%%%%%%%%%%%%%%%%%%%%%%%%%%%%%%%%%%%%%%%%%%%%%%%%%%%%%
%%%%%                           		Experiment                         		 %%%%%
%%%%%%%%%%%%%%%%%%%%%%%%%%%%%%%%%%%%%%%%%%%%%%%%%%%%%%%%%%%%%%%%%%%%%%%%%%%%%%%%

\begin{figure}[h]
    \centering
    \includegraphics[scale=0.35]{images/rov}
    \caption{The modified BlueROV-2 used in the  experiment.}
    \label{fig:rov}
\end{figure}
% \vspace{-2ex}

\section{Experiment Results}
\label{sec:experiment}


\begin{figure*}[t] \centering
 \vspace{-5ex}
 \includegraphics[width=0.99\textwidth,height=0.2\textwidth]{images/metashape.png}
    \vspace{-6ex}
    \caption{The Metashape reconstructed result. The largest ice-hole on the left side is the starting point for the vehicle, the total length of this reconstructed result is roughly 40 meters} \label{fig:metashape}
\end{figure*}

\subsection{Experiment data set}
% hardware 
In March 2021, we have conducted an under-ice experiment under the frozen Keweenaw Waterway in Michigan using a modified BlueROV2~\cite{UnderIce-ROV_Zhao_2021} with a suite of sensors shown in Fig.~\ref{fig:rov}. 
The ice thickness is about 30 cm.
An ice hole (about 1m by 1m) was cut for deploying the ROV while several small ice holes (shown in Fig. \ref{fig:metashape}) were also drilled along the transect. They are spaced at 10 meters except the second one.
During the experiment, the ROV is remotely controlled by the pilot to drive along a straight line multiple times (roughly 40 meters each way) from a position at 4 meters deep, resulting in a total traveling distance of about 200 m and a total duration of about 20 minutes.

We used the experimental data set to validate our proposed sensor fusion framework.
In the data set, the up-looking stereo-camera is running at 15 Hz with a raw image size of 1616 by 1240 pixels. 
Only one camera was used for this experiment. The upward-looking DVL is pinging at 4 Hz, the IMU is running at 100 Hz, and the pressure sensor on the DVL is sampling at 2 Hz.
The standard deviation (SD) of the DVL single ping at 3 m/s is about 0.005 m/s from Norteck technical specification. During the data collection, the vehicle moves at about 0.4 m/s and the transformation between DVL and IMU is roughly measured. Therefore, we set the velocity SD to between 0.0375 - 0.1 m/s for the experiment.
The original result shown in \cite{UnderIce-ROV_Zhao_2021} used the robot localization without correcting the time delays (about 10 seconds) between the IMU and DVL due to the DVL driver issue.
Even though the localization in \cite{UnderIce-ROV_Zhao_2021} shows a low drift, it may be a coincidence.
In this data set, we have corrected the delays during the validation process.
One unique feature in this data set is that, occasionally, the ROV is controlled to hover in place.
Such maneuvers will challenge the visual SLAM performance since during the hovering no significant translation is available for feature triangulation. 
In application, hovering may be needed in several key locations during an under-ice exploration to collect more measurements on abnormal biogeochemical processes, e.g., a salt brine injection and algae bloom.

\subsection{Results}
The ground truth vehicle path is generated using Agisoft Metashape based on SfM technique, a rendering of the ice surface is shown in Fig.~\ref{fig:metashape}.
For comparison, we use the evo~\cite{evo_Grupp_2017} toolbox to align (recovery orientation and scale) Metashape ground truth and estimated odometry created from different sensor fusion methods.
We only selected a short amount of time (90 seconds about 10 meters) at the beginning for alignment.
Herein, we compare the localization results from 10 settings, as shown in Table~\ref{tab:odom_setting} against the ground truth path.
\begin{table}[h]
    \caption{Setup with different sensor suites and features. "Y" means used and "N" means not used.}
    \label{tab:odom_setting}
    \centering
    \begin{tabular}{c|c|c|c|c|c|c|c|c|c|c}
    \hline
    \textbf{Case \#} &\textbf{1} &\textbf{2} &\textbf{3} &\textbf{4} &\textbf{5} 
                     &\textbf{6} &\textbf{7} &\textbf{8} &\textbf{9} &\textbf{10} \\
    \hline
    Visual      &Y  &Y  &Y  &Y  &Y  &N  &\textit{Y}  &\textit{Y}  &\textit{Y}  &\textit{Y}  \\    
    \hline
    DVL         &Y  &Y  &Y  &Y  &Y  &Y  &\textit{N}  &\textit{N}  &\textit{N}  &\textit{N}  \\
    \hline
    IMU         &Y  &Y  &Y  &Y  &Y  &Y  &\textit{Y}  &\textit{Y}  &\textit{Y}  &\textit{Y}  \\
    \hline
    Pressure    &Y  &Y  &Y  &Y  &N  &Y  &\textit{N}  &\textit{N}  &\textit{Y}  &\textit{Y}  \\
    \hline
    Enhancement &Y  &N  &Y  &N  &Y  &N  &\textit{N}  &\textit{N}  &\textit{N}  &\textit{N}  \\
    \hline
    Keyframe    &Y  &Y  &N  &N  &Y  &N  &\textit{Y}  &\textit{N}  &\textit{Y}  &\textit{N}  \\
    \hline
    \end{tabular}
\end{table} 
% \vspace{-2ex}

For all tracks, we used identical parameters in the MSCKF and system initialization is conducted using the method from~\cite{INS-DVL-Pressure_Zhao_2022}. 
We used CLAHE~\cite{CLAHE_Pizer_1987} with KLT~\cite{KLT_Lucas_1981} method for the front-end feature tracking because the descriptor based methods, such as the ORB and KAZE, didn't provide us with a consistent tracking result. 
We found that the descriptor-based method can be confused by the air bubbles in the ice which appear in similar shapes and sizes.
We present all the resulting vehicle paths estimated from case $1\thicksim6$ in Fig.~\ref{fig:plot_result}(a) with different colors. Noted that VIO options (case $7\thicksim10$) are not visualized since those runs failed quickly at the beginning because of the hovering maneuvers.  
From Fig.~\ref{fig:plot_result}(a), we can easily observe that the odometry generated without the visual assist (case 6) is drifting away from the ground truth. 
In contrast, the paths generated with visual assistant stay closer to the ground truth path, especially, during the first and the second transects.
We believe that the angle offset between tracks and the ground truth during the third and fourth segments may due to the hovering maneuverings (2-3 minutes) near the ROV deployment hole at the end of the second segment. 

% \vspace{-2ex}
% \textbf{\begin{figure}[h] \centering
% \includegraphics[width=0.5\textwidth,height=0.3\textwidth]{images/aligned_traj.png}
%     \caption{The resulting trajectories for all the cases.} \label{fig:aligned_traj}
% \end{figure}}
% \vspace{-2ex}

\begin{figure*}[t] \centering
 % \vspace{-1ex}
    \makebox[0.49\textwidth]{\small (a) }
    \makebox[0.48\textwidth]{\small (b) }
    % \vspace{+2mm}
    % \\
    \includegraphics[width=0.48\textwidth]{images/plot_trajectories.png} 
    \hspace{+2mm}
    \includegraphics[width=0.48\textwidth]{images/plot_errors.png}
    \caption{The evaluation result. (a) The aligned trajectories for case 1 to case 6. (b) The ATE translation RMSE for X-Y axis. } 
    \label{fig:plot_result}
\end{figure*}

To further compare the performance in different cases, we computed the Absolute Trajectory Error (ATE)~\cite{VIO-EVO_Zhang_2018} in X and Y and the X-Y plane between the ground truth path and aligned each odometry.
The statistical values are listed in Table~\ref{tab:rmse}
and the X-Y errors are presented in Fig.~\ref{fig:plot_result}(b).
Based on that, we could see that the integration of visual measurement into the MSCKF will help with reducing errors. 
Overall, the drift in the Y direction (transversal to the vehicle transects) is higher than in the x direction. 
This may be the fact that the vehicle's sway velocity is slightly small than its surge speed.
Therefore, a lower SNR may be expected in the transversal direction, causing the increased drift.

\begin{table}[h!]
  \begin{center}
    \caption{ATE with RMSE metric for different cases.}
    \label{tab:rmse}
    \begin{tabular}{c|c|c|c|c|c|c} 
      \hline
      \textbf{Case \#} &\textbf{1} &\textbf{2} &\textbf{3} &\textbf{4} &\textbf{5} &\textbf{6}\\
      \hline
      RMSE(X)   &\textbf{0.39}  &0.42  &1.34  &1.23   &1.45  &3.54  \\
      RMSE(Y)   &\textbf{1.82}  &1.86  &2.09  &2.29   &1.45  &2.87  \\
      RMSE(X-Y) &\textbf{1.11}  &1.14  &1.71  &1.76   &1.45  &3.21  \\
      \hline
    \end{tabular}
  \end{center}
\end{table}
\vspace{-1ex}

When comparing the statistical values in Table~\ref{tab:rmse}, we have several findings.
First, visual-fused solutions are better than case 6 which only used the DVL, IMU, and pressure measurements.
Second, case 1 and 2 are better than case 3 and 4. 
This comparison allows us to highlight the benefit of having keyframe selection mechanism which allows a longer translation for a better result in feature triangulation, ultimately affecting the localization.
Third, case 1 is better than case 5 means pressure update actually helped the 2D pose estimation.
Fourth, our method (case 1 with DVL-aided feature enhancement and keyframe selection enabled) produced the lowest RMSE. 
However, case 2 (with DVL-aided feature enhancement disabled but keyframe selection enabled) is only slightly worse than case 1.
This small improvement may be mainly caused by two following reasons.
First, the detected visual features are too close to the vehicle (roughly between 1-2 meters). 
Therefore, the feature's position in z corrected by the DVL measurements is relatively small (even though the improvement is visible in Fig.~\ref{fig:depth_enhance}), resulting in a small impact on the state estimation.
Second, the feature measurements noise is set to 0.09 pixel which is relatively high compared to 0.0035 we set when testing the VIO on simulated data from OpenVINS.
We also tried 0.01 for our data, the localization error was larger than the shown result.
Therefore, we think there are still room for improvement, especially, in the front-end feature tracking.

% \begin{figure}[h]
%     \centering
%     \includegraphics[scale=0.35]{images/error.png}
%     \caption{The errors in X-axis, Y-axis and X-Y plane.}
%     \label{fig:errors}
% \end{figure}
% \vspace{-2ex}




% init alignment RMSE:
%   - DIPO:  0.04372366500998966
%   - VDIPO: 0.042506709947047125
%   - VDIPO-wo-enhance: 0.041838946871800786
%   - VDIPO-wo-enhance-keyframe: 0.04293661481755104
%   - VDIPO-wo-keyframe: 0.04353596573555296 


\section{Conclusion}
Throughout the paper, we first analyze the current evaluation methods for diffusion-based adversarial purification and then propose a recommendation for the reliable evaluation of the robustness of adversarial purification. We further investigate the influence of hyperparameters of the diffusion model on the robustness of the purification. Based on our analysis, we propose a new strategy to maximize the benefit of the purification methods.


% \end{thebibliography}
%\IEEEtriggeratref{20}
\bibliographystyle{IEEEtran}
\bibliography{ref}

\end{document}
