\section{introduction}

To date, many \sota{} data-driven grasp sampling methods have achieved high grasp success rates on completely unknown objects by learning distributions of successful grasps all over the object \cite{morrisonClosingLoopRobotic2018b,mahlerDexNetDeepLearning2017a,zhou2018fully,satish2019policy,kumra2020antipodal,zhu2022sample}. Although impressive, it is worth highlighting that the grasping conditions for these methods are very favorable as the workspace for the robot is often only restricted by the surface the robot and the object rest on. If the focus is instead shifted towards grasping from confined spaces, such as the example of picking from the shelf shown in \figref{fig:shelf_exp_with_grasps}, the grasp success rate fall. One reason the success rate falls when grasping an object resting on a shelf is that the often highly successful top-down approach direction is now impossible to reach, and the robot should instead grasp the object from the side.
%The grasping direction is thus important.  

% \begin{figure}[tb]
%     \centering
%         \includegraphics[width=1\linewidth]{figures/shelf_exp.jpg}
%     \caption{Picking bleach cleanser from a shelf. Given visual information and feasible direction constraint, VCGS proposes grasp for execution.}
%     \label{fig:shelf_exp}
% \end{figure}

In this work, we address the problem of generating approach-constrained grasps. Towards such a goal, we propose \methodname{}: the first \ac{dl}-based data-driven grasp sampler capable of generating approach-constrained grasps in all of $\mathrm{SO}(3)$, and a geometrical method to extract the specific grasp approach directions automatically from the object's \pc{}. To train \methodname{}, we first discretize $\mathrm{SO}(3)$ into a preset number of bins, each associated with a pitch-and-yaw angle. Then, from a given object \pc{} and bin, \methodname{} is trained to reconstruct successful grasps on that object with approach directions within that bin. Together, \methodname{} and the \ac{pc}-based bin selection approach can be seen as a geometrical approach-constrained data-driven grasp sampler. 

We empirically evaluate the performance of \methodname{} to the unconstrained \sota{} 6-\ac{dof} \graspnet{} \cite{mousavian20196} on over 17 million grasps in simulation, and over 100 grasps in two real-world experiments. The simulation results indicate that our method with the \ac{pc} inspired sampling direction generally achieves higher success-over-coverage than the baseline. Furthermore, in the real world, our method achieves a 12\% higher grasp success rate on a shelf-picking task and an 18\% higher grasp success rate on a table-picking task.   

To summarize, the main contributions of this work are:

\begin{itemize}
    \item \methodname{}: a novel 6-\ac{dof} generative grasp sampler that can constrain grasps to approach the object from specific directions in $\mathrm{SO}(3)$ (\secref{sec:generator}).
    \item A geometrically inspired approach for choosing the specific approach directions (\secref{sec:pc_selection}). 
    \item A dataset for training \methodname{} and a thorough discussion on the choice of coordinate systems for representing the grasps in the dataset (\secref{sec:dataset}). 
    \item Extensive simulation and real-world experiments of \methodname{} and \sota{} 6-\ac{dof} \graspnet{}, highlighting specific benefits and drawbacks of our constrained grasp sampler to an unconstrained sampler (\secref{sec:exp}). 
\end{itemize}