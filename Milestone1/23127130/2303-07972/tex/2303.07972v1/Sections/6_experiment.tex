\section{experiment}\label{sec:exp}

We quantitatively evaluate the performance of \methodname{} in both simulated and real-world scenarios. Our evaluations aim to answer the following three key questions: 
\begin{enumerate}
\item How efficient is \methodname{} at sampling successful approach-constrained grasps compared to an unconstrained grasp sampler?
\item What is the impact on grasp success rate when sampling a direction bin based on the \ac{pc} direction over a random bin?
\item How successful is \methodname{} compared to an unconstrained generative grasp sampler in grasping real-world objects in an unconfined and confined space? 
\end{enumerate}

To answer these questions, we conduct three experiments, one in simulation and two in the real world. In the simulation, we benchmark several ablations of \methodname{}, whereas in the real-world experiments, only one is used. Both \methodname{} and \graspnet{} are trained with a latent space size $L=4$ on the \datasetacronomy{} dataset. 

We use the grasp success metric to quantify whether a grasp was successful, which is a binary variable evaluating 1 for a successful and 0 for an unsuccessful grasp. In all experiments, a grasp was deemed successful if the object was picked and remained within the gripper during a predefined movement. Moreover, to only evaluate grasps from specific directions, we removed all grasps outside the specified bins prior to evaluation. Consequently, if a specific direction is given and all generated grasps fall outside the specified bin, we deem that grasp trial unsuccessful.  

\begin{figure}[tb]
    \centering
        % \includesvg[width=0.89\linewidth]{figures/cov_suc_all_10D_1000_1000.svg}
        \includegraphics[width=0.89\linewidth]{figures/cov_suc_all_10D_1000_1000.pdf}
    \caption{The ablation study results with varying discretization resolutions for yaw (y) and pitch (p). The \ac{auc} for each method is depicted in the image legend \bestcolor{}.}
    \label{fig:exp1_ablation_cov_suc}
\end{figure}

% \begin{figure*}[tb]
%     \centering
%         \includegraphics[width=1\linewidth]{figures/shelf_procedure_with_grasps.png}
%     \caption{Real robot experiment of shelf object picking (with grasps).}
%     \label{fig:shelf_exp_with_grasps}
% \end{figure*}


\subsection{Simulation Experiments}
In the simulation experiments, we used the Issac-Gym  simulator \cite{makoviychuk2021isaac} to evaluate  all methods. In this simulator, we initialized the test object as free-floating, used a side-facing depth camera to capture \pcs{}, and a virtual 7-DoF Franka Panda robot for grasping. Gravity was activated after the object was grasped. 

Besides evaluating the grasp success rate, we also evaluated the success-over-coverage rate in simulation, a metric originally proposed for grasping in \cite{mousavian20196}. To evaluate the coverage, we first discretized $\mathrm{SO}(3)$ for each test object into 128 so-called ground-truth bins. Then, we associated each ground-truth grasps for each object with the correct ground-truth bin based on the approach direction, forming 128 different grasp subsets $\matr{G}^{*}=\{\matr{G}_{1}^*,~\matr{G}_{2}^*,~\dots,~\matr{G}_{128}^*\}$. Bins with no ground-truth grasps were ignored. These split allowed us to evaluate grasp coverage between grasps from ground-truth bins that overlap with the bin in \methodname{} that grasps were generated from. A generated grasp covers a ground-truth one if the angle between their approach vectors is less than 10 degrees and the translation distance is less than 2 cm. 

In the simulator, we first performed an ablation over different discretization resolutions to study its effect on the success-over-coverage rate. To this end, we evaluated the following 6 \methodname{} models with varying numbers of yaw and pitch (y/p) bins: (4/1), (4/2), (8/1), (8/4), (16/1), and (16/8). For both \methodname{} and \graspnet{} on each object, we sample 1000 grasps for comparison.

The results of the ablation study are presented in \figref{fig:exp1_ablation_cov_suc}. These results clearly indicate that all ablations of \methodname{} achieve a much higher success-over-coverage rate than the \graspnet{} baseline. The reason \graspnet{} scores a much lower success-over-coverage rate is that it generates grasps all over the object, while \methodname{} leveraged orientation information to generate approach-constrained grasps. We observed that \methodname{} with large pitch resolutions generally performed worse than the other methods. One reason for this is that the pitch bins were generally placed in occluded regions on the object \pc{}, and grasps sampled from such directions generally had lower quality and tended to align with the mean grasping pose from the training data. Interestingly, \methodname{} (16/8) with the highest pitch and yaw resolution performed even worse than \graspnet{}, which we hypothesize stems from many low-quality grasps sampled from the occluded regions.

Finally, we compared the success rates of the different ablations of \methodname{} with the \ac{pc}-based bin selection, random bin selection, and \graspnet{}. We sampled and evaluated 400 grasps for each model and executed the 20 best grasps according to their grasp score. The results, presented in \figref{fig:simresult_suc_rate}, demonstrate that \methodname{} with the \ac{pc}-based bin selection consistently outperforms the other models, achieving, on average, a 7\% higher grasp success rate than a random bin and an 8\% better success rate than \graspnet{}, highlighting, once again, the effectiveness of generating grasps from geometrically meaningful approach directions as discussed in \cite{balasubramanian2012physical}. Interestingly, the average success rate of \methodname{} with random bin selection is on par with \graspnet{}, indicating that our method can find good grasps even from random approach directions. 

\begin{figure}
	\centering
	% \input{figures/sim_result.tikz}
        % \includesvg[width=1\linewidth]{figures/suc-cov.svg}
        \includegraphics[width=1\linewidth]{figures/suc-cov.pdf}
	\vspace{-1em}
	\caption{The per model grasp success rates with varying yaw and pitch (y/p) resolution. The last column shows the average grasp success rate for all constrained models and \graspnet{}.}
	\label{fig:simresult_suc_rate}
	\vspace{-1em}
\end{figure}

\subsection{Real-world Experiments}

\begin{figure}[tb]
    \centering
        \includegraphics[width=0.8\linewidth]{figures/object_set_background_number.png}
    \caption{Experimental objects. From 1 to 10: bleach cleanser, mustard bottle, sugar box, cracker box, Windex bottle, metal bowl, tennis ball, metal mug, banana, and spring clamp.}
    \label{fig:full_object_set}
\end{figure}

\begin{figure}[tb]
    \centering
        \includegraphics[width=1\linewidth]{figures/table_procedure_highlight.png}
    \caption{An example of \methodname{} generating and picking the bleach cleanser from the table. (a) The object is placed in a specific pose. (b) The $\mathrm{SO}(3)$ space around the object is discretized into a pre-defined yaw and pitch resolution where the red bin is the \ac{pc} one and the blue is the top-down one. (c) The sampled grasps with the highest-scoring one in red. (d) The robot successfully reaches the red grasp pose and (e) moves back to its start pose.}
    \label{fig:table_exp_pca}
\end{figure}

%Given the selected bins and partial point cloud, VCGS proposes grasps which are ranked by the evaluator. We choose the generated grasp with the highest score (red) for robot execution. Finally, the robot executes the acceleration test.

\begin{figure}[tb]
    \centering
        \includegraphics[width=1\linewidth]{figures/suc_fail.png}
    \caption{Some successful and failed grasp examples for table-picking (top row) and shelf-picking (bottom row). The first three columns depict successful grasps, while the last column shows two different failure cases: collision (top image) and slip (bottom image).}
    \label{fig:suc-fail-case}
\end{figure}

\begin{table}[ht]
    \centering
    \begin{adjustbox}{max width=\linewidth}
    \setlength{\tabcolsep}{4pt}
         \begin{tabular}{lccccccc}
            \toprule
            \textbf{Experiment}& \multicolumn{4}{c}{\textbf{Table-picking}} & \multicolumn{3}{c}{\textbf{Shelf-picking}}\\
            \cmidrule(l){2-5} \cmidrule(l){6-8}
            Object & GN & GN-f & \methodname{}-t & \methodname{} & GN & GN-f & \methodname{}\\ \midrule
            1. Bleach cleanser & 4/5 & 1/5  & 1/5   & 4/5  & 0/5  & 3/5 & 3/5 \\ 
            2. Mustard bottle  & 1/5 & 1/5  & 1/5   & 3/5  & 1/5  & 2/5 & 1/5 \\ 
            3. Sugar box       & 5/5 & 4/5  & 5/5   & 5/5  & 0/5  & 1/5 & 4/5 \\ 
            4. Cracker box     & 4/5 & 5/5  & 5/5   & 5/5  & 1/5  & 0/5 & 3/5 \\
            5. Windex bottle   & 4/5 & 4/5  & 5/5   & 5/5  & 0/5  & 1/5 & 0/5 \\
            6. Metal bowl      & 5/5 & 5/5  & 5/5   & 5/5  & 1/5  & 5/5 & 4/5 \\
            7. Tennis ball     & 1/5 & 4/5  & 4/5   & 5/5  & 1/5  & 0/5 & 3/5 \\
            8. Metal mug       & 5/5 & 5/5  & 5/5   & 5/5  & 3/5  & 3/5 & 3/5 \\
            9. Banana          & 0/5 & 2/5  & 3/5   & 3/5  & 0/5  & 0/5 & 0/5 \\
            10. Spring clamp    & 0/5 & 2/5  & 2/5   & 2/5  & 0/5  & 0/5 & 0/5 \\
            \midrule 
            Avg. success rate  &58\%&66\% &72\% & \textbf{84\%}  & 14\% & 30\% & \textbf{42\%}\\
            Ratio grasps kept    & -- & 8.63\% &99.80\% &100\%  & -- & 13.28\% & 99.59\%\\
            \bottomrule
        \end{tabular}
    \end{adjustbox}
    \caption{Experiment results for table-picking and shelf-picking. GN-f denotes \graspnet{} with filtering and \methodname{}-t denotes \methodname{} with top-down only grasping.}
    \label{result_exp}
    \end{table}

In the real-world experiments, we conducted two experiments: a table-picking experiment and a shelf-picking experiment. The table picking experiments shown in \figref{fig:table_exp_pca} is an example of an unconfined picking environment while the shelf picking experiment is an example of a confined picking environment. In both experiments, we used a 7-\ac{dof} Franka Panda robot for grasping, and a Kinect V2 mounted on a tripod for capturing \pcs{} of the objects. An Aruco marker \cite{arucomarker} placed on the table was used for extrinsic camera calibration.% Hardware communication was established using ROS2. 

The objects used in the experiments are shown in \figref{fig:full_object_set} and were selected due to their diverse geometries. All of these objects except the blue metal mug were from the YCB dataset~\cite{calli2017yale}. In both experiments, each object was placed on the plane in five equally distributed orientations (0\textdegree, 72\textdegree, 144\textdegree, 216\textdegree, and 288\textdegree). For each object orientation, we sampled 200 grasps and removed all that would collide with the environment. We then scored the remaining grasps with the grasp evaluator and executed the first reachable grasps of the top 10 scoring grasps. In total, this amounts to 50 grasp trials per method per experiment. A grasp trial was successful if the object remained within the gripper after the robot returned to the starting position and rotated the hand by ±90 degrees. If the object dropped or if no reachable grasps were found the trial was a failure.

We compare the 8/4 \methodname{} from the simulation experiments against two \graspnet{} versions: one where all sampled grasps were kept, and another one where all grasps with approach directions outside the bins specified by \methodname{} were removed. In both real-world experiments, we evaluated the grasp success rate and the ratio of grasps kept, which measures the sample efficiency of the methods. In the table-picking experiment, we select the bins that aligned with the second \ac{pc} and the top-down grasping direction. In the shelf-picking experiment, we chose the bins shown in \figref{fig:shelf_exp_with_grasps} that would result in collision-free grasps. 

The results for both experiments are presented in Table\ref{result_exp}, and some example grasps are shown in \figref{fig:suc-fail-case}. Overall, \methodname{} outperformed \graspnet{}, with an 18\% higher average grasp success rate on table-picking and 12\% higher on shelf-picking. We attribute this improvement to \methodname{}'s ability to efficiently generate high-quality grasps in specific directions, highlighted by a 99.5\% ratio of grasps kept compared to less than 13.3\% for \graspnet{}. We also observe that the reason \graspnet{} with no removed grasps has such a low grasp success rate is because a large portion of the generated grasps was in collision with the environment, especially for shelf picking.

Another interesting finding is that \methodname{} with top-down and \ac{pc}-based bins achieved a higher success rate on the Bleach cleanser and Mustard bottle in table picking than when the top-down only bin was used. For such objects, the grasps approaching along the second \ac{pc} are more stable than the top-down direction. However, the combination of \ac{pc} and top-down does not improve the success rate on box-shaped objects, as both approach directions are good for grasping such objects. 
