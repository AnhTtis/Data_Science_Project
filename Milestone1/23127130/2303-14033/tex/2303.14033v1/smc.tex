\documentclass[pdflatex,sn-aps]{sn-jnl}
%%%% Standard Packages
%%<additional latex packages if required can be included here>
%%%%
\usepackage{amsmath}
\usepackage{amssymb}
\usepackage{graphicx}

\jyear{2022}%

\begin{document}

    \title[Data-driven estimation of transfer integrals in undoped cuprates]{Data-driven estimation of transfer integrals in undoped cuprates}

    %%=============================================================%%
    %% Prefix   -> \pfx{Dr}
    %% GivenName        -> \fnm{Joergen W.}
    %% Particle -> \spfx{van der} -> surname prefix
    %% FamilyName       -> \sur{Ploeg}
    %% Suffix   -> \sfx{IV}
    %% NatureName       -> \tanm{Poet Laureate} -> Title after name
    %% Degrees  -> \dgr{MSc, PhD}
    %% \author*[1,2]{\pfx{Dr} \fnm{Joergen W.} \spfx{van der} \sur{Ploeg} \sfx{IV} \tanm{Poet Laureate} 
    %%                 \dgr{MSc, PhD}}\email{iauthor@gmail.com}
    %%=============================================================%%

    \author*[1]{\fnm{Denys Y.} \sur{Kononenko}}\email{d.kononenko@ifw-dresden.de}
    \author[1]{\fnm{Ulrich K.} \sur{R{\"o}{\ss}ler}}
    \author[1,2]{\fnm{Jeroen} \spfx{van den} \sur{Brink}}
    \author*[1]{\fnm{Oleg} \sur{Janson}}\email{olegjanson@gmail.com}

    \affil[1]{\orgname{Institute for Theoretical Solid State Physics}, \orgaddress{\city{Dresden}, \postcode{01069}, \country{Germany}}}
    \affil[2]{\orgdiv{Institute for Theoretical Physics}, \orgname{TU Dresden}, \orgaddress{\city{Dresden}, \postcode{01069}, \country{Germany}}}

    \abstract{Undoped cuprates are an abundant class of magnetic insulators, in which the synergy of rich chemistry and sizable quantum fluctuations leads to a variety of magnetic behaviors. Understanding the magnetism of these materials is impossible without the knowledge of the underlying spin model. The typically dominant antiferromagnetic superexchanges can be accurately estimated from the respective electronic transfer integrals. Density functional theory calculations mapped onto an effective one-orbital model in the Wannier basis are an accurate, albeit computationally cumbersome method to estimate such transfer integrals in cuprates. We demonstrate that instead a Gaussian Process Regression (GPR) model, trained on the results of high-throughput calculations, can predict the transfer integrals using the crystal structure as the only input. Descriptors of our GPR model encode the spatial configuration and the chemical composition of the local crystalline environment. A virtual toolbox employing our model can be readily employed to determine leading superexchange paths as well as for rapidly assessing the relevant spin model in yet unknown cuprates.}

    \keywords{quantum magnetism, machine learning, high-throughput calculations, transfer integrals}

    \maketitle

    \section{Introduction}\label{sec:intro}
    The data-driven approach accompanied by modern machine-learning (ML) techniques
    becomes an increasingly important tool of scientific investigations across many
    domains of physics. From quantum to fluid mechanics~\cite{tsubaki.prl.20,
    brunton.annurev-fluid.20} learning from data facilitates descriptions of
    complex phenomena for which analytical approach is prohibitively challenging.
    The employment of ML in solid-state physics and material science is
    particularly contemplated: the sheer amount of collected experimental and
    computed records propels the community to design data-driven frameworks for prediction
    of materials properties~\cite{scmidt.npj.compmat.19, xie.prl.20}.

    The application of ML for problems of solid-state physics is not
    straightforward. One of the key challenges is to represent periodic (crystal)
    or finite (molecule, local crystal environment, etc.) atomic systems as
    descriptors -- data structures amenable to ML methods. Such descriptors must be
    invariant with respect to the choice of the unit cell (crystals) or to global
    rotations in a finite system.  Several classes of descriptors have been
    developed for material properties prediction: Coulomb
    matrix~\cite{rupp.prl.12}, partial radial distribution
    function~(PRDF)~\cite{schtt.prb.2014}, smooth overlap of atomic
    positions~(SOAP)~\cite{bartok.prb.2013}, diffraction
    fingerprint~(DF)~\cite{ziletti.natcomm.2018} and three-dimensional~(3D) Zernike
    descriptor (3DZRD)~\cite{venkatraman.joc.09}.  The latter were designed for the
    characterization of 3D shapes~\cite{novotni2003, novotni.shape.04} and
    successfully employed for comparison of molecules~\cite{sael.prot.2008,
    mak.jmgm.2008, sael2008}. These descriptors are invariant with respect to the
    number of chemical species in the data set, they store detailed information
    about the spatial arrangement, and do not require additional simulation
    software. These features as well as the compact size of the resulting data
    structures make 3DZRD ideally suited for a description of diverse, dissimilar
    crystalline environments.

    Another key challenge is the construction of descriptors that represent
    material properties. For instance, it is widely accepted that electronic,
    magnetic, and topological properties of bulk materials are rooted -- in a
    highly nontrivial way --- in their electronic structure. However, using the
    complete band structure of a material as a universal descriptor is not possible
    for a number of reasons: different number of bands, non-universal
    discretization of the Brillouin zone, huge dimensionality etc. A more practical
    approach is to restrict the description to the states relevant for the physical
    quantity of interest. Naturally, this is possible only for a certain class of
    materials and only for specific physical property.

Following this idea, we apply a data-driven approach to assess spin models in
undoped cuprates -- stoichiometric inorganic materials containing divalent
copper and oxygen atoms. In contrast to their doped counterparts -- the
high-temperature cuprate superconductors~\cite{Plakida2010htc} -- undoped
cuprates are magnetic insulators with the $3d^9$ electronic configuration of
Cu$^{2+}$. The sizable Jahn-Teller distortion lifts the orbital degeneracy,
giving rise to half filling and localized $S$=$\frac12$ spins. Owing to the
plethora of structure types and the quantum limit assured by $S$=$\frac12$,
undoped cuprates exhibit a variety of magnetic
behaviors~\cite{sahadasgupta2021}, from simple quantum dimers and spin chains
-- to exotic collective behaviors such as the spin-liquid regime in
herbertsmithtite $\gamma$-Cu$_3$Zn(OH)$_6$Cl$_2$~\cite{helton07}, Bose-Einstein
condensation of magnons in Han purple BaCuSi$_2$O$_6$~\cite{jaime04}, or bound
magnon states in volborthite
Cu$_3$V$_2$O$_7$(OH)$_2\cdot2$H$_2$O~\cite{kohama19}.

Understanding the magnetic properties of cuprates requires the knowledge of the
underlying spin model. While exchange anisotropies are generally present and
can alter the magnetic properties, the backbone of spin models are isotropic
interactions, and the relevant Heisenberg Hamiltonian is the following sum:
       \begin{align}\label{eq:heisenberg}
           \mathcal{H} = \frac12\sum_{ij} J_{ij}\left(\mathbf{S}_i \cdot \mathbf{S}_j\right),
       \end{align}
where $\mathbf{S}_i$ and $\mathbf{S}_j$ are spin operators on sites $i$ and
$j$. The set of relevant magnetic exchange integrals $\{J_{ij}\}$ determines
the spin model. It is important to note that each individual $J_{ij}$ term is a
sum of antiferromagnetic ($J^\text{AF}_{ij} < 0$) and ferromagnetic
($J^\text{FM}_{ij} < 0$) contributions that are driven by competing
processes~\cite{Goodenough_Magnetism}. Commonly, $J^\text{AF}_{ij} \gg
\lvert{}J^\text{FM}_{ij}\rvert$, with the exception of short-range exchanges
for which the ferromagnetic contribution can become dominant.

The antiferromagnetic contribution is a textbook example of the
superexchange mechanism and can be derived via second-order perturbation theory
of the Hubbard model in the strong-coupling limit at half-filling as
$J^\textsc{AF}_{ij} = 4 t_{ij}^2 /U_\text{eff}$~\cite{hubbard63, anderson59,
auerbach2000}. Here, $U_\text{eff}$ is the Coulomb repulsion within an
effective molecularlike orbital, which in most cuprates is dominated by
$3d_{x^2 - y^2}$ orbital of Cu and $\sigma$-bonded $2p$ orbitals of O. There is
empirical evidence that $U_\text{eff}$ from the range 4--5 eV gives a proper
description of the magnetism of cuprates~\cite{belik2004, johannes2006,
janson2009}. Hence, the knowledge of transfer integrals $t_{ij}$ paves the way
to a quantitative assessment of the spin model in the majority of cuprate
materials. Yet, extracting $t_{ij}$ directly from the structural information is
essentially impossible; instead, it requires first-principles calculations
followed by an additional modeling.

To overcome this challenge, we propose a data-driven approach for prediction of
transfer integrals in cuprates, which requires the crystal structure as the
only input.  Our approach is based on the local crystal environment description
utilizing 3DZRD.  The local crystal environment descriptor is used as input for
the ML model which is trained on the results of high-throughput
density-functional-theory (DFT) calculations for hundreds of cuprate materials.
DFT calculation for each material is followed by automatized Wannierization and
a manual quality control. The trained model is wrapped into a freely accessible
web application \footnote{https://smc-t.ifw-dresden.de/} that can be used for a
quick estimation of relevant transfer paths in new cuprate materials.

    \section{Data set generation}
    We start with the description of the high-throughput DFT calculations employed
    for the generation of the data set of transfer integrals. The list of materials
    contains 672 unique structures of undoped cuprates. The structures were
    filtered out from the 10\,710 cuprate structures stored in the Inorganic
    Crystal Structure Database (ICSD)~\cite{bergerhoff.crystallographic.1987}. For this screening, the
    following criteria were consecutively applied: (i) the presence of Cu$^{2+}$
    ions, (ii) electroneutrality (zero total charge), (iii) absence of sites with
    fractional occupancies, (iv) the minimal inter-atomic distance of 0.5\,\r{A},
    and (v) the absence of other magnetic atoms beyond Cu~\cite{suppl}. The latter criterion is necessary to
    filter out compounds with multiple magnetic atoms, where the presence of
    additional bands in the relevant energy range may render the effective
    one-orbital model inapplicable and its results misleading.  For the
    analysis of the crystal structures we used the pymatgen python
    library~\cite{Ong2013pymatgen}.

    For each structure, we performed DFT calculations to construct Wannier
    Hamiltonians and consequently determined the transfer integrals. All DFT
    calculations were performed using the generalized gradient approximation
    (GGA)~\cite{perdew.prl.1996} with the full potential code FPLO of version
    18.00-52~\cite{koepernik.prb.1999}. The computational workflow comprised
    several steps. First, scalar-relativistic nonmagnetic DFT calculations were
    carried out and the Hellmann-Feynman forces were calculated. For further processing, we considered only those structures where the calculated forces do not exceed 0.1\,eV/\r{A}.
    %only converged calculations with the individual
    %forces on atoms below 0.1\,eV/\r{A}. 
    Second, we calculated the orbital-resolved density of states (DOS) and band structure.
    From the orbital-resolved DOS, the energy interval which contains the copper $3d_{x^2 - y^2}$ bands was
    determined. The energy interval is selected such that the contribution of the magnetic $3d_{x^2 - y^2}$ orbital in the total density of Cu $3d$ states exceeds 5 \%.
    The determined energy window $[\mathscr{E}_\textsc{min}, \mathscr{E}_\textsc{max}]$
    was adopted for Wannierization in the next step.
    Third, the Wannier transformation procedure was performed to obtain the
    effective one-orbital Hamiltonian $H$ in the Wannier basis. We used copper
    $3d_{x^2-y^2}$ orbitals as projectors and the interval $[\mathscr{E}_\textsc{min},
    \mathscr{E}_\textsc{max}]$ as the energy window to construct the Wannier
    functions~(WF). The latter is necessary to discriminate the target antibonding
    orbital (crossing the Fermi energy) from its bonding sibling at the
    bottom of the valence band. The transfer integral between two WF $w_i$ and $w_j$
    placed at copper sites $i$, $j$ is determined as (real) Hamiltonian matrix
    element $t_{ij} = \langle{}w_i\lvert{}H\rvert{}w_j\rangle$. The details on the construction
    of Wannier functions and the construction of the respective tight-binding
    models are provided in the papers~\cite{koepernik2021arxiv, eschrig2009}.

    After the calculation pipeline was completed, we obtained a list of transfer
    integrals $\{t_{i j}\}$ that connected $i$-th and $j$-th copper sites
    situated at the distance $r_{ij}$ from each other for all valid structures~\cite{suppl}.
    % ,enumerated by index $p$~\cite{suppl}.  
    For construction of the data set we select transfer integrals larger than 5 meV with Cu..Cu spacing less than 8 \r{A}.
    The distribution of calculated transfer integrals $t_{ij}$ among Cu..Cu distances
    is shown in Fig.~\ref{fig:fig_data}. The crystal chemistry of cuprates sets a
    natural lower limit for the bond lengths; accordingly, there is no transfer
    integral with the distance less than 2.4~\r{A}\ in the data set. Remarkably,
    for the vast majority of transfer integrals, the absolute values are below
    0.2\,eV. This natural imbalance of the data set will inevitably affect the
    performance of predictions.

    %==================================================================\
        \begin{center}
            \begin{figure}
                % fig 1
                \includegraphics[width=.667\textwidth]{fig_data.pdf}
                \caption{(Color online) Transfer integrals obtained from the DFT calculations  as a function of the Cu..Cu distance. Region I harbors hoppings between edge-sharing (a) and corner-sharing (b) CuO$_4$ plaquettes, while region II is dominated by transfer integrals between CuO$_4$ plaquettes that do not share oxygen atoms (c). The inset shows the distribution of the computed transfer integrals.
                }
                \label{fig:fig_data}
                % fig 2
                \includegraphics[width=.667\textwidth]{fig_main.pdf}
                \caption{(Color online) Schematics of the workflow: selection of the local crystal environment from the cuprate crystal structure, generation of rotationally invariant descriptor $\vec{D}$ via decomposition of the local crystal environment function in the truncated basis of 3DZF and prediction of the transfer integral $t_{ij}$ with ML algorithm trained on the data set from DFT calculations. The illustrating example Ba$_2$CuHgO$_4$ (ICSD Identifier 75724) hosts pairs of corner-sharing CuO$_4$ square-like plaquettes.
                }
                \label{fig:fig_crenv}
            \end{figure}
        \end{center}
    %==================================================================/

        % Introduce the crystal environment function 

        \section{Crystal Environment Descriptor}
        To describe the crystal environment, we first determine the midpoint $\vec{p}$  between a given pair of copper atoms and build a sphere with the empirically determined threshold radius $R^\textsc{th} = \max(3, r_{i j} / 2 + 0.2)$ \r{A} centered at $\vec{p}$. Next, all atoms in the sphere are enlisted in the crystal environment alongside with nearest neighbors of $i$ and $j$. We consider nearest neighbors as atoms distanced from $i$ or $j$ not farther than 2.5 \r{A}. After the local crystal environment is assembled, we shift the coordinate system origin to the centroid (the point between copper pair) $\vec{p}$ and normalize atoms coordinates by $r_0 = 6$ \r{A} to fit the crystal environment in the unit ball.
        To construct a robust representation of the local crystal environment we introduce the piecewise function of site positions $\mathcal{I}(\vec{r})$. The function $\mathcal{I}$ equal to the $q$-th atom oxidation number $O_q$ in the ball with center at the position of  $q$-th atom $\vec{r}_q$ and radius $R_q$ equals to the ionic radius of the atom
        \begin{align}\label{eq:cryst_env_func}
        \mathcal{I}(\vec{r}) =
        \begin{cases}
        O_q & \Vert \vec{r}_q - \vec{r}\Vert \leq R_q, \\
        0 & \text{otherwise}.
        \end{cases}
        \end{align}
        The normalization factor $r_0$ is a sum of the maximal considered Cu-Cu distance $\max \Vert \vec{r}_{i j} \Vert = 4$ \r{A} and a maximal considered ionic radius $2$ \r{A}.
        The function $\mathcal{I}(x,y,z)$ describes the spatial configuration and chemical composition of the crystal environment placed in the unit ball with $x^2 + y^2 + z^2 \leq 1$. An example of the local crystal environment defined by~\eqref{eq:cryst_env_func} is shown in Fig.~\ref{fig:fig_crenv}.
     
        We describe the selected crystal environment $\mathcal{I}$ in the form of a finite-dimensional vector.
        Such representation provides a robust way for numerical operations with crystal environments, e.g. similarity and sorting.
        To obtain the finite vector representation of the crystal environment we decompose the $\mathcal{I}(x,y,z)$ in the truncated basis of three-dimensional (3D) Zernike functions (3DZF)  $Z^{m}_{nl}$ which are defined as follows~\cite{canterakis96, canterakis99, novotni.shape.04}
        \begin{align}\label{eq:zernike_func_3d}
        \begin{split}
        Z_{nl}^{m}(r, \theta, \phi) &= R_{nl}(r) Y_{lm}(\theta, \phi), \\
        R_{nl}(r) &= \sum_{\nu = 0}^{(n-l)/2} Q_{l\nu} r^{2 \nu + l}, \\
        Q_{l\nu} &= \dfrac{(-1)^{k + \nu}}{4^{k}} \sqrt{\dfrac{2l + 4 k + 3}{3}} \dfrac{ \binom{2 k}{k} \binom{k}{\nu} \binom{2 (k + l + \nu)+1}{2 k} }{\binom{k + l + \nu}{k}},
        \end{split}
        \end{align}
        where indices $n$ and $l$ are positive integers which satisfy condition $n \geq l$; $m$ changes from  $-l$ to $l$ with constraint $(n-l)$ is even number; $k = (n - l) / 2$ and $Y_{lm}(\theta, \phi)$ are spherical harmonics, and $(r, \theta, \phi)$ are spherical coordinates~\cite{nist.math.handbook}.
        For convenience, we use Cartesian coordinates $(x,y,z)$ representation
        of 3DZF implying change of coordinates: $\mathcal{Z}_{nl}^{m}(x, y, z) = Z_{nl}^{m}\left(\sqrt{x^2 + y^2 + z^2}, \arctan{\sqrt{x^2 + y^2} / z}, \arctan{y / x}\right)$.
        3DZF form the complete basis of orthogonal functions in the unit ball, so that the function $\mathcal{I}(x,y,z)$ defined in the unit ball $x^2 + y^2 + z^2 \leq 1$ can be expanded in the introduced basis~\cite{morais.mathematics-of-computation.2014}.

        The decomposition coefficients read
        \begin{align}\label{eq:zernike_moment_3d}
        c_{nl}^{m} = \dfrac{1}{V} \int_{-1}^{1}\int_{-1}^{1}\int_{-1}^{1} \mathcal{I}(x, y, z) \mathcal{Z}_{nl}^{m}\left(x,y,z\right) \mathrm{d}x\mathrm{d}y \mathrm{d}z,
        \end{align}
    where the normalization factor is the volume of the unit ball $V = 4 \pi / 3$~\cite{suppl}.


        Note, $c_{nl}^{m}$ is not invariant with respect to rotations of the crystal environment $\mathcal{I}$. Rotationally invariant characteristics can be obtained by assembling the vector $\vec{C}_{nl}$ whose components are all $(2 l + 1)$ coefficients with different $m$ for given pair of $n$ and $l$. The norm of obtained vector $\Vert\vec{C}_{nl}\Vert = C_{nl}$ determined as
        \begin{align}\label{eq:zernike_rot_inv_moment_3d}
        C_{nl} = \left\Vert c_{nl}^{0}, ..., c_{nl}^{l-1}, c_{nl}^{l} \right\Vert,
        \end{align}
        is invariant with respect to the rotation of the crystal environment, thus the pre-alignment is not required.

        We introduce the finite dimensional vector-descriptor of the crystal environment $\mathcal{I}$ as
        \begin{align}\label{eq:zernike3d_descriptor}
        \vec{D} = \left( C_{00}, C_{11}, C_{20}, ..., C_{n_\textsc{max}l_\textsc{max}}, r_{ij} \right),
        \end{align}
    where copper-copper distance $r_{ij}$ is incorporated into the descriptor as well.
    The size of the descriptor $\vec{D}$ is determined by the cut-off order of the Zernike 3D moments $n_\textsc{max}$ and corresponding $l_\textsc{max}$ in the truncated basis. The size of the 3DZF basis grows with $n_\textsc{max}$
    as the sum of the series $\sum_{n=0}^{n_\textsc{max}}(n^2 + 3 n + 2) / 2$. In the present work, we chose the cut-off order $n_\textsc{max} = 25$.
    The vector  $\vec{D}$ encodes the information about spatial configuration and chemical composition of the crystal environment, allowing the introduction of the mapping of $\vec{D}$ on the transfer integral.
% A description of such a problem is given in the next section.

        %==================================================================\
        \begin{center}
                \begin{figure}[!t]
                        \includegraphics[width=.667\textwidth]{fig_gpr_eval.pdf}
                        \caption{(Color online) Performance of the GPR model on the test and training data sets for random train-test split.
                        The GPR shows $R^2 = 0.79$, RMSE = 27 meV and MAE = 20 meV on the training data set and $R^2 = 0.81$, RMSE = 36 meV and MAE = 25 meV on the test data set.
                        The inset figure shows the distribution of the GPR model error for the test data set with $\mu$ and $\sigma$ are mean value and standard deviation of the errors. The solid line corresponds to the normal distribution with parameters $\mu$ and $\sigma$.
                        }
                        \label{fig:model_eval}
                \end{figure}
        \end{center}
        %==================================================================/

        \section{Transfer Integral Prediction}
        Our high-throughput DFT calculations yielded $N = 1513$ local crystal environments $\{\vec{D}\}$ with corresponding transfer integrals $\{t_{ij}\}$. The obtained data set is imbalanced: by taking 40\,meV as the cutoff, we end up with 405 ``large'' versus 1108 ``small'' transfer integrals. To balance the data set, we randomly pick 405 environments featuring ``small'' transfer integrals, and build the model for the prediction of the continuous-valued attribute $t_{ij}$ associated with the local crystal environment descriptor $\vec{D}$.
        To solve this regression problem we tested the following models: (i) linear (LIR), (ii) random forest~(RFR)~\cite{Breiman_2001} (iii) Gaussian process regression~(GPR)~\cite{rasmussen2005} models.
        As a metric for the regression model performance with predictions $\tau$ we use:
        (i) the coefficient of determination
        \begin{align}\label{eq:r2}
        R^2 = 1 - \dfrac{S_\textsc{reg}}{S_\textsc{tot}},
        \end{align}
        where $S_\textsc{reg} = \sum_{p=1}^M (t^p_{ij} - \tau^p)^2$ is a sum of squared residuals of the regression model and $S_\textsc{tot} = \sum_{p=1}^M (t^p_{ij} - \overline{t_{ij}})^2$ is a total sum of squares with $\overline{t_{ij}}$ being the mean value of transfer integral in the test data set with $M$ samples.
        (ii) the root mean squared error $R^2$
        \begin{align}\label{eq:rmse}
        \text{RMSE} = \sqrt{\dfrac{1}{M} \sum_{p=1}^M (t^p_{ij} - \tau^p)^2},
        \end{align}
    and (iii) mean absolute error
    \begin{equation}\label{eq:mae}
        \text{MAE} = \dfrac{1}{M} \sum_{p=1}^M \vert t^p_{ij} - \tau^p \vert.
    \end{equation}
    For RMSE squared errors of the model are included in the average makes
    this measure more sensitive to outliers. As more variance in predictions a larger RMSE. The MAE provides mean of linear
    scores with all errors weighted equally.

    \begin{table}[h]
        \centering
        \begin{tabular}{|l|r|r|r}
                \textbf{Model} & $\overline{\text{RMSE}}$, \text{meV} & $\overline{\text{MAE}}$, \text{meV} \\
                \hline
                LIR       & 50 $\pm$ 2.4 & 38 $\pm$ 2.2  \\
                RFR       & 42 $\pm$ 5.4 & 30 $\pm$ 2.5  \\
                GPR       & 38 $\pm$ 4   & 28 $\pm$ 2
        \end{tabular}
        \caption{Results of k-fold cross-validation of selected models. The average value of RMSE and MAE on folds is given alongside with the standard deviation.}
        \label{tab:cv_res}
    \end{table}

        For model selection, we used the $k$-fold cross-validation procedure.
        According to this method, the entire data set is split into $k$ approximately equal parts (folds).
        Each ML model is trained on the $k-1$ folds and evaluated on one fold. The $k$-fold procedure was implemented using the scikit-learn library~\cite{scikit-learn} with a number of folds $k=6$. Thus the number of test points $M$ for evaluation of the model during the cross-validation process equals the number of points in the folds: 252 for the first five folds and 253 for the last fold. The results of the cross-validation are presented in the Table~\ref{tab:cv_res}.
        During the training and testing of the model on each cross-validation step, we apply the scaling of the features and targets to lie between zero and one. Furthermore, the model-based feature selection approach is employed to select features that firmly influence the value of the transfer integral. For this purpose, the auxiliary RFR model is trained on the test data and used for the determination of feature importance in the input $\vec{D}$. The descriptors with features whose importance exceeds the threshold of 0.05 are used for the training and validation of each ML model.
        The model selection procedure shows that GPR has the best performance among the selected models, see Table.~\ref{tab:cv_res}. In particular, GPR has the lowest average errors, $\overline{\text{MAE}}$ = 28 meV and $\overline{\text{RMSE}}$ = 38 meV  with the lowest standard deviation of 4 and 2 meV respectively.
        We also evaluated the GPR model on the random test-train split with 20 \% of the data from each stratum (``small'' and ``large'' transfer integrals) allocated for the test subset.
        The prediction of transfer integrals for the test set is shown in Fig.~\ref{fig:model_eval} as a scatter plot of the calculated values versus predicted ones.

    \section{Discussion}
    For parent compounds of high-temperature superconductors, such as La$_2$CuO$_4$, the GPR model yields -- as expected -- the frustrated square lattice model with a dominant $t_1$ and a considerably smaller $t_2$. Similarly, the dominance of $t_1$ is correctly reproduced for the quasi-one-dimensional Sr$_2$CuO$_3$~\cite{rosner97}, another compound with corner-sharing connections between CuO$_4$ squares. For linarite PbCuSO$_4$(OH)$_2$ featuring edge-sharing chains, it correctly recognizes the relevance of first- and second-neighbor transfer integrals along the chains, and correctly identifies the leading interchain coupling~\cite{heinze22}.

    As a less trivial case, we consider two isostructural natural minerals in which Cu$^{2+}$ atoms form a kagome lattice: kapellasite $\alpha$-Cu$_3$Zn(OH)$_6$Cl$_2$ and haydeeite $\alpha$-Cu$_3$Mg(OH)$_6$Cl$_2$. The relevance of the cross-hexagon coupling $t_d$ and the corresponding magnetic exchange was suggested based on DFT results~\cite{janson08, jeschke13} and confirmed experimentally~\cite{fak12, boldrin15}. (As a side note, the $J_d$ exchange is the principal source of frustration in these systems, because the nearest-neighbor exchange $J_1$ is ferromagnetic.) Here, we consider crystal structures of kapellasite and haydeeite that were determined by neutron diffraction; these structures are not in the ICSD and hence were not used for training.  The GPR model yields nearly identical results for both materials, suggesting the leading $t_1\simeq130$\,meV, plus sizable $t_2$ and $t_d$ of about 60\,meV each. The $t_1$ and $t_d$ values are 25-30\,\% larger than in first-principles calculations~\cite{janson08}. Given that the latter used a different functional and a different structural input, the agreement is very good. Yet, the relevance of $t_2$ in the GPR model is a spurious result which is at odds with the DFT calculations and experiments.

    Is the accuracy of the GPR model sufficient to eliminate the need of doing first-principle calculations? Unfortunately, not. One problem is the strong correlation between the magnitude of the transfer integral and on the Cu..Cu separation. While on the average shorter distances indeed correspond to larger $\|t_{ij}\|$, in many materials it is not the case. For instance, in Bi$_2$CuO$_4$ the leading coupling operates between the structural chains formed by stacks of CuO$_4$ squares~\cite{janson07}. Yet, the GPR model suggests that two strongest couplings connect first and second neighbors in these chains. Also for the spin-dimer compound Cu$_2$TeO$_5$, the intradimer coupling is incorrectly assigned to the shortest Cu..Cu connection, while DFT calculations reveal that magnetic dimers do not coincide with the structural ones~\cite{das08, ushakov09}.

    Another aspect is the scarcity or even uniqueness of particular local environments, which inevitably leads to undersampling and hence sizable error bars for respective transfer integrals. A prominent example is the nearest-neighbor coupling in the quasi-one-dimensional CuSe$_2$O$_5$~\cite{janson2009}. This transfer integral is facilitated by the $[$Se$_2$O$_5]^{2-}$ anionic group connecting two CuO$_4$ squares that are at an angle to each other. The GPR model underestimates this coupling by a factor of four, which is remarkable and surprising given that the Cu..Cu separation is rather short. We believe that this discrepancy can be traced back to the scarcity of such structural motive in the data set.

    What are possible ways to enhance the accuracy? An apparent solution is to extend the data set by including structures that are not represented in the ICSD. Also a revision of materials that were filtered out due to failed Wannierization can make the data set bigger. Less obvious, but arguably more significant improvement can be achieved by amending the crystalline environments descriptors. This can be done by taking the connectivity of atoms within a chosen sphere into account and a more explicit consideration of charge densities.

        \section{Conclusions}
We constructed a Gaussian Process Regression (GPR) model that estimates the
magnitude of transfer integrals in undoped cuprates. These terms underlie the
leading mechanism of the magnetic exchange, and their knowledge is crucial for
the correct determination of the microscopic magnetic model. To describe
crystalline environments that correspond to individual transfer integrals, we
employed a mapping onto a three-dimensional Zernike descriptor.  The resulting
GPR model trained on the results of our high-throughput DFT calculations can
predict transfer integrals with reasonable error MAE = 28 meV.  The model
efficiently differentiates between weak and sizable transfer integrals, which
is most important for estimating the relevant spin model. We discuss limitations
of this approach and outline ways of improving the numerical accuracy.

\bmhead{Acknowledgments}
We thank Markus Wallerberger, Roman Rezaev, and Dmitry Chernyavsky for fruitful discussions. This project was funded by the Leibniz Association through the Leibniz
Competition.  Authors acknowledge financial support from the DFG through the
Collaborative Research Center SFB 1143 (Project-Id247310070). We thank Ulrike Nitzsche for the technical assistance. 

\section*{Methods}

\bmhead{DFT calculations}
    For high-throughput DFT calculations, we used the crystal structures of cuprates downloaded from the ICSD~\cite{bergerhoff.crystallographic.1987} employing the application programming interface.
    We performed nonmagnetic DFT calculations with the full potential local orbital code FPLO of version 18.00-52~\cite{koepernik.prb.1999}. Electron exchange-correlation interactions were described using the Perdew-Burke-Ernzerof (PBE) GGA functional~\cite{perdew.prl.1996}. The electron density was converged to within 10$^{-6}$. The reciprocal space mesh was calculated for each structure, accounting for the size of the reciprocal cell~\cite{suppl}.

    One shot calculation of Hellmann–Feynman forces was performed for each cuprate structure. The Wannier fit of the band structure was performed only for those structures where calculated forces are below 0.1\,eV/\r{A}~\cite{suppl}.
    The python library pymatgen~\cite{Ong2013pymatgen} has been used broadly in the high-throughput pipeline code to achieve complete automation.

    \bmhead{Machine learning model}
    In the work we consider regression problem $t_{ij} = f(\vec{D})$ which we handle with ML methods.
    For selection of ML predictive model we consider $k$-fold cross validation procedure. The cross-validation procedure with 6 folds was implemented with scikit-learn python library~\cite{scikit-learn}.
    The best performance was shown by the GPR model. We implemented GPR using the Matern kernel utilizing the scikit-learn. 
    For training and testing of the model, we standardize the features and targets so that the maximum absolute value of individual features is scaled to unit size. 
    Furthermore, we employ model-based dimensionality reduction for $\vec{D}$ using the RFR model to improve the predictive qualities of the model.
    The Supplementary Information 1~\cite{suppl} provides the details on configuration of the GPR model. 

    \begin{thebibliography}{10}
    	\providecommand{\url}[1]{{#1}}
    	\providecommand{\urlprefix}{URL }
    	\providecommand{\doi}[1]{\url{https://doi.org/#1}}
    	\bibcommenthead
    	
    	\bibitem{tsubaki.prl.20}
    	M.~Tsubaki, T.~Mizoguchi, {Quantum Deep Field: Data-Driven Wave Function,
    		Electron Density Generation, and Atomization Energy Prediction and
    		Extrapolation with Machine Learning}.
    	\newblock Phys. Rev. Lett. \textbf{125}, 206401 (2020).
    	
    	\bibitem{brunton.annurev-fluid.20}
    	S.~L. Brunton, B.~R. Noack, P.~Koumoutsakos, {Machine Learning for Fluid
    		Mechanics}.
    	\newblock Annu. Rev. Fluid Mech. \textbf{52}(1), 477--508 (2020).
    	
    	\bibitem{scmidt.npj.compmat.19}
    	J.~Schmidt, M.~R.~G. Marques, S.~Botti, M.~A.~L. Marques, {Recent advances and
    		applications of machine learning in solid-state materials science}.
    	\newblock npj Comput. Mater. \textbf{5}(1), 83 (2019).
    	
    	\bibitem{xie.prl.20}
    	T.~Xie, J.~C. Grossman, {Crystal Graph Convolutional Neural Networks for an
    		Accurate and Interpretable Prediction of Material Properties}.
    	\newblock Phys. Rev. Lett. \textbf{120}, 145301 (2018).
    	
    	\bibitem{rupp.prl.12}
    	M.~Rupp, A.~Tkatchenko, K.~R. M\"uller, O.~A. von Lilienfeld, {Fast and
    		Accurate Modeling of Molecular Atomization Energies with Machine Learning}.
    	\newblock Phys. Rev. Lett. \textbf{108}, 058301 (2012).
    	
    	\bibitem{schtt.prb.2014}
    	K.~T. Sch\"{u}tt, H.~Glawe, F.~Brockherde, A.~Sanna, K.~R. M\"{u}ller, E.~K.~U.
    	Gross, {How to represent crystal structures for machine learning: Towards
    		fast prediction of electronic properties}.
    	\newblock Phys. Rev. B \textbf{89}(20), 205118 (2014).
    	
    	\bibitem{bartok.prb.2013}
    	A.~P. Bart{\'{o}}k, R.~Kondor, G.~Cs{\'{a}}nyi, {On representing chemical
    		environments}.
    	\newblock Phys. Rev. B \textbf{87}(18), 184115 (2013).
    	
    	\bibitem{ziletti.natcomm.2018}
    	A.~Ziletti, D.~Kumar, M.~Scheffler, L.~M. Ghiringhelli, {Insightful
    		classification of crystal structures using deep learning}.
    	\newblock Nat. Commun. \textbf{9}(1), 2775 (2018).
    	
    	\bibitem{venkatraman.joc.09}
    	V.~Venkatraman, P.~R. Chakravarthy, D.~Kihara, {Application of 3D Zernike
    		descriptors to shape-based ligand similarity searching}.
    	\newblock J. Cheminformatics \textbf{1}(1), 19 (2009).
    	
    	\bibitem{novotni2003}
    	M.~Novotni, R.~Klein, in \emph{{Proceedings of the eighth {ACM} symposium on
    			Solid modeling and applications - {SM} {\textquotesingle}03}} ({ACM} Press,
    	2003).
    	
    	\bibitem{novotni.shape.04}
    	M.~Novotni, R.~Klein, {Shape retrieval using 3D Zernike descriptors}.
    	\newblock Comput. Aided Des. \textbf{36}(11), 1047--1062 (2004).
    	
    	\bibitem{sael.prot.2008}
    	L.~Sael, B.~Li, D.~La, Y.~Fang, K.~Ramani, R.~Rustamov, D.~Kihara, {Fast
    		protein tertiary structure retrieval based on global surface shape
    		similarity}.
    	\newblock Proteins: Structure, Function, and Bioinformatics \textbf{72}(4),
    	1259--1273 (2008).
    	
    	\bibitem{mak.jmgm.2008}
    	L.~Mak, S.~Grandison, R.~J. Morris, {An extension of spherical harmonics to
    		region-based rotationally invariant descriptors for molecular shape
    		description and comparison}.
    	\newblock J. Mol. Graph. Model. \textbf{26}(7), 1035--1045 (2008).
    	
    	\bibitem{sael2008}
    	L.~Sael, D.~La, B.~Li, R.~Rustamov, D.~Kihara, {Rapid comparison of properties
    		on protein surface}.
    	\newblock Proteins: Struct. Funct. Genet. \textbf{73}(1), 1--10 (2008).
    	
    	\bibitem{Plakida2010htc}
    	N.~Plakida, \emph{{High-temperature cuprate superconductors}}, 2010th edn.
    	\newblock Springer Series in Solid-State Sciences (Springer, Berlin, Germany,
    	2010).
    	
    	\bibitem{sahadasgupta2021}
    	T.~Saha-Dasgupta, {The Fascinating World of Low-Dimensional Quantum Spin
    		Systems: Ab Initio Modeling}.
    	\newblock Molecules \textbf{26}(6), 1522 (2021).
    	
    	\bibitem{helton07}
    	J.~S. Helton, K.~Matan, M.~P. Shores, E.~A. Nytko, B.~M. Bartlett, Y.~Yoshida,
    	Y.~Takano, A.~Suslov, Y.~Qiu, J.~H. Chung, D.~G. Nocera, Y.~S. Lee, {Spin
    		Dynamics of the Spin-1/2 Kagome Lattice Antiferromagnet
    		ZnCu$_3$(OH)$_6$Cl$_2$}.
    	\newblock Phys. Rev. Lett. \textbf{98}, 107204 (2007).
    	
    	\bibitem{jaime04}
    	M.~Jaime, V.~F. Correa, N.~Harrison, C.~D. Batista, N.~Kawashima, Y.~Kazuma,
    	G.~A. Jorge, R.~Stern, I.~Heinmaa, S.~A. Zvyagin, Y.~Sasago, K.~Uchinokura,
    	{Magnetic-Field-Induced Condensation of Triplons in Han Purple Pigment
    		BaCuSi$_2$O$_6$}.
    	\newblock Phys. Rev. Lett. \textbf{93}, 087203 (2004).
    	
    	\bibitem{kohama19}
    	Y.~Kohama, H.~Ishikawa, A.~Matsuo, K.~Kindo, N.~Shannon, Z.~Hiroi, {Possible
    		observation of quantum spin-nematic phase in a frustrated magnet}.
    	\newblock Proc. Nat. Acad. Sci. \textbf{116}, 10686--10690 (2019).
    	
    	\bibitem{Goodenough_Magnetism}
    	J.~B. Goodenough, \emph{{Magnetism And The Chemical Bond}} (John Wiley And
    	Sons, 1963).
    	
    	\bibitem{hubbard63}
    	J.~Hubbard, {Electron correlations in narrow energy bands}.
    	\newblock Proc. R. Soc. Lond. A \textbf{276}(1365), 238--257 (1963).
    	
    	\bibitem{anderson59}
    	P.~W. Anderson, {New Approach to the Theory of Superexchange Interactions}.
    	\newblock Phys. Rev. \textbf{115}, 2--13 (1959).
    	
    	\bibitem{auerbach2000}
    	A.~Auerbach, F.~Berruto, L.~Capriotti, \emph{{Quantum Magnetism Approaches to
    			Strongly Correlated Electrons}} (Springer Berlin Heidelberg, 2000), pp.
    	143--170.
    	
    	\bibitem{belik2004}
    	A.~Belik, M.~Azuma, M.~Takano, {Characterization of quasi-one-dimensional S=1/2
    		Heisenberg antiferromagnets Sr$_2$Cu(PO$_4$)$_2$ and Ba$_2$Cu(PO$_4$)$_2$
    		with magnetic susceptibility, specific heat, and thermal analysis}.
    	\newblock J. Solid State Chem. \textbf{177}(3), 883--888 (2004).
    	
    	\bibitem{johannes2006}
    	M.~D. Johannes, J.~Richter, S.~L. Drechsler, H.~Rosner,
    	{${\mathrm{Sr}}_{2}\mathrm{Cu}{(\mathrm{P}{\mathrm{O}}_{4})}_{2}$: A real
    		material realization of the one-dimensional nearest neighbor Heisenberg
    		chain}.
    	\newblock Phys. Rev. B \textbf{74}, 174435 (2006).
    	
    	\bibitem{janson2009}
    	O.~Janson, W.~Schnelle, M.~Schmidt, Y.~Prots, S.~L. Drechsler, S.~K. Filatov,
    	H.~Rosner, {Electronic structure and magnetic properties of the spin-1/2
    		Heisenberg system CuSe$_2$O$_5$}.
    	\newblock New J. Phys. \textbf{11}(11), 113034 (2009).
    	
    	\bibitem{bergerhoff.crystallographic.1987}
    	G.~Bergerhoff, I.~Brown, F.~Allen, et~al., {Crystallographic databases}.
    	\newblock International Union of Crystallography, Chester \textbf{360}, 77--95
    	(1987).
    	
    	\bibitem{suppl}
    	See Supplementary Information 1 for details of Wannierization, the reciprocal
    	space mesh calculation, and the numerical implementation of functions
    	$\mathcal{I}(x,y,z)$ and $\mathcal{Z}_{nl}^{m}(x, y, z)$. See Supplementary
    	Information 2 for the summary of Wannier fits.
    	
    	\bibitem{Ong2013pymatgen}
    	S.~P. Ong, W.~D. Richards, A.~Jain, G.~Hautier, M.~Kocher, S.~Cholia,
    	D.~Gunter, V.~L. Chevrier, K.~A. Persson, G.~Ceder, {Python Materials
    		Genomics (pymatgen): A robust, open-source python library for materials
    		analysis}.
    	\newblock Comput. Mater. Sci. \textbf{68}, 314--319 (2013).
    	
    	\bibitem{perdew.prl.1996}
    	J.~P. Perdew, K.~Burke, M.~Ernzerhof, {Generalized Gradient Approximation Made
    		Simple}.
    	\newblock Phys. Rev. Lett. \textbf{77}(18), 3865--3868 (1996).
    	
    	\bibitem{koepernik.prb.1999}
    	K.~Koepernik, H.~Eschrig, {Full-potential nonorthogonal local-orbital
    		minimum-basis band-structure scheme}.
    	\newblock Phys. Rev. B \textbf{59}(3), 1743--1757 (1999).
    	
    	\bibitem{koepernik2021arxiv}
    	K.~Koepernik, O.~Janson, Y.~Sun, J.~{van den Brink}.
    	\newblock {Symmetry Conserving Maximally Projected Wannier Functions} (2021).
    	\newblock {a}rXiv:2111.09652.
    	
    	\bibitem{eschrig2009}
    	H.~Eschrig, K.~Koepernik, {Tight-binding models for the iron-based
    		superconductors}.
    	\newblock Phys. Rev. B \textbf{80}, 104503 (2009).
    	
    	\bibitem{canterakis96}
    	N.~Canterakis, {Complete moment invariants and pose determination for
    		orthogonal transformations of 3D objects}.
    	\newblock Internal Report 1/96, Technische Informatik I, Technische
    	Universität Hamburg-Harburg  (1996).
    	
    	\bibitem{canterakis99}
    	N.~Canterakis, {3D Zernike moments and Zernike affine invariants for 3D image
    		analysis and recognition}.
    	\newblock 11th Scandinavian Conference on Image Analysis  (1999).
    	
    	\bibitem{nist.math.handbook}
    	F.~W.~J. Olver, D.~W. Lozier, R.~F. Boisvert, C.~W. Clark, \emph{{NIST Handbook
    			of Mathematical Functions}} (Cambridge University Press, 2010).
    	
    	\bibitem{morais.mathematics-of-computation.2014}
    	J.~Morais, I.~Ca{\c{c}}{\~{a}}o, {Quaternion Zernike spherical polynomials}.
    	\newblock Math. Comput. \textbf{84}(293), 1317--1337 (2014).
    	
    	\bibitem{Breiman_2001}
    	L.~Breiman, Random forests.
    	\newblock {Mach. Learn.} \textbf{45}(1), 5--32 (2001).
    	
    	\bibitem{rasmussen2005}
    	C.~E. Rasmussen, C.~K.~I. Williams, \emph{{Gaussian processes for machine
    			learning}}.
    	\newblock Adaptive Computation and Machine Learning series (MIT Press, London,
    	England, 2005).
    	
    	\bibitem{scikit-learn}
    	F.~Pedregosa, G.~Varoquaux, A.~Gramfort, V.~Michel, B.~Thirion, O.~Grisel,
    	M.~Blondel, P.~Prettenhofer, R.~Weiss, V.~Dubourg, J.~Vanderplas, A.~Passos,
    	D.~Cournapeau, M.~Brucher, M.~Perrot, E.~Duchesnay, {Scikit-learn: Machine
    		Learning in Python}.
    	\newblock J. Mach. Learn. Res. \textbf{12}, 2825--2830 (2011).
    	
    	\bibitem{rosner97}
    	H.~Rosner, H.~Eschrig, R.~Hayn, S.~L. Drechsler, J.~M\'alek, {Electronic
    		structure and magnetic properties of the linear chain cuprates
    		${\mathrm{Sr}}_{2}{\mathrm{CuO}}_{3}\phantom{\rule{0ex}{0ex}}$ and
    		${\mathrm{Ca}}_{2}{\mathrm{CuO}}_{3}$}.
    	\newblock Phys. Rev. B \textbf{56}, 3402--3412 (1997).
    	
    	\bibitem{heinze22}
    	L.~Heinze, M.~D. Le, O.~Janson, S.~Nishimoto, A.~U.~B. Wolter, S.~S{\"u}llow,
    	K.~C. Rule, {Low-energy spin excitations of the frustrated ferromagnetic
    		$J_1-J_2$ chain material linarite PbCuSO$_4$(OH)$_2$ in applied magnetic
    		fields parallel to the $b$ axis}.
    	\newblock Phys. Rev. B \textbf{106}, 144409 (2022).
    	
    	\bibitem{janson08}
    	O.~Janson, J.~Richter, H.~Rosner, {Modified Kagome Physics in the Natural
    		Spin-1/2 Kagome Lattice Systems: kapellasite Cu$_3$Zn(OH)$_6$Cl$_2$ and
    		Haydeeite Cu$_3$Mg(OH)$_6$Cl$_2$}.
    	\newblock Phys. Rev. Lett. \textbf{101}, 106403 (2008).
    	
    	\bibitem{jeschke13}
    	H.~O. Jeschke, F.~Salvat-Pujol, R.~Valent\'{\i}, {First-principles
    		determination of Heisenberg Hamiltonian parameters for the spin-$\frac{1}{2}$
    		kagome antiferromagnet ZnCu${}_{3}$(OH)${}_{6}$Cl${}_{2}$}.
    	\newblock Phys. Rev. B \textbf{88}, 075106 (2013).
    	
    	\bibitem{fak12}
    	B.~F\aa{}k, E.~Kermarrec, L.~Messio, B.~Bernu, C.~Lhuillier, F.~Bert,
    	P.~Mendels, B.~Koteswararao, F.~Bouquet, J.~Ollivier, A.~D. Hillier,
    	A.~Amato, R.~H. Colman, A.~S. Wills, {Kapellasite: A Kagome Quantum Spin
    		Liquid with Competing Interactions}.
    	\newblock Phys. Rev. Lett. \textbf{109}, 037208 (2012).
    	
    	\bibitem{boldrin15}
    	D.~Boldrin, B.~F\aa{}k, M.~Enderle, S.~Bieri, J.~Ollivier, S.~Rols, P.~Manuel,
    	A.~S. Wills, {Haydeeite: A spin-$\frac{1}{2}$ kagome ferromagnet}.
    	\newblock Phys. Rev. B \textbf{91}, 220408 (2015).
    	
    	\bibitem{janson07}
    	O.~Janson, R.~O. Kuzian, S.~L. Drechsler, H.~Rosner, Electronic structure and
    	magnetic properties of the spin-1/2 {H}eisenberg magnet {Bi$_2$CuO$_4$}.
    	\newblock Phys. Rev. B \textbf{76}, 115119 (2007).
    	
    	\bibitem{das08}
    	H.~Das, T.~Saha-Dasgupta, C.~Gros, R.~Valent\'{\i}, {Proposed low-energy model
    		Hamiltonian for the spin-gapped system ${\text{CuTe}}_{2}{\text{O}}_{5}$}.
    	\newblock Phys. Rev. B \textbf{77}, 224437 (2008).
    	
    	\bibitem{ushakov09}
    	A.~V. Ushakov, S.~V. Streltsov, {Electronic and magnetic structure for the
    		spin-gapped system CuTe$_2$O$_5$}.
    	\newblock J. Phys.: Condens. Matter \textbf{21}, 305501 (2009).
    	
    \end{thebibliography}
    

\end{document}