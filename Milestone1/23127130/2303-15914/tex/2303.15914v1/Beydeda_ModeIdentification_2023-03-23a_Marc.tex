\documentclass[aip,rsi,amsmath,reprint,amssymb,groupedaddress]{revtex4-2}

\pdfoutput=1

\usepackage{graphicx}
%\usepackage{siunitx}
%\usepackage{amssymb,amsmath}
\usepackage{textcomp}


%---- For changes by Marc Scheffler ---------------
\usepackage[normalem]{ulem}
\usepackage{color}
\newcommand{\MS}[1]{\textcolor{black}{#1}}
%---- For changes by Cenk Beydeda ---------------
\newcommand{\CB}[1]{\textcolor{green}{#1}}
%--------------------------------------------------



\begin{document}

\title{Characterization of harmonic modes and parasitic resonances in multi-mode superconducting coplanar resonators}

\author{Cenk~Beydeda}
\email[]{cenk.beydeda@pi1.uni-stuttgart.de}
\author{Konstantin~Nikolaou}
\author{Marius~Tochtermann}
\author{Nikolaj~G.~Ebensperger}
\author{Gabriele~Untereiner}
\author{Ahmed~Farag}
\author{Philipp~Karl}
\author{Monika~Ubl}
\author{Harald~Giessen}
\author{Martin~Dressel}
\author{Marc~Scheffler}
\email[]{marc.scheffler@pi1.physik.uni-stuttgart.de}
\affiliation{Physikalisches Institut, Universit\"at Stuttgart, 70569 Stuttgart, Germany}

\date{28.03.2023}

\begin{abstract}
Planar superconducting microwave transmission line resonators can be operated at multiple harmonic resonance frequencies. This allows covering wide spectral regimes with high sensitivity, as it is desired e.g.\ for cryogenic microwave spectroscopy. 
A common complication of such experiments is the presence of undesired \lq spurious\rq{} additional resonances, which are due to standing waves within the resonator substrate or housing box.
Identifying the nature of individual resonances (\lq designed\rq{} vs.\ \lq spurious\rq ) can become challenging for higher frequencies or if elements with unknown material properties are included, as is common for microwave spectroscopy.
Here we discuss various experimental strategies to distinguish designed and spurious modes in coplanar superconducting resonators that are operated in a broad frequency range up to 20~GHz. These strategies include tracking resonance evolution as a function of temperature, magnetic field, and microwave power. We also demonstrate that local modification of the resonator, by applying minute amounts of dielectric or ESR-active materials, lead to characteristic signatures in the various resonance modes, depending on the local strength of the electric or magnetic microwave fields.

\end{abstract}

\maketitle

%\MS{Who grew the Nb films/did the lithography? only Monika Ubl? Or was also Philipp Karl involved?}
%\MS{Labels for the resonators: \lq R$_\textrm{f}$\rq{} or \lq Res$_\textrm{f}$\rq{}}

\section{Introduction}

%\begin{itemize}
%	\item Planar superconducting resonators eg. for quantum information (cQED), as KIDs, for probing of material properties
%	\item Often: use only one mode
%	\item for certain applications: use various harmonics. Advantage of transmission line resonators: harmonics evenly spaced
%	\item Spectroscopy: want many harmonics \cite{DiIorio1988,Oates1991,Scheffler2013,Thiemann2018a,Thiemann2018b}
%	\item problem: spurious/undesired modes in this frequency range.
%	\item in spectroscopy of unconventional materials (cite some more papers, e.g. Corbino or cavity studies), the harmonics could be non-equidistant
%\end{itemize}

%\MS{[MS: Still to do overall: Make sure, that the indices of $f_\textrm{n}$ etc. make sense throughout the manuscript: this index somehow indicates $n$ of the harmonics as well as some index of the parasitic modes. Thus, maybe one should have a generic index (maybe $f_\textrm{m}$ or $f_\textrm{i}$), that includes both harmonics and parasitic modes?]}

%\CB{[CB: Noch zu erledigen: Dateigröße der pdf runterschrauben]}

%\MS{[MS: Next steps: (1) Cenk, please implement the changes that are noted below, and then send the manuscript back to me, ideally by tomorrow noon. (2) I will go through the manuscript again and send it back to you on short notice. (3) You send it to all coauthors and ask whether they have any comments, with deadline e.g. March 27th or 28th. (4) If coauthors have comments, we implement them. (5) You submit the manuscript to arXiv and RSI, ideally by March 28th.\\
%Cenk: Already now you can prepare the email to the coauthors, and send it to me.
%]\\}


%\MS{[MS: Somewhat optional things for Marc that are still on the list: Reference for suppressing slotline modes by wirebonds.]}


Planar superconducting resonators, fabricated from superconducting thin films on insulating substrates, play an important role for cryogenic on-chip applications and in various research fields. In quantum information processing, superconducting resonators couple microwave photons to individual solid-state quantum bits or ensembles of quantum systems.\cite{Wallraff2004,Sillanpaeae2007,Majer2007,Schuster2010,Kubo2010,Huebl2013,Ghirri2015,Gu2017,Hattermann2017}
In astronomy and particle physics, highly sensitive kinetic inductance detectors (KIDs) can easily be multiplexed.\cite{Day2003,Zmuidzinas2012,Battistelli2015,Adam2018}
In solid state spectroscopy,\cite{Scheffler2013,Hafner2014,McRae2020} planar superconducting resonators probe the microwave properties of numerous material classes of interest, ranging from conventional\cite{DiIorio1988,Oates1991,Andreone1993,Zemlicka2015,Driessen2012,Beutel2016,Thiemann2018a,Thiemann2018b,Manca2019} and unconventional superconductors \cite{Anlage1989,Langley1991,Revenaz1994,Porch1995,Zaitsev2001,Wang2007,Ghigo2012,Scheffler2015,Ghigo2016} to heavy-fermion metals,\cite{Scheffler2013,Parkkinen2015} quantum paraelectrics,\cite{Davidovikj2017,Engl2019} various magnetic and spin systems,\cite{Wallace1998,Bushev2011,Malissa2013,Bondorf2018,Golovchanskiy2018,Ranjan2020,Miksch2021} and dielectric thin films.\cite{OConnell2008,Wisbey2019,Ebensperger2019}

Realization of on-chip superconducting resonators can follow different approaches, such as lumped element resonators\cite{Doyle2008,FornDiaz2010} or transmission line resonators.\cite{Frunzio2005,Goeppl2008} The latter employs one of various transmission line geometries (e.g.\ coplanar, microstrip, or stripline); here a line segment of a certain length with open or short ends defines a one-dimensional resonator.
The higher resonance modes of transmission line resonators are harmonics, which in the simplest case are spaced equally in frequency, and they have transverse field distributions corresponding to the fundamental mode. These properties are advantageous for microwave spectroscopy applications, because they allow to conveniently cover a rather wide frequency range combined with high sensitivity and straightforward data analysis.\cite{DiIorio1988,Andreone1997,Scheffler2013,Hafner2014,Zou2017} 
%\MS{(some more references on analysis or on multi-mode spectroscopy?)}
Typical spectral ranges span from 1 to 20~GHz and beyond.\cite{Wang2007,Davidovikj2017,Rausch2018,Thiemann2018b,Manca2019}
If one operates a superconducting on-chip resonator in such a broad frequency range, one typically encounters various additional resonances that are undesired and that stem e.g.\ from standing waves in the dielectric substrate or in the metallic sample holder box, or from 
%slotline modes where the two grounds on either side of a coplanar transmission line feature opposite microwave voltage
asymmetric slotline modes.\cite{Schuster2010,Wenner2011,McRae2020}
For a superconducting microwave device operating at a single frequency or in a narrow frequency range,\cite{Adam2018} the microwave environment (e.g.\ sample box) can often be optimized such that all the parasitic modes are shifted to frequency ranges that are not relevant for the particular device, usually this means to higher frequencies.\cite{Wenner2011,McConkey2018} Also slotline modes can often be eliminated by e.g.\ bridging wirebonds.
But for spectroscopy studies, avoiding such parasitic resonances completely usually is not possible. 
Then it is crucial to identify which of the detected resonances are the designed resonator harmonics and which are the parasitic modes. This is straightforward if the harmonics are evenly distributed in frequency. 
But the material properties to be determined in microwave spectroscopy can exhibit substantial frequency dependence,\cite{Viana1994,Sluchanko2000,Tran2002,Turner2003,Scheffler2005c,Engl2019} and thus the resulting resonator frequencies are not known beforehand and might not be spaced evenly in frequency. 
In such cases, identifying whether an observed resonance is one of the designed modes or parasitic, can become challenging.
Here we present various strategies how one can characterize such higher-frequency modes and determine their nature.

\section{Experiment and Data Analysis}

\begin{figure}
	\centering
	\includegraphics[width=\linewidth]{plots/setup/setup5_adj_2023_03_20.png}
	\caption{Schematic resonator designs of the (a) flip-chip setup (labeled R$_\textrm{f}$) and the (b) on-plane setup (labeled R$_\textrm{p}$) and (c) photograph and (d) design of the actual devices. For R$_\textrm{f}$ the resonator chip (Nb on TiO$_2$) is held face-to-face slightly above a separate feedline chip (Cu on sapphire), while for R$_\textrm{p}$ two resonators and a feedline are fabricated from the same Nb layer on a sapphire substrate. The dashed line in (c) indicates the position of the resonator on the lower side of the TiO$_2$ substrate. The two resonators in (d) are labeled R$_\text{p1}$ and R$_\text{p2}$ for distinction. $l_c$ is the coupling length of the resonators.}
%Resonator designs of the (a) flip-chip setup (labeled R$_\textrm{f}$) and the (b) on-plane setup (labeled R$_\textrm{p}$). The sapphire substrate is shown in dark red, the TiO$_2$ substrate in light blue, the copper film in orange, and the thin Nb-layer in grey. In the flip-chip setup the resonator is fabricated on TiO$_2$ whereas in the on-plane setup the feedline and the resonator are on the same sapphire substrate.  $l_c$ is the coupling length. The photographs on the right show the respective devices.
	\label{fig:Setup}
\end{figure}

We present and discuss data that are mostly obtained with two different coplanar waveguide (CPW) resonator designs, labeled R$_\textrm{f}$ and R$_\textrm{p}$, as shown in Fig.\ \ref{fig:Setup}. Each resonator is fabricated by optical lithography from a Nb layer with 300\,nm thickness on a dielectric substrate. We employ $\lambda/4$-resonators (with $\lambda$ the wavelength in the CPW) that are coupled to their feedlines by parallel straight sections of the CPWs of length $l_c$.

The first case, R$_\textrm{f}$ shown in Figs.\ \ref{fig:Setup}(a) and (c), consists of a coplanar resonator fabricated on a TiO$_2$ substrate, forming the so-called flip-chip, which is mounted above a copper feedline that is deposited on a separate sapphire (Al$_2$O$_3$) chip. The distance between both chips is $\approx$~50~$\mu$m, and the coupling arm of the resonator on the flip-chip and the copper feedline face each other.\cite{Wendel2020} The dielectric constant of TiO$_2$, between 110 and 260 depending on crystallographic direction,\cite{Klein1995,Zuccaro1997,Tobar1998} is rather high, and thus harmonic and parasitic modes incorporating the TiO$_2$ can occur at comparably low frequencies. 
%\MS{[MS: From Cenk's thesis: Tc is 8.6 K from jump in S21 vs. T in off-resonant background.]}

The second case, R$_\textrm{p}$ shown in Figs.\ \ref{fig:Setup}(b) and (d), employs a sapphire chip with feedline and two resonators respectively labeled R$_\text{p1}$ and R$_\text{p2}$ arranged in the same plane, i.e.\ both resonators can be addressed with a single microwave line like multiplexed devices.\cite{Day2003,Adam2018} 
These CPW resonators have meander shape to allow low fundamental frequencies for a small chip area, which in fact is a strategy to suppress parasitic box modes.
%In the experiments presented below, only five of the six resonators worked properly, and most of the analysis will focus on the harmonics of two of these resonators.
%\MS{[MS: From Cenk's thesis: Tc is 8.6 K from jump in S21 vs. T in off-resonant background.]}
%\MS{[MS: From Konstantin's thesis: Tc maybe 7.0 K (page 58). Could we determine Tc also here via off-resonant background signal like in Cenk's case?]} \CB{[Determining Tc from the off-resonant background should be possible also in Konstantins case but it requires some analysis and thus time]}

%Both chips, R$_\textrm{f}$ and R$_\textrm{p}$, use $\lambda/4$-resonators, which support resonances at odd multiples $n=1,3,5,...$ of the fundamental mode frequency $f_0$:

The $\lambda/4$-resonators of both chips, R$_\textrm{f}$ and R$_\textrm{p}$, support resonances at odd multiples $n=1,3,5,...$ of the fundamental mode frequency $f_0$:
%
\begin{equation}
f_\textrm{n} = n f_0 = n\cdot c(4l\sqrt{\epsilon_\textrm{eff}})^{-1}\label{eq:ResonanceFrequencies}
\end{equation}
%
where $c$ is the vacuum speed of light, $l$ the resonator's total length, and $\epsilon_\textrm{eff}$ the effective dielectric constant, which depends on the CPW geometry and the dielectric constants $\epsilon$ of the materials that are used, e.g. sapphire or TiO$_2$. The finite and temperature-dependent penetration depth for the superconducting film, which also affects the resonant frequency, we incorporate into the generic parameter $\epsilon_\textrm{eff}$. In spectroscopy applications, the frequency dependence of $\epsilon_\textrm{eff}$ is a key piece of information.

The microwave chips were mounted in brass boxes, and measurements of the complex transmission coefficient $\hat{S}_{21}$ through the feedlines were performed using a vector network analyzer (VNA) and a $^4$He cryostat with superconducting magnet and variable-temperature insert for temperatures $T$ down to 2\,K. 
The superconducting transition $T_\textrm{c}$ is around 8.6~K for device R$_\textrm{f}$ and around 7.1~K 
%(see below) \CB{[CB: what is there to see below?]} 
for device R$_\textrm{p}$.

For the microwave power-dependent measurements, amplifier and attenuator were used to reach higher power levels up to 17\,dBm.
Since our highest employed frequency is 20~GHz while the low-temperature superconducting energy gap of Nb is around 750~GHz,\cite{Pronin1998} we restrict our analysis using the assumption of frequency being much smaller than the energy gap, which might not rigorously hold for temperatures close to %the critical temperature 
$T_\textrm{c}$.

From the $\hat{S}_{21}$ spectra, each resonance is fitted using the following function\cite{Thiemann_phd}:
%\CB{[Ich habe die Thiemann-Phd zitiert. Die Gleichung schaut bei Thiemann bisschen anders aus, aber diese Form ist meiner Meinung nach sinnvoller anstatt $\hat{v}_1$ und $\hat{v}_2$ einzuführen was nur nötig ist wenn man die komplexe Fitmethode erklären will. Ich habe eine kurze Recherche gemacht, die Fitfunktion habe ich in keinem paper gefunden.]}
%
\begin{equation}
\hat{S}_{21} = e^{\text{i}2\pi f\hat{\tau}} \left[ \frac{\hat{A}}{\left(f-f_\text{m}\right) +\text{i}\frac{f_\text{b,m}}{2}} +\hat{v}_3 + \hat{v}_4\left(f-f_\text{m}\right) \right]  \label{eq:FitFunctionResonances}
\end{equation}
%\MS{[MS: Add actual function.]}
%
Here $f_\textrm{m}$ is the resonance frequency, $f_\text{b,m}$ is the bandwidth where the generic index m includes designed and spurious resonances, $\hat{A}$ is a complex amplitude, $\hat{\tau}$ is a complex time constant, and the complex coefficients $\hat{v}_3$ and $\hat{v}_4$ model the background as first-order Taylor expansion. $Q_\textrm{m} = f_\text{m}/f_\text{b,m}$ is the experimentally observed loaded quality factor of the resonance. Real and imaginary parts of $\hat{S}_{21}(f)$ are fitted simultaneously. The discussion below will concentrate on $f_\textrm{m}$ and $Q_\textrm{m}$.

For a clear presentation of the numerous observed resonance modes, we use the following color coding in the figures below: 
for data obtained with R$_\textrm{f}$, the harmonic modes are plotted in shades of blue and black, with dashed and straight lines to distinguish adjacent modes.
For the very numerous modes analyzed for the R$_\textrm{p}$ device, the harmonic modes of the first resonator R$_\textrm{p1}$ are plotted in shades of blue, green, and yellow, and the modes of the second resonator R$_\textrm{p2}$ are plotted in shades of grey. Parasitic modes are plotted in shades of red for both R$_\textrm{f}$ and R$_\textrm{p}$ resonators. 



\section{Results and Discussion}

%\MS{[Somewhere comment on Tc of both chips? Should be in Section II.]}

\subsection{Spectra}

\begin{figure}
	\centering
	\includegraphics[width=\linewidth]{plots/spektrum1/spectrumswapped_adj_2023_03_03.pdf}
	\caption{
	%\MS{[Now there are photographs of the devices in both Fig. 1 and Fig. 2, but we need them only once. Remove them here, and instead show zoom-ins of the spectra with fits to resonances.]}
	%\MS{[Now in Fig.\ \ref{fig:Spectra}(a) the resonances are labeled with Roman numerals while in the latter figures it is 'p=1' etc. In principle I like the Roman labeling, but in the present version there is no consistency between figures. The simplest solution is to replace the Roman label, e.g. PIV, by the Arabic numbers, e.g. p=4.]}
	%\MS{[For the R$_\textrm{p}$ case: which of the resonances are designed and which are parasitic? Maybe one can use different colors of the arrows to indicate this?]}
	Transmission coefficient $|\hat{S}_{21}|$ of the (a) flip-chip setup R$_\textrm{f}$ and (b) on-plane setup R$_\textrm{p}$ measured at temperature $T=2\,$K. The insets show zoom-ins with fits to exemplary resonances. In (a), the labels of the resonances indicate the number $n$ of the harmonics (following Eq.\ \ref{eq:ResonanceFrequencies}) for the designated resonator modes whereas the number $p$ simply enumerates the parasitic modes that were analyzed. 
	%In the R$_\textrm{f}$ case the harmonic modes are colour-coded in black and blue where the parasitic modes are colour-coded in shades of red with dashed and straight lines for better distinguishability. In the R$_\textrm{p}$ case the great amount of harmonic modes makes clear distinguishability difficult.
	}
	\label{fig:Spectra}
\end{figure}

In Fig.\ \ref{fig:Spectra} broadband spectra of the flip-chip setup R$_\textrm{f}$ and the on-plane setup R$_\textrm{p}$ are shown for $T=$ 2\,K. In both cases the background signal shows an overall decrease due to the transmission-line losses generally increasing with frequency for the CPW feedline and for the coaxial cables that connect the VNA and the cryogenic chip. 
%Standing waves on the background signal form because of small reflections at both ends of the coplanar feedline. 
Characteristic sharp minima in the spectra, indicated by arrows, arise for the designed harmonic modes as well as for the undesired parasitic resonances. In Fig.\ \ref{fig:Spectra}(a) the desired harmonic resonator modes are identified by their frequencies roughly equaling odd multiples of the fundamental frequency of 0.75~GHz, and the remaining resonances are labeled parasitic.
One reason why in Fig.\ \ref{fig:Spectra}(a) the frequencies of the harmonics are not exactly multiples of the fundamental frequency is the anisotropy of the TiO$_2$ combined with the varying contributions of the different crystallographic directions to the resonator response due to the standing wave pattern of the modes within the resonator. 
The assignment in Fig.\ \ref{fig:Spectra}(b) is complicated by the presence of two resonators but somewhat simplified by the less pronounced anisotropy of the sapphire substrate, therefore the odd multiples of the two fundamental resonances can be established straightforwardly where the remaining resonances are labeled parasitic again.
While all expected resonator harmonics are observed for the covered spectral range, the $n=7$ mode
of R$_\textrm{f}$ is very weak and thus this particular harmonic will not be considered below.
%The spectrum in Fig.\ \ref{fig:Spectra}(b) for setup R$_\textrm{p}$ shows many more resonances than setup R$_\textrm{f}$ because two resonators simultaneously contribute harmonics.

%Below, this assignment will be confirmed by separate experiments, and thus those procedures might help in this assignments for cases where the harmonics are not expected to be equidistant due to material properties.

\subsection{Temperature Dependence} \label{sec:TemperatureDependence}

%\MS{Maybe Fig. spectra: e.g. based on Nikolaou 5.2/5.3}
%\MS{Fig.: based on Beydeda Fig.4.5 and Fig. 4.7/4.8 and Nikolaou 5.4 and 5.6.}
%\MS{Use (a) and (b) to distinguish between plots}

\begin{figure}
	\centering
	\includegraphics[width=\linewidth]{plots/Tdependence/total_fshift_adj_2023_03_20.pdf}
	\caption{Resonance frequencies $f_\textrm{m}$ in dependence of the temperature $T$ for the R$_\textrm{f}$ case in (a) and R$_\textrm{p}$ case in (b). The resonance frequencies are normalized to the respective value at 2~K. The main panels show the complete temperature range, from 2~K up to the highest temperature where the modes were detected, and the insets show in more detail smaller temperature ranges.}
 \label{fig:TemperatureDependenceFrequency}
\end{figure}

The strong temperature dependence of superconducting properties can be used to distinguish designed and parasitic modes, as shown in Fig.\ \ref{fig:TemperatureDependenceFrequency}. For simpler comparison, the resonance frequencies are normalized to their respective values at 2~K, and while the main panels show the data for the full temperature range (from 2~K up to the highest temperature where the modes can still be properly distinguished from the background), the insets show in more detail the temperature ranges where the temperature-dependent evolution of the modes becomes evident.
One clearly sees that the data for the different modes form bundles of curves with similar behavior, and both for R$_\textrm{f}$ and for R$_\textrm{p}$ the parasitic modes have a weaker temperature dependence than the designated resonator modes. For a simple superconducting resonator based on a transmission line such as CPW, the transverse field distribution for all designated modes is equivalent, and therefore the temperature-dependent penetration depth of the superconductor will affect all resonator modes in the same fashion, via $\epsilon_\textrm{eff}$ in Eq.\ (\ref{eq:ResonanceFrequencies}),\cite{Hafner2014} and this is basically what one sees in Fig.\ \ref{fig:TemperatureDependenceFrequency}. Then it might come as a surprise that for the two CPW resonators of R$_\textrm{p}$  in Fig.\ \ref{fig:TemperatureDependenceFrequency}(b), which are fabricated within the same Nb layer and have the same lateral dimension of the CPW, the temperature evolution of the resonance frequencies is different with separating bundles of curves towards $T_\textrm{c}$. This can be explained if one assumes that the film quality of the Nb layer differs throughout different parts of the overall chip, and thus the \lq local $T_\textrm{c}$\rq{} might differ between resonators 1 and 2. Minute quality and thus $T_\textrm{c}$ variations within the Nb layer can also explain why the designated modes for each resonator, including the case in Fig.\ \ref{fig:TemperatureDependenceFrequency}(a), slightly differ in their temperature evolution.

%\MS{[Somewhere, either here or above when the anisotropic substrate is mentioned, discuss more about the field distribution of the different standing wave patterns of the modes.]}

\begin{figure}[t]
	\centering
	\includegraphics[width=\linewidth]{plots/Tdependence/total_Qshift_absolute_adj.pdf}
	\caption{Measured quality factor $Q$ in dependence of the temperature $T$ for the R$_\textrm{f}$ case in (a) and R$_\textrm{p}$ case in (b) of the modes measured in Fig.\ \ref{fig:Spectra}.}
	\label{fig:TemperatureDependenceQ}
\end{figure}

\begin{figure}[t]
	\centering
	\includegraphics[width=\linewidth]{plots/Tdependence/total_Qshift_rescaled_adj.pdf}
	\caption{Measured quality factor $Q$ in dependence of the temperature $T$ for the R$_\textrm{f}$ case in (a) and R$_\textrm{p}$ case in (b) of the modes measured in Fig.\ \ref{fig:Spectra}. The quality factor is normalized to $Q(T=2\,\textrm{K})$}
	\label{fig:TemperatureDependenceQnormalized}
\end{figure}


Here we have assumed that the temperature dependence of $f_\textrm{n}$ is fully governed by the superconducting film. This assumption is justified in the present study because all other parameters that enter in Eq.\ \ref{eq:ResonanceFrequencies} can be assumed constant in this temperature range, e.g.\ the dielectric constants of TiO$_2$ or sapphire. \cite{Rausch2018, Sabinsky1962}

Also for the parasitic modes, which reside within the resonator chip and/or the housing box and thus can have as relevant further materials only metals, the superconducting film will have the strongest temperature dependence.
Indeed, the temperature dependence for parasitic modes in Fig.\ \ref{fig:TemperatureDependenceFrequency} is much less than the designated resonances. This means that for parasitic modes a much smaller fraction of the mode volume concerns the superconducting film. This matches the expectations for either undesired one-dimensional slotline modes of the CPW or three-dimensional modes that include the bulk of the substrate and/or the volume within the sample box.
%\MS{[About slotline modes: are there any parasitic modes that have \lq harmonics\rq ?]} \CB{I didn't observe parasitc modes that have harmonics}
The situation might be different if other strongly temperature-dependent materials are involved.\cite{Engl2019,Hering2007,Pompeo2007,Geiger2012}

The temperature dependence of the quality factor $Q$ is shown in Fig.~\ref{fig:TemperatureDependenceQ} for the different modes, and in Fig.~\ref{fig:TemperatureDependenceQnormalized} as normalized $Q(T)/Q(2\ \textrm{K})$.
Again one can clearly distinguish the desired resonator modes from the undesired parasitic ones, as they appear separated in Fig.\ \ref{fig:TemperatureDependenceQnormalized}: while the former decrease with increasing temperature already starting around 2\ K, the latter have much weaker temperature dependence and decrease substantially only close to $T_\textrm{c}$. But compared to the resonance frequencies in Fig.\ \ref{fig:TemperatureDependenceFrequency}, the $Q$ data do not assemble closely to bundles, and this has several reasons: firstly, the microwave losses of a superconducting resonator strongly depend on frequency, which is due to the characteristic low-frequency properties of the complex optical conductivity $\hat{\sigma}$ of superconductors,\cite{Pracht2013} and thus the absolute $Q$ of designated harmonics shown in Fig.\ \ref{fig:TemperatureDependenceQ} have a very strong frequency dependence in the low-temperature limit, roughly corresponding to $1/f$.\cite{Goeppl2008} In a similar fashion, the temperature evolution of $\hat{\sigma}$ also varies for different frequencies,\cite{Pracht2013,Steinberg2008} and thus no matching temperature dependence can be expected for the $Q$ of different resonant frequencies even when normalized (Fig.\ \ref{fig:TemperatureDependenceQnormalized}). Furthermore, there are various physical phenomena affecting $Q$. If there are separate loss mechanisms, one can assign a characteristic $Q$ to each of those, and the total, loaded $Q_\textrm{total}$ that we determine from the experiment is the inverse sum of the inverse respective $Q$s. For superconducting planar resonators, this might read as: 
\begin{equation}
  \frac{1}{Q_\textrm{total}} = \frac{1}{Q_\textrm{sc}} + \frac{1}{Q_\textrm{coupl}} + \frac{1}{Q_\textrm{diel}} + ... \label{eq:SumQs}
\end{equation}
where $Q_\textrm{sc}$ quantifies Ohmic losses in the superconductor, $Q_\textrm{coupl}$ coupling losses to the microwave readout, $Q_\textrm{diel}$ dielectric losses (in the substrate), and further contributions might consider radiation losses or losses in metallic components within the respective mode volume.
As discussed, $Q_\textrm{sc}$ strongly depends on frequency and temperature, and we have clear expectations based on the well-known $\hat{\sigma}(f,T)$ of conventional superconductors.\cite{Pracht2013,Steinberg2008} 
For the designated CPW modes, $Q_\textrm{diel}$ should be negligible here due to the choice of low-loss substrates, and $Q_\textrm{coupl}$, which is governed by geometrical parameters like $l_\textrm{c}$, for spectroscopy applications usually is designed to be rather high. In this case, $Q_\textrm{sc}$ represents the dominant loss channel and should obtain the strong frequency and temperature dependences discussed above. But if other mechanisms also contribute, e.g.\ quantified by a term $Q_\textrm{spur}$ of unclear origin that affect the spurious resonances and limit their $Q$ to values of order a few hundred, then the strong temperature dependence of possible $Q_\textrm{sc}$ contributions with absolute values above e.g.\ 1000 for temperatures well below $T_\textrm{c}$ will not affect much the $Q_\textrm{total}$ of the spurious modes, exactly as we see in Fig.\ \ref{fig:TemperatureDependenceQ}. Furthermore, the $Q$s of parasitic modes change little with temperature except close to $T_\textrm{c}$, in stark contrast to the designed CPW resonances.
These characteristics lead to the various intersecting curves in Fig.\ \ref{fig:TemperatureDependenceQ}, where $Q$s of designated modes clearly decrease with increasing temperature whereas $Q$s of the parasitic modes are almost constant.


%This section deals with the temperature dependence of the resonance frequency $f$ and the quality factor $Q$ of the modes. The analysis will focus on the superconducting layer as there is no $T$-dependence of $\epsilon$ in the microwave regime of TiO$_2$ and Al$_2$O$_3$.

%In Fig.\ \ref{fig:T1} the $T$-dependence of $f$ where $f$ is normalized to $f(T=2\textrm{K})$ can be seen. The $T$-dependent London penetration depth $\lambda_L(T)$ effectively alters the resonator's geometry and therefore also $\epsilon_\textrm{eff}$ resulting in a decrease of all harmonic modes $f$ according to equation\ref{eq:ResonanceFrequencies} towards $T_c$.\cite{Hafner2014} In both cases R$_\textrm{f}$ and R$_\textrm{p}$ the harmonic modes of the resonators and the parasitic modes respectively form bundles, also the relative decrease is in the same order and goes down from $1.00$ to almost $f_\textrm{normalized}=0.98$. Higher harmonics of different resonators also respectively form bundles as seen in the R$_p$ case which is a result of the manufacturing process and the geometry as the resonators have different lengths \MS{kann man das so sagen? Zumindest sollte man kurz darauf eingehen dass auch die Resonatoren untereinander Buendel formen koennen weil ja dann auch Verwechslungsgefahr mit den Parasitaeren bestehen kann}. The parasitic modes approach $T_c$ showing that they are supported by the superconducting Nb-layer although their delayed decrease show a weaker influence of the limiting effects of $\lambda_L$. This leads to the assumption that the parasitic modes are supported by the whole superconducting layer on the flip-chip whereas the harmonic modes are located in the $\lambda/4$-resonator.

%In Fig.\ \ref{fig:T2} and \ref{fig:T3} the $T$-dependence of the quality factor $Q$ can be seen where in Fig.\ \ref{fig:T3} $Q(T)$ is normalized to $Q(T=2\,\textrm{K})$. Here and throughout this paper the loaded quality factor $Q = Q_L = \frac{f}{f_B}$ is considered with the half power frequency width $f_B$. The quality factor $Q(T)$ of all modes decrease due to increased losses for greater temperature $T$ in the superconducting layer. In Fig.\ \ref{fig:T2} it can be seen that the initial values $Q(T=2\,\textrm{K})$ of the harmonic harmonic modes decrease for greater $n$.\cite{Hafner2014} Additionally it can be seen in Fig.\ \ref{fig:T3} that the decrease of $Q$ is relatively stronger for lower $n$. This can be explained if the quality factor is split into a constant $Q_1$ and a $T$-dependent $Q_2(T)$. With respect to the inverse addition of quality factors $Q^{-1} = Q_1^{-1} + Q_2^{-1}$ it can be shown that the effects of $Q_2$ are stronger on $Q$ for greater $Q_1$ explaining how greater initial quality factors $Q(T=2\,\textrm{K})$ decrease relatively stronger than lower $Q(T=2\,\textrm{K})$. It is important to consider these effects because normalizing the quality factors $Q$ to their values at $T=2\,$K would not result in bundles similar to those seen in Fig.\ \ref{fig:T1}. So for a distinction between harmonic modes and parasitic modes it is important to consider $Q(T)$ of modes that intersect. In that case the $Q$-values are comparable but with different $T$-dependence showing a different loss mechanism.

%In Fig.\ \ref{fig:T2} intersecting quality factors $Q$ can be seen showing different loss mechanisms although most parasitic modes already start at low $Q(T=2\,\textrm{K})$. As mentioned before normalizing $Q$ to $T=2\,$K is not suitable to distinguish harmonic modes and parasitic modes, nevertheless Fig.\ \ref{fig:T3} shows in an illustrative way the $T$-dependence of the modes and that the parasitic modes are relatively weaker affected by increasing $T$. \MS{Mode p=6 benimmt sich in der Guete wie die Resonatormoden, darauf gesondert eingehen?}

\subsection{Magnetic Field Dependence}\label{sec:MagneticFieldDependence}

\begin{figure}
	\centering
	\includegraphics[width=\linewidth]{plots/Bdependence/total_frequency_q_adj_2023_03_23.pdf}
	\caption{Resonance frequency $f_\text{m}$ in (a) and quality factor $Q_\textrm{m}$ in (d) normalized to the data at $B = 0$\,T for the TiO$_2$ flip-chip setup R$_\textrm{f}$ with the harmonic modes measured in Fig.\ \ref{fig:Spectra}, measured at $T=$ 2 ~K. (b) shows $f_\textrm{m}$ for $1\,\textrm{T}<B<2$\,T and (c) for $0.1\,\textrm{T}<B<0.6$\,T. The black arrow marks the starting point of decrease of $f_\textrm{m}$ at $B=0.2$\,T and the red arrow marks an abrupt change in decrease of $f_\textrm{m}$ at $B=0.4$\,T.}
%The fluctuations for $B>1$\,T are due to the decrease of the quality factor $Q(B)$ in dependence of $B$ that result in a stronger disturbance of the harmonic modes as the harmonic modes shift in their frequencies over the standing waves of the background. 
	%\MS{[MS: This caption has to be updated with the new labeling of the different subfigures. Make sure that the labels (a), (b) etc. are used consistently. Also, we now know that the arrows indicate Bc1 and Bc2.]}
	\label{fig:BfieldDependence}
\end{figure}

Applying an external static magnetic field $B$ has strong effects on superconductors, but basically leaves the other materials, such as the dielectric substrates, unaffected. For Nb, as type-II superconductor, the external magnetic field penetrates as quantized vortices for fields higher than the lower critical field $B_\textrm{c1}$ until superconductivity is fully suppressed (in the bulk) at the upper critical field $B_\textrm{c2}$. In our experiment, the external static magnetic field is applied roughly parallel to the Nb thin film of the resonator, and thus strong changes in the CPW performance are expected for fields of order 100~mT (in contrast to order 1~mT for perpendicular field).\cite{Bothner2012}
%, with an expected slight misalignment of order 1\textdegree. 
Due to this field arrangement and the strong dependence of superconducting properties on Nb material quality,\cite{Halbritter2005} it is difficult to quantitatively relate observed field-dependent effects to theoretical expectation. Still, the magnetic field dependence can help to assign resonator modes.

Fig.\ \ref{fig:BfieldDependence} shows the field dependence of $f_\textrm{m}$ and $Q_\textrm{m}$, both normalized to the respective zero-field values, at temperature $T = 2$~K for the R$_\textrm{f}$ device.
With increasing external static magnetic field and thus increasing vortex density, the microwave losses in the superconductor increase and thus result in a decreasing $f_\textrm{m}$, as clearly visible in Fig.~ \ref{fig:BfieldDependence}(a). 
%\CB{[I changed this sentenced, before it was argued with an increasing $\lambda_\text{eff}$ to explain the changing $f_\text{m}$]} 
Like for the temperature dependence in Fig.\ \ref{fig:TemperatureDependenceFrequency}, the designed and the parasitic modes assemble as well-separated bundles. The designated modes decrease more strongly with field due to the larger filling fraction of superconducting material within the mode volume compared to the parasitic modes. As indicated by arrows in Fig.~ \ref{fig:BfieldDependence}(a) and Fig.~\ref{fig:BfieldDependence}(c), two kinks in the data can be identified around 0.2~T and 0.4~T. From comparison with literature,\cite{Halbritter2005,Karasik1970,DasGupta1976} the first kink can be assigned to the first critical field $B_\textrm{c1}$, where vortices start to enter. The second kink indicates the second critical field $B_\textrm{c2}$, where superconductivity ceases in the bulk, while surface superconductivity continues for much higher static magnetic fields.

The field dependence of $Q_\textrm{m}$ in Fig.\ \ref{fig:BfieldDependence}(d) shows related behavior: again two kinks, for $B_\textrm{c1}$ and $B_\textrm{c2}$, can be identified around 0.2~T and 0.4~T. Above $B_\textrm{c1}$ the designed CPW modes exhibit strongly suppressed $Q_\textrm{m}$, while most of the parasitic modes are much less affected and hardly have any decrease in $Q_\textrm{m}$. Here one should keep in mind that the absolute zero-field $Q_\textrm{m}$ of the parasitics is already much lower than for the CPW modes.
The additional \lq oscillatory\rq{} field dependence of some of the modes (e.g.\ $n=11$, $n=13$, $p=3$, $p=5$, $p=6$) is due to overlap of the resonances in the microwave spectra with standing wave contributions of the background of the microwave spectra, which in these cases has not been fully covered by the fitting procedure and which changes as a function of field (and temperature) as the resonances move in frequency. This effect becomes more pronounced for broader resonances, and thus for higher fields and temperatures.


%The superconducting magnet in the cryostat provides a $B$-field parallel to the flip-chip. Data were only obtained for the R$_f$ setup. As in section \ref{sec:T} no $B$-dependent effects due to the $\epsilon$ of TiO$_2$ substrate are expected but due to the superconducting Nb-layer.

%In Fig.\ \ref{fig:B1} in (a) the frequency $f$ normalized to \mbox{$f(T=2\,\textrm{K})$} can be seen. The decrease of the frequency $f$ can be explained with the effectively altered geometry caused by the $B$-dependent $\lambda_L$ that impacts $\epsilon_\textrm{eff}$ and therefore $f$ as it could be also seen when investigating the $T$-dependence. Harmonic modes and parasitic modes respectively form bundles indicating a different distribution on the superconducting layer. 

%Previous experiments showed that for a perpendicular orientation of the $B$-field to the Nb-layer a decrease of $f$ already occurs for $B>0.5\,$mT, our data shows a decrease for $B>0.2\,$T.\cite{Bothner2012b} In a parallel orientation the superconducting Nb-layer can repel a $B$-field more effectively than in a perpendicular orientation, this explains the greater $B$-field necessary for a distortion in the investigated R$_\textrm{f}$ setup. 

%In Fig.\ \ref{fig:B1} in (b) the quality factor $Q$ normalized to $Q(T=2\,\textrm{K})$ can be seen. For the harmonic modes a weaker decrease of $Q$ can be seen for higher harmonics $n$. Again the weaker decrease of $Q$ is a result of the lower initial quality factor $Q(T=2\,\textrm{K})$ as it could already be seen in $T$-dependent data of $Q$. When comparing the $B$-dependence and the $T$-dependence of $Q$ a stronger decrease of $Q$ can be seen for the $B$-dependence. The $B$-field induces vortices in the superconducting layer which results in increased losses whereas the effective geometry remains unchanged \MS{Der Satz passt so?}. The parasitic modes are weaker affected by the $B$-field indicating a different distribution on the superconducting layer.

%In Fig.\ \ref{fig:B1} fluctuating frequencies and quality factors can be seen. These are the results of standing waves that add up with the resonances. With the strong decrease of the quality factor and the shifting frequencies, the fluctuations of some modes can be explained. In addition an interesting oberservation shall be mentioned: Sudden changes of $f$ in dependence of $B$ can be seen in Fig.\ \ref{fig:B1} in (I) in (b) which are marked with a black and red arrow. They can be seen for both harmonic modes and parasitic modes. It could be speculated that additional mechanisms have to be accounted for this behaviour.
%\MS{sollte ich auf den roten und schwarzen Pfeil eingehen?}

%\begin{figure}[h]
%	\centering
%	\includegraphics[width=\linewidth]{plots/Bdependence/total_q.pdf}
%	\caption{Beydeda Fig 4.11.}
%	\label{fig:B2}
%\end{figure}


%\MS{Fig.: based on Beydeda Fig.4.10/4.11}



\subsection{Power Dependence}

%\CB{[CB: Did not use any of Konstantins BSc-Plots, maybe more of the spectra and analysis can be shown (?)]}
%\MS{[MS: Yes, also the presently shown spectra for different powers, for Konstantin's case, start at much too high powers; here one does not even see the undistorted cases for low powers.]}


\begin{figure}
	\centering
	\includegraphics[width=\linewidth]{plots/Pdependence/Duffingosciallator_adj_2023_03_20.pdf}
	\caption{Measured spectra of transmission coefficient $|\hat{S}_{21}|$ for different harmonic and parasitic modes from the two setups $R_\textrm{f}$, with powers between -2~dBm (blue curve) and +17~dBm (brown curve) in steps of 1~dB, and $R_\textrm{p}$, with powers between -50~dBm (blue curve) and +17~dBm (brown curve) in steps of 3~dB. For greater powers $P$ anharmonicities arise which are characterized by sharp changes of $|\hat{S}_{21}|$. The decreased background of the spectra for high power is a result of the nonlinearity of the employed amplifier. % in another way than raising the power.
		%\MS{[MS: Combine the two figures about power dependence into one. For the R$_p$ case also show spectra at much lower power levels, where the spectra are undistorted and fully superconducting.]}
		%\MS{[MS: The caption has to be redone after the two figures are combined. Then also comment on the absolute value (background) for the spectra, whether this includes an amplifier.]}
	}
	\label{fig:PowerDependenceSpectra}
\end{figure}

\begin{figure}
	\centering
	\includegraphics[width=\linewidth]{plots/Pdependence/total_Q_adj_2023_03_20.pdf}
	\caption{Quality factor $Q$ in dependence of the power $P$ for (a) the R$_\mathrm{f}$ case and (b) the R$_\mathrm{p}$ case, measured at $T =$ 2~K. }
	\label{fig:PowerDependenceQ}
\end{figure}

Another strategy to probe the nature of the different resonances concerns power dependence, i.e.\ studying the nonlinear behavior of the superconducting element. Here we focus on the behavior at temperature 2~K, i.e.\ much lower than $T_{\textrm{c}}$. 
With increasing power, basically three regimes are expected:\cite{Chin1992,Cohen2002,Abdo2006,deVisser2010} for low probing power, the resonator response is linear and $f_{\textrm{m}}$ and $Q_{\textrm{m}}$ are independent of power. 
For higher powers, the losses e.g\ due to thermally excited quasiparticles lead to a temperature increase, which in turn leads to a reduction of $f_{\textrm{m}}$ and $Q_{\textrm{m}}$, following the behavior discussed in Section \ref{sec:TemperatureDependence}. 
For even higher powers, the current density induced by the microwave field overcomes the critical current density of the superconductor at some position within the resonator. 
(The local microwave power depends strongly on the mode and its standing-wave pattern.) 
In this moment the superconductor turns normal at this position, and then the resonator properties change dramatically and exhibit certain characteristics of a metallic resonator, such as much lower $Q_{\textrm{m}}$ and $f_{\textrm{m}}$.

This generic behavior is indeed found in our data, as seen in Fig.\ \ref{fig:PowerDependenceSpectra} for various exemplary resonances of both devices R$_\text{f}$ and R$_\text{p}$. Considering the harmonic mode $n=11$ of R$_\text{f}$ in Fig.\ \ref{fig:PowerDependenceSpectra}(b), then one sees that for powers below +3 dBM (cyan curve) the resonance is basically unchanged for all powers. 
For the range from +3 dBM to +10 dBM (orange curve), the resonance becomes broader and weaker for increasing powers and shifts to lower frequencies, but the lineshape is still Lorentzian and can be properly fitted by Eq.\ \ref{eq:FitFunctionResonances} 
%(sometimes called \lq weak nonlinearity\rq{} \cite{Chin1992})
. For higher powers the situation changes drastically: e.g.\ for +17 dBM, there is an abrupt jump in the spectrum at 7.988 GHz 
%(\lq strong nonlinearity\rq{} \cite{Chin1992} \CB{[Cenk: About removing the \lq strong\rq /\lq weak\rq{} comments; I also don't know, I did not take a look into the paper yet.]})
. 
Above this frequency, the data follow a much broader resonance curve that is characteristic of the resonator being (at least partially) not superconducting any more but in the metallic state. In some cases, one can also clearly identify a second jump, back into the superconducting state, e.g.\ in Figs.\ \ref{fig:PowerDependenceSpectra}(c), (e), and (g) for resonator R$_\textrm{p}$.
Whenever jumps occur in the resonance spectra, it is not possible to determine unique values for $f_{\textrm{n}}$ and $Q_{\textrm{n}}$ for the full spectrum.
Comparable nonlinear behavior in planar superconducting resonators has been studied for various cases,\cite{Chin1992,Cohen2002,Abdo2006,deVisser2010} and different microscopic origins and theoretical descriptions have been discussed, but for our goal of just distinguishing different types of resonator modes we do not aim at a quantitative description of the nonlinear behavior.

If one fits the observed resonance spectra for all modes and powers to Eq.\ \ref{eq:FitFunctionResonances}, thus disregarding that the fit will not work well for spectra that include jumps, then one obtains power-dependent values of $f_{\textrm{m}}$ and $Q_{\textrm{m}}$. Here we focus on the behavior of $Q_{\textrm{m}}$ as shown in Fig.\ \ref{fig:PowerDependenceQ} for both devices, R$_\textrm{f}$ and R$_\textrm{p}$. For lowest tested powers, all modes have power-independent $Q_{\textrm{m}}$, thus indicating the linear regime. For higher powers, there are cases where $Q_{\textrm{m}}$ smoothly evolves into decreasing behavior and others where this decrease starts abruptly. The latter are those where jumps in the spectra set in at a critical power, and thus the spectra are not fitted properly any more.

When it comes to distinguishing regular resonator harmonics from parasitic modes, we find the general trend that the nonlinear behavior (decreasing $Q_{\textrm{m}}$) for resonator harmonics starts at lower powers than for the parasitic modes. This can be explained as follows: the nonlinear behavior sets in if the microwave-induced current density locally overcomes a certain threshold. 
For the designed modes, the microwave signal is directly induced into the CPW of the resonator and thus the largest current density is destined to flow in the center conductor with its rather small cross section. Even if we do not know the actual field distribution for the parasitic modes, we can assume that the current densities induced locally in the superconducting film are substantially smaller. 
This holds for \lq three-dimensional cavity modes\rq, where the microwave field is distributed throughout the comparably large volume of substrate(s) of the chip(s) as well as the sample box.
Also for the case of undesired slotline modes, the microwave electric field and thus the induced current density in the center conductor is smaller compared to the CPW mode, and thus a higher power has to be supplied to the overall device to induce strong nonlinearity for such a resonance.


%This section deals with the dependence of the quality factor $Q$ on the microwave power $P$. Both setups $R_\textrm{f}$ and $R_\textrm{p}$ are investigated.

%In Fig.\ \ref{fig:P1} the microwave power dependence of the quality factor $Q(P)$ can be seen. Note here that the power is given in units of $[P]=$dBm which is a logarithmic scale. Up to a certain power $P\approx -10\,$dBm in the R$_\textrm{f}$ setup and $P\approx -30\,$dBm in the R$_\textrm{p}$ setup the quality factor $Q$ of the harmonic modes remain unchanged and decrease for greater powers. The parasitic modes are relatively unchanged. For intersecting quality factors $Q(P)$ a different behavior of parasitic and harmonic modes can be deduced and allows a distinction. The measurement of higher powers is limited by the Duffing behavior of the resonator.

%In Fig.\ \ref{fig:P2} the transmission coefficients $S_{21}(f)$ of single resonances of the harmonic modes and parasitic modes can be seen. Whereas for lower power $p=3\,$dBm the continuous Lorentz-behavior of the modes can be seen, the Duffing behavior becomes apparent for higher power $p>3\,$dBm. The Lorentz-behavior characterizes the excitation of harmonic modes and parasitic modes as a driven damped oscillator. The Duffing oscillator incorporates an additional non-continuous dependence of $S_{21}(f)$ on $f$ that scales with $P$. In general heating effects and vortex losses can be addressed to explain the Duffing-behavior \MS{Wir sollten darüber reden was zitiert werden kann}. A greater power $P$ is equivalent to a stronger current $j$ in the Nb-layer that causes non-linear effects of the surface impedance $R_s$. This results in a non-continuous dependence of the transmission coefficient $S_{21}$ as it is seen.

%The quality factor $Q(P)$ of the parasitic modes show a weaker dependence on the microwave power $P$. It can be assumed that the microwave field of the parasitic modes are weaker bounded by the resonator geometry as the microwave field of the harmonic modes. The current $j$ increases weaker in dependence of $P$ for a broader distribution of the microwave field on the superconducting plane, resulting in a weaker susceptibility of $Q$ in dependence of $P$ for parasitic modes.

%Please note that the decreased background transmission signal in Fig.\ \ref{fig:P2} is a result due to the non-linear amplifier that was used in the measurement. The measured quality factor is unaffected by this non-linearity.

\subsection{Dielectric Markers}\label{sec:DielectricMarkers}

%\begin{figure}[h]
%	\centering
%	\includegraphics[width=\linewidth]{plots/Dielectric_markers/shift_thesis_moden4.pdf}
%	\caption{Nikolaou Fig 5.13 \MS{How to do a compact version?}.}
%	\label{fig:DIM1}
%\end{figure}

%\begin{figure}
%	\centering
%	\includegraphics[width=0.8\linewidth]{plots/Dielectric_markers/Bemalung.png}
%	\caption{Schematic sketch of the $R_p$ resonators with the feedline. The resonators are partially marked with a customary permanent marker with $\epsilon >1$. Marking results in an altered $\epsilon_\textrm{eff}$ with shifted frequencies $f$. (I) and (II) show two sketches where in (II) two more resonators are marked. Resonator (0) is broken and therefore is not investigated}
%	\label{fig:DIM1}
%\end{figure}

\begin{figure}
	\centering
	\includegraphics[width=\linewidth]{plots/Dielectric_markers/shift_thesis_und_zuordnung_paperversion_adj_2023_03_27.pdf}
	\caption{Identifying resonator modes by application of dielectric markers. The left column shows the design of the device R$_\text{p,diel}$ for three different cases: original device (upper row), device with dielectric markers attached to two of the resonators (middle row; area of dielectric markers shaded in red), device with dielectric markers attached to two further resonators (middle row; area of dielectric markers shaded in green).
The middle and right columns show the transmission spectra$|\hat{S}_\textrm{21}|$, measured at temperature 2~K, for the three different states of the device. Going from upper to middle row, two resonances shift due to the dielectric markers, and the same happens from middle to lower row.
%Transmission spectrum  of the $R_p$ setup showing 5 resonance signals where all of them refer to the $n=1$ signal of a respective resonator since one resonator is broken. (0) refers to the unmarked chip and (I), (II) to the marked chips that were respectively investigated in 3 measurements. Marking the resonators causes a shift of the respective frequencies $f$ towards lower values due to an altered $\epsilon_\textrm{eff}$. The resonators are partially marked with a customary permanent marker with $\epsilon >1$. %On the left side schematic sketches of the $R_p$ chips can be seen. Marking results in an altered $\epsilon_\textrm{eff}$ with shifted frequencies $f$. %On the chips that are labeled (0), (I) and (II) respectively 0, 2 and 4 resonators are marked. Resonator (0) is broken and therefore is not investigated
	%\MS{[MS: This caption needs to be updated with the combined figure.]}
}
	\label{fig:DielectricMarkers}
\end{figure}

If the in-situ strategies of the previous sections do not suffice to unambiguously assign observed resonances to specific modes of the device, one can minutely modify the resonator structure and observe which modes in the spectra then behave as expected. Considering Eq.~\ref{eq:ResonanceFrequencies}, one approach is changing $\epsilon_\textrm{eff}$ in a controlled fashion. 
For the designated CPW resonator modes, $\epsilon_\textrm{eff}$ includes a contribution due to the temperature-dependent penetration depth of the superconductor but to lowest order is the arithmetic mean of the dielectric functions of the dielectric substrate (here: TiO$_2$ or Al$_2$O$_3$) and vacuum/air/helium gas above the substrate. One can tune this by adding a small amount of another dielectric material on top of the CPW.\cite{Ebensperger2019,Wisbey2019}
Here we follow this strategy by using a conventional permanent marker pen.

In this case, we use a separate device, R$_\text{p,diel}$, that follows the overall design of the R$_\text{p}$ setup, but this new chip features six resonators as it can be seen in Fig.\ \ref{fig:DielectricMarkers}. All six resonators were designed to be at the same frequency, and thus the original spectrum of this device, shown in Fig.\ \ref{fig:DielectricMarkers}, features five resonances very close in frequency. The sixth resonator did not work properly. Here the task of mode assignment is extended such that one wants to identify which of the designated modes belongs to which resonator.
So two of the resonators were \lq marked\rq{} in a first step and two other resonators in a second step. 
The respective spectra with the fundamental modes around 1.54~GHz in Fig.\ \ref{fig:DielectricMarkers} clearly show how in each of these steps two of the resonances move to lower frequencies. 
These thus belong to the resonators where pigments of the marker pen were added, and therefore the respective modes can be assigned. 
This particular strategy resembles procedures that are being used in the field of KIDs, where resonator frequencies can be permanently adjusted e.g.\ by laser-trimming.\cite{Liu2017}
Our approach with a marker pen is less quantitatively predictable, but it can be implemented more easily and reversibly.

%The previous sections focuse on the dependencies of the resonance frequencies and parasitic frequencies $f$ on the Nb-layer. An effective change of the resonator geometry due to disturbing effects on the superconducting Nb-layer causes an altered $\epsilon_\textrm{eff}$ although the dielectric constant of the materials in use do not change.

%Here the $R_\textrm{p}$ setup is investigated. The effective dielectric constant $\epsilon_\textrm{eff}$ is changed by adding an additional layer with $\epsilon>1$ on the resonator. As result $\epsilon_\textrm{eff}$ increases. Adding the layer is done by partially marking the resonator with a customary permanent marker. The marked resonator can be seen in Fig.\ \ref{fig:DIM1}. This technique allows to distinct multiple resonators on a chip by marking them individually and observing the effect on the transmission spectrum.

%As $\epsilon_\textrm{eff}$ increases, the resonance frequencies $f$ of the respective resonators decrease since $f\propto \epsilon_\textrm{eff}^{-1/2}$. The transmission spectrum (0) in Fig.\ \ref{fig:DIM2} shows 5 signals which are the $n=1$ resonances of the respective resonators in Fig.\ \ref{fig:DIM1} since the resonator (0) is broken. The transmission signal I in Fig.\ \ref{fig:DIM2} refers to the marked chip I in Fig.\ \ref{fig:DIM1} and shows that two signal shifted towards lower frequencies as expected. The transmission signal II shows two more shifted frequencies as expected from the marked chip II with two more marked resonators. This allows a clear distinction of multiple resonators on the same chip.

%A few general remarks should be added. The shifted frequency $f$ is not the only effect, also the frequency bandwidth $f_B$ increases resulting in a decreased quality factor $Q$. This is not unusual as the permanent marker also introduces increased losses. Therefore caution should be exercised as covering the whole resonator could accidentally wipe out the entire resonance. Also marking with a marker does not allow to define an amount that is placed on the resonator. In that sense it is not suitable to use this technique to estimate the $\epsilon$ of the marker but it can used to clearly distinct marked and unmarked resonators by observing the resonance before and after marking.





\subsection{ESR Markers}

\begin{figure}
	\centering
	\includegraphics[width=\linewidth]{plots/ESR/ESR_Q_3_adj_2023_03_27.pdf}
	\caption{Normalized quality factor $Q$ of the harmonic and parasitic modes of the TiO$_2$ flip-chip resonator R$_\textrm{f}$ at $T = 2\,K$ in dependence of the static magnetic field $B$. Here $Q$ is normalized to its value at $B = 0\,$T. The DPPH sample is placed at the short-circuited end of the resonator (the end that is not adjacent to the feedline), as shown in the inset.
	}
  \label{fig:ESR1}
\end{figure}

\begin{figure}
	\centering
	\includegraphics[width=\linewidth]{plots/ESR/ESR_comparison_5_adj_2023_03_27.pdf}
	\caption{Normalized quality factor $Q$ for the $n = 1$, $n = 3$ and $n = 5$ modes for two separate measurements, where DPPH is applied to different locations, positions 1 and 2, as indicated on photographs on the right.}
	\label{fig:ESR2}
\end{figure}
%
%\MS{[MS: Cenk: Deshalb setze bitte eine der beiden folgenden Optionen um:\\
%(a) Ersetze ueberall, wo bisher $B$ fuer das statische Magnetfeld steht, dieses $B$ durch $B_0$.\\
%oder (b) ersetze ueberall, wo jetzt $B_0$ fuer das statische Magnetfeld steht, dieses $B_0$ durch $B$ und schreibe ueberall, wo es notwendig ist, explizit dazu, dass es sich bei diesem $B$ um das \lq static magnetic field\rq{} handelt.]}
%
%\CB{[Ich wähle Option (b)]}
%
The \lq dielectric marker\rq{} approach as presented above is hard to implement for the distant flip-chip design of R$_\textrm{f}$ because it would require removing the flip-chip from the sample box and later reattaching it, which for our way of mounting typically slightly changes the coupling between feedline and resonator chip, and thus basically all resonance frequencies, designed as well as parasitic, change somewhat.

Here, a different approach is possible that employs a \lq magnetic marker\rq{}. More specific, we use electron spin resonance (ESR) of the well-known paramagnet DPPH (1,1-diphenyl-2-picryl-hydrazyl) that is commonly used as reference material in ESR spectroscopy.\cite{Ghirri2015}
Magnetic effects are neglected in Eq.\ \ref{eq:ResonanceFrequencies} and in all discussions presented so far. This is justified because the frequency-dependent magnetic permeability for the relevant materials and settings of our experiments are very close to unity. This changes if the ESR condition holds:
\begin{equation}
	hf = g\mu_B B \label{eq:ESRCondition}
\end{equation}
with $f$ the frequency of a driving microwave magnetic field, $h$ Planck's constant, $g$ the Land\'{e} factor of the material (for DPPH $g\approx 2$), $\mu_B$ Bohr's magneton, and $B$ the external static magnetic field. 
For this combination of  $f$ and $B$, the microwave magnetic field component that is perpendicular to the external static magnetic field $B$ can induce transitions between the Zeeman-split energy levels of the material, and this means characteristic absorption of the microwave signal. For our case of resonance modes with almost fixed respective frequencies $f=f_\textrm{m}$ this means that if one sweeps the external static magnetic field $B$ and then fulfills Eq.\ \ref{eq:ESRCondition}, the microwave losses due to ESR will reduce the $Q_\textrm{m}$ of the mode at this particular $B$.

To take advantage of ESR for resonator mode identification, we apply a small amount of DPPH at a certain position of the resonator chip where we expect certain resonance modes to have strong microwave magnetic fields and thus strong ESR signal whereas other modes with weaker or absent microwave magnetic field at this position should exhibit weaker or absent ESR. Our resonator thus acts like an on-chip ESR spectrometer.\cite{Scheffler2013,Samkharadze2016,JavaheriRahim2016}

Fig.\ \ref{fig:ESR1} shows such an experiment, where the DPPH is deposited at \lq position 1\rq{} at the short-circuited end of the $\lambda/2$-type resonator of R$_\textrm{f}$: for all harmonics of the the CPW resonator, this position features a maximum of current and microwave magnetic field, and thus all harmonics should exhibit a clear ESR signal. This is indeed the case, see Fig.\ \ref{fig:ESR1}: the quality factors of the different modes as function of static external magnetic field show the overall evolution already known from Fig.\ \ref{fig:BfieldDependence}(a), but in addition there are pronounced, sharp minima. 
These occur at combinations of $f_\textrm{n}$ and $B$ according to the ESR condition, and thus their presence demonstrates that ESR can be used to encode information about certain resonance modes. As expected, all investigated CPW harmonics feature a clear ESR signal. 
In contrast, most of the observed ESR signals for parasitic modes at their respective $f_\textrm{m}$-$B$-combinations are weak. This is expected for three-dimensional cavity modes where the mode extends over a much larger volume than the one-dimensional CPW modes, and thus the microwave magnetic field at the position of the DPPH sample should be much weaker than for the CPW modes, leading to absence of observed ESR. 
%\CB{ESR-Signale sind auch bei den Parasitären gemessen worden. Die absoluten Q-Werte sind bei den Parasitären allerdings viel geringer und deswegen verändert sich das Q bei erfüllter ESR-condition nicht so stark wie bei den Resonatormoden (ich habe im Kopf gerade $Q_1^{-1} + Q_2^{-1} = Q_\text{ers}^{-1}$). Ich war damals in meiner BSc-Arbeit zum Schluss gekommen dass die ESR-Methoden grundsätzlich funktioniert aber nicht erlaubt Parasitäre und Resonatormoden zu trennen}.
One exception is the $p=6$ parasitic, which indeed features a pronounced ESR signal. This could mean that this mode is a slotline mode or a three-dimensional mode within the TiO$_2$ subtrate that \lq accidentally\rq{} features a substantial microwave magnetic field at the DPPH position. 
%\MS{[MS: is the previous sentence sound?]}

To further investigate the information that can be gained by ESR markers, we have performed another experiment with the DPPH deposited at a different position: we now choose \lq position 2\rq{} such that it should correspond to a microwave magnetic field node of the $n=5$ harmonic, and thus this mode should barely excite ESR. For the $n=3$ harmonic the microwave magnetic field should be substantially weaker compared to position~1 whereas for the fundamental $n=1$ harmonic there should only be a slight reduction and thus still strong ESR as before. In Fig.\ \ref{fig:ESR2} we show a close-up on the normalized quality factor $Q(B)$ for the $n=1,3,5$ modes for both discussed positions. As expected the ESR signal is almost completely suppressed for the $n=5$ mode, strongly reduced for the $n=3$ mode and slightly reduced for the $n=1$ mode when going with DPPH from position 1 to position 2. A small shift of the ESR signal to lower static magnetic fields $B$ can be observed which can be attributed to a small offset field of the superconducting magnet in the setup.

The ESR-marker technique is an elegant way to evaluate the microwave field strengths of different resonant modes at certain geometrical positions, and thus to verify the assignment of the modes as being dedicated resonator modes or parasitic. Compared to the dielectric markers of Section \ref{sec:DielectricMarkers} it has the advantage that the quality factor of any designated mode is substantially affected only near a single value of the external static magnetic field $B$, when Eq.~\ref{eq:ESRCondition} is met, and not affected for all other values of the external static magnetic field $B$ and thus possibly not interfering with other main experiments of interest.

%Here the $R_\textrm{f}$ setup is investigated. Electron Spin Resonance (abbreviated ESR) can be used to investigate the local magnetic field strength on the resonator. Therefore 1,1-diphenyl-2-picryl-hydrazyl (abbreviated DPPH) is deposited locally on the resonator. DPPH is often used in microwave experiments as it shows a strong ESR signal. The quality factor $Q(B)$ of the modes is measured in dependence of a varied external magnetic field $B$. According to the Zeeman effect an energy difference $\Delta E$ between two electron states with opposite spins occurs in a magnetic field $B$. The microwave photons in the resonator can excite transitions between the split energy states when the condition
%\begin{equation}
%	hf = g\mu_B B \label{eq:ESR1}
%\end{equation}
%is met. Here $f$ is the microwave frequency, $h$ Planck's constant, $g\approx 2$ the Landé factor and $\mu_B$ Bohr's magneton.

%When the condition in equation \ref{eq:ESR1} is met, a sharp decrease of the quality factor $Q(B)$ can be observed as the losses are increased. This can be seen in Fig.\ \ref{fig:ESR1} where DPPH is deposited on the open end of the resonator. Fig.\ \ref{fig:ESR1} shows the normalized $Q(B)$ for the considered modes. The ESR-Signal is visible for every mode because the magnetic field $B_\textrm{photon}$ of the microwave photon in the resonator has a maximum at the open end. The ESR-signal is also visible for the parasitic modes but strongly decreased due to the initial low quality factor $Q(B=0)$.


\section{Summary}
This study examines differences between harmonic and parasitic modes of superconducting CPW resonators on a phenomenological level. Distinguishing these different types of modes can be important for the reliable interpretation of cryogenic microwave resonator data, and it can be particularly challenging if unconventional device geometries and/or materials with unknown microwave characteristics are involved.\cite{Engl2019} Therefore different mode assignment strategies have been presented, which can be grouped into those that analyze typically accessible microwave data of a given resonator structure and those that slightly modify the resonator structure to enable clearer mode assignment.

Tracking the resonance frequency $f_\textrm{m}$ of various modes as a function of temperature $T$ and external static magnetic field $B$ showed that designed harmonics and parasitic modes respectively form separate bundles in their decrease for increasing $T$ and $B$. This is due to the superconductor having a much larger filling fraction of the resonance mode volumes for the designed CPW modes compared to the parasitic modes. For the same reason, the quality factor $Q_\textrm{m}$ of the resonator harmonics exhibits stronger temperature and magnetic-field dependence compared to parasitic modes. Also in the power dependence much stronger nonlinear effects are observed for the designed harmonic modes compared to the parasitic ones. If such data sets are not sufficient to unambiguously assign the modes, one can add small amounts of dielectric and/or ESR markers to selectively tune some of the modes, and then check for the expected changes in the microwave response.
While any of the presented techniques might be sufficient for mode assignment, we found that in the more challenging cases the combination of several of them is most convincing.


%Using the magnetic field to identify individual modest could also be done in a different way, by designing dedicated vortex pinning sites in pattern such that e.g.\ matching fields show up differently in different modes.\cite{Kroll2019}

%Typically used setups in the research of quantum computing were investigated. It was shown that on this phenomenological level with feasible microwave measurement techniques it is possible to distinct harmonic modes from parasitic modes.

%The behavior of the frequency $f$ in the presence of varying temperature $T$ and magnetic field $B$ shows that harmonic and parasitic modes respectively form bundles and decrease for greater $T$ and $B$. This can be explained with the superconducting Nb-layer and the London penetration depth $\lambda_L$ that is susceptible to $T$ and $B$. An altered $\lambda_L$ causes an effectively changed resonator geometry resulting in a slightly changed distribution of the microwave field in the resonator which returns a different $\epsilon_\textrm{eff}$ which can be seen in the data as decreasing resonance frequency $f$. The quality factor $Q$ showes a decrease for greater $T$, $B$ as the losses of the superconducting Nb-layer increases whereas the decrease of the harmonic modes is substantially stronger than that of the parasitic modes. The power dependence of the quality factor $Q$ showes similar behavior.

%Apart from the dependencies from the frequency $f$ and quality factor $Q$ the resonator itself can be adjusted to check for features in the spectrum. Depositing a dielectric on the resonator by using a customary permanent marker increases $\epsilon_\textrm{eff}$ resulting in a lowered frequency $f$. This way multiple resonator with the same length on the resonator can be distinguished by marking. By locally depositing the magnetic material DPPH on the resonator ESR spectroscopy can be performed. The quality factor of the harmonic and parasitic modes is measured in dependence of an external magnetic field $B$. If the condition to excite a transition between the split energy states in DPPH is met by the microwave frequency $f$ and the external magnetic field $B$, the ESR-signal can be seen as a sharp decrease of the quality factor $Q$. The strength of the ESR-signal depends on the local magnetic field strength of the microwave field in the resonator. This can be used to examine the magnetic field strength of the harmonic and parasitic modes on the resonator for a clear distinction.

%The presented techniques allow a reliable distinction of harmonic and parasitic modes. The wide range of presented techniques ensures that if one approach is not suitable, another approach may be very well suitable. Open questions remain about the origin of the parasitic modes which is not discussed in this paper. From a technical point of view this paper shows that a deeper understanding of the nature of parasitic modes is not necessarily required for the distinction of parasitic modes from harmonic modes. From a physicist's point of view the question about the origin of parasitic modes may not be intriguing enough to focus on it. In that sense the presented techniques treat the problem of mode identification in an appropriate fashion.


%-----------
%This study...
%This study examines differences between harmonic modes and parasitic modes of superconducting resonators on a phenomenological level. Typically used setups in the research of quantum computing were investigated. It was shown that on this phenomenological level with feasible microwave measurement techniques it is possible to distinct harmonic modes from parasitic modes.


%----
%\MS{The strategies presented in this either can be applied during experiments (temperature, magnetic field, power dependence) or by slight modifications of the resonator that then require an additional cooldown (dielectric and ESR markers). If none of these techniques is sufficient, further efforts could be applied that include fabrication of purpose-designed resonators, e.g.\ using dedicated vortex pinning sites only in certain parts of the superconductor to take of advantage of commensurability effects if the field distribution is known.\cite{Kroll2019}
%[Marc:Some relevant papers: Bothner \cite{Bothner2011,Bothner2012a,Bothner2012};. One could have controlled flux pinning cite in the resonator and observe explicit field dependence; but difficult in fabrication and requires field-cooling strategy.\cite{Kroll2019}. Maybe not mention these here at all?]}


\section*{Acknowledgments}
We thank D. Bothner for helpful discussions and Deutsche Forschungsgemeinschaft (DFG) for financial support.

%\section*{Appendix}
%\MS{If there is anything else that we want to report, Maybe anisotropic epsilon. Comment by Cenk: Anisotropic epsilon is probably not important}


\begin{thebibliography}{99}


\bibitem{Wallraff2004} A. Wallraff, D. I. Schuster, A. Blais, L. Frunzio, R.- S. Huang, J. Majer, S. Kumar, S. M. Girvin, and R. J. Schoelkopf,
Nature \textbf{431}, 162 (2004).
%Strong coupling of a single photon to a superconducting qubit using circuit quantum electrodynamics

\bibitem{Sillanpaeae2007} M. A. Sillanp\"a\"a, J. I. Park, and R. W. Simmonds,
Nature \textbf{449}, 438 (2007).
%Coherent quantum state storage and transfer between two phase qubits via a resonant cavity

\bibitem{Majer2007} J. Majer, J. M. Chow, J. M. Gambetta, Jens Koch, B. R. Johnson, J. A. Schreier, L. Frunzio, D. I. Schuster, A. A. Houck, A. Wallraff, A. Blais, M. H. Devoret, S. M. Girvin, and R. J. Schoelkopf,
Nature \textbf{449}, 443 (2007).
%Coupling superconducting qubits via a cavity bus

\bibitem{Schuster2010} D. I. Schuster, A. P. Sears, E. Ginossar, L. DiCarlo, L. Frunzio, J. J. L. Morton, H. Wu, G. A. D. Briggs, B. B. Buckley, D. D. Awschalom, and R. J. Schoelkopf,
Phys. Rev. Lett. \textbf{105}, 140501 (2010).
%High-Cooperativity Coupling of Electron-Spin Ensembles to Superconducting Cavities

\bibitem{Kubo2010} Y. Kubo, F. R. Ong, P. Bertet, D. Vion, V. Jacques, D. Zheng, A. Dr\'{e}au, J.-F. Roch, A. Auffeves, F. Jelezko, J. Wrachtrup, M. F. Barthe, P. Bergonzo, and D. Esteve,
Phys. Rev. Lett. \textbf{105}, 140502 (2010).
%Strong Coupling of a Spin Ensemble to a Superconducting Resonator

\bibitem{Huebl2013} H. Huebl, C. W. Zollitsch, J. Lotze, F. Hocke, M. Greifenstein, A. Marx, R. Gross, and S. T. B. Goennenwein,
Phys. Rev. Lett. \textbf{111}, 127003 (2013).
%High Cooperativity in Coupled Microwave Resonator Ferrimagnetic Insulator Hybrids

\bibitem{Ghirri2015} A. Ghirri, C. Bonizzoni, D. Gerace, S. Sanna, A. Cassinese, and M. Affronte
Appl. Phys. Lett. \textbf{106}, 184101 (2015).
%YBa2Cu3O7 microwave resonators for strong collective coupling with spin ensembles

\bibitem{Gu2017} X. Gu, A. F. Kockum, A. Miranowicz Y.-x. Liu, and F. Nori,
Phys. Rep. \textbf{718} \textbf{-} \textbf{719}, 1 (2017).
%Microwave photonics with superconducting quantum circuits

\bibitem{Hattermann2017} H. Hattermann, D. Bothner, L. Y. Ley, B. Ferdinand, D. Wiedmaier, L. S\'ark\'any, R. Kleiner, D. Koelle, and J. Fort\'agh,
Nature Commun. \textbf{8}, 2254 (2017).
%Coupling ultracold atoms to a superconducting coplanar waveguide resonator

%\bibitem{Bonizzoni2018} C. Bonizzoni, A. Ghirri, and M. Affronte,
%Adv. Phys. X, \textbf{3}, 1435305 (2018).
%Coherent coupling of molecular spins with microwave photons in planar superconducting resonators

\bibitem{Day2003} P. K. Day, H. G. LeDuc, B. A. Mazin, A. Vayonakis, J, Zmuidzinas,
Nature \textbf{425}, 817 (2003).
%A broadband superconducting detector suitable for use in large arrays

\bibitem{Zmuidzinas2012} J. Zmuidzinas,
Annu. Rev. Condens. Matter Phys. \textbf{3}, 169 (2012).
%Superconducting Microresonators: Physics and Applications

\bibitem{Adam2018} R. Adam \textit{et al.},
% R. Adam, A. Adane, P. A. R. Ade, P. André, A. Andrianasolo, H. Aussel, A. Beelen, A. Benoît, A. Bideaud, N. Billot, O. Bourrion, A. Bracco, M. Calvo, A. Catalano, G. Coiffard, B. Comis, M. De Petris, F.-X. Désert, S. Doyle, E. F. C. Driessen, R. Evans, J. Goupy, C. Kramer, G. Lagache, S. Leclercq, J.-P. Leggeri, J.-F. Lestrade, J. F. Macías-Pérez, P. Mauskopf, F. Mayet, A. Maury, A. Monfardini, S. Navarro, E. Pascale, L. Perotto, G. Pisano, N. Ponthieu, V. Revéret, A. Rigby, A. Ritacco, C. Romero, H. Roussel, F. Ruppin, K. Schuster, A. Sievers, S. Triqueneaux, C. Tucker, and R. Zylka,
Astron. Astrophys. \textbf{609},  A115 (2018).
%The NIKA2 large-field-of-view millimetre continuum camera for the 30 m IRAM telescope

\bibitem{Battistelli2015} E. S. Battistelli \textit{et al.},
%E. S. Battistelli, F. Bellini, C. Bucci, M. Calvo, L. Cardani, N. Casali, M. G. Castellano, I. Colantoni, A. Coppolecchia, C. Cosmelli, A. Cruciani, P. de Bernardis, S. Di Domizio, A. D’Addabbo, M. Martinez, S. Masi, L. Pagnanini, C. Tomei, and M. Vignati,
Eur. Phys. J. C \textbf{75}, 353 (2015).
%CALDER: neutrinoless double-beta decay identification in TeO2 bolometers with kinetic inductance detectors


\bibitem{Scheffler2013} M. Scheffler, K. Schlegel, C. Clauss, D. Hafner, C. Fella, M. Dressel, M. Jourdan, J. Sichelschmidt, Cornelius Krellner, C. Geibel, and F. Steglich,
Phys. Status Solidi B \textbf{250}, 439 (2013).
%Microwave spectroscopy on heavy-fermion systems: Probing the dynamics of charges and magnetic moments

\bibitem{Hafner2014} D. Hafner, M. Dressel, and M. Scheffler,
Rev. Sci. Instrum. \textbf{85} 014702 (2014).
%Surface-resistance measurements using superconducting stripline resonators.

\bibitem{McRae2020} \MS{C. R. H. McRae, H. Wang, J. Gao, M. R. Vissers, T. Brecht, A. Dunsworth, D. P. Pappas, and J. Mutus,
Rev. Sci. Instrum. \textbf{91}, 091101 (2020).}
%Materials loss measurements using superconducting microwave resonators

\bibitem{DiIorio1988} M. S. DiIorio, A. C. Anderson, and B.-Y. Tsaur,
Phys. Rev. B \textbf{38}, 7019 (1988).
%rf surface resistance of Y-Ba-Cu-O thin films

\bibitem{Oates1991} D. E. Oates, Alfredo C. Anderson, C. C. Chin, J. S. Derov, G. Dresselhaus, and M. S. Dresselhaus,
Phys. Rev. B \textbf{43}, 7655 (1991).
%Surface-impedance measurements of superconducting NbN films

\bibitem{Andreone1993} A. Andreone, A. DiChiara, G. Peluso, M. Santoro, C. Attanasio, L. Maritato, and R. Vaglio,
J. Appl. Phys. \textbf{73}, 4500 (1993).
%Surface impedance measurements of superconducting (NbTi)N films by a ring microstrip resonator technique

\bibitem{Zemlicka2015} M. \v{Z}emli\v{c}ka, P. Neilinger, M. Trgala, M. Reh\'ak, D. Manca, M. Grajcar, P. Szab\'o, P. Samuely, \v{S}. Ga\v{z}i, U. H\"ubner, V. M. Vinokur, and E. Il'ichev,
Phys. Rev. B \textbf{92}, 224506 (2015).
%Finite quasiparticle lifetime in disordered superconductors

\bibitem{Driessen2012} E. F. C. Driessen, P. C. J. J. Coumou, R. R. Tromp, P. J. de Visser, and T. M. Klapwijk,
Phys. Rev. Lett. \textbf{109}, 107003 (2012).
%Strongly Disordered TiN and NbTiN s-Wave Superconductors Probed by Microwave Electrodynamics

\bibitem{Beutel2016} M. H. Beutel, N. G. Ebensperger, M. Thiemann, G. Untereiner, V. Fritz, M. Javaheri, J. N\"{a}gele, Roland R\"{o}sslhuber, Martin Dressel, and M. Scheffler,
Supercond. Sci. Technol. \textbf{29}, 085011 (2016).
%Microwave study of superconducting Sn films above and below percolation

\bibitem{Thiemann2018a} M. Thiemann, M. H. Beutel, M. Dressel, N. R. Lee-Hone, D. M. Broun, E. Fillis-Tsirakis, H. Boschker, J. Mannhart, and M. Scheffler,
Phys. Rev. Lett. \textbf{120}, 237002 (2018).
%Single-Gap Superconductivity and Dome of Superfluid Density in Nb-Doped SrTiO3

\bibitem{Thiemann2018b} M. Thiemann, M. Dressel, and M. Scheffler,
Phys. Rev. B \textbf{97}, 214516 (2018).
%Complete electrodynamics of a BCS superconductor with μeV energy scales: Microwave spectroscopy on titanium at mK temperatures

\bibitem{Manca2019} N. Manca, D. Bothner, A. M. R. V. L. Monteiro, D. Davidovikj, Y. G. Sa\u{g}lam, M. Jenkins, M. Gabay, G. Steele, and A. D. Caviglia,
Phys. Rev. Lett. \textbf{122}, 036801 (2019).
%Bimodal Phase Diagram of the Superfluid Density in LaAlO3/SrTiO3 Revealed by an Interfacial Waveguide Resonator




\bibitem{Anlage1989} S. M. Anlage, H. Sze, H. J. Snortland, S. Tahara, B. Langley, C.-B. Eom, M. R. Beasley, and R. Taber,
Appl. Phys. Lett. \textbf{54}, 2710 (1989).
%Measurements of the magnetic penetration depth in YBa2Cu3O7−δ thin films by the microstrip resonator technique

\bibitem{Langley1991} B. W. Langley, S. M. Anlage, R. F. W. Pease, and M. R. Beasley,
Rev. Sci. Instrum. \textbf{62}, 1801 (1991).
%Magnetic penetration depth measurements of superconducting thin films by a microstrip resonator technique

\bibitem{Revenaz1994} S. Revenaz, D. E. Oates, D. Labb\'{e}-Lavigne, G. Dresselhaus, and M. S. Dresselhaus,
Phys. Rev. B \textbf{50}, 1178 (1994).
%Frequency dependence of the surface impedance of YBa2Cu3O7−δ thin films in a dc magnetic field: Investigation of vortex dynamics

\bibitem{Porch1995} A. Porch, M. J. Lancaster, and R. G. Humphreys,
IEEE Trans. Microw. Theory Techn. \textbf{43}, 306 (1995).
%The coplanar resonator technique for determining the surface impedance of YBa/sub 2/Cu/sub 3/O/sub 7-/spl delta// thin films

\bibitem{Zaitsev2001} A. G. Zaitsev, R. Schneider, G. Linker, F. Ratzel, R. Smithey, and J. Geerk,
Appl. Phys. Lett. \textbf{79}, 4174 (2001).
%Nonlinear effects in YBa2Cu3O7Àx microstrip resonators on sapphire

\bibitem{Wang2007} Y. Wang, H. T. Su, F. Huang, and M. J. Lancaster,
IEEE Trans. Appl. Supercond. \textbf{17}, 3632 (2007).
%Measurement of YBCO Thin Film Surface Resistance Using Coplanar Line Resonator Techniques From 20 MHz to 20 GHz

\bibitem{Ghigo2012} G. Ghigo, F. Laviano, R. Gerbaldo, and L. Gozzelino,
Supercond. Sci. Technol. \textbf{25}, 115007 (2012).
%Tuning the response of YBCO microwave resonators by heavy-ion patterned micro-channels

\bibitem{Scheffler2015} M. Scheffler, M. M. Felger, M. Thiemann, D. Hafner, K. Schlegel, M. Dressel, K. S. Ilin, M. Siegel, S. Seiro, C. Geibel, and F. Steglich,
Acta IMEKO \textbf{4}, 47 (2015).
%Broadband Corbino spectroscopy and stripline resonators to study the microwave properties of superconductors

\bibitem{Ghigo2016} G. Ghigo, R. Gerbaldo, L. Gozzelino, F. Laviano, and T. Tamegai,
IEEE Trans. Appl. Supercond. \textbf{26}, 7300104 (2016).
%Penetration Depth and Quasiparticle Conductivity of Co- and K-Doped BaFe2As2 Crystals, Investigated by a Microwave Coplanar Resonator Technique

\bibitem{Parkkinen2015} K. Parkkinen, M. Dressel, K. Kliemt, C. Krellner, C. Geibel, F. Steglich, and M. Scheffler,
Physics Procedia \textbf{75}, 340 (2015).
%Signatures of Phase Transitions in the Microwave Response of YbRh2Si2 

\bibitem{Davidovikj2017} D. Davidovikj, N. Manca, H. S. J. van der Zant, A. D. Caviglia, and G. A. Steele,
Phys. Rev. B \textbf{95}, 214513 (2017).
%Quantum paraelectricity probed by superconducting resonators

\bibitem{Engl2019} V. T. Engl, N. G. Ebensperger, L. Wendel, and M. Scheffler,
arXiv:1911.11456 [cond-mat.supr-con]
%Planar GHz Resonators on SrTiO3: Suppressed Losses at Temperatures below 1 K

\bibitem{Wallace1998} W. J. Wallace and R. H. Silsbee,
Rev. Sci. Instrum. \textbf{62}, 1754 (1991).
%Microstrip resonators for electron‐spin resonance

\bibitem{Bushev2011} P. Bushev, A. K. Feofanov, H. Rotzinger, I. Protopopov, J. H. Cole, C. M. Wilson, G. Fischer, A. Lukashenko, and A. V. Ustinov,
Phys. Rev. B \textbf{84}, 060501 (2011).
%Ultralow-power spectroscopy of a rare-earth spin ensemble using a superconducting resonator

\bibitem{Malissa2013} H. Malissa, D. I. Schuster, A. M. Tyryshkin, A. A. Houck, and S. A. Lyon,
Rev. Sci. Instrum. \textbf{84}, 025116 (2013).
%Superconducting coplanar waveguide resonators for low temperature pulsed electron spin resonance spectroscopy

\bibitem{Bondorf2018} L. Bondorf, M. Beutel, M. Thiemann, M. Dressel, D. Bothner, J. Sichelschmidt, K. Kliemt, C. Krellner, and M. Scheffler,
Physica B \textbf{536}, 331 (2018).
%Angle-dependent electron spin resonance of YbRh2Si2 measured with planar microwave resonators and in-situ rotation

\bibitem{Golovchanskiy2018}I. A. Golovchanskiy , N. N. Abramov, V. S. Stolyarov , I. V. Shchetinin, P. S. Dzhumaev, A. S. Averkin, S. N. Kozlov, A. A. Golubov, V. V. Ryazanov, and A. V. Ustinov,
J. Appl. Phys. \textbf{123}, 173904 (2018).
%Probing dynamics of micro-magnets with multi-mode superconducting resonator

\bibitem{Ranjan2020} V. Ranjan, S. Probst, B. Albanese, T. Schenkel, D. Vion, D. Esteve, J. J. L. Morton, and P. Bertet,
Appl. Phys. Lett. \textbf{116}, 184002 (2020).
%Electron spin resonance spectroscopy with femtoliter detection volume

\bibitem{Miksch2021} B. Miksch, A. Pustogow, M. Javaheri Rahim, A. A. Bardin, K. Kanoda, J. A. Schlueter, R. H\"ubner, M. Scheffler, and M. Dressel,
Science \textbf{372}, 276 (2021).
%Gapped magnetic ground state in quantum spin liquid candidate κ-(BEDT-TTF)2Cu2(CN)3

\bibitem{OConnell2008}\MS{A. D. O’Connell, M. Ansmann, R. C. Bialczak, M. Hofheinz, N. Katz, E. Lucero, C. McKenney, M. Neeley, H. Wang, E. M. Weig, A. N. Clelanda, and J. M. Martinis,
Appl. Phys. Lett. 92, 112903 (2008)}
%Microwave dielectric loss at single photon energies and millikelvin temperatures

\bibitem{Wisbey2019} \MS{D. S. Wisbey, M. R. Vissers, J. Gao, J. S. Kline, M. O. Sandberg, M. P. Weides, M. M. Paquette, S. Karki, J. Brewster, D. Alameri, I. Kuljanishvili, A. N. Caruso, and D. P. Pappas,
J. Low Temp. Phys. \textbf{195}, 474 (2019).}
%Dielectric Loss of Boron-Based Dielectrics on Niobium Resonators

\bibitem{Ebensperger2019} \MS{N. G. Ebensperger, B. Ferdinand, D. Koelle, R. Kleiner, M. Dressel and M. Scheffler,
Rev. Sci. Instrum. \textbf{90}, 114701 (2019).}
%Characterizing dielectric properties of ultra-thin films using superconducting coplanar microwave resonators


\bibitem{Doyle2008} S. Doyle, P. Mauskopf, J. Naylon, A. Porch  and C. Duncombe, 
J. Low Temp. Phys. \textbf{151}, 530 (2008).
%Lumped element kinetic inductance detectors

\bibitem{FornDiaz2010} P. Forn-D\'{\i}az, J. Lisenfeld, D. Marcos, J. J. Garc\'{\i}a-Ripoll, E. Solano, C. J. P. M. Harmans, and J. E. Mooij,
Phys. Rev. Lett. \textbf{105}, 237001 (2010).
%Observation of the Bloch-Siegert Shift in a Qubit-Oscillator System in the Ultrastrong Coupling Regime

\bibitem{Frunzio2005} L. Frunzio, A. Wallraff, D. Schuster, J. Majer, and R. Schoelkopf,
IEEE Trans. Appl. Supercond. \textbf{15}, 860 (2005).
%Fabrication and characterization of superconducting circuit QED devices for quantum computation

\bibitem{Goeppl2008} M. G\"{o}ppl, A. Fragner, M. Baur, R. Bianchetti, S. Filipp, J. M. Fink, P. J. Leek, G. Puebla, L. Steffen, and A. Wallraff,
J. Appl. Phys. \textbf{104}, 113904 (2008).
%Coplanar waveguide resonators for circuit quantum electrodynamics

\bibitem{Andreone1997} A. Andreone, A. Cassinese, A. Di Chiara, M. Iavarone, F. Palomba, A. Ruosi, and R. Vaglio,
IEEE Trans. Appl. Supercond. \textbf{7}, 1772 (1997).
%Microwave measurements of superconducting Nb/sub 3/Sn films by a microstrip resonator technique

\bibitem{Zou2017} S. Zou, Y. Cao, V. Gupta, B. Yelamanchili, J. A. Sellers, C. D. Ellis, D. B. Tuckerman, and M. C. Hamilton,
IEEE Trans. Appl. Supercond. \textbf{27}, 1700405 (2017).
%High-Quality Factor Superconducting Flexible Resonators Embedded in Thin-Film Polyimide HD-4110

\bibitem{Rausch2018} D. S. Rausch, M. Thiemann, M. Dressel, D. Bothner, D. Koelle, R. Kleiner, and M. Scheffler, 
J. Phys. D \textbf{51}, 465301 (2018).
%Superconducting coplanar microwave resonators with operating frequencies up to 50 GHz

\bibitem{Wenner2011} J. Wenner, M. Neeley, R. C. Bialczak, M. Lenander, E. Lucero, A. D. O’Connell, D. Sank, H. Wang, M. Weides, A. N. Cleland, and J. M. Martinis,
Supercond. Sci. Technol. \textbf{24}, 065001 (2011).
%Wirebond crosstalk and cavity modes inlarge chip mounts for superconducting qubits


\bibitem{McConkey2018} T. G. McConkey, J. H. B\'ejanin, C. T. Earnest, C. R. H. McRae, Z. Pagel, J. R. Rinehart, and M. Mariantoni,
Quantum Sci. Technol. \textbf{3}, 034004 (2018).
%Mitigating leakage errors due to cavity modes in a superconducting quantum computer

\bibitem{Viana1994} R. Viana, P. Lunkenheimer, J. Hemberger, R. B\"ohmer, and A. Loidl,  
Phys. Rev. B \textbf{50}, 601 (1994).
%Dielectric spectroscopy in SrTi03

\bibitem{Sluchanko2000} N. E. Sluchanko, V. V. \.{G}lushkov, B. P. Gorshunov, S. V. Demishev, M. V. Kondrin, A. A. Pronin, A. A. Volkov, A. K. Savchenko, G. Gr\"uner, Y. Bruynseraede, V. V. Moshchalkov, and S. Kunii,
Phys. Rev. B \textbf{61}, 9906 (2000).
%Intragap states in SmB6

\bibitem{Tran2002}P. Tran, S. Donovan, and G. Gr\"uner,
Phys. Rev. B \textbf{65}, 205102 (2002).
%Charge excitation spectrum in UPt$_3$

\bibitem{Turner2003} P.\,J. Turner, R. Harris, S. Kamal, M.\,E. Hayden, D.\,M. Broun, D.\,C. Morgan, A. Hosseini, P. Dosanjh, G.\,K. Mullins, J.\,S. Preston, R. Liang, D.\,A. Bonn, and W.\,N. Hardy,
Phys. Rev. Lett. \textbf{90}, 237005 (2003).
%Observation of Weak-Limit Quasiparticle Scattering via Broadband Microwave Spectroscopy of a d-Wave Superconductor

\bibitem{Scheffler2005c} M. Scheffler, M. Dressel, M. Jourdan, and H. Adrian,
Nature \textbf{438}, 1135 (2005).
%Extremely slow Drude relaxation of correlated electrons


\bibitem{Wendel2020} L. Wendel, V. T. Engl, G. Untereiner, N. G. Ebensperger, M. Dressel, A. Farag, M. Ubl, H. Giessen, and M. Scheffler,
Rev. Sci. Instrum. \textbf{91}, 054702 (2020).
%Microwave probing of bulk dielectrics using superconducting coplanar resonators in distant-flip-chip geometry

\bibitem{Klein1995} \MS{N. Klein, C. Zuccaro, U. D\"ahne, H. Schulz, N. Tellmann, R. Kutzner, A. G. Zaitsev, and R. W\"ordenweber,
J. Appl. Phys. \textbf{78}, 6683 (1995).}

\bibitem{Zuccaro1997} \MS{C. Zuccaro, I. Ghosh, K. Urban, N. Klein, S. Penn, and N. McN. Alford,
IEEE Trans. Appl. Supercond. \textbf{7}, 3715 (1997).}

\bibitem{Tobar1998} \MS{M. E. Tobar, J. Krupka, E. N. Ivanov, and R. A. Woode,
J. Appl. Phys. \textbf{83}, 1604 (1998).}

\bibitem{Pronin1998} \MS{A. V. Pronin, M. Dressel, A. Pimenov, A. Loidl, I. V. Roshchin, and L. H. Greene,
Phys. Rev. B \textbf{57}, 14416 (1998).}
%Direct observation of the superconducting energy gap developing in the conductivity spectra of niobium

\bibitem{Thiemann_phd} M.Thiemann, \textit{Microwave investigations on superconducting Nb-doped SrTiO$_3$} (Doctor
thesis, University of Stuttgart, 2018).

\bibitem{Sabinsky1962} E. S. Sabinsky and H. J. Gerritsen, J. Appl. Phys. \textbf{33}, 1450 (1962).
%TiO2 temperature dependence down too 4.2K

\bibitem{Hering2007} \MS{M. Hering, M. Scheffler, M. Dressel, and H. v. L\"{o}hneysen,
Phys. Rev. B \textbf{75}, 205203 (2007).}
%Signature of electronic correlations in the optical properties of the doped semiconductor Si:P

\bibitem{Pompeo2007} \MS{N. Pompeo, L. Muzzi, V. Galluzzi, R. Marcon, and E. Silva,
Supercond. Sci. Technol. \textbf{20} 1002 (2007).}
%Measurements and removal of substrate effects on the microwave surface impedance of YBCO films on SrTiO3

\bibitem{Geiger2012} \MS{D. Geiger, M. Scheffler, M. Dressel, M. Schneider, and P. Gegenwart,
J. Phys.: Conf. Ser. \textbf{391}, 012091 (2012).}
%Broadband microwave study of SrRuO3 and CaRuO3 thin films


\bibitem{Pracht2013} \MS{U. S. Pracht, E. Heintze, C. Clauss, D. Hafner, R. Bek, D. Werner, S. Gelhorn, M. Scheffler, M. Dressel, D. Sherman, B. Gorshunov, K. S. Il’in, D. Henrich, and M. Siegel
IEEE Trans. THz Sci. Technol. \textbf{3}, 269 (2013).}
%Electrodynamics of the Superconducting State in Ultra-Thin Films at THz Frequencies

\bibitem{Steinberg2008} \MS{K. Steinberg, M. Scheffler, and M. Dressel,
Phys. Rev. B \textbf{77}, 214517 (2008).}
%Quasiparticle response of superconducting aluminum to electromagnetic radiation


\bibitem{Bothner2012} D. Bothner, T. Gaber, M. Kemmler, D. Koelle, R. Kleiner, S. W\"{u}nsch, and M. Siegel,
Phys. Rev. B \textbf{86}, 014517 (2012).
%Magnetic hysteresis effects in superconducting coplanar microwave resonators

\bibitem{Halbritter2005} \MS{J. Halbritter,
J. Appl. Phys. \textbf{97}, 083904 (2005).}
%Transport in superconducting niobium films for radio frequency applications

\bibitem{Karasik1970} \MS{V. R. Karasik and I. Yu. Shebalin,
Sov. Phys. JETP \textbf{30}, 1068 (1970).}
%Supeconducting propeties of pure niobium

\bibitem{DasGupta1976} \MS{A. Das Gupta, W. Gey, J. Halbritter, H. K\"upfer, and J. A. Yasaitis,
J. Appl. Phys. \textbf{47}, 2146 (1976).}
%Inhomogeneities in superconducting niobium surfaces

\bibitem{Chin1992} \MS{C. C. Chin, D. E. Oates, G. Dresselhaus, and M. S. Dresselhaus,
Phys. Rev. B \textbf{45}, 4788 (1992).}
%Nonlinear electrodynamics of superconducting NbN and Nb thin films at microwave frequencies

\bibitem{Cohen2002} \MS{L. F. Cohen, A. L. Cowie, A. Purnell, N. A. Lindop, S. Thiess, and J. C. Gallop,
Supercond. Sci. Technol. \textbf{15}, 559 (2002).}
%Thermally induced nonlinear behaviour of HTS films at high microwave power

\bibitem{Abdo2006} \MS{B. Abdo, E. Segev, O. Shtempluck, and E. Buks,
Phys. Rev. B \textbf{73}, 134513 (2006).}
%Nonlinear dynamics in the resonance line shape of NbN superconducting resonators

\bibitem{deVisser2010} \MS{P. J. de Visser, S. Withington, and D. J. Goldie,
J. Appl. Phys. \textbf{108}, 114504 (2010).}
%Readout-power heating and hysteretic switching between thermal quasiparticle states in kinetic inductance detectors 

\bibitem{Liu2017} \MS{X. Liu, W. Guo, Y. Wang, M. Dai, L. F. Wei, B. Dober, C. M. McKenney, G. C. Hilton, J. Hubmayr, J. E. Austermann, J. N. Ullom, J. Gao, and M. R. Vissers,
Appl. Phys. Lett. \textbf{111}, 252601 (2017).} 
%Superconducting micro-resonator arrays with ideal frequency spacing

\bibitem{Samkharadze2016} N. Samkharadze, A. Bruno, P. Scarlino, G. Zheng, D. P. DiVincenzo, L. DiCarlo, and L. M. K. Vandersypen,
Phys. Rev. Applied \textbf{5}, 044004 (2016).
%High-Kinetic-Inductance Superconducting Nanowire Resonators for Circuit QED in a Magnetic Field

\bibitem{JavaheriRahim2016} \MS{M. Javaheri Rahim, T. Lehleiter, D. Bothner, C. Krellner, D. Koelle, R. Kleiner, M. Dressel, and M. Scheffler,
J. Phys. D: Appl. Phys. 49, 395501 (2016).}
%Metallic coplanar resonators optimized for low-temperature measurements


%\bibitem{CommentMarc1} \MS{Are the references below still used?}

%\bibitem{Bothner2011} D. Bothner, T. Gaber, M. Kemmler, D. Koelle, and R. Kleiner,
%Appl. Phys. Lett. \textbf{98}, 102504 (2011).
%Improving the performance of superconducting microwave resonators in magnetic fields

%\bibitem{Bothner2012a} D. Bothner, C. Clauss, E. Koroknay, M. Kemmler, T. Gaber, M. Jetter, M. Scheffler, P. Michler, M. Dressel, D. Koelle, and R. Kleiner,
%Appl. Phys. Lett. \textbf{100}, 012601 (2012).
%Reducing vortex losses in superconducting microwave resonators with microsphere patterned antidot arrays

%\bibitem{Kroll2019} J. G. Kroll, F. Borsoi, K. L. van der Enden, W. Uilhoorn, D. de Jong, M. Quintero-P\'erez, D. J. van Woerkom, A. Bruno, S. R. Plissard, D. Car, E. P. A. M. Bakkers, M. C. Cassidy, and L. P. Kouwenhoven,
%Phys. Rev. Appl. \textbf{11}, 064053 (2019).
%Magnetic-Magnetic-Field-Resilient Superconducting Coplanar-Waveguide Resonators for Hybrid Circuit Quantum Electrodynamics Experiments

%\bibitem{deGraaf2012} S. E. de Graaf, A. V. Danilov, A. Adamyan, T. Bauch, and S. E. Kubatkin,
%J. Appl. Phys. \textbf{112}, 123905 (2012).
%Magnetic field resilient superconducting fractal resonators for coupling to free spins




\end{thebibliography}

\end{document}