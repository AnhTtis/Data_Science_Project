% !TEX root =  main.tex
%%%%%%%%%%%%%%%%%%%%%%%%%%%%%%%%%%%%%%%%%%%%%%%%%%%%%%%%%%
%%%%%%%%%%%%%%%%%%%%%%%%%%%%%%%%%%%%%%%%%%%%%%%%%%%%%%%%%%
%%%%%%%%%%%%%%%% input performance table
% !TEX root =  main.tex
%%%%%%%%%%%%%%%%%%%%%%%%%%%%%%%%%%%%%%%%%%%%%%%%%%%%%%%%%%


\begin{figure}[!t]
\renewcommand{\arraystretch}{0.9}
\centering
	%%%%%%%%%%%%%%%%%
	\scalebox{0.89}{\begin{tabular}{l|c|c|rr}
		    \multicolumn{1}{c|}{Name, source for ${T^{\text{in}}}$} &
		    \multicolumn{1}{c|}{
		    %$\mathbf{T^{\text{in}}}$\textbf{: } 
		  %  \textbf{ Source},
		    ${C^{\text{in,}\star}_\text{size}}$, ${S^{\text{in,}\star}}$ for ${T^{\text{in}}}$}
		    &
		  %  \multicolumn{1}{c}{$\mathbf{S^{\text{in,}\star}}$} &
		  %  \multicolumn{1}{c}{$\mathbf{C^{\text{in,}\star}_\text{size}}$} &
		  %  \multicolumn{1}{c|}{\textbf{Source}} &
		    \multicolumn{1}{c|}{${S^\text{quiz}}$ \scriptsize$\in \substructures({S^{\text{in,}\star}})$\normalsize} &
		 \multicolumn{1}{c}{${\#C^\text{quiz}}$} & \multicolumn{1}{c}{${\#T^\text{quiz}}$}
			 \\
			\toprule
			\multirow{3}{*}{
			\shortstack{
			T-1
			\\
		    \footnotesize{\textbf{HOC:Maze08}~\cite{hourofcode_maze}}
			}
			}
			& 
			\multirow{3}{*}{
			\shortstack[c]{
% 			Source = \footnotesize{HOC:Maze08}
% 		    \quad \quad \quad \quad \quad \ \hspace{0.25em}
% 			$C^{\text{in,}\star}_\text{size} = 6$ 
			$6$ 
			\\
% 			$S^{\text{in,}\star} = $ \text{\footnotesize{\{\DSLRun\{\DSLRepeat; \DSLRepeat{}\}\}}}
		   \text{\scriptsize{\{\DSLRun\{\DSLRepeat; \DSLRepeat{}\}\}}}
			}
			}
			& 
			\text{\scriptsize{\{\DSLRun{}}\}} 
			& 
           	$22$ & $220$
			\\
			&&
			\text{\scriptsize{\{\DSLRun\{\DSLRepeat{}}\}\}} 
			& $34$ & $340$
			\\
			&&
			\ $S^{\text{in,}\star}$
			& $179$ & $1790$
			\\
			\hline
			%%%%%%%%
			\multirow{3}{*}{\shortstack{
			T-2
			\\
			\footnotesize{\textbf{HOC:Maze16}~\cite{hourofcode_maze}}
			}}&
			\multirow{3}{*}{\shortstack[c]{
% 			Source = \text{\footnotesize{HOC:Maze16}}
% 			\quad \quad \quad \quad \quad \ \hspace{0.25em}
% 			$C^{\text{in,}\star}_\text{size} = 5$ 
			$5$ 
			\\
% 			$S^{\text{in,}\star} = $\text{\footnotesize{\{\DSLRun\{\DSLRUntil\{\DSLIf{}\}\}\}}}
		\text{\scriptsize{\{\DSLRun\{\DSLRUntil\{\DSLIf{}\}\}\}}}
			}
			}
			&
			\text{\scriptsize{\{\DSLRun{}}\}} 
			& 
			$10$ & $100$
			\\
            &&
            \text{\scriptsize{\{\DSLRun\{\DSLRUntil{}\}\}}} 
			& $6$ & $60$
			\\
			&& 
			\ $S^{\text{in,}\star}$
			& $19$ & $190$
			\\
			\hline
			%%%%%%%%%
			\multirow{3}{*}{\shortstack{
			T-3
			\\
			\footnotesize{\textbf{HOC:Maze18}~\cite{hourofcode_maze}}
			}}&
			\multirow{3}{*}{\shortstack[c]{
% 			Source = \text{\footnotesize{HOC:Maze18}}
% 			\quad \quad \quad \quad \quad \ \ \
		    $5$
			\\
		 \text{\scriptsize{\{\DSLRun\{\DSLRUntil\{\DSLIfElse{}\}\}}}
			}
			}
			& 
			\text{\scriptsize{\{\DSLRun{}\}}} 
			& 
			$10$ & $100$
			\\
			&&
			\text{\scriptsize{\{\DSLRun\{\DSLRUntil{}\}\}}} 
			& $6$ & $60$
			\\
			&&
			\ $S^{\text{in,}\star}$
			& $9$ & $90$
			\\
			\hline
			%%%%%%%%
			\multirow{4}{*}{\shortstack{
			T-4
			\\
		    \footnotesize{\textbf{HOC:Maze20}~\cite{hourofcode_maze}}
			}}
			&
			\multirow{4}{*}{
			\shortstack[c]{
		  %  Source = \footnotesize{HOC:Maze20}
		  %  \quad \quad \quad \quad \quad \ \
			$7$ 
			\\
			\scriptsize{\{\DSLRun\{\DSLRUntil\{\DSLIfElse\{\{\};\{\DSLIfElse{}\}\}\}\}\}}
			}
			}
			&
			\text{\scriptsize{\{\DSLRun{}}\}} 
			& $10$ & $100$
			\\
			&&
			\text{\scriptsize{\{\DSLRun\{\DSLRUntil{}\}\}}} 
			& $6$ & $60$
			\\
			&&
			\text{\scriptsize{\{\DSLRun\{\DSLRUntil\{\DSLIfElse{}\}\}\}}} 
			& $9$ & $90$
			\\
			&& 
			\ $S^{\text{in,}\star}$
			& $10$ & $100$
			\\
			\hline
			%%%%%%%
			\multirow{3}{*}{\shortstack{
			T-5
			\\
			\footnotesize{\textbf{Karel:Opposite}~\cite{intro_to_karel_codehs}}
			}}
			 &
			\multirow{3}{*}{
			\shortstack[c]{
% 			Source = \footnotesize{Karel:BallSpot}
% 			\quad \quad \quad \quad \quad
			$6$
			\\
			\text{\scriptsize{\{\DSLRun\{\DSLRepeat\{\DSLIfElse{}\}\}\}}}
			}
			}
			&
			\text{\scriptsize{\{\DSLRun{}}\}} 
			& $73$ & $730$
			\\
			&&
			\text{\scriptsize{\{\DSLRun\{\DSLRepeat{}\}\}}} 
			& $118$ & $1180$
			\\
			&& 
			\ $S^{\text{in,}\star}$
			& $343$ & $3430$
			\\
            \hline
			%%%%%%%%%%%%%
			\multirow{2}{*}{\shortstack{
			T-6
			\\
			\footnotesize{\textbf{Karel:Diagonal}~\cite{intro_to_karel_codehs}}
			}}
			&
			\multirow{2}{*}{
			\shortstack[c]{
% 			Source = \text{\footnotesize{Karel:Diagonal}}
% 			\quad \quad \quad \quad \quad
			$8$
			\\
			\text{\scriptsize{\{\DSLRun\{\DSLWhile{}\}\}}}
			}
			}
			&
			\text{\scriptsize{\{\DSLRun{}}\}} 
			& $447$ & $4470$
			\\
			&&
			\ $S^{\text{in,}\star}$
			& $579$ & $5790$
			\\
		\bottomrule
   \end{tabular}
   }
	\caption{\looseness-1\algmultihop~applied to six HOC and Karel reference tasks; see Section~\ref{sec:evaluation} for details. For brevity, sketches have been abbreviated, e.g., \DSLRepeatUntil{}(\DSLBoolGoal) as \DSLRUntil.
	%
    }
\label{fig:experiments.analysis}
%\vspace{-3mm}
\end{figure}
%reference

%%%%%%%%%%%%%%%%%%%%%%%%%%%%%%%%%%%%%%%%
\section{\algmultihop~on Real-World Tasks}\label{sec:evaluation}
\looseness-1In this section, we present the performance of \algmultihop~on six reference tasks taken from real-world block-based programming platforms: HOC~\cite{hourofcode_maze} and Karel~\cite{intro_to_karel_codehs}. The set of these tasks along with their sources are mentioned in Fig.~\ref{fig:experiments.analysis}. These tasks differ in complexity, measured in terms of the programming constructs of their solution code as illustrated by the diversity of their respective solution sketches ${S^{\text{in,}\star}}$. For the exhaustive set of substructures of $S^{\text{in,}\star}$, 
% of these tasks
Fig.~\ref{fig:experiments.analysis} lists the total number of pop quizzes, in the form of unique task-code pairs ($T^\text{quiz}, C^\text{quiz}$), generated by our algorithm.
As can be seen in the figure, our algorithm generates $50$ to $1000$s of pop quizzes for each substructure. For any potential student attempt on these tasks, Stage 1 of \algmultihop~ would generate one of these task-specific substructures by design -- hence, for every attempt we can present several unique yet adaptive pop quizzes to the student. Note that, our algorithm generates higher number of tasks than codes for each substructure. This is because the task synthesis methodology used in Stage 2(ii) can generate more than one task for a single code in Stage 2(ii) of Fig.~\ref{fig:pipeline.abstract}. In particular, for each new code, we obtain $10$ diverse tasks. For instance, Fig.~\ref{fig:intro} and Fig.~\ref{fig:karel.illustration} illustrate pop quizzes generated by \algmultihop{} for the specific student attempts on tasks T-4 and T-5, respectively.
%































