% !TEX root =  main.tex
%%%%%%%%%%%%%%%%%%%%%%%%%%%%%%%%%%%%%%%%%%%%%%%%%%%%%%%%%%
%%%%%%%%%%%%%%%%%%%%%%%%%%%%%%%%%%%%%%%%%%%%%%%%%%%%%%%%%% V4
\begin{abstract}
\looseness-1Block-based programming environments are increasingly used to introduce computing concepts to beginners. However, novice students often struggle in these environments, given the conceptual and open-ended nature of programming tasks. To effectively support a student struggling to solve a given task, it is important to provide adaptive scaffolding that guides the student towards a solution.
%
We introduce a scaffolding framework based on pop quizzes presented as multi-choice programming tasks. To automatically generate these pop quizzes, we propose a novel algorithm, \algmultihop. More formally, given a reference task with a solution code and the student's current attempt, \algmultihop~synthesizes new tasks for pop quizzes with the following features: (a) \ind~(i.e., individualized to the student's current attempt), (b) \interp~(i.e., easy to comprehend and solve), and (c) \conceal~(i.e., do not reveal the solution code). Our algorithm synthesizes these tasks using techniques based on symbolic reasoning and graph-based code representations.
%
We show that our algorithm can generate hundreds of pop quizzes for different student attempts on reference tasks from \emph{Hour of Code: Maze Challenge}~\cite{hourofcode_maze} and \emph{Karel}~\cite{intro_to_karel_codehs}. We assess the quality of these pop quizzes through expert ratings using an evaluation rubric. Further, we have built an online platform for practicing block-based programming tasks empowered via pop quiz based feedback, and report results from an initial user study.
%
% \keywords{Block-based visual programming \and Scaffolding \and Task synthesis}
\end{abstract}

%%%%%%%%%%%%%%%%%%%%%%%%%%%%%%%%%%%%%%%%%%%%%%%%%%%%%%%%%%
%%%%%%%%%%%%%%%%%%%%%%%%%%%%%%%%%%%%%%%%%%%%%%%%%%%%%%%%%%
