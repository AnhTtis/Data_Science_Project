% !TEX root =  main.tex
%%%%%%%%%%%%%%%%%%%%%%%%%%%%%%%%%%%%%%%%%%%%%%%%%%%%%%%%%%
%%%%%%%%%%%%%%%%%%%%%%%%%%%%%%%%%%%%%%%%%%%%%%%%%%%%%%%%%%
% 


\section{Expert Study via Multi-Dimensional Rubric
}\label{sec:surveystudy}


\looseness-1In this section, we evaluate \algmultihop~w.r.t. the desired features specified in the objective, i.e., \ind, \interp, and \conceal~(see Section~\ref{sec:notation.setup}). In particular, we seek to compare \algmultihop~with its variants resulting from different design choices in Section~\ref{sec:model}. To this end, we conduct an expert study via a multi-dimensional rubric.


% \vspace{1mm}
\looseness-1\textbf{Variants of \algmultihop~algorithm.} We compare the performance of \algmultihop{} with the following variants: \algsame, \alghintpolicystruct, and \algmincode{}. \algsame~and \alghintpolicystruct~differ from \algmultihop~only in the \hintpolicy{} routine used in Stage 1  of Fig.~\ref{fig:pipeline.abstract} when generating $S^\text{quiz}$. In particular, Stage 1 of \algsame~always returns the sketch of the solution code, i.e., $S^\text{quiz} := S^{\text{in,}\star}$; Stage 1 of \alghintpolicystruct~returns a sketch directly from the $1$-hop neighborhood of $S^\text{in,stu}$, i.e., $S^\text{quiz} \in \mathcal{N}_\sketchspace(S^\text{in,stu},1)$. The third baseline, \algmincode, differs from \algmultihop~only in Stage 2(i) of Fig.~\ref{fig:pipeline.abstract} when generating $C^\text{quiz}$ from $S^\text{quiz}$. In particular, Stage 2(i) of \algmincode~generates $C^\text{quiz}$ as a direct reduction of the solution code w.r.t. the sketch obtained in Stage 1, i.e., $C^\text{quiz} \in \reducedcodes(\solutioncode~|~S^\text{quiz})$.


% \vspace{1mm}
\looseness-1\textbf{Simulated student attempts.} For this expert evaluation, we simulated unsuccessful student attempts as seen in block-based programming domains~\cite{DBLP:conf/lats/PiechSHG15}. In particular, for each reference task, we manually created four student attempts as follows: (a) Stu-A: $C^\text{in,stu}$ uses only action blocks, i.e., (\DSLMove, \DSLTurnLeft, \DSLTurnRight, \DSLPickMarker, \DSLPutMarker); (b) Stu-B: $C^\text{in,stu}$ uses a subset of programming constructs in \solutioncode; (c) Stu-C: $C^\text{in,stu}$ is structurally the same as \solutioncode, i.e., $S^\text{in,stu}= S^{\text{in,}\star}$; (d) Stu-D: $C^\text{in,stu}$ has a structure more complex than \solutioncode. These four types of attempts exhaustively cover all the scenarios that an algorithm might encounter when deployed (see Section~\ref{sec:userstudy}).


%%%%%%%%%% figure for Karel-task-6 here
% !TEX root =  main.tex
%%%%%%%%%%%%%%%%%%%%%%%%%%%%%%%%%%%%%%%%%%%%%%%%%%%%%%%%%%
\begin{figure*}[!t]
\centering
% \raggedleft
\begin{tikzpicture}[
        box/.style={rectangle,draw=black, minimum size=0.1cm},scale=1, every node/.style={transform shape}
        ]
       
        \foreach \x in {-7.6,-7.4,...,-5.6}{
            \foreach \y in {1.9,2.1,...,3.9}
                \node[box, fill=white] at (\x,\y){};
        }



        
\foreach \x in {-5.2,-5,...,-3.4}{
            \foreach \y in {1.9,2.1,...,3.9}
                \node[box, fill=white] at (\x,\y){};
        }

%% add the walls and the markers
\node[draw, fill=yellow, diamond, inner sep=1.25pt,minimum size=0.4pt] at(-7.2, 2.1){};
\node[draw, fill=yellow, diamond, inner sep=1.25pt,minimum size=0.4pt] at(-6.8, 2.1){};
\node[draw, fill=yellow, diamond, inner sep=1.25pt,minimum size=0.4pt] at(-6.6, 2.1){};
\node[draw, fill=yellow, diamond, inner sep=1.25pt,minimum size=0.4pt] at(-5, 2.1){};
\node[draw, fill=yellow, diamond, inner sep=1.25pt,minimum size=0.4pt] at(-4.6, 2.1){};
\node[draw, fill=yellow, diamond, inner sep=1.25pt,minimum size=0.4pt] at(-4, 2.1){};
\node[draw, fill=yellow, diamond, inner sep=1.25pt,minimum size=0.4pt] at(-3.8, 2.1){};
\node[draw, fill=yellow, diamond, inner sep=1.25pt,minimum size=0.4pt] at(-3.6, 2.1){};

% add the karel agent
\node[draw, fill=blue!50, dart, rotate=0 ,inner sep=0.01pt, minimum size=4pt] at (-7.4, 2.1){};
\node[draw, fill=blue!50, dart, rotate=0 ,inner sep=0.01pt, minimum size=4pt] at (-3.4, 2.1){};

% bounding boxes in the task-grids
\draw[draw=black, thick] (-7.7,1.8) rectangle (-5.7,3.8);
\draw[draw=black, thick] (-5.3,1.8) rectangle (-3.3,3.8);
\draw[-stealth, line width=0.5mm] (-5.7, 2.8) -- (-5.3,2.8);
        %%%% student code
        \node[text width=0.5cm, anchor=west, right] at (-3.25, 2.85)
    {\begin{boxcode}{2.98cm}{0.70}{1.0}
				\textcode{def }\DSLRun\textcode{()\{}\\
                \quad \DSLPickMarker\\
                \quad \DSLMove\\
                \quad \DSLPickMarker\\
                \quad \ldots\\
                \quad \text{several more} \\ \quad \text{action blocks}\\
				\textcode{\}}
				\\
				%\vspace{1em}
			\end{boxcode}};
	%%%%% bounding box around the task-code pair
% 	\draw[draw=black, thick] (-7.35,0.5) rectangle (-2.1, 3.8);
	%%%%%%%%% solution code
	\node[text width=0.5cm, anchor=east, left] at (-10.35, 2.85)
    {\begin{boxcode}{3.68cm}{0.68}{0.8}
				\textcode{def }\DSLRun\textcode{()\{}\\
				\quad \DSLRepeat\textcode{(8}\textcode{)\{}\\
                \quad \quad
              \DSLIf\textcode{(}\DSLBoolNoMarker\textcode{)\{}\\
				\quad \quad \quad \DSLPutMarker\\
				\quad \quad \textcode{\}}\\
                \quad \quad \DSLElse\textcode{}\textcode{\{}\\
                \quad \quad \quad \DSLPickMarker\\
                % \quad \quad \quad	\framebox[8.0\width]{?}\\
                \quad \quad \textcode{\}
                	}\\
                \quad \quad \DSLMove\\
				\quad \textcode{\}}\\
				\textcode{\}}
            %\vspace{-6mm}				
			\end{boxcode}
			%\vspace{-6mm}
			};
	%%%%% bounding box around the solution-code pair
	\draw[draw=black, thick, pattern=north east lines, pattern color=gray!30] (-10.9,1.1) rectangle (-7.9, 4.4);
	%%%% captions for the task-codes
	\node[text width=0.5cm] at (-5.4,4.2) {$T^\text{in}$};
	\node[text width=0.5cm] at (-6.8,4) {\scriptsize\textsc{{Pregrid}}};
	\node[text width=0.5cm] at (-4.5,4) {\scriptsize\textsc{{Postgrid}}};
	\node[text width=0.5cm] at (-1.96,4.2) {\studentcode};
	\node[text width=0.5cm] at (-9.4,4.2) {\solutioncode};
	%%%%%%%%%%%% Intervention boxes
    \draw[draw=black, thick] (-7.8,1.1) rectangle (-0.2, 4.4);
	\draw[draw=black, thick] (-10.9,-2) rectangle (-0.2, 1);
	%%%%%% T-out: intervention task
[
        box/.style={rectangle,draw=black, minimum size=0.2cm},
        ]
        
        \foreach \x in {-10.7,-10.5,...,-8.9}{
            \foreach \y in {-1.4,-1.2,...,0.4}
                \node[box, fill=white] at (\x,\y){};
        }


\foreach \x in {-8.3,-8.1,...,-6.5}{
             \foreach \y in {-1.4,-1.2,...,0.4}
                \node[box, fill=white] at (\x,\y){};
        }


% %% add the walls and the markers
% \node[draw, fill=yellow, diamond, inner sep=1.5pt,minimum size=0.5pt] at(-10.5, 0.2){};
% \node[draw, fill=yellow, diamond, inner sep=1.5pt,minimum size=0.5pt] at(-10.3, 0.2){};
\node[draw, fill=yellow, diamond, inner sep=1.25pt,minimum size=0.5pt] at(-10.1, 0.2){};
\node[draw, fill=yellow, diamond, inner sep=1.25pt,minimum size=0.5pt] at(-9.9, 0.2){};
\node[draw, fill=yellow, diamond, inner sep=1.25pt,minimum size=0.5pt] at(-9.7, 0.2){};
\node[draw, fill=yellow, diamond, inner sep=1.25pt,minimum size=0.5pt] at(-9.5, 0.2){};
\node[draw, fill=yellow, diamond, inner sep=1.25pt,minimum size=0.5pt] at(-9.3, 0.2){};
\node[draw, fill=yellow, diamond, inner sep=1.25pt,minimum size=0.5pt] at(-9.1, 0.2){};
% \node[draw, fill=yellow, diamond, inner sep=1.5pt,minimum size=0.5pt] at(-8.3, 0.2){};
\node[draw, fill=yellow, diamond, inner sep=1.25pt,minimum size=0.5pt] at(-7.9, 0.2){};

% % add the karel agent
\node[draw, fill=blue!50, dart, rotate=-180 ,inner sep=0.01pt, minimum size=4pt] at (-9.1, 0.2){};
\node[draw, fill=blue!50, dart, rotate=90 ,inner sep=0.01pt, minimum size=4pt] at (-7.9, 0.2){};



% bounding boxes in the task-grids
\draw[draw=black, thick] (-10.8,-1.5) rectangle (-8.8,0.5);
\draw[draw=black, thick] (-8.4,-1.5) rectangle (-6.4,0.5);
\draw[-stealth, line width=0.5mm] (-8.8, -0.5) -- (-8.4,-0.5);
 
    %%%%%%%% C-out: intervention code
    \node[text width=0.5cm, anchor=west, right] at (-6.45, -0.4)
    {\begin{boxcode}{3.58cm}{0.70}{1.0}
				\textcode{def }\DSLRun\textcode{()\{}\\
				\quad \DSLRepeat\textcode{(6}\textcode{)\{}\\
                \quad \quad \DSLPickMarker\\
                %  \quad \quad \DSLMove\\
                \quad \quad \framebox[8.0\width]{?}\\
				\quad \textcode{\}}\\
				\quad \DSLPutMarker\\
				\quad \DSLTurnRight\\
				\textcode{\}}
			\end{boxcode}};
	%%%%%%%%% Multiple Choice Question
	 \node[text width=0.5cm, anchor=west, right] at (-3.45, -0.4)
	 {\begin{boxcode}{3.92cm}{0.65}{1}
				\textcode{Q.} \text{Fill in the blank from: }\\
				\quad \quad \tikz\draw[black,fill=none] (0,0) circle (.5ex); \DSLMove\\
				\quad \quad \tikz\draw[black,fill=none] (0,0) circle (.5ex); \DSLTurnLeft\\
				\quad \quad \tikz\draw[black,fill=none] (0,0) circle (.5ex); \DSLTurnRight\\
				\quad \quad \tikz\draw[black,fill=none] (0,0) circle (.5ex); \DSLPickMarker\\
				\quad \quad \tikz\draw[black,fill=none] (0,0) circle (.5ex); \DSLPutMarker
			\\
			\\
			\\
			\end{boxcode}
	 };
	 %%%% captions for the task-codes
\node[text width=0.5cm] at (-8.5,0.8) {$T^\text{quiz}$};
\node[text width=0.5cm] at (-10,0.7) {\scriptsize\textsc{{Pregrid}}};
	\node[text width=0.5cm] at (-7.6,0.7) {\scriptsize\textsc{{Postgrid}}};
\node[text width=2.6cm] at (-4.75,0.75) {$C^\text{quiz}\text{ with }1\text{ blank}$};
\node[text width=1cm] at (-1.6,0.75) {Quiz};

\tikzstyle{doublearr}=[latex-latex, black, line width=1.5pt]
\draw [doublearr, bend left]    (-0.2, 2.5) to (-0.2,-0.7);

    %%%%%%%%%%%\draw (-7.5,2.7) rectangle (-6.5,3.2) 
    \node[text width=1cm] at (-12,2.5){\textbf{Task}};
    %%%%%%%%%%%%%%\draw (1.9,2.7) rectangle (2.9,3.2) 
    \node[text width=1.7cm] at (-12,-0.7){\textbf{Pop Quiz}};
    \end{tikzpicture}
    %\setlength{\belowcaptionskip}{-16pt} 
    %\vspace{-5mm}
\caption{Analogous to Fig.~\ref{fig:intro}, here we illustrate our framework on a Karel task, T-5 (see Fig.~\ref{fig:experiments.analysis}). Karel tasks~\cite{pattis1981karel} comprise of a pair of visual grids, (\textsc{Pregrid}, \textsc{Postgrid}), and the objective is to write code that, when executed, transforms \textsc{Pregrid} to \textsc{Postgrid}. 
% Compared to the HOC tasks, these tasks additionally have \DSLPickMarker~and~\DSLPutMarker actions.
}
\label{fig:karel.illustration}
%\vspace{-2.8mm}
\end{figure*}













% %%%%%%%%%%%%%%%%%%%%%%%%%%%%%%%% OLD
% \begin{figure*}[!t]
% \begin{subfigure}{1\textwidth}% for solution code/task
% \begin{minipage}{0.5\textwidth}
% \centering{
% \begin{tikzpicture}[
%         box/.style={rectangle,draw=black, minimum size=0.2cm},
%         ]
% \draw[draw=black, thick] (0.0,1.35) rectangle (8, 4);% outer-box
% \foreach \x in {0.4,0.6,...,2.4}{
%             \foreach \y in {1.6,1.8,...,3.4}
%                 \node[box, fill=white] at (\x,\y){};
%         }
% \foreach \x in {2.8,3,...,4.6}{
%             \foreach \y in {1.6,1.8,...,3.4}
%                 \node[box, fill=white] at (\x,\y){};
%         }
% %% add the walls and the markers
% \node[draw, fill=yellow, diamond, inner sep=1.5pt,minimum size=0.5pt] at(0.8, 1.8){};
% \node[draw, fill=yellow, diamond, inner sep=1.5pt,minimum size=0.5pt] at(1.2, 1.8){};
% \node[draw, fill=yellow, diamond, inner sep=1.5pt,minimum size=0.5pt] at(1.4, 1.8){};
% \node[draw, fill=yellow, diamond, inner sep=1.5pt,minimum size=0.5pt] at(3, 1.8){};
% \node[draw, fill=yellow, diamond, inner sep=1.5pt,minimum size=0.5pt] at(3.4, 1.8){};
% \node[draw, fill=yellow, diamond, inner sep=1.5pt,minimum size=0.5pt] at(4, 1.8){};
% \node[draw, fill=yellow, diamond, inner sep=1.5pt,minimum size=0.5pt] at(4.2, 1.8){};
% \node[draw, fill=yellow, diamond, inner sep=1.5pt,minimum size=0.5pt] at(4.4, 1.8){};
% % add the karel agent
% \node[draw, fill=blue!50, dart, rotate=0 ,inner sep=0.01pt, minimum size=4pt] at (0.6, 1.8){};
% \node[draw, fill=blue!50, dart, rotate=0 ,inner sep=0.01pt, minimum size=4pt] at (4.6, 1.8){};

% % bounding boxes in the task-grids
% \draw[draw=black, thick] (0.3,1.5) rectangle (2.3,3.5);
% \draw[draw=black, thick] (2.7,1.5) rectangle (4.7,3.5);
% \draw[-stealth, line width=0.5mm] (2.3, 2.5) -- (2.7,2.5);
% % task title
% \node[text width=0.5cm] at (2.6,3.8) {$T^\text{in}$};
% % solution code
% \node[text width=0.5cm, anchor=east, left] at (5.4, 2.75)
%     {\begin{boxcode}{3.68cm}{0.65}{0.60}
% 				\textcode{def }\DSLRun\textcode{()\{}\\
% 				\quad \DSLRepeat\textcode{(8}\textcode{)\{}\\
%                 \quad \quad
%               \DSLIf\textcode{(}\DSLBoolNoMarker\textcode{)\{}\\
% 				\quad \quad \quad \DSLPutMarker\\
% 				\quad \quad \textcode{\}}\\
%                 \quad \quad \DSLElse\textcode{}\textcode{\{}\\
%                 	\quad \quad \quad \DSLPickMarker\\
%                 \quad \quad \textcode{\}}\\
%                 \quad \quad \DSLMove\\
% 				\quad \textcode{\}}\\
% 				\textcode{\}}
% 			\end{boxcode}};
% % code title
% \node[text width=0.5cm] at (6.2,3.8) {$C^{\text{in,}\star}$};
% \end{tikzpicture}
% }
% \vspace{-1.5mm}
% \subcaption{T-5: Reference task and solution code}
% \end{minipage}
% %%% example 2
% \begin{minipage}{0.5\textwidth}
% \centering{
% \begin{tikzpicture}
% \draw[draw=black, thick] (1.5,1.35) rectangle (5, 4);
% % add the options box
% \node[text width=0.5cm, anchor=west, right] at (1.45, 2.73)
% 	 {\begin{boxcode}{3.88cm}{0.70}{1.0}
% 				\quad \textcode{Q.} \text{Fill in the blank from: }\\
% 				\quad \quad \tikz\draw[black,fill=none] (0,0) circle (.5ex); \DSLMove\\
% 				\quad \quad \tikz\draw[black,fill=none] (0,0) circle (.5ex); \DSLTurnLeft\\
% 				\quad \quad \tikz\draw[black,fill=none] (0,0) circle (.5ex); \DSLTurnRight\\
% 				\quad \quad \tikz\draw[black,fill=none] (0,0) circle (.5ex); \DSLPickMarker\\
% 				\quad \quad \tikz\draw[black,fill=none] (0,0) circle (.5ex); \DSLPutMarker\\
% 			\end{boxcode}
% 	 };
% 	 \node[text width=1cm] at (3.44,3.7) {Quiz};
% \end{tikzpicture}
% }
% \vspace{-1.5mm}
% \subcaption{Multi-choice question}
% \end{minipage}
% \end{subfigure}% end of solution code and task
% \\
% \vspace{1mm}
% \begin{subfigure}{1\textwidth}% for Quiz pairs: size =0
% \begin{minipage}{0.5\textwidth}
% \centering{
% \begin{tikzpicture}[
%         box/.style={rectangle,draw=black, minimum size=0.2cm},
%         ]
% \draw[draw=black, thick] (0,1.4) rectangle (8, 4);
% \foreach \x in {0.4,0.6,...,2.4}{
%             \foreach \y in {1.6,1.8,...,3.4}
%                 \node[box, fill=white] at (\x,\y){};
%         }
% \foreach \x in {2.8,3,...,4.6}{
%             \foreach \y in {1.6,1.8,...,3.4}
%                 \node[box, fill=white] at (\x,\y){};
%         }
        
        
% %% add the walls and the markers
% \node[draw, fill=yellow, diamond, inner sep=1.5pt,minimum size=0.5pt] at(2, 1.8){};
% \node[draw, fill=yellow, diamond, inner sep=1.5pt,minimum size=0.5pt] at(2, 2){};
% \node[draw, fill=yellow, diamond, inner sep=1.5pt,minimum size=0.5pt] at(2, 2.2){};
% \node[draw, fill=yellow, diamond, inner sep=1.5pt,minimum size=0.5pt] at(2, 2.4){};
% \node[draw, fill=yellow, diamond, inner sep=1.5pt,minimum size=0.5pt] at(2, 2.6){};
% \node[draw, fill=yellow, diamond, inner sep=1.5pt,minimum size=0.5pt] at(2, 2.8){};
% \node[draw, fill=yellow, diamond, inner sep=1.5pt,minimum size=0.5pt] at(2, 3){};


% % add the karel agent
% \node[draw, fill=blue!50, dart, rotate=90 ,inner sep=0.01pt, minimum size=4pt] at (2, 1.8){};
% \node[draw, fill=blue!50, dart, rotate=90 ,inner sep=0.01pt, minimum size=4pt] at (4.4, 3.2){};
        
% % bounding boxes in the task-grids
% \draw[draw=black, thick] (0.3,1.5) rectangle (2.3,3.5);
% \draw[draw=black, thick] (2.7,1.5) rectangle (4.7,3.5);
% \draw[-stealth, line width=0.5mm] (2.3, 2.5) -- (2.7,2.5);
% % task title
% \node[text width=0.5cm] at (2.6,3.8) {$T^\text{quiz}$};
% % solution code
% \node[text width=0.5cm, anchor=east, left] at (5.4, 3)
% {\begin{boxcode}{3.68cm}{0.65}{0.88}
% 				\textcode{def }\DSLRun\textcode{()\{}\\
% 				\quad \DSLRepeat\textcode{(7}\textcode{)\{}\\
%                 \quad \quad \DSLPickMarker\\
%                 %  \quad \quad \DSLMove\\
%                 \quad \quad \framebox[8.0\width]{?}\\
% 				\quad \textcode{\}}\\
% 				\textcode{\}}
% 			\end{boxcode}};
% % code title
% \node[text width=0.5cm] at (6.2,3.8){$C^{\text{quiz}}$};
% \end{tikzpicture}
% }
% \vspace{-1.5mm}
% \subcaption{$S^\text{quiz} = \footnotesize{\{\DSLRun\{\DSLRepeat{}\}\}}$; a $C^\text{quiz}$ with size $=4$}
% \end{minipage}
% %%% example 2
% \begin{minipage}{0.5\textwidth}
% \centering{
% \begin{tikzpicture}[
%         box/.style={rectangle,draw=black, minimum size=0.2cm},
%         ]
% \draw[draw=black, thick] (0,1.4) rectangle (8, 4);
% %% task
% \foreach \x in {0.4,0.6,...,2.4}{
%             \foreach \y in {1.6,1.8,...,3.4}
%                 \node[box, fill=white] at (\x,\y){};
%         }
% \foreach \x in {2.8,3,...,4.6}{
%             \foreach \y in {1.6,1.8,...,3.4}
%                 \node[box, fill=white] at (\x,\y){};
%         }
% %% add the walls and the markers
% \node[draw, fill=yellow, diamond, inner sep=1.5pt,minimum size=0.5pt] at(0.6, 3.2){};
% \node[draw, fill=yellow, diamond, inner sep=1.5pt,minimum size=0.5pt] at(0.8, 3.2){};
% \node[draw, fill=yellow, diamond, inner sep=1.5pt,minimum size=0.5pt] at(1, 3.2){};
% \node[draw, fill=yellow, diamond, inner sep=1.5pt,minimum size=0.5pt] at(1.2, 3.2){};
% \node[draw, fill=yellow, diamond, inner sep=1.5pt,minimum size=0.5pt] at(1.4, 3.2){};
% \node[draw, fill=yellow, diamond, inner sep=1.5pt,minimum size=0.5pt] at(1.6, 3.2){};
% \node[draw, fill=yellow, diamond, inner sep=1.5pt,minimum size=0.5pt] at(1.8, 3.2){};
% \node[draw, fill=yellow, diamond, inner sep=1.5pt,minimum size=0.5pt] at(2, 3.2){};
% \node[draw, fill=yellow, diamond, inner sep=1.5pt,minimum size=0.5pt] at(2.8, 3.2){};


% % add the karel agent
% \node[draw, fill=blue!50, dart, rotate=-180 ,inner sep=0.01pt, minimum size=4pt] at (2, 3.2){};
% \node[draw, fill=blue!50, dart, rotate=90 ,inner sep=0.01pt, minimum size=4pt] at (2.8, 3.2){};
        
% % bounding boxes in the task-grids
% \draw[draw=black, thick] (0.3,1.5) rectangle (2.3,3.5);
% \draw[draw=black, thick] (2.7,1.5) rectangle (4.7,3.5);
% \draw[-stealth, line width=0.5mm] (2.3, 2.5) -- (2.7,2.5);
% % task title
% \node[text width=0.5cm] at (2.6,3.8) {$T^\text{quiz}$};
% % solution code
% \node[text width=0.5cm, anchor=east, left] at (5.4, 2.75){\begin{boxcode}{3.68cm}{0.65}{0.88}
% 				\textcode{def }\DSLRun\textcode{()\{}\\
% 				\quad \DSLRepeat\textcode{(8}\textcode{)\{}\\
%                 \quad \quad \DSLPickMarker\\
%                 %  \quad \quad \DSLMove\\
%                 \quad \quad \framebox[8.0\width]{?}\\
% 				\quad \textcode{\}}\\
% 				\quad \DSLPutMarker\\
% 				\quad \DSLTurnRight\\
% 				\textcode{\}}
% 			\end{boxcode}};
% % code title
% \node[text width=0.5cm] at (6.2,3.8) {$C^{\text{quiz}}$};
% \end{tikzpicture}
% }
% \vspace{-1.5mm}
% \subcaption{$S^\text{quiz} = \footnotesize{\{\DSLRun\{\DSLRepeat{}\}\}}$; a $C^\text{quiz}$ with size $=6$}
% \end{minipage}
% \end{subfigure}% for Quiz pairs: size =0
% \\
% \vspace{1mm}
% \begin{subfigure}{1\textwidth}% for Quiz pairs: size =2
% \begin{minipage}{0.5\textwidth}
% \centering{
% \begin{tikzpicture}[
%         box/.style={rectangle,draw=black, minimum size=0.2cm},
%         ]
% \draw[draw=black, thick] (0,1.1) rectangle (8, 4.4);
% %% task
% \foreach \x in {0.4,0.6,...,2.4}{
%             \foreach \y in {2,2.2,...,3.8}
%                 \node[box, fill=white] at (\x,\y){};
%         }
% \foreach \x in {2.8,3,...,4.6}{
%             \foreach \y in {2,2.2,...,3.8}
%                 \node[box, fill=white] at (\x,\y){};
%         }
        
% %% add the walls and the markers

% \node[draw, fill=yellow, diamond, inner sep=1.5pt,minimum size=0.5pt] at(2, 2.2){};
% \node[draw, fill=yellow, diamond, inner sep=1.5pt,minimum size=0.5pt] at(2, 2.4){};
% \node[draw, fill=yellow, diamond, inner sep=1.5pt,minimum size=0.5pt] at(2, 2.8){};
% \node[draw, fill=yellow, diamond, inner sep=1.5pt,minimum size=0.5pt] at(2, 3){};
% \node[draw, fill=yellow, diamond, inner sep=1.5pt,minimum size=0.5pt] at(2, 3.2){};
% \node[draw, fill=yellow, diamond, inner sep=1.5pt,minimum size=0.5pt] at(4.4, 2.6){};
% \node[draw, fill=yellow, diamond, inner sep=1.5pt,minimum size=0.5pt] at(4.4, 3.4){};

% % add the karel agent
% \node[draw, fill=blue!50, dart, rotate=90 ,inner sep=0.01pt, minimum size=4pt] at (2, 2.2){};
% \node[draw, fill=blue!50, dart, rotate=90 ,inner sep=0.01pt, minimum size=4pt] at (4.4, 3.6){};
% % bounding boxes in the task-grids
% \draw[draw=black, thick] (0.3,1.9) rectangle (2.3,3.9);
% \draw[draw=black, thick] (2.7,1.9) rectangle (4.7,3.9);
% \draw[-stealth, line width=0.5mm] (2.3, 2.9) -- (2.7,2.9);
% % task title
% \node[text width=0.5cm] at (2.6,4.2) {$T^\text{quiz}$};
% % solution code
% \node[text width=0.5cm, anchor=east, left] at (5.4, 2.9) {\begin{boxcode}{3.68cm}{0.65}{0.85}
% 				\textcode{def }\DSLRun\textcode{()\{}\\
% 				\quad \DSLRepeat\textcode{(7}\textcode{)\{}\\
%                 \quad \quad
%               \DSLIf\textcode{(}\DSLBoolMarker\textcode{)\{}\\
% 				\quad \quad \quad \DSLPickMarker\\
% 				\quad \quad \textcode{\}}\\
%                 \quad \quad \DSLElse\textcode{}\textcode{\{}\\
%                 % 	\quad \quad \quad \DSLPutMarker
%                 \quad \quad \quad	\framebox[8.0\width]{?}\\
%                 \quad \quad \textcode{\}
%                 	}\\
%                 \quad \quad \DSLMove\\
% 				\quad \textcode{\}}\\
% 				\textcode{\}}
% 			\end{boxcode}};
% % code title
% \node[text width=0.5cm] at (6.2,4.2) {$C^{\text{quiz}}$};
% \end{tikzpicture}
% }
% \vspace{-1.5mm}
% \subcaption{$S^\text{quiz} = \footnotesize{\{\DSLRun\{\DSLRepeat\{\DSLIfElse{}\}\}\}}$; a code $C^\text{quiz}$ with size $=6$}
% \end{minipage}
% %%% example 2
% \begin{minipage}{0.5\textwidth}
% \centering{
% \begin{tikzpicture}[
%         box/.style={rectangle,draw=black, minimum size=0.2cm},
%         ]
% \draw[draw=black, thick] (0,1.1) rectangle (8, 4.4);
% %% task
% \foreach \x in {0.4,0.6,...,2.4}{
%             \foreach \y in {2,2.2,...,3.8}
%                 \node[box, fill=white] at (\x,\y){};
%         }
% \foreach \x in {2.8,3,...,4.6}{
%             \foreach \y in {2,2.2,...,3.8}
%                 \node[box, fill=white] at (\x,\y){};
%         }

% %% add the walls and the markers
% \node[draw, fill=yellow, diamond, inner sep=1.5pt,minimum size=0.5pt] at(2, 3.6){};
% \node[draw, fill=yellow, diamond, inner sep=1.5pt,minimum size=0.5pt] at(1.6, 3.6){};
% \node[draw, fill=yellow, diamond, inner sep=1.5pt,minimum size=0.5pt] at(1.4, 3.6){};
% \node[draw, fill=yellow, diamond, inner sep=1.5pt,minimum size=0.5pt] at(1.2, 3.6){};

% \node[draw, fill=yellow, diamond, inner sep=1.5pt,minimum size=0.5pt] at(4.2, 3.6){};
% \node[draw, fill=yellow, diamond, inner sep=1.5pt,minimum size=0.5pt] at(3.4, 3.6){};
% \node[draw, fill=yellow, diamond, inner sep=1.5pt,minimum size=0.5pt] at(3.2, 3.6){};

% % add the karel agent
% \node[draw, fill=blue!50, dart, rotate=-180 ,inner sep=0.01pt, minimum size=4pt] at (2, 3.6){};
% \node[draw, fill=blue!50, dart, rotate=90 ,inner sep=0.01pt, minimum size=4pt] at (3, 3.8){};
        
        
% % bounding boxes in the task-grids
% \draw[draw=black, thick] (0.3,1.9) rectangle (2.3,3.9);
% \draw[draw=black, thick] (2.7,1.9) rectangle (4.7,3.9);
% \draw[-stealth, line width=0.5mm] (2.3, 2.9) -- (2.7,2.9);
% % task title
% \node[text width=0.5cm] at (2.6,4.2) {$T^\text{quiz}$};
% % solution code
% \node[text width=0.5cm, anchor=east, left] at (5.4, 2.8){\begin{boxcode}{3.68cm}{0.65}{0.68}
% 				\textcode{def }\DSLRun\textcode{()\{}\\
% 				\quad \DSLRepeat\textcode{(7}\textcode{)\{}\\
%                 \quad \quad
%               \DSLIf\textcode{(}\DSLBoolNoMarker\textcode{)\{}\\
% 				\quad \quad \quad \DSLPutMarker\\
% 				\quad \quad \textcode{\}}\\
%                 \quad \quad \DSLElse\textcode{}\textcode{\{}\\
%                 % 	\quad \quad \quad \DSLPickMarker
%                 \quad \quad \quad \framebox[8.0\width]{?}\\
%                 \quad \quad \textcode{\}}\\
%                 \quad \quad \DSLMove\\
% 				\quad \textcode{\}}\\
% 				\quad \DSLTurnRight\\
% 				\quad \DSLMove\\
% 				\textcode{\}}
% 			\end{boxcode}};
% % code title
% \node[text width=0.5cm] at (6.2,4.2) {$C^{\text{quiz}}$};
% \end{tikzpicture}
% }
% \vspace{-1.5mm}
% \subcaption{$S^\text{quiz} = \footnotesize{\{\DSLRun\{\DSLRepeat\{\DSLIfElse{}\}\}\}}$; a $C^\text{quiz}$ with size $=8$}
% \end{minipage}
% \end{subfigure}% for Quiz pairs: size =2
% \vspace{-2.8mm}
% \caption{Illustration of pop quizzes generated by \algmultihop~when considering two different substructures of Karel task T-5 (see Fig.~\ref{table:ref.tasks}). (a) illustrates the original task and its solution code. (b) shows the multi-choice question presented for a Karel task based pop quiz. (c, d) illustrate task-code pairs with substructure \begin{small}\{\DSLRun\{\DSLRepeat{}\}\}\end{small}, having different code sizes; we show two out of the $1220$ different pop quizzes that were generated (see Fig.~\ref{fig:experiments.analysis}). (e, f) illustrate task-code pairs with substructure \begin{small}\{\DSLRun\{\DSLRepeat\{\DSLIfElse{}\}\}\}\end{small}, having different code sizes; we show two out of the $3500$ different pop quizzes that were generated (see Fig.~\ref{fig:experiments.analysis}).}
% \label{fig:karel.illustration}
% \vspace{-2.65mm}
% \end{figure*}






%%%%%%%%%%%%%%%%%%%%%%%%%%%%%%%%

% \vspace{1mm}
\looseness-1\textbf{Multi-dimensional evaluation rubric.} Inspired by the evaluation rubric in \cite{DBLP:conf/aied/PriceZB17,DBLP:conf/edm/ZhiMDLPB19},  we assess pop quizzes on a multi-dimensional rubric with three attributes, each rated on a three-point Likert scale (with higher scores being better). More concretely, we have: (i) \ind~attribute measuring the degree of individualization of the pop quiz to the current student attempt ($3$: high; $2$: medium; $1$: low); (ii) \interp~attribute measuring how easy the pop quiz is to comprehend/solve ($3$: easy; $2$: might confuse the student sometimes; $1$: either incorrect or is very difficult to solve.); (iii) \conceal~attribute measuring the extent to which the pop quiz conceals the solution code ($3$: sufficiently conceals; $2$: reveals the solution to some extent; $1$: reveals the solution to a large extent). \overall{} denotes the sum of scores across three attributes for a pop quiz.


% \vspace{1mm}
\looseness-1\textbf{Expert study setup.} We picked three tasks spanning different types of constructs and complexity: T-1, T-4, and T-5 from Fig.~\ref{fig:experiments.analysis}. Thus, in total we evaluated $48$ scenarios: $4$ algorithm variants $\times$ $4$ student types $\times$ $3$ tasks (see Figs.~\ref{fig:intro} and~\ref{fig:karel.illustration} as example scenarios). Two researchers, with experience in block-based programming, evaluated each of the $48$ scenarios independently.
%
The evaluation was done through a web survey where a scenario was introduced at random, and assessed based on the rubric.



%%%%%%%%%%% input results table
% !TEX root =  main.tex
%%%%%%%%%%%%%%%%%%%%%%%%%%%%%%%%%%%%%%%%%%%%%%%%%%%%%%%%%%

\begin{wrapfigure}[11]{r}{0.62\textwidth}
% \renewcommand{\arraystretch}{1.1}
\vspace{-3.5mm}
\centering
\scalebox{0.85}{
    \setlength\tabcolsep{2pt}
    \renewcommand{\arraystretch}{1.1}
    \begin{tabular}{c|ccc|c}
        Algorithm & \small{\ind} & \small{\interp} & \small{\conceal} & \small{\overall} \\
        \toprule
        \footnotesize{\algsame} & \cellcolor{red!35}$2.0 (0.7)$ & $2.8 (0.1)$ & $3.0 (0.0)$ & $7.8 (0.8)$ \\
        \footnotesize{\alghintpolicystruct} & $2.8 (0.1)$ & \cellcolor{red!35}$2.5 (0.6)$ & $3.0 (0.0)$ & $8.3 (0.7)$ \\
        % & 
        \footnotesize{\algmincode} & $2.7 (0.3)$ & $3.0 (0.0)$ & \cellcolor{red!35}$1.5 (0.4)$ & $7.2 (0.7)$ \\
        % & 
        \hline
        \cellcolor{blue!15}\footnotesize{\algmultihop} & \cellcolor{blue!15}${2.7 (0.2)}$ & \cellcolor{blue!15}${3.0 (0.0)}$ & \cellcolor{blue!15}${2.9 (0.1)}$ & 
        \cellcolor{blue!15}${8.6 (0.3)}$ \\
        \bottomrule
    \end{tabular}
}
%\vspace{-2.5mm}
\caption{\looseness-1Mean (Variance) attribute ratings for different algorithms. Higher scores are better. \algmultihop~performs well across all three attributes and has the highest \overall~score; 
see  Section~\ref{sec:surveystudy} for details.}
\label{fig:surveystudy}
%\vspace{4mm}
\end{wrapfigure}
%

%%%%%%%%%%%%%%%%%%%%%%%%%%%%%%%%%%%%%


% \vspace{1mm}
\looseness-1\textbf{Expert study results.} First, we validate the expert ratings using the quadratic-weighted Cohen's kappa inter-agreement reliability value~\cite{DBLP:conf/aied/PriceZB17} for each attribute: $0.62$ (\ind), $0.69$ (\interp), $0.79$ (\conceal), and $0.7$ (\overall). The values indicate \textit{substantial agreement} between the raters. The average ratings are presented in Fig.~\ref{fig:surveystudy} and \algmultihop~has the highest \overall~score. We analyze these ratings per attribute based on the Kruskal-Wallis significance test~\cite{macfarland2016kruskal}; the results discussed next are statistically significant with $p < 0.01$. On the~\ind~attribute, \algsame{} performs significantly worse because it does not account for the student attempt (see Section~\ref{sec:model.stage1}). On the ~\interp~attribute, \alghintpolicystruct~performs significantly worse because there are instances where no valid code reduction of \solutioncode~w.r.t. $S^\text{quiz}$ is found (see Footnote~\ref{footnote:sec3}, Section~\ref{sec:model.stage2}). 
%
Finally, on the \conceal~attribute, \algmincode~performs significantly worse because it obtains $C^\text{quiz}$ via a direct reduction of \solutioncode{} without any mutation (see Section~\ref{sec:model.stage2}). 
