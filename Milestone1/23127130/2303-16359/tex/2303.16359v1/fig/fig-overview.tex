% !TEX root =  main.tex
%%%%%%%%%%%%%%%%%%%%%%%%%%%%%%%%%%%%%%%%%%%%%%%%%%%%%%%%%%
\begin{figure*}[!t]
\centering
\begin{tikzpicture}[
        box/.style={rectangle,draw=black, minimum size=0.25cm},
        ]
        \foreach \x in {-9.2,-8.95,...,-6.95}{
            \foreach \y in {2,2.25,...,4.25}
                \node[box, fill=gray!40] at (\x,\y){};
        }
       
        \foreach \y in {2.5,2.75,3}{
                \node[box, fill=white] at (-8.45,\y){};
        }
        % \foreach \x in {-8.75,-8.5, -8.25}{
        %         \node[box, fill=white] at (\x,3){};
        % }
        \node[box, fill=white] at (-7.95,2.75){};
        \node[box, fill=white] at (-7.7,2.75){};
          \foreach \x in {-8.45,-8.2,-7.95}{
                \node[box, fill=white] at (\x,3){};
        }
        \foreach \y in {2.75,3,3.25, 3.5}{
                \node[box, fill=white] at (-7.45,\y){};
        }
        \foreach \x in {-7.45,-7.7,-7.95,-8.2}{
                \node[box, fill=white] at (\x,3.5){};
        }
        \node[box, fill=white] at (-8.2,3.75){};
         %%% bounding box
        \draw[draw=black, thick] (-9.3,1.9) rectangle (-6.85,4.35);
        \node[draw, fill=red, star, star points=5,inner sep=0pt,minimum size=5pt] at (-8.2,3.75){};
        \node[draw, fill=blue!50, dart, rotate=90, inner sep=0.2pt,minimum size=4pt] at (-8.45,2.5){};
        %%%% code
        \node[text width=0.5cm, anchor=west, right] at (-6.45, 3.5)
    {\begin{boxcode}{3.68cm}{0.80}{0.975}
				\textcode{def }\DSLRun\textcode{()\{}\\
                \quad \DSLMove\\
                \quad \DSLMove\\
                \quad \DSLTurnRight\\
                \quad \ldots\\
                \quad \text{21 more action blocks}\\
				\textcode{\}}
				% \vspace{0.5em}
			\end{boxcode}};
	%%%%% bounding box around the task-code pair
	\draw[draw=black, thick] (-9.95,1.1) rectangle (-2.1, 4.9);
	%%%%%%%%% solution code
	\node[text width=0.5cm, anchor=east, left] at (-12.9, 3.1)
    {\begin{boxcode}{3.68cm}{0.70}{0.55}
				\textcode{def }\DSLRun\textcode{()\{}\\
				\quad \DSLRepeatUntil\textcode{(}\DSLBoolGoal\textcode{)\{}\\
                \quad \quad
              \DSLIf\textcode{(}\DSLBoolPathAhead\textcode{)\{}\\
				\quad \quad \quad \DSLMove\\ 
				% \textcode{\}}\\
				\quad \quad \textcode{\}}\\
				\quad \quad 
				\DSLElse\textcode{\{}\\
				\quad \quad \quad \DSLIf\textcode{(}\DSLBoolPathRight\textcode{)\{}\\
				\quad \quad \quad \quad \DSLTurnRight\\
				% \textcode{\}}\\
				\quad \quad \quad \textcode{\}}\\
				\quad \quad \quad
				\DSLElse\textcode{\{}\\
				\quad \quad \quad \quad \DSLTurnLeft\\
				% \textcode{\}}\\
				\quad \quad \quad \textcode{\}}\\
			     \vspace{-3.5mm}
				\quad \quad \textcode{\}}\\
		     	\vspace{-4.5mm}				
				\quad \textcode{\}}\\
				\vspace{-5.5mm}				
				\textcode{\}}
            %\vspace{-40mm}				
			\end{boxcode}
		    %\vspace{-6mm}
			};
	%%%%% bounding box around the solution-code pair
	\draw[draw=black, thick, pattern=north east lines, pattern color=gray!30] (-13.5,1.1) rectangle (-10.3, 4.9);
	%%%% captions for the task-codes
	\node[text width=0.5cm] at (-7.975,4.7) {$T^\text{in}$};
	\node[text width=0.5cm] at (-4.46,4.7) {\studentcode};
	\node[text width=0.5cm] at (-11.8,4.7) {\solutioncode};
	%%%%%%%%%%%% Intervention boxes
% 	\draw[draw=black, thick] (-1.2,0.5) rectangle (6.9, 3.8);
	\draw[draw=black, thick] (-13.5,-2.2) rectangle (-2.1, 0.9);
	%%%%%% T-out: intervention task
	[
        box/.style={rectangle,draw=black, minimum size=0.25cm},
        ]
    
        \foreach \x in {-13,-12.75,...,-10.75}{
            \foreach \y in {-2,-1.75,...,0.25}
                \node[box, fill=gray!40] at (\x,\y){};
            
        }
    	
     \foreach \y in {-2,-1.75,...,-0.25, 0}{
    		\node[box, fill=white] at (-13,\y){};
    		
    	}
    	
    \foreach \x in {-13,-12.75}{
    	\node[box, fill=white] at (\x,0){};
    		
    }

        \node[box, fill=white] at (-13,-2){};
        %%% bounding box
        \draw[draw=black, thick] (-13.1,-2.1) rectangle (-10.65,0.35);
         \node[draw, fill=red, star, star points=5,inner sep=0pt,minimum size=5pt] at (-13,-2){};
         \node[draw, fill=blue!50, dart, rotate=180, inner sep=0.2pt,minimum size=4pt] at (-12.75,0){};
    %%%%%%%% C-out: intervention code
    \node[text width=0.5cm, anchor=west, right] at (-10.3, -0.6)
    {\begin{boxcode}{3.68cm}{0.80}{1}
				\textcode{def }\DSLRun\textcode{()\{}\\
				\quad \DSLMove\\
			   \quad \DSLTurnLeft\\
			   \quad
		 \DSLRepeatUntil\textcode{(}\DSLBoolGoal\textcode{)\{}\\
				\quad \quad \framebox[8.0\width]{?}\\
				\quad \textcode{\}}\\
				% \vspace{-5mm}
				\textcode{\}}
				% \vspace{-10mm}
			\end{boxcode}};
	%%%%%%%%% Multiple Choice Question
	 \node[text width=0.5cm, anchor=west, right] at (-6.4, -0.6)
	 {\begin{boxcode}{3.98cm}{0.8}{1.15}
				\textcode{Q.} \text{Fill in the blank from: }\\
				\quad \quad \tikz\draw[black,fill=none] (0,0) circle (.5ex); \DSLMove\\
				\quad \quad \tikz\draw[black,fill=none] (0,0) circle (.5ex); \DSLTurnLeft\\
				\quad \quad \tikz\draw[black,fill=none] (0,0) circle (.5ex); \DSLTurnRight\\
				\\
			\end{boxcode}
	 };
	 %%%% captions for the task-codes
	\node[text width=0.5cm] at (-11.7,0.7) {$T^\text{quiz}$};
	\node[text width=2.6cm] at (-8.25,0.7) {$C^\text{quiz}\text{ with }1\text{ blank}$};
	\node[text width=1cm] at (-4.1,0.7) {Quiz};
% 	 %%%%%%%% connector arrows between task and intervention --- OLD
% 	\draw[draw=black,solid,line width=1mm,
% preaction={-triangle 90,thin,draw,shorten >=-1mm}] (-2.10, 2.4) to [out=60,in=120](-1.3,2.4);
% 	\draw[draw=black,solid,line width=1mm,
% preaction={-triangle 90,thin,draw,shorten >=-1mm}] (-1.25, 1.4) to [out=-120,in=-60] (-2.05, 1.4);
 %%%%%%%% connector arrows between task and intervention --- new
	\tikzstyle{doublearr}=[latex-latex, black, line width=1.5pt]
    \draw [doublearr, bend left]    (-2.1, 2.8) to (-2.1,-1);
    %%%%%%%%%%\draw (-7.5,2.7) rectangle (-6.5,3.2) 
    \node[text width=1cm] at (-14.5,2.8){\textbf{Task}};
    %%%%%%%%%%%%%%\draw (1.9,2.7) rectangle (2.9,3.2) 
    \node[text width=1.7cm] at (-14.25,-1){\textbf{Pop Quiz}};
    \end{tikzpicture}
    %\setlength{\belowcaptionskip}{-16pt} 

%\vspace{-1.5mm}
\caption{Illustration of our pop quiz based framework. The ``Task'' panel shows an input task $T^\text{in}$ from HOC~\cite{hourofcode_maze}, the student's current attempt \studentcode, and the solution code \solutioncode~(not revealed to the student). The student is currently unsuccessful in solving the task: the current attempt \studentcode~does not solve the visual puzzle within the maximal number of permitted blocks ($7$ blocks) and does not use any of the required constructs  (\DSLRepeatUntil and \DSLIfElse constructs). The ``Pop Quiz'' panel shows a pop quiz generated by our algorithm in the form of task-code pair ($T^\text{quiz}, C^\text{quiz}$) along with a multiple choice question, introducing the \DSLRepeatUntil~construct. After the student solves the pop quiz, they resume working on the input task. The framework would be invoked when a student needs help; importantly, the pop quizzes presented to the student are adaptive w.r.t. the student's current attempt \studentcode. Moreover, our algorithm generates pop quizzes that are easy to comprehend and solve, and $C^\text{quiz}$ sufficiently conceals \solutioncode.
}
% 
\label{fig:intro}
%\vspace{-4.2mm}
\end{figure*}






























%%%%%%%%%%%%%%%%%%%%%%%%%%%%%%%% OLD
% \begin{figure*}[!t]
% \centering
% \begin{tikzpicture}[
%         box/.style={rectangle,draw=black, minimum size=0.2cm},
%         ]
%         \foreach \x in {-7.1,-6.9,...,-5.1}{
%             \foreach \y in {1.5,1.7,...,3.3}
%                 \node[box, fill=gray!40] at (\x,\y){};
%         }
       
%         \foreach \y in {1.9,2.1,2.3}{
%                 \node[box, fill=white] at (-6.5,\y){};
%         }
%         \foreach \x in {-6.5,-6.3, -6.1}{
%                 \node[box, fill=white] at (\x,2.3){};
%         }
%         \node[box, fill=white] at (-6.1,2.1){};
%           \foreach \x in {-6.1,-5.9,-5.7}{
%                 \node[box, fill=white] at (\x,2.1){};
%         }
%         \foreach \y in {2.1,2.3,2.5, 2.7}{
%                 \node[box, fill=white] at (-5.7,\y){};
%         }
%         \foreach \x in {-5.7,-5.9,-6.1,-6.3}{
%                 \node[box, fill=white] at (\x,2.7){};
%         }
%         \node[box, fill=white] at (-6.3,2.9){};
%          %%% bounding box
%         \draw[draw=black, thick] (-7.2,1.4) rectangle (-5.2,3.4);
%         \node[draw, fill=red, star, star points=5,inner sep=0pt,minimum size=5pt] at (-6.3,2.9){};
%         \node[draw, fill=blue!50, dart, rotate=90, inner sep=0.2pt,minimum size=4pt] at (-6.5,1.9){};
%         %%%% code
%         \node[text width=0.5cm, anchor=west, right] at (-5.35, 2.6)
%     {\begin{boxcode}{3.68cm}{0.70}{1.0}
% 				\textcode{def }\DSLRun\textcode{()\{}\\
%                 \quad \DSLMove\\
%                 \quad \DSLMove\\
%                 \quad \DSLTurnRight\\
%                 \quad \ldots\\
%                 \quad \textcode{21 more action blocks}\\
% 				\textcode{\}}
% 				% \vspace{0.5em}
% 			\end{boxcode}};
% 	%%%%% bounding box around the task-code pair
% 	\draw[draw=black, thick] (-7.35,0.5) rectangle (-2.1, 3.8);
% 	%%%%%%%%% solution code
% 	\node[text width=0.5cm, anchor=east, left] at (-9.88, 2.25)
%     {\begin{boxcode}{3.68cm}{0.62}{0.65}
% 				\textcode{def }\DSLRun\textcode{()\{}\\
% 				\quad \DSLRepeatUntil\textcode{(}\DSLBoolGoal\textcode{)\{}\\
%                 \quad \quad
%               \DSLIf\textcode{(}\DSLBoolPathAhead\textcode{)\{}\\
% 				\quad \quad \quad \DSLMove\\ 
% 				% \textcode{\}}\\
% 				\quad \quad \textcode{\}}\\
% 				\quad \quad 
% 				\DSLElse\textcode{\{}\\
% 				\quad \quad \quad \DSLIf\textcode{(}\DSLBoolPathRight\textcode{)\{}\\
% 				\quad \quad \quad \quad \DSLTurnRight\\
% 				% \textcode{\}}\\
% 				\quad \quad \quad \textcode{\}}\\
% 				\quad \quad \quad
% 				\DSLElse\textcode{\{}\\
% 				\quad \quad \quad \quad \DSLTurnLeft\\
% 				% \textcode{\}}\\
% 				\quad \quad \quad \textcode{\}}\\
% 				%\vspace{-4.5mm}
% 				\quad \quad \textcode{\}}\\
% 				%\vspace{-6mm}				
% 				\quad \textcode{\}}\\
% 				%\vspace{-8mm}				
% 				\textcode{\}}
%             %\vspace{-6mm}				
% 			\end{boxcode}
% 			%\vspace{-6mm}
% 			};
% 	%%%%% bounding box around the solution-code pair
% 	\draw[draw=black, thick, pattern=north east lines, pattern color=gray!30] (-10.5,0.5) rectangle (-7.5, 3.8);
% 	%%%% captions for the task-codes
% 	\node[text width=0.5cm] at (-5.975,3.6) {$T^\text{in}$};
% 	\node[text width=0.5cm] at (-3.86,3.6) {\studentcode};
% 	\node[text width=0.5cm] at (-9.1,3.6) {\solutioncode};
% 	%%%%%%%%%%%% Intervention boxes
% 	\draw[draw=black, thick] (-1.2,0.5) rectangle (6.9, 3.8);
% 	%%%%%% T-out: intervention task
% 	[
%         box/.style={rectangle,draw=black, minimum size=0.2cm},
%         ]
%         \foreach \x in {-1,-0.8,...,0.8}{
%             \foreach \y in {1.5,1.7,...,3.3}
%                 \node[box, fill=gray!40] at (\x,\y){};
            
%         }
    	
%     	 \foreach \y in {3.1,2.9,...,1.5}{
%     		\node[box, fill=white] at (-1,\y){};
    		
%     	}
    	
%     	 \foreach \x in {-1,-0.8}{
%     		\node[box, fill=white] at (\x,3.1){};
    		
%     	}
%         \node[box, fill=white] at (-1,1.5){};
%         %%% bounding box
%         \draw[draw=black, thick] (-1.1,1.4) rectangle (0.9,3.4);
%          \node[draw, fill=red, star, star points=5,inner sep=0pt,minimum size=5pt] at (-1,1.5){};
%          \node[draw, fill=blue!50, dart, rotate=180, inner sep=0.2pt,minimum size=4pt] at (-0.8,3.1){};
%     %%%%%%%% C-out: intervention code
%     \node[text width=0.5cm, anchor=west, right] at (0.7, 2.65)
%     {\begin{boxcode}{3.68cm}{0.65}{1.0}
% 				\textcode{def }\DSLRun\textcode{()\{}\\
% 				\quad \DSLMove\\
% 			   \quad \DSLTurnLeft\\
% 			   \quad
% 			   \DSLRepeatUntil\textcode{(}\DSLBoolGoal\textcode{)\{}\\
% 				\quad \quad \framebox[8.0\width]{?}\\
% 				\quad \textcode{\}}\\
% 				\textcode{\}}
% 			\end{boxcode}};
% 	%%%%%%%%% Multiple Choice Question
% 	 \node[text width=0.5cm, anchor=west, right] at (3.45, 2.7)
% 	 {\begin{boxcode}{3.88cm}{0.70}{1.0}
% 				\quad \textcode{Q.} \text{Fill in the blank from: }\\
% 				\quad \quad \tikz\draw[black,fill=none] (0,0) circle (.5ex); \DSLMove\\
% 				\quad \quad \tikz\draw[black,fill=none] (0,0) circle (.5ex); \DSLTurnLeft\\
% 				\quad \quad \tikz\draw[black,fill=none] (0,0) circle (.5ex); \DSLTurnRight\\
% 			\\
% 			\end{boxcode}
% 	 };
% 	 %%%% captions for the task-codes
% 	\node[text width=0.5cm] at (0.125,3.6) {$T^\text{quiz}$};
% 	\node[text width=2.6cm] at (2.5,3.6) {$C^\text{quiz}\text{ with }1\text{ blank}$};
% 	\node[text width=1cm] at (5.44,3.6) {Quiz};
% % 	 %%%%%%%% connector arrows between task and intervention 
% 	\draw[draw=black,solid,line width=1mm,
% preaction={-triangle 90,thin,draw,shorten >=-1mm}] (-2.10, 2.4) to [out=60,in=120](-1.3,2.4);
% 	\draw[draw=black,solid,line width=1mm,
% preaction={-triangle 90,thin,draw,shorten >=-1mm}] (-1.25, 1.4) to [out=-120,in=-60] (-2.05, 1.4);
% 	%%%%% intervention box
%     %%%%%%%%%%%\draw (-7.5,2.7) rectangle (-6.5,3.2) 
%     \node[text width=2cm] at (-7,4.2){\textbf{Current Task}};
%     %%%%%%%%%%%%%%\draw (1.9,2.7) rectangle (2.9,3.2) 
%     \node[text width=1.4cm] at (2.7,4.2){\textbf{Pop Quiz}};
%     \end{tikzpicture}
%     %\setlength{\belowcaptionskip}{-16pt} 
%     \vspace{-3.2mm}
% \caption{Illustration of our pop quiz based framework. The ``Current Task'' panel shows the input task $T^\text{in}$ from HOC~\cite{hourofcode_maze}, the student's current attempt \studentcode, and the solution code \solutioncode~(not revealed to the student). The student is currently unsuccessful in solving the task with the maximal number of permitted blocks ($7$ blocks) as \studentcode~does not use any of the constructs \DSLRepeatUntil, \DSLIf~and \DSLElse~needed to solve the task correctly. The ``Pop Quiz'' panel shows a pop quiz generated by our algorithm in the form of task-code pair ($T^\text{quiz}, C^\text{quiz}$) along with a multiple choice question, introducing the \DSLRepeatUntil~construct. After the student solves the pop quiz, they resume working on the input task. The framework would be invoked when a student seeks help on their attempt; importantly, the pop quizzes presented to the student are adaptive w.r.t. the student's current attempt \studentcode. }
% %successfully
% %\vspace{-3mm}
% %Throughout this process, the solution code~\solutioncode~is never exposed to the student. 
% \label{fig:intro}
% \vspace{-2.8mm}
% \end{figure*}

