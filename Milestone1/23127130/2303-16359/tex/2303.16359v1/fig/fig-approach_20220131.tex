% !TEX root =  main.tex
%%%%%%%%%%%%%%%%%%%%%%%%%%%%%%%%%%%%%%%%%%%%%%%%%%%%%%%%%%
%%%%%%%%%%%%%%%%%%%%%%%%%%%%%%%%%%%%%%%%%% Begin minipage for Figure 2
\begin{figure*}[!t]
\centering
\begin{minipage}{0.65\textwidth} 
%%%%%%%%%%%%%%%%%%%%%%%
\begin{minipage}{1\textwidth}
    \begin{subfigure}[b]{1.0\textwidth}
    	\centering
        \begin{tikzpicture}[scale=0.75, every node/.style={transform shape}]
            %%% input
             \draw (0,0) rectangle (0.8,0.8) node[pos=0.5] {\small$\studentcode$};
             \draw (0,1.5) rectangle (0.8,2.3) node[pos=0.5] {$\solutioncode$};
            %  \draw (-1.2,1.5) rectangle (-0.4,2.3) node[pos=0.5] {$T^\text{in}$};
            %  \draw [decorate,decoration={brace,amplitude=5pt, mirror},xshift=-4pt,yshift=0pt] (0.15,1.4) -- (1.05,1.4) node [black,midway,xshift=-0.6cm] {};
            %  \draw [decorate,decoration={brace,amplitude=5pt},xshift=-4pt,yshift=0pt] (0.15,2.4) -- (1.05,2.4) node [black,midway,xshift=-0.6cm] {};
             \draw (-1.2,0.6) rectangle (-0.4,1.4) node[pos=0.5] {$T^\text{in}$};
             %%% sketchspace
             \draw[pattern=dots, pattern color=gray!40] (1.3,-0.2) rectangle (5.8,2.5) node[pos=0.5] {};
             \draw[fill=white] (1.5,0) rectangle (2.3,0.8) node[pos=0.5] {$S^\text{in,stu}$};
             \draw[fill=white] (1.5,1.5) rectangle (2.3,2.3) node[pos=0.5] {$S^{\text{in,}\star}$};
             \node [trapezium, trapezium angle=85, minimum width=15mm, fill=white, draw] at (3.25,1.2){${\hintpolicy}$};
             \draw[fill=white] (4.8,0.85) rectangle (5.6,1.65) node[pos=0.5] {$S^\text{quiz}$};
             \node[] at (5.3,2.25) {\textbf{\large\sketchspace}};
             \draw (3.1, -0.1) rectangle (4.2, 0.4) node[pos=0.5] {Stage 1};
             \draw (6.05, -0.1) rectangle (7.45, 0.4) node[pos=0.5] {Stage 2(i)};
             \draw (7.65, -0.1) rectangle (9.15, 0.4) node[pos=0.5] {Stage 2(ii)};
             \draw (6.3, 2.2) rectangle (7.7, 2.6) node[pos=0.5] {Stage 3};
             %%%% task-code pair
             \draw (6.5,0.85) rectangle (7.3,1.65) node[pos=0.5] {$C^\text{quiz}$};
             \draw (8,0.85) rectangle (8.8,1.65) node[pos=0.5] {$T^\text{quiz}$};
             \draw (8,1.85) rectangle (8.8,2.65) node[pos=0.5] {\small$C^{\text{\tiny{quiz,}}k}$};
             %%%% arrows
             \draw[thick,->] (0.8,0.4) to node[pos=0.5, above]{$\sketchmap$} (1.3,0.4);
             \draw[thick,->] (0.8,1.9) to node[pos=0.5, above]{$\sketchmap$} (1.3,1.9);
             %%%%%%%%%%% arrows inside and outside Sketch space
             \draw[thick,->] (5.8,1.25) to node[pos=0.5, above]{$\sketchmap^{-1}$} (6.5,1.25);
             \draw[thick,->] (2.3,0.5) to node[pos=0.5, above]{} (2.8,0.95); % slant arrow inside
             \draw[thick,->] (2.3,2) to node[pos=0.5, above]{} (2.9,1.5);% slant arrow inside
             \draw[thick,->] (4.4,1.25) to node[pos=0.5, above]{} (4.8,1.25);
             \draw[thick,->] (7.3,1.25) to node[pos=0.5, above]{} (8,1.25);
        	\draw[thick,->] (7,1.65) to node[pos=0.25, above]{\exposureparam} (8,2.25);
             %%%% surrounding box
             %\draw (-1.4,-0.25) rectangle (10.1,2.9);
        \end{tikzpicture}
        \caption{Our algorithm \algmultihop}
        \label{fig:pipeline.abstract}
        \vspace{2mm}
    \end{subfigure}
\end{minipage}
%%%%%%%%%%%%%%%%%%%%%%%
\begin{minipage}{1\textwidth}
    \begin{minipage}{0.31\textwidth}
        \begin{subfigure}[b]{1\textwidth}
        	\centering
            \begin{boxcode}{3.62cm}{0.70}{0.95}
                \textcode{def }\DSLRun\textcode{()\{}\\
        		\ \ \DSLRepeatUntil\textcode{(}\DSLBoolGoal\textcode{)\{}\\
                \quad \quad
                \DSLIf\textcode{(}\SDSLBool\textcode{)\{\}}\\
        		\quad \quad 
        		\DSLElse\textcode{\{}\\
        		\quad \quad \quad \DSLIf\textcode{(}\SDSLBool\textcode{)\{\}}\\
        		\quad \quad \quad
        		\DSLElse\textcode{\{\}}\\
        		\quad \quad \textcode{\}}\\
        		%\vspace{-4.5mm}
        		\ \  \textcode{\}}\\
        		%\vspace{-6mm}				
        		\textcode{\}}
            \end{boxcode}
            \vspace{-3.3mm}
            \caption{$S^{\text{in,}\star}$}    
            \label{fig:pipeline.solsketch}
        \end{subfigure}
    \end{minipage}
    \begin{minipage}{0.33\textwidth}
        \begin{subfigure}[b]{1\textwidth}
        	\centering
            \begin{boxcode}{3.68cm}{0.70}{1.0}
        		\textcode{def }\DSLRun\textcode{()\{\}}
            \end{boxcode}
            \vspace{-3mm}
            \caption{$S^{\text{in,stu}}$}
            \vspace{3mm}
            \label{fig:pipeline.stusketch}
        \end{subfigure}
        \begin{subfigure}[b]{1\textwidth}
        	\centering
        	\begin{boxcode}{3.68cm}{0.70}{1.0}
        		\textcode{def }\DSLRun\textcode{()\{}\\
        		\ \ \DSLRepeatUntil\textcode{(}\DSLBoolGoal\textcode{)\{\}}\\
        		\textcode{\}}
            \end{boxcode}
            \vspace{-3mm}
            \caption{$S^{\text{quiz}}$}
            \label{fig:pipeline.sketchhint}    
        \end{subfigure}
    \end{minipage}
    \begin{minipage}{0.32\textwidth}
        \begin{subfigure}[b]{1\textwidth}
        	\centering
            \begin{boxcode}{3.68cm}{0.70}{1.0}
        		\textcode{def }\DSLRun\textcode{()\{}\\
        		\quad \DSLMove\\
        		\quad \DSLTurnLeft\\
        		\quad \DSLRepeatUntil\textcode{(}\DSLBoolGoal\textcode{)\{}\\
        		\quad \quad \DSLMove\\
        		\quad \textcode{\}}\\
        		\textcode{\}}
        		\vspace{5.4mm}
            \end{boxcode}
            \vspace{-3mm}
            \caption{$C^{\text{quiz}}$}
            \label{fig:pipeline.codequiz}
        \end{subfigure}
    \end{minipage}
\end{minipage}
\vspace{-3mm}
\caption{ (a) illustrates \algmultihop. In particular, we can instantiate the presented algorithm using input task~$T^\text{in}$, its solution code \solutioncode, and the current student attempt \studentcode~from Fig.~\ref{fig:intro}. The sketch of \solutioncode~is shown in (b), sketch of \studentcode~is shown in (c), sketch of $C^\text{quiz}$ is shown in (d), and the code of the pop quiz $C^\text{quiz}$ is shown in (e).}
%
%In all the sketches \DSLRepeatUntil is abbreviated as \DSLRUntil.
\label{fig:pipeline}
\end{minipage}
\hspace{2mm}
%%%%%%%%%%%%%%%%%%%%%%%
%%%%%%%%%%%%%%%%%%%%%%%%%%%%%%%%%%%%%%%%%% end minipage for Figure 2
%%%%%%%%%%%%%%%%%%%%%%%%%%%%%%%%%%%%%%%%%% Begin minipage for Figure 3
\begin{minipage}{0.32\textwidth}% figure 3
    \begin{subfigure}[b]{1.0\textwidth}
    	\centering
        	\begin{tikzpicture}[scale=0.75, every node/.style={transform shape}]
                \draw[pattern=dots, pattern color=gray!40] (-0.1,0) rectangle (5.4,7.5) node[pos=0.5] {};
                \node[] at (5,6.5) {\textbf{\large\sketchspace}};
                 %% add the node for the student sketch
                \node[circle, fill=black, minimum size=0.15cm, inner sep=0pt, outer sep=0pt, label=right:{$S^{\text{in,stu}}$}] (nstu) at (2.75,4.7) {};
                %%%%% add the node for the solution sketch
                % \node[circle, fill=black, minimum size=0.15cm, inner sep=0pt, outer sep=0pt, 
                % label=right:{$S^{\text{in,}\star}$}] (nsol) at (2.75,0.5) {};
                \node [outer sep=1pt,fill=white,above] at (2.85,0.4) {\footnotesize{$S^{\text{in,}\star}$}};
                \draw (2.75,1.1) node[cross=4] {};
                %%%%%% draw the neighborhood ovals
                \draw (2.75,4.5) ellipse (1.5cm and 1.5cm); % nbd-1
                 %%%%%% draw the neighborhood ovals
                \draw (2.75,4.2) ellipse (2cm and 2.5cm); % nbd-2
                  %%%%%% draw the neighborhood ovals
                \draw (2.75,3.8) ellipse (2.4cm and 3.6cm); % nbd-3
                %%%%%%% add the nodes for the substructures
                \draw (3,3.9) node[cross=4] {}; % repeat_until
                \node [outer sep=1pt,fill=white,above] at (2.8,3.2) {\scriptsize{\{\DSLRun\{\DSLRepeatUntil{(\DSLBoolGoal)}}\}\}};
                \draw (3.2,2.6) node[cross=4] {}; %repeat_until;if_else
                 \node [outer sep=1pt,fill=white,above] at (2.9,1.8) {\scriptsize{\{\DSLRun\{\DSLRepeatUntil(\DSLBoolGoal)\{\DSLIfElse(\SDSLBool){}\}}\}\}};
                %%%%%%% Add the labels showing the hops
                \node [outer sep=1pt,fill=white,above] at (2.75,5.25) {$l=1$};
                 \node [outer sep=1pt,fill=white,above] at (2.75,6) {$l=2$};
                  \node [outer sep=1pt,fill=white,above] at (2.75,6.8) {$l=3$};
            \end{tikzpicture}
    \end{subfigure}
\vspace{-4mm}    
\caption{
\algmultihop{} Stage~1 for the scenario shown in Fig.~\ref{fig:intro}.  $\textnormal{X}$ shows substructures of $S^{\text{in,}\star}$ in $l$-hop neighborhoods of $S^\text{in,stu}$ for $l \in \{1, 2, 3\}$. Details are provided in Section~\ref{sec:model.stage1}.
%$S^\text{quiz}$ selection details are in Section~\ref{sec:model.stage1}.
%Stage 1 of \algmultihop~for the scenario shown in Fig.~\ref{fig:intro}.  $\textnormal{X}$ shows substructures of $S^{\text{in,}\star}$ in $l$-hop neighborhoods of $S^\text{in,stu}$. $S^\text{quiz}$ selection details are in Section~\ref{sec:model.stage1}.
}
\label{fig.approach.algmultihop}
\end{minipage}
%%%%%%%%%%%%%%%%%%%%%%%%%%%%%%%%%%%%%%%%%% End minipage for Figure 3
\end{figure*}








% %%%%%%%%%%%%%%%%%%%%%%%%%%%%%%%%%%
% \begin{minipage}{0.30\textwidth}
% 	\centering

% \\
% %%%%%%%%%%%%%%%%%

% \end{minipage}
% %%%%%%%%%%%%%%%%%%%%%%%%%%%%%%%%%%




% \begin{minipage}{1\textwidth} % pipeline abstract
% \begin{tikzpicture}[scale=0.75, every node/.style={transform shape}]
%     %%% input
%      \draw (0,0) rectangle (0.8,0.8) node[pos=0.5] {$\studentcode$};
%      \draw (0,1.5) rectangle (0.8,2.3) node[pos=0.5] {$\solutioncode$};
%      \draw (-1.2,1.5) rectangle (-0.4,2.3) node[pos=0.5] {$T^\text{in}$};
%      \draw [decorate,decoration={brace,amplitude=5pt, mirror},xshift=-4pt,yshift=0pt] (-1.15,1.4) -- (1.05,1.4) node [black,midway,xshift=-0.6cm] {};
%      \draw [decorate,decoration={brace,amplitude=5pt},xshift=-4pt,yshift=0pt] (-1.15,2.4) -- (1.05,2.4) node [black,midway,xshift=-0.6cm] {};
%      %%% sketchspace
%      \draw[pattern=dots, pattern color=gray!40] (1.3,-0.2) rectangle (5.8,2.8) node[pos=0.5] {};
%      \draw[fill=white] (1.5,0) rectangle (2.3,0.8) node[pos=0.5] {$S^\text{in,stu}$};
%      \draw[fill=white] (1.5,1.5) rectangle (2.3,2.3) node[pos=0.5] {$S^{\text{in,}\star}$};
%      \node [trapezium, trapezium angle=85, minimum width=15mm, fill=white, draw] at (3.55,1.2){$\hintpolicy$};
%      \draw[fill=white] (4.8,0.75) rectangle (5.6,1.55) node[pos=0.5] {$S^\text{quiz}$};
%      \node[] at (5.3,2.25) {\textbf{\sketchspace}};
%      \draw (3.1, -0.1) rectangle (4.2, 0.4) node[pos=0.5] {Stage 1};
%      \draw (6.1, -0.1) rectangle (7.4, 0.4) node[pos=0.5] {Stage 2(i)};
%      \draw (7.7, -0.1) rectangle (9.1, 0.4) node[pos=0.5] {Stage 2(ii)};
%      \draw (6.3, 2.2) rectangle (7.7, 2.6) node[pos=0.5] {Stage 3};
%      %%%% task-code pair
%      \draw (6.5,0.75) rectangle (7.3,1.55) node[pos=0.5] {$C^\text{quiz}$};
%      \draw (8,0.75) rectangle (8.8,1.55) node[pos=0.5] {$T^\text{quiz}$};
%      \draw (8,1.85) rectangle (8.9,2.75) node[pos=0.5] {$C^{\text{quiz},k}$};
%      %%%% arrows
%      \draw[thick,->] (0.8,0.4) to node[pos=0.5, above]{$\sketchmap$} (1.3,0.4);
%      \draw[thick,->] (0.8,1.9) to node[pos=0.5, above]{$\sketchmap$} (1.3,1.9);
%      %%%%%%%%%%% arrows inside and outside Sketch space
%      \draw[thick,->] (5.8,1.3) to node[pos=0.5, above]{$\sketchmap^{-1}$} (6.5,1.3);
%      \draw[thick,->] (2.3,0.5) to node[pos=0.5, above]{} (2.8,0.95); % slant arrow inside
%      \draw[thick,->] (2.3,2) to node[pos=0.5, above]{} (2.9,1.5);% slant arrow inside
%      \draw[thick,->] (4.5,1.25) to node[pos=0.5, above]{} (4.8,1.25);
%      \draw[thick,->] (7.3,1.25) to node[pos=0.5, above]{} (8,1.25);
% 	\draw[thick,->] (7,1.55) to node[pos=0.25, above]{\exposureparam} (8,2.35);
%      %%%% surrounding box
%      %\draw (-1.4,-0.25) rectangle (10.1,2.9);
% \end{tikzpicture}
% \vspace{-1mm}
% \subcaption{Our algorithm: \algmultihop}
% \label{fig:pipeline.abstract}
% \end{minipage}% pipeline abstract
% \\
% \begin{minipage}{1\textwidth}
% \begin{minipage}{0.13\textwidth}
% \begin{boxcode}{3.48cm}{0.70}{0.95}
% 				\textcode{def }\DSLRun\textcode{()\{}\\
% 				\quad \DSLRUntil\textcode{(}\DSLBoolGoal\textcode{)\{}\\
%                 \quad \quad
%               \DSLIf\textcode{(}\SDSLBool\textcode{)\{\}}\\
% 				\quad \quad 
% 				\DSLElse\textcode{\{}\\
% 				\quad \quad \quad \DSLIf\textcode{(}\SDSLBool\textcode{)\{\}}\\
% 				\quad \quad \quad
% 				\DSLElse\textcode{\{\}}\\
% 				\quad \quad \textcode{\}}\\
% 				%\vspace{-4.5mm}
% 				\quad \textcode{\}}\\
% 				%\vspace{-6mm}				
% 				\textcode{\}}
% 			\end{boxcode}
% \vspace{-3.3mm}
% \subcaption{$S^{\text{in,}\star}$}
% \label{fig:pipeline.solsketch}
% \end{minipage} % end of first sketch box
% %
% \hspace{3mm}
% %
% \begin{minipage}{0.65\textwidth}
% \begin{minipage}{1\textwidth}
% \begin{boxcode}{1.9cm}{0.80}{1.0}
% 				\textcode{def }\DSLRun\textcode{()\{\}}
% \end{boxcode}
% \vspace{-3mm}
% \subcaption{$S^{\text{in,stu}}$}
% \label{fig:pipeline.stusketch}
% \end{minipage}\\
% \vspace{10mm}
% \begin{minipage}{1\textwidth}
% \begin{boxcode}{2.8cm}{0.70}{1.0}
% 				\textcode{def }\DSLRun\textcode{()\{}\\
% 				\quad \DSLRUntil\textcode{(}\DSLBoolGoal\textcode{)\{\}}\\
% 				\textcode{\}}
% \end{boxcode}
% \vspace{-3mm}
% \subcaption{$S^{\text{quiz}}$}
% \label{fig:pipeline.sketchhint}
% \end{minipage}
% \end{minipage} % end of second-third sketch
% %
% \hspace{-15mm}
% %
% \begin{minipage}{0.15\textwidth}
% \begin{boxcode}{3.38cm}{0.70}{1.0}
% 				\textcode{def }\DSLRun\textcode{()\{}\\
% 				\quad \DSLMove\\
% 			   \quad \DSLTurnLeft\\
% 			   \quad
% 			   \DSLRepeatUntil\textcode{(}\DSLBoolGoal\textcode{)\{}\\
% 				\quad \quad \DSLMove\\
% 				\quad \textcode{\}}\\
% 				\textcode{\}}
% \end{boxcode}
% \vspace{-3mm}
% \subcaption{$C^{\text{quiz}}$}
% \label{fig:pipeline.codequiz}
% \end{minipage}% end of fourth sketch
% \end{minipage} % end of sketch boxes minipage
% \vspace{-2.5mm}
% \caption{ (a) illustrates \algmultihop~. In particular, we can instantiate the presented algorithm using input task~$T^\text{in}$, its solution code \solutioncode, and current student attempt \studentcode~from Fig.~\ref{fig:intro}. The sketch of \solutioncode~is shown in (b), sketch of \studentcode~is shown in (c), sketch of $C^\text{quiz}$ is shown in (d), and the code of the pop quiz is shown in (e). In all the sketches \DSLRepeatUntil is abbreviated as \DSLRUntil.}
% \label{fig:pipeline}
% \end{minipage} % end figure 2
% \hfill
% %%%%%%%%%%%%%%%%% Begin the minipage for the abstract diagram of the quiz sketch
% % \hspace{5mm}
% % \begin{minipage}{0.32\textwidth}% figure 3
% % \centering{
% % \begin{tikzpicture}[scale=0.75, every node/.style={transform shape}]
% %     \draw[pattern=dots, pattern color=gray!40] (0,0.15) rectangle (5.5,7.5) node[pos=0.5] {};
% %     \node[] at (5,6.5) {\textbf{\sketchspace}};
% %     %% add the node for the student sketch
% %     \node[circle, fill=black, minimum size=0.15cm, inner sep=0pt, outer sep=0pt, label=right:{$S^{\text{in,stu}}$}] (nstu) at (2.75,4.7) {};
% %     %%%%% add the node for the solution sketch
% %     \node[circle, fill=black, minimum size=0.15cm, inner sep=0pt, outer sep=0pt, 
% %     label=right:{$S^{\text{in,}\star}$}] (nsol) at (2.75,0.5) {};
% %     %%%%%% draw the neighborhood ovals
% %     \draw (2.75,4.5) ellipse (1.5cm and 1.5cm); % nbd-1
% %      %%%%%% draw the neighborhood ovals
% %     \draw (2.75,4.2) ellipse (2cm and 2.5cm); % nbd-2
% %       %%%%%% draw the neighborhood ovals
% %     \draw (2.75,3.8) ellipse (2.4cm and 3.6cm); % nbd-3
% %     %%%%%%% add the nodes for the substructures
% %     \draw (3,3.9) node[cross=4] {}; % repeat_until
% %     \node [outer sep=1pt,fill=white,above] at (2.8,3.2) {\footnotesize{\{\DSLRun\{\DSLRUntil{}}\}\}};
    
% %     \draw (3.2,2.6) node[cross=4] {}; %repeat_until;if_else
% %      \node [outer sep=1pt,fill=white,above] at (3.1,1.8) {\footnotesize{\{\DSLRun\{\DSLRUntil\{\DSLIfElse{}\}}\}\}};
% %     %%%%%%% Add the labels showing the hops
% %     \node [outer sep=1pt,fill=white,above] at (2.75,5.25) {$l=1$};
% %      \node [outer sep=1pt,fill=white,above] at (2.75,6) {$l=2$};
% %       \node [outer sep=1pt,fill=white,above] at (2.75,6.8) {$l=3$};
% % \end{tikzpicture}
% % }
% % \vspace{-2.8mm}
% % \caption{
% % Stage 1 of \algmultihop~for the scenario shown in Fig.~\ref{fig:intro}. $\textnormal{X}$ shows other substructures of $S^{\text{in,}\star}$. See further details in Section~\ref{sec:model.stage1}.
% % % $\textnormal{X}$ shows other substructures of $S^{\text{in,}\star}$ in $l$-hop neighborhoods of $S^\text{in,stu}$. $S^\text{quiz}$'s selection is detailed in 
% % }
% % \label{fig.approach.algmultihop}
% % \end{minipage}
% \vspace{-3mm}
% \end{figure*}
