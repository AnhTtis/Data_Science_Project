\section{Gradient formulations}
\noindent
Here we provide the formulations of the manually derived gradient expressions which are used for backpropagation. For each, we provide the Jacobians ($J_\cdot$) with respect to its input variables, left-multiplied by the gradient ($u$) of its output.

\subsection{M\"obius addition}
\noindent
The M\"obius addition operation is defined as
\begin{equation}
    x \oplus_c y = \frac{(1 + 2c \langle x, y \rangle + c ||y||^2) x + (1 - c ||x||^2) y}{1 + 2c \langle x, y \rangle + c^2 ||x||^2 ||y||^2}.
\end{equation}
Its Jacobians, left-multiplied by the output gradient, can be written as
\begin{equation}\label{eq:mob_add_grad_x}
    u^T J_x (x \oplus_c y) = \frac{a}{d} u^T - \frac{2c}{d} \Big( u^T y + \frac{\theta c ||y||^2}{d} \Big) x^T + \frac{2c}{d} \Big( u^T x - \frac{\theta}{d} \Big) y^T,
\end{equation}
\begin{equation}\label{eq:mob_add_grad_y}
    u^T J_y (x \oplus_c y) = \frac{b}{d} u^T + \frac{2c}{d} \Big( u^T x - \frac{\theta}{d} \Big) x^T + \frac{2c}{d} \Big( u^T x - \frac{c ||x||^2 \theta}{d} \Big) y^T,
\end{equation}
where
\begin{equation}
    a = 1 + 2c \langle x, y \rangle + c ||y||^2,
\end{equation}
\begin{equation}
    b = 1 - c ||x||^2,
\end{equation}
\begin{equation}
    d = 1 + 2c \langle x, y \rangle + c^2 ||x||^2 ||y||^2,
\end{equation}
\begin{equation}
    \theta = a u^T x + b u^T y.
\end{equation}

\subsection{Exponential map at the origin}
\noindent
The exponential map at the origin is given by
\begin{equation}
    \exp_0^c (v) = \tanh(\sqrt{c} ||v||) \frac{v}{\sqrt{c} ||v||}.
\end{equation}
Its Jacobian, left-multiplied by the output gradient, can be written as
\begin{equation}
    u^T J_v \exp_0^c (v) = u^T v \Big( \frac{1}{||v||^2 \cosh(\sqrt{c} ||v||)^2} - \frac{\tanh(\sqrt{c} ||v||)}{\sqrt{c} ||v||^3} \Big) v^T + \frac{\tanh(\sqrt{c} ||v||)}{\sqrt{c} ||v||} u^T.
\end{equation}

\subsection{Logarithmic map at the origin}
\noindent
The logarithmic map at the origin is given by
\begin{equation}
    \log_0^c (y) = \tanh^{-1}(\sqrt{c} ||y||) \frac{y}{\sqrt{c} ||y||}.
\end{equation}
Its Jacobian, left-multiplied by the output gradient, can be written as
\begin{equation}
    u^T J_y \log_0^c (y) = u^T y \Big( \frac{1}{||y||^2 (1 - c ||y||^2)} - \frac{\tanh^{-1} (\sqrt{c} ||y||)}{\sqrt{c} ||y||^3} \Big) y^T + \frac{\tanh^{-1}(\sqrt{c} ||y||)}{\sqrt{c} ||y||} u^T.
\end{equation}

\subsection{Exponential map}
\noindent
The exponential map at $x$ is defined as 
\begin{equation}
    \exp_x^c (v) = x \oplus_c \Big(\tanh\big(\frac{\sqrt{c} \lambda_x^c ||v||}{2}\big) \frac{v}{\sqrt{c} ||v||}\Big),
\end{equation}
which we can reformulate as
\begin{equation}
    \exp_x^c (v) = x \oplus_c z_c (x, v),
\end{equation}
where
\begin{equation}
    z_c (x, v) = \tanh\big(\frac{\sqrt{c} \lambda_x^c ||v||}{2}\big) \frac{v}{\sqrt{c} ||v||}.
\end{equation}
Now we can backpropagate through this operation in two steps. First, the Jacobians of $z_c$, left-multiplied by the output gradient, can be written as
\begin{equation}
    u^T J_x z_c (x, v) = \frac{2c u^T v}{\cosh(\frac{\sqrt{c} ||v||}{1 - c ||x||^2})^2 (1 - c ||x||^2)^2} x^T,
\end{equation}
\begin{equation}
    u^T J_v z_c (x, v) = u^T v \Big( \frac{1}{||v||^2 \cosh (\frac{\sqrt{c} ||v||}{1 - c||x||^2})^2 (1 - c ||x||^2)^2} - \frac{\tanh(\frac{\sqrt{c} ||v||}{1 - c ||x||^2})}{\sqrt{c} ||v||^3} \Big) v^T + \frac{\tanh(\frac{\sqrt{c}||v||}{1 - c ||x||^2})}{\sqrt{c} ||v||} u^T.
\end{equation}
Next, for backpropagating through the M\"obius addition, we can use the expressions given in equations (\ref{eq:mob_add_grad_x}, \ref{eq:mob_add_grad_y}).

\subsection{Logarithmic map}
\noindent
The logarithmic map at $x$ is defined as
\begin{equation}
    \log_x^c (y) = \frac{2}{\sqrt{c} \lambda_x^c} \tanh^{-1} \big(\sqrt{c} ||-x \oplus_c y||\big) \frac{-x \oplus_c y}{||-x \oplus_c y||},
\end{equation}
which we can reformulate as
\begin{equation}
    \log_x^c (y) = f_c(x, z_c (x, y)) = \frac{2}{\sqrt{c} \lambda_x^c} \tanh^{-1} \big(\sqrt{c} ||z_c (x, y)||\big) \frac{z_c (x, y)}{||z_c (x, y)||},
\end{equation}
where
\begin{equation}
    z_c (x, y) = -x \oplus_c y.
\end{equation}
Again, we can backpropagate through this operation in two steps. First, we backpropagate through $z_c (x, y)$ using equations (\ref{eq:mob_add_grad_x}, \ref{eq:mob_add_grad_y}). Then, the Jacobians of $f_c (x, z)$, left-multiplied by the output gradient, can be written as
\begin{equation}
    u^T J_x f_c (x, z) = - \tanh^{-1} (\sqrt{c} ||z||) \frac{2c u^T z}{\sqrt{c} ||z||} x^T,
\end{equation}
\begin{equation}
    u^T J_z f_c (x, z) = u^T z \Big( \frac{1 - c ||x||^2}{(1 - c ||z||^2) ||z||^2} - \tanh^{-1}(\sqrt{c} ||z||) \frac{1 - c ||x||^2}{\sqrt{c} ||z||^3} \Big) z^T + \tanh^{-1}(\sqrt{c} ||z||) \frac{1 - c||x||^2}{\sqrt{c}  ||z||} u^T.
\end{equation}

\subsection{Conformal factor}
\noindent
The conformal factor is given as
\begin{equation}
    \lambda_x^c = \frac{2}{1 - c||x||^2}.
\end{equation}
Its Jacobian, multiplied by the output gradient (which is a scalar here), can be written as
\begin{equation}
    u J_x \lambda_x^c = \frac{4c u}{(1 - c||x||^2)^2} x^T.
\end{equation}

\subsection{Projection onto the Poincar\'e ball}
\noindent
An operation that is often applied in hyperbolic geometry, but rarely mentioned, is projection onto the Poincar\'e ball. This operation can be used to ensure numerical stability. It is defined as
\begin{equation}
    \text{Proj}_c (x) = x \mathbbm{1}_{\{c ||x||^2 < 1\}} (x) + \frac{x}{\sqrt{c}||x||} \mathbbm{1}_{\{c||x||^2 > 1\}} (x),
\end{equation}
where $\mathbbm{1}_A (x)$ is the indicator function, which is 1 if $x \in A$ and 0 if $x \notin A$. The Jacobian of this projection operation, left-multiplied by the output gradient, can be computed as
\begin{equation}
    u^T J_x \text{Proj}_c (x) = \Big( \mathbbm{1}_{\{c ||x||^2 < 1\}} (x) + \frac{1}{\sqrt{c} ||x||} \mathbbm{1}_{\{c||x||^2 > 1\}} (x) \Big) u^T - \Big( \frac{u^T x}{\sqrt{c} ||x||^3} \mathbbm{1}_{\{c||x||^2 > 1\}}(x) \Big) x^T.
\end{equation}