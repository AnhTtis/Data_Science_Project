\documentclass[10pt,twocolumn,letterpaper]{article}

\usepackage{iccv}
\usepackage{times}
\usepackage{epsfig}
\usepackage{graphicx}
\usepackage{amsmath}
\usepackage{amssymb}
\usepackage{bbm}
\usepackage{mathtools}
\usepackage{booktabs}
\usepackage{xcolor,colortbl}
\usepackage{algorithm,algorithmic}
\usepackage{caption, subcaption}
\usepackage{comment}
\usepackage{multirow}
\usepackage{authblk}
\usepackage{bm}
\usepackage[accsupp]{axessibility}  %

\usepackage{svg}


\usepackage[pagebackref=true,breaklinks=true,letterpaper=true,colorlinks,bookmarks=false]{hyperref}

\definecolor{Gray}{gray}{0.9}
\newcommand{\theHalgorithm}{\arabic{algorithm}}
\newcommand{\psmm}[1]{\textcolor{orange}{[\textbf{PM}: #1]}}

\renewcommand\Authands{, }
\renewcommand\Affilfont{\normalsize}

\iccvfinalcopy %

\def\iccvPaperID{10230} %
\def\httilde{\mbox{\tt\raisebox{-.5ex}{\symbol{126}}}}

\ificcvfinal\pagestyle{empty}\fi

\begin{document}

\title{Poincar\'e ResNet}

\author[1]{Max van Spengler}
\author[2]{Erwin Berkhout}
\author[1]{Pascal Mettes}

\affil[1]{
    VIS Lab\\
    Informatics Institute\\
    University of Amsterdam\\
}
\affil[2]{
    Department of Oral Radiology\\
    Academic Center for Dentistry\\
    University of Amsterdam \& VU Amsterdam
}

\maketitle
\ificcvfinal\thispagestyle{empty}\fi


\begin{abstract}
This paper introduces an end-to-end residual network that operates entirely on the Poincar\'e ball model of hyperbolic space. Hyperbolic learning has recently shown great potential for visual understanding, but is currently only performed in the penultimate layer(s) of deep networks. All visual representations are still learned through standard Euclidean networks. In this paper we investigate how to learn hyperbolic representations of visual data directly from the pixel-level. We propose Poincar\'e ResNet, a hyperbolic counterpart of the celebrated residual network, starting from Poincar\'e 2D convolutions up to Poincar\'e residual connections. We identify three roadblocks for training convolutional networks entirely in hyperbolic space and propose a solution for each: (i) Current hyperbolic network initializations collapse to the origin, limiting their applicability in deeper networks. We provide an identity-based initialization that preserves norms over many layers. (ii) Residual networks rely heavily on batch normalization, which comes with expensive Fr\'echet mean calculations in hyperbolic space. We introduce Poincar\'e midpoint batch normalization as a faster and equally effective alternative. (iii) Due to the many intermediate operations in Poincar\'e layers,
the computation graphs of deep learning libraries blow up, limiting our ability to train on deep hyperbolic networks. We provide manual backward derivations of core hyperbolic operations to maintain manageable computation graphs.
\end{abstract}

\section{Introduction}
\section{Introduction}


Recent years have witnessed the rise of human digitization~\cite{habermannDeepCapMonocularHuman2020,alexanderCREATINGPHOTOREALDIGITAL,pengNeuralBodyImplicit2021,alldieckDetailedHumanAvatars2018, rajANRArticulatedNeural2020}. This technology greatly impacts the entertainment, education, design, and engineering industry.
There is a well-developed industry solution for this task.
High-fidelity reconstruction of humans can be achieved either with full-body laser scans~\cite{saitoSCANimateWeaklySupervised2021}, dense synchronized multi-view cameras~\cite{xiangModelingClothingSeparate2021a,xiangDressingAvatarsDeep2022a}, or light stages~\cite{alexanderCREATINGPHOTOREALDIGITAL}.
However, these settings are expensive and tedious to deploy and consist of a complex processing pipeline, preventing the technology's democratization.

Another solution is to view the problem as inverse rendering and learn digital humans directly from custom-collected data.
Traditional approaches directly optimize explicit mesh representation~\cite{loperSMPLSkinnedMultiperson2015, fangRMPERegionalMultiperson2018, pavlakosExpressiveBodyCapture2019} which suffers from the problems of smooth geometry and coarse textures~\cite{prokudinSMPLpixNeuralAvatars2020,alldieckVideoBasedReconstruction2018}. Besides, they require professional artists to design human templates, rigging, and unwrapped UV coordinates.
Recently, with the help of volumetric-based implicit representations~\cite{mildenhallNeRFRepresentingScenes2020, parkDeepSDFLearningContinuous2019, meschederOccupancyNetworksLearning2019} and neural rendering~\cite{laineModularPrimitivesHighPerformance2020, liuSoftRasterizerDifferentiable2019, thiesDeferredNeuralRendering2019}, 
one can easily digitize a quality-plausible human avatar from video footage~\cite{jiangNeuManNeuralHuman2022,wengHumanNeRFFreeviewpointRendering}.
Particularly, volumetric-based implicit representations~\cite{mildenhallNeRFRepresentingScenes2020, pengNeuralBodyImplicit2021} can reconstruct scenes or objects with much higher fidelity against previous neural renderer~\cite{thiesDeferredNeuralRendering2019,prokudinSMPLpixNeuralAvatars2020}, and is more user-friendly as it does not need any human templates, pre-set rigging, or UV coordinates.
Captured visual footage and corresponding skeleton tracking are enough for training.
However, better reconstructions and more friendly usability are at the expense of the following factors.
1) \textbf{Inefficiency:}
They require longer optimization times (typically tens of hours or days) and inference slowly.
Volume rendering~\cite{mildenhallNeRFRepresentingScenes2020,lombardiNeuralVolumesLearning2019} formulates images by querying the densities and colors of millions of spatial coordinates. 
In the training stage, due to memory constraints, only a small fraction of points are sampled which leads to slow convergence speed.
2) \textbf{Entangled representations}:
The geometry, materials, and motion dynamics are entangled in the neural networks. 
Due to the implicit nature of neural nets, one can hardly edit one property without touching the others~\cite{yuanNeRFEditingGeometryEditing2022a,liuEditingConditionalRadiance2021}.
3) \textbf{Graphics incompatibility}:
Volume rendering is incompatible with the current popular graphic pipeline,
which renders triangular/quadrilateral meshes efficiently with the rasterization technique.
Many downstream applications require mesh rasterization in their workflow (\eg, editing~\cite{foundationBlenderOrgHome}, simulation~\cite{benderPositionBasedSimulationMethods2015}, real-time rendering~\cite{akenine2019real}, ray-tracing~\cite{waldRTXRayTracing}).
Although there are approaches~\cite{lorensenMarchingCubesHigh,labelleIsosurfaceStuffingFast2007} can convert volumetric fields into meshes, the gaps from discrete sampling degrade the output quality in terms of both meshes and textures.


To address these issues, we present \textbf{EMA}, a method based on \textbf{E}fficient \textbf{M}eshy neural fields to reconstruct animatable human \textbf{A}vatars.
Our method enjoys flexibility from implicit representations and efficiency from explicit meshes, yet still maintains high-fidelity reconstruction quality.
Given video sequences and the corresponding pose tracking, our method digitizes humans in terms of canonical triangular meshes, physically-based rendering (PBR) materials, and skinning weights \textit{w.r.t.} skeletons.
We jointly learn the above components via inverse rendering~\cite{laineModularPrimitivesHighPerformance2020,chenDIBRLearningPredict2021,chenLearningPredict3D2019} in an end-to-end manner.
Each of them is derived from a separate neural field, which relaxes the requirements of a preset human template, rigging, or UV coordinates.
Specifically, we predict a canonical mesh out of a signed distance field (SDF) by differentiable marching tetrahedra~\cite{shenDeepMarchingTetrahedra2021,gaoGET3DGenerativeModel,gaoLearningDeformableTetrahedral2020,munkbergExtractingTriangular3D2022}, then we extend the marching tetrahedra~\cite{shenDeepMarchingTetrahedra2021} for spatial-varying materials by utilizing a neural field to predict PBR materials \textit{on the mesh surfaces} after rasterization~\cite{munkbergExtractingTriangular3D2022,hasselgrenShapeLightMaterial2022,laineModularPrimitivesHighPerformance2020}.
To make the canonical mesh animatable, we take another neural field to model the forward linear blend skinning for the meshes. 
Given a posed skeleton, the canonical mesh is then transformed into the corresponding poses.
Finally, we shade the mesh with a rasterization-based differentiable renderer~\cite{laineModularPrimitivesHighPerformance2020} and train our models with a photo-metric loss.
After training, we export the mesh with materials and discard the neural fields.

\looseness=-1
There are several merits of our method design.
1) \textbf{Efficiency}:
Powered by efficient mesh rendering, our method can render in real-time.
Besides, the training speed is boosted as well, 
since we compute loss holistically on the whole image and the gradients only flow on the mesh surface. In contrast, volume rendering takes limited pixels for loss computation and back-propagates the gradients in the whole space.
Our method only needs about an hour of training and minutes of optimization are enough for plausible avatar reconstruction.
2) \textbf{Disentangled representations}:
Our shape, materials, and motion modules are disentangled naturally by design, which facilitates editing. 
Besides, Canonical meshes with forward skinning modeling handle the out-of-distribution poses better.
3) \textbf{Graphics compatibility}:
Our derived mesh representation is compatible with 
the prominent graphic pipeline, which leads to instant downstream applications (\eg, the shape and materials can be edited directly in design software~\cite{foundationBlenderOrgHome}).
To further improve reconstruction quality, we additionally optimize image-based environment lights and non-rigid motions.


We conduct extensive experiments on standards benchmarks H36M~\cite{ionescuHuman36MLarge2014b} and ZJU-MoCap~\cite{pengNeuralBodyImplicit2021}.
Our method achieves very competitive performance for novel view synthesis, generalizes better for novel poses, 
and significantly improves both training time and inference speed against previous arts.
Our research-oriented code reaches real-time inference speed (100+ FPS for rendering $512\times512$ images).
We in addition showcase applications including novel pose synthesis, material editing, and relighting.

\section{Background and related work}
\subsection{Poincar\'e ball model of hyperbolic space}
This paper operates on the most commonly used model of hyperbolic geometry in deep learning, namely the Poincar\'e ball model. We will therefore restrict the background discussion to this model and refer to Peng \etal~\cite{peng2021} for a more comprehensive discussion on the different isometric models of hyperbolic space.
The $n$-dimensional Poincar\'e ball model with constant negative curvature $-c$ is defined as the Riemannian manifold $(\mathbb{B}_c^n, \mathfrak{g}_c)$, where 
\begin{equation}
    \mathbb{B}_c^n = \{x \in \mathbb{R}^n : ||x||^2 < \frac{1}{c}\},
\end{equation}
and where
\begin{equation}
    \mathfrak{g}_c = \lambda_x^c I_n, \quad \lambda_x^c = \frac{2}{1 - c ||x||^2},
\end{equation}
with $I_n$ being the $n$-dimensional identity matrix. 
The Poincar\'e ball model can be turned into a gyrovector space \cite{ungar2009} by endowing it with M\"obius addition and M\"obius scalar multiplication, respectively defined as
\begin{equation}
\begin{split}
    x \oplus_c y = & \frac{(1 + 2c \langle x, y \rangle + c ||y||^2) x + (1 - c ||x||^2) y}{1 + 2c \langle x, y \rangle + c^2 ||x||^2 ||y||^2},\\
    r \otimes_c x = &\frac{1}{\sqrt{c}} \tanh \big(r \tanh^{-1} (\sqrt{c} ||x||)\big) \frac{x}{||x||},
\end{split}
\end{equation}
where $x, y \in \mathbb{B}_c^n$, $r \in \mathbb{R}$ and where $||\cdot||$ and $\langle \cdot, \cdot \rangle$ denote the Euclidean norm and inner product, respectively. An important map related to this gyrovector space is the gyrator $\text{gyr} : \mathbb{B}_c^n \times \mathbb{B}_c^n \rightarrow \text{Aut}(\mathbb{B}_c^n, \oplus_c)$, where $\text{Aut}(\mathbb{B}_c^n, \oplus_c)$ denotes the set of automorphisms on $\mathbb{B}_c^n$ \cite{ungar2009}. This map is implicitly defined as
\begin{equation}
    \text{gyr}[x, y] z = - (x \oplus_c y) \oplus_c \big(x \oplus_c (y \oplus_c z)\big),
\end{equation}
where $x, y, z \in \mathbb{B}_c^n$, which can be used to measure the extent to which M\"obius addition deviates from commutativity. It will be used later on to define parallel transport. Furthermore, we can compute the distance between any two points $x, y \in \mathbb{B}_c^n$ as
\begin{equation}
    d_c(x, y) = \frac{2}{\sqrt{c}} \tanh^{-1} (\sqrt{c} ||-x \oplus_c y||).
\end{equation}
For an in-depth analysis of this gyrovector space approach to the Poincar\'e ball see \cite{ungar2009}.
Using the definition of M\"obius addition, the exponential and logarithmic maps can be written as \cite{ganea2018}
\begin{equation*}
\begin{split}
    \exp_x^c (v) &= x \oplus_c \Big(\tanh\big(\frac{\sqrt{c} \lambda_x^c ||v||}{2}\big) \frac{v}{\sqrt{c} ||v||}\Big),\\
    \log_x^c (y) &= \frac{2}{\sqrt{c} \lambda_x^c} \tanh^{-1} \big(\sqrt{c} ||-x \oplus_c y||\big) \frac{-x \oplus_c y}{||-x \oplus_c y||},
\end{split}
\end{equation*}
where $x, y \in \mathbb{B}_c^n$ and $v \in \mathcal{T}_x \mathbb{B}_c^n$. Moreover, we can define parallel transport $P_{x \rightarrow y}^c : \mathcal{T}_x \mathbb{B}_c^n \rightarrow \mathcal{T}_y \mathbb{B}_c^n$ as follows \cite{shimizu2021}
\begin{equation}
    P_{x \rightarrow y}^c (v) = \frac{\lambda_x^c}{\lambda_y^c} \text{gyr}[y, -x]v,
\end{equation}
which allows us to transport a tangent vector at a point $x \in \mathbb{B}_c^n$ to the tangent space at another point $y \in \mathbb{B}_c^n$, used for example in batch normalization.


\subsection{The Poincar\'e ball model in neural networks}
\label{subsect:poincare_nn}
To perform deep learning on the Poincar\'e ball model, Ganea \etal~\cite{ganea2018} outline a theoretical framework for incorporating this model into core layers of neural networks, such as hyperbolic logistic regression, hyperbolic fully-connected, and hyperbolic recurrent layers.
More recently, Shimizu \etal~\cite{shimizu2021} made important improvements to this framework to ensure that the hyperbolic geometry was fully taken advantage of without the need for additional learnable parameters. We will therefore use this work as a starting point for the rest of this paper and provide a short overview here.

As a foundation, Poincar\'e multinomial logistic regression is defined by computing the score for each of $n$ classes for some input $x \in \mathbb{B}_c^m$ as
\begin{align*}
    v_k (x) = \frac{2}{\sqrt{c}} ||z_k|| \sinh^{-1} \Big(\lambda_x^c \langle \sqrt{c} x, \frac{z_k}{||z_k||} \rangle \cosh(2 \sqrt{c} r_k) \\
    - (\lambda_x^c - 1) \sinh(2 \sqrt{c} r_k)\Big),
\end{align*}
where $z_k \in \mathcal{T}_0 \mathbb{B}_c^m = \mathbb{R}^m$ and $r_k \in \mathbb{R}$ are the parameters for the $k$-th class. These scores are equivalent to the distances between the input $x$ and the $n$ different Poincar\'e hyperplanes determined by the parameters $\{(z_k, r_k)\}_{i=1}^n$. Here, $z_k$ determines the orientation of the hyperplane while $r_k$ determines its offset with respect to the origin. A Poincar\'e fully connected layer mapping input $x \in \mathbb{B}_c^m$ to $\mathbb{B}_c^n$ is in turn defined as
\begin{equation}\label{eq:FC_layer}
    y = \mathcal{F}^c (x; Z, r) = \frac{w}{1 + \sqrt{1 + c ||w||^2}},
\end{equation}
with 
\begin{equation}
    w = \Big(\frac{1}{\sqrt{c}} \sinh(\sqrt{c} v_k (x))\Big)_{k=1}^n,
\end{equation}
where the $v_k (\cdot)$ are the scores from the Poincar\'e multinomial logistic regression and where $Z = [z_1 | \ldots | z_n] \in (\mathcal{T}_0 \mathbb{B}_c^m)^n = \mathbb{R}^{m \times n}$ and $r = (r_k)_{k=1}^n \in \mathbb{R}^m$ are the parameters of the layer. Given hyperbolic fully connected layers, Shimizu \etal~\cite{shimizu2021} outline general formulations for self-attention and convolutional operations in hyperbolic space. We take such investigations to the visual domain and arrive at Poincar\'e ResNets, which require 2D convolutions, fast batch normalization, residual blocks, norm-preserving initialization and derived backpropagation of core operations in order to be realized.

\subsection{Hyperbolic learning in computer vision}
Khrulkov \etal~\cite{khrulkov2020hyperbolic} have shown that both image data and labels contain hierarchical structures and introduced Hyperbolic Image Embeddings to exploit these observations. In their approach, embeddings of images obtained through standard networks are mapped to hyperbolic space, followed by a final classification layer based on hyperbolic logistic regression or hyperbolic prototypical learning, directly improving few-shot learning and uncertainty quantification.

A wide range of works have further investigated hyperbolic embeddings of images and videos. Several works have proposed prototypes-based hyperbolic embeddings for few-shot learning \cite{fang2021kernel,gao2021curvature,guo2022clipped,ma2022adaptive,zhang2022hyperbolic}, where hyperbolic space consistently outperforms Euclidean space. Hyperbolic embeddings of classes based on their hierarchical relations has also shown to be effective for zero-shot learning \cite{liu2020hyperbolic,xu2022meta} and hierarchical recognition \cite{dhall2020hierarchical,ghadimi2021hyperbolic,long2020searching,yu2022skin}. Hyperbolic embeddings have furthermore been effective in metric learning \cite{ermolov2022hyperbolic,zhang2021learning}, object detection \cite{valada2022hyperbolic}, image segmentation \cite{chen2022hyperbolic,atigh2022hyperbolic} and future prediction in videos~\cite{suris2021learning}. 

In generative learning, hyperbolic variational auto-encoders \cite{hsu2021capturing,mathieu2019continuous,nagano2019wrapped}, generative adversarial networks \cite{lazcano2021hgan} and normalizing flows \cite{bose2020latent,mathieu2020riemannian} have been shown to obtain competitive results in data-constrained settings. A number of recent works have proposed unsupervised hyperbolic learning approaches \cite{hsu2021capturing,monath2019gradient,weng2021unsupervised,yan2021unsupervised}, allowing for learning and discovering hierarchical representations.

This body of literature highlights that hyperbolic geometry is fruitful for visual understanding. In current literature, however, hyperbolic learning is restricted to the final embedding layers, with all visual representations being learned by standard networks. This paper strives to learn hyperbolic representations in an end-to-end manner, from pixels to labels, complementing current research on computer vision with hyperbolic embeddings.

\section{Poincar\'e residual networks for images}\label{sect:method}
\section{Methods}



This paper aims to utilize pre-trained diffusion generation models for downstream tasks by proposing a two-stage synthesis-exploitation framework. 
In \secref{Problem Formulation}, we start by describing the preliminary. 
In \secref{Synthesizing Labeled Data}, we detail the synthesis stage to generate sufficient labeled data. 
In \secref{Diffusion Features and Synthetic Data Supervision}, we detail the exploitation stage to close the structural gap between generative models and discriminative tasks.










\subsection{Preliminary and Overview}
\label{Problem Formulation}
{\noindent \bf Problem Definition.} Object Discovery (OD), \ie, saliency segmentation and object localization, as a fundamental and typical discriminative task, is studied in this paper. 
Concretely, object discovery aims to train one pixel-level segmentation model $\Phi_{\mathrm{OD}}$ that partitions one image $\mathcal{I}$ into two disjoint groups, namely, foreground and background.
\vspace{-0.5em}
\begin{equation}
    \mathcal{M}_{\mathrm{seg}} = \Phi_{\mathrm{OD}}(\mathcal{I}) \in\{0,1\}^{H \times W \times 1}, \  \mathcal{I} \in \mathbb{R}^{H \times W \times 3}, 
\vspace{-0.5em}
\end{equation}
where $\mathcal{M}_{\mathrm{seg}}$ refers to the binary segmentation mask. 

Here, to clearly evaluate the effectiveness of our method, we focus on the strict \textit{unsupervised} setting, {\em i.e.}, the model is trained \textit{without} any manually annotated data. 


\vspace{0.1cm}
{\noindent \bf Motivation.} 
\hspace{1pt} This paper aims to exploit pixel-level visual knowledge from pre-trained diffusion generation models, for downstream discriminative tasks, \eg, OD. To achieve this goal, we design a novel synthesis-exploitation framework (\figref{fig:framework}). Specifically, at the synthesis stage, we explicitly construct one free (infinite-size) discriminative synthetic dataset, to obtain sufficient labeled samples. At the exploitation stage, we enable diffusion to be compatible with OD tasks, by extracting implicit diffusion features, and training one discovery decoder with the synthetic dataset.

























\vspace{0.2cm}
{\noindent \bf Diffusion}~\cite{sohl2015deep,ddpm} is one recently popular generative idea, containing forward and reverse processes. 
The \textit{forward process} is a Markov chain where noise is gradually added to the data.
The \textit{reverse process} is a denoising procedure that can be decomposed into a linear combination of a noisy image $\boldsymbol{x}_t$ and a noise approximator $\epsilon_\theta(\cdot)$. $t=1,\dots,T$ refers to the denoising timesteps.
The key to diffusion models is to learn the function $\epsilon_\theta(\cdot)$, typically using a UNet~\cite{ronneberger2015u}.


Particularly, we build on a variant of the text-to-image diffusion model, namely, Stable Diffusion~\cite{ldm}. During the synthesis process, it's sampled by iteratively denoising $\boldsymbol{x}_t$ conditioned on the input text prompt $y$ for timestep $t=1,\dots,T$. The conditional denoising UNet $\boldsymbol\epsilon_\theta(\boldsymbol{x}_t, t, y)$ stacks layers of self- and cross-attentions. 
$y$ is first encoded to text embeddings by a pre-trained text encoder, then text embeddings are mapped to intermediate layers as $K$ and $V$ via the attention mechanism, and the noisy image $\boldsymbol{x}_t$ is mapped as $Q$. 
For step $t$ and layer $l$, we call cross-attention as $\mathcal{A}_c^{t,l}$, self-attention as $\mathcal{A}_s^{t,l}$, and intermediate features as $\mathcal{F}^{t,l}$.
{\bf Note that}, this paper freezes Stable Diffusion pre-trained on LAION-5B~\cite{schuhmann2022laion} (5 billion image-text pairs), as a knowledge provider. This diffusion model involves both low-level object details and high-level class semantics, enabling us to achieve unsupervised object discovery.

































\subsection{Synthesis Stage:  Free Data Generation}
\label{Synthesizing Labeled Data}
As illustrated in \figref{fig:framework} (1), this stage aims to synthesize large and free image-mask pairs through Stable Diffusion, solving the lack of labeled training data under unsupervised settings. 
We detail image synthesis in \secref{sec:Image Generation}, and mask generation in \secref{synthetic mask generation}.
 





\vspace{-0.25cm}
\subsubsection{Image Generation}
\label{sec:Image Generation}
For one pre-trained text-to-image Stable Diffusion~\cite{ldm}, we here freeze it, then generate images through inputting random Gaussian noise and class text prompts. Class names are sampled from ImageNet~\cite{imagenet}. 



For text input, a simple way is to simply use class names, but this may limit diversity and cause bottlenecks for downstream tasks. 
Hence, to adaptively generate various text prompts for each class, we interact with ChatGPT~\cite{chatGPT}. 
For example, we ask ChatGPT to list prompts about ``aeroplane'', then it could give some generative-style prompts like: ``\textit{A aeroplane soaring through a vibrant sunset sky, fluffy clouds, warm lighting, viewed from a low angle, realistic style.}'' 
The generated prompts introduce richer context, thus can better unleash the potential of the Stable Diffusion to synthesis high-fidelity, more diverse images. One noise reduction strategy is also applied following~\cite{he2022synthetic}.


\vspace{-0.25cm}
\subsubsection{Mask Generation}
\label{synthetic mask generation}




















Here, we generate high-quality masks by leveraging attentions in pre-trained diffusion models as clues, following two non-trivial observations.
(1) Cross-attention $\mathcal{A}_c$ indicates locality between the conditioning text and noisy image, thus $\mathcal{A}_c$ can coarsely describe \textit{objectness}.
(2) Self-attention $\mathcal{A}_s$ inside one image indicates pairwise semantic similarity between pixels, thus $\mathcal{A}_s$ could roughly describe \textit{coherence}.
Inspired by these, we propose AttentionCut, a training-free strategy to generate masks guided by attention maps.







\vspace{0.2cm}
{\noindent \bf Preparations.}
We first extract $\mathcal{A}_c$ and $\mathcal{A}_s$ at the position of category token in the prompt sentence, then aggregate different resolutions and timesteps considering multi-scale objects and avoiding focus shift during diffusion.
Formally,
\vspace{-0.5em}
\begin{equation}
    \mathcal{A}_c=\frac{1}{kT}\sum_{l=1}^{k}\sum_{t=0}^{T-1} \mathcal{A}_c^{t,l}; \
    \mathcal{A}_s=\frac{1}{LT}\sum_{l=1}^{L}\sum_{t=0}^{T-1} \mathcal{A}_s^{t,l}, 
    \vspace{-0.5em}
\end{equation}
where $t=T-1, \dots, 0$ is for each reverse step and $l = 1, \dots, L$ is for intermediate layers. $\mathcal{A}_c$ is averaged among the top-$k$ of the standard variation from all $\mathcal{A}_c^l$, while $\mathcal{A}_s$ is averaged among all layers and time steps.


\vspace{0.2cm}
{\noindent \bf Objectness.}
Intuitively, the pixel-level cross-attention $\mathcal{A}_c$ under a specific category can roughly be seen as segmentation masks, as it indicates how likely a pixel belongs to the category. However, in practice we found $\mathcal{A}_c$ is sparse and inattentive near the boundary, which can seriously damage segmentation results. 
To handle this issue, we improve $\mathcal{A}_c$ by strengthening the edge area with the self-attention $\mathcal{A}_s$. It indicates semantic connectivity, \ie, how semantically two pixels belong to one group. 
Specifically, we first randomly select a set of initial seeds $\mathcal{B}$ from the boundary of the binary mask $\left[\mathcal{A}_c>\tau\right]$. 
Then each selected seed $b \in \mathcal{B}$ can expand as a confidence map $\mathcal{A}_s(b, \cdot)$, which is the self-attention between $b$ and other pixels, indicating weights of the boundary area. 
We assume $\mathcal{A}_s(\cdot, b)=\mathcal{A}_s(b, \cdot)$, as $\mathcal{A}_s$ is symmetric theoretically.
For pixel $p$, these maps are averaged as a refined map $r(p)$, to reinforce the boundary pixels: 
\vspace{-0.5em}
\begin{equation}
    r(p) = 1/{|\mathcal{B}|}\cdot \sum\nolimits_{b \in {B}} \mathcal{A}_s(p,b).
\vspace{-0.5em}
\end{equation}

Combining cross-attention $\mathcal{A}_c$ and the refined map $r(p)$ with a balance weight $\lambda_\phi$, the pixel-level objectness $\phi$ are:
\vspace{-0.5em}
\begin{equation}
    \phi(p) = \left\{
    \begin{aligned}
        -\log(\mathcal{A}_c(p)+\lambda_\phi r(p)),\  \text{if }p \in\text{foreground},\\
        \log(1-\mathcal{A}_c(p)-\lambda_\phi r(p)),\  \text{if }p\in\text{background},
    \end{aligned}
    \right.
    \vspace{-0.5em}
\end{equation}
where $\mathcal{A}_c(p)$ is the cross-attention at pixel $p$.


\vspace{0.2cm}
{\noindent \bf Inner Coherence.}
With only objectness, we found that the masks tend to lose local information, for example, irregular corners, mis-segmented holes, or jagged contours.
This can be solved by taking local consistency into account, \ie, how likely two neighboring pixels belong to one group.
Here we design an inner coherence term that can help to enforce continuity, proximity and smoothness of segments belonging to the same object, and penalize those who deviate.




The proposed inner coherence consists of two parts: semantic and spatial.
As mentioned above, $\mathcal{A}_s$ can indicate semantic coherence, as self-attention is calculated in semantic feature space.
Spatial coherence is designed to indicate pixels pairwise distance in both RGB and Euclidian space.
This coherence is obtained by absorbing the form of geodesic distance on the surface of image intensity, then by negative exponential transformation.
The inner coherence $\psi$ can be formalized as:
\vspace{-0.5em}
\begin{equation}
    \begin{aligned}
        \psi(p,q) &= \mathcal{A}_s(p,q) + \lambda_\psi e^{-\mathcal{D}(p,q)},\\
        \mathcal{D}(p,q) &= \min_{P}\int_0^1\|\nabla I\left(P(s)\right)\cdot v(s)\|ds,
    \end{aligned}
    \label{eq:psi}
    \vspace{-0.5em}
\end{equation}
where for pixel $p$ and $q$, $\mathcal{A}_s(p,q)$ is the self-attention and $\mathcal{D}(p,q)$ is the geodesic distance;
$P$ is an arbitrary path from $p$ to $q$ parameterized by $s\in[0,1]$;
$v(s)$ denotes the unit vector $P'(s)/\|P'(s)\|$ that is tangent to the path direction; $I(\cdot)$ is image RGB intensity.






\vspace{0.2cm}
{\noindent \bf Calculating Mask.}
Given objectness and inner coherence, we define an energy function $E$ for each potential mask $\mathcal{M}$:
\vspace{-0.5em}
\begin{equation}
E(\mathcal{M}) = \sum\nolimits_p\phi(p)+\lambda\sum\nolimits_{\mathcal{M}(p)\neq\mathcal{M}(q)}\psi(p,q),
\vspace{-0.5em}
\end{equation}
where $\lambda$ denotes the weight between $\phi$ and $\psi$; $\mathcal{M}(\cdot)\in\{0,1\}$ means the pixel in this mask. The binary mask $\mathcal{M}$ is generated by minimizing $E(\mathcal{M})$, \ie, use Ford-Fulkerson algorithm~\cite{ford1956maximal} to find a minimum cut in the image graph. And after further post-processing and denoising~\cite{barron2016fast, zhang2021datasetgan,li2022bigdatasetgan}, we can obtain the final synthetic mask (see \figref{fig:teaser} Right for some examples).









\vspace{0.2cm}
{\noindent \bf Discussion.}
Compared with other training-free mask generation methods like NCut~\cite{shi2000normalized} and K-means~\cite{lloyd1982},
they only consider pairwise similarly, thus cannot decide fore/background for each partition. 
Compared with DenseCRF~\cite{NIPS2011_beda24c1},
AttentionCut has well-designed objectness and inner coherence terms, which is more suitable for diffusion models and guarantees convergence.
In \tabref{tab:raw_cut}, we have conducted experiments to validate the superiority of AttentionCut.



\subsection{Exploitation Stage: Diffusion Knowledge }
\label{Diffusion Features and Synthetic Data Supervision}


This stage aims to bridge the architectural gap between pre-trained diffusion models and discriminative tasks, \eg, object discovery. 
As shown in \figref{fig:framework} (2), we achieve this in two steps: in \secref{Extracting Diffusion Features}, we treat diffusion models as a universal feature extractor to distill explicit visual knowledge; in \secref{Segment Decoder}, we feed diffusion features into one flexible decoder, and train with ``infinite'' synthetic data.









\subsubsection{Extracting Diffusion Knowledge}
\label{Extracting Diffusion Features}






For diffusion models, they are fed with noise and text to output synthesis images; while for object discovery models, they are fed with images to output pixel-level masks. Such an architectural gap blocks direct feature extraction from diffusion. 
To solve this, given one image, we are required to find the corresponding input noise of diffusion models under some conditioning text, then features can be extracted through diffusion reverse process.
To get input noise, we combine diffusion inversion~\cite{ddim}
with the conditional UNet. To get the conditioning text, we simply classify images by CLIP~\cite{clip}.











\vspace{0.2cm}
{\noindent \bf Diffusion Inversion and Feature Extraction.} 
Given pre-trained diffusion models, we here inverse one image back to its corresponding noise under the conditioning text. 
This diffusion inversion can be seen as a special forward process.





One trivial solution is to use the typical DDPM~\cite{ddpm}. Although it can yield latent variables (\ie, noise) through the forward process, 
these variables are stochastic and cannot reconstruct the image through the reverse process.
So it is not suitable for feature extraction.
Inspired by DDIM~\cite{ddim}, we modify each step by combining it with conditional denoising UNet $\boldsymbol\epsilon_\theta(\boldsymbol{x}_t, t, y)$ in Stable Diffusion, making the forward/reverse non-Markovian to enjoy deterministic. 
Now the forward/reverse process for each step is:
\vspace{-0.5em}
\begin{equation}
\resizebox{0.90\linewidth}{!}{$
\begin{aligned}
    &\boldsymbol{x}_{t+1}=\sqrt{\alpha_{t+1}} \boldsymbol{f}_\theta\left(\boldsymbol{x}_t, t, y\right)+\sqrt{1-\alpha_{t+1}} \boldsymbol{\epsilon}_\theta\left(\boldsymbol{x}_t, t, y\right),
    \\
    &\boldsymbol{x}_{t-1}=\sqrt{\alpha_{t-1}} \boldsymbol{f}_\theta\left(\boldsymbol{x}_t, t,y\right)+\sqrt{1-\alpha_{t-1}} \boldsymbol{\epsilon}_\theta\left(\boldsymbol{x}_t, t,y\right),
    \end{aligned}$
    }
    \label{equ: deterministic reverse}
    \vspace{-0.5em}
\end{equation}
where $\boldsymbol{f}_\theta\left(\boldsymbol{x}_t, t, y\right)=\left({\boldsymbol{x}_t-\sqrt{1-\bar{\alpha}_t} \boldsymbol{\epsilon}_\theta\left(\boldsymbol{x}_t, t, y\right)}\right)\,/\, {\sqrt{\bar{\alpha}_t}}$, $\alpha_t=1-\beta_t$, $\bar{\alpha}_t=\prod_{s=1}^t\left(1-\beta_s\right)$, $\beta_t$ is a variance schedule. $y$ denotes the conditional text, and $t$ means timesteps.

After diffusion inversion, to get the corresponding noise, features $\mathcal{F}^{t,l}$ can be extracted from $\boldsymbol\epsilon_\theta(\boldsymbol{x}_t, t, y)$ during each reverse step $t=T-1, \dots, 0$ and intermediate layer $l = 1, \dots, L$.
To cover long range and multi-level features of multi-scale objects, they are aggregated in all time steps:
\vspace{-0.5em}
\begin{equation}
\mathcal{F}^l=1/T\cdot\sum\nolimits_{t=0}^{T-1} \mathcal{F}^{t,l}.
\vspace{-0.5em}
\end{equation}
In practice, we choose the output of the ``SpatialTransformer'' block in Stable Diffusion, where $L=6$ with resolutions $16 \times 16$, $32 \times 32$, and $64 \times 64$, two of each.


\begin{table*}
\begin{center}
\caption{Comparison with \sota\ methods on the public crowd analysis benchmarks: \jhu, ShanghaiTech, UCF, and \nwpu. 
The best results are shown in \first{red}. The second-best results are shown in \second{blue}. 
}
\vspace{\tablegap}
\resizebox{0.95\textwidth}{!}{
\begin{tabular}{l c c c c c c c c c c c c c}
\toprule
 \multirow{2}{*}{Method} & \multirow{2}{*}{Venue} &\multicolumn{2}{c}{\jhu} &\multicolumn{2}{c}{\shha} &\multicolumn{2}{c}{\shhb} &\multicolumn{2}{c}{\ucf} &\multicolumn{2}{c}{\qnrf} &\multicolumn{2}{c}{\nwpu}\\[0.2ex]
 \cmidrule(lr){3-4}\cmidrule(lr){5-6}\cmidrule(lr){7-8}\cmidrule(lr){9-10}\cmidrule(lr){11-12}\cmidrule(lr){13-14}
& & MAE$\downarrow$ & MSE$\downarrow$ & MAE$\downarrow$ & MSE$\downarrow$ & MAE$\downarrow$ & MSE$\downarrow$ & MAE$\downarrow$ & MSE$\downarrow$ & MAE$\downarrow$ & MSE$\downarrow$ & MAE$\downarrow$ & MSE$\downarrow$\\[0.2ex]
\midrule\midrule
TopoCount \cite{abousamra2021localization}	& AAAI'21	& {60.9}	& {267.4}	& {61.2}	& {104.6}	& {7.8}	& {13.7}	& {184.1}	& {258.3}	& {89.0}	& {159.0}	& {107.8}	& {438.5}	\\[0.2ex]
SUA \cite{meng2021spatial}	& ICCV'21	& {80.7}	& {290.8}	& {68.5}	& {121.9}	& {14.1}	& {20.6}	& {-}	& {-}	& {130.3}	& {226.3}	& {111.7}	& {443.2}	\\[0.2ex]
ChfL \cite{shu2022crowd}	& CVPR'22	& {57.0}	& {235.7}	& {57.5}	& {94.3}	& {6.9}	& {11.0}	& {-}	& {-}	& {80.3}	& {137.6}	& {76.8}	& {343.0}	\\[0.2ex]
MAN \cite{lin2022boosting}	& CVPR'22	& {53.4}	& \second{209.9}	& {56.8}	& {90.3}	& {-}	& {-}	& {-}	& {-}	& {77.3}	& {131.5}	& {76.5}	& {323.0}	\\[0.2ex]
GauNet \cite{cheng2022rethinking}	& CVPR'22	& {58.2}	& {245.1}	& {54.8}	& {89.1}	& {6.2}	& {9.9}	& {186.3}	& {256.5}	& {81.6}	& {153.7}	& {-}	& {-}	\\[0.2ex]
CLTR \cite{liang2022end}	& ECCV'22	& {59.5}	& {240.6}	& {56.9}	& {95.2}	& {6.5}	& {10.6}	& {-}	& {-}	& {85.8}	& {141.3}	& {74.3}	& {333.8}	\\[0.2ex]
CrwodHat \cite{wu2023boosting}	& CVPR'23	& \second{52.3}	& {211.8}	& {51.2}	& {81.9}	& \first{5.7}	& {9.4}	& {-}	& {-}	& {75.1}	& \second{126.7}	& {68.7}	& \second{296.9}	\\[0.2ex]
STEERER \cite{han2023steerer}	& ICCV'23	& {54.3}	& {238.3}	& {54.5}	& {86.9}	& {5.8}	& \second{8.5}	& {-}	& {-}	& {74.3}	& {128.3}	& \second{63.7}	& {309.8}	\\[0.2ex]
PET \cite{liu2023point}	& ICCV'23	& {58.5}	& {238.0}	& \second{49.3}	& \second{78.8}	& {6.2}	& {9.7}	& {-}	& {-}	& {79.5}	& {144.3}	& {74.4}	& {328.5}	\\[0.2ex]
\rowcolor{black!10}\method\	& 	& \first{47.3}	& \first{198.9}	& \first{47.4}	& \first{75.0}	& \first{5.7}	& \first{8.2}	& \first{160.8}	& \first{225.0}	& \first{68.9}	& \first{125.6}	& \first{57.8}	& \first{221.2}	\\[0.2ex]
\bottomrule
\end{tabular}
}
\vspace{\tablegap}
\label{table: crowd counting performance}
\end{center}
\end{table*}


\vspace{0.2cm}
{\noindent \bf CLIP-classifiable Prior.}
Notice that in \equref{equ: deterministic reverse}, the diffusion inversion should be done under some conditional text $y$.
We choose $y$ to be the CLIP-classified category of the input image, because of the following observations:
(1) humans take pictures by naturally framing an object of interest near the center of the image~\cite{judd2009learning} (center prior); 
(2) most background regions can be easily connected to image boundaries, while difficult for object regions~\cite{wei2012geodesic} (background prior); 
(3) CLIP is pre-trained on a large corpus of web-curated data, and most of which is human-token images with saliency objects~\cite{clip} (source prior). 
It is easy to classify images with the center and background priors, and the source prior enables us to classify using CLIP~\cite{clip}.
We summarize this as \textit{CLIP-classifiable prior}.

In practice, we choose the label set in ImageNet~\cite{imagenet}, and combine semantically similar classes, \eg, poodle and Chihuahua as dogs, etc.
Besides, multiple prompt templates are used, \eg, ``A photo of \{category\}'' to boost performance.



\subsubsection{Segment Decoder}
\label{Segment Decoder}
To enable diffusion compatible with object discovery, we here propose two options for preference. 
One is to attach a flexible decoder to the pre-trained diffusion models, and train using the synthesised data to achieve object discovery. This option costs many parameters and rich training data, bringing superior performance, and we denote it as {\textit{DiffusionSeg}} in  {\tabref{table:main_all}}. 
The other is to extract cross- and self-attention during diffusion inversion, and generate pseudo-masks using AttentionCut in \secref{synthetic mask generation}.
Such an option costs no trainable parameters and data, thus showing faster inference speeds, and we call it {\textit{AttentionCut}} in \tabref{table:main_all}. 

 
















\subsection{Discussion}
This paper uses pre-trained diffusion models for unsupervised object discovery.
Comparing with discriminative pre-training~\cite{lost,tokencut,selfmask}, generative pre-training has additional pixel-level understanding, which is more suitable for object discovery. 
Compared with MAE-style~\cite{mae} generative pre-training, which learns reconstruction representations to help object discovery, diffusion models show a clear advantage, \ie, synthesis abundant data, which is valuable to improve performance (see \tabref{tab:train_syn} and \figref{fig:scale}). 
Comparing with GANs in image synthesizing, diffusion models have significant advantages in higher sample quality and diversity, more stability and robustness~\cite{dhariwal2021diffusion}.
Compared to a few early GAN-based works that struggle to synthesise  mask with manual annotations~\cite{zhang2021datasetgan,li2022bigdatasetgan}, diffusion model can obtain mask using AttentionCut, without manually labeling.















\section{Experiments}
\begin{table*}[t!]
\begin{minipage}{0.175\linewidth}
\centering
% \hspace{1.8mm}
\captionof{table}{\small Datasets statistics \label{graph_datasets}}
\begin{tiny}
\begin{tabular}{c||c|c}
      {\bf Graph} & {\bf \#nodes} & {\bf \#edges} \\ \hline
      {\em FL} & 80\,513     & 5\,899\,882 \\
      {\em YT} & 1\,138\,499 & 2\,990\,443 \\
      {\em LJ} & 2\,238\,731 &14\,608\,137 \\
      {\em OR} & 3\,072\,441 & 117\,185\,083 \\
      {\em TW} & 41\,652\,230 & 1\,468\,365\,182 \\
\end{tabular}
\end{tiny}
\end{minipage}%
 \quad
 \begin{minipage}{.265\linewidth}
\centering
\tabcolsep=0.05cm

\captionof{table}{\small Avg. memory footprint (GB) of {\sf DistGER} and {\sf KnightKing} on each machine, where $\sigma$ is the standard deviation}
\label{Memory_usage}
\begin{tiny}
\newcommand{\tabincell}[2]{\begin{tabular}{@{}#1@{}}#2\end{tabular}}
  % \caption{\small {\color{blue} Avg. memory footprint (GB) of {\sf DistGER} and {\sf KnightKing} on each machine, where $\sigma$ is the standard deviation.}}
  \begin{tabular}{c|cc|cc}
    %\hline
    { }&\multicolumn{2}{c|}{\bfseries{ Sampling}}&\multicolumn{2}{c}{\bfseries{Training}}\\
    \hline
    {\bf{Graph}} &{\sf KnightKing} &{\sf DistGER} &{\sf KnightKing} &{\sf DistGER} \\
    \hline
     {\em FL} & 0.66($\pm$0.06)	&{\bf 0.41($\pm$0.02)}	&1.31($\pm$0.17) 	&{\bf 0.86($\pm$0.06)} 	\\

     {\em YT} &4.11($\pm$0.55)	&{\bf 1.36($\pm$0.23)} 	&4.73($\pm$0.72) 	&{\bf 4.26($\pm$0.63)} \\

     {\em LJ} & 7.65($\pm$0.82)	&{\bf 1.95($\pm$0.16)}	&6.38($\pm$0.97) 	&{\bf 5.49($\pm$0.85)} 	\\

     {\em CO} &10.98($\pm$1.03)	&{\bf 3.27($\pm$0.79)} 	&8.52($\pm$1.01) 	&{\bf 6.86($\pm$0.69)} 	\\

     {\em TW} & out-of-memory	&{\bf 20.18($\pm$3.62)} 	&out-of-memory 	& {\bf 67.16($\pm$5.18)} 	\\
  %\hline
\end{tabular}
\end{tiny}

\end{minipage}
\quad
\begin{minipage}{.25\linewidth}
    \centering
    \includegraphics[width= 1.85in, height = 1.2in]{./Figures/Dist_total_time_partition.eps}%
    \captionof{figure}
      {\small Efficiency: {\sf PBG} \cite{PBG_2019}, {\sf DistDGL} \cite{DistDGL_2020}, {\sf KnightKing} \cite{KnighKing_2019}, {\sf HuGE-D} (baseline), {\sf DistGER} (ours)
        \label{overall_performance}
      }
\end{minipage}%\hfill
\quad
\begin{minipage}{.25\linewidth}
    \centering
    \includegraphics[width= 1.85in, height = 1.2in]{./Figures/Dist_scalability_partition.eps}%
    \captionof{figure}
      {\small Scalability: {\sf PBG} \cite{PBG_2019}, {\sf DistDGL} \cite{DistDGL_2020}, {\sf KnightKing} \cite{KnighKing_2019}, {\sf HuGE-D} (baseline), {\sf DistGER} (ours)
        \label{Dist_scalability}
      }
\end{minipage}
\end{table*}


\section{Experimental Results}
\label{sec:experiments}
We evaluate the efficiency (\S \ref{sec:overall}) and scalability (\S \ref{sec:scalability}) of our proposed method, {\sf DistGER}
by comparing with {\sf HuGE-D} (baseline),
{\sf KnightKing} \cite{KnighKing_2019}, {\sf PyTorch-BigGraph} ({\sf PBG}) \cite{PBG_2019}, and {\sf Distributed DGL}
({\sf DistDGL}) \cite{DistDGL_2020}. We also compare the effectiveness (\S \ref{sec:effectiveness}) of generated embeddings
on link prediction.
% and multi-label classification tasks. 
Finally, we analyze efficiency due to individual
parts of {\sf DistGER} (\S \ref{sec:individual})
and the generality of {\sf DistGER} for other random walk-based embeddings (\S \ref{sec:generality}).
Our codes and datasets are at \cite{code}.
%
\subsection{Experimental Setup}
\label{sec:setup}
%
\spara{Environment.} We conduct experiments on a cluster of 8 machines with 2.60GHz Intel $^\circledR$ Xeon $^\circledR$ Gold 6240 CPU with 72 cores (hyper-threading)
in a dual-socket system, and each machine is equipped with 192GB DDR4 memory and connected by a 100Gbps network.
The machines run Ubuntu 16.04 with Linux kernel 4.15.0. We use GCC v9.4.0 for compiling {\sf DistGER}, {\sf KnightKing}, and {\sf HuGE-D},
and use Python v3.6.15 and torch v1.10.2 as the backend deep learning framework for {\sf Pytorch-BigGraph} and {\sf DistDGL}.

\spara{Datasets. } We employ five widely-used, real-world graphs
(Table~\ref{graph_datasets}): {\em Flickr} (FL) \cite{Flickr_Youtube_Graph},
{\em Youtube} (YT) \cite{Flickr_Youtube_Graph},
{\em LiveJournal} (LJ) \cite{BlogCatalog_Twitter_LiveJournal_Graph},
{\em Com-Orkut} (OR) \cite{com-orkut_2012}, and {\em Twitter} (TW) \cite{twitter_2010}.
The first two graphs are selected for multi-label node classification with distinct number of node labels 195 and 47, respectively, %in {\em Flickr} and {\em Youtube},
where labels in {\em Flickr} represent interest groups of users, and {\em Youtube}'s labels represent groups of viewers that enjoy common video genres. The last four graphs are used in link prediction. We also use synthetic graphs \cite{RMAT_2004} (up to 1 billion nodes, 10 billion edges) and a real-world {\em UK graph} \cite{BSVLTAG} (100M nodes, 3.7B edges) to assess the scalability of {\sf DistGER}.
Considering the default settings of popular random walk-based methods (e.g., Deepwalk, node2vec, HuGE), we use their undirected version.

\spara{Competitors.} We compare {\sf DistGER} against three state-of-the-art distributed graph embedding frameworks: the distributed random walk engine, {\sf KnightKing} {\scriptsize\url{https://github.com/KnightKingWalk/KnightKing}}
\cite{KnighKing_2019}; the distributed multi-relations based graph embedding system, {\sf PyTorch-BigGraph} ({\sf PBG})
{\scriptsize\url{https://github.com/facebookresearch/PyTorch-BigGraph}} \cite{PBG_2019} -- designed by Facebook; and
the distributed graph neural networks-based system, {\sf DistDGL} {\scriptsize\url{https://github.com/dmlc/dgl}}
\cite{DistDGL_2020} -- recently proposed by Amazon. We also implement {\sf HuGE-D}, a distributed version of
information-centric random walk-based graph embedding ({\sf HuGE} \cite{HuGE_2021}), on top of {\sf KnightKing},
served as our baseline. Since {\sf KnightKing} and {\sf HuGE-D} provide distributed support only for
random walk without that for embedding learning, we generate their node embeddings using
{\sf Pword2vec} {\scriptsize\url{https://github.com/IntelLabs/pWord2Vec}} \cite{Pword2vec_2019},
the most popular distributed {\sf Skip-Gram} system released by Intel.
%We find that {\sf pSGNScc} \cite{pSGNSCC_2017} (\S \ref{sec:learning})
%only provides a single-machine implementation, thus we do not include it in our distributed experiments.

\spara{Parameters.} For {\sf DistGER} and {\sf HuGE-D} random walks, we set
parameters $\mu$=0.995, $\delta$=0.001 based on information measurements (\S \ref{sec:preliminaries}),
while {\sf KnightKing} uses $L$=80 and $r$=10 that are routine configurations in the traditional
random walk-based graph embedding \cite{node2vec_2016, DeepWalk_2014, KnighKing_2019}. For {\sf DistGER}, {\sf KnightKing}, and {\sf HuGE-D} training,
we set the sliding window size $w$=10, number of negative samples $K$=5, and synchronization period=0.1 sec \cite{Pword2vec_2019},
and additionally, multi-windows number=2, $\gamma$=2 for {\sf DisrGER}.
%For {\sf Pytorch-BigGraph} ({\sf PBG}), we set the number of partitions to 16 following \cite{PBG_2019}, that is, using $2m$ partitions for the number of machines $m$ = 8
%in our case. %For {\sf DistDGL}, the deployed {\sf GaphSAGE} model uses three graph convolutional layers.
For fair comparison across all systems, %the efficiency performance of all systems involved in the experiments,
we set the embedding dimension $d$=128 that is commonly used \cite{HuGE_2021,node2vec_2016,DeepWalk_2014,Line_2015,Verse_2018,ProNE_2019},
and report the average running time for each epoch. For task effectiveness evaluations,
we find the best results from a grid search over learning rates from 0.001-0.1, \# epochs from 1-30,
and \# dimensions from 128-512.


%
\eat{
\begin{table}
\newcommand{\tabincell}[2]{\begin{tabular}{@{}#1@{}}#2\end{tabular}}
  \caption{\small Avg. memory footprint (GB) of {\sf DistGER} and {\sf KnightKing} on each machine, where $\sigma$ is the standard deviation.}
  \label{Memory_usage}
  \begin{center}
   \footnotesize
  \begin{tabular}{c|cc|cc}
    %\hline
    { }&\multicolumn{2}{c|}{\bfseries{ Sampling}}&\multicolumn{2}{c}{\bfseries{Training}}\\
    \hline
    {\bf{Graph}} &{\sf KnightKing} &{\sf DistGER} &{\sf KnightKing} &{\sf DistGER} \\
    \hline
     {\em Flickr} & 0.66($\pm$0.06)	&{\bf 0.41($\pm$0.02)}	&1.31($\pm$0.17) 	&{\bf 0.86($\pm$0.06)} 	\\

     {\em Youtube} &4.11($\pm$0.55)	&{\bf 1.36($\pm$0.23)} 	&4.73($\pm$0.72) 	&{\bf 4.26($\pm$0.63)} \\

     {\em LiveJournal} & 7.65($\pm$0.82)	&{\bf 1.95($\pm$0.16)}	&6.387($\pm$0.97) 	&{\bf 5.49($\pm$0.85)} 	\\

     {\em Com-Orkut} &10.98($\pm$1.03)	&{\bf 3.27($\pm$0.79)} 	&8.52($\pm$1.01) 	&{\bf 6.86($\pm$0.69)} 	\\

     {\em Twitter} & out-of-memory	&{\bf 37.1($\pm$5.28)} 	&out-of-memory 	& {\bf 79.5($\pm$7.27)} 	\\
  %\hline
\end{tabular}
\end{center}
\end{table}
%
}
%


%
\subsection{Efficiency and Memory Use w.r.t. Competitors}
\label{sec:overall}
%\begin{figure}
%  \centering
%  \includegraphics[width= 3 in]{Dist_total_time.eps}
%  \caption{\small Overall performance of PBG, DistDGL, KnightKing, HuGE-D and DistGER for generating embeddings on different read-word graphs, {\color{blue}for Twitter graph, DistDGL cannot finish in one day, and KnightKing fails to perform due to memory issue, where the y axis is in log-scale.}}
%  \label{overall_performance}
%\end{figure}
%
We report the end-to-end running times of {\sf PBG}, {\sf DistDGL}, {\sf KnightKing}, {\sf HuGE-D}, and {\sf DistGER}
on five real-world graphs with the cluster of 8 machines in Figure~\ref{overall_performance}.
The reported end-to-end time includes the running time of partitioning, random walks (for random walk-based frameworks), and training procedures.
%{\color{blue} Noted that the reported end-to-end time in our experiments excludes the partition time for all evaluated frameworks due to the all used partition schemes are executed as a preprocessing component, and we separately evaluate the partition efficiency in Section 6.5, thus the end-to-end time refers to the running time of random walk (only for random walk-based framework) and training procedure.}
{\sf DistGER} significantly outperforms the competitors
on all these graphs, achieving a speedup ranging from 2.33$\times$ to 129$\times$. %, by an average acceleration of $39.78 \times$.
Recall that {\sf DistGER} is a similar type of system as {\sf KnightKing} and {\sf HuGE-D},
and our key improvements are discussed in \S \ref{sec:DistGER} and in \S \ref{sec:learning}.
Analogously, Figure~\ref{overall_performance} exhibits that our system, %designs are more effective (see more evaluation details in \S 6.3),
{\sf DistGER} achieves an average speedup of 9.25$\times$ and 6.56$\times$ compared with {\sf KnightKing} and {\sf HuGE-D}.
Notice that we fail to run {\sf KnightKing} on the largest {\em Twitter} dataset
because its routine random walk strategy requires more main memory space.
%Although Huge-D achieves comparable performance,
The advantage of information-centric random walk in {\sf HuGE} is almost wiped out in {\sf HuGE-D}
due to on-the-fly information measurements and the higher communication costs in a distributed setting.
The multi-relation-based {\sf PBG} leverages a parameter server to synchronize embeddings between clients,
resulting in more load on the communication network. As a result, {\sf PBG} is on average
26.22$\times$ slower than {\sf DistGER}. For graph neural network-based system {\sf DistDGL},
due to the long running time of graph sampling (e.g., taking 80\% of the overhead for the {\sf GraphSAGE}),
it is highly inefficient than other systems. For the billion-edge {\em Twitter} graph, it does not terminate in 1 day.
%
%Considering the resource consumption that affects scalability,
{Table ~\ref{Memory_usage}} shows {\sf DistGER}'s average memory footprint on each machine of the 8-machine cluster. %from a cluster of 8 machines.
%and the standard deviation %($\sigma$) %of the results
%in 
%Table ~\ref{Memory_usage}. 
Compared
to %other methods, %with the 
same type of system
{\sf KnightKing}, 
% that is of the same system type, 
{\sf DistGER} requires less memory for sampling and training.


\subsection{Scalability w.r.t. Competitors}
\label{sec:scalability}
%
%\begin{figure}
%  \centering
%  \includegraphics[width= 3.2 in]{Dist_scalability.eps}
%  \caption{\small Scalability comparison on LiveJournal graph, where the y axis is in log-scale.}
% \label{Dist_scalability}
%\end{figure}
%
Figure~\ref{Dist_scalability} shows end-to-end running times of all competing
systems on the {\em LiveJournal} graph, as we increase \# machines
from 1 to 8 to evaluate scalability. {\sf DistGER} achieves better scalability than the other
four distributed systems.
%Due to space limitation, we omit results on other graph datasets,
%which exhibit similar trends.
{\sf PBG} leverages a parameter server and a shared network filesystem
to synchronize the parameters in the distributed model. %The edges are partitioned into $m^2$ buckets
%and training can be performed in parallel using up to $m/2$ machines. After one bucket completes
%the training, it needs to communicate with the parameter server.
When the number of machines increases, {\sf PBG} puts more load
on the communications network, resulting in poor scalability. Likewise, {\sf DistDGL}
is bounded by the synchronization overhead for gradient updates,
limiting its scalability.
%Since {\sf DistDGL} uses mini-batches for sampling, %features %for GraphSAGE,
%if the mini-batch samples cannot be generated on time, the trainer will be delayed on the forward pass, and all other
%machines need to wait before starting the backward pass. Thus, increasing the number of
%machines also affects the efficiency of backward pass. %Being the random walk-based distributed systems,
Both {\sf KnightKing} and {\sf HuGE-D} suffer from higher communication costs during random walks,
due to their only workload-balancing partitioning scheme (\S \ref{sec:dRand}, \S \ref{sec:individual}).
%Their scalability is relatively poor as the number of machines increases.
%{\sf KnightKing} partitions the graph by a workload-balancing scheme, inevitably introducing higher
%cross-machine communications due to the randomness inherent in the random walking procedure (\S \ref{sec:dRand}, \S \ref{sec:individual}).
%With more machines, the inefficiency of the partitioning scheme is further magnified.
Since {\sf HuGE-D} is implemented on top of {\sf KnigtKing},
it exhibits worse scalability due to high communication costs and on-the-fly information measurements in a distributed setting (\S \ref{sec:HUGED}).
%In contrast, its performance is much better than all the competitors in a single machine.
In comparison, {\sf DistGER} incorporates multi-proximity-aware streaming graph partitioning and incremental computations
to reduce both communication and computation costs, it also employs hotness-block based parameters synchronization
during training to dramatically reduce the pressure on network bandwidth. Hence, {\sf DistGER} achieves better scalability than other systems.
Due to space limitations, we omit {\sf DistGER}'s scalability results on other graphs, which exhibit similar trends. On {\em Twitter}, the end-to-end running times {\sf DistGER} on 1, 2, 4, and 8 machines are 3090s, 1739s, 1197s, and 746s, respectively,
while on {\em Com-Orkut}, the results are 304s, 204s, 149s, and 89s, respectively. 
The results show a good linear relationship.
% The results demonstrate a desired scalability with the increase of the machines.

\begin{table}
\quad
\begin{minipage}{0.46\linewidth}
    \centering
    \includegraphics[width= 1.6 in]{./Figures/Dist_scalability_datasize.eps}%
    \captionof{figure}
      {\small {Scalability of {\sf DistGER} on synthetic graphs, where Y-axis is in log-scale}}
      %The lines depict the running time required for random walk (blue line) and training (red line), respectively. Pentagrams show the time cost of six real-world graphs,
        \label{Dist_scalability_data}
      
\end{minipage}\hfill
\quad
\begin{minipage}{.46\linewidth}
    \centering
    \includegraphics[width= 1.6 in]{./Figures/Dist_time_auc.eps}%
    \captionof{figure}
      {\small {The influence of running time on embedding quality for {\sf DistGER} and competitors}}
        \label{Dist_time_auc}
\end{minipage}
\end{table}


To further assess the scalability of {\sf DistGER}, we generate synthetic graphs \cite{RMAT_2004} with a fixed node degree of 10 and the number of nodes from $10^5$ to $10^9$. Figure~\ref{Dist_scalability_data} presents the running times for random walks and training on these synthetic graphs using a cluster of 8 machines, suggesting that the running time increases linearly with the size of a graph, and {\sf DistGER} has the capability to handle even billion-node graphs. Moreover, the running times for six real-world graphs (including the {\em UK graph} with $|E|=3.7B$, $|V|=100M$, for which the competing systems do not terminate in 1 day or crash due to hardware and memory limitation) are inserted into the plot, which is consistent with the trend on synthetic data.

%
%
\subsection{Effectiveness w.r.t. Competitors}
\label{sec:effectiveness}
%
\spara{Link prediction.} To perform link prediction on a given graph $G$, following \cite{HuGE_2021,node2vec_2016,Verse_2018,NRP_2020},
we first uniformly at random remove 50\% edges as positive test edges, and the rest are used as positive training edges.
We also provide negative training and test edges by considering those node pairs between which no edge exists in $G$.
We ensure that the positive and negative set sizes are similar. %For a pair of nodes $(u, v)$, let $\varphi(u)$ and
%$\varphi(v)$ be the vectors learned by embedding methods.
The link prediction is conducted as a classification task
based on the similarity of $u$ and $v$, i.e., $\varphi(u)\cdot\varphi(v)$.
The effectiveness of link prediction is measured via the $AUC$ (Area Under Curve) score \cite{AUC_kdd} -- the higher the better.
We repeat this procedure 50 times to offset the randomness of edge removal and report the average $AUC$ in
Table~\ref{AUC_results}.
%shows $AUC$ for all the methods on five real-world graphs.
%, respectively, where a ``$-$'' indicates that the method fails due to the limitation of computing resources or because its running time exceeds 1 day.
{\sf DistGER} outperforms all competitors on these graphs, except for {\sf PBG} on {\em Com-Orkut}, where {\sf DistGER} ranks second.
On average, {\sf DistGER} has an 11.7\% higher $AUC$ score compared with the other three systems, thanks to our
information-centric random walks. {\sf PBG} is the best on {\em Com-Orkut} because this graph is much denser
and is friendly to the multi-relationship-based model in {\sf PBG}.
Figure~\ref{Dist_time_auc} exhibits accuracy-efficiency tradeoffs of {\sf DistGER} and competitors, i.e., their $AUC$ convergence curves w.r.t. increasing running times of random walks and training, over {\em LiveJournal}, further indicating
that {\sf DistGER} has better efficiency and effectiveness than the competitors.
%As a system of the same type, DistGER achieves better accuracies on all graphs than KnightKing which leverages the routine random walk configuration, thanks to its information-centric random walk strategies. We do not report the effectiveness of HuGE-D here because it uses the same random walk model as DistGER.
%
\begin{table}[h!]
\newcommand{\tabincell}[2]{\begin{tabular}{@{}#1@{}}#2\end{tabular}}
  \caption{\small $AUC$ scores of {\sf DistGER} and competitors for link prediction}
  \label{AUC_results}
  \begin{center}
  \footnotesize
  \begin{tabular}{cccccc}
%    \hline
    {Method}&\tabincell{c}{Youtube}&{LiveJournal}&\tabincell{c}{Com-Orkut}&{ Twitter}\\
    \hline
    {\sf PBG}        & 0.753           &0.882            &\bfseries{0.955} &0.912\\

    {\sf DistDGL}    &0.894            &0.718            &0.815            & running time $>$ 1 day \\

    {\sf KnightKing} &0.904            &0.963            & $0.918$         & out-of-memory\\

    {\sf DistGER}    &\bfseries{0.966} &\bfseries{0.976} &0.921            &\bfseries{0.919}\\
%  \hline
\end{tabular}
\end{center}
\end{table}

% \eat{
\spara{Multi-label node classification.}
This task predicts one or more labels for each graph node and has applications in %modern applications ranging from
text categorization \cite{zhang2006multilabel} and bioinformatics \cite{zhang2018ontological}.
We use embedding vectors and a one-vs-rest logistic regression classifier
with L2 regularization \cite{MLC_LIBLINEAR_2008}, %(using the LIBLINEAR library),
then evaluate the effectiveness by micro-averaged F1 ($Micro-F1$) and macro-averaged F1 ($Macro-F1$) \cite{WangC016}
scores, where $Micro-F1$ gives equal weight to each test instance and $Macro-F1$ assigns equal weight to each label category \cite{keikha2018community}.
%To train a classifier, nodes are uniformly at random split into training and test sets.
Following \cite{HuGE_2021,node2vec_2016,DeepWalk_2014,Line_2015,Verse_2018},
we select 10\% to 90\% training data ratio on {\em Flickr}, and 1\% to 9\% training ratio on {\em Youtube}.
%and the remaining nodes for testing.
We report the averaged $Macro-F1$ and $Micro-F1$ scores from 50 trials in Figure~\ref{Dist_MLC_mac_mic_F1}.
% shows the $Macro-F1$ and $Micro-F1$ scores achieved by each system as a
%function of the training ratio variation, respectively.
We find that {\sf DistGER} has better $Macro-F1$ and $Micro-F1$ scores
than existing frameworks, %on these graphs, %. In particular, compared with the KnightKing,
%DistGER consistently outperforms the other random walk-based systems on all graphs in $Macro-F1$ and $Micro-F1$ scores,
gaining 9.2\% and 3.3\% average improvements, respectively, due to its more effective information-centric random walks.
%Definition of $Macro-F1$ and $Micro-F1$ are as the following:

\begin{figure}[h!]
  \centering
  \includegraphics[width= 3.45 in]{./Figures/Dist_MLC_mac_mic_F1_1.eps}
  \caption{\small $Macro-F1$ (a1, b1) and $Micro-F1$ (a2, b2) scores for multi-label node classification. $X$-axis: training data ratio}
  \label{Dist_MLC_mac_mic_F1}
\end{figure}

% }
%\begin{equation}
%Precision = \frac{\sum\nolimits_{i}^{K}TP(i)}{\sum\nolimits_{i}^{K}(TP(i)+FP(i))}
%\end{equation}
%
%\begin{equation}
%Recall = \frac{\sum\nolimits_{i}^{K}TP(i)}{\sum\nolimits_{i}^{K}(TP(i)+FN(i))}
%\end{equation}
%
%\begin{equation}
%Micro-F1 = \frac{2\times Precision\times Recall}{Precision+Recall}
%\end{equation}
%
%\begin{equation}
%Macro-F1 = \frac{\sum\nolimits_{i}^{K}Micro-F1(i)}{|K|}
%\end{equation}
%
%where $TP(i)$, $FP(i)$ and $FN(i)$ are the number of true positives, false positives and false negatives in the instances which are predicted as $i$, respectively. Suppose $K$ is the overall label set, $Micro-F1$($i$) and $Macro-F1$ are the measure of $Micro-F1$ and $Macro-F1$ for the label $i$, respectively.
%\begin{table}
%\setlength{\abovecaptionskip}{0.cm}
%\setlength{\belowcaptionskip}{-0.cm}
%\newcommand{\tabincell}[2]{\begin{tabular}{@{}#1@{}}#2\end{tabular}}
%  \caption{$Macro-F1$ and $Micro-F1$ for multi-label classification on Flickr and Youtube graph, where train ratio is 0.5.}
%  \label{cluster_results}
%  \begin{center}
%  \small
%  \begin{tabular}{ccccc}
%    \hline
%    { }&\multicolumn{2}{c}{\bfseries{ \scriptsize Flickr}}&\multicolumn{2}{c}{\bfseries{\scriptsize Youtube}}\\
%    \hline
%    { }&Macro-F1 &Micro-F1&Macro-F1 &Micro-F1\\
%
%    \hline
%    \small PBG & 0.225	&0.387 	&0.295 	&0.406 	\\
%
%    \small DistDGL &0.205 	&0.378 	&0.283 	&0.403 	\\
%
%    \small KnightKing &0.239    &0.386 &0.285 	&0.402 	\\
%
%    \small DistGER  &\bfseries{0.277} &\bfseries{0.409}&\bfseries{0.298} &\bfseries{0.417}\\
%
%  \hline
%\end{tabular}
%\end{center}
%\end{table}
%

\begin{figure}
  \centering
  \includegraphics[width= 3.2 in]{./Figures/Dist_sampling_training_Mpad_efficiency.eps}
  \caption{\small {(a) Random walk efficiency, (b) training efficiency, (c) \# cross-machine messages, (d) random walk efficiency for {\sf MPGP} (ours) and workload-balancing scheme ({\sf KnightKing})}}
  \label{Dist_efficiency_sampling_training_MPGP}
\end{figure}

\begin{table*}[t!]
\begin{minipage}{0.275\linewidth}
\centering
\renewcommand\arraystretch{1.2}
\captionof{table}{\small Performance evaluation of partitioning for {\sf DistGER} and Competitors } %{\sf PBG} and {\sf DistDGL}
\label{Partition_sechme_overhead}
\begin{scriptsize}
\begin{tabular}{ccccc}
    \multicolumn{5}{c}{{\bfseries (a) Partitioning time for {\sf DistGER} and competitors }} \\
    \hline
    {\bf graph} & {\sf PBG} & {\sf DistDGL} & {\sf DistGER}\\
                &           & ({\sf METIS}) &  ({\sf MPGP}) \\
    \hline
    {\sf FL} & 383.28 s & 127.72 s & \bfseries{15.96 s} \\
    {\sf YT} & 349.15 s & 116.30 s & \bfseries{13.56 s} \\
    {\sf LJ} & 458.52 s & 425.19 s & \bfseries{36.42 s} \\
    {\sf OR} & 2662.62 s & 2761.25 s &\bfseries{294.68 s}\\
    {\sf TW} & 22 hour s & $>$ 1 day &\bfseries{9 hours}\\
    \hline
%    \multicolumn{5}{c}{} \\
    \multicolumn{5}{c}{{\bfseries (b) Evaluation of {\sf Parallel MPGP} }} \\
    \hline
    {\bf graph} & {\sf Streaming} & {\sf Partitioning} & {\sf Walking}\\
    \hline
  %  {\sf MPGP}   &DFS+deg  & 9 hours & \bfseries{575.22 s} \\
    \multirow{2}{*}{\sf LJ} &DFS+deg & 21.86 s & \bfseries{23.78 s} \\
           & BFS+deg & \bfseries{21.25 s} & 24.79 s \\
    \multirow{2}{*}{\sf OR} &DFS+deg & \bfseries{151.29 s} & 77.12 s \\
           & BFS+deg & 156.37 s & \bfseries{46.55 s} \\
    \multirow{2}{*}{\sf TW} &DFS+deg & \bfseries{1940.65} s & 683.81 s \\
           & BFS+deg & 2034.21 s & \bfseries{590.36 s}
\end{tabular}
\end{scriptsize}
\end{minipage}%\hfill
\quad
\begin{minipage}{.3\linewidth}
    \centering
    \includegraphics[width= 2.5in, height = 1.45 in]{./Figures/Dist_Mpad_streaming_vertex_time.eps}%
    \captionof{figure}
      {\small The distribution of local computations and cross-machine communications for different streaming orders on {\em LiveJournal}. The top table reports their running times for partitioning and random walks
        \label{Dist_MPaD_streaming}
      }
\end{minipage}%\hfill
\qquad
\begin{minipage}{.37\linewidth}
    \centering
    \includegraphics[width= 2.5 in, height = 1.45 in]{./Figures/Dist_generality_table_HuGE+.eps}%
    \captionof{figure}
      {\small Generality of {\sf DistGER} vs. {\sf KnightKing}. The bars show random walk efficiency ($-R$) and training efficiency ($-T$) for {\sf Deepwalk} ({\sf DW}), {\sf node2vec} ({\sf n2v}) and {\sf HuGE+}. The top table shows the ratio $\frac{\text{{\em AUC} for {\sf DistGER}}}{\text{{\em AUC} of {\sf KnightKing}}}$, with {\sf DW} and {\sf n2v}, task: link prediction
        \label{Dist_generality}
      }
\end{minipage}
\end{table*}


\subsection{Efficiency due to Individual Parts of DistGER}
\label{sec:individual}
\spara{Random walk and training efficiency.}
To evaluate the system design of {\sf DistGER} (\S \ref{sec:DistGER}, \S \ref{sec:learning}),
we first compare the efficiency of random walks and training with those of {\sf KnighKing} and {\sf HuGE-D}.
%For fair comparison, the running times that we reported for {\sf KnightKing} and {\sf HuGE-D} exclude the
%time of vocabulary table construction, since it is a serial process in {\sf Pwode2vec}, while {\sf DistGER}
%pipelines the construction during random walks.
For random walks (Figure~\ref{Dist_efficiency_sampling_training_MPGP}(a)),
{\sf DistGER} significantly outperforms {\sf KnightKing} and {\sf HuGE-D} on all our graph
datasets, achieving an average speedup of $3.32\times$ and $3.88\times$, respectively.
Although {\sf HuGE-D} implements information-oriented random walks on {\sf KnightKing},
due to additional computation and communication overheads during on-the-fly information
measurements (\S \ref{sec:HUGED}), its efficiency can be lower than that of {\sf KnightKing}.
We also notice that the random walk lengths ($L$) and the number of random walks ($r$) reduce (on average)
63.2\% and 18\%, respectively, in our information-oriented random walks, compared to {\sf KnightKing}'s
routine random walk configuration.
%which supports the traditional
%random walk methods. %To provide a straightforward adaptation for the information-oriented approach, DistGER leverages the incremental information-centric computation mechanism to mitigate the redundant computation and high communication cost in HuGE-D, then it achieves an average speedup of $3.32\times$ and $3.88\times$ in random walk procedure compared to KnightKing and HuGE-D.

Another benefit of information-centric random walks is that it generates concise and effective corpus to improve 
training efficiency. Compared to {\sf KnightKing}, {\sf DistGER} achieves $17.37\times$-$27.95\times$ acceleration
in training over all our graphs. Next, considering the same corpus size, we compare the training efficiency of {\sf Pword2vec} and {\sf DSGL}
(trainer in {\sf DistGER}). Figure~\ref{Dist_efficiency_sampling_training_MPGP}(b) shows that {\sf DSGL} achieves $4.31\times$ average speedup
compared to {\sf Pword2vec}. We also notice that the average throughput (number of nodes processed per second) for {\sf DSGL} is up to 49.5 million/s,
while that of {\sf Pword2vec} is only up to 16.1 million/s. These results indicate that our distributed {\sf Skip-Gram} learning model (\S \ref{sec:learning})
is more efficient than {\sf Pword2vec}.
%
%\begin{figure}
%  \centering
%  \includegraphics[width= 2.5 in]{Dist_Mpad_efficiency.eps}
%  \caption{\small (a) exhibits the number of cross-machine computation for DistGER on workload-balancing and MPGP partition scheme, respectively, and (b) shows the random walk time of DistGER on the two schemes, where y axis is in log-scale.}
%  \label{Dist_efficiency_MPaD}
%\end{figure}

\spara{Partitioning efficiency.} Considering the %large number cross-machine computing introduced by the
randomness inherent in random walks, the partitioning scheme is critical to overall efficiency. %of the distributed framework.
%To validate the efficiency of our multi-proximity-aware streaming graph partitioning (MPGP),
%we deploy the workload balancing scheme used in KnightKing and MPGP on DistGER,
%respectively,
%and report the number of cross-machine computations during the random walk procedure for the two schemes.
%We also present the efficiency performance of MPGP compared with the workload-balancing scheme.
For {\sf DistGER},
Figure~\ref{Dist_efficiency_sampling_training_MPGP}(c) exhibits that our multi-proximity-aware streaming graph partitioning ({\sf MPGP})
significantly reduces (avg. reduction $45\%$) the number of cross-machine messages than the workload-balancing partition of {\sf KnightKing}
on five graphs. Moreover, it improves the efficiency by 38.9\% for the random walking procedure
(Figure~\ref{Dist_efficiency_sampling_training_MPGP}(d)) over the same set of walks.
We report in Table~\ref{Partition_sechme_overhead}(a) the time required for graph partitioning in competing systems,
where {\sf DistDGL} uses the {\sf METIS} algorithm \cite{METIS_1998} for partitioning.
The results show that {\sf MPGP} performs partitioning with very little overhead in most cases, and
the partitioning efficiency is on average $25.1\times$ faster than competitors.
In Figure~\ref{Dist_MPaD_streaming}, we exhibit the distribution of local computations and cross-machine communications
on four machines for different streaming orders, and the top table reports their running times for partitioning and random walks.
For sequential {\sf MPGP}, we find that the {\sf DFS+degree}-based streaming order (\S \ref{sec:partition}) is more efficient than other streaming orders,
and it also strikes the best balance between cross-machine communications reduction and workload balancing.
Table~\ref{Partition_sechme_overhead}(b) exhibits the performance evaluation of {\sf parallel MPGP} on the small- ({\em LiveJournal}), medium- ({\em Com-Orkut}) and large-scale ({\em Twitter}) graphs. The results show that {\sf DFS+Degree} in {\sf parallel MPGP} is still the best or comparable in terms of partition time, due to the same reason as stated in our third optimization scheme (\S \ref{sec:partition}). On the other hand, {\sf BFS+Degree} in {\sf parallel MPGP} works the best in terms of random walk time due to preserving the locality of the graph structure (our fourth optimization scheme in \S \ref{sec:partition}).
%as using its streaming order to parallel partitioning can reduce the influence of relevance between each segment.
We ultimately recommend {\sf BFS+Degree} for {\sf parallel MPGP}, since it reduces the partition time greatly, while the random walk time is comparable to that obtained from sequential {\sf MPGP}.
%
%\begin{figure}
%  \centering
%  \includegraphics[width= 3 in]{Dist_Mpad_streaming_vertex_time.eps}
%  \caption{\small The distribution of local computations and cross-machine communications for different streaming orders on {\em LiveJournal}. The top table reports their running times for partitioning and random walks.}
%  \label{Dist_MPaD_streaming}
%\end{figure}
%
%\begin{table}
%\setlength{\abovecaptionskip}{0.cm}
%\setlength{\belowcaptionskip}{-0.cm}
%\newcommand{\tabincell}[2]{\begin{tabular}{@{}#1@{}}#2\end{tabular}}
%  \caption{\small Time execution time (seconds) of the partition scheme in PBG, DistDGL, and DistGER, ``$-$'' means the scheme fails under constrains of computation resource.}
%  \label{Partition_sechme_overhead}
%  \begin{center}
%  \small
%  \begin{tabular}{ccccc}
%    \hline
%    {Graph}&{PBG}&{DistDGL(METIS)}&{DistGER}\\
%    \hline
%    Flickr& 383.28 &127.72 &\bfseries{15.96} \\
%
%    Youtube& 349.15 &116.30&\bfseries{13.56} \\
%
%    LiveJournal& 458.52 &425.19 &\bfseries{36.42} \\
%
%    Com-Orkut& 2662.62 &2761.25 &\bfseries{294.68}\\
%
%    Twitter&78986.85 &$-$&\bfseries{35500.41}\\
    %\hline
%    \multicolumn{5}{l}{* HuGE+ generates the smallest corpus size for training among all methods tested.} \\
%
%  \hline
%\end{tabular}
%\end{center}
%\end{table}
%


\subsection{Generality of DistGER}
\label{sec:generality}
%\begin{figure}
%  \centering
%  \includegraphics[width= 3 in, height= 1.65 in]{Dist_generality_table.eps}
%  \caption{\small Generality comparison for DistGER and KnightKing, %on real-word graphs,
%  The bars display random walk (denoted as $-R$) and training efficiency (denoted as $-T$) for {\sf Deepwalk} (DW) and {\sf node2vec} (n2v), respectively.%, and the y axis is in log-scale.
%  Top table shows the ratio $\frac{\text{{\em AUC} for {\sf DistGER}}}{\text{{\em AUC} of {\sf KnightKing}}}$, both with Deepwalk and node2vec, respectively, considering link prediction.}
%  \label{Dist_generality}
%\end{figure}
%
%Since our proposed information-oriented random walk framework DistGER aims to address the redundant computations and high communication cost introduced by the effectiveness measurement of the generated walking information in distributed setting, it provides a good systematic support for the information-centric approach HuGE as shown by the previous experimental results. A natural question arises: can DistGER also support the traditional random-walk-based methods?
To demonstrate the generality of {\sf DistGER}, we deploy {\sf Deepwalk} \cite{DeepWalk_2014}, {\sf node2vec} \cite{node2vec_2016} and {
\sf HuGE+} \cite{HuGE+_2022}
on {\sf DistGER}. While the original {\sf Deepwalk} and {\sf node2vec} follow
traditional random walks, in {\sf DistGER} the walk length and the number of walks are decided via information-centric measurements.
Next, we also deploy both {\sf Deepwalk} and {\sf node2vec} on {\sf KnightKing} which supports the routine configuration random walk.
Figure~\ref{Dist_generality} illustrates that {\sf DistGER} reduces the random walks time by 41.1\% and 51.6\% on average for
{\sf Deepwalk} and {\sf node2vec}, respectively. For training, {\sf DistGER} is on average $17.7\times$ and $21.3\times$ faster than {\sf KnightKing}+{\sf Pword2vec}
for {\sf Deepwalk} and {\sf node2vec}, respectively.
Moreover, we also show the {\em AUC} ratio of {\sf DistGER} and {\sf KnightKing}, considering {\sf Deepwalk} and {\sf node2vec}, for link prediction.
% tasks, where performing multi-label classification on Flickr graph  and link prediction on other graphs, the accuracy metric for the two task are $Miro-F1$ and $AUC$ score, respectively, it can be found from
Our results depict that {\sf DistGER} has comparable (in most cases, higher) {\em AUC} scores, while it improves the efficiency significantly
even for traditional random walk-based graph embedding methods.
{\sf HuGE+} is an extension of {\sf HuGE}, and it uses the same {\sf HuGE} information-centric method to determine the walk length and the number of walks per node. Figure~\ref{Dist_generality} exhibits the compatibility of {\sf HuGE+} on {\sf DistGER} via its general API.

\section{Discussion}
\section{Conclusion and Future Works}

We present a novel formulation for a differentiable DTW layer that can be embedded into a deep network, based on continuous time warping and implicit deep layers. Unlike existing approaches, DecDTW yields the \textit{optimal} alignment path as a differentiable function of layer inputs. We demonstrate the effectiveness of DecDTW in utilising path information in challenging, real-world alignment tasks compared to existing methods. One promising direction for future work is in exploring more compact, alternative parameterisations of the warp function to improve scalability, inspired by~\cite{zhou12gtw}. Another direction would be to integrate more sophisticated DTW variants such as jumpDTW~\citep{jumpdtw}, which allows for repeats and discontinuities in the warp, into the GDTW (and DecDTW) formulation. Finally, we presented methodology for allowing regularisation and constraints to be learnable parameters in a deep network but did not explore this in our experiments. Future work will also explore this capability in more detail.


\section{Acknowledgements}
Max van Spengler acknowledges the University of Amsterdam Data Science Centre for financial support.

{\small
\bibliographystyle{ieee_fullname}
\bibliography{refs}
}

\appendix
\onecolumn
\section{Appendix for Proofs}

\paragraph{Proof of Theorem \ref{thm:main}.}

\begin{proof}
\label{proof:main}
Our proof has two steps. In Step 1, we will show that SimCLR is equivalent to minimizing the cross entropy loss defined in Eqn.~(\ref{eqn:cross-entropy}). 
In Step 2, we will show  that minimizing the cross-entropy loss 
is equivalent to spectral clustering on $\bfpi$. 
Combining the two steps together, we have proved our theorem. 

\textbf{Step 1: } SimCLR is equivalent to minimizing the cross entropy loss.

The cross-entropy loss takes expectation over 
$\bfW_\bfX\sim \mathbb{P}(\cdot ; \bfpi)$, 
which means $\bfW_\bfX$ has exactly one non-zero entry in each row $i$. By Lemma~\ref{lem:multinomial}, we know every row $i$ of $\bfW_\bfX$ is independent of other rows. Moreover, 
$\bfW_{\bfX,i}\sim \mathcal{M}(1, \bfpi_i/\sum_j \bfpi_{i,j})=\mathcal{M}(1, \bfpi_i)$, because $\bfpi_i$ itself is a probability distribution.
Similarly, we know $\bfW_\bfZ$ also has the row-independent property by sampling over $\mathbb{P}(\cdot;\bfK_\bfZ)$.
Therefore, by Lemma~\ref{lem:cross_split}, we know Eqn.~(\ref{eqn:cross-entropy}) is equivalent to:
\[
 -\sum_{i=1}^n \mathbb{E}_{\bfW_{\bfX,i}}[\log \mathbb{P}(\bfW_{\bfZ,i}=\bfW_{\bfX,i};\bfK_\bfZ)],
\]

This expression takes expectation over $\bfW_{\bfX,i}$ for the given row $i$. Notice that 
$\bfW_{\bfX,i}$ has exactly one non-zero entry, which equals $1$ (same for $\bfW_{\bfZ,i}$). 
As a result
we expand the above expression to be:
\begin{equation}
 -\sum_{i=1}^n \sum_{j\neq i} \Pr(\bfW_{\bfX,i,j}=1)\log \Pr(\bfW_{\bfZ,i,j}=1).
\label{eqn:detailed-expansion}    
\end{equation}


By Lemma~\ref{lem:multinomial}, $\Pr(\bfW_{\bfZ,i,j}=1)=\bfK_{\bfZ,i,j}/\|\bfK_{\bfZ,i}\|_1$ for $j\neq i$. Recall that $\bfK_\bfZ=(k(\bfZ_i-\bfZ_j))_{(i,j)\in[n]^2}$, which means 
$\bfK_{\bfZ,i,j}/\|\bfK_{\bfZ,i}\|_1=\frac{\exp(-\|\bfZ_i-\bfZ_j\|^2/{2\tau})}{\sum_{k\neq i}
\exp(-\|\bfZ_i-\bfZ_k\|^2/{2\tau})
}$ for $j\neq i$, when $k$ is the Gaussian kernel with variance $\tau$. 

Notice that $\bfZ_i=f(\bfX_i)$, so we know
\begin{equation}
-\log \Pr(\bfW_{\bfZ,i,j}=1)=
-\log \frac{\exp(-\|f(\bfX_i)-f(\bfX_j)\|^2/{2\tau})}{\sum_{k\neq i}
\exp(-\|f(\bfX_i)-f(\bfX_k)\|^2/{2\tau}),
}
\label{eqn:infonce-equivalence}    
\end{equation}


The right hand side is exactly the InfoNCE loss defined in Eqn.~(\ref{eqn:infonce}).
Inserting Eqn.~(\ref{eqn:infonce-equivalence}) into Eqn.~(\ref{eqn:detailed-expansion}), we get the SimCLR algorithm, which first samples augmentation pairs $(i,j)$ with $\Pr(\bfW_{\bfX,i,j}=1)$ for each row $i$, and then optimize the InfoNCE loss. 

\textbf{Step 2: } minimizing the cross entropy loss 
is equivalent to spectral clustering on $\bfpi$.


By Lemma~\ref{lem:convert_to_spectral}, we may further convert the loss to 
\begin{equation}
\label{eqn:main-theorem-repul-attr}
\min_{\bfZ}
-\sum_{(i,j)\in [n]^2} \mathbf{P}_{i,j}
\log k (\bfZ_i-\bfZ_j)+\log \mathbf{R}(\bfZ).
\end{equation}
Since $k$ is the Gaussian kernel, this reduces to \[
\min_\bfZ \mathrm{tr}(\bfZ^\top \mathbf{L}(\bfpi) \bfZ)
+\log \mathbf{R}(\bfZ),
\]

where we use the fact that $\mathbb{E}_{\bfW_\bfX\sim \mathbb{P}(\cdot; \bfpi)}[\mathbf{L}(\bfW_\bfX)]
=\mathbf{L}(\bfpi)
$, because the Laplacian operator is linear and $
\mathbb{E}_{\bfW_\bfX\sim \mathbb{P}(\cdot; \bfpi)}(\bfW_\bfX)=\bfpi
$.
\end{proof}

\paragraph{Proof of Theorem \ref{thm:clip}.}
\begin{proof}
Since $\bfW_\bfX\sim \mathbb{P}(\cdot;\bfpi_{\mathbf{A}, \mathbf{B}})$, we know 
$\bfW_\bfX$ has exactly one non-zero entry in each row, denoting the pair that got sampled. 
A notable difference compared to the previous proof is we now have $n_\mathcal{A}+n_\mathcal{B}$ objects in our graph. CLIP deals with this by taking a mini-batch of size $2N$, 
such that $n_\mathcal{A}=n_\mathcal{B}=N$, and adding the $2N$ InfoNCE losses together. We label the objects in $\mathcal{A}$ as $[n_\mathcal{A}]$, and the objects in $\mathcal{B}$ as $\{n_\mathcal{A}+1, \cdots, n_\mathcal{A}+n_\mathcal{B}\}$. 

Notice that $\bfpi_{\mathbf{A}, \mathbf{B}}$ is a bipartite graph, so the edges of objects in $\mathcal{A}$ will only connect to object in $\mathcal{B}$ and vice versa. We can define the similarity matrix in $\cZ$ as $\bfK_\bfZ$, 
where $\bfK_\bfZ(i, j+n_\mathcal{A})=\bfK_\bfZ(j+n_\mathcal{A},i)= k(\bfZ_i-\bfZ_j)$ for $i\in [n_\mathcal{A}], j\in [n_\mathcal{B}]$, and otherwise we set $\bfK_\bfZ(i,j)=0$. 
The rest is same as the previous proof. 
\end{proof}

\paragraph{Proof of Theorem \ref{thm:exponential}.}

\begin{proof}
\label{proof:exponential}
Since the objective function consists of a linear term combined with an entropy regularization, which is a strongly concave function, the maximization problem is a convex optimization problem. Owing to the implicit constraints provided by the entropy function, the problem is equivalent to having only the equality constraint. We then introduce the Lagrangian multiplier $\lambda$ and obtain the following relaxed problem:

$$
\widetilde{E}(\boldsymbol{\alpha})=\psi_{1}-\sum_{i=1}^n \alpha_{i} \psi_{i}+\tau \sum_{i=1}^n \alpha_{i}\log \alpha_{i}+\lambda\left(\boldsymbol{\alpha}^{\top} \mathbf{1}_n-1\right).
$$

As the relaxed problem is unconstrained, taking the derivative with respect to $\alpha_{i}$ yields

$$
\frac{\partial \widetilde{E}(\boldsymbol{\alpha})}{\partial \alpha_{i}}=-\psi_{i}+\tau\left(\log \alpha_{i}+\alpha_{i} \frac{1}{\alpha_{i}}\right)+\lambda=0.
$$

Solving the above equation implies that $\alpha_{i}$ takes the form
$
\alpha_{i}=\exp \left(\frac{1}{\tau} \psi_{i}\right) \exp \left(\frac{-\lambda}{\tau}-1\right).
$ Since $\alpha_{i}$ lies on the probability simplex, the optimal $\alpha_{i}$ is explicitly given by
$
\alpha^{*}_{i}=\frac{\exp \left(\frac{1}{\tau} \psi_{i}\right)}{\sum_{i^{\prime}=1}^n \exp \left(\frac{1}{\tau} \psi_{i^{\prime}}\right)} .
$ Substituting the optimal point into the objective function, we obtain
$$
\begin{aligned}
E\left(\boldsymbol{\alpha}^*\right)  &=\psi_1-\sum_{i=1}^n \frac{\exp \left(\frac{1}{\tau} \psi_{i}\right)}{\sum_{i^{\prime}=1}^n \exp \left(\frac{1}{\tau} \psi_{i^{\prime}}\right)} \psi_{i}+\tau \sum_{i=1}^n \frac{\exp \left(\frac{1}{\tau} \psi_{i}\right)}{\sum_{i^{\prime}=1}^n \exp \left(\frac{1}{\tau} \psi_{i^{\prime}}\right)}\log \frac{\exp \left(\frac{1}{\tau} \psi_{i}\right)}{\sum_{i^{\prime}=1}^n \exp \left(\frac{1}{\tau} \psi_{i^{\prime}}\right)} \\
& =\psi_1 - \tau \log \left(\sum_{i=1}^n \exp \left(\frac{1}{\tau} \psi_{i}\right)\right).
\end{aligned}
$$
Thus, the Lagrangian dual function is given by
\begin{equation*}
-E\left(\boldsymbol{\alpha}^*\right)= -\tau \log \frac{\exp \left(\frac{1}{\tau} \psi_{1}\right)}{\sum_{i=1}^n \exp \left(\frac{1}{\tau} \psi_{i}\right)}.\qedhere
\end{equation*}
\end{proof}



\section{More on Experiments} \label{section: experiment_details}

\paragraph{CIFAR-10 and CIFAR-100} CIFAR-10 ~\citep{krizhevsky2009learning} and CIFAR-100 ~\citep{krizhevsky2009learning} are well-known classic image classification datasets. Both CIFAR-10 and CIFAR-100 contain a total of 60k $32 \times 32$ labeled images of different classes, with 50k for training and 10k for testing. CIFAR-10 is similar to CIFAR-100, except there are 10 different classes in CIFAR-10 and 100 classes in CIFAR-100.

\paragraph{TinyImageNet} TinyImageNet ~\citep{le2015tiny} is a subset of ImageNet ~\citep{deng2009imagenet}. There are 200 different object classes in TinyImageNet, with 500 training images, 50 validation images, and 50 test images for each class. All the images in TinyImageNet are colored and labeled with a size of $64 \times 64$.

\textbf{Pseudo-code.} Algorithm \ref{alg:Training Procedure} presents the pseudo-code for our empirical training procedure.

\begin{algorithm}[!htbp]
\caption{Training Procedure}
\label{alg:Training Procedure}
\begin{algorithmic}[1]
\REQUIRE trainable encoder network $f$, batch size $N$, augmentation strategy \textit{aug}, loss function $L$ with hyperparameters \textit{args}
\FOR {sampled minibatch ${x_i}_{i=1}^N$}
\FORALL{$i \in { 1, ..., N }$}
\STATE draw two augmentations $t_i = \textit{aug}\left(x_i\right) $, $t_i' = \textit{aug}\left(x_i\right) $
\STATE $z_i = f\left(t_i\right)$, $z_i' = f\left(t_i'\right)$
\ENDFOR
\STATE compute loss $\mathcal{L} = L(N, z, z', \textit{args})$
\STATE update encoder network $f$ to minimize $\mathcal{L}$
\ENDFOR
\STATE \textbf{Return} encoder network $f$
\end{algorithmic}
\end{algorithm}

We also provide the pseudo-code for our core loss function used in the training procedure in Algorithm \ref{alg:Core loss}. The pseudo-code is almost identical to SimCLR's loss function, with the exception of an extra parameter $\gamma$.

\begin{algorithm}[!htbp]
\caption{Core loss function $\mathcal{C}$}
\label{alg:Core loss}
\begin{algorithmic}[1]
\REQUIRE batch size $N$, two encoded minibatches $z_1, z_2$, $\gamma$, temperature $\tau$
\STATE $z = \textit{concat}\left(z_1, z_2\right)$
\FOR {$i \in {1, ..., 2N }, j \in {1, ..., 2N}$ }
\STATE $s_{i,j} = \Vert z_i - z_j \Vert_2^{\gamma}$
\ENDFOR
\STATE \textbf{define} $l(i, j)$ \textbf{as} $l(i, j) = - \log \frac{exp\left(s_{i,j}/\tau \right)}{\sum_{k=1}^{2N} \mathbf{1}{[k \ne i]} exp\left(s{i, j} / \tau \right)} $
\STATE \textbf{Return} $\frac{1}{2N} \sum_{k=1}^N\left[l(i, i+N) + l(i+N, i)\right]$
\end{algorithmic}
\end{algorithm}

Utilizing the core loss function $\mathcal{C}$, we can define all kernel loss functions used in our experiments in Table \ref{table: loss definition}. For all $z_i \in z$ with even dimensions $n$, we define $z_{L_i} = z_i\left[0:n/2\right]$ and $z_{R_i} = z_i\left[n/2:n\right]$.

\begin{table}[ht]
\centering
\begin{tabular}{{@{}l|l@{}}}
Kernel  &  Loss function \\ \midrule
Laplacian & $\mathcal{C}\left(N, z, z', \gamma=1, \tau\right)$\\ \midrule
Sum       & $\lambda * \mathcal{C}\left(N, z, z', \gamma=1, \tau_1\right) + (1-\lambda) * \mathcal{C}\left(N, z, z', \gamma=2, \tau_2\right)$  \\ \midrule
Concatenation Sum&$\lambda * \mathcal{C}\left(N, z_L, z'_L, \gamma=1, \tau_1\right) + (1-\lambda) * \mathcal{C}\left(N, z_R, z'_R, \gamma=2, \tau_2\right)$\\ \midrule
$\gamma = 0.5$ & $\mathcal{C}\left(N, z, z', \gamma=0.5, \tau\right)$          \\ 

\end{tabular}

\caption{Definition of kernel loss functions in our experiments}
\label {table: loss definition}
\end{table}

\textbf{Baselines.} We reproduce the SimCLR algorithm using PyTorch Lightning~\citep{PytorchLightning}.

\textbf{Encoder details.}
The encoder $f$ consists of a backbone network and a projection network. We employ ResNet50~\citep{ResNet} as the backbone and a 2-layer MLP (connected by a batch normalization~\citep{ioffe2015batch} layer and a ReLU \cite{nair2010rectified} layer) with hidden dimensions 2048 and output dimensions 128 (or 256 in the concatenation kernel case).

\textbf{Encoder hyperparameter tuning.}
For each encoder training case, we randomly sample 500 hyperparameter groups (sample details are shown in Table \ref{table: Hyperparameter sample}) and train these samples simultaneously using Ray Tune ~\citep{RayTune}, with the ASHA scheduler~\citep{li2018massively}. Ultimately, the hyperparameter group that maximizes the online validation accuracy (integrated in PyTorch Lightning) within 5000 validation steps is chosen for the given encoder training case.

\begin{table}[ht]
\centering

\begin{tabular}{@{}l|l|l@{}}
\midrule
Hyperparameter  & Sample Range & Sample Strategy \\ \midrule
start learning rate & $\left[10^{-2}, 10\right]$ & log uniform \\ \midrule
$\lambda$       & $\left[0, 1\right]$ & uniform \\ \midrule
$\tau$, $\tau_1$, $\tau_2$ & $\left[0, 1\right]$ & log uniform \\ \midrule
\end{tabular}

\caption{Hyperparameters sample strategy}
\label {table: Hyperparameter sample}
\end{table}

\textbf{Encoder training.} 
We train each encoder using the LARS optimizer~\citep{LARSOptimizer}, LambdaLR Scheduler in PyTorch, momentum 0.9, weight decay $10^{-6}$, batch size 256, and the aforementioned hyperparameters for 400 epochs on a single A-100 GPU.

\textbf{Image transformation.} The image transformation strategy, including augmentation, is identical to the default transformation strategy provided by PyTorch Lightning.

\textbf{Linear evaluation.}
The linear head is trained using the SGD optimizer with a cosine learning rate scheduler, batch size 64, and weight decay $10^{-6}$ for 100 epochs. The learning rate starts at $0.3$ and ends at $0$.

\textbf{Moco Experiments.} We also tested our method based on MoCo~\citep{he2019moco}. The results are summarized in Table \ref{tab:results-moco}. Here we choose ResNet18~\citep{ResNet} as the backbone and set a temperature of $0.1$ as default. For our simple sum kernel, we set $\lambda=0.8$. The results show that our method outperforms the original MoCo method.

\begin{table}[thb]
\centering
\caption{MoCo Experiment Results on CIFAR-10 and CIFAR-100.}
\label{tab:results-moco}
\resizebox{\textwidth}{!}{%
\begin{tabular}{@{}c|ccc|ccc@{}}
\toprule
\multirow{3}{*}{Method} & \multicolumn{3}{c|}{CIFAR-10} & \multicolumn{3}{c}{CIFAR-100} \\ \cmidrule(lr){2-4} \cmidrule(lr){5-7} 
                        & 200 epochs & 400 epochs    & 1000 epochs   & 200 epochs & 400 epochs & 1000 epochs         \\ \midrule
MoCo (repro.)         & $76.41 \pm 0.12$    & $80.01 \pm 0.15$          & $84.45 \pm 0.08$    & $\mathbf{47.02 \pm 0.11}$ & $52.50 \pm 0.07$ & $57.62 \pm 0.15$            \\
\midrule
Laplacian Kernel        & ${78.09 \pm 0.10}$    & $\mathbf{83.85 \pm 0.09}$          & $\mathbf{88.34 \pm 0.16}$    & $46.12 \pm 0.22$   & $53.44 \pm 0.17$ & $59.10 \pm 0.14$        \\
Simple Sum Kernel & $\mathbf{78.12 \pm 0.15}$   & $83.23 \pm 0.18$ & $87.50 \pm 0.20$ & $46.65 \pm 0.06$ & $\mathbf{53.62 \pm 0.19}$ & $\mathbf{59.83 \pm 0.12}$\\
\bottomrule
\end{tabular}
}
\end{table}



\section{More Experiments on Synthetic Data}


Consider a scenario with $n$ clusters, each containing $k$ vertices. Let the probability of vertices $u$ and $v$ from the same cluster belonging to $\bfpi$ be $p$. Conversely, for vertices $u$ and $v$ from different clusters, let the probability of belonging to $\pi$ be $q$. We generate the graph $\bfpi$ randomly, based on $p$ and $q$. We experiment with values of $k=100$ and $n=6$ for ease of visualization, embedding all points in a two-dimensional space. Each vertex's initial position originates from a normal distribution. In each iteration, we sample a subgraph of $\bfpi$ uniformly, ensuring each vertex has an out-degree of $1$. We then optimize the corresponding vectors using InfoNCE loss with an SGD optimizer and iterate until convergence. Our experimental setup consists of an SGD learning rate of $1$, an InfoNCE loss temperature of $0.5$, and a batch size of $50$. We evaluate two scenarios with different $p$ and $q$ values: $p=1$, $q=0$, and $p=0.75$, $q=0.2$. The results of these experiments are visualized in Figure \ref{fig:vis-spectral-cluster}. The obtained embeddings exhibit the hallmark pattern of spectral clustering of graph $\bfpi$.

\begin{figure}[!tb]
\centering
\subfigure{
\includegraphics[width=1\textwidth]{Figures/cluster_pi.png}
\label{fig:vis-cluster}
}
\subfigure{
\includegraphics[width=1\textwidth]{Figures/noised_cluster_pi.png}
\label{fig:vis-noised-cluster}
}
\caption{Visualizations of the optimization process using InfoNCE Loss on the vectors corresponding to $\bfpi$. Points of identical color belong to the same cluster within $\bfpi$. To showcase the internal structure of $\bfpi$, we randomly select 10 vertices from each cluster to display the edge distribution of $\bfpi$.}
\label{fig:vis-spectral-cluster}
\end{figure}



\end{document}
