\label{sec:4.1}

\begin{table}[t]
  \centering
%   \small
\setlength{\tabcolsep}{8pt}
% \renewcommand{\arraystretch}{0.80}
% \begin{center}
  \begin{tabular}{l|c|c c}
    \toprule
    Methods & Year & mAcc & OA \\
    \midrule
    PointNet~\cite{qi2017pointnet} & 2017 &  63.4 &68.2\\
    PointNet++~\cite{qi2017pointnet++} & 2017 & 75.4 & 77.9 \\
    % 3DmFV~\cite{ben20183dmfv} & 2018 & 58.1 & 63\\
    SpiderCNN~\cite{xu2018spidercnn}& 2018& 69.8 & 73.7\\
    PointCNN~\cite{li2018pointcnn}&2018& 75.1 & 78.5 \\
    DGCNN~\cite{wang2019dynamic}&2019& 73.6 & 78.1\\
    DRNet~\cite{qiu2021dense}&2021& 78.0 & 80.3 \\
    GBNet~\cite{qiu2021geometric}&2021& 77.8 & 80.5 \\
    SimpleView~\cite{goyal2021revisiting}&2021 &-& 80.5\\
    PRA-Net~\cite{cheng2021net}&2021& 77.9 & 81.0 \\
    MVTN~\cite{hamdi2021mvtn}&2021 & - & 82.8 \\
    % PointMLP~\cite{ma2022rethinking} & 2022 & 83.9±0.5 & 85.4±0.3 \\
    CT~\cite{mazur2021cloud} & 2021 & 83.1 & 85.5 \\
    PointMLP~\cite{ma2022rethinking} & 2022 & 84.4 & 85.7 \\
    RepSurf-U~\cite{ran2022surface} & 2022 & 83.1 & 86.0 \\
    \rowcolor{LightYellow}PointNeXt~\cite{qian2022pointnext}& 2022 & 85.8±0.6 & 87.7±0.4\\
    \midrule
    \rowcolor{LightRed}\textbf{SPoTr}& 2023 & \textbf{86.8} &\textbf{88.6} \\
    \bottomrule
  \end{tabular}
  \caption{\textbf{Shape classification results on PB\_T50\_RS in SONN.}
  \label{tab:sonn}
  mAcc is the mean of class accuracy and OA is the overall accuracy.
  }
  % \end{center}
\end{table} 
%86.0



\paragraph{Shape Classification.}
% \subsubsection{Datasets.} 
For the shape classification, we validate SPoTr on a real-world dataset ScanObjectNN (\textbf{SONN})~\cite{uy2019revisiting}. 
% and synthetic dataset ModelNet40 (\textbf{MN40})~\cite{wu20153d}.
SONN has 2,902 objects categorized into 15 classes from SceneNN~\cite{hua2016scenenn} and ScanNet~\cite{dai2017scannet}.
Among diverse variants of SONN, we use PB\_T50\_RS (\textbf{SONN\_PB}), which is the most challenging version with random perturbation and contains 14,510 objects in total.
We follow the official split of \cite{uy2019revisiting}, where they divide SONN into 80\% for training and 20\% for evaluation.
Also, we sample 1,024 points for training and evaluating the models. 
% MN40 is a synthetic dataset with 12,311 meshed CAD models from 40 categories. 
% We use 9,843 models for training and 2,468 models for evaluation.

% Our model achieves state-of-the-art performance SONN\_PB.
\Cref{tab:sonn} shows that SPoTr outperforms all baselines with the mean of class accuracy (mAcc) of 86.8\% and overall accuracy (OA) of 88.6\% (+1.0\% mAcc, +0.9\% OA).
This result shows that capturing long-range context is important for recognizing 3D shapes in real-world datasets.
% In \Cref{tab:mn40}, SPoTr achieves an overall accuracy of 94.1\% on MN40 without any additional information~(\eg, normal vector). %or voting strategy in~\cite{liu2019relation}.
% Compared to baselines with the identical setting as ours, \ie, without a normal vector, SPoTr outperforms all baseline models.