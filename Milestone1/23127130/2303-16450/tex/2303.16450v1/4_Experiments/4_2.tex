% \subsection{Part Segmentation}
% \label{sec:4.2}
\begin{table}[t]
  \centering
\setlength{\tabcolsep}{5pt}
  \begin{tabular}{l|c|cc}
    \toprule
    Methods & Year & cls. mIoU & ins. mIoU\\
    \midrule
    PointNet~\cite{qi2017pointnet}&2017 & 80.4& 83.7\\
    PointNet++~\cite{qi2017pointnet++}&2017 & 81.9 & 85.1 \\
    PointCNN~\cite{li2018pointcnn}&2018& 84.6 & 86.1 \\
    DGCNN~\cite{wang2019dynamic}&2019 & 82.3 & 85.1 \\
    RSCNN~\cite{liu2019relation} & 2019 & 84.0 & 86.2 \\
    KPConv~\cite{thomas2019kpconv}&2019 & 85.1&86.4 \\
    PointConv~\cite{wu2019pointconv}&2019 & 82.8 & 85.7 \\
    PointASNL~\cite{yan2020pointasnl}&2020 & - & 86.1 \\
    PCT~\cite{guo2021pct}&2021 & - & 86.4\\
    PAConv~\cite{xu2021paconv}&2021 & 84.6  &  86.1 \\
    AdaptConv~\cite{zhou2021adaptive}&2021 & 83.4 & 86.4\\
    PointTransformer~\cite{zhao2021point}&2021& 83.7 & 86.6 \\
    CurveNet~\cite{xiang2021walk}&2021 & - & 86.8 \\
    PointMLP~\cite{ma2022rethinking} & 2022 &84.6 & 86.1\\
    \rowcolor{LightYellow}PointNeXt~\cite{qian2022pointnext} & 2022& 85.2 $\pm$ 0.1 & 87.0 $\pm$ 0.1 \\
    \midrule
    \rowcolor{LightRed}\textbf{SPoTr} & 2023& \textbf{85.4} & \textbf{87.2}\\


    \bottomrule
  \end{tabular}
  \caption{\textbf{Part segmentation results on SN-Part.} ins. mIoU is the mean of instance IoU. cls. mIoU is the mean of the class IoU. 
  }
  \label{tab:shapenet}
% \vspace{-5pt}
\end{table} 

\begin{table}[t]
  \centering
% \setlength{\tabcolsep}{5pt}
\setlength{\tabcolsep}{3pt}
\resizebox{\columnwidth}{!}{
  \begin{tabular}{l|c|ccc}
    \toprule
    Methods & Year & OA & mAcc & mIoU\\
    \midrule
    PointNet~\cite{qi2017pointnet} & 2017  & - & - & 41.1 \\
    PointCNN~\cite{li2018pointcnn} & 2018 &  85.9 & 63.9 & 57.3\\
    PointWeb~\cite{zhao2019pointweb} & 2019 & 87.0 & 66.6 & 60.3\\
    KPConv~\cite{thomas2019kpconv} & 2019 & - & 72.8 & 67.1 \\
    PCT~\cite{guo2021pct} & 2021 & - & 67.7 & 61.3 \\
    CT~\cite{mazur2021cloud} & 2021 & - & - & 67.9 \\
    PointTransformer~\cite{zhao2021point} & 2021 & \textbf{90.8} & - & 70.4 \\
    RepSurf-U~\cite{ran2022surface} & 2022 & 90.2 & 76.0 & 68.9 \\
    \rowcolor{LightYellow}PointNeXt~\cite{qian2022pointnext} & 2022 & 90.6 $\pm$ 0.1& - & 70.5 $\pm$ 0.3 \\
    \midrule
    \rowcolor{LightRed}\textbf{SPoTr}  & 2023 & 90.7 & \textbf{76.4} & \textbf{70.8}\\ % 32 - [1,1,1,1] [2,2,2,2]
    \bottomrule
  \end{tabular}}
  \caption{\textbf{Semantic segmentation results on S3DIS.} OA is the overall accuracy, mAcc is the mean of class accuracy, and mIoU is the mean of instance IoU.
  \label{tab:s3dis}
  }

\end{table} 


\paragraph{Part Segmentation.}
For part segmentation, we use \textbf{SN-Part}~\cite{snpart,lee2022sagemix}, which is a synthetic dataset with 16,881 shapes from 16 categories with 50 part labels.
We follow the split used in \cite{qi2017pointnet}, where 14,006 samples are for training and 2,874 samples are for validation. 
On each shape, 2,048 points are randomly sampled.

The results are reported in~\Cref{tab:shapenet}, where we evaluate the performance with the mean of instance IoU (ins. mIoU) and class IoU (cls. mIoU). Following previous works~\cite{liu2019relation,xiang2021walk,xu2021paconv}, we report the results with a multi-scale inference setting. 
Although the performance in SN-Part is quite saturated, SPoTr achieves the best performance 87.2\% with considerable improvements (+0.2\% mIoU).
% We also observed that each SP point of SPoTr recognizes the shape part of where it belongs without its explicit part label.
% See \Cref{sec:4.3} for visualizations.

\paragraph{Scene Segmentation.}
For comparison with previous methods~\cite{qi2017pointnet,thomas2019kpconv,zhao2021point,choe2022pointmixer} on scene segmentation, we validate SPoTr on the widely used benchmark dataset \textbf{S3DIS}~\cite{armeni20163d}. S3DIS is the large-scale dataset containing 271 rooms from 6 indoor areas with 13 semantic categories. In our experiments, we largely follow the settings of PointTransformer~\cite{zhao2021point} and consider Area-5 as the test set.

As shown in~\Cref{tab:s3dis}, SPoTr outperforms all previous methods in every metric (\ie, overall accuracy (OA), mean of class accuracy (mAcc), and mean of instance IoU (mIoU)).
The superior performance over previous Transformer architecture~\cite{zhao2021point} (+0.4\% mIoU) proves the importance of long-range dependency in the semantic segmentation as well as the shape classification.