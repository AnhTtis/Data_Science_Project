\label{sec:4.2} 

\begin{table}[t]
  \centering
%   \small
\setlength{\tabcolsep}{5pt}
% \renewcommand{\arraystretch}{0.80}
  \begin{tabular}{l|ccc|cc}
    \toprule
    Method & $g$ & $h$ & SP&  OA\\
    \midrule
    w/o SPA (\textit{baseline}) &    &  &   &  87.9\\
    w/o self-positioning &  \checkmark  & \checkmark &&87.7\\
    %  \checkmark  & & \checkmark & \checkmark &78.9 &82.5\\
    w/o disentangled attention &\checkmark &  &  \checkmark  &88.2\\
    SPoTr (\textit{ours}) & \checkmark &\checkmark & \checkmark &\textbf{88.6}\\
    \bottomrule
  \end{tabular}
  
    \caption{\textbf{Ablations on SONN\_PB.} $g$: spatial kernel, $h$: semantic kernel, SP: self-positioning points. OA is the overall accuracy.}
\label{table5}
% \vspace{-5pt}
\end{table} 


\begin{table}[t]
  \centering
%   \small
\setlength{\tabcolsep}{5pt}
% \renewcommand{\arraystretch}{0.80}
  \begin{tabular}{l|c|c}
    \toprule
    Attention type& Semantic rel. $\mathcal{R}$ &   OA\\
    \midrule
    Standard Att.  &-- & 86.1\\
    CWPA &$\mathbf{f}_k$    & 88.1\\
    CWPA & $\mathbf{f}_q + \mathbf{f}_k$    & 86.4\\
    CWPA & $\mathbf{f}_q \odot \mathbf{f}_k$&  85.4\\
    CWPA &$\mathbf{f}_q - \mathbf{f}_k$ & \textbf{88.6}\\
    \bottomrule
  \end{tabular}
  
    \caption{\textbf{Performance comparisons of different attention types and semantic relation $\mathcal{R}$ on SONN\_PB.} Attention types : Standard Attention in Transformer~\cite{vaswani2017attention} and channel-wise point attention~(CWPA) with Semantic relation : $\mathcal{R}(\mathbf{f}_q, \mathbf{f}_k)$} %= $ $\mathbf{f}_k$, $\mathbf{f}_q+\mathbf{f}_k$, $\mathbf{f}_q\odot\mathbf{f}_k$, and $\mathbf{f}_q-\mathbf{f}_k$.}
  % \vspace{-5pt}
  \label{table6}
\end{table} 

\begin{figure*}[ht] 
\centering
\includegraphics[trim=20 20 20 20,clip,width=1.0\textwidth]{Figures/SPpoints_compressed.pdf}
\caption{{\textbf{Self-positioning points~(SP points).} \textcolor{cyan}{SP points} are adaptively self-positioned according to each shape. \textcolor[rgb]{1,0,0}{Red points} correspond to specific SP points. Under the same class, the red points are located at \textit{semantically similar} positions.
}}
\label{fig:SPpoints}
% \vspace{0pt}
\end{figure*}
\begin{figure}[t] 
\centering

\includegraphics[trim=50 20 30 10, width=1.0\columnwidth]{Figures/dis_att_compressed.pdf}
% \vspace{-20pt}
\caption{
\textbf{Visual comparison of disentangled attention with a spatial kernel.}
{
% SP point aggregates feature with disentangled attention which filters out irrelevant information.
A spatial kernel (Middle) only considers spatial proximity without considering semantic relevance. Differently, Disentangled Attention~(Right) filters out irrelevant information.}}


% SPA aggregates feature with disentangled attention and then non-locally distributes the information using (c) global cross-attention.
% \textcolor{blue}{Blue} arrows illustrate information aggregation from points to the focal point.
% \textcolor[RGB]{96,96,96}{Gray} arrows stand for suppressed influence by $g \cdot h$.
% \textcolor[RGB]{0,153,0}{Green} arrows represent the focal point influence on semantically related points.}
\label{fig:dis_att}
% \vspace{-15pt}
\end{figure}

% Unlike (a) spatial kernel that only considers spatial proximity, SPA aggregates features with (b) disentangled attention $g \cdot h$.
% SPA aggregation with (a) the spatial kernel $g$ and (b) disentangled attention $g \cdot h$. 
% Then, \textcolor{cyan}{focal points} distribute information non-locally using global cross-attention (c).
% from semantically dissimilar parts are removed
% are from the area, where points have high proximity to focal point
% A self-positioning \textcolor{cyan}{focal point} consider spatial proximity with spatial kernel $g$~(Blue arrows are from the area, where points have high proximity to focal point). (b) With semantic kernel $h$, disentangled attention $g \cdot h$ improves the descriptive power by aggregating semantically similar local points~(Gray arrows from semantically dissimilar parts are removed).  (c) Then, \textcolor{cyan}{focal point} distributes the information non-locally using cross-attention~(Green arrows from focal point represent the area of the influence)


\paragraph{Ablation studies.}
We explore how self-positioning positions (SP) and disentangled attention contribute to SPA.
\Cref{table5} shows the final results on SONN, where the baseline~(\textit{w/o SPA}) learns only with local point attention.
In the case of \textit{w/o self-positioning}, we use FPS to randomly select a small set of points for cross-attention, and for \textit{w/o disentangled attention}, we only adopt the spatial kernel function $g$.
Our model with all the components of SPA achieves the best performance of 88.6\% in overall accuracy.
This superior performance verifies that every component is crucial for SPA.
In particular, when we use FPS instead of SP, the performance is even worse than the baseline as overall accuracy dropped from 87.9\% to 87.7\%.
This observation suggests the positions of SP points \textit{matter} for global cross-attention.
Rather than simple sampling, our learnable approach successfully locates SP points and makes global cross-attention effective.
Next, with \textit{w/o disentangled attention}, the performance gain in OA is minimal (0.3\%) over the baseline compared to using disentangled attention (0.7\%).
It indicates that disentangled attention improves the descriptive power by filtering semantically irrelevant information.

\paragraph{Attention types and semantic relation $\mathcal{R}$.}
In \Cref{table6}, we conduct experiments to compare the models with different attention types (Standard attention in Transformer~\cite{vaswani2017attention} and our CWPA) and semantic relations ($\mathcal{R}(\mathbf{f}_q, \mathbf{f}_k) = $ $\mathbf{f}_k$, $\mathbf{f}_q+\mathbf{f}_k$, $\mathbf{f}_q\odot\mathbf{f}_k$, and $\mathbf{f}_q-\mathbf{f}_k$).
The models adopting the CWPA outperform the model with the standard attention, which shows that the channel-wise point attention operation is more powerful to represent point clouds compared to the standard attention.
Furthermore, the results demonstrate that Sub~($\mathbf{f}_q-\mathbf{f}_k$) is most appropriate to model the semantic relation between points. 

% \SH{We verify that every component is critical for SPA as utilizing both components achieves the best performance of 85.1\% in overall accuracy.}


% SP, disentangled 강조, 키워드 살리기, represntative, long-range 

% This observation suggests positions learned by SP
% Also, We can know that learning positions of focal points is more effective than simply locating them via heuristic without considering semantic information.  
% Also, in the case of using only the spatial kernel, the performance gain is minimal (0.2\%) over the baseline compared to using disentangled attention (0.6\%).

% \SH{
% To better understand the contribution of each component in our model, we conduct ablation studies on PB\_T50\_RS variant of ScanObjectNN. 
% % Specifically, we measure the effect of SPA, self-positioning, and disentangeld attention  of the SPA.
% % The results are provided in~\Cref{table5}.
% \Cref{table5} reports the performance after excluding each component of SPoTr.
% The table shows that all components contribute to improve performance.
% % Compared to SPoTr with the model without SPA, SPA achieves 0.6\% improvement on overall accurac.
% % Model A is set to SPoTr block without SPA while model B is SPoTr block without GPA.
% % Both models show better performance compared to the other baselines in \Cref{table1}.
% % \paragraph{Ablation studies.} 
% % All SPoTr variants without some components show performance drops compared to the full architecture, which means that each component contributes to the performance improvement of SPoTr. 
% % Compared to model C, which uses FPS for determining the position of the focal point instead of self-positioning~(SP), model E gets 1.9\% higher overall accuracy~(OA).
% In particular, compared to SPoTr and  
% This indicates that the focal points are adaptively located with the self-positioning mechanism. 
% % Also, we can observe that disentangled attention kernel contributes the improvement of performance by comparing SPoTr and the model without semantic kernel $h\left(\cdot\right)$. 
% Lastly, we observe that using the disentangled attention kernel~(Model E), instead of spatial kernel~(Model D), improves the overall accuracy with 0.4\%.
% }
%by comparing SPoTr and the model without $h$.  
% For the model without self-positioning~(SP), we use farthest point sampling~(FPS) for selecting global points.
% Finally, the best accuracy 85.1\% is obtained when applying both LPA and SPA. 
% As mentioned in~\Cref{sec:3.2}, we think that the performance improvement is achieved by capturing local structural information and resolving the limitation of the sole local-attention.}
% Otherwise, if model considers only global information without LPA, it achieves 84.0\% which is still surpassing the previous state-of-the-art models. 
% Further, we deeper explore the self-positioning and bilateral filter in SPA.
% \begin{itemize}
%     \item LGPA
%     \item DGPA
% \end{itemize}


% \paragraph{Number of Self-positioning Points.}
%relation between \# of points -> graph로 그림하나 넣기


%keyword : scalability, long-range dependency 
\section{Complexity Analysis}
\label{sec:complexity_analysis}

{\bf Size bounds.} For a join query $Q$, its hypergraph $H(Q)$ has one node per variable in $Q$ and one hyperedge per relation in $Q$.  Figures~\ref{fig:example_intro_varorder} depicts a query hypergraph.

An edge cover is a subset of (hyper)edges of $H(Q)$ such that each node appears in at least one edge. Edge cover can be formulated as an integer programming problem by assigning to each edge $R_i$ a weight $w_{R_i}$ that can be $1$ if $R_i$ is part of the cover and $0$ otherwise. The size of an edge cover upper bounds the size of the query result, since the Cartesian product of the relations in the cover includes the
query result: $|Q(\db)| \leq |R_1|^{w_{R_1}}\cdot\ldots\cdot|R_n|^{w_{R_n}}$, where the database $\db$ is $(R_1,\ldots,R_n)$. By minimizing the size of the edge cover, we can obtain a lower upper bound on the size of the query result. This bound becomes tight for fractional weights~\cite{AGM:2013}.  Minimizing the sum of the weights thus becomes the objective of a linear program.

\begin{definition}[\cite{AGM:2013}]\label{def:agm}
Given a join query $Q$ over a database $(R_1,\ldots,R_n)$, the {\em fractional edge cover number} $\rho^*(Q)$ is the cost of an optimal solution to the linear program with variables $(w_{R_i})_{i\in[n]}$ representing weights of $(R_i)_{i\in[n]}$:
\begin{flalign*}
\textrm{minimize} &\prod_{i\in[n]} |R_i|^{w_{R_i}}\\
\textrm{subject to} &\sum_{R\textrm{ is relation of } X} w_R \geq 1~~\textrm{for each variable } X \\
&~~~\forall i\in[n]: \omega_{R_i}\geq 0.
\end{flalign*}
\end{definition}

\begin{example}
\em
Consider the triangle query:
\begin{align*}
Q_{\vartriangle} = R(A,B), S(B,C), T(C,A)
\end{align*}
Figure~\ref{fig:triangle_hypergraph_viewtree} gives the hypergraph of $Q_{\vartriangle}$. The linear program is: 
\begin{flalign*}
\textrm{\em minimize} & \quad |R|^{w_{R}} \cdot |S|^{w_{S}} \cdot |T|^{w_{T}} \\ 
\textrm{\em subject to} & \quad
\begin{tabular}[t]{@{}c@{\hspace*{.5em}}c@{\hspace*{.25em}}c@{\hspace*{.25em}}c@{\hspace*{.25em}}c@{\hspace*{.25em}}c@{\hspace*{.25em}}c}
$A:$ & $w_{R}$ & & & $+$& ${w_{T}}$& $\geq 1$ \\
$B:$ & $w_{R}$ & $+$ & ${w_{S}}$ & & & $\geq 1$ \\
$C:$ & & & $w_{S}$ & $+$ & ${w_{T}}$ & $\geq 1$
\end{tabular}
\end{flalign*}
% 
For $|R|=|S|=|T|=N$, setting $w_{R}=w_{S}=w_{T}=1/2$ gives the optimal solution $\rho^*(Q_{\vartriangle}) = N^{3/2}$. Consequently, the query result has $\bigO{N^{3/2}}$ tuples. This bound is tight in the sense that there exist classes of databases for which the result size is at least $\Omega(N^{3/2})$. For the acyclic query $Q$ in Section~\ref{sec:introduction}, setting the weights $1$ to each of the three relations gives $\rho^*(Q)=N^3$ if all relations have size $N$.
\punto
\end{example}

\nop{
Cardinality constraints can be used to lower the size bounds of query results. For instance, if the number of distinct $A$-values in $R(A,B)$ is $k \ll N$, then we can refine  $Q_\vartriangle$ as $R(A,B),S(B,C),T(C,A),U(A)$ with the new size bound $\rho^*(Q_\vartriangle) = N \cdot k$, where $w_{S}=1$ and $w_{U}=1$.

Join selectivities can also be incorporated to obtain a size {\em estimate} (in contrast to an upper bound). For instance, assume the selectivity of the join on $A$ between $R$ and $T$ is very low: $sel(A) = \frac{|R(A,B),T(C,A)|}{|R|\cdot|T|} = \frac{k}{N}$. Then, we consider a relation $U(A,B,C)=R(A,B),T(C,A)$ whose size estimate is $k \cdot N$ and use this as a cardinality constraint to obtain an estimate of $k \cdot N$ for $Q_\vartriangle$'s size since the size of the join of $S$ and $U$ cannot exceed the size of $U$.
}

Similarly to $\rho^*(Q)$, the {\em factorization width} $\fw(Q)$ governs the sizes of the factorized results of a join query $Q$~\cite{Olteanu:FactBounds:2015:TODS}. In a factorized join over a variable order $\omega$, the values of a variable $X$  depend on the tuples of values of its $\mathit{key}(X)$ variables and are independent of the values for other variables. A tight bound on this number is then given by the size of a join query that covers the variables in $\mathit{key}(X)\cup\{X\}$. We denote this restriction of $Q$ by $Q_{\mathit{key}(X)\cup\{X\}}$. An upper bound on the size of the factorization is then given by the maximum over all variables in $\omega$ of their number of values. This can be improved by going over all possible variable orders of $Q$ and taking the minimum upper bound. This is the factorization width of the query.

\begin{definition}\label{def:fw}
Given a join query $Q$, the {\em factorization width} of $Q$ is  $\fw(Q) = \min_{\omega\in\Omega(Q)} \max_{v\in\mathit{vars}(Q)} \rho^*(Q_{\mathit{key}(X)\cup\{X\}})$.
\end{definition}

\begin{example}\em
For acyclic queries $Q$ over relations $R_1,\ldots,R_n$, $\fw(Q)=\max_{i\in[n]}(|R_i|)$, while $\rho^*(Q)$ can be as much as $\prod_{i\in[n]}|R_i|$ as in our running example. Here are examples of restrictions of our natural join $Q$ in Section~\ref{sec:intro_example}: $\mathit{key}(D)\cup\{D\}=\{C,D\}$ is covered by the query restriction $Q_{\{C,D\}}$ that is the relation $T$; $\mathit{key}(C)\cup\{C\}=\{A,C\}$ is covered by the query restriction $Q_{\{A,C\}}$ that is the relation $S$. For the triangle query $Q_\vartriangle$ and variable order $A-B-C$: $\mathit{key}(C)\cup\{C\}=\{A,B,C\}$ is covered by $Q_\vartriangle$, while $\mathit{key}(B)\cup\{B\}=\{A,B\}$ is covered by relation $R$.
\punto
\end{example}

For any join query $Q$, its factorization width is the fractional hypertree width~\cite{Olteanu:FactBounds:2015:TODS}, a parameter that captures tracta\-bility for a host of computational problems~\cite{FAQ:PODS:2016}.

\begin{proposition}
\label{prop:factorization}
Given a join query $Q$, for every database $\db$, the result $Q(\db)$ admits:
\begin{itemize}
\item a flat representation of size $\bigO{\rho^*(Q)}$~{\em\cite{AGM:2013}};
\item a factorized representation of size $\bigO{\fw(Q)}$~{\em\cite{Olteanu:FactBounds:2015:TODS}}.
\end{itemize}

There are classes of databases $\db$ for which the above size bounds are tight. The flat and factorized representations of $Q(\db)$ can be computed worst-case optimally{\em~\cite{Ngo:SIGREC:2013,Olteanu:FactBounds:2015:TODS}}.
\end{proposition}


\subsection{Dynamic Factorization Width}
\label{sec:dynamic_width}

\milos{Doesn't consider other rings (only LR), indicator projections, and factorizable updates}

As in the non-incremental case, different variable orders may lead to wildly different performance of our IVM approach. In this section, we settle the question of which variable orders can best support IVM under updates to a given set of relations and thereby pinpoint the complexity of maintaining query results under updates. This is captured by a novel notion called {\em dynamic factorization width}, which is a refinement of the factorization width.

We first recall the complexities in the non-incremental case. There, we only materialize the root view of a view tree over a variable order with the smallest factorization width, and we thus have the time data complexity $\bigO{\fw(Q)}$ for computing factorized joins~\cite{Olteanu:FactBounds:2015:TODS} and aggregates over them~\cite{BKOZ:PVLDB:2013,FAQ:PODS:2016}; for cofactor matrices over factorized joins, there is an additional $\bigO{m^2}$ factor, since the sizes of these matrices can be quadratic in the number $m$ of variables (features)~\cite{SOC:SIGMOD:2016}. The space complexity is $\bigO{1}$ or $\bigO{m^2}$ to store the aggregate or cofactor matrix in addition to the database (modulo logarithmic factors in the data size for data iterators).

We next discuss the IVM case. Let $Q$ be any join query. For any variable order $\omega \in \Omega(Q)$, let $\tau(\omega)$ be the view tree inferred from $\omega$. This view tree has exactly one leaf for each relation symbol in $Q$.

We consider updates to relations whose relation symbols in $Q$ form a set ${\mathcal{U}}$; a relation may have several relation symbols  if it is involved in self-joins in $Q$, in which case all of them are in ${\mathcal{U}}$. For a relation symbol $R\in{\mathcal{U}}$, let $\Upsilon_{\tau(\omega)}(R)$ be the set of views that are ancestors of the leaf $R$ in $\tau(\omega)$, i.e., it consists of all the views (recursively) defined using $R$. 

The time needed to compute the delta for a view $\VIEW[keys]{V^{@X}_{rels}}$ is upper bounded by that of a join query $Q^{\sf rels}_{\sf keys \cup \{X\}-\sigma(R)}$ over relations in {\sf rels} that cover $X$ and the variables in {\sf keys} but excluding the variables in $R$. The reason for the exclusion is that a single-tuple update to $R$ binds the variables in $R$ to constants. The overall time to compute the deltas of all views in $\Upsilon_{\tau(\omega)}(R)$ is then
\begin{align*}
T(\omega,R) = \sum_{\VIEW[keys]{V^{@X}_{rels}}\in\Upsilon_{\tau(\omega)}(R)} \rho^*(Q^{\sf rels}_{{\sf keys\cup\{x\}}-\sigma(R)}).
\end{align*}

We are now ready to define the dynamic factorization width that captures the time complexity of incremental maintenance of $Q$ under updates to relations in ${\mathcal{U}}$.

\begin{definition}
Given a join query $Q$ and a set of relation symbols ${\mathcal{U}}$ in $Q$. Then, the {\em dynamic factorization width} of $Q$ and ${\mathcal{U}}$ is $\dfw(Q,{\mathcal{U}}) = \min_{\omega\in\Omega(Q)}\max_{R\in\mathcal{U}} T(\omega,R).$
\end{definition}

\begin{theorem}
Given a query $Q$ with $m$ variables, database $\db$, a payload ring $\RING$, and a set of relations ${\mathcal{U}}$ in $\db$. The time complexity of incrementally maintaining the result of $Q$ over the ring $\RING$ under single-tuple updates to relations in ${\mathcal{U}}$ is $\bigO{\dfw(Q,{\mathcal{U}})\cdot T_\RING}$, where $T_\RING$ is $\bigO{1}$ for rings of numbers and $\bigO{m^2}$ for the degree-$1$ matrix ring.
\end{theorem}

\begin{example}\label{ex:time-complexity}
\em
For our query $Q$ in Section~\ref{sec:intro_example} and database $\db$, the (static) factorization width is $\fw(Q)=O(|R|+|S|+|T|)$. Under single-tuple updates to relations in a set ${\mathcal{U}}_1\subseteq\{R,S\}$, the dynamic factorization width is $\dfw(Q,{\mathcal{U}}_1)=1$ since there are no free variables of the views over $R$ or $S$ in the variable order in Figure~\ref{fig:example_intro_varorder}. This means that we can maintain the result of a sum aggregate over $Q$ in $\bigO{1}$ time under ${\mathcal{U}}_1$ updates. The same holds for ${\mathcal{U}}_2\subseteq\{S,T\}$, i.e., $\dfw(Q,{\mathcal{U}}_2)=1$, as supported by the variable order $C-\{ D, A - \{ B, E \}\}$. However, $\dfw(Q,{\mathcal{U}}_3)=\bigO{|\db|}$ for ${\mathcal{U}}_3=\{R,S,T\}$ since there is no variable order without free variables above all three relations and some variable orders have one free variable above at least one of the three relations. Under the variable order in Figure~\ref{fig:example_intro_varorder}, $\dfw(Q,{\mathcal{U}}_3)=\min(|R|,|S|)$.

The triangle query $Q_\vartriangle$ has the (static) factorization width $\fw(Q_\vartriangle)= \rho^*(Q_\vartriangle)$. For any relation $U \in \{ R, S, T \}$, the dynamic factorization width is $\dfw(Q,\{ U \})=1$ as supported by a path variable order that has the variables in $U$ as prefix. We can thus maintain an aggregate over the triangle query in $\bigO{1}$ under single-tuple updates to exactly one of its three relations. For updates to at least two relations ${\mathcal{U}}_4$, $\dfw(Q,{\mathcal{U}}_4)=O(|\db|)$. For instance, assume a variable order $A-B-C$. We need to cover: no variable under updates to $R$; one of the variables $A$ or $B$ under updates to $S$ or $T$ respectively (the case for other permutations of this variable order is analog). Maintenance has thus lower time cost than recomputation.
\punto
\end{example}

We next analyze the space complexity $S(Q)$ of our approach. This is the sum of the sizes of the views in a view tree. The space needed by the keys of a view $\VIEW[keys]{V^{@X}_{rels}}$ is given by the fractional edge cover of a join query built using relation symbols {\sf rels} to cover the variables in {\sf keys}. To obtain the minimum size, we go over all variable orders of $Q$:
\begin{align*}
 S(Q) = \min_{\omega\in\Omega(Q)}\sum_{\VIEW[keys]{V^{@X}_{rels}}\in\tau(\omega)} \rho^*(Q^{\sf rels}_{\sf keys}).
\end{align*}

\begin{theorem}
Given a query $Q$ with $m$ variables, database $\db$, a payload ring $\RING$. The space complexity required by the materialization of a view tree for $Q$ over the ring $\RING$ is $\bigO{S(Q)\cdot T_\RING}$, where $T_\RING$ is $\bigO{1}$ for the sum ring and $\bigO{m^2}$ for the degree-$m$ matrix ring.
\end{theorem}

There are three differences between the formula $S(Q)$ and Definition~\ref{def:fw} of the factorization width $\fw(Q)$: (1) the use of summation vs. maximum, though the gap between them is linear in $m$ and thus independent of the database size; (2) the cover for $S(Q)$ can only use relation symbols of the view; (3) for $S(Q)$, we only need to cover $\sf keys$ and not also the variable at the view as in the case of $\fw(Q)$. The interplay of (2) and (3) can in fact make $S(Q)$ larger than $\fw(Q)$.
For acyclic queries, both complexities are linear if all relations have the same size and $S(Q)$ can be smaller than $\fw(Q)$ in case some relations are asymptotically smaller than others. 
For cyclic queries, however, $S(Q)$ can be larger than $\fw(Q)$. We show this for the triangle query $Q_\vartriangle$ and relations of the same size $N$. Under any variable order, there is a view of size $\bigO{N^2}$, whereas $\fw(Q_\vartriangle)=N^{3/2}$. For instance, for the variable order $A-B-C$, we materialize the view $\VIEW[A,B]{V^{@C}_{ST}} = \VSUM_{C} \VIEW[B,C]{S} \VPROD \VIEW[C,A]{T} \VPROD \VIEW[C]{\VLIFT_{C}}$, which may create $\bigO{N^2}$ pairs $(A,B)$ as we need both $S$ and $T$ to cover the variables $A$ and $B$. To avoid the large intermediate result, we join all three relations at the same time~\cite{Ngo:SIGREC:2013}, so as to cover $(A,B)$ using $R$. That would, however, require recomputation of this 3-way join for each update. This takes $\bigO{N}$ time since only two of the three variables are bound to constants. In contrast, our IVM approach trades off space for time: We need $\bigO{N^2}$ space but then support $\bigO{1}$ updates to one of the three relations (Example~\ref{ex:time-complexity}).

%%%%%%%%%%%%%%%%%%%%%%%%%%%%%%%%%%%%%%%%%%%%

% \paragraph{Efficiency comparison with baselines.}
\paragraph{Complexity analysis on SN-Part.} 
We analyze the space and time complexity to validate the computational efficiency of SPoTr during inference time with a batch size of 8.
For a baseline, SPA in SPoTr is replaced by the standard global self-attention (abbreviated in GSA) with CWPA. 
For a comparison with GSA requiring the quadratic complexity, we inevitably use the variants of SPoTr, where the channel size of each layer is reduced by $\times 1/4$.
For space complexity, we measure  the number of parameters and total memory usage, and for time complexity, we measure FLOPs and throughput performance.
\Cref{tab:complexity} empirically proves the efficiency over GSA.
For space complexity, GSA shares a similar number of parameters with SPA but introduces a large memory usage of 24.2 (GB). Instead, Our SPA only uses 2.5 (GB) (-89.7\%). Also, SPA largely reduces the time complexity from 114.0 GFLOPS with a throughput of 17.7 (shapes/s) to 10.8 GFLOPS (-90.5\%) with a throughput of 281.5 (shapes/s) ($\times$15.9).
% To investigate the efficiency of SPoTr, we compare our method with recent baselines~\cite{ma2022rethinking, ran2022surface} on ScanObjectNN~\cite{uy2019revisiting}. Specifically, we use DeepSpeed~\cite{rasley2020deepspeed} library as a profiler for calculating the number of parameters and FLOPS during inference. For comparison, we also opt a light version of SPoTr (SPoTr*), where the channel size of each layer is 2/3. The results are summarized in~\Cref{tab:efficiency}. It is worth noting that SPoTr achieves the best performance (88.6\%) with fewer parameters of 3.3(M) than 13.2(M) of PointMLP and 6.8(M) of RepSurf. Further, although SPoTr* only requires 1.6(M) parameters and 5.5 GFLOPS, it still shows significant gains over the previous best methods (+2.2\%). In short, we demonstrate that SPoTr is a computation-efficient and memory-efficient method. 
% \begin{table}[t!]
    \centering
    \caption{
    \textbf{Efficiency comparison of optimization algorithms.}
    R@1 scores evaluated on MSRVTT-7k for video retrieval are recorded.
    Multi-task learning simultaneously trains all tasks with even loss weights. 
    CG and FP are abbreviations of conjugate gradient and fixed-point optimization. 
    In terms of time costs, average training time per epoch is reported. 
    $^\dagger$ refers to our optimization algorithm which approximates $\nabla^2_w \aux$ as the identity matrix $\mathrm{I}$.}
    \begin{adjustbox}{width=\linewidth}
    \begin{tabular}{l |c| c  c}
        \toprule
        \textbf{Method}  & \textbf{Opt. Scheme}  & \textbf{R@1} &  \textbf{Time} \\
        \midrule
        \midrule
        Multi-task Learning   & 
        - &  
        26.1 \scriptsize(+0.0)    & 
        547 \scriptsize(+0.0\%) \\
        
        \textbf{MELTR} + Meta-Weight Net~\cite{shu2019meta}  & 
        ITD &  
        27.3 \scriptsize(\textcolor{red}{+1.2})  & 
        1,296 \scriptsize(\textcolor{red}{+136.9\%}) \\ 
        
        \textbf{MELTR} + StocBIO~\cite{ji2021bilevel} & 
        N/A  &  
        26.8 \scriptsize(\textcolor{red}{+0.7})   &   
        686 \scriptsize(\textcolor{red}{+25.4\%})\\
        
        \textbf{MELTR} + CG & 
        AID-CG &  
        28.0 \scriptsize(\textcolor{red}{+1.9})   &   
        624 \scriptsize(\textcolor{red}{+14.1\%})\\
        
        \textbf{MELTR} + AuxiLearn~\cite{navon2020auxiliary} &  
        AID-FP    &  
        27.9 \scriptsize(\textcolor{red}{+1.8})    &
        638 \scriptsize(\textcolor{red}{+16.6\%})      \\
        
        \textbf{MELTR} + \textbf{AID-FP-Lite}$^\dagger$ & 
        AID-FP &  
        28.5 \scriptsize(\textcolor{red}{+2.4})   &   
        574 \scriptsize(\textcolor{red}{+4.9\%})\\
        \bottomrule
    \end{tabular}
    \end{adjustbox}
    \label{tab:efficiency}
    \vspace{-3mm}
\end{table}

% SPA introduces 188 seconds additional time with $+1.8$MB of memory usage, whereas 
% In sum, our SPA reduces space and time overhead about ${3.5}\times$ and $5\times$ lower than GSA, respectively.

% \SH{
% To probe the efficiency of SPA, we analyze the space and time complexity on S3DIS during training with a single Nvidia RTX A6000 GPU.
% For a baseline, SPA in SPoTr is replaced with global self-attention(GSA).
% For space complexity, we measure memory usage with a batch size of 8.
% Also, the time complexity is measured by considering the latency per epoch of SPA and GSA block, respectively.
% % To probe the efficiency of SPA, we compare SPoTr with the model
% % From the \Cref{table6}
% SPA introduces 188(s) additional computational time with 10.8(GB) of memory usage.
% Whereas, GSA requires much more extra costs: 35.6(GB) memory usage and +910 seconds.
% Remarkably, our SPA reduces extra space and time overhead with about ${3.5}\times$ and $5\times$ lower than GSA, respectively.
% % The model B is SPoTr, which uses both LPA and SPA.  
% % From the table, 
% % In \Cref{table6}, we report the memory usage and  
% % For complexity analysis, using single Nvidia RTX A6000, we measure the cost on S3DIS with the batch size of 8 during training.
% % We measure the costs on a single Nvidia RTX A6000 GPU with the batch size of 8 during training for complexity analysis.
% % Baseline is model A that only considers local shape context with LPA.
% % Compared to baseline, model B~(SPoTr) introduces 188(s) additional computation with +1GB increment of memory usage. 
% % Whereas, with model C, where we add global self-attention(GSA) instead SPA, it requires much more extra costs: +25.8GB memory and +910 seconds. 
% % Remarkably, our SPA reduces extra space and time overhead with ${26}\times$ and $5\times$, respectively, lower cost than model C, respectively.
% % Whereas, although global self-attention(GSA) is natural for capturing global shape context, it is ineffective in both space and time complexity. 
% % As shown in table, compared to baseline model without SPA, 
% % we observe that SPoTr $w/$ SPA outperforms SPoTr $w/$ GSA with fewer latency. 
% }
% %Flops & time complexity table using shapenet part

