% CVPR 2023 Paper Template
% based on the CVPR template provided by Ming-Ming Cheng (https://github.com/MCG-NKU/CVPR_Template)
% modified and extended by Stefan Roth (stefan.roth@NOSPAMtu-darmstadt.de)

\documentclass[10pt,twocolumn,letterpaper]{article}

\usepackage[accsupp]{axessibility}
%%%%%%%%% PAPER TYPE  - PLEASE UPDATE FOR FINAL VERSION
% \usepackage[review]{cvpr}      % To produce the REVIEW version
\usepackage{cvpr}              % To produce the CAMERA-READY version
%\usepackage[pagenumbers]{cvpr} % To force page numbers, e.g. for an arXiv version

% Include other packages here, before hyperref.
\usepackage{graphicx}
\usepackage{amsmath}
\usepackage{amssymb}
\usepackage{booktabs}
\usepackage{color}
\usepackage{colortbl}
\usepackage{multirow}
% \usepackage{subfigure}


% It is strongly recommended to use hyperref, especially for the review version.
% hyperref with option pagebackref eases the reviewers' job.
% Please disable hyperref *only* if you encounter grave issues, e.g. with the
% file validation for the camera-ready version.
%
% If you comment hyperref and then uncomment it, you should delete
% ReviewTempalte.aux before re-running LaTeX.
% (Or just hit 'q' on the first LaTeX run, let it finish, and you
%  should be clear).
\usepackage[pagebackref,breaklinks,colorlinks]{hyperref}


% Support for easy cross-referencing
\usepackage[capitalize]{cleveref}
\crefname{section}{Sec.}{Secs.}
\Crefname{section}{Section}{Sections}
\Crefname{table}{Table}{Tables}
\crefname{table}{Tab.}{Tabs.}
\crefname{figure}{Fig.}{Figs.}
\Crefname{figure}{Figure}{Figures}
\crefname{equation}{Eq.}{Eqs.}
\Crefname{equation}{Equation}{Equations}

%%%%%%%%%%%%
\newtheorem{remark}{Remark}

\newcommand{\yx}[1]{\textcolor{red}{yx: #1}}

%%%%%%%%% PAPER ID  - PLEASE UPDATE
\def\cvprPaperID{6339} % *** Enter the CVPR Paper ID here
\def\confName{CVPR}
\def\confYear{2023}


\begin{document}

%%%%%%%%% TITLE - PLEASE UPDATE
% \title{SPoTr: Transformer with Self-positioning Points for Point Cloud Understanding}
\title{Self-positioning Point-based Transformer for Point Cloud Understanding}
% \title{SPoTr: Self-positioning Point-based Transformer for Point Cloud Understanding}

\author{Jinyoung Park$^1$\thanks{First two authors have equal contribution.} , Sanghyeok Lee$^{1*}$, Sihyeon Kim$^1$, Yunyang Xiong$^2$, Hyunwoo J. Kim$^1$\thanks{is the corresponding author.}\\
$^1$Korea University, $^2$Meta Reality Labs\\
{\tt\small \{lpmn678, cat0626, sh\_bs15, hyunwoojkim\}@korea.ac.kr}\\
{\tt\small yunyang@fb.com}
% For a paper whose authors are all at the same institution,
% omit the following lines up until the closing ``}''.
% Additional authors and addresses can be added with ``\and'',
% just like the second author.
% To save space, use either the email address or home page, not both
% \and
% Second Author\\
% Institution2\\
% First line of institution2 address\\
% {\tt\small secondauthor@i2.org}
}
\maketitle
% \documentclass[a4paper]{amsart}%[a4paper]
% %%%%% GENERAL MATH COMMANDS
% Reals
\newcommand{\R}{{\mathbb R}}
% Integers
\newcommand{\Z}{{\mathbb Z}}
% Naturals
\newcommand{\N}{{\mathbb N}}
% Expectation
\DeclareMathOperator*{\E}{\mathbb{E}}
% ^th notation
\newcommand{\tth}{^{\text{th}}}
% Small dots for integer range [a .. b]
\newcommand{\sdots}{\,..\,}
% Vectorized version of matrix
\newcommand{\matvec}{\mbox{vec}}

% := sign
\newcommand{\defeq}{\vcentcolon=}
% Zero function
\newcommand{\zf}{\mathbf{0}}
% Vector of ones
\newcommand{\ones}{\mathbf{1}}

% Argmin and argmax definitions
\DeclareMathOperator*{\argmax}{arg\,max}
\DeclareMathOperator*{\argmin}{arg\,min}


%%%%% PROBLEM STATEMENT NOTATION 
% \newcommandtwoopt{\St}[2][t][]{{S_{#1}^{#2}}} % State
\newcommand{\task}[1][i]{{\mathcal{T}_{#1}}} % Task, optionally takes index
\newcommand{\tasks}{\{ \task \}_{i=1}^N}
\newcommand{\losst}[1][i]{{l_{#1}}}
\newcommand{\lossv}[1][i]{{l_{#1}^{\textrm{val}}}}
\newcommand{\tasktarget}{{\mathcal{T}_{\textrm{target}}}}
\newcommand{\lossttarget}{l_{\textrm{target}}}
\newcommand{\lossvtarget}{l_{\textrm{target}}^{\textrm{val}}}
\newcommand{\lossttargetit}{l_{\textrm{target}}^{(k)}}
\newcommand{\losstotal}{l^{\textrm{total}}}
\newcommand{\lossopt}{l^*}

\newcommand{\thetait}[2]{\theta_{#1}^{(#2)}}
\newcommand{\phit}[1]{\phi^{(#1)}}
\newcommand{\hist}[2]{S_{#1}^{(#2)}}
\newcommand{\grad}[2]{G_{#1}^{(#2)}}

\newcommand{\Alg}{\textup{\textbf{Opt}}}
\newcommand{\MetaAlg}{\textup{\textbf{MetaOpt}}}

%%%%% Theorems
\newtheoremstyle{mytheoremstyle} % name
    {\topsep}                    % Space above
    {\topsep}                    % Space below
    {\itshape}                   % Body font
    {}                           % Indent amount
    {\scshape}                   % Theorem head font
    {.}                          % Punctuation after theorem head
    {.5em}                       % Space after theorem head
    {}  % Theorem head spec (can be left empty, meaning ‘normal’)
\theoremstyle{mytheoremstyle}
\theoremstyle{plain}
\newtheorem{theorem}{Theorem}
\newtheorem{proposition}{Proposition}
\newtheorem{assumption}{Assumption}
\newtheorem{definition}{Definition}
\newtheorem{lemma}{Lemma}
\theoremstyle{remark}
\newtheorem{remark}{Remark}

%
% \begin{document}
% \section{notation}\label{sec:notation}
For a positive integer $d$, we define $[d]:=\{1,2,\ldots,d\}$. 
The set of non-negative integers is denoted by $\NN:=\{0,1,2,\ldots\}$.
The cardinality of a set $S$ is denoted by $|S|$.
%Operations on $[d]$ cyclically.

Our \emph{graphs} are finite and undirected. We allow multiple edges and loops. A \emph{simple graph} is a graph without multiple edges or loops. 


A \emph{plane map} is a connected planar graph drawn in the plane without edge crossing, considered up to continuous deformation. 
The \emph{faces} of a plane map are the connected components of the complement of the graph. The infinite face is called \emph{outer face}, and the finite faces are called \emph{inner faces}. The vertices and edges incident to the outer face are called \emph{outer} while the other are called \emph{inner}. 
The numbers $\vv$, $\ee$ and $\ff$ of vertices, edges and faces of a plane map are related by the \emph{Euler relation}  $\vv+\ff=\ee+2$. 


We now define the class of plane maps which will be relevant for this article.
\begin{definition}\label{def:d-adapted}
A \emph{$d$-map} is a plane map such that the inner faces have degree at most $d$, and the outer face has degree $d$ and is incident to $d$ distinct vertices (in other words, the contour of the outer face is a simple cycle). 
We will assume that the outer vertices of a $d$-map are labeled $v_1,v_2,\ldots, v_d$ in clockwise order along the boundary of the outer face. %, as in Figure \ref{???}.\\
A \emph{$d$-adapted map} is a $d$-map such that any simple cycle which is not the contour of a face has length at least $d$.\\
\end{definition}
We point out that $d$-adapted maps are necessarily 2-connected (because a cut point in a $d$-map $G$ implies the existence of a simple cycle of length strictly less than the degree of an inner face of $G$, which shows that $G$ is not $d$-adapted).


In a plane map, a \emph{corner} is the sector delimited by two consecutive (half-)edges around a vertex. It is called an \emph{inner corner} if it lies in an inner face, and an \emph{outer corner} otherwise.
The \emph{degree} of a vertex or face is its number of incident corners. A  \emph{$d$-angulation} is a plane map with all faces of degree $d$. A \emph{$d$-angulation of the $k$-gon} is a plane map with inner faces of degree $d$, and outer face of degree $k$. 
A graph is \emph{bipartite} if it admits a bicoloring of its vertices such that adjacent vertices have different colors. It is known that a plane map is bipartite if and only if all its faces have even degree. For $k\geq 2$, a graph is called \emph{$k$-connected} if it is connected and the deletion of any subset of $(k-1)$ vertices does not disconnect it (loops are forbidden for $k\geq 2$, multiple edges are forbidden for $k\geq 3$). 




Let $G$ be an undirected graph. An \emph{arc} of $G$ is an edge $e$ of $G$ together with a chosen orientation of $e$ (so each edge of $G$ correspond to two arcs). The arc \emph{opposite} to an arc $a$, denoted by $-a$, is the arc corresponding to the same edge as $a$ but with the opposite direction. 
The endpoints of an arc $a$ are called the \emph{initial} and \emph{terminal} vertices of $a$ (with $a$ oriented from the initial vertex to the terminal vertex).  If $v$ is the initial (resp. terminal) vertex of the arc $a$, then we say that $a$ is an \emph{outgoing arc} (resp. \emph{ingoing arc}) at $v$. 
\\

%In a graph, a \emph{walk} (of length $k$) is a sequence $v_1,e_1,v_2,\ldots,e_k,v_{k+1}$ that alternates vertices and edges, such that $e_i$ connects $v_i$ to $v_{i+1}$ for $i\in[k]$. It is called a \emph{closed walk} if $v_1=v_{k+1}$. 
%\OB{Made a change in the def of walk (talking about arcs instead). Should we call them ``paths'' rather than ``walks''?}
A \emph{path} in an undirected graph $G$ is a sequence of arcs $a_1,a_2,\ldots,a_k$ such that the terminal vertex of $a_i$ is the initial vertex of $a_{i+1}$ for all $i\in[k-1]$. It is called a \emph{closed path} if the terminal vertex of $a_k$ is the initial vertex of $a_1$. A \emph{cycle} is a (cyclically ordered) closed path. A path or cycle is called \emph{simple} if it does not pass twice by the same vertex. The \emph{girth} of a graph is the minimum length of its simple cycles.   In a plane map, a closed path formed by the arcs around a face is called \emph{contour} of that face. It is known that face contours are simple cycles if the plane map is 2-connected. 
A simple cycle on a plane map is called \emph{counterclockwise} (resp. \emph{clockwise}) if the direction of arcs is counterclockwise (resp. clockwise) around the cycle.

Let $G$ be a graph.  Given an orientation of $G$, a \emph{directed path} (resp. \emph{directed cycle}) is a path (resp. cycle) $a_1,a_2,\ldots,a_k$ such that every arc $a_i$ is oriented according to the orientation of $G$.
A \emph{weighted orientation} of $G$ is an assignment of a non-negative integer to each arc of $G$. Given a weighted orientation $\cW$ of $G$, we call \emph{weight} of an edge the sum of the weights of the two corresponding arcs. 
Weighted orientations are a generalization of the classical (unweighted) orientations of $G$. Indeed the ``unweighted'' orientations of $G$ can be identified to the weighted orientations of $G$ such that the weight of every edge is 1 (for each edge, the arc of weight 1 is taken as the orientation of the edge). The \emph{outgoing weight} (shortly, the \emph{weight}) of a vertex $v$ is the sum of the weights of the arcs going out of $v$. Given a weighted orientation, we call \emph{positive path} (resp. \emph{positive cycle}) a path (resp. cycle) $a_1,a_2,\ldots,a_k$ such that the weight of every arc is positive (this generalizes the notion of \emph{directed path} and \emph{directed cycle}).  




A \emph{tree} is a connected, acyclic graph. For a tree $T$ with a vertex $v$ distinguished as its \emph{root}, we apply the usual ``genealogy'' vocabulary about trees, where $v$ is an \emph{ancestor} of all the other vertices, and every non-root vertex incident to $T$ has a \emph{parent} in $T$, etc. 
We say that we \emph{orient the tree $T$ toward its root} by orienting every edge from child to parent. With this orientation, every non-root vertex of $T$ is incident to one outgoing edge in $T$ (the edge leading to its parent).
%\OB{changed: calling ``subtree'' instead of ``tree''}
A \emph{subtree} of a graph $G$ is a subset of edges of $G$ such that this set of edges together with the incident vertices forms a tree. A \emph{spanning tree} of $G$ is a subtree of $G$ incident to every vertex of $G$. 





%\end{document}

%%%%%%%%% ABSTRACT
\begin{abstract}
    \begin{abstract}
The current study investigated possible human-robot kinaesthetic interaction using a variational recurrent neural network model, called PV-RNN, which is based on the free energy principle.
Our prior robotic studies using PV-RNN showed that the nature of interactions between top-down expectation and bottom-up inference is strongly affected by a parameter, called the meta-prior, which regulates the complexity term in free energy.
% The current study examines how the behaviours of robots alter by changing the meta-prior $w$ in human-robot kinaesthetic interaction.
The current study examines how changing the meta-prior $w$ in the interaction phase affects the counter force generated when an experimenter attempts to induce movement pattern transitions familiar to the robot through its prior training.
The study also compares the counter force generated when trained transitions are induced by a human experimenter and when untrained transitions are induced.
Our experimental results indicated that (1) the human experimenter needs more/less force to induce trained transitions when $w$ is set with larger/smaller values, (2) the human experimenter needs more force to act on the robot when he attempts to induce untrained as opposed to trained movement pattern transitions.
Our analysis of time development of essential variables and values in PV-RNN during bodily interaction clarified the mechanism by which gaps in actional intentions between the human experimenter and the robot can be manifested as reaction forces between them.


%% Hiroki writing 2022-11-4
%Current study investigates the dynamics of the latent states during human-robot kinaesthetic interaction using PV-RNN.
%We have achieved to observe and analyse the internal state of an RNN model based on the free energy principle, during real-time human-robot interaction.
%Essential characteristics observed in the previous study of this variational recurrent neural network model, PV-RNN, is that by changing a meta prior $w$, the balance between the top-down intention and the bottom-up perceptual reality changes.
%In the current study, we examined how changing the weighting parameter $w$ between accuracy and complexity in free energy principle affects the humanoid robot's behaviour through human-robot interaction. We have conducted some human-robot kinaesthetic interaction experiments with various $w$ and quantitatively analysed the latent variable and the force applied to the humanoid robot. We have observed that the force required to change the robot's intention has increased, both when the top-down intention was strengthened by changing the $w$ and when corresponding switch of its primitive was against the experience of the RNN during its training. The study confirms through quantitative analysis that by increasing or decreasing the $w$ in PV-RNN, humanoid robot leads or follows the human counterpart during the human-robot kinaesthetic interaction.

\begin{comment}
Comment from Jun #2
・最後にQualitativeな結果(インパクト)が欲しい
・Current study investigates the problem on~と書き出すのが一般的
・最初の一文と最後の一文を対応させる
・最後の一文はもう少しAbstractかつ包括的に
\end{comment}

\begin{comment}
Comment from Jun #1
We investigated how the kinaesthetic human-robot interaction can affect the internal state of a model based on the free energy principle. 
=> how the internal state is affected is not the most important point in this study. This part should be rewritten.

The key function of this variational recurrent neural network model, PV-RNN, is that by changing a meta prior $w$, it takes a balance between the "complexity” term and the ”accuracy” term which corresponds to a top-down intention and a bottom-up perceptual reality in the free energy principle, respectively. 
=> This is not key function of PV-RNN. It is an essential characteristics observed in the previous study. The grammar after $w$ is something strange. Rewrite these.

This research has conducted a human-robot interaction experiment with a robotic agent in a kinaesthetic sense.
=> The sentence is not good. "in a kinaesthetic sense" is grammatically wrong.
MODIFIED => "In the current study human-robot interaction experiments using the kinaesthetic sense were conducted."

We investigated that when human forces the agent to switch primitives from one to another, larger force was required both when the human intention is conflictive against the top-down the intention of the agent and when the agent has a stronger top-down intention by modifying the $w$.
=> You should write the essential results of the experiments rather than what we investigated and also how these results could contribute to the studies on human-robot interaction.
\end{comment}

\end{abstract}
\end{abstract}

%%%%%%%%% BODY TEXT
\section{Introduction}
    \label{sec:1}
% Enabling convolutional neural networks (CNNs) to directly operate on point clouds has 
Point clouds have been widely applied in various areas such as autonomous driving, robotics, and augmented reality.  
Since the point cloud is an unordered set of points with irregular structures, adopting convolutional neural networks~(CNNs) on point clouds is challenging.
Some works devoted effort to transforming point clouds into regular structures, such as projection to multi-view images~\cite{su2015multi,chen2017multi} and voxelization~\cite{maturana2015voxnet,zhou2018voxelnet}.
Others have tried to preserve the structure and design a convolution on the point space~\cite{liu2019relation,thomas2019kpconv,li2018pointcnn,atzmon2018point,xu2018spidercnn,wu2019pointconv,xu2021paconv}.
% However, the ability to capture long-range dependencies is limited in most convolution-based approaches while it is crucial to understand global shape context, especially with noisy and real-world data~\cite{uy2019revisiting}.
However, the ability to capture long-range dependencies is limited in most convolution-based approaches, while it is crucial to understand global shape context, especially with real-world data~\cite{uy2019revisiting}.

Transformer~\cite{vaswani2017attention} tackled the long-range dependency issue in natural language processing and later it has been actively extended to 2D image processing~\cite{dosovitskiy2020image,liu2021swin,chu2021Twins,wang2021pyramid}.
Early works tried to replace convolutional layers with self-attention~\cite{dosovitskiy2020image,ramachandran2019stand,hu2019local,parmar2018image,chen2020generative,cordonnier2019relationship}, but they struggled with the quadratic computational cost of self-attention to the number of pixels.
To mitigate the scalability issue, self-attention in local neighborhoods~\cite{liu2021swin,wang2021pyramid} or approximating a self-attention with a reduced set of queries or keys~\cite{chu2021Twins,jaegle2021perceiver,zhu2020deformable} have been studied.
% To mitigate the scalability issue, several efforts have been made such as applying self-attention in local neighborhoods~\cite{liu2021swin,wang2021pyramid} or approximating it with a reduced set of queries or keys~\cite{chu2021Twins,jaegle2021perceiver,zhu2020deformable,xiong2021nystromformer}.
For point clouds, Point Transformer~\cite{zhao2021point} applies a local attention operation~(\Cref{fig:local_att}) and PointASNL \cite{yan2020pointasnl} employs a global attention module in a non-local manner~(\Cref{fig:global_att}).
Still, in point clouds, Transformer, which tackles both long-range dependency and scalability issues, has been less explored.

% \begin{figure}[!t]
\centering
\setlength{\abovecaptionskip}{0pt}
\setlength{\belowcaptionskip}{0pt}
\includegraphics[width=\linewidth]{fig/teaser.pdf}
\caption{\textbf{Our proposed workflow framework.} We seek to optimise, through a coordinate MLP, the mapping $\mathbf{\Phi}$ to align the coordinates between the source and target images. Our highlight is a new regulariser, whose effect is illustrated in the middle part. Our proposed conformal-invariant hyperelastic regulariser enforces volume presentation, controls changes in length and area, and ensures smoothness of deformation yielding to a better optimisation outcome.}
\label{FigOverviewFigOverview}
\vspace{0pt}
\end{figure}


\begin{figure}[t]
     \centering
     \begin{subfigure}[b]{0.156\textwidth}
         \centering
         \includegraphics[trim=70 70 70 70,clip,width=\textwidth]{Figures/local_att.pdf}
         \caption{Local attention}
         \label{fig:local_att}
     \end{subfigure}
     % \hfill
     \begin{subfigure}[b]{0.156\textwidth}
         \centering
         \includegraphics[trim=70 70 70 70,clip,width=\textwidth]{Figures/global_att.pdf}
         \caption{Global attention}
         \label{fig:global_att}
     \end{subfigure}
     \begin{subfigure}[b]{0.156\textwidth}
         \centering
         \includegraphics[trim=70 70 70 70,clip,width=\textwidth]{Figures/spotr_att.pdf}
         \caption{SP attention}
         \label{fig:spotr_att}
     \end{subfigure}
     % \hfill
     % \begin{subfigure}[b]{0.3\textwidth}
     %     \centering
     %     \includegraphics[width=\textwidth]{graph3}
     %     \caption{$y=5/x$}
     %     \label{fig:five over x}
     % \end{subfigure}
        \caption{{\textbf{Comparison of attention methods.} (a) Local attention, (b) Global attention, (c) Self-positioning point-based attention~(SP attention).}}
        \label{fig:attentions}

% \vspace{}
\end{figure}

In this paper, we propose \textbf{S}elf-\textbf{Po}sitioning point-based \textbf{Tr}ansformer (SPoTr) to capture both local and global shape contexts with reduced complexity.
% SPoTr consists of two attention modules: local attention and global attention, \yunyang{where the latter is a newly proposed attention mechanism to alleviate the complexity of global attention}.
SPoTr block consists of two attention modules: (i) \textit{local points attention}~(LPA) to learn local structures and (ii) \textit{self-positioning point-based attention}~(SPA) to embrace global information via self-positioning points. 
% where the latter is a newly proposed attention mechanism for point clouds to alleviate the complexity of global attention.
% To address the scalability issue, SPA computes global attention weights with only a small set of self-positioning focal points instead of the whole input points
SPA performs global attention by computing attention weights with only a small set of Self-Positioning points~(SP points) instead of the whole input points different from the standard global attention as illustrated in \Cref{fig:spotr_att}.
Specifically, SP points are adaptively located based on the input shape to cover the overall shape with only a small set of points. 
SP points learn its representation  considering both spatial and semantic proximity through \textit{disentangled attention}.
% Specifically, SPA aggregates information on each SP point through disentangled attention.
% As shown in the bottom row of \Cref{fig:sprf}, disentangled attention improves the descriptive power of the receptive field by considering both spatial and semantic proximity with separated kernels.
{Then, SPA non-locally distributes information of SP points to each input point.
% Also, we analyze that our SPoTr block has the stronger expressive power to represent point clouds by showing other point-based layers are its special case.
We also show that our SPoTr block generalizes set abstraction~\cite{qi2017pointnet++} with improved expressive power.}

Further, we propose SPoTr architecture for standard point cloud tasks (\eg, shape classification and semantic segmentation).
We conduct extensive experiments with three datasets: ScanObjectNN~\cite{uy2019revisiting}, SN-Part~\cite{snpart}, and S3DIS~\cite{armeni20163d}.
Our proposed method shows its effectiveness on all datasets compared to other attention-based methods.
In particular, our architecture achieves an accuracy improvement of 2.6\% over the previous best model in shape classification with a real-world dataset ScanObjectNN.
Additionally, we demonstrate the effectiveness and interpretability of self-positioning point-based attention with qualitative analyses.  

The \textbf{contribution} of our paper can be summarized as the following:
\begin{itemize}
    \item[\textbullet] We design a novel Transformer architecture (SPoTr) to tackle the long-range dependency issues and the scalability issue of Transformer for point clouds.
    \item[\textbullet] We propose a global cross-attention mechanism with flexible self-positioning points (SPA). SPA aggregates information on a few self-positioning points via disentangled attention and non-locally distributes information to semantically related points.
    \item[\textbullet] SPoTr achieves the best performance on three point cloud benchmark datasets (SONN, SN-Part, and S3DIS) against strong baselines.
    \item[\textbullet] Our qualitative analyses show the effectiveness and interpretability of SPA.
    % \item[\textbullet] We design a novel Transformer architecture (SPoTr), which captures both local and global shape information and tackles the scalability issue of Transformer for point clouds.
    % \item[\textbullet] We propose a global cross-attention mechanism with flexible self-positioning receptive fields (SPA). SPA aggregates information on each focal point via disentangled attention and non-locally distributes information to semantically related points.
    % \item[\textbullet] SPoTr achieves competitive performance regarding standard point cloud tasks such as shape classification, part segmentation, and scene segmentation.
    % \item[\textbullet] Our qualitative results demonstrate the effectiveness and interpretability of SPA.
\end{itemize}

% Still, scalable Transformer for long-range dependency in point cloud have been less explored.
% Still, point cloud works trying to emulate vision transformers have been less explored.
% - keyword 추가, scailable, vision transformers 어색, point cloud works 어색
% 
% still transformers for point clouds.
% 'scalable and long-range dependency aware'

% Although recent point cloud works try to emulate vision transformers in 2D vision, they still have been less explored.
% , it has become a \textit{de facto} architecture in natural language processing. 
% Henceforth, transformer-based models have been actively proposed to computer vision~\cite{dosovitskiy2020image,ramachandran2019stand,carion2020end,parmar2018image,liu2021swin,chu2021Twins}.

% \begin{itemize}
%     % \item We propose SPoTr, transformer-based architectures for point cloud processing. 
%     % SPoTr captures both global long-range distance information and local short-range distance information using self-positioning receptive fields. 
%     % \item We propose self-positioning receptive fields for attention operation, which reduces computational complexity of standard attention operations while  
%     % \item We propose Self-Positioning points based Attention (SPA) blocks which consist of local self-attention and global cross-attention. 
%     % SPA eliminates the quadratic complexity issue of attention in the transformer with a small set of self-positioning points.


%     \item We design a novel Transformer architecture~(SPoTr) for point clouds based on the SPoTr block, which captures both local and global information only using the attention modules and feed-forward networks.
    
%     \item We propose a global cross-attention mechanism with flexible self-positioning receptive fields.
%     The self-positioning receptive fields take spatial/semantic information into account being interpretable.
%     \SH{
%     \item  We propose a new global cross-attention mechanism with flexible self-positioning receptive fields.
%     The 
%     that consider spatial/semantic proximity to the focal point. 
%     }
    
%     \item Our proposed method shows consistently \SH{good} performance on 3D shape classification and 3D segmentation.
%     In particular, SPoTr achieves state-of-the-art performance on the hardest version of ScanObjectNN dataset. 
%     We also demonstrate the effectiveness of SPA through qualitative analysis providing interpretability. 
% \end{itemize}

% except for the hierarchical architecture from~\cite{qi2017pointnet++} to enlarge the receptive field.
% Especially with noisy real-world data like~\cite{uy2019revisiting}, capturing long-range dependencies are crucial to understand the global shape context.

% \jyp{
% In recent years, 3D point clouds has attracted a lot of attention in various applications such as robotics, autonomous driving, and augmented reality. 
% With the success of deep learning on image recognition, many works explore deep learning on point clouds.
% However, learning representations on point clouds can be challenging compared to learning on 2D images.
% Unlike 2D images, which are represented in regular grid, the 3D point clouds consist of point sets expressed in irregular continuous space. 
% This irregular property makes difficult to apply standard discrete convolution neural networks on point clouds.
% }
% \jyp{
% In recent years, 3D point clouds has attracted a lot of attention in various applications such as robotics, autonomous driving, and augmented reality. 
% With the success of convolutional neural networks~(CNNs) on image recognition~\cite{simonyan2014very,he2016deep}, a variety of approaches have been proposed to directly applying CNNs on point clouds by modifying point clouds into regular voxels~\cite{wu20153d,maturana2015voxnet,zhou2018voxelnet} or projected images~\cite{su2015multi,chen2017multi}.
% % However, learning representations on point clouds can be challenging compared to learning on 2D images.
% % Unlike 2D images, which are represented in regular grid, the 3D point clouds consist of point sets expressed in irregular continuous space. 
% However, voxelization and projection not only require heavy computational and memory costs but also induce the loss of geometric information.
% % This irregular property makes difficult to apply standard discrete convolution neural networks on point clouds.
% }

% \jyp{
% A variety of deep networks have been proposed to address this challenge.
% Some prior works modify point clouds into regular voxels~\cite{wu20153d,maturana2015voxnet,zhou2018voxelnet} or projected images~\cite{su2015multi,chen2017multi} for applying standard convolution operation.
% However, voxelization and projection not only require heavy computational and memory costs but also induce the loss of geometric information.
% To address them, many networks that operate directly on point clouds have been presented. 
% Most of them are based on local convolutional neighborhoods without considering long-range dependencies.
% % Most of them continuous convolution~\cite{liu2019relation,thomas2019kpconv}, mlp-based~\cite{qi2017pointnet,qi2017pointnet++} and graph-based methods.
% % sentence 반전
% }
% \jyp{
% To address these challenges, most recent methods~\cite{qi2017pointnet,qi2017pointnet++,liu2019relation,wang2019dynamic} have been proposed to learn point representation by directly processing point cloud data.
% Most of them utilize the spatial proximity between points to capture short-range dependencies without considering long-range dependencies.
% % More recently, some works have designed convolution-like operations on point clouds considering spatial proximity between points.
% % Beyond MLP-based works, 
% % Though impressive, 
% % Most of them continuous convolution~\cite{liu2019relation,thomas2019kpconv}, mlp-based~\cite{qi2017pointnet,qi2017pointnet++} and graph-based methods.
% % sentence 반전
% }

% % Recently, Transformer, which applies an attention to capture long-range dependencies, have proven their effectiveness in natural language processing. 
% % % Inspired by its success, many researchers extend it to computer vision and demonstrate remarkable performance.
% % Inspired by its success, many researchers have tried to extend it to computer vision and achieves remarkable performance~\cite{dosovitskiy2020image,ramachandran2019stand,carion2020end,parmar2018image,liu2021swin,chu2021Twins}.
% % Different from the natural language processing, it is challenging to directly apply self-attention on images since the self-attention operation requires quadratic computational costs in the number of pixels.
% One of the important works, ViT~\cite{dosovitskiy2020image}, successfully applied a Transformer-based backbone for image recognition. 
% % by merging pixels into fixed-size patches and applying the Transformer encoder on them.
% Even if it achieved an impressive speed-accuracy tradeoff compared to convolutional networks, it requires quadratic computational costs with respect to the image size.
% To avoid the heavy computational costs of the self-attention, several works have been proposed by applying self-attention operation only in local neighborhoods~\cite{liu2021swin,wang2021pyramid} or approximating it with reduced set of queries or keys~\cite{chu2021Twins,jaegle2021perceiver,zhu2020deformable}.
% % Twins~\cite{chu2021Twins}
% % Swins~\cite{liu2021swin}
% For processing point clouds, Point-ASNL~\cite{yan2020pointasnl} applies self-attention to capture long-range dependencies.
% However, self-attention have quadratic computation complexity to the number of input points.                                                                    
% % \jyp{
% % In this paper, we design architectures for point clouds processing motivated by the success of Transformers in various tasks such as natural language processing~\cite{vaswani2017attention,devlin2018bert} and 2D image recognition~\cite{dosovitskiy2020image,liu2021swin}.
% % }
% \jyp{
% In this paper, we propose a \textbf{S}elf-\textbf{Po}sitioning receptive fields based \textbf{Tr}ansformer~(SPoTr), which can capture both local and global information with linear complexity to the number of points.
% % It learns representation of points considering long-range information as well as short-range information using attention operations in the Transformer.
% % We design a Transformer-based architecture to process point clouds motivated by the success of Transformers on image recognition~\cite{dosovitskiy2020image,liu2021swin,carion2020end} and natural language processing~\cite{vaswani2017attention,devlin2018bert} with their abilities to capture long-range dependencies.
% SPoTr constructs powerful point cloud representations with local points self-attention~(LPA) and self-positioning receptive fields based global cross-attention~(SPA), in which LPA captures short-range information and SPA focuses on long-range information.
% }

% \jyp{
% % A key design component of SPoTr is self-positioning receptive fields, which are dynamically positioned according to the shape of sample, as illustrated in \Cref{fig:sprf}.
% % The goal of the self-positioning receptive fields is to propagate information of their peripherals to all points in the cross-attention manner.
% A key element of SPoTr is self-positioning receptive field, which aims to propagate its peripheral information to all points in the cross-attention manner.
% As illustrated in \Cref{fig:sprf}, the model simultaneously learns the position and shape of the self-positioning receptive fields by considering the overall point set and spatial/semantic information with bilateral filters.  
% % 예시든 더 구체적인거 추가
% % For example, the self-positioning receptive field, which is denoted by
% % The figure shows that a small set of self-positioning receptive fields, which are denoted by red star, are adaptively located according to the sample and constructed 
% % Before propagating, they aggregate information of their local parts with bilateral filters.
% }
% We validate our model in three tasks such as shape classification, object part segmentation and scene segmentation with four datasets.
% Our proposed methods consistently shows their effectiveness on all datasets compared to other attention-based methods.
% In particular, our architecture achieves an accuracy gain of 2.5\% over the current state-of-the-art model on shape classification with realworld dataset ScanObjectNN.
% Also, we demonstrate the effectiveness of self-positioning receptive fields with qualitative analyses.  
\section{Related works}
    \label{sec:2}
% Some recent methods need to be added.
% This includes deep learning methods for point cloud analysis and Transformer  
\paragraph{Deep learning on point clouds.}
The success of CNNs has encouraged adapting CNNs to operate on point clouds rather than using hand-designed features.
Early approaches aim to transform the unstructured point cloud data into a structured form for directly applying convolution.
These include~\cite{su2015multi,guo2016multi,qi2016volumetric,jaritz2019multi,goyal2021revisiting,hamdi2021mvtn}, where they project 3D point clouds to 2D multi-view images for applying 2D convolution.
Other approaches~\cite{liu2019point,maturana2015voxnet,zhou2018voxelnet} convert point clouds to voxel grids, then apply 3D convolution.
However, both approaches have difficulty in preserving intrinsic geometries of point clouds.
To address this issue,  PointNet \cite{qi2017pointnet} directly processes point clouds with multi-layer perceptrons and a max-pooling function. 
However, it blindly aggregates all points without considering local information.
Thus, PointNet++~\cite{qi2017pointnet++} proposes utilizing local information through set abstraction and local grouping.
% Then PointNet++~\cite{qi2017pointnet++} proposed the hierarchical architecture to gradually enlarge the receptive field with set abstraction and local grouping.
For further understanding of local contexts, recent works~\cite{atzmon2018point,li2018pointcnn,wu2019pointconv,thomas2019kpconv,wang2018deep,xu2021paconv,xu2018spidercnn,zhou2021adaptive,ma2022rethinking, ran2022surface} have proposed explicit convolution kernels on the point space.
KPConv~\cite{thomas2019kpconv} has applied deformable convolution~\cite{dai2017deformable,zhu2019deformable} to capture local information of point clouds.
PointNeXt~\cite{qian2022pointnext} has revisited PointNet++ by fully exploring its potential with improved training and augmentation schemes.
Although the representation power has been improved by capturing local information, the ability to capture long-range dependencies is limited.
% Although convolution-based approaches improve the representation power by capturing local information, the ability to capture long-range dependencies is limited.
SPoTr is the Transformer for point clouds equipped with global cross-attention to capture long-range dependencies.   

\paragraph{Attention-based methods on 2D images.}
Following the success of self-attention and Transformers~\cite{vaswani2017attention} in natural language understanding, many efforts have been made in the computer vision to replace convolution layers with self-attention layers~\cite{dosovitskiy2020image,ramachandran2019stand,hu2019local,parmar2018image,chen2020generative,cordonnier2019relationship}.
Despite the success, self-attention requires the quadratic computational cost with respect to the input image size.
To address the scalability issue, several works adopt self-attention within local neighborhoods~\cite{liu2021swin,wang2021pyramid}.
Swin Transformer~\cite{liu2021swin} utilizes non-overlapping windows and performs self-attention within each local window to get linear computational complexity in the number of input pixels.
Other works explore global attention mechanisms with a small set of queries or keys to reduce complexity~\cite{chu2021Twins,jaegle2021perceiver,zhu2020deformable}.
Twins~\cite{chu2021Twins} applies attention with a small set of representatives. %for efficiency.
Inspired by recent works, we suggest an efficient global cross-attention with only a small set of self-positioning points for point clouds.

\paragraph{Attention-based methods on point clouds.}
Recently, \cite{yan2020pointasnl,xie2018attentional,lee2019set,guo2021pct,zhao2021point,ran2021learning,mazur2021cloud,choe2022pointmixer, xiang2021walk} have adopted attention operations for point cloud processing. 
PointASNL~\cite{yan2020pointasnl} leverages the attention operation to non-locally influence entire points.
RPNet~\cite{ran2021learning} proposes attention-based modules for capturing local semantic and positional relations.
PointTransformer~\cite{zhao2021point} performs self-attention only within local neighborhoods.
CloudTransformer~\cite{mazur2021cloud} inspired by spatial transformer~\cite{jaderberg2015spatial}, uses an attention mechanism to transform the point cloud into a voxel grid for convolutional operation.
Although these works have proven to be effective, most works have neglected the capability of Transformers to capture long-range dependencies due to their quadratic computational cost to the number of input points.
In this paper, we aim to design Transformer architecture to capture both local and global information with a modest computational cost.

% With the advent of convolutional neural networks~(CNNs), a variety of methods have been proposed to apply CNNs on point clouds. 
% %  adopts CNN on point cloud.
% These include view-based methods~\cite{su2015multi, guo2016multi, qi2016volumetric, jaritz2019multi, goyal2021revisiting, hamdi2021mvtn} where they change the 3D point clouds into 2D images by projecting them on 2D spaces.
% Other works~\cite{liu2019point, wu20153d, maturana2015voxnet,zhou2018voxelnet} rasterize the point clouds into voxel images to directly apply 3D convolution.  
% But, both methods have a difficulty preserving the geometric properties of point clouds.
% To address this issue, point-based methods have been presented.



% To address this issue, \cite{qi2017pointnet} proposes a PointNet that directly processes point clouds with multi-layer perceptrons and max-pooling function. 
% However, it blindly aggregates all points without considering local information.
% Thus, PointNet++~\cite{qi2017pointnet++} proposed the hierarchical architecture with to utilize local information through set abstraction and grouping.

% extracts point-wise features with shared multi-layer perceptrons and aggregates them using symmetric functions such as max-pooling.
% Several methods change point clouds into two-dimensional grid sapce  
% These methods inevitably depends on projecting operation onto two-dimensional space (\eg bird's-eye view or multi-view projection) to apply CNN, which brings them intrinsic problems like occlusion or loss of density. 
% In another way, \cite{liu2019point, wu20153d, maturana2015voxnet,zhou2018voxelnet} rasterize the point cloud to voxel images to apply 3D CNN.
% Yet, they still suffer from the trade-offs between the characteristic of the shape and the cubic computational cost due to the resolution. 
% To handle this issue, OCNN~\cite{wang2017cnn} and OctNet~\cite{riegler2017octnet} adopt octree structure, and MinkowskiNet~\cite{choy20194d} neglects the empty grid with sparse convolution.
% \SH{
% As a pioneering work to understand unordered point set itself, \cite{qi2017pointnet} proposes a PointNet that extracts point-wise features with shared multi-layer perceptrons and aggregates them using symmetric functions.
% However, it failed to capture local structure, and PointNet++ \cite{qi2017pointnet++} extends \cite{qi2017pointnet} to hierarchical structure using set abstraction. 
% Following \cite{qi2017pointnet++}, diverse convolutional methods for extracting representations from locally-grouped points have been introduced \cite{liu2019relation,thomas2019kpconv, li2018pointcnn, atzmon2018point, xu2018spidercnn, wu2019pointconv, xu2021paconv}. 
% In parallel with these, other studies~\cite{yan2020pointasnl, wang2019dynamic} also focus on the importance of long-range dependency.
% DGCNN~\cite{wang2019dynamic} proposes EdgeConv that dynamically updates graph on feature space to group semantically similar points. 
% A-SCN~\cite{xie2018attentional} uses self-attention on entire points, but it does not get good performance due to the lack of local information.
% On the contrary, PointASNL~\cite{yan2020pointasnl} leverages the global features by non-local module~\cite{wang2018non} after local operation.
% Following these attempts, recent studies explore self-attention and Transformer to guarantee long-range dependencies for point clouds.}
% More recently, CurveNet~\cite{xiang2021walk} tackles long-range relations with the guided walks on the point cloud generating curve rather using k-nearest neighbors. 

% employ self-attention to replace convolution layers and achieves successful speed-accuracy trade-off compared to convolutional networks. 
% One of the important works, ViT~\cite{dosovitskiy2020image}, directly applies a Transformer encoder for image recognition with a successful speed-accuracy trade-off compared to convolutional networks.
% Most of them /

% or approximating it with a reduced set of queries or keys~\cite{chu2021Twins,jaegle2021perceiver,zhu2020deformable}.


% self-attention with local neighborhoods and global cross-attention to perform attention operations with reduced complexity. 
% To directly perform attention on points, w
% With \cite{dosovitskiy2020image}, Transformer based approaches~\cite{liu2021swin,chu2021Twins,wang2021pyramid} bring the noticeable expansions of self-attention and Transformer in computer vision.

% A-SCN~\cite{xie2018attentional} performs self-attention on entire points, but it does not get good performance due to the lack of local information.

% PointTransformer~\cite{zhao2021point} introduces the Transformer architecture encoding the points with nearest neighbors. 
% Another work RPNet~\cite{ran2021learning} explored a aggregator to understand both positional relations and semantic relations in local point sets by integrating RS-CNN~\cite{liu2019relation} and self-attention. 

% Point2Sequence~\cite{liu2019point2sequence}, considering point set as a sequence, consists of encoder with RNN and attention-based decoder.

% \SH{\paragraph{Transformers on Point Cloud} Following the early success of self-attention~\cite{vaswani2017attention} in NLP, ViT~\cite{dosovitskiy2020image} considers images as a set of patches and
% successfully propose the Transformer in computer vision. With \cite{dosovitskiy2020image}, Transformer based approaches~\cite{liu2021swin,chu2021Twins,wang2021pyramid} bring the noticeable expansions of self-attention and Transformer in computer vision.
% concurrently, \cite{yan2020pointasnl, xie2018attentional, lee2019set, guo2021pct, zhao2021point, ran2021learning, mazur2021cloud} propose the network to encodes the representations of the point cloud using self-attention. Recent study PointTransformer~\cite{zhao2021point} introduces the Transformer architecture encoding the points with nearest neighbors. Another work RPNet~\cite{ran2021learning} explore a aggregator to understand both positional relations and semantic relations in local point sets by integrating RS-CNN~\cite{liu2019relation} and self-attention. Although these works have shown compelling competitiveness compared with previous studies, long-range dependency, key intuition of Transformer, is less explored in point cloud domain due to the high computational cost of global self-attention.
% }
% \SH{
% To this end, we present SPoTr that sequentially reflects the local and global information inspired by Twins~\cite{chu2021Twins} that introduces more inductive bias using locally-grouped self-attention and globally sampled representative.
% SPoTr dynamically extracts global representative in sample-aware way with self-positioned receptive fields, and it makes our model more applicable for irregular domains. 
% Furthermore, we achieve this with the computational cost linear to the number of points alleviating the original quadratic cost of global self-attention.
% }
\section{Method}
    \label{sec:3}
\begin{figure*}[t] 
\centering
% \includegraphics[width=1.0\columnwidth, trim = 0 25 0 0]{Figures/new_fig2.pdf}
\includegraphics[width=1.0\textwidth]{Figures/selfpositioning_compressed.pdf}
\caption{{{\textbf{Illustration of self-positioning point-based attention (SPA).} Given input points $\Cen$ and their corresponding features $\Feat$, self-positioning points~(SP points) $\Vpoint$ are adaptively placed through the learnable latent $\Zfeat$. SP points aggregate features considering both spatial and semantic proximity and constructs $\Vfeat$ via disentangled attention. Then, SPA performs channel-wise point attention (CWPA) between input points and SP points to generate the output features $\Feat^{\text{SPA}}$.
}}}
\label{fig:fig2}
% \vspace{-10pt}
\end{figure*}

The goal of our framework SPoTr is to learn point representations for various point cloud processing tasks with a Transformer architecture using \textit{self-positioning points}.
First, we shortly describe the background regarding the point-based approaches including PointNet++ and Point Transformer~(\Cref{sec:3.1}). 
Second, we propose self-positioning point-based attention to efficiently capture the global context~(\Cref{sec:3.2}).
% \jyp{Second, we demonstrate a SPoTr block, which is composed of two types of attention modules: (i) \emph{local points attention}~(LPA) and (ii) \emph{self-positioning receptive fields based attention}~(SPA), where LPA captures local information and GSA deals with long-range and global information~(\Cref{sec:3.2}).} 
Third, we delineate the SPoTr block, which compromises both global cross-attention and local self-attention, and discuss the relation with a popular point-based network~(\Cref{sec:3.3}). 
Finally, we present the overall architecture of SPoTr, which is composed of multiple SPoTr blocks, for shape classification and segmentation tasks~(\Cref{sec:3.4}).

% To capture global long-range information beyond local short-range information, we present self-positioning points based attention (SPA) block, which consists of local self-attention and global cross-attention with dynamically positioning points.

% Then, we introduce our deformable \textbf{po}int \textbf{tr}ansformer~(Deformable PoTR). 
% Our proposed Deformable PoTR is composed of two attention techniques. In \textbf{LGP-A} (\Cref{sec:3.2}), self-attention is computed on spatially grouped patches. 
% After that, cross-attention over deformable patches is used considering long-range dependencies in \textbf{DGP-A} (\Cref{sec:3.3}). 
% With \textbf{LGP-A} and \textbf{DGP-A}, we present entire architecture of Deformable PoTR in \Cref{sec:3.4}. \SH{Finally, we will discuss about the relationship and competitiveness to prior works in \Cref{sec:3.5}}.

    \subsection{Backgrounds}
    \label{sec:3.1}
% PointNet++
% PointTransformer <- point transformer
% We briefly revisit the formulation of multi-head attention~(MHA) in Transformer.
% Let $\mathbf{z}_q \in \mathbb{R}^{c}$ be a query vector and $\mathcal{F} = \left\{\mathbf{f}_k \right\}_k$ be a set of key vectors, where $q \in \mathrm{\Omega}_q$ is an index of the query, $k \in \mathrm{\Omega}_k$ is an index of the key, and $\mathrm{\Omega}_q, \mathrm{\Omega}_k$ are the set of query and key indices, respectively. 
% Then, the multi-head attention~(MHA) operation can be formulated as follows~\cite{vaswani2017attention,zhu2020deformable}:

% \begin{equation}
% \label{eq:mha}
%     \text{MHA}\left(\mathbf{z}_q, \mathcal{F}\right) = \sum_{m=1}^M \mathbf{W}_m \left[\sum_{k \in \mathrm{\Omega}_k} \textbf{A}_{mqk}\cdot \mathbf{W}_m^\prime \mathbf{f}_k \right],
% \end{equation}
% where each $\mathbf{W}_m, \mathbf{W}_m^\prime \in \mathbb{R}^{c_v \times c}$ are learnable weight matrices, $\mathbf{A}_{mqk}$ is an attention weight between $q$-th query and $k$-th key and $m$ denotes the index of the $M$ attention heads.

% {In this subsection, we briefly revisit the attention in Transformer and the point-based approaches such as  PointNet++~\cite{qi2017pointnet++} and Point Transformer~\cite{zhao2021point}.
% }
In this subsection, we briefly revisit the point-based approaches such as  PointNet++~\cite{qi2017pointnet++} and Point Transformer~\cite{zhao2021point}.


% Assume that a point set $\mathcal{P} = \left\{x_i, \mathbf{f}_i \right\}_{i=1}^N$, where $x_i$ is the position of $i$-th point and $\mathbf{f}_i$ is corresponding feature.


% \paragraph{Standard attention~\cite{vaswani2017attention}}proposed in Transformer has shown success in various domains.
% Let $\mathbf{z}_q \in \mathbb{R}^{c}$ be a query vector and $\mathcal{F} = \left\{\mathbf{f}_k \right\}_k$ be a set of key vectors, where $q \in \mathrm{\Omega}_\text{query}$ is an index of the query, $k \in \mathrm{\Omega}_\text{key}$ is an index of the key, and $\mathrm{\Omega}_\text{query}, \mathrm{\Omega}_\text{key}$ are the set of query and key indices, respectively. 
% Then, the attention operation can be formulated as follows~\cite{vaswani2017attention,zhu2020deformable}:
% \begin{equation}
% \label{eq:mha}
%     \text{Attention}\left(\mathbf{z}_q, \mathcal{F}\right) = \sum_{k \in \mathrm{\Omega}_{\text{key}}} \textbf{A}_{qk}\cdot \left(\mathbf{W} \mathbf{f}_k \right),
% \end{equation}
% {where $\mathbf{W} \in \mathbb{R}^{C \times C}$ is a learnable weight matrix, $\mathbf{A}_{qk}$ is an attention weight between $q$-th query and $k$-th key. 
% The attention weights are computed as $\text{SoftMax}\left({\mathbf{z}_q^\top \mathbf{U}^\top \mathbf{V} \mathbf{f}_k} \right)$, where $\mathbf{U}, \mathbf{V} \in \mathbb{R}^{C \times C}$ are learnable weight matrices.}

% where each $\mathbf{W} \in \mathbb{R}^{c \times c}$ are learnable weight matrices, $\mathbf{A}_{mqk}$ is an attention weight between $q$-th query and $k$-th key. The attention weights are computed as $\exp\left({\mathbf{z}_q^\top \mathbf{U}_m^\top \mathbf{V}_m \mathbf{f}_k} \right)$, where $\mathbf{U}_m, \mathbf{V}_m \in \mathbb{R}^{c \times c}$ are learnable weight matrices. 

\paragraph{PointNet++~\cite{qi2017pointnet++}} captures local shape information through set abstraction and local grouping.
Given that a point set $\mathcal{P} = \left\{x_i \right\}_{i=1}^N$, where $x_i$ is the position of the $i$-th point, and its corresponding feature $\mathbf{f}_i$, PointNet++ proposed local set abstraction as follows:
\begin{equation}
    \mathbf{f}_i' = \mathcal{A}\left(\left\{\mathcal{M}\left([\mathbf{f}_j;\phi_{ij}] \right),\  \forall j \in \mathcal{G}_i \right\}\right),
\end{equation}
where $\mathcal{M}$ is the mapping function~(\eg, MLP), $\mathcal{A}$ is the aggregation function such as max-pooling, and $\mathcal{G}_i$ is the index set of the local group centered on the $i$-th point.


\paragraph{Point Transformer~\cite{zhao2021point}} leverages self-attention operations~\cite{zhao2020exploring} to represent local point groups.
% Different from standard self-attention, Point Transformer applies vector subtraction attention to calculating attention map.
Similar to PointNet++, Point Transformer leverages local grouping to represent local point groups with a self-attention mechanism as follows:
% \jyp{Several works~\cite{zhao2021point,ran2021learning} tried to harness the power of the attention on the point cloud domain.
% For instance, Point Transformer~\cite{zhao2021point} leverages vector self-attention operations~\cite{zhao2020exploring} to represent local point groups.
% It is based on vector self-attention~\cite{zhao2020exploring}.
% Given a point set $\left\{x_i\right\}$ and its corresponding feature $\left\{\mathbf{f}_i\right\}$, the self-attention in~\cite{zhao2021point} operates on \textit{local} point group $\mathcal{G}_i$} as below:
% \begin{equation}
%     % \mathbf{y}_q = \sum_{k \in \mathrm{\Omega}_k} \mathbf{A}_{qk} \odot \alpha\left(\mathbf{f}_k\right), \quad \text{where } 
%     \mathbf{A}_{mqk} = \rho\left(\eta\left(\varphi\left(\mathbf{z}_q\right) - \zeta\left(\mathbf{f}_k\right) + \phi_{qk} \right)\right),    
% \end{equation}
%%%%%%%%%%%%%%
\begin{equation}
    % \mathbf{y}_q = \sum_{k \in \mathrm{\Omega}_k} \mathbf{A}_{qk} \odot \alpha\left(\mathbf{f}_k\right), \quad \text{where } 
    \begin{split}
    &\mathbf{f}_i^\prime = \sum_{j \in \Gc_i} \mathbf{A}_{ij} \odot\left(\mathbf{W}_1\mathbf{f}_j + \Delta_{ij} \right), \\
    &
    \mathbf{A}_{ij} = \text{SoftMax}\left(\mathcal{M}\left(\mathbf{W}_2\mathbf{f}_i - \mathbf{W}_3\mathbf{f}_j + \Delta_{ij} \right)\right),  
    \end{split}
\end{equation}
where $\odot$ denotes element-wise multiplication,  $\mathbf{W}_1$, $\mathbf{W}_2$, $\mathbf{W}_3$ are learnable weight matrices, $\mathcal{M}$ is a mapping function such as multi-layer perceptron, and $\Delta$ is a positional encoding.
% where $\varphi, \zeta$ is linear projection functions, $\eta$ is a mapping function such as multi-layer perceptron, $\phi$ is a positional encoding and $\rho$ is a normalization function (\ie, softmax).
Point Transformer~\cite{zhao2021point} has shown the advantage of the attention mechanism on point clouds only with `local attention' since computing the global attention on whole input points is almost infeasible on large-scale data.

% While several methods have shown the advantage of attention mechanism on point clouds, they are limited to `local attention'.
% While it is possible to apply existing attention methods such as scaled dot product attention on whole input points, it is challenging to naively apply it due to heavy computation cost with the complexity of $\mathcal{O}\left(nd^2 + n^2d\right)$ when $n$ is the number of input points.

% Why global attention is important 
% interpretability
% However, most works are limited to `local attention' due to heavy computation for global attention on the whole input points.
% \jyp{However, most works are limited to `local attention' without considering  of point clouds.}

% The input of the multi-head attention consists of a query and a set of keys and values.

% the multi-head attention operation adaptively aggregates a set of values with attention weights, which represent the compatibility between the query and the key.


% The success of self-attention has led researchers to study its adoption on point clouds.
% There has been papers employing self-attention for point cloud analysis.
% However, prior works~\cite{zhao2021point,ran2021learning}, naturally trying to harness the power of self-attention on the point cloud domain, are limited to \textit{local} self-attention due to heavy computation for global attention on the whole input points.

% \jyp{
% % With multi-head attention and self-attention operations, the transformers have shown remarkable success in the 2D computer vision as well as natural language processing by learning long-range dependencies between representations.
% Even if the transformers have shown remarkable success with the attention operations, most existing transformer architectures for point cloud are designed for only capturing local dependencies without considering long-range dependencies between point representations.
% To address this limitation, our proposed transformer architecture is based on SPA block, which consists of local self-attention operation and global cross-attention operation with self-positioning representative points to capture fine-grained short-range information and long-range information.  
% % To address this limitation, our proposed transformer architecture is composed of two attention blocks : \emph{(i) locally-grouped point attention block}~(\Cref{sec:3.2}) and \emph{(ii) deformable global point attention block}~(\Cref{sec:3.3}) to capture fine-grained short-distance information and global long-range information.
% }


    \subsection{Self-positioning point-based attention}
    \input{3_Method/3_2.tex}
    \subsection{Self-positioning point-based Transformer}
    \label{sec:3.3}

\begin{figure*}[ht] 
\centering
\includegraphics[width=1\textwidth]{Figures/SPoTr_arch6_compressed.pdf}
\caption{
\textbf{Overall Architecture of SPoTr.} For classification~(bottom), four SPoTr blocks run consecutively, followed by a max-pooling and a multi-layer perceptron. 
For segmentation~(top), a U-net style architecture is adopted with SPoTr blocks for downsampling and feature propagation for upsampling, followed by a multi-layer perceptron.}
\label{fig:fig4}
% \vspace{-10pt}
\end{figure*}


% \jyp{Next, we propose the SPoTr block that utilizes \emph{local points attention}~(LPA) module and \emph{self-postioning receptive fields based attention}~(SPA) module to capture not only local fine-grained and short-distance information but also long-distance and global information}.
We now propose the SPoTr block that utilizes \emph{self-positioning point-based attention}~(SPA) with \emph{local point attention}~(LPA).
By combining LPA and SPA, it captures not only local and short-distance information but also long-distance and global information.
% Next, we propose the SPoTr block that utilizes SPA with \emph{local points self-attention}~(LPA).
% Each component operates in order as illustrated in \Cref{fig:fig3}.
% First, we capture the local shape context by LPA, then conduct PGE to abstract local point groups into higher-level feature representations. 
% Lastly, we capture the overall shape context with SPA.\\
% The overall architecture of the SPoTr block is illustrated in \Cref{fig:fig3}.
% First, we capture the local shape context by LPA, then conduct PGE to abstract local point groups into higher-level feature representations. 
% Lastly, we capture the overall shape context with SPA.
% \jyp{LPA captures the locally grouped information with attention mechanism and SPA deals with long-range and global information using self-positioning receptive fields.}\\

\paragraph{Local points attention (LPA).} 
We adopt local points attention~(LPA) defined on a local group to learn local shape context.
% A local point group $\Pointgroup_i$ consists of $k$ nearest neighbors from an anchor point $x_i$, which is selected by the farthest point sampling~(FPS). 
A local point group consists of neighbors in ball query centered on an anchor point $x_i$.
The attention for each local point group $\Pointgroup_i$ and points $\left\{x_j | \forall j \in \Pointgroup_i\right\}$ is defined as
% \yunyang{
\begin{equation}
\label{eq:lpa}
    \Feat_{i}^{\text{LPA}} = \mbox{CWPA}\left(x_{i}, \Feat_i, \left\{x_j\right\}_{j \in \Pointgroup_i}, \left\{\Feat_j\right\}_{j \in \Pointgroup_i}\right),
\end{equation}
% \begin{equation}
% \label{eq:lpa}
%     \Feat_{i}^{\text{LPA}} = \sum_{m=1}^M \mathbf{W}_m \left[\sum_{x_{j} \in \mathcal{G}_i} \textbf{A}_{mij}\cdot \left(\mathbf{W}_m^\prime \mathbf{f}_j + \phi_{ij}\right) \right],
% \end{equation}
% where $\Feat_{j}$ is an input feature vector, $\Feat^\text{LPA}_{i}$ is an output feature vector of LPA, $\Cen_{ij}$ is a 3D coordinate, and $\mathcal{T}$ is a function for injecting the positional information.
where $\Feat_{j}$ is the feature vector of point $x_j$, $\Feat^\text{LPA}_{i}$ is an output feature vector of LPA.
We adopt channel-wise point attention operation~(CWPA) same as SPA.
% The positional encoding can be categorized into two groups: absolute positional encoding and relative positional encoding.
% Herein, we use the \textit{relative} positional encoding for better capturing relationship within local groups as in~\cite{qi2017pointnet++,zhao2021point}.
% Our positional encoding is defined as 
% Following \cite{zhao2021point}, the attention weight $\mathbf{A}_{mij}$ is defined as $\text{softmax}_j\left(\left\{\text{MLP}\left(\varphi(\mathbf{f}_i) - \alpha(\mathbf{f}_{j^\prime}) +\PE_{ij^\prime} \right)\right\}_{j^\prime}\right)$, where $\varphi, \alpha$ are linear transformations and $\delta$ is a positional encoding.
% Note that $i$ is an anchor point index and $j$ is a local point index.
% Positional Encoding
% We use a multi-layer perceptron~(MLP) with one hidden layer.
% We apply LayerNorm~(LN) and residual connections.

% \paragraph{Positional encoding.}
% Generally speaking, Transformers utilize positional encoding such as sinusoidal positional encoding~\cite{vaswani2017attention,dosovitskiy2020image} to capture positional information between data points.
% The positional encoding can be categorized into two groups: absolute positional encoding and relative positional encoding.
% We here use the \textit{relative} positional encoding for capturing better relationships within local groups as in~\cite{qi2017pointnet++,zhao2021point}.
% The relative positional encoding $\PE_{ij} $ of the point $\Cen_{j}$ is defined as
% % \yunyang{
% \begin{equation}
% \label{eq:posemb}
%     \PE_{ij} = \text{MLP}\left(\left[\left(\Cen_{j}-\Cen_{i}\right)^\top||d_{ij}  \right]\right)\text{, where }d_{ij}= \left\lVert \Cen_{j} - \Cen_{i} \right\rVert_2,
% \end{equation}
% where $||$ indicates a concatenation operation.
% \jyp{
% Following \cite{zhao2021point}, we use positional encoding for computing attention weights and feature transformations. 
% So the positional encoding is added to key vectors of attention operations.
% }
% % , $\Cen_{i}$ is an anchor point of the local point group $\Pointgroup_i$,
% %$d_{ij}$ is the Euclidean distance between $\Cen_{ij}$ and $\Cen_{i}$, which is calculated as $\left\lVert \Cen_{ij} - \Cen_{i} \right\rVert_2$. 
% To inject positional information as well as semantic features, we aggregate the positional encoding $\PE_{ij}$ and the feature ${\Feat}_{ij}$ by the function $\mathcal{T}$.
% There are several aggregation methods such as element-wise addition, multiplication, and concatenation.
% Here we apply concatenation followed by a multi-layer perceptron (MLP), \ie, $\mathcal{T}\left(\Cen_{ij}, \Feat_{ij} \right) = \text{MLP}\left(\left[\PE_{ij}||\Feat_{ij}  \right]\right)$.\\

% \noindent\textbf{Point group embedding~(PGE).}
% For downsampling, we apply point group embedding~(PGE), which is the set abstraction~\cite{qi2017pointnet++} with a skip-connection. 
% PGE abstracts the points $\Cen_{ij}$ in the local point group~$\Pointgroup_i$ with the representation vector ${\Feat}_{ij}$.
% It plays a similar role as the patch merging in hierarchical Vision Transformers~\cite{liu2021swin,wang2021pyramid}.
% In our works, the point group embedding is implemented as 
% \begin{equation}
% \label{eq:ffn_pool}
%     \Feat_i^{\text{PGE}} = \mathcal{A}\left(\left\{\FFN\left(\LN\left({\Feat}_{ij} \right) \right) + \Feat_{ij}\right\}_{j \in \Gc_i}\right),
% \end{equation}

% where $\mathcal{A}$ is a symmetric function (\eg, max-pooling) and $\Feat_{ij}$ is the output of LPA defined in \eqref{eq:lpa}.
% $\Feat_i^{\text{PGE}}$ is the representation for encoding shape information of group $\Pointgroup_i$.\\

% \begin{figure*}[t]
    \centering
    \includegraphics[width=\linewidth]{images/architecture-v9.pdf}
    \caption{Our proposed method, PASS, consists of three classifiers trained in a round-robin fashion, with two classifiers (\(h_{\gamma_{2},\gamma_{3}}\) in green) being used to select samples for training the other classifier (\(h_{\gamma_{1}}\) in red).
    The training process begins with a warm-up of all classifiers, followed by the sample selection stage. During the selection stage, the peer classifiers calculate the prediction agreement using cosine similarity between their posterior distributions, followed by Otsu's thresholding~\cite{otsu1979threshold} to automatically find the threshold $t$ to select the clean set ${D}_{\text{clean}}$ and noisy set ${D}_{\text{noisy}}$. In the training stage, we follow the robust noisy-label training algorithm.}
    \label{fig:architecture}
\end{figure*}

% \noindent\textbf{SPA: Global Cross-Attention for Point Cloud.} %Self-positioning Receptive Fields based Attention

% \begin{figure}[!t]
\centering
\setlength{\abovecaptionskip}{0pt}
\setlength{\belowcaptionskip}{-2pt}
\subfigure[] {\ \ \includegraphics [width=0.47\linewidth]{fig/stats_2_4DCT.pdf}} \ 
\subfigure[] {\includegraphics [width=0.47\linewidth]{fig/stats_1_COPD.pdf}}
\caption{Statistical Analysis of our technique and existing methods. We performed the Friedman test for multiple comparisons along the Wilcoxon test for pair-wise comparison for (a) 4DCT and (b) COPD datasets. }
\label{FigTest}
\vspace{0pt}
\end{figure}



\paragraph{SPoTr block.}
We construct a SPoTr block by combining the local points attention~(LPA) module and self-positioning point-based attention (SPA) module to capture local and global information simultaneously.
As shown in \Cref{fig:fig4}~(bottom right), the SPoTr block is defined as follows:
\begin{equation}
    \hat{\Feat}_i = \alpha \cdot \Feat_i^{\text{SPA}} + (1-\alpha) \cdot \Feat_i^{\text{LPA}} 
\end{equation}
where $\alpha$ is a learnable parameter that softly selects the representations generated by self-positioning point-based attention and local points attention.
% As shown in~\Cref{fig:fig3a}, the SPoTr block first linearly transforms input features followed by batch normalization.
% Then, the generated representations are processed by both LPA and SPA with batch normalization and residual connections.
% Next, we perform farthest point sampling~(fps), kNN, and local pooling~(\ie, max-pooling) to reduce the cardinality of the point set and aggregate local points similar to Pointnet++~\cite{qi2017pointnet++}. 
Finally, LPA and MLP with batch normalization and the residual connection are applied to extract point-wise high-level representations.

\paragraph{Connection between SPoTr block and Set abstraction.}
% We find a interesting connection between point channel-wise attention and set abstraction.
% By discussing the point channel-wise attention operation with set abstraction, we demonstrate why the point channel-wise attention is more powerful than the standard attention.
We show the superiority of the SPoTr block by discussing it with set abstraction in PointNet++~\cite{qi2017pointnet++}. 
% Unlike standard attention, since CWPA in SPoTr block computes the attention value for each channel, our SPoTr block can express the set abstraction in PointNet++~\cite{qi2017pointnet++}: 
\begin{remark}
A SPoTr block with proper $\alpha, \mathcal{M},\mathcal{M}',\mathcal{R},\mathcal{R}'$ can express set abstraction.
\end{remark}
When the value of $\tau$ is sufficiently low, the function $\mathcal{M}^\prime$ is the same as $\mathcal{M}$, $\alpha=0$, and $\mathcal{R}(\mathbf{f}_q, \mathbf{f}_k)=\mathcal{R}^\prime(\mathbf{f}_q, \mathbf{f}_k)=\mathbf{f}_k$, the channel-wise point attention becomes equivalent to the set abstraction.   
In this setting, the attention weight between the query point and $k$-th key point on $c$-th channel becomes 1 if $k = \underset{k^\prime \in \Omega_k}{\mathrm{argmax}}\  \mathcal{M}\left([\mathbf{f}_{k^\prime};\phi_{qk^\prime}] \right)_{q,k^\prime,c}$.
Otherwise, the attention score is 0.
It means that the attention only activates the maximum channels alike a max-pooling operation. 
Therefore, the SPoTr block can play a role as a max-pooling operation following the mapping function, which is the set abstraction.
This fact supports the improved expressive power of SPoTr on point cloud analysis.
% The detailed proof is in the supplement.

% \paragraph{SPoTr-U block.}
% For semantic segmentation tasks, we present a SPoTr-U block as illustrated in \Cref{fig:fig3b}.
% To combine features of downsampled points $\Pc_1 \subset \Pc_2$ with its superset $\Pc_2$, which is provided by a skip-connection, trilinear interpolation is applied to upsample points.
% Then, we adopt multi-layer perceptron~(MLP) with batch normalization.
% To the end, similar to the SPoTr block, the combined features are updated by both LPA and SPA with batch normalization and residual connections.
% We construct a SPoTr block by combining the local points attention~(LPA) module and self-positioning receptive fields based attention (SPA) module to capture local and global information simultaneously.
% As shown in Figure~\Cref{fig:fig4}, the SPoTr block first linearly transforms input features followed by batch normalization.
% Then, the generated representations are processed by both LPA and SPA with batch normalization and residual connections.
% Next, we perform farthest point sampling~(fps), kNN graphs, and local pooling~(\ie, max-pooling) to reduce the cardinality of the point set and aggregate local points similar to Pointnet++~\cite{qi2017pointnet++}. 
% Finally, multi-layer perceptron with batch normalization and residual connection is applied to extract deep aggregated point features.
% Latency comparison experiments.
% For instance, in the last block of scene segmentation, we opt $m=128$ compared to the original $n=4096$ points, making the global attention approximately ${\frac{4096}{128}}=\mathbf{32}\times$ faster.

    \subsection{SPoTr architectures}
    \label{sec:3.4}

We design a Transformer-based architecture called SPoTr for point cloud tasks as illustrated in~\Cref{fig:fig4}. 
% Similar to vision Transformer architectures~\cite{dosovitskiy2020image,liu2021swin} in image domains, 
% we do \emph{not} utilize separate convolutional layers.
% the entire architecture is only based on attention blocks and multi-layer perceptions (MLP).

\paragraph{Classification.}
For the shape classification task, we build our Transformer encoder by stacking the SPoTr blocks  described in \Cref{sec:3.3}.
To increase the representational power, we first apply an MLP before operating the attention blocks following \cite{zhao2021point}.
Then, the SPoTr blocks are sequentially applied on sampled points, which are sampled through the farthest point sampling~(FPS).
In shape classification, we set $l=0$ for the SPoTr block since we empirically found that it is enough in shape classification task.
Besides, the sampling rates are 1/4 for every stage.
We use the ball query that selects points within a radius (an upper limit of the number of neighborhoods is set in implementation) to generate a local point group $\Pointgroup_i$ centered at point $x_i$ following \cite{qi2017pointnet++}.
After the last stage, the features are aggregated by a max-pooling function and processed by an MLP.
% with a dropout probability of 0.5.

\paragraph{Segmentation.}
The encoder for semantic segmentation contains the SPoTr blocks and FPS.
Following previous studies~\cite{qi2017pointnet++}, we apply a U-net designed architecture, which contains the feature propagation layers and the SPoTr blocks for dense prediction.
Same to the classification, we use a ball query to generate a local group.
% For the number of neighbors, we use $k=32$ as the number of input points which is usually larger than classification tasks.
% k 추가로 작성 
% Since the local attention block includes the aggregate operation with a symmetric function, we only use the global attention blocks for upsampling.
The outputs of the final block are processed by an MLP.
More details on SPoTr architectures are in the supplement.
% with a dropout probability of 0.5.

\section{Experiments}
    \label{sec:4}
In this section, we demonstrate the effectiveness of SPoTr and provide quantitative and qualitative results for further analyses.
% We evaluate our architectures for object classification (\Cref{sec:4.1}), shape part segmentation and scene segmentation (\Cref{sec:4.2}).
First, we conduct shape classification and semantic segmentation~(\Cref{sec:4.1}).
We also provide ablation studies and complexity analysis of SPoTr~(\Cref{sec:4.2}).
Lastly, we provide visualizations to better understand how SPA behaves~(\Cref{sec:4.3}). Implementation details are available in the supplement.
% We also perform the ablation studies to better understand the contribution of each component~(\Cref{sec:4.3}), followed by qualitatively and quantitatively analyses of SPoTr~(\Cref{sec:4.3}) with the visualization. 

% \input{4_Experiments/4-5}
    \subsection{Shape classification and semantic segmentation}
    \label{sec:4.1}

\begin{table}[t]
  \centering
%   \small
\setlength{\tabcolsep}{8pt}
% \renewcommand{\arraystretch}{0.80}
% \begin{center}
  \begin{tabular}{l|c|c c}
    \toprule
    Methods & Year & mAcc & OA \\
    \midrule
    PointNet~\cite{qi2017pointnet} & 2017 &  63.4 &68.2\\
    PointNet++~\cite{qi2017pointnet++} & 2017 & 75.4 & 77.9 \\
    % 3DmFV~\cite{ben20183dmfv} & 2018 & 58.1 & 63\\
    SpiderCNN~\cite{xu2018spidercnn}& 2018& 69.8 & 73.7\\
    PointCNN~\cite{li2018pointcnn}&2018& 75.1 & 78.5 \\
    DGCNN~\cite{wang2019dynamic}&2019& 73.6 & 78.1\\
    DRNet~\cite{qiu2021dense}&2021& 78.0 & 80.3 \\
    GBNet~\cite{qiu2021geometric}&2021& 77.8 & 80.5 \\
    SimpleView~\cite{goyal2021revisiting}&2021 &-& 80.5\\
    PRA-Net~\cite{cheng2021net}&2021& 77.9 & 81.0 \\
    MVTN~\cite{hamdi2021mvtn}&2021 & - & 82.8 \\
    % PointMLP~\cite{ma2022rethinking} & 2022 & 83.9±0.5 & 85.4±0.3 \\
    CT~\cite{mazur2021cloud} & 2021 & 83.1 & 85.5 \\
    PointMLP~\cite{ma2022rethinking} & 2022 & 84.4 & 85.7 \\
    RepSurf-U~\cite{ran2022surface} & 2022 & 83.1 & 86.0 \\
    \rowcolor{LightYellow}PointNeXt~\cite{qian2022pointnext}& 2022 & 85.8±0.6 & 87.7±0.4\\
    \midrule
    \rowcolor{LightRed}\textbf{SPoTr}& 2023 & \textbf{86.8} &\textbf{88.6} \\
    \bottomrule
  \end{tabular}
  \caption{\textbf{Shape classification results on PB\_T50\_RS in SONN.}
  \label{tab:sonn}
  mAcc is the mean of class accuracy and OA is the overall accuracy.
  }
  % \end{center}
\end{table} 
%86.0



\paragraph{Shape Classification.}
% \subsubsection{Datasets.} 
For the shape classification, we validate SPoTr on a real-world dataset ScanObjectNN (\textbf{SONN})~\cite{uy2019revisiting}. 
% and synthetic dataset ModelNet40 (\textbf{MN40})~\cite{wu20153d}.
SONN has 2,902 objects categorized into 15 classes from SceneNN~\cite{hua2016scenenn} and ScanNet~\cite{dai2017scannet}.
Among diverse variants of SONN, we use PB\_T50\_RS (\textbf{SONN\_PB}), which is the most challenging version with random perturbation and contains 14,510 objects in total.
We follow the official split of \cite{uy2019revisiting}, where they divide SONN into 80\% for training and 20\% for evaluation.
Also, we sample 1,024 points for training and evaluating the models. 
% MN40 is a synthetic dataset with 12,311 meshed CAD models from 40 categories. 
% We use 9,843 models for training and 2,468 models for evaluation.

% Our model achieves state-of-the-art performance SONN\_PB.
\Cref{tab:sonn} shows that SPoTr outperforms all baselines with the mean of class accuracy (mAcc) of 86.8\% and overall accuracy (OA) of 88.6\% (+1.0\% mAcc, +0.9\% OA).
This result shows that capturing long-range context is important for recognizing 3D shapes in real-world datasets.
% In \Cref{tab:mn40}, SPoTr achieves an overall accuracy of 94.1\% on MN40 without any additional information~(\eg, normal vector). %or voting strategy in~\cite{liu2019relation}.
% Compared to baselines with the identical setting as ours, \ie, without a normal vector, SPoTr outperforms all baseline models.
    % \subsection{Part Segmentation}
% \label{sec:4.2}
\begin{table}[t]
  \centering
\setlength{\tabcolsep}{5pt}
  \begin{tabular}{l|c|cc}
    \toprule
    Methods & Year & cls. mIoU & ins. mIoU\\
    \midrule
    PointNet~\cite{qi2017pointnet}&2017 & 80.4& 83.7\\
    PointNet++~\cite{qi2017pointnet++}&2017 & 81.9 & 85.1 \\
    PointCNN~\cite{li2018pointcnn}&2018& 84.6 & 86.1 \\
    DGCNN~\cite{wang2019dynamic}&2019 & 82.3 & 85.1 \\
    RSCNN~\cite{liu2019relation} & 2019 & 84.0 & 86.2 \\
    KPConv~\cite{thomas2019kpconv}&2019 & 85.1&86.4 \\
    PointConv~\cite{wu2019pointconv}&2019 & 82.8 & 85.7 \\
    PointASNL~\cite{yan2020pointasnl}&2020 & - & 86.1 \\
    PCT~\cite{guo2021pct}&2021 & - & 86.4\\
    PAConv~\cite{xu2021paconv}&2021 & 84.6  &  86.1 \\
    AdaptConv~\cite{zhou2021adaptive}&2021 & 83.4 & 86.4\\
    PointTransformer~\cite{zhao2021point}&2021& 83.7 & 86.6 \\
    CurveNet~\cite{xiang2021walk}&2021 & - & 86.8 \\
    PointMLP~\cite{ma2022rethinking} & 2022 &84.6 & 86.1\\
    \rowcolor{LightYellow}PointNeXt~\cite{qian2022pointnext} & 2022& 85.2 $\pm$ 0.1 & 87.0 $\pm$ 0.1 \\
    \midrule
    \rowcolor{LightRed}\textbf{SPoTr} & 2023& \textbf{85.4} & \textbf{87.2}\\


    \bottomrule
  \end{tabular}
  \caption{\textbf{Part segmentation results on SN-Part.} ins. mIoU is the mean of instance IoU. cls. mIoU is the mean of the class IoU. 
  }
  \label{tab:shapenet}
% \vspace{-5pt}
\end{table} 

\begin{table}[t]
  \centering
% \setlength{\tabcolsep}{5pt}
\setlength{\tabcolsep}{3pt}
\resizebox{\columnwidth}{!}{
  \begin{tabular}{l|c|ccc}
    \toprule
    Methods & Year & OA & mAcc & mIoU\\
    \midrule
    PointNet~\cite{qi2017pointnet} & 2017  & - & - & 41.1 \\
    PointCNN~\cite{li2018pointcnn} & 2018 &  85.9 & 63.9 & 57.3\\
    PointWeb~\cite{zhao2019pointweb} & 2019 & 87.0 & 66.6 & 60.3\\
    KPConv~\cite{thomas2019kpconv} & 2019 & - & 72.8 & 67.1 \\
    PCT~\cite{guo2021pct} & 2021 & - & 67.7 & 61.3 \\
    CT~\cite{mazur2021cloud} & 2021 & - & - & 67.9 \\
    PointTransformer~\cite{zhao2021point} & 2021 & \textbf{90.8} & - & 70.4 \\
    RepSurf-U~\cite{ran2022surface} & 2022 & 90.2 & 76.0 & 68.9 \\
    \rowcolor{LightYellow}PointNeXt~\cite{qian2022pointnext} & 2022 & 90.6 $\pm$ 0.1& - & 70.5 $\pm$ 0.3 \\
    \midrule
    \rowcolor{LightRed}\textbf{SPoTr}  & 2023 & 90.7 & \textbf{76.4} & \textbf{70.8}\\ % 32 - [1,1,1,1] [2,2,2,2]
    \bottomrule
  \end{tabular}}
  \caption{\textbf{Semantic segmentation results on S3DIS.} OA is the overall accuracy, mAcc is the mean of class accuracy, and mIoU is the mean of instance IoU.
  \label{tab:s3dis}
  }

\end{table} 


\paragraph{Part Segmentation.}
For part segmentation, we use \textbf{SN-Part}~\cite{snpart,lee2022sagemix}, which is a synthetic dataset with 16,881 shapes from 16 categories with 50 part labels.
We follow the split used in \cite{qi2017pointnet}, where 14,006 samples are for training and 2,874 samples are for validation. 
On each shape, 2,048 points are randomly sampled.

The results are reported in~\Cref{tab:shapenet}, where we evaluate the performance with the mean of instance IoU (ins. mIoU) and class IoU (cls. mIoU). Following previous works~\cite{liu2019relation,xiang2021walk,xu2021paconv}, we report the results with a multi-scale inference setting. 
Although the performance in SN-Part is quite saturated, SPoTr achieves the best performance 87.2\% with considerable improvements (+0.2\% mIoU).
% We also observed that each SP point of SPoTr recognizes the shape part of where it belongs without its explicit part label.
% See \Cref{sec:4.3} for visualizations.

\paragraph{Scene Segmentation.}
For comparison with previous methods~\cite{qi2017pointnet,thomas2019kpconv,zhao2021point,choe2022pointmixer} on scene segmentation, we validate SPoTr on the widely used benchmark dataset \textbf{S3DIS}~\cite{armeni20163d}. S3DIS is the large-scale dataset containing 271 rooms from 6 indoor areas with 13 semantic categories. In our experiments, we largely follow the settings of PointTransformer~\cite{zhao2021point} and consider Area-5 as the test set.

As shown in~\Cref{tab:s3dis}, SPoTr outperforms all previous methods in every metric (\ie, overall accuracy (OA), mean of class accuracy (mAcc), and mean of instance IoU (mIoU)).
The superior performance over previous Transformer architecture~\cite{zhao2021point} (+0.4\% mIoU) proves the importance of long-range dependency in the semantic segmentation as well as the shape classification.
    \subsection{Quantitative analysis}
    \label{sec:4.2} 

\begin{table}[t]
  \centering
%   \small
\setlength{\tabcolsep}{5pt}
% \renewcommand{\arraystretch}{0.80}
  \begin{tabular}{l|ccc|cc}
    \toprule
    Method & $g$ & $h$ & SP&  OA\\
    \midrule
    w/o SPA (\textit{baseline}) &    &  &   &  87.9\\
    w/o self-positioning &  \checkmark  & \checkmark &&87.7\\
    %  \checkmark  & & \checkmark & \checkmark &78.9 &82.5\\
    w/o disentangled attention &\checkmark &  &  \checkmark  &88.2\\
    SPoTr (\textit{ours}) & \checkmark &\checkmark & \checkmark &\textbf{88.6}\\
    \bottomrule
  \end{tabular}
  
    \caption{\textbf{Ablations on SONN\_PB.} $g$: spatial kernel, $h$: semantic kernel, SP: self-positioning points. OA is the overall accuracy.}
\label{table5}
% \vspace{-5pt}
\end{table} 


\begin{table}[t]
  \centering
%   \small
\setlength{\tabcolsep}{5pt}
% \renewcommand{\arraystretch}{0.80}
  \begin{tabular}{l|c|c}
    \toprule
    Attention type& Semantic rel. $\mathcal{R}$ &   OA\\
    \midrule
    Standard Att.  &-- & 86.1\\
    CWPA &$\mathbf{f}_k$    & 88.1\\
    CWPA & $\mathbf{f}_q + \mathbf{f}_k$    & 86.4\\
    CWPA & $\mathbf{f}_q \odot \mathbf{f}_k$&  85.4\\
    CWPA &$\mathbf{f}_q - \mathbf{f}_k$ & \textbf{88.6}\\
    \bottomrule
  \end{tabular}
  
    \caption{\textbf{Performance comparisons of different attention types and semantic relation $\mathcal{R}$ on SONN\_PB.} Attention types : Standard Attention in Transformer~\cite{vaswani2017attention} and channel-wise point attention~(CWPA) with Semantic relation : $\mathcal{R}(\mathbf{f}_q, \mathbf{f}_k)$} %= $ $\mathbf{f}_k$, $\mathbf{f}_q+\mathbf{f}_k$, $\mathbf{f}_q\odot\mathbf{f}_k$, and $\mathbf{f}_q-\mathbf{f}_k$.}
  % \vspace{-5pt}
  \label{table6}
\end{table} 

\begin{figure*}[ht] 
\centering
\includegraphics[trim=20 20 20 20,clip,width=1.0\textwidth]{Figures/SPpoints_compressed.pdf}
\caption{{\textbf{Self-positioning points~(SP points).} \textcolor{cyan}{SP points} are adaptively self-positioned according to each shape. \textcolor[rgb]{1,0,0}{Red points} correspond to specific SP points. Under the same class, the red points are located at \textit{semantically similar} positions.
}}
\label{fig:SPpoints}
% \vspace{0pt}
\end{figure*}
\begin{figure}[t] 
\centering

\includegraphics[trim=50 20 30 10, width=1.0\columnwidth]{Figures/dis_att_compressed.pdf}
% \vspace{-20pt}
\caption{
\textbf{Visual comparison of disentangled attention with a spatial kernel.}
{
% SP point aggregates feature with disentangled attention which filters out irrelevant information.
A spatial kernel (Middle) only considers spatial proximity without considering semantic relevance. Differently, Disentangled Attention~(Right) filters out irrelevant information.}}


% SPA aggregates feature with disentangled attention and then non-locally distributes the information using (c) global cross-attention.
% \textcolor{blue}{Blue} arrows illustrate information aggregation from points to the focal point.
% \textcolor[RGB]{96,96,96}{Gray} arrows stand for suppressed influence by $g \cdot h$.
% \textcolor[RGB]{0,153,0}{Green} arrows represent the focal point influence on semantically related points.}
\label{fig:dis_att}
% \vspace{-15pt}
\end{figure}

% Unlike (a) spatial kernel that only considers spatial proximity, SPA aggregates features with (b) disentangled attention $g \cdot h$.
% SPA aggregation with (a) the spatial kernel $g$ and (b) disentangled attention $g \cdot h$. 
% Then, \textcolor{cyan}{focal points} distribute information non-locally using global cross-attention (c).
% from semantically dissimilar parts are removed
% are from the area, where points have high proximity to focal point
% A self-positioning \textcolor{cyan}{focal point} consider spatial proximity with spatial kernel $g$~(Blue arrows are from the area, where points have high proximity to focal point). (b) With semantic kernel $h$, disentangled attention $g \cdot h$ improves the descriptive power by aggregating semantically similar local points~(Gray arrows from semantically dissimilar parts are removed).  (c) Then, \textcolor{cyan}{focal point} distributes the information non-locally using cross-attention~(Green arrows from focal point represent the area of the influence)


\paragraph{Ablation studies.}
We explore how self-positioning positions (SP) and disentangled attention contribute to SPA.
\Cref{table5} shows the final results on SONN, where the baseline~(\textit{w/o SPA}) learns only with local point attention.
In the case of \textit{w/o self-positioning}, we use FPS to randomly select a small set of points for cross-attention, and for \textit{w/o disentangled attention}, we only adopt the spatial kernel function $g$.
Our model with all the components of SPA achieves the best performance of 88.6\% in overall accuracy.
This superior performance verifies that every component is crucial for SPA.
In particular, when we use FPS instead of SP, the performance is even worse than the baseline as overall accuracy dropped from 87.9\% to 87.7\%.
This observation suggests the positions of SP points \textit{matter} for global cross-attention.
Rather than simple sampling, our learnable approach successfully locates SP points and makes global cross-attention effective.
Next, with \textit{w/o disentangled attention}, the performance gain in OA is minimal (0.3\%) over the baseline compared to using disentangled attention (0.7\%).
It indicates that disentangled attention improves the descriptive power by filtering semantically irrelevant information.

\paragraph{Attention types and semantic relation $\mathcal{R}$.}
In \Cref{table6}, we conduct experiments to compare the models with different attention types (Standard attention in Transformer~\cite{vaswani2017attention} and our CWPA) and semantic relations ($\mathcal{R}(\mathbf{f}_q, \mathbf{f}_k) = $ $\mathbf{f}_k$, $\mathbf{f}_q+\mathbf{f}_k$, $\mathbf{f}_q\odot\mathbf{f}_k$, and $\mathbf{f}_q-\mathbf{f}_k$).
The models adopting the CWPA outperform the model with the standard attention, which shows that the channel-wise point attention operation is more powerful to represent point clouds compared to the standard attention.
Furthermore, the results demonstrate that Sub~($\mathbf{f}_q-\mathbf{f}_k$) is most appropriate to model the semantic relation between points. 

% \SH{We verify that every component is critical for SPA as utilizing both components achieves the best performance of 85.1\% in overall accuracy.}


% SP, disentangled 강조, 키워드 살리기, represntative, long-range 

% This observation suggests positions learned by SP
% Also, We can know that learning positions of focal points is more effective than simply locating them via heuristic without considering semantic information.  
% Also, in the case of using only the spatial kernel, the performance gain is minimal (0.2\%) over the baseline compared to using disentangled attention (0.6\%).

% \SH{
% To better understand the contribution of each component in our model, we conduct ablation studies on PB\_T50\_RS variant of ScanObjectNN. 
% % Specifically, we measure the effect of SPA, self-positioning, and disentangeld attention  of the SPA.
% % The results are provided in~\Cref{table5}.
% \Cref{table5} reports the performance after excluding each component of SPoTr.
% The table shows that all components contribute to improve performance.
% % Compared to SPoTr with the model without SPA, SPA achieves 0.6\% improvement on overall accurac.
% % Model A is set to SPoTr block without SPA while model B is SPoTr block without GPA.
% % Both models show better performance compared to the other baselines in \Cref{table1}.
% % \paragraph{Ablation studies.} 
% % All SPoTr variants without some components show performance drops compared to the full architecture, which means that each component contributes to the performance improvement of SPoTr. 
% % Compared to model C, which uses FPS for determining the position of the focal point instead of self-positioning~(SP), model E gets 1.9\% higher overall accuracy~(OA).
% In particular, compared to SPoTr and  
% This indicates that the focal points are adaptively located with the self-positioning mechanism. 
% % Also, we can observe that disentangled attention kernel contributes the improvement of performance by comparing SPoTr and the model without semantic kernel $h\left(\cdot\right)$. 
% Lastly, we observe that using the disentangled attention kernel~(Model E), instead of spatial kernel~(Model D), improves the overall accuracy with 0.4\%.
% }
%by comparing SPoTr and the model without $h$.  
% For the model without self-positioning~(SP), we use farthest point sampling~(FPS) for selecting global points.
% Finally, the best accuracy 85.1\% is obtained when applying both LPA and SPA. 
% As mentioned in~\Cref{sec:3.2}, we think that the performance improvement is achieved by capturing local structural information and resolving the limitation of the sole local-attention.}
% Otherwise, if model considers only global information without LPA, it achieves 84.0\% which is still surpassing the previous state-of-the-art models. 
% Further, we deeper explore the self-positioning and bilateral filter in SPA.
% \begin{itemize}
%     \item LGPA
%     \item DGPA
% \end{itemize}


% \paragraph{Number of Self-positioning Points.}
%relation between \# of points -> graph로 그림하나 넣기


%keyword : scalability, long-range dependency 
\section{Complexity Analysis}
\label{sec:complexity_analysis}

{\bf Size bounds.} For a join query $Q$, its hypergraph $H(Q)$ has one node per variable in $Q$ and one hyperedge per relation in $Q$.  Figures~\ref{fig:example_intro_varorder} depicts a query hypergraph.

An edge cover is a subset of (hyper)edges of $H(Q)$ such that each node appears in at least one edge. Edge cover can be formulated as an integer programming problem by assigning to each edge $R_i$ a weight $w_{R_i}$ that can be $1$ if $R_i$ is part of the cover and $0$ otherwise. The size of an edge cover upper bounds the size of the query result, since the Cartesian product of the relations in the cover includes the
query result: $|Q(\db)| \leq |R_1|^{w_{R_1}}\cdot\ldots\cdot|R_n|^{w_{R_n}}$, where the database $\db$ is $(R_1,\ldots,R_n)$. By minimizing the size of the edge cover, we can obtain a lower upper bound on the size of the query result. This bound becomes tight for fractional weights~\cite{AGM:2013}.  Minimizing the sum of the weights thus becomes the objective of a linear program.

\begin{definition}[\cite{AGM:2013}]\label{def:agm}
Given a join query $Q$ over a database $(R_1,\ldots,R_n)$, the {\em fractional edge cover number} $\rho^*(Q)$ is the cost of an optimal solution to the linear program with variables $(w_{R_i})_{i\in[n]}$ representing weights of $(R_i)_{i\in[n]}$:
\begin{flalign*}
\textrm{minimize} &\prod_{i\in[n]} |R_i|^{w_{R_i}}\\
\textrm{subject to} &\sum_{R\textrm{ is relation of } X} w_R \geq 1~~\textrm{for each variable } X \\
&~~~\forall i\in[n]: \omega_{R_i}\geq 0.
\end{flalign*}
\end{definition}

\begin{example}
\em
Consider the triangle query:
\begin{align*}
Q_{\vartriangle} = R(A,B), S(B,C), T(C,A)
\end{align*}
Figure~\ref{fig:triangle_hypergraph_viewtree} gives the hypergraph of $Q_{\vartriangle}$. The linear program is: 
\begin{flalign*}
\textrm{\em minimize} & \quad |R|^{w_{R}} \cdot |S|^{w_{S}} \cdot |T|^{w_{T}} \\ 
\textrm{\em subject to} & \quad
\begin{tabular}[t]{@{}c@{\hspace*{.5em}}c@{\hspace*{.25em}}c@{\hspace*{.25em}}c@{\hspace*{.25em}}c@{\hspace*{.25em}}c@{\hspace*{.25em}}c}
$A:$ & $w_{R}$ & & & $+$& ${w_{T}}$& $\geq 1$ \\
$B:$ & $w_{R}$ & $+$ & ${w_{S}}$ & & & $\geq 1$ \\
$C:$ & & & $w_{S}$ & $+$ & ${w_{T}}$ & $\geq 1$
\end{tabular}
\end{flalign*}
% 
For $|R|=|S|=|T|=N$, setting $w_{R}=w_{S}=w_{T}=1/2$ gives the optimal solution $\rho^*(Q_{\vartriangle}) = N^{3/2}$. Consequently, the query result has $\bigO{N^{3/2}}$ tuples. This bound is tight in the sense that there exist classes of databases for which the result size is at least $\Omega(N^{3/2})$. For the acyclic query $Q$ in Section~\ref{sec:introduction}, setting the weights $1$ to each of the three relations gives $\rho^*(Q)=N^3$ if all relations have size $N$.
\punto
\end{example}

\nop{
Cardinality constraints can be used to lower the size bounds of query results. For instance, if the number of distinct $A$-values in $R(A,B)$ is $k \ll N$, then we can refine  $Q_\vartriangle$ as $R(A,B),S(B,C),T(C,A),U(A)$ with the new size bound $\rho^*(Q_\vartriangle) = N \cdot k$, where $w_{S}=1$ and $w_{U}=1$.

Join selectivities can also be incorporated to obtain a size {\em estimate} (in contrast to an upper bound). For instance, assume the selectivity of the join on $A$ between $R$ and $T$ is very low: $sel(A) = \frac{|R(A,B),T(C,A)|}{|R|\cdot|T|} = \frac{k}{N}$. Then, we consider a relation $U(A,B,C)=R(A,B),T(C,A)$ whose size estimate is $k \cdot N$ and use this as a cardinality constraint to obtain an estimate of $k \cdot N$ for $Q_\vartriangle$'s size since the size of the join of $S$ and $U$ cannot exceed the size of $U$.
}

Similarly to $\rho^*(Q)$, the {\em factorization width} $\fw(Q)$ governs the sizes of the factorized results of a join query $Q$~\cite{Olteanu:FactBounds:2015:TODS}. In a factorized join over a variable order $\omega$, the values of a variable $X$  depend on the tuples of values of its $\mathit{key}(X)$ variables and are independent of the values for other variables. A tight bound on this number is then given by the size of a join query that covers the variables in $\mathit{key}(X)\cup\{X\}$. We denote this restriction of $Q$ by $Q_{\mathit{key}(X)\cup\{X\}}$. An upper bound on the size of the factorization is then given by the maximum over all variables in $\omega$ of their number of values. This can be improved by going over all possible variable orders of $Q$ and taking the minimum upper bound. This is the factorization width of the query.

\begin{definition}\label{def:fw}
Given a join query $Q$, the {\em factorization width} of $Q$ is  $\fw(Q) = \min_{\omega\in\Omega(Q)} \max_{v\in\mathit{vars}(Q)} \rho^*(Q_{\mathit{key}(X)\cup\{X\}})$.
\end{definition}

\begin{example}\em
For acyclic queries $Q$ over relations $R_1,\ldots,R_n$, $\fw(Q)=\max_{i\in[n]}(|R_i|)$, while $\rho^*(Q)$ can be as much as $\prod_{i\in[n]}|R_i|$ as in our running example. Here are examples of restrictions of our natural join $Q$ in Section~\ref{sec:intro_example}: $\mathit{key}(D)\cup\{D\}=\{C,D\}$ is covered by the query restriction $Q_{\{C,D\}}$ that is the relation $T$; $\mathit{key}(C)\cup\{C\}=\{A,C\}$ is covered by the query restriction $Q_{\{A,C\}}$ that is the relation $S$. For the triangle query $Q_\vartriangle$ and variable order $A-B-C$: $\mathit{key}(C)\cup\{C\}=\{A,B,C\}$ is covered by $Q_\vartriangle$, while $\mathit{key}(B)\cup\{B\}=\{A,B\}$ is covered by relation $R$.
\punto
\end{example}

For any join query $Q$, its factorization width is the fractional hypertree width~\cite{Olteanu:FactBounds:2015:TODS}, a parameter that captures tracta\-bility for a host of computational problems~\cite{FAQ:PODS:2016}.

\begin{proposition}
\label{prop:factorization}
Given a join query $Q$, for every database $\db$, the result $Q(\db)$ admits:
\begin{itemize}
\item a flat representation of size $\bigO{\rho^*(Q)}$~{\em\cite{AGM:2013}};
\item a factorized representation of size $\bigO{\fw(Q)}$~{\em\cite{Olteanu:FactBounds:2015:TODS}}.
\end{itemize}

There are classes of databases $\db$ for which the above size bounds are tight. The flat and factorized representations of $Q(\db)$ can be computed worst-case optimally{\em~\cite{Ngo:SIGREC:2013,Olteanu:FactBounds:2015:TODS}}.
\end{proposition}


\subsection{Dynamic Factorization Width}
\label{sec:dynamic_width}

\milos{Doesn't consider other rings (only LR), indicator projections, and factorizable updates}

As in the non-incremental case, different variable orders may lead to wildly different performance of our IVM approach. In this section, we settle the question of which variable orders can best support IVM under updates to a given set of relations and thereby pinpoint the complexity of maintaining query results under updates. This is captured by a novel notion called {\em dynamic factorization width}, which is a refinement of the factorization width.

We first recall the complexities in the non-incremental case. There, we only materialize the root view of a view tree over a variable order with the smallest factorization width, and we thus have the time data complexity $\bigO{\fw(Q)}$ for computing factorized joins~\cite{Olteanu:FactBounds:2015:TODS} and aggregates over them~\cite{BKOZ:PVLDB:2013,FAQ:PODS:2016}; for cofactor matrices over factorized joins, there is an additional $\bigO{m^2}$ factor, since the sizes of these matrices can be quadratic in the number $m$ of variables (features)~\cite{SOC:SIGMOD:2016}. The space complexity is $\bigO{1}$ or $\bigO{m^2}$ to store the aggregate or cofactor matrix in addition to the database (modulo logarithmic factors in the data size for data iterators).

We next discuss the IVM case. Let $Q$ be any join query. For any variable order $\omega \in \Omega(Q)$, let $\tau(\omega)$ be the view tree inferred from $\omega$. This view tree has exactly one leaf for each relation symbol in $Q$.

We consider updates to relations whose relation symbols in $Q$ form a set ${\mathcal{U}}$; a relation may have several relation symbols  if it is involved in self-joins in $Q$, in which case all of them are in ${\mathcal{U}}$. For a relation symbol $R\in{\mathcal{U}}$, let $\Upsilon_{\tau(\omega)}(R)$ be the set of views that are ancestors of the leaf $R$ in $\tau(\omega)$, i.e., it consists of all the views (recursively) defined using $R$. 

The time needed to compute the delta for a view $\VIEW[keys]{V^{@X}_{rels}}$ is upper bounded by that of a join query $Q^{\sf rels}_{\sf keys \cup \{X\}-\sigma(R)}$ over relations in {\sf rels} that cover $X$ and the variables in {\sf keys} but excluding the variables in $R$. The reason for the exclusion is that a single-tuple update to $R$ binds the variables in $R$ to constants. The overall time to compute the deltas of all views in $\Upsilon_{\tau(\omega)}(R)$ is then
\begin{align*}
T(\omega,R) = \sum_{\VIEW[keys]{V^{@X}_{rels}}\in\Upsilon_{\tau(\omega)}(R)} \rho^*(Q^{\sf rels}_{{\sf keys\cup\{x\}}-\sigma(R)}).
\end{align*}

We are now ready to define the dynamic factorization width that captures the time complexity of incremental maintenance of $Q$ under updates to relations in ${\mathcal{U}}$.

\begin{definition}
Given a join query $Q$ and a set of relation symbols ${\mathcal{U}}$ in $Q$. Then, the {\em dynamic factorization width} of $Q$ and ${\mathcal{U}}$ is $\dfw(Q,{\mathcal{U}}) = \min_{\omega\in\Omega(Q)}\max_{R\in\mathcal{U}} T(\omega,R).$
\end{definition}

\begin{theorem}
Given a query $Q$ with $m$ variables, database $\db$, a payload ring $\RING$, and a set of relations ${\mathcal{U}}$ in $\db$. The time complexity of incrementally maintaining the result of $Q$ over the ring $\RING$ under single-tuple updates to relations in ${\mathcal{U}}$ is $\bigO{\dfw(Q,{\mathcal{U}})\cdot T_\RING}$, where $T_\RING$ is $\bigO{1}$ for rings of numbers and $\bigO{m^2}$ for the degree-$1$ matrix ring.
\end{theorem}

\begin{example}\label{ex:time-complexity}
\em
For our query $Q$ in Section~\ref{sec:intro_example} and database $\db$, the (static) factorization width is $\fw(Q)=O(|R|+|S|+|T|)$. Under single-tuple updates to relations in a set ${\mathcal{U}}_1\subseteq\{R,S\}$, the dynamic factorization width is $\dfw(Q,{\mathcal{U}}_1)=1$ since there are no free variables of the views over $R$ or $S$ in the variable order in Figure~\ref{fig:example_intro_varorder}. This means that we can maintain the result of a sum aggregate over $Q$ in $\bigO{1}$ time under ${\mathcal{U}}_1$ updates. The same holds for ${\mathcal{U}}_2\subseteq\{S,T\}$, i.e., $\dfw(Q,{\mathcal{U}}_2)=1$, as supported by the variable order $C-\{ D, A - \{ B, E \}\}$. However, $\dfw(Q,{\mathcal{U}}_3)=\bigO{|\db|}$ for ${\mathcal{U}}_3=\{R,S,T\}$ since there is no variable order without free variables above all three relations and some variable orders have one free variable above at least one of the three relations. Under the variable order in Figure~\ref{fig:example_intro_varorder}, $\dfw(Q,{\mathcal{U}}_3)=\min(|R|,|S|)$.

The triangle query $Q_\vartriangle$ has the (static) factorization width $\fw(Q_\vartriangle)= \rho^*(Q_\vartriangle)$. For any relation $U \in \{ R, S, T \}$, the dynamic factorization width is $\dfw(Q,\{ U \})=1$ as supported by a path variable order that has the variables in $U$ as prefix. We can thus maintain an aggregate over the triangle query in $\bigO{1}$ under single-tuple updates to exactly one of its three relations. For updates to at least two relations ${\mathcal{U}}_4$, $\dfw(Q,{\mathcal{U}}_4)=O(|\db|)$. For instance, assume a variable order $A-B-C$. We need to cover: no variable under updates to $R$; one of the variables $A$ or $B$ under updates to $S$ or $T$ respectively (the case for other permutations of this variable order is analog). Maintenance has thus lower time cost than recomputation.
\punto
\end{example}

We next analyze the space complexity $S(Q)$ of our approach. This is the sum of the sizes of the views in a view tree. The space needed by the keys of a view $\VIEW[keys]{V^{@X}_{rels}}$ is given by the fractional edge cover of a join query built using relation symbols {\sf rels} to cover the variables in {\sf keys}. To obtain the minimum size, we go over all variable orders of $Q$:
\begin{align*}
 S(Q) = \min_{\omega\in\Omega(Q)}\sum_{\VIEW[keys]{V^{@X}_{rels}}\in\tau(\omega)} \rho^*(Q^{\sf rels}_{\sf keys}).
\end{align*}

\begin{theorem}
Given a query $Q$ with $m$ variables, database $\db$, a payload ring $\RING$. The space complexity required by the materialization of a view tree for $Q$ over the ring $\RING$ is $\bigO{S(Q)\cdot T_\RING}$, where $T_\RING$ is $\bigO{1}$ for the sum ring and $\bigO{m^2}$ for the degree-$m$ matrix ring.
\end{theorem}

There are three differences between the formula $S(Q)$ and Definition~\ref{def:fw} of the factorization width $\fw(Q)$: (1) the use of summation vs. maximum, though the gap between them is linear in $m$ and thus independent of the database size; (2) the cover for $S(Q)$ can only use relation symbols of the view; (3) for $S(Q)$, we only need to cover $\sf keys$ and not also the variable at the view as in the case of $\fw(Q)$. The interplay of (2) and (3) can in fact make $S(Q)$ larger than $\fw(Q)$.
For acyclic queries, both complexities are linear if all relations have the same size and $S(Q)$ can be smaller than $\fw(Q)$ in case some relations are asymptotically smaller than others. 
For cyclic queries, however, $S(Q)$ can be larger than $\fw(Q)$. We show this for the triangle query $Q_\vartriangle$ and relations of the same size $N$. Under any variable order, there is a view of size $\bigO{N^2}$, whereas $\fw(Q_\vartriangle)=N^{3/2}$. For instance, for the variable order $A-B-C$, we materialize the view $\VIEW[A,B]{V^{@C}_{ST}} = \VSUM_{C} \VIEW[B,C]{S} \VPROD \VIEW[C,A]{T} \VPROD \VIEW[C]{\VLIFT_{C}}$, which may create $\bigO{N^2}$ pairs $(A,B)$ as we need both $S$ and $T$ to cover the variables $A$ and $B$. To avoid the large intermediate result, we join all three relations at the same time~\cite{Ngo:SIGREC:2013}, so as to cover $(A,B)$ using $R$. That would, however, require recomputation of this 3-way join for each update. This takes $\bigO{N}$ time since only two of the three variables are bound to constants. In contrast, our IVM approach trades off space for time: We need $\bigO{N^2}$ space but then support $\bigO{1}$ updates to one of the three relations (Example~\ref{ex:time-complexity}).

%%%%%%%%%%%%%%%%%%%%%%%%%%%%%%%%%%%%%%%%%%%%

% \paragraph{Efficiency comparison with baselines.}
\paragraph{Complexity analysis on SN-Part.} 
We analyze the space and time complexity to validate the computational efficiency of SPoTr during inference time with a batch size of 8.
For a baseline, SPA in SPoTr is replaced by the standard global self-attention (abbreviated in GSA) with CWPA. 
For a comparison with GSA requiring the quadratic complexity, we inevitably use the variants of SPoTr, where the channel size of each layer is reduced by $\times 1/4$.
For space complexity, we measure  the number of parameters and total memory usage, and for time complexity, we measure FLOPs and throughput performance.
\Cref{tab:complexity} empirically proves the efficiency over GSA.
For space complexity, GSA shares a similar number of parameters with SPA but introduces a large memory usage of 24.2 (GB). Instead, Our SPA only uses 2.5 (GB) (-89.7\%). Also, SPA largely reduces the time complexity from 114.0 GFLOPS with a throughput of 17.7 (shapes/s) to 10.8 GFLOPS (-90.5\%) with a throughput of 281.5 (shapes/s) ($\times$15.9).
% To investigate the efficiency of SPoTr, we compare our method with recent baselines~\cite{ma2022rethinking, ran2022surface} on ScanObjectNN~\cite{uy2019revisiting}. Specifically, we use DeepSpeed~\cite{rasley2020deepspeed} library as a profiler for calculating the number of parameters and FLOPS during inference. For comparison, we also opt a light version of SPoTr (SPoTr*), where the channel size of each layer is 2/3. The results are summarized in~\Cref{tab:efficiency}. It is worth noting that SPoTr achieves the best performance (88.6\%) with fewer parameters of 3.3(M) than 13.2(M) of PointMLP and 6.8(M) of RepSurf. Further, although SPoTr* only requires 1.6(M) parameters and 5.5 GFLOPS, it still shows significant gains over the previous best methods (+2.2\%). In short, we demonstrate that SPoTr is a computation-efficient and memory-efficient method. 
% \begin{table}[t!]
    \centering
    \caption{
    \textbf{Efficiency comparison of optimization algorithms.}
    R@1 scores evaluated on MSRVTT-7k for video retrieval are recorded.
    Multi-task learning simultaneously trains all tasks with even loss weights. 
    CG and FP are abbreviations of conjugate gradient and fixed-point optimization. 
    In terms of time costs, average training time per epoch is reported. 
    $^\dagger$ refers to our optimization algorithm which approximates $\nabla^2_w \aux$ as the identity matrix $\mathrm{I}$.}
    \begin{adjustbox}{width=\linewidth}
    \begin{tabular}{l |c| c  c}
        \toprule
        \textbf{Method}  & \textbf{Opt. Scheme}  & \textbf{R@1} &  \textbf{Time} \\
        \midrule
        \midrule
        Multi-task Learning   & 
        - &  
        26.1 \scriptsize(+0.0)    & 
        547 \scriptsize(+0.0\%) \\
        
        \textbf{MELTR} + Meta-Weight Net~\cite{shu2019meta}  & 
        ITD &  
        27.3 \scriptsize(\textcolor{red}{+1.2})  & 
        1,296 \scriptsize(\textcolor{red}{+136.9\%}) \\ 
        
        \textbf{MELTR} + StocBIO~\cite{ji2021bilevel} & 
        N/A  &  
        26.8 \scriptsize(\textcolor{red}{+0.7})   &   
        686 \scriptsize(\textcolor{red}{+25.4\%})\\
        
        \textbf{MELTR} + CG & 
        AID-CG &  
        28.0 \scriptsize(\textcolor{red}{+1.9})   &   
        624 \scriptsize(\textcolor{red}{+14.1\%})\\
        
        \textbf{MELTR} + AuxiLearn~\cite{navon2020auxiliary} &  
        AID-FP    &  
        27.9 \scriptsize(\textcolor{red}{+1.8})    &
        638 \scriptsize(\textcolor{red}{+16.6\%})      \\
        
        \textbf{MELTR} + \textbf{AID-FP-Lite}$^\dagger$ & 
        AID-FP &  
        28.5 \scriptsize(\textcolor{red}{+2.4})   &   
        574 \scriptsize(\textcolor{red}{+4.9\%})\\
        \bottomrule
    \end{tabular}
    \end{adjustbox}
    \label{tab:efficiency}
    \vspace{-3mm}
\end{table}

% SPA introduces 188 seconds additional time with $+1.8$MB of memory usage, whereas 
% In sum, our SPA reduces space and time overhead about ${3.5}\times$ and $5\times$ lower than GSA, respectively.

% \SH{
% To probe the efficiency of SPA, we analyze the space and time complexity on S3DIS during training with a single Nvidia RTX A6000 GPU.
% For a baseline, SPA in SPoTr is replaced with global self-attention(GSA).
% For space complexity, we measure memory usage with a batch size of 8.
% Also, the time complexity is measured by considering the latency per epoch of SPA and GSA block, respectively.
% % To probe the efficiency of SPA, we compare SPoTr with the model
% % From the \Cref{table6}
% SPA introduces 188(s) additional computational time with 10.8(GB) of memory usage.
% Whereas, GSA requires much more extra costs: 35.6(GB) memory usage and +910 seconds.
% Remarkably, our SPA reduces extra space and time overhead with about ${3.5}\times$ and $5\times$ lower than GSA, respectively.
% % The model B is SPoTr, which uses both LPA and SPA.  
% % From the table, 
% % In \Cref{table6}, we report the memory usage and  
% % For complexity analysis, using single Nvidia RTX A6000, we measure the cost on S3DIS with the batch size of 8 during training.
% % We measure the costs on a single Nvidia RTX A6000 GPU with the batch size of 8 during training for complexity analysis.
% % Baseline is model A that only considers local shape context with LPA.
% % Compared to baseline, model B~(SPoTr) introduces 188(s) additional computation with +1GB increment of memory usage. 
% % Whereas, with model C, where we add global self-attention(GSA) instead SPA, it requires much more extra costs: +25.8GB memory and +910 seconds. 
% % Remarkably, our SPA reduces extra space and time overhead with ${26}\times$ and $5\times$, respectively, lower cost than model C, respectively.
% % Whereas, although global self-attention(GSA) is natural for capturing global shape context, it is ineffective in both space and time complexity. 
% % As shown in table, compared to baseline model without SPA, 
% % we observe that SPoTr $w/$ SPA outperforms SPoTr $w/$ GSA with fewer latency. 
% }
% %Flops & time complexity table using shapenet part


    \subsection{Qualitative analysis}
    \label{sec:4.3}
For a deeper understanding of each component in SPA, such as self-positioning points~(SP points) and disentangled attention, we provide qualitative results in this section.
We use SN-Part for visualizations.

\paragraph{Self-positioning points.}
As mentioned in Section~\ref{sec:3.2}, it is important that SP points are adaptively located considering the input shape.
% We visualized SP points (colored in cyan) of the first SpoTr block in our model.
\Cref{fig:SPpoints} shows that the SP points are adaptively located on various samples from different categories.
Interestingly, a specific SP point (colored in red) appears at a \textit{semantically similar} place for each category.
For example, red dots from airplanes are always near the left-wing.
% , red dots from the table are always on the leg.
This consistent placement of SP points implies that each SP point learns to represent semantically similar regions.

\paragraph{Disentangled attention.}
SPA aggregates feature considering spatial proximity as well as semantic proximity via disentangled attention as introduced in \Cref{sec:3.2}. 
\Cref{fig:dis_att} shows the weights of the spatial kernel $g$ and the effective receptive field of the disentangled attention $g\cdot h$.
Cyan-colored points are selected SP points and kernel weights are illustrated in heatmaps.
With only the spatial kernel $g$, SP point blindly aggregates the information of neighbor points inducing \textit{irrelevant} information from close regions (\eg, a wing and a body of an airplane are strongly colored in the second row of the figure).
Conversely, with our disentangled attention $g\cdot h$, the same SP point selectively aggregates information considering both spatial proximity and semantic proximity~(\eg, the right-wing is only colored in the figure).
The obvious difference suggests disentangled attention is crucial for enhancing the descriptive power by suppressing irrelevant information.

% Furthermore, we qualitatively validate the global cross-attention in the distribution step.
% \Cref{fig:fig6}(c) displays the attention weight, which indicates how the SP point distributes information to entire input points.
% As shown in the figure, the SP points \textit{non-locally} influence semantically related points.
% For example, on the airplane in \Cref{fig:fig6}(c), the SP point conveys the information aggregated from one wing to semantically related points including the remote points at the other wings.


% \SH{focal points need to be} adaptively located considering the input shape.
%it is important that focal points are adaptively located considering the input shape.

% For each shape, cyan-colored points are entire points from the latent vector $\Zfeat_k$, and the red-colored point is a point from the specific index of the $\Zfeat_k$.
% By and large, the focal points are self-positioned in accordance with the various shape from the learned latent vector $\Zfeat_k$ without any explicit information regarding their positions. 
% In addition, we observed that red dots appear at a \textit{semantically similar} place for each category.
% For example, red dots from airplanes are always near the right wing, red dots from guitars are always on the neck.
% This consistent observation implies each element from the latent vector implicitly represents the specific semantics.
% points from the same latent index understand the same semantics.

% Interestingly, our attention weights look similar to part segmentation labels as if part labels have emerged from the global cross-attention.

% \jyp{
% To validate the effectiveness of disentangled attention, we compute the receptive field of the self-positioning global point by measuring the weight of spatial kernel $h$ and disentangled attention $h\cdot g$. 
% \Cref{fig:fig6} illustrates the measured weights between input points of the shape~(colored with a white) and a single global point~(colored with a cyan) with the intensity of yellow color.
% % As shown in the leftmost of the figure of each sample, when applying only the spatial kernel, the weight has a high value when the input point is just located near the global point.
% % smoothing이야기 같이 위에 녹여내야 한다.
% % Whereas, the disentangled attention refines the local information and receives information from the semantically similar points.
% As shown in the figure, the receptive fields of spatial kernel only considers local information, whereas the disentangled attention seems to simultaneously adapt spatial and semantic information.
% In the case of the airplane in the figure, the global point aggregates noisy information such as a body part when applying spatial kernel.  
% On the contrary, the disentangled attention refines local information and receives semantically similar information such as a wing part.
% This flexible receptive fields show that disentangled attention helps self-positioning global points construct powerful representations by receiving information from meaningful points.
% }
% \paragraph{Global cross-attention visualization.}

% \jyp{
% Further, based on the representations of self-positioning points $\Vfeat$, we analyze the behaviour of cross-attention~\eqref{eq:globalatt} in SPA.
% \Cref{fig:fig6}(c) shows the attention weight, which is measure of how global points influence on input points.  
% We can see that the global points non-locally influence a lot on related points.
% As shown in the figure, higher weights are assigned in arms of chair, tires of motorcycle, upper part of desks, and wings of airplane.
% This adaptive behaviour demonstrates that a small set of global points globally convey important information to relevant input points.  
% % Indeed, the global point aggregates information from 
% }

\section{Conclusion}
    \label{sec:5}
In this paper, we propose SPoTr, a Transformer for point clouds, which captures both local and global shape context without the quadratic complexity of input points.
SPoTr includes two attention modules: self-positioning point-based attention~(SPA) and local points attention~(LPA).
SPA is a novel global cross-attention, which aggregates information via disentangled attention and non-locally distributes information to entire points.
% Our self-positioning mechanism allows a small number of SP points to cover overall shapes and provides improved expressive power.
% Properly located self-positioning points enable improved expressive power with a small number of
Our experiments show superior performance across various tasks, including ScanObjectNN, SN-Part, and S3DIS.
% and visualizations demonstrate the effectiveness and interpretability of SPA. 
% Our codes are available in the supplement.

% Limitations and negative societal impacts are discussed in the supplement.

% With self-positioning receptive fields based attention~(SPA) and local points attention~(LPA), SPoTr captures both local and global information.
% aggregates local inforation and distributes the global shape context.
% We also analyze SPoTr in both quantitative and qualitative manner with various benchmark datasets to probe the applicability.
% Further, as shown in visualizations, the behaviour of self-positioning receptive fields is fully interpretable.
% Further, the visualization shows the 
% We present a novel semi-supervised strategy with Metropolis-Hastings algorithm based augmentation
% method. This is the first work to impose data augmentation on graph-structured data from a perspective
% of a Markov chain Monte Carlo sampling. We theoretically and experimentally show the convergence
% of augmented samples to target distribution and demonstrate its consistent performance improvement
% over baselines across five benchmark datasets.
% In this paper, we present a Transformer architecture called SPoTr for point clouds, which efficiently captures both local and global shape context.
% SPoTr includes two attention modules: self-positioning receptive fields based attention~(SPA) and local points attention~(LPA).
% SPA is a novel global cross-attention mechanism with self-positioning receptive fields.
% SPA aggregates information of local neighborhoods via disentangled attention and non-locally influence on entire points. 
% In our experiment, SPoTr consistently shows competitive performance on various tasks.
% Especially, SPoTr achieves new state-of-the-art performance in the ScanObjectNN dataset with reduced complexity.
% Our visualizations demonstrate the effectiveness and interpretability of SPA. 
% Limitations and negative societal impacts are discussed in the supplement.

\paragraph{Acknowledgements.}
This work was supported by the MSIT, Korea, under the ICT Creative Consilience program (IITP-2023-2020-0-01819) supervised by the IITP and the Virtual Engineering Platform Project  (P0022336) of the Ministry of Trade, Industry and Energy (MOTIE), Korea.



%%%%%%%%% REFERENCES
{\small
\bibliographystyle{unsrt}
\bibliography{egbib}
}

\end{document}
