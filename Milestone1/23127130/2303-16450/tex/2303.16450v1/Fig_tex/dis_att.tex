\begin{figure}[t] 
\centering

\includegraphics[trim=50 20 30 10, width=1.0\columnwidth]{Figures/dis_att_compressed.pdf}
% \vspace{-20pt}
\caption{
\textbf{Visual comparison of disentangled attention with a spatial kernel.}
{
% SP point aggregates feature with disentangled attention which filters out irrelevant information.
A spatial kernel (Middle) only considers spatial proximity without considering semantic relevance. Differently, Disentangled Attention~(Right) filters out irrelevant information.}}


% SPA aggregates feature with disentangled attention and then non-locally distributes the information using (c) global cross-attention.
% \textcolor{blue}{Blue} arrows illustrate information aggregation from points to the focal point.
% \textcolor[RGB]{96,96,96}{Gray} arrows stand for suppressed influence by $g \cdot h$.
% \textcolor[RGB]{0,153,0}{Green} arrows represent the focal point influence on semantically related points.}
\label{fig:dis_att}
% \vspace{-15pt}
\end{figure}

% Unlike (a) spatial kernel that only considers spatial proximity, SPA aggregates features with (b) disentangled attention $g \cdot h$.
% SPA aggregation with (a) the spatial kernel $g$ and (b) disentangled attention $g \cdot h$. 
% Then, \textcolor{cyan}{focal points} distribute information non-locally using global cross-attention (c).
% from semantically dissimilar parts are removed
% are from the area, where points have high proximity to focal point
% A self-positioning \textcolor{cyan}{focal point} consider spatial proximity with spatial kernel $g$~(Blue arrows are from the area, where points have high proximity to focal point). (b) With semantic kernel $h$, disentangled attention $g \cdot h$ improves the descriptive power by aggregating semantically similar local points~(Gray arrows from semantically dissimilar parts are removed).  (c) Then, \textcolor{cyan}{focal point} distributes the information non-locally using cross-attention~(Green arrows from focal point represent the area of the influence)
