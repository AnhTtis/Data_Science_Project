\begin{table}[t]
  \centering
\setlength{\tabcolsep}{5pt}
  \begin{tabular}{l|c|cc}
    \toprule
    Methods & Year & cls. mIoU & ins. mIoU\\
    \midrule
    PointNet~\cite{qi2017pointnet}&2017 & 80.4& 83.7\\
    PointNet++~\cite{qi2017pointnet++}&2017 & 81.9 & 85.1 \\
    PointCNN~\cite{li2018pointcnn}&2018& 84.6 & 86.1 \\
    DGCNN~\cite{wang2019dynamic}&2019 & 82.3 & 85.1 \\
    RSCNN~\cite{liu2019relation} & 2019 & 84.0 & 86.2 \\
    KPConv~\cite{thomas2019kpconv}&2019 & 85.1&86.4 \\
    PointConv~\cite{wu2019pointconv}&2019 & 82.8 & 85.7 \\
    PointASNL~\cite{yan2020pointasnl}&2020 & - & 86.1 \\
    PCT~\cite{guo2021pct}&2021 & - & 86.4\\
    PAConv~\cite{xu2021paconv}&2021 & 84.6  &  86.1 \\
    AdaptConv~\cite{zhou2021adaptive}&2021 & 83.4 & 86.4\\
    PointTransformer~\cite{zhao2021point}&2021& 83.7 & 86.6 \\
    CurveNet~\cite{xiang2021walk}&2021 & - & 86.8 \\
    PointMLP~\cite{ma2022rethinking} & 2022 &84.6 & 86.1\\
    \rowcolor{LightYellow}PointNeXt~\cite{qian2022pointnext} & 2022& 85.2 $\pm$ 0.1 & 87.0 $\pm$ 0.1 \\
    \midrule
    \rowcolor{LightRed}\textbf{SPoTr} & 2023& \textbf{85.4} & \textbf{87.2}\\


    \bottomrule
  \end{tabular}
  \caption{\textbf{Part segmentation results on SN-Part.} ins. mIoU is the mean of instance IoU. cls. mIoU is the mean of the class IoU. 
  }
  \label{tab:shapenet}
% \vspace{-5pt}
\end{table} 
