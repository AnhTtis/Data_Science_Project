\pdfoutput=1
\documentclass[letterpaper, 10 pt, conference]{ieeeconf}

\IEEEoverridecommandlockouts
\overrideIEEEmargins

%%%%%%%%% PAPER TYPE  - PLEASE UPDATE FOR FINAL VERSION
% \usepackage[review]{cvpr}      % To produce the REVIEW version
%\usepackage{cvpr}              % To produce the CAMERA-READY version
%\usepackage[pagenumbers]{cvpr} % To force page numbers, e.g. for an arXiv version

% Include other packages here, before hyperref.
\usepackage{graphicx}
\usepackage{amsmath}
\usepackage{amssymb}
% \usepackage{setspace}
%\usepackage{booktabs}
\usepackage{svg}
\newcommand{\etal}{\textit{et al.}}


% It is strongly recommended to use hyperref, especially for the review version.
% hyperref with option pagebackref eases the reviewers' job.
% Please disable hyperref *only* if you encounter grave issues, e.g. with the
% file validation for the camera-ready version.
%
% If you comment hyperref and then uncomment it, you should delete
% ReviewTempalte.aux before re-running LaTeX.
% (Or just hit 'q' on the first LaTeX run, let it finish, and you
%  should be clear).
% \usepackage[pagebackref,breaklinks,colorlinks]{hyperref}
\usepackage{hyperref}
\hypersetup{breaklinks,colorlinks}


% Support for easy cross-referencing
\usepackage[capitalize]{cleveref}
\crefname{section}{Sec.}{Secs.}
\Crefname{section}{Section}{Sections}
\Crefname{table}{Table}{Tables}
\crefname{table}{Tab.}{Tabs.}


%%%%%%%%% PAPER ID  - PLEASE UPDATE
% \def\cvprPaperID{10301} % *** Enter the CVPR Paper ID here
% \def\confName{CVPR}
% \def\confYear{2023}

%%%%%%%%% TITLE - PLEASE UPDATE
\title{\LARGE \bf
Co-Speech Gesture Synthesis using Discrete Gesture Token Learning
}
% Another candidate for the title
% Co-Speech Gesture Synthesis using Discrete Gesture Token Learning

\author{Shuhong Lu$^{1}$, Youngwoo Yoon$^{2}$, and Andrew Feng$^{1}$
%\thanks{This work was supported by ...}
\thanks{$^{1}$Shuhong Lu and Andrew Feng are with Institute for Creative Technologies, University of Southern California, Los Angeles, USA
{\tt\small \{shuhongl, feng\}@usc.edu}}%
\thanks{$^{2}$Electronics and Telecommunications Research Institute, Daejeon, Republic of Korea
{\tt\small youngwoo@etri.re.kr}}%
}

\begin{document}
\maketitle
\thispagestyle{empty}
\pagestyle{empty}

%%%%%%%%% ABSTRACT
\begin{abstract}
    
   Synthesizing realistic co-speech gestures is an important and yet unsolved problem for creating believable motions that can drive a humanoid robot to interact and communicate with human users. Such capability will improve the impressions of the robots by human users and will find applications in education, training, and medical services. One challenge in learning the co-speech gesture model is that there may be multiple viable gesture motions for the same speech utterance. The deterministic regression methods can not resolve the conflicting samples and may produce over-smoothed or damped motions. We proposed a two-stage model to address this uncertainty issue in gesture synthesis by modeling the gesture segments as discrete latent codes. Our method utilizes RQ-VAE in the first stage to learn a discrete codebook consisting of gesture tokens from training data. In the second stage, a two-level autoregressive transformer model is used to learn the prior distribution of residual codes conditioned on input speech context. Since the inference is formulated as token sampling, multiple gesture sequences could be generated given the same speech input using top-k sampling. The quantitative results and the user study showed the proposed method outperforms the previous methods and is able to generate realistic and diverse gesture motions. 
\end{abstract}

%%%%%%%%% BODY TEXT
\vspace{-10pt}
%%%%%%%%% BODY TEXT

\section{Introduction}
\label{section:introduction}
%% 1. why should someone care?

%The advent of advanced interactive computer vision systems~\cite{hololens} and recent progress in vision-language and multi-modal models~\cite{} opens doors for such next generation of assistive agents. 
% We envision that the future assistive agents would build up on these visual and language reasoning capabilities of today and empower users to achieve goals in their everyday lives. In particular, such agents would be able to reason about \emph{unseen} human goals... 
% We posit that such agents would require the ability to understand user goals described in natural language at high-level i.e., without complete details about as well as unseen user goals. 

%Recent progress in augmented reality systems~\cite{hololens, magicleap}, as well as vision-language and multi-modal models~\cite{}, opens doors for the next generation of assistive agents. 
Inspired by recent progress in visual systems~\cite{MagicLeap, ungureanu2020hololens}, we consider an assistive egocentric agent capable of reasoning about daily activities. When invoked via natural language commands, for e.g., while baking a cake, the agent understands the steps involved in baking, tracks progress through the various stages of the task, detects and proactively prevents mistakes by making suggestions. Such an agent would empower users to learn new skills and accomplish tasks efficiently.
% One could envision invoking such an agent merely through natural language descriptions of tasks similar to how present day assistants such as Alexa, Siri etc.~\cite{voice_assistants} are invoked. 
%We envision such agents to empower users in daily life by  invoking them naturally through 

%% 2. Why is it challenging? 
%While recent progress in vision-language and multi-modal models~\cite{} opens doors for such next generation of assistive agents, various challenges remain in making such agents a reality. 
%To make such agents a reality, 

Developing such an egocentric agent capable of tracking and verifying everyday tasks based on their natural language specification is challenging for multiple reasons. First, such an agent must reason about various ways of doing a \emph{multi-step} task specified in natural language. This entails decomposing the task into relevant actions, state changes, object interactions as well as any necessary causal and temporal relationships between these entities. Secondly, the agent must ground these entities in egocentric observations to track progress and detect mistakes. Lastly, to truly be useful, such an agent must support tracking and verification for a combination of tasks and, ideally, even unseen tasks. These three challenges -- causal and temporal reasoning about task structure from natural language, visual grounding of sub-tasks, and compositional generalization -- form the core goals of our work.

% %% 3. What are we doing? What is our approach?
% \aks{I think this is a matter of preference, but I personally don't like related work in intro. I would make this paragraph be about EgoTV and NSG. Starting with something like - "To this end, we propose...", ie, your next paragraph.}
% \nk{+1, we should move parts of this para to lit review and delete the rest.}
% Recent research on language modeling enables decomposing tasks into multiple steps from natural language descriptions~\cite{llm_zero_shot_planning,proscript}. However, such \emph{task decompositions} cannot directly be leveraged for task tracking in egocentric agents because of lack of grounding into the visual observations or context. In parallel, the computer vision community has advanced action recognition~\cite{}, object detection and tracking~\cite{}, hand object interaction and object state change detection~\cite{ego_4d,change_it,}, step classification in procedural tasks~\cite{}, and even vision language reasoning~\cite{nsvqa,nscl,star_situated_reasoning,clevrer}, which may help with the grounding challenge. However, majority of current research on identifying actions, objects, steps, or state changes does not account for the overall task structure. Likewise, predominant research on vision language understanding~\cite{} and multi-modal grounding~\cite{} does not consider the temporal and causal constraints that emerge in task tracking and verification. We therefore focus on the order-aware visual grounding problem in our work, with an eye towards compositional generalization to scale usability of these agents. In particular, we aim to achieve visual grounding of the actions and objects corresponding to each step or sub-task obtained from the task description decomposition in an order-aware manner.

%% 4. What are our results/contributions?
As our first contribution, we propose a benchmark -- \emph{\textbf{Ego}centric \textbf{T}ask \textbf{V}erification} (\etv \inlineimg{figures/TV}) -- and a corresponding dataset in the AI2-THOR~\cite{ai2thor} simulator. % \emoji{tv}
Given a natural language (NL) task description and a corresponding egocentric video of an agent, the goal of \etv is to verify whether the task was successfully completed in the video or not.
\etv contains multi-step tasks with \emph{ordering} constraints on the steps and \emph{abstracted} NL task descriptions with omitted low-level task details inspired by the needs of real-world assistants. We also provide splits of the dataset focused on different generalization aspects, e.g., unseen visual contexts, compositions of steps, and tasks (see Figure~\ref{figure:dataset}).
% Next, we create splits of the dataset focused on different aspects of generalization, ranging from generalization to unseen visual context to unseen compositions of steps and tasks. Figure~\ref{figure:dataset} shows an example task and overview of generalization splits from \etv. Succeeding at \etv tasks requires decomposing tasks into partially-ordered steps from the NL description and order-aware visual grounding of these steps into the video. 

Our second contribution is a novel approach for order-aware visual grounding~--~\emph{\textbf{N}euro-\textbf{S}ymbolic \textbf{G}rounding} (NSG), capable of compositional reasoning and generalizing to unseen tasks owing to its ability to leverage abstract NL descriptions and compositional structure of tasks (task decomposition, ordering).~In contrast, state-of-the-art vision-language models~\cite{coca,clip,videoclip,clip_hitchiker} struggle to ground NL descriptions in egocentric videos, and do not generalize to unseen tasks.~NSG outperforms these models by~$\mathbf{33.8}\%$~on compositional generalization and~$\mathbf{32.8}\%$~on abstractly described task verification. Finally, to evaluate \nsg on real-world data, we instantiate \etv on the CrossTask~\cite{cross_task} instructional video dataset. %Specifically, we synthetically create videos with mistakes in CrossTask. 
We find that it also outperforms state-of-the-art models at task verification on CrossTask. We hope that the \etv~benchmark and dataset will enable future research on egocentric agents capable of aiding in everyday tasks.

% We experiment with many for the \etv tasks. We find that while these models generalize well to unseen visual context, they struggle to perform grounding from abstracted task descriptions and to generalize to new compositions of tasks. To deal with these challenges, we take inspiration from recent research on and develop . ~\rd{unclear why neurosymbolic models would do well on abstraction.} 

% To summarize, our main contributions are:~1)~\etv: a benchmark and synthetic dataset to systematically study egocentric task verification.
% 2)~\nsg: a novel neuro-symbolic approach to enable the core reasoning capability for \etv -- order-aware visual grounding. We demonstrate \nsg's capability on our synthetic \etv dataset as well as a real-world dataset derived from CrossTask. We will release both of these datasets and our models for future research on egocentric task tracking and verification. 


% Assistive agents require the ability to track actions and state changes from an egocentric perspective for effective assistance in day-to-day tasks. For example, an agent helping a user prepare a recipe would need to both generate the steps of the recipe (\textit{plan generation}) and track the user's actions to ensure the plan is executed correctly (\textit{plan verification}). We formulate this as a Video Entailment task~\cite{violin_dataset,9710490} \rd{should we call our task video-based goal entailment?}, wherein, given an egocentric video of an agent (or human) performing a task (\textit{premise}) and a NL task description (\textit{hypothesis}), the objective is to learn a model to track whether the given task was successfully executed in the video. 
% An ideal model should also be able to seamlessly generalize to novel compositions (of actions and objects) unseen during training. \rd{add a line about what we mean by abstraction and why is it important.} To this end, we generate a novel Vision-Language dataset on the AI2-THOR simulator~\cite{ai2thor} to study compositional and abstraction-based generalization. Our dataset provides effective evaluation measures in a controlled setting, while closely reflecting the diversity of real-world events. We implement and train a variety of end-to-end models based on existing state-of-the-art approaches. We empirically demonstrate that neural models suffer from overfitting and cannot effectively generalize to novel compositions of actions, objects, and scenes. 
% To address this problem, we propose an end-to-end Neuro-Symbolic (NeSy) framework that performs plan generation and verification. At the heart of our approach is the hypothesis that symbolic reasoning models are good at generalization and capturing compositional substructure, while neural models are good at learning representations from sensory data~\cite{10.5555/3326943.3327039,nscl,clevrer}. \rd{summarize contributions in a bulleted list.} \rd{also add a line about the main result e.g., x\% improvement as compared to end-to-end models}. 

% \rd{we also evaluate NeSy with real-world data: add briefly about CrossTask experiments.}

% % \fbox{\begin{minipage}{\linewidth}
% % \textbf{Problem Statement}

% % Given:
% % (i) Premise: Egocentric video of an agent performing a task.
% % (ii) Hypothesis: NL description of the task.

% % Learn: A model to track whether the premise entails the hypothesis. The output of the model is True if the given task is executed successfully in the video.
% % \end{minipage}}

% \textbf{Contributions:} 
% \begin{itemize}
%     \item We generate a benchmark video-language dataset to study compositional and abstraction-based generalization.
%     \item We evaluate the performance of a variety of state-of-the-art models and show that these (baseline) models cannot effectively generalize to novel compositions of actions.
%     \item We propose a novel end-to-end NeSy approach that significantly outperforms the baselines on some compositional generalization splits while performing on par with them on the rest.
%     \item We also evaluate our NeSy approach with real-world data showing similar performance improvements.
% \end{itemize}

\section{Related Work}
% To-do : need to rewrite the related work text since it's currently copied from GENEA paper

%We first review traditional gesture generation methods and later introduce some work related to VQ-VAE. 
\subsection{Co-Speech Gesture Generation}
The early works in co-speech gesture synthesis utilized pre-defined sets of manually created gesture units to build a gesture database. The new gestures were generated via keyword matching or prosody analysis to find the best corresponding gesture units from the database \cite{kopp2003max, kopp2009, marsella2013}. The gesture unit database can also be created automatically from speech-gesture data by segmenting and clustering gesture motions based on the similarity of motions and speech contents. During synthesis, the desired gesture property and speech attributes are used to search for the best gesture segment in the database that matches the input speech content. Our method is motivated by these ideas from early works, but instead of manually building a gesture database, we used residual quantization to implicitly learn the discrete codebook of gesture tokens. 

Recent learning-based methods train an end-to-end model from speech-gesture datasets to predict gesture motions from speech. The methods based on direct regressions find a deterministic mapping from speech to gestures \cite{kucherenko2020gesticulator, ginosar2019gestures, Yoon2020Speech, bhattacharya2021speech2affectivegestures}. Since these methods do not handle the issue of one-to-many mapping, the adversarial scheme is sometimes utilized by training the model with an additional discriminator to improve the resulting motion qualities. Recent methods further improve synthesized results by using hierarchical architecture to model multi-level skeletal poses \cite{liu2022learning} or adding semantic prompter to force semantic alignment in output gestures \cite{liang2022seeg}. 

Probabilistic frameworks were also used in the recent gesture generation works to handle the gesture ambiguities \cite{ahuja2019language2pose, Alexanderson2020, Qian2021, Li2021, bhattacharya2021text2gestures}. This type of methods learns a latent space generative model and could generate multiple gesture motions from the same speech input using conditional sampling from the latent space during inference. Similar to the previous methods, our method utilizes a latent space model and conditional sampling to handle the one-to-many problems for gesture synthesis. However, instead of learning a continuous latent space, we applied residual quantization to learn \textit{discrete} latent codes. The discrete latent codebook provides a compact representation for gesture units and the inference process is reduced to selecting the most probable latent code from the codebook. Thus by learning the conditional prior distribution over the discrete latent codes, the method naturally handles the mapping from one speech input to multiple different gestures with varying probabilities.  

\subsection{Discrete Latent Space Learning}

Vector-quantized variational autoencoder (VQ-VAE) \cite{van2017neural} learns discrete representations as a codebook from images. In the two-staged generative architecture like in Video GPT \cite{yan2021videogpt}, an autoregressive prior can then be trained to model the categorical distributions for these discrete latent codes. It was first introduced for image synthesizing or compression tasks and is able to produce sharper and higher quality image synthesis results. It was further improved in \cite{razavi2019generating} using a multi-scale hierarchical architecture to model higher-resolution images. 

%One issue that often affects the reconstruction quality when training VQ-VAE is codebook collapse. It happens when the model only learns to use a small subset of the codes in the codebook, leaving a majority of the codes unused. This limits the expressiveness of the model and results in lower-quality results. Several methods and techniques have been proposed to prevent codebook collapse. Jukebox \cite{dhariwal2020jukebox} introduced re-initializing unused codes to a random vector to prevent dead codes during each training iteration. Video GPT \cite{yan2021videogpt} finds normalizing MSE for the reconstruction loss also mitigates codebook collapse. Hierarchical models were proposed for better codebook utilization in VQ-VAE2 \cite{razavi2019generating} by first extracting bottom and top features unconditionally to mitigate the codebook collapse. RQ-VAE~\cite{lee2022autoregressive} uses a fixed size of codebook to recursively quantize the feature map represented as a stacked map of discrete codes, which reduces the codebook size and stabilizes the codebook training. 

One issue that often affects the reconstruction quality when training VQ-VAE is codebook collapse, which leaves a majority of the codes unused and limits the expressiveness of the model. Several methods and techniques have been proposed to prevent codebook collapse. Jukebox \cite{dhariwal2020jukebox} introduced re-initializing unused codes to a random vector to prevent dead codes during each training iteration. Video GPT \cite{yan2021videogpt} finds normalizing MSE for the reconstruction loss also mitigates codebook collapse. Hierarchical models were proposed for better codebook utilization in VQ-VAE2 \cite{razavi2019generating} by first extracting bottom and top features unconditionally to mitigate the codebook collapse. RQ-VAE~\cite{lee2022autoregressive} uses a fixed size of codebook to recursively quantize the feature map represented as a stacked map of discrete codes, which reduces the codebook size and stabilizes the codebook training. 

The discrete latent space has also been applied for text-to-image synthesis, which generates new images based on input textual description using a two-stage architecture. It first learns a discrete representation for image patches and modeling the auto-regressive priors using transformers. Learning with discrete codes is more efficient over raw pixels since the transformer may not learn the fully dependencies between pixels. The work by Esser \etal \cite{esser2021} further applied adversarial training to learn VQ-GAN that produces a perceptually rich codebook. 

In addition to image synthesis, discrete latent space model is also known to be one of the state-to-the-art methods for modeling time-series data such as audio. Jukebox \cite{dhariwal2020jukebox} utilized VQ-VAE to generate singing music. It trained multi-level networks to compress audio in different resolutions into discrete space and then used autoregressive transformers to learn the latent codes for music generation. The same idea was also adapted to generate repetitive rhythms of music by learning from extracted music loops. Multi-Instrumentalist Net \cite{su2020multi} was proposed to generate multi-instrumental music from videos, which trained VQ-VAE along with an autoregressive prior conditioned on the musician’s body key points movements. 

%Our method is motivated by the recent success of applying a discrete latent space model in cross-modal synthesis tasks. 
Compared to previous works, the proposed co-speech gesture synthesis methods utilize RQ-VAE for modeling discrete gesture tokens and a two-level RQ-Transformer architecture to model conditional priors for gesture-generating tasks. The evaluation results show its potential for retaining motion quality while allowing non-deterministic motion synthesis from the same speech input.


% In our work, we do suffers from the codebook collapse problem in the beginning, output gestures are limited. We then applied used random restart of the codebook, exponential moving average updates for the codebook and adjusting MSE weight, and the results are much better. We may consider hierarchical architecture for VQ-VAE in the future as well.

%\subsection{Multi-modal Text-to-Image Synthesis}



%\section{Background VQ-VAE}
% the background text for introducing VQ-VAE & RQ-VAE?
%The proposed method is motivated by the recent works in cross-modal text-to-image synthesis \cite{dalle2021, esser2021} that utilize VQ-VAE as latent space representation for image patches and generate new images via autoregressive models to predict discrete tokens for each patch. As a high-level analogy for gesture generations, this could be seen as extracting a smaller set of gesture units from the training motions and learning the conditional probability distribution for these gesture units based on speech context and previous gestures. One motivation for learning the gesture units as discrete latent vectors is because the generation process can be seen as sampling from the codebook instead of interpolation within a continuous latent space and thus we hypothesize that the resulting gestures are more likely to retain their motion quality from original data. Moreover, learning the probability distribution in the discrete codebook will naturally handle the issue when different gesture motions are associated with the same speech in the training data since during inference the model can randomly pick one of these gesture units instead of outputting their average. 




% \begin{table}[t]
% \centering
% \caption{Details of the datasets used in our work.}
% \label{tab:datasets}
% \begin{tabular}{lccc}
% \hline
% Datasets & \begin{tabular}[c]{@{}c@{}}Exposure\\ Ratios\end{tabular} & \begin{tabular}[c]{@{}c@{}}Training\\ images\end{tabular} & \begin{tabular}[c]{@{}c@{}}Validation\\ images\end{tabular} \\ \hline
% Sony \cite{chen2018learning} & 90,150,300 & 161 &  36\\
% Nikon &  100, 300 & 53 &  24 \\
% Canon \cite{CanonLSID} & 50, 150, 300 & 44 &  21\\ \hline
% \end{tabular}
% \end{table}
%--------------------------------------------------------------------------------------------------------------------
% \begin{figure*}
% \begin{center}
% \includegraphics[width=\textwidth, clip, trim=0cm 15.55cm 0.75cm 0.05cm]{figures/FDA-LSID.png}
% \end{center}
%   \caption{Proposed few-shot domain adaptation model architecture.}
% \label{fig:model}
% \end{figure*}
% % \begin{figure*}
% % \begin{center}
% % \includegraphics[scale=0.42]{figures/convert.png}
% % \end{center}
% %   \caption{Source camera specific 16-to-8-bit converter.}
% % \label{fig:convert}
% % \end{figure*}
\begin{figure}[t]
\begin{tabular}{cc}
\includegraphics[page=1, width=5.8cm]{Images/nikon.png}&\hspace{+1mm}
\includegraphics[page=1, clip, trim=0.6cm 16.75cm 7.5cm 0.15cm, scale=0.9]{Images/block_diagram.pdf}\\
(a)&(b)
\end{tabular}
\caption{(a) Example short-exposure and long-exposure image pairs from the Nikon dataset. The short exposure images are almost entirely dark whereas the long-exposure images have immense scene information. (b) Overview of our few-shot domain adaptation method.}
\label{fig:nikon}
\label{fig:prop_overview}
\end{figure}
With a noisy raw image captured with low-exposure time (i.e., shutter speed) as input, our CNN-based approach is trained to predict a clean long-exposure sRGB output of the same scene. The input is multiplied by an exposure factor calculated by the ratio of output and input exposure times. For example, to generate a 10-second long exposure output, the input 0.1-second low exposure image must be multiplied by 100. As a result of this operation, along with illumination, the noise is also amplified proportionally. Since we multiply the factor in the unprocessed raw domain and expect the output in the sRGB domain, the network must learn camera hardware-specific enhancement as well as its entire ISP pipeline (lens correction, demosaicing, white balancing, color manipulation, tone curve application, color space transform, and Gamma correction). Thus, a model trained on one specific camera data (source domain) does not translate similar performance to a different camera (target domain), hence the domain gap. In this paper, we propose to transfer the enhancement task from large labeled source data and generate output in the target domain using few labeled target data.

\textbf{Problem formulation}: We denote source domain ($\mathbf{S}$) with input short-exposure images as $\{S_n\}$ and corresponding long-exposure ground truth as $\widehat{\mathbf{S}}\!=\!\widehat{S}_n$, $\forall n=1,\cdots,N$. Similarly, the target domain ($\mathbf{T}$) consists of input images $\{T_m\}$ and corresponding ground truth, $\widehat{\mathbf{T}}\!=\!\widehat{T}_m$, $\forall m=1,\cdots,M$. Note that $N$ is much greater than $M$, $N\gg M$. With both $\mathbf{S}$ and $\mathbf{T}$ as input, we train a CNN model ($\mathbb{N}$) to generate enhanced long-exposure output ($\widetilde{\mathbf{S}}$ and $\widetilde{\mathbf{T}}$). Our method is illustrated in Fig. \ref{fig:prop_overview}(b) with the source and target training pipelines. It is an end-to-end trainable deep network that takes the raw sensor arrays as input and performs image enhancement utilizing the source data for few-shot domain adaptation to the target data.

% \begin{figure*}[t]
%     \centering
%     \includegraphics[width=\linewidth]{3_BMVC/Images/Nikon-Results-1.pdf}
%     \caption{Qualitative comparison with methods tested on Nikon target images. (b) HDRCNN and (c) Unprocess are trained on Sony source and fine-tuned on 4-Nikon target images, LSID with (d) 4-Nikon target images and (e) full ($k$=53) Nikon training dataset, (f) Our few-shot domain adaptation approach with 4-Nikon target images and 161 Sony source images.}
%     \label{fig:nikon_eg1}
% \end{figure*}
% \begin{table}[t]
% \centering
% \caption{Quantitative comparison of Sony as source and Nikon as target dataset. The improvement of proposed method over only $k$ shot trained model is shown in brackets. The LSID model trained with full Nikon dataset ($k$=53) achieves 30.74dB PSNR and 0.803 SSIM.}
% \label{tab:nikon}
% \scalebox{0.735}{
% \begin{tabular}{@{}lccc|ccc@{}}
% \hline
%  & \multicolumn{3}{c|}{PSNR} & \multicolumn{3}{c}{SSIM} \\ \hline
% $k$ ($\rightarrow$) & 1 & 2 & 4 & 1 & 2 & 4 \\ \hline
% \begin{tabular}[c]{@{}l@{}}LSID\\ (only $k$ target)\end{tabular} & 23.20 $\pm$ 3.06 & 27.27 $\pm$ 0.384 & 28.05 $\pm$ 1.53 & 0.679 $\pm$ 0.172 & 0.819 $\pm$ 0.031 & 0.864 $\pm$ 0.0111 \\ \hline
% \begin{tabular}[c]{@{}l@{}}Proposed\\ ($k$ target + source)\end{tabular} & \begin{tabular}[c]{@{}c@{}}\textbf{25.27} $\pm$ 0.58\\ (+2.07)\end{tabular} &
% \begin{tabular}[c]{@{}c@{}}\textbf{28.06} $\pm$ 0.671\\ (+0.79)\end{tabular} &
% \begin{tabular}[c]{@{}c@{}}\textbf{30.30} $\pm$ 0.52\\ (+2.25)\end{tabular} & \begin{tabular}[c]{@{}c@{}}\textbf{0.860} $\pm$ 0.010\\ (+0.181)\end{tabular} &
% \begin{tabular}[c]{@{}c@{}}\textbf{0.909} $\pm$ 0.0028\\ (+0.090)\end{tabular} &
% % \begin{tabular}[c]{@{}c@{}}\textbf{0.913} $\pm$ 0.006\\ (+0.049)\end{tabular} \\ \midrule
% % \begin{tabular}[c]{@{}l@{}}LSID\\ (full target, $k$ = 53)\end{tabular} & \multicolumn{3}{c|}{30.74} & \multicolumn{3}{c}{0.803} \\ \bottomrule
% \begin{tabular}[c]{@{}c@{}}\textbf{0.913} $\pm$ 0.006\\ (+0.049)\end{tabular} \\ \hline
% \end{tabular}
% }
% \end{table}

\textbf{Encoders}: \label{sec:pipeline} The significant domain gap between the source and target domains necessitates the extraction of separate and independent features from each domain before processing with a shared enhancement network ($\mathbb{N}$). Hence, we use a source encoder ($\mathcal{E}_S$) and a target encoder ($\mathcal{E}_T$). We first pack the input raw sensor arrays into a four-channel vector (for Bayer arrays from Sony, Nikon, and Canon cameras) and subtract the black level (reference voltage). Then, the packed array is multiplied by the exposure ratio factor and passed as input to the respective domain encoder. It should be noted that the exposure ratio factors need not be the same between the source and the target domain (See Table \ref{tab:datasets}). For the encoder network, we use three convolutional layers with $\{16,32,64\}$ filters and 3$\times$3 kernel size. 

\textbf{Enhancement Network}: The source and target domain encoder features are passed separately to a shared common enhancement network, $\mathbb{N}$. By having a common enhancement network, the large pool of source data helps to improve the enhancement quality of $\mathbb{N}$, while the few target samples ensure that the output is in the target domain. We use U-Net architecture for the enhancement network. Further, the network has a pixel shuffle layer to convert 12-channel prediction to 16-bit three channel sRGB output. The objective of $\mathbb{N}$ is to enhance, denoise, perform other ISP operations (AWB, color manipulation, etc.), and finally demosaicking to generate an sRGB output. $\mathbb{N}$ generates enhanced output $\widetilde{\mathbf{T}}$ for the target domain data as, $\widetilde{\mathbf{T}} = \mathbb{N}\big(\mathcal{E}_T(\mathbf{T}) \big)$. Similarly, $\widetilde{\mathbf{S}}$ for the source domain as, $\widetilde{\mathbf{S}} = \mathbb{N}\big(\mathcal{E}_S(\mathbf{S}) \big)$.
%\footnote{Please refer to the supplementary material for detailed network definition.}

\textbf{Losses}: For the target domain, we compute the $\ell_1$ loss between the prediction ($\widetilde{\mathbf{T}}$) and the ground truth ($\widehat{\mathbf{T}}$) as, $\mathcal{L}_{target} = \ell_1\big(\widetilde{\mathbf{T}},\widehat{\mathbf{T}} \big)$. The source domain loss consists of two components: cosine similarity loss and SSIM loss. We compute cosine similarity between $\widetilde{\mathbf{S}}$ and $\widehat{\mathbf{S}}$ as, 
$
    \mathcal{L}_{CS}(\widetilde{\mathbf{S}},\widehat{\mathbf{S}})= 1 -  \frac{{\widetilde{\mathbf{S}} \cdotp \widehat{\mathbf{S}}}}{\|\widetilde{\mathbf{S}}\|\times\|\widehat{\mathbf{S}}\|}
$.
% \begin{equation}
%     \mathcal{L}_{CS}(\widetilde{\mathbf{S}},\widehat{\mathbf{S}})= 1 -  \frac{{\widetilde{\mathbf{S}} \cdotp \widehat{\mathbf{S}}}}{\|\widetilde{\mathbf{S}}\|\times\|\widehat{\mathbf{S}}\|}
% \end{equation}
Cosine similarity loss is weak supervision for the source domain and is used instead of $\ell_1$ loss since $N\gg M$, and using a strong supervision loss like $\ell_1$ optimizes for pixel values to train $\mathbb{N}$, making the network predict the output in the source domain even for target domain input. Cosine similarity loss ensures that the prediction and the ground truth are in a similar direction. Hence with $\mathcal{L}_{CS}$, $\mathbb{N}$ can still perform enhancement while predicting in target domain even for source domain input. Further, when trained with Sony as source and 4-shot Nikon as target (Table \ref{tab:ablation}) with $L_1$ loss for the source, we obtain only 27.14dB PSNR for target domain validation, whereas using $\mathcal{L}_{CS}$ loss for source achieves 30.30dB PSNR.

% \begin{figure*}[t]
%     \centering
%     \includegraphics[width=\linewidth]{3_BMVC/Images/Canon-Results-1.pdf}
%     \caption{Qualitative comparison with methods tested on Canon target images. (b)HDRCNN and (c) Unprocess are trained on Sony source and fine-tuned on 6-Canon target images, LSID with (d) 6-Canon target images and (e) full ($k$=44) Canon training dataset, (f) Proposed few-shot domain adaptation approach with 6-Canon target images and 161 Sony source images.}
%     \label{fig:canon_eg2}
% \end{figure*}

\begin{figure*}[t]
\centering
\subfigure{\includegraphics[width=\linewidth]{Images/Nikon-Results-1.pdf}}\\ \vspace{-1.65\baselineskip}
\subfigure{\includegraphics[width=\linewidth]{Images/Canon-Results-1.pdf}}
\caption{Qualitative comparison with methods tested on Nikon (top row) and Canon (bottom row) target images. (a) Input after multiplying by exposure factor, results from (b) HDRCNN and (c) Unprocess methods are after training on full Sony source and fine-tuning on $k$-shot target images, LSID with (d) $k$-shot target images and (e) full target training dataset ($k$=53 for Nikon and $k$=44 for Canon), (f) Proposed few-shot domain adaptation method with 161 Sony source images and 4-shot Nikon (top row) and 6-Canon (bottom row) target images.}
\label{fig:nikon_eg1}
\label{fig:canon_eg2}
\end{figure*}

% \begin{table}[t]
% \centering
% \caption{Quantitative comparison of Sony as source and Canon as target dataset. The improvement of proposed method over only $k$ shot trained model is shown in brackets. The LSID model trained with full Canon dataset ($k$=44) attains 32.32dB PSNR and 0.899 SSIM.}
% \label{tab:canon}
% \scalebox{0.73}{
% \begin{tabular}{@{}lccc|ccc@{}}
% \hline
%  & \multicolumn{3}{c|}{PSNR} & \multicolumn{3}{c}{SSIM} \\ \hline
% $k$ ($\rightarrow$) & 1 & 3 & 6 & 1 & 3 & 6 \\ \hline
% \begin{tabular}[c]{@{}l@{}}LSID\\ (only $k$ target)\end{tabular} & 21.54 $\pm$ 2.89 & 26.9 $\pm$ 2.37 & 29.36 $\pm$ 0.763 & 0.588 $\pm$ 0.182 & 0.785 $\pm$ 0.0051 & 0.829 $\pm$ 0.0073 \\ \hline
% % \begin{tabular}[c]{@{}l@{}}Proposed\\ ($k$ target + source)\end{tabular} & \begin{tabular}[c]{@{}c@{}}\textbf{24.29} $\pm$ 3.16\\ (+2.75)\end{tabular} & \begin{tabular}[c]{@{}c@{}}\textbf{28.78} $\pm$ 3.54\\ (+1.8)\end{tabular} & \begin{tabular}[c]{@{}c@{}}\textbf{33.22} $\pm$ 0.45\\ (+3.86)\end{tabular} & \begin{tabular}[c]{@{}c@{}}\textbf{0.623} $\pm$ 0.0074\\ (+0.035)\end{tabular} & \begin{tabular}[c]{@{}c@{}}\textbf{0.841} $\pm$ 0.0335\\ (+0.056)\end{tabular} & \begin{tabular}[c]{@{}c@{}}\textbf{0.896} $\pm$ 0.015\\ (+0.067)\end{tabular} \\ \midrule
% % \begin{tabular}[c]{@{}l@{}}LSID\\ (full target, $k$ = 45)\end{tabular} & \multicolumn{3}{c|}{32.32} & \multicolumn{3}{c}{0.899} \\ \bottomrule
% \begin{tabular}[c]{@{}l@{}}Proposed\\ ($k$ target + source)\end{tabular} & \begin{tabular}[c]{@{}c@{}}\textbf{24.29} $\pm$ 3.16\\ (+2.75)\end{tabular} & \begin{tabular}[c]{@{}c@{}}\textbf{28.78} $\pm$ 3.54\\ (+1.8)\end{tabular} & \begin{tabular}[c]{@{}c@{}}\textbf{33.22} $\pm$ 0.45\\ (+3.86)\end{tabular} & \begin{tabular}[c]{@{}c@{}}\textbf{0.623} $\pm$ 0.0074\\ (+0.035)\end{tabular} & \begin{tabular}[c]{@{}c@{}}\textbf{0.841} $\pm$ 0.0335\\ (+0.056)\end{tabular} & \begin{tabular}[c]{@{}c@{}}\textbf{0.896} $\pm$ 0.015\\ (+0.067)\end{tabular} \\ \hline
% \end{tabular}
% }
% \end{table}

From experiments (in section \ref{sec:exp}), we find better enhancement (in terms of PSNR) using the structural similarity index measure (SSIM) \cite{wang2004image} to compute perceived degradation and preserve the spatial structure in the source output with respect to the ground truth. We do not use SSIM directly on the 16-bit data as that causes the source data to heavily influence the domain adaptation since the source dataset is much larger. Hence, we apply SSIM in JPEG compressed 8-bit domain, where the structural domain difference is less. Since type-casting the 16-bit data to 8-bit will still possess domain-specific details, we train a 16-to-8-bit U-net model ($\mathcal{D}$ in Fig. \ref{fig:prop_overview}) to convert the output from 16-bit to post-processed 8-bit representation. 

The $\mathcal{D}$ network is trained to perform the following non-linear operations: White balancing, Gamma correction, Quantization, and JPEG compression. Even after JPEG compression, the prediction may have traces of source domain specific color information. Further, the SSIM loss is a strong pixel-wise supervision, and in order to avoid the source domain from heavily influencing $\mathbb{N}$, we compute SSIM loss only in grayscale space, not in RGB color space. Also, it follows the intuition that the structure and edge information of a scene will remain the same across images captured with different cameras, while the color space representation may vary. We find that without SSIM loss for the source, we obtain 29.38dB PSNR on target domain validation, whereas using SSIM loss achieves 30.30dB PSNR (Table \ref{tab:ablation}). For computing the SSIM loss, the ground truth ($\widehat{\mathbf{S}}$) is also converted offline to post-processed 8-bit data ($\widehat{\mathbf{S}}_{PP}$) using the rawpy post process function. Hence, the loss is obtained by computing SSIM loss between $\mathcal{D}(\widetilde{\mathbf{S}})$ and $\widehat{\mathbf{S}}_{PP}$,
$
    \mathcal{L}_{SSIM} = 1 - SSIM\Big(\mathcal{D}(\widetilde{\mathbf{S}}), \widehat{\mathbf{S}}_{PP}\Big)
$.
% \begin{equation}
%     \mathcal{L}_{SSIM} = 1 - SSIM\Big(\mathcal{D}(\widetilde{\mathbf{S}}), \widehat{\mathbf{S}}_{PP}\Big)
% \end{equation}
In Fig. \ref{fig:prop_overview}, the top branch guided by the deep red arrows shows the entire source camera training pipeline. It should be noted that $\mathcal{D}$ is used only to compute the loss but not in inference. Finally, we use the sum of cosine similarity loss ($\mathcal{L}_{CS}$) as well as the SSIM loss calculated in the 8-bit domain as the total loss for the source camera pipeline: $\mathcal{L}_{source} = \mathcal{L}_{CS} + \mathcal{L}_{SSIM}$. The total loss is the sum of target and source domain losses: $\mathcal{L}_{total}=\mathcal{L}_{target}+\mathcal{L}_{source}$.
% Source and target models are trained jointly. An
% epoch consists of 161 batches, with one source patch and
% one target domain patch per batch. Both source and target
% patches (of size 512 512, lines 478-480) are cropped at a
% random location from a randomly chosen source and target
% image.

% For the proposed method, we use the respective short-exposure raw sensor data as input to the source ($\mathbf{S}$) and target ($\mathbf{T}$) encoder networks. We first pack the input raw sensor arrays into a four-channel vector (for Bayer arrays from Sony, Nikon, and Canon cameras), subtract the black level (reference voltage), and multiply the input with the exposure ratio. There is one input for the source camera encoder ($\mathcal{E}_S$) and one for the target camera encoder ($\mathcal{E}_T$) in every training step. We pass the output from these camera-specific encoders through a shared $\mathbb{N}$ network to allow the model to learn both camera-specific and camera invariant properties. The output of the $\mathbb{N}$ network is a 12-channel image with half the spatial resolution. % , comprising of a U-net \cite{ronneberger2015u} followed by three CNN layers, 

% \begin{figure}[t]
%     \centering
%     \label{fig:prop_overview}
%     \includegraphics[page=1, clip, trim=0.6cm 16.75cm 7.5cm 0.15cm, scale=0.9]{3_BMVC/Images/block_diagram.pdf}
%     \caption{Overview of our few-shot domain adaptation model. }
% \end{figure}


% \subsection{Source and Target camera pipeline}\label{sec:pipeline}
% The source and target pipelines are trained jointly in an end-to-end manner. An epoch consists of 161 batches, with one source domain patch and one target domain patch per batch 
%given as input to the model. Both source and target patches are $512\times512$ random crops. % are cropped at a random location from a random source and target domain image at each train step.

% \textbf{Source camera pipeline.}
% \label{subsec:source}
% The packed raw input arrays from the source domain are passed to the source encoder, which learns camera-specific parameters to obtain an intermediate representation. We find visible denoising performance from experiments with the encoder heads, and subsequent analysis suggests effective learning of the camera's non-uniform noise model in extreme low-light conditions. The output from the source encoder is passed through the shared $\mathbb{N}$ network to obtain the source output feature maps, $\widetilde{\mathbf{S}}$. Bayer conversion with a sub-pixel layer \cite{shi2016real} unpacks the 12-channel data into a full resolution sRGB image.

% We compute the standard Cosine Similarity loss ($\mathcal{L}_{CS}$) between the source domain output ($\widetilde{\mathbf{S}}$) and the corresponding source domain long-exposure ground-truth image ($\widehat{\mathbf{S}}$),
% % \begin{equation}
% % \mathcal{L}_{CS}(\widetilde{\mathbf{S}},\widehat{\mathbf{S}})= 1 -  \frac{{\widetilde{\mathbf{S}} \cdotp \widehat{\mathbf{S}}}}{\bf \text{max}( \sqrt{({\bf \tilde{S}})^2} \cdotp \sqrt{({\bf \widehat{S}})^2})}
% % \end{equation}
% \begin{align}
%     \widetilde{\mathbf{S}} &= \mathbb{N}\big(\mathcal{E}_S(\mathbf{S}) \big), &    \mathcal{L}_{CS}(\widetilde{\mathbf{S}},\widehat{\mathbf{S}})&= 1 -  \frac{{\widetilde{\mathbf{S}} \cdotp \widehat{\mathbf{S}}}}{\|\widetilde{\mathbf{S}}\|\times\|\widehat{\mathbf{S}}\|}
% \end{align}
% % \begin{equation}
% % \mathcal{L}_{CS}(\widetilde{\mathbf{S}},\widehat{\mathbf{S}})= 1 -  \frac{{\widetilde{\mathbf{S}} \cdotp \widehat{\mathbf{S}}}}{\|\widetilde{\mathbf{S}}\|\times\|\widehat{\mathbf{S}}\|}
% % \end{equation}
  
% Cosine similarity loss is a weak supervision for the source domain and is used instead of $\ell_1$ loss since $N\gg M$, and using a strong supervision loss like $\ell_1$ optimizes for pixel values to train $\mathbb{N}$, making the network predict the output in the source domain even for target domain input. Cosine similarity loss ensures that the prediction and the ground truth are in a similar direction. Hence with $\mathcal{L}_{CS}$, $\mathbb{N}$ can still perform enhancement while predicting in target domain even for source domain input. Further, when trained with Sony as source and 4-shot Nikon as target (Table \ref{tab:ablation}) with $L_1$ loss for the source, we obtain only 27.14dB PSNR for target domain validation, whereas using $\mathcal{L}_{CS}$ loss for source achieves 30.30dB PSNR.

% From experiments (discussed in section \ref{sec:exp}), we find better enhancement (in terms of PSNR) using the structural similarity index measure (SSIM) \cite{wang2004image} to compute perceived degradation and preserve the spatial structure in the source output with respect to the ground truth. We do not use SSIM directly on the 16-bit data as that causes the source data to heavily influence the domain adaptation since the source dataset is much larger. Thus, we apply SSIM in JPEG compressed 8-bit domain, where the structural domain difference is less. Since type-casting the 16-bit data to 8-bit will still possess domain-specific details, we train a 16-to-8-bit U-net model ($\mathcal{D}$ in Fig. \ref{fig:prop_overview}) to convert the output from 16-bit to post-processed 8-bit representation. We find that without SSIM loss for the source, we obtain 29.38dB PSNR on target domain validation, whereas using SSIM loss achieves 30.30dB PSNR (Table \ref{tab:ablation}).

% % (we discuss the converter in section \ref{sec:ablation})

% For computing the SSIM loss, the ground truth ($\widehat{\mathbf{S}}$) is also converted offline to post-processed 8-bit data ($\widehat{\mathbf{S}}_{PP}$) using the rawpy post process function. Hence, the loss is obtained by computing SSIM loss between $\mathcal{D}(\widetilde{\mathbf{S}})$ and $\widehat{\mathbf{S}}_{PP}$.
% \begin{equation}
%     \mathcal{L}_{SSIM} = 1 - SSIM\Big(\mathcal{D}(\widetilde{\mathbf{S}}), \widehat{\mathbf{S}}_{PP}\Big)
% \end{equation}
% In Fig. \ref{fig:prop_overview}, the top branch guided by the deep red arrows shows the entire source camera training pipeline. Finally, we use the sum of cosine similarity loss ($\mathcal{L}_{CS}$) as well as the SSIM loss calculated in the 8-bit domain as the total loss for the source camera pipeline. 
% \begin{equation}
%     \mathcal{L}_{source} = \mathcal{L}_{CS} + \mathcal{L}_{SSIM}
% \end{equation}


% % \begin{figure*}[t]
% %     \centering
% %     \includegraphics[width=17.4cm, height=4.75cm]{ICCV/figures/Nikon-Results-1.pdf}
% %     \caption{Qualitative comparison between different methods tested on images from the Nikon dataset (target). The models are trained on (b) only 4 Nikon images, (c) full Nikon training dataset, and (d) 4 Nikon images and 161 Sony images with our proposed approach.}
% %     \label{fig:nikon_eg1}
% % \end{figure*}

% \textbf{Target camera pipeline.} 
% \label{subsec:target}
% We pass the packed target input raw data to the target encoder ($\mathcal{E}_T$). We then pass the encoded feature maps through the $\mathbb{N}$ network and then through the Bayer converter and calculate the target pipeline's loss in the 16-bit space. From several experiments with various loss functions, we have found the best image enhancement for the target pipeline is achieved with the $\ell_{1}$ loss between the predicted ($\widetilde{\mathbf{T}}$) and ground truth ($\widehat{\mathbf{T}}$),
% \begin{align}
%     \widetilde{\mathbf{T}} &= \mathbb{N}\big(\mathcal{E}_T(\mathbf{T}) \big), &    \mathcal{L}_{target} &= \ell_1\big(\widetilde{\mathbf{T}},\widehat{\mathbf{T}} \big)
% \end{align}

\section{Experiment}
\label{sec:experiment}
In this section, we demonstrate that~\salad{} outperforms other baselines in shape \emph{generation} (Section~\ref{sec:shape_generation}) and enables intuitive \emph{manipulation}, such as part completion (Section~\ref{sec:shape_completion}) and part mixing and refinement (Section~\ref{sec:part_mixing}), where the combination of part-level representation and diffusion models is essential.
Lastly, we also demonstrate that~\salad{} outperforms other baselines in text-guided shape generation (Section~\ref{sec:text_conditional_generation}) and can leverage part-level representation for text-guided part completion (Section~\ref{sec:text_driven_manipulation}).

\subsection{Shape Generation}
\label{sec:shape_generation}
\paragraph{Evaluation Setup.}
For evaluation and comparison, we follow the settings of Hui~\etal~\cite{Hui:2022NeuralWavelet}. We use \emph{airplane} and \emph{chair} classes from the ShapeNet~\cite{Chang:2015Shapenet} dataset and the train-test split from Chen~\etal~\cite{Chen:2019ImNet}. The model is trained for each class. At inference time, we sample \num{2000} shapes for each class, and measure three evaluation metrics~\cite{Achlioptas:2018LatentGAN,Lopez-paz:20171nna} to assess quality and diversity of the generated shapes:
Coverage (COV), Minimum Matching Distance (MMD), and 1-Nearest Neighbor Accuracy (1-NNA).
We compare \salad{} with existing 3D generative models~\cite{Chen:2019ImNet,Kleineberg:2020VoxelGAN,Luo:2021DPM,Hertz:2022Spaghetti,Hui:2022NeuralWavelet}.%



\vspace{-15pt}
\paragraph{Results.} 
The quantitative and qualitative results, including ablation studies, are summarized in Table~\ref{tbl:quantitative_comparison_of_shape_generation} and Figure~\ref{fig:shape_generation_qualitative_results}.
For more results, refer to the \textbf{supplementary material}.
We reproduced the results of SPAGHETTI~\cite{Hertz:2022Spaghetti} and Neural Wavelet~\cite{Hui:2022NeuralWavelet} using the official code,
and the other quantitative results are directly borrowed from Hui~\etal~\cite{Hui:2022NeuralWavelet}, marked with ``$*$'' in Table~\ref{tbl:quantitative_comparison_of_shape_generation}.
(We also display the results of SPAGHETTI~\cite{Hertz:2022Spaghetti} and Neural Wavelet~\cite{Hui:2022NeuralWavelet} reported by Hui~\etal~\cite{Hui:2022NeuralWavelet}~\cite{Hui:2022NeuralWavelet} in the gray-colored rows. Note that SPAGHETTI results are similar, while there is a gap in the Neural Wavelet results.)
To ease qualitative comparisons in Figure~\ref{fig:shape_generation_qualitative_results},
we retrieve the generated shapes using the same query ground truth shape and compare them.

As shown in Table~\ref{tbl:quantitative_comparison_of_shape_generation},~\salad{} achieves SotA results or is on par with the baselines. In particular, we outperform Neural Wavelet~\cite{Hui:2022NeuralWavelet}, which is a SotA diffusion-based 3D generative model, on 1-NNA by a large margin: \num{65.04} vs. \num{57.82} for \emph{chair} CD, and \num{75.77} vs. \num{73.92} for \emph{airplane} CD (lower is better).

Qualitatively,~\salad{} produces clean high-resolution meshes with fine details as shown in Figure~\ref{fig:shape_generation_qualitative_results}. When comparing ``Diffusion of $\B{z}$'' (in Section~\ref{sec:method}) with SPAGHETTI~\cite{Hertz:2022Spaghetti}, we demonstrate that our simple latent diffusion already produces much better quality shapes than sampling $\B{z}$ from the unit Gaussian distribution as SPAGHETTI does. 
``Diffusion of $\{\B{p}_i\}_{i=1}^N$'' uses Transformer~\cite{Vaswani:2017Attention} instead of simple MLPs and outperforms ``Diffusion of $\B{z}$'', clearly showing how our Transformer-based architecture is the key to learning the distribution of high-dimensional latents represented as a set.

When comparing our final model~\salad{} with ``Diffusion of $\{\B{p}_i\}_{i=1}^N$'',~\salad{} outperforms ``Diffusion of $\{\B{p}_i\}_{i=1}^N$'' by a large margin across all metrics. It shows that our cascaded diffusion training is crucial to improve shape generation quality.




\begin{table*}[t!]
\centering
\newcolumntype{Y}{>{\centering\arraybackslash}X}
\caption{\textbf{Quantitative comparison of shape generation.} The numbers directly from Hui~\etal\cite{Hui:2022NeuralWavelet} are marked with *.
MMD-CD scores and MMD-EMD scores are scaled by $10^3$ and $10^2$, respectively. The best results are highlighted without considering the gray-colored rows. 
The ablation study results are presented in rows 8-9.}
\footnotesize
{
\setlength{\tabcolsep}{0.2em}
\renewcommand{\arraystretch}{1.0}
\definecolor{LightCyan}{rgb}{0.88,1,1}
\definecolor{Gray}{gray}{0.85}
\begin{tabularx}{\linewidth}{>{\centering}m{0.5cm} |>{\centering}m{3.5cm}| Y Y Y Y Y Y | Y Y Y Y Y Y }
  \toprule
  \multirow{3}{*}{Id}       &
  \multirow{3}{*}{Method} & \multicolumn{6}{c|}{Chair} & \multicolumn{6}{c}{Airplane} \\
   &              & \multicolumn{2}{c}{COV $\uparrow$} & \multicolumn{2}{c}{MMD $\downarrow$} & \multicolumn{2}{c|}{1-NNA $\downarrow$}  & \multicolumn{2}{c}{COV $\uparrow$} & \multicolumn{2}{c}{MMD $\downarrow$} & \multicolumn{2}{c}{1-NNA $\downarrow$} \\
   &                      &   CD   &   EMD   &   CD   &   EMD   &   CD   &   EMD   &   CD   &   EMD   &   CD   &   EMD   &   CD   &   EMD \\
  \midrule
  1 & IM-NET$*$~\cite{Chen:2019ImNet}    & \textbf{56.49}  & 54.50   & 11.79  & 14.52   & 61.98  & 63.45   & 61.55  & 62.79   & \textbf{3.320}  & 8.371   & 76.21  & 76.08 \\
  2 & Voxel-GAN$*$~\cite{Kleineberg:2020VoxelGAN}       & 43.95  & 39.45   & 15.18  & 17.32   & 80.27  & 81.16   & 38.44  & 39.18   & 5.937  & 11.69   & 93.14  & 92.77 \\
  3 & DPM$*$~\cite{Luo:2021DPM}      & 51.47  & \textbf{55.97}   & 12.79  & 16.12   & 61.76  & 63.72   & 60.19  & 62.30   & 3.543  & 9.519   & 74.60  & 72.31 \\
  \rowcolor{Gray}
  4 & SPAGHETTI$*$~\cite{Hertz:2022Spaghetti} & 49.19  & 51.92   & 14.90  & 15.90   & 70.72  & 68.95   & 58.34  & 58.38   & 4.062  & 8.887   & 78.24  & 77.01 \\
  \rowcolor{Gray}
  5 & Neural Wavelet$*$~\cite{Hui:2022NeuralWavelet}  & 58.19  & 55.46   & 11.70  & 14.31   & 61.47  & 61.62   & 64.78  & 64.40   & 3.230  & 7.756   & 71.69  & 66.74 \\
  6 & SPAGHETTI & 49.48 &	50.22 &	14.7  &	15.85 &	72.34 &	69.46 &	56.86 &	58.83 &	4.260 &	8.930 &	79.36 &	78.86 \\
  7 & Neural Wavelet   & 49.63 &	50.15 &	12.12 &	14.25 &	65.04 &	62.87 & 60.94 &	59.09 &	3.528 &	\textbf{7.964} & 75.77 &	72.93 \\
  \midrule
  8 & Diff. of $\B{z}$  & 49.71  &	48.75 &	11.71 &	\textbf{14.12} &	62.72 &	61.25 & 54.88 &	59.33 &	3.877 &	8.958 &	82.20 &	80.35\\
  9 & Diff. of $\{\B{p}_i\}_{i=1}^N$ & 50.96	&51.40	&13.57	&15.41	&66.19	&67.04& 58.59	& 61.80	& 4.264	& 9.230	& 78.80	& 76.14\\  
  \midrule 
  10 & \salad{} (Ours)                  & 56.42	& 55.16	& \textbf{11.69}	& 14.29	& \textbf{57.82}	&\textbf{ 58.41}	& \textbf{63.16}	& \textbf{65.39}	& 3.636	& 8.238	& \textbf{73.92}	& \textbf{71.08} \\
  \bottomrule
\end{tabularx}
}
\vspace{-0.5\baselineskip}
\label{tbl:quantitative_comparison_of_shape_generation}
\end{table*}

\def\arraystretch{0.0}
\begin{figure*}
\centering
{
\scriptsize
\setlength{\tabcolsep}{0em}
\renewcommand\tabularxcolumn[1]{m{#1}}
\newcolumntype{Y}{>{\centering\arraybackslash}X}
\begin{tabularx}{0.85\textwidth}{YYYYYYYYYY}
\centering
DPM~\cite{Luo:2021DPM} & PVD~\cite{Zhou:2021PVD} & LION~\cite{Zeng:2022LION} & Voxel-GAN~\cite{Kleineberg:2020VoxelGAN} & \makecell{Neural \\Wavelet\cite{Hui:2022NeuralWavelet}} & \makecell{SPAGHETTI\\\cite{Hertz:2022Spaghetti}} & \makecell{Diff. of\\$\B{z}$} & \makecell{Diff. of\\$\{\B{p}_i\}_{i=1}^N$} & Gaussians & \makecell{\salad{}\\(Ours)} \\
\multicolumn{10}{c}{\includegraphics[width=0.85\textwidth]{figures/fig1_0309.png}} \\
\bottomrule
\end{tabularx}
}
\caption{\textbf{Qualitative comparison of the shape generation.} Given a query ground truth shape, we retrieve the closest generated shape by measuring EMD in each method. \salad{} produces highly detailed 3D shapes compared to the baselines.}
\label{fig:shape_generation_qualitative_results}
\vspace{-0.5\baselineskip}
\end{figure*}


\subsection{Part Completion}
\label{sec:shape_completion}


\begin{figure*}
\centering
{
\scriptsize
\setlength{\tabcolsep}{0em}
\renewcommand\tabularxcolumn[1]{m{#1}}
\newcolumntype{W}{>{\centering\arraybackslash}m{0.094444\textwidth}}
\newcolumntype{Z}{>{\centering\arraybackslash}m{0.188888\textwidth}}
\begin{tabularx}{0.85\textwidth}{WWWZZZ}
GT & Bounding Box & Gaussians & ShapeFormer~\cite{Yan:2022ShapeFormer} &Neural Wavelet~\cite{Hui:2022NeuralWavelet}& {\salad} (Ours) \\ 
\multicolumn{6}{c}{\includegraphics[width=0.85\textwidth]{figures/fig2_0309.png}} 
\end{tabularx}
}
\caption{\textbf{Qualitative comparison of the part completion.} We examine \salad{} and other baselines in part completion after ablating semantic parts or regions, highlighted in red in columns 2 and 3. \salad{} produces realistic completions for missing parts. The baselines fail to preserve observed parts or introduce noticeable seams at bounding box boundaries.
}
\label{fig:shape_completion}
\vspace{-0.5\baselineskip}
\end{figure*}



Here, we describe how~\salad{}, which was trained in an \emph{unconditional} setup, can be employed to part completion. We compare the results against the most recent diffusion model, Neural Wavelet~\cite{Hui:2022NeuralWavelet} and the SotA of shape completion, ShapeFormer~\cite{Yan:2022ShapeFormer}.

\sisetup{group-separator={,}}

\vspace{-10pt}
\paragraph{Experiment Setup.}
For completion using diffusion models, we run \emph{guided} reverse process proposed by Meng~\etal~\cite{Meng:2022SDEdit}.
Specifically, given the input data $\B{x}\in \mathbb{R}^d$ and a mask of the region to be reconstructed $m \in [0, 1]^d$, each step of the reverse process of the diffusion is performed as follows:
\begin{equation}
 \begin{aligned}
 \B{x}^{(t-1)}_{\text{unmasked}}&\sim\mathcal{N}(\sqrt{\bar\alpha^{(t)}}\B{x}^{(0)},(1-\bar\alpha^{(t)})\B{I}) \\
 \B{x}^{(t-1)}_{\text{masked}}&\sim\mathcal{N}(\boldsymbol{\mu}_\theta(\B{x}^{(t)},t),\beta^{(t)}\B{I}) \\
 \B{x}^{(t-1)}&=m\odot\B{x}^{(t-1)}_{\text{unmasked}}+(1-m)\odot\B{x}^{(t-1)}_{\text{masked}}.
 \end{aligned}
\end{equation}
Unlike previous methods such as ShapeFormer~\cite{Yan:2022ShapeFormer}, this approach guarantees to preserve the unmasked region.
In our experiments, we randomly remove and regenerate a semantic part of \emph{chairs} and \emph{airplanes}. While we can simply select ($\B{e}_i$, $\B{s}_i$) pairs of parts we want to remove in \salad{}, in feature-voxel representation like Neural Wavelet~\cite{Hui:2022NeuralWavelet}, it is not trivial to specify the regions that would include the completed part. This limits their generation output to only occupy the masked voxels, while a larger mask could interfere with or even break unwanted parts leading to seams in the final output. 
For the guided reverse process of Neural Wavelet~\cite{Hui:2022NeuralWavelet} in our experiments, we use the axis-aligned bounding box of a part as a mask and transform the mask to the wavelet domain. Refer to the \textbf{supplementary material} for more details on mask construction.

We first randomly choose 100 shapes from our training set. Then, for all methods, we randomly select a semantic part from each shape and generate five variations.
For quantitative comparisons, we report the reconstruction loss, MMD and FPD (Fréchet PointNet Distance)~\cite{Shu:2019Treegan} indicating the quality and diversity of completions. Note that we measure MMD \emph{from} completions \emph{to} groundtruth shapes to quantify the proximity of the completed shapes to the groundtruth shapes.
We use the official pre-trained models for ShapeFormer~\cite{Yan:2022ShapeFormer} and Neural Wavelet~\cite{Hui:2022NeuralWavelet}. We also report the results from Neural Wavelet trained by ourselves.




\vspace{-\baselineskip}
\paragraph{Results.}
The quantitative results and qualitative results are summarized in Table~\ref{tbl:part_regeneration} and Figure~\ref{fig:shape_completion}, respectively. 
For more results, refer to the \textbf{supplementary material}.
As shown in Table~\ref{tbl:part_regeneration},~\salad{}, trained solely for \emph{unconditional} shape generation, outperforms the baselines in most of the metrics by large margins, especially in FPD which is the metric of how plausible the shapes are. 


The qualitative results presented in Figure~\ref{fig:shape_completion} further manifests the advantages of employing a part-level 3D representation in \salad{}. In row 1 of Figure~\ref{fig:shape_completion}, ShapeFormer~\cite{Yan:2022ShapeFormer} introduces noticeable artifacts at the back of the chair that lies outside the binary mask (column 2).
In contrast, \salad{} completes the seat seamlessly while preserving the other parts, benefiting from the spatial correspondence between the binary mask and the shape representation. 
Even with such spatial correspondences, the limitation of specifying regions instead of parts persists in Neural Wavelet~\cite{Hui:2022NeuralWavelet}. In particular, the row 2 of Figure~\ref{fig:shape_completion} shows visible seams at the bounding box boundary while \salad{} generates the missing part consistent with the surrounding parts. 



































\begin{table*}[t!]
\centering
\newcolumntype{Y}{>{\centering\arraybackslash}X}
\caption{\textbf{Quantitative comparison of part completion.} The metrics based on CD and EMD are scaled by $10^3$ and $10^2$, respectively. The result from the pre-trained Neural Wavelet is marked with *.  
}
{
\footnotesize
\setlength{\tabcolsep}{0.2em}
\renewcommand{\arraystretch}{1.0}
\begin{tabularx}{\linewidth}{>{\centering}m{3.4cm}| Y Y Y Y Y | Y Y Y Y Y}
  \toprule
  \multirow{3}{*}{Method} & \multicolumn{5}{c|}{Chair} & \multicolumn{5}{c}{Airplane} \\
                          & \multicolumn{2}{c}{\emph{reverse}-MMD $\downarrow$} & \multicolumn{2}{c}{Reconstruction $\downarrow$} & \multirow{2}{*}{\makecell{\\FPD $\downarrow$}}  & \multicolumn{2}{c}{\emph{reverse}-MMD $\downarrow$} & \multicolumn{2}{c}{Reconstruction $\downarrow$} & \multirow{2}{*}{\makecell{\\FPD $\downarrow$}} \\  
                          &   CD   &   EMD                       &   CD   &   EMD                                  &                       & CD   &   EMD                       &   CD   &   EMD   & \\
  \midrule
  ShapeFormer~\cite{Yan:2022ShapeFormer} &  32.83 & 22.8 & 55.05 & 25.49 & 83.56 & 5.43 & 10.87 & 10.83 & 11.81 & 79.18 \\
  Neural Wavelet$*$ ~\cite{Hui:2022NeuralWavelet}  & 13.46 & 15.65 & 8.72 & 12.92 & 18.83 & 3.81 & 9.07 & 3.84 & 8.85 & 31.38 \\
  Neural Wavelet & \textbf{11.87} & 15.07 & 8.93 & 12.44 & 18.78 & 3.56 & 8.79 & 3.90 & 9.02 & 36.17 \\
  \midrule
  \salad{} (Ours)                    & 12.1 & \textbf{14.56} & \textbf{5.45} & \textbf{9.22} & \textbf{16.75} & \textbf{3.55} & \textbf{8.68} & \textbf{2.12} & \textbf{6.53} & \textbf{29.44} \\
  \bottomrule
\end{tabularx}
}
\vspace{-0.5\baselineskip}
\label{tbl:part_regeneration}
\end{table*}


\subsection{Part Mixing and Refinement}
\label{sec:part_mixing}

\begin{figure}
\centering
\includegraphics[width=0.8\linewidth]{figures/fig_part_mixing.pdf}
\caption{\textbf{Qualitative results of Part mixing and refinement.} \salad{} improves quality of part mixing outputs.}
\label{fig:part_mixing}
\vspace{-\baselineskip}
\end{figure}




While Hertz~\etal~\cite{Hertz:2022Spaghetti} demonstrates creating new shapes by combining parts from existing shapes, naively mixing part representations is prone to produce failure cases as illustrated in Figure~\ref{fig:part_mixing} and Figure~\ref{fig:teaser}. Cracks or discontinuities at joint regions are one type of failure case as shown in row 3 of Figure~\ref{fig:part_mixing} and (b) of Figure~\ref{fig:teaser}. Another type of failures is the dissonance between combined parts that results in undesired distortions or the vanishing of parts. 
\salad{} can remedy this issue by refining both the extrinsic and intrinsic vectors through the guided reverse process. Refer to the \textbf{supplementary material} for more qualitative results and \textbf{quantitative comparisons}.













\subsection{Text-Guided Shape Generation}
\label{sec:text_conditional_generation}
We further demonstrate \salad{} can perform \emph{conditional} generation, specifically generating 3D shapes given an input text. To condition a text to the model, we concatenate a language feature and an input of AdaLN, $\gamma(t)$, and optionally $\mathcal{E}(\B{e}_i)$.
We experiment with the text and shape pair dataset from ShapeGlot~\cite{Achlioptas:2019Shapeglot} and compare the generation quality of our text-conditioned model with the one by AutoSDF~\cite{Mittal:2022Autosdf}, which is the SotA text-to-shape generative model. 
The train-test split used in AutoSDF is used.
Also, following AutoSDF, we measure the following three metrics for the evaluation: CLIP-Similarity-Score (CLIP-S)~\cite{Radford:2021CLIP}, Neural-Evaluator-Preference (NEP), and Fréchet Point Cloud Distance (FPD)~\cite{Shu:2019Treegan}.

NEP proposed by Mittal~\etal~\cite{Mittal:2022Autosdf} is a preference rate obtained from a neural evaluator. The neural evaluator is pre-trained on a text-conditioned binary classification task where the model distinguishes the target shape corresponding to the input text. 
Since the neural evaluator used in AutoSDF has not publicly been released, we train our neural evaluator based on PartGlot~\cite{Koo:2022Partglot}, a simpler architecture trained only on point clouds without images. More details of the experiment setup is in the \textbf{supplementary material}. 

As shown in Table~\ref{tbl:lang_quantitative_result}, our generated shapes are more preferred by the neural evaluator over the shapes generated by AutoSDF. Also, Figure~\ref{fig:text_generation} and FPD results reflect that ~\salad{} produces more plausible shapes, and our generated shapes conform to given texts more than the shapes of AutoSDF. 


% \begin{table}[h!]
\begin{figure}
\scriptsize
\renewcommand\tabularxcolumn[1]{m{#1}}
\newcolumntype{Y}{>{\centering\arraybackslash}X}
\begin{tabularx}{\linewidth}{>{\centering}m{2.7cm}| Y Y| Y Y}
    % \toprule
    Text & \multicolumn{2}{c|}{AutoSDF~\cite{Mittal:2022Autosdf}} & \multicolumn{2}{c}{\salad{} (Ours)} \\
    % \multicolumn{5}{c}{``\texttt{circular back}'',\,\,  ``\texttt{curved seat}'',\,\,  ``\texttt{long legs}''}\\
    \midrule
    \footnotesize
    \makecell{``\texttt{chair has round}\\\texttt{arms and wheels.}''} &
    \includegraphics[width=0.8\linewidth]{figures/language/16286_var0_autosdf_resized.png} &
    \includegraphics[width=0.8\linewidth]{figures/language/16286_var1_autosdf_resized.png} &
    \includegraphics[width=0.8\linewidth]{figures/language/16286_var0_ours_resized.png} &
    \includegraphics[width=0.8\linewidth]{figures/language/16286_var1_ours_resized.png} \\ 
    \midrule
    \footnotesize
    \makecell{``\texttt{its the one}\\\texttt{with gaps}\\\texttt{in the back.}''} &
    \includegraphics[width=0.8\linewidth]{figures/language/16058_var0_autosdf_resized.png} &
    \includegraphics[width=0.8\linewidth]{figures/language/16058_var1_autosdf_resized.png} &
    \includegraphics[width=0.8\linewidth]{figures/language/16058_var0_ours_resized.png} &
    \includegraphics[width=0.8\linewidth]{figures/language/16058_var1_ours_resized.png} \\
    % \bottomrule
\end{tabularx}
% \caption{\textbf{Qualitative comparison of text-guided generation.} Given the same texts, \salad{} generates more clean and higher quality meshes compared to AutoSDF, and the generated shapes more conform to the given texts.}
\caption{\textbf{Qualitative comparison of text-guided generation.} \salad{} generates high-quality 3D shapes conforming to the input texts compared to AutoSDF~\cite{Mittal:2022Autosdf}.}
\label{fig:text_generation}
% \end{table}
\end{figure}
\def\arrvline{\hfil\kern\arraycolsep\vline\kern-\arraycolsep\hfilneg}
% \vspace{-0.5\baselineskip}
\begin{table}[t!]
\centering
\caption{\textbf{Quantitative comparison of text-guided generation.} Overall, \salad{} achieves better performance than AutoSDF. Specifically, it improves FPD by a large margin.}
{
\scriptsize
\setlength{\tabcolsep}{0.2em}
\renewcommand{\arraystretch}{1.0}
\begin{tabularx}{\linewidth}{>{\centering}m{2.3cm}| Y | Y | Y}
\toprule
Methods & CLIP-S $\uparrow$ & NEP $\uparrow$ & FPD $\downarrow$ \\
\midrule 
AutoSDF~\cite{Mittal:2022Autosdf} & \textbf{30.98} & 38.98          & 31.53 \\ 
\salad{} (Ours)                              & 30.92 & \textbf{42.22} &  \textbf{4.043} \\ 

\bottomrule

\end{tabularx}
}
% \vspace{-0.5\baselineskip}
% \vspace{-1.5\baselineskip}
\label{tbl:lang_quantitative_result}
\end{table}



\vspace{-10pt}
\subsection{Text-Guided Part Completion}
\label{sec:text_driven_manipulation}
We further demonstrate how~\salad{} can be integrated with a text-driven semantic part segmentation network to aid user interactive shape editing. Following PartGlot~\cite{Koo:2022Partglot} architecture, we design \gaussglot{}, a model that uses $\{\B{e}_i\}_{i=1}^N$ as a part representation and predicts semantic part labels of those from texts. More details of \gaussglot{} architecture and training results can be found in the \textbf{supplementary material}. 
Figure~\ref{fig:text_completion} shows examples that the parts of input shapes selected by \gaussglot{} are completed according to given texts by a reverse process of text-conditioned~\salad{} introduced in Section~\ref{sec:text_conditional_generation}. It demonstrates that users can freely manipulate 3D shapes with texts in an end-to-end manner by leveraging~\salad{} with~\gaussglot{}.
% \begin{figure}
% \scriptsize
% \renewcommand\tabularxcolumn[1]{m{#1}}
% \newcolumntype{Y}{>{\centering\arraybackslash}X}
% \begin{tabularx}{\linewidth}{>{\centering}m{2.0cm}| Y Y Y Y}
% Text & Input & \makecell{Input\\Gaus.} & Output & \makecell{Output\\Gaus.} \\ 
%     \makecell{``\texttt{four leg}\\\texttt{and two arm}''} & \multicolumn{4}{c}{\includegraphics[width=0.75\linewidth]{figures/language/5_mesh.png}} \\
%     \midrule 
%     \makecell{``\texttt{has leg}\\\texttt{seat back of}\\\texttt{office chair}''} & \multicolumn{4}{c}{\includegraphics[width=0.75\linewidth]{figures/language/17_mesh.png}}
% \end{tabularx}
% \caption{\textbf{Qualitative results of text-guided shape completion.} Gaussians that are predicted as semantic parts in input texts are highlighted.}
% \label{fig:text_completion}
% % \end{table}
% \end{figure}


\begin{figure}
% \scriptsize
\footnotesize
\renewcommand\tabularxcolumn[1]{m{#1}}
\newcolumntype{Y}{>{\centering\arraybackslash}X}
% \begin{tabularx}{\linewidth}{>{\centering}m{2.0cm}| Y}
\begin{tabularx}{\linewidth}{Y| Y}
    \includegraphics[width=\linewidth]{figures/language/text_guided_0_0309.png} & \includegraphics[width=\linewidth]{figures/language/text_guided_1_0309.png} \\ 
    % \midrule 
    \makecell{``\texttt{four legs and}\\\texttt{two arms.}''} & \makecell{``\texttt{solid back.}''} 
\end{tabularx}
\caption{\textbf{Qualitative results of text-guided part completion.} The part of the left mesh selected by \gaussglot{}, highlighted by red, is completed to fit a given text by a reverse process of text-guided \salad{}.}
\label{fig:text_completion}
% \end{table}
% \vspace{-\baselineskip}
\end{figure}








%\section{Conclusion}
% Future work
% our finding 




%This study is limited by the number of participants and the lengths of the interactions. In addition, the displayed emotions were not the central focus of the conversations. Instead, they were used spontaneously whenever they matched the context leading to a varying experience for each of the participants.
%Still, it could prove a fantastic potential for using the display of emotion to deepen the connection between humans and robots. Nevertheless, it cannot be disregarded that humanizing robots also carries a significant risk. When the distinctions between humans and robots become blurred, it may soon become impossible for some people to tell the two apart. Therefore, the risks and benefits need to be carefully evaluated, and it might also be reasonable to establish internationally binding guidelines to differentiate robots and humans visually.
%For now, it has to be clearly stated, though, that any of the participants experienced no difficulty in telling that they were not, in fact, interacting with a human, indicating that the robot is not that human-like after all 
%Many participants reported a positive reaction to being smiled at. However, the display of other, potentially more negative emotions, like sadness, fear, or anger, has to be evaluated in further study.
%Future research could solidify our result with quantitative data, directly comparing the interaction with robots who do and do not show emotions.
%It can also be concluded that context awareness and the feeling of being understood by the robot were reported as a more significant benefit than the idea of the robot feeling or communicating joy. It encouraged the participants to talk more, and it could be further investigated whether this effect could be achieved through other visual or auditory cues that are not directly related to a human-typical expression of happiness.
%The study could show that the display of emotion by an Android was generally seen as positive and beneficial but also shed light on the fact that there many ethical questions that need to be investigated. 

%%%%%%%%% REFERENCES
{\small
\bibliographystyle{ieee_fullname}
\bibliography{shortstrings,bibliography}
}

\end{document}
