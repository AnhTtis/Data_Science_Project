\documentclass[12pt,showpacs,bibnotes,prl,onecolumn]{revtex4}
%\documentclass[12pt]{article}
%\documentclass[twocolumn,showpacs,twoside,superscriptaddress]{revtex4}
\usepackage{amssymb, amsbsy, amsmath, latexsym, dsfont, array, layout, graphicx,bm}
\usepackage{color,soul}
\usepackage[colorlinks=true, citecolor=red, urlcolor=blue ]{hyperref}
\newcommand{\fig}[1]{Fig.~\ref{#1}}
\newcommand{\tab}[1]{Table~\ref{#1}}
\newcommand{\one}{\mathds{1}}
\newcommand{\ket}[1]{\left|{#1}\right\rangle}
\newcommand{\bra}[1]{\left\langle{#1}\right|}
\newcommand{\braket}[2]{\langle{#1}|{#2}\rangle}
\newcommand{\ketbrad}[1]{\left|{#1}\rangle\!\langle{#1}\right|}
\newcommand{\ketbra}[2]{\left|{#1}\rangle\!\langle{#2}\right|}

\newcommand{\red}{\color[rgb]{0.8,0,0}}
\newcommand{\blue}{\color[rgb]{0,0,0.6}}
\newcommand{\green}{\color[rgb]{0.0,0.7,0.0}}

\newcommand{\tc}[1]{{\color{magenta}{{#1}}}}

% commenting commands
\usepackage[normalem]{ulem}%This enables strike through text using the \sout{} command.
\usepackage{color}
\newcommand{\Tin}[1]{{\color{red} #1}}
\newcommand{\Tout}[1]{{\color{blue} \sout{#1}}}

\begin{document}

\begin{center}
{\bf {Response to the review on the manuscript AE12341 titled ``Production of genuine multimode entanglement in circular waveguides
     with long-range coupling''}}
\end{center}

\noindent Dear Dr. Thomas Pattard,

Thank you very much for sending us the second report on the manuscript AE12341 titled ``Production of genuine multimode entanglement in circular waveguides
     with long-range coupling''. We would first like to thank all the Reviewers for their comments and criticisms that helped us to further improve our work.
% \sout{We have addressed all the comments by  both the Referees.} 
We have incorporated all the suggestions by the Referees, and have made the corresponding changes (marked in blue except the Introduction) in the manuscript. 

Herewith we resubmit the manuscript for your consideration as a regular article in Physical Review A. 
%We hope that after all these changes, the current form of the manuscript is now suitable for publication. 
In the following, we attach a detailed response to the Referees' comments.

\noindent Yours sincerely,\\
Authors


\newpage

\begin{center}
  \textbf{Reply to the Report of the First Referee}
\end{center}

%We thank the Referee for his/her insightful comments on our manuscript, and also for the suggestions to improve the presentation. 
%In the following, we reproduce the Referee’s report verbatim and address the Referee’s comments, providing details of the changes made
%in the revised version.



{\red{\bf Referee's Comment:}} {\it I have reviewed the revised manuscript and would like to recommend its
publication in Physical Review A now, with some suggested improvements. 

The authors have made substantial enhancements to their manuscript,
addressing previous concerns and providing a clearer understanding of
their work. The manuscript now explicitly emphasizes that it is a
quantitative study examining the effects of long-range couplings on
multimode entanglements in waveguide systems. The authors successfully
demonstrated that, despite the seemingly intuitive nature of their
results, quantifying the Genuine Multimode Entanglement (GME) enhances
our understanding of the role of coupling strengths (J and n) in
entanglement production.

Moreover, the inclusion of additional discussions throughout the paper
significantly contributes to the overall readability and comprehension
of the manuscript. This augmentation aids readers in interpreting the
results and grasping the significance of the findings.}


{\blue{\bf Author's Response:}} We thank the Referee for recommending our revised manuscript for publication.


{\red{\bf Referee's Comment:}}  \textit{While significant improvements are made, I would suggest that the
authors focus on enhancing the readability of the introduction
section. Specifically, they should emphasize the quantitative analysis
aspect of their work and avoid unnecessary repetition of the same
ideas.}

{\blue{\bf Author's Response:}} We thank the Referee for this suggestion. We have now revised our introduction to the manuscript by making it more precise.

\newpage

\begin{center}
  \textbf{Reply to the Report of the Second Referee}
\end{center}

{\red{\bf Referee's Comment:}}  \textit{I have read the resubmitted manuscript “Production of genuine multimode entanglement in
circular waveguides with long-range coupling” by T. Anuradha, Ayan Patra, Rivu Gupta, Amit
Rai, Aditi Sen (De) and their response to the comments emerged in the first round of review.
The authors have addressed the concerns raised in the first report and the manuscript has
improved in quality and readability.}

{\blue{\bf Author's Response:}} We thank the Referee for reviewing our manuscript again. We will now respond to his/her queries in detail.

{\red{\bf Referee's Comment:}}  {\it I have a brief comment: in the response to the first Referee’s comment on block entropy the
authors say

The investigation of the behavior of block entanglement entropy is, moreover, carried, out with the help of the Renyi entropy. In contrast, in the analysis of area law, the quantity typically employed is the von-Neumann entropy. As a result, sufficient physical intuition in this case is complicated to present.

While Renyi entropy and von Neumann entropy are different quantities, their qualitative behavior is generally the same, so physical intuition can be gained from the Renyi entropy as well.}

{\blue{\bf Author's Response:}} We thank the Referee for this comment. We agree with the Referee that physical intuition from the trends of the Renyi entropy can also be derived. However, in our work, we made use of the variations of block entanglement entropy to explain the contribution of different bipartitions to the GGM and to illustrate the difference in dynamics between the nearest-neighbor and long-range couplings. We have mentioned in the main text that our analysis is not concerned with the physical intuition behind the system following the area law, in particular it is used to determine the bipartition which contributes to genuine multimode entanglement, specifically in terms of GGM.

{\red{\bf Referee's Comment:}}  {\it However, I still feel that the authors have not really addressed my comment regarding block
entropy. In particular, in what is now Fig. $4b$, the authors still calculate and plot the entropy for
system sizes up to $N/2$. While it is reasonable to think that the behavior of $S(\rho_L)$ is symmetrical
for $L > N/2$, it is interesting/concerning that some of the curves do not have a flat derivative at
$L = N/2$, meaning the entropy would exhibit a cusp point in $L = N/2$. This is something that
should be explained/investigated. Moreover, the authors talk about volume law of the entanglement entropy, but to truly establish that one should perform a scaling with $N$ of the bipartite entropy $S(\rho_{N/2})$. I believe the authors should try to address this point.}

{\blue{\bf Author's Response:}} We thank the Referee for this comment. We wish to point out that in Fig. $4(b)$, we plot the variation of the block entanglement entropy $S(\rho_L)$ for system with all-to-all long-range couplings. Since the entanglement dynamics in the presence of long-range interactions do not follow the area law, $S(\rho_L)$ increases monotonically, leading to a cusp at $L= N/2$. Note that, since we are dealing with a one-dimensional system, obeying the area law would mean a constant value of $S(\rho_L)$ with the variation of \(L\), which we obtain for large block sizes in the nearest-neighbor case. The non-existence of flatness for long-range interactions signals the violation of the area law. We have now added this discussion in Sec. III. C. 1. of the revised manuscript.

We have now also removed the comment on the volume law from the introduction of our manuscript to avoid confusion regarding our results.

{\red{\bf Referee's Comment:}}  {\it With this minor revision addressed, I now think the manuscript is suitable for publication in
Physical Review A}

{\blue{\bf Author's Response:}} We thank the Referee for the suggestions and recommending our revised manuscript for publication. 



\bibliography{bib}
\bibliographystyle{apsrev4-1}

\end{document}