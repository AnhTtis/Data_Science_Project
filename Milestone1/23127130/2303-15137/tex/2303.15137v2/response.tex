\documentclass[12pt,showpacs,bibnotes,prl,onecolumn]{revtex4}
%\documentclass[12pt]{article}
%\documentclass[twocolumn,showpacs,twoside,superscriptaddress]{revtex4}
\usepackage{amssymb, amsbsy, amsmath, latexsym, dsfont, array, layout, graphicx,bm}
\usepackage{color,soul}
\usepackage[colorlinks=true, citecolor=red, urlcolor=blue ]{hyperref}
\newcommand{\fig}[1]{Fig.~\ref{#1}}
\newcommand{\tab}[1]{Table~\ref{#1}}
\newcommand{\one}{\mathds{1}}
\newcommand{\ket}[1]{\left|{#1}\right\rangle}
\newcommand{\bra}[1]{\left\langle{#1}\right|}
\newcommand{\braket}[2]{\langle{#1}|{#2}\rangle}
\newcommand{\ketbrad}[1]{\left|{#1}\rangle\!\langle{#1}\right|}
\newcommand{\ketbra}[2]{\left|{#1}\rangle\!\langle{#2}\right|}

\newcommand{\red}{\color[rgb]{0.8,0,0}}
\newcommand{\blue}{\color[rgb]{0,0,0.6}}
\newcommand{\green}{\color[rgb]{0.0,0.7,0.0}}

\newcommand{\tc}[1]{{\color{magenta}{{#1}}}}

% commenting commands
\usepackage[normalem]{ulem}%This enables strike through text using the \sout{} command.
\usepackage{color}
\newcommand{\Tin}[1]{{\color{red} #1}}
\newcommand{\Tout}[1]{{\color{blue} \sout{#1}}}

\begin{document}

\begin{center}
{\bf {Response to the review on the manuscript AE12341 titled ``Production of genuine multimode entanglement in circular waveguides
     with long-range coupling''}}
\end{center}

\noindent Dear Dr. Gabriele De Chiara,

Thank you very much for sending us the report on the manuscript AE12341 titled ``Production of genuine multimode entanglement in circular waveguides
     with long-range coupling''. We would first like to thank all the Reviewers for their comments and criticisms that helped us to further improve our work.
% \sout{We have addressed all the comments by  both the Referees.} 
As suggested by the Referees, we have now provided additional explanations and interpretations of our results to further improve our work. We have answered all the comments by all the Referees, and have made the corresponding changes (marked in red) in the manuscript. 

Herewith we resubmit the manuscript for your consideration as a regular article in Physical Review A. We hope that after all these changes, the current form of the manuscript is now suitable for publication. In the following, we attach a detailed response to the Referees' comments.

\noindent Yours sincerely,\\
Authors


\newpage

\begin{center}
  \textbf{Reply to the Report of the First Referee}
\end{center}

%We thank the Referee for his/her insightful comments on our manuscript, and also for the suggestions to improve the presentation. 
%In the following, we reproduce the Referee’s report verbatim and address the Referee’s comments, providing details of the changes made
%in the revised version.



{\red{\bf Referee's Comment:}}  \textit{In the present manuscript, the authors primarily discussed the impact
of long-range coupling on the entanglement between interconnected
modes of photonic waveguides. Notably, the Hamiltonian exhibits a
generality that extends its applicability to a broad range of coupled
photonic systems. While the authors conducted an extensive numerical
analysis on this subject, I would like to raise certain reservations
regarding their work's fundamental novelty and significance. Please
find below my specific comments for further consideration and
discussion.
}

{\blue{\bf Author's Response:}} We thank the Referee for reviewing our results. We now proceed to address his/her comments in order.
\\

{\red{\bf Referee's Comment:}}  \textit{In sections II and III, the primary outcome of the analysis reveals that the presence of "long-range interactions" amplifies entanglement between different modes, with the most significant enhancement observed when all-to-all coupling with equal strength is established, in comparison to nearest-neighbor couplings. This outcome is not a surprise, considering the "area law" of entanglement entropy, which states that entropy scales proportionally to the boundary of the region. Consequently, it is only natural to anticipate enhanced entanglement with all-to-all interaction compared to nearest-neighbor interaction. The novelty of this particular discussion segment in the present work remains unclear.}

{\blue{\bf Author's Response:}} We thank the Referee for this comment. We acknowledge and agree with the understanding that the enhancement of entanglement with all-to-all interaction compared to nearest-neighbor interaction aligns with the expected outcome based on the area law of entanglement entropy. As shown in the manuscript, the results obtained here for multimode entanglement cannot be straightforwardly obtained from the laws of block entanglement.  

Further, let us briefly discuss the motivations for this work. 
\begin{enumerate}
    \item In continuous variable systems, a majority of the previous works analyzed whether genuine multimode entanglement creation was successful (qualitative query) \cite{Asjad_PRAl_2021, Barral_PRA_2018}, through the application of the van-Loock Furusawa inequalities \cite{vanLoock_PRA_2003}, for systems with up to five modes \cite{Rai_PRA_2012}. 
%Furthermore, analysis involving long-range interactions has mostly been restricted to discrete systems (c.f. \cite{Lakkaraju_PLA_2021}), although it is known that waveguide systems can support interactions beyond nearest-neighbor couplings \cite{Dreisow_OL_2008}. 
    Our work quantifies the entanglement created and explicitly shows how long-range interactions are beneficial for generating genuine multimode entanglement (GME). 

    \item Such analysis reveals that although for moderate J values, GGM is higher for the long-range model than that of the short-range model, the maximum GGM can be achieved for both short- and long-range models. Hence the results obtained here demonstrates that the enhancement of entanglement  due to all to all interactions is not ubiquitous.  
    %, thereby illustrating that the intuition from the entanglement area law cannot be applied in such systems in a straightforward manner (see also Sec. III. C). 
    Moreover, we provide a method to compute the GME in such systems, which is not restricted with regard to the number of involved modes and is scalable to an arbitrarily high number of modes when all-to-all interactions are implemented.

    \item Further, we introduce disorder in the interaction strength and study the interplay of entanglement and disorder in CV systems, which might help in the production of genuine multimode entanglement through waveguide setups in laboratories.

    \item Our work also provides an idea about the interaction strength required to create a given amount of multimode entanglement GGM. \\
     

\end{enumerate}

We have now included the above discussions in the Introduction of the revised manuscript.

  In Sections II and III, we have presented the results by examining all possible interactions and by computing GGM to demonstrate the aspects of entanglement creation in these systems. We also compute the block entanglement entropy in the system and try to connect it to GME in Sec. II. C. $1$.
  
  {\red{\bf Referee's Comment:}}  \textit{The authors argue that their work is novel since “earlier research
works relating to the continuous variable (CV) multimode entanglement involves the use of bulk optical elements, which are large and inherently sensitive to decoherence resulting in a reduction of entanglement content”. However, since this is mostly a theoretical piece that starts with a general Hamiltonian featuring various "photon swapping" terms, their analysis should be applicable regardless of the experimental setup. So, this claim doesn't quite justify the novelty of their work.}

{\blue{\bf Author's Response:}} We thank the Referee for this comment. Our Hamiltonian would lead to the generation of genuine multimode entanglement in any experimental setup capable of simulating the proposed evolution, even with the help of bulk optical elements. In our study, we utilize evanescently coupled waveguides that are fabricated using the femtosecond laser direct-writing method, as demonstrated in Ref. \cite{szameit_2010}. It should be noted, however, that with an increase in the number of modes, implementation of the protocol through bulk optical elements would increase the scope of decoherence. On the other hand, waveguide setups provide decoherence resistance and interferometric stability \cite{meany_2015,Perets_PRL_2008} even for arbitrary sizes and hence also provide scalability. Furthermore, higher-order couplings have already been engineered in such setups \cite{Dreisow_OL_2008}, thereby making them suitable candidates for the efficient realization of the proposed scheme for entanglement generation. 

We have now incorporated the aforementioned discussion at the beginning of Sec. II of the revised manuscript.
\\ 

{\red{\bf Referee's Comment:}}  \textit{In part V, the calculation of block entropy does not follow any order with respect to J in the case of all-to-all interaction. Insufficient physical intuition is provided within this section, making it unclear what insights one can gain from these findings.}

{\blue{\bf Author's Response:}} We thank the Referee for this comment. In our analysis, we employed the concept of block entanglement entropy to justify the bipartition responsible for the computation of  GGM. Although the calculation of block entropy does not adhere to any specific order in the case of all-to-all interaction, it is worth noting that the interaction strengths, $J$s, exhibit an increasing trend starting from a block of size $L=1$, which corresponds to the contribution from the $2:$rest bipartition. Consequently, the subsequent eigenvalues derived from this bipartition are responsible for GGM.  The investigation of the behavior of block entanglement entropy is, moreover, carried, out with the help of the Renyi entropy. In contrast, in the analysis of area law, the quantity typically employed is the von-Neumann entropy. As a result, sufficient physical intuition in this case is complicated to present. However, the present analysis suffices to explain why the single-mode reduced state contributes to the genuine multimode entanglement and provides some physical justification to our claim that it is enough to investigate the $2:$rest bipartition of any $N$-mode system. 
We presented analytical evidence indicating that a particular bipartition, which is not predetermined by the input, is accountable for GGM. In the case of discrete systems, such studies have been performed \cite{Kumar_PLA_2017, Demianowicz_NJP_2021}, and we provide a similar analysis for CV systems. This is significant when the number of modes is large, since, without the knowledge of the contributing bipartition, analytical calculations would involve finding the eigenvalues of a large number of reduced density matrices, namely ($\binom{N}{1} + \binom{N}{2} + \cdots + \binom{N}{[N/2]}$). Furthermore, with the increase in the dimension of the reduced systems, it becomes analytically intractable to estimate their eigenvalues. The proof that the contribution to the genuine multimode entanglement comes from the single-mode reduced state thus helps in calculating the exact GGM produced in a system containing an arbitrary number of modes. 

Moreover, as correctly pointed out by the Referee, such analysis also reveals that for moderate interaction strength, the block entanglement entropy is almost constant with the block size, $L$, while it steadily increases with $L$ for systems with long-range interactions, thereby indicating the violation of the area law in this system. 

We have now included the analysis on the block entanglement entropy as Subsection $1$. in Sec. III. C where we have added the above discussion.
\newpage

\begin{center}
  \textbf{Reply to the Report of the Second Referee}
\end{center}

{\red{\bf Referee's Comment:}}  \textit{I have read the manuscript “Production of genuine multimode entanglement in circular
waveguides with long-range coupling” by T. Anuradha, Ayan Patra, Rivu Gupta, Amit Rai, and Aditi Sen (De).\\
The authors study genuine multimode entanglement (GME) in a system of circular waveguide modes coupled through interactions with tunable range. To this end, they study the generalized geometric measure (GGM) – i.e. the geometric distance of the state of the system from a genuinely unentangled pure state – as well as the accumulated GGM (AcGGM), i.e. the average GGM over a range of coupling strengths, and the bipartite entanglement entropy (EE).\\
The system is initialized in an input state where the state is a product state of a squeezed coherent state on one mode and the vacuum state on the other modes. The system is then evolved for a time t (which is incorporated into the coupling strength J) and the GME is studied for different sizes of the system and for different types of long-range couplings. The authors find that in general long-range interactions lead to larger GME, as well as initial states with larger squeezing. GGM seems to show a periodic behavior with J, while the AcGGM converges for. The GME also goes to zero in the thermodynamics limit. The authors also study what happens if there is a disorder in the coupling strength and find that the periodic behavior of GGM is smoothed out. Furthermore, they investigate the behavior of entanglement entropy as a function of the partition size for different coupling strengths.\\
The main idea and objectives of the manuscript are clearly enounced in the introduction and conclusion of the paper. The idea that long-range interactions may be more effective in creating entanglement is not new but it has not been tested in waveguides settings. However, the results are not presented in a clear way and the conclusions they suggest are not discussed enough. Moreover, the presentation seems at times to be missing important details or steps in the analysis.\\
I do not recommend publication of the manuscript in Physical Review A in its present form, although I would consider it for publication if improvements are made. Below is a more detailed list of comments and questions for the authors.}

{\blue{\bf Author's Response:}} We thank the Referee for reviewing our manuscript and summarising our results. We will now respond to his/her queries in detail.

{\red{\bf Referee's Comment:}}  \textit{In Fig. (2a) both $\mathcal{G}_{4}^{\text{NN}}$ and $ \mathcal{G}_{4}^{\text{NNN}}$ are plotted, but it is not specified for which value of n, $ \mathcal{G}_{4}^{\text{NNN}}$ is obtained.}

{\blue{\bf Author's Response:}} We thank the Referee for this comment. We would like to clarify that in Fig. $2 (a)$, both $\mathcal{G}_{4}^{\text{NN}}$ and $\mathcal{G}_{4}^{\text{NNN}}$ are plotted with the assumption of $n=1$, as explained in Sec. III.A.2, specifically in point $3$. We agree that we should have mentioned this explicitly in the caption, and we have now added this omission in the caption of Fig. $2 (a)$.
\\

{\red{\bf Referee's Comment:}} The statement in point 3 of Sec. III.A.2 (page 4) makes sense intuitively, but I cannot
reconcile it with Fig. (2b). In other words, for n = 1 there is all-to-all coupling and all modes are equally interacting which intuitively would create the largest GME, but Fig. (2b) offers incomplete proof of this statement.

{\blue{\bf Author's Response:}} We thank the Referee for this comment. In Fig. $2(b)$, we present the plot of $\mathcal{G}_{4}^{\text{NNN}}$ (ordinate) against the variation of $n$ (abscissa) for different values of $J$ ($J = 0.5$, $J = 1.0$, $J = 2.0$, and $J = 4.0$) where the initial squeezing strength is set to $s = 1.0$. Notably, we can observe that the curves exhibit periodic behavior for certain values of $J$, with varying GGM at different $n$ values. By using Fig. $2(b)$, we wanted to establish that regardless of the chosen value of $J$, GGM consistently reaches its maximum at $n = 1$ although different values of $n$ show different trends in GGM. As a consequence, it is concluded that all-to-all interactions with equal strength furnish the best possible production of genuine multimode entanglement.  We have now incorporated this explanation in the manuscript in point $4$ of Sec. III.A.2 and have also modified Fig. $2(b)$ by explicitly highlighting the aforementioned phenomenon to make the explanation more coherent.
\\ 

{\red{\bf Referee's Comment:}} Moreover, there is no comment on the behavior of GGM as a function of J and n in Fig. (2b). The periodic behavior with n is not explained. Moreover, for N=$5$, there is a symmetry $n \iff \frac{1}{n}$ and $J \iff \frac{J}{n}$, since the two NNN sites can be exchanged with the two NN sites when n becomes larger than one. Did the authors explore or use this symmetry?

{\blue{\bf Author's Response:}} We thank the Referee for this valuable comment. The periodicity of GGM with $J$ is apparent from Fig. $2(a)$. Fig. $2(b)$. illustrates that for a fixed number of modes $N$ and a fixed squeezing strength, $s$, of the initial input, the maximum genuine multimode entanglement is created at $n = 1$, regardless of the coupling parameter $J$. 
%We have shown curves corresponding to different $J$ values to demonstrate that the beneficial effect of equal coupling strengths is universal.  Moreover, studying the variation of GGM (at a fixed $J, N$, and $s$) with $n$ helps to demonstrate the fact that the optimum generation of genuine entanglement occurs when nearest-neighbor and long-range couplings have equal strength, i.e. $n = 1$. This is true for all values of $J$ since all the different curves share their maximum at $n = 1$. 
As mentioned in the previous response, Fig. $2(b)$ motivates us to consider all-to-all couplings with equal strength, a configuration that yields a high genuine multimode entanglement content upon evolution.

Our analytical results reveal that all the symplectic eigenvalues of the reduced subsystems are sine and cosine functions of the Hamiltonian parameters, thereby leading to the oscillatory nature of GGM with respect to $n$ and $J$. Moreover, in terms of $n$, the next-nearest-neighbor coupling reads $J_2 = nJ$. For a fixed value of $J$, the NNN coupling parameter is a function of $n$ which furthermore enters into the evolution operator $\exp(-i \hat{H} t) = f(J,n)$. Therefore, the periodic behavior of  GGM with $n$ is again due to the unitary evolution, similar to the oscillatory nature of $\mathcal{G}$ with $J$.  We have now included the explanation below point $3$ of Sec. III.A.2 as a remark in the revised manuscript.

Fig. $2(b)$ illustrates clearly that for $N = 4$, optimal GGM is generated when the nearest-neighbor and long-range couplings have the same strength. We note that for $N = 5$ as well, we can only consider up to next-nearest-neighbor couplings and also, we need to consider only the single-mode and two-mode reduced subsystems to estimate $\mathcal{G}_5^{\text{int}}$ similar to the case for the four-mode system. Therefore, $N = 4$ and $N = 5$ offer qualitatively similar insights, due to which we have set $n = 1$ from the beginning while calculating GGM for the five-mode waveguide case. We have not considered the symmetry $n \iff 1/n$ and $J \iff J/n$ although for the exact GGM analysis, this would highly simplify the calculations. We have now added this discussion at the end of Note $2$ in the revised manuscript.
\\

{\red{\bf Referee's Comment:}} The authors say that GGM is periodic in J with period $\frac{2\pi}{4}$ for N=$4$, with period $\frac{2\pi}{5}$ for N=$5$, with period $\frac{2\pi}{6}$ for N=$6$ and with period $\frac{2\pi}{N}$ for generic N (at n=$1$). It would be nice to see a more detailed proof of this statement. I believe it is true for generic N at n = 1 given the symmetric nature of the Hamiltonian.

{\blue{\bf Author's Response:}} We thank the Referee for this comment. We have indeed observed that, for $N = 4, 5,$ and $6$, GGM exhibits periodicity in $J$ with a period of $\frac{2\pi}{N}$. Upon further analysis, we have determined that GGM is predominantly influenced by a single eigenvalue, specifically that of the $2:$rest bipartition. This eigenvalue is expressed in Eq. $(5)$ of the revised manuscript as
\begin{equation}
    \nu = \frac{\sqrt{\frac{1}{4} f_1(N)-2 \sinh ^2s \left[f_2(N) \cos \left(J N\right)+\cos\left(2 J N\right)\right]+ f_3(N) \cosh 2 s}}{N^2}.
\end{equation}
It can be observed that the eigenvalue and thus GGM is a function of $\cos \left(J N\right)$ and $\cos\left(2 J N\right)$. Therefore, a period in $J$ equalling $\frac{2 \pi}{N}$ causes the eigenvalue and hence GGM to repeat its magnitude. Based on this observation, it becomes evident that GGM possesses a period of $\frac{2 \pi}{N}$ for $N$ number of modes. We have now presented the above explanation at the end of Sec. III. C. 2. of our work.
\\

{\red{\bf Referee's Comment:}} It would be useful to know what kind of analytic calculations led to Eq. (6). Is the problem exactly solvable in that case?

{\blue{\bf Author's Response:}} We thank the Referee for this question.  The problem of calculating GGM for an arbitrary number of modes is exactly solvable when we consider all-to-all interactions with equal interaction strength. In such a scenario, the Hamiltonian matrix becomes highly symmetric, $H_{p,q} = J(1 - \delta_{p,q})$ (where $\delta$ represents the Kronecker delta function), with vanishing diagonal entries and all off-diagonal entries being equal to $J$, thereby simplifying the analytical solution. In order to derive Eq. $(5)$ (in the revised manuscript), we applied the all-to-all interaction assumption to mode number $N$ ranging from $4 ~\text{to} ~10$ and observed a recursion relation for the eigenvalue corresponding to the $2:$rest bipartition which contributes to GGM throughout the entire time evolution. In the computation of the GGM, the symplectic eigenvalues of only the $2:\text{rest}$ bipartition matter. This is justified through numerical simulations. To ensure the validity of the recursion relation, we verified its applicability for $N = 11, \cdots, 15$ modes and subsequently extended it to arbitrary $N$ modes. We have now added this discussion as Note $3$ at the end of Sec. III. C. 2 of our revised manuscript.
\\

{\red{\bf Referee's Comment:}}  In general, can the problem be solved analytically for generic N and n? Since all couplings beyond NN have the same strength the Hamiltonian has a rather symmetric form that may allow exact calculations.

{\blue{\bf Author's Response:}} We thank the Referee for this comment. While it is possible to solve the problem analytically for a general $n$ and $N = 3, 4, 5, 6$, it becomes more challenging when $n \neq 1$ and for large $N$. As mentioned in the previous response, with $n = 1$, the Hamiltonian becomes highly symmetric - $H_{p,q} = J(1 - \delta_{p,q})$ (with $\delta$ being the Kronecker delta function) and hence the GGM can be studied. Let us highlight the problems for the Hamiltonian with $n \neq 1$. In such cases, the Hamiltonian loses the particular symmetry that has been employed to calculate GGM. Moreover, for such values of $n$, the bipartition that contributes to GGM varies for different numbers of total modes, which was observed from our numerical simulations. This additional complexity makes it more difficult to obtain solutions when the eigenvalues of the relevant bipartitions cannot be obtained analytically as the system size increases. Consequently, the still symmetric nature of the Hamiltonian does not necessarily facilitate a straightforward analytical solution when considering generic values of the parameter $n$. The above discussion has now been included as a Remark at the end of Sec. III. C. 2 of the revised manuscript.
\\

{\red{\bf Referee's Comment:}}  The AcGGM does not appear a meaningful observable to me. It is essentially the average over J from 0 to $J_{0}$ of GGM. Since GGM appears to be periodic in J (it is for sure in the n = 1 case), then the AcGGM converges to the average over one period of GGM, hence the behavior reported in Fig. 2c) seems trivial. Moreover, I would physically interpret the AcGGM as the average over time of the multimode entanglement, which would give information on the peak of entanglement, but not on the value of entanglement at a certain time. On the other hand, interpreting the AcGGM as an average over coupling strength is the same as the disorder averaging from Section IV, only with a uniform (instead of Gaussian) distribution of J.

{\blue{\bf Author's Response:}}  We thank the Referee for this important suggestion. We have understood the concerns of the Referee and have omitted the quantity AcGGM in the modified manuscript.
\\


{\red{\bf Referee's Comment:}}  The authors refer to fluctuations in the periodic behavior of GGM, but to me they look more like natural oscillations of GGM due to the unitary evolution of the system. In the absence of noise or dissipation, this seems expected to me.

{\blue{\bf Author's Response:}} We thank the Referee for this comment. We have now omitted the use of the word fluctuations in the revised manuscript and have referred to the variation of the GGM as oscillations.
%While this behavior is expected in the absence of dissipation, the introduction of disorder allows us to exhibit a constructive behavior of noise on entanglement production, contrary to the expectation that noise would inevitably cause correlations to degrade.
\\

{\red{\bf Referee's Comment:}}  Can the results of Fig. 5 be interpreted as the system having a finite amount of GME so that with an increasing number of waveguides, the multimode entanglement has to be shared among a larger number of modes and is thus smaller?

{\blue{\bf Author's Response:}} We thank the Referee for this question.
%In our work, we have primarily focused on the scenario where a given input is applied to a single waveguide. As we introduce more waveguides into the system to create genuine entanglement among several modes, the propagation of photons in these additional waveguides becomes more challenging. Consequently,
We agree with the Referee's argument. The distribution of multimode entanglement among a larger number of modes results in a reduction in the overall extent of entanglement. Hence, it can be noted that this decrease in multimode entanglement is due to a finite amount of GME (multimode entanglement) that needs to be shared among an expanding number of modes, leading to its diminishment. For a large number of involved modes, it can be made to increase by increasing the squeezing strength $s$ of the input state. However, since the experimental application of very high $s$ values is challenging, an arbitrary increase in the number of modes would result in the generation of smaller genuine multimode entanglement. We have included this discussion as SubsubSec. III. C. $3$ in the revised manuscript.
\\

{\red{\bf Referee's Comment:}}  I do not understand the point of the accumulated quenched average GGM. It looks like averaging twice over the disorder.

{\blue{\bf Author's Response:}} We thank the Referee for this important suggestion. We have understood the concerns of the Referee and have not included the accumulated quenched average GGM in the revised manuscript. 
\\

{\red{\bf Referee's Comment:}}  Can the results of Fig. 5a) be thought of as the disorder just smoothing out the oscillations in J, so that when $\sigma$ covers at least one period of oscillation then GGM is more or less independent of  $J_{m}$? Would that mean that for large N s small disorder, $\sigma \sim 1/N$ is enough to erase the dependence from  $J_{m}$?

{\blue{\bf Author's Response:}} We thank the Referee for these very interesting arguments. In Figure $6(a)$ (in the revised manuscript), the presence of disorder in $J$ does indeed lead to smoother behavior in GGM. When the standard deviation $\sigma$ of the disorder covers at least one period of oscillation in $J$, GGM becomes more or less independent of $J_{m}$. This is because GGM in the presence of equal all-to-all coupling is periodic and has a non-zero average value, and thus its quenched average value over an entire period is constant and independent of $J_m$. It is important to note that this independence is only achieved beyond a certain threshold value of $\sigma \sim 1/N$. Thus, for large values of $N$ and small disorder, the relation $\sigma \sim 1/N$ is sufficient to completely eliminate the dependence on $J_{m}$. We have added this discussion in our revised manuscript just before Sec. IV. A
\\

{\red{\bf Referee's Comment:}}  I do not understand the meaning of Fig. 6) and the results on entanglement entropy. How does $S(\rho_{L})$ behave for L varying up to N? Does the linear behavior of Fig. 6b) continue for larger L? Why is the slope non-monotone in J?

{\blue{\bf Author's Response:}}  We thank the Referee for this question. In Fig. $4$ (in the revised manuscript), we investigated the behavior of the entanglement entropy $S(\rho_{L})$ by varying the block sizes, $L$, up to the system size $N/2$. We restrict our analysis to $L \leq N/2$, since
%since we are interested in studying the Renyi-$2$ entropy between different bipartitions. Subsystems of size $L \geq N/2$ would correspond to a similar behavior to those of $L \leq N/2$ which is the length of the remaining system of the bipartition.
 due to the circular arrangement of waveguides, a similar behavior emerges for $L \geq N/2$.

%Regarding the non-monotone slope in $J$, it arises from the interplay between the nearest-neighbor coupling and the long-range couplings. 

We have now added the above two points as a remark at the end of Sec. III. C. 1.

The block entropy of entanglement has been studied with the help of the Renyi entropy. It is studied to provide a justification for the observation that in the presence of all-to-all interactions with equal strength, the single-mode reduced density matrix (more precisely, the $2:\text{rest}$ bipartition) contributes to GGM. As explained in the manuscript, the behavior of another entanglement measure, the Renyi entropy, actually helps us in gauging that our conjecture is true and allows us to motivate our results physically and not merely treat them as a numerical by-product. Moreover, as mentioned in the response to Referee $1$, the behavior of the entanglement entropy also supports the area law in the case of short-range interactions, and the volume law for long-range interactions, as expected.
\bibliography{bib}
\bibliographystyle{apsrev4-1}

\end{document}