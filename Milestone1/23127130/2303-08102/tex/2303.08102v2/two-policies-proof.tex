%\begin{proof}

We will consider two environments $\mu_1$ and $\mu_2$. For environment $\mu_1$ and arm $j$, we choose\footnote{Note that $\sqrt{\theta_1(j)} + \sqrt{\theta_2(j)}$ is always positive by our assumption that each arm is in the support of at least one policy.} 
\[ \mu_1(j) = \frac{1}{2} - \Delta \frac{\sqrt{\theta_1(j)} - \sqrt{\theta_2(j)}}{\sqrt{\theta_1(j)} + \sqrt{\theta_2(j)}},
\]
where $0 \leq \Delta < \frac{1}{2}$ is to be tuned later. Environment $\mu_2$ is defined analogously. Additionally let $\mu_0$ be an environment such that $\mu_0(j) = \frac{1}{2}$ for any arm $j$. Note that $\theta_1$ ($\theta_2$) is the optimal policy in $\mu_1$ ($\mu_2$). Indeed, 
\begin{align*}
    &\sum_{j=1}^K (\theta_2(j) - \theta_1(j)) \mu_1(j) \\
&\quad= \Delta \sum_{j=1}^K  (\theta_1(j) - \theta_2(j)) \frac{\sqrt{\theta_1(j)} - \sqrt{\theta_2(j)}}{\sqrt{\theta_1(j)} + \sqrt{\theta_2(j)}} \\
&\quad= \Delta \sum_{j=1}^K  (\sqrt{\theta_1(j)} - \sqrt{\theta_2(j)})^2 = 2 \Delta H^2(\theta_1,\theta_2).
\end{align*}
Therefore,  
    \begin{align*}
        &R_T(\mu_1)\\
        &\quad= 2 \Delta H^2(\theta_1,\theta_2) (T - N_{\mu_1}(\theta; T)) \\
        &\quad\geq 2 \Delta H^2(\theta_1,\theta_2) \left(T - N_{\mu_0}(\theta; T) - T \sqrt{\frac{1}{2}D(P_{\mu_0},P_{\mu_1})}\right),
    \end{align*}
    where the inequality follows by using that $N_{\mu_1}(\theta; T) - N_{\mu_0}(\theta; T) \leq T \tv(P_{\mu_0},P_{\mu_1})$ followed by an application of Pinsker's inequality. Note that for $\Delta \leq 1/4$ and $c=8\log(4/3)$, we have that 
    \begin{align*}
    &d(\mu_0(j),\mu_1(j)) \\
    &\quad= d\left(\frac{1}{2},\frac{1}{2}-\Delta \frac{\sqrt{\theta_1(j)} - \sqrt{\theta_2(j)}}{\sqrt{\theta_1(j)} + \sqrt{\theta_2(j)}} \right)\\
    &\quad= -\frac{1}{2} \log\left(1-4\Delta^2\left(\frac{\sqrt{\theta_1(j)}- \sqrt{\theta_2(j)}}{\sqrt{\theta_1(j)} + \sqrt{\theta_2(j)}}\right)^2\right)\\
    &\quad\leq c \Delta^2 \left(\frac{\sqrt{\theta_1(j)}- \sqrt{\theta_2(j)}}{\sqrt{\theta_1(j)} + \sqrt{\theta_2(j)}}\right)^2.
    \end{align*}
    While on the other hand
    \begin{align*}
        &\sum_{j=1}^K \theta_1(j) \left(\frac{\sqrt{\theta_1(j)}- \sqrt{\theta_2(j)}}{\sqrt{\theta_1(j)} + \sqrt{\theta_2(j)}}\right)^2 \\
        &\quad= \sum_{j=1}^K  (\sqrt{\theta_1(j)}- \sqrt{\theta_2(j)})^2 \frac{\theta_1(j)}{(\sqrt{\theta_1(j)} + \sqrt{\theta_2(j)})^2}\\
        &\quad\leq \sum_{j=1}^K  (\sqrt{\theta_1(j)}- \sqrt{\theta_2(j)})^2\\
        &\quad= 2H^2(\theta_1,\theta_2).
    \end{align*}
    With the analogous inequalities for $\mu_2$ and $\theta_2$ we get that 
    \begin{align*}
         D(P_{\mu_0}, P_{\mu_1}) 
         &= 2 c \Delta^2 H^2(\theta_1,\theta_2) (N_{\mu_1}(\theta'; T)+N_{\mu_2}(\theta'; T)) \\
         &= 2 c \Delta^2 H^2(\theta_1,\theta_2)T.
    \end{align*}
    Putting everything together, we get that
        \begin{align*}
        &\sup_{\mu} R_T(\mu)\\ 
        &\quad\geq \frac{1}{2}( R_T(\mu_1)+R_T(\mu_2))\\
        &\quad\geq2 \Delta H^2(\theta_1,\theta_2) \left(T - \frac{T}{2} - T \sqrt{c \Delta^2 H^2(\theta_1,\theta_2)T}\right)\\
        &\quad= \Delta H^2(\theta_1,\theta_2) T \left(1 - 2\Delta \sqrt{c H^2(\theta_1,\theta_2)T}\right),
    \end{align*}
    The theorem then follows by setting $\Delta = \frac{1}{4\sqrt{c H^2(\theta_1,\theta_2)T}}$ and verifying that the stated condition on $T$ ensures that $\Delta$ is less than $1/4$.
%\end{proof}