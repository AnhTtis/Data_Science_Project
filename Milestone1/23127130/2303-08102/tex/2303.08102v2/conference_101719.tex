\documentclass[conference]{IEEEtran}
%\IEEEoverridecommandlockouts
% The preceding line is only needed to identify funding in the first footnote. If that is unneeded, please comment it out.
\usepackage{hyperref}
\usepackage{cite}
\usepackage{amsmath,amssymb,amsfonts}
%\usepackage{algorithmic}
\usepackage{amsthm}
\usepackage{mathtools}
\usepackage{thmtools}
%\usepackage{bm}
\usepackage{algorithm} 
\usepackage[noend]{algpseudocode}
\newtheorem{theorem}{Theorem}
\newtheorem{lemma}[theorem]{Lemma}
\newtheorem{cor}[theorem]{Corollary}
\theoremstyle{definition}
\newtheorem{exmp}{Example}
\usepackage{graphicx}
\usepackage{textcomp}
\usepackage{xcolor}
\usepackage{macros2}
\def\BibTeX{{\rm B\kern-.05em{\sc i\kern-.025em b}\kern-.08em
    T\kern-.1667em\lower.7ex\hbox{E}\kern-.125emX}}
\hypersetup{
    colorlinks,
    linkcolor={black},
    citecolor={black},
    urlcolor={black}
}

\begin{document}

\title{
 Information-Theoretic Regret Bounds \\ for Bandits with Fixed Expert Advice
}

\author{\IEEEauthorblockN{Khaled Eldowa\IEEEauthorrefmark{1}, Nicolò Cesa-Bianchi\IEEEauthorrefmark{1}, Alberto Maria Metelli\IEEEauthorrefmark{2}, Marcello Restelli\IEEEauthorrefmark{2}}
\IEEEauthorblockA{\IEEEauthorrefmark{1}Università degli Studi di Milano, Milan, Italy}
\IEEEauthorblockA{\IEEEauthorrefmark{2}Politecnico di Milano, Milan, Italy}
\IEEEauthorblockA{\{khaled.eldowa, nicolo.cesa-bianchi\}@unimi.it, \{albertomaria.metelli, marcello.restelli\}@polimi.it}
}






\maketitle

\begin{abstract}
We investigate the problem of bandits with expert advice when the experts are fixed and known distributions over the actions. Improving on previous analyses, we show that the regret in this setting is controlled by information-theoretic quantities that measure the similarity between experts. In some natural special cases, this allows us to obtain the first regret bound for EXP4 that
can get arbitrarily close to zero if the experts are similar enough.
While for a different algorithm, we provide another bound that describes the similarity between the experts in terms of the KL-divergence, and we show that this bound can be smaller than the one of EXP4 in some cases.  
Additionally, we provide lower bounds for certain classes of experts showing that the algorithms we analyzed are nearly optimal in some cases.
\end{abstract}

%\begin{IEEEkeywords}
%\end{IEEEkeywords}

\section{Introduction}
Bandits with expert advice (see, e.g., \cite{lattimore2020bandit}) is a well-known variant of the non-stochastic bandits problem in which, at the beginning of each round, $N$ experts each make a recommendation to the learner in the form of a distribution over the $K$ available actions. The algorithm EXP4 \cite{adversarial} solves this problem with a regret against the best expert bounded by $\sqrt{2TK\log N}$, where $T$ is the horizon. %%
When $N \gg K$, this bound shows the ability of EXP4 to leverage the structure of the problem, as opposed to running a bandit algorithm over the $N$ experts achieving a bound of $\sqrt{TN}$. An almost matching lower bound of order $\sqrt{TK\log N/\log K}$ was proved in \cite{expertslowerbound} (for deterministic experts).  %%
In this work, we study a variant of bandits with expert advice in which the distributions recommended by the experts are \emph{fixed} and \emph{known}.
In the following, we will use the term \emph{policies} to denote these fixed experts.
Our goal is to determine the best possible dependence of the regret on the structure of the policy set $\Theta$ irrespective of the assigned sequence of losses. 

This problem is closely related to linear bandits \cite{linear-bandits} (with finite decision sets), where 
the structure of the decision/policy set can be provably leveraged. %%%
Our problem can also be viewed as a non-stochastic version of \textit{bandits with mediator feedback} \cite{metelli2021policy}, where the learner's access to actions is mediated by the fixed policy set.
When losses are stochastic rather than being adversarial,
regret bounds were proved in \cite{metelli2021policy} that scale with the largest pair-wise exponentiated 2-Rényi divergence\footnote{This divergence is related to the chi-squared divergence. Note that pairwise, these divergences can be infinite in non-trivial cases, see Example~\ref{example-flower}.} between the policies in the context of policy-based reinforcement learning. Comparable bounds were also proved in \cite{chi-squared} in the setting of contextual bandits.
In our setting, where losses are adversarially generated, the best known bound is $\sqrt{2T\mathcal{S}(\Theta)\log N}$ from \cite{McMahanS09}, where $\mathcal{S}(\Theta) \le \min\{K,N\}$ %%
is a notion describing the similarity between policies, see Section \ref{sec:S-interp} for its definition and a new information-theoretic interpretation.
Since $\mathcal{S}(\Theta) \ge 1$ for all $\Theta$, this bound cannot get arbitrarily small no matter how similar the policies are, and becomes vacuous when the policies are identical. %%%%%%%%

Our first contribution (Theorem~\ref{upper:alt-losses:thm}) is a new regret bound for EXP4 of the form $\sqrt{2T\mathcal{S}^*(\Theta)\log N}$, where $\mathcal{S}^*(\Theta)$ is a new index of similarity between policies that is never larger than $\mathcal{S}(\Theta)$ and reduces to twice the total variation distance when $N = |\Theta| = 2$. In particular, we show that $\mathcal{S}^*(\Theta)$ can indeed become arbitrarily small, depending on the policy set.
Note that such guarantees cannot be obtained only as a consequence of the reduced range of the losses caused by the similarity of the policies, see \cite{lattimore-lowerbounds}.
Additionally, we show in Theorem~\ref{cvxhull-thm} an algorithm whose regret is bounded by $\sqrt{2 T K D^*(\Theta)}$, where $D^*(\Theta)$ is a notion of the ``width'' of $\Theta$ in terms of the KL-divergence that reduces to the information radius when $\Theta$ is symmetric. This bound is never worse than the EXP4 bound $\sqrt{2TK\log N}$. Moreover, we construct sets $\Theta$ where $K D^*(\Theta) < \mathcal{S}^*(\Theta) \log N$. 
Finally, we prove lower bounds for a number of policy set structures and contrast them with the upper bounds we derived. 
We illustrate, in particular, some examples where the bounds are nearly matching.

\section{Problem Formulation}
We consider a non-stochastic multi-armed bandits problem with a finite action set $\mathcal{A} = [K]$ containing $K$ actions, and a (fixed) policy set $\Theta \in \Delta_{K-1}$ consisting of $N$ probability distributions over the actions. Here $\Delta_{K-1}$ denotes the probability simplex in $\mathbb{R}^K$ and, for a policy $\theta \in \Theta$ and $j \in [K]$, $\theta(j)$ is the probability with which policy $\theta$ picks action $j$. We additionally assume that each arm is in the support of at least one policy.
With a time horizon of $T$ rounds, an instance of the problem is characterized by an unknown sequence of loss vectors $(\ell_t)_{t=1}^T$, where $\ell_t(j) \in [0,1]$, for $j \in [K]$, denotes the loss assigned to action $j$ at round $t \in [T]$. A decision maker interacts with the environment as follows: at each round $t$, the decision maker selects a policy $\theta_t \in \Theta$; an action $A_t \in [K]$ is then sampled from $\theta_t$; the decision maker subsequently suffers the loss $\ell_t(A_t)$ and observes the pair $\big(A_t,\ell_t(A_t)\big)$. With a slight abuse of notation we denote by $\ell_t(\theta)$ the expected loss (at round $t$) given that policy $\theta$ was selected; that is, $\ell_t(\theta) = \sum_{j=1}^K \theta(j) \ell_t(j)$. The objective is to minimize the regret, which we define as follows:
\begin{equation*}
    R_T = \E \sum_{t=1}^T \ell_t(\theta_t) - \sum_{t=1}^T \ell_t(\theta^*),
\end{equation*}
where $\theta^* \in \argmin_{\theta \in \Theta} \sum_{t=1}^T \ell_t(\theta)$ and the expectation is over the internal randomization of the player.  

\section{EXP4 Regret Analysis}
It is possible to show\footnote{See also Theorem 18.3 in \cite{lattimore2020bandit}.} that the EXP4 algorithm with a suitable tuning of the learning rate satisfies a similar regret bound to the one proven for the algorithm developed in \cite{McMahanS09}. In our setting, this bound is $\sqrt{2T\mathcal{S}(\Theta)\log N}$, where
$
    \mathcal{S}(\Theta) = \sum_{j=1}^K \max_{\theta \in \Theta} \theta(j)
$
is a notion of similarity for the policy set. It is easy to see that $\mathcal{S}(\Theta) \leq \min\{K,N\}$, and thus, it can be interpreted as the effective ``number'' of policies.

\begin{algorithm} [t]
    \caption{EXP4 With Fixed Expert Advice}
    \label{exp4}
    \begin{algorithmic}[1]
        \State \textbf{Input:} $K$, $\Theta$, $\eta$
        \State \textbf{Initialize:} $\forall \theta \in \Theta$, $\hat{\ell}_0(\theta)=0$
        \For{$t=1,\dotsc,T$}
            \State Draw $\theta_t \sim P_t$, where $P_{t}(\theta) = \frac{\exp(-\eta\sum_{s=0}^{t-1}\hat{\ell}_s(\theta))}{\sum_{\xi \in \Theta}  \exp(-\eta\sum_{s=0}^{t-1}\hat{\ell}_s(\xi))}$
            \State Draw $A_t \sim \theta_t$, and observe loss $\ell_t (A_t)$
            \State $\forall \theta \in \Theta$, set $\hat{\ell}_t(\theta) = \frac{\theta(A_t)}{\sum_{\xi \in \Theta} P_{t} (\xi) \xi(A_t)} \ell_t(A_t)$
        \EndFor
    \end{algorithmic}
\end{algorithm}

\subsection{An Information-Theoretic Interpretation of \texorpdfstring{$\mathcal{S}(\Theta)$}{S(Theta)}} \label{sec:S-interp}
An alternative characterization of the policy set similarity can be derived by observing that $\mathcal{S}(\Theta) = 1 + \mathcal{TV}(\Theta)$, where for any ordering of the policies $(\theta_i)_{i=1}^N$ we define:
\begin{equation}\label{tv}
    \mathcal{TV}(\Theta) = \sum_{i=2}^N \sum_{j:\theta_i(j) > \theta_{[i-1]}(j)} (\theta_i(j)-\theta_{[i-1]}(j)),
\end{equation}
where $\theta_{[i-1]}(j) = \max_{\theta \in \{\theta_1,\cdots,\theta_{i-1}\}}\theta(j)$, and the second sum is the upper variation of the signed measure $\theta_i-\theta_{[i-1]}$.
$\mathcal{TV}(\Theta)$ can be seen as a generalization of the total variation distance to describe the overall divergence of the policy set. Indeed, it is easy to see that when $\Theta=\{\theta_1, \theta_2\}$,  $\mathcal{TV}(\Theta)$ reduces to the total variation between the two policies:
\[
\tv(\theta_1,\theta_2) = \sum_{j: \theta_2(j) > \theta_1(j)} (\theta_2(j)-\theta_1(j))~.
\]
Moreover, an upper bound on $\mathcal{TV}(\Theta)$ can be derived by noting that for any $\tau \in \Delta_{K-1}$ we have:
\begin{equation*}
    \mathcal{TV}(\Theta) = \mathcal{S}(\Theta) - 1 = \sum_{\theta \in \Theta} \sum_{j \in B(\theta)} (\theta(j) - \tau(j)),
\end{equation*}
where $(B(\theta))_{\theta \in \Theta}$ is any partition of $[K]$ such that for  $j \in B(\theta)$ we have $\theta \in \argmax_{\theta'\in \Theta} \theta'(j)$. Then, it follows that:
\begin{equation}\label{tv-bound}
         \mathcal{TV}(\Theta) \leq \min_{\tau \in \Delta_{K-1}} \sum_{\theta \in \Theta} \tv(\theta,\tau).
\end{equation}
It should be noted that this bound can be loose. For instance, if the policy set is partitioned into clusters of similar policies, then it is not hard to see that the right-hand side of \eqref{tv-bound} will be wasteful compared to \eqref{tv}. It is also noteworthy that quantities related to $\mathcal{S}(\Theta)$ are used when studying the minimax risk in statistical estimation problems. In particular, Theorem II.1 in \cite{minimax-risk} can be used to derive upper bounds similar to \eqref{tv-bound} in terms of any $f$-divergence, though often in implicit form.

\subsection{An Improved Bound}
Nevertheless, the stated regret bound scales with $\mathcal{S}(\Theta)$, not with $\mathcal{TV}(\Theta)$. The main limitation is that 
$\mathcal{S}(\Theta) \geq 1$, no matter how close $\mathcal{TV}(\Theta)$ is to zero. Thus, the bound $\sqrt{2T\mathcal{S}(\Theta)\log N}$ is never smaller than $\sqrt{2T\log N}$ regardless of the structure. One might wonder if this is necessary.
The following theorem provides the first regret bound for EXP4 that can get arbitrarily close to zero if the policies are similar enough. Our bound depends on the key quantity:
\[
        \mathcal{S}^*(\Theta) = \sum_{j=1}^K \left(\max_{\theta \in \Theta} \theta(j) - \min_{\theta' \in \Theta} \theta'(j)\right),
\]
that is easily seen to satisfy
$\mathcal{TV}(\Theta) \le \mathcal{S}^*(\Theta) \le \mathcal{S}(\Theta)$.
\begin{theorem} \label{upper:alt-losses:thm}
    Algorithm \ref{exp4} run with $\eta = \sqrt{ \frac{2\log N}{T \mathcal{S}^*(\Theta)}}$ satisfies
$
        R_T \leq \sqrt{2 T \mathcal{S}^*(\Theta) \log N}.
$
\end{theorem}
%
\begin{proof}
    For a policy $\theta$, we define a modified version of the loss at time $t$ as $\zeta_t (\theta) = \sum_{j=1}^K (\theta(j) - q(j)) \ell_t(j)$, where $q(j) = \min_{\theta' \in \Theta} \theta'(j)$. Notice that for any two policies $\theta_i$ and $\theta_j$, $\zeta_t(\theta_i) - \zeta_t(\theta_j) = \ell_t(\theta_i) - \ell_t(\theta_j)$. An estimate for this modified loss can be constructed in a standard manner:  $\hat{\zeta}_t (\theta) = (\theta(A_t)-q(A_t)) \frac{\ell_t (A_t)}{\psi_t (A_t)}$, where $\psi_t(j)=\sum_{\theta' \in \Theta} P_{t} (\theta') \theta'(A_t)$ is the probability of playing arm $j$ at time $t$ given $P_t$. Let $\E_{t-1}$ be the expectation conditioned on the events up to round $t-1$, it can be verified\footnote{See the proof of Theorem 4.2 in \cite{bubeck2012regret} for similar arguments.} that (for any $\theta$) $\mathbb{E}_{t-1}\hat{\zeta}_t (\theta)
= \zeta_t (\theta)$ and  $\mathbb{E}_{t-1}\zeta_t (\theta_t)
= \mathbb{E}_{t-1}\sum_{\theta \in \Theta}  P_t(\theta) \hat{\zeta}_t (\theta)$. Hence, we have that
\begin{align}
    R_T &= \E\left[ \sum\nolimits_t \zeta_t (\theta_t) -  \sum\nolimits_t \zeta_t (\theta^*) \right] \nonumber\\ \label{upper:alt-losses:regret}
        &= \E\left[ \sum\nolimits_t \sum\nolimits_{\theta \in \Theta}  P_t(\theta) \hat{\zeta}_t (\theta) -  \sum\nolimits_t \hat{\zeta}_t (\theta^*)\right].
\end{align}
Notice that:
    $
        \sum_{s=1}^{t-1} \hat{\ell}_s (\theta) = \sum_{s=1}^{t-1} \hat{\zeta}_s (\theta) + \sum_{s=1}^{t-1} q(A_s) \frac{\ell_s (A_s)}{\psi_s (A_s)}.
    $
    And since the second sum does not depend on $\theta$, we can rewrite\footnote{Like $\hat{\ell}_0(\theta)$, we initialize $\hat{\zeta}_0(\theta)$ as zero.} $P_t$ as follows:
    \begin{equation*}
        P_{t}(\theta) = \frac{\exp(-\eta\sum_{s=0}^{t-1} \hat{\zeta}_s (\theta))}{\sum_{\theta' \in \Theta}  \exp(-\eta\sum_{s=0}^{t-1} \hat{\zeta}_s (\theta'))}.
    \end{equation*}
    One can then bound (\ref{upper:alt-losses:regret}) using a standard manipulation (see, for example, Theorem 1.5 in \cite{hazan}) to yield that:\footnote{This requires that $\hat{\zeta}_t(\theta)$ is non-negative, which is indeed the case. 
    %Moreover, the $1/2$ factor in front of the second moment term is achieved by bounding $e^{-\eta \hat{\zeta}_t(\theta)}$ by $1-\eta \hat{\zeta}_t(\theta)+\frac{\eta^2}{2} \hat{\zeta}^2_t(\theta)$ instead of $1-\eta \hat{\zeta}_t(\theta)+\eta^2 \hat{\zeta}^2_t(\theta)$ as in the proof of Theorem 1.5 in \cite{hazan}.
    }
    \begin{equation} \label{upper:alt-losses:hedge}
        R_T \leq  \frac{\eta}{2} \E \sum\nolimits_t \sum\nolimits_{\theta \in \Theta}  P_t(\theta) \hat{\zeta}^2_t (\theta) + \frac{\log N}{\eta}.
    \end{equation}
    Define $s(j) = \max_{\theta \in \Theta} \theta(j)$, we now bound the second moment term at step $t$:
     \begin{align*}
         &\mathbb{E}_{t-1}\sum\nolimits_{\theta\in\Theta}  P_t(\theta) \hat{\zeta}^2_t (\theta)\\ &= \mathbb{E}_{t-1}\sum\nolimits_{\theta\in\Theta} P_t(\theta) (\theta(A_t)-q(A_t))^2 \frac{\ell^2_t (A_t)}{\psi^2_t (A_t)}  \\
         &\leq \mathbb{E}_{t-1}(s(A_t)-q(A_t)) \frac{\ell^2_t (A_t)}{\psi_t (A_t)} \frac{\sum_{\theta\in\Theta} P_t(\theta) (\theta(A_t)-q(A_t)) }{\sum_{\theta\in\Theta} P_t(\theta) \theta(A_t)}  \\
         &\leq \mathbb{E}_{t-1}(s(A_t)-q(A_t)) \frac{\ell^2_t (A_t)}{\psi_t (A_t)}  \\
         & = \sum\nolimits_j (s(j)-q(j)) \ell^2_t (j) \leq \mathcal{S}^*(\Theta).
     \end{align*}
     Combining this with \eqref{upper:alt-losses:hedge} and the specified choice of $\eta$ concludes the proof.
\end{proof}
%
In general, $\mathcal{S}^*(\Theta)$ is not guaranteed to be strictly smaller $\mathcal{S}(\Theta)$; they can be equal in some cases regardless of how small $\mathcal{TV}(\Theta)$ is. In other cases, however, there can be an improvement. To see this, note that for any $\tau \in \Delta_{K-1}$,
\begin{equation*}
    \mathcal{S}^*(\Theta) = \sum_{\theta \in \Theta} \sum_{j \in B(\theta)} (\theta(j) - \tau(j)) + \sum_{j \in B'(\theta)} (\tau(j)-\theta(j)),
\end{equation*} 
where $(B'(\theta))_{\theta \in \Theta}$ is a partition of $[K]$ such that for every $j \in B'(\theta)$ we have $\theta \in \argmin_{\theta'\in \Theta} \theta'(j)$. Then, analogously to~\eqref{tv-bound}, we have that:
\begin{equation*} 
    \mathcal{S}^*(\Theta) \leq 2\min_{\tau \in \Delta_{K-1}} \sum_{\theta \in \Theta} \tv(\theta,\tau).
\end{equation*}
Like \eqref{tv-bound}, this bound can be loose, but it serves to indicate that if $\min_{\tau \in \Delta_{K-1}} \sum_{\theta \in \Theta} \tv(\theta,\tau)$ is small (it can get arbitrarily so), $\mathcal{S}^*(\Theta)$ is guaranteed to be of at most the same order.

\subsection{Examples}
In the following, we compare the quantities $\mathcal{S}(\Theta)$, $\mathcal{S}^*(\Theta)$, and $\mathcal{TV}(\Theta)$ for a selection of policy set structures. 
\begin{exmp}[Two Policies] \label{example-two}
    As we have seen before, for the two policies case, i.e., $\Theta = \{\theta_1,\theta_2\}$, $\mathcal{S}(\Theta) = 1 + D_{TV}(\theta_1,\theta_2)$. Whereas
$
    \mathcal{S}^*(\Theta) = \sum_{j=1}^K |\theta_1(j) - \theta_2(j)| = 2 D_{TV}(\theta_1,\theta_2).
$
\end{exmp}
The next two examples concern the case in which each policy is a uniform distribution over a support of $M\leq K$ arms. In this scenario, we get that\footnote{Recall that we assume that an arm is in the support of at least one policy.} $\mathcal{S}(\Theta) = \frac{K}{M}$, while $\mathcal{S}^*(\Theta)$ depends on the number of arms common to all policies.

\begin{exmp}[Radially Symmetric Uniform Policies] \label{example-flower}
    Consider a structure where the intersection of the supports of any pair of policies is the same.\footnote{This means that any arm is either in the support of all policies or exclusively in the support of a single one.} 
    Let $V \leq M$ be the number of arms common to all policies, we have that $\mathcal{S}(\Theta) = \frac{N(M-V)+V}{M}$, that is bounded from below by $1$, while $\mathcal{S}^*(\Theta)=N\frac{M-V}{M}$ and $\mathcal{TV}(\Theta)=(N-1)\frac{M-V}{M}$ are not.\footnote{Note that $\frac{M-V}{M}$ is the total variation distance between any two policies.}
    \end{exmp}

\begin{exmp}[Failure of $\mathcal{S}^*$] \label{example-stripes}    
    On the other hand, if $K=M+1$, and the policy set contains all possible $M$-supported uniform policies, then $\mathcal{S}(\Theta) = \frac{M+1}{M}$ which approaches $1$ as $M$ increases, thus $\mathcal{TV}(\Theta)=\frac{1}{M}$ approaches $0$. However, $\mathcal{S}^*(\Theta)$ is always equal to $\mathcal{S}(\Theta)$ since for each arm $j$, $\min_{\theta' \in \Theta} \theta'(j) = 0$. 
\end{exmp}

\begin{exmp}[$\epsilon$-Uniform Policies] \label{example-epsilon}
    If $N=K$ and each policy $\theta$ is associated (one-to-one) with an arm $a_\theta$ so that, for an arm $j$, $\theta(j) = \frac{1-\epsilon}{K} + \epsilon \mathbb{I}\{j=a_\theta\}$, where $0 \leq \epsilon \leq 1$.
    Then, $\mathcal{S}(\Theta) = \epsilon K + 1 - \epsilon$, while $\mathcal{S}^*(\Theta) = \epsilon K$ and $\mathcal{TV}(\Theta) = \epsilon (K-1)$.
\end{exmp}

\section{An Alternative Approach} \label{sec:osmd}
Since we can randomize our policy choice at each round, we can interpret our setting as a bandits problem in which the player has to randomize over the actions choosing a distribution from 
(and also competing with)
the convex hull $\mathrm{co}(\Theta)$ of the available policies. A simple approach, outlined in Algorithm~\ref{osmd}, is to adapt the Online Stochastic Mirror Descent (OSMD) interpretation of EXP3 \cite{adversarial} to our setting. The main distinction is that we need to project onto $\mathrm{co}(\Theta)$ at each step. Denote by $D(P,Q)$ the KL-divergence between distributions $P$ and $Q$, and for $\tau \in \Delta_{K-1}$, define $D(\Theta||\tau)=\max_{\theta\in\Theta} D(\theta, \tau)$. 
The following regret bound for Algorithm~\ref{cvxhull-thm} uses a notion of the ``width'' of $\Theta$ in terms of the KL-divergence defined by:\footnote{Note that the minimum value can only be attained in $\mathrm{co}(\Theta)$; see Theorem 11.6.1 in \cite{cover2006}.}
\[
    D^*(\Theta) =  \min_{\tau\in\Delta_{K-1}} D(\Theta||\tau)~.
\]
\begin{theorem} \label{cvxhull-thm}
Algorithm \ref{osmd} run with
\[
\tau^* \in \argmin_{\tau\in\Delta_{K-1}} D(\Theta||\tau) \quad\text{and}\quad \eta=\sqrt{ \frac{2D^*(\Theta)}{T K}}
\]
satisfies
$
       R_T \leq \sqrt{2TD^*(\Theta)K}
$.
\end{theorem}
\begin{proof}
    Noting  that $\hat{\ell}_t$ is unbiased given $x_t$, we get that
    $
         R_T 
        = \E \sum\nolimits_t \langle x_t - \theta^*, \hat{\ell}_t \rangle.
    $
    This expression is the regret of OMD on the estimated losses with a decision set $\text{co}(\Theta)$ and the negative entropy function $\psi(x) = \sum_{j} x(j) \log x(j)$ as the regularizer. Using Lemma 6.14 in \cite{orabona},\footnote{See also, still in \cite{orabona}, the discussion leading up to Theorem 10.2 concerning the negative entropy regularizer.} we get that:     
    \begin{equation} \label{cvxhull-omd}
        R_T \leq \frac{D(\theta^*, \tau^*)}{\eta} + \frac{\eta}{2} \E \sum\nolimits_t\sum\nolimits_j x_t(j) \hat{\ell}^2_t(j),
    \end{equation}
    Similar to the last step of the proof of Theorem \ref{upper:alt-losses:thm}, we can show that
    $\E \sum_{j} x_t(j) \hat{\ell}^2_t(j) \leq K$. 
    The proof concludes by bounding $D(\theta^*, \tau^*)$ with $D^*(\Theta)$ and plugging in the value of $\eta$.
\end{proof}


    %\begin{equation*} %\label{cvxhull-omd}
    %    R_T \leq \frac{D(\theta^*, \tau)}{\eta} + \frac{\eta}{2} \E \sum_{t=1}^T\sum_{j=1}^K z_t(j) \hat{\ell}^2_t(j),
    %\end{equation*}
    %where $z_t$ lies on the line segment between $x_t$ and $\tilde{x}_{t+1} = \argmin_{x \in \text{dom}(\psi)} \eta \langle x, \hat{\ell}_t \rangle + D(x, x_t)$. It can be shown that $\tilde{x}_{t+1}(j) = x_t(j) \exp(-\eta \hat{\ell}_t(j))$, and since $\hat{\ell}_t(j)$ is non-negative, we get that $\tilde{x}_{t+1}(j) \leq x_t(j)$, and thus $\E \sum_{j=1}^K z_t(j) \hat{\ell}^2_t(j) \leq \E \sum_{j=1}^K x_t(j) \hat{\ell}^2_t(j) \leq K$, where the last inequality follows from the definition of $\hat{\ell}_t(j)$ and the fact that $\ell_t(j)\leq 1$. 
It can be shown that $D^*(\Theta)\leq \log N$. Moreover, if the policy set is symmetric, in the sense that the KL-divergence between any policy and the uniform mixture is the same, then $D^*(\Theta)$ (attained at the uniform mixture) coincides with the Jensen-Shannon divergence~\cite{jensen-shannon} (or the information radius) of the policy set.

\begin{algorithm} [thb]
    \caption{OSMD on the Convex Hull of Policies}
    \label{osmd}
    \begin{algorithmic}[1]
        \State \textbf{Input:} $K$, $\Theta$, $\eta$, $\tau \in \text{co}(\Theta): \tau(j)>0 \: \forall j \in [K]$
        \State \textbf{Initialize:} $x_1=\tau$
        \For{$t=1,\dotsc,T$}
            \State Pick distribution $P_t$ on $\Theta$ such that $\sum_{\theta \in \Theta} P_t(\theta) \theta = x_t$
            \State Draw $\theta_t \sim P_t$, then $A_t \sim \theta_t$, and observe loss $\ell_t (A_t)$
            \State $\forall j \in [K]$, set $\hat{\ell}_t(j) = \frac{\mathbb{I}\{j=A_t\}}{x_t(j)} \ell_t(A_t)$
            \State Update $x_{t+1} = \argmin_{x \in \text{co}(\Theta)} \eta \langle x, \hat{\ell}_t \rangle + D(x, x_t)$
        \EndFor
    \end{algorithmic}
\end{algorithm}
Algorithm~\ref{exp4} can also be seen as an instance of OSMD with the negative entropy regularizer. The main difference is that the decision space is the entire probability simplex over the policies. Hence, the regularization in Algorithm~\ref{exp4} favors exploring uniformly over the policies, whereas it favors exploring uniformly over actions in Algorithm~\ref{osmd}. Analysis-wise (compare in particular \eqref{upper:alt-losses:hedge} and \eqref{cvxhull-omd}), Algorithm~\ref{exp4} takes advantage of the similarity between policies to reduce the variance of their loss estimates (compared to $K$), whereas Algorithm~\ref{osmd} takes advantage of the (possibly) limited size of $\text{co}(\Theta)$ to reduce the bias term (compared to $\log N$).

Consider the $\epsilon$-uniform structure of Example \ref{example-epsilon}.
%$\tau^*$ is the uniform distribution, and thus, 
$D^*(\Theta)$ in this case is given by
\begin{equation*} 
    \frac{K-1}{K} (1-\epsilon) \log(1-\epsilon) + \frac{1+\epsilon(K-1)}{K} \log(1+\epsilon(K-1)).
\end{equation*}
If we now compare $K D^*(\Theta)$ and $\epsilon K \log K$, 
we see that both are equal when $\epsilon\in\{0,1\}$. However, the former is strictly convex for $\epsilon\in (0,1)$, while the latter is linear. Thus, in this case, the bound of Theorem \ref{cvxhull-thm} is better than that of Theorem \ref{upper:alt-losses:thm}.
However, if we consider the structure of Example \ref{example-flower}, we have that
%$\tau^*$ is the uniform mixture of the policies, and 
$D^*(\Theta) = \frac{M-V}{M} \log N$. Thus, the bound of Theorem \ref{cvxhull-thm} becomes $\sqrt{2T\frac{M-V}{M}K\log N}$ which is worse than the $\sqrt{2T\frac{M-V}{M}N\log N}$ bound of Theorem \ref{upper:alt-losses:thm} since $K \geq N$.

\section{Lower Bounds}
In this section, we prove minimax lower bounds for specific classes of policy sets and contrast them with the regret bounds discussed thus far. More precisely, with a fixed policy set $\Theta$, we prove lower bounds on $\inf_\pi \sup_{(\ell_t)_{t=1}^T} R_T$, 
where $\pi$ is the player's strategy. To this end, we will consider a class of stochastic environments, each identified by the vector $\mu \in [0,1]^K$ such that, for $j \in [K]$ and any $t$, $\ell_t(j)$ is drawn from a Bernoulli distribution with mean $\mu(j)$. For any $t \leq T$, let $H_t = (\theta_1, A_1, \ell_1(A_1), \dotsc, \theta_t, A_t, \ell_t(A_t))$ be the interaction history up to round $T$. Each environment $\mu$ (together with strategy $\pi$) induces a probability distribution $P_\mu$ on $H_T$. Define: 
\begin{equation*}
    R_T(\mu) = \max_{\theta^* \in \Theta} \mathbb{E}_\mu  \sum_{t=1}^T \sum_{j=1}^K (\theta_t(j) - \theta^*(j)) \mu(j),
\end{equation*}
where the subscript in $\mathbb{E}_\mu$ emphasizes the dependence on $P_\mu$. 
%In the following, we will use the fact that a lower bound on $\sup_\mu R_T(\mu)$ that is independent of $\pi$ is also a lower bound on $\inf_\pi \sup_{(\ell_t)_{t=1}^T} R_T$. 
In the following, we will prove lower bounds on $\sup_\mu R_T(\mu)$ that hold for any algorithm. 
We will rely on the following lemma, which is an immediate extension of a standard result (see Lemma 15.1 in \cite{lattimore2020bandit}) to our setting.
\begin{lemma} \label{low:kl-decomp}
Fix a strategy $\pi$, a policy set $\Theta$, and a horizon $T$; and let $\mu$ and $\mu'$ be two environments. Then, 
\begin{equation*}
    D(P_\mu, P_{\mu'}) = \sum\nolimits_{\theta \in \Theta} N_\mu(\theta; T) \sum\nolimits_j \theta(j) d(\mu(j),\mu'(j)),
\end{equation*}
where $N_\mu(\theta; T) = \E_\mu\sum_{t=1}^T\mathbb{I}\{\theta_t=\theta\}$, and $d(a,b)$ is the KL-divergence between two Bernoulli distributions with means $a$ and $b$.
\end{lemma} 


\subsection{Radially Symmetric Uniform Policies}
The first lower bound concerns the case discussed in Example \ref{example-flower}. The construction of the lower bound (which is an adaptation of the standard approach in \cite{adversarial}) leverages the fact that the policies are uniform and equidistant.
\begin{theorem} \label{low:flower:thm}
If $\Theta$ conforms to the structure of Example \ref{example-flower} such that $M > V$. Then for any algorithm and $T \geq \frac{N M}{4 \log(4/3)(M-V)}$, there exists a sequence of losses such that
$
    R_T \geq \frac{1}{18} \sqrt{N\frac{M-V}{M} T}.
$
\end{theorem}
Since $\mathcal{S}^*(\Theta)=N\frac{M-V}{M}$, it follows that the bound of Theorem \ref{upper:alt-losses:thm} is optimal in this case, up to a logarithmic factor. 
\begin{proof}
    We denote by $U(\theta) \subseteq [K]$ the support of $\theta$ and  $C = \bigcap_{\theta \in \Theta} U(\theta)$. We consider $N$ environments $\{\mu_{\theta}\}_{\theta \in \Theta}$ such that for $\mu_{\theta}$ and arm $j$, $\mu_{\theta}(j) = \frac{1}{2} - \Delta\mathbb{I}\{j \in U(\theta) \backslash C\}$, where $0 < \Delta < \frac{1}{2}$ is to be tuned later. Let $\mu_0$ be an environment such that $\mu_0(j) = \frac{1}{2}$ for any arm $j$. Note that $\theta$ is the optimal policy in $\mu_{\theta}$, and for $\theta' \in \Theta \backslash \{\theta\}$, we have that:
    \begin{equation*}
        \sum_{j=1}^K (\theta'(j) - \theta(j)) \mu(j) = \Delta \sum_{j\in U(\theta) \backslash C} \left(\frac{1}{M} - 0\right) = \Delta \frac{M-V}{M}.
    \end{equation*}
    And thus, 
    \begin{align*}
        R_T(\mu_\theta) &= \Delta \frac{M-V}{M} (T - N_{\mu_\theta}(\theta; T)) \\
        &\geq \Delta \frac{M-V}{M} \bigg(T - N_{\mu_0}(\theta; T) - T \sqrt{\frac{1}{2}D(P_{\mu_0},P_{\mu_\theta})}\bigg),
    \end{align*}
    where the inequality follows from the fact that\footnote{See Exercise 14.4 in \cite{lattimore2020bandit} for a general version of this inequality.} $N_{\mu_\theta}(\theta; T) - N_{\mu_0}(\theta; T) \leq T \tv(P_{\mu_0},P_{\mu_\theta})$ and by Pinsker's inequality. Starting from Lemma \ref{low:kl-decomp}, we have that:
    \begin{align*}
         D(P_{\mu_0}, P_{\mu_\theta}) &= \sum_{\theta' \in \Theta} N_{\mu_0}(\theta'; T) \sum_{j=1}^K \theta'(j) d(\mu_0(j),\mu_\theta(j)) \\
         &= N_{\mu_0}(\theta; T) \sum_{j\in U(\theta) \backslash C} \frac{1}{M} d\left(\frac{1}{2},\frac{1}{2}-\Delta\right)\\
         &= \frac{M-V}{M} N_{\mu_0}(\theta; T) \left(-\frac{1}{2} \log(1-4\Delta^2)\right)\\
         &\leq \frac{M-V}{M} N_{\mu_0}(\theta; T) c \Delta^2, 
    \end{align*}
    where the second equality holds since the only arms whose means have changed between the two environments lie exclusively in the support of $\theta$,
    and the inequality holds for $\Delta \leq \frac{1}{4}$ with $c = 8\log{\frac{4}{3}}$. Hence:
    \begin{align*}
        \sup_{\mu} R_T(\mu) &\geq \frac{1}{N}\sum_{\theta \in \Theta} R_T(\mu_\theta)\\
        &\geq \Delta \frac{M-V}{M} \bigg(T - \frac{T}{N} - T \sqrt{\frac{1}{2}\frac{M-V}{M} \frac{T}{N} c \Delta^2}\bigg)\\
        &\geq \Delta \frac{M-V}{M} T \bigg(\frac{1}{2} - \Delta \sqrt{\frac{1}{2}\frac{M-V}{M} \frac{T}{N} c}\bigg),
    \end{align*}
    where the second inequality holds by the concavity of the square root, and the third since $N\geq2$. The theorem then follows by setting $\Delta = \frac{1}{4} \sqrt{\frac{2MN}{c(M-V)T}}$ and verifying that the condition on $T$ ensures that $\Delta \leq \frac{1}{4}$.
\end{proof}

\subsection{\texorpdfstring{$\epsilon$}{Epsilon}-Uniform Policies}
The next bound concerns the $\epsilon$-Uniform structure of Example \ref{example-epsilon}. While all policies are still equidistant, this case does not enjoy the peculiar discrete structure of Example~\ref{example-flower}.
\begin{theorem} \label{thm:eps-uniform}
    If $\Theta$ conforms to the $\epsilon$-uniform structure of Example \ref{example-epsilon}. Then for $T \geq \frac{K}{4\log(4/3)}$ and any algorithm, there exists a sequence of losses such that
    $
    R_T \geq \frac{1}{18}\epsilon\sqrt{KT}.
    $
\end{theorem}
The proof, see Appendix \ref{appendix-eps-uni-proof}, is similar to that of Theorem~\ref{low:flower:thm} apart from the fact that all policies contribute to the KL-divergence between any two environments (see Lemma~\ref{low:kl-decomp}) 
since all policies have full support.
Notice that the bound is of order $\sqrt{\epsilon^2KT}$ instead of $\sqrt{\epsilon KT} = \sqrt{\mathcal{S}^*(\Theta)T}$, which would have nearly matched the bound of Theorem~\ref{upper:alt-losses:thm}. This was expected since we have shown that Algorithm~\ref{osmd} enjoys a better regret bound in this case. It is interesting to see if a matching lower bound could be proved.  

\subsection{The Two Policies Case}
For the two policies case, we can prove a lower bound of order $\sqrt{H^2(\theta_1,\theta_2)T}$ as asserted by the following theorem, where $H^2(\theta_1,\theta_2)=\frac{1}{2}\sum_j (\sqrt{\theta_1(j)}-\sqrt{\theta_2(j)})^2$ is the squared Hellinger distance. Relative to the total variation, we have that in general:
$
    \frac{1}{2}D^2_{TV}(\theta_1,\theta_2) \leq H^2(\theta_1,\theta_2) \leq D_{TV}(\theta_1,\theta_2).
$ 
%Note that when the two policies conform to the structure of Example~\ref{example-flower}, $H^2(\theta_1,\theta_2)$ and $\tv(\theta_1,\theta_2)$ coincide.

\begin{theorem} \label{thm:H2}
    Assume that $\Theta=\{\theta_1,\theta_2\}$ and $H^2(\theta_1,\theta_2) > 0$. Then for any algorithm and $T \geq \frac{1}{8\log(4/3)H^2(\theta_1,\theta_2)}$, there exists a sequence of losses such that
    $
    R_T \geq \frac{1}{13}\sqrt{H^2(\theta_1,\theta_2)T}.$
\end{theorem}

The proof follows a similar scheme as before, so we only sketch the main distinctions and defer the full proof to Appendix \ref{appendix-two-policies-proof}. For an arm $j$, we define $z_1(j) = \frac{\sqrt{\theta_1(j)} - \sqrt{\theta_2(j)}}{\sqrt{\theta_1(j)} + \sqrt{\theta_2(j)}}$, and $z_2(j) = -z_1(j)$.
We use two environments $\mu_1$ and $\mu_2$, where 
$ \mu_1(j) = 1/2 - \Delta z_1(j),
$
with $\mu_2(j)$ defined analogously. 
Subsequently, focusing on $\mu_1$ and $\theta_1$, 
%we can lower bound $R_T(\mu_1)$ by
we can bound $R_T(\mu_1)$ from below by
\begin{equation*}
    2 \Delta H^2(\theta_1,\theta_2) \bigg(T - N_{\mu_0}(\theta; T) - T \sqrt{\frac{1}{2}D(P_{\mu_0},P_{\mu_1})}\bigg),
\end{equation*}
whereas we can show that $D(P_{\mu_0}, P_{\mu_1}) \leq 2 c \Delta^2 H^2(\theta_1,\theta_2) T$. To see the latter, it suffices to start from Lemma \ref{low:kl-decomp} and to use that 
$
    d(\mu_0(j),\mu_1(j)) \leq c \Delta^2 z_1(j)^2
$
for sufficiently small $\Delta$, and that 
$
    \sum\nolimits_j \theta_1(j) z_1(j)^2 \leq 2H^2(\theta_1,\theta_2).
$
%What remains then is to tune the value of $\Delta$ to maximize the average of the lower bounds on $R_T(\mu_1)$ and $R_T(\mu_2)$.
%%\begin{proof}

We will consider two environments $\mu_1$ and $\mu_2$. For environment $\mu_1$ and arm $j$, we choose\footnote{Note that $\sqrt{\theta_1(j)} + \sqrt{\theta_2(j)}$ is always positive by our assumption that each arm is in the support of at least one policy.} 
\[ \mu_1(j) = \frac{1}{2} - \Delta \frac{\sqrt{\theta_1(j)} - \sqrt{\theta_2(j)}}{\sqrt{\theta_1(j)} + \sqrt{\theta_2(j)}},
\]
where $0 \leq \Delta < \frac{1}{2}$ is to be tuned later. Environment $\mu_2$ is defined analogously. Additionally let $\mu_0$ be an environment such that $\mu_0(j) = \frac{1}{2}$ for any arm $j$. Note that $\theta_1$ ($\theta_2$) is the optimal policy in $\mu_1$ ($\mu_2$). Indeed, 
\begin{align*}
    &\sum_{j=1}^K (\theta_2(j) - \theta_1(j)) \mu_1(j) \\
&\quad= \Delta \sum_{j=1}^K  (\theta_1(j) - \theta_2(j)) \frac{\sqrt{\theta_1(j)} - \sqrt{\theta_2(j)}}{\sqrt{\theta_1(j)} + \sqrt{\theta_2(j)}} \\
&\quad= \Delta \sum_{j=1}^K  (\sqrt{\theta_1(j)} - \sqrt{\theta_2(j)})^2 = 2 \Delta H^2(\theta_1,\theta_2).
\end{align*}
Therefore,  
    \begin{align*}
        &R_T(\mu_1)\\
        &\quad= 2 \Delta H^2(\theta_1,\theta_2) (T - N_{\mu_1}(\theta; T)) \\
        &\quad\geq 2 \Delta H^2(\theta_1,\theta_2) \left(T - N_{\mu_0}(\theta; T) - T \sqrt{\frac{1}{2}D(P_{\mu_0},P_{\mu_1})}\right),
    \end{align*}
    where the inequality follows by using that $N_{\mu_1}(\theta; T) - N_{\mu_0}(\theta; T) \leq T \tv(P_{\mu_0},P_{\mu_1})$ followed by an application of Pinsker's inequality. Note that for $\Delta \leq 1/4$ and $c=8\log(4/3)$, we have that 
    \begin{align*}
    &d(\mu_0(j),\mu_1(j)) \\
    &\quad= d\left(\frac{1}{2},\frac{1}{2}-\Delta \frac{\sqrt{\theta_1(j)} - \sqrt{\theta_2(j)}}{\sqrt{\theta_1(j)} + \sqrt{\theta_2(j)}} \right)\\
    &\quad= -\frac{1}{2} \log\left(1-4\Delta^2\left(\frac{\sqrt{\theta_1(j)}- \sqrt{\theta_2(j)}}{\sqrt{\theta_1(j)} + \sqrt{\theta_2(j)}}\right)^2\right)\\
    &\quad\leq c \Delta^2 \left(\frac{\sqrt{\theta_1(j)}- \sqrt{\theta_2(j)}}{\sqrt{\theta_1(j)} + \sqrt{\theta_2(j)}}\right)^2.
    \end{align*}
    While on the other hand
    \begin{align*}
        &\sum_{j=1}^K \theta_1(j) \left(\frac{\sqrt{\theta_1(j)}- \sqrt{\theta_2(j)}}{\sqrt{\theta_1(j)} + \sqrt{\theta_2(j)}}\right)^2 \\
        &\quad= \sum_{j=1}^K  (\sqrt{\theta_1(j)}- \sqrt{\theta_2(j)})^2 \frac{\theta_1(j)}{(\sqrt{\theta_1(j)} + \sqrt{\theta_2(j)})^2}\\
        &\quad\leq \sum_{j=1}^K  (\sqrt{\theta_1(j)}- \sqrt{\theta_2(j)})^2\\
        &\quad= 2H^2(\theta_1,\theta_2).
    \end{align*}
    With the analogous inequalities for $\mu_2$ and $\theta_2$ we get that 
    \begin{align*}
         D(P_{\mu_0}, P_{\mu_1}) 
         &= 2 c \Delta^2 H^2(\theta_1,\theta_2) (N_{\mu_1}(\theta'; T)+N_{\mu_2}(\theta'; T)) \\
         &= 2 c \Delta^2 H^2(\theta_1,\theta_2)T.
    \end{align*}
    Putting everything together, we get that
        \begin{align*}
        &\sup_{\mu} R_T(\mu)\\ 
        &\quad\geq \frac{1}{2}( R_T(\mu_1)+R_T(\mu_2))\\
        &\quad\geq2 \Delta H^2(\theta_1,\theta_2) \left(T - \frac{T}{2} - T \sqrt{c \Delta^2 H^2(\theta_1,\theta_2)T}\right)\\
        &\quad= \Delta H^2(\theta_1,\theta_2) T \left(1 - 2\Delta \sqrt{c H^2(\theta_1,\theta_2)T}\right),
    \end{align*}
    The theorem then follows by setting $\Delta = \frac{1}{4\sqrt{c H^2(\theta_1,\theta_2)T}}$ and verifying that the stated condition on $T$ ensures that $\Delta$ is less than $1/4$.
%\end{proof}

The squared Hellinger distance can be related to other measures of divergence. For instance, it is of the same order as the Jensen-Shannon divergence and the triangular discrimination \cite{topsoe, H2-JSD}. Thus, the bound of Theorem \ref{thm:H2} can be stated, up to small constants, in terms of these measures as well. 

\subsection{A Matching Lower Bound for a Class of Policy Sets} \label{sec-multi-task}
Lastly, we provide a lower bound that almost matches both Theorems  \ref{upper:alt-losses:thm} and \ref{cvxhull-thm} for a certain class of policy sets that we discuss shortly. This bound is analogous to the $\sqrt{KT\log N/\log K}$ lower bound proved in \cite{expertslowerbound}. However, unlike \cite{expertslowerbound}, we rely on \textit{fixed} sets of \textit{stochastic} policies.
\begin{theorem}
\label{th:worst-case-lower}
For any integer $q\geq2$, there exists a problem structure where $K\geq2$, $\mathcal{S}^*(\Theta) = q$, and $\mathcal{S}^*(\Theta) \log N = K D^*(\Theta)$; such that any algorithm, for sufficiently large $T$, suffers $\Omega\left(\sqrt{\mathcal{S}^*(\Theta) T \frac{\log N}{\log \mathcal{S}^*(\Theta)}}\right)$ regret.
%Moreover, when $K\geq4$, the policy set in question can be chosen such that $\mathcal{S}^*(\Theta) < \min(K,N)$.
\end{theorem}
%In the type of structure referred to in the theorem, the arms are divided into $M$ sections
%\footnote{This construction can also be done with $M=1$, although this case is not interesting as it is equivalent to a standard bandits problem.}
(where $\frac{K}{M}=q$) and each policy is a uniform distributions supported over $M$ arms such that its support contains an arm from each section. When the policy set contains all possible $(\frac{K}{M})^M$ policies, this problem becomes equivalent to playing $M$ bandit problems simultaneously with the choice of policy at each round dictating an arm choice at each game. The distinction is that only the loss of one such arm is observed, while the player suffers the average regret of the $M$ games. This structure is commonly used to prove lower bounds for combinatorial bandits, see \cite{comb_audibert} for example. A straightforward adaptation of the proof of Theorem 5 in \cite{comb_audibert} to our case\footnote{In the setting of Theorem 5 in \cite{comb_audibert}, the scale of the incurred losses is $M$, while for us it is $1$, which is why their bound has an extra factor of $M$. The other main difference is the feedback model, which matters in the proof when bounding the KL-divergence between two environments, which, for us, can be easily bounded via Lemma \ref{low:kl-decomp} yielding a similar result.}
leads to a lower bound of $\Omega(\sqrt{K T})$, from which the theorem follows by using that $\mathcal{S}^*(\Theta) = \mathcal{S}(\Theta)=\frac{K}{M}$ and that $N=(\frac{K}{M})^M$. For what concerns the bound of Theorem~\ref{cvxhull-thm}, we have that
%$\tau^*$ is the uniform distribution (with full support) and 
$D^*(\Theta)=\log(\frac{K}{M})$. Thus, Theorems \ref{upper:alt-losses:thm} and \ref{cvxhull-thm} provide the same bound since $\mathcal{S}^*(\Theta) \log N = K D^*(\Theta)$. 
In the type of structure referred to in the theorem, the arms are divided into $M$ sections
(where $\frac{K}{M}=q$) and each policy is a uniform distributions supported over $M$ arms such that its support contains an arm from each section. When the policy set contains all such policies 
($(\frac{K}{M})^M$ in total),
this problem becomes equivalent to playing $M$ bandit problems simultaneously with the choice of policy at each round dictating an arm choice at each game. The distinction is that only the loss of one such arm is observed, while the player 
%suffers
aims to minimize
the average regret of the $M$ games. This type of structure
(albeit with a different type of feedback)
is commonly used to prove lower bounds for combinatorial bandits, see \cite{comb_audibert} for example. 
An adaptation of the proof of Theorem 5 in \cite{comb_audibert} to our case (see Appendix \ref{appendix-multi-task-proof}) leads
to a lower bound of $\Omega(\sqrt{K T})$, from which the theorem follows by using that $\mathcal{S}^*(\Theta) = \mathcal{S}(\Theta)=\frac{K}{M}$ and that $N=(\frac{K}{M})^M$. For what concerns the bound of Theorem~\ref{cvxhull-thm}, we have that
%$\tau^*$ is the uniform distribution (with full support) and 
$D^*(\Theta)=\log(\frac{K}{M})$. Thus, Theorems \ref{upper:alt-losses:thm} and \ref{cvxhull-thm} provide the same bound since $\mathcal{S}^*(\Theta) \log N = K D^*(\Theta)$. 

\section{Conclusion}
We analyzed two algorithms providing regret bounds that depend on information-theoretic quantities describing the policy set. We proved lower bounds for certain classes of policy sets highlighting instances where our regret bounds are nearly matched. Nevertheless, it remains to be seen if better guarantees can be proved in cases like Example \ref{example-stripes} where $\mathcal{S}^*(\Theta)\geq1$
even if $\mathcal{TV}(\Theta)$ can be smaller. It is also interesting to see what are the optimal rates for cases like Example~\ref{example-epsilon} where the bound of Theorem~\ref{upper:alt-losses:thm} is suboptimal, as we learned in Section~\ref{sec:osmd}, even if $\mathcal{S}^*(\Theta)$ is of the same order as $\mathcal{TV}(\Theta)$. Another direction is 
%to see if it is possible to adapt
to investigate the possibility of adapting
to the policy set structure if the distributions are not known beforehand.
  

%\section*{Acknowledgment}

\bibliographystyle{IEEEtran}
\bibliography{IEEE}

\section{Appendix for Proofs}

\paragraph{Proof of Theorem \ref{thm:main}.}

\begin{proof}
\label{proof:main}
Our proof has two steps. In Step 1, we will show that SimCLR is equivalent to minimizing the cross entropy loss defined in Eqn.~(\ref{eqn:cross-entropy}). 
In Step 2, we will show  that minimizing the cross-entropy loss 
is equivalent to spectral clustering on $\bfpi$. 
Combining the two steps together, we have proved our theorem. 

\textbf{Step 1: } SimCLR is equivalent to minimizing the cross entropy loss.

The cross-entropy loss takes expectation over 
$\bfW_\bfX\sim \mathbb{P}(\cdot ; \bfpi)$, 
which means $\bfW_\bfX$ has exactly one non-zero entry in each row $i$. By Lemma~\ref{lem:multinomial}, we know every row $i$ of $\bfW_\bfX$ is independent of other rows. Moreover, 
$\bfW_{\bfX,i}\sim \mathcal{M}(1, \bfpi_i/\sum_j \bfpi_{i,j})=\mathcal{M}(1, \bfpi_i)$, because $\bfpi_i$ itself is a probability distribution.
Similarly, we know $\bfW_\bfZ$ also has the row-independent property by sampling over $\mathbb{P}(\cdot;\bfK_\bfZ)$.
Therefore, by Lemma~\ref{lem:cross_split}, we know Eqn.~(\ref{eqn:cross-entropy}) is equivalent to:
\[
 -\sum_{i=1}^n \mathbb{E}_{\bfW_{\bfX,i}}[\log \mathbb{P}(\bfW_{\bfZ,i}=\bfW_{\bfX,i};\bfK_\bfZ)],
\]

This expression takes expectation over $\bfW_{\bfX,i}$ for the given row $i$. Notice that 
$\bfW_{\bfX,i}$ has exactly one non-zero entry, which equals $1$ (same for $\bfW_{\bfZ,i}$). 
As a result
we expand the above expression to be:
\begin{equation}
 -\sum_{i=1}^n \sum_{j\neq i} \Pr(\bfW_{\bfX,i,j}=1)\log \Pr(\bfW_{\bfZ,i,j}=1).
\label{eqn:detailed-expansion}    
\end{equation}


By Lemma~\ref{lem:multinomial}, $\Pr(\bfW_{\bfZ,i,j}=1)=\bfK_{\bfZ,i,j}/\|\bfK_{\bfZ,i}\|_1$ for $j\neq i$. Recall that $\bfK_\bfZ=(k(\bfZ_i-\bfZ_j))_{(i,j)\in[n]^2}$, which means 
$\bfK_{\bfZ,i,j}/\|\bfK_{\bfZ,i}\|_1=\frac{\exp(-\|\bfZ_i-\bfZ_j\|^2/{2\tau})}{\sum_{k\neq i}
\exp(-\|\bfZ_i-\bfZ_k\|^2/{2\tau})
}$ for $j\neq i$, when $k$ is the Gaussian kernel with variance $\tau$. 

Notice that $\bfZ_i=f(\bfX_i)$, so we know
\begin{equation}
-\log \Pr(\bfW_{\bfZ,i,j}=1)=
-\log \frac{\exp(-\|f(\bfX_i)-f(\bfX_j)\|^2/{2\tau})}{\sum_{k\neq i}
\exp(-\|f(\bfX_i)-f(\bfX_k)\|^2/{2\tau}),
}
\label{eqn:infonce-equivalence}    
\end{equation}


The right hand side is exactly the InfoNCE loss defined in Eqn.~(\ref{eqn:infonce}).
Inserting Eqn.~(\ref{eqn:infonce-equivalence}) into Eqn.~(\ref{eqn:detailed-expansion}), we get the SimCLR algorithm, which first samples augmentation pairs $(i,j)$ with $\Pr(\bfW_{\bfX,i,j}=1)$ for each row $i$, and then optimize the InfoNCE loss. 

\textbf{Step 2: } minimizing the cross entropy loss 
is equivalent to spectral clustering on $\bfpi$.


By Lemma~\ref{lem:convert_to_spectral}, we may further convert the loss to 
\begin{equation}
\label{eqn:main-theorem-repul-attr}
\min_{\bfZ}
-\sum_{(i,j)\in [n]^2} \mathbf{P}_{i,j}
\log k (\bfZ_i-\bfZ_j)+\log \mathbf{R}(\bfZ).
\end{equation}
Since $k$ is the Gaussian kernel, this reduces to \[
\min_\bfZ \mathrm{tr}(\bfZ^\top \mathbf{L}(\bfpi) \bfZ)
+\log \mathbf{R}(\bfZ),
\]

where we use the fact that $\mathbb{E}_{\bfW_\bfX\sim \mathbb{P}(\cdot; \bfpi)}[\mathbf{L}(\bfW_\bfX)]
=\mathbf{L}(\bfpi)
$, because the Laplacian operator is linear and $
\mathbb{E}_{\bfW_\bfX\sim \mathbb{P}(\cdot; \bfpi)}(\bfW_\bfX)=\bfpi
$.
\end{proof}

\paragraph{Proof of Theorem \ref{thm:clip}.}
\begin{proof}
Since $\bfW_\bfX\sim \mathbb{P}(\cdot;\bfpi_{\mathbf{A}, \mathbf{B}})$, we know 
$\bfW_\bfX$ has exactly one non-zero entry in each row, denoting the pair that got sampled. 
A notable difference compared to the previous proof is we now have $n_\mathcal{A}+n_\mathcal{B}$ objects in our graph. CLIP deals with this by taking a mini-batch of size $2N$, 
such that $n_\mathcal{A}=n_\mathcal{B}=N$, and adding the $2N$ InfoNCE losses together. We label the objects in $\mathcal{A}$ as $[n_\mathcal{A}]$, and the objects in $\mathcal{B}$ as $\{n_\mathcal{A}+1, \cdots, n_\mathcal{A}+n_\mathcal{B}\}$. 

Notice that $\bfpi_{\mathbf{A}, \mathbf{B}}$ is a bipartite graph, so the edges of objects in $\mathcal{A}$ will only connect to object in $\mathcal{B}$ and vice versa. We can define the similarity matrix in $\cZ$ as $\bfK_\bfZ$, 
where $\bfK_\bfZ(i, j+n_\mathcal{A})=\bfK_\bfZ(j+n_\mathcal{A},i)= k(\bfZ_i-\bfZ_j)$ for $i\in [n_\mathcal{A}], j\in [n_\mathcal{B}]$, and otherwise we set $\bfK_\bfZ(i,j)=0$. 
The rest is same as the previous proof. 
\end{proof}

\paragraph{Proof of Theorem \ref{thm:exponential}.}

\begin{proof}
\label{proof:exponential}
Since the objective function consists of a linear term combined with an entropy regularization, which is a strongly concave function, the maximization problem is a convex optimization problem. Owing to the implicit constraints provided by the entropy function, the problem is equivalent to having only the equality constraint. We then introduce the Lagrangian multiplier $\lambda$ and obtain the following relaxed problem:

$$
\widetilde{E}(\boldsymbol{\alpha})=\psi_{1}-\sum_{i=1}^n \alpha_{i} \psi_{i}+\tau \sum_{i=1}^n \alpha_{i}\log \alpha_{i}+\lambda\left(\boldsymbol{\alpha}^{\top} \mathbf{1}_n-1\right).
$$

As the relaxed problem is unconstrained, taking the derivative with respect to $\alpha_{i}$ yields

$$
\frac{\partial \widetilde{E}(\boldsymbol{\alpha})}{\partial \alpha_{i}}=-\psi_{i}+\tau\left(\log \alpha_{i}+\alpha_{i} \frac{1}{\alpha_{i}}\right)+\lambda=0.
$$

Solving the above equation implies that $\alpha_{i}$ takes the form
$
\alpha_{i}=\exp \left(\frac{1}{\tau} \psi_{i}\right) \exp \left(\frac{-\lambda}{\tau}-1\right).
$ Since $\alpha_{i}$ lies on the probability simplex, the optimal $\alpha_{i}$ is explicitly given by
$
\alpha^{*}_{i}=\frac{\exp \left(\frac{1}{\tau} \psi_{i}\right)}{\sum_{i^{\prime}=1}^n \exp \left(\frac{1}{\tau} \psi_{i^{\prime}}\right)} .
$ Substituting the optimal point into the objective function, we obtain
$$
\begin{aligned}
E\left(\boldsymbol{\alpha}^*\right)  &=\psi_1-\sum_{i=1}^n \frac{\exp \left(\frac{1}{\tau} \psi_{i}\right)}{\sum_{i^{\prime}=1}^n \exp \left(\frac{1}{\tau} \psi_{i^{\prime}}\right)} \psi_{i}+\tau \sum_{i=1}^n \frac{\exp \left(\frac{1}{\tau} \psi_{i}\right)}{\sum_{i^{\prime}=1}^n \exp \left(\frac{1}{\tau} \psi_{i^{\prime}}\right)}\log \frac{\exp \left(\frac{1}{\tau} \psi_{i}\right)}{\sum_{i^{\prime}=1}^n \exp \left(\frac{1}{\tau} \psi_{i^{\prime}}\right)} \\
& =\psi_1 - \tau \log \left(\sum_{i=1}^n \exp \left(\frac{1}{\tau} \psi_{i}\right)\right).
\end{aligned}
$$
Thus, the Lagrangian dual function is given by
\begin{equation*}
-E\left(\boldsymbol{\alpha}^*\right)= -\tau \log \frac{\exp \left(\frac{1}{\tau} \psi_{1}\right)}{\sum_{i=1}^n \exp \left(\frac{1}{\tau} \psi_{i}\right)}.\qedhere
\end{equation*}
\end{proof}



\section{More on Experiments} \label{section: experiment_details}

\paragraph{CIFAR-10 and CIFAR-100} CIFAR-10 ~\citep{krizhevsky2009learning} and CIFAR-100 ~\citep{krizhevsky2009learning} are well-known classic image classification datasets. Both CIFAR-10 and CIFAR-100 contain a total of 60k $32 \times 32$ labeled images of different classes, with 50k for training and 10k for testing. CIFAR-10 is similar to CIFAR-100, except there are 10 different classes in CIFAR-10 and 100 classes in CIFAR-100.

\paragraph{TinyImageNet} TinyImageNet ~\citep{le2015tiny} is a subset of ImageNet ~\citep{deng2009imagenet}. There are 200 different object classes in TinyImageNet, with 500 training images, 50 validation images, and 50 test images for each class. All the images in TinyImageNet are colored and labeled with a size of $64 \times 64$.

\textbf{Pseudo-code.} Algorithm \ref{alg:Training Procedure} presents the pseudo-code for our empirical training procedure.

\begin{algorithm}[!htbp]
\caption{Training Procedure}
\label{alg:Training Procedure}
\begin{algorithmic}[1]
\REQUIRE trainable encoder network $f$, batch size $N$, augmentation strategy \textit{aug}, loss function $L$ with hyperparameters \textit{args}
\FOR {sampled minibatch ${x_i}_{i=1}^N$}
\FORALL{$i \in { 1, ..., N }$}
\STATE draw two augmentations $t_i = \textit{aug}\left(x_i\right) $, $t_i' = \textit{aug}\left(x_i\right) $
\STATE $z_i = f\left(t_i\right)$, $z_i' = f\left(t_i'\right)$
\ENDFOR
\STATE compute loss $\mathcal{L} = L(N, z, z', \textit{args})$
\STATE update encoder network $f$ to minimize $\mathcal{L}$
\ENDFOR
\STATE \textbf{Return} encoder network $f$
\end{algorithmic}
\end{algorithm}

We also provide the pseudo-code for our core loss function used in the training procedure in Algorithm \ref{alg:Core loss}. The pseudo-code is almost identical to SimCLR's loss function, with the exception of an extra parameter $\gamma$.

\begin{algorithm}[!htbp]
\caption{Core loss function $\mathcal{C}$}
\label{alg:Core loss}
\begin{algorithmic}[1]
\REQUIRE batch size $N$, two encoded minibatches $z_1, z_2$, $\gamma$, temperature $\tau$
\STATE $z = \textit{concat}\left(z_1, z_2\right)$
\FOR {$i \in {1, ..., 2N }, j \in {1, ..., 2N}$ }
\STATE $s_{i,j} = \Vert z_i - z_j \Vert_2^{\gamma}$
\ENDFOR
\STATE \textbf{define} $l(i, j)$ \textbf{as} $l(i, j) = - \log \frac{exp\left(s_{i,j}/\tau \right)}{\sum_{k=1}^{2N} \mathbf{1}{[k \ne i]} exp\left(s{i, j} / \tau \right)} $
\STATE \textbf{Return} $\frac{1}{2N} \sum_{k=1}^N\left[l(i, i+N) + l(i+N, i)\right]$
\end{algorithmic}
\end{algorithm}

Utilizing the core loss function $\mathcal{C}$, we can define all kernel loss functions used in our experiments in Table \ref{table: loss definition}. For all $z_i \in z$ with even dimensions $n$, we define $z_{L_i} = z_i\left[0:n/2\right]$ and $z_{R_i} = z_i\left[n/2:n\right]$.

\begin{table}[ht]
\centering
\begin{tabular}{{@{}l|l@{}}}
Kernel  &  Loss function \\ \midrule
Laplacian & $\mathcal{C}\left(N, z, z', \gamma=1, \tau\right)$\\ \midrule
Sum       & $\lambda * \mathcal{C}\left(N, z, z', \gamma=1, \tau_1\right) + (1-\lambda) * \mathcal{C}\left(N, z, z', \gamma=2, \tau_2\right)$  \\ \midrule
Concatenation Sum&$\lambda * \mathcal{C}\left(N, z_L, z'_L, \gamma=1, \tau_1\right) + (1-\lambda) * \mathcal{C}\left(N, z_R, z'_R, \gamma=2, \tau_2\right)$\\ \midrule
$\gamma = 0.5$ & $\mathcal{C}\left(N, z, z', \gamma=0.5, \tau\right)$          \\ 

\end{tabular}

\caption{Definition of kernel loss functions in our experiments}
\label {table: loss definition}
\end{table}

\textbf{Baselines.} We reproduce the SimCLR algorithm using PyTorch Lightning~\citep{PytorchLightning}.

\textbf{Encoder details.}
The encoder $f$ consists of a backbone network and a projection network. We employ ResNet50~\citep{ResNet} as the backbone and a 2-layer MLP (connected by a batch normalization~\citep{ioffe2015batch} layer and a ReLU \cite{nair2010rectified} layer) with hidden dimensions 2048 and output dimensions 128 (or 256 in the concatenation kernel case).

\textbf{Encoder hyperparameter tuning.}
For each encoder training case, we randomly sample 500 hyperparameter groups (sample details are shown in Table \ref{table: Hyperparameter sample}) and train these samples simultaneously using Ray Tune ~\citep{RayTune}, with the ASHA scheduler~\citep{li2018massively}. Ultimately, the hyperparameter group that maximizes the online validation accuracy (integrated in PyTorch Lightning) within 5000 validation steps is chosen for the given encoder training case.

\begin{table}[ht]
\centering

\begin{tabular}{@{}l|l|l@{}}
\midrule
Hyperparameter  & Sample Range & Sample Strategy \\ \midrule
start learning rate & $\left[10^{-2}, 10\right]$ & log uniform \\ \midrule
$\lambda$       & $\left[0, 1\right]$ & uniform \\ \midrule
$\tau$, $\tau_1$, $\tau_2$ & $\left[0, 1\right]$ & log uniform \\ \midrule
\end{tabular}

\caption{Hyperparameters sample strategy}
\label {table: Hyperparameter sample}
\end{table}

\textbf{Encoder training.} 
We train each encoder using the LARS optimizer~\citep{LARSOptimizer}, LambdaLR Scheduler in PyTorch, momentum 0.9, weight decay $10^{-6}$, batch size 256, and the aforementioned hyperparameters for 400 epochs on a single A-100 GPU.

\textbf{Image transformation.} The image transformation strategy, including augmentation, is identical to the default transformation strategy provided by PyTorch Lightning.

\textbf{Linear evaluation.}
The linear head is trained using the SGD optimizer with a cosine learning rate scheduler, batch size 64, and weight decay $10^{-6}$ for 100 epochs. The learning rate starts at $0.3$ and ends at $0$.

\textbf{Moco Experiments.} We also tested our method based on MoCo~\citep{he2019moco}. The results are summarized in Table \ref{tab:results-moco}. Here we choose ResNet18~\citep{ResNet} as the backbone and set a temperature of $0.1$ as default. For our simple sum kernel, we set $\lambda=0.8$. The results show that our method outperforms the original MoCo method.

\begin{table}[thb]
\centering
\caption{MoCo Experiment Results on CIFAR-10 and CIFAR-100.}
\label{tab:results-moco}
\resizebox{\textwidth}{!}{%
\begin{tabular}{@{}c|ccc|ccc@{}}
\toprule
\multirow{3}{*}{Method} & \multicolumn{3}{c|}{CIFAR-10} & \multicolumn{3}{c}{CIFAR-100} \\ \cmidrule(lr){2-4} \cmidrule(lr){5-7} 
                        & 200 epochs & 400 epochs    & 1000 epochs   & 200 epochs & 400 epochs & 1000 epochs         \\ \midrule
MoCo (repro.)         & $76.41 \pm 0.12$    & $80.01 \pm 0.15$          & $84.45 \pm 0.08$    & $\mathbf{47.02 \pm 0.11}$ & $52.50 \pm 0.07$ & $57.62 \pm 0.15$            \\
\midrule
Laplacian Kernel        & ${78.09 \pm 0.10}$    & $\mathbf{83.85 \pm 0.09}$          & $\mathbf{88.34 \pm 0.16}$    & $46.12 \pm 0.22$   & $53.44 \pm 0.17$ & $59.10 \pm 0.14$        \\
Simple Sum Kernel & $\mathbf{78.12 \pm 0.15}$   & $83.23 \pm 0.18$ & $87.50 \pm 0.20$ & $46.65 \pm 0.06$ & $\mathbf{53.62 \pm 0.19}$ & $\mathbf{59.83 \pm 0.12}$\\
\bottomrule
\end{tabular}
}
\end{table}



\section{More Experiments on Synthetic Data}


Consider a scenario with $n$ clusters, each containing $k$ vertices. Let the probability of vertices $u$ and $v$ from the same cluster belonging to $\bfpi$ be $p$. Conversely, for vertices $u$ and $v$ from different clusters, let the probability of belonging to $\pi$ be $q$. We generate the graph $\bfpi$ randomly, based on $p$ and $q$. We experiment with values of $k=100$ and $n=6$ for ease of visualization, embedding all points in a two-dimensional space. Each vertex's initial position originates from a normal distribution. In each iteration, we sample a subgraph of $\bfpi$ uniformly, ensuring each vertex has an out-degree of $1$. We then optimize the corresponding vectors using InfoNCE loss with an SGD optimizer and iterate until convergence. Our experimental setup consists of an SGD learning rate of $1$, an InfoNCE loss temperature of $0.5$, and a batch size of $50$. We evaluate two scenarios with different $p$ and $q$ values: $p=1$, $q=0$, and $p=0.75$, $q=0.2$. The results of these experiments are visualized in Figure \ref{fig:vis-spectral-cluster}. The obtained embeddings exhibit the hallmark pattern of spectral clustering of graph $\bfpi$.

\begin{figure}[!tb]
\centering
\subfigure{
\includegraphics[width=1\textwidth]{Figures/cluster_pi.png}
\label{fig:vis-cluster}
}
\subfigure{
\includegraphics[width=1\textwidth]{Figures/noised_cluster_pi.png}
\label{fig:vis-noised-cluster}
}
\caption{Visualizations of the optimization process using InfoNCE Loss on the vectors corresponding to $\bfpi$. Points of identical color belong to the same cluster within $\bfpi$. To showcase the internal structure of $\bfpi$, we randomly select 10 vertices from each cluster to display the edge distribution of $\bfpi$.}
\label{fig:vis-spectral-cluster}
\end{figure}







\end{document}
