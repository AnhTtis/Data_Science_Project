\renewcommand{\thesectiondis}[2]{\Alph{section}:}
\section{About $D^*(\Theta)$} \label{appendix-d-star}
In this short section, we justify some of the claims made about $D^*(\Theta)$. Fix a policy set $\Theta$, in what follows we denote by $\bar{\tau}$ the uniform mixture of the policies; that is, $\bar{\tau}(j) = \frac{1}{N}\sum_{\theta \in \Theta} \theta(j)$ for any arm $j$. Firstly, to see that $D^*(\Theta)\leq \log N$, it suffices to use that $D^*(\Theta) \leq D(\Theta||\bar{\tau})$, and that for any $\theta \in \Theta$ we have that
\begin{equation*}
    D(\theta, \bar{\tau}) = \sum\nolimits_j \theta(j) \log \frac{\theta(j)}{ \frac{1}{N}\sum_{\theta' \in \Theta} \theta'(j)} \leq \log N.
\end{equation*}
Secondly, we show that $D^*(\Theta) = D(\Theta||\bar{\tau})$ (i.e., $\bar{\tau}$ minimizes $D(\Theta||\cdot)$ in $\text{co}(\Theta)$) whenever $D(\theta, \bar{\tau})$ is the same for every $\theta \in \Theta$. Define, for $\theta \in \Theta$, $f_\theta(\tau)=D(\theta,\tau)$. Then, $D(\Theta||\tau) = \max_\theta f_\theta(\tau)$, which is convex in $\tau$. Denote by $\partial D(\Theta||\tau)$ the subdifferential (the set of subgradients) at $\tau$, which is given by (see Theorem 2.19 in \cite{orabona})
$
    \partial D(\Theta||\tau) = ~\text{co}(\{\nabla f_\theta(\tau)\}_{\theta \in A(\tau)})
$ for $\tau$ with full support, where $A(\tau) = \{\theta \in \Theta: D(\Theta||\tau) = f_\theta(\tau)\}$ is the set of ``active'' functions at $\tau$. Since $D(\theta, \bar{\tau})$ is the same for every $\theta \in \Theta$, we get that
$
    \partial D(\Theta||\bar{\tau}) = \text{co}(\{\nabla f_\theta(\bar{\tau})\}_{\theta \in \Theta})
$. For any subgradient $g \in \partial D(\Theta||\bar{\tau})$, we have that 
\begin{equation*}
    D(\Theta||\tau) \geq D(\Theta||\bar{\tau}) + \langle g, \tau - \bar{\tau} \rangle
\end{equation*}
for every $\tau \in \Delta_{K-1}$. So, it is enough to find one such $g$ such that $\langle g, \tau - \bar{\tau} \rangle \geq 0$ for every $\tau \in \text{co}(\Theta)$. Let $\Bar{g} = \frac{1}{N} \sum_{\theta \in \Theta} \nabla f_\theta(\bar{\tau})$, which belongs to $\partial D(\Theta||\bar{\tau})$. Notice that for any $j \in [K]$,
\begin{equation*}
    \Bar{g}(j) = -\frac{1}{N} \sum_{\theta \in \Theta}  \frac{\theta(j)}{\bar{\tau}(j)} = -1.
\end{equation*}
We conclude by observing that for any $\tau \in \Delta_{K-1}$, we have that $\langle \Bar{g}, \tau - \bar{\tau} \rangle = 0$.