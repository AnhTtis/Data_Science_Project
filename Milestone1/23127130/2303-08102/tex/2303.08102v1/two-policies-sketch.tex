The proof follows a similar scheme as before, so we only sketch the main distinctions and defer the full proof to Appendix \ref{appendix-two-policies-proof}. For an arm $j$, we define $z_1(j) = \frac{\sqrt{\theta_1(j)} - \sqrt{\theta_2(j)}}{\sqrt{\theta_1(j)} + \sqrt{\theta_2(j)}}$, and $z_2(j) = -z_1(j)$.
We use two environments $\mu_1$ and $\mu_2$, where 
$ \mu_1(j) = 1/2 - \Delta z_1(j),
$
with $\mu_2(j)$ defined analogously. 
Subsequently, focusing on $\mu_1$ and $\theta_1$, 
%we can lower bound $R_T(\mu_1)$ by
we can bound $R_T(\mu_1)$ from below by
\begin{equation*}
    2 \Delta H^2(\theta_1,\theta_2) \bigg(T - N_{\mu_0}(\theta; T) - T \sqrt{\frac{1}{2}D(P_{\mu_0},P_{\mu_1})}\bigg),
\end{equation*}
whereas we can show that $D(P_{\mu_0}, P_{\mu_1}) \leq 2 c \Delta^2 H^2(\theta_1,\theta_2) T$. To see the latter, it suffices to start from Lemma \ref{low:kl-decomp} and to use that 
$
    d(\mu_0(j),\mu_1(j)) \leq c \Delta^2 z_1(j)^2
$
for sufficiently small $\Delta$, and that 
$
    \sum\nolimits_j \theta_1(j) z_1(j)^2 \leq 2H^2(\theta_1,\theta_2).
$
%What remains then is to tune the value of $\Delta$ to maximize the average of the lower bounds on $R_T(\mu_1)$ and $R_T(\mu_2)$.