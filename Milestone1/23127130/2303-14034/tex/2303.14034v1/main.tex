\documentclass[onecolumn,superscriptaddress,amsmath,amssymb,showpacs,prb]{revtex4-2}

\usepackage{graphicx}
\usepackage{color}
%\usepackage{hyperref}
\usepackage{makecell}
\renewcommand{\phi}{\varphi}

\begin{document}

\title{Effect of nitrogen doping and pressure on the stability of cubic LuH$_3$}

\author{Yang Sun}
	\affiliation{Department of Physics, Xiamen University, Xiamen 361005, China}
	\affiliation{Department of Physics, Iowa State University, Ames, IA 50011, USA}
\author{Feng Zhang}
	\affiliation{Department of Physics, Iowa State University, Ames, IA 50011, USA}
\author{Shunqing Wu}
	\affiliation{Department of Physics, Xiamen University, Xiamen 361005, China}
\author{Vladimir Antropov}
	\affiliation{Department of Physics, Iowa State University, Ames, IA 50011, USA}
\author{Kai-Ming Ho}
	\affiliation{Department of Physics, Iowa State University, Ames, IA 50011, USA}



\begin{abstract}

The report on the near-ambient superconductivity in an nitrogen-doped lutetium hydride has stimulated great interest in this material (Dasenbrock-Gammon et al. 2023). While its superconductivity is still a subject of debate, the structure of the claimed cubic phase remains uncertain. In this work, we study the effect of nitrogen doping and pressure on the energetic and dynamic stability of cubic LuH$_3$. Our findings indicate that both pressure and nitrogen doping can enhance the stability of the cubic LuH$_3$ phase. We propose a Lu$_8$H$_{21}$N structure that exhibits stable phonon, reasonable thermodynamics stability at 1 GPa, and a similar XRD pattern to the experimental data. However, we do not observe electron-phonon coupling in the zone-center phonon modes of these crystal structures.
\end{abstract}

\maketitle
\section{Introduction}
The search for room-temperature superconductors under ambient conditions is a huge scientific challenge that can lead to many applications. More than 50 years ago Ashcroft and Ginzburg already discussed the possibility of having such superconductivity in the metallic phase of hydrogen  \cite{1,2}. Only recently, it became possible to predict and observe new hydrogen-rich metallic phases with such superconductivity  \cite{3}.

High-pressure experiments have had great success in discovering many high-$T_c$ hydrides, such as H$_3$S with $T_c$ above 200 K under 200 GPa \cite{4}, LaH$_{10}$ with $T_c$ $\sim$ 250K at 170GPa  \cite{5,6}, ThH$_{10}$ with $T_c$ $\sim$ 160K at $\sim$170GPa  \cite{7}, CeH$_{10}$ with $T_c$ $\sim$ 115K at 95GPa  \cite{8}, YH$_6$ with  $T_c$ $\sim$ 224K at $\sim$166GPa  \cite{9,10}, YH$_9$ with $T_c$ $\sim$ 262K at $\sim$180GPa \cite{10,11}, CaH$_6$ with $T_c$ $\sim$ 215K at 172GPa  \cite{12}. However, such high pressures are still too extreme for practical applications. Therefore, doping or substituting hydrogen with other light elements such as B, C, and N has been proposed to lower the required high pressure  \cite{13,14,15,16,17,18,19,20,21,CaBH}. 

Recently, Dasenbrock-Gammon et al. reported experimental evidence of superconductivity in N-doped Lutetium hydride near ambient conditions \cite{22}. The material was synthesized at 2 GPa and claimed to achieve the $T_c$ of 274 K at 1 GPa. However, the superconducting behavior of this phase is being questioned  \cite{23,24}, and the crystal structure of the synthesized phase is unclear. Dasenbrock-Gammon et al. described the structure as a cubic LuH$_{3-\delta}$N$_{\epsilon}$, where $\delta$ and $\epsilon$ are unknown values that account for the hydrogen vacancy defect and nitrogen contents, respectively. As the X-ray diffraction pattern is not sensitive to hydrogen and nitrogen, only the lattice parameter for the Lu sublattice was solved, as a cubic structure with a=5.0289(4) $\text{\AA}$. Elemental analysis revealed a weight percent of 0.8-0.9$\%$ N in the materials. However, the cubic LuH$_3$ is known to be unstable under ambient conditions. Therefore, the unknown crystal structure makes the study of superconductivity in this compound difficult. In this work, we use the first-principle calculation to investigate the effects of pressure and N doping on the thermodynamic and dynamical stability of cubic LuH$_3$. Our aim is to find a stable structure model that can explain the experimental data reported in Dasenbrock-Gammon et al.

\section{Methods}
Density functional theory (DFT) calculations were carried out using the projector augmented wave (PAW) method \cite{25} implemented in the VASP code \cite{26,27}. PAW potentials with valence electronic configurations 6s$^2$5d$^1$5p$^6$, 1s$^1$, and 2s$^2$2p$^3$ were used for Lu, H, and N atoms, respectively. The exchange and correlation energy was treated with the generalized gradient approximation (GGA) and parameterized by the Perdew-Burke-Ernzerhof formula (PBE) \cite{28}. A plane-wave basis was used with a kinetic energy cutoff of 520 eV, and the convergence criterion for the total energy was set to 10$^{-8}$ eV. The $\Gamma$-centered Monkhorst-Pack grid was adopted for Brillouin zone sampling with a spacing of $2\pi \times 0.025 ~\text{\AA}^{-1}$. The full Brillouin zone phonon spectrums were computed by the finite displacement method implemented in the Phonopy code  \cite{29} using $3\times3\times3$ supercell for LuH$_3$ (108 atoms) and $2×\times2\times2$ supercell for Lu$_8$H$_{21}$N (240 atoms). The zone-center electron-phonon coupling strength was calculated by the frozen-phonon method developed in \cite{30}.

\section{Results}

\subsection{Lu-H-N phase diagram}

The binary LuN  \cite{31}, LuH$_2$ \cite{32} and LuH$_3$ \cite{33} phases have been synthesized and documented in the experimental ICSD database, but no stable ternary Lu-H-N phase has been reported yet. Computational databases, such as Material Project \cite{34} and OQMD \cite{35}, yield consistent results. As shown in Fig.~\ref{fig:fig1}, the Lu-H-N phase diagram at 1 GPa is relatively simple that only LuN, LuH$_2$, and LuH$_3$ are stable phases. Both LuN and LuH$_2$ have $Fm\bar{3}m$ structures, with Lu occupying the fcc sublattice. In LuN, N occupies the octahedral site, while in LuH$_2$, H occupies the tetrahedral site. The stable LuH$_3$ phase exhibits a rhombohedral structure (noted as r-LuH$_3$) with space group $P\bar{3}c1$, where Lu atoms pack in an hcp sublattice. The cubic LuH$_3$ (noted as c-LuH$_3$) shows a relatively high enthalpy,  85 meV/atom higher than r-LuH$_3$ at 1 GPa (Table~\ref{table:tab1}). 

\begin{figure}[h]
\includegraphics[width=0.6\textwidth]{fig1.pdf}
\caption{\label{fig:fig1} The phase diagram of Lu-N-H system. (b) The crystal structure of LuH$_2$, LuN, and rhombohedral LuH$_3$. }
\end{figure}

\begin{table}[h]
\caption{\label{table:tab1} Enthalpy above the convex hull of LuNH phases at 1GPa.}
\centering
\begin{tabular} {c | c}
\hline
\hline
System       & $\Delta H_{\text{hull}}$ (meV/atom)      \\
\hline
LuH$_2$	&0 \\
\hline
Rhombohedral LuH$_3$&	0 \\
\hline
Cubic LuH$_3$	&85\\
\hline
Lu$_4$H$_{11}$N, N at octahedral site	&278\\
\hline
Lu$_4$H$_{11}$N, N at tetrahedral site	&134\\
\hline
Lu$_8$H$_{23}$N, N at octahedral site	&150\\
\hline
Lu$_8$H$_{23}$N, N at tetrahedral site	&130\\
\hline
Lu$_8$H$_{21}$N, N at tetrahedral site	&31 \\
\hline
\hline
\end{tabular}
\end{table}

\subsection{Pressure effect on cubic LuH$_3$}
We first investigate the pressure effect on the stability of c-LuH$_3$. The phonon spectrums of c-LuH$_3$ at various pressures are shown in Fig. 2. At pressures of 1-5 GPa, the phonon modes show strong instability. As the pressure increases to $\sim$20 GPa, the imaginary phonon modes are significantly reduced. At 25GPa, c-LuH$_3$ does not show any imaginary phonon modes. The phonon density of states shows that the H atoms at the tetrahedral site have much higher frequencies than those at octahedral sites.  At 25 GPa, the enthalpy of c-LuH$_3$ is only 12 meV/atom higher than that of r-LuH$_3$, which is significantly reduced compared to 85 meV/atom at 1 GPa. These results suggest that high-pressure conditions can stabilize c-LuH$_3$ both energetically and mechanically. 


\begin{figure}
\includegraphics[width=0.8\textwidth]{fig2.pdf}
\caption{\label{fig:fig2} The phonon spectrum for cubic LuH$_3$ as a function of pressures. The imaginary modes disappear when the pressure is higher than $\sim$25 GPa. The phonon density of states (Ph. DOS) is shown for the data at 25 GPa. In the atomic structure of cubic LuH$_3$, red shows H at the tetrahedral sites, while green shows H at the octahedral sites. Blue is Lu atoms.}
\end{figure}



\subsection{N doping effect on cubic LuH$_3$}
Although pressure can stabilize LuH$_3$, the condition of 25 GPa is too high compared to the near-ambient condition reported in Dasenbrock-Gammon et al. \cite{22}. Hence, we dope the N in the c-LuH$_3$ to study its effect on the stability at 1 GPa. We first use a conventional c-LuH$_3$ cell  (4 f.u.) and substitute the H at octahedral and tetrahedral sites, respectively, resulting in Lu$_4$H$_{11}$N. At 1 GPa, the enthalpies of both phases above the convex hull are largely positive (Table~\ref{table:tab1}), even higher than c-LuH$_3$. The enthalpy of N at tetrahedral sites is lower than that at the octahedral site, indicating N's preference for the tetrahedral sites of c-LuH$_3$. Doping N at the octahedral site did not significantly alter the structure. However, doping at the tetrahedral site causes a change in the local H positions. As shown in Fig. 3(a), N repels the surrounding H atoms, leading to the formation of short H-H bonds and a local H tetrahedron. The N content in Lu$_4$H$_{11}$N is 1.9 wt.$\%$, which is twice larger than the ones (0.8-0.9 wt.$\%$) reported in Dasenbrock-Gammon et al. \cite{22}. To be closer to experimental condition, we use a $\sqrt{2}\times \sqrt{2} \times 1$ supercell (8 f.u.) and replace one H with N to reduce the doping concentration, yielding Lu$_8$H$_{23}$N. As shown in Table. 1, Lu$_8$H$_{23}$N with N at the tetrahedral site still shows a lower enthalpy than that at the octahedral site. Thus, if N is doped into the c-LuH$_3$, it is expected to occupy the tetrahedral sites. Both Lu$_4$H$_{11}$N and Lu$_8$H$_{23}$N show strong imaginary phonons in Fig.~\ref{fig:fig3} and higher relative enthalpy above the convex hull compared to c-LuH$_3$. Therefore, these phases are still unstable and cannot be used to model the experimentally claimed LuH$_{3-\delta}$N$_{\epsilon}$ phase \cite{22}.




\begin{figure}[h]
\includegraphics[width=0.8\textwidth]{fig3.pdf}
\caption{\label{fig:fig3} The structural change in (a) Lu$_4$H$_{11}$N and (b) Lu$_8$H$_{23}$N due to the N doping at the tetrahedral site at 1 GPa. The connected H-H bonds are the shortened bonds caused by the N substitution. The right panels show the phonon spectrum and phonon density of states of doped structures.}
\end{figure}


By analyzing the phonon instability at the zone center in Lu$_8$H$_{23}$N, we find it’s mainly caused by H atoms with shortened H bonds. To eliminate these instabilities, we introduce vacancies at the H sites that are closely related to these short bonds (circled sites in Fig.~\ref{fig:fig3}b). This generates a model with stoichiometry Lu$_8$H$_{21}$N. After relaxing this model, we find the phonon instability disappears, as shown in Fig.~\ref{fig:fig4}. Moreover, by introducing the defects, the enthalpy above the convex hull decreases to 31 meV/atom (Table~\ref{table:tab1}). Thus, these defects significantly improve both thermodynamic and dynamic stability. As this structure contains 14 tetrahedral sites with 2 vacancies, the N atom can also replace other tetrahedral sites equivalently. It could form solid solutions at finite temperatures, introducing configurational entropy to further lower the free energy. If we estimate the configurational entropy with the ideal solid solution model, the free energy above the convex hull is $\sim$24 meV/atom at 294 K, which is within the room temperature fluctuation. Therefore, from the perspective of energetics and dynamical stabilities, this Lu$_8$H$_{21}$N structure can exist under near-ambient conditions.

\begin{figure}
\includegraphics[width=0.6\textwidth]{fig4.pdf}
\caption{\label{fig:fig4} The phonon spectrum and atomic structure of Lu$_8$H$_{21}$N at 1 GPa.}
\end{figure}

\subsection{Comparison of XRD patterns}
We compare the XRD of Lu$_8$H$_{21}$N structures with the experimental data reported in Ref. \cite{22}. The peak of Lu$_8$H$_{21}$N XRD shows good agreement with experimental XRD without any refinement, although some small differences exist. Firstly, it shows small peak splitting at $2\theta=35^{\circ}, 51^{\circ}$ and $61^{\circ}$. These splitting are relatively small, even less than the FWHM of corresponding experimental peaks. They are mainly due to the change of lattice symmetry. With full DFT relaxation, the stoichiometric Lu$_8$H$_{21}$N model is a tetragonal phase with space group $P\bar{4}m2$ (see crystallographic data in Supplementary Material Table S1). The a/c ratio is 1.382, which deviates from the original $\sqrt{2}$ ratio in the supercell of the cubic lattice. However, if it forms a solid solution at a finite temperature and N is randomly distributed to the tetrahedral sites, the phase should recover the cubic lattice. Secondly, there is a systematical shift of the peak positions of the fully relaxed structure compared to the experimental data. This is due to the lattice parameter difference. The experimental XRD can be solved by the cubic lattice with a=5.0289 $\text{\AA}$, while the fully relaxed cell has c=5.1951 $\text{\AA}$. This may be attributed to the PBE functional that constantly overestimates the cell volume compared to the experimental measurement. To eliminate the effect of these two factors, we change the Lu$_8$H$_{21}$N lattice to experimental one a=b=$5.0289\times\sqrt{2}$ $\text{\AA}$ and c=5.0289 $\text{\AA}$, while keeping atom positions fixed. This leads to an XRD that ma$T_c$hes all the main peaks of the experimental data.

\begin{figure}
\includegraphics[width=0.5\textwidth]{fig5.pdf}
\caption{\label{fig:fig5} Comparison of XRD between experimental data from \cite{22} and Lu$_8$H$_{21}$N model with different lattices. Red arrows indicate the indexed peaks. The remaining peak can be explained by LuN+Lu$_2$O$_3$ as described in  \cite{22}.}
\end{figure}

\subsection{Zone-center electron-phonon coupling strength}
We have demonstrated that both high pressure and N doping can stabilize the c-LuH$_3$ phase. To examine the possibility of superconductivity in this phase, we employ a recently developed frozen phonon method to compute the zone-center electron-phonon coupling (EPC) strength in the present structures. This efficient method can identify the strong EPC candidates in many hydrides because the zone-center EPC shows a strong correlation with the full Brillouin zone EPC in these materials \cite{30}. In Fig. 6, we compute the screened ($\omega$) and unscreened ($\tilde{\omega}$) zone-center phonon frequencies for LuH$_3$ at 1GPa and 25 GPa, as well as Lu$_8$H$_{21}$N at 1 GPa, respectively. The zone-center EPC strength ($\lambda_\Gamma$) can be computed by the relative difference between two frequencies  \cite{30} as
\begin{equation}
\lambda_\Gamma=\frac{\tilde{\omega}^2-\omega ^2}{4\omega ^2}
\label{eq1}.
\end{equation}

We find the screened and unscreened phonon frequencies are almost the same, resulting in $\lambda_\Gamma \sim 0 $ for all these materials. Therefore, the c-LuH$_3$ phase, regardless of whether it is compressed or N-doped, does not show strong EPC in the zone center.


\begin{figure}[h]
\includegraphics[width=0.7\textwidth]{fig6.pdf}
\caption{\label{fig:fig6} Zone-center EPC strength. Upper panels show the screened and unscreened phonon frequencies and lower panels show the zone-center EPC strength.}
\end{figure}

\section{Summary}
In summary, we show that both pressure and nitrogen doping can improve the thermodynamic and structural stability of cubic LuH$_3$. A pressure of ~25 GPa can stabilize the phonon spectrum of the cubic LuH$_3$ and reduce its enthalpy difference with respect to the stable rhombohedral LuH$_3$ to 12 meV/atom. However, this condition is too far from the recently reported experimental condition in Dasenbrock-Gammon et al. We find that a more realistic way to stabilize the cubic LuH$_3$ is to substitute hydrogen atoms at the tetrahedral sites with nitrogen atoms and introduce vacancies to release the local distortion of hydrogen atoms. This can simultaneously stabilize the phonon spectrum and improve thermodynamic stability. We propose the Lu$_8$H$_{21}$N model to address the experimentally synthesized material. In addition, by examining the zone-center electron-phonon coupling, we cannot find any theoretical evidence of strong superconductivity in these structures. 



%\begin{acknowledgments}

%\end{acknowledgments}

\bibliographystyle{apsrev4-2}
\begin{thebibliography}{46}%
\bibitem{1}	N. W. Ashcroft, Metallic Hydrogen: A High-Temperature Superconductor?, Phys Rev Lett 21, 1748 (1968).
\bibitem{2}	V. L. Ginzburg, Superfluidity and Superconductivity in the Universe, Soviet Physics Uspekhi 12, 241 (1969).
\bibitem{3}	L. Boeri et al., The 2021 Room-Temperature Superconductivity Roadmap, J Phys Condens Matter 34, 183002 (2022).
\bibitem{4}	A. P. Drozdov, M. I. Eremets, I. A. Troyan, V. Ksenofontov, and S. I. Shylin, Conventional Superconductivity at 203 Kelvin at High Pressures in the Sulfur Hydride System, Nature 525, 73 (2015).
\bibitem{5}	M. Somayazulu, M. Ahart, A. K. Mishra, Z. M. Geballe, M. Baldini, Y. Meng, V. V. Struzhkin, and R. J. Hemley, Evidence for Superconductivity above 260 K in Lanthanum Superhydride at Megabar Pressures, Phys Rev Lett 122, 027001 (2019).
\bibitem{6}	A. P. Drozdov et al., Superconductivity at 250 K in Lanthanum Hydride under High Pressures, Nature 569, 528 (2019).
\bibitem{7}	D. V. Semenok, A. G. Kvashnin, A. G. Ivanova, V. Svitlyk, V. Y. Fominski, A. V. Sadakov, O. A. Sobolevskiy, V. M. Pudalov, I. A. Troyan, and A. R. Oganov, Superconductivity at 161 K in Thorium Hydride ThH$_{10}$: Synthesis and Properties, Materials Today 33, 36 (2020).
\bibitem{8}	W. Chen, D. V. Semenok, X. Huang, H. Shu, X. Li, D. Duan, T. Cui, and A. R. Oganov, High-Temperature Superconducting Phases in Cerium Superhydride with a $T_c$ up to 115 K below a Pressure of 1 Megabar, Phys Rev Lett 127, 117001 (2021).
\bibitem{9}	I. A. Troyan et al., Anomalous High‐Temperature Superconductivity in YH$_6$, Adv Mater 33, 2006832 (2021).
\bibitem{10}	P. Kong et al., Superconductivity up to 243 K in the Yttrium-Hydrogen System under High Pressure, Nat Comm 12, 5075 (2021).
\bibitem{11}	E. Snider, N. Dasenbrock-Gammon, R. McBride, X. Wang, N. Meyers, K. V. Lawler, E. Zurek, A. Salamat, and R. P. Dias, Synthesis of Yttrium Superhydride Superconductor with a Transition Temperature up to 262 K by Catalytic Hydrogenation at High Pressures, Phys Rev Lett 126, 117003 (2021).
\bibitem{12}	L. Ma et al., High-Temperature Superconducting Phase in Clathrate Calcium Hydride CaH$_6$ up to 215 K at a Pressure of 172 GPa, Phys Rev Lett 128, 167001 (2022).
\bibitem{13}	Y. Hou, B. Li, Y. Bai, X. Hao, Y. Yang, F. Chi, S. Liu, J. Cheng, and Z. Shi, Superconductivity in CeBeH$_8$ and CeBH$_8$ at Moderate Pressures, J Phys Condens Matter 34, 505403 (2022).
\bibitem{14}	T. T. Gai, P. J. Guo, H. C. Yang, Y. Gao, M. Gao, and Z. Y. Lu, Van Hove Singularity Induced Phonon-Mediated Superconductivity above 77 K in Hole-Doped SrB$_3$C$_3$, Phys Rev B 105, 224514 (2022).
\bibitem{15}	X. Li, X. Zhang, A. Bergara, Y. Liu, and G. Yang, Structural and Electronic Properties of Na-B-H Compounds at High Pressure, Phys Rev B 106, 174104 (2022).
\bibitem{16}	H. B. Ding, Y. J. Feng, M. J. Jiang, H. L. Tian, G. H. Zhong, C. L. Yang, X. J. Chen, and H. Q. Lin, Ambient-Pressure High-T$_c$ Superconductivity in Doped Boron-Nitrogen Clathrates La(BN)$_5$ and Y(BN)$_5$, Phys Rev B 106, 104508 (2022).
\bibitem{17}	Z. Zhang, T. Cui, M. J. Hutcheon, A. M. Shipley, H. Song, M. Du, V. Z. Kresin, D. Duan, C. J. Pickard, and Y. Yao, Design Principles for High-Temperature Superconductors with a Hydrogen-Based Alloy Backbone at Moderate Pressure, Phys Rev Lett 128, 047001 (2022).
\bibitem{18}	N. Geng, T. Bi, and E. Zurek, Structural Diversity and Superconductivity in S–P–H Ternary Hydrides under Pressure, J Phys Chem C 126, 7208 (2022).
\bibitem{19}	S. Di Cataldo, W. Von Der Linden, and L. Boeri, Phase Diagram and Superconductivity of Calcium Borohyrides at Extreme Pressures, Phys Rev B 102, 014516 (2020).
\bibitem{20}	S. Di Cataldo, W. von der Linden, and L. Boeri, First-Principles Search of Hot Superconductivity in La-X-H Ternary Hydrides, npj Comput Mater 8, 2 (2022).
\bibitem{21}	F. Zheng, Y. Sun, R. Wang, Y. Fang, F. Zhang, S. Wu, C.-Z. Wang, V. Antropov, and K.-M. Ho, Superconductivity in the Li-B-C System at 100 GPa, Phys Rev B 107, 014508 (2023).
\bibitem{CaBH}	 S. Di Cataldo and L. Boeri, Metal borohydrides as ambient-pressure high-$T_c$ superconductors, Phys. Rev. B 107, L060501 (2023).
\bibitem{22}	N. Dasenbrock-Gammon et al., Evidence of Near-Ambient Superconductivity in a N-Doped Lutetium Hydride, Nature 615, 244 (2023).
\bibitem{23}	P. Shan, N. Wang, X. Zheng, Q. Qiu, Y. Peng, and J. Cheng, Pressure-Induced Color Change in the Lutetium Dihydride LuH$_2$, Chin Phys Lett (2023).
\bibitem{24}	X. Ming, Y.-J. Zhang, X. Zhu, Q. Li, C. He, Y. Liu, B. Zheng, H. Yang, and H.-H. Wen, Absence of Near-Ambient Superconductivity in LuH$_{2\pm\text{x}}$N$_y$, arXiv:2303.08759 (2023).
\bibitem{25}	P. E. Blöchl, Projector Augmented-Wave Method, Phys Rev B 50, 17953 (1994).
\bibitem{26}	G. Kresse and J. Furthmüller, Efficiency of Ab-Initio Total Energy Calculations for Metals and Semiconductors Using a Plane-Wave Basis Set, Comput Mater Sci 6, 15 (1996).
\bibitem{27}	G. Kresse and J. Furthmüller, Efficient Iterative Schemes for $Ab~Initio$ Total-Energy Calculations Using a Plane-Wave Basis Set, Phys Rev B 54, 11169 (1996).
\bibitem{28}	J. P. Perdew, K. Burke, and M. Ernzerhof, Generalized Gradient Approximation Made Simple, Phys Rev Lett 77, 3865 (1996).
\bibitem{29}	A. Togo and I. Tanaka, First Principles Phonon Calculations in Materials Science, Scr Mater 108, 1 (2015).
\bibitem{30}	Y. Sun, F. Zhang, C.-Z. Wang, K.-M. Ho, I. I. Mazin, and V. Antropov, Electron-Phonon Coupling Strength from $Ab~Initio$ Frozen-Phonon Approach, Phys Rev Mater 6, 074801 (2022).
\bibitem{31}	G. L. Olcese, Interconfiguration Fluctuation of Cerium in CeN as a Function of Temperature and Pressure, J Physics F: Metal Physics 9, 569 (1979).
\bibitem{32}	A. Pebler and W. E. Wallace, Crystal Structures of Some Lanthanide Hydrides, J Phys Chem 66, 148 (1962).
\bibitem{33}	M. Mansmann and W. E. Wallace, The Structure of HoD$_3$, Journal de Physique 25, 454 (1964).
\bibitem{34}	A. Jain et al., Commentary: The Materials Project: A Materials Genome Approach to Accelerating Materials Innovation, APL Mater 1, 011002 (2013).
\bibitem{35}	S. Kirklin, J. E. Saal, B. Meredig, A. Thompson, J. W. Doak, M. Aykol, S. Rühl, and C. Wolverton, The Open Quantum Materials Database (OQMD): Assessing the Accuracy of DFT Formation Energies, npj Comput Mater 1, 15010 (2015).



\end{thebibliography}%


%%% Supplementary
\pagebreak
\widetext
\begin{center}
\textbf{\large Supplemental Material } 
\end{center}


\onecolumngrid
%%rename with s
\setcounter{equation}{0}
\setcounter{figure}{0}
\setcounter{table}{0}
%\setcounter{page}{1}
\makeatletter
\renewcommand{\theequation}{S\arabic{equation}}
\renewcommand{\thefigure}{S\arabic{figure}}
\renewcommand{\bibnumfmt}[1]{[S#1]}
\renewcommand{\citenumfont}[1]{S#1}
\renewcommand{\thesection}{S\arabic{section}}
%%

\setcounter{section}{0}
%\section{Crystallographic data of fully relaxed Lu8H21N at 0GPa}

\begin{table}[h]
\begin{ruledtabular}

%\centering
\begin{tabular} {c | c| c| c| c| c c c}

Phase    & Space group & Lattice parameter & Wyckoff site & Atoms & x & y & z     \\
\hline
Lu$_8$H$_{21}$N & $P\bar{4}m2$ & \makecell{a= 7.1778 $\text{\AA}$ \\ b= 7.1778 $\text{\AA}$\\ c= 5.1951 $\text{\AA}$ \\ $\alpha=90^{\circ}$ \\ $\beta=90^{\circ}$ \\$\gamma=90^{\circ}$ } & \makecell{4k\\	4j\\	4j\\	4h\\	4i\\	1c\\	2g\\	4k\\	1d\\	1a\\	1b}	 & \makecell{Lu1\\	Lu2\\	H1\\	H2\\	H3\\	H4\\	H5	\\H6	\\H7\\	H8	\\N1	} & \makecell{0.251\\	0.737\\	0.683\\	0.240\\	0.749	\\0.500	\\0.000	\\0.838	\\0.000	\\0.000	\\0.500} & \makecell{0.500\\	0.000\\	0.000\\	0.240\\	0.749\\	0.500\\	0.500\\	0.500	\\0.000	\\0.000\\	0.500} & \makecell{0.757\\	0.730\\	0.317	\\0.000	\\0.500	\\0.500	\\0.009	\\0.343\\	0.500\\	0.000 \\ 0.000} \\

\end{tabular}
\end{ruledtabular}
\caption{\label{table:tabs1} Crystallographic data of fully relaxed Lu$_8$H$_{21}$N  at 0 GPa.}
\end{table}

\end{document}