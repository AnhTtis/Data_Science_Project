\vspace{-5mm}

\section{Related Work}

\label{sec:related_work}
\noindent\textbf{Dense Queries with One-To-Many Assignment.}  One-stage detectors such as RetinaNet\cite{lin2017focal} and FCOS \cite{tian2019fcos} use densely distributed queries for regression and classification.The same manner is also applied to the region proposal network (RPN) of multi-stage models \cite{ren2015faster, cai2018cascade}. 
And one-to-many assignments are a common practice for these traditional detectors. Despite the fast development of one-to-many assignments from static label assignments (such as IOU-based~\cite{lin2017focal, ren2015faster, cai2018cascade} and center-based ones~\cite{tian2019fcos, kong2020foveabox}) to prediction-aware dynamic label assignments \cite{zhang2019freeanchor, zhu2020autoassign, ge2021ota, feng2021tood, li2022dual, chen2022diffusiondet}, these strategies are also long criticized for they pair each ground truth with multiple queries and thus require additional postprocessing to remove duplicate predictions at inference, which prevents the pipeline from being end-to-end. \\
\noindent\textbf{Sparse Queries with One-To-One Assignment.} DETR \cite{carion2020end} designs a small set of learned positional embeddings that represent the position in an image to focus on. These queries are then optimized with one-to-one assignments, making an end-to-end pipeline. Sparse R-CNN~\cite{sun2021sparse} reformulates queries in the traditional R-CNN framework as a bounding box and its corresponding embedding. Anchor DETR~\cite{wang2021anchor} provides the correspondence between anchor points and query position. DAB-DETR ~\cite{liu2021dab} explicitly learns a set of 4-D anchor boxes as queries. Though the formulation of queries varies, they share the same core idea of sparse queries and one-to-one assignments. Therefore, a low recall rate is an expected issue for these detectors. \\
\noindent\textbf{Dense Queries with One-To-One Assignment.} Both DeFCN \cite{wang2021end} and OneNet \cite{sun2021makes} try to integrate one-to-one assignment with dense queries. %Both methods found that the classification cost does matter. 
Despite their competitive performance compared to FCOS \cite{tian2019fcos}, there is still a clear performance gap with recent detectors with dynamic one-to-many assignment strategies~\cite{kim2020probabilistic, zhu2020autoassign, ge2021ota, feng2021tood, li2022dual}. It is the optimization difficulty of similar queries under one-to-one assignments that accounts for the performance gap. Efficient DETR~\cite{yao2021efficient}, and Two-Stage Deformable DETR \cite{zhu2020deformable} can also be regarded as a multi-stage version of this paradigm. Although DINO~\cite{zhang2022dino}, Group DETR~\cite{chen2022group}, and H-DeformableDETR~\cite{jia2022detrs} have introduced more positive samples to speed up convergence, the hindrance effect between similar queries and one-to-one assignments still remains unrevealed.