In this work, we presented an \gls{mor} concept for parametric time-dependent problems defined in deforming domains that are allowed to even entail spatial topology changes. There are two main building blocks in this approach. First, we make use of the time-continuous space-time setting. Here, the \gls{csst}-\gls{fem} leads to fixed finite-dimensional subspaces for the entire space-time domain, implicitly accounting for current domain deformations. Working on these fixed subspaces, we make use of a projection-based \gls{mor} technique based on \gls{pod}. In contrast to other \gls{mor} methods applied to deforming domain problems, this particular approach proposed in this paper can be applied in a straightforward manner. Taken together, we argued that it can reduce the computational complexity for the aforementioned class of problems and, at the same time, maintain the desired level of accuracy in the results.
\par
To confirm this claim, we investigated two representative test cases from the fields of engineering and biomedical applications. In particular, we carried out error and performance analyses for both cases. The first test case was composed of a two-dimensional valve-like geometry containing a moving plug. In the resulting three-dimensional space-time domain, we considered the flow of plastics melt, which occurs in many polymer processing techniques. The parametric character of the problem originated from incorporating uncertainty in the viscosity model, i.e., fluctuations in the model parameters. The error analysis showed that we are able to reduce the original model effectively while keeping control over the magnitude of the error in the relevant physical fields. Moreover, a significant speed-up in terms of computational time was proven.
By means of the second test case, we extended our analysis to four-dimensional space-time domains that arise when considering three-dimensional bodies in space. Specifically, we studied blood flow in an artery-like volume that undergoes compression, yielding the domain deformation. The problem was varied via a parametric prescribed inflow velocity. The results of the error and performance analysis confirmed the findings from the previous test case and demonstrates the applicability of the approach, also when interested in spatially three-dimensional problems.
\par
Overall, the approach presented here extends the collection of existing \gls{mor} techniques with a simple but elegant approach for deforming domain problems. Despite the fact that it shows methodologically favorable characteristics and yields plausible results for the presented test cases, practical difficulties may arise. For example, in three-dimensional application cases, the generation of the required four-dimensional meshes can be challenging and limits the applicability as discussed in~\Cref{subsec:methodologicalBackgroundFOM}. Furthermore, the nature of the resulting reduced system may influence and potentially limit the speed-up that can be realized. Especially, the latter observation remains to be elucidated. Nevertheless, the proposed approach is able to make the advantages of \gls{mor} accessible for the class of complex problems that include domain deformations, potentially with spatial topology changes.
