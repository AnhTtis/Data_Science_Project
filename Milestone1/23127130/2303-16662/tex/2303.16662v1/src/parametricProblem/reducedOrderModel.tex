Now that the \gls{fom} has been defined, we can turn to the \gls{mor}. We will start with a description of the underlying ideas of \gls{mor} in \Cref{subsec:methodologicalBackgroundROM} before the \gls{rom} is eventually constructed via \gls{pod} with subsequent projection in \Cref{subsec:discreteFormulationROM}.

\subsection{Methodological Background for the Reduced-Order Model}
\label{subsec:methodologicalBackgroundROM}
In the context of \gls{mor} for numerical schemes, one can distinguish between \textit{interpolation}- and \textit{projection}-based approaches \cite{Manzoni2012}. Methods of the former type try to directly capture input-output relations on the basis of data that comes from numerical simulations or measurements. The latter project the underlying governing equations onto a reduced space that has been constructed before. The class of projection-based methods further entails \textit{certified \gls{rb}} methods \cite{Hesthaven2015} or \textit{\gls{pod}-projection} methods \cite{Chinesta2011,Hesthaven2015,Manzoni2012}. It is also worthwhile to mention the \textit{\gls{pgd}}~\cite{Chinesta2011} here.
\par
In view of the fact that we would like to focus on the class of unsteady problems in deforming domains, one can note that the application of \gls{mor} techniques requires some thorough treatment in this case.
\par
To account for the unsteady character of the problem in the reduction process, there exist two alternative ways of handling time \cite{Glas2017}. The first one follows the classical time-stepping approach, leading to so-called greedy or \gls{pod}-greedy methods, which have been applied to linear as well as to nonlinear problems \cite{Drohmann2012, Fick2018, Grepl2005a, Grepl2012, Haasdonk2008, Haasdonk2013, Sleeman2022}. The second one is rather based on space-time formulations for which reduction techniques for steady problems have been extended appropriately \cite{Urban2012, Urban2013, Tamma2018, Yano2014a, Yano2014b}. This perspective has been used to tackle, e.g., time-dependent optimal control problems \cite{Strazzullo2020, Strazzullo2022}. In addition, a space-time \gls{rom} for sub-intervals of the total simulation time has been presented in \cite{Fritzen2018a}. In distinction to the former methods, we will apply the \gls{pod}-projection approach to the time-continuous space-time formulation stated beforehand.
\par
The construction of a \gls{rom} for deforming domain problems is usually quite involved and requires some careful treatment of the deformations applied to the computational mesh \cite{Anttonen2001,Forti2014}. This is due to the fact that the \gls{fem} function spaces are inherently linked with the geometry of the underlying grid. A strategy based on a mapping functional to relate the time-dependent solution in one- and two-dimensional deforming domains to fixed reference domains has been presented in \cite{Izadi2013}. Furthermore, temporally-local eigenfunctions have been used for \gls{mor} in \cite{Narasingam2017}. 
To this respect, the \gls{csst} approach (cf. \Cref{subsec:discreteFormulationFOM}) offers an appealing but straightforward alternative, since all deformations --- as long as they are prescribed or known a-priori --- are already integrated in the computational mesh and, thereby, considered in the original function spaces. Also, it is worthwhile to note that this approach even works in the presence of spatial topology changes and for two- and three-dimensional spatial domains without any further adaptions.

\subsection{Discrete Formulation for the Reduced-Order Model}
\label{subsec:discreteFormulationROM}
Before we can construct an efficient \gls{rom} whose complexity is independent of the \gls{fom}, we will have to address the non-linearity appearing in our problem through the viscosity model as well as through the formulation of the stabilization parameter.  
For this purpose, we make use of the \gls{eim} \cite{Barrault2004, Grepl2007} here and introduce the following approximations for the viscosity and the stabilization parameter:
\begin{align*}
    \visc\left(\trialVelocityDiscrete, \paramVec\right) 
    \approx
    \sum_{q=1}^{Q_{\visc}} c_q^{\visc}\lp\trialVelocityDiscrete, \paramVec\rp h_q^{\visc}\lp\x\rp,
    &&
    \tMom%\left(\trialVelocityDiscrete, \paramVec\right) 
    \approx
    \sum_{q=1}^{Q_{\tau}} c_q^{\tau}\lp\trialVelocityDiscrete, \paramVec\rp  h_q^{\tau}\lp\x\rp
    .
\end{align*}
Note that each of the $Q_{\star}$ terms consists of parameter-dependent coefficients $c_q^{\star}\lp\trialVelocity, \paramVec\rp$ and parameter-independent basis functions $h_q^{\star}\lp\x\rp$, where $\star\in\{\visc,\tau\}$.
Consequently, the viscous term is replaced by
\begin{align}
    \bilinearFormViscousStressParam{\weightVelocityDiscrete}{\trialVelocityHomDiscrete}{\trialVelocityDiscrete}{\paramVec}_{\domainSpaceTime}
    \approx
    \sum_{q=1}^{Q_{\visc}} c_q^{\visc}\lp\trialVelocityDiscrete, \paramVec\rp 
    a_q\lp\weightVelocityDiscrete, \trialVelocityHomDiscrete\rp_{\domainSpaceTime},
\end{align}
with
\begin{align}
    a_q\lp\weightVelocityDiscrete, \trialVelocityHomDiscrete\rp_{\domainSpaceTime}
    =
    \intDomainSpaceTime
    {
        \ros{\weightVelocityDiscrete} : 2 h_q^{\visc}\lp\x\rp \ros{\trialVelocityHomDiscrete}
    }.
\end{align}
Moreover, the stabilization term $s^{\trialVelocityHom}$ is approximated via
\begin{align}
    s^{\trialVelocityHom}\lp\weightPressureDiscrete,\trialVelocityHomDiscrete;\trialVelocityDiscrete, \paramVec\rp_{\domainSpaceTime}
    \approx
    \sum_{q=1}^{Q_{\tau}} c_q^{\tau}\left(\trialVelocityDiscrete, \paramVec\right)
    s^{\trialVelocityHom}_q\lp\weightPressureDiscrete,\trialVelocityHomDiscrete\rp_{\domainSpaceTime},
\end{align}
where
\begin{align}
    s^{\trialVelocityHom}_q\lp\weightPressureDiscrete,\trialVelocityHomDiscrete\rp_{\domainSpaceTime}
    =
    \sum_{e} \intElementSpaceTime{\basisFunctionEIM{q}{}^{\tau}\lp\x\rp \frac{1}{\rho}
    \lp
    -\gr{\weightPressureDiscrete}
    \rp
    \cdot
    \lp
    \rho \ddt{\trialVelocityHomDiscrete}
    \rp
    }.
\end{align}
Analogously, the remaining stabilization terms 
$s^{\vek{l}}
    (\weightPressureDiscrete,
    \liftingVelocityDiscrete;
    \trialVelocityDiscrete,
    \paramVec)_{\domainSpaceTime}$ 
    and
$s^{\trialPressure}(
    \weightPressureDiscrete,
    \trialPressureDiscrete;
    \trialVelocityDiscrete
    \paramVec)_{\domainSpaceTime}$ 
can be approximated using parameter-independent terms
$s_q^{\vek{l}}
    (\weightPressureDiscrete,
    \liftingVelocityDiscrete)_{\domainSpaceTime}$
    and
$s_q^{\trialPressure}(
    \weightPressureDiscrete,
    \trialPressureDiscrete)_{\domainSpaceTime}$,
respectively. As a consequence, all the matrices and vectors in the algebraic system of the \gls{fom} relying on these forms, i.e., $\matrixViscousStress$, $\matrixStabMomStokesTemporal$, $\matrixStabMomStokes$, $\vectorRHSVelocityNonlinear$, and $\vectorRHSStabMomStokesTemporal$ are replaced by approximations using $\matrixViscousStress^q$, $\matrixStabMomStokesTemporal^q$, $\matrixStabMomStokes^q$, $\vectorRHSVelocityNonlinear^q$, and $\vectorRHSStabMomStokesTemporal^q$ corresponding to the respective terms above. For example, this means $\matrixViscousStress^q = \lb \matrixViscousStressSymbol^q_{i,j}\rb,
        \text{ with } 
        \matrixViscousStressSymbol^q_{i,j} 
        = 
        a_q\lp\basisFunctionVelocityFOM{i},\basisFunctionVelocityFOM{j}\rp_{\domainSpaceTime}$.
\bigskip\par
To perform the projection step later, a basis spanning the reduced spaces is needed. To that end, we apply the \gls{pod} using the method of snapshots~\cite{Sirovich1987}, i.e., solutions of the \gls{fom}. In particular, this is done individually for the homogeneous velocity $\trialVelocityHomDiscrete$ and the pressure $\trialPressureDiscrete$ leading to the \textit{reduced finite-dimensional function spaces} $\weightingSpaceVelocityReduced \subset \weightingSpaceVelocityDiscrete$ and $\weightingSpacePressureReduced \subset \trialSpacePressureDiscrete$.
As a result of the \gls{pod}, we obtain $\nBasisVelocityHomROM$ and $\nBasisPressureROM$ basis functions for the reduced representation $(\trialVelocityHomReduced,\trialPressureReduced)$ of the homogeneous velocity and pressure field, respectively. To account for the Dirichlet boundary conditions, the basis for $\weightingSpaceVelocityReduced$ is augmented with the lifting function(s) $\liftingVelocityDiscrete$ yielding the reduced space $\trialSpaceVelocityReduced \subset \trialSpaceVelocityDiscrete$ for the reduced velocity $\trialVelocityReduced$. Therefore, we use $\nBasisVelocityROM \ge \nBasisVelocityHomROM$ to denote the number of basis functions for $\trialVelocityReduced$. We sort the basis functions in descending order of significance --- indicated by the magnitude of the corresponding eigenvalues --- while the lifting functions are always leading to ensure that the Dirichlet boundary conditions are met, even if we only use a subset of these basis functions.
\par
For all basis functions, we collect the coefficients with respect to the \gls{fom} function spaces in the so-called \textit{basis function matrices} $\basisFunctionMatrixVelocity\in\mathbb{R}^{\nBasisVelocityFOM \times \nBasisVelocityROM}$ and $\basisFunctionMatrixPressure\in\mathbb{R}^{\nBasisPressureFOM \times \nBasisPressureROM}$. These matrices are multiplied with those from the algebraic system of the \gls{fom} given in \Cref{eq:algebraicSystemFOM}, which yields the projection of the corresponding operators onto the reduced space. Thus, one can formulate the algebraic system for the vectors of unknowns $\velocityDOFVectorReduced\in\mathbb{R}^{\nBasisVelocityROM}$ and $\pressureDOFVectorReduced\in\mathbb{R}^{\nBasisPressureROM}$, where $\nBasisVelocityROM$ and $\nBasisPressureROM$ are the number of unknowns of the \gls{rom}. Key assumption for an effective reduction is that it holds that $\nBasisROM = \nBasisVelocityROM + \nBasisPressureROM \ll \nBasisFOM = \nBasisVelocityFOM + \nBasisPressureFOM$. The system finally reads:
\begin{align}
    \lb
    \begin{array}{cc}
        \matrixTemporalReduced + \matrixViscousStressReducedParam
         & -\matrixPressureReducedTrans
        \\
        \matrixPressureReduced + \matrixStabMomStokesTemporalReducedParam& \matrixStabMomStokesReducedParam
    \end{array}
    \rb
    \lb
    \begin{array}{c}
        \velocityDOFVectorReducedParam\\
        \pressureDOFVectorReducedParam
    \end{array}
    \rb&
    \nonumber
    \\
    =
    \lb
    \begin{array}{c}
        \vectorRHSTemporalReduced +
        \vectorRHSVelocityReduced + \vectorRHSVelocityNonlinearReducedParam \\
        \vectorRHSPressureReduced
        + \vectorRHSStabMomStokesTemporalReducedParam
    \end{array}
    \rb&,
    \label{eq:ROMStokesShearThinningDeformingDomainsAlgebraicSystem}
\end{align}
where the \gls{lhs} matrices read
\begin{align*}
    \begin{aligned}[t]
        \matrixTemporalReduced
        &=
        \basisFunctionMatrixVelocityTrans
        \matrixTemporal
        \basisFunctionMatrixVelocity,
        \\
        \matrixPressureReduced
        &=
        \basisFunctionMatrixPressureTrans
        \matrixPressure
        \basisFunctionMatrixVelocity,
    \end{aligned}
    \quad
    \begin{aligned}[t]
        \matrixViscousStressReducedParam
        &=
        \sum_{q=1}^{Q_{\visc}}
        \coeffEIM{q}^{\visc}\lp\trialVelocityDiscrete,\paramVec\rp
        \mat{A}^q_{\superscriptROM},
        \text{ and }
        \mat{A}^q_{\superscriptROM}
        =
        \basisFunctionMatrixVelocityTrans
        \mat{A}^q
        \basisFunctionMatrixVelocity,
        \\
        \matrixStabMomStokesTemporalReducedParam
        &=
        \sum_{q=1}^{Q_{\tau}}
        \coeffEIM{q}^{\tau}\lp\trialVelocityDiscrete,\paramVec\rp
        \matrixStabMomStokesTemporalReduced^q,
        \text{ and }
        \matrixStabMomStokesTemporalReduced^q
        =
        \basisFunctionMatrixPressureTrans
        \matrixStabMomStokesTemporal^q
        \basisFunctionMatrixVelocity,
        \\
        \matrixStabMomStokesReducedParam
        &=
        \sum_{q=1}^{Q_{\tau}}
        \coeffEIM{q}^{\tau}\lp\trialVelocityDiscrete,\paramVec\rp
        \mat{S}^q_{\superscriptROM},
        \text{ and }
        \mat{S}^q_{\superscriptROM}
        =
        \basisFunctionMatrixPressureTrans
        \mat{S}^q
        \basisFunctionMatrixPressure,
    \end{aligned}
\end{align*}
and the \gls{rhs} vectors are given as
\begin{align*}
    \begin{aligned}[t]
        \vectorRHSTemporalReduced
        &=
        \basisFunctionMatrixVelocityTrans
        \vectorRHSTemporal,
        \\
        \vectorRHSVelocityReduced
        &=
        \basisFunctionMatrixVelocityTrans
        \vectorRHSVelocity,
        \\
        \vectorRHSPressureReduced
        &=
        \basisFunctionMatrixPressureTrans
        \vectorRHSPressure,
    \end{aligned}
    \quad
    \begin{aligned}[t]
        \vectorRHSVelocityNonlinearReducedParam
        &=
        \sum_{q=1}^{Q_{\visc}}
        \coeffEIM{q}^{\visc}\lp\trialVelocityDiscrete,\paramVec\rp
        \vectorRHSVelocityNonlinearReduced^q,
        \text{ and }
        \vectorRHSVelocityNonlinearReduced^q
        =
        \basisFunctionMatrixVelocityTrans
        \vectorRHSVelocityNonlinear^q,
        \\
        \vectorRHSStabMomStokesTemporalReducedParam
        &=
        \sum_{q=1}^{Q_{\tau}}
        \coeffEIM{q}^{\tau}\lp\trialVelocityDiscrete,\paramVec\rp
        \vectorRHSStabMomStokesTemporalReduced^q,
        \text{ and }
        \vectorRHSStabMomStokesTemporalReduced^q
        =
        \basisFunctionMatrixPressureTrans
        \vectorRHSStabMomStokesTemporal^q.
    \end{aligned}
\end{align*}
\Cref{tab:dimensionsROM} gives an overview over the dimensions of the matrices and vectors for the \gls{rom}. Due to the reduced dimensions, the solution of the algebraic system of the \gls{rom} for any new parameter sample $\paramVec$ is possible with significantly less computational resources. Thanks to the \gls{eim}, the same holds for the assembly process of this system. Note, however, that due to the specific implementation of the full-order solver, we keep the dependency on $\trialVelocityDiscrete$ in the \gls{eim} coefficients.
\begin{table}
    \centering
    \begin{tabular}{lclc}
        \toprule
        \acrshort{lhs} Matrices & Dimensions & \acrshort{rhs} Vectors & Dimensions\\
        \midrule
        $\matrixTemporalReduced$, $\matrixViscousStressReduced$ & $\mathbb{R}^{\nBasisVelocityROM \times \nBasisVelocityROM}$ & $\vectorRHSTemporalReduced$, $\vectorRHSVelocityReduced$, $\vectorRHSVelocityNonlinearReduced$ & $\mathbb{R}^{\nBasisVelocityROM}$  \\
        $\matrixPressureReduced$, $\matrixStabMomStokesTemporalReduced$ & $\mathbb{R}^{\nBasisPressureROM \times \nBasisVelocityROM}$ & $\vectorRHSPressureReduced$, $\vectorRHSStabMomStokesTemporalReduced$ & $\mathbb{R}^{\nBasisPressureROM}$  \\
        $\matrixStabMomStokesReduced$ & $\mathbb{R}^{\nBasisPressureROM \times \nBasisPressureROM}$ \\
        \bottomrule
    \end{tabular}
    \caption{Dimensions of \acrshort{lhs} matrices and \acrshort{rhs} vectors for the \acrshort{rom}.}
    \label{tab:dimensionsROM}
\end{table}