% GENERAL BACKGROUND
The use of simulation-based methods is nowadays widespread in scientific and engineering applications. Commonly, these methods are intended to serve purposes like enhanced insight and prediction capabilities with respect to the processes under investigation. In this way, they can be useful tools in the context of product design or optimization procedures. Furthermore, they may also help to quantify or enhance the reliability of a given process by techniques such as \gls{uq}. Lastly, they further allow to be integrated into ongoing operations, e.g., as digital twins or to set up optimal control.
\bigskip\par
% SPECIFIC BACKGROUND
Arising in the context of simulation-based methods, we will focus on two specific types of complexity in this work. First, we consider applications that show a \textit{transient} behavior involving a \textit{deforming domain} and, potentially, even \textit{topology changes}. These aspects need to be treated appropriately by the computational approach to cover all the relevant effects.
In the following, we restrict ourselves to methods that make use of a computational mesh although there exists a variety of further simulation methods. In the case discussed here, the deformation of the domain requires methods that can appropriately handle both the computational mesh and the unsteady solution field. Existing methods will be discussed in more detail in \Cref{subsec:methodologicalBackgroundFOM}.
The second kind of complexity is related to the \textit{expense} --- in terms of computational resources and time --- needed for an evaluation of the computational model. Especially the design and development phase of a process or a product may entail the assessment of various operating points or configurations, the optimization of process settings or component features, as well as uncertainty regarding involved quantities. In all cases, one ends up with a so-called many query scenario in which a great number of model evaluations needs to be performed. Furthermore, the integration into an automatic control environment may demand fast feedback from the simulation model. Typically, one can formulate the involved problems in a parametric manner where each of the model evaluations is characterized by a certain sample of parameter values.
In these situations, employing the original model, which is often referred to as \textit{\gls{fom}}, for each sample may easily exceed available resources or required feedback times. Here, the application of \gls{mor} can provide a remedy. Based on the original model, a \textit{\gls{rom}} with decreased computational complexity is constructed, while keeping its accuracy in the desired range. Common \gls{mor} techniques will be presented later in \Cref{subsec:methodologicalBackgroundROM}.
\bigskip\par
% MAIN POINT OF THIS PAPER
The work presented here addresses both types of complexity described above and entails a \gls{mor} approach for problems that are characterized by a transient solution field in a deforming domain with possibly changing topology. Regarding the question of deforming domain problems, we rely on the \textit{time-continuous space-time} setting. To significantly decrease the computational demands and/or response times of the underlying model in a next step, we apply a \textit{projection-based} \gls{mor} technique for which we make use of \gls{pod}. Key of the proposed approach is the particular combination of this \gls{mor} technique with the choice of a time-continuous space-time formulation. This connection allows us to construct a corresponding \gls{rom} for the aforementioned class of problems in a straightforward manner. In this way, the benefits of \gls{mor} are made easily accessible even when dealing with deforming domain problems that involve an unsteady solution and, if necessary, changes in the spatial topology. Alternative \gls{mor} approaches that exist and are applicable to transient or deformation-driven scenarios will also be discussed later, i.e., in \Cref{subsec:methodologicalBackgroundROM}.
\par
As examples for the type of problems in focus, we consider the simulation of two specific transient fluid flow scenarios like they may appear in engineering or biomedical applications. The domain deformation in these examples results from a moving valve plug or the narrowing of flexible artery walls. Furthermore, the parameterization is induced through a variation of the material properties of the fluid or of the boundary conditions.
\bigskip\par
% PAPER ORGANIZATION
The remainder of the article is structured as follows: in \Cref{sec:parametricFormulation}, we derive the parametric formulation for fluid flow problems in deforming domains using the space-time perspective. Next, \Cref{sec:fullOrderModel} contains the description of the respective \gls{fom}, which will be based on the \gls{fem}. Subsequently, the construction of the corresponding \gls{rom} using \gls{pod} and projection is outlined in \Cref{sec:reducedOrderModel}. Results for the two fluid flow test cases covering three- and four-dimensional space-time domains are presented in \Cref{sec:numericalResults} to illustrate the aptitude of the approach introduced in this work. Finally, our findings are summarized and discussed in \Cref{sec:conclusion}. 
