%\documentclass[12pt,thmsb,a4paper]{article}
\documentclass[12pt,a4paper]{article}
%%%%%%%%%%%%%%%%%%%%%%%%%%%%%%%%%%%%%%%%%%%%%%%%%%
\usepackage{mathrsfs}
\usepackage{graphics}
\usepackage{mathptmx}
\usepackage{times}
\usepackage{epstopdf}
\usepackage{color}
\usepackage{enumerate}
\usepackage{float}
%\usepackage{AMSFonts}
\usepackage{indentfirst}
\usepackage{multirow}
\usepackage[noadjust]{cite}
%\usepackage{natbib}
%\usepackage{doi}
%\usepackage{natbib}
%\usepackage{cases}
%\usepackage{mathbbold}
%\usepackage{amsthm,amsmath,amssymb}
\usepackage{amsthm,amsmath,amssymb,amsfonts}
\usepackage{psfrag}
\usepackage{epsfig}
\usepackage{graphicx,subfigure}
\usepackage{multirow}
\usepackage{setspace}
\usepackage{caption}
\usepackage{booktabs} %需要加载宏包{booktabs}
%\usepackage[linesnumbered]{algorithm}
\usepackage{algorithm}
\usepackage[noend]{algpseudocode}
\usepackage{dsfont}

\usepackage{url}
\usepackage{tikz}
\usepackage{subeqnarray}
\usepackage{cases}
\usepackage{color}

\usepackage[shortlabels]{enumitem}
\usepackage{exscale}
\usepackage{relsize}
\usepackage{bm}
\usepackage{enumerate}
\usepackage{enumitem}




\baselineskip=12pt
\normalbaselineskip=\baselineskip
\renewcommand{\baselinestretch}{1.2}
\hoffset=-2.75cm
\topmargin=-2cm
\textheight=25cm
\textwidth=18cm
\footskip=0.8cm
\oddsidemargin=1.8cm
\evensidemargin=1.8cm
\marginparwidth=1.8cm
\parindent=15pt
\setlength{\baselineskip}{24pt}
\renewcommand{\thefootnote}{\fnsymbol{footnote}}
%\newtheorem{theorem}{Theorem}
%\newtheorem{definition}{Definition}
%\newtheorem{corollary}{Lemma}
%\newtheorem{example}{Example}

%\newtheorem{hypothesis}{Hypothesis}
%\newtheorem{lemma}{Lemma}
%\newtheorem{remark}{Remark}
%\newtheorem{algorithm}{Algorithm}
%\newtheorem{proposition}{Proposition}
%\newtheorem{myDef}{Definition}
\newtheorem{myTheo}{Theorem}


%\newtheorem{thm}{Theorem}[section] %如果不采用章节号做前缀,则不用[section]
\newtheorem{myDef}[myTheo]{Definition} %这句定义使得环境和thm共享编号
\newtheorem{lemma}[myTheo]{Lemma} %这句定义使得lem环境和thm共享编号
\newtheorem{myCollo}[myTheo]{Corollary}
\newtheorem{remark}[myTheo]{Remark}
%\newtheorem{lemma}{Lemma}
\newtheorem{myPro}[myTheo]{Proposition}
\newtheorem{assumption}[myTheo]{Assumption}
\newtheorem{property}[myTheo]{Property}
%\newtheorem{algorithm}{Algorithm}
\def\proof{\par{ \textbf{Proof}}. \ignorespaces}
\def\endproof{\vbox{\hrule height0.6pt\hbox{   \vrule height1.3ex width0.6pt\hskip0.8ex
   \vrule width0.6pt}\hrule height0.6pt
  }}

\begin{document}

\author{\textbf{Xin Gong}\footnotemark[2]\ \ \ \ \ \textbf{Tieniu Wang}\footnotemark[3]\ \ \ \ \ \textbf{Yukang Cui}\footnotemark[3]}
%Double-Integrator \ \ \ \ \ \textbf{James Lam}\footnotemark[2]
\title{{\LARGE \textbf{Distributed Prescribed-time Consensus Observer for High-order Integrator Multi-agent Systems on Directed Graphs}}\vspace{0.5cm}}
\date{\today}       %日期
\maketitle
%
\begin{abstract}
\ \ \ This brief deals with the distributed consensus observer design problem for high-order integrator multi-agent systems on directed graphs, which intends to estimate accurately the leader state in a prescribed time interval. A new kind of distributed prescribed-time observers (DPTO) on directed graphs is first formulated for the followers, which is implemented in a cascading manner. Then, the prescribed-time zero-error estimation performance of the above DPTO is guaranteed for both time-invariant and time-varying directed interaction topologies, based on strictly Lyapunov stability analysis and mathematical induction method. Finally, the practicability and validity of this new distributed observer are illustrated via a numerical simulation example.

\vspace{0.2cm}

\noindent \textbf{Keywords:} Consensus observer, Directed graphs, High-order Multi-agent systems, Prescribed-time stability
% Nano-quadcopters formation,
% Linear matrix inequality, Linear programming, Leader-follower consensus, Multi-agent systems, Positive consensus, Positive linear systems, Robust consensus, Static output-feedback
\end{abstract}

%\bigskip \footnotetext[1]{Corresponding author.}

\footnotetext[2]{Department of Mechanical Engineering, The University of Hong Kong, Pokfulam Road, Hong Kong (Email: gongxin@connect.hku.hk).}
\footnotetext[3]{College of Mechatronics and Control Engineering, Shenzhen University, Shenzhen, 518060, China (Email: cuiyukang@gmail.com).}
%liujinrjason@connect.hku.hk; yaminwang1994@hotmail.com;
%;  james.lam@hku.hk
%\footnotetext[3]{School of Engineering Sciences, University of Southampton, Southampton SO17 1BJ, U.K. (Email: Z.Shu@soton.ac.uk (Z. Shu)).}
%\footnotetext[3]{%
%School of Automation, Nanjing University of Science and Technology, Nanjing
%210094, Jiangsu, China (Email:baoyongzhang@njust.edu.cn).}
%\footnotetext[2]{%
%This work was partially supported by GRF HKU 7140/11E.}

%%%%%%%%%%%%%%%%%%%%%%%%%%%%%%%%%%%%%%%%%%%%%%%%%%%%%%%%%%%%%%%%%%%%%%%%%%%%%%%%%%%%%%%%%%%%%%%%%%%%%%%%%%%%%%%%%%%%%%%%%%%%%%%%%%%%%%%%%%%%%%
\section{Introduction}


% such as formation control of multi-robots , intelligent transportation \cite{satunin2014multi}, mobile robots swarming \cite{} and large-scale sensor networks \cite{}.
% Most of the reported works are in the realm of the consensus problem of MAS \cite{lewis2013cooperative,mesbahi2010graph,ren2010distributed}, that is, all agents are required to reach an agreement on certain variables of interests.
The last decade has witnessed substantial progresses contributed by various industries (see \cite{satunin2014multi, dongITCST2015, MCYY2017, JZSZM2017,liang2016leader, xu2020distributed} and references therein) on distributed coordination of multi-agent systems (MAS). 
In this brief, we consider the leader-following scenarios \cite{liang2016leader, xu2020distributed} rather than leaderless ones \cite{dongITCST2015, MCYY2017}, where only a portion of followers are pinned (directly connected) to the leader. As pointed out in \cite{de2014controlling}, the formation error (consensus error) suffers from biased measuring noise among robots. Moreover, the actuation faults may also propagate among the MAS networks, which pose a non-negligible threat to the collective control of MAS. An alternative and easily realized method to overcome the above harmful measuring noise and fault propagation is to introduce a distributed observer for each agent and reassign the control input according to the estimated information. Similar to the distributed consensus observer defined in \cite{zuo2019distributed}, herein we intend to design a kind of distributed prescribed-time observers (DPTO), which could reconstruct the leader state for all followers, especially these unpinned ones. Notice that the above distributed observers are different from the traditional Luenberger observers \cite{9311845} focusing on reconstruction on the unmeasured states. Also, this kind of distributed observers \cite{de2014controlling,fu2017finite,fu2016fixed,zuo2017fixed2,zuo2019distributed} could eradicate the notorious communication loop problem \cite{khoo2014multi} encountered in traditional convey communication mechanism.

%In previous works \cite{satunin2014multi, dongITCST2015, MCYY2017, JZSZM2017,lewis2013cooperative,mesbahi2010graph,ren2010distributed}, two types of consensus: 1) leaderless and 2) leader-following consensus w.r.t. a leader agent have been widely studied, where all agents reach an agreement on specific states of interests. 
%Compared with the above two types of consensus, a more complex, practical yet challenging issue named formation-containment (FC) control problem has become an emerging popular topic with multiple interactive and non-autonomous leaders considered. When it comes to the FC problem, it signifies such a role-play scenario that the leaders achieve a predefined formation configuration centered by a predefined trajectory, while the followers converge into the above formation pattern formed by leaders. Image the following escort scenario: For a fleet convoy of ships sailing across the Aden Gulf, the warships (leaders) equipped with advanced weapons and armors should achieve a compact configuration cooperatively to detect and expel the Somali pirates; the merchant ships (followers)  without essential defenses are required to enter the moving protective hull formed by the warships. The above FC mission can also be extended to numerous applications with multiple autonomous leaders, such as combined UAV operations in smart agriculture \cite{maddikunta2021unmanned} and collective transportation of platoon systems \cite{hu2020cooperative}.
% , including the combat aerial vehicles'  hybrid formation of lead aircraft and its wingmen  \cite{humphreys2015optimal}, cooperative penetration of multiple missiles \cite{wang2017composite}, and cooperative transportation of multi-robot systems \cite{alonso2017multi}.


%$1$%See $\url{http://www.baidu.com}$.


One of the key properties of a superior distributed observer is to achieve distributed coordination quickly with little constraint on the communication topologies. Unlike the distributed observer in \cite{de2014controlling} which only owns asymptotical convergence, Fu \emph{et. al.} first proposed a kind of distributed fixed-time observer for first-order \cite{fu2017finite} or second-order MAS \cite{fu2016fixed}. However, the above two works employ two fractional power feedback terms, that is, $\frac{p}{q}$ and  $2-\frac{p}{q}$ with $q>p>0$, which are too cumbersome to extend to high-order dynamics. Then, Zuo \emph{et. al.} further employed single fractional power feedback ($\gamma>1$) in the design of distributed finite-time observer, which achieved the distributed fixed-time estimation w.r.t. the leader state on both undirected graphs \cite{zuo2017fixed2,zuo2019distributed} and directed ones \cite{zuo2019distributed}. As pointed in \cite{zuo2019distributed}, the proof based on the symmetry of Laplacian matrices is not applicable for directed topologies. Thus, it is not trivial to study the distributed finite-time observer on directed graphs. Notice that the fixed-time zero-error convergence is only guaranteed on undirected topologies \cite[Theorem 1]{zuo2019distributed}, while only fixed-time attractiveness of an error domain is proven in  \cite[Theorem 2]{zuo2019distributed}. Thus, it remains a big challenge to achieve distributed finite-time zero-error observation on a general directed topology. Moreover, the needed time interval to achieve distributed observation in the above works \cite{fu2017finite,fu2016fixed,zuo2017fixed2,zuo2019distributed} depends on the settings of initial conditions and/or the network algebraic connectivity \cite{wu2005algebraic}, which put barriers on the way of their applications.






%In this aspect, various finite-time protocols, including but not limited to  fractional power feedback protocols \cite{wang2007finitefir, xiao2011finitefir, xiao2008fastfir}, iterative-learning protocols \cite{meng2012iterative} and switching protocols \cite{liu2016finitesw}, have been formulated to achieve finite-time consensus of MAS. However, the settling time of the above works \cite{wang2007finitefir, xiao2011finitefir, xiao2008fastfir, meng2012iterative, liu2016finitesw} depends directly on the initial conditions of all agents. Consequently, it is nontrivial to regulate the consensus time arbitrarily without priori information of the initial states of all agents, particularly for the second- or high-order MAS with tedious initial conditions. Moreover, it has been verified by \cite{shang2012finite} that the Lipschitz condition of control protocols must be violated in most previous works. For example, the fractional power feedback protocols \cite{wang2007finitefir, xiao2011finitefir, xiao2008fastfir} suffer from inevitable chattering phenomena. Recently, Song \emph{et al.} \cite{song2017time, song2019time} have proposed a new prescribed finite-time method to achieve finite-time regulation via exploiting a series of time scaling function featured by parameters denoting the start time instant and lasting time interval. Inspired by \cite{song2017time}, prescribed-time protocols for MAS have been investigated in \cite{wang2018leader, wang2018prescribed, colunga2018predefined, 1wang2019zero}, which are prescribed finite-time convergent as well as at least $C^1$ smooth. These works \cite{wang2018leader, wang2018prescribed, colunga2018predefined, 1wang2019zero} mainly investigated MAS on some basic graphs, such as undirected graphs \cite{wang2018prescribed, 1wang2019zero}, directed and acyclic spanning tree with a single leader\cite{wang2018leader}, or multiple static leaders \cite{wang2018prescribed} serving as the tree root. In this brief, we will extend the prescribed-time protocols to a more general scope of communication topology, that is, directed and two-layered graphs whose subgraph of leaders is strongly-connected and cyclic.

Two inevitable yet challenging difficulties arise when actuation faults are involved in the framework of distributed finite-time observer of MAS.

\begin{enumerate}[(i)]%[1),itemsep= 0 pt, topsep =1ex, itemindent=-0em, listparindent = 0 pt]
  \item How can all followers obtain finite-time zero-error convergence since the actual states of each pinned leader are only available to only a portion of followers, especially on a directed topology?
  \item How to regulate the consensus observation time in (i) arbitrarily without priori information of the initial states of the MAS and network algebraic connectivity, particularly for high-order MAS?
  \end{enumerate}


% First,  Second, how to regulate the consensus observation time arbitrarily without priori information of the initial states of all agents and network algebraic connectivity, particularly for high-order MAS?

Fortunately, recently proposed prescribed-time protocols in \cite{song2017time, song2019time,wang2018leader, wang2018prescribed, 1wang2019zero, gong2020distributed2} shed light on the solution of the above problem. In this brief, we formulate a kind of DPTO for each follower, featured by a new hybrid constant and time-varying scaling function. This design is fundamentally different from the previous distributed finite-time observer in \cite{fu2017finite,fu2016fixed,zuo2017fixed2,zuo2019distributed} based fractional power feedback. This DPTO can provide an accurate estimation of the leader's full state after a prescribed time interval. Compared with the distributed finite-time observer on directed graphs in a recently distinguished work \cite[Theorem 2]{zuo2019distributed}, this new kind of DPTO on directed graphs possesses the following two highlights: 1) zero-error convergence rather than uniformly bounded convergence; 2) prescribed-time convergence rather than just finite-time convergence.
 %, by which the FC error can be driven into an adjustable residual set around zero within a predefined time interval. Furthermore, the associated residual error vanishes into zero asymptotically.

 The above discussions indicate that distributed prescribed-time consensus observer design problem for networked high-order integrator MAS on directed graphs has not been addressed or solved. Herein, we provide a solution for the above problem, which has the following main contributions and characteristics:



\begin{enumerate}[1)]%[1),itemsep= 0 pt, topsep =1ex, itemindent=-0em, listparindent = 0 pt]
%\item The prescribed-time FC control of the double-integrator MAS subject to actuation faults is first considered, to the best of our knowledge.
\item \textbf{hybrid constant and time-varying feedback}: A new kind of cascaded DPTO, based on a hybrid constant and time-varying feedback, is developed for the agents on directed graphs, which achieves the distributed accurate estimation on each order of leader state in a cascaded manner. 
\item \textbf{Zero-error convergence on time-invariant/varying directed graphs}: Unlike previous work \cite{zuo2019distributed} which only guarantees finite-time attractive of a fixed error bound, we manage to regulate the observation error on directed graphs into zero in a finite-time sense. It is further demonstrated that the feasibility of this DPTO can be extended from time-invariant directed graphs to time-varying ones.
%The difficulties caused by the asymmetrical Laplacian matrix under the circumstance of single-way directed communication topology are circumvented in the frameworks of distributed prescribed-time fault-tolerant control.
\item \textbf{Prescribed-time convergence}: This DPTO could achieve distributed zero-error estimation within a predefined time interval on a directed communication topology, in spite of the initial conditions and the exact network topology. Thus, the observer design procedure is much more easily-grasped for the new users than that in \cite{zuo2019distributed}.%Also, the parameter design procedure is much simpler than \cite{zuo2019distributed}. These user-friendly merits bring great convergence to real applications of this distribute observer.  
\end{enumerate}


% The remainder of this paper is listed as follows. Section $2$ introduces some preliminaries concerned with commonly used notations, graph theory, and some useful lemmas and definitions. Section $3$ formulates the problem of fault-tolerant prescribed-time FC control of MAS on directed graphs, then proves the effectiveness of newly developed protocols for both leaders and followers in the MAS to achieve FC control. Section $4$ uses a simulation example of prescribed-time FC control of MAS subject to actuation faults to verify the validity of the new proposed protocols. Section $5$ concludes the whole paper.
%Section $5$ uses an experiment of bipartite \ldots\ldots\ldots\ldots\ldots\ldots. to show the practicability of this new protocol.

\noindent\textbf{Notations:}
In this brief, $\boldsymbol{1}_m$ (or $\boldsymbol{0}_m$) denotes a column vector of size $m$ filled with $1$ (respectively, 0). Denote the index set of sequential integers as $\textbf{I}[m,n]=\{m,m+1,\ldots~,n\}$ where $m<n$ are two natural numbers. Define the set of real numbers, positive real numbers and nonnegative real numbers as $\mathbb{R}$, $\mathbb{R}_{>0}$ and $\mathbb{R}_{\geq 0}$, respectively. ${\rm diag}({b})$ means a diagonal matrix whose diagonal elements equals to a given vector ${b}$.
For a given symmetric matrix $A\in \mathbb{R}^{n\times n}$, its spectrum can be sorted as: $\lambda_1(A)\leq\lambda_2(A) \leq\ldots \leq\lambda_n(A)$; moreover, $A>0$ means that $\lambda_1(A)>0$.
%For a time-varying function $x(t): \mathbb{R}_{\geq 0 }\mapsto \mathbb{R}$, denote that $\sup_{t\in [t_0, t_1]} x(t) $ and $\inf_{t\in [t_0, t_1]} x(t)$ as the upper bound and lower bound of $x(t)$ over the time interval $[t_0, t_1]$, respectively. Moreover, denote that $\|x(t)\|_{[t_0, t_1]} =\sup_{t\in [t_0, t_1]} \|x(t)\| $. Define that $L_{\infty}:=\{x(t)|x(t): \mathbb{R}_{\geq 0 }\mapsto \mathbb{R}^n,\ \|x(t)\|_{[t_0, t_1]}<\infty\}$. In the following sections, $x(t) \in L_{\infty}$, $t\in [t_0, t_1]$, represents that the variable $x$ is uniformly bounded over $[t_0, t_1]$.   %$A\succeq 0$ (or $A\succ 0$) denotes that $A$ is a nonnegative matrix (positive matrix, respectively), which means all elements of $A$ are nonnegative (positive, respectively).
 %${\rm span}(x)$ denotes the span vector of a given vector $x=[p_1, p_2,\ldots~, p_n]^{\mathrm{T}}\in \mathbb{R}^n$.

\label{introduction}


\section{Preliminaries}\label{section2}

%\subsection{Notations}





\subsection{Graph Theory}
We employ a directed graph (`digraph' for short) to denote the communication topology of the MAS.
For the MAS, a digraph $\mathcal{G } =(\mathcal{V}, \mathcal{E}, \boldsymbol{A} )$ is a ternary,  which consists of a node set $\mathcal{V}=\{ 1, 2, \ldots~ , N \}$; an edge set $\mathcal{E} \subset \mathcal{V} \times \mathcal{V}=\{(v_j,\ v_i)\mid\ v_i,\ v_j \in \mathcal{V}\}$ such that $(v_j,\ v_i)$ is a directed edge from $v_j$ to $v_i$; and an adjacency matrix $\boldsymbol{A}=[a_{ij}] \in \mathbb{R}^{N\times N} $ is a weight matrix such that $a_{ij}\neq 0\Leftrightarrow (v_j,\ v_i) \in \mathcal{E}$ and $a_{ij}= 0$, otherwise.  
%Each node on $\mathcal{G }$ can be uniquely labeled by an integer $i$ belonging to a finite index set $\mathcal{I}=\{1, 2,  \ldots ~ , n \}$.
%Assume the signed digraph $\mathcal{G }$ is digon sign-symmetric, that is, $a_{ij}a_{ji}\geq 0,\ \forall i,\ j\in  \mathcal{I}$.
The neighbor set of node $v_i$ is defined by $\mathcal{N}_{i}=\{v_{j}\in \mathcal{V}\mid (v_j,\ v_i)\in \mathcal{E} \}$.
%An adjacency matrix $A=[a_{ij}]$ of graph $\mathcal{G}$ with order $n$ is defined as $a_{ij}=1$ if $ (v_i,v_j)\in \mathcal{E}$, but $0$ otherwise.
The Laplacian matrix ${L}\in \mathbb{R}^{n\times n}$ of graph $\mathcal{G}$ with order $N$ is defined as ${[L]}_{ii}=  \sum_{j \in \mathcal{N}_{i}} a_{ij}  $ and  ${[L]}_{ij}= -a_{ij} $ for any $i\neq j$.
A directed path of length $m$ from node $v_{s_0}$ to $v_{s_m}$ is an ordered sequence of distinct nodes $\{v_{s_0}, v_{s_1}, \ldots~ , v_{s_m}   \}$ where $(v_{s_i}, v_{s_{i+1}} ) \in \mathcal{E}$, $ \forall i$.
A digraph $\mathcal{G}$ has a directed spanning tree if and only if there exists at least one node $v_{i}$ owning a directed path to any other node $v_{j}$ on $\mathcal{G}$.
% Two nodes $v_{i}$ and $v_{j}$ on a digraph $\mathcal{G}$ are \emph{path equivalent} if there exists a directed path from $v_i$ to $v_j$ and another return path from $v_j$ to $v_i$. The node set $\mathcal{V}$ can be partitioned into a unique series of disjoint node subsets consisting of path equivalent nodes, that is, \emph{strongly connected components} of $\mathcal{G }$. A digraph is called \emph{strongly connected} if and only if all nodes belong to one strongly connected component.
To solve the problem induced by the asymmetrical Laplacian matrix ${L}$ under general directed topology, a mirror matrix ${M}$ \cite{gong2020distributed2} corresponding to ${L}$ is established:
\begin{equation}\label{1Qmat}
{M}(L)=\frac{1}{2}[{\rm diag}({\rho})L+L^{\mathrm{T}}{\rm diag}({\rho})],
\end{equation}
where $\rho=[\rho_1, \rho_2, \ldots~,\rho_n]^{\mathrm{T}}={\rm diag}((L^{\mathrm{T}})^{-1}\boldsymbol{1}_n)$.


\subsection{Some Useful Definitions and Lemmas}






%\begin{myDef}\label{def41}
%{\color{blue}
\begin{myDef}\label{def41}
Consider the system in the form of
\begin{equation}\label{EQ1}
\begin{cases}
\dot{x}=f(x,t),\\
x(0)=x_0,
\end{cases}
\end{equation}
where $x\in \mathbb{R}^n$ and $f\colon \mathbb{R}^{n}\times \mathbb{R}_{>0}\mapsto \mathbb{R}^{n}$ is a given nonlinear function. The solutions of (\ref{EQ1}) can be understood from the viewpoint of Flippov \cite{filippov2013differential} if $f(x,t)$ is discontinuous. Suppose that the equilibrium point of (\ref{EQ1}) is the origin. The origin of system (\ref{EQ1}) is said to be \textbf{globally prescribed-time stable}, if it is globally asymptotical stable and any solution $x(t)$ arrives at the origin no later than a prescribed time instant, that is, $x(t)\equiv 0$, $\forall t\geq T$, where $T$ denotes a prescribed time interval.
%\begin{enumerate}[{\rm 1)},itemsep= 0 pt, topsep =1ex, itemindent=-0em, listparindent = 0 pt]
%\item The origin of system (\ref{EQ1}) is said to be \emph{globally prescribed-time stable}, if it is globally asymptotically stable and any solution $x(t)$ reaches the origin no later than a prescribed time instant, that is, $x(t)=0$, $\forall t\geq T$, where $T$ denotes a prescribed time interval;
%\item The set $\mathcal{S}$ is said to be \emph{globally prescribed-time attractive}, if  any solution $x(t)$ reaches $\mathcal{S}$ no later than a prescribed time instant and remains there, that is, $x(t)\in \mathcal{S}$, $\forall t\geq T$, where $T$ denotes a prescribed time interval.
%\end{enumerate}
\end{myDef}

The following Lemma provides a key judgment criterion for the globally prescribed-time stability in Definition \ref{def41}:%}



\begin{lemma}[{\cite[Lemma 1]{wang2018prescribed}}]\label{Plemma001}
For a scalar function $V:\mathbb{R}^n \times \mathbb{R}_{>0} \mapsto \mathbb{R}_{\geq0}$, if there exists a positive scalar $c$ such that
 \begin{equation*}
\dot{V}(x,\ t)=-\left(c+2\frac{\dot{\varsigma} (t_0,T)}{{\varsigma (t_0,T)}}\right){V}(x,\ t)
,\ t\in [t_0,\ t_0+\infty),
\end{equation*}
with
  \begin{equation*}
 {\varsigma (t_0,T)}=
\begin{cases}
  (\frac{T}{t_0+T-t})^{h}, & \mbox{if }  t\in [t_0,\ t_0+T), \\
  1, & \mbox{otherwise},
\end{cases}
\end{equation*}
and
 \begin{equation*}\label{roudot}
 \dot{\varsigma} (t_0,\ T)=
\begin{cases}
 \frac{h}{T} \varsigma(t_0+T)^{1+\frac{1}{h}}, & \mbox{if }  t\in [t_0,\ t_0+T), \\
  0, & \mbox{otherwise},
\end{cases}
\end{equation*}
where $t_0$ is the start time point, $T$ denotes the duration of the time-varying period of $\varsigma (t_0,T)$, and $h$ is a scalar bigger than 2. Then, we can conclude
\begin{enumerate}[{\rm 1)},itemsep= 0 pt, topsep =1ex, itemindent=-0em, listparindent = 0 pt]
\item $\lim \limits_{t\rightarrow (t_0+T)^{-}} V(x,\ t)=0$,
\item $V(x,\ t)\equiv0,\ t\in [t_0+T,\ \infty)$,
\end{enumerate}
which shows $V(t)$ is globally prescribed-time stable with needed time interval $T$. Furthermore, we have
 \begin{equation}\label{Vrealm}
V(t)\leq {\varsigma (t_0,T)}^{-2}\exp(-c(t-t_0))V(t_0),~t\in [t_0,\ t_0+\infty).
\end{equation}
\end{lemma}








%For a strongly-connected digraph, it follows from \cite[Lemma 5]{5wang2016fault} that $\lambda_2({M}(L))>0$.

% {\color{blue}The following two lemmas will be useful in the proof of the main conclusions.}

% %\begin{lemma} \label{Plemma064}
% %For a given time-varying function $\mu: \mathbb{R}_{\geq 0 }\mapsto \mathbb{R}$, which satisfies $\inf_{t\in [0,\ \infty)} \mu(t)\geq 1$, it holds that
% %%\begin{equation*}\label{intexpbound}
% %%\int_{t_0}^{t}{\rm e}^{-c\int_{\tau}^{t}\mu(s)ds}\mu(\tau)d\tau \leq \frac{1}{c}
% %%\end{equation*}
% %%\begin{equation*}\label{intexpbound}
% %%\int_{t_0}^{t}e^{-c\int_{\tau}^{t}\mu(s)ds}\mu(\tau)d\tau \leq \frac{1}{c}
% %%\end{equation*}
% %\begin{equation*}\label{intexpbound}
% %\int_{0}^{t}{\rm e}^{-c\int_{\tau}^{t}\mu(s)ds}\mu(\tau)d\tau \leq \frac{1}{c},
% %\end{equation*}
% %where $c\in \mathbb{R}_{>0}$.
% %\end{lemma}
% %
% %\textbf{Proof.}
% %By setting $d(\tau)$ in \cite[Equation 12]{song2017time} as $1$, Lemma \ref{Plemma064} is proved.
% %$\hfill \hfill \square $
% %\vspace{0.2cm}




% % Based on the above two steps, Lemma 2 is proven.
% \begin{lemma}[{\cite[Lemma 4]{5wang2016fault}}]\label{Plemmainq}
%   For a given series of positive scalars $x_i\in \mathbb{R}_{>0}$, $i=1,\ldots~,m$, and $p\in(0,1]$, we have
%   \begin{equation*}\label{stochaticbound}
%    (\sum_{i=1}^{m}|x_i|)^p \leq \sum_{i=1}^{m}|x_i|^p\leq m^{1-p}(\sum_{i=1}^{m}|x_i|)^p.
%   \end{equation*}
% \end{lemma}

\subsection{Problem Formulation}

%One of recent distinguished distributed finite-time observers on digraphs is summarized in \cite[Theorem 2]{zuo2019distributed}, which achieved distributed estimation w.r.t. the leader in finite-time CUUB sense. The needed time interval to achieve uniform bounded estimation error depended directly on the initial conditions as well as the generalized network algebraic connectivity \cite{wu2005algebraic} of digraphs. Compared with \cite[Theorem 2]{zuo2019distributed}, our DPTO on digraphs has the merit of prescribed-time zero-error convergence.

In this brief, we consider a MAS with one leader and $N$ followers on directed graphs $\mathcal{G}$, which can be denoted as agent $0$ and agents $1\sim N$, respectively.
Consider the same dynamics of the non-autonomous leader as described in \cite{zuo2019distributed}:
\begin{equation}\label{condition_variate}
  \dot{x}_{0,1}(t)=x_{0,2}(t),\ \dot{x}_{0,2}(t)=x_{0,3}(t),\ \ldots,\ \dot{x}_{0,n}(t)=f_0(x_0,t),
\end{equation}
where $x_0 =[x_{0,1},x_{0,2},\ldots~,x_{0,n}]^{\mathrm{T}} \in \mathbb{R}^n$ and $f_0(x_0,t)\in \mathbb{R}$ denote the state vector and the input of the leader, respectively.

\begin{assumption} \label{assumption035}
  $f_0(x_0,t)$ is continuous w.r.t. t, which satisfies ${{\|f_0(x_0,t)\|}_{\infty}} \leq \bar{f}_0$, where $\bar{f}_0$ is a positive scalar denoting the upper bound of the leader input.
\end{assumption}



Let $\hat{x}^i_{0,k}$, for all $i\in \textbf{I}[1,N]$ and $k\in \textbf{I}[1,n]$, denote as the $i$th follower's estimation of the $k$th state of the leader.
Based on the above settings, the design issue of the DPTO of high-order MAS on digraphs will be studied, whose details are summarized as follows:

\vspace{0.2cm}

\noindent \textbf{Problem DPTOHD} (Distributed Prescribed-Time Observer for multi-agent systems with
High-order integrator dynamics and Directed topologies) : For any initial bounded estimations, design a distributed observer $\hat{x}^i_{0,k}$, $k\in \textbf{I}[1,n]$, such that all followers on the digraph achieve distributed zero-error estimation after a predefined time interval $T_{obs}$ w.r.t. the high-order integrator leader (\ref{condition_variate}), that is,
% \begin{equation}
%   1
% \end{equation}
\begin{subequations}\label{PTfocusall}
\begin{numcases}{}
\lim_{t\rightarrow  t^{*}}\hat{x}^i_{0}= x_0,\label{PTfocusalla}\\
\hat{x}^i_{0}(t)= x_0(t), \forall t\geq t^{*},\label{PTfocusallb}
\end{numcases}
\end{subequations}
where $t^{*}=t_0+T_{obs}$ with $t_0$ denoting the initial time instant and $\hat{x}^i_{0}=[\hat{x}^i_{0,1},\ldots~,\hat{x}^i_{0,n}]^{\mathrm{T}}$.
$\hfill \hfill \square $
\vspace{0.2cm}




\section{Main Results}

To solve \textbf{Problem DPTOHD}, a kind of DPTO on time-invariant/varying digraphs, inspired by the distributed finite-time observer in \cite{fu2017finite,fu2016fixed,zuo2017fixed2,zuo2019distributed}, is proposed in this section, which only utilizes neighbors' information. 

\subsection{Time-invariant Topology Cases}







In this subsection, we consider the time-invariant directed topologies satisfying the following assumptions:

\begin{assumption} \label{assumption040}
  The graph $\mathcal{G}$ contains a spanning tree rooted by the leader, that is, agent 0.
\end{assumption}

Based on Assumption \ref{assumption040}, the Laplacian matrix corresponding to $\mathcal{G }$ can be divided into four blocks:

\begin{equation*}
L=
	\left[
	\begin{array}{c|c}
0&0_{1\times N}\\ \hline
-\boldsymbol{B} &L_0 \\
	\end{array}
	\right]\in \mathbb{R}^{(N+1)\times (N+1)},
\end{equation*}
where $L_0\in \mathbb{R}^{N\times N}$ denotes the communication topology among followers, $\boldsymbol{B}=(b_{1},\ldots~, b_{N})^{\mathrm{T}}\in \mathbb{R}_{\geq 0}^{N\times 1}$ represents the pinned vector, with $b_i=1,\ i\in {\textbf{I}[1,N]}$, if the $i$th follower connects directly to the leader, otherwise, $b_i=0$. $L_0$ has the property given in Lemma \ref{Plemma74}. 
% $\boldsymbol{D}$, and $\boldsymbol{A}$, represent the associated sub-matrices w.r.t. $L$, and $\boldsymbol{B}=(b_{1},\ldots~, b_{n})^{\mathrm{T}}$, with $b_i=1,\ i\in {\textbf{I}[1,N]}$, if the $i$th follower connects directly to the leader, otherwise, $b_i=0$. 
%$L_0$ has the property given in Lemma \ref{Plemma74}:
%Assumption 5. f0(p0,v0,t) is continuous w.r .t. t, which satisfies kf0(p0,v0,t)k∞≤¯f0.
%Assumption 7. There exists at least one leader pinned to the leader, that is, it can communicate directly with
%the leader. The subgraph GLamong all leaders is strongly-connected. F or each follower , there exists at least
%one leader owning a directed path to it.
Furthermore, $L_0$ owns the following property:
\begin{lemma}[{\rm \cite[Lemma 3]{gong2020distributed2}}]\label{Plemma74}
  Under Assumption \ref{assumption040}, the matrix $L_0$ is invertible.
  Define that $P:={\rm diag(\rho)}$ where $\rho=(L_0^{\mathrm{T}})^{-1}\boldsymbol{1}_N = [\rho_1,\ldots,\rho_N]^{\mathrm{T}}$, then we have
  \begin{equation*}
    M(L_0) = \frac{1}{2}(PL_0+L_0^{\mathrm{T}}P) > 0.
  \end{equation*}
\end{lemma}





\begin{figure}[!htbp]
  %\begin{minipage}[t]{1\linewidth}
  \centering
  \includegraphics[width=1\textwidth]{pic/P4.pdf}
  \caption{Timeflow graph of the cascaded DPTO}
  \label{fig:figure0}
  \end{figure}

Define $\tilde{x}^i_{0,k}=\hat{x}^i_{0,k}-{x}_{0,k}$, $k\in \textbf{I}(1,n)$, as the global state estimation error of the $i$th follower on the leader's $k$th state. Moreover, let $\psi^i_{k}=b_i(\hat{x}^i_{0,k}-x_{0,k})+\sum_{j \in \mathcal{N}_{i}}a_{ij}(\hat{x}^i_{0,k}-\hat{x}^j_{0,k})$ denote the local state estimation error of the $i$th follower on the leader's $k$th state. The DPTO for the $i$th follower, $i\in\textbf{I}[1,N]$, can be designed as

%Revised version:
\begin{equation}\label{DPTO}
  \begin{cases}
   \dot{\hat{x}}^i_{0,k}=\hat{x}^i_{0,k+1}-\left(\alpha+\beta\frac{\dot{\varsigma}(t_{n-k}, T^k_{obs})}{\varsigma(t_{n-k}, T^k_{obs})}\right)\psi^i_{k},~\forall k\in \textbf{I}(1,n-1),\\
   \dot{\hat{x}}^i_{0,n}=-\sigma{\rm sign}(\psi^i_{n}) -\left(\alpha+\beta\frac{\dot{\varsigma}(t_0, T^n_{obs})}{\varsigma(t_0, T^n_{obs})}\right)\psi^i_{n},\\
  \end{cases}
\end{equation}
where $T^k_{obs}\in \mathbb{R}_{> 0}$, $\forall k \in\textbf{I}[1,n]$, satisfying $T_{obs}=\sum_{k=1}^{n}T^k_{obs}$ and thus $t_{n-k}=t_0+\sum_{j=k+1}^{n}T^j_{obs}$, $\forall k\in \textbf{I}[0,n-1]$. To be more specific, the timeflow graph of the DPTO (\ref{DPTO}) is depicted in Fig. \ref{fig:figure0}, which shows that the cascaded estimation process on each leader state in an inverted order.


%
\begin{myTheo}\label{Plemma800}
Consider the leader with high-order dynamics (\ref{condition_variate}), if Assumptions \ref{assumption035} and \ref{assumption040} hold, then \emph{\textbf{Problem DPTOHD}} can be solved via the DPTO (\ref{DPTO}) satisfying
\begin{subequations}\label{DPTOparam1}
  \begin{numcases}{}
    \alpha >0,\label{DPTOparam1a}\\
    \beta\geq \frac{\max\{\rho\}}{\lambda_1(M(L_0))},\label{DPTOparam1b}\\
    \sigma \geq \bar{f}_0.\label{DPTOparam1c}
  \end{numcases}
  \end{subequations}
%can ensure the estimation errors of all followers w.r.t. the leader converge into zero in a prescribed time interval $T_{obs}$.
\end{myTheo}

\textbf{Proof.} First, define that $\psi_k:=[\psi^1_k,\ldots~,\psi^N_k]^{\mathrm{T}}$ and $\tilde{x}_{0,k}:=[\tilde{x}^1_{0,k},\ldots~,\tilde{x}^N_{0,k}]^{\mathrm{T}}$, $\forall k\in  \textbf{I}[1,n]$. Then we have $\psi_k=L_0\tilde{x}_{0,k}$ and $\psi_k=\boldsymbol{0}_{N} \Leftrightarrow \tilde{x}_{0,k}=\boldsymbol{0}_{N}$ since $L_0$ is nonsingular by recalling Lemma \ref{Plemma74}. Thus, the effectiveness of the DPTO can be ensured via proving the proposition that $\psi_k$, $\forall k\in  \textbf{I}[1,n]$, converges into zero. Theorem \ref{Plemma800} is proven step by step via \emph{Mathematical Induction}:

\textbf{Step 1:}
Consider the second equation in (\ref{DPTO}), define that
\begin{equation*}
u^{i}_{0,n}=\dot{\hat{x}}^i_{0,n}=u^{i,a}_{0,n}+u^{i,b}_{0,n},~i\in\textbf{I}[1,N],
\end{equation*}
where
\begin{equation*}
\begin{cases}
u^{i,a}_{0,n}=-\left(\alpha+\beta\frac{\dot{\varsigma}(t_0, T^n_{obs})}{\varsigma(t_0, T^n_{obs})}\right)\psi^i_{n},\\
u^{i,b}_{0,n}=-\sigma{\rm sign}(\psi^i_{n}).\\
\end{cases}
\end{equation*}
The time derivative of $\psi_{n}^i$ equals to
\begin{equation*}
\dot{\psi}_n^i=\sum_{j \in \mathcal{N}_{i}}a_{ij}(u^{i}_{0,n}-u^{j}_{0,n})+b_{i}(u^{i}_{0,n}-f_0).
\end{equation*}
Consider the following Lyapunov function candidate:
\begin{equation*}
V_n=  \frac{1}{2}\psi_{n}^{\mathrm{T}}P\psi_{n}=\frac{1}{2}\sum_{i=1}^{N}\rho_i( |{\psi}_n^i|^2),
\end{equation*}
where $\psi_{n}:=[\psi^1_{n},\ldots~,\psi^N_{n}]^{\mathrm{T}}$, $P$ and $\rho$ are defined in Lemma \ref{Plemma74}. Notice that $V_n$ is well-defined due to the positive definiteness of $P$. It can be computed that
\begin{align}\label{9eq}
\dot{V}_n=&\sum_{i=1}^{N}\rho_i {\psi}_n^i\times\dot{\psi}_n^i\nonumber\\
=&\sum_{i=1}^{N}\rho_i {\psi}_n^i\times\left(\sum_{j \in \mathcal{N}_{i}}a_{ij}(u^{i}_{0,n}-u^{j}_{0,n})+b_{i}(u^{i}_{0,n}-f_0)\right)\nonumber\\
=&\sum_{i=1}^{N}\rho_i {\psi}_n^i\times\left(\sum_{j \in \mathcal{N}_{i}}a_{ij}(u^{i,a}_{0,n}-u^{j,a}_{0,n})+b_{i}(u^{i,a}_{0,n})\right)\nonumber\\
&+\underbrace{\sum_{i=1}^{N}\rho_i {\psi}_n^i\times\left(\sum_{j \in \mathcal{N}_{i}}a_{ij}(u^{i,b}_{0,n}-u^{j,b}_{0,n})+b_{i}(u^{i,b}_{0,n}-f_0)\right)}_{\dot{V}^{b}_{n}}.
\end{align}
%By recalling the fact that $u^*_{ib}-u^*_{jb}\leq -\sigma{\rm sign}({\psi}_v^i)+\sigma$ and $u^*_{ib}-f_0\leq -\sigma{\rm sign}({\psi}_v^i)+\sigma$, we come to a conclusion that
%\begin{equation}\label{10eq}
%  \dot{V}_{1b}\leq0
%\end{equation}
%\begin{remark}
The proposition under Condition (\ref{DPTOparam1c}) that
\begin{equation}\label{10eq}
  \dot{V}^{b}_{n}\leq0,
\end{equation}
 can be verified via the following analysis:
\begin{enumerate}[{\rm 1)},itemsep= 0 pt, topsep =1ex, itemindent=-0em, listparindent = 0 pt]
\item When ${\psi}_n^i<0$, we obtain that $u^{i,b}_{0,n}=\sigma$, $u^{j,b}_{0,n}\leq\sigma$, $f_0\leq\bar{f}_0\leq\sigma$, and thus $u^{i,b}_{0,n}-u^{j,b}_{0,n}\geq 0$, $u^{i,b}_{0,n}-f_0\geq 0$. Consequently, one can easily obtain that $\dot{V}^{b}_{n}\leq0$;
\item When ${\psi}_n^i=0$, we have $\dot{V}^{b}_{n}=0$;
\item When ${\psi}_n^i>0$, we obtain that $u^{i,b}_{0,n}=-\sigma$, $-u^{j,b}_{0,n}\leq\sigma$, $-f_0\leq\bar{f}_0\leq\sigma$, and thus $u^{i,b}_{0,n}-u^{j,b}_{0,n}\leq 0$, $u^{i,b}_{0,n}-f_0\leq 0$. To this end, it follows that $\dot{V}^{b}_{n}\leq0$.%$\hfill \hfill \square $
\end{enumerate}
%\end{remark}
It can be obtained from (\ref{9eq}), (\ref{10eq}) and Lemma \ref{Plemma74} that
\begin{align}\label{11eq}
\dot{V}_n
\leq&\sum_{i=1}^{N}\rho_i( {\psi}_n^i)\times\left(\sum_{j \in \mathcal{N}_{i}}a_{ij}(u^{i,a}_{0,n}-u^{j,a}_{0,n})+b_{i}(u^{i,a}_{0,n})\right)\nonumber\\
=&-\left(\alpha+\beta\frac{\dot{\varsigma}(t_0, T^n_{obs})}{\varsigma(t_0, T^n_{obs})}\right) {\psi}_n^{\mathrm{T}}PL_0 {\psi}_n \nonumber\\
%=&-\left(\alpha+\beta\frac{\dot{\varsigma}(t_0, T^n_{obs})}{\varsigma(t_0, T^n_{obs})}\right) {\psi}_v^{\mathrm{T}}M(L_0){\psi}_v \nonumber\\
=&-\left(\alpha+\beta\frac{\dot{\varsigma}(t_0, T^n_{obs})}{\varsigma(t_0, T^n_{obs})}\right) {\psi}_n^{\mathrm{T}}M(L_0){\psi}_n \nonumber\\
\leq&-2\left(\alpha+\beta\frac{\dot{\varsigma}(t_0, T^n_{obs})}{\varsigma(t_0, T^n_{obs})}\right) \frac{\lambda_1(M(L_0))}{{\max}\{\rho\}}{V}_n(t).\nonumber
\end{align}
By recalling Condition (\ref{DPTOparam1b}), it follows that
\begin{equation*}
 \dot{V}_n(t)
\leq -2\alpha\frac{\lambda_1(M(L_0))}{{\max}\{\rho\}}{V}_n(t)-2\frac{\dot{\varsigma}(t_0, T_{obs}^n)}{\varsigma(t_0, T_{obs}^n)}{V}_n(t).
\end{equation*}
By recalling Lemma  \ref{Plemma001}, we learn that ${V}_n(t)\equiv 0$ over $[t_0+ T_{obs}^n,\infty)$. Herein, the DPTO can ensure that the velocity estimation errors of all followers w.r.t. the leader converge into zero in a prescribed time interval $T_{obs}^n$.

\textbf{Step 2:} Suppose that $V_{k+1}=  \frac{1}{2}\psi_{k+1}^{\mathrm{T}}P\psi_{k+1}=\frac{1}{2}\sum_{i=1}^{N}\rho_i( |{\psi}_{k+1}^i|^2)$, $k=n-1,~n-2,\ldots~,1,$ can be globally prescribed-time stabilized no later than $t_{n-k}=t_0+\sum_{j=k+1}^{n}T^j_{obs}$, that is, 
\begin{equation}\label{Inferk}
\psi_{k+1}=\boldsymbol{0}_{N}, ~\forall t\geq t_{n-k}.
\end{equation}
It can be easily checked that the first $V_{k+1}$ term, that is, $V_n$ satisfies (\ref{Inferk}) according the analysis of Step 1.

\textbf{Step 3:} Selecting the Lyapunov function as $V_k=  \frac{1}{2}\psi_k^{\mathrm{T}}P\psi_k=\frac{1}{2}\sum_{i=1}^{N}\rho_i( |{\psi}_k^i|^2)$, we compute that
\begin{align*}%\label{9eq1}
\dot{V}_k
=&\sum_{i=1}^{N}\rho_i {\psi}_k^i\times\left(\sum_{j \in \mathcal{N}_{i}}a_{ij}(u^{i,a}_{0,k}-u^{j,a}_{0,k})+b_{i}(u^{i,a}_{0,k})\right)\nonumber\\
&+\underbrace{\sum_{i=1}^{N}\rho_i {\psi}_k^i \times\left(\sum_{j \in \mathcal{N}_{i}}a_{ij}(u^{i,b}_{0,k}-u^{*}_{jb,k})+b_{i}(u^{i,b}_{0,k}-x_{0,k})\right)}_{\dot{V}^{b}_{k}},
\end{align*}
where
\begin{equation*}
\begin{cases}
u^{i,a}_{0,k}=-\left(\alpha+\beta\frac{\dot{\varsigma}(t_{n-k}, T^k_{obs})}{\varsigma(t_{n-k}, T^k_{obs})}\right)\psi^i_{k},\\
u^{i,b}_{0,k}=\hat{x}^i_{0,k+1}.\\
\end{cases}
\end{equation*}
It should be noted that equation (\ref{Inferk}) holds, which leads to ${\dot{V}^{b}_{k}}=0$, $\forall t\geq t_{n-k}$. Thus, we have
\begin{align}\label{10eq1}
\dot{V}_k(t)
=&\sum_{i=1}^{N}\rho_i {\psi}_k^i\times\left(\sum_{j \in \mathcal{N}_{i}}a_{ij}(u^{i,a}_{0,k}-u^{j,a}_{0,k})+b_{i}(u^{i,a}_{0,k})\right)\nonumber\\
\leq&-2\left(\alpha+\beta\frac{\dot{\varsigma}(t_{n-k}, T^k_{obs})}{\varsigma(t_{n-k}, T^k_{obs})}\right) \frac{\lambda_1(M(L_0))}{{\max}\{\rho\}}{V}_k(t), ~\forall t\geq t_{n-k}.\nonumber
\end{align}
Following the similar procedure as Step 1, we could obtain that ${V}_k=0$ and thus $\psi_k= \tilde{x}_{0,k}=\boldsymbol{0}_{N}$, $\forall t\geq t_{n+1-k}$. Thus, the DPTO (\ref{DPTO}) can reconstruct the $k$th leader state no latter than a prescribed time instant $t_{n+1-k}$, as long as the $(k+1)$th leader state can be accurately estimated within $t_{n-k}$, $\forall k\in \textbf{I}(1,n-1)$.
%$\tilde{x}^i_{0,q}$

Based on the above analysis, we obtain $\tilde{x}_{0,k}=\boldsymbol{0}_{N}$, $\forall t\geq t_0+T_{obs}$ and $k\in \textbf{I}(1,n)$. Therefore, Theorem \ref{Plemma800} is proved.
$\hfill \hfill \blacksquare $
\vspace{0.2cm}

\begin{remark}
  For simplicity, herein we investigate the design of DPTO w.r.t. the leader with high-order integrator dynamics (\ref{condition_variate}). In fact, this DPTO could be extended to the leader with certain high-order strict-feedback dynamics in a canonical form \cite{zhang2015leader, holloway2019prescribed}, whose estimation process can also be implemented in a cascaded manner.
  $\hfill \hfill \square $
\end{remark}















































\subsection{Time-varying Topology Cases}

Let $\mathcal{G}_{\sigma(t)}$ denote a time-varying digraph switching among $p$ kinds of possible directed topologies, where $\sigma(t)\in\textbf{I}[1,p]$ denotes the index associated with the directed topology at the time instant $t$. Denote the corresponding sub-Laplcian matrix of $\mathcal{G}_{\sigma(t)}$ as $L_{\sigma(t)}$, which is a counterpart of $L_0$ associated with $\mathcal{G}$.

The protocol (\ref{DPTO}) could also be extended to some time-varying digraphs satisfying the following assumption:% \cite{liu2019leader}

\begin{assumption} \label{assumption0402}
  The graph $\mathcal{G}_{\sigma(t)}$ contains a spanning tree rooted by the leader all the time. Furthermore, there exists a diagonal matrix $H:={\rm diag}(\eta)>0$ with $\eta=[\eta_1,\ldots~,\eta_N]^{\mathrm{T}}$ such that $M(L_j):=\frac{1}{2}(HL_j+L_j^{\mathrm{T}}H) > 0$, $\forall j\in\emph{\textbf{I}}[1,p]$.
\end{assumption}

\begin{remark}
Notice that Assumption \ref{assumption040} is a special case of Assumption \ref{assumption0402}. Also, the assumption used in \cite{hong2008distributed} can also been seen as a special case of Assumption \ref{assumption0402}, where each $\mathcal{G}_i$, $i\in\emph{\textbf{I}}[1,p]$, is reachable from the leader and all
  these subgraphs among followers are undirected and connected. Actually, the constraint brought by Assumption \ref{assumption0402} is not too strict since it is feasible to obtain a diagonal matrix $H$ in many time-varying directed topology cases. One such example can be found in our simulation part in Section \ref{SecSm}.$\hfill \hfill \square $
\end{remark}

By replacing $\psi^i_{k}$ in (\ref{DPTO}) with $\psi^{i,{\sigma(t)}}_{k}=b_i^{\sigma(t)}(\hat{x}^i_{0,k}-x_{0,k})+\sum_{i=1}^Na_{ij}^{\sigma(t)}(\hat{x}^i_{0,k}-\hat{x}^j_{0,k})$, we have the following corollary, which is a natural extension of Theorem \ref{Plemma800}:

\begin{myCollo}\label{Theo3}
  Consider the leader with high-order dynamics (\ref{condition_variate}), if Assumptions \ref{assumption035} and \ref{assumption0402} hold, then the DPTO (\ref{DPTO}) satisfying
\begin{subequations}\label{DPTOparam2}
  \begin{numcases}{}
    \alpha >0,\label{DPTOparam2a}\\
    \beta\geq \frac{\max_{i\in \emph{\textbf{I}}[1,N]}\{\eta_i\}}{\min_{j\in \emph{\textbf{I}}[1,p]}\lambda_1(M(L_j))},\label{DPTOparam2b}\\
    \sigma \geq \bar{f}_0.\label{DPTOparam2c}
  \end{numcases}
  \end{subequations}
  can ensure the estimation error of each follower w.r.t. the leader state converges into zero in a prescribed time interval $T_{obs}$.
\end{myCollo}

\textbf{Proof.}
The proving process is similar to Theorem \ref{Plemma800} and hence omitted here.
$\hfill \hfill \blacksquare $
\vspace{0.2cm}

\begin{remark}
  The main merits of the DPTO (\ref{DPTO}) can be summarized as:
   \begin{enumerate}[{\rm 1)},itemsep= 0 pt, topsep =1ex, itemindent=-0em, listparindent = 0 pt]
    \item One of recent distinguished distributed finite-time observers on digraphs is summarized in \cite[Theorem 2]{zuo2019distributed}, which achieved distributed estimation w.r.t. the leader in finite-time cooperatively ultimately uniformly boundedness sense. The needed time interval to achieve uniform bounded estimation error depended directly on the initial conditions as well as the generalized network algebraic connectivity \cite{wu2005algebraic} of digraphs. Compared with \cite[Theorem 2]{zuo2019distributed}, our DPTO on digraphs in Theorem \ref{Plemma800} has the merit of prescribed-time zero-error convergence. Also, the parameter design principle in Theorem \ref{Plemma800} is much simpler than that in \cite[Theorem 2]{zuo2019distributed}. These user-friendly merits bring great convergence to the real applications of distributed consensus observers. 
\item Compared with our previous DPTO in \cite[Eq. 34]{9311845}, this new DPTO can achieve prescribed-time performance without using the real-time control input of the leader, which facilitates the real applications of DPTO since the control input is generally not available for the pinned followers. Herein, only the assumption on the upper boundedness of the leader input is needed, which is quite reasonable and general since all force-driven robots in real applications are actuation bounded.%Also, the subgraph among followers in \cite{wang2018leader} is confined in undirected and connected graphs, which is a special case of the topology studied in this brief.
\item This work first considers the extension of the topologies of distributed finite-time consensus observer from time-invariant digraphs to time-varying ones. Besides, the obtained parameter design principle (\ref{DPTOparam2}) in Corollary \ref{Theo3} is also quite simple and clear to be deployed, as the one (\ref{DPTOparam1}) in Theorem \ref{Plemma800}.
  %\item The above observer could work anonymously, that is, the agent can achieve prescribed-time estimation without anticipating the exact identity (follower or leader)  of its neighbors.
  \item The communication loop problem stated in \cite{khoo2014multi} can be avoided.$\hfill \hfill \square $
  %eradicate the notorious communication loop problem \cite{khoo2014multi} encountered in traditional convey communication mechanism.
\end{enumerate}
\end{remark}



% Assumption 5.1: For any p ∈ P, every node i = 1,...,N is reachable from node 0 in the digraph G¯p and there exists a positive definite
% diagonal matrix D = diag(d1,...,dN ) such that DHp + HT
% p D is
% positive definite.



% \begin{remark}
% The main merits of our distributed observer-based protocol are summarized as follows:
% \begin{enumerate}[{\rm 1)},itemsep= 0 pt, topsep =1ex, itemindent=-0em, listparindent = 0 pt]
% \item \textbf{Detected graph}: Most of the previous works \cite{2zuo2014adaptive, 4chen2014fault, 7deng2019distributed} concerned about distributed fault-tolerant are developed under the premise that the subgraph among the followers is undirected, for the technical difficulties induced by the asymmetrical Laplacian matrices.  Here, we focus on the distributed fault-tolerant control on a more general scope of topology, that is, directed and two-layered graphs whose subgraph of followers is strongly-connected;
% \item  \textbf{PT characteristics}:  The settling time interval to achieve CUUB performance in traditional distributed finite-time control approach \cite{jin2016fault, sakthivel2019finite,5wang2016fault, 8wu2020fuzzy} depends heavily on the two factors, that is, the specific topology among agents and initial states of all agents. Our work can achieve distributed fault-tolerant control in a prescribed-time CUUB sense, whose needed time interval is immune to the above two factors. This merit can bring great convenience to the controller designers.
% \item \textbf{Zero-error Convergence}:  Most of the previous works \cite{chen2015robust, wang2015robust, deng2016distributed, jin2017adaptive, khalili2018distributed,3xiao2021distributed, 4chen2014fault, 5wang2016fault, 6chen2015robust, 8wu2020fuzzy} could only achieve CUUB convergence, while little works \cite{2zuo2014adaptive, 7deng2019distributed} have asymptotical convergence. However, it should be noted that only multiplicative fault is considered in \cite{2zuo2014adaptive, 7deng2019distributed}. This work manages to achieve zero-error convergence against both the additive and multiplicative faults. Also, unlike most of the previous works \cite{2zuo2014adaptive, 3xiao2021distributed, 5wang2016fault, 6chen2015robust, 7deng2019distributed, 8wu2020fuzzy}, we do not need the assumption that $\bar{\varrho}\leq 1$, that is, the multiplicative fault must be under-actuated.
% \item Here, we consider the distributed fault-tolerant problem in a two-layered FC framework, where the followers and the leader are all non-autonomous (the input signals are non-zero, time-varying, and fault-polluted). This setting makes the design of the DPTO and the estimation of the upper bound of the residual set much more complex and challenging.
% \end{enumerate}
% \end{remark}
%\cite{1wang2019zero, 2zuo2014adaptive, 3xiao2021distributed, 4chen2014fault, 5wang2016fault, 6chen2015robust, 7deng2019distributed, 8wu2020fuzzy}

% \section{Numerical Simulation}

% \begin{figure}[!htbp]
% %\begin{minipage}[t]{1\linewidth}
% \centering
% \includegraphics[width=0.6\textwidth]{4Ag.pdf}
% \caption{Directed communication topology among all agents}
% \label{fig:figure1}
% \end{figure}

% %{\color{blue}
% %\begin{figure}[htbp]
% %\centering
% %\subfigure[Performance of observer w.r.t. the leader]{
% %\begin{minipage}[t]{0.475\textwidth}
% %\centering
% %\includegraphics[width=0.85\textwidth]{pic/pobs.eps}
% %%\caption{fig1}
% %\end{minipage}\label{fig:figure2:1}
% %}
% %%\hspace{-0.1in}
% %\subfigure[Performance of observer w.r.t. the first leader]{
% %\begin{minipage}[t]{0.475\textwidth}
% %\centering
% %\includegraphics[width=0.85\textwidth]{pic/vobs.eps}
% %%\caption{fig2}
% %\end{minipage}\label{fig:figure2:2}
% %}\\%
% %\centering
% %\caption{Performance of two observers}
% %\label{fig:figure2}
% %\end{figure}

% \begin{figure}[htbp]
% \centering
% \subfigure[Estimation performance w.r.t. the acceleration of L0]{
% \begin{minipage}[t]{0.475\textwidth}
% \centering
% \includegraphics[width=0.85\textwidth]{pic/est0a.eps}
% \label{fig:figure2:1}
% %\caption{fig1}
% \end{minipage}%
% }%
% \\
% %\hspace{-0.2in}
% \subfigure[Estimation performance w.r.t. the velocity of L0]{
% \begin{minipage}[t]{0.475\textwidth}
% \centering
% \includegraphics[width=0.85\textwidth]{pic/est0v.eps}
% \label{fig:figure2:2}
% %\caption{fig2}
% \end{minipage}%
% }%
% %\subfigure[Estimation performance w.r.t. the displacement of leader0]{
% %\begin{minipage}[t]{0.475\textwidth}
% %\centering
% %\includegraphics[width=0.85\textwidth]{pic/est0p.eps}
% %\label{fig:figure2:3}
% %%\caption{fig1}
% %\end{minipage}%
% %}%
% \\
% %\subfigure[Estimation performance w.r.t. the velocity of L1]{
% %\begin{minipage}[t]{0.475\textwidth}
% %\centering
% %\includegraphics[width=0.85\textwidth]{pic/est1v.eps}
% %\label{fig:figure2:4}
% %%\caption{fig2}
% %\end{minipage}
% %}%
% %\hspace{-0.2in}
% \subfigure[Estimation performance w.r.t. the displacement of L0]{
% \begin{minipage}[t]{0.475\textwidth}
% \centering
% \includegraphics[width=0.85\textwidth]{pic/est0p.eps}
% \label{fig:figure2:3}
% %\caption{fig2}
% \end{minipage}
% }%
% \centering
% \caption{ {\color{blue}Performance of the DPTO: the shadow areas denote time intervals using time-varying gains.}}
% \label{fig:figure2}
% \end{figure}







% Consider a triple-integrator MAS including $1$ leader and $3$ followers. Their communication topology is shown in Fig. \ref{fig:figure1}, which satisfies Assumption \ref{assumption040}. To be more concrete, agent $0$ represents the leader, while agents $1\sim 3$ stand for the three followers. 
% The Laplace matrix corresponding to the above MAS is
% \begin{equation*}
% L
% =
%  \left[
% 	\begin{array}{c|ccc}
% 0& 0 &  0  & 0  \\ \hline
%  -1& 2 &  0  & -1  \\
%  0& -1 &  1  & 0   \\
%  0& 0 &  -1  & 1  \\ 
% 	\end{array}
% 	\right].
% \end{equation*}
% %0& 0 &  0  & 0 & 1 & 0 & -1 \\
% % 0& 0 &  0  & 0 & -1 & 1 & 0 \\
% % 0& 0 &  0  & 0 & 0 & -1 & 1\\
%  %The formation pattern of the three leaders is defined as
% %\begin{equation*}\label{leaderformpattern}
% %\varpi_i=[0.5\cos((i-4)\pi/3),0.5\sin((i-4)\pi/3)]^{\mathrm{T}},\ i\in \textbf{I}[1,n].
% %\end{equation*}
% %v00-1/2*sin(t/2); v00-1/2*sin(t/2);-1/4*cos(t/2);-1/4*cos(t/2)
% Let $t_0=0$ s. 
% The dynamic of the leader is $\dot{x}_{0,1}=x_{0,2}$, $\dot{x}_{0,2}=x_{0,3}$, and $\dot{x}_{0,3}=\frac{1}{8}\sin(\frac{t}{2}) $ with ${x}_{0,1}(t_0)=1$, ${x}_{0,2}(t_0)=0$ and ${x}_{0,3}(t_0)=0$. Thus, $\bar{f}_0$ can be chosen as $\frac{1}{8}$. The following design parameters are applied in the simulation: $T_{obs}=0.6$ s, $T_{obs}^1=T_{obs}^2=T_{obs}^3=\frac{T_{obs}}{3}=0.2$ s and $h=2.01>2$.
% %$\bar{\varrho}_i$ and $\underline{\varrho}_i$ denotes the upper bound and lower bound of the modulus of $\varrho_i(t)$, respectively. Besides, the modulus of the additive actuator fault $r_i$ is upper bound, that is, $|r_i|\leq \bar{r_i}

% For the DPTO (\ref{DPTO}) to estimate the states of the leader, the initial displacement estimations of all leaders are set as: $\hat{x}_{0,1}(0)=[\hat{x}^{1}_{0,1}(0),\ \hat{x}^{2}_{0,1}(0),\ \hat{x}^{3}_{0,1}(0)]^{\mathrm{T}}=[0.4,\ 0.8,\ 0.6]^{\mathrm{T}}$. Similarly, the initial velocity estimation of all leaders is selected as $\hat{x}_{0,2}(0)=[0.6,\ 0.5,\ 0.4]^{\mathrm{T}}$, and the initial acceleration estimation of all leaders is selected as $\hat{x}_{0,3}(0)=[0.3,\ 0.7,\ 0.5]^{\mathrm{T}}$. By recalling Theorem \ref{Plemma800}, we choose the parameters of the DPTO in (\ref{DPTO}) as: $\alpha=1.05$, $\beta=5.692$ and $\sigma=0.125$. The performance of this observer is shown in Fig. \ref{fig:figure2}. As shown in Fig. \ref{fig:figure2:1}, the global acceleration estimation error of this observer will remain zero after the predefined time point $t_0+T^1_{obs}=0.2$ s. Similarly, the global velocity estimation error of this observer will remain zero after the predefined time point $t_0+T^1_{obs}+T^2_{obs}=0.4$ s, and the global displacement estimation error of this observer will remain zero after the predefined time point $t_0+T_{obs}=0.6$ s, as shown in Fig. \ref{fig:figure2:3} and Fig. \ref{fig:figure2:3}, respectively.


\section{Numerical Simulation}\label{SecSm}

% \begin{figure}[!htbp]
% %\begin{minipage}[t]{1\linewidth}
% \centering
% \includegraphics[width=0.6\textwidth]{4Ag.pdf}
% \caption{Time-varying directed communication topology among all agents}
% \label{fig:figure1}
% \end{figure}

\begin{figure}[htbp]
  \centering
  \subfigure[Digraph 1]{
  \begin{minipage}[t]{0.475\textwidth}
  \centering
  \includegraphics[width=0.85\textwidth]{pic/4Ag.pdf}
  \label{fig:figure1:1}
  %\caption{fig1}
  \end{minipage}%
  }%
  %\hspace{-0.2in}
  \subfigure[Digraph 2]{
  \begin{minipage}[t]{0.475\textwidth}
  \centering
  \includegraphics[width=0.85\textwidth]{pic/4Bg.pdf}
  \label{fig:figure1:2}
  %\caption{fig2}
  \end{minipage}%
  }
  \centering
  \caption{Time-varying directed communication topology among all agents: The periodically switching sequence is that the initial topology is digraph 1 and the switching occurs every $0.1$ s.}
  \label{fig:figure1}
  \end{figure}

%{\color{blue}
%\begin{figure}[htbp]
%\centering
%\subfigure[Performance of observer w.r.t. the leader]{
%\begin{minipage}[t]{0.475\textwidth}
%\centering
%\includegraphics[width=0.85\textwidth]{pic/pobs.eps}
%%\caption{fig1}
%\end{minipage}\label{fig:figure2:1}
%}
%%\hspace{-0.1in}
%\subfigure[Performance of observer w.r.t. the first leader]{
%\begin{minipage}[t]{0.475\textwidth}
%\centering
%\includegraphics[width=0.85\textwidth]{pic/vobs.eps}
%%\caption{fig2}
%\end{minipage}\label{fig:figure2:2}
%}\\%
%\centering
%\caption{Performance of two observers}
%\label{fig:figure2}
%\end{figure}

\begin{figure}[htbp]
\centering
\subfigure[Estimation performance w.r.t. the acceleration of L0]{
\begin{minipage}[t]{0.475\textwidth}
\centering
\includegraphics[width=0.85\textwidth]{pic/est0a.eps}
\label{fig:figure2:1}
%\caption{fig1}
\end{minipage}%
}%
\\
%\hspace{-0.2in}
\subfigure[Estimation performance w.r.t. the velocity of L0]{
\begin{minipage}[t]{0.475\textwidth}
\centering
\includegraphics[width=0.85\textwidth]{pic/est0v.eps}
\label{fig:figure2:2}
%\caption{fig2}
\end{minipage}%
}%
%\subfigure[Estimation performance w.r.t. the displacement of leader0]{
%\begin{minipage}[t]{0.475\textwidth}
%\centering
%\includegraphics[width=0.85\textwidth]{pic/est0p.eps}
%\label{fig:figure2:3}
%%\caption{fig1}
%\end{minipage}%
%}%
\\
%\subfigure[Estimation performance w.r.t. the velocity of L1]{
%\begin{minipage}[t]{0.475\textwidth}
%\centering
%\includegraphics[width=0.85\textwidth]{pic/est1v.eps}
%\label{fig:figure2:4}
%%\caption{fig2}
%\end{minipage}
%}%
%\hspace{-0.2in}
\subfigure[Estimation performance w.r.t. the displacement of L0]{
\begin{minipage}[t]{0.475\textwidth}
\centering
\includegraphics[width=0.85\textwidth]{pic/est0p.eps}
\label{fig:figure2:3}
%\caption{fig2}
\end{minipage}
}%
\centering
\caption{Performance of the DPTO: the shadow areas denote time intervals using time-varying gains.}
\label{fig:figure2}
\end{figure}







Consider a triple-integrator MAS, where agent $0$ represents the leader, while agents $1\sim 3$ stand for the three followers. Their communication topology is shown in Fig. \ref{fig:figure1}, which satisfies Assumption \ref{assumption0402} together with the selection of $H={\rm diag}(3,5,4)$. 
The Laplace matrices corresponding to the above MAS are
\begin{equation*}
L_1
=
 \left[
	\begin{array}{ccc}
 2 &  0  & -1  \\
-1 &  1  & 0   \\
 0 &  -1  & 1  \\ 
	\end{array}
	\right],~L_2
  =
   \left[
    \begin{array}{ccc}
   1 &  0  & 0  \\
  -1 &  1  & 0   \\
   -1 & 0  & 1  \\ 
    \end{array}
    \right].
\end{equation*}
%0& 0 &  0  & 0 & 1 & 0 & -1 \\
% 0& 0 &  0  & 0 & -1 & 1 & 0 \\
% 0& 0 &  0  & 0 & 0 & -1 & 1\\
 %The formation pattern of the three leaders is defined as
%\begin{equation*}\label{leaderformpattern}
%\varpi_i=[0.5\cos((i-4)\pi/3),0.5\sin((i-4)\pi/3)]^{\mathrm{T}},\ i\in \textbf{I}[1,n].
%\end{equation*}
%v00-1/2*sin(t/2); v00-1/2*sin(t/2);-1/4*cos(t/2);-1/4*cos(t/2)
Let $t_0=0$ s. 
The dynamic of the leader is $\dot{x}_{0,1}=x_{0,2}$, $\dot{x}_{0,2}=x_{0,3}$, and $\dot{x}_{0,3}=\frac{1}{8}\sin(\frac{t}{2}) $ with ${x}_{0,1}(t_0)=1$, ${x}_{0,2}(t_0)=0$ and ${x}_{0,3}(t_0)=0$. Thus, $\bar{f}_0$ can be chosen as $\frac{1}{8}$. The following design parameters are applied in the simulation: $T_{obs}=0.6$ s, $T_{obs}^1=T_{obs}^2=T_{obs}^3=\frac{T_{obs}}{3}=0.2$ s and $h=2.01>2$.
%$\bar{\varrho}_i$ and $\underline{\varrho}_i$ denotes the upper bound and lower bound of the modulus of $\varrho_i(t)$, respectively. Besides, the modulus of the additive actuator fault $r_i$ is upper bound, that is, $|r_i|\leq \bar{r_i}

For the DPTO (\ref{DPTO}) to estimate the states of the leader, the initial displacement estimations of all leaders are set as: $\hat{x}_{0,1}(0)=[\hat{x}^{1}_{0,1}(0),\ \hat{x}^{2}_{0,1}(0),\ \hat{x}^{3}_{0,1}(0)]^{\mathrm{T}}=[0.4,\ 0.8,\ 0.6]^{\mathrm{T}}$. Similarly, the initial velocity estimation of all leaders is selected as $\hat{x}_{0,2}(0)=[0.6,\ 0.5,\ 0.4]^{\mathrm{T}}$, and the initial acceleration estimation of all leaders is selected as $\hat{x}_{0,3}(0)=[0.3,\ 0.7,\ 0.5]^{\mathrm{T}}$. By recalling Corollary \ref{Theo3}, we choose the parameters of the DPTO in (\ref{DPTO}) as: $\alpha=1.05$, $\beta=5.692$ and $\sigma=0.125$. The performance of this observer is shown in Fig. \ref{fig:figure2}. As shown in Fig. \ref{fig:figure2:1}, the global acceleration estimation error of this observer will remain zero after the predefined time point $t_0+T^3_{obs}=0.2$ s. Similarly, the global velocity estimation error of this observer will remain zero after the predefined time point $t_0+T^3_{obs}+T^2_{obs}=0.4$ s, and the global displacement estimation error of this observer will remain zero after the predefined time point $t_0+T_{obs}=0.6$ s, as shown in Fig. \ref{fig:figure2:2} and Fig. \ref{fig:figure2:3}, respectively.


\section{Conclusion}





We investigate the distributed prescribed-time consensus observer for multi-agent systems with
high-order integrator dynamics and directed topologies in this brief.  To our best knowledge, the DPTO on time-invariant/varying digraphs with prescribed-time zero-error convergence has been formulated for the first time, which could achieve distributed estimation w.r.t. the leader state within an arbitrary time interval predefined by the users. An illustrative simulation example has been conducted, which confirms the prescribed-time performance of the above DPTO. In future work, it will be an interesting topic to use this prescribed-time observer to deal with fault-tolerant \cite{3xiao2021distributed} and attack-resilient \cite{9382951} control problems.



\bibliographystyle{IEEEtran}


\bibliography{PIDFR}

\end{document}
