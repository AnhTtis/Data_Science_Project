


\documentclass[letterpaper, 12pt, journal]{support/IEEEtran}
\usepackage[fleqn]{amsmath}
\usepackage{times}
\usepackage[pdftex]{graphicx}
\usepackage{subfigure}
\usepackage{amsmath,amssymb,amsopn,amstext,amsfonts}
\usepackage{cancel}
\usepackage[noadjust]{cite}
\usepackage{soul}
\usepackage{caption}

\captionsetup{font={small}}

\captionsetup[figure]{labelfont={},textfont={}}


\usepackage{balance}
\usepackage{color}
\usepackage{mathtools}
% \usepackage{algorithm}
% \usepackage{algorithmic}
\usepackage{bm}
%\newtheorem{theorem}{Theorem}
\usepackage{diagbox}
\usepackage{float}
\usepackage{epstopdf}
\usepackage{url}
\usepackage{multirow}
\usepackage{tikz}
\usepackage{subeqnarray}
\usepackage{cases}
\usepackage{booktabs}
\usepackage[linkcolor=black,citecolor=black,urlcolor=black,colorlinks=true]{hyperref}
\usepackage{algorithm}
\usepackage[noend]{algpseudocode}
\newtheorem{myTheo}{Theorem}
%\newtheorem{thm}{Theorem}[section] %如果不采用章节号做前缀,则不用[section]
\newtheorem{myDef}{Definition} %这句定义使得defn环境和thm共享编号
\newtheorem{lemma}{Lemma} %这句定义使得lem环境和thm共享编号
\newtheorem{myCollo}{Corollary}
\newtheorem{remark}{Remark}
%\newtheorem{lemma}{Lemma}
\newtheorem{myPro}{Proposition}
\newtheorem{assumption}{Assumption}
\newtheorem{example}{Example}
\soulregister\cite7
\soulregister\citep7
\soulregister\citet7
\soulregister\ref7
\soulregister\it7
\soulregister\pageref7

\bibliographystyle{support/IEEEtran}

\newcommand\px{\mathrel{/\mkern-5mu/}}  %平行
\newcommand{\ann}[1]{%
    \begin{tikzpicture}[remember picture, baseline=-0.75ex]%
        \node[coordinate] (inText) {};%
    \end{tikzpicture}%
    \marginpar{%
        \renewcommand{\baselinestretch}{1.0}%
        \begin{tikzpicture}[remember picture]%
            \definecolor{orange}{rgb}{1,0.5,0}%
            \draw node[fill=red!20,rounded corners,text width=\marginparwidth] (inNote){\footnotesize#1};%
    \end{tikzpicture}%
    }%
    \begin{tikzpicture}[remember picture, overlay]%
        \draw[draw = orange, thick]
            ([yshift=-0.2cm] inText)
                -| ([xshift=-0.2cm] inNote.west)
                -| (inNote.west);%
    \end{tikzpicture}%
}%

\graphicspath{{figures/}}
\DeclareGraphicsExtensions{.pdf,.png,.jpg,.eps}
\IEEEoverridecommandlockouts
%\overrideIEEEmargins

\title{\LARGE \bf Output Leader-following Consensus of Multi-agent system against Byzantine Attacks: A Hierarchical Approach}
%\title{Distributed Optimization in Prescribed-Time: Theory and Experiment}%
\author{
  \vskip 1em
  { 
   Xin Gong, \emph{Graduate Student Member, IEEE}, Yiwen Liang,
   Yukang Cui, \emph{Member, IEEE},
 
  }

  \thanks{
    This work was partially supported by the National Natural Science Foundation of China under Grant 61903258, 61973156, 61603180, Qatar National Research Fund NPRP12C-0814-190012. %(\emph{Corresponding author: Yukang Cui.}) %the National Natural Science Foundation of China under Grant 61903258

X. Gong is with the Department of Mechanical Engineering, The University of Hong Kong, Pokfulam Road, Hong Kong (e-mail: {\tt\small gongxin@connect.hku.hk}).


Y. Cui and Y. Liang are with the College of Mechatronics and Control Engineering, Shenzhen University, Shenzhen, 518060, China (e-mail: {\tt\small cuiyukang,szuwtn@gmail.com}).


  
%J. He is with the Department of Mechanical Engineering, The University of Hong Kong, Pokfulam Road, Hong Kong (e-mail: {\tt\small esmehe@connect.hku.hk}). 

%X. Gong is with the Department of Mechanical Engineering, The University of Hong Kong, Pokfulam Road, Hong Kong, and also with the College of Mechatronics and Control Engineering, Shenzhen University, Shenzhen 518060, China. (e-mail: {\tt\small gongxin@connect.hku.hk}).
%China, and also
%with the Department of Mechanical Engineering, University of Hong Kong,
%Hong Kong
    
  }
%\thanks{$^{*}$ means the corresponding author.}
}

%\maketitle
%\author{}%\vspace{-0.0cm}
%%\thanks{This work was partially supported by.}% <-this % stops a space
%\thanks{$^{*}$These authors contribute equally and share the first authorship.}
%\thanks{$^{1}$Author is with the Group Robotics with Intelligent Planning (GRIP) Lab, Department of Mechanical Engineering, University of Hong Kong, Hong Kong,
%   {\tt\small gongxin@connect.hku.hk}}
%\thanks{Digital Object Identifier (DOI): see the top of this page.}
%\vspace{-0.5cm}}

% The note headers
%\markboth{Journal of \LaTeX\ Class Files,~Vol.~14, No.~8, August~2015}%
%{Shell \MakeLowercase{\textit{et al.}}: Bare Demo of IEEEtran.cls for IEEE Journals}
\markboth{IEEE Transactions on ...}{GONG \MakeLowercase{\textit{et al.}}: Distributed Prescribed-time Consensus Observer on Directed Graphs}%{He \MakeLowercase{\textit{et al.}}: Resilient Path Planning of UAVs against Covert Attacks on UWB Sensors}



\begin{document}
  \maketitle
  \begin{abstract}
    This paper studies the output consensus of heterogeneous multi-agent systems against Byzantine attacks. In order to achieve consensus, agents need to share information with neighbors through communication channels, which may be subject to malicious network attacks. There is a type of Byzantine nodes in the network, which can send different false messages to neighbors. We decompose Byzantine attacks into node attacks on the cyber-physical layer (CPL) and edge attacks on the 
   twin layer (TL). On the twin layer, we use mean-sequential-reduction (MSR) algorithm to remove the extreme values received by the agents. On the cyber-physical layer, we design a  decentralized controller to resist Byzantine node attacks. Our goal is to enable each follower to track the desired signal. Finally, the stability of the multi-intelligent system is analyzed by Lyapunov method. The effectiveness of the simulation results is verified by comparing simulation examples.
\end{abstract}
\begin{IEEEkeywords}
     Output consensus, Multi-agent systems, Byzantine attacks, Resilient control
% Periodic positive systems, hyper-pyramid,
% reachable set estimation, S-procedure, state-feedback control.
%Formation-containment control,  high-order multi-agent systems,  observer-type protocols,  time-varying formation configuration
\end{IEEEkeywords}
\section{Introduction}
\IEEEPARstart{T}{he}  %; moreover, $A>0$ means that $\lambda_1(A)>0$.
%For a time-varying function $x(t): \mathbb{R}_{\geq 0 }\mapsto \mathbb{R}$, denote that $\sup_{t\in [t_0, t_1]} x(t) $ and $\inf_{t\in [t_0, t_1]} x(t)$ as the upper bound and lower bound of $x(t)$ over the time interval $[t_0, t_1]$, respectively. Moreover, denote that $\|x(t)\|_{[t_0, t_1]} =\sup_{t\in [t_0, t_1]} \|x(t)\| $. Define that $L_{\infty}:=\{x(t)|x(t): \mathbb{R}_{\geq 0 }\mapsto \mathbb{R}^n,\ \|x(t)\|_{[t_0, t_1]}<\infty\}$. In the following sections, $x(t) \in L_{\infty}$, $t\in [t_0, t_1]$, represents that the variable $x$ is uniformly bounded over $[t_0, t_1]$.   %$A\succeq 0$ (or $A\succ 0$) denotes that $A$ is a nonnegative matrix (positive matrix, respectively), which means all elements of $A$ are nonnegative (positive, respectively).
 %${\rm span}(x)$ denotes the span vector of a given vector $x=[p_1, p_2,\ldots~, p_n]^{\mathrm{T}}\in \mathbb{R}^n$.






%; moreover, $A>0$ means that $\lambda_1(A)>0$.
%For a time-varying function $x(t): \mathbb{R}_{\geq 0 }\mapsto \mathbb{R}$, denote that $\sup_{t\in [t_0, t_1]} x(t) $ and $\inf_{t\in [t_0, t_1]} x(t)$ as the upper bound and lower bound of $x(t)$ over the time interval $[t_0, t_1]$, respectively. Moreover, denote that $\|x(t)\|_{[t_0, t_1]} =\sup_{t\in [t_0, t_1]} \|x(t)\| $. Define that $L_{\infty}:=\{x(t)|x(t): \mathbb{R}_{\geq 0 }\mapsto \mathbb{R}^n,\ \|x(t)\|_{[t_0, t_1]}<\infty\}$. In the following sections, $x(t) \in L_{\infty}$, $t\in [t_0, t_1]$, represents that the variable $x$ is uniformly bounded over $[t_0, t_1]$.   %$A\succeq 0$ (or $A\succ 0$) denotes that $A$ is a nonnegative matrix (positive matrix, respectively), which means all elements of $A$ are nonnegative (positive, respectively).
 %${\rm span}(x)$ denotes the span vector of a given vector $x=[p_1, p_2,\ldots~, p_n]^{\mathrm{T}}\in \mathbb{R}^n$.






%新开始
\color{black}{
\section{Introduction}\label{section1}

Based on the above analyses, this paper focuses on resilient 
output consensus problems for heterogeneous MASs against Byzantine attacks. In order to solve this issue, the main contributions can be summarized as follows: 
\begin{enumerate}
\item Inspired by digital twin technology, we design a double-layer elastic control architecture, including TL and CPL. The TL can be arranged in virtual space, and there will be no Byzantine node attacks. Because of this feature of the TL, the resilient control strategy can be decomposed into  the defense against  Byzantine edge attacks on the TL and  the defense against Byzantine node attacks on the CPL.
\item  Improve the robustness of the network. We strictly prove that strongly $(2f+1)$-robustness is still effective, rather than $(3f+1)$-robustness. 
\item A decentralized controller is designed. Different from the existing work, we ensure that Byzantine agents and normal agents perform well. The control goal is to make the error of each follower and leader is  uniformly ultimately bounded (UUB).
\end{enumerate}


\noindent\textbf{Notations:}
In this paper, $\sigma_{{\rm min}}(S)$, $\sigma_{{\rm max}}(S)$, $\sigma(S)$ are the minimum singular value,
the maximum singular value, and the spectrum of matrix $S$,
respectively. $|S|$ is the cardinality of a set S. $\Vert \cdot \Vert$ is the Euclidean norm of a vector. Define the sets of positive real numbers, positive integers as $\mathbb{R}_{> 0}$ and $\mathbb{Z}_{> 0}$.

\section{Preliminaries}
\subsection{Robust Graph Theory}
In this article, a group of agents are be considered. Define a digraph $\mathcal{G } =(\mathcal{V}, \mathcal{E}, \mathcal{A} )$, where $\mathcal{V}=\{ 1, 2, \ldots~ , N \}$ indicates  the edge set the vertex set, $\mathcal{E} \subset \mathcal{V} \times \mathcal{V}$ indicates the edge set. And the associated adjacency matrix $\mathcal{A} = [a_{ij}]$. An edge rooting from node $j$ and ending at node $i$ is denoted by ($v_j$, $v_i$), meaning the information flow from node
$j$ to node $i$. $[a_{ij}]$ is the weight of edge
($v_j$, $v_i$), and $a_{ij} > 0$ if $(v_j, v_i) \in  \mathcal{E} $, otherwise $a_{ij} = 0$. $b_{i0}> 0 $ if there is a link between the leader and the ith follower, otherwise $b_{i0} = 0$.


\begin{myDef}[r-reachable set]
Consider a graph $\mathcal{G } =(\mathcal{V}, \mathcal{E})$, a nonempty node subset $S \in \mathcal{V}$,  We say that $S$ is an r-reachable set if there exists at least one node in set $S$ with at least $r$ neighbors outside set $S$. 
\end{myDef}\label{def41}

The notion of $r$-reachable can be expressed as follows: Let $r \in \mathbb{Z}_{> 0}$, then define $\mathcal{X}_S \subseteq S$ to be the subset of nodes in S, that is 
\begin{equation}
\mathcal{X}_S=\{i \in \mathcal{S}:|\mathcal{N}_i \backslash S| \geq r  \}.
\end{equation}
\begin{myDef}[strongly r-robust w.r.t. $\mathcal{X}$]
Consider a graph $\mathcal{G } =(\mathcal{V}, \mathcal{E})$ and a nonempty set $\mathcal{X} \subseteq \mathcal{V}$. $\mathcal{G }$ is strongly r-robust w.r.t. $\mathcal{X}$ if every nonempty subset $\mathcal{Y} \subseteq \mathcal{V} \backslash \mathcal{X}$ is a $r$-reachable set.
\end{myDef}

\subsection{Metzler matrix}
Consider a Metzler matrix $A \in \mathbb{R}^{n \times n}$ with zero row sums. We denote the $\alpha$-digraph associated with it as $\mathcal{G}_\alpha = (\mathcal{V},\mathcal{E})$ and all edge weights are larger than $\alpha$. 
\begin{lemma}(\cite{1429377})
Consider the linear system 
\begin{equation}
\dot{x}=A(t)x.
\end{equation}

It is assumed that the system matrix is a bounded piecewise continuous function of time. Moreover, the system matrix $A(t)$ is a Metzler with zero row sums for every time $t$.  If there exists an index $k = \{1,\dots,n\}$,  a threshold value $\alpha \in \mathbb{R}_{> 0}$ and an interval length $T \in \mathbb{R}_{> 0}$ such that for all $t \in \mathbb{R}$ the $\alpha$-digraph associated with 
\begin{equation}
\int_t^{t+T}A(s)ds
\end{equation}
has the property that all nodes may be reached from the node $k$, then all components of any solution $x(t)$ of (1)  exponentially converge
to a common value as $t \rightarrow \infty$. 
\end{lemma}

\begin{lemma}([2])
Given any matrix $A$, let us define $\beta(A) \triangleq {\rm max}\{{\rm Re}(\lambda)|\lambda \in \sigma(A)  \}$. Then 
 for any $\kappa > 0$ and any $t \geq 0$, there exists $\Vert e^{At} \Vert \leq \psi(\kappa)e^{(\beta(A)+\kappa)t}$.

\end{lemma}



\section{Problem Formulation}
\subsection{System description and Related Assumptions}
Consider a group of $N + 1$ identical agents, where an agent labelled as 0 is assigned as the leader and the agents indexed by $1, …, N$ are referred to as followers.
The dynamics of the followers are represented by
\begin{equation}
\begin{split}
&\dot{x}_i(t)=A_ix_i (t)+B_i u_i(t),   \\
&y_i(t)=C_ix_i(t),
\end{split}
\end{equation}
where $x_i(t) \in \mathbb{R}^{n_i}$, $u_i(t) \in \mathbb{R}^{m_i}$, and $y_i(t) \in \mathbb{R}^p$ are the state, control
input, and output of the $i$th follower, respectively;  $A_i, B_i$ and $C_i$ are
the system, input, and output matrices, respectively.


   The dynamic of the leader is described as
\begin{equation}
\begin{cases}
\dot{x}_0(t)=S x_0(t),\\
y_0(t)=R x_0(t),
\end{cases}
\end{equation}
where $x_0(t) \in \mathbb{R}^q$ and $y_0(t) \in \mathbb{R}^p$ are the state and output of the  leader, respectively.
In (4) and (5), $A_i \in \mathbb{R}^{n_i \times n_i}$, $B_i \in \mathbb{R}^{n_i \times m_i}$, $C_i \in \mathbb{R}^{p_i \times n_i}$, $S \in \mathbb{R}^{n \times n}$, $R \in \mathbb{R}^{p\times n}$. 

Then we next make some assumptions.
\begin{assumption}
All the eigenvalues of $S$ are with zero real part.
\end{assumption}
\begin{assumption}
The pair ($A_i$, $B_i$) is stabilizable and ($A_i$, $C_i$) is detectable.
\end{assumption}
\begin{assumption}
For any $\lambda \in \sigma(S)$, it holds that\\
\begin{equation*}
{\rm rank} \bigg( \left[
  \begin{array}{ccc}
 A_i-\lambda I &  B_i   \\
C_i &  0   \\
  \end{array}
  \right]\bigg)=n_i+q.
\end{equation*}

\end{assumption}






\begin{lemma}([3])
With Assumptions $2$-$4$, for each follower $i$, there exist some matrices $\Gamma_i$ and $\Pi_i$ that satisfy the following regulator equations:
\begin{equation}
\begin{cases}
A_i \Pi_i +B_i\Gamma_i= \Pi_i S, \\
C_i \Pi_i-R=0.
\end{cases}
\end{equation}
\end{lemma}

In this paper, it is assumed that only some of the follower systems can observe $y_0(t)$. For the sake of illustration, we take the followers where the leader can be directly observed as the pinned followers. The set of these pinned followers is described as set $\mathcal{V}_p$. On the other hand, the remaining followers are regarded as non-pinned followers are collected in set $\mathcal{V}_{np}$.

Define the following local output consensus error:
\begin{equation}
e_i(t)=y_i(t)-y_0(t)
\end{equation}

Based on the above settings, the heterogeneous MASs in (4) are said to achieve output consensus if $\lim_{t \to \infty} e_i(t) = 0, \forall i \in \mathcal{G}$. 

\subsection{Attack Model}

\begin{figure}
  \centering
  \includegraphics[height=9cm, width=7cm]{a.PNG}
  \caption{Distributed attack-resilient against Composite Attacks}\label{image1}
\end{figure}

\subsubsection{Edge Attack}
We consider a subset $\mathcal{B} \subseteq \mathcal{V}$ of the nodes in the network to be adversarial. We assume that the nodes are completely aware of the network topology and the system dynamics of any agent. Let us denote the set of  agents as Byzantine agents. On the other hand, the normal agents are collected in the set $\mathcal{K}=\{1,2,\dots,k\}$, where $k$ is the number of normal agents in network. 

Byzantine agent can fully control the communication channel and can be allowed to send different false information to its in-neighbors and out-neighbors at any time.

\begin{assumption}[f-local Attack]
There exist at most $f $  Byzantine agents in the neighborhood of any agent. Namely, $|\mathcal{B}\cap \mathcal{N}^{in} _i|\leq f , \ \forall i \in (\mathcal{B} \cup \mathcal{K})$.
\end{assumption}

Here, we mainly consider the f-local attack model to deal with lots of Byzantine nodes in network. It is reasonable to assume that there exists at most f adjacent Byzantine nodes. Otherwise, it will be too pessimistic to protect the network.

\textbf{Remark 1.}
The maximum number of Byzantine agents is related to the network topology, so we can derive this upper bound $f$ from the communication topology.

\textbf{Remark 2.}
The Byzantine nodes can completely understand the network topology and the system
dynamics. Different from general malicious nodes, they can send arbitrary and different false data to different in-neighbors and  cooperate with other Byzantine nodes.


\subsubsection{Node Attack}
For Byzantine agents, attacker attacks the actuator to change the input signal of the system by adding the unbounded signal.  Hence, instead of the control input $u_i$ in (2), we may only obtain the following corrupted input signal:
\begin{equation}
\hat{u}_i=u_i+\psi_i
\end{equation}
where $\psi_i$ is the unknown and unbounded attack signal caused in the actuator and we assume that $[({\rm d}\Vert \psi_i \Vert )/ {\rm d}t]$ is bounded and $[({\rm d}\Vert \psi_i \Vert )/ {\rm d}t] \leq \bar{\kappa}$.

\textbf{Problem}(Attack-resilient output consensus control of MASs against Byzantine Attacks):
Consider the MASs in (4) and (5) on graphs suffered the compound attack, design  distributed protocols $u_i$ such that the global error $e_i$ is UUB.

\section{Main Results}
 


\subsection{Resilient Twin Layer Design}
In the paper, we use the virtual state $z_i(t)$ to introduce MSR algorithm. At any time $t$, each follower  always makes updates as below:

(1) Collect the status of all neighbor agents (except the leader if $i \in \mathcal{V}_p$) in a list $\Delta_i(t)$.

(2) The agents in $\Delta_i(t)$ are divided into $\overline{\Delta}_i(t)$ and $\underline{\Delta}_i(t)$ as follows :
\begin{equation}
\begin{split}
&\overline{\Delta}_i(t)=\{j \in \Delta_i:z_j(t) < z_i(t)\}   \\
&\underline{\Delta}_i(t)=\{j \in \Delta_i:z_j(t) > z_i(t)\} 
\end{split}
\end{equation}

Remove $f$ largest state values in $\overline{\Delta}_i(t)$ that are greater than $z_i(t)$. Remove all values  if the number of agents in $\overline{\Delta}_i(t)$ is less than $f$.
 
(3) Similarly, remove $f$ smallest state values in $\overline{\Delta}_i(t)$ that are lower than $z_i(t)$. Remove all values  if the number of agents in $\overline{\Delta}_i(t)$ is less than $f$.

(4) Denote $\Omega_i (t)$, termed as an admitting set, as the collection of agents whose values are retained after (2) and (3). Agent $i$ makes updata  with the following rule:
\begin{equation}
\dot{z}_i (t) =\sum_{j \in \Omega_i(t)} \bar{\alpha} m_{i j}(t)[z_j(t)-z_i(t)]+\bar{\alpha}m_{i0}(t)[z_0(t)-z_i(t)],
\end{equation}
where $z_i(t)$ is the local state of the virtual layer. $\bar{\alpha}> \alpha$ and $m_{ij}$ denotes the number of adjacent followers of agent i and $m_{i0}$ denotes the number of adjacent leader of agent $i \in \Omega_i$.

\begin{algorithm}
\caption{MSR algorithm}
\begin{algorithmic}[1]
   \State Receive the virtual states $z_j$ from all in-neighboring agents in a list $\Delta_i(t)$
   \State  Agents with state values greater than $z_i(t)$ are divided into set $\overline{\Delta}_i(t)$ and the rest are divided into set $\underline{\Delta}_i(t)$.
   \State  Compute the number of values that are in $\overline{\Delta}_i(t)$ and  $\underline{\Delta}_i(t)$  as $k$ and $q$, respectively.
   \If  {$k \geq f$}
   \State Remove $f$ largest state values in $\overline{\Delta}_i(t)$ .
   \Else   
   \State Remove all largest state values $\overline{\Delta}_i(t)$.
   \EndIf
   \If  {$q \geq f$}
   \State Remove $f$ smallest state values in $\underline{\Delta}_i(t)$ .
   \Else 
   \State Remove all smallest state values in $\underline{\Delta}_i(t)$.
   \EndIf
   \State Collect the agents retained after above process as a list $\Omega_i (t)$.
\end{algorithmic}
\end{algorithm}

We can resist edge attacks in this way. At any time, the agent removes at most 2$f$ extreme values and updates itself with the remaining information.  

Agents of the twin layer can communicate with agents of the physical layer. Compared to the output $y_i$ of the cyber-physical layer, the virtual state $z_i(t)$ of the twin layer has no physical meaning and is hard to find by attackers. Therefore, we can think that the nodes on the twin layer will not be attacked. 

\begin{lemma}
Consider the MASs satisfying the assumption 4. Suppose each follower makes updates with MSR and there exists at least $2f+1$ in-neighbors outside the agent $i \in \mathcal{V}_{np}$. Then we have two statements as follows:

(1) There exists a nonempty set $\Upsilon_i(t) \subseteq N_i^{in}$ and some positive weights $\bar{m}_{ij}(t)$ and $\bar{m}_{i0}(t)$, so that the (10) is equivalent to the following form:
\begin{equation}
\dot{z}_i (t) =\sum_{j \in \Upsilon_i(t)} \bar{m}_{i j}(t)[z_j(t)-z_i(t)]+\bar{m}_{i0}(t)[z_0(t)-z_i(t)],
\end{equation}

(2) For any agent $j \in \Upsilon_i(t)$, it holds that ${\rm max}\{\bar{m}_{i j} (t) \} > \frac{\alpha}{2}$. 
\end{lemma}

\textbf{Proof.} 
There are some cases can be discussed.

\textbf{Case 1:}  $i \in \mathcal{V}_{np}$. Since the in-neighbors of each agent $i$ are at least $2f+1$, it holds that $\Omega_i(t) \neq \varnothing$ as each agent
removes at most $2f$ values. If there exists some Byzantine agents in $\Omega_i(t)$, we denote that $\vartheta(t)=1$; otherwise, $\vartheta(t)=0$.

\begin{enumerate}
\item $\vartheta(t)=0$.
In the  case, it is obvious that the two statements are satisfied by letting $\bar{m}_{ij}(t) = \bar{\alpha}m_{ij}(t)$. Thus we mainly consider the second case where there are some Byzantine agents in $\Omega_i(t)$.


\item $\vartheta(t)=1$.
Consider the Byzantine agent $b \in \Omega_i(t)$. Since $z_b(t)$ is retained in $\Omega_i(t)$, it means that there exists $f$ agents whose states are greater than $z_b(t)$ and $f$ agents whose states are less than $z_b(t)$. The Byzantine agent pretend to be a normal agent. We can find a pair of normal nodes $p, q \in \mathcal{R}$ such that $z_p(t) \leq z_b(t) \leq z_q(t)$. There exists $0 \leq \gamma \leq 1 $ such that $z_b(t) = \gamma z_p(t) + (1-\gamma) z_q(t)$.  By setting $\bar{m}_{ip}(t) =\bar{\alpha}( m_{ip}(t) +\gamma m_{ib}(t))$ and $\bar{m}_{iq}(t) = \bar{\alpha}(m_{iq}(t) +(1-\gamma) m_{ib}(t))$, each Byzantine agent $b$ can be decomposed into two normal agents $p$ and $q$.
By repeating the above steps, we can obtain that (11) is hold. Moreover, we can find that
\begin{equation}
{\rm max}_{j \in \Omega_i(t)}\{\bar{m}_{ij}(t)\} \geq {\rm max}\{(1-\gamma), \gamma \}\bar{\alpha}\geq \frac{1}{2}\bar{\alpha}\geq
\frac{1}{2}{\alpha}.
\end{equation}
\end{enumerate}

\textbf{Case 2:} $i \in \mathcal{V}_p$. This case is similar to the Case 1 except that the information from leader will be preserved. The proof is completed.   \qquad $\blacksquare$

\begin{lemma}
Consider the MASs satisfying Assumption 5. Suppose $\mathcal{G} =(\mathcal{V}, \mathcal{E})$  is strongly $(2f+1)$-robust w.r.t. $\mathcal{V}_p$ and all followers make updates according to MSR algorithm. At any time $t$, there always exists a secure and equivalent $\frac{1}{2}\alpha$-directed path from leader to any follower in $\mathcal{G}$.
\end{lemma}

\textbf{Proof.} 
The first statement in Lemma 4 ensures that there exists a secure spanning tree rooted by the leader to all followers in $\mathcal{G}$. Moreover, the second statement in Lemma 4 ensures that the above
secure spanning tree is an equivalent $\frac{1}{2}\alpha$-directed spanning tree. Consequently, Lemma 5 can be proven
by following a similar procedure of Lemma 4.
The proof is completed.     \qquad $\blacksquare$

Note that MSR is proposed in the one-dimensional systems where the system state is simply a scalar. In order to cope with the multi-dimensional case, we discuss the extension of MSR to vector spaces. Towards the end, for any vector $z_i(t) \in \mathbb{R}^q$, let us introduce the notation:
\begin{equation}
\Theta_{f}(\{z_j(t)\}_{j \in N_i^{in}}) \triangleq
\begin{bmatrix} \sum_{j \in \Omega_i^1(t)} {m}_{i j}^1(t)[z_j^1(t)-z_i^1(t)]+m_{i0}^1(t)[z_0^1(t)-z_i^1(t)] \\ \vdots \\  \sum_{j \in \Omega_i^q(t)} {m}_{i j}^q(t)[z_j^q(t)-z_i^q(t)]+m_{i0}^q(t)[z_0^q(t)-z_i^q(t)] \end{bmatrix}.\quad
\end{equation}

\begin{lemma}
Consider the following updating law:
\begin{equation}
 \dot{z}_i(t) = \Theta_{f}(\{z_j(t)\}_{j \in N_i^{in}}), 
\end{equation}
where $\Theta_{f}(\{z_j(t)\}_{j \in N_i^{in}})$ is defined in (14). Suppose the lemma 1 and lemma 5 hold, $z_i(t)$ exponentially converges to $z_0(0)$.
\end{lemma}

\textbf{Proof.}
Since the lemma 5 hold. Hence, the conditions of lemma 4 are met. By applying lemma 4 on (15), (15) is equivalent to
\begin{equation}
\dot{z}_i(t) =
\begin{bmatrix} \sum_{j \in \Upsilon_i^1(t)} \bar{m}_{i j}^1(t)[z_j^1(t)-z_i^1(t)]+\bar{m}_{i0}^1(t)[z_0^1(t)-z_i^1(t)] \\ \vdots \\  \sum_{j \in \Upsilon_i^q(t)} \bar{m}_{i j}^q(t)[z_j^q(t)-z_i^q(t)]+\bar{m}_{i0}^q(t)[z_0^q(t)-z_i^q(t)] \end{bmatrix},\quad
\end{equation}
where $\Upsilon_i^s(t) \subseteq N_i^{in}, \forall s =1,2,\dots,q.$ Let us define $z^s(t) \triangleq [z_0^s(t), z_1^s(t), z_2^s(t),\dots, z_N^s(t)]$. From former discussion, we have
\begin{equation}
\dot{z}^s(t)=\bar{\Phi}^s(t)z^s(t),
\end{equation}
where $\bar{\Phi}^s(t)=[\bar{\phi}_{ij}^s(t)]$ such that
\begin{equation}
\begin{split}
\bar{\phi}_{ij}^s=
\begin{cases}
\bar{m}_{i0}+\sum_{j \in \Upsilon_i(t)} \bar{m}_{ij}, &j=i\\
-\bar{m}_{ij},        & j \in \Upsilon_i(t)\\
0,& {\rm otherwise}.
\end{cases}
\end{split}
\end{equation}

$\bar{\Phi}^s(t)$ is a Metzler matrix with zero row sums. According to lemma 5, each $\bar{\Phi}^s(t)$ has the property that any follower can be reached from leader. Recalling lemma 1, $z_i^s(t)$ exponentially converges to a common values as $t \rightarrow \infty$. Leader can not collect states from followers, so it holds that $z_0^s(t)=z_0^s(0)$. Thus, $z_i^s(t)$ exponentially converges to $z_0^s(0)$ for any $i$. The proof is completed.   \qquad $\blacksquare$

}
\color{black}{
\subsection{Leader-Following Consensus against $f$-local Attack}
Define the following state tracking error:
\begin{equation}
\varepsilon_i=x_i-\Pi_i v_i,
\end{equation}
where  $v_i=e^{St} z_i$.

We then present the following control protocols:
\begin{align}
u_i&=K_ix_i+H_i v_i-\hat{\psi}_i \\
\hat{\psi}_i&=\frac{B^T_i P_i\varepsilon_i}{\Vert \varepsilon^T_i P_i B_i \Vert+\omega}\hat{\rho}_i\\
\dot{\hat{\rho}}_i&=
\begin{cases}
\Vert \varepsilon^T_i P_i B_i \Vert +2\omega, \quad  {\rm if} \   \Vert \varepsilon^T_i P_i B_i \Vert \geq \bar{\kappa}\\
\Vert \varepsilon^T_i P_i B_i \Vert +2\omega(\frac{\Vert \varepsilon^T_i P_i B_i \Vert}{\bar{\kappa}}), \quad {\rm otherwise}
\end{cases}
\end{align}
where  $\hat{\psi}_i$ is an adaptive compensational signal designed. 

\begin{myTheo}
Suppose that Assumptions 1-3 hold.
Consider the heterogeneous MASs with (4) and (5), and the communication topology satisfying lemma 5. 
The  leader-following consensus problem can be solved  by using the controller (19), if the following local conditions hold for each  follower.

(1) The controller gain matrices $K_i$ and $H_i$ are designed as:
\begin{align}
K_i&=-U_i^{-1}B_i^T P_i, \\
H_i&=\Gamma_i-K_i \Pi_i,
\end{align}
where $P_i$ is positive definite matrix and satisfies the following Riccati equation:
\begin{equation}
P_i A_i+A_i^T P_i-P_i B_i U_i^{-1}B_i^T P_i +Q_i-\frac{1}{\mu_i}I= 0,
\end{equation}
where $U_i > 0$ is a symmetric matrix. $Q_i >0$ and $\mu_i>0$  satisfy $Q_i-\frac{1}{\mu_i}I > 0$. 

(2) Suppose $\mathcal{G} = (\mathcal{V}, \mathcal{E})$ is strongly $(2f+1)$-robust w.r.t. $\mathcal{V}_p$.
\end{myTheo}

\textbf{Proof.} 

Note that (7) can be written as
\begin{equation}
\begin{split}
e_i(t)&=C_ix_i(t)-Rx_0(t)\\
&=C_ix_i(t)-C_i\Pi_ix_0(t)\\
&=C_i(x_i(t)-\Pi_ix_0(t))\\
&=C_i\big(x_i(t)-\Pi_i v_i(t)+\Pi_i(v_i(t)-x_0(t))\big)\\
&=C_i\varepsilon_i(t)+C_i\Pi_i(v_i(t)-x_0(t)),
\end{split}
\end{equation}

Then we have
\begin{equation}
\begin{split}
\Vert e_i(t) \Vert& = \Vert C_i \varepsilon_i(t)+C_i\Pi_i(v_i(t)-x_0(t)) \Vert\\
&\leq \Vert C_i\varepsilon_i(t) \Vert+\Vert C_i\Pi_i(v_i(t)-x_0(t)) \Vert\\
&\leq \Vert C_i\Vert \Vert \varepsilon_i(t) \Vert+\Vert C_i\Pi_i\Vert \Vert (v_i(t)-x_0(t)) \Vert.
\end{split}
\end{equation}

To show that $ e_i(t)$ is UUB, in the following part, we shall prove that $\varepsilon_i(t) $ is UUB and $(v_i(t)-x_0(t)) \rightarrow 0$ as $t \rightarrow \infty$. Then there exists a  positive constant $\varrho$ such that
\begin{equation}
\Vert z_i(t)-z_0(0) \Vert \leq e^{-\varrho t} \Vert z_i(0)-x_0(0) \Vert
\end{equation}

Since $z_0(0)=x_0(0)$ and $z_i(0)= v_i(0)$, one has
\begin{equation}
\Vert z_i(t)-x_0(0) \Vert \leq e^{-\varrho t} \Vert z_i(0)-x_0(0) \Vert
\end{equation}


Using (5) and lemma 6, we can obtain
\begin{equation}
\begin{split}
\Vert v_i(t)-x_0(t) \Vert &= \Vert e^{St}(z_i(t)-x_0(0)) \Vert \\
&\leq \Vert e^{St} \Vert  \Vert (z_i(t)-x_0(0)) \Vert \\
&\leq e^{-\varrho t} \Vert e^{St} \Vert \Vert (v_i(0)-x_0(0)) \Vert \\
\end{split}
\end{equation}

From lemma 2, there exists $\varphi(\epsilon)$ for $0 < \epsilon < \varrho$ such that 
\begin{equation}
\Vert e^{St} \Vert  \leq \varphi(\epsilon)e^{\epsilon t}
\end{equation}

Thus, we have 
\begin{equation}
\begin{split}
\Vert v_i(t)-x_0(t) \Vert &\leq 
e^{-\varrho t}   \varphi(\epsilon)e^{\epsilon t} \Vert (v_i(0)-x_0(0)) \Vert \\
&\leq \varphi(\epsilon) e^{-(\varrho-\epsilon)t} \Vert (v_i(0)-x_0(0)) \Vert .
\end{split}
\end{equation}

Then $\Vert v_i(t)-x_0(t) \Vert$ converges to 0 exponentially. 

Next, we prove that $\Vert \varepsilon(t) \Vert$ is UUB. From (4),(6),(14),(21), we obtain the time derivative of (11) as
\begin{equation}
\begin{split}
\dot{\varepsilon}_i &=\dot{x}_i-\Pi_i (S v_i+e^{St}\Theta_{f}(\cdot))\\
&=A_i x_i+B_i K_i x_i+B_i H_i v_i-B_i \hat{\psi}_i+B_i \psi_i-\Pi_i S v_i-\Pi_i e^{St}\Theta_{f}(\cdot)\\
&=A_i x_i+B_i K_i x_i+B_i (\Gamma_i-K_i \Pi_i)v_i-B_i \hat{\psi}_i+B_i \psi_i-\Pi_i S v_i-\Pi_i  e^{St}\Theta_{f}(\cdot)\\
&=(A_i+B_i K_i)x_i+(\Pi_i S- A_i \Pi_i-B_i K_i \Pi_i)v_i-B_i \hat{\psi}_i+B_i \psi_i-\Pi_i S v_i-\Pi_i  e^{St}\Theta_{f}(\cdot)\\
&=(A_i+B_i K_i)(x_i-\Pi_i v_i)-B_i \hat{\psi}_i+B_i \psi_i-\Pi_i  e^{St}\Theta_{f}(\cdot)\\
&=\bar{A}_i \varepsilon_i-B_i \hat{\psi}_i+B_i \psi_i-\Pi_i  e^{St}\Theta_{f}(\cdot),
\end{split}
\end{equation}
where $\bar{A}_i = A_i+B_i K_i$. Let $\bar{Q}_i=Q_i+K_i^T U_i K_i$. It can be obtained from (22) and (24) that
\begin{equation}
P_i \bar{A}_i+\bar{A}_i^T P_i= -\bar{Q}_i+\frac{1}{\mu_i}I
\end{equation}

Consider the following Lyapunov function:
\begin{equation}
V(t)=\sum_{i=1}^N \varepsilon_i^T P_i \varepsilon_i.
\end{equation}

The derivative of $V(t)$ can be calculated as
\begin{equation}
\begin{split}
\dot{V}(t)&=2\sum_{i=1}^N \varepsilon_i^T P_i(\bar{A}_i \varepsilon_i-B_i \hat{\psi}_i+B_i \psi_i-\Pi_i e^{St}\Theta_{f}(\cdot))\\
&=\sum_{i=1}^N \varepsilon_i^T(P_i \bar{A}_i+\bar{A}_i^T P_i) \varepsilon_i+2\sum_{i=1}^N \varepsilon_i^T P_i (B_i \psi_i-B_i \hat{\psi}_i)-2\sum_{i=1}^N \varepsilon_i^T P_i \Pi_i e^{St}\Theta_{f}(\cdot)\\
&=\sum_{i=1}^N \varepsilon_i^T(-\bar{Q}_i+\frac{1}{\mu_i}I)\varepsilon_i+2\sum_{i=1}^N \varepsilon_i^T P_i (B_i \psi_i-B_i \hat{\psi}_i)-2\sum_{i=1}^N \varepsilon_i^T P_i \Pi_i e^{St}\Theta_{f}(\cdot).
\end{split}
\end{equation}

Using (19) to obtain
\begin{equation}
\begin{split}
\varepsilon_i^T P_i B_i (\psi_i-\hat{\psi}_i)&=\varepsilon_i^T P_i B_i \psi_i-\frac{{\Vert \varepsilon_i^T P_i B_i \Vert}^2 }{\Vert \varepsilon_i^T P_i B_i \Vert+\omega}\hat{\rho}_i\\
&\leq \Vert \varepsilon_i^T P_i B_i \Vert \Vert \psi_i \Vert -\frac{{\Vert \varepsilon_i^T P_i B_i \Vert}^2 }{\Vert \varepsilon_i^T P_i B_i \Vert+\omega}\hat{\rho}_i\\
&\leq \frac{{\Vert \varepsilon_i^T P_i B_i \Vert}^2(\Vert \psi_i \Vert- \hat{\rho}_i)+\Vert \varepsilon_i^T P_i B_i \Vert \Vert \psi_i \Vert \ \omega}{\Vert \varepsilon_i^T P_i B_i \Vert+\omega}\\
& \leq  \frac{{\Vert \varepsilon_i^T P_i B_i \Vert}^2(\frac{\Vert \varepsilon_i^T P_i B_i \Vert +\omega}{\Vert \varepsilon_i^T P_i B_i \Vert}\Vert \psi_i \Vert-\hat{\rho}_i)}{\Vert \varepsilon_i^T P_i B_i \Vert+\omega}.
\end{split}
\end{equation}




Noting that $[({\rm d}\Vert \psi_i \Vert )/ {\rm d}t]$ is bounded by $\bar{\kappa}$.  Choose $\dot{\hat{\rho}}_i \geq  \bar{\kappa} \geq [({\rm d}\Vert \psi_i \Vert )/ {\rm d}t]$,that is, $\frac{\bar{\kappa}+\omega}{\bar{\kappa}} \frac{d \Vert \psi_i \Vert}{dt}-\dot{\hat{\rho}}_i \leq \bar{\kappa}+\omega-\dot{\hat{\rho}}_i \leq -\omega < 0$. Then, $\exists$ $ t_1 > 0$  such that for all $t \geq t_1$, we have
\begin{equation}
\frac{\Vert \varepsilon_i^T P_i B_i \Vert +\omega} {\Vert \varepsilon_i^T P_i B_i \Vert}\Vert \psi_i \Vert -\hat{\rho}_i \leq  \frac{\bar{\kappa}+\omega}{\bar{\kappa}}\Vert \psi_i \Vert-\hat{\rho}_i < 0.
\end{equation}

Thus, we have 
\begin{equation}
\varepsilon_i^T P_i B_i \psi_i-\varepsilon_i^T P_i B_i \hat{\psi}_i \leq 0  \qquad  \forall t \geq t_1.
\end{equation}

Substituting (38) into (35) yields
\begin{equation}
\dot{V}(t) \leq \sum_{i=1}^N \varepsilon_i^T(-\bar{Q}_i+\frac{1}{\mu_i}I)\varepsilon_i-2\sum_{i=1}^N \varepsilon_i^T P_i \Pi_i e^{St}\Theta_{f}(\cdot)      \qquad  \forall t \geq t_1
\end{equation}



By using Yong's inequality, it holds that
\begin{equation}
\begin{split}
2\varepsilon_i^T P_i \Pi_i e^{St}\Theta_{f}(\cdot) \leq \frac{1}{\mu_i}\varepsilon_i^T\varepsilon_i+\mu_i\Theta_{f}^T(\cdot) e^{St}\Pi_i^T P_i^T P_i \Pi_i  e^{St}\Theta_{f}(\cdot).
\end{split}
\end{equation}

By using (39) and (40), we can obtain that
\begin{equation}
\dot{V}(t) \leq -\sum_{i=1}^N \varepsilon_i^T \bar{Q}_i \varepsilon_i-\sum_{i=1}^N\mu_i\Theta_{f}^T(\cdot) e^{St}\Pi_i^T P_i^T P_i \Pi_i  e^{St}\Theta_{f}(\cdot)  \qquad  \forall t \geq t_1.
\end{equation}

In light of (29), $\Theta_{f}(\cdot)$  exponentially tends to 0 with $\varrho$. According to (28), $e^{St}\Theta_{f}(\cdot)$ also exponentially tends to 0. We conclude that there exist $\epsilon_1$
 and $\epsilon_2$ such that
 \begin{equation}
 \sum_{i=1}^N \mu_i\Theta_{f}(\cdot)^T e^{St}\Pi_i^T P_i^T P_i \Pi_i  e^{St}\Theta_{f}(\cdot) \leq \epsilon_1 e^{-\epsilon_2 t}. 
\end{equation}

Thus,
\begin{equation}
\begin{split}
\dot{V}(t) \leq -\eta \sum_{i=1}^N   \varepsilon_i^T \varepsilon_i(t)- \epsilon_1 e^{-\epsilon_2 t}  \qquad  \forall t \geq t_1,
\end{split}
\end{equation}
where $\eta = \sigma_{min}(\bar{Q}_i)$. From (43), we have
\begin{equation}
V(t)-V(t_1) \leq -\eta \sum_{i=1}^N \int_{t_1}^t \Vert \varepsilon_i(\tau) \Vert^2 d\tau +\frac{\epsilon_1}{\epsilon_2}(e^{-\epsilon_2 t}-e^{-\epsilon_2 t_1})
\end{equation}

Thus, 
\begin{equation}
\sum_{i=1}^N \Vert \varepsilon_i(t) \Vert^2 \leq -\sum_{i=1}^N \int_{t_1}^t \Vert \varepsilon_i(\tau) \Vert^2 d\tau + F
\end{equation}
where $F=\frac{1}{\eta}(\frac{\epsilon_1}
{\epsilon_2}(e^{-\epsilon_2 t}-e^{-\epsilon_2 t_1})+V(t_1))$. 

Using Bellman–Gronwall Lemma, we obtain
\begin{equation}
\sum_{i=1}^N \Vert \varepsilon_i(t) \Vert^2 \leq -\sqrt{F}e^{-(t-t_1)}
\end{equation}

That is, $\Vert \varepsilon_i(t) \Vert$ is UUB.
Hence, the output consensus problem for MASs (4) and (5)  has been solved. This completes the proof.   \qquad $\blacksquare$

\section{Conclusion}
 In this paper, we study the consensus problem of heterogeneous multi-agent systems. By decoupling the Byzantine attack into two types of attacks, we design corresponding defense strategies on the TL and the CPL. For Byzantine edge attacks, we use the MSR algorithm to remove extreme values. For Byzantine node attacks, we design a distributed resilient control protocol to ensure the stability of the system. Through simulation, the effectiveness of the method is verified. In future work, it is interesting to assume that the system matrix of the follower is unknown.






}



























% Proof: Consider the Lyapunov function candidate
% $$
% V_{1}=\frac{1}{2} \sum_{i=1}^{N} \xi_{i}^{T} P \xi_{i}+\sum_{i=1}^{N} \sum_{j=1, j \neq i}^{N} \frac{\left(c_{i j}-\alpha\right)^{2}}{8 \kappa_{i j}}
% $$
% where $\alpha$ is a positive constant that is to be determined later. Evidently, $V_{1}$ is positive definite. The time derivative of $V_{1}$ along the trajectory of (5) is given by
% $$
% \begin{aligned}
% \dot{V}_{1}=& \sum_{i=1}^{N} \xi_{i}^{T} P \dot{\xi}_{i}+\sum_{i=1}^{N} \sum_{j=1, j \neq i}^{N} \frac{c_{i j}-\alpha}{4 \kappa_{i j}} \dot{c}_{i j} \\
% =& \sum_{i=1}^{N} \xi_{i}^{T} P A \xi_{i}+\sum_{i=1}^{N} \xi_{i}^{T} P B K \sum_{j=1}^{N} c_{i j} a_{i j}\left(\tilde{x}_{i}-\tilde{x}_{j}\right) \\
% &+\sum_{i=1}^{N} \sum_{j=1, j \neq i}^{N} \frac{c_{i j}-\alpha}{4 \kappa_{i j}} \dot{c}_{i j}
% \end{aligned}
% $$
% Since $a_{i j}=a_{j i}$ and $c_{i j}(t)=c_{j i}(t)$, it can be easily verified that
% $$
% \begin{array}{rl}
% \sum_{i=1}^{N} \xi_{i}^{T} & P B K \sum_{j=1}^{N} c_{i j} a_{i j}\left(\tilde{x}_{i}-\tilde{x}_{j}\right) \\
% & =-\frac{1}{2} \sum_{i=1}^{N} \sum_{j=1}^{N} c_{i j} a_{i j}\left(\xi_{i}-\xi_{j}\right)^{T} \Gamma\left(\tilde{x}_{i}-\tilde{x}_{j}\right)
% \end{array}
% $$

 \bibliography{PIDFR}
 \end{document}\grid







