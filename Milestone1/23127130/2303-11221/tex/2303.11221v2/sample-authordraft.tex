%%
%% This is file `sample-manuscript.tex',
%% generated with the docstrip utility.
%%
%% The original source files were:
%%
%% samples.dtx  (with options: `manuscript')
%% 
%% IMPORTANT NOTICE:
%% 
%% For the copyright see the source file.
%% 
%% Any modified versions of this file must be renamed
%% with new filenames distinct from sample-manuscript.tex.
%% 
%% For distribution of the original source see the terms
%% for copying and modification in the file samples.dtx.
%% 
%% This generated file may be distributed as long as the
%% original source files, as listed above, are part of the
%% same distribution. (The sources need not necessarily be
%% in the same archive or directory.)
%%
%% Commands for TeXCount
%TC:macro \cite [option:text,text]
%TC:macro \citep [option:text,text]
%TC:macro \citet [option:text,text]
%TC:envir table 0 1
%TC:envir table* 0 1
%TC:envir tabular [ignore] word
%TC:envir displaymath 0 word
%TC:envir math 0 word
%TC:envir comment 0 0
%%
%%
%% The first command in your LaTeX source must be the \documentclass command.
%%%% Small single column format, used for CIE, CSUR, DTRAP, JACM, JDIQ, JEA, JERIC, JETC, PACMCGIT, TAAS, TACCESS, TACO, TALG, TALLIP (formerly TALIP), TCPS, TDSCI, TEAC, TECS, TELO, THRI, TIIS, TIOT, TISSEC, TIST, TKDD, TMIS, TOCE, TOCHI, TOCL, TOCS, TOCT, TODAES, TODS, TOIS, TOIT, TOMACS, TOMM (formerly TOMCCAP), TOMPECS, TOMS, TOPC, TOPLAS, TOPS, TOS, TOSEM, TOSN, TQC, TRETS, TSAS, TSC, TSLP, TWEB.
% \documentclass[acmsmall]{acmart}

%%%% Large single column format, used for IMWUT, JOCCH, PACMPL, POMACS, TAP, PACMHCI
% \documentclass[acmlarge,screen]{acmart}

%%%% Large double column format, used for TOG
% \documentclass[acmtog, authorversion]{acmart}

%%%% Generic manuscript mode, required for submission
%%%% and peer review
%\documentclass[manuscript,screen,review]{acmart}
%% Fonts used in the template cannot be substituted; margin 
%% adjustments are not allowed.
%%
%% \BibTeX command to typeset BibTeX logo in the docs
%\AtBeginDocument{%
  %\providecommand\BibTeX{{%
    %\normalfont B\kern-0.5em{\scshape i\kern-0.25em b}\kern-0.8em\TeX}}}

%% Rights management information.  This information is sent to you
%% when you complete the rights form.  These commands have SAMPLE
%% values in them; it is your responsibility as an author to replace
%% the commands and values with those provided to you when you
%% complete the rights form.
%\setcopyright{acmcopyright}
%\copyrightyear{2018}
%\acmYear{2018}
%\acmDOI{XXXXXXX.XXXXXXX}

%% These commands are for a PROCEEDINGS abstract or paper.
%\acmConference[CCAI '23]{Child-Centred AI Design: Definition, Operation, and Consideration}{April 23,
  %2023}{}
%
%  Uncomment \acmBooktitle if th title of the proceedings is different
%  from ``Proceedings of ...''!
%
%\acmBooktitle{CCAI '23: Child-Centred AI Design: Definition, Operation, and Considerations (ACM CHI '23 Workshop),
%April 23, 2023} 
%\acmPrice{15.00}
%\acmISBN{978-1-4503-XXXX-X/18/06}

\documentclass[manuscript,screen]{acmart}

\AtBeginDocument{%
  \providecommand\BibTeX{{%
	\normalfont B\kern-0.5em{\scshape i\kern-0.25em b}\kern-0.8em\TeX}}}

\setcopyright{rightsretained}
\copyrightyear{2023}
\acmYear{2023}
\acmDOI{}

\acmConference[CCAI 2023]{CHI 2023 Workshop on Child-centred AI Design: Definition, Operation and Considerations}{April 23, 2023}{Hamburg, Germany}

\acmBooktitle{CHI 2023 Workshop on Child-centred AI Design: Definition, Operation and Considerations, April 23, 2023, Hamburg, Germany}
%%
%% Submission ID.
%% Use this when submitting an article to a sponsored event. You'll
%% receive a unique submission ID from the organizers
%% of the event, and this ID should be used as the parameter to this command.
%%\acmSubmissionID{123-A56-BU3}

%%
%% For managing citations, it is recommended to use bibliography
%% files in BibTeX format.
%%
%% You can then either use BibTeX with the ACM-Reference-Format style,
%% or BibLaTeX with the acmnumeric or acmauthoryear sytles, that include
%% support for advanced citation of software artefact from the
%% biblatex-software package, also separately available on CTAN.
%%
%% Look at the sample-*-biblatex.tex files for templates showcasing
%% the biblatex styles.
%%

%%
%% The majority of ACM publications use numbered citations and
%% references.  The command \citestyle{authoryear} switches to the
%% "author year" style.
%%
%% If you are preparing content for an event
%% sponsored by ACM SIGGRAPH, you must use the "author year" style of
%% citations and references.
%% Uncommenting
%% the next command will enable that style.
%%\citestyle{acmauthoryear}

%%
%% end of the preamble, start of the body of the document source.
\begin{document}

%%
%% The "title" command has an optional parameter,
%% allowing the author to define a "short title" to be used in page headers.
\title{Towards Goldilocks Zone in Child-centered AI}

%%
%% The "author" command and its associated commands are used to define
%% the authors and their affiliations.
%% Of note is the shared affiliation of the first two authors, and the
%% "authornote" and "authornotemark" commands
%% used to denote shared contribution to the research.

\author{Tahiya Chowdhury}
\affiliation{%
  \institution{Davis Institute for Artificial Intelligence}
  %\streetaddress{1 Th{\o}rv{\"a}ld Circle}
  \city{Waterville, Maine}
  \country{USA}}
\email{tahiya.chowdhury@colby.edu}


%%
%% By default, the full list of authors will be used in the page
%% headers. Often, this list is too long, and will overlap
%% other information printed in the page headers. This command allows
%% the author to define a more concise list
%% of authors' names for this purpose.
\renewcommand{\shortauthors}{Chowdhury et al.}

%%
%% The abstract is a short summary of the work to be presented in the
%% article.
\begin{abstract}

Using YouTube Kids as an example, in this work, we argue the need to understand a child's interaction process with AI and its broader implication on a child's emotional, social, and creative development. We present several design recommendations to create value-driven interaction in child-centric AI that can guide designing compelling, age-appropriate, beneficial AI experiences for children.


\end{abstract}

%%
%% The code below is generated by the tool at http://dl.acm.org/ccs.cfm.
%% Please copy and paste the code instead of the example below.
%%

\begin{CCSXML}
<ccs2012>
   <concept>
       <concept_id>10003120.10003123</concept_id>
       <concept_desc>Human-centered computing~Interaction design</concept_desc>
       <concept_significance>500</concept_significance>
       </concept>
   <concept>
       <concept_id>10010405.10010455</concept_id>
       <concept_desc>Applied computing~Law, social and behavioral sciences</concept_desc>
       <concept_significance>300</concept_significance>
       </concept>
 </ccs2012>
\end{CCSXML}

\ccsdesc[500]{Human-centered computing~Interaction design}
\ccsdesc[300]{Applied computing~Law, social and behavioral sciences}

%%
%% Keywords. The author(s) should pick words that accurately describe
%% the work being presented. Separate the keywords with commas.
\keywords{Interaction, technology and social behavior, creativity, child and AI}

%% A "teaser" image appears between the author and affiliation
%% information and the body of the document, and typically spans the
%% page.


\iffalse
\begin{teaserfigure}
  \includegraphics[width=\textwidth]{sampleteaser}
  \caption{Seattle Mariners at Spring Training, 2010.}
  \Description{Enjoying the baseball game from the third-base
  seats. Ichiro Suzuki preparing to bat.}
  \label{fig:teaser}
\end{teaserfigure}

\received{20 February 2007}
\received[revised]{12 March 2009}
\received[accepted]{5 June 2009}
\fi
%%
%% This command processes the author and affiliation and title
%% information and builds the first part of the formatted document.
\maketitle

\section{YouTube Kids: A case of child-centered AI}
My 7-year-old nephew loves paper crafts and once asked for my help to make an origami frog. Sitting next to him, I created an account on YouTube Kids to find origami tutorials. As the homepage showed up with 28 videos neatly organized in a grid, I immediately found him interested in several videos, none of which is related to origami. %and asking me to click on them.%
Meanwhile, I noticed one video on the homepage containing adolescents wearing provocative clothes, which is age inappropriate for him according to his parents. All this happened without a single click made or a video watched on a platform designed for children.

As I am writing this, YouTube Kids has 35 million kids viewers every week \cite{youtube_blog}. Children's content creation on YouTube has become a lucrative business as evidenced by hundreds of YouTube videos available on the topic `How to Make Money on YouTube Kids'. The kids' contents are created by parents, kids, teenagers, families -- almost everyone. The top content categories watched on YouTube Kids are nursery rhymes and animated songs, created by a selected few channels with high viewership. The videos often include a child actor or cartoon character depicting a fantasy-filled life, with magical creatures, family, and friends doing fun-filled activities (interested readers can search for `Cocomelon' or `Like Nastya Show' on YouTube).

Parents find value in YouTube for both entertainment and educational purposes. 50\% of US parents of a child aged under four years let their kids watch YouTube videos daily primarily for entertainment purposes \cite{pew_research}. Half of these parents have found that their children have encountered inappropriate content on the app. Exposure to such content occurs through videos \cite{10.1145/3341161.3342913}, comments \cite{10.1145/3442442.3452314}, video language (and subtitle) and advertisements \cite{https://doi.org/10.48550/arxiv.2211.02356} presented to the user by the app's recommendation algorithm.

We are currently at a pivotal moment. For decades we have known that every child is unique, in their learning style and personality. With artificial intelligence-enabled applications and services, we now have an opportunity to cater to the unique needs of a child, a feat difficult to achieve through traditional classroom learning. Through connected devices, we can collect ubiquitous data about a child's behavior, identify their interests and struggles, and develop personalized curricula or therapy centered around nurturing the child's abilities. This offers an opportunity for early detection of child development issues such as autism spectrum disorder and to design personalized therapy robots to help the parents enable the child to become an active part of society \cite{Rudovic_2018}. In reality, our children's daily interaction with AI is largely dominated by apps such as YouTube.

A child's interaction with YouTube is purely one-sided, as YouTube disables comments on videos targeted towards child audience due to COPPA requirements \cite{coppa}. %Regardless, the contents are surprise-filled, brightly colored, and generated specifically to appeal to and engage children to screen for prolonged hours.%
Note that much of the content available on the platform are auto-generated videos made by bots, often combining child-friendly characters with adult content involving violence and published with uncomfortable titles such as 'Wrong Heads Disney Wrong Ears Wrong Legs Kids Learn Colors Finger Family 2017 Nursery Rhymes'. Hundreds of videos can thus be uploaded per week (the per day upload limit for an account is 30 videos), and slip through YouTube's automatic content moderation mechanism. Prolonged screen exposure has several negative consequences such as disrupted sleep, poor communication skills, speech delay, etc. It is perhaps not a coincidence that autism rates have nearly tripled in the last decade, with an increased rate for symptoms such as impaired social skills and difficulty in communication \cite{autism}. 

YouTube is a powerful platform with immense potential. Despite the negative impacts on mental health, it is the only social media platform reported to have a positive impact on young individuals, providing them access to skill development and community building \cite{rsph}. To reap the potential of any child-AI interaction, we need to identify the values we seek from such interaction and embed these values into the design principles. %A modern family can benefit from its resources by engaging children with education and play.% However, social media boosted recommendation powered experience can easily cause information overload to children. To reap the potential of any child-AI interaction, we need to identify the values we seek from such interaction and embed these values into the design principles.

To understand the nature of child-AI interaction, let us consider a child's information and entertainment-seeking process off (condition 1) and on YouTube (condition 2). %the author followed two children of the author's network during their interaction with YouTube videos.%
Ayana, who is 12 years old interested to learn origami. In condition 1, Ayana goes to their local library to find 5-6 books on this topic. They found 2-3 books from this collection irrelevant to their interest after a quick exploration and then checks out the other 3 books. They encountered issues following the guidelines in the books and felt the need for someone to demonstrate the steps for them to follow. The overall process takes longer but remains focused on a single activity throughout the interaction.

In condition 2, Ayana uses YouTube to learn about origami. On their first visit to the app (in the presence of an adult guardian), videos ranked by high view counts are displayed on the homepage. %(I created fresh accounts on both YouTube and YouTube kids. on both websites I found inappropriate content on the homepage, without even watching a single video or making a single click).%
%Considering children's curiosity to explore, they will find a couple of videos interesting on the homepage and will click to view them.%
According to YouTube, `there’s an audience for almost every video, and the job of our recommendation system is to find that audience' \cite{YouTube_rec}. The two key signals YouTube uses in its recommendation are: clicks and watch time. These signals are used to populate users' homepage with new videos ranked by the relevance to the clicks users made and the videos they watched. The homepage contents will be curated by activities of others in their demographic group, using `collaborative filtering' technique.




%I imagine my 10 year old self many years ago watching children of my age talking about topics of my interest or simply watching a children's cartoon. I will be spending hours listening to what my friends have done or what mystery my favorite cartoon character is going to solve today. Sometimes when excited, I would answer to some question they may have asked. But I never receive a response.  What kind of interaction is this? Mass media communication? Yes, one that operates on a schedule for airing and contents are moderated for screening in day time.

 %What form of interaction is this one? Let me call this social media communication. Why? Who has control over the schedule? Nobody, as it is available to you on-demand. And as for the content, a high percentage is user generated. In the main Youtube site, contents can be filtered by using restricted  mode or based on other users' report on inappropriate videos. It is of this unwanted contents that may appear on a young viewer's device screen that I want to focus on. 

In condition 1, while the user spends more time finding the information they seek, they do not experience the impulse for task switching and overcome resistance to return to their original goal. In case they indeed explore other contents, that interaction remains independent, not influencing their future experience (e.g. it is unlikely that the author will find their regular study desk at the library filled with books on cats because they opened and peeked into a few pages of a title
`Knit your Own Cat` on their last visit). It is also highly improbable that the bookshelves around Ayana will be filled with new content on teenage singers as that is %ranked by the algorithm most relevant to% 
what people from similar demographic and locations are watching. The user thus has more control over their interaction and experience in condition 1. 

The form of social media interaction we see in condition 2  %Why? Who has control over the schedule? Nobody, as it% 
has implications that can span many aspects of children's life. 
The children interacting with AI are among the first generation being surveilled right from their birth, through their or their parent's digital history. Worldwide, children start viewing YouTube content as young as 2 years old and gain repeated exposure to omni-available, user-generated, interaction-driven entertainment platforms from a very young age. At this age, they also begin their journey for emotional, social, creative, and physical growth, and slowly shape their view of how the world works.

\section{Towards value-driven interaction: A Goldilocks Zone}

%develop their individual identity. They learn information by interacting with their peers, family, environment, and materials that slowly shape their existing view of how the world works. 
A healthy child development process involves self-concept, curiosity, confidence, and motivation to learn and explore. And this process fosters an optimal learning environment where a child is free to make errors and improve. %correct it by demonstration or by exploration and learn to improve.%
The consequences of being exposed to `what million others of their age are thinking' %At an age when they are still developing self-esteem and processing their understanding of the world, what are t They %
can impair their self-esteem development and their emerging interests can be dominated by thousands of others currently in trend. %pursued by others (recall that feeling of 'how do i make my mark when everything has been done' when we entered the research world).%
The lack of agency in AI-driven automated experience can exacerbate as their momentary wandering will influence their interaction experience and achieving the values sought from the platform will become difficult. 

We argue that this is a concern that needs attention from parents, policymakers, and designers of children-AI interaction. The concern stems from the lack of design principles for child-appropriate technology rather than issues found within a certain platform such as YouTube. At the heart of any recommendation system lies collaborative filtering, which finds content that other similar users are enjoying. Based on this, we point out that such a social recommendation system used in child-AI interaction is detrimental to a child's early cognitive development, social skills, self-concept, creativity, and curiosity. To that end, some recommendations for child-centered AI design are presented below:


\begin{itemize}
    \item Re-purposing adult-focused technology for children relies on a reactive mechanism for safeguarding and this has to change. On YouTube, content creators are responsible for the content and the user (parents in the case of a child user) for their activities (see YouTube's user agreement)\cite{YouTube_terms}. Between the responsibility of the creator (the sender) and the viewer (the receiver) lies the responsibility of the platform (the service provider), responsible for the recommendation algorithm selecting content for the viewer. Child-AI interaction should have a clear accountability principle for negative impacts resulting from AI algorithms.

    \item {Understanding the child-centered AI context requires involving both children and their parents (or guardians) in the design process. Parents are the primary caregiver responsible for child safety and happiness. As AI algorithms favor majority groups in the data, on culture and value conflict issues (contents on YouTube kids in 2022 largely reflect western cultures), incorporating parents can help achieve fairness and value-driven interaction.}

    \item{Child-centric AI experience should be free of collaborative recommendation which takes away control of interaction from the child. Design choices should be guided by values expected by the service and by age-appropriate consideration of the cognitive development they experience.}

    \item{Realizing the potential of child-centric AI will require large-scale, diverse data collected from child users. Though children's interactions and responses are different from adults, the lack of child data persists due to consent and privacy constraints.  As AI systems will be more pervasive in how children play, learn, and transition into adults, an appropriate framework needs to be formulated in collaboration with children, parents, designers, child psychologists, and policymakers for ethical data collection and user studies.} 
    
    %apps, toys and services children use in consideratuon of their age and the type of cognitive development they may be experiencing. To create such framework, trchnologists nbeed to work with child phychologists.
   % however, this may result in creation of AI based digital experience created without adequate ethical and cognitive consideration.}
    
\end{itemize} 

%children over a separate website does not address the potential consequences the interaction between the platform and the user. The party responsible for any negative impacts this interaction will bring falls in a gray area. See youTube user agreement where the platform mentions 'Content is the responsibility of the person or entity that provides it to the Service. YouTube is under no obligation to host or serve Content.' Also for parent of a user under 18, parents are responsible for the child’s activity on the Service.. As parents of younger children rely on YouTube as a form of 'digital nanny', the expectation that child's activity on the site is supervised by a parent can hardly met. For any appropriate content reported by a parent, the platform takes action by removing the content. Between these two tails of responsibility: responsibility of the creator (from an information standpoint, the sender) and  responsibility of the viewer (the receiver) lies the responsibility of the platform (the service provider) that is responsible of the recommendation system used to select contents to present to the viewer.

%Can children of young age, at the stage of making sense of their world, be called 'user'?  This can be up for debate as children interact with the service, without often understanding them. What is noticeable is that this frees the platform for any responsibility for their algorithm. THere is demand for the service, so parents, regardless of concerns, lean towards allowing their child on the platform despite the lack of accountability or regulation.

%But is this the only way? The expectation of the child and parent can continue to be met if they can find content of educational and entertainment value. However, the recommendations they receive can solely based on their search query, analogous to a search in the library or tv guide. If recommendations are curated based on collaborative filtering, the current framework of the platform will still work while being mindful of the child's own uniqueness.

%Existing concerns about fairness in AI is another factor for child-cenetred AI. Contents on YouTube kids largely reflect western cultures (based on the content provided by the top youtube channel of Youtube Kids in 2022), that can often pose opposing views to viewers from other culture and the parents may find inappropriate.

%AI system will be more pervasive in how children play, learn, and transition into adult in the years to come. Realizing technologies built for adult is note best suited for children is the first step towards realizing its potential to bring richer and more useful experience. One reason for doing so is the lack of data to understand children's needs due to consent and privacy.  however, this may result in creation of AI based digital experience created without adequate ethical and cognitive consideration. On efundamnetal key can be ensuring platforms should be devoid of social media component through collaboartive filtering which takes away control from child and the guardian over the technology. We also need to critically think about appropriate framework for apps, toys and services children use in consideratuon of their age and the type of cognitive development they may be experiencing. To create such framework, trchnologists nbeed to work with child phychologists.

For designing child-centered AI, or any user-centered AI, it is vital to identify the core values to be embedded and prioritized in the interaction design. Child-centered AI research can benefit from developing a framework of 'Contextual Safeguarding' \cite{10547/624844} for researchers and practitioners working in this space. As a community focused on improving human interaction and creative processes through technology, we hope child-centered AI platforms such as YouTube Kids can provide practical lessons to inform the design of future experiences.

%\section{Arguments and evidence}
%\section{Opposing and qualifying ideas}
%\section{Conclusion}




%%
%% The next two lines define the bibliography style to be used, and
%% the bibliography file.
\bibliographystyle{ACM-Reference-Format}
\bibliography{sample-base}


\end{document}
\endinput
%%
%% End of file `sample-authordraft.tex'.
