\documentclass[aps,10pt,prb,preprintnumbers,floatfix,showpacs,citeautoscript,superscriptaddress]{revtex4-2}

%%% TEXEXPAND: \input FILE MARKER ./tex-files/settings.tex
\usepackage{amsmath}
\usepackage{amssymb}
\usepackage{graphicx}
\usepackage{tabularx}
\usepackage{dcolumn}
\usepackage{xcolor}
%\usepackage{units}
\usepackage{booktabs}
\usepackage{comment}
\usepackage{enumitem}
\usepackage{multirow}
\usepackage[acronym]{glossaries}
\usepackage[version=4]{mhchem}
%\usepackage[utf8]{inputenc}


\usepackage{siunitx}
\DeclareSIUnit\angstrom{\text {Å}}
\definecolor{cadmiumgreen}{rgb}{0.0, 0.42, 0.24}
\newcommand{\BZO}{BaZrO$_\mathrm{3}$~}
\newcommand{\red}[1]{\textcolor{red}{#1}}
\newcommand{\orange}[1]{\textcolor{orange}{#1}}
\newcommand{\green}[1]{\textcolor{cadmiumgreen}{#1}}
\usepackage{braket}

\usepackage[colorlinks=true]{hyperref}
\hypersetup{
%    bookmarks=true,
%    unicode=false,          % non-Latin characters in Acrobat’s bookmarks
    pdftoolbar=true,        % show Acrobat’s toolbar?
    pdfmenubar=true,        % show Acrobat’s menu?
    pdffitwindow=false,     % window fit to page when opened
    pdfstartview={FitH},    % fits the width of the page to the window
    pdftitle={My title},    % title
    pdfauthor={Author},     % author
    pdfsubject={Subject},   % subject of the document
    pdfcreator={Creator},   % creator of the document
    pdfproducer={Producer}, % producer of the document
    pdfkeywords={keyword1, key2, key3}, % list of keywords
    pdfnewwindow=true,      % links in new PDF window
    colorlinks=true,        % false: boxed links; true: colored links
    linkcolor=black,      % color of internal links (change box color with linkbordercolor)
    citecolor=blue,        % color of links to bibliography
    filecolor=blue,      % color of file links
    urlcolor=blue        % color of external links
}


% location for image files
\graphicspath{{/article-v2/figures/}} 

\setacronymstyle{long-short}


% acronyms
\newacronym{aimd}{AIMD}{ab-initio molecular dynamics}
\newacronym{ardr}{ARDR}{automatic relevance detection regression}
\newacronym{bcc}{BCC}{body-centered cubic}
\newacronym{cv}{CV}{cross-validation}
\newacronym{ce}{CE}{cluster expansion}
\newacronym{dft}{DFT}{density functional theory}
\newacronym{dof}{DOF}{degrees of freedom}

\newacronym{dos}{DOS}{density of states}
\newacronym{edos}{EDOS}{electronic density of states}
\newacronym{vdos}{VDOS}{vibrational density of states}

\newacronym{eam}{EAM}{embedded atom method}
\newacronym{eci}{ECI}{effective cluster interaction}
\newacronym{emt}{EMT}{effective medium theory}
\newacronym{ehm}{EHM}{effective harmonic model}
\newacronym{ha}{HA}{harmonic approximation}
\newacronym{qha}{QHA}{quasi-harmonic approximation}
\newacronym{fc}{FC}{force constant}
\newacronym{fcc}{FCC}{face-centered cubic}
\newacronym{fcp}{FCP}{force constant potential}
\newacronym{gc}{GC}{grand canonical}
\newacronym{lasso}{LASSO}{least absolute shrinkage and selection operator}
\newacronym{loocv}{LOOCV}{leave-one-out cross-validation}
\newacronym{mae}{MAE}{mean absolute error}
\newacronym{mc}{MC}{Monte Carlo}
\newacronym{md}{MD}{molecular dynamics}
\newacronym{msd}{MSD}{mean squared displacement}
\newacronym{msrd}{MSRD}{mean squared relative displacement}
\newacronym{ols}{OLS}{ordinary least squares}
\newacronym{omp}{OMP}{orthogonal matching pursuit}
\newacronym{pes}{PES}{potential energy surface}
\newacronym{rfe}{RFE}{recursive feature elimination}
\newacronym{rmse}{RMSE}{root-mean-square error}
\newacronym{scp}{SCP}{self-consistent phonon}
\newacronym{hp}{HP}{harmonic phonon}
\newacronym{sgc}{SGC}{semi-grand canonical}
\newacronym{sed}{SED}{spectral energy density}


\newacronym{svd}{SVD}{singular value decomposition}
\newacronym{tmd}{TMD}{transition metal di\-chal\-co\-ge\-ni\-de}
\newacronym{vacf}{VACF}{velocity auto-correlation function}
\newacronym{fep}{FEP}{free energy perturbation}
\newacronym{tdep}{TDEP}{temperature dependent effective potential}


% softwares
\newcommand{\hiphive}{\textsc{hiphive}}
\newcommand{\icet}{\textsc{icet}}
\newcommand{\shengbte}{\textsc{shengBTE}}
\newcommand{\spglib}{\textsc{spglib}}
\newcommand{\phonopy}{\textsc{phonopy}}
\newcommand{\phonothreepy}{\textsc{phono3py}}
\newcommand{\sklearn}{\textsc{scikit-learn}}
\newcommand{\tdep}{\textsc{TDEP}}
\newcommand{\ase}{\textsc{ASE}}
\newcommand{\dynasor}{\textsc{dynasor}}
\newcommand{\vasp}{\textsc{vasp}}

% comments
\newcommand{\cmt}[1]{\emph{\color{red}#1}}

% overriding defaults in math mode 
\renewcommand{\vec}[1]{\ensuremath\boldsymbol{#1}}
\renewcommand{\epsilon}[0]{\varepsilon}

% commands for structuring
\newcommand{\sect}[1]{Sect.~\ref{#1}}
\newcommand{\fig}[1]{Fig.~\ref{#1}}
\newcommand{\Fig}[1]{Figure~\ref{#1}}
\newcommand{\eq}[1]{Eq.~\eqref{#1}}
\newcommand{\Eq}[1]{Equation~\eqref{#1}}
\newcommand{\tab}[1]{Table~\ref{#1}}


% autoref prefixes
\def\sectionautorefname{Sect.}
\def\figureautorefname{Fig.}
\def\tableautorefname{Table}
\def\equationautorefname{Eq.}

\hyphenation{
  ex-pan-sion
  pho-non
}%%% TEXEXPAND: END FILE ./tex-files/settings.tex

\renewcommand\thefigure{S\arabic{figure}}
\renewcommand{\thetable}{S\arabic{table}}
\renewcommand{\t}[1]{\text{#1}}

\begin{document}

\title{Supplemental Material: \\ Anharmonicity of the antiferrodistortive soft mode in barium zirconate BaZrO$_3$}

\author{Petter Rosander}
\author{Erik Fransson}
\affiliation{Department of Physics, Chalmers University of Technology, SE-412 96  G\"oteborg, Sweden}

\author{Cosme Milesi-Brault}
\affiliation{Department of physics and materials science, University of Luxembourg, 41 Rue du Brill, L-4422 Belvaux, Luxembourg}
\affiliation{Materials Research and Technology Department, Luxembourg Institute of Science and Technology, 41 rue du Brill, L-4422 Belvaux, Luxembourg}
\affiliation{Institute of Physics of the Czech Academy of Sciences, Na Slovance 1999/2, 182 21 Prague, Czech Republic}
\author{Constance Toulouse}
\affiliation{Department of physics and materials science, University of Luxembourg, 41 Rue du Brill, L-4422 Belvaux, Luxembourg}

% ILL and ESRF scientists (inelastic measurements)
%\author{Fr\'ed\'eric Bourdarot}
%\author{Andrea Piovano}
%\affiliation{Institut Laue-Langevin (ILL), 6 Rue Jules Horowitz, 38043 %Grenoble, France}
%\author{Alexei Bossak}
%\affiliation{European Synchrotron Radiation Facility, BP 220, F-38043 Grenoble Cedex, France}

% Luxembourgish team (experimental lead, data analysis, paper writing)
%\author{Cosme Milesi-Brault}
%\affiliation{Department of physics and materials science, University of Luxembourg, 41 Rue du Brill, L-4422 Belvaux, Luxembourg}
%\affiliation{Materials Research and Technology Department, Luxembourg Institute of Science and Technology, 41 rue du Brill, L-4422 Belvaux, Luxembourg}
%\affiliation{Institute of Physics of the Czech Academy of Sciences, Na Slovance 1999/2, 182 21 Prague, Czech Republic}
%\author{Constance Toulouse}
%\author{Mael Guennou}
%\email{mael.guennou@uni.lu}
%\affiliation{Department of physics and materials science, University of Luxembourg, 41 Rue du Brill, L-4422 Belvaux, Luxembourg}

% ILL and ESRF scientists (inelastic measurements)
\author{Fr\'ed\'eric Bourdarot}
\author{Andrea Piovano}
\affiliation{Institut Laue-Langevin (ILL), 6 Rue Jules Horowitz, 38043 Grenoble, France}
\author{Alexei Bossak}
\affiliation{European Synchrotron Radiation Facility, BP 220, 38043 Grenoble, France}

\author{Mael Guennou}
\email{mael.guennou@uni.lu}
\affiliation{Department of physics and materials science, University of Luxembourg, 41 Rue du Brill, L-4422 Belvaux, Luxembourg}

\author{G\"oran Wahnstr\"om}
\email{goran.wahnstrom@chalmers.se}
\affiliation{Department of Physics, Chalmers University of Technology, SE-412 96  G\"oteborg, Sweden}



\date{\today}

\maketitle

\vspace{1.0cm}
\section{Model validation}
The performance of the \gls{fcp} is estimated from a 10-Fold cross validation split.
The parity plot for this analysis is shown in \autoref{fig:model_validation_forces}.
It depicts the \gls{fcp} vs \gls{dft} forces for the validation data.
The obtained average \gls{rmse} over the splits is $\SI{0.034}{eV/Å}$ and $\SI{0.035}{eV/Å}$ for PBE and PBEsol respectively.
This shows that there is a good agreement between \gls{dft} and our \gls{fcp}.
\begin{figure}[ht]
    \includegraphics{sfig1a.png}
    \includegraphics{sfig1b.png}
    \caption{Forces from \gls{dft} compared to predicted forces with the \gls{fcp} for the validation sets from \gls{cv} analysis for a) PBE and b) PBEsol.}
     \label{fig:model_validation_forces}
\end{figure}

\clearpage
\section{Phonon dispersions}
\begin{figure}[ht]
\centering
\includegraphics[width=.57\textwidth]{sfig2.pdf}
\caption{
    Phonon dispersions based on PBE. Comparison between our harmonic phonon (\gls{hp}) model based on the derived force constant potential (\gls{fcp}) and the standard small displacement method with $\pm$ 0.01 \AA\ as implemented in \textsc{phonopy}. The R-tilt mode frequency is 3.77 meV and 2.66 meV, respectively. We expect that the deviations at higher frequencies, as seen in the figure, can be reduced by using more \gls{dft} training configurations and possibly a higher order expansion.
}
\label{fig:compare-dispersion-with-phonopy}
\end{figure}

\begin{figure}[ht]
\centering
\includegraphics[width=.57\textwidth]{sfig3.pdf}
\caption{
    Phonon dispersions based on PBE. The data are obtained using the self consistent phonon (\gls{scp}) method using the derived force constant potential (\gls{fcp}). Data are shown at four different temperatures; 0, 100, 200 and 300 K. It is seen the temperature dependence is mainly localized at the R point (out of phase tilting) and the M point (in phase tilting). At the R point the tilting frequency is increased from 5.63 meV to 9.03 meV when increasing the temperature to 300 K.
}
\label{fig:compare-dispersion-with-phonopy}
\end{figure}


\section{Details for fitting of inelastic neutron scattering spectra}

Fitting of the inelastic neutron scattering (INS) data is performed by using the Takin software\cite{Weber2016, Weber2017, Weber2021} using damped harmonic oscillators. This takes into account the resolution function of the diffraction setup (cf.\ the resolution ellipsoids on Fig. \ref{fig:ellipsoid}). In a first step, a parabolic dispersion around the R point was introduced for the tilt mode in a attempt to fit properly the asymmetry of this peak at low temperatures. This unfortunately was not successful, probably because of the very steep dispersion. We therefore added an additional ad-hoc mathematical asymmetry $\gamma$ to the damping $\Gamma$, i.e. $\Gamma=2\Gamma_0 /(1+e^{\gamma (E-E_0)})$, for the damped harmonic oscillator of lowest energy. This allowed us to obtain reasonable fits for the mode energies.

%In our case, fitting is complicated by the strong asymmetry that develops for the lowest temperatures measured, probably due to the high dispersion around the R-point, skewing the damped harmonic oscillator line-shapes. %An example of fitting for \SI{250}{\kelvin} is shown on Fig.\ \ref{fig:INS_fit_250K}.


\begin{figure}[ht]
    \centering
    \includegraphics[width=.65\textwidth]{sfig4.png}
    
    \caption{Ellipsoids of resolution based on the TAS file from the Institut Laue Langevin (ILL) IN8 spectrometer. The green line represents HWHM contour of the projected ellipse while the blue one represents the HWHM contour of the sliced ellipse.}
    \label{fig:ellipsoid}
\end{figure}


% \begin{center}
% \begin{tabular}{| c || c | c |}
% \hline
% \textbf{Temperature (K)} & \textbf{Frequency of tilt-mode (meV)} & \textbf{Frequency of acoustic mode (meV)} \\ 
% \hline
% \multicolumn{3}{|c|}{Q = (0.5 0.5 2.5)} \\
% \hline
% 296 & 9.43 & 13.15 \\ \hline
% 250 & 9.01 & 13.10 \\ \hline
% 200 & 8.32 & 13.13 \\ \hline
% 150 & 7.58 & 12.92 \\ \hline
% 100 & 6.50 & 13.17 \\ \hline
% 50  & 5.91 & 13.20 \\ \hline
% 2   & 5.62 & 13.07 \\ \hline
% \end{tabular}
% \end{center}
% \clearpage

% \section{Inelastic x-ray scattering: temperature dependence}

% Inelastic x-ray scattering (IXS) data is fitted by the sum of a standard damped harmonic oscillator for the R-acoustic mode and a Lorentzian centered around \SI{0}{\milli\electronvolt} for the pseudo-elastic contribution.

% \begin{center}
% \begin{tabular}{| c || c |}
% \hline
% \textbf{Temperature (K)} & \textbf{Frequency of acoustic mode (meV)} \\ 
% \hline
% \multicolumn{2}{|c|}{Q = (0.5 0.5 3.5)} \\
% \hline
% 300 & 13.41 \\ \hline
% 230 & 13.34 \\ \hline
% 180 & 13.23 \\ \hline
% 130 & 13.23 \\ \hline
% 80  & 13.10  \\ \hline
% \end{tabular}
% \end{center}

% \clearpage





\clearpage
\section{Dispersion data}
\subsection{Dispersion from INS spectra at 2 K}

\begin{figure}[h]
    \centering
    \includegraphics[height=.65\textwidth]{sfig5.pdf}
    \caption{INS spectra taken at 2 K along the (a) R-$\Gamma$ and (b) R-M directions of the Brillouin zone. Black lines show fits used to plot the dispersion of Fig.~5 of the main text. Stars ($\ast$) indicate spurious peaks. Their spurious nature has every time been checked by measuring a spectrum at an equivalent point in a different Brillouin zone.} %Spectra displayed in red are the ones that have been fitted to plot the dispersion curves in Fig.~3 of the main text.}
    \label{fig:INS-spectra}
\end{figure}


% \begin{center}
% \begin{tabular}{| c | c | c |}
% \hline
% \textbf{Q (h k l)} & \textbf{Frequency of tilt-mode (meV)} & \textbf{Frequency of acoustic mode (meV)} \\ 
% \hline
% \multicolumn{3}{|c|}{$R \rightarrow  \Gamma$ direction, T = \SI{2}{\kelvin}} \\
% \hline
% (0.50 0.50 2.50) & 5.62 & 13.15 \\ \hline
% (0.55 0.55 2.45) & 7.49 & 12.69 \\ \hline
% (0.60 0.60 2.40) & 10.83 & 12.44 \\ \hline
% (0.65 0.65 2.35) & - & 12.20 \\ \hline
% (0.70 0.70 2.30) & - & 11.71 \\ \hline
% (0.75 0.75 2.25) & - & 10.64  \\ \hline
% (0.80 0.80 2.20) & - & 9.95  \\ \hline
% (0.85 0.85 2.15) & - & 8.69 \\ \hline
% (0.90 0.90 2.10) & - & 7.03 \\
% \hline
% \multicolumn{3}{|c|}{$R \rightarrow M$ direction, T = \SI{2}{\kelvin}} \\
% \hline
% (0.50 0.50 2.50) & 5.62 & 13.15 \\ \hline
% (0.50 0.50 2.45) & 6.33 & 12.92 \\ \hline
% (0.50 0.50 2.40) & 10.69 & 12.93 \\ \hline
% (0.50 0.50 2.35) & - & 12.84 \\ \hline
% (0.50 0.50 2.30) & - & 12.47 \\ \hline
% (0.50 0.50 2.25) & - & 11.97  \\ \hline
% (0.50 0.50 2.20) & - & 11.81  \\ \hline
% (0.50 0.50 2.15) & - & 11.31 \\ \hline
% (0.50 0.50 2.10) & - & 10.94 \\ \hline
% (0.50 0.50 2.05) & - & 10.77 \\ \hline
% \end{tabular}
% \end{center}


% \begin{figure}[htpb]
%     \centering
%     \includegraphics[height=.5\textwidth]{fig6.png}
%     \caption{INS spectra R $\to$ M}
%     \label{fig:INS-spectra}
% \end{figure}

\clearpage
\subsection{Dispersion from IXS spectra at 80 K}

\begin{figure}[htpb]
    \centering
    \includegraphics[height=.65\textwidth]{sfig7.pdf}
    \caption{IXS spectra taken at 80 K along the (a) R-$\Gamma$ and (b) R-M directions of the Brillouin zone. Black lines represent fits used to extract frequency of the R-acoustic mode, shown on the dispersion curves of Fig.~5 in the main text.}
    \label{fig:IXS_dispersion}
\end{figure}

% \begin{center}
% \begin{tabular}{| c || c || c |}
% \hline
% \textbf{Q (h k l)} & \textbf{Frequency of acoustic mode (meV)} & \textbf{Extra modes frequency (meV)}\\ 
% \hline
% \multicolumn{3}{|c|}{$R \rightarrow  \Gamma$ direction, T = \SI{80}{\kelvin}} \\
% \hline
% (0.5 0.5 3.5) & 13.14 & - \\ \hline
% (0.6 0.6 3.6) & 12.16 & - \\ \hline
% (0.7 0.7 3.7) & 11.43 & 25.44 \\ \hline
% (0.8 0.8 3.8) & 9.75 & - \\ \hline 
% \hline
% \multicolumn{3}{|c|}{$R \rightarrow M$ direction, T = \SI{80}{\kelvin}} \\
% \hline
% (0.5 0.5 3.5) & 13.14 & - \\ \hline
% (0.5 0.5 3.6) & 12.60 & - \\ \hline
% (0.5 0.5 3.7) & 11.92 & 24.76 \\ \hline
% (0.5 0.5 3.8) & 11.36 & 20.14 \\ \hline
% (0.5 0.5 3.9) & 11.41 & 19.52 \\ \hline
% (0.5 0.5 4.0) & 11.97 & 23.67 \\ \hline
% \end{tabular}
% \end{center}




\clearpage
\section{Peak fitting}

The calculated dynamical structure factor, $S(\boldsymbol{q},\omega)$, from the MD simulation is fitted to a damped harmonic oscillator model
\begin{equation}
    f(\omega) = A\ \frac{2\Gamma\omega_0^2}{(\omega^2 - \omega_0^2)^2 + (\Gamma\omega)^2}\ ,
    \label{eq:damped}
\end{equation}
where $\omega_0$ is the bare frequency, $\Gamma$ the damping and $A$ the amplitude of the mode \cite{FraSlaErhWah2021}.
These fits are shown in \autoref{fig:peak_fitting} for different temperatures. In \autoref{tab:SQ} we show the fitted values for the bare frequency $\omega_0$ and the damping coefficient $\Gamma$. The frequency at the maximum of the peak is given by $\omega_{\text{max}} = \omega_0 \sqrt{1 - \Gamma^2/(2 \omega_0^2})$.

\begin{figure}[ht]
\centering
\includegraphics[width=.45\textwidth]{sfig8.pdf}
\caption{
    The dynamical structure factor $S(\boldsymbol{q},\omega)$ at different temperatures for PBE. The damped harmonic oscillator model in \autoref{eq:damped} is fitted to the MD data.
}
\label{fig:peak_fitting}
\end{figure}
\begin{table}[h!]
    \centering
    \begin{tabular}{ccccc}\toprule\toprule
                                        & \multicolumn{2}{c}{R-tilt} & \multicolumn{2}{c}{R-acoustic} \\
                                  T (K) & $\omega_0$ (meV) & $\Gamma$ (meV)      & $\omega_0$ (meV) & $\Gamma$ (meV)          \\ \midrule
                                  10    &   $4.08$   &  $0.07$       &  $12.14$   &  $0.00$           \\
                                  50    &   $5.06$   &  $0.30$       &  $12.22$   &  $0.02$           \\
                                 100    &   $5.91$   &  $0.40$       &  $12.31$   &  $0.06$           \\
                                 200    &   $7.21$   &  $0.87$       &  $12.50$   &  $0.14$           \\
                                 300    &   $8.26$   &  $1.04$       &  $12.66$   &  $0.16$          \\
        \bottomrule\bottomrule
    \end{tabular}
    \caption{Fitted values for the bare frequency $\omega_0$ and the damping coefficient $\Gamma$ from the damped harmonic oscillator in \autoref{eq:damped} for the dynamical structure factor from the MD simulation.}
    \label{tab:SQ}
\end{table}

%\clearpage
%\section{PBEsol results}
%\autoref{fig:dynasor_spectra_PBEsol} depicts the dynamical structure factor for PBEsol at $q=[0.5, 0.5, 2.5]$ in the frequency range $\SI{0}-\SI{14}{meV}$
%\begin{figure}[ht]
%\centering
%\includegraphics{fig9.pdf}
%\caption{
%    Dynamical structure factor $S(q, \omega)$ at q=[0.5, 0.5, 2.5] for PBEsol.
%}
%\label{fig:dynasor_spectra_PBEsol}
%\end{figure}


\clearpage
\section{Frequencies obtained from an effective harmonic model}
It is possible to construct an \gls{ehm} directly from the \gls{md} simulation \cite{hellmanLatticeDynamicsAnharmonic2011}.
This can be done by minimizing the differences between the forces for the harmonic model and the \gls{fcp} in the MD simulation,
\[
    \min_{\mathbf{x}}\lVert {\mathbf{A}}(\mathbf{u}) {\mathbf x} - {\mathbf{f}}(\mathbf{u}) \rVert,
\]
which is the same minimization problem as for the \gls{scp} problem.
Hence, the only difference between the two methods is how the displaced structures are obtained.
The frequency obtain from the EHM method is compared with the frequencies presented in the paper.
It compares very well with the bare frequency of the antiferrodistortive R-tilt mode from the dynamical structure factor (the MD model) as can be seen in \autoref{fig:MD-EHM-freqs}.
The fact that the \gls{scp} gives a slightly larger frequency compared to the \gls{ehm} has been observed in previous studies as well, e.g., in Ref.~
\cite{korotaevReproducibilityVibrationalFree2018,metsanurkSamplingdependentSystematicErrors2019,castellanoInitioCanonicalSampling2022,franssonProbingLimitsPhonon2022}
\begin{figure}[ht]
\centering
\includegraphics[width=0.6\textwidth]{sfig10.pdf}
\caption{
    Calculated frequencies for four different models using PBE.
    Only the SCP model includes the quantum fluctuations of the atomic motions.
}
\label{fig:MD-EHM-freqs}
\end{figure}


\begin{thebibliography}{9}%
\makeatletter
\providecommand \@ifxundefined [1]{%
 \@ifx{#1\undefined}
}%
\providecommand \@ifnum [1]{%
 \ifnum #1\expandafter \@firstoftwo
 \else \expandafter \@secondoftwo
 \fi
}%
\providecommand \@ifx [1]{%
 \ifx #1\expandafter \@firstoftwo
 \else \expandafter \@secondoftwo
 \fi
}%
\providecommand \natexlab [1]{#1}%
\providecommand \enquote  [1]{``#1''}%
\providecommand \bibnamefont  [1]{#1}%
\providecommand \bibfnamefont [1]{#1}%
\providecommand \citenamefont [1]{#1}%
\providecommand \href@noop [0]{\@secondoftwo}%
\providecommand \href [0]{\begingroup \@sanitize@url \@href}%
\providecommand \@href[1]{\@@startlink{#1}\@@href}%
\providecommand \@@href[1]{\endgroup#1\@@endlink}%
\providecommand \@sanitize@url [0]{\catcode `\\12\catcode `\$12\catcode
  `\&12\catcode `\#12\catcode `\^12\catcode `\_12\catcode `\%12\relax}%
\providecommand \@@startlink[1]{}%
\providecommand \@@endlink[0]{}%
\providecommand \url  [0]{\begingroup\@sanitize@url \@url }%
\providecommand \@url [1]{\endgroup\@href {#1}{\urlprefix }}%
\providecommand \urlprefix  [0]{URL }%
\providecommand \Eprint [0]{\href }%
\providecommand \doibase [0]{https://doi.org/}%
\providecommand \selectlanguage [0]{\@gobble}%
\providecommand \bibinfo  [0]{\@secondoftwo}%
\providecommand \bibfield  [0]{\@secondoftwo}%
\providecommand \translation [1]{[#1]}%
\providecommand \BibitemOpen [0]{}%
\providecommand \bibitemStop [0]{}%
\providecommand \bibitemNoStop [0]{.\EOS\space}%
\providecommand \EOS [0]{\spacefactor3000\relax}%
\providecommand \BibitemShut  [1]{\csname bibitem#1\endcsname}%
\let\auto@bib@innerbib\@empty
%</preamble>
\bibitem [{\citenamefont {Weber}\ \emph {et~al.}(2016)\citenamefont {Weber},
  \citenamefont {Georgii},\ and\ \citenamefont {Böni}}]{Weber2016}%
  \BibitemOpen
  \bibfield  {author} {\bibinfo {author} {\bibfnamefont {T.}~\bibnamefont
  {Weber}}, \bibinfo {author} {\bibfnamefont {R.}~\bibnamefont {Georgii}},\
  and\ \bibinfo {author} {\bibfnamefont {P.}~\bibnamefont {Böni}},\ }\bibfield
   {title} {\bibinfo {title} {Takin: An open-source software for experiment
  planning, visualisation, and data analysis},\ }\href
  {https://doi.org/10.1016/j.softx.2016.06.002} {\bibfield  {journal} {\bibinfo
   {journal} {SoftwareX}\ }\textbf {\bibinfo {volume} {5}},\ \bibinfo {pages}
  {121} (\bibinfo {year} {2016})}\BibitemShut {NoStop}%
\bibitem [{\citenamefont {Weber}(2017)}]{Weber2017}%
  \BibitemOpen
  \bibfield  {author} {\bibinfo {author} {\bibfnamefont {T.}~\bibnamefont
  {Weber}},\ }\bibfield  {title} {\bibinfo {title} {Update 1.5 to “takin: An
  open-source software for experiment planning, visualisation, and data
  analysis”, (pii: S2352711016300152)},\ }\href
  {https://doi.org/10.1016/j.softx.2017.06.002} {\bibfield  {journal} {\bibinfo
   {journal} {SoftwareX}\ }\textbf {\bibinfo {volume} {6}},\ \bibinfo {pages}
  {148} (\bibinfo {year} {2017})}\BibitemShut {NoStop}%
\bibitem [{\citenamefont {Weber}(2021)}]{Weber2021}%
  \BibitemOpen
  \bibfield  {author} {\bibinfo {author} {\bibfnamefont {T.}~\bibnamefont
  {Weber}},\ }\bibfield  {title} {\bibinfo {title} {Update 2.0 to “takin: An
  open-source software for experiment planning, visualisation, and data
  analysis”, (pii: S2352711016300152)},\ }\href
  {https://doi.org/10.1016/j.softx.2021.100667} {\bibfield  {journal} {\bibinfo
   {journal} {SoftwareX}\ }\textbf {\bibinfo {volume} {14}},\ \bibinfo {pages}
  {100667} (\bibinfo {year} {2021})}\BibitemShut {NoStop}%
\bibitem [{\citenamefont {Fransson}\ \emph {et~al.}(2021)\citenamefont
  {Fransson}, \citenamefont {Slabanja}, \citenamefont {Erhart},\ and\
  \citenamefont {Wahnstr\"{o}m}}]{FraSlaErhWah2021}%
  \BibitemOpen
  \bibfield  {author} {\bibinfo {author} {\bibfnamefont {E.}~\bibnamefont
  {Fransson}}, \bibinfo {author} {\bibfnamefont {M.}~\bibnamefont {Slabanja}},
  \bibinfo {author} {\bibfnamefont {P.}~\bibnamefont {Erhart}},\ and\ \bibinfo
  {author} {\bibfnamefont {G.}~\bibnamefont {Wahnstr\"{o}m}},\ }\bibfield
  {title} {\bibinfo {title} {dynasor {\textemdash}a tool for extracting
  dynamical structure factors and current correlation functions from molecular
  dynamics simulations},\ }\href {https://doi.org/10.1002/adts.202000240}
  {\bibfield  {journal} {\bibinfo  {journal} {Adv. Theory Simul.}\ }\textbf
  {\bibinfo {volume} {4}},\ \bibinfo {pages} {2000240} (\bibinfo {year}
  {2021})}\BibitemShut {NoStop}%
\bibitem [{\citenamefont {Hellman}\ \emph {et~al.}(2011)\citenamefont
  {Hellman}, \citenamefont {Abrikosov},\ and\ \citenamefont
  {Simak}}]{hellmanLatticeDynamicsAnharmonic2011}%
  \BibitemOpen
  \bibfield  {author} {\bibinfo {author} {\bibfnamefont {O.}~\bibnamefont
  {Hellman}}, \bibinfo {author} {\bibfnamefont {I.~A.}\ \bibnamefont
  {Abrikosov}},\ and\ \bibinfo {author} {\bibfnamefont {S.~I.}\ \bibnamefont
  {Simak}},\ }\bibfield  {title} {\bibinfo {title} {Lattice dynamics of
  anharmonic solids from first principles},\ }\href
  {https://doi.org/10.1103/PhysRevB.84.180301} {\bibfield  {journal} {\bibinfo
  {journal} {Phys. Rev. B}\ }\textbf {\bibinfo {volume} {84}},\ \bibinfo
  {pages} {180301(R)} (\bibinfo {year} {2011})}\BibitemShut {NoStop}%
\bibitem [{\citenamefont {Korotaev}\ \emph {et~al.}(2018)\citenamefont
  {Korotaev}, \citenamefont {Belov},\ and\ \citenamefont
  {Yanilkin}}]{korotaevReproducibilityVibrationalFree2018}%
  \BibitemOpen
  \bibfield  {author} {\bibinfo {author} {\bibfnamefont {P.}~\bibnamefont
  {Korotaev}}, \bibinfo {author} {\bibfnamefont {M.}~\bibnamefont {Belov}},\
  and\ \bibinfo {author} {\bibfnamefont {A.}~\bibnamefont {Yanilkin}},\
  }\bibfield  {title} {\bibinfo {title} {Reproducibility of vibrational free
  energy by different methods},\ }\href
  {https://doi.org/10.1016/j.commatsci.2018.03.057} {\bibfield  {journal}
  {\bibinfo  {journal} {Computational Materials Science}\ }\textbf {\bibinfo
  {volume} {150}},\ \bibinfo {pages} {47} (\bibinfo {year} {2018})}\BibitemShut
  {NoStop}%
\bibitem [{\citenamefont {Metsanurk}\ and\ \citenamefont
  {Klintenberg}(2019)}]{metsanurkSamplingdependentSystematicErrors2019}%
  \BibitemOpen
  \bibfield  {author} {\bibinfo {author} {\bibfnamefont {E.}~\bibnamefont
  {Metsanurk}}\ and\ \bibinfo {author} {\bibfnamefont {M.}~\bibnamefont
  {Klintenberg}},\ }\bibfield  {title} {\bibinfo {title} {Sampling-dependent
  systematic errors in effective harmonic models},\ }\href
  {https://doi.org/10.1103/PhysRevB.99.184304} {\bibfield  {journal} {\bibinfo
  {journal} {Phys. Rev. B}\ }\textbf {\bibinfo {volume} {99}},\ \bibinfo
  {pages} {184304} (\bibinfo {year} {2019})}\BibitemShut {NoStop}%
\bibitem [{\citenamefont {Castellano}\ \emph {et~al.}(2022)\citenamefont
  {Castellano}, \citenamefont {Bottin}, \citenamefont {Bouchet}, \citenamefont
  {Levitt},\ and\ \citenamefont
  {Stoltz}}]{castellanoInitioCanonicalSampling2022}%
  \BibitemOpen
  \bibfield  {author} {\bibinfo {author} {\bibfnamefont {A.}~\bibnamefont
  {Castellano}}, \bibinfo {author} {\bibfnamefont {F.}~\bibnamefont {Bottin}},
  \bibinfo {author} {\bibfnamefont {J.}~\bibnamefont {Bouchet}}, \bibinfo
  {author} {\bibfnamefont {A.}~\bibnamefont {Levitt}},\ and\ \bibinfo {author}
  {\bibfnamefont {G.}~\bibnamefont {Stoltz}},\ }\bibfield  {title} {\bibinfo
  {title} {{\emph{Ab Initio}} canonical sampling based on variational
  inference},\ }\href {https://doi.org/10.1103/PhysRevB.106.L161110} {\bibfield
   {journal} {\bibinfo  {journal} {Phys. Rev. B}\ }\textbf {\bibinfo {volume}
  {106}},\ \bibinfo {pages} {L161110} (\bibinfo {year} {2022})}\BibitemShut
  {NoStop}%
\bibitem [{\citenamefont {Fransson}\ \emph {et~al.}(2022)\citenamefont
  {Fransson}, \citenamefont {Rosander}, \citenamefont {Eriksson}, \citenamefont
  {Rahm}, \citenamefont {Tadano},\ and\ \citenamefont
  {Erhart}}]{franssonProbingLimitsPhonon2022}%
  \BibitemOpen
  \bibfield  {author} {\bibinfo {author} {\bibfnamefont {E.}~\bibnamefont
  {Fransson}}, \bibinfo {author} {\bibfnamefont {P.}~\bibnamefont {Rosander}},
  \bibinfo {author} {\bibfnamefont {F.}~\bibnamefont {Eriksson}}, \bibinfo
  {author} {\bibfnamefont {J.~M.}\ \bibnamefont {Rahm}}, \bibinfo {author}
  {\bibfnamefont {T.}~\bibnamefont {Tadano}},\ and\ \bibinfo {author}
  {\bibfnamefont {P.}~\bibnamefont {Erhart}},\ }\href@noop {} {\bibinfo {title}
  {Probing the limits of the phonon quasi-particle picture: {{The}} transition
  from underdamped to overdamped dynamics in {{CsPbBr}}$_3$}} (\bibinfo {year}
  {2022}),\ \Eprint {https://arxiv.org/abs/2211.08197} {arXiv:2211.08197
  [cond-mat]} \BibitemShut {NoStop}%
\end{thebibliography}%

\end{document}
