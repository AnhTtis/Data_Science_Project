\documentclass{article}

\setlength{\parindent}{12pt}

% Language setting
% Replace `english' with e.g. `spanish' to change the document language
\usepackage[english]{babel}

% Set page size and margins
% Replace `letterpaper' with `a4paper' for UK/EU standard size
\usepackage[letterpaper,top=2cm,bottom=2cm,left=3cm,right=3cm,marginparwidth=1.75cm]{geometry}

% Useful packages
\usepackage{physics}
\usepackage{schemata}
\usepackage{subcaption}
\usepackage{array}
\usepackage{float}
\usepackage{amsfonts}
\usepackage{amssymb}
\usepackage{amsmath}
\usepackage{graphicx}
\usepackage[colorlinks=true, allcolors=blue]{hyperref}

\title{Supplementary Information}
\author{L. Escalera-Moreno}

\begin{document}
\maketitle

\tableofcontents

\newpage

\section{Theory development}

\subsection{Spin-vibration coupling Hamiltonian}

The target systems are magnetic objects described by a single ground electron spin quantum number $S\geq 1/2$. Depending on the magnitude of the spin-orbit coupling, $S$ may be replaced by the quantum number $J$ which also comprises any possible orbital contribution $L$. From now on, we will use the $J$ quantum number where we include the particular case in which $L=0$ and $J=S$. 
\bigbreak
In case the electronic structure of the magnetic object is more complex such that it contains excited $J_{\text{ex}}$ quantum numbers, we note at this point that all of them are neglected in our approach. For this approximation to work, we suppose that the energy gap between $J$ and the first excited $J_{\text{ex}}$ is large enough as compared to the working temperature. On the other hand, the magnetic object could also be characterized by several $J_i$ quantum numbers such that, for low enough temperatures, they all couple in an effective giant ground electron spin $J$. Under this circumstance, $J$ can be employed in our approach as the single ground electron spin quantum number described above. 
\bigbreak
All in all, the elementary building-block is a single ground electron spin quantum number $J$ whose $2J+1$ states (i) are well-isolated from the states of excited $J_{\text{ex}}$ quantum numbers which are neglected in case of existing, and (ii) interact with an external, weak, and static magnetic field $\va{B}=(B_x,B_y,B_z)$. Being $\{\hat{J}_{\alpha}\}_{\alpha=x,y,z}$ the Cartesian components of the ground electron spin operator $\hat{\va{J}}=(\hat{J}_x,\hat{J}_y,\hat{J}_z)$, this situation is simply described by a spin Hamiltonian $\hat{H}$ consisting in a Zeeman term as follows, where $\mu_B$ is the Bohr magneton and $\hbar$ is the reduced Planck's constant:

\begin{equation}
\hat{H}=\frac{\mu_B}{\hbar}\sum_{\alpha=x,y,z}g_{\alpha}B_{\alpha}\hat{J}_{\alpha}
\label{zeeman}
\end{equation} 

Note that whenever spin-orbit coupling is absent and orbital degrees of freedom do not mix with the spin degrees of freedom one sets $g_{\alpha}=g_J$ for every $\alpha=x,y,z$, where $g_J$ is the free-$J$ Landé factor. The fact of letting $g_{\alpha}$ be different among them and different from $g_J$ is just an effective way of recovering the electron spin anisotropy in case of a non-negligible spin-orbit coupling. 
\bigbreak
A particular kind of magnetic objects of our interest is that of molecular coordination complexes, where one or several magnetic atoms or magnetic atomic ions are surrounded by and linked to a set of donor atoms via coordinate covalent bonds. Either the number of magnetic atoms or magnetic atomic ions is (i) greater than one or (ii) just one, let us recall that the working conditions must met the requirement of describing the low-lying energy scheme of the molecular coordination complex with a single ground electron spin quantum number $J$ as a good approximation. In the case of (i), $J$ should be the ground quantum number of coupling the electron spin quantum numbers $J_i$ of the several magnetic atoms or magnetic atomic ions as mentioned above, while in the case of (ii) $J$ is the ground electron spin quantum number of the unique magnetic atom or magnetic atomic ion. In both cases, the degeneracy of the $2J+1$ states is lifted by the electrostatic field of the coordinating donor atoms thus producing a zero-field splitting. This fact is modeled by adding the Crystal Field Hamiltonian as follows, written in terms of the Crystal Field Parameters (CFPs) $B_k^q$ and the Extended Stevens Operators (ESOs) $\hat{O}_k^q$ with $k=2,4,6$ and $q=-k,-k+1,...,k-1,k$:

\begin{equation}
\hat{H}=\sum_{k,q}B_k^q\hat{O}_k^q+\frac{\mu_B}{\hbar}\sum_{\alpha=x,y,z}g_{\alpha}B_{\alpha}\hat{J}_{\alpha}
\label{zeemancrystal}
\end{equation}
\bigbreak

The Crystal Field Hamiltonian can also be employed when working with other magnetic objects such as nitrogen-vacancy centers in nanodiamonds, which also show a zero-field splitting of the ground electron spin quantum number $S=1$. Note that when $J=1/2$ and the magnetic field is absent, the $2J+1=2$ states are degenerate in virtue of the Kramers' theorem which translates into $B_k^q\equiv 0$ for every $k,q$. In case of existing a ground nuclear spin quantum number $I\neq 0$, the Hilbert space can be further expanded by coupling $J$ and $I$ via a hyperfine interaction. The expanded Hilbert space is now composed of $(2J+1)(2I+1)$ states, and the definitive general spin Hamiltonian $\hat{H}$ that we use in our framework includes the additional terms of (i) the hyperfine interaction between $\hat{\va{J}}$ and the ground nuclear spin operator $\hat{\va{I}}$ with parameters $\{A_{\beta}\}_{\beta=x,y,z}$, and (ii) the quadrupole interaction describing a possible zero-field splitting of $I$ with parameter $P$:

\begin{equation}
\hat{H}=\sum_{k,q}B_k^q\hat{O}_k^q+\frac{\mu_B}{\hbar}\sum_{\alpha=x,y,z}g_{\alpha}B_{\alpha}\hat{J}_{\alpha}+\sum_{\beta=x,y,z}A_{\beta}\hat{I}_{\beta}\hat{J}_{\beta}+P\hat{I}_z^2
\label{spinH}
\end{equation}

Note that when $J=1/2$ -hence $B_k^q \equiv 0$-, the parameters $g_{\alpha}$, $A_{\beta}$ and $P$ are often taken as free parameters to fit the relevant cw-EPR spectrum. On the other hand, when computing the $g$ and $A$ tensors, one would set $g_x=g_{xx}$, $g_y=g_{yy}$, $g_z=g_{zz}$, $A_x=A_{xx}$, $A_y=A_{yy}$, $A_z=A_{zz}$ from the diagonal elements of $g$ and $A$ thus non-using the off-diagonal ones. In case $J>1/2$ and not all CFPs are neglected, it is common to take $g_{\alpha}=g_J$ such that the electron spin anisotropy is encoded in the crystal field Hamiltonian.
\bigbreak
To derive the spin-vibration coupling matrix elements, we first Taylor-develop the CFPs $B_k^q$ and the effective Landé factors $g_{\alpha}$ up to second order as a function of the vibration normal coordinates $\hat{Q}_1,\ldots,\hat{Q}_R$, being $R$ the number of vibration normal modes. When $g_{\alpha}=g_J$, the Taylor development is only performed on the CFPs and $g_J$ is kept constant. The parameters $A_{\beta}$ and $P$ do not depend so much on the environment distortions. Hence, we suppose that the variations of $A_{\beta}$ and $P$ with $\hat{Q}_i$ will be small enough as compared to those of $B_k^q$ and $g_{\alpha}$, and no Taylor expansion is performed on $A_{\beta}$ and $P$. We just use for $A_{\beta}$ and $P$ the numerical values as determined either by fitting the relevant cw-EPR spectrum or by first-principles calculations. Whenever $\va{B}=\va{0}$ and $J=1/2$ -hence $B_k^q \equiv 0$-, the remaining terms in the spin Hamiltonian would be those of the hyperfine and quadrupole interactions. In this case, we would have to Taylor-develop $A_{\beta}$ and $P$ in order to not give up on studying the spin-vibration coupling. Notwithstanding, our focus is put on EPR experiments where the magnetic field $\va{B}$ is always activated with a high enough magnitude to dominate over the hyperfine and quadrupole terms and thus vibrational relaxation will always be determined either by the modulation of the Zeeman term or by the said modulation together with that of the Crystal Field Hamiltonian.\cite{LunghiHow}
\bigbreak
Let now $T_l$ be either $B_k^q$ or $g_{\alpha}$ and let us consider the function $T_l=T_l(\hat{Q}_1,\ldots,\hat{Q}_R)$. The Taylor-development is performed around $\hat{Q}_1=\ldots=\hat{Q}_R=0$ which corresponds to the relaxed geometry of the magnetic object. Note that this relaxed geometry is obtained after the minimization of the potential energy function often performed under the harmonic approximation. In this case, the output of this relaxation process is composed of the following three sets: (i) the linear harmonic frequencies $\nu_i$, (ii) the reduced masses $\mu_i$, and (iii) the displacement vectors $\va{v}_i$. Each vibration normal mode $i$ is characterized by the three elements $\{\nu_i,\mu_i,\va{v}_i\}$. The relationship between the relaxed geometry -whose 3D position is denoted by the 3N-dimensional vector $\va{v}_{eq}$ where $N$ is the number of atoms- and the displacement vector $\va{v}_i$ of a normal mode $i$ is $\va{v}_i^{dist}=\va{v}_{eq}+Q_i\va{v}_i$, where $\va{v}_i^{dist}$ is the resulting distorted geometry produced when giving $\hat{Q}_i$ a particular value $Q_i$. Note that when $Q_i=0$, the relaxed geometry is recovered. The Taylor expansion is performed up to second order to include both the direct and the Raman mechanisms in the spin-vibration relaxation. 
\bigbreak
To compute the spin-vibration matrix elements, we need first to express the normal coordinates as a function of the second-quantization creation $\hat{a}_i^{\dagger}$ and annihilation $\hat{a}_i$ operators, namely:

\begin{equation}
\hat{Q}_i=\sqrt{\frac{\hbar}{2m_i\omega_i}}\left(\hat{a}_i+\hat{a}_i^{\dagger}\right)
\label{secondquant}
\end{equation}

The angular harmonic frequency is $\omega_i=2\pi\nu_i$. Note that $\hat{n}_i=\hat{a}_i^{\dagger}\hat{a}_i$ is the so-called phonon number operator of normal mode $i$, and the action of $\hat{n}_i$, $\hat{a}_i$ and $\hat{a}_i^{\dagger}$ on a given Fock state $\ket{n_i}$ with $n_i$ phonons of the normal mode $i$ is:

\begin{equation}
\hat{n}_i\ket{n_i}=n_i\ket{n_i}\qquad
\hat{a}_i\ket{0}=0\qquad
\hat{a}_i\ket{n_i}=\sqrt{n_i}\ket{n_i-1}\qquad
\hat{a}_i^{\dagger}\ket{n_i}=\sqrt{n_i+1}\ket{n_i+1}
\label{actionop}
\end{equation}

The evaluation of the following strain-tensor matrix elements is now straightforward:

\begin{equation}
\bra{n_i-1}\hat{Q}_i\ket{n_i}=\sqrt{\frac{\hbar}{2m_i\omega_i}n_i}
\label{stte1}
\end{equation}

\begin{equation}
\bra{n_i+1}\hat{Q}_i\ket{n_i}=\sqrt{\frac{\hbar}{2m_i\omega_i}\left(n_i+1\right)}
\label{stte2}
\end{equation}

The vibration bath is in thermal equilibrium at a given temperature $T$ since its dynamics is much faster than the one involved in the spin relaxation. This fact allows implementing the relevant adiabatic approximation and replacing each given phonon number $n_i$ by the thermally-averaged value $\langle n_i\rangle=\left(\text{exp}\left(\hbar\omega_i/k_BT\right)-1\right)^{-1}$ (Bose-Einstein statistics).
\bigbreak
The above-mentioned Taylor expansion is:

\begin{equation}
T_l\left(\hat{Q}_1,\ldots,\hat{Q}_R\right)\approx T_{l,eq}+\sum_{i=1}^RT_{l,i,eq}\hat{Q}_i+\frac{1}{2}\sum_{i'=1}^R\sum_{i=1}^RT_{l,i,i',eq}\hat{Q}_i\hat{Q}_{i'}
\label{Tayexp}
\end{equation}

$T_{l,eq}$ is the value of $T_l$ at $Q_1=\ldots=Q_R=0$. After implementing this development, the spin Hamiltonian $\hat{H}$ is approximated as:

\begin{equation}
\Hat{H}\approx\hat{H}_{eq}+\sum_{i=1}^R\hat{H}_i^{(1)}+\sum_{i'=1}^R\sum_{i=1}^R\hat{H}_{ii'}^{(2)}=\hat{H}_{eq}+\hat{H}_{eq-vib}
\label{Happrox}
\end{equation}

\begin{equation}
\hat{H}_{eq}=\sum_{k,q}B_{k,eq}^q\hat{O}_k^q+\frac{\mu_B}{\hbar}\sum_{\alpha=x,y,z}g_{\alpha,eq}B_{\alpha}\hat{J}_{\alpha}+\sum_{\beta=x,y,z}A_{\beta}\hat{I}_{\beta}\hat{J}_{\beta}+P\hat{I}_z^2
\label{Heq}
\end{equation}

\begin{equation}
\hat{H}_i^{(1)}=\left(\sum_{k,q}B_{k,i,eq}^q\hat{O}_k^q+\frac{\mu_B}{\hbar}\sum_{\alpha=x,y,z}g_{\alpha,i,eq}B_{\alpha}\hat{J}_{\alpha}\right)\hat{Q}_i
\label{Hi1}
\end{equation}

\begin{equation}
\hat{H}_{ii'}^{(2)}=\frac{1}{2}\left(\sum_{k,q}B_{k,i,i',eq}^q\hat{O}_k^q+\frac{\mu_B}{\hbar}\sum_{\alpha=x,y,z}g_{\alpha,i,i',eq}B_{\alpha}\hat{J}_{\alpha}\right)\hat{Q}_i\hat{Q}_{i'}
\label{Hii2}
\end{equation}

The derivatives, shown below, are evaluated at the equilibrium geometry, namely at $Q_1=\ldots=Q_R=0$:

\begin{equation}
B_{k,i,eq}^q=\left(\frac{\partial B_k^q}{\partial\hat{Q}_i}\right)_{eq} \qquad g_{\alpha,i,eq}=\left(\frac{\partial g_{\alpha}}{\partial\hat{Q}_i}\right)_{eq}
\label{T1der}
\end{equation}

\begin{equation}
B_{k,i,i',eq}^q=\left(\frac{\partial^2 B_k^q}{\partial\hat{Q}_i\partial\hat{Q}_{i'}}\right)_{eq} \qquad g_{\alpha,i,i',eq}=\left(\frac{\partial^2 g_{\alpha}}{\partial\hat{Q}_i\partial\hat{Q}_{i'}}\right)_{eq}
\label{T2der}
\end{equation}

We suppose that the second derivatives are symmetric, namely $B_{k,i,i',eq}^q=B_{k,i',i,eq}^q$ and $g_{\alpha,i,i',eq}=g_{\alpha,i',i,eq}$. This is something reasonable because each distorted geometry has a linear dependence on the associated displacement vector such that, since summation is commutative, the resulting distorted geometry is the same both if the normal mode $i$ is firstly applied and then the $i'$ normal mode, and if the reverse order is applied. Details on first-principles computation of first and second derivatives involving a single normal mode can be found elsewhere.\cite{firstder1,firstder2} Concerning cross second derivatives involving two different normal modes, their first-principles computation is more sophisticated but yet feasible. Given two different normal modes $i$ and $j$, the key idea is again to generate a set of distorted geometries $\{\va{v}_k^{dist}\}_k$ around the relaxed geometry by giving $\hat{Q}_i$ and $\hat{Q}_j$ a proper set of values around $Q_i=0$ and $Q_j=0$. Specifically, $\va{v}_k^{dist}=\va{v}_{eq}+Q_i\va{v}_i+Q_j\va{v}_j$, where each given $k$ corresponds to particular values $Q_i$ and $Q_j$. At each one of these distorted geometries, one computes $T_l^k$. Then, the set $\{T_l^k\}_k$ is fitted to a two-variable polynomial $P_{ij}=P_{ij}(Q_i,Q_j)$ which can be employed to determine the cross second derivative $\partial^2P_{ij}/\partial Q_j\partial Q_i$ evaluated at $Q_i=Q_j=0$. All in all, the complete Hamiltonian reads as follows:

\begin{equation}
\hat{H}_c = \hat{H}_{eq} + \hat{H}_{vib} + \hat{H}_{eq-vib}
\label{complworkH}
\end{equation}

where $\hat{H}_{vib}$ is the Hamiltonian describing the vibration bath composed of $R$ harmonic normal modes with angular frequencies $\omega_i$ and phonon number operators $\hat{n}_i$:

\begin{equation}
\hat{H}_{vib} = \sum_{i=1}^R\hbar\omega_i\left(\hat{n}_i+\frac{1}{2}\right)
\label{vibbathH}
\end{equation}

The diagonalization of $\hat{H}_{eq}$ provides a set of $\left(2J+1\right)\left(2I+1\right)$ spin states and energies. Inside this energy scheme, two eigenstates are selected to be the two qubit states which are denoted as $\ket{u_-}$ and $\ket{u_+}$ with energies $u_- < u_+$. Now, we define the so-called pseudo-spin $\tilde{S}=1/2$ operators $\tilde{S}_x$, $\tilde{S}_y$, $\tilde{S}_z$ which satisfy the formalism: 

\begin{equation}
\tilde{S}_x\ket{u_+}=\frac{\hbar}{2}\ket{u_-} \quad \tilde{S}_x\ket{u_-}=\frac{\hbar}{2}\ket{u_+}
\label{pseudox}
\end{equation}
\begin{equation}
\tilde{S}_y\ket{u_+}=\frac{i\hbar}{2}\ket{u_-} \quad
\tilde{S}_y\ket{u_-}=-\frac{i\hbar}{2}\ket{u_+}
\label{pseudoy}
\end{equation}
\begin{equation}
\tilde{S}_z\ket{u_+}=\frac{\hbar}{2}\ket{u_+} \quad 
\tilde{S}_z\ket{u_-}=-\frac{\hbar}{2}\ket{u_-}
\label{pseudoz}
\end{equation}
\begin{equation}
\braket{u_+}=\braket{u_-}=1, \bra{u_+}\ket{u_-}=0
\label{orthocond}
\end{equation}

By considering the effective ordered basis set $\{\ket{u_+},\ket{u_-}\}$ from now on, we have the vector and matrix representation:

\begin{equation}
\ket{u_+} = \mqty(1 \\ 0) \qquad \ket{u_-} = \mqty(0 \\ 1)
\label{vecrep}
\end{equation}

\begin{equation}
S_x = \frac{\hbar}{2}\mqty(0 & 1 \\ 1 & 0) \qquad 
S_y = \frac{\hbar}{2}\mqty(0 & -i \\ i & 0) \qquad 
S_z = \frac{\hbar}{2}\mqty(1 & 0 \\ 0 & -1)
\label{matrep}
\end{equation}

The quantization axis given by $\tilde{S}_z$ is chosen to be parallel to the direction of the magnetic field $\va{B}$. The qubit effective Hamiltonian (two-level system with an energy gap $u_+-u_-$) is simply:

\begin{equation}
\hat{H}_{\text{eff}}=\frac{u_+-u_-}{\hbar}\tilde{S}_z=\frac{1}{2}\left(u_+-u_-\right)\left(\ket{u_+}\bra{u_+}-\ket{u_-}\bra{u_-}\right)
\label{Heffqubit}
\end{equation}

\subsection{One-phonon transition rates}

The so-called Fermi golden rule applied to one-phonon processes provides the absorption $\left(\ket{u_-}\rightarrow{}\ket{u_+}\right)$ and emission $\left(\ket{u_+}\rightarrow{}\ket{u_-}\right)$ transition rates given by ($\hbar\omega_{+-}=u_+-u_-$, $\omega_{+-}$ is the Larmor frequency):

\begin{equation}
\Gamma_{\ket{u_-}\rightarrow{}\ket{u_+}}^{\text{ab.1p}}=\frac{2\pi}{\hbar^2}\sum_{i=1}^R\left|\bra{n_i-1,u_+}\hat{H}_i^{(1)}\ket{u_-,n_i}\right|^2\delta\left(\omega_{+-}-\omega_i\right)
\label{abs1ph}
\end{equation}

\begin{equation}
\Gamma_{\ket{u_+}\rightarrow{}\ket{u_-}}^{\text{em.1p}}=\frac{2\pi}{\hbar^2}\sum_{i=1}^R\left|\bra{n_i+1,u_-}\hat{H}_i^{(1)}\ket{u_+,n_i}\right|^2\delta\left(\omega_{+-}-\omega_i\right)
\label{em1ph}
\end{equation}

The computation of the matrix elements leads to:

\begin{equation}
\bra{n_i-1,u_+}\hat{H}_i^{(1)}\ket{u_-,n_i}=\sqrt{\frac{\hbar}{2m_i\omega_i}\langle n_i\rangle}\left(\sum_{k,q}B_{k,i,eq}^q\bra{u_+}\hat{O}_k^q\ket{u_-}+\mu_B\sum_{\alpha=x,y,z}g_{\alpha,i,eq}B_{\alpha}\bra{u_+}\frac{\hat{J}_{\alpha}}{\hbar}\ket{u_-}\right)
\label{mateleabs1ph}
\end{equation}

\begin{equation}
\bra{n_i+1,u_-}\hat{H}_i^{(1)}\ket{u_+,n_i}=\sqrt{\frac{\hbar}{2m_i\omega_i}\left(\langle n_i\rangle+1\right)}\left(\sum_{k,q}B_{k,i,eq}^q\bra{u_-}\hat{O}_k^q\ket{u_+}+\mu_B\sum_{\alpha=x,y,z}g_{\alpha,i,eq}B_{\alpha}\bra{u_-}\frac{\hat{J}_{\alpha}}{\hbar}\ket{u_+}\right)
\label{mateleem1ph}
\end{equation}

\subsection{Two-phonon transition rates}

We consider the six possible different types of two-phonon processes through intermediate $\ket{c}$ -both real (eigenstates of $\hat{H}_{eq}$) and virtual- states: raising and lowering direct transition (R-Direct and L-Direct), Stokes transition (Stokes), anti-Stokes transition (anti-Stokes), and spontaneous emission followed by absorption (R-Spont. and L-Spont.), see figures below.

\begin{figure}[H]
    \centering
    \includegraphics[scale=0.35]{Raising2phononprocesses.jpg}
    \hspace{1cm}
    \includegraphics[scale=0.35]{Lowering2phononprocesses.jpg}
   \caption{2-phonon raising (left) and lowering (right) processes.}
   \label{RL2pp}
\end{figure}

After applying the Fermi golden rule to these two-phonon processes, the expressions of the several transition rates involve matrix elements of the form $V_{\ket{i}}^{\ket{f}}=\bra{i}\hat{H}_{eq-vib}\ket{f}$ with initial $\ket{i}=\ket{e,n_i+p_1,n_{i'}+p_2}$ and final $\ket{f}=\ket{t,n_i+p_3,n_{i'}+p_4}$ states, and $\ket{e},\ket{t}\in\{\ket{u_-},\ket{u_+}\}$, $p_1,p_2,p_3,p_4\in\{-1,0,1\}$, $\ket{e}\rightarrow{}\ket{c}\rightarrow{}\ket{t}$. In the calculation of $V_{\ket{i}}^{\ket{f}}$ two normal modes are involved. Both of them change their phonon number at once for the virtual process, while this change is produced one after the other one in the real processes. This means that in the virtual process, since the terms $\hat{H}_i^{(1)}$ only contain one normal mode, the evaluation of the matrix elements will produce some summands proportional to $\bra{n_{i'}}\ket{n_{i'}\pm1}\equiv 0$. Hence, only the terms $\hat{H}_{ii'}^{(2)}$ have a non-vanishing contribution to the matrix elements in the virtual process. In this case, $V_{\ket{i}}^{\ket{f}}=\bra{i}\hat{H}_{eq-vib}\ket{f}=\bra{i}\hat{H}_{ii'}^{(2)}\ket{f}$ with $p_1\neq p_3$ and $p_2 \neq p_4$. On the other hand, in the real processes, the non-vanishing contributions come from the terms $\hat{H}_{i}^{(1)}$. Since only one normal mode changes its phonon number in each one of the two steps, the terms $\hat{H}_{ii'}^{(2)}$ give rise to summands proportional to $\bra{n_{i'}+p}\hat{Q}_{i'}\ket{n_{i'}+p}$ which are identically zero if one recalls that $\hat{Q}_{i'}\propto \hat{a}_{i'} + \hat{a}_{i'}^{\dagger}$. Hence, in this case, $V_{\ket{i}}^{\ket{f}}=\bra{i}\hat{H}_{eq-vib}\ket{f}=\bra{i}\hat{H}_{i}^{(1)}\ket{f}$ with either $p_1\neq p_3$, $p_2=p_4$ or $p_1=p_3$, $p_2\neq p_4$. These matrix elements were already calculated in the one-phonon processes section above.
\bigbreak
As mentioned above, the remaining matrix elements are simple to compute and are of the form ($p_1,p_2,p_3,p_4\in\{-1,0,1\}$, $p_1 \neq p_3$, $p_2 \neq p_4$):

\begin{align*}
V_{\ket{e,n_i+p_1,n_{i'}+p_2}}^{\ket{t,n_i+p_3,n_{i'}+p_4}}=\bra{n_{i'}+p_2,n_i+p_1,e}\hat{H}_{ii'}^{(2)}\ket{t,n_i+p_3,n_{i'}+p_4}= \\
=\frac{1}{2}\left(\sum_{k,q}B_{k,i,i',eq}^q\bra{e}\hat{O}_k^q\ket{t}+\frac{\mu_B}{\hbar}\sum_{\alpha=x,y,z}g_{\alpha,i,i',eq}B_{\alpha}\bra{e}\hat{J}_{\alpha}\ket{t}\right)\cdot
\end{align*}
\begin{align}
\cdot\bra{n_i+p_1}\hat{Q}_i\ket{n_i+p_3}\bra{n_{i'}+p_2}\hat{Q}_{i'}\ket{n_{i'}+p_4}
\label{matelev2ppc}
\end{align}

where:

\begin{align}
\schema
    {
    \schemabox{$\bra{n_i+p_1}\hat{Q}_i\ket{n_i+p_3}=$}
    }
    {
    \schemabox{$\sqrt{\frac{\hbar}{2m_i\omega_i}\langle n_i\rangle}\hspace{0.2cm}$ R-Direct, Stokes, anti-Stokes (if normal mode $i$ loses 1 phonon) \\ $\sqrt{\frac{\hbar}{2m_i\omega_i}\left(\langle n_i\rangle+1\right)}\hspace{0.2cm}$ L-Direct, R-Spont., L-Spont. (if normal mode $i$ gains 1 phonon)}
    }
\label{matelem1}
\end{align}

\begin{align}
\schema
    {
    \schemabox{$\bra{n_{i'}+p_2}\hat{Q}_{i'}\ket{n_{i'}+p_4}=$}
    }
    {
    \schemabox{$\sqrt{\frac{\hbar}{2m_{i'}\omega_{i'}}\langle n_{i'}\rangle}\hspace{0.5cm}$ R-Direct, R-Spont., L-Spont. (if normal mode $i'$ loses 1 phonon) \\ $\sqrt{\frac{\hbar}{2m_{i'}\omega_{i'}}\left(\langle n_{i'}\rangle+1\right)}\hspace{0.5cm}$ L-Direct, Stokes, anti-Stokes (if normal mode $i'$ gains 1 phonon)}
    }
\label{matelem2}
\end{align}

The R-Direct and L-Direct transition rates are characterized by the resonance condition $\omega_i+\omega_{i'}=\omega_{+-}$ and are given by:

\begin{align}
\Gamma_{\ket{u_-}\rightarrow{}\ket{u_+}}^{\text{R-Direct}}=\frac{2\pi}{\hbar^2}\sum_{ii'=1}^R\left|V_{\ket{u_-,n_i,n_{i'}}}^{\ket{u_+,n_i-1,n_{i'}-1}}+\sum_{\substack{\ket{c}\neq \ket{u_-},\ket{u_+} \\ u_- \leq E_c \leq u_+}}\frac{V_{\ket{u_+,n_i-1,n_{i'}-1}}^{\ket{c,n_i-1,n_{i'}}}V_{\ket{c,n_i-1,n_{i'}}}^{\ket{u_-,n_i,n_{i'}}}}{u_--E_c+\hbar\omega_i}\right|^2\delta\left(\omega_{+-}-\omega_i-\omega_{i'}\right)
\label{RDirect}
\end{align}

\begin{align}
\Gamma_{\ket{u_+}\rightarrow{}\ket{u_-}}^{\text{L-Direct}}=\frac{2\pi}{\hbar^2}\sum_{ii'=1}^R\left|V_{\ket{u_+,n_i,n_{i'}}}^{\ket{u_-,n_i+1,n_{i'}+1}}+\sum_{\substack{\ket{c}\neq \ket{u_-},\ket{u_+} \\ u_- \leq E_c \leq u_+}}\frac{V_{\ket{u_-,n_i+1,n_{i'}+1}}^{\ket{c,n_i+1,n_{i'}}}V_{\ket{c,n_i+1,n_{i'}}}^{\ket{u_+,n_i,n_{i'}}}}{u_+-E_c-\hbar\omega_i}\right|^2\delta\left(\omega_{+-}-\omega_i-\omega_{i'}\right)
\label{LDirect}
\end{align}

The Stokes and anti-Stokes transition rates contain the resonance condition $\left|\omega_i-\omega_{i'}\right|=\omega_{+-}$ and read as follows:

\begin{align}
\Gamma_{\ket{u_-}\rightarrow{}\ket{u_+}}^{\text{Stokes}}=\frac{2\pi}{\hbar^2}\sum_{ii'=1}^R\left|V_{\ket{u_-,n_i,n_{i'}}}^{\ket{u_+,n_i-1,n_{i'}+1}}+\sum_{\substack{\ket{c}\neq \ket{u_-},\ket{u_+} \\ u_+ \leq E_c}}\frac{V_{\ket{u_+,n_i-1,n_{i'}+1}}^{\ket{c,n_i-1,n_{i'}}}V_{\ket{c,n_i-1,n_{i'}}}^{\ket{u_-,n_i,n_{i'}}}}{u_--E_c+\hbar\omega_i}\right|^2\delta\left(\omega_{+-}-\omega_i+\omega_{i'}\right)
\label{Stokes}
\end{align}

\begin{align}
\Gamma_{\ket{u_+}\rightarrow{}\ket{u_-}}^{\text{anti-Stokes}}=\frac{2\pi}{\hbar^2}\sum_{ii'=1}^R\left|V_{\ket{u_+,n_i,n_{i'}}}^{\ket{u_-,n_i-1,n_{i'}+1}}+\sum_{\substack{\ket{c}\neq \ket{u_-},\ket{u_+} \\ u_+ \leq E_c}}\frac{V_{\ket{u_-,n_i-1,n_{i'}+1}}^{\ket{c,n_i-1,n_{i'}}}V_{\ket{c,n_i-1,n_{i'}}}^{\ket{u_+,n_i,n_{i'}}}}{u_+-E_c+\hbar\omega_i}\right|^2\delta\left(\omega_{+-}+\omega_i-\omega_{i'}\right)
\label{antiStokes}
\end{align}

Last but not least, the transition rates for spontaneous emission followed by absorption are shown below also with $\left|\omega_i-\omega_{i'}\right|=\omega_{+-}$ as a resonance condition:

\begin{align}
\Gamma_{\ket{u_-}\rightarrow{}\ket{u_+}}^{\text{R-Spont.}}=\frac{2\pi}{\hbar^2}\sum_{ii'=1}^R\left|V_{\ket{u_-,n_i,n_{i'}}}^{\ket{u_+,n_i+1,n_{i'}-1}}+\sum_{\substack{\ket{c}\neq \ket{u_-},\ket{u_+} \\ E_c \leq u_-}}\frac{V_{\ket{u_+,n_i+1,n_{i'}-1}}^{\ket{c,n_i+1,n_{i'}}}V_{\ket{c,n_i+1,n_{i'}}}^{\ket{u_-,n_i,n_{i'}}}}{u_--E_c-\hbar\omega_i}\right|^2\delta\left(\omega_{+-}+\omega_i-\omega_{i'}\right)
\label{RSpont}
\end{align}

\begin{align}
\Gamma_{\ket{u_+}\rightarrow{}\ket{u_-}}^{\text{L-Spont.}}=\frac{2\pi}{\hbar^2}\sum_{ii'=1}^R\left|V_{\ket{u_+,n_i,n_{i'}}}^{\ket{u_-,n_i+1,n_{i'}-1}}+\sum_{\substack{\ket{c}\neq \ket{u_-},\ket{u_+} \\ E_c \leq u_-}}\frac{V_{\ket{u_-,n_i+1,n_{i'}-1}}^{\ket{c,n_i+1,n_{i'}}}V_{\ket{c,n_i+1,n_{i'}}}^{\ket{u_+,n_i,n_{i'}}}}{u_+-E_c-\hbar\omega_i}\right|^2\delta\left(\omega_{+-}-\omega_i+\omega_{i'}\right)
\label{LSpont}
\end{align}

We now define the global absorption and emission 2-phonon transition rates as follows:

\begin{equation}
\Gamma_{\ket{u_-}\rightarrow{}\ket{u_+}}^{\text{ab.2p}}=\Gamma_{\ket{u_-}\rightarrow{}\ket{u_+}}^{\text{R-Direct}}+\Gamma_{\ket{u_-}\rightarrow{}\ket{u_+}}^{\text{Stokes}}+\Gamma_{\ket{u_-}\rightarrow{}\ket{u_+}}^{\text{R-Spont.}}
\label{gloab2ptr}
\end{equation}

\begin{equation}
\Gamma_{\ket{u_+}\rightarrow{}\ket{u_-}}^{em.2p}=\Gamma_{\ket{u_+}\rightarrow{}\ket{u_-}}^{\text{L-Direct}}+\Gamma_{\ket{u_+}\rightarrow{}\ket{u_-}}^{\text{anti-Stokes}}+\Gamma_{\ket{u_+}\rightarrow{}\ket{u_-}}^{\text{L-Spont.}}
\label{gloem2ptr}
\end{equation}

\subsection{Spectral density of the vibration bath}

The harmonic approximation employed to describe the vibration bath as a finite set of normal modes produces a collection of well-defined harmonic frequencies $\omega_i$. Nonetheless, the non-zero temperature $T$ and the consequential anharmonicity widens each $\omega_i$ and makes the use of delta functions $\delta$ as a spectral density be unrealistic. Anharmonicity can be introduced \textit{ad hoc} by replacing each $\delta$ by an appropriate function centered at $\omega_i$ and involving a given width $\sigma_i$. We choose to use a Gaussian line shape which for the one-phonon and two-phonon processes is shown below ($\omega=2\pi\nu$):

\begin{align}
\delta\left(\omega_{+-}-\omega_i\right)\rightarrow{}\frac{1}{2\pi}\frac{1}{\sigma_i\sqrt{2\pi}}\text{exp}\left(-\frac{1}{2}\left(\frac{\nu_{+-}-\nu_i}{\sigma_i}\right)^2\right) \\
\delta\left(\omega_{+-}-\omega_i-\omega_{i'}\right)\rightarrow{}\frac{1}{2\pi}\frac{1}{\sigma_{ii'}\sqrt{2\pi}}\text{exp}\left(-\frac{1}{2}\left(\frac{\nu_{+-}-\nu_i-\nu_{i'}}{\sigma_{ii'}}\right)^2\right) \\
\delta\left(\omega_{+-}\pm\omega_i\mp\omega_{i'}\right)\rightarrow{}\frac{1}{2\pi}\frac{1}{\sigma_{ii'}\sqrt{2\pi}}\text{exp}\left(-\frac{1}{2}\left(\frac{\nu_{+-}\pm\nu_i\mp\nu_{i'}}{\sigma_{ii'}}\right)^2\right)
\label{spde}
\end{align}

with $\sigma_{ii'}=\sigma_i+\sigma_{i'}$ and we have used the property $\delta(\alpha x)=\frac{1}{|\alpha|}\delta(x)$. 

\subsection{Spin relaxation dynamics}

\subsubsection{Lindblad master equation}

Herein, we analyze the master equation that we employ to model the user-driven and relaxation-mediated time evolution of the given spin qubit which is pictured as an effective spin doublet (see pseudo-spin formalism above). We choose the Markovian Lindblad master equation to determine the time-evolution of the spin qubit reduced density operator $\hat{\rho}(t\geq t_0)=\text{Tr}_{\text{bath}}(\hat{\rho}_{\text{qubit+bath}})$.\cite{maestra2022,maestra2018} As usual, it is considered that at time $t=t_0$ the global density operator is given by the uncorrelated state $\hat{\rho}_{\text{qubit+bath}}(t_0)=\hat{\rho}_{\text{qubit}}(t_0)\otimes\hat{\rho}_{\text{bath}}(t_0)$. The use of the Markovian approximation is legit if one considers -as already mentioned- that the thermal bath dynamics is much faster than the spin dynamics, so no important memory effects are expected. In the so-called weak-coupling limit, the master equation results in:

\begin{equation}
\frac{\partial\hat{\rho}}{\partial t}\equiv\dot{\hat{\rho}}=\frac{1}{i\hbar}[\hat{H}_{\text{eff}},\hat{\rho}]+\mathcal{L}_{\text{1-p}}\hat{\rho}+\mathcal{L}_{\text{2-p}}\hat{\rho}+\mathcal{L}_{\text{mag}}\hat{\rho}
\label{mastereq}
\end{equation}

The first term describes the unitary dynamics of the spin qubit driven by the effective Hamiltonian $\hat{H}_{\text{eff}}$. The second and third terms shown below describe the relaxation induced by the one-phonon and two-phonon processes:

\begin{align}
\mathcal{L}_{\text{1-p}}\hat{\rho}= \Gamma_{\ket{u_-}\rightarrow{}\ket{u_+}}^{\text{ab.1p}}\mathcal{L}[L]\hat{\rho}+\Gamma_{\ket{u_+}\rightarrow{}\ket{u_-}}^{\text{em.1p}}\mathcal{L}[L^{\dagger}]\hat{\rho} \\
\mathcal{L}_{\text{2-p}}\hat{\rho}= \Gamma_{\ket{u_-}\rightarrow{}\ket{u_+}}^{\text{ab.2p}}\mathcal{L}[L]\hat{\rho}+\Gamma_{\ket{u_+}\rightarrow{}\ket{u_-}}^{\text{em.2p}}\mathcal{L}[L^{\dagger}]\hat{\rho}
\label{lindterms}
\end{align}

where the Lindblad superoperator is defined as:

\begin{equation}
\mathcal{L}[\hat{O}]\hat{\rho}=\hat{O}\hat{\rho}\hat{O}^{\dagger}-\frac{1}{2}\{\hat{O}^{\dagger}\hat{O},\hat{\rho}\}
\label{lindsup}
\end{equation}

and the jump operators are:

\begin{equation}
L=\ket{u_+}\bra{u_-} \qquad L^{\dagger}=\ket{u_-}\bra{u_+}
\label{spinop}
\end{equation}

The last term describes a phenomenological dynamics produced by an isotropic magnetic noise (which we often refer to as the spin bath):

\begin{equation}
\mathcal{L}_{\text{mag}}\hat{\rho}=-\frac{\Gamma_{\text{mag}}}{2\hbar^2}\sum_{\alpha=x,y,z}[\tilde{S}_{\alpha},[\tilde{S}_{\alpha},\hat{\rho}]]
\label{isomagnoise}
\end{equation}

E.g. the origin of this noise is often a thermal bath composed of electron/nuclear spins surrounding the spin qubit that are non-resonant with the energy of the pulses employed to drive the spin qubit. $\Gamma_{\text{mag}}$ is named as magnetic relaxation rate, depends on temperature and proportionally on the volume concentration of the electron/nuclear spins in the bath, and can be obtained via first-principles calculations by employing an adapted version of the software package SIMPRE,\cite{simpre2.0,espesp2019} which operates with Eq.\ref{spinH} and provides both the nuclear and electron spin bath relaxation times $T_n$ and $T_e$ such that:

\begin{equation}
\Gamma_{\text{mag}}=\frac{1}{T_n}+\frac{1}{T_e}
\label{gmagsimpre}
\end{equation}

By defining the global absorption and emission transition rates as:

\begin{equation}
\Gamma_{\text{ab}}=\Gamma_{\ket{u_-}\rightarrow{}\ket{u_+}}^{\text{ab.1p}}+\Gamma_{\ket{u_-}\rightarrow{}\ket{u_+}}^{\text{ab.2p}} \qquad \Gamma_{\text{em}}=\Gamma_{\ket{u_+}\rightarrow{}\ket{u_-}}^{\text{em.1p}}+\Gamma_{\ket{u_+}\rightarrow{}\ket{u_-}}^{\text{em.2p}}
\label{globabemtr}
\end{equation}

we rewrite the master equation as:

\begin{equation}
\dot{\hat{\rho}}=\frac{1}{i\hbar}[\hat{H}_{\text{eff}},\hat{\rho}]+\Gamma_{\text{ab}}\mathcal{L}[L]\hat{\rho}+\Gamma_{\text{em}}\mathcal{L}[L^{\dagger}]\hat{\rho}+\mathcal{L}_{\text{mag}}\hat{\rho}
\label{rwmaseq}
\end{equation}

\subsubsection{Driving oscillating magnetic field and Rabi frequency}

In order to be able to exert control on the spin qubit and rotate its state on the Bloch sphere, we must first expand the effective Hamiltonian $\hat{H}_{\text{eff}}$ with an oscillating magnetic field $\va{B}_1$. The $\va{B}_1$ direction -as determined by its wave-vector $\hat{k}$- must be perpendicular to the $\va{B}$ direction according to how an EPR apparatus is built. Let us now determine two orthogonal directions to the $\va{B}$ direction in order to use them as a basis set for $\hat{k}$. The unitary vector defining the $\va{B}$ direction is ($0\leq\phi\leq 2\pi$, $0\leq\theta\leq\pi$):

\begin{equation}
\hat{u}=(\text{sen}\theta\text{cos}\phi,\text{sen}\theta\text{sen}\phi,\text{cos}\theta)
\label{stmagfdir}
\end{equation}

A way to build an orthogonal direction $\hat{n}_1$ to $\hat{u}$ is by considering the Z-axis direction $\hat{z}=(0,0,1)$ and computing the normalized cross product:

\begin{equation}
\hat{n}_1=\frac{\hat{z}\times\hat{u}}{||\hat{z}\times\hat{u}||}=(-\text{sen}\phi,\text{cos}\phi,0)
\label{n1dir}
\end{equation}

The remaining normalized direction $\hat{n}_2$ is given by the cross product between $\hat{u}$ and $\hat{n}_1$. We choose the order $\hat{n}_1\times\hat{u}$:

\begin{equation}
\hat{n}_2=\hat{n}_1\times\hat{u}=(\text{cos}\theta\text{cos}\phi,\text{cos}\theta\text{sen}\phi,-\text{sen}\theta)
\label{n2dir}
\end{equation}

We select the ordered orthonormal basis set $\{\hat{n}_2,\hat{n}_1,\hat{u}\}$ which is appropriately right-handed since its determinant $\text{det}[\hat{n}_2,\hat{n}_1,\hat{u}]=1$ is positive. The normalized direction $\tilde{X}\equiv\hat{n}$ -perpendicular to $\va{B}$- is defined as ($\epsilon\in[0,2\pi[$):

\begin{equation}
\tilde{X}\equiv\hat{n}=\hat{n}_2\text{cos}\epsilon+\hat{n}_1\text{sen}\epsilon
\label{udrotdir}
\end{equation}

The wave-vector $\hat{k}$ is parallel to $\tilde{X}\equiv\hat{n}$. We also consider that $\va{B}_1$ -perpendicular to $\hat{k}$- is linearly polarized such that $\va{B}_1$ is contained in the plane generated by $\{\tilde{Y},\tilde{Z}\}$ with a polarization angle $\alpha\in[-\pi/2,+\pi/2]$ measured respect to $\tilde{Y}$, being $\tilde{Z}\equiv\hat{u}$ with $\tilde{Y}$ given by:

\begin{equation}
\tilde{Y}\equiv\hat{n}(\epsilon\rightarrow{}\epsilon+\pi/2)=\hat{n}_2\text{cos}(\epsilon+\pi/2)+\hat{n}_1\text{sen}(\epsilon+\pi/2)=-\hat{n}_2\text{sen}\epsilon+\hat{n}_1\text{cos}\epsilon
\label{ytildir}
\end{equation}

The basis set $\{\tilde{X},\tilde{Y},\tilde{Z}\}$ is obviously orthonormal and right-handed, and is depicted in the figure below. It defines the axes where the pseudo-spin $\tilde{S}=1/2$ operators $\tilde{S}_x,\tilde{S}_y,\tilde{S}_z$ are referred to. 

\begin{figure}[H]
    \centering
    \includegraphics[scale=0.8]{EffectiveAxes.jpg}
   \caption{Pseudo-spin $\tilde{S}=1/2$ axes.}
   \label{psaxes}
\end{figure}

The effect of $\va{B}_1=\left(B_{\text{1x}},B_{\text{1y}},B_{\text{1z}}\right)$ -also known as driving field- with given angular frequency $\omega_{\text{MW}}$ and magnitude $B_1=|\va{B}_1|$ is described by the driving Hamiltonian -written in the ordered basis set $\{\ket{u_+},\ket{u_-}\}$-:

\begin{equation}
\hat{H}_1=2\frac{\hbar}{2}\mqty(0 & \Omega_{\text{R}}^* \\ \Omega_{\text{R}} & 0)\text{cos}(\omega_{\text{MW}}(t-t_0))=\left(\va{\mu}_{+-}^*\cdot\va{B}_1\ket{u_+}\bra{u_-}+\va{\mu}_{+-}\cdot\va{B}_1\ket{u_-}\bra{u_+}\right)\text{cos}(\omega_{\text{MW}}(t-t_0))
\label{driHam}
\end{equation}

where the so-called long-wavelength approximation has been implemented and $\Omega_{\text{R}}$ is the so-called Rabi frequency:

\begin{equation}
\Omega_{\text{R}}=\frac{\va{\mu}_{+-}\cdot\va{B}_1}{\hbar}=\frac{\mu_{\text{B}}g_{\text{I}}}{\hbar^2}\sum_{\gamma=x,y,z}\bra{u_+}\hat{J}_{\gamma}\ket{u_-}B_{1\gamma}
\label{rabfreq}
\end{equation}

The magnetic dipolar moment associated to the transition between the two qubit states is:

\begin{equation}
\va{\mu}_{+-}=
\frac{\mu_{\text{B}}g_{\text{I}}}{\hbar}\left(\bra{u_+}\hat{J}_x\ket{u_-},\bra{u_+}\hat{J}_y\ket{u_-},\bra{u_+}\hat{J}_z\ket{u_-}\right)
\label{magdipmom}
\end{equation}

Since $\va{B}_1$ is contained in the plane generated by $\{\tilde{Y},\tilde{Z}\}$, we can easily build an expression as a linear combination of $\tilde{Y}$ and $\tilde{Z}$ in terms of $\alpha$:

\begin{align}
\va{B}_1=B_1\left(\tilde{Y}\text{cos}\alpha+\tilde{Z}\text{sen}\alpha\right)=B_1\left(\text{cos}\alpha\left(-\hat{n}_2\text{sen}\epsilon+\hat{n}_1\text{cos}\epsilon\right)+\hat{u}\text{sen}\alpha\right)
\label{lincombB1}
\end{align}

where one finds:

\begin{align*}
B_{\text{1x}}/B_1=-\text{cos}\alpha\text{sen}\epsilon\text{cos}\theta\text{cos}\phi-\text{cos}\alpha\text{cos}\epsilon\text{sen}\phi+\text{sen}\alpha\text{sen}\theta\text{cos}\phi
\end{align*}
\begin{align*}
B_{\text{1y}}/B_1=-\text{cos}\alpha\text{sen}\epsilon\text{cos}\theta\text{sen}\phi+\text{cos}\alpha\text{cos}\epsilon\text{cos}\phi+\text{sen}\alpha\text{sen}\theta\text{sen}\phi
\end{align*}
\begin{align}
B_{\text{1z}}/B_1=\text{cos}\alpha\text{sen}\epsilon\text{sen}\theta+\text{sen}\alpha\text{cos}\theta
\label{B1comp}
\end{align}

All in all, the expanded effective Hamiltonian we were looking for is:

\begin{align*}
\hat{H}_{\text{eff}}=\hat{H}_0+\hat{H}_1=\frac{1}{2}\left(u_+-u_-\right)\left(\ket{u_+}\bra{u_+}-\ket{u_-}\bra{u_-}\right)+
\end{align*}
\begin{align}
+\frac{\mu_{\text{B}}g_{\text{I}}}{\hbar}\sum_{\gamma=x,y,z}\left(\bra{u_-}\hat{J}_{\gamma}\ket{u_+}\ket{u_+}\bra{u_-}+\bra{u_+}\hat{J}_{\gamma}\ket{u_-}\ket{u_-}\bra{u_+}\right)B_{1\gamma}\text{cos}(\omega_{\text{MW}}(t-t_0))
\label{expHeff}
\end{align}

As we will see in a section below devoted to examples, the rotation axis is contained in the equatorial plane of the Bloch sphere and its direction can be controlled by changing the value of $\epsilon$ (see also main text).

\subsubsection{Master equation resolution: change of picture, rotation, and free evolution}

The expanded $\hat{H}_{\text{eff}}$ is time-dependent and this fact hampers the resolution of the master equation Eq.\ref{rwmaseq} written in the Schrödinger picture. The step to undertake in this situation is to devise a clever change of picture that allows eliminating the mentioned time-dependency. A change of picture consists in a unitary operator $\hat{\mathcal{U}}$ that transforms $\hat{\rho}$ into $\overline{\hat{\rho}}=\hat{\mathcal{U}}\hat{\rho}\hat{\mathcal{U}}^{\dagger}$ (i.e. $\hat{\rho}=\hat{\mathcal{U}}^{\dagger}\overline{\hat{\rho}}\hat{\mathcal{U}}$). In our case, the right choice is:

\begin{equation}
\hat{\mathcal{U}}=\text{exp}\left(i(t-t_0)\frac{\omega_{\text{MW}}}{2}\right)\ket{u_+}\bra{u_+}+\text{exp}\left(-i(t-t_0)\frac{\omega_{\text{MW}}}{2}\right)\ket{u_-}\bra{u_-}
\label{chapic}
\end{equation}

By using $\dot{\hat{\rho}}=\dot{\hat{\mathcal{U}}}^{\dagger}\overline{\hat{\rho}}\hat{\mathcal{U}}+\hat{\mathcal{U}}^{\dagger}\dot{\overline{\hat{\rho}}}\hat{\mathcal{U}}+\hat{\mathcal{U}}^{\dagger}\overline{\hat{\rho}}\dot{\hat{\mathcal{U}}}$, we now replace $\hat{\rho}$ by $\hat{\mathcal{U}}^{\dagger}\overline{\hat{\rho}}\hat{\mathcal{U}}$ in Eq.\ref{rwmaseq} and then multiply by $\hat{\mathcal{U}}$ on the left and by $\hat{\mathcal{U}}^{\dagger}$ on the right. Since $\hat{\mathcal{U}}\hat{\mathcal{U}}^{\dagger}=\hat{\mathcal{U}}^{\dagger}\hat{\mathcal{U}}=\hat{I}$ being $\hat{I}$ the identity operator, we obtain (to alleviate notation, we avoid using the symbols $\hat{}$ and $\tilde{}$ from now on):

\begin{equation}
\dot{\overline{\rho}}=\frac{1}{i\hbar}\left(\mathcal{U}H_{\text{eff}}\mathcal{U}^{\dagger}\overline{\rho}-\overline{\rho}\mathcal{U}H_{\text{eff}}\mathcal{U}^{\dagger}\right)-\mathcal{U}\dot{\mathcal{U}}^{\dagger}\overline{\rho}-\overline{\rho}\dot{\mathcal{U}}\mathcal{U}^{\dagger}+
\label{mechpiu}
\end{equation}
\begin{align}
+\Gamma_{\text{ab}}\left(\mathcal{U}L\mathcal{U}^{\dagger}\overline{\rho}\mathcal{U}L^{\dagger}\mathcal{U}^{\dagger}-\frac{1}{2}\mathcal{U}\{L^{\dagger}L,\mathcal{U}^{\dagger}\overline{\rho}\mathcal{U}\}\mathcal{U}^{\dagger}\right)+
\label{mechpirab}
\end{align}
\begin{align}
+\Gamma_{\text{em}}\left(\mathcal{U}L^{\dagger}\mathcal{U}^{\dagger}\overline{\rho}\mathcal{U}L\mathcal{U}^{\dagger}-\frac{1}{2}\mathcal{U}\{LL^{\dagger},\mathcal{U}^{\dagger}\overline{\rho}\mathcal{U}\}\mathcal{U}^{\dagger}\right)-
\label{mechpirem}
\end{align}
\begin{align}
-\frac{\Gamma_{\text{mag}}}{2\hbar^2}\sum_{\alpha=x,y,z}\mathcal{U}[S_{\alpha},[S_{\alpha},\mathcal{U}^{\dagger}\overline{\rho}\mathcal{U}]]\mathcal{U}^{\dagger}
\label{mechpirmag}
\end{align}

We will now proceed to work out the four lines on the right side of the above equality.
\bigbreak
-First line. Since $I$ is constant and $\mathcal{U}\mathcal{U}^{\dagger}=I$, the time derivative of the previous equality gives $\mathcal{U}\dot{\mathcal{U}}^{\dagger}=-\dot{\mathcal{U}}\mathcal{U}^{\dagger}$ which is now employed below:

\begin{equation*}
\frac{1}{i\hbar}\left(\mathcal{U}H_{\text{eff}}\mathcal{U}^{\dagger}\overline{\rho}-\overline{\rho}\mathcal{U}H_{\text{eff}}\mathcal{U}^{\dagger}\right)-\mathcal{U}\dot{\mathcal{U}}^{\dagger}\overline{\rho}-\overline{\rho}\dot{\mathcal{U}}\mathcal{U}^{\dagger}=
\end{equation*}
\begin{equation*}
=\frac{1}{i\hbar}\left(\mathcal{U}H_{\text{eff}}\mathcal{U}^{\dagger}\overline{\rho}-\overline{\rho}\mathcal{U}H_{\text{eff}}\mathcal{U}^{\dagger}-i\hbar\mathcal{U}\dot{\mathcal{U}}^{\dagger}\overline{\rho}-i\hbar\overline{\rho}\dot{\mathcal{U}}\mathcal{U}^{\dagger}\right)=
\end{equation*}
\begin{equation*}
=\frac{1}{i\hbar}\left(\left(\mathcal{U}H_{\text{eff}}\mathcal{U}^{\dagger}-i\hbar\mathcal{U}\dot{\mathcal{U}}^{\dagger}\right)\overline{\rho}-\overline{\rho}\left(\mathcal{U}H_{\text{eff}}\mathcal{U}^{\dagger}+i\hbar\dot{\mathcal{U}}\mathcal{U}^{\dagger}\right)\right)=
\end{equation*}
\begin{equation*}
=\frac{1}{i\hbar}\left(\left(\mathcal{U}H_{\text{eff}}\mathcal{U}^{\dagger}+i\hbar\dot{\mathcal{U}}\mathcal{U}^{\dagger}\right)\overline{\rho}-\overline{\rho}\left(\mathcal{U}H_{\text{eff}}\mathcal{U}^{\dagger}+i\hbar\dot{\mathcal{U}}\mathcal{U}^{\dagger}\right)\right)=
\end{equation*}
\begin{equation*}
=\frac{1}{i\hbar}[\mathcal{U}H_{\text{eff}}\mathcal{U}^{\dagger}+i\hbar\dot{\mathcal{U}}\mathcal{U}^{\dagger},\overline{\rho}]=\frac{1}{i\hbar}[\overline{H}_{\text{eff}},\overline{\rho}]
\end{equation*}

The qubit effective Hamiltonian $\overline{H}_{\text{eff}}$ in the new picture is:

\begin{equation}
\overline{H}_{\text{eff}}=\mathcal{U}H_{\text{eff}}\mathcal{U}^{\dagger}+i\hbar\dot{\mathcal{U}}\mathcal{U}^{\dagger}
\label{qbeffHnp}
\end{equation}

Let us compute $\overline{H}_{\text{eff}}$ :

\begin{equation*}
\overline{H}_{\text{eff}}=\mathcal{U}H_{\text{eff}}\mathcal{U}^{\dagger}+i\hbar\dot{\mathcal{U}}\mathcal{U}^{\dagger}=\mathcal{U}H_0\mathcal{U}^{\dagger}+\mathcal{U}H_1\mathcal{U}^{\dagger}+i\hbar\dot{\mathcal{U}}\mathcal{U}^{\dagger}=
\end{equation*}
\begin{equation*}
=\mathcal{U}H_0\mathcal{U}^{\dagger}+\mathcal{U}H_1\mathcal{U}^{\dagger}+
\end{equation*}
\begin{equation*}
+i\hbar\left(i\frac{\omega_{\text{MW}}}{2}\text{exp}\left(i(t-t_0)\frac{\omega_{\text{MW}}}{2}\right)\ket{u_+}\bra{u_+}-i\frac{\omega_{\text{MW}}}{2}\text{exp}\left(-i(t-t_0)\frac{\omega_{\text{MW}}}{2}\right)\ket{u_-}\bra{u_-}\right)\cdot
\end{equation*}
\begin{equation*}
\cdot\left(\text{exp}\left(-i(t-t_0)\frac{\omega_{\text{MW}}}{2}\right)\ket{u_+}\bra{u_+}+\text{exp}\left(i(t-t_0)\frac{\omega_{\text{MW}}}{2}\right)\ket{u_-}\bra{u_-}\right)=
\end{equation*}
\begin{equation*}
=\mathcal{U}H_0\mathcal{U}^{\dagger}+\mathcal{U}H_1\mathcal{U}^{\dagger}-\frac{\hbar\omega_{\text{MW}}}{2}\left(\ket{u_+}\bra{u_+}-\ket{u_-}\bra{u_-}\right)=
\end{equation*}
\begin{equation*}
=\left(\text{exp}\left(i(t-t_0)\frac{\omega_{\text{MW}}}{2}\right)\ket{u_+}\bra{u_+}+\text{exp}\left(-i(t-t_0)\frac{\omega_{\text{MW}}}{2}\right)\ket{u_-}\bra{u_-}\right)\frac{\hbar\omega_{+-}}{2}\left(\ket{u_+}\bra{u_+}-\ket{u_-}\bra{u_-}\right)\mathcal{U}^{\dagger}+
\end{equation*}
\begin{equation*}
+\mathcal{U}H_1\mathcal{U}^{\dagger}-\frac{\hbar\omega_{\text{MW}}}{2}\left(\ket{u_+}\bra{u_+}-\ket{u_-}\bra{u_-}\right)=
\end{equation*}
\begin{equation*}
=\frac{\hbar\omega_{+-}}{2}\left(\text{exp}\left(i(t-t_0)\frac{\omega_{\text{MW}}}{2}\right)\ket{u_+}\bra{u_+}-\text{exp}\left(-i(t-t_0)\frac{\omega_{\text{MW}}}{2}\right)\ket{u_-}\bra{u_-}\right)\cdot
\end{equation*}
\begin{equation*}
\cdot\left(\text{exp}\left(-i(t-t_0)\frac{\omega_{\text{MW}}}{2}\right)\ket{u_+}\bra{u_+}+\text{exp}\left(i(t-t_0)\frac{\omega_{\text{MW}}}{2}\right)\ket{u_-}\bra{u_-}\right)+
\end{equation*}
\begin{equation*}
+\mathcal{U}H_1\mathcal{U}^{\dagger}-\frac{\hbar\omega_{\text{MW}}}{2}\left(\ket{u_+}\bra{u_+}-\ket{u_-}\bra{u_-}\right)=
\end{equation*}
\begin{equation*}
=\frac{\hbar\omega_{+-}}{2}\left(\ket{u_+}\bra{u_+}-\ket{u_-}\bra{u_-}\right)+\mathcal{U}H_1\mathcal{U}^{\dagger}-\frac{\hbar\omega_{\text{MW}}}{2}\left(\ket{u_+}\bra{u_+}-\ket{u_-}\bra{u_-}\right)=
\end{equation*}
\begin{equation*}
=\frac{\hbar\delta}{2}\left(\ket{u_+}\bra{u_+}-\ket{u_-}\bra{u_-}\right)+\mathcal{U}H_1\mathcal{U}^{\dagger}
\end{equation*}

where we have defined the detuning $\delta=\omega_{+-}-\omega_{\text{MW}}$. Let us now proceed with $\mathcal{U}H_1\mathcal{U}^{\dagger}$: 

\begin{equation*}
\mathcal{U}H_1\mathcal{U}^{\dagger}=\left(\text{exp}\left(i(t-t_0)\frac{\omega_{\text{MW}}}{2}\right)\ket{u_+}\bra{u_+}+\text{exp}\left(-i(t-t_0)\frac{\omega_{\text{MW}}}{2}\right)\ket{u_-}\bra{u_-}\right)\cdot
\end{equation*}
\begin{equation*}
\cdot\left(\hbar\Omega_{\text{R}}^*\ket{u_+}\bra{u_-}+\hbar\Omega_{\text{R}}\ket{u_-}\bra{u_+}\right)\text{cos}(\omega_{\text{MW}}(t-t_0))\cdot
\end{equation*}
\begin{equation*}
\cdot\left(\text{exp}\left(-i(t-t_0)\frac{\omega_{\text{MW}}}{2}\right)\ket{u_+}\bra{u_+}+\text{exp}\left(i(t-t_0)\frac{\omega_{\text{MW}}}{2}\right)\ket{u_-}\bra{u_-}\right)=
\end{equation*}
\begin{equation*}
=\left(\hbar\Omega_{\text{R}}^*\text{exp}\left(i(t-t_0)\frac{\omega_{\text{MW}}}{2}\right)\ket{u_+}\bra{u_-}+\hbar\Omega_{\text{R}}\text{exp}\left(-i(t-t_0)\frac{\omega_{\text{MW}}}{2}\right)\ket{u_-}\bra{u_+}\right)\cdot
\end{equation*}
\begin{equation*}
\cdot\left(\text{exp}\left(-i(t-t_0)\frac{\omega_{\text{MW}}}{2}\right)\ket{u_+}\bra{u_+}+\text{exp}\left(i(t-t_0)\frac{\omega_{\text{MW}}}{2}\right)\ket{u_-}\bra{u_-}\right)\cdot\text{cos}(\omega_{\text{MW}}(t-t_0))=
\end{equation*}
\begin{equation*}
=\left(\hbar\Omega_{\text{R}}^*\text{exp}\left(i(t-t_0)\omega_{\text{MW}}\right)\ket{u_+}\bra{u_-}+\hbar\Omega_{\text{R}}\text{exp}\left(-i(t-t_0)\omega_{\text{MW}}\right)\ket{u_-}\bra{u_+}\right)\cdot
\end{equation*}
\begin{equation*}
\cdot\frac{\text{exp}\left(i(t-t_0)\omega_{\text{MW}}\right)+\text{exp}\left(-i(t-t_0)\omega_{\text{MW}}\right)}{2}
\end{equation*}

We now make use of the so-called rotating wave approximation (RWA) which leads to remove the fast-rotating terms, namely all terms that are proportional to $\text{exp}\left(\pm2i(t-t_0)\omega_{\text{MW}}\right)$. Thus, the term $\mathcal{U}H_1\mathcal{U}^{\dagger}$ is approximated as:

\begin{equation}
\mathcal{U}H_1\mathcal{U}^{\dagger}\approx \frac{\hbar}{2}\left(\Omega_{\text{R}}\ket{u_-}\bra{u_+}+\Omega_{\text{R}}^*\ket{u_+}\bra{u_-}\right)
\label{rwah1}
\end{equation}

All in all, the effective Hamiltonian $\overline{H}_{\text{eff}}$ -with no time-dependency- is approximated as:

\begin{equation}
\overline{H}_{\text{eff}}\approx\frac{\hbar\delta}{2}\left(\ket{u_+}\bra{u_+}-\ket{u_-}\bra{u_-}\right)+\frac{\hbar}{2}\left(\Omega_{\text{R}}\ket{u_-}\bra{u_+}+\Omega_{\text{R}}^*\ket{u_+}\bra{u_-}\right)=\frac{\hbar}{2}\mqty(\delta & \Omega_{\text{R}}^* \\ \Omega_{\text{R}} & -\delta)
\label{rwaheff}
\end{equation}

Let us write the matrix representation of $\overline{\rho}$ in the ordered basis set $\{\ket{u_+},\ket{u_-}\}$ as: 

\begin{equation}
\overline{\rho}=\mqty(\overline{\rho}_{11} & \overline{\rho}_{12} \\ \overline{\rho}_{12}^* & \overline{\rho}_{22})=\overline{\rho}_{11}\ket{u_+}\bra{u_+}+\overline{\rho}_{12}\ket{u_+}\bra{u_-}+\overline{\rho}_{12}^*\ket{u_-}\bra{u_+}+\overline{\rho}_{22}\ket{u_-}\bra{u_-}
\label{mrdenmatnp}
\end{equation}

where $\overline{\rho}_{11}+\overline{\rho}_{22}=1$, $\overline{\rho}_{11},\overline{\rho}_{22}\in\mathbb{R}$ and $\overline{\rho}_{12}=\overline{\rho}_{12,r}+i\overline{\rho}_{12,i}\in\mathbb{C}$, $\overline{\rho}_{12,r},\overline{\rho}_{12,i}\in\mathbb{R}$. Hence, $0\leq\overline{\rho}_{11}\leq 1$ is the spin population of $\ket{u_+}$, and $0\leq\overline{\rho}_{22}\leq 1$ represents the spin population of $\ket{u_-}$. The matrix representation of the first line \ref{mechpiu} after RWA is:

\begin{equation}
\frac{1}{i\hbar}[\overline{H}_{\text{eff}},\overline{\rho}]=\frac{1}{i\hbar}\left(\overline{H}_{\text{eff}}\overline{\rho}-\overline{\rho}\overline{H}_{\text{eff}}\right)\approx-\frac{1}{2}\mqty(2\text{Im}(\Omega_{\text{R}}\overline{\rho}_{12}) & -i(\Omega_{\text{R}}^*(\overline{\rho}_{11}-\overline{\rho}_{22})-2\delta\overline{\rho}_{12}) \\ i(\Omega_{\text{R}}(\overline{\rho}_{11}-\overline{\rho}_{22})-2\delta\overline{\rho}_{12}^*) & -2\text{Im}(\Omega_{\text{R}}\overline{\rho}_{12}))
\label{firstlrwa}
\end{equation}

being $\text{Im}(\cdot)$ the imaginary part. Before proceeding with the second and third lines, we will need the following easy-to-compute expressions:

\begin{equation*}
\mathcal{U}L^{\dagger}=\text{exp}\left(-i(t-t_0)\frac{\omega_{\text{MW}}}{2}\right)\ket{u_-}\bra{u_+} \qquad \mathcal{U}L=\text{exp}\left(i(t-t_0)\frac{\omega_{\text{MW}}}{2}\right)\ket{u_+}\bra{u_-}
\end{equation*}
\begin{equation}
\left(\mathcal{U}L^{\dagger}\right)^{\dagger}=\text{exp}\left(i(t-t_0)\frac{\omega_{\text{MW}}}{2}\right)\ket{u_+}\bra{u_-} \qquad \left(\mathcal{U}L\right)^{\dagger}=\text{exp}\left(-i(t-t_0)\frac{\omega_{\text{MW}}}{2}\right)\ket{u_-}\bra{u_+}
\label{etce1}
\end{equation}

Let us also note that $\mathcal{U}L^{\dagger}=\left(\mathcal{U}L\right)^{\dagger}$ and $\mathcal{U}L=\left(\mathcal{U}L^{\dagger}\right)^{\dagger}$.

\bigbreak
-Second line.

\begin{equation*}
\Gamma_{\text{ab}}\left(\mathcal{U}L\mathcal{U}^{\dagger}\overline{\rho}\mathcal{U}L^{\dagger}\mathcal{U}^{\dagger}-\frac{1}{2}\mathcal{U}\{L^{\dagger}L,\mathcal{U}^{\dagger}\overline{\rho}\mathcal{U}\}\mathcal{U}^{\dagger}\right)=
\end{equation*}
\begin{equation*}
=\Gamma_{\text{ab}}\left(\mathcal{U}(\mathcal{U}L^{\dagger})^{\dagger}\overline{\rho}(\mathcal{U}L^{\dagger})\mathcal{U}^{\dagger}-\frac{1}{2}\mathcal{U}\left(L^{\dagger}L\mathcal{U}^{\dagger}\overline{\rho}\mathcal{U}+\mathcal{U}^{\dagger}\overline{\rho}\mathcal{U}L^{\dagger}L\right)\mathcal{U}^{\dagger}\right)=
\end{equation*}
\begin{equation*}
=\Gamma_{\text{ab}}\left(\mathcal{U}(\mathcal{U}L^{\dagger})^{\dagger}\overline{\rho}(\mathcal{U}L^{\dagger})\mathcal{U}^{\dagger}-\frac{1}{2}\left(\mathcal{U}L^{\dagger}L\mathcal{U}^{\dagger}\overline{\rho}+\overline{\rho}\mathcal{U}L^{\dagger}L\mathcal{U}^{\dagger}\right)\right)=
\end{equation*}
\begin{equation*}
=\Gamma_{\text{ab}}\left(\mathcal{U}(\mathcal{U}L^{\dagger})^{\dagger}\overline{\rho}(\mathcal{U}L^{\dagger})\mathcal{U}^{\dagger}-\frac{1}{2}\left(\mathcal{U}L^{\dagger}(\mathcal{U}L^{\dagger})^{\dagger}\overline{\rho}+\overline{\rho}\mathcal{U}L^{\dagger}(\mathcal{U}L^{\dagger})^{\dagger}\right)\right)=*
\end{equation*}

Now, we replace $\mathcal{U}L^{\dagger}$ by $\left(\mathcal{U}L\right)^{\dagger}$ and $\left(\mathcal{U}L^{\dagger}\right)^{\dagger}$ by $\mathcal{U}L$, and use the fact $\mathcal{U}^{\dagger}\mathcal{U}=I$:

\begin{equation*}
*=\Gamma_{\text{ab}}\left(\mathcal{U}(\mathcal{U}L)\overline{\rho}(\mathcal{U}L)^{\dagger}\mathcal{U}^{\dagger}-\frac{1}{2}\left((\mathcal{U}L)^{\dagger}\mathcal{U}L\overline{\rho}+\overline{\rho}(\mathcal{U}L)^{\dagger}\mathcal{U}L\right)\right)=
\end{equation*}
\begin{equation*}
=\Gamma_{\text{ab}}\left(\mathcal{U}(\mathcal{U}L)\overline{\rho}(\mathcal{U}L)^{\dagger}\mathcal{U}^{\dagger}-\frac{1}{2}\left(L^{\dagger}\mathcal{U}^{\dagger}\mathcal{U}L\overline{\rho}+\overline{\rho}L^{\dagger}\mathcal{U}^{\dagger}\mathcal{U}L\right)\right)=
\end{equation*}
\begin{equation*}
=\Gamma_{\text{ab}}\left(\mathcal{U}(\mathcal{U}L)\overline{\rho}(\mathcal{U}L)^{\dagger}\mathcal{U}^{\dagger}-\frac{1}{2}\left(L^{\dagger}L\overline{\rho}+\overline{\rho}L^{\dagger}L\right)\right)=
\end{equation*}
\begin{equation*}
=\Gamma_{\text{ab}}\left(\mathcal{U}(\mathcal{U}L)\overline{\rho}(\mathcal{U}(\mathcal{U}L))^{\dagger}-\frac{1}{2}\left(L^{\dagger}L\overline{\rho}+\overline{\rho}L^{\dagger}L\right)\right)=*
\end{equation*}

By employing the expressions of $\mathcal{U}$, $\mathcal{U}L$, and $\overline{\rho}$, one easily finds:

\begin{equation}
\mathcal{U}(\mathcal{U}L)\overline{\rho}(\mathcal{U}(\mathcal{U}L))^{\dagger}=\overline{\rho}_{22}\ket{u_+}\bra{u_+}
\label{absimp}
\end{equation}

Hence, by using now the expressions of $L^{\dagger}$ and $L$:

\begin{equation*}
*=\Gamma_{\text{ab}}\left(\overline{\rho}_{22}\ket{u_+}\bra{u_+}-\frac{1}{2}\left(\ket{u_-}\bra{u_-}\overline{\rho}+\overline{\rho}\ket{u_-}\bra{u_-}\right)\right)=
\end{equation*}
\begin{equation*}
=\Gamma_{\text{ab}}\left(\overline{\rho}_{22}\ket{u_+}\bra{u_+}-\frac{1}{2}\left(\overline{\rho}_{12}^*\ket{u_-}\bra{u_+}+\overline{\rho}_{22}\ket{u_-}\bra{u_-}+\overline{\rho}_{12}\ket{u_+}\bra{u_-}+\overline{\rho}_{22}\ket{u_-}\bra{u_-}\right)\right)=
\end{equation*}
\begin{equation*}
=\Gamma_{\text{ab}}\left(\overline{\rho}_{22}\ket{u_+}\bra{u_+}-\frac{1}{2}\overline{\rho}_{12}\ket{u_+}\bra{u_-}-\frac{1}{2}\overline{\rho}_{12}^*\ket{u_-}\bra{u_+}-\overline{\rho}_{22}\ket{u_-}\bra{u_-}\right)=
\end{equation*}
\begin{equation*}
=\Gamma_{\text{ab}}\mqty(\overline{\rho}_{22} & -\overline{\rho}_{12}/2 \\ -\overline{\rho}_{12}^*/2 & -\overline{\rho}_{22})
\end{equation*}

All in all, the matrix representation of the second line \ref{mechpirab} is:

\begin{equation}
\Gamma_{\text{ab}}\left(\mathcal{U}L\mathcal{U}^{\dagger}\overline{\rho}\mathcal{U}L^{\dagger}\mathcal{U}^{\dagger}-\frac{1}{2}\mathcal{U}\{L^{\dagger}L,\mathcal{U}^{\dagger}\overline{\rho}\mathcal{U}\}\mathcal{U}^{\dagger}\right)=\Gamma_{\text{ab}}\mqty(\overline{\rho}_{22} & -\overline{\rho}_{12}/2 \\ -\overline{\rho}_{12}^*/2 & -\overline{\rho}_{22})
\label{secondl}
\end{equation}

\bigbreak
-Third line.

\begin{equation*}
\Gamma_{\text{em}}\left(\mathcal{U}L^{\dagger}\mathcal{U}^{\dagger}\overline{\rho}\mathcal{U}L\mathcal{U}^{\dagger}-\frac{1}{2}\mathcal{U}\{LL^{\dagger},\mathcal{U}^{\dagger}\overline{\rho}\mathcal{U}\}\mathcal{U}^{\dagger}\right)=
\end{equation*}
\begin{equation*}
=\Gamma_{\text{em}}\left(\mathcal{U}(\mathcal{U}L)^{\dagger}\overline{\rho}(\mathcal{U}L)\mathcal{U}^{\dagger}-\frac{1}{2}\mathcal{U}\left(LL^{\dagger}\mathcal{U}^{\dagger}\overline{\rho}\mathcal{U}+\mathcal{U}^{\dagger}\overline{\rho}\mathcal{U}LL^{\dagger}\right)\mathcal{U}^{\dagger}\right)=
\end{equation*}
\begin{equation*}
=\Gamma_{\text{em}}\left(\mathcal{U}(\mathcal{U}L)^{\dagger}\overline{\rho}(\mathcal{U}L)\mathcal{U}^{\dagger}-\frac{1}{2}\left(\mathcal{U}LL^{\dagger}\mathcal{U}^{\dagger}\overline{\rho}+\overline{\rho}\mathcal{U}LL^{\dagger}\mathcal{U}^{\dagger}\right)\right)=
\end{equation*}
\begin{equation*}
=\Gamma_{\text{em}}\left(\mathcal{U}(\mathcal{U}L)^{\dagger}\overline{\rho}(\mathcal{U}L)\mathcal{U}^{\dagger}-\frac{1}{2}\left(\mathcal{U}L(\mathcal{U}L)^{\dagger}\overline{\rho}+\overline{\rho}\mathcal{U}L(\mathcal{U}L)^{\dagger}\right)\right)=*
\end{equation*}

Now, we replace $\left(\mathcal{U}L\right)^{\dagger}$ by $\mathcal{U}L^{\dagger}$ and $\mathcal{U}L$ by $\left(\mathcal{U}L^{\dagger}\right)^{\dagger}$, and use again the fact $\mathcal{U}^{\dagger}\mathcal{U}=I$:

\begin{equation*}
*=\Gamma_{\text{em}}\left(\mathcal{U}(\mathcal{U}L^{\dagger})\overline{\rho}(\mathcal{U}L^{\dagger})^{\dagger}\mathcal{U}^{\dagger}-\frac{1}{2}\left((\mathcal{U}L^{\dagger})^{\dagger}\mathcal{U}L^{\dagger}\overline{\rho}+\overline{\rho}(\mathcal{U}L^{\dagger})^{\dagger}\mathcal{U}L^{\dagger}\right)\right)=
\end{equation*}
\begin{equation*}
=\Gamma_{\text{em}}\left(\mathcal{U}(\mathcal{U}L^{\dagger})\overline{\rho}(\mathcal{U}L^{\dagger})^{\dagger}\mathcal{U}^{\dagger}-\frac{1}{2}\left(L\mathcal{U}^{\dagger}\mathcal{U}L^{\dagger}\overline{\rho}+\overline{\rho}L\mathcal{U}^{\dagger}\mathcal{U}L^{\dagger}\right)\right)=
\end{equation*}
\begin{equation*}
=\Gamma_{\text{em}}\left(\mathcal{U}(\mathcal{U}L^{\dagger})\overline{\rho}(\mathcal{U}L^{\dagger})^{\dagger}\mathcal{U}^{\dagger}-\frac{1}{2}\left(LL^{\dagger}\overline{\rho}+\overline{\rho}LL^{\dagger}\right)\right)=
\end{equation*}
\begin{equation*}
=\Gamma_{\text{em}}\left(\mathcal{U}(\mathcal{U}L^{\dagger})\overline{\rho}(\mathcal{U}(\mathcal{U}L^{\dagger}))^{\dagger}-\frac{1}{2}\left(LL^{\dagger}\overline{\rho}+\overline{\rho}LL^{\dagger}\right)\right)=*
\end{equation*}

By employing the expressions of $\mathcal{U}$, $\mathcal{U}L^{\dagger}$, and $\overline{\rho}$, we have now:

\begin{equation}
\mathcal{U}(\mathcal{U}L^{\dagger})\overline{\rho}(\mathcal{U}(\mathcal{U}L^{\dagger}))^{\dagger}=\overline{\rho}_{11}\ket{u_-}\bra{u_-}
\label{emsimp}
\end{equation}

Hence, by using again the expressions of $L^{\dagger}$ and $L$:

\begin{equation*}
*=\Gamma_{\text{em}}\left(\overline{\rho}_{11}\ket{u_-}\bra{u_-}-\frac{1}{2}\left(\ket{u_+}\bra{u_+}\overline{\rho}+\overline{\rho}\ket{u_+}\bra{u_+}\right)\right)=
\end{equation*}
\begin{equation*}
=\Gamma_{\text{em}}\left(\overline{\rho}_{11}\ket{u_-}\bra{u_-}-\frac{1}{2}\left(\overline{\rho}_{11}\ket{u_+}\bra{u_+}+\overline{\rho}_{12}\ket{u_+}\bra{u_-}+\overline{\rho}_{11}\ket{u_+}\bra{u_+}+\overline{\rho}_{12}^*\ket{u_-}\bra{u_+}\right)\right)=
\end{equation*}
\begin{equation*}
=\Gamma_{\text{em}}\left(-\overline{\rho}_{11}\ket{u_+}\bra{u_+}-\frac{1}{2}\overline{\rho}_{12}\ket{u_+}\bra{u_-}-\frac{1}{2}\overline{\rho}_{12}^*\ket{u_-}\bra{u_+}+\overline{\rho}_{11}\ket{u_-}\bra{u_-}\right)=
\end{equation*}
\begin{equation*}
=\Gamma_{\text{em}}\mqty(-\overline{\rho}_{11} & -\overline{\rho}_{12}/2 \\ -\overline{\rho}_{12}^*/2 & \overline{\rho}_{11})
\end{equation*}

All in all, the matrix representation of the third line \ref{mechpirem} is:

\begin{equation}
\Gamma_{\text{em}}\left(\mathcal{U}L^{\dagger}\mathcal{U}^{\dagger}\overline{\rho}\mathcal{U}L\mathcal{U}^{\dagger}-\frac{1}{2}\mathcal{U}\{LL^{\dagger},\mathcal{U}^{\dagger}\overline{\rho}\mathcal{U}\}\mathcal{U}^{\dagger}\right)=\Gamma_{\text{em}}\mqty(-\overline{\rho}_{11} & -\overline{\rho}_{12}/2 \\ -\overline{\rho}_{12}^*/2 & \overline{\rho}_{11})
\label{thirdl}
\end{equation}

\bigbreak
-Fourth line.

\begin{equation*}
-\frac{\Gamma_{\text{mag}}}{2\hbar^2}\sum_{\alpha=x,y,z}\mathcal{U}[S_{\alpha},[S_{\alpha},\mathcal{U}^{\dagger}\overline{\rho}\mathcal{U}]]\mathcal{U}^{\dagger}=
\end{equation*}
\begin{equation*}
=-\frac{1}{8}\Gamma_{\text{mag}}\sum_{\alpha=x,y,z}\mathcal{U}[\sigma_{\alpha},[\sigma_{\alpha},\mathcal{U}^{\dagger}\overline{\rho}\mathcal{U}]]\mathcal{U}^{\dagger}
\end{equation*}

Let us recall the facts $\sigma_{\alpha}^2=I$ and $\mathcal{U}\mathcal{U}^{\dagger}=I$. Straightforwardly, one gets:

\begin{equation}
\mathcal{U}[\sigma_{\alpha},[\sigma_{\alpha},\mathcal{U}^{\dagger}\overline{\rho}\mathcal{U}]]\mathcal{U}^{\dagger}=2\left(\overline{\rho}-\mathcal{U}\sigma_{\alpha}\mathcal{U}^{\dagger}\overline{\rho}\mathcal{U}\sigma_{\alpha}\mathcal{U}^{\dagger}\right)
\label{termthirdl}
\end{equation}

By using the expressions of $\mathcal{U}$ and $\sigma_{\alpha}$, it is easy to obtain:

\begin{equation}
\mathcal{U}\sigma_x\mathcal{U}^{\dagger}=\text{exp}\left(i(t-t_0)\omega_{\text{MW}}\right)\ket{u_+}\bra{u_-}+\text{exp}\left(-i(t-t_0)\omega_{\text{MW}}\right)\ket{u_-}\bra{u_+}
\label{termx}
\end{equation}
\begin{equation}
\mathcal{U}\sigma_y\mathcal{U}^{\dagger}=i\text{exp}\left(-i(t-t_0)\omega_{\text{MW}}\right)\ket{u_-}\bra{u_+}-i\text{exp}\left(i(t-t_0)\omega_{\text{MW}}\right)\ket{u_+}\bra{u_-}
\label{termy}
\end{equation}
\begin{equation}
\mathcal{U}\sigma_z\mathcal{U}^{\dagger}=\ket{u_+}\bra{u_+}-\ket{u_-}\bra{u_-}
\label{termz}
\end{equation}

We now work out the term $\alpha=x$ in the sum above:

\begin{equation*}
\mathcal{U}[\sigma_x,[\sigma_x,\mathcal{U}^{\dagger}\overline{\rho}\mathcal{U}]]\mathcal{U}^{\dagger}=
\end{equation*}
\begin{equation*}
=2(\overline{\rho}-\left(\text{exp}\left(i(t-t_0)\omega_{\text{MW}}\right)\ket{u_+}\bra{u_-}+\text{exp}\left(-i(t-t_0)\omega_{\text{MW}}\right)\ket{u_-}\bra{u_+}\right)\cdot 
\end{equation*}
\begin{equation*}
\cdot\left(\overline{\rho}_{11}\ket{u_+}\bra{u_+}+\overline{\rho}_{12}\ket{u_+}\bra{u_-}+\overline{\rho}_{12}^*\ket{u_-}\bra{u_+}+\overline{\rho}_{22}\ket{u_-}\bra{u_-}\right)\cdot
\end{equation*}
\begin{equation*}
\cdot\left(\text{exp}\left(i(t-t_0)\omega_{\text{MW}}\right)\ket{u_+}\bra{u_-}+\text{exp}\left(-i(t-t_0)\omega_{\text{MW}}\right)\ket{u_-}\bra{u_+}\right))=
\end{equation*}
\begin{equation*}
=2(\overline{\rho}-(\text{exp}\left(i(t-t_0)\omega_{\text{MW}}\right)\overline{\rho}_{12}^*\ket{u_+}\bra{u_+}+\text{exp}\left(i(t-t_0)\omega_{\text{MW}}\right)\overline{\rho}_{22}\ket{u_+}\bra{u_-}+
\end{equation*}
\begin{equation*}
+\text{exp}\left(-i(t-t_0)\omega_{\text{MW}}\right)\overline{\rho}_{11}\ket{u_-}\bra{u_+}+\text{exp}\left(-i(t-t_0)\omega_{\text{MW}}\right)\overline{\rho}_{12}\ket{u_-}\bra{u_-})\cdot
\end{equation*}
\begin{equation*}
\cdot\left(\text{exp}\left(i(t-t_0)\omega_{\text{MW}}\right)\ket{u_+}\bra{u_-}+\text{exp}\left(-i(t-t_0)\omega_{\text{MW}}\right)\ket{u_-}\bra{u_+}\right)\approx
\end{equation*}
\begin{equation*}
\approx 2\left(\overline{\rho}-\overline{\rho}_{11}\ket{u_-}\bra{u_-}-\overline{\rho}_{22}\ket{u_+}\bra{u_+}\right)
\end{equation*}

where, in the last equality, we have eliminated the fast-rotating terms. Proceeding likewise with the terms $\alpha=y$ and $\alpha=z$ (no fast-rotating terms are found when working out the case $\alpha=z$):

\begin{equation*}
\mathcal{U}[\sigma_y,[\sigma_y,\mathcal{U}^{\dagger}\overline{\rho}\mathcal{U}]]\mathcal{U}^{\dagger}\approx 2\left(\overline{\rho}-\overline{\rho}_{22}\ket{u_+}\bra{u_+}-\overline{\rho}_{11}\ket{u_-}\bra{u_-}\right)
\end{equation*}

\begin{equation*}
\mathcal{U}[\sigma_z,[\sigma_z,\mathcal{U}^{\dagger}\overline{\rho}\mathcal{U}]]\mathcal{U}^{\dagger}=2\left(\overline{\rho}-\overline{\rho}_{11}\ket{u_+}\bra{u_+}+\overline{\rho}_{12}^*\ket{u_-}\bra{u_+}+\overline{\rho}_{12}\ket{u_+}\bra{u_-}-\overline{\rho}_{22}\ket{u_-}\bra{u_-}\right)
\end{equation*}

All in all, if the equations derived above are employed in the fourth line \ref{mechpirmag} together with $\overline{\rho}_{11}+\overline{\rho}_{22}=1$, $3\overline{\rho}_{11}-1-\overline{\rho}_{22}=2(\overline{\rho}_{11}-\overline{\rho}_{22})$, $3\overline{\rho}_{22}-1-\overline{\rho}_{11}=2(\overline{\rho}_{22}-\overline{\rho}_{11})$, the matrix representation after RWA is:

\begin{equation}
-\frac{\Gamma_{\text{mag}}}{2\hbar^2}\sum_{\alpha=x,y,z}\mathcal{U}[S_{\alpha},[S_{\alpha},\mathcal{U}^{\dagger}\overline{\rho}\mathcal{U}]]\mathcal{U}^{\dagger}\approx-\frac{\Gamma_{\text{mag}}}{2}\mqty(\overline{\rho}_{11}-\overline{\rho}_{22} & 2\overline{\rho}_{12} \\ 2\overline{\rho}_{12}^* & \overline{\rho}_{22}-\overline{\rho}_{11})
\label{fourthl}
\end{equation}

Let us now write the matrix representation of the master equation in the ordered basis set $\{\ket{u_+},\ket{u_-}\}$ after RWA:

\begin{equation*}
\mqty(\dot{\overline{\rho}}_{11} & \dot{\overline{\rho}}_{12} \\ \dot{\overline{\rho}}_{12}^* & \dot{\overline{\rho}}_{22})=-\frac{1}{2}\mqty(2\text{Im}(\Omega_{\text{R}}\overline{\rho}_{12}) & -i(\Omega_{\text{R}}^*(\overline{\rho}_{11}-\overline{\rho}_{22})-2\delta\overline{\rho}_{12}) \\ i(\Omega_{\text{R}}(\overline{\rho}_{11}-\overline{\rho}_{22})-2\delta\overline{\rho}_{12}^*) & -2\text{Im}(\Omega_{\text{R}}\overline{\rho}_{12}))+
\end{equation*}
\begin{equation}
+\Gamma_{\text{ab}}\mqty(\overline{\rho}_{22} & -\overline{\rho}_{12}/2 \\ -\overline{\rho}_{12}^*/2 & -\overline{\rho}_{22})+\Gamma_{\text{em}}\mqty(-\overline{\rho}_{11} & -\overline{\rho}_{12}/2 \\ -\overline{\rho}_{12}^*/2 & \overline{\rho}_{11})-\frac{\Gamma_{\text{mag}}}{2}\mqty(\overline{\rho}_{11}-\overline{\rho}_{22} & 2\overline{\rho}_{12} \\ 2\overline{\rho}_{12}^* & \overline{\rho}_{22}-\overline{\rho}_{11})
\label{matrepmerwa}
\end{equation}

The above matrix equation provides four scalar equations, namely two equations from the two diagonal elements and two more equations from the real and imaginary parts of the upper off-diagonal elements (note that the lower off-diagonal element does not provide any independent information since it is just the complex-conjugate of the upper one). Let us now write the above-mentioned four equations.
\medbreak
-For $\dot{\overline{\rho}}_{11}$:
\begin{equation*}
\dot{\overline{\rho}}_{11}=-\left(\Gamma_{\text{em}}+\frac{\Gamma_{\text{mag}}}{2}\right)\overline{\rho}_{11}+\left(\Gamma_{\text{ab}}+\frac{\Gamma_{\text{mag}}}{2}\right)\overline{\rho}_{22}-\text{Im}(\Omega_{\text{R}}\overline{\rho}_{12})
\end{equation*}

By recalling the expression of $\Omega_{\text{R}}$, we can write:

\begin{equation*}
\text{Im}(\Omega_{\text{R}}\overline{\rho}_{12})=\text{Q}\overline{\rho}_{12,r}+\text{P}\overline{\rho}_{12,i}
\end{equation*}

where:

\begin{equation*}
\text{Q}=\frac{\mu_{\text{B}}g_{\text{I}}}{\hbar}\sum_{\gamma=x,y,z}B_{1\gamma}q_{\gamma}
\end{equation*}
\begin{equation*}
\text{P}=\frac{\mu_{\text{B}}g_{\text{I}}}{\hbar}\sum_{\gamma=x,y,z}B_{1\gamma}p_{\gamma}
\end{equation*}

and $p_{\gamma}$ and $q_{\gamma}$ are the real and imaginary parts of the following matrix elements:

\begin{equation*}
\bra{u_+}\hat{J}_{\gamma}/\hbar\ket{u_-}=p_{\gamma}+iq_{\gamma} \qquad \gamma=x,y,z
\end{equation*}

Hence, the equation is:

\begin{equation}
\dot{\overline{\rho}}_{11}=\text{A}_{11}\overline{\rho}_{11}+\text{B}_{11}\overline{\rho}_{22}-\text{Q}\overline{\rho}_{12,r}-\text{P}\overline{\rho}_{12,i}
\label{eqde1me}
\end{equation}

with:

\begin{equation}
\text{A}_{11}=-\left(\Gamma_{\text{em}}+\frac{\Gamma_{\text{mag}}}{2}\right) \qquad \text{B}_{11}=\Gamma_{\text{ab}}+\frac{\Gamma_{\text{mag}}}{2}
\label{eqdemepar}
\end{equation}

\bigbreak
Likewise:

\bigbreak
-For $\dot{\overline{\rho}}_{12,r}$:
\begin{equation}
\dot{\overline{\rho}}_{12,r}=\frac{\text{Q}}{2}\overline{\rho}_{11}-\frac{\text{Q}}{2}\overline{\rho}_{22}-\text{D}\overline{\rho}_{12,r}+\delta\overline{\rho}_{12,i}
\label{eqoer}
\end{equation}

\bigbreak
-For $\dot{\overline{\rho}}_{12,i}$:
\begin{equation}
\dot{\overline{\rho}}_{12,i}=\frac{\text{P}}{2}\overline{\rho}_{11}-\frac{\text{P}}{2}\overline{\rho}_{22}-\delta\overline{\rho}_{12,r}-\text{D}\overline{\rho}_{12,i}
\label{eqoei}
\end{equation}

being:

\begin{equation}
\text{D}=\frac{\Gamma_{\text{ab}}+\Gamma_{\text{em}}}{2}+\Gamma_{\text{mag}} \qquad \delta=\omega_{+-}-\omega_{\text{MW}}
\label{eqoemepar}
\end{equation}

\bigbreak
-For $\dot{\overline{\rho}}_{22}$:
\begin{equation}
\dot{\overline{\rho}}_{22}=-\text{A}_{11}\overline{\rho}_{11}-\text{B}_{11}\overline{\rho}_{22}+\text{Q}\overline{\rho}_{12,r}+\text{P}\overline{\rho}_{12,i}
\label{eqde2me}
\end{equation}

All in all, we obtain the following first-order homogeneous linear differential equation system with constant coefficients:

\begin{equation}
\mqty(\dot{\overline{\rho}}_{11} \\ \dot{\overline{\rho}}_{22} \\ \dot{\overline{\rho}}_{12,r} \\ \dot{\overline{\rho}}_{12,i}) = \mqty(\text{A}_{11} & \text{B}_{11} & -\text{Q} & -\text{P} \\
-\text{A}_{11} & -\text{B}_{11} & \text{Q} & \text{P} \\
\text{Q}/2 & -\text{Q}/2 & -\text{D} & \delta \\
\text{P}/2 & -\text{P}/2 & -\delta & -\text{D})\mqty(\overline{\rho}_{11} \\ \overline{\rho}_{22} \\ \overline{\rho}_{12,r} \\ \overline{\rho}_{12,i}) 
\label{diffeqsys}
\end{equation}

Note that the coefficient matrix is singular (determinant=0) since the first and second rows are proportional, which means that one of the eigenvalues -say $\lambda_1$- is $\lambda_1=0$. This fact is crucial as it indicates that there exists a stationary solution at $t\rightarrow{}+\infty$ that corresponds to the attainment of the thermodynamic equilibrium where the qubit state populations are given by the Boltzmann distribution. Note also that, since $\overline{\rho}_{11}+\overline{\rho}_{22}=1$, $\dot{\overline{\rho}}_{11}+\dot{\overline{\rho}}_{22}=0$ $\forall t$ after summing up Eq.\ref{eqde1me} and Eq.\ref{eqde2me}. The solutions are functions of time $t\geq t_0$: $\overline{\rho}_{11}=\overline{\rho}_{11}(t)$, $\overline{\rho}_{22}=\overline{\rho}_{22}(t)$, $\overline{\rho}_{12,r}=\overline{\rho}_{12,r}(t)$, 
$\overline{\rho}_{12,i}=\overline{\rho}_{12,i}(t)$, with initial conditions $\overline{\rho}_{11}^0=\overline{\rho}_{11}(t_0)$, $\overline{\rho}_{22}^0=\overline{\rho}_{22}(t_0)$, $\overline{\rho}_{12,r}^0=\overline{\rho}_{12,r}(t_0)$, $\overline{\rho}_{12,i}^0=\overline{\rho}_{12,i}(t_0)$.
\bigbreak
After diagonalizing the coefficient matrix, one obtains eigenvalues $\lambda_1=0,\lambda_2,\lambda_3,\lambda_4$ and associated eigenvectors $\va{v}_1,\va{v}_2,\va{v}_3,\va{v}_4$. We must exclude the case $\Gamma_{\text{ab}}=\Gamma_{\text{em}}=\Gamma_{\text{mag}}\equiv 0$ so we can suppose that the eigenvectors are all linearly independent if at least one of the three rates is non-zero (if all three rates are zero, the solution is given by the rotation operator, see below). In that case, the solution to Eq.\ref{diffeqsys} is:

\begin{equation}
\va{\overline{\rho}}=\sum_{i=1}^4c_i\text{exp}\left(\lambda_i(t-t_0)\right)\va{v}_i
\label{gensol}
\end{equation}

where the constants $c_i$ are determined from the initial condition $\va{\overline{\rho}}^0=\va{\overline{\rho}}(t_0)$ by solving the linear equation system:

\begin{equation}
\va{\overline{\rho}}^0=[\va{v}_1\quad\va{v}_2\quad\va{v}_3\quad\va{v}_4]\mqty(c_1 \\ c_2 \\ c_3 \\ c_4)
\label{inicond}
\end{equation}

Once the eigenvectors are linearly independent, the column-wise matrix $[\va{v}_1\quad\va{v}_2\quad\va{v}_3\quad\va{v}_4]$ is invertible and the constants $c_i$ have a unique value given by:

\begin{equation}
\mqty(c_1 \\ c_2 \\ c_3 \\ c_4)=\text{inv}[\va{v}_1\quad\va{v}_2\quad\va{v}_3\quad\va{v}_4]\va{\overline{\rho}}^0
\label{inicondsol}
\end{equation}

Note that always $\text{A}_{11}<0$, $\text{B}_{11}>0$, $\text{D}>0$. This fact can be used to prove that one eigenvalue is real and negative -say $\lambda_2<0$- and the two remaining eigenvalues are complex-conjugate with a negative real part: $\lambda_4=\overline{\lambda}_3$, $\text{Re}(\lambda_4)=\text{Re}(\lambda_3)<0$. Hence, $\va{v}_3$ and $\va{v}_4$ are complex-conjugate: $\va{v}_4=\overline{\va{v}}_3$. Let us decompose $\va{v}_3$ into real and imaginary parts: $\va{v}_3=\va{v}_{3,r}+i\va{v}_{3,i}$. It can also be proven that $c_1,c_2\in\mathbb{R}$ and $c_4=\overline{c}_3\in\mathbb{C}$, being $c_3$ and $c_4$ complex-conjugate and $c_3=c_{3,r}+ic_{3,i}$ with $c_{3,r}$, $c_{3,i}$ real and imaginary parts of $c_3$. 
\bigbreak
With these considerations, the solution $\va{\overline{\rho}}$ is:

\begin{equation*}
\va{\overline{\rho}}=c_1\va{v}_1+c_2\text{exp}\left(\lambda_2(t-t_0)\right)\va{v}_2+2\text{exp}\left(\lambda_{3,r}(t-t_0)\right)\cdot
\end{equation*}
\begin{equation*}
\cdot[(c_{3,r}\text{cos}(\lambda_{3,i}(t-t_0))-c_{3,i}\text{sen}(\lambda_{3,i}(t-t_0)))\va{v}_{3,r}-
\end{equation*}
\begin{equation}
-(c_{3,r}\text{sen}(\lambda_{3,i}(t-t_0))+c_{3,i}\text{cos}(\lambda_{3,i}(t-t_0)))\va{v}_{3,i}]\in\mathbb{R}^4
\label{gensolpar}
\end{equation}

Note that since $\lambda_2<0$ and $\lambda_{3,r}<0$, the stationary solution $\va{\overline{\rho}}_{\text{stat}}$ for $t\rightarrow{}+\infty$ is $\va{\overline{\rho}}_{\text{stat}}=c_1\va{v}_1$.

\subsubsection{Initialization and measure of longitudinal and in-plane magnetization}
Given the initial time $t=t_0$, an initial density matrix can be set up by providing the four real numbers $\overline{\rho}_{11}^0$, $\overline{\rho}_{22}^0$, $\overline{\rho}_{12,r}^0$, $\overline{\rho}_{12,i}^0$ with $\overline{\rho}_{11}^0+\overline{\rho}_{22}^0=1$ and $\overline{\rho}_{12}^0=\overline{\rho}_{12,r}^0+i\overline{\rho}_{12,i}^0$. Note that $\mathcal{U}=I$ at $t=t_0$ and the Schrödinger and new pictures coincide. Hence, it is enough to provide the initial density matrix in the Schrödinger picture $\rho_{11}^0$, $\rho_{22}^0$, $\rho_{12,r}^0$, $\rho_{12,i}^0$ which coincide respectively with $\overline{\rho}_{11}^0$, $\overline{\rho}_{22}^0$, $\overline{\rho}_{12,r}^0$, $\overline{\rho}_{12,i}^0$. It must also fulfil $\rho_{11}^0+\rho_{22}^0=1$ and $\rho_{12}^0=\rho_{12,r}^0+i\rho_{12,i}^0$. 
\bigbreak
In case at $t=t_0$ the qubit state is pure with vector state $\ket{\psi(t_0)}=a_0\ket{u_+}+b_0\ket{u_-}$ with $|a_0|^2+|b_0|^2=1$, $a_0,b_0\in\mathbb{C}$, the density matrix is $\ket{\psi(t_0)}\bra{\psi(t_0)}$, where:

\begin{equation}
\rho_{11}^0=|a_0|^2,\rho_{22}^0=|b_0|^2,\rho_{12,r}^0=a_{0,r}b_{0,r}+a_{0,i}b_{0,i},\rho_{12,i}^0=a_{0,i}b_{0,r}-a_{0,r}b_{0,i}
\label{dmpuresta}
\end{equation}

For instance, in the determination of the spin relaxation times $T_1$, $T_m$ and in the production of Rabi oscillations with $t_0=0$, one common option is to use $\ket{\psi(t_0=0)}=\ket{u_-}$, $a_0=0$, $b_0=1$ such that $\rho_{11}^0=0$, $\rho_{22}^0=1$, $\rho_{12,r}^0=0$, $\rho_{12,i}^0=0$.
\bigbreak
After a time evolution, the master equation provides a density matrix, see Eq.\ref{mrdenmatnp}, in the new picture. Important physical observables of interest are the longitudinal and in-plane magnetization which respectively are crucial at determining the $T_1$ and $T_m$ spin relaxation times and producing Rabi oscillations. The measure of these observables is conducted by computing the expectation values of the operators $\sigma_z$ and $\sigma_x+i\sigma_y$, respectively. In the ordered basis set $\{\ket{u_+},\ket{u_-}\}$, the matrix representations in the Schrödinger picture are:

\begin{equation}
\sigma_z=\mqty(1 & 0 \\ 0 & -1) \qquad \sigma_x+i\sigma_y=\mqty(0 & 1 \\ 1 & 0)+i\mqty(0 & -i \\ i & 0)=\mqty(0 & 2 \\ 0 & 0)
\label{matrepmagobs}
\end{equation}

Given an observable represented by a Hermitian operator $O$, its expectation value $\langle O \rangle$ when the quantum system is described by a density operator $\rho$ is $\langle O \rangle =\text{Tr}[O\rho]$, being $\text{Tr}$ the trace. To compute the expectation values of $\sigma_z$ and $\sigma_x+i\sigma_y$, we first need to determine the density matrix $\rho$ in the Schrödinger picture. By using the expressions of $\overline{\rho}$ and $\mathcal{U}$, $\rho$ is written in the ordered basis set $\{\ket{u_+},\ket{u_-}\}$ as:

\begin{equation}
\rho(t\geq t_0)=\mathcal{U}^{\dagger}\overline{\rho}\mathcal{U}=\mqty(\overline{\rho}_{11} & \text{exp}\left(-i(t-t_0)\omega_{\text{MW}}\right)\overline{\rho}_{12} \\ \text{exp}\left(i(t-t_0)\omega_{\text{MW}}\right)\overline{\rho}_{12}^* & \overline{\rho}_{22})
\label{denmatschpic}
\end{equation}

The longitudinal magnetization $\text{M}_z$ is:

\begin{equation}
\text{M}_z(t\geq t_0)=\text{Tr}[\sigma_z\rho(t\geq t_0)]=\text{Tr}\mqty(\overline{\rho}_{11} & \text{exp}\left(-i(t-t_0)\omega_{\text{MW}}\right)\overline{\rho}_{12} \\ -\text{exp}\left(i(t-t_0)\omega_{\text{MW}}\right)\overline{\rho}_{12}^* & -\overline{\rho}_{22})=\overline{\rho}_{11}-\overline{\rho}_{22}
\label{longmag}
\end{equation}

The in-plane magnetization $\text{M}_{xy}$ is:

\begin{equation}
\text{M}_{xy}(t\geq t_0)=\text{Tr}[(\sigma_x+i\sigma_y)\rho(t\geq t_0)]=\text{Tr}\mqty(2\text{exp}\left(i(t-t_0)\omega_{\text{MW}}\right)\overline{\rho}_{12}^* & 2\overline{\rho}_{22} \\ 0 & 0)=
\label{inplanemag}
\end{equation}
\begin{equation*}
=\text{Re}(\text{M}_{xy}(t-t_0))+i\text{Im}(\text{M}_{xy}(t-t_0))
\end{equation*}

where:

\begin{equation}
\text{Re}(\text{M}_{xy}(t-t_0))=2\left(\overline{\rho}_{12,r}\text{cos}((t-t_0)\omega_{\text{MW}})+\overline{\rho}_{12,i}\text{sen}((t-t_0)\omega_{\text{MW}})\right)
\label{mxyre}
\end{equation}

\begin{equation}
\text{Im}(\text{M}_{xy}(t-t_0))=2\left(\overline{\rho}_{12,r}\text{sen}((t-t_0)\omega_{\text{MW}})-\overline{\rho}_{12,i}\text{cos}((t-t_0)\omega_{\text{MW}})\right)
\label{mxyim}
\end{equation}

Note that $\text{M}_{xy}(t\geq t_0)$ -with oscillatory real and imaginary parts- is complex but in order to compare with experimental data we must provide a real magnitude. The option that we consider is to take the absolute value $|\text{M}_{xy}(t\geq t_0)|$:\cite{absinpm1,absinpm2,absinpm3}

\begin{equation}
|\text{M}_{xy}(t\geq t_0)|=\sqrt{\text{Re}(\text{M}_{xy}(t-t_0))^2+\text{Im}(\text{M}_{xy}(t-t_0))^2}=2|\overline{\rho}_{12}^*|=2\sqrt{(\overline{\rho}_{12,r})^2+(\overline{\rho}_{12,i})^2}
\label{mxyabs}
\end{equation}

In case of powder samples or frozen solutions, we integrate the magnetization over every single spatial direction $(\epsilon,\theta,\phi)$ to provide an average magnetization $\text{M}_{av}$, where $\text{M}$ can be either $\text{M}_z$ or $|\text{M}_{xy}|$ depending on the case of interest:

\begin{equation}
\text{M}_{av}(t\geq t_0)=\int_{-\pi}^{\pi}\left(\int_0^{\pi}\left(\int_0^{2\pi}\text{M}(t-t_0,\epsilon,\theta,\phi)\text{w}(\epsilon,\theta,\phi)d\epsilon\right) \text{sen}\theta d\theta\right)d\phi
\label{magav}
\end{equation}

where $\text{w}=\text{w}(\epsilon,\theta,\phi)$ is the weight function that determines the importance of each spatial direction and it is normalized: 

\begin{equation}
\int_{-\pi}^{\pi}\int_0^{\pi}\int_0^{2\pi}\text{w}(\epsilon,\theta,\phi)\text{sen}\theta d\epsilon d\theta d\phi =1
\label{weigen}
\end{equation}

The simplest option is to consider a uniform distribution where each direction has the same weight:

\begin{equation}
\text{w}=\frac{1}{4\pi}\frac{1}{2\pi}
\label{weiuni}
\end{equation}

\begin{equation}
\text{M}_{av}(t\geq t_0)=\frac{1}{4\pi}\int_{-\pi}^{\pi}\left(\int_0^{\pi}\left(\frac{1}{2\pi}\int_0^{2\pi}\text{M}(t-t_0,\epsilon,\theta,\phi)d\epsilon\right)\text{sen}\theta d\theta\right)d\phi
\label{magavuni}
\end{equation}

We will use the case studies below to test this option. Now, let us define the function:

\begin{equation}
f(t-t_0,\theta,\phi)=\frac{1}{2\pi}\int_0^{2\pi}\text{M}(t-t_0,\epsilon,\theta,\phi)d \epsilon
\label{funcf}
\end{equation}

We approximate this integral by means of a Lebedev rule:\cite{lebedev}

\begin{equation}
\text{M}_{av}(t\geq t_0)=\frac{1}{4\pi}\int_{-\pi}^{\pi}\left(\int_0^{\pi}f(t-t_0,\theta,\phi)\text{sen}\theta d\theta\right)d\phi\approx\sum_{i=1}^{\text{N}}\text{p}_if(t-t_0,\theta_i,\phi_i)
\label{intmapprox}
\end{equation}

being $\text{N}$ the number of points of the rule and the tabulated weights $\text{p}_i$ are such that $\sum_{i=1}^{\text{N}}\text{p}_i=1$. For each pair $(\theta_i,\phi_i)$, we have the integral:

\begin{equation}
f(t-t_0,\theta_i,\phi_i)=\frac{1}{2\pi}\int_0^{2\pi}\text{M}(t-t_0,\epsilon,\theta_i,\phi_i)d \epsilon
\label{intmeps}
\end{equation}

which we simply approximate by partitioning the unit circle with a set of $\text{L}\geq 2$ angles $0 \leq \epsilon_j \leq 2\pi$ equally-spaced, namely $\epsilon_j=2\pi\frac{j-1}{\text{L}}$, and equally-weighted:

\begin{equation}
f(t-t_0,\theta_i,\phi_i)\approx\frac{1}{\text{L}}\sum_{j=1}^{\text{L}}\text{M}(t-t_0,\epsilon_j,\theta_i,\phi_i)
\label{intmepsapprox}
\end{equation}

\subsubsection{Analytical and explicit solution of the free evolution}

The free evolution refers to the case in which the oscillating magnetic field $\va{B}_1=0$ and $\omega_{\text{MW}}=0$. This means that the time evolution of the qubit state is just determined by its unitary dynamics given by Eq.\ref{Heffqubit} but constrained by the relaxation rates $\Gamma_{\text{ab}}$, $\Gamma_{\text{em}}$, $\Gamma_{\text{mag}}$. This particular case is of interest because it allows finding an analytical solution for the master equation -without using numerical methods- and explicitly relating $\Gamma_{\text{ab}}$, $\Gamma_{\text{em}}$, $\Gamma_{\text{mag}}$ with the relaxation time that produces the decay in time of both the longitudinal and the in-plane magnetization. This is relevant since the last step in the pulse sequences employed to determine the spin relaxation times $T_1$ and $T_m$ is a free evolution. 
\bigbreak
When $\omega_{\text{MW}}=0$, the change of picture Eq.\ref{chapic} is $\mathcal{U}=I$ and we can just work in the Schrödinger picture. Since $\va{B}_1=0$ as well, the equation system Eq.\ref{diffeqsys} is now:

\begin{equation}
\mqty(\dot{\rho}_{11} \\ \dot{\rho}_{22} \\ \dot{\rho}_{12,r} \\ \dot{\rho}_{12,i}) = \mqty(\text{A}_{11} & \text{B}_{11} & 0 & 0 \\
-\text{A}_{11} & -\text{B}_{11} & 0 & 0 \\
0 & 0 & -\text{D} & \omega_{+-} \\
0 & 0 & -\omega_{+-} & -\text{D})\mqty(\rho_{11} \\ \rho_{22} \\ \rho_{12,r} \\ \rho_{12,i}) 
\label{diffeqsysfe}
\end{equation}

The coefficient matrix becomes block-diagonal and is simple to diagonalize. The eigenvalues are:

\begin{equation}
\lambda_1=0 \quad \lambda_2=-(\Gamma_{\text{ab}}+\Gamma_{\text{em}}+\Gamma_{\text{mag}}) \quad \lambda_3=-\frac{\Gamma_{\text{ab}}+\Gamma_{\text{em}}+2\Gamma_{\text{mag}}}{2}+i\omega_{+-} \quad \lambda_4=\overline{\lambda}_3
\label{eigenvafe}
\end{equation}

Note that $\lambda_2$ is real and negative, $\text{Re}(\lambda_3)<0$, and two eigenvalues are complex-conjugate. The normalized eigenvectors are:

\begin{equation}
\va{v}_1=\frac{1}{\sqrt{(b/a)^2+1}}\mqty(-b/a \\ 1 \\ 0 \\ 0) \quad \va{v}_2=\frac{1}{\sqrt{2}}\mqty(-1 \\ 1 \\ 0 \\ 0) \quad \va{v}_3=\va{v}_{3,r}+i\va{v}_{3,i}=\frac{1}{\sqrt{2}}\mqty(0 \\ 0 \\ 0 \\ 1)+i\frac{1}{\sqrt{2}}\mqty(0 \\ 0 \\ -1 \\ 0) \quad \va{v}_4 = \overline{\va{v}}_3
\label{eigenvefe}
\end{equation}

where $a=\text{A}_{11}$ and $b=\text{B}_{11}$. It is also found that:

\begin{equation}
\va{c}=\mqty(c_1 \\ c_2 \\ c_3 \\ c_4)=\mqty(p \\ q\left(a\rho_{11}^0+b\rho_{22}^0\right) \\ \left(\rho_{12,i}^0+i\rho_{12,r}^0\right)/\sqrt{2} \\ \left(\rho_{12,i}^0-i\rho_{12,r}^0\right)/\sqrt{2}) \quad\text{with}\quad p=\frac{a\sqrt{(b/a)^2+1}}{a-b} \quad q=-\frac{\sqrt{2}}{a-b}
\end{equation}

Note that $c_1,c_2\in\mathbb{R}$ and $c_3,c_4$ are complex-conjugate.
Now, regarding the longitudinal magnetization, one obtain:

\begin{equation}
\rho_{11}=c_1(\va{v}_1)_1+c_2\text{exp}\left(\lambda_2(t-t_0)\right)(\va{v}_2)_1 \qquad \rho_{22}=c_1(\va{v}_1)_2+c_2\text{exp}\left(\lambda_2(t-t_0)\right)(\va{v}_2)_2
\label{poplongmagfe}
\end{equation}

Hence:

\begin{equation}
\text{M}_z(t\geq t_0)=\rho_{11}-\rho_{22}=c_1((\va{v}_1)_1-(\va{v}_1)_2)+c_2\text{exp}\left(\lambda_2(t-t_0)\right)((\va{v}_2)_1-(\va{v}_2)_2)
\label{longmagfe}
\end{equation}

with a characteristic decay time:

\begin{equation}
-\lambda_2^{-1}=(\Gamma_{\text{ab}}+\Gamma_{\text{em}}+\Gamma_{\text{mag}})^{-1}
\label{decayrlongmag}
\end{equation}

Regarding the in-plane magnetization Eq.\ref{mxyabs}, we have:

\begin{equation*}
\rho_{12,r}=c_1(\va{v}_1)_3+c_2\text{exp}\left(\lambda_2(t-t_0)\right)(\va{v}_2)_3+2\text{exp}\left(\lambda_{3,r}(t-t_0)\right)\cdot
\end{equation*}
\begin{equation*}
\cdot[(c_{3,r}\text{cos}(\lambda_{3,i}(t-t_0))-c_{3,i}\text{sen}(\lambda_{3,i}(t-t_0)))(\va{v}_{3,r})_3-
\end{equation*}
\begin{equation}
-(c_{3,r}\text{sen}(\lambda_{3,i}(t-t_0))+c_{3,i}\text{cos}(\lambda_{3,i}(t-t_0)))(\va{v}_{3,i})_3]
\label{inpmag12rfe}
\end{equation}

\begin{equation*}
\rho_{12,i}=c_1(\va{v}_1)_4+c_2\text{exp}\left(\lambda_2(t-t_0)\right)(\va{v}_2)_4+2\text{exp}\left(\lambda_{3,r}(t-t_0)\right)\cdot
\end{equation*}
\begin{equation*}
\cdot[(c_{3,r}\text{cos}(\lambda_{3,i}(t-t_0))-c_{3,i}\text{sen}(\lambda_{3,i}(t-t_0)))(\va{v}_{3,r})_4-
\end{equation*}
\begin{equation}
-(c_{3,r}\text{sen}(\lambda_{3,i}(t-t_0))+c_{3,i}\text{cos}(\lambda_{3,i}(t-t_0)))(\va{v}_{3,i})_4]
\label{inpmag12ife}
\end{equation}

We observe that there are two characteristic decay times involved:

\begin{equation}
-\lambda_2^{-1}=\left(\Gamma_{\text{ab}}+\Gamma_{\text{em}}+\Gamma_{\text{mag}}\right)^{-1} \qquad -\lambda_{3,r}^{-1}=\left(\frac{\Gamma_{\text{ab}}+\Gamma_{\text{em}}}{2}+\Gamma_{\text{mag}}\right)^{-1}
\label{decayrsinpmag}
\end{equation}

Since $\lambda_{3,i}=\omega_{+-}\neq 0$, one can expect oscillations in $|\text{M}_{xy}(t\geq t_0)|$ with time, with $\omega_{+-}$ -the Larmor frequency (the angular frequency of the effective spin $\tilde{S}=1/2$ rotating around the $\va{B}$ direction)- as a frequency. In some situations, the spin where the qubit is encoded is coupled to a neighbouring proton nuclear spin in such a way that $|\text{M}_{xy}(t\geq t_0)|$ oscillates with time but with the Larmor frequency $\omega_{^1\text{H}}$ of the given proton. This possibility in which $\omega_{+-}$ is rather replaced by $\omega_{^1\text{H}}$ -or the Larmor frequency of the relevant coupled neighbouring spin- is not collected in our theory development. On the other hand, when $\Gamma_{\text{ab}}=\Gamma_{\text{em}}\equiv 0$, $|\text{M}_{xy}(t\geq t_0)|$ decays in time with the single characteristic time given by $\Gamma_{\text{mag}}^{-1}=\left(1/T_n+1/T_e\right)^{-1}$. 

\subsubsection{Semi-empirical transition rates}

To collect relaxation effects not covered in the computed rates $\Gamma_{\text{ab}}$, $\Gamma_{\text{em}}$, $\Gamma_{\text{mag}}$ (see main text), we add up effective rates $\Gamma_{\text{ab,add}}$, $\Gamma_{\text{em,add}}$, $\Gamma_{\text{mag,add}}$ such that we replace:

\begin{equation}
\Gamma_{\text{ab}}\rightarrow{}\Gamma_{\text{ab}}+\Gamma_{\text{ab,add}}
\label{addrab}
\end{equation}

\begin{equation}
\Gamma_{\text{em}}\rightarrow{}\Gamma_{\text{em}}+\Gamma_{\text{em,add}}
\label{addrem}
\end{equation}

\begin{equation}
\Gamma_{\text{mag}}\rightarrow{}\Gamma_{\text{mag}}+\Gamma_{\text{mag,add}}
\label{addrmag}
\end{equation}

While $\Gamma_{\text{ab}}$, $\Gamma_{\text{em}}$, $\Gamma_{\text{mag}}$ are kept with their fixed values given by the computation explained above, $\Gamma_{\text{ab,add}}$, $\Gamma_{\text{em,add}}$, $\Gamma_{\text{mag,add}}$ are intended to be employed as free parameters. This set can be actually reduced to just two parameters by relating $\Gamma_{\text{ab,add}}$ and $\Gamma_{\text{em,add}}$ with the so-called detailed-balance condition at a given temperature $T$. In this case, given $\Gamma_{\text{em,add}}$ as an input, $\Gamma_{\text{ab,add}}$ is determined according to:

\begin{equation}
\Gamma_{\text{ab,add}}=\Gamma_{\text{em,add}}\text{exp}\left(-\left(u_+-u_-\right)/k_{\text{B}}T\right)
\label{detbalcond}
\end{equation}

\subsubsection{Rotation operator and rotation without decoherence}

Herein, we provide another solution of interest of the equation system Eq.\ref{diffeqsys} consisting in $\Gamma_{\text{ab}}=\Gamma_{\text{em}}=\Gamma_{\text{mag}}\equiv 0$, namely $\text{A}_{11}=\text{B}_{11}=\text{D}\equiv 0$. With no relaxation terms activated, the spin qubit is now a closed system with a unitary dynamics governed by the effective Hamiltonian $\overline{H}_{\text{eff}}$ Eq.\ref{rwaheff}. This time evolution is given by the master equation Eq.\ref{matrepmerwa} which has been reduced to the well-known Schrödinger equation:

\begin{equation}
\dot{\overline{\rho}}=\frac{1}{i\hbar}[\overline{H}_{\text{eff}},\overline{\rho}]
\label{scheqdm}
\end{equation}

The solution of the above equation after a time $t\geq t_0$ is:

\begin{equation}
\overline{\rho}(t)=\mathcal{R}\overline{\rho}(t_0)\mathcal{R}^{\dagger} \quad \text{with} \quad \mathcal{R}=\text{exp}\left(-i\frac{t-t_0}{\hbar}\overline{H}_{\text{eff}}\right)
\label{solscheqdm}
\end{equation}

where the time evolution operator $\mathcal{R}$ is known as the rotation operator. Let us recall that in order to use the solution Eq.\ref{gensolpar} of the equation system Eq.\ref{diffeqsys} -particularly, to guarantee the linear independence of the eigenvectors $\va{v}_1,\va{v}_2,\va{v}_3,\va{v}_4$- we cannot set up the three relaxation rates $\Gamma_{\text{ab}}$, $\Gamma_{\text{em}}$, $\Gamma_{\text{mag}}$ to be all of them zero (actually, we cannot set $\Gamma_{\text{ab}}+\Gamma_{\text{ab,add}}\equiv 0, \Gamma_{\text{em}}+\Gamma_{\text{em,add}}\equiv 0, \Gamma_{\text{mag}}+\Gamma_{\text{mag,add}}\equiv 0$). That is why we would use the solution Eq.\ref{solscheqdm} in case of deactivating all the relaxation rates. Notwithstanding, in practice, we can still use the solution Eq.\ref{gensolpar} if all these rates are given the zero value but one of them which should have an effectively zero value such as $10^{-10}$ $\mu s^{-1}$ (see the section devoted to examples of one-qubit gates). Hence, we have not code the solution Eq.\ref{solscheqdm} in the software package QBithm. 
\bigbreak
If one is interested in performing qubit rotations by employing Eq.\ref{solscheqdm}, let us note that this rotation would not include any relaxation meaning, for instance, it is not appropriate to reproduce processes -such as decaying-damped Rabi oscillations- whose timescale is comparable or longer than the relaxation timescale. For a proper description of these phenomena, one should use the solution Eq.\ref{gensolpar} where the above-mentioned relaxation is included. As just said, the use of Eq.\ref{solscheqdm} is legit when the relaxation timescale is long enough as compared to the rotation time. Only in this case, relaxation can be neglected and the qubit dynamics can be approximated as a unitary dynamics given by Eq.\ref{solscheqdm}. For instance, this situation can be encountered in the $\pi/2$ and $\pi$ rotations -which are fast enough (few tens of ns) compared to the typical relaxation times above 1 $\mu$s found in molecular spin qubits- performed in the pulse sequences employed to determine the spin relaxation times $T_1$ and $T_m$. Notwithstanding, since the solution Eq.\ref{solscheqdm} is not implemented in QBithm, we will perform these $\pi/2$ and $\pi$ rotations -and any rotation- by employing Eq.\ref{gensolpar} in any case, regardless whether the rotation time is much shorter compared to the relaxation timescale or not.
\bigbreak
All in all, the rotations given by Eq.\ref{solscheqdm} can be considered to be already included in the solution Eq.\ref{gensolpar}. That is why it is enough to proceed with Eq.\ref{gensolpar} and hence this section could have been omitted. Nonetheless, we think it could be interesting for the curiosity of the reader to provide the computation of $\mathcal{R}$ and show how it rotates the qubit state on the Bloch sphere when the said state is particularly defined as a pure vector state. To compute $\mathcal{R}$, we first introduce in $\mathcal{R}$ the matrix representation of $\overline{H}_{\text{eff}}$ in the ordered basis set $\{\ket{u_+},\ket{u_-}\}$:

\begin{equation}
\mathcal{R}=\text{exp}\left(-i\frac{t-t_0}{2}\mqty(\delta & \Omega_{\text{R}}^* \\ \Omega_{\text{R}} & -\delta)\right)=\text{e}^{\text{A}}
\label{rotopA}
\end{equation}

being $t-t_0$ the rotation time. Now, $\text{A}$ is diagonalized. The eigenvalues are:

\begin{equation}
\lambda=\pm i\frac{t-t_0}{2}\Omega_{\text{g}}=\pm i\frac{t-t_0}{2}\sqrt{|\Omega_{\text{R}}|^2+\delta^2}
\label{eigenvarotop}
\end{equation}

where $\Omega_{\text{g}}=\sqrt{|\Omega_{\text{R}}|^2+\delta^2}$ is known as generalized Rabi frequency. This is the frequency that appears in the horizontal axis of the Fourier Transform spectrum of a Rabi oscillations plot. On the other hand, the eigenvectors are:

\begin{equation}
\ket{\pm i\frac{t-t_0}{2}\Omega_{\text{g}}}=\frac{1}{c_{\pm}}\left(\ket{u_+}\mp\left(D_1\pm D_2\right)\ket{u_-}\right)
\label{eigenverotop}
\end{equation}

where:

\begin{equation}
c_{\pm}=\sqrt{1+\left|\frac{\Omega_{\text{g}}\pm\delta}{\Omega_{\text{R}}^*}\right|^2} \quad D_1=\frac{\Omega_{\text{g}}}{\Omega_{\text{R}}^*} \quad D_2=\frac{\delta}{\Omega_{\text{R}}^*}
\label{pareigenrotop}
\end{equation}

Hence, the matrix representation of $\mathcal{R}$ in the ordered basis set $\{\ket{u_+},\ket{u_-}\}$ is:

\begin{equation*}
\mathcal{R}=\mqty(\frac{1}{c_-} & \frac{1}{c_+} \\ \frac{D_1-D_2}{c_-} & -\frac{D_1+D_2}{c_+})\mqty(\text{exp}\left(-i\frac{t-t_0}{2}\Omega_{\text{g}}\right) & 0 \\ 0 &  \text{exp}\left(i\frac{t-t_0}{2}\Omega_{\text{g}}\right))\mqty(\frac{1}{c_-} & \frac{1}{c_+} \\ \frac{D_1-D_2}{c_-} & -\frac{D_1+D_2}{c_+})^{\dagger}=
\end{equation*}
\begin{equation}
=\mqty(R_{++} & R_{+-} \\ R_{-+} & R_{--})=R_{++}\ket{u_+}\bra{u_+}+R_{+-}\ket{u_+}\bra{u_-}+R_{-+}\ket{u_-}\bra{u_+}+R_{--}\ket{u_-}\bra{u_-}
\label{matreprotop}
\end{equation}

where:

\begin{equation*}
R_{++}=\frac{1}{c_-^2}\text{exp}\left(-i\frac{t-t_0}{2}\Omega_{\text{g}}\right)+\frac{1}{c_+^2}\text{exp}\left(i\frac{t-t_0}{2}\Omega_{\text{g}}\right)
\end{equation*}
\begin{equation*}
R_{+-}=\frac{D_1^*-D_2^*}{c_-^2}\text{exp}\left(-i\frac{t-t_0}{2}\Omega_{\text{g}}\right)-\frac{D_1^*+D_2^*}{c_+^2}\text{exp}\left(i\frac{t-t_0}{2}\Omega_{\text{g}}\right)
\end{equation*}
\begin{equation*}
R_{-+}=\frac{D_1-D_2}{c_-^2}\text{exp}\left(-i\frac{t-t_0}{2}\Omega_{\text{g}}\right)-\frac{D_1+D_2}{c_+^2}\text{exp}\left(i\frac{t-t_0}{2}\Omega_{\text{g}}\right)
\end{equation*}
\begin{equation}
R_{--}=\frac{|D_1-D_2|^2}{c_-^2}\text{exp}\left(-i\frac{t-t_0}{2}\Omega_{\text{g}}\right)+\frac{|D_1+D_2|^2}{c_+^2}\text{exp}\left(i\frac{t-t_0}{2}\Omega_{\text{g}}\right)
\label{elemrotop}
\end{equation}

Let us analyze the particular case in which there is zero detuning $\delta =0$, i.e., $\omega_{+-}=\omega_{\text{MW}}$. In this case, $c_{\pm}=\sqrt{2}$, $D_2=0$, $D_1=|\Omega_{\text{R}}^*|/\Omega_{\text{R}}^*=\text{e}^{i\eta}$ for some $\eta\in\mathbb{R}$, and: 

\begin{equation}
\mathcal{R}=\mqty(\text{cos}\left(\frac{t-t_0}{2}\Omega_{\text{g}}\right) & -iD_1^*\text{sen}\left(\frac{t-t_0}{2}\Omega_{\text{g}}\right) \\ -iD_1\text{sen}\left(\frac{t-t_0}{2}\Omega_{\text{g}}\right) & \text{cos}\left(\frac{t-t_0}{2}\Omega_{\text{g}}\right))
\label{rotoppart}
\end{equation}

Let us now consider the particular example in which all the spin population is located in the lowest energy qubit state $\ket{u_-}$ which can be assigned to the north pole of the Bloch sphere. Hence, the qubit initial state is the pure state vector $\ket{\psi(t=t_0)}=\ket{u_-}$. The result after applying the rotation $\mathcal{R}$ is:

\begin{equation*}
\mathcal{R}\ket{\psi(t=t_0)}=\mqty(-iD_1^*\text{sen}\left(\frac{t-t_0}{2}\Omega_{\text{g}}\right) \\ \text{cos}\left(\frac{t-t_0}{2}\Omega_{\text{g}}\right))=\mqty(\text{e}^{i\left(\frac{3\pi}{2}-\eta\right)}\text{sen}\left(\frac{t-t_0}{2}\Omega_{\text{g}}\right) \\ \text{cos}\left(\frac{t-t_0}{2}\Omega_{\text{g}}\right))=
\end{equation*}
\begin{equation}
=\text{cos}\left(\frac{t-t_0}{2}\Omega_{\text{g}}\right)\ket{u_-}+\text{e}^{i\left(\frac{3\pi}{2}-\eta\right)}\text{sen}\left(\frac{t-t_0}{2}\Omega_{\text{g}}\right)\ket{u_+}
\label{exrotop}
\end{equation}

This corresponds to the parameterization of a qubit state $\ket{q}=\text{cos}\left(\frac{\theta}{2}\right)\ket{0}+\text{e}^{i\phi}\text{sen}\left(\frac{\theta}{2}\right)\ket{1}$ on the Bloch sphere with the identification: $\ket{u_-}\equiv\ket{0}$ (north pole), $\ket{u_+}\equiv\ket{1}$ (south pole), $\frac{3\pi}{2}-\eta\equiv$ azimuthal angle, $(t-t_0)\Omega_{\text{g}}\equiv$ rotation zenithal angle. Note that the key parameter in the qubit rotation is the rotation zenithal angle $(t-t_0)\Omega_{\text{g}}$ in the Bloch sphere. For instance, the value $(t-t_0)\Omega_{\text{g}}=\pi/2$ corresponds to a $\pi/2$-pulse, namely a $\pi/2$ rotation around some axis contained in the equatorial plane of the Bloch sphere that transforms the state $\ket{u_-}$ of the previous example into some state contained in the above-mentioned plane. In case the rotation zenithal angle is $(t-t_0)\Omega_{\text{g}}=\pi$, we have a $\pi$-pulse that transforms $\ket{u_-}$ into $\ket{u_+}$ up to a global phase. The expression $(t-t_0)\Omega_{\text{g}}$ is omitted in some research articles such that it is replaced by the rotation zenithal angle as the only input. In other words, these research articles assume that the combination between $t-t_0$ and $\Omega_{\text{g}}$ will appropriately be chosen in the lab to produce the desired rotation zenithal angle. If generally the initial qubit state is $\ket{\psi(t=t_0)}=a\ket{u_+}+b\ket{u_-}$ with $|a|^2+|b|^2=1$, $a,b\in\mathbb{C}$, the result of the rotation is:

\begin{equation*}
\mathcal{R}\ket{\psi(t=t_0)}=\mqty(a\text{cos}\left(\frac{t-t_0}{2}\Omega_{\text{g}}\right)-ibD_1^*\text{sen}\left(\frac{t-t_0}{2}\Omega_{\text{g}}\right) \\-iaD_1\text{sen}\left(\frac{t-t_0}{2}\Omega_{\text{g}}\right)+b\text{cos}\left(\frac{t-t_0}{2}\Omega_{\text{g}}\right))=
\end{equation*}
\begin{equation*}
=\left(\text{e}^{i\left(\frac{3\pi}{2}+\eta\right)}a\text{sen}\left(\frac{t-t_0}{2}\Omega_{\text{g}}\right)+b\text{cos}\left(\frac{t-t_0}{2}\Omega_{\text{g}}\right)\right)\ket{u_-}+
\end{equation*}
\begin{equation}
+\left(a\text{cos}\left(\frac{t-t_0}{2}\Omega_{\text{g}}\right)+\text{e}^{i\left(\frac{3\pi}{2}-\eta\right)}b\text{sen}\left(\frac{t-t_0}{2}\Omega_{\text{g}}\right)\right)\ket{u_+}
\label{genqbrotop}
\end{equation}

It may be less obvious that the above expression can also be written of the form $\ket{q}=\text{cos}\left(\frac{\theta}{2}\right)\ket{0}+\text{e}^{i\phi}\text{sen}\left(\frac{\theta}{2}\right)\ket{1}$ for some $\theta$ and $\phi$. Nonetheless, since $||\ket{\psi(t=t_0)}||=1$ and $\mathcal{R}$ is norm-conserving because $\mathcal{R}$ is a unitary matrix, we also have that $||\mathcal{R}\ket{\psi(t=t_0)}||=1$. Hence, $\mathcal{R}\ket{\psi(t=t_0)}$ can also somehow be written in the form $\ket{q}=\text{cos}\left(\frac{\theta}{2}\right)\ket{0}+\text{e}^{i\phi}\text{sen}\left(\frac{\theta}{2}\right)\ket{1}$ for some $\theta$ and $\phi$. If we now write $\ket{\psi(t=t_0)}=\text{cos}\left(\frac{\theta}{2}\right)\ket{u_-}+\text{e}^{i\phi}\text{sen}\left(\frac{\theta}{2}\right)\ket{u_+}$ for some $\theta$ and $\phi$, the resulting expression Eq.\ref{genqbrotop} is:

\begin{equation*}
\mathcal{R}\ket{\psi(t=t_0)}=\left(\text{e}^{i\left(\frac{3\pi}{2}+\eta+\phi\right)}\text{sen}\left(\frac{\theta}{2}\right)\text{sen}\left(\frac{t-t_0}{2}\Omega_{\text{g}}\right)+\text{cos}\left(\frac{\theta}{2}\right)\text{cos}\left(\frac{t-t_0}{2}\Omega_{\text{g}}\right)\right)\ket{u_-}+
\end{equation*}
\begin{equation}
+\left(\text{e}^{i\phi}\text{sen}\left(\frac{\theta}{2}\right)\text{cos}\left(\frac{t-t_0}{2}\Omega_{\text{g}}\right)+\text{e}^{i\left(\frac{3\pi}{2}-\eta\right)}\text{cos}\left(\frac{\theta}{2}\right)\text{sen}\left(\frac{t-t_0}{2}\Omega_{\text{g}}\right)\right)\ket{u_+}
\label{genqbrotoppar}
\end{equation}

Let us suppose that $\text{e}^{i\phi}=\text{e}^{i(\frac{3\pi}{2}-\eta)}$:

\begin{equation*}
\mathcal{R}\ket{\psi(t=t_0)}=\left(-\text{sen}\left(\frac{\theta}{2}\right)\text{sen}\left(\frac{t-t_0}{2}\Omega_{\text{g}}\right)+\text{cos}\left(\frac{\theta}{2}\right)\text{cos}\left(\frac{t-t_0}{2}\Omega_{\text{g}}\right)\right)\ket{u_-}+
\end{equation*}
\begin{equation*}
+\text{e}^{i\phi}\left(\text{sen}\left(\frac{\theta}{2}\right)\text{cos}\left(\frac{t-t_0}{2}\Omega_{\text{g}}\right)+\text{cos}\left(\frac{\theta}{2}\right)\text{sen}\left(\frac{t-t_0}{2}\Omega_{\text{g}}\right)\right)\ket{u_+}=
\end{equation*}
\begin{equation}
=\text{cos}\left(\frac{\theta+(t-t_0)\Omega_{\text{g}}}{2}\right)\ket{u_-}+\text{e}^{i\phi}\text{sen}\left(\frac{\theta+(t-t_0)\Omega_{\text{g}}}{2}\right)\ket{u_+}
\label{genqbrotoppar2}
\end{equation}

What we have obtained is that $\mathcal{R}\ket{\psi(t=t_0)}$ is a rotation that has been performed along the meridian defined by the unaltered azimuthal angle $\phi$ with a rotation angle $\theta+(t-t_0)\Omega_{\text{g}}-\theta=(t-t_0)\Omega_{\text{g}}$ that indeed can be interpreted as a zenithal angle. In other words, the condition $\text{e}^{i\phi}=\text{e}^{i(\frac{3\pi}{2}-\eta)}$ characterizes the fact that the rotation axis is contained in the equatorial plane of the Bloch sphere and the rotation is realized along a given meridian of the mentioned sphere. This condition contains the particular case $\ket{\psi(t=t_0)}=\ket{u_-}$ studied above in which $\theta=0$. 
\bigbreak
In case $\delta\neq 0$, $\mathcal{R}$ is still unitary and $\mathcal{R}\ket{\psi(t=t_0)}$ -with a general $\ket{\psi(t=t_0)}=\text{cos}\left(\frac{\theta}{2}\right)\ket{u_-}+\text{e}^{i\phi}\text{sen}\left(\frac{\theta}{2}\right)\ket{u_+}$- will also be contained in the Bloch sphere. The difference is that, if for instance $\ket{\psi(t=t_0)}=\ket{u_-}$ again, a value $(t-t_0)\Omega_{\text{g}}=\pi/2$ will transform $\ket{u_-}$ into a state $\mathcal{R}\ket{\psi(t=t_0)}$ such that the more $\delta$ is different from zero the more $\mathcal{R}\ket{\psi(t=t_0)}$ will be far away from the particular state obtained with $\delta=0$. If additionally relaxation is included, the norm is not conserved anymore in the rotation and the resulting state is a rather mixed state now not contained in the Bloch sphere anymore.
\bigbreak
To conclude this section, let us now consider that the initial qubit state is a mixed state described by a density matrix $\overline{\rho}(t_0)$ (with $\overline{\rho}_{11}^0+\overline{\rho}_{22}^0=1$):

\begin{equation}
\overline{\rho}(t_0)=\overline{\rho}_{11}^0\ket{u_+}\bra{u_+}+\overline{\rho}_{12}^0\ket{u_+}\bra{u_-}+\overline{\rho}_{12}^{0*}\ket{u_-}\bra{u_+}+\overline{\rho}_{22}^0\ket{u_-}\bra{u_-}
\label{rotqbinidm}
\end{equation}

The result $\overline{\rho}(t)$ of the rotation is given by Eq.\ref{solscheqdm}. By recalling the expressions Eq.\ref{matreprotop} and Eq.\ref{elemrotop}, we obtain:

\begin{equation}
\overline{\rho}(t)=\overline{\rho}_{++}(t)\ket{u_+}\bra{u_+}+\overline{\rho}_{+-}(t)\ket{u_+}\bra{u_-}+\overline{\rho}_{+-}^*(t)\ket{u_-}\bra{u_+}+\overline{\rho}_{--}(t)\ket{u_-}\bra{u_-}
\label{dmaftrot}
\end{equation}

where:

\begin{equation*}
\overline{\rho}_{++}(t)=|R_{++}|^2\overline{\rho}_{11}^0+2\text{Re}\left(R_{++}R_{+-}^*\overline{\rho}_{12}^0\right)+|R_{+-}|^2\overline{\rho}_{22}^0
\end{equation*}
\begin{equation*}
\overline{\rho}_{+-}(t)=R_{++}R_{-+}^*\overline{\rho}_{11}^0+R_{++}R_{--}^*\overline{\rho}_{12}^0+R_{+-}R_{-+}^*\overline{\rho}_{12}^{0*}+R_{+-}R_{--}^*\overline{\rho}_{22}^0
\end{equation*}
\begin{equation}
\overline{\rho}_{--}(t)=|R_{-+}|^2\overline{\rho}_{11}^0+2\text{Re}\left(R_{-+}R_{--}^*\overline{\rho}_{12}^0\right)+|R_{--}|^2\overline{\rho}_{22}^0
\label{eledmaftrot}
\end{equation}

\newpage

\section{QBithm software package}

The above theory development is coded in the form of a free and open-source FORTRAN77 software package called QBithm. The code file is qbithm.f (see folder SI), and it consists of three input files -qb.ddata, qb.mdata, qb.adata- and three output files -qb.out, qb.mxy.out, qb.mz.out- that we describe below together with some toy made-up examples. One can use gfortran as a compiler as we do. LAPACK library is required. We also include as a Supplementary Information the script script\_exe\_gf.sh and the job job-qbithm that we employ to run and compile the code (see also folder SI).

\subsection{Code and input} 

Let us start by describing the code as it contains some of the input parameters. After opening the file qbithm.f, one finds the following parameters that must be firstly set up:

\begin{itemize}
\item id: Hilbert space size, namely the numerical value $(2J+1)(2I+1)$ of spin states and energies obtained after diagonalizing $\hat{H}_{eq}$ in Eq.\ref{Heq}, being $J$ and $I$ the ground electron and nuclear spin quantum numbers.
\item ig: qubit ground state, selected such that $1\leq\text{ig}\leq\text{id}-1$.
\item ie: qubit excited state, selected such that $\text{ig}+1\leq\text{ie}\leq\text{id}$.
\item nm: number of vibration normal modes ($\geq 1$) in qb.ddata and qb.mdata. One can choose to work with no normal modes by selecting nm=0. This will make QBithm set $\Gamma_{\text{ab}}=\Gamma_{\text{em}}=0$.
\item ic: number of static magnetic field values ($\geq 1$) in qb.ddata. 
\item nd: number of static magnetic field directions ($\geq 1$) in qb.ddata. 
\item temp: temperature ($>0$).
\item sfgw: scaling factor ($>0$) for gaussian width in qb.mdata. This parameter multiplies each phonon mode half-width $\sigma_i$ in qb.mdata in case one desires to scale this magnitude. Otherwise, one sets sfgw=1.  
\item top: when Eq.\ref{abs1ph} and Eq.\ref{em1ph} have significant values a1ph and e1ph, respectively -say above numerical precision-, the contribution percentage of each normal mode $i$ is (ia1ph/a1ph)100 and (ie1ph/e1ph)100, where ia1ph and ie1ph is the corresponding summand in Eq.\ref{abs1ph} and Eq.\ref{em1ph}, respectively. Only the normal modes with a contribution percentage larger than top ($>0$) will be listed in qb.out. 
\item ttp: only the pairs of normal modes in Eq.\ref{RDirect}, Eq.\ref{LDirect}, Eq.\ref{Stokes}, Eq.\ref{antiStokes}, Eq.\ref{RSpont}, Eq.\ref{LSpont} with a contribution percentage larger than ttp ($>0$) will be listed in qb.out. 
\item geme: numerical value ($\geq 0$) given to $\Gamma_{\text{em,add}}$ in Eq.\ref{addrem}.
\item gabe: numerical value ($\geq 0$) given to $\Gamma_{\text{ab,add}}$ in Eq.\ref{addrab}. If one sets gabe=-1.0, QBithm gives $\Gamma_{\text{ab,add}}$ the value provided by Eq.\ref{detbalcond}.
\item tmage: numerical value ($\geq 0$) given to $\Gamma_{\text{mag,add}}$ in Eq.\ref{addrmag}.
\item gfi: free-$J$ Landé factor $g_{\text{I}}>0$ in Eq.\ref{rabfreq}. 
\item bcm: magnitude $|\va{B}_1|>0$ of the oscillating magnetic field. QBithm sets automatically $|\va{B}_1|=0$ during a free evolution and resets to the given bcm value after finishing it. 
\item firr: linear driving frequency $\omega_{\text{MW}}/2\pi>0$ in Eq.\ref{driHam} of the oscillating magnetic field. Again, QBithm sets automatically $\omega_{\text{MW}}/2\pi=0$ during a free evolution and resets to the given firr value after finishing it.
\item alp: polarization angle $0\leq\alpha<360$ in Eq.\ref{ytildir} of the oscillating magnetic field. 
\item nang: number $\text{L}\geq 1$ of angles $\epsilon_j$ in Eq.\ref{intmepsapprox}. In case of no integration, one sets nang=1.
\bigbreak
When writting $\text{M}_z(t)$ and $|\text{M}_{xy}(t)|$ -$t$ is the time taken by the whole sequence/algorithm- against $m\tau$ in the output files qb.mz.out and qb.mxy.out (see below) -$m\geq 1$ is the number of steps in qb.adata (see below) with a variable duration time (which is the same for all the mentioned steps and equal to $\tau$)-, $\tau$ is swept from esta to eend through npe point times (the sum over the time taken by each step with a fixed duration is often negligible compared to $m\tau$ as soon as the initial value of $\tau$, namely esta, is a large enough positive value (see below); hence, $t\approx m\tau$). 
\item esta: initial value of $\tau$ ($\geq 0$).
\item eend: final value of $\tau$ ($>$esta).
\item npe: number ($\geq 2$) of point times taken by $\tau$. If no steps with variable duration time are selected in qb.adata, these three parameters are not employed by QBithm and what one finds in qb.mz.out and qb.mxy.out is just the longitudinal and in-plane magnetization evaluated at $t$.
\bigbreak
At the beginning of the algorithm/pulse sequence in qb.adata (Schrödinger picture and ordered basis set $\{\ket{u_+},\ket{u_-}\}$):
\item ro11: density matrix upper diagonal element $0\leq\rho_{11}(t=0)\leq 1$ (population of $\ket{u_+}(t=0)$).
\item ro22: density matrix lower diagonal element $0\leq\rho_{22}(t=0)\leq 1$ (population of $\ket{u_-}(t=0)$) such that ro11+ro22=1. 
\item ro12r: real part $\rho_{12,r}(t=0)$ of the density matrix upper off-diagonal element $\rho_{12}(t=0)$.
\item ro12i: imaginary part $\rho_{12,i}(t=0)$ of the density matrix upper off-diagonal element $\rho_{12}(t=0)$. 
\item nsa: number ($\geq 1$) of steps in qb.adata. 
\end{itemize}
\bigbreak

As mentioned above, for instance, in the determination of the spin relaxation times $T_1$, $T_m$ and in the production of Rabi oscillations, one would set ro11=0, ro22=1, ro12r=0, ro12i=0. 
\bigbreak
The three input files are now described below:

\begin{itemize}
\item qb.ddata
This input file is broken down into as many blocks as values of $|\va{B}|$. The first line of each one of these blocks contains the value of $|\va{B}|$. This value is not used by QBithm, it is just written with an informative purpose. Given one of these blocks, it is itself broken down into as many sub-blocks as directions $\left(\theta_i,\phi_i\right)$ of $\va{B}$. The corresponding line contains the values of $\phi_i$ (azimuthal angle), $\theta_i$ (zenithal angle), as well as both $\text{p}_i$ (weight in Eq.\ref{intmapprox}) and $1/T_n+1/T_e$ (see Eq.\ref{gmagsimpre}) corresponding to the given direction. The following line contains the $(2J+1)(2I+1)$ eigenenergies obtained after diagonalizing $\hat{H}_{eq}$ in Eq.\ref{Heq} with the direction $\left(\theta_i,\phi_i\right)$ of $\va{B}$. The two following lines contain the real and imaginary parts of the matrix elements $\bra{u_+}\hat{J}_{\gamma}/\hbar\ket{u_-}$, $\gamma=x,y,z$. The following line contains the $R$ one-phonon process matrix elements associated to $\Gamma_{\ket{u_-}\rightarrow\ket{u_+}}^{\text{ab,1p}}$ and given by (from left $i=1$ to right $i=R$):

\begin{equation}
\bra{u_+}\sum_{k,q}B_{k,i,eq}^q\hat{O}_k^q\ket{u_-}+\mu_B\sum_{\alpha=x,y,z}g_{\alpha,i,eq}B_{\alpha}\bra{u_+}\hat{J}_{\alpha}/\hbar\ket{u_-}
\label{1pmesimpre}
\end{equation}

There is no need to introduce the matrix elements associated to $\Gamma_{\ket{u_+}\rightarrow\ket{u_-}}^{\text{em,1p}}$ since they are complex-conjugate of Eq.\ref{1pmesimpre}, namely QBithm already computes the complex-conjugate value of Eq.\ref{1pmesimpre} to build $\Gamma_{\ket{u_+}\rightarrow\ket{u_-}}^{\text{em,1p}}$. The last part of the sub-block corresponds to the two-phonon process matrix elements. In particular, for each normal mode $i=1,\ldots,R$, it is provided:
\bigbreak
\subitem *for the virtual process, for all $i'\geq i$, from left to right: 

\begin{equation}
\bra{u_-}\sum_{k,q}B_{k,i,i',eq}^q\hat{O}_k^q\ket{u_+}+\mu_B\sum_{\alpha=x,y,z}g_{\alpha,i,i',eq}B_{\alpha}\bra{u_-}\hat{J}_{\alpha}/\hbar\ket{u_+}
\label{2pmevirtsimpre}
\end{equation}

The values of Eq.\ref{2pmevirtsimpre} are employed in R-Direct, Stokes and R-Spont. For L-Direct, anti-Stokes and L-Spont., QBithm uses the complex-conjugate of Eq.\ref{2pmevirtsimpre}. 
\bigbreak
\subitem *for the real processes, for all $i'=1,\ldots,R$, from left to right:
\bigbreak
\subsubitem -for R-Direct (for L-Direct, QBithm uses the complex-conjugate):

\begin{equation*}
\left(\bra{u_+}\sum_{k,q}B_{k,i',eq}^q\hat{O}_k^q\ket{c}+\mu_B\sum_{\alpha=x,y,z}g_{\alpha,i',eq}B_{\alpha}\bra{u_+}\hat{J}_{\alpha}/\hbar\ket{c}\right)\cdot
\end{equation*}
\begin{equation}
\cdot\left(\bra{c}\sum_{k,q}B_{k,i,eq}^q\hat{O}_k^q\ket{u_-}+\mu_B\sum_{\alpha=x,y,z}g_{\alpha,i,eq}B_{\alpha}\bra{c}\hat{J}_{\alpha}/\hbar\ket{u_-}\right)
\label{2pmerealRDsimpre}
\end{equation}

with as many lines as eigenenergies $E_c$ such that $u_-\leq E_c\leq u_+$ with $\ket{c}\neq\ket{u_-},\ket{u_+}$. 
\bigbreak
\subsubitem -for Stokes (for anti-Stokes, QBithm uses the complex-conjugate), same as Eq.\ref{2pmerealRDsimpre} but with as many lines as eigenenergies $E_c$ such that $E_c\geq u_+ > u_-$ with $\ket{c}\neq\ket{u_-},\ket{u_+}$. 
\bigbreak
\subsubitem -for R-Spont. (for L-Spont., QBithm uses the complex-conjugate), same as Eq.\ref{2pmerealRDsimpre} but with as many lines as eigenenergies $E_c$ such that $u_+ > u_- \geq E_c$ with $\ket{c}\neq\ket{u_-},\ket{u_+}$. 
\bigbreak
Note that, for a well-defined qubit, the eigenenergies $u_-$ and $u_+$ must no be degenerate, which means there should not be any other eigenstate $\ket{c}$ with energy $E_c$ such that $E_c=u_-$ or $E_c=u_+$. 
\bigbreak
This input file is generated by an adapted version of the software package SIMPRE.\cite{simpre2.0} This file must always contain at least one block corresponding to one value of $|\va{B}|$ and at least one sub-block corresponding to one direction $\left(\theta_i,\phi_i\right)$ of $\va{B}$. For instance, in a calculation for a single-crystal sample, it will contain right one block (ic=1) and right one sub-block (nd=1) where the only weight should be $\text{p}_1=1$. On the other hand, for a powder sample, a case can be just one block (ic=1) for one value of $|\va{B}|$ and several sub-blocks (nd$>$1) corresponding to a representative set of Lebedev directions $\{\left(\theta_i,\phi_i\right)\}_{i=1}^{\text{nd}}$ each one with its particular weight $\text{p}_i$. In either case, SIMPRE is always in charge of writing the proper weight $\text{p}_i$ for each given direction $\left(\theta_i,\phi_i\right)$ whose angles $\theta_i$ and $\phi_i$ are also written by SIMPRE. The value of nd -along with the particular values $\left(\theta_i,\phi_i\right)$- is selected in the relevant input file of SIMPRE, and the same value of nd must be employed in qbithm.f. The same applies for ic. Since qb.ddata is generated by SIMPRE, one must also use in ig and ie (see above) the same values employed in SIMPRE when this code was employed to generate qb.ddata. Importantly, when nm=0, SIMPRE will not generate qb.mdata nor write in qb.ddata any line corresponding to the one-phonon and two-phonon processes since in this case all the matrix elements are set to zero by QBithm. If one desires to build qb.ddata by hand, the mentioned lines must not be written either. 

\item qb.mdata

This input is also generated by SIMPRE, although it can be much more easily built by hand. It consists of three columns: on the left one finds the harmonic linear frequencies of the several normal modes, sorted from top to bottom by increasing numerical value; and the columns in the center and on the right resp. contain the reduced mass and the half-width $\sigma_i$ (see Eq.\ref{spde}) of the corresponding normal mode in the same line. This input is not read -nor generated by SIMPRE- when nm=0 so that its construction can be avoided in this case. The dimension of the reduced mass is chemical atomic mass unit (c.a.m.u.), which is employed in the output files of some software packages such as GAUSSIAN when employed to compute the vibration spectrum. 
\bigbreak
Note that the dimension (international system) of some input parameters such as vibration linear frequencies is s$^{-1}$ but they are introduced in cm$^{-1}$. The implicit constant employed for the conversion s$^{-1}\leftrightarrow$cm$^{-1}$ is c=29979245800 cm/s. On the other hand, other parameters are energies but instead of being introduced in Joule (international system) they are also introduced in cm$^{-1}$. In this case, the constant employed in the conversion J$\leftrightarrow$cm$^{-1}$ is h(J·s)c(cm/s)=1.9864459...·10$^{-23}$ J·cm. 

\item qb.adata

This input is always built by hand. In this file, one sets the several pulses or gates (steps) of the desired sequence or one-qubit algorithm. It will contain as many lines as the numerical value of nsa. Each line is characterized by three parameters. In the column on the left, one writes the numerical code defining the selected step: 0, 1, 2, 3 (see below). In the central column, the duration time of the step is established. The column on the right contains the value of the angle $\epsilon$ that determines the direction $\tilde{X}$ of the rotation axis (see Eq.\ref{udrotdir}). A simple option can be to use the same value of $\epsilon$ for all the steps, e.g. $\epsilon=0$. 
\bigbreak
The step sequence will be run by QBithm as many times as the numerical value of npe. Of course, every time the run is repeated, the qubit state is initialized to the one specified in the file qbithm.f via ro11, ro22, ro12r, and ro12i. If the steps 0 and 3 are not employed in qb.adata, QBithm will authomatically set npe=1. In this case, this means that qb.adata is composed of the steps 1 and 2 which have a fixed duration time. On the other hand, when the step 0 or 3 (or both) appears at least once in qb.adata, the step sequence is run npe times by QBithm and, in each run, the duration time $\tau$ of the steps 0 and/or 3 (both have the same duration time $\tau$ for each $\tau$ value) is increased from esta to eend by the amount (eend-esta)/npe. This leads to writing the longitudinal and in-plane magnetizations against the npe values of $m\tau$ -being $m\geq1$ the number of steps in qb.adata with variable duration time- in the output files qb.mz.out and qb.mxy.out. If npe=1, one finds in qb.mz.out and qb.mxy.out the longitudinal and in-plane magnetizations computed after the only run of the step sequence. If npe$>$1 but no steps 0 and 3 are employed in qb.adata, QBithm will also internally set npe=1 and one will also find in qb.mz.out and qb.mxy.out the longitudinal and in-plane magnetizations computed after the only run of the step sequence.
\bigbreak
The steps that can be selected are found among the following set of four elements: 

\subitem 0 = variable free evolution. In this step, QBithm itself sets $|\va{B}_1|=0$ and $\omega_{\text{MW}}=0$ (no driving field) to time-evolve the density matrix. In the corresponding line of qb.adata, one must write dummy values for the duration time and for the $\epsilon$ angle although they both will not be employed by QBithm. As mentioned above, the duration time of this step is variable from esta to eend. 

\subitem 1 = fixed free evolution. In this step, QBithm sets $|\va{B}_1|=0$ and $\omega_{\text{MW}}=0$ as well. Now, QBithm uses the value of the duration time in the central column but not the (dummy) value (that must be) given to $\epsilon$ within the same line.

\subitem 2 = rotation. In this step, QBithm uses the values bcm and firr to set $|\va{B}_1|$ and $\omega_{\text{MW}}$, resp. Now, both the values of the duration time and of the $\epsilon$ angle provided in the relevant line are employed by QBithm to time-evolve the density matrix. 

\subitem 3 = variable rotation. In this step, a rotation as above is implemented but with a variable duration time from esta to eend. The given value of the $\epsilon$ angle is employed but not the (dummy) value (that must be) given to the duration time in the central column of the same line. 

\end{itemize}
\bigbreak
The three output files are described in the following two sections by relying on the toy examples. 

\subsection{One-qubit quantum gates: CNOT, Hadamard, and Phase}

Herein, we implement a selection of one-qubit quantum gates such as CNOT, Hadamard and Phase. Let us parameterize the qubit state $\ket{q}$ in the ordered basis set $\{\ket{1}\equiv\ket{u_+},\ket{0}\equiv\ket{u_-}\}$ in terms of the zenithal $0\leq\theta\leq\pi$ and azimuthal $0\leq\phi\leq 2\pi$ angles of the Bloch sphere (do not confuse these $\theta$ and $\phi$ angles with the ones employed to describe the direction of the static magnetic field $\va{B}$:

\begin{equation}
\ket{q}=\text{cos}\frac{\theta}{2}\ket{0}+\text{e}^{i\phi}\text{sen}\frac{\theta}{2}\ket{1}
\label{qbparam}
\end{equation}

Let us initialize the qubit in the lower energy state $\ket{0}\equiv\ket{u_-}$. We use this state as a starting point to implement the above-mentioned quantum gates. The corresponding density matrix to set in qbithm.f is given by: ro11=0, ro22=1, ro12r=0, ro12i=0; and the remaining parameters are: id=7, ig=3, ie=6, nm=0, ic=1, nd=1, geme=0, gabe=0, and tmage=$10^{-10}$ $\mu\text{s}^{-1}$ (we follow the description explained above to drive the qubit -here in the form of one-qubit gates- without relaxation), gfi=2.0, bcm=1.5 mT, firr=8.99377 GHz (intended zero detuning with the qubit gap), alp=0.0, nang=1, nsa=2. Since nm=0, the values of sfgw, top and ttp, the input qb.mdata, and the one-phonon and two-phonon matrix elements in qb.ddata are not employed. In addition, since geme=0, one must give temp a dummy value. Note that if nm=0 but ic$>$1 and/or nd$>$1, one must not write in qb.ddata any line corresponding to matrix elements of both one-phonon and two-phonon processes to avoid error readings by QBithm (e.g., see input qb.ddata\_nd\_110\_nopb\_nosb in folder VOdmit2 of the case study [VO(dmit)$_2$]$^{2-}$). On the other hand, since the steps 0 and 3 are not used, the values of esta, eend and npe are not employed either. The made-up input qb.ddata is found in the folder Examples, while the input files qb.adata -implementing the several gates- and output files qb.out are found in the folder Examples/Gates. In this section, the output files qb.mxy.out and qb.mz.out are not of interest. 

\bigbreak

The CNOT gate can be implemented in the form of either Pauli X gate or Pauli Y gate whose representations are:

\begin{equation}
X=\ket{1}\bra{0}+\ket{0}\bra{1} \qquad Y=i\ket{1}\bra{0}-i\ket{0}\bra{1}
\label{XYPgates}
\end{equation}

The resulting density matrices of the actions $X\ket{0}$ and $Y\ket{0}$ are:

\begin{equation}
X\ket{0}\left(X\ket{0}\right)^{\dagger}=\ket{1}\bra{1} \qquad Y\ket{0}\left(Y\ket{0}\right)^{\dagger}=\ket{1}\bra{1}
\label{XYPgatesres}
\end{equation}

In both cases, the resulting density matrix is ro11=1, ro22=0, ro12r=0, ro12i=0. Indeed, this corresponds to a 180º rotation around an axis contained in the equatorial plane of the Bloch sphere from $\ket{0}\equiv\ket{u_-}$ to $\ket{1}\equiv\ket{u_+}$ (up to a global phase). As seen in the input files qb.adata\_PauliXgate\_rotdir0 and qb.adata\_PauliYgate\_rotdir90, the rotation direction in the first case is $\epsilon=0$º, while in the second case is $\epsilon=90$º. Actually, a 180º rotation starting from one of the two poles of the Bloch sphere will always produce the same result as a final state (up to a global phase), namely the other pole regardless the employed equatorial rotation axis. This means we could have used any value of $\epsilon$ in either case. In both cases, the rotation time is 23.816 ns which is the proper duration for a 180º rotation given the selected values of gfi=2 and bcm=1.5 mT. The right rotation time can be found by means of the generalized Rabi frequency $\Omega_{\text{g}}$ that appears in the output qb.out. In our case, $\Omega_{\text{g}}$=20.99437 MHz. Hence, a 180º rotation -one half-period- takes $0.5/\Omega_{\text{g}}$=23.816 ns. We see in the output files qb.out\_PauliXgate\_rotdir0 and qb.out\_PauliYgate\_rotdir90 that the resulting density matrices are, resp., ro11=1.00000, ro22=0.00000, ro12r=0.00001, ro12i=0.00018, and ro11=1.00000, ro22=0.00000, ro12r=0.00018, ro12i=-0.00001. Both coincide with the expected ro11=1, ro22=0, ro12r=0, ro12i=0. The deviations in ro12r and ro12i can be attributed to (i) the small detuning produced after truncating the value of firr, (ii) the truncation of the value $0.5/\Omega_{\text{g}}$, and (iii) the finite precision involved in the numerical diagonalization of the coefficient matrix Eq.\ref{diffeqsys}. The small yet finite value tmage=$10^{-10}$ $\mu\text{s}^{-1}$ should have a negligible effect on these deviations given that 1/tmage is much larger than the rotation time. 
\bigbreak
Regarding the rest of the output file qb.out, after the heading we see the four rates in Eq.\ref{globabemtr}, as well as the two rates $\Gamma_{\text{ab,add}}$ and $\Gamma_{\text{em,add}}$. Since nm=0, geme=0, gabe=0, all rates are zero. The firstly mentioned four rates are computed from and correspond to the only magnetic field block mf: 1 and direction sub-block az(º): 0.00000; ze(º): 0.00000. Note that mf: 1 is not the value of $|\va{B}|$ but just the list position of the magnetic field block in qb.ddata from 1 to ic. 
\bigbreak
In the Time evolution of density matrix section, we see again first the list position of the corresponding magnetic field block in qb.ddata, namely magnetic field: 1. Note that magnetization: 1 simply refers to the only run of the step sequence written in qb.adata (no steps 0 or 3 are employed in qb.adata for this example). Whenever the step 0 or 3 (or both) is introduced in qb.adata, there will be as many magnetization lines as the numerical value of npe. For each magnetization line, we first find some information such as the angles az(º): 0.00000; and ze(º): 0.00000; and the corresponding Gap(GHz): 8.99377; and Detuning(GHz): 0.00000. 
\bigbreak
Then, we see the several steps of the sequence sorted from top to bottom. Note that the step column contains the step position in the sequence from 1 to nsa, not to confuse with the above-mentioned step numerical code 0, 1, 2, 3. Within the same line of a given step, one finds the step duration time -either fixed (for numerical code 1 and 2) or variable (for numerical code 0 and 3)-, the rotation direction $\epsilon$ (for numerical code 2 and 3), the generalized Rabi frequency $\Omega_{\text{g}}$ (for numerical code 2 and 3), and the resulting density matrix after the time evolution given by the specific step. 

\bigbreak

The Hadamard gate consists in a 90º rotation of $\ket{0}$ around an axis contained in the equatorial plane of the Bloch sphere. The resulting state is contained in the mentioned equatorial plane in the form of an equally-weighted superposition between $\ket{0}$ and $\ket{1}$. Let us consider two particular examples, namely the $\phi=\pi/2$ Hadamard gate $H\left(\pi/2\right)$ and the $\phi=\pi$ Hadamard gate $H\left(\pi\right)$ whose representations are:

\begin{equation*}
H\left(\pi/2\right)=\frac{1}{\sqrt{2}}\left(-i\ket{1}\bra{1}+i\ket{1}\bra{0}+\ket{0}\bra{1}+\ket{0}\bra{0}\right)
\end{equation*}
\begin{equation}
H\left(\pi\right)=\frac{1}{\sqrt{2}}\left(\ket{1}\bra{1}-\ket{1}\bra{0}+\ket{0}\bra{1}+\ket{0}\bra{0}\right)
\label{Hgates}
\end{equation}

The action on $\ket{0}$ of each gate is:

\begin{equation}
H\left(\pi/2\right)\ket{0}=\frac{1}{\sqrt{2}}\left(\ket{0}+i\ket{1}\right) \qquad H\left(\pi\right)\ket{0}=\frac{1}{\sqrt{2}}\left(\ket{0}-\ket{1}\right)
\label{Hgatesres}
\end{equation}

Indeed, according to the parameterization Eq.\ref{qbparam}, the result on the left corresponds to $\phi=\pi/2$ and the one on the right to $\phi=\pi$. The density matrices are:

\begin{equation*}
H\left(\pi/2\right)\ket{0}\left(H\left(\pi/2\right)\ket{0}\right)^{\dagger}=\frac{1}{2}\left(\ket{1}\bra{1}+i\ket{1}\bra{0}-i\ket{0}\bra{1}+\ket{0}\bra{0}\right)
\end{equation*}
\begin{equation}
H\left(\pi\right)\ket{0}\left(H\left(\pi\right)\ket{0}\right)^{\dagger}=\frac{1}{2}\left(\ket{1}\bra{1}-\ket{1}\bra{0}-\ket{0}\bra{1}+\ket{0}\bra{0}\right)
\label{Hgatesresdm}
\end{equation}

In both cases, ro11=ro22=0.5, which means that the states $\ket{1}\equiv\ket{u_+}$ and $\ket{0}\equiv\ket{u_-}$ are equally populated. On the other hand, ro12r=0, ro12i=0.5 for $H\left(\pi/2\right)\ket{0}$; and ro12r=-0.5, ro12i=0 for $H\left(\pi\right)\ket{0}$. As seen in the input files qb.adata\_Hadamard\_rotdir90 and qb.adata\_Hadamard\_rotdir0, $H\left(\pi/2\right)$ is implemented with a rotation direction $\epsilon=90$º, while $H\left(\pi\right)$ requires $\epsilon=0$º. Since now we perform a 90º rotation -around an axis contained in the equatorial plane starting from $\ket{0}$-, the rotation time is $0.25/\Omega_{\text{g}}$=23.816/2=11.908 ns. After both rotations, the resulting density matrices shown in the output files qb.out\_Hadamard\_rotdir90 and qb.out\_Hadamard\_rotdir0 are, resp.,: ro11=ro22=0.50000, ro12r=0.00009, ro12i=0.50000; and ro11=ro22=0.50000, ro12r=-0.50000, ro12i=0.00009, both coinciding with the expected results once finite numerical precision is unconsidered. 

\bigbreak

The Phase gate $P(\gamma)$ consists in a rotation of a given angle $\gamma$ around the axis joining the two poles $\ket{1}$ and $\ket{0}$ of the Bloch sphere. The representation of this gate is:

\begin{equation}
P(\gamma)=\ket{0}\bra{0}+\text{e}^{i\gamma}\ket{1}\bra{1}
\label{Pgate}
\end{equation}

Note that this gate has no effect -up to global phase- on $\ket{1}$ and $\ket{0}$. That is why it will be more interesting to apply this gate on linear combinations of $\ket{1}$ and $\ket{0}$ to produce a non-trivial result. Let us consider $H\left(0\right)\ket{0}=\frac{1}{\sqrt{2}}\left(\ket{0}+\ket{1}\right)$ whose density matrix is ro11=ro22=0.5, ro12r=0.5, ro12i=0. At this point, the qubit state is located in the equatorial plane of the Bloch sphere. Hence, the action of $P(\gamma)$ on $H\left(0\right)\ket{0}$ is to perform a rotation of angle $\gamma$ around the above-mentioned axis while mantaining the qubit state in the equatorial plane. 
\bigbreak
We now consider the S gate $S=P(\gamma=\pm\pi/2)$ with $\gamma=-\pi/2$ and the T gate $T=P(\gamma=\pm\pi/4)$ with $\gamma=-\pi/4$ as particular implementations of $P(\gamma)$. Their representations are:

\begin{equation}
S=\ket{0}\bra{0}-i\ket{1}\bra{1} \qquad T=\ket{0}\bra{0}+\frac{1-i}{\sqrt{2}}\ket{1}\bra{1}
\label{STgates}
\end{equation}

The density matrices of the actions on $H\left(0\right)\ket{0}$ of the S and T gates are:

\begin{equation*}
S\left(H\left(0\right)\ket{0}\right)\left(S\left(H\left(0\right)\ket{0}\right)\right)^{\dagger}=\frac{1}{2}\left(\ket{1}\bra{1}-i\ket{1}\bra{0}+i\ket{0}\bra{1}+\ket{0}\bra{0}\right)
\end{equation*}
\begin{equation}
T\left(H\left(0\right)\ket{0}\right)\left(T\left(H\left(0\right)\ket{0}\right)\right)^{\dagger}=\frac{1}{2}\left(\ket{1}\bra{1}+\frac{1-i}{\sqrt{2}}\ket{1}\bra{0}+\frac{1+i}{\sqrt{2}}\ket{0}\bra{1}+\ket{0}\bra{0}\right)
\label{STgatesdm}
\end{equation}

In the first case, ro11=ro22=0.5, ro12r=0, ro12i=-0.5; in the second case, ro11=ro22=0.5, ro12r=-ro12i=$1/2\sqrt{2}\approx$0.3536. 
\bigbreak
Since in an EPR experiment $\va{B}$ and the wave-vector $\va{k}$ of $\va{B}_1$ are perpendicular, it is not possible to use the driving field $\va{B}_1$ to perform qubit rotations around the axis joining the poles $\ket{1}$ and $\ket{0}$ of the Bloch sphere. Instead, one take advantage of the Larmor precession around the $\va{B}$ direction such that after a proper waiting time the qubit state is rotated a given angle around the firstly mentioned axis. Hence, to implement the gates S and T starting from $H\left(0\right)\ket{0}$, we need to determine the two waiting times corresponding to the respective rotation angles $\gamma=-\pi/2$ and $\gamma=-\pi/4$. 
\bigbreak
The frequency of the Larmor precession -known as Larmor frequency- is determined by the qubit gap frequency $\omega_{+-}/2\pi=u_+-u_-$=0.3 cm$^{-1}$=8.99377 GHz (let us recall that $u_+$=0.6 cm$^{-1}$ and $u_-$=0.3 cm$^{-1}$ since ig=3 and ie=6). Hence, one period -a $2\pi$ rotation- takes $2\pi/\omega_{+-}$=0.111... ns. This means that a $\pi/2$ rotation is produced after a waiting time of $(2\pi/\omega_{+-})/4$=0.0278 ns, while the $\pi/4$ rotation is achieved after $(2\pi/\omega_{+-})/8$=0.0139 ns. This two times may seem too short to be experimentally implemented. In this case, to produce the same rotations, one can wait for these times plus $m2\pi/\omega_{+-}$, being $m>0$ a large enough integer to provide a sufficiently long waiting time but yet short enough compared to the relevant relaxation timescale. 
\bigbreak
Note that the strategy of implementing a proper waiting time allows taking the qubit to any point on the Bloch sphere. Indeed, by starting in the state $\ket{0}\equiv\ket{u_-}$, one first performs a rotation of the desired zenithal angle $\theta$ around an axis contained in the equatorial plane of the Bloch sphere. For instance, one could always use $\epsilon=0$. Then, one waits for a proper time until the Larmor precession produces the desired azimuthal angle $\phi$. The clock-wise or anti-clock-wise azimuthal rotation can be selected depending on which convention is used: either $\{\ket{1}\equiv\ket{u_+},\ket{0}\equiv\ket{u_-}\}$ (clock-wise) or $\{\ket{0}\equiv\ket{u_+},\ket{1}\equiv\ket{u_-}\}$ (anti-clock-wise) resp. We are employing the clock-wise convention and that is why the S and T gates produce rotations of angles $-\pi/2$ and $-\pi/4$ instead of $\pi/2$ and $\pi/4$. 
\bigbreak
In the input files qb.adata\_S\_gate and qb.adata\_T\_gate we first perform a 90º rotation -with the above-established time duration of 11.908 ns- around the axis $\epsilon=180$º to produce the qubit state $H\left(0\right)\ket{0}=\frac{1}{\sqrt{2}}\left(\ket{0}+\ket{1}\right)$. Then, free evolutions with fixed durations of 0.0278 ns and 0.0139 ns are allowed to respectively implement the $S=S(\gamma=-\pi/2)$ and $T=T(\gamma=-\pi/4)$ gates. 
\bigbreak
We can now check the evolution of these two step sequences in the output files qb.out\_S\_gate and qb.out\_T\_gate. After the 90º rotation around the axis $\epsilon=180$º, the density matrix becomes ro11=ro22=0.50000, ro12r=0.50000, ro12i=-0.00009, which compares to ro11=ro22=0.5, ro12r=0.5, ro12i=0 of the state $\frac{1}{\sqrt{2}}\left(\ket{0}+\ket{1}\right)$. Then, the 0.028 ns free evolution produces the density matrix ro11=ro22=0.50000, ro12r=-0.00017, ro12i=-0.50000; while the 0.0139 ns free evolution provides ro11=ro22=0.50000, ro12r=0.35346, ro12i=-0.35365. Both density matrices coincide -within the finite numerical precision- with the expected results ro11=ro22=0.5, ro12r=0, ro12i=-0.5; and ro11=ro22=0.5, ro12r=-ro12i=$1/2\sqrt{2}\approx$0.3536.

\bigbreak

The above calculations have been performed under ideal circumstances, namely a zero detuning $\delta=0$ and zero relaxation rates $\Gamma_{\text{a}}=\Gamma_{\text{ab}}+\Gamma_{\text{ab,add}}=0$, $\Gamma_{\text{e}}=\Gamma_{\text{em}}+\Gamma_{\text{em,add}}=0$, $\Gamma_{\text{m}}=\Gamma_{\text{mag}}+\Gamma_{\text{mag,add}}=0$. In the main text, by setting nm=0 (meaning $\Gamma_{\text{ab}}=\Gamma_{\text{em}}=0$) and $\Gamma_{\text{mag}}=0$, we study how the fidelity of the X, $H(\pi)$ and S gates is affected when increasing $\delta$, $\Gamma_{\text{em,add}}$ (with $\Gamma_{\text{ab,add}}$ given by Eq.\ref{detbalcond}), and $\Gamma_{\text{mag,add}}$, as well as when $\epsilon$ and rotation/waiting times are deviated from their ideal values (see folders detuning, epsilon, rotation\_time, waiting\_time, tmage, geme\_5K, geme\_50K, geme\_200K in SI/Examples/Gates). Given two density matrices $\rho$ and $\sigma$, their fidelity can be defined as:\cite{fidelity1}

\begin{equation}
F\left(\rho,\sigma\right)=\left(\text{Tr}\sqrt{\sqrt{\rho}\sigma\sqrt{\rho}}\right)^2
\label{fid1}
\end{equation}

For a positive semidefinite matrix M, $\sqrt{\text{M}}$ is its unique positive square root as provided by the spectral theorem. In case $\rho$ and $\sigma$ determine qubit states, Eq.\ref{fid1} takes the following expression,\cite{fidelity1,fidelity2} which is the one we use in the main text:

\begin{equation}
F\left(\rho,\sigma\right)=\text{Tr}\left(\rho\sigma\right)+2\sqrt{\text{det}(\rho)\text{det}(\sigma)}
\label{fid2}
\end{equation}

In the above expression, $\rho$ and $\sigma$ are 2x2 matrices. Note that if $\rho=\ket{\psi_{\rho}}\bra{\psi_{\rho}}$ (or $\sigma$) describes a pure state, $\text{det}(\rho)=0$ ($\text{det}(\sigma)=0$). Hence, $F\left(\rho,\sigma\right)=\text{Tr}\left(\rho\sigma\right)=\bra{\psi_{\rho}}\sigma\ket{\psi_{\rho}}$. If additionally $\sigma=\ket{\psi_{\sigma}}\bra{\psi_{\sigma}}$ is a pure state, one recovers the more known expression: $F\left(\rho,\sigma\right)=\left|\bra{\psi_{\rho}}\ket{\psi_{\sigma}}\right|^2$. 
\bigbreak
Let us write the matrix representations of $\rho$ and $\sigma$ in the ordered basis set $\{\ket{u_+},\ket{u_-}\}$ as follows:

\begin{equation}
\rho=\mqty(\rho_{11} & \rho_{12,r}+i\rho_{12,i} \\ \rho_{12,r}-i\rho_{12,i} & \rho_{22}) \qquad \sigma=\mqty(\sigma_{11} & \sigma_{12,r}+i\sigma_{12,i} \\ \sigma_{12,r}-i\sigma_{12,i} & \sigma_{22})
\label{fidrhosigma}
\end{equation}

To compute the fidelities, we use the expressions:

\begin{equation*}
\text{Tr}\left(\rho\sigma\right)=\rho_{11}\sigma_{11}+\rho_{22}\sigma_{22}+2\left(\rho_{12,r}\sigma_{12,r}+\rho_{12,i}\sigma_{12,i}\right)
\end{equation*}
\begin{equation*}
\text{det}\left(\rho\right)=\rho_{11}\rho_{22}-\rho_{12,r}^2-\rho_{12,i}^2
\end{equation*}
\begin{equation}
\text{det}\left(\sigma\right)=\sigma_{11}\sigma_{22}-\sigma_{12,r}^2-\sigma_{12,i}^2
\label{exprfid}
\end{equation}

\subsection{Pulse sequences: determination of \texorpdfstring{$T_1$, } \texorpdfstring{$T_m$} and Rabi oscillations}

Herein, we show some examples on how standard pulse sequences employed in EPR spectroscopy are implemented, such as the ones devoted to produce the so-called Rabi oscillations and to determine the spin relaxation times $T_1$ and $T_m$. We also use the input qb.ddata employed in the section above. For the Rabi oscillation example, we set nm=5 and hence include (i) the lines in qb.ddata where the spin-phonon coupling matrix elements are written and (ii) the made-up input qb.mdata -with 5 vibration normal modes- located in the same folder Examples. In the $T_1$ and $T_m$ examples, we set back nm=0. The input files qb.adata and output files qb.out, qb.mz.out, qb.mxy.out are found in the folder Examples/Pulse\_sequences. In the examples below, we employ as input parameters in qbithm.f: id=7, ig=3, ie=6, ic=1, nd=1, gabe=-1 (thus using Eq.\ref{detbalcond}), gfi=2.0, bcm=1.5 mT, firr=8.99377 GHz (intended zero detuning with the qubit gap), alp=0.0, nang=1, ro11=0, ro22=1, ro12r=0, ro12i=0 (i.e., the initial state is $\ket{u_-}$). Importantly, note that when coding QBithm we decided to define the longitudinal magnetization as $\text{M}_z(t\geq t_0)=\overline{\rho}_{22}-\overline{\rho}_{11}$ instead of $\text{M}_z(t\geq t_0)=\overline{\rho}_{11}-\overline{\rho}_{22}$ as it appears in Eq.\ref{longmag}. This re-definition just implies a global sign change and does not affect at all the shape -nor its properties such as the decay timescale- of the longitudinal magnetization. 

\bigbreak

The first example consists in the production of Rabi oscillations. Here, nm=5, temp=5 K, sfgw=1, top=5\%, ttp=5\%, geme=1 $\mu\text{s}^{-1}$, tmage=1 $\mu\text{s}^{-1}$, esta=0 $\mu\text{s}$, eend=0.5 $\mu\text{s}$, npe=201, nsa=1. The input file qb.adata\_Rabi only contains one step: a variable rotation with rotation direction $\epsilon=0$º. Since the rotation is time-variable, the (dummy) duration time in the second column is not employed. The output file to read is qb.mz.out\_Rabi (longitudinal magnetization) -qb.mxy.out is not of interest in this example- and is plotted from esta to eend in Fig.\ref{Rabi_ejemplo}. As long as the generalized Rabi frequency $\Omega_{\text{g}}$ is greater enough than the relaxation rate, some Rabi oscillations can still be observed before they decay and vanish with time. Since nm=5$>$0, the output qb.out\_Rabi contains some additional information. In particular, one can check which normal modes and pairs of normal modes have a contribution equal or above top=5\% and ttp=5\% in the relaxation rates of the one-phonon and two-phonon processes resp., being emdi=L-Direct, emanst=anti-Stokes, emsp=L-Spont., abdi=R-Direct, abstok=Stokes, absp=R-Spont. In the section Time evolution of density matrix, we find as many magnetization lines (runs of the step sequence in qb.adata\_Rabi) as the numerical value of npe. In each line, we find the same step sequence but with a different value of the duration time -here, nutation time- between esta and eend and the resulting density matrix. Note that ro11 -here, representing the population of $\ket{u_+}$ at each time- decreases while ro22 -here, representing the population of $\ket{u_-}$ at each time- increases, and vice-versa, with time. For a long enough time, one would see the attainment of thermal equilibrium -oscillations already vanished- with ro11 and ro22 similar to 0.5 being ro11 a bit below ro22 since ro11 describes the population of $\ket{u_+}$ which is higher in energy than $\ket{u_-}$ whose population is given by ro22. Hence, the oscillation tends with time to a constant value a bit above zero according to the definition $\text{M}_z(t\geq t_0)=\overline{\rho}_{22}-\overline{\rho}_{11}$ stated earlier.

\begin{figure}[H]
    \centering
    \includegraphics[scale=1.0]{Rabi.jpg}
   \caption{Example of production of (damped) Rabi oscillations.}
   \label{Rabi_ejemplo}
\end{figure}

\bigbreak

In the second example, we determine the $T_1$ spin-lattice relaxation time with nm=0, geme=1 $\mu\text{s}^{-1}$, tmage=1 $\mu\text{s}^{-1}$, esta=0 $\mu\text{s}$, eend=2.0 $\mu\text{s}$, npe=101, nsa=2. The input qb.adata\_T1 is composed of two steps, namely, a $\pi$ rotation around the axis $\epsilon=0$º with a fixed duration time of 23.816 ns, and a variable free evolution (second and third columns of its line with dummy values are not employed). Again, the magnetization $\text{M}$ of interest is the longitudinal one $\text{M}_z$ found in the output qb.mz.out\_T1 -qb.mxy.out is not of interest either-. Note that since the state after the $\pi$ rotation is $\ket{\psi(t=t_{\pi})}=\ket{u_+}$, ideally $\text{M}_z(t=t_{\pi})=\overline{\rho}_{22}-\overline{\rho}_{11}=0-1=-1<0$ if there was no relaxation (since relaxation is active, $\text{M}_z(t=t_{\pi})$ is a bit above $-1$), and then $\text{M}_z$ increases with further time $t>t_{\pi}$ until finding the corresponding equilibrium value for a long enough time. In particular, $\overline{\rho}_{11}$ decreases and $\overline{\rho}_{22}$ increases -while keeping $\overline{\rho}_{11}+\overline{\rho}_{22}$=1 all time- since $\overline{\rho}_{11}$ represents the population of the highest energy state $\ket{u_+}$ and $\overline{\rho}_{22}$ represents the population of the lowest energy state $\ket{u_-}$. For the same reason, the equilibrium value of $\overline{\rho}_{22}$ will be a bit above the one of $\overline{\rho}_{11}$ and, since $\text{M}_z=\overline{\rho}_{22}-\overline{\rho}_{11}$, the equilibrium $\text{M}_z$ value will lie a bit above zero. The equilibrium values of $\overline{\rho}_{11}$ and $\overline{\rho}_{22}$ are more or less close to 0.5 depending on how large the gap $u_+-u_-$ is. The less large it is, the closer they are to 0.5. In Fig.\ref{T1_ejemplo} we plot, against the duration time $\text{esta}\leq t\leq\text{eend}$ of the only step with variable duration time, the opposite magnetization $-\text{M}_z(t)$ since a decaying magnetization is easier and faster to fit (with no affection on the value of $T_1$ to be extracted) than an increasing one. The following model in the fitting procedure is employed (we fixed $x=1$):

\begin{equation}
\text{M}_z(t)=\text{M}_{z,0}+a\text{exp}\left(-\left(t/T_1\right)^x\right)
\label{modelexpdecT1}
\end{equation}

The output qb.out\_T1 also contains as many lines as the value of npe (runs of the step sequence in qb.adata\_T1). In each run, one sees how the duration time of the second step -the variable free evolution- is increased starting from esta=0. Note that the $\pi$ pulse -first step of the sequence- inverts the spin population from $\overline{\rho}_{11}=0$ and $\overline{\rho}_{22}=1$ to $\overline{\rho}_{11}\approx 1$ and $\overline{\rho}_{22}\approx 0$. Then, the longer the variable free evolution time is -second step of the sequence-, the lower $\overline{\rho}_{11}$ is and the higher $\overline{\rho}_{22}$ is until, for a long enough time, $\overline{\rho}_{11}\approx\overline{\rho}_{22}\approx 0.5$ and $\text{M}_z(t)=\overline{\rho}_{11}-\overline{\rho}_{22}\approx 0$, with $\overline{\rho}_{11}$ being a bit lower than $\overline{\rho}_{22}$ as already explained (attainment of thermal equilibrium). 

\begin{figure}[H]
    \centering
    \includegraphics[scale=1.0]{T1.jpg}
   \caption{Example of determination of $T_1$.}
   \label{T1_ejemplo}
\end{figure}

\bigbreak

The third and last example is devoted to determine the phase memory time $T_m$ with nm=0, geme=1 $\mu\text{s}^{-1}$, tmage=1 $\mu\text{s}^{-1}$, esta=0 $\mu\text{s}$, eend=1.0 $\mu\text{s}$, npe=51, nsa=4. The input file qb.adata\_Tm contains now four steps, namely, a $\pi/2$ rotation of 11.908 ns around the axis $\epsilon=0$º, a first variable free evolution, a $\pi$ rotation (refocusing pulse) of 2$\cdot$11.908=23.816 ns around the same axis $\epsilon=0$º, and a second variable free evolution. Now, the magnetization of interest is the in-plane magnetization found in qb.mxy.out\_Tm -qb.mz.out is ignored- and plotted in Fig.\ref{Tm_ejemplo}. At this point, one important consideration must be noted. In qb.mxy.out\_Tm, each magnetization value is the result of a run of the step sequence in qb.adata\_Tm where the two free evolution steps are activated for the same duration time $\tau$ swept from esta to eend. Hence, the total free evolution time is $2\tau$ in this particular sequence. The important thing to note is that the associated time to each magnetization in qb.mxy.out\_Tm is $2\tau$ and not $\tau$. The plot and fit in Fig.\ref{Tm_ejemplo} is against $2\tau$ which is right the column "time" in qb.mxy.out\_Tm so the user does not have to do any modification, just to directly plot and fit the two columns of qb.mxy.out\_Tm. 
\bigbreak
In general, if $m>1$ is the number of steps in qb.adata with a variable duration time (thus, these steps have numerical code 0 and/or 3 and are all of them activated for the same time $\tau$ for each given $\tau$ value), both the longitudinal and in-plane magnetizations are written in qb.mz.out and qb.mxy.out resp. against $m\tau$ -the total free evolution time- being $\tau$ the time that is swept from esta to eend. Hence, when plotting and fitting one just directly use the "time" and "magnetization" columns in qb.mz.out and qb.mxy.out without any further modification. The input files qb.adata to produce Rabi oscillations and to determine $T_1$ only contain one step with a variable duration time. That is why the construction of qb.mz.out and its plot and fit is directly against the time that is swept from esta to eend. To fit the plot in Fig.\ref{Tm_ejemplo}, we use the following model, where $t=2\tau$ and $\text{M}_{xy}(t)$ are directly the columns "time" and "magnetization" in qb.mxy.out\_Tm (we fixed $x=1$):

\begin{equation}
\text{M}_{xy}(t)=\text{M}_{xy,0}+a\text{exp}\left(-\left(t/T_m\right)^x\right)
\label{modelexpdecTm}
\end{equation}

Actually, each magnetization value that appears in each line of qb.mz.out and qb.mxy.out is the magnetization computed right at the final time of the algorithm/pulse sequence. The point is that the sum $\Sigma$ over the time taken by each step with a fixed duration is negligible in comparison with the variable $m\tau$ since the initial value of $\tau$ is always a large enough positive value such that $\Sigma\ll m\tau$ at each $\tau$ value. That is why both calculated and experimental magnetization are always written and plotted against $t=m\tau>0$ instead of $m\tau+\Sigma>0$. The reason of using an initial positive value for $\tau$ is due to the fact that EPR spectrometers do not allow starting at $\tau=0$. Instead, when performing calculations, nothing prevents us from starting at $\tau=0$ as we do. In this case, even if $m\tau$ is not negligible respect to $\Sigma$ when $\tau$ is still close enough to zero, we can in any case neglect $\Sigma$ since the timescale where $m\tau$ lies is usually longer than $\Sigma$. If one still wants to plot the magnetization against the total runtime $\Sigma+m\tau$ when starting at $\tau=0$, the time $t$ in Eq.\ref{modelexpdecT1} and Eq.\ref{modelexpdecTm} must be replaced by $t-\Sigma$ as the smallest runtime is $\Sigma$ (which happens when $\tau=0$). Since $\Sigma$ is just a constant that redefines the time origin, one would obtain the same values for the relaxation times as when plotting against $m\tau$ and using Eq.\ref{modelexpdecT1} and Eq.\ref{modelexpdecTm}. In this last case when the magnetization is plotted against $m\tau$, the fact that $\Sigma>0$ manifests itself in the form of a magnetization value that is strictly below 1 at $\tau=0$ as seen in Fig.\ref{T1_ejemplo} and Fig.\ref{Tm_ejemplo}. Indeed, despite $\Sigma>0$ is short, it is long enough when $\tau=0$ for relaxation and experimental imperfections to take place thus producing the said value below 1 (or above -1 for some cases regarding $\text{M}_z$ depending on the initial condition of the algorithm/pulse sequence). On the other hand, the simplified sequence to produce Rabi oscillations does not contain any step with a fixed duration. That is why Rabi oscillations produced with this sequence -as those shown in Fig.\ref{Rabi_ejemplo}- start with a magnetization value equal to 1 at $\tau=0$. The starting magnetization would start from a value below 1 (or above -1 depending on the initial condition) as soon as the simplified sequence is followed by the fixed-duration Hahn sequence employed for the experimental production of Rabi oscillations.
\medbreak
As can be seen in the output qb.out\_Tm, the $\pi/2$ rotation tips the magnetization by taking the qubit state from the pole $\ket{u_-}$ to a state contained in the equatorial plane of the Bloch sphere with $\overline{\rho}_{11}\approx\overline{\rho}_{22}\approx 0.5$, $\overline{\rho}_{12,r}\approx -0.5$, $\overline{\rho}_{12,i}\approx 0$. Once the spin population is inverted $\overline{\rho}_{11}\leftrightarrow\overline{\rho}_{22}$ after implementing the $\pi$ pulse, for a long enough free evolution time the coherence terms $\overline{\rho}_{12,r}$ and $\overline{\rho}_{12,i}$ of the final density matrix (fourth step) are close to zero, which means that the in-plane magnetization has vanished and so the qubit coherence. 

\begin{figure}[H]
    \centering
    \includegraphics[scale=1.0]{Tm.jpg}
   \caption{Example of determination of $T_m$.}
   \label{Tm_ejemplo}
\end{figure}

We have employed the same values of geme and tmage with nm=0 in the determination of both $T_1$ and $T_m$ but we have obtained $T_1$=0.343 $\mu\text{s}$ $<$ $T_m$= 0.512 $\mu\text{s}$. Hence, the fact that experimentally one always observe $T_1>T_m$ indicates that calculations performed with our theoretical development will require rather different sets of values for the relaxation rates $\Gamma_{\text{ab}}$, $\Gamma_{\text{em}}$, $\Gamma_{\text{mag}}$, $\Gamma_{\text{ab,add}}$, $\Gamma_{\text{em,add}}$, $\Gamma_{\text{mag,add}}$ when determining $T_1$ and $T_m$. This fact should be reasonable since after all they both are determined by means of rather different experiments where all the processes that affect the spin dynamics do not have to be necessarily the same. Some of them may not be covered by the \textit{ab initio} rates $\Gamma_{\text{ab}}$, $\Gamma_{\text{em}}$, $\Gamma_{\text{mag}}$, and that is why the additional rates $\Gamma_{\text{ab,add}}$, $\Gamma_{\text{em,add}}$, $\Gamma_{\text{mag,add}}$ may be needed. The above-mentioned difference is what would effectively manifest itself in the form of different values of geme and tmage when it comes to reproduce the experimental fact $T_m\leq T_1$. In general, each specific implementation of a pulse sequence and one-qubit algorithm would require a particular set of values for the said six relaxation rates in order to produce a satisfactory match with the experimental data (see main text for further discussion). Last, a fourth example could be the production of an ESE(electron spin echo)-detected spectrum. The sequence is composed of the following four steps: $\pi/2-\tau-\pi-\tau$ but with a fixed free evolution time $\tau$. The magnitude that is swept is $|\va{B}|$ and then one plots the in-plane magnetization against $|\va{B}|$.

\newpage

\section{Case studies}

In this section, we describe the input parameters employed in the study of the proposed case studies as well as the results produced when calculating the Rabi oscillations and the spin relaxation times $T_1$ and $T_m$. Note that in all cases we use gabe=-1, alp=0, gfi=2 (V$^{4+}$ and Cu$^{2+}$ ions), ro11=0, ro22=1, ro12r=0, ro12i=0 ($\ket{\psi(t=0)}=\ket{u_-}$), and nm=0. The latter means that (i) sfgw, top, ttp, qb.mdata will not be read nor employed at all, and (ii) QBithm sets $\Gamma_{\text{ab}}=\Gamma_{\text{em}}=0$. We will employ $\Gamma_{\text{em,add}}$ (namely geme) as a free fitting parameter with $\Gamma_{\text{ab,add}}$ given by Eq.\ref{detbalcond}. Unless indicated something different, we will also use tmage=0 -hence, $\Gamma_{\text{mag,add}}=0$- while $\Gamma_{\text{mag}}$ -appearing below 1/Tn+1/Te(=$>$0) in qb.ddata- is calculated by employing an adapted version of the software package SIMPRE (recall that this adapted version actually makes the whole input file qb.ddata).\cite{simpre2.0,espesp2019} Further details on the calculation of $\Gamma_{\text{mag}}=1/T_n+1/T_e$ for each case study will be published elsewhere.
\bigbreak
In the calculation of $\Gamma_{\text{mag}}$, the concentrations $y$ of the electron spin bath are: $y=0.100$ for [(VO)$_y$(MoO)$_{1-y}$(dmit)$_2$]$^{2-}$, $y=0.060$ for [V$_y$Ti$_{1-y}$(dmit)$_3$]$^{2-}$, $y=0.001$ for (VO)$_y$(TiO)$_{1-y}$Pc, and $y=0.00001$, $y=0.0001$, $y=0.003$ for [Cu$_y$Ni$_{1-y}$(mnt)$_2$]$^{2-}$. To make this bath, one first selects a given molecule in the X-ray crystallographic structure. Then, all the copies of the given molecule that are present in the crystalline structure below a given maximum distance from it are replaced by its diamagnetic version with probability $1-y$. The electron spin bath is composed of all the copies -whose positions are employed to compute $T_e$- which have not been replaced. On the other hand, the nuclear spin bath is composed of the magnetic nuclei -whose positions are now employed to compute $T_n$- of all the atoms that are found in the crystallographic structure below a given maximum distance from the position of the above-selected molecule. The maximum distances $d_{\text{max}}$ are increased until the computed $T_e$ and $T_n$ reach a converged value. These converged values are found in the ranges $d_{\text{max}}$=500-700 \AA$\text{}$ for the electron spin bath and $d_{\text{max}}$=50-60 \AA$\text{}$ for the nuclear spin bath. 

\subsection{[VO(dmit)\texorpdfstring{$_2$}{}]\texorpdfstring{$^{2-}$}{} and [V(dmit)\texorpdfstring{$_3$}{}]\texorpdfstring{$^{2-}$}{}}

We first diagonalized the Hamiltonian Eq.\ref{spinH} by employing the SIMPRE package,\cite{simpre2.0} with $J=S=1/2$, $I=7/2$, $B_k^q\equiv 0$, $|\va{B}|=345$ mT, $P=0$, and the parameters provided in \cite{keyrole}. For [VO(dmit)$_2$]$^{2-}$, these are $g_x=1.986$, $g_y=1.988$, $g_z=1.970$, $A_x=138$ MHz, $A_y=128$ MHz, $A_z=413$ MHz. For [V(dmit)$_3$]$^{2-}$, $g_x=1.961$, $g_y=1.971$, $g_z=1.985$, $A_x=299$ MHz, $A_y=230$ MHz, $A_z=40$ MHz. From the $(2J+1)(2I+1)=16$ spin eigenstates, the qubit is given by the two states $\ket{m_J=-1/2,m_I=-1/2}$ and $\ket{m_J=+1/2,m_I=-1/2}$, which correspond to ig=5 and ie=12. For both molecules, we employ id=16, ic=1, nd=1, nang=1.
\bigbreak
The Y axis direction has the largest $g$ factor for [VO(dmit)$_2$]$^{2-}$ and hence provides the largest qubit gap with a value of 9.611 GHz. This value is the closest one to the experimentally-employed irradiation frequency 9.70 GHz. That is why we use firr=9.611 GHz and 90º as azimuthal and zenithal angles (which correspond to the Y axis direction) in qb.ddata\_nd\_1\_nopb\_sb (see folder VOdmit2 and below for explanation of setting firr=9.611 GHz). On the other hand, in the case of [V(dmit)$_3$]$^{2-}$, it is the Z axis direction the one with the largest $g$ factor thus providing the largest qubit gap with a value of 9.622 GHz. Since this value is the closest one to the experimental irradiation frequency 9.70 GHz, we now use firr=9.622 GHz and 0º as azimuthal and zenithal angles (which correspond to the Z axis direction) in qb.ddata\_nd\_1\_nopb\_sb (see folder Vdmit3 and also below for explanation of using firr=9.622 GHz) for this molecule. 
\bigbreak
The determination of $|\va{B}_1|$, namely bcm, is performed by making use of the experimental Rabi oscillations found in figures 6 ([VO(dmit)$_2$]$^{2-}$) and S22 ([V(dmit)$_3$]$^{2-}$) of \cite{keyrole}. Let us start with [VO(dmit)$_2$]$^{2-}$ and the Figure 3 of our main text, where we plot both our calculated Rabi oscillations (solid line, see output files qb.mz.out in folder VOdmit2/Figure6/only\_with\_G\_emadd) and the experimental ones found in Figure 6 of \cite{keyrole}. To proceed with these calculations, we used qb.adata\_Rabi (see folder VOdmit2), set temp=293 K, tmage=0, and started with the experimental oscillation determined at 9 dB. We varied bcm until the period of the calculated oscillation (namely, the calculated generalized Rabi frequency) matched as best as possible that of the experimental oscillation (namely, the experimental generalized Rabi frequency provided in the Fourier transform spectrum of the said experimental oscillation). This process gave bcm=5.020 mT. Then, we adjusted geme to reproduce the decay with time of the experimental oscillation thus providing geme=26 $\mu$s$^{-1}$. Note that a continuum of frequencies may contribute to the experimental Rabi oscillation due to the non-negligible width of its Fourier transform observed in Figure 6b of \cite{keyrole}. Since the oscillation produced by our theoretical model just contain a single frequency, one could sometimes expect a non satisfactory match in the whole time range between the experimental and the calculated oscillation. If this should happen in our case studies, we prioritize to get a better match in the short time range. We also note that, generally, Rabi oscillations do not necessarily tend to a zero magnetization value for long times. This can be understood by recalling that the longitudinal magnetization $\langle\text{M}_z\rangle$ -the one we employ to produce Rabi oscillations with the simplified sequence- is proportional to the difference between $\overline{\rho}_{11}$ and $\overline{\rho}_{22}$. The fact that the two qubit states are non-degenerate makes $\overline{\rho}_{11}$ and $\overline{\rho}_{22}$ not get the same equilibrium values $\overline{\rho}_{11}=\overline{\rho}_{22}=0.5$ thus preventing $\langle\text{M}_z\rangle$ from tending to zero at long times. Nonetheless, this non-zero limit value is unimportant as one can set the magnetization origin to any magnetization value by performing the appropriate vertical displacement of the Rabi oscillation. 
\bigbreak
The bcm values for the rest of oscillations are now totally determined by making use of the Figure 6c in \cite{keyrole}. Indeed, in this figure, the relative intensity $B_1=1$ a.u. corresponds to the highest attenuation of 9 dB. The larger $B_1$ values accordingly correspond to the smaller attenuations. The absolute $|\va{B}_1|$ value for each given attenuation is obtained by multiplying bcm=5.020 mT by the corresponding $B_1$ value. The $B_1$ values for 7, 3, 2, 0 dB are 1.25, 2, 2.25, 2.8 a.u., which provide bcm=6.275, 10.040, 11.295, 14.056 mT, respectively. The remaining task to perform with these fixed bcm values is just to adjust geme until getting the best match of the experimental time decay. This is accomplished with geme=30 (7 dB), 40 (3 dB), 44 (2 dB), 48 (0 dB) $\mu$s$^{-1}$. 
\bigbreak
As an additional test, we repeated the above calculations (see the output files qb.mz.out in folder VOdmit2/Figure6 /only\_with\_G\_magadd) with the same bcm values but now by keeping geme=0 and using tmage as a free fitting parameter to adjust the time decay of the experimental oscillations. The results are shown in the figure below. The experiment-theory agreement is qualitatively the same as when geme was employed as a free fitting parameter with tmage=0. If anything, the amplitude of the experimental oscillations at intermediate times is kind of overestimated. In this particular calculation, note that, either with geme or with tmage as a free fitting parameter, there always exists the contribution 1/Tn+1/Te to relaxation of the spin bath given in the input qb.ddata\_nd\_1\_nopb\_sb. Hence, even with tmage=0, the case where geme is used as a free fitting parameter is more realistic because in this situation both the vibration and the spin bath are considered as a relaxation source. Instead, when employing tmage as a free fitting parameter, since qb.ddata\_nd\_1\_nopb\_sb does not contain any information about the vibration bath and geme=0, the spin bath is unrealistically considered as the only relaxation source in this case, and that is why we need larger values in tmage -compared to the ones employed for geme- to reproduce the same time decay in each experimental Rabi oscillation. 

\begin{figure}[H]
    \centering
    \includegraphics[scale=0.70]{Rabi_VOdmit2_solo_Gmadd.jpg}
   \caption{Experimental (Figure 6 in \cite{keyrole}) and calculated (solid line) Rabi oscillations of [VO(dmit)$_2$]$^{2-}$ where geme=0, bcm values are the ones employed in Figure 3 of main text, and tmage is used as a free fitting parameter to adjust the experimental time decay.}
   \label{RabiVOdmit2Gmadd}
\end{figure}

We now use the calculated oscillation at 9 dB attenuation (appearing in Figure 3 of main text and being the red curve in the figure below) as a reference to perform extra tests (see main text for further discussion). In particular, we test the effect of detuning and also compute the oscillation as an average magnetization by considering a set of equally-weighted $\va{B}$ directions according to Eq.\ref{weiuni} and Eq.\ref{magavuni} as if it was the case of a powder sample or a frozen solution. For the detuning test, we use firr=9.700 GHz which coincides with the experimentally-employed irradiation frequency, hence giving a detuning of 9.700-9.611 GHz = 0.089 GHz. On the other hand, the average magnetization is computed by using the set of 110 Lebedev directions found in the input file qb.ddata\_nd\_110\_nopb\_nosb (see folder VOdmit2) together with nang=72. Note that this input contains a zero magnetic rate 1/Tn+1/Te for every single $\va{B}$ direction. The fact of cancelling now 1/Tn+1/Te does not preclude the comparison with the previous calculation performed with the input qb.ddata\_nd\_1\_nopb\_sb -where 1/Tn+1/Te=0.04 $\mu$s$^{-1}$ is non-zero- since 1/Tn+1/Te is ca. 0.04 $\mu$s$^{-1}$ as seen in qb.ddata\_nd\_1\_nopb\_sb, which is a value negligible respect to the employed value geme=26 $\mu$s$^{-1}$ for the four curves in the figure below. The reason of not calculating 1/Tn+1/Te for each $\va{B}$ direction in qb.ddata\_nd\_110\_nopb\_nosb is that of decreasing the computational time required by SIMPRE when making the input qb.ddata. Since the electron spin bath of [VO(dmit)$_2$]$^{2-}$ is of a high enough concentration ($y=10\%$), the mentioned time would be rather long. 

\bigbreak

As can be concluded from the figure below, the proper reproduction of the experimental Rabi oscillations crucially requires the participation of only those $\va{B}$ directions that give a small enough detuning $\delta=\omega_{+-}-\omega_{\text{MW}}$ to make the employed $\omega_{\text{MW}}$ resonate with $\omega_{+-}$ of each one of the mentioned particular $\va{B}$ directions. This means that the weight function Eq.\ref{weiuni} -which provides the same weight to all $\va{B}$ directions and is coded in QBithm- should not be used. A more realistic weight function would weigh each $\va{B}$ direction according to the value of $\delta$ produced in that $\va{B}$ direction in such a way that a small $\delta$ value should result in a large weight close to 1. Instead, a large $\delta$ value should result in a small weight close to 0. An option could be to use the resonance probability -which is actually a function of $\delta$ and behaves as required- as a weight function. As explained in main text, as an intermediate alternative, we drop all $\va{B}$ directions except those ones where $|\delta|$ attains the minimum value $|\delta|_{\text{min}}$. Even in these particular $\va{B}$ directions, it could happen that $|\delta|_{\text{min}}$ is still too large for a significant enough resonance. Despite this fact, one can still conduct a faithful driving of the spin in an experiment since $\omega_{\text{MW}}$ and $\omega_{+-}$ are somewhat widened. The role of the widenings in our theoretical model can be effectively reproduced just by setting $\omega_{\text{MW}}$ (firr in qbithm.f) with the value of $\omega_{+-}$ that happens in the said particular $\va{B}$ directions. This is the solution that we have adopted in our case studies where, due to their anisotropy, there is only one $\va{B}$ direction with $|\delta|=|\delta|_{\text{min}}$. In case of existing more $\va{B}$ directions with $|\delta|=|\delta|_{\text{min}}$ (systems with some isotropy), one would have to integrate the relevant magnetization only over these particular $\va{B}$ directions as it may happen in powder samples and frozen solutions. In this case, beside also setting $\omega_{\text{MW}}$ with the $\omega_{+-}$ value of those particular $\va{B}$ directions, one can give the same weight to all these particular $\va{B}$ directions. 

\begin{figure}[H]
    \centering
    \includegraphics[scale=0.80]{VOdmit2_EstudioRabi.jpg}
   \caption{Calculated Rabi oscillation of [VO(dmit)$_2$]$^{2-}$ at 9 dB attenuation (red curve and appearing in Figure 3 of main text) as a reference to (i) test the effect of detuning and (ii) re-compute the same oscillation as an average magnetization of 110 Lebedev $\va{B}$ directions and nang=72.}
   \label{VOdmit2EstudioRabi}
\end{figure}

Following now with [V(dmit)$_3$]$^{2-}$, its experimental Rabi oscillations (Figure S22 of \cite{keyrole}) along with the corrresponding calculated ones (solid line, see output files qb.mz.out in folder Vdmit3/FigureS22) are shown in Figure 3 of main text. The procedure with calculations is similar as for [VO(dmit)$_2$]$^{2-}$. We used qb.adata\_Rabi (see folder Vdmit3), set tmage=0 and now temp=60 K, and started with the Rabi oscillation measured at 15 dB attenuation. The best match for the period of the experimental oscillation was achieved for bcm=0.35 mT, while the amplitude decay with time was reproduced with geme=1.70 $\mu$s$^{-1}$. The bcm values for the remaining oscillations are determined as did above by employing now Figure S22c in \cite{keyrole}. The $B_1$ values for 11, 5, 1 dB are 1.6, 3.2, 5 a.u. thus giving bcm=0.56, 1.12, 1.75 mT, respectively. With these bcm values, the geme values that provide the closest time decays to the experimental ones are 1.80, 2.70, 3.30 $\mu$s$^{-1}$. 
\bigbreak
Note that at 5 dB attenuation and from 200 ns one observes the onset of the so-called Hartmann-Hahn condition in which the spin nutation gives rise to a long-lived oscillation. According to the Fourier transform in Figure S22b of \cite{keyrole}, the frequency of this oscillation is the proton $^1$H Larmor frequency which, at the working static magnetic field $|\va{B}|$=345 mT, is 14.69 MHz. Recall that the proton $^1$H gyromagnetic factor is 42.58 MHz/T. If we take 68.1 ns as a period for the long-lived oscillation, the corresponding frequency is 14.68 MHz which matches the previous calculation. A key ingredient to meet the Hartmann-Hahn condition is the presence, in this case, of a close proton $^1$H that couples to the [V(dmit)$_3$]$^{2-}$ molecular spin. Under this situation, the other condition for the Hartmann-Hahn condition to be met is to drive the spin with a generalized Rabi frequency similar to the proton $^1$H Larmor frequency at the given $|\va{B}|$. Our theoretical model does not consider any proton explicitly in the molecular spin Hamiltonian Eq.\ref{expHeff} (but rather as a part of the spin bath). That is why it is unable to incorporate the above-mentioned condition and reproduce the long-lived oscillation. 
\bigbreak
We now focus on reproducing the experimental spin relaxation times $T_m^{\text{exp}}$ of [VO(dmit)$_2$]$^{2-}$ and [V(dmit)$_3$]$^{2-}$ found in Tables S2 and S8 of \cite{keyrole}. In particular, we consider the experimental measures performed with a $t_{\pi/2}$ and $t_{\pi}$ pulse length of 16 and 32 ns, respectively. This means that $t_{2\pi}$=64 ns. We use qb.adata\_Tm (see folders VOdmit2 and Vdmit3), set tmage=0, and the first step prior to perform the calculations is to determine which bcm value corresponds to the given pulse lengths both for [VO(dmit)$_2$]$^{2-}$ and for [V(dmit)$_3$]$^{2-}$. This task is simple once we have found the bcm value of at least one experimental Rabi oscillation in each molecule. Indeed, let us consider the reference Rabi oscillation of [VO(dmit)$_2$]$^{2-}$ determined at 9 dB attenuation where bcm-ref=5.020 mT. The point now is to estimate the period $T$ of this oscillation (which corresponds to a full 2$\pi$ spin nutation). As mentioned above, an experimental Rabi oscillation may be composed of several frequencies (hence, several periods). Either this happens or not, we just focus on the very first oscillation starting from zero time. This method provides a period of ca. $T$=15 ns. Let us recall that $|\va{B}_1|$ and $T$ are inversely proportional such that $|\va{B}_1|\cdot T$=constant. Hence, in particular, bcm-ref·T = bcm·$t_{2\pi}$ thus resulting in bcm=1.177 mT. In the case of [V(dmit)$_3$]$^{2-}$, we use as a reference the one measured at 15 dB attenuation, where bcm-ref=0.35 mT and we estimate $T$=200 ns for the very first oscillation. These data provide bcm=1.094 mT. 
\bigbreak
All in all, with bcm=1.177 mT for [VO(dmit)$_2$]$^{2-}$ and bcm=1.094 mT for [V(dmit)$_3$]$^{2-}$, for each given experimental temperature we use geme as a fitting parameter to match $T_m^{\text{exp}}$. Specifically, for each value of geme explored (until finding the satisfactory one), we plot the magnetization in qb.mxy.out against the time therein (see folders VOdmit2/TableS2\_Tm\_16\_32ns and Vdmit3/TableS8\_Tm\_16\_32ns) and fit the resulting magnetization curve to the following mathematical model consisting in a stretched monoexponential decaying function with x as a stretch factor. The results are shown in Table \ref{TmVOdmit2Vdmit3}. 

\begin{equation}
y=y_0+a\cdot\text{exp}\left(-\left(t/T_m^{\text{calc}}\right)^\text{x}\right)
\label{Tmmodel}
\end{equation}

\begin{table}[H]
    \centering
    \begin{tabular}{c | c c c | c c c}
 & &[VO(dmit)$_2$]$^{2-}$& & &[V(dmit)$_3$]$^{2-}$& \\\hline
T (K) & $T_m^{\text{exp}}$ ($\mu$s) & $T_m^{\text{calc}}$ ($\mu$s)/x & geme ($\mu$s$^{-1}$) & $T_m^{\text{exp}}$ ($\mu$s) & $T_m^{\text{calc}}$ ($\mu$s)/x & geme ($\mu$s$^{-1}$) \\\hline
4.5 & 1.270 & 1.270/0.948 & 0.752 & 0.542 & 0.542/1.0048 & 1.914 \\
10 & 1.368 & 1.369/0.948 & 0.678 & 0.543 & 0.543/1.0045 & 1.860\\
25 & 1.317 & 1.316/0.948 & 0.696 & 0.539 & 0.539/1.0043 & 1.848\\
40 & 1.279 & 1.279/0.948 & 0.714 & 0.495 & 0.495/1.0044 & 2.008\\
60 & 1.206 & 1.207/0.947 & 0.756 & 0.402 & 0.402/1.0048 & 2.476\\
100 & 1.042 & 1.041/0.945 & 0.876 & 0.231 & 0.231/1.0081 & 4.336\\
150 & 0.825 & 0.825/0.940 & 1.102 & 0.144 & 0.144/1.0163 & 7.008\\
294 & 0.491 & 0.491/0.917 & 1.818 & - & - & - 
    \end{tabular}
    \caption{Experimental and calculated spin relaxation time $T_m$ and stretching factor x of [VO(dmit)$_2$]$^{2-}$ and [V(dmit)$_3$]$^{2-}$ -with $t_{\pi/2}$=16 ns and $t_{\pi}$=32 ns- as a function of temperature. Experimental data is extracted from Tables S2 and S8 in \cite{keyrole}.}
    \label{TmVOdmit2Vdmit3}
\end{table}

We also performed similar extra tests by taking the calculated $T_m$ magnetization curve at T = 294 K of [VO(dmit)$_2$]$^{2-}$ as a reference (see Table \ref{TmVOdmit2Vdmit3} and black curve in the figure below). Here, we also test the effect of detuning as well as the computation of the $T_m$ magnetization curve as an average magnetization by considering the set of 110 Lebedev $\va{B}$ equally-weighted directions (same input qb.ddata\_nd\_110\_nopb\_nosb in folder VOdmit2) with nang=72 as if it was the case of a powder sample or a frozen solution (see folder VOdmit2/TableS2\_Tm\_16\_32ns and blue, green, and red curves in the figure below which were also fitted with the model Eq.\ref{Tmmodel} to determine $T_m^{\text{calc}}$ and x). Let us recall again that the input qb.ddata\_nd\_110\_nopb\_nosb contains a zero magnetic rate 1/Tn+1/Te for every single $\va{B}$ direction. This does not imply any issue either since this rate, which is ca. 0.04 $\mu$s$^{-1}$ as seen in qb.ddata\_nd\_1\_nopb\_sb (folder VOdmit2), can still also be considered as small enough as compared to the value geme=1.818 $\mu$s$^{-1}$ employed at T = 294 K (see Table \ref{TmVOdmit2Vdmit3}). As we check again, it is crucial to consider only those $\va{B}$ directions where the detuning is small enough to account for the experimental $T_m$ value (in this case study, only one $\va{B}$ direction attains the minimum value $|\delta|_{\text{min}}$, namely the Y axis direction, where set firr with the value $\omega_{+-}=9.611$ GHz produced in that particular direction). 

\begin{figure}[H]
    \centering
    \includegraphics[scale=0.80]{VOdmit2_EstudioTm.jpg}
   \caption{Calculated $T_m$ magnetization curves of [VO(dmit)$_2$]$^{2-}$ at T = 294 K to (i) test the effect of detuning and (ii) re-compute the same $T_m$ magnetization as an average magnetization of 110 Lebedev $\va{B}$ directions and nang=72. The black curve, whose $T_m^{\text{calc}}$ and x parameters appear in Table \ref{TmVOdmit2Vdmit3}, is taken as a reference to compare with.}
   \label{VOdmit2EstudioTm}
\end{figure}

The last task of this section devoted to [VO(dmit)$_2$]$^{2-}$ and [V(dmit)$_3$]$^{2-}$ is the computation of the spin relaxation time $T_1$ whose experimental values $T_1^{\text{exp}}$ are found in \cite{keyrole}, particularly, in Table S2 for [VO(dmit)$_2$]$^{2-}$ and in Table S8 for [V(dmit)$_3$]$^{2-}$. Before proceeding with the calculations, we first note that the model employed in \cite{keyrole} to fit the experimental $T_1$ magnetization curves is a stretched monoexponential decaying function: $\text{exp}\left(-\left(t/T_1^{\text{exp}}\right)^x\right)$. The employment of this model gives rise, as shown in the above-mentioned tables, to significantly low stretch factors x, specially at temperatures below 25 K where 0.4$<$x$<$0.8. According to \cite{jorisCumnt2,millicohtime}, such a low stretch factor may be given a more sensible alternative interpretation in the form of a fast $T_{1,\text{fast}}^{\text{exp}}$ and a slow $T_{1,\text{slow}}^{\text{exp}}$ component, both with a fixed x=1 (no stretching). The former operates at short times in the experimental magnetization curve, while the latter operates at longer times. 
\bigbreak
Following this interpretation, we first plotted the model function $\text{exp}\left(-\left(t/T_1^{\text{exp}}\right)^x\right)$ against time $t$ for each pair $\left(T_1^{\text{exp}},x\right)$. Then, the plot is fitted to the model Eq.\ref{T1model} below to determine the fast and slow components $T_{1,\text{fast}}^{\text{exp}}$ and $T_{1,\text{slow}}^{\text{exp}}$ which are shown in Table \ref{T1VOdmit2} for [VO(dmit)$_2$]$^{2-}$ and in Table \ref{T1Vdmit3} for [V(dmit)$_3$]$^{2-}$. To avoid over-parameterization, we kept $y_0=0$ in all fits. 

\begin{equation}
y=y_0+a_f\cdot\text{exp}\left(-t/T_{1,\text{fast}}^{\text{exp}}\right)+a_s\cdot\text{exp}\left(-t/T_{1,\text{slow}}^{\text{exp}}\right)
\label{T1model}
\end{equation}

The goal now is to reproduce both components $T_{1,\text{fast}}^{\text{exp}}$ and $T_{1,\text{slow}}^{\text{exp}}$ at each given temperature for [VO(dmit)$_2$]$^{2-}$ and [V(dmit)$_3$]$^{2-}$ (we employ tmage=0 and the input files qb.adata\_T1 in folders VOdmit2 and Vdmit3). With a first calculation, we vary geme to match the calculated $T_{1,\text{fast}}^{\text{calc}}$ with $T_{1,\text{fast}}^{\text{exp}}$. To obtain $T_{1,\text{fast}}^{\text{calc}}$, we fit the calculated magnetization curve in qb.mz.out to a monoexponential decaying function with no stretch (fixed stretch factor x=1). Then, in a second calculation, we proceed likewise until making $T_{1,\text{slow}}^{\text{calc}}$ match with $T_{1,\text{slow}}^{\text{exp}}$. The calculated magnetization curves are found in folders VOdmit2/TableS2\_T1 and Vdmit3/TableS8\_T1. The resulting geme values are shown in Tables \ref{T1VOdmit2} and \ref{T1Vdmit3}.
\bigbreak
We implemented an important consideration in the above-mentioned $T_1$ calculations both for [VO(dmit)$_2$]$^{2-}$ and for [V(dmit)$_3$]$^{2-}$. As we mentioned in a previous section devoted to toy examples (see Figures \ref{T1_ejemplo} and \ref{Tm_ejemplo}), the values of tmage and geme employed to determine $T_m$ may not be necessarily the same as the ones employed to determine $T_1$ even under the same working conditions. In the determination of $T_m$, we have employed tmage=0 and geme as a free fitting parameter, as well as the magnetic rate 1/Tn+1/Te values found in qb.ddata\_nd\_1\_nopb\_sb (with values ca. 0.039 $\mu$s$^{-1}$ for [VO(dmit)$_2$]$^{2-}$ and ca. 0.024 $\mu$s$^{-1}$ for [V(dmit)$_3$]$^{2-}$, see folders VOdmit2 and Vdmit3). Recall that $\Gamma_{\text{ab}}$ and $\Gamma_{\text{em}}$ do not play any role here since we are using nm=0; hence $\Gamma_{\text{ab}}=\Gamma_{\text{em}}=0$.
\bigbreak
The key point to note is that the non-zero value of 1/Tn+1/Te in qb.ddata imposes an upper bound for the calculated values of $T_1$ and $T_m$. Namely, if all rates ($\Gamma_{\text{ab}}$(nm=0), $\Gamma_{\text{em}}$(nm=0), gabe, geme, tmage) but 1/Tn+1/Te are zero, the calculated $T_1$ and $T_m$ will always be below 1/(1/Tn+1/Te) ($\text{M}_z$ is plotted against $\tau$ in the determination of $T_1$) and 2/(1/Tn+1/Te) ($\text{M}_{xy}$ is plotted against $2\tau$ in the determination of $T_m$) resp. (see main text, Eq.\ref{decayrlongmag}, Eq.\ref{decayrsinpmag}). In the case of [VO(dmit)$_2$]$^{2-}$, the upper bounds are 1/0.039 $\sim$ 25.6 $\mu$s and 2/0.039 $\sim$ 51.2 $\mu$s, while for [V(dmit)$_3$]$^{2-}$ they are 1/0.024 $\sim$ 41.7 $\mu$s and 2/0.024 $\sim$ 83.4 $\mu$s. Since the experimental spin relaxation times $T_m^{\text{exp}}$ (see Table \ref{TmVOdmit2Vdmit3}) are all below these upper bounds in the explored temperature range, there is still room for using geme as a fitting parameter to make $T_m^{\text{calc}}$ match with $T_m^{\text{exp}}$. However, this is not the case of the experimental fast $T_{1,\text{fast}}^{\text{exp}}$ and slow $T_{1,\text{slow}}^{\text{exp}}$ components (see Tables \ref{T1VOdmit2} and \ref{T1Vdmit3}) since some of their values ($\leq$100-150 K for [VO(dmit)$_2$]$^{2-}$ and $\leq$25-40 K for [V(dmit)$_3$]$^{2-}$) are above the mentioned upper bounds. Under this situation, there is no room for using geme to match $T_{1,\text{fast}}^{\text{calc}}$ and $T_{1,\text{slow}}^{\text{calc}}$ with $T_{1,\text{fast}}^{\text{exp}}$ and $T_{1,\text{slow}}^{\text{exp}}$ unless we employ negative values for geme which is completely meaningless.  

\begin{table}[H]
    \centering
    \begin{tabular}{c | c c c | c c c}
T (K) & $T_{1,\text{fast}}^{\text{exp}}$ ($\mu$s) & $T_{1,\text{fast}}^{\text{calc}}$ ($\mu$s) & geme ($\mu$s$^{-1}$) & $T_{1,\text{slow}}^{\text{exp}}$ ($\mu$s) & $T_{1,\text{slow}}^{\text{calc}}$ ($\mu$s) & geme ($\mu$s$^{-1}$) \\\hline
4.5 & 3400 & 3400 & 1.54E-4 & 26100 & 26200 & 2.01E-5\\
10 & 2210 & 2210 & 2.31E-4 & 10980 & 10980 & 4.66E-5\\
25 & 433 & 435 & 1.16E-3 & 1789 & 1789 & 2.82E-4\\
40 & 152 & 152 & 3.31E-3 & 556 & 556 & 9.05E-4\\
60 & 53.2 & 53.2 & 9.45E-3 & 205.8 & 205.7 & 2.44E-3\\
100 & 17.4 & 17.5 & 2.87E-2 & 57.11 & 57.14 & 8.77E-3\\
150 & 8.47 & 8.47 & 5.92E-2 & 25.73 & 25.68 & 1.95E-2\\
294 & 1.16 & 1.16 & 4.32E-1 & 3.210 & 3.207 & 1.56E-1
    \end{tabular}
    \caption{Experimental and calculated fast and slow components of the spin relaxation time $T_1$ of [VO(dmit)$_2$]$^{2-}$ -with $t_{\pi}$=32 ns (bcm=1.177 mT)- as a function of temperature. Experimental data is extracted from Table S2 in \cite{keyrole}.}
    \label{T1VOdmit2}
\end{table}

\begin{table}[H]
    \centering
    \begin{tabular}{c | c c c | c c c}
T (K) & $T_{1,\text{fast}}^{\text{exp}}$ ($\mu$s) & $T_{1,\text{fast}}^{\text{calc}}$ ($\mu$s) & geme ($\mu$s$^{-1}$) & $T_{1,\text{slow}}^{\text{exp}}$ ($\mu$s) & $T_{1,\text{slow}}^{\text{calc}}$ ($\mu$s) & geme ($\mu$s$^{-1}$) \\\hline
10 & 152 & 152 & 3.380E-3 & 1860 & 1860 & 2.755E-4\\
25 & 45.7 & 45.7 & 1.104E-2 & 204.9 & 204.9 & 2.464E-3\\
40 & 12.7 & 12.7 & 3.995E-2 & 46.66 & 46.69 & 1.077E-2\\
60 & 3.48 & 3.48 & 1.442E-1 & 12.89 & 12.89 & 3.895E-2\\
100 & 0.641 & 0.641 & 7.820E-1 & 2.392 & 2.392 & 2.095E-1\\
150 & - & - & - & 0.6798 & 0.6798 & 7.366E-1
    \end{tabular}
    \caption{Experimental and calculated fast and slow components of the spin relaxation time $T_1$ of [V(dmit)$_3$]$^{2-}$ -with $t_{\pi}$=32 ns (bcm=1.094 mT)- as a function of temperature. Experimental data is extracted from Table S8 in \cite{keyrole}.}
    \label{T1Vdmit3}
\end{table}

To fix this issue, besides setting tmage=0 as also done in the computation of $T_m^{\text{calc}}$, we now additionally set 1/Tn+1/Te=0 in qb.ddata\_nd\_1\_nopb\_sb for all the explored temperatures in the determination of $T_{1,\text{fast}}^{\text{calc}}$ and $T_{1,\text{slow}}^{\text{calc}}$ shown in Tables \ref{T1VOdmit2} and \ref{T1Vdmit3}. Once the upper bound provided by 1/Tn+1/Te is removed, there is plenty of room to use geme to match $T_{1,\text{fast}}^{\text{exp}}$ and $T_{1,\text{slow}}^{\text{exp}}$. We find that not only the tmage and geme values employed in the determination of $T_m$ may be different from those employed to compute $T_1$, the 1/Tn+1/Te value may also be different between $T_m$ and $T_1$. In case the spin bath has some contribution to $T_1$ in our case studies, it may not be related to the mechanisms that we have employed to calculate 1/Tn+1/Te since these ones produce a too large value of 1/Tn+1/Te thus making the reproduction of $T_{1,\text{fast}}^{\text{exp}}$ and $T_{1,\text{slow}}^{\text{exp}}$ be impossible with the restriction geme$\geq$0. We could have tried to use the non-zero values of 1/Tn+1/Te in qb.ddata\_nd\_1\_nopb\_sb for the high temperatures where $T_{1,\text{fast}}^{\text{exp}}$ and $T_{1,\text{slow}}^{\text{exp}}$ are below 1/(1/Tn+1/Te) and hence there is room for using geme. However, since 1/Tn+1/Te has a rather temperature-independent value according to the mechanisms employed to compute 1/Tn+1/Te, if we are forced to set 1/Tn+1/Te=0 at low temperature, we must use the same setting at high temperature. All in all, by recalling that tmage=0, we here consider the vibration bath -whose effect is here collected by geme and Eq.\ref{detbalcond}- as the only mechanism determining $T_1$.
\bigbreak
To end up, we also performed the same extra tests (see folder VOdmit2/TableS2\_T1 and green, blue and red magnetization curves in the figure below which were also calculated with 1/Tn+1/Te=0 in all three tests and fitted to a non-stretched monoexponential decaying function) by taking the calculated slow component $T_{1,\text{slow}}^{\text{calc}}$=3.207 $\mu$s of [VO(dmit)$_2$]$^{2-}$ at T = 294 K as a reference (see Table \ref{T1VOdmit2} and black curve in the figure below). The $T_{1,\text{slow}}^{\text{calc}}$ values determined in the three extra tests do not show any significant change as they are also 3.207 $\mu$s. 

\begin{figure}[H]
    \centering
    \includegraphics[scale=0.80]{VOdmit2_EstudioT1.jpg}
   \caption{Calculated $T_1$ magnetization curves of [VO(dmit)$_2$]$^{2-}$ at T = 294 K to (i) test the effect of detuning and (ii) re-compute the same $T_1$ magnetization as an average magnetization of 110 Lebedev $\va{B}$ directions and nang=72. The monoexponential decaying black curve, with characteristic relaxation time $T_{1,slow}^{\text{calc}}$ appearing in Table \ref{T1VOdmit2}, is taken as a reference to compare with.}
   \label{VOdmit2EstudioT1}
\end{figure}

\subsection{VOPc}

The Hamiltonian Eq.\ref{spinH} is again diagonalized with the SIMPRE package,\cite{simpre2.0} by employing $J=S=1/2$, $I=7/2$, $B_k^q\equiv 0$, $|\va{B}|$=345 mT, $P=0$, together with the parameters provided in \cite{roomtcoh}: $g_x=g_y$=1.987, $g_z$=1.966, $A_x=A_y=0.0056$ cm$^{-1}$, $A_z$=0.0159 cm$^{-1}$. The qubit is also defined by the two states $\ket{m_J=-1/2,m_I=-1/2}$ and $\ket{m_J=+1/2,m_I=-1/2}$ which correspond to ig=5 and ie=12 in the $(2J+1)(2I+1)$=16 spin eigenstates. We employ id=16, ic=1, nd=1, nang=1. The X and Y axes directions have the largest g factor, namely  $g_x=g_y$=1.987. Hence, they provide the largest qubit gap with the value 9.613 GHz which is closest one to the experimental irradiation frequency 9.70 GHz. That is why we use firr=9.613 GHz. As a $\va{B}$ direction, we could use either the X or the Y axis direction. We choose again the Y axis direction and set 90º as azimuthal and zenithal angles in qb.ddata\_nd\_1\_nopb\_sb (see folder VOPc). As explained above, since $g_x=g_y$, we should actually perform an average of the calculated magnetization curves over all the $\va{B}$ directions contained in the XY plane given by the fixed zenithal angle $\theta$=90º and the variable azimuthal angle 0$\leq\phi<$360º. To avoid the computational cost of using a representative set of $\va{B}$ directions in the said XY plane, we only consider the Y axis direction given by $\theta$=90º and $\phi$=90º. We do not expect any significant difference in the results respect to performing the mentioned average since the $A$ tensor is also axial, namely $A_x=A_y$.
\bigbreak
The determination of $|\va{B}_1|$, namely bcm, is performed again by employing the experimental Rabi oscillations, the input qb.adata\_Rabi (see folder VOPc), and tmage=0. In particular, we use the Figures S17 (temp=300 K) and S19 (temp=4.3 K) found in \cite{roomtcoh}. In the two figures below, we plot both the experimental and calculated (solid line, see output files qb.mz.out in folders VOPc/FigureS17 and VOPc/FigureS19) oscillations. In Figures S17 and S19, we take as a reference the experimental traces determined at 0 dB and 1 dB attenuation, resp. First, bcm is varied until matching the period of the very first experimental oscillation of the two above-mentioned traces. Then, geme is adjusted to reproduce the experimental amplitude decay. The bcm values for the resting experimental traces are determined by employing the relative $B_1$ values of Figures S17c (1.99 and 3.98 a.u. for 6 and 0 db, resp.) and S19c (2.52 and 4.43 a.u. for 6 and 1 dB, resp.) in \cite{roomtcoh}. Once the mentioned bcm values are obtained, the task left to do is to adjust geme until finding the amplitude decay that fits the best. 
\bigbreak
The impossibility of reproducing the rest of experimental oscillations at longer times comes from the fact that more frequencies might be involved in the experimental traces while, as preoviously mentioned, our theoretical model only contains a single generalized Rabi frequency. The involvement of extra frequencies is reflected in the non-zero width of the Fourier transform shown in the Figures S17b and Figures S19b of \cite{roomtcoh}. The success at reproducing the experimental oscillation period in the whole nutation time range depends on how wide the Fourier transform is. A wide Fourier transform involves many frequencies thus making the reproduction with a single calculated frequency be unlikely. Instead, a narrow Fourier transform concentrates almost all the weight in a single frequency. In this case, a theoretical model containing only one calculated frequency works much better. As seen in Figures S17b and S19b of \cite{roomtcoh}, the Fourier transform is narrowed as the attenuation is increased, especially at T = 4.3 K (Figure S19b). That might be the reason why the calculated trace fits much better at the highest attenuation. Regarding [VO(dmit)$_2$]$^{2-}$ and [V(dmit)$_3$]$^{2-}$, there exists a satisfactory reproduction of the experimental traces both at short and at longer times in the whole attenuation range explored (see Figure 3 of main text). Indeed, while the width of the Fourier transforms of VOPc is around 5 MHz (Figures S17b and S19b of \cite{roomtcoh}), the width in the case of [VO(dmit)$_2$]$^{2-}$ and [V(dmit)$_3$]$^{2-}$ seems to be somewhat below 5 MHz thus providing narrower Fourier transforms (see Figures 6b and S22b of \cite{keyrole}). 

\begin{figure}[H]
    \centering
    \includegraphics[scale=0.70]{Rabi_VOPc_solo_Geadd_300K.jpg}
   \caption{Experimental (Figure S17 in \cite{roomtcoh}) and calculated (solid line) Rabi oscillations of VOPc at \textit{T}=300 K.}
   \label{RabiVOPcGeadd300K}
\end{figure}

The Fourier transforms in Figures S17b and S19b of \cite{roomtcoh} also reveal the participation of the proton $^1$H Larmor frequency located at around 15 MHz in the two 6 dB attenuation experimental traces. While the meeting of the Hartmann-Hahn condition is less obvious in the T=300 K trace, it becomes much clearer in the T=4.3 K one. Indeed, as seen in the Fourier transform at T=4.3 K (Figure S19b of \cite{roomtcoh}), the proton $^1$H Larmor frequency has the largest intensity respect to the rest of frequencies around. Instead, in the Fourier transform at T=300 K (Figure S17b of \cite{roomtcoh}), this frequency is clearly not the one with the highest intensity.

\begin{figure}[H]
    \centering
    \includegraphics[scale=0.70]{Rabi_VOPc_solo_Geadd_4_3K.jpg}
   \caption{Experimental (Figure S19 in \cite{roomtcoh}) and calculated (solid line) Rabi oscillations of VOPc at \textit{T}=4.3 K.}
   \label{RabiVOPcGeadd4_3K}
\end{figure}

The last task here would be to reproduce the experimental spin relaxation times $T_m^{\text{exp}}$ ($t_{\pi/2}$=20 ns, $t_{\pi}$=40 ns) and $T_1^{\text{exp}}$ ($t_{\pi/2}$=16 ns, $t_{\pi}$=32 ns) together with the stretch factors x, all of it found in the Figures 3 and S21 of \cite{roomtcoh} and obtained by fitting the experimental $T_m$ and $T_1$ magnetization curves to an stretched monoexponential decaying function (Equations S1 and S2 of \cite{roomtcoh}). As one sees in Figure S21 of \cite{roomtcoh}, the stretch factor x of $T_m^{\text{exp}}$ becomes larger than 1 as temperature is increased. This indicates the participation of relaxation mechanisms -such as Nuclear Spin Diffusion- which actually produce x values larger than 1 but are not included in our theoretical treatment (see main text). Hence, we avoid the calculation of these experimental $T_m^{\text{exp}}$ and x values and remit the reader to other research articles where models are developed to reproduce both $T_m^{\text{exp}}$ and x when x$>$1.\cite{dasSarma,JorisQuant}
\bigbreak
Regarding $T_1^{\text{exp}}$, we observe x values in Figure S21 of \cite{roomtcoh} from ~0.6 to ~0.8 with increasing temperature. As did with [VO(dmit)$_2$]$^{2-}$ and [V(dmit)$_3$]$^{2-}$, since $x$ is also below 1, we would first re-interpret such a low x values in the form of experimental $T_1$ magnetization curves as a sum of two non-stretched (x=1 fixed) exponential decaying functions with respective fast $T_{1,\text{fast}}^{\text{exp}}$ and slow $T_{1,\text{slow}}^{\text{exp}}$ components. Then, at each given temperature, one can independently reproduce the values $T_{1,\text{fast}}^{\text{exp}}$ and $T_{1,\text{slow}}^{\text{exp}}$ with different geme values by also fitting the calculated $T_1$ magnetization curves to a non-stretched monoexponential decaying function. One would also set tmage=0 and 1/Tn+1/Te=0 in qb.ddata\_nd\_1\_nopb\_sb. The bcm values at temperatures close to 4.3 K could be determined by employing the 4.3 K and 14 dB experimental Rabi oscillation with 160 ns as a period $T$ of the very first full $2\pi$ nutation, while the bcm values at temperatures close to 300 K would be determined with the 300 K and 12 dB experimental Rabi oscillation by taking now 120 ns as the period $T$ for the very first full $2\pi$ nutation. In the first case, bcm=1.050 mT; in the second one, bcm=1.088 mT. 

\subsection{[Cu(mnt)\texorpdfstring{$_2$}{}]\texorpdfstring{$^{2-}$}{}}

We address in this section the molecular spin qubit in [Cu(mnt)$_2$]$^{2-}$ studied in two articles.\cite{jorisCumnt2,CPMGCumnt2} We employ the first one to study the experimental spin relaxation times $T_m$ and $T_1$, while the second one is employed to study the experimental spin coherence time $T_{\text{dd}}$ measured with the so-called CPMG sequence. This pulse sequence is known for lengthening the coherence time by dynamically decoupling the spin qubit from the spin bath in the form of a long sequence of refocusing $\pi$ pulses. 
\bigbreak
Let us start with the article \cite{jorisCumnt2}. The diagonalization of Eq.\ref{spinH} was carried out with SIMPRE,\cite{simpre2.0} with $J=S=1/2$, $I=3/2$, $B_k^q\equiv 0$, $|\va{B}|$=1.199 T, $P$=0, $g_{xy}$=2.0227, $g_z$=2.0925, $A_{xy}$=118 MHz, $A_z$=500 MHz. Among the $(2J+1)(2I+1)$=8 spin eigenstates, the two qubit states are $\ket{m_J=-1/2,m_I=+3/2}$, $\ket{m_J=+1/2,m_I=+3/2}$ corresponding to ig=1, ie=8. Also, id=8, ic=1, nd=1, nang=1, tmage=0. As mentioned in \cite{jorisCumnt2}, the experimental measurements therein are performed on a powder sample at $|\va{B}|$=1.199 T which corresponds to the $g_{xy}$ region. Since, $g_x=g_y$, we should perform again an average of the calculated magnetization curves over all the $\va{B}$ directions contained in the XY plane given by the fixed zenithal angle $\theta$=90º and the variable azimuthal angle 0$\leq\phi<$360º. To avoid the computational cost of using a representative set of $\va{B}$ directions in the said XY plane, we only consider the X axis direction given by $\theta$=90º and $\phi$=0º. We do not expect either any significant difference in the results respect to performing the mentioned average since the $A$ tensor is also axial, namely $A_x=A_y$. In the $g_{xy}$ region, the qubit gap is 34.147 GHz which is close to the experimental irradiation frequency $\sim$34 GHz. We use firr=34.147 GHz. The bcm values are determined by employing the experimental Rabi oscillations of the Figure S9 in \cite{jorisCumnt2}. We use temp=15 K and the input files qb.ddata\_0.001pc\_15K\_nd\_1 and qb.adata\_Rabi\_1 in folder Cumnt2. The two experimental traces of the Figure S9 in \cite{jorisCumnt2} were reproduced (see green and yellow traces in figure below) by employing bcm=0.345 mT, geme=3.50 $\mu$s$^{-1}$ (low power), and bcm=1.510 mT, geme=9.00 $\mu$s$^{-1}$ (saturation power). The two calculated traces (qb.mz.out\_LowPower and qb.mz.out\_SatPower) are found in folder Cumnt2/FigureS9.  

\begin{figure}[H]
    \centering
    \includegraphics[scale=0.55]{Rabi_Cumnt2_solo_Geadd_Joris.jpg}
   \caption{Experimental (Figure S9 in \cite{jorisCumnt2}) and calculated (green and yellow solid lines) Rabi oscillations of [Cu(mnt)$_2$]$^{2-}$ at \textit{T}=15 K.}
   \label{RabiCumnt2GeaddJoris}
\end{figure}

We now proceed to reproduce the experimental fast $T_{m,\text{fast}}^{\text{exp}}$ and slow $T_{m,\text{slow}}^{\text{exp}}$ components of the spin relaxation time $T_m$ which are found in Table S3 of \cite{jorisCumnt2}. The relevant pulse lengths are $t_{\pi/2}$=20 ns and $t_{\pi}$=40 ns, and we use the input files qb.adata\_Tm\_1, qb.ddata\_0.01pc\_7K\_nd\_1, qb.ddata\_0.01pc\_50K\_nd\_1, qb.ddata\_0.01pc\_150K\_nd\_1 (see folder Cumnt2). To determine the bcm value corresponding to the pulse lengths above, we take the experimental low power trace -where bcm=0.345 mT- and assign it a period of 200 ns for the very first Rabi oscillation. Since $t_{2\pi}$=80 ns in the $T_m$ determination, the resulting bcm value is 0.8625 mT. The parameter geme is employed to fit the experimental values $T_{m,\text{fast}}^{\text{exp}}$ and $T_{m,\text{slow}}^{\text{exp}}$. The calculated $T_m$ magnetization traces in qb.mxy.out -fitted to a non-stretched monoexponential decaying function- are found in folder Cumnt2/TableS3, while the calculated components $T_{m,\text{fast}}^{\text{calc}}$ and $T_{m,\text{slow}}^{\text{calc}}$ are shown in Table \ref{TmCumnt2joris}. 

\begin{table}[H]
    \centering
    \begin{tabular}{c | c c c | c c c}
T (K) & $T_{m,\text{fast}}^{\text{exp}}$ ($\mu$s) & $T_{m,\text{fast}}^{\text{calc}}$ ($\mu$s) & geme ($\mu$s$^{-1}$) & $T_{m,\text{slow}}^{\text{exp}}$ ($\mu$s) & $T_{m,\text{slow}}^{\text{calc}}$ ($\mu$s) & geme ($\mu$s$^{-1}$) \\\hline
7 & 4.2 & 4.21 & 2.660E-1 & 68 & 68.0 & 1.640E-2\\
50 & - & - & - & 19.8 & 19.79 & 5.135E-2\\
150 & - & - & - & 4.2 & 4.19 & 2.400E-1
    \end{tabular}
    \caption{Experimental and calculated fast and slow components of the spin relaxation time $T_m$ of [Cu(mnt)$_2$]$^{2-}$ -with $t_{\pi/2}$=20 ns and $t_{\pi}$=40 ns (bcm=0.8625 mT)- as a function of temperature. Experimental data is extracted from Table S3 in \cite{jorisCumnt2}.}
    \label{TmCumnt2joris}
\end{table}

Now, in the reproduction of the experimental fast $T_{1,\text{fast}}^{\text{exp}}$ and slow $T_{1,\text{slow}}^{\text{exp}}$ components of the spin relaxation time $T_1$ (see also Table S3 of \cite{jorisCumnt2}), we also use the three input files qb.ddata\_0.01pc\_7K\_nd\_1, qb.ddata\_0.01pc\_50K\_nd\_1, qb.ddata\_0.01pc\_150K\_nd\_1 together with the input qb.adata\_T1\_1 (see folder Cumnt2). As did for [VO(dmit)$_2$]$^{2-}$ and [V(dmit)$_3$]$^{2-}$, we set 1/Tn+1/Te=0 in all qb.ddata in these $T_1$-related computations. The calculated $T_1$ magnetization traces in qb.mz.out -also fitted to a non-stretched monoexponential decaying function- are found in folder Cumnt2/TableS3, while the calculated components $T_{1,\text{fast}}^{\text{calc}}$ and $T_{1,\text{slow}}^{\text{calc}}$ by also using geme as a fitting parameter are shown in Table \ref{T1Cumnt2joris}.

\begin{table}[H]
    \centering
    \begin{tabular}{c | c c c | c c c}
T (K) & $T_{1,\text{fast}}^{\text{exp}}$ ($\mu$s) & $T_{1,\text{fast}}^{\text{calc}}$ ($\mu$s) & geme ($\mu$s$^{-1}$) & $T_{1,\text{slow}}^{\text{exp}}$ ($\mu$s) & $T_{1,\text{slow}}^{\text{calc}}$ ($\mu$s) & geme ($\mu$s$^{-1}$) \\\hline
7 & 7 & 7 & 8.000E-2 & 96197 & 96202 & 5.803E-6\\
50 & - & - & - & 83.5 & 83.5 & 6.087E-3\\
150 & - & - & - & 1.60 & 1.60 & 3.140E-1
    \end{tabular}
    \caption{Experimental and calculated fast and slow components of the spin relaxation time $T_1$ of [Cu(mnt)$_2$]$^{2-}$ -with $t_{\pi}$=40 ns (bcm=0.8625 mT)- as a function of temperature. Experimental data is extracted from Table S3 in \cite{jorisCumnt2}.}
    \label{T1Cumnt2joris}
\end{table}

The pulse sequence that we have employed so far to produce the calculated $T_1$ magnetization curves consists in the following simplified pulse sequence, namely a $\pi$ rotation with a predefined time length (numerical code 2 in qb.adata) to invert the equilibrium magnetization, and then a free evolution with a variable waiting time (numerical code 0). We carry out at this point another extra test to check how the calculated $T_1$ is affected by using the complete pulse sequence that is employed in the experimental measurement. This complete sequence consists of (i) a $\pi$ rotation, (ii) a free evolution with a variable waiting time, and (iii) a Hahn sequence with a fixed waiting time, namely a $\pi/2$ rotation, a free evolution with a fixed waiting time (numerical code 1), a $\pi$ rotation, and another free evolution with the same fixed waiting time. In the present experiment according to \cite{jorisCumnt2}, the $\pi/2$ and $\pi$ rotations last 20 and 40 ns, respectively; while the fixed free evolutions last 140 ns. We repeat the same $T_1$ calculations as above but by keeping fixed the geme values found to make the Table \ref{T1Cumnt2joris} and by replacing the input qb.adata\_T1\_1 by qb.adata\_T1\_1\_Compl (see folder Cumnt2). The corresponding calculated magnetization curves are found in folder Cumnt2/TableS3, and the resulting $T_{1,\text{fast}}^{\text{calc,compl.}}$ and $T_{1,\text{slow}}^{\text{calc,compl.}}$ values are shown in Table \ref{T1complCumnt2joris}. Note that these calculated curves are the ones given in the qb.mxy.out output files instead of in the qb.mz.out output files. This is because the Hahn sequence tips the magnetization so that it is measured in the XY plane in the form of an in-plane magnetization instead of as a longitudinal magnetization as it happens in the simplified pulse sequence. What we find is that the use of the complete sequence does not lead to any significant difference respect to the $T_1$ values calculated with the simplified sequence. The maximum relative deviation is $\sim$1.4\%.

\begin{table}[H]
    \centering
    \begin{tabular}{c | c c c | c c c}
T (K) & $T_{1,\text{fast}}^{\text{exp}}$ ($\mu$s) & $T_{1,\text{fast}}^{\text{calc,compl.}}$ ($\mu$s) & geme ($\mu$s$^{-1}$) & $T_{1,\text{slow}}^{\text{exp}}$ ($\mu$s) & $T_{1,\text{slow}}^{\text{calc,compl.}}$ ($\mu$s) & geme ($\mu$s$^{-1}$) \\\hline
7 & 7 & 6.9 & 8.000E-2 & 96197 & 95500 & 5.803E-6\\
50 & - & - & - & 83.5 & 83.4 & 6.087E-3\\
150 & - & - & - & 1.60 & 1.60 & 3.140E-1
    \end{tabular}
    \caption{Experimental and calculated fast and slow components of the spin relaxation time $T_1$ of [Cu(mnt)$_2$]$^{2-}$ -with $t_{\pi/2}$=20 ns, $t_{\pi}$=40 ns (bcm=0.8625 mT) and fixed waiting time of 140 ns- as a function of temperature. The components $T_{1,\text{fast}}^{\text{calc,compl.}}$ and $T_{1,\text{slow}}^{\text{calc,compl.}}$ have been calculated with the complete $T_1$ pulse sequence and the same geme values found to make Table \ref{T1Cumnt2joris}. Experimental data is extracted from Table S3 in \cite{jorisCumnt2}.}
    \label{T1complCumnt2joris}
\end{table}

Now, let us focus on the second article.\cite{CPMGCumnt2} The Hamiltonian Eq.\ref{spinH} is now diagonalized by employing $|\va{B}|$=335.7 mT, $g_{xy}$=2.0215, $g_z$=2.0898, $A_{xy}$=118 MHz, $A_z$=495.4 MHz. We also consider only the X axis direction given by $\theta$=90º and $\phi$=0º. The corresponding qubit gap is 9.6857 GHz, the closest one to the experimental irradiation frequency 9.70 GHz, and we use firr=9.6857 GHz and the input file qb.ddata\_0.3pc\_8K\_nd\_1 in folder Cumnt2. The temperature is kept fixed at $T$=8 K.
\bigbreak
We employ the 10 W attenuation Rabi trace in Figure 2 of \cite{CPMGCumnt2} to determine the bcm values, where we assign a value of 200 ns to the period of its very first oscillation. The calculated Rabi trace (yellow curve in the figure below, see also qb.mz.out\_Rabi\_8K\_10W in folder Cumnt2/CPMG) is computed by running the extended pulse sequence used in \cite{CPMGCumnt2} (see input qb.adata\_Rabi\_2 in folder Cumnt2), namely: (i) variable rotation (numerical code 3 in qb.adata), (ii) 400 ns free evolution (numerical code 1), (iii) 48 ns rotation (numerical code 2), and (iv) 400 ns free evolution (numerical code 1). Note that, with this extended sequence, the initial magnetization at a zero nutation pulse length does not reach the maximum amplitude value. Essentially, the Rabi trace calculated either with the simplified sequence or with the extended one is the same but with a relative horizontal offset between them. As seen in Figure 2b of \cite{CPMGCumnt2}, the Fourier transform of the experimental 10 W attenuation Rabi trace is the narrowest one yet with a non-negligible width that manifests itself in the noticeable deviations of the computed yellow trace below. The optimal bcm and geme values found are 0.364 mT and 1.1 $\mu$s$^{-1}$. 

\begin{figure}[H]
    \centering
    \includegraphics[scale=0.35]{Rabi_Cumnt2_solo_Geadd_Jiangfeng.jpg}
   \caption{Experimental (Figure 2 in \cite{CPMGCumnt2}) and calculated (yellow solid line) Rabi oscillations of [Cu(mnt)$_2$]$^{2-}$ at \textit{T}=8 K.}
   \label{RabiCumnt2GeaddJiangfeng}
\end{figure}

The $\pi/2$ and $\pi$ pulse lengths that we employ in the $T_m$ and CPMG sequences according to \cite{CPMGCumnt2} are 24 and 48 ns, respectively. This means that a full $2\pi$ rotation takes 96 ns. If we now use the above-determined data when we reproduced one of the experimental Rabi traces (bcm=0.364 mT for a $2\pi$ rotation of 200 ns), the bcm value to employ for a $2\pi$ rotation of 96 ns is 0.7583 mT. The experimental $T_m$ value in \cite{CPMGCumnt2} of 6.8 $\mu$s can be reproduced with geme=0.1487 $\mu$s$^{-1}$ (we employ the input qb.adata\_Tm\_2 in folder Cumnt2, and the calculated magnetization curve -fitted to a monoexponential decaying function- is found in the output qb.mxy.out\_8K\_Tm in folder Cumnt2/CPMG). On the other hand, the experimental $T_1$ value of 25000 $\mu$s is reproduced with geme=2.047E-5 $\mu$s$^{-1}$ and by setting 1/Tn+1/Te=0 in qb.ddata\_0.3pc\_8K\_nd\_1. To calculate the $T_1$ magnetization curve (see output qb.mxy.out\_8K\_T1 in folder Cumnt2/CPMG), we employ the following extended pulse sequence as done in \cite{CPMGCumnt2} (see input qb.adata\_T1\_2 in folder Cumnt2): (i) 48 ns $\pi$ rotation (numerical code 2 in qb.adata), (ii) variable free evolution (numerical code 0), (iii) 24 ns $\pi/2$ rotation (numerical code 2), (iv) 400 ns free evolution (numerical code 1), (v) 48 ns $\pi$ rotation (numerical code 2), and (vi) 400 ns free evolution (numerical code 1). Note that we use again the output file qb.mxy.out in this extended pulse sequence, namely the in-plane magnetization, since the $\pi/2$ pulse after the variable free evolution takes the magnetization from the longitudinal Z axis to the equatorial XY plane.
\bigbreak
The input files qb.adata\_CPMG employed in the study of the CPMG sequence for several values of the number $k$ of refocusing $\pi$ pulses are found in folder Cumnt2. As explained in main text and as a first step, we find an appropriate value of geme. Then, as a second step, we use tmage to reproduce the experimental spin coherence times $T_{\text{dd}}$ that result from applying the CPMG sequence with the several values of $k$. The output files qb.mxy.out\_CPMG of the first step are located in folder Cumnt2/CPMG/geme\_fixed, while the ones of the second step can be found in folder Cumnt2/CPMG/tmage\_var. We use the output files qb.mxy.out since the initial $\pi/2$ pulse of the CPMG sequence takes the initial magnetization (initially placed along the longitudinal Z axis according to the initial condition of the sequence) to the equatorial XY plane. As did in \cite{CPMGCumnt2}, to determine $T_{\text{dd}}$, we fit our calculated magnetization curves of the qb.mxy.out\_CPMG output files to an exponential function with an stretch factor $\text{x}$:
\begin{equation}
y=y_0+a\cdot\text{exp}\left(-\left(t/T_{\text{dd}}\right)^\text{x}\right)
\label{Tddmodel}
\end{equation}

The experimental values $T_{\text{dd}}^{\text{exp}}$, $x^{\text{exp}}$ are not directly provided in \cite{CPMGCumnt2}. Instead, the authors provide the expression $T_{\text{dd}}^{\text{fit}}=T_{\text{dd},k=1}^{\text{fit}}k^{\alpha}$ with $\alpha=0.67$ that results from fitting the plot $T_{\text{dd}}^{\text{exp}}$ vs $k$. While \cite{CPMG2Cumnt2} reports $T_{\text{dd},k=1}^{\text{fit}}=5.2$ $\mu$s, we rather use $T_{\text{dd},k=1}^{\text{fit}}=T_m=6.8$ $\mu$s since both the Hahn and the CPMG sequence essentially coincide for $k=1$. Hence, ideally $T_{\text{dd},k=1}^{\text{fit}}=T_m$. We use $T_{\text{dd}}^{\text{fit}}$ -evaluated with $T_{\text{dd},k=1}^{\text{fit}}=6.8$ $\mu$s and $\alpha=0.67$- as a reference to compare with our calculated $T_{\text{dd}}$. For $k=2048$, instead of using $T_{\text{dd}}^{\text{fit}}$, we use $T_{\text{dd}}^{\text{exp}}$ -whose value is $1400$ $\mu$s- to compare with. In the table below, we show $T_{\text{dd}}^{\text{fit}}$ and the calculated pairs $T_{\text{dd}}$, $x$ in the two steps mentioned above.

\begin{table}[H]
    \centering
    \begin{tabular}{c | c c c c c c c c}
$k$ & 1 & 8 & 32 & 128 & 256 & 512 & 1024 & 2048 \\\hline
nsa & 4 & 25 & 97 & 385 & 769 & 1537 & 3073 & 6145 \\
$T_{\text{dd}}^{\text{fit}}$ ($\mu$s) & 6.8* & 27.4 & 69.3 & 175.5 & 279.3 & 444.3 & 707.0 & 1400.0** \\\hline
$T_{\text{dd}}^{\text{geme}}$ ($\mu$s) & 1407.1 & 1406.7 & 1406.3 & 1406.1 & 1406.3 & 1405.1 & 1405.0 & 1400.6 \\
$\text{x}^{\text{geme}}$ & 0.99948 & 1.00010 & 0.99982 & 0.99973 & 0.99934 & 0.99878 & 0.99837 & 0.99347 \\
eend ($\mu$s) & 5120.0 & 640.0 & 160.0 & 40.0 & 20.0 & 10.0 & 5.0 & 2.3 \\\hline
$T_{\text{dd}}^{\text{tmage}}$ ($\mu$s) & 6.8 & 27.4 & 69.3 & 175.5 & 279.0 & 444.2 & 707.4 & 1400.9 \\
$\text{x}^{\text{tmage}}$ & 0.99961 & 0.99997 & 0.99986 & 0.99930 & 0.99743 & 0.99768 & 0.99679 & 0.99320 \\
tmage ($\mu$s$^{-1}$) & 1.463E-1 & 3.577E-2 & 1.372E-2 & 4.983E-3 & 2.866E-3 & 1.537E-3 & 7.000E-4 & 0.0 \\
eend ($\mu$s) & 20.0 & 10.0 & 6.5 & 4.2 & 3.2 & 2.6 & 2.0 & 2.5
    \end{tabular}
    \caption{$T_{\text{dd}}^{\text{fit}}$ (see text and \cite{CPMGCumnt2}) as a reference and $T_{\text{dd}}$, $\text{x}$ spin coherence time and stretch factor calculated by us (see main text) when applying the CPMG sequence with $k$ refocusing $\pi$ pulses to [Cu(mnt)$_2$]$^{2-}$. All calculations have been performed with nm=0, gabe=-1.0, by setting 1/Tn+1/Te=0 in qb.ddata\_0.3pc\_8K\_nd\_1, esta=0, npe=201. Each pair $T_{\text{dd}}^{\text{geme}}$ and $\text{x}^{\text{geme}}$ are computed with tmage=0 and geme=7.31E-4 $\mu$s$^{-1}$ (the value of geme employed to reproduce $T_{\text{dd}}^{\text{fit}}$ for $k=2048$), while each pair $T_{\text{dd}}^{\text{tmage}}$ and $\text{x}^{\text{tmage}}$ are computed with geme=7.31E-4 $\mu$s$^{-1}$ and tmage as a free parameter. The values * and ** are respectively $T_m^{\text{exp}}$ and $T_{\text{dd}}^{\text{exp}}$ (see text).}
    \label{CPMGCumnt2SecI}
\end{table}

\newpage

\begin{thebibliography}{20}

\bibitem{LunghiHow}
A. Lunghi, S. Sanvito, \textit{Sci. Adv.} \textbf{5}, 9 (2019).

\bibitem{firstder1}
L. Escalera-Moreno, N. Suaud, A. Gaita-Ariño, E. Coronado, \textit{J. Phys. Chem. Lett.} \textbf{8}, 1695-1700 (2017).

\bibitem{firstder2}
L. Escalera-Moreno, José J. Baldoví, A. Gaita-Ariño, E. Coronado, \textit{Chem. Sci.} \textbf{11}, 1593-1598 (2020).

\bibitem{maestra2022}
P. Szańkowski, arXiv:2209.10928v1 (2022).

\bibitem{maestra2018}
A. Norambuena, E. Muñoz, H. T. Dinani, A. Jarmola, P. Maletinsky, D. Budker, J. R. Maze, \textit{Phys. Rev. B} \textbf{97}, 094304 (2018).

\bibitem{simpre2.0}
S. Cardona-Serra, L. Escalera-Moreno, J. J. Baldoví, A. Gaita-Ariño, J. M. Clemente-Juan, E. Coronado, \textit{J. Comput. Chem.} \textbf{37(13)}, 1238-1244 (2016).

\bibitem{espesp2019}
L. Escalera-Moreno, A. Gaita-Ariño, E. Coronado, \textit{Phys. Rev. B} \textbf{100}, 064405 (2019).

\bibitem{lebedev}
V. Lebedev, \textit{USSR Comp. Math. and Phys.} \textbf{16}, 10-24 (1976).

\bibitem{absinpm1}
W. M. Witzel, R. de Sousa, S. Das Sarma, \textit{Phys. Rev. B} \textbf{72}, 161306(R) (2005).

\bibitem{absinpm2}
F. Troiani, V. Bellini, M. Affronte, \textit{Phys. Rev. B} \textbf{77}, 054428 (2008).

\bibitem{absinpm3}
S. J. Balian, G. Wolfowicz, John J. L. Morton, T. S. Monteiro, \textit{Phys. Rev. B} \textbf{89}, 045403 (2014).

\bibitem{fidelity1}
R. Jozsa, \textit{J. Mod. Opt.} \textbf{41}, 2315-2323 (1994).

\bibitem{fidelity2}
M. Hübner, \textit{Phys. Lett. A}, \textbf{163}, 239-242 (1992).

\bibitem{keyrole}
M. Atzori, E. Morra, L. Tesi, A. Albino, M. Chiesa, L. Sorace, R. Sessoli, \textit{J. Am. Chem. Soc.} \textbf{138}, 35, 11234-11244 (2016).

\bibitem{jorisCumnt2}
K. Bader, D. Dengler, S. Lenz, B. Endeward, S.-Da Jiang, P. Neugebauer, J. van Slageren, \textit{Nat. Commun.} \textbf{5}, 5304 (2014).

\bibitem{millicohtime}
J. M. Zadrozny, J. Niklas, O. G. Poluektov, D. E. Freedman, \textit{ACS Cent. Sci.} \textbf{1}, 9, 488-492 (2015).

\bibitem{roomtcoh}
M. Atzori, L. Tesi, E. Morra, M. Chiesa, L. Sorace, R. Sessoli, \textit{J. Am. Chem. Soc.} \textbf{138}, 2154-2157 (2016).

\bibitem{dasSarma}
W. M. Witzel, S. Das Sarma, \textit{Phys. Rev. B} \textbf{74}, 035322 (2006).

\bibitem{JorisQuant}
S. Lenz, K. Bader, H. Bamberger, J. van Slageren, \textit{Chem. Commun.} \textbf{53}, 4477-4480 (2017).

\bibitem{CPMGCumnt2}
Y. Dai, Z. Shi, Y. Fu, X. Qin, S. Mu, Y. Wu, J.-Hu Su, L. Qin, Y.-Qi Zhai, Y.-Fei Deng, X. Rong, J. Du, arXiv:1706.09259v1 (2017).

\bibitem{CPMG2Cumnt2}
Y. Dai, Y. Fu, Z. Shi, X. Qin, S. Mu, Y. Wu, J.-Hu Su, Y.-Fei Deng, L. Qin, Y.-Qi Zhai, Y.-Zhen Zheng, X. Rong, J. Du, \textit{Chin. Phys. Lett.} \textbf{38}, 030303 (2021).

\end{thebibliography}

\end{document}