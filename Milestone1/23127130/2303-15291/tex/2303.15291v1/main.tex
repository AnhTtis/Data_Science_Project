\documentclass{article}


\usepackage{PRIMEarxiv}

\usepackage[utf8]{inputenc} % allow utf-8 input
\usepackage[T1]{fontenc}    % use 8-bit T1 fonts
\usepackage{hyperref}       % hyperlinks
\usepackage{url}            % simple URL typesetting
\usepackage{booktabs}       % professional-quality tables
\usepackage{amsfonts}       % blackboard math symbols
\usepackage{nicefrac}       % compact symbols for 1/2, etc.
\usepackage{microtype}      % microtypography
\usepackage{lipsum}
\usepackage{fancyhdr}       % header
\usepackage{graphicx}       % graphics
\graphicspath{{media/}}     % organize your images and other figures under media/ folder
\usepackage{ptdr-definitions}  % standard units and kinematic variables 
\usepackage{heppennames2}      % Latex friendly particle names and other useful definitions for HEP.


%Header
\pagestyle{fancy}
\thispagestyle{empty}
\rhead{ \textit{ }} 

% Update your Headers here
%\fancyhead[LO]{Running Title for Header}
% \fancyhead[RE]{Firstauthor and Secondauthor} % Firstauthor et al. if more than 2 - must use \documentclass[twoside]{article}



\newcommand{\bbox}{\text{bbox}}
\newcommand{\alphapck}{\alpha_\bbox}
\newcommand{\kcycle}{\text{k-CyPCK}}
\newcommand{\cycle}{\text{-CyPCK}}

\newcommand{\I}{\mathbf{I}}
\newcommand{\Ia}{\I^\text{a}}
\newcommand{\Ib}{\I^\text{b}}
\newcommand{\Iatob}{\I^\text{a $\rightarrow$ b}}
\newcommand{\F}{\mathbf{F}}
\newcommand{\Fa}{\F^\text{a}}
\newcommand{\Fb}{\F^\text{b}}
\newcommand{\f}{\mathbf{f}}
\newcommand{\fa}{\f^\text{a}}
\newcommand{\fb}{\f^\text{b}}
\newcommand{\p}{\mathbf{p}}
\newcommand{\pa}{\p^\text{a}}
\newcommand{\pb}{\p^\text{b}}
\newcommand{\A}{\boldsymbol{\Phi}_\text{align}}
\newcommand{\G}{\mathbf{G}}
\newcommand{\C}{\mathbf{C}}
\newcommand{\Ca}{\C^\text{a}}
\newcommand{\Cb}{\C^\text{b}}
\newcommand{\cc}{\mathbf{c}}
\newcommand{\cca}{\cc^\text{a}}
\newcommand{\ccb}{\cc^\text{b}}
\newcommand{\Irec}{\I_\text{Recon}}
\newcommand{\M}{\mathbf{M}}
\newcommand{\Mrec}{\M_\text{Recon}}
\newcommand{\loss}{\mathcal{L}}
\newcommand{\T}{\mathcal{T}}
\newcommand{\W}{\mathcal{W}}
\newcommand{\Id}{\mathcal{I}}

\input{Utils/ju_math.tex}
  
%% Title
\title{Predicting the Future of the CMS  Detector: Crystal Radiation Damage and Machine Learning at the LHC
%%%% Cite as
%%%% Update your official citation here when published 
%\thanks{\textit{\underline{Citation}}: 
%\textbf{Authors. Title. Pages.... DOI:000000/11111.}} 
}



\author{}
% \author{
%   Buyun Liang$^{\dag}$ \\
%   Department of Computer Science and Engineering\\
%   University of Minnesota \\
%   Minneapolis, MN 55455, USA\\
%   \texttt{liang664@umn.edu }\\
%   \And
%   Bhargav Joshi$^{\dag}$ \\ % (alphabetical)
%   % Affiliation \\
%   Department of Physics\\
%   University of Minnesota \\
%   Minneapolis, MN 55455, USA \\
%   \texttt{joshib@umn.edu} \\
%   \And
%   Taihui Li$^{\dag}$ \\
%   Department of Computer Science and Engineering\\
%   University of Minnesota \\
%   Minneapolis, MN 55455, USA\\
%   \texttt{lixx5027@umn.edu} \\
%   \And
%   Roger Rusack$^{\dag}$ \\
%   Department of Physics\\
%   University of Minnesota \\
%   Minneapolis, MN 55455, USA\\
%   \texttt{rusack@umn.edu} \\
%   \And
%   Ju Sun$^{\dag}$ \\
%   Department of Computer Science and Engineering\\
%   University of Minnesota \\
%   Minneapolis, MN 55455, USA \\  
%   \texttt{jusun@umn.edu} \\
%   \\
%   $\dag$, \textit{These authors contributed equally to this work.}
% }



\begin{document}
\maketitle
\vspace{-25mm}
\begin{center}
    \textbf{
    Bhargav Joshi\textsuperscript{2,\dag},
    Taihui Li\textsuperscript{1,\dag},
    Buyun Liang\textsuperscript{1,\dag},
    Roger Rusack\textsuperscript{2,\dag},
    Ju Sun\textsuperscript{1,\dag}
    }
    \\
    % \vspace{3mm}
    \textsuperscript{1} Department of Computer Science \& Engineering, University of Minnesota, Minneapolis, USA
    \\
    \textsuperscript{2} Department of Physics, University of Minnesota, Minneapolis, USA
    \\
    \textsuperscript{\dag} These authors contributed equally to this work 
    \\
    \textit{\{liang664, joshib, lixx5027, rusack, jusun\}@umn.edu}
\end{center}
\vspace{10mm}

\begin{abstract}
  The 75,848 lead tungstate crystals in CMS experiment at the CERN Large Hadron Collider are used to measure the energy of electrons and photons produced in the proton-proton collisions. The optical transparency of the crystals degrades slowly with radiation dose due to the beam-beam collisions. The transparency of each crystal is monitored with a laser monitoring system that tracks changes in the optical properties of the crystals due to radiation from the collision products. Predicting the optical transparency of the crystals, both in the short-term and in the long-term, is a critical task for the CMS experiment. We describe here the public data release, following FAIR principles~\cite{fairmetrics}, of the crystal monitoring data collected by the CMS Collaboration between 2016 and 2018.  Besides describing the dataset and its access, the problems that can be addressed with it are described, as well as an example solution based on a Long Short-Term Memory neural network developed to predict future behavior of the crystals.
  \keywords{Time Series Prediction \and Machine Learning \and FAIR Data \and LSTM}
\end{abstract}

% keywords can be removed


\section{Introduction}
% Importance and appeal of children's drawings
Children's depictions of the human figure are highly expressive and varied.
As one of the very first subjects children attempt to draw, the representation begins as an almost unintelligible cloud of scribbles. 
As the child grows, their representation of the human figure becomes more developed and is extended to graphically represent many different types of characters: people, animals, and even personified objects (see Figure 1).

Who among us has not wished, either as a child or as an adult, to see such figures come to life and move around on the page?
Sadly, while it is relatively fast to produce a single drawing, creating the sequence of images necessary for animation is a much more tedious endeavor, requiring discipline, skill, patience, and sometimes complicated software.
As a result, most of these figures remain static upon the page.

% We built a system to animate them.
Inspired by the importance and appeal of the drawn human figure, we design and build a system to automatically animate it given an in-the-wild photograph of a child's drawing. 
Our system is fast, intuitive, and robust to much of the variation present in these types of drawings, making it well-suited to allow our target audience--children--to see their own characters coming to life.
The system is comprised of four stages: figure detection, segmentation masking, pose estimation/rigging, and animation. 
We describe each stage and identify common causes of failure in each. 
For object detection and pose estimation, we make use of existing computer vision models designed to detect human figures and joints in photographs; we fine-tune these models for use with children's drawings.
For segmentation, we present a straightforward, image processing-based method that, for animation purposes, is more useful and accurate than segmentation masks obtained from a fine-tuned object detection model.
During the animation step, we take advantage of the \textit{twisted perspective} commonly seen in children’s drawings to retarget motion capture data onto the character in a novel and appealing way.

% We use existing machine learning models. However, given the wide domain gap it's not clear how much fine-tuning data was needed. So we ran some experiments to find out and report it.
While our system leverages existing models and techniques, most are not directly applicable to the task due to the many differences between photographic images and simple pen and paper representations. 
To this end, we couple the presentation of our system with a set of experiments exploring the relationship between fine-tuning training set size and success rates.
We also include a perceptual study validating viewer preference for incorporating \textit{twisted perspective} into the motion retargeting step.

We validate the desirability and appeal of our system by building and publicly releasing a version of it as the \AD Demo \,\cite{animateddrawings}.
Launched in December 2021, this demo has been used by millions of people around the world to animate their children's drawings.
Inspired by this reception, our second contribution is The Amateur Drawings Dataset: \hjs{180,000 drawings and user-accepted annotations collected, with consent, through the demo. See Section \ref{sec:UI} for a description of how the annotations were generated.}
We believe this dataset will be a resource to researchers from various fields seeking to better understand the space of amateur drawings, evaluate new algorithms in this domain, or develop new drawing-based tools in general.

To summarize, our contributions are as follows:
\begin{enumerate}
    \item 
    We explore the problem of automatic sketch-to-animation for children's drawings of human figures and present a framework that achieves this effect. We also present a set of experiments determining the amount of training data necessary to achieve high levels of success and a perceptual study validating the usefulness of our motion retargeting technique.
    \item To encourage additional research in the domain of amateur drawings, we present a first-of-its-kind dataset of 180,000 user-submitted amateur drawings, along with user-accepted bounding box, segmentation mask, and joint location annotations.
\end{enumerate}

Upon acceptance of this paper, we plan to publicly release the Amateur Drawings Dataset, project code, and fine-tuned model weights.


\section{Data Challenge}
The details of the color center formation under radiation are poorly understood. It is thought that they are primarily determined by the atomic-level defects in the crystals when they were first grown and, thus are determined by the individual history of each crystal. From this it follows that the best way to predict the future behavior of a crystal is by examining their past behavior under radiation. 

It is valuable to the CMS collaboration to be able to accurately predict how the ensemble of crystals will respond to future irradiation cycles with damage and recovery. In particular there is interest in predicting the level of light-loss for a group of crystals after a duration of the order one day, when the integrated luminosity is $\approx 1 \rm{fb}^{-1}$. Another point of interest is to predict the performance of the crystals in the barrel ECAL up to the end of operation of the High-Luminosity LHC in 2040, when the total delivered luminosity is expected to be 3000 fb$^{-1}$

In the examples given below the short-term problem is tackled. The long term problem is left as an open challenge.

\section{Datasets}
\label{datasets}
The response to injected laser pulses for every crystal is measured approximately every 40 minutes and stored in an offline database. Data corresponding to the laser response of all 75848 crystals from 2016 through 2018 were extracted from the database.  Each entry has a timestamp corresponding to the time when the measurement was taken, and the current intensity of the beam-beam collisions (the instantaneous luminosity) which is a measure of the radiation dose being received by the crystals.  

The dataset consists of the following elements corresponding to a single measurement.

\begin{itemize}
    \item \textbf{xtal\_id}: Crystal identification number within ECAL ranging from [0, 75848]. 
    \item \textbf{start\_ts}: Start of the Interval of Validity (IOV). An Interval of Validity corresponds to a time during which a measurement is taken for a single crystal. In other words, each IOV contains one measurement per crystal.
    \item \textbf{stop\_ts}: End of the Interval of Validity (IOV).
    \item \textbf{laser\_datetime}: Timestamp of the measurement for a given crystal within an IOV. The timestamp lies between the start of IOV and end of IOV.
    \item \textbf{calibration}: APD/PD ratio taken at laser\_datetime. This value is used to quantify the transparency of the crystal at the time of measurement.
    \item \textbf{time}: Time corresponding to the luminosity measurement (obtained from BRIL) closest to the time when the laser measurement was taken.
    \item \textbf{int\_deliv\_inv\_ub}: Approximate integrated luminosity delivered up to the measurement, in the units of inverse microbarns.
\end{itemize}

To ensure the FAIR-ness of the publication of the dataset, it has been published~\cite{bhargav_joshi_2022_7510572} on Zenodo\footnote{\href{https://zenodo.org/}{https://zenodo.org/}} platform, which was launched in May 2013 as part of the OpenAIRE project, in partnership with CERN. 
%It is a repository funded by the European Commision and is widely used by the scientific community to publish large datasets. 
The dataset consists of 26 files in \textit{tar gzip} format, each file consisting of up to 10 csv files that contain measurements of up to 360 crystals. The files corresponding to the +z side of the ECAL are labelled as "plus" and those corresponding to -z side are labelled "minus". The list of $i\eta$ rings along with their position in terms of pseudo-rapidity ($\eta$) and azimuthal angle ($\phi$) in each tar file are included in form of json files under the metadata section in Zenodo.

\section{Machine Learning Solutions}

We describe in this and subsequent sections two ML solutions that we have developed to address this problem, their training and the results we have obtained. The source code is published on GitHub at \textcolor{blue}{\href{https://github.com/FAIR-UMN/fair_ecal_monitoring}{https://github.com/FAIR-UMN/fair\_ecal\_monitoring}}.

Neural networks (NNs) are machine learning (ML) methods that mimic the biological structure and functioning of neurons in a brain. A NN that does not involve any cyclic connections is called a Feedforward neural network (FNN). Deep neural networks (DNNs) are structures that consist of many stages of interconnected neurons. DNNs perform well for hard learning tasks, such as object identification and speech recognition. However, they require that the inputs and outputs of the task be encoded into vectors with fixed dimensionality \cite{sutskever2014sequence}. 

%It consists of units called \textit{neurons}, each activating to produce an output based on the inputs received. The inputs received by a neuron are weighted according to the weight assigned for each input connection and in some cases a bias is added. Following the input transformation, an activation function is applied to the input to produce an output. The model is \textit{trained} to tune the weights assigned to the input connections to maximize the accuracy of the final output. The number of neurons and their inter-connectivity can vary depending on the given problem. Due to the non-linearity introduced by the \textit{activation functions} at each node, determining the weights becomes analytically non-trivial. In NN models which are trained using teacher-based \textit{Supervised Learning}, an algorithm based on gradient descent method, known as \textit{backpropagation}\cite{backpropagation} efficiently determines the optimum set of weights.  

Many problems, like machine translation and speech recognition, have a sequential structure, since their input and output lengths are not known \textit{a-priori}~\cite{sutskever2014sequence}. 
%The same architectures have also been implemented to solve time series prediction problems. 
A class of neural networks called recurrent neural networks (RNNs) are a type of FNNs that pass the data sequentially between different nodes. This architecture allows the network to learn and to retain past knowledge when processing data points from a given data series. RNNs have some shortcomings---in particular, the vanishing gradient problem~\cite{vanishing-gradient} while training the network. Other architectures have been developed in the past to address these issues.

% This feature allows them learn the trends in data series over long ranges and make accurate forecasts. 
\subsection{Long Short Term Memory (LSTM) Models}

LSTM models are a type of RNNs that include feedback components. RNNs are good at tracking arbitrary long-term dependencies in a sequence but have a tendency to be unstable during training. LSTMs solve the vanishing-gradient problem through an additive gradient structure. The LSTM cell is shown in Figure \ref{fig:encoder-decoder}, which is the key part of the Seq2Seq model. For each element in the input sequence, each layer of the LSTM computes the following functions: 
\begin{align}
    \begin{split}
        & \vi_t = \sigma(\mW_{ii}\vx_t + \mathbf{b}_{ii} + \mW_{hi}\vh_{t-1} + \mathbf{b}_{hi} ) \\
        & \vf_t = \sigma(\mW_{if}\vx_t + \mathbf{b}_{if} + \mW_{hf}\vh_{t-1} + \mathbf{b}_{hf} ) \\ 
        & \vg_t = \tanh(\mW_{ig}\vx_t + \mathbf{b}_{ig} + \mW_{hg}\vh_{t-1} + \mathbf{b}_{hg} ) \\
        & \vo_t = \sigma(\mW_{io}\vx_t + \mathbf{b}_{io} + \mW_{ho}\vh_{t-1} + \mathbf{b}_{ho} ) \\
        & \vc_t = \vf_t \odot \vc_{t-1} + \vi_t \odot \vg_t \\
        & \vh_t = \vo_t \odot \tanh{\vc_t} 
    \end{split}
\end{align}

where $\vh_t,\vc_t$ are the hidden and cell state at time $t$, $\vx_t$ is the input at time $t$, $\vi_t,\vf_t,\vg_t,\vo_t$ are the input, forget, cell, and output gates, respectively. $\sigma$ is the sigmoid function and $\odot$ is the Hadamard product~\cite{lstm}.

% In multi layer LSTM, the input $\vx_t^{(l)}$ of the $l$-th layer ($l\geq 2$) is the hidden state $\vh_t^{(l-1)}$ of the previous layer multiplied by dropout $\delta_t^{(l-1)}$.
% \newcommand{\bbox}{\text{bbox}}
\newcommand{\alphapck}{\alpha_\bbox}
\newcommand{\kcycle}{\text{k-CyPCK}}
\newcommand{\cycle}{\text{-CyPCK}}

\newcommand{\I}{\mathbf{I}}
\newcommand{\Ia}{\I^\text{a}}
\newcommand{\Ib}{\I^\text{b}}
\newcommand{\Iatob}{\I^\text{a $\rightarrow$ b}}
\newcommand{\F}{\mathbf{F}}
\newcommand{\Fa}{\F^\text{a}}
\newcommand{\Fb}{\F^\text{b}}
\newcommand{\f}{\mathbf{f}}
\newcommand{\fa}{\f^\text{a}}
\newcommand{\fb}{\f^\text{b}}
\newcommand{\p}{\mathbf{p}}
\newcommand{\pa}{\p^\text{a}}
\newcommand{\pb}{\p^\text{b}}
\newcommand{\A}{\boldsymbol{\Phi}_\text{align}}
\newcommand{\G}{\mathbf{G}}
\newcommand{\C}{\mathbf{C}}
\newcommand{\Ca}{\C^\text{a}}
\newcommand{\Cb}{\C^\text{b}}
\newcommand{\cc}{\mathbf{c}}
\newcommand{\cca}{\cc^\text{a}}
\newcommand{\ccb}{\cc^\text{b}}
\newcommand{\Irec}{\I_\text{Recon}}
\newcommand{\M}{\mathbf{M}}
\newcommand{\Mrec}{\M_\text{Recon}}
\newcommand{\loss}{\mathcal{L}}
\newcommand{\T}{\mathcal{T}}
\newcommand{\W}{\mathcal{W}}
\newcommand{\Id}{\mathcal{I}}

% \input{Utils/ju_math.tex}

\subsection{Seq2Seq Model}

A sequence-to-sequence (Seq2Seq) model is an architecture that combines two or more LSTMs. It consists of two parts---the encoder and the decoder, each of which is built by using separate LSTMs. This type of model has been developed for automatic language translation, where a sentence from one language is translated to another language \cite{sutskever2014sequence}. The encoder is used to process each token in the input sentence, and encode all the input sequence information into a fixed-length vector. The transform vector, known as context vector, is a vector in a latent space and it encapsulates the whole meaning of the input sequence. The decoder reads the context vector and predicts the target sequence token by token.

Figure \ref{fig:encoder-decoder} shows the basic architecture of the encoder-decoder network used for this problem. The encoder block consists of LSTM units connected in series that take in a set of calibration values and the luminosity differences between those calibration values as input. All the information from the input sequence is encapsulated in the internal states $\vh_t$ (hidden state) and $\vc_t$ (cell state). The decoder block is another block of LSTM units connected in series. The final states $(\vh_t,\vc_t)$ of the encoder are used as the initial states $(\vh_0,\vc_0)$ of the decoder, which is the context vector used to predict the target sequence. The decoder network also takes input along with the initial states to predict the target sequence. The input to the decoder varies according to the method used for the training.

The Seq2Seq model can be trained using teacher forcing method, where the decoder is trained using the target output (ground truth output) instead of the output generated by the decoder in the previous step of the sequence. However, during the evaluation step, the decoder generates the output sequentially using the output generated in the previous step. Using teacher forcing during training has been shown to improve the training process. To determine if the training process has converged, the model was trained for 300 epochs and the validation e
%In this problem, the input $X$ is an English sentence (e.g., "nice to meet you"), and the Seq2Seq model is used to find a prediction $Y_{pred}$ that is closest to its corresponding French translation $Y_{true}$ (e.g., "ravi de vous rencontrer"). A token in this case is a single word in the sentence.

\begin{figure}[!ht]
  \centering
  \includegraphics[width=1.0\textwidth]{img/Seq2Seq_arch.png}
  \caption{Seq2Seq model for used for predicting future calibration values (\textbf{lower left}). The Encoder block (\textbf{upper left}) and Decoder block (\textbf{lower right}) are a set of sequentially connected LSTM units (\textbf{upper right}).}
  \label{fig:encoder-decoder}
\end{figure}

\section{Training}
% \section{Training}

For training AI models, PyTorch packages were used. The code is maintained in a public Github repository (\url{https://github.com/FAIR-UMN/FAIR-UMN-ECAL}). Conda environments are provided so that the users of the datasets can use any API of their liking. The project details can also be found in \url{https://fair-umn.github.io/FAIR-UMN-Docs}.

%\subsection{Training LSTM Model}
%The LSTM model was trained using the Tensorflow library. 

\subsection{Data Pre-Processing}

% todo list
%\todo{Reference to the actual notebook could be useful here.}

1080 csv files from the datasets mentioned in \cref{datasets} are used for the training and prediction. 360 csv files (\texttt{df\_skimmed\_xtal\_54000\_2016} to \texttt{df\_skimmed\_xtal\_54359\_2016}; size 203 MB) are used for training, and 720 csv files (\texttt{df\_skimmed\_xtal\_54000\_2017} to \texttt{df\_skimmed\_xtal\_}\\\texttt{54359\_2017} and \texttt{df\_skimmed\_xtal\_54000\_2018} to \texttt{df\_skimmed\_xtal\_54359\_2018}; size 406 MB) are used for prediction. 

The difference between subsequent entries in the dataset is the integrated luminosity delivered between the two consecutive measurements. Before training the networks, the measured calibration values and the luminosity differences in the training dataset are normalized to unity using the \textit{StandardScaler} from the sklearn library. Next, to obtain the input $X$ and the true output $Y_{\text{true}}$ used for model training, we performed the following steps:
\begin{itemize}
\item Define the input length $L_E$ (e.g., $L_E=24$), corresponding to the number of LSTM units in the encoder, and the output length $L_D$ (e.g., $L_D=24$), corresponding to the number of LSTM units in the decoder for each individual sample. The Seq2Seq model will be trained to use a sequence of calibration values and luminosity differences of length $L_E$ and learn to predict the next $L_D$ calibration values. 
\item To avoid any overlap between the prediction sequences, a separation stride $L_S$ is set to be the same as $L_{D}$. Therefore, the total number of samples is 
\[N_{sample} = \frac{N-L_{E}-L_{D}}{L_{S}}+1,\]
where N is the total number of entries in the training dataset. For each individual sample, the input is a sequence starting from $T$ to $T+L_{E}-1$ and the output is a sequence starting from $T+L_{E}$ to $T+L_{E}+L_{D}-1$. 
\item In PyTorch, the LSTM module takes a 3D tensor as the input whose dimensions are given by (sequence, batch, features). In this problem, the input to encoder (X$_{\text{encoder}}$), the input to decoder (X$_{\text{decoder}}$) and the output of the decoder (Y$_{\text{decoder}}$) can be represented as below:
\begin{align*}
    &X_{\text{encoder}} \in \R^{L_{E}\times N_{\text{sample}}\times N_{E}},\\
    &X_{\text{decoder}} \in \R^{L_{D}\times N_{\text{sample}}\times N_{D}},\\
    &Y_{\text{decoder}} \in \R^{L_{D}\times N_{\text{sample}}\times 1},
\end{align*}
where N$_{E}$ and N$_{D}$ are the number of features used in the encoder and decoder, respectively. In this study, N$_{E}$ and N$_{D}$ are set to 2, which represents features calibration value and luminosity difference. But more features, such as the difference in the timestamps between two entries, can be added if needed.
\end{itemize}

\subsection{Training Seq2Seq Model}
The Seq2Seq model was built using the PyTorch library. Both encoder and decoder blocks were set with 1024 hidden layers. The number of LSTM cells in the encoder and the decoder is varied to scan for the optimal input and output lengths. The LSTMs were initialized with a sigmoid activation for input, forget, and output gates and a hyper-tangent activation for the cell gates. The Mean Squared Error (MSE) loss function along with the Adam~\cite{kingma2014adam} optimizer is used to train the model. The model is trained for 200 epochs with a batch size of 128 and a learning rate of $10^{-3}$. A higher number of epochs (3000) were used to check if the model performance improves, but it was found that it converges after about 200 epochs. All trainings and predictions were performed on machine with Intel Xeon Silver 4214R@2.40GHz, and Nvidia RTX A6000 graphics card with 48 GB memory.

With a large number of data points, there are several options available for training a model. A single model can be developed for each of the individual crystals. On the other hand, the data points of different crystals that are at equal distances from the center of ECAL, i.e., within one i$\eta$-ring of the ECAL, can be combined together. The assumption is that these crystals receive an equal amount of radiation dose because of the radial symmetry and hence will have similar behavior in laser response over a course of time. Then this model trained with more data points would be able to predict calibrations for all the crystals in the corresponding i$\eta$-ring. In addition to changing the number of crystals, three different strategies were used which are given as follows:
\begin{enumerate}
\item Recursive: In this setting, as shown in Figure \ref{fig:train_strg} (left), we feed the token from $Y_{\text{pred}}$ from the previous time step as the input to the current time step.
\item Teacher Forcing: In this setting, as shown in Figure \ref{fig:train_strg} (right), we feed the token from $Y_{\text{true}}$ (instead of the token from $Y_{\text{pred}}$) from the previous time step as the input to the current time step.
\item Mixed: In this setting, the previous two strategies can be combined in different ratios. For example, a mixed training with a teacher forcing ratio of 0.7 means that only 70\% of the batches in the decoder training use the teacher forcing strategy. 
\end{enumerate}

\begin{figure}[ht]
  \centering
  \includegraphics[width=0.45\textwidth]{img/our_seq2seq_recursive.png}
  \includegraphics[width=0.45\textwidth]{img/our_seq2seq_tf.png}
  \caption{\textbf{(left)}: Seq2Seq model with recursive training; \textbf{(right)}: Se2Seq model with teacher forcing.}
  \label{fig:train_strg}
\end{figure}

% \newpage

\section{Results}
\section*{Results}
We started by assembling a dataset derived from public hikes. This process included an iterative data cleaning process to remove erroneous/false data, identify and remove breaks (e.g. Fig \ref{Fig2}) to give us a final usable dataset containing 7,636 GPS tracks, with over 1.4 million individual data points and covering almost 88,000 km of travel in the U.K. 

Our curated hike dataset allowed us to create a data-driven model which we can directly compare with existing walking speed algorithms. The model formulation was selected using a small-scale exploratory study which considered data from Scotland (see \nameref{S3_Appendix}). In this exploratory study, multiple different model types were explored which could fit the data, and which matched existing knowledge about walking speeds. Cross-validation methods showed that there was very little difference in performance of the best models, therefore the final model was a Generalised Linear Model (GLM), which was chosen as it was the simplest of those tested (we had no evidence that a more complex model would be superior). This choice also meant that our model was both easy to interpret, and simple to apply to future work.

This final GLM model included all three of the variables suggested by Arnet \cite{Arnet2009ArithmeticalJapan}:

\begin{equation}
    v = exp(a+b\phi+c\theta+d\theta^2)
\end{equation}
where
\begin{quote}
$v = \text{walking speed (km/h)}$\\
$\phi = \text{hill slope angle (degrees)}$\\
$\theta = \text{walking slope angle (degrees)}$
\end{quote}

Terrain obstruction level was included as a factor variable, while we considered the road types as both factor variables and interaction terms. Not all terms had a significant effect on all variables; we therefore created a model with all possible terms, and removed them one at a time (in order of least significance) until all remaining terms were significant to at least 95\% confidence  level (using Wald test). The final values for a, b, c and d are given in Table \ref{tab:2ROUK model variable values} for each of the terrain obstruction levels and road types. The critical gradient for this model is between 14 -- 16 degrees when walking uphill and -16 -- -18 degrees when walking downhill (depending on road and obstruction conditions), which is in line with previous findings. 

Fig \ref{Fig3} shows the predicted walking speeds under different conditions. The importance of including both the hill slope and terrain obstruction variables can be clearly seen when looking at the Off Road Light Obstruction speed predictions. When directly ascending or descending a slope, the walking speed is comparable to walking on a road. However, when traversing a slope while off road, the walking speed is comparable to traversing a slope of double the gradient while on a road or path. Similarly, comparing the walking speed predictions of Off Road Light Obstruction and Off Road Heavy Obstruction reveals that just 10 cm of vegetation (our cutoff point for heavy obstruction) can reduce the walking speed by more than 0.5 km/h.

\begin{table}[!ht]
\begin{adjustwidth}{-0.5in}{0in}
    \centering
    \caption{Final walking speed model variable coefficients}
    \begin{tabular}{|l+c|c|c|c|}
    \hline
    & $a$ & $b$ & $c$  & $d$ \\ 
    \thickhline
    Paved road & 1.580 & -0.00389 & -0.00726 & -0.00218 \\ 
    \hline
    Unpaved road & 1.580 & -0.00389 & -0.00965 & -0.00248 \\
    \hline
    Off-road (obstruction unknown) & 1.536 & -0.00731 & -0.00965 & -0.00187 \\
    \hline
    Off-road (light obstruction) & 1.580 & -0.00731 & -0.00965 & -0.00187 \\ 
    \hline
    Off-road (heavy obstruction) & 1.400 & -0.00731 & -0.00965 & -0.00187 \\ 
    \hline
    \end{tabular}
    \label{tab:2ROUK model variable values}
\end{adjustwidth}
\end{table}

\begin{figure}[!h]
\begin{adjustwidth}{-2.25in}{0in} 
    \includegraphics[width=\linewidth]{Images/Paper/Fig3.eps}
    \captionsetup{width=1\linewidth}
    \caption[width=\textwidth]{{\bf Walking speed predictions under different terrain conditions.}  When: (A) travelling directly up or down hills of varying slope, (B) traversing across hills of varying slope.}
    \label{Fig3}
    \end{adjustwidth}
\end{figure}

Fig \ref{Fig4} compares the Paved Road and Off Road Heavy Obstruction speed predictions from our model against the existing functions from Naismith, Tobler and Campbell et al. When looking at the walking slope, the largest areas of deviation between our model and Naismith's rule occurs when descending a slope, as Naismith's rule does not predict a reduced speed in this scenario. For both Tobler's and Campbell et al.'s functions, the shape of the walking slope component is relatively similar to our new model, with the main distinction being the peak predicted speed on flat ground. None of the existing functions account for the hill slope, which leads to large disparities when predicting the walking speed for slope traversals. A further example of this can be seen in \nameref{S6_Appendix}, which shows the walking speeds for a simulated off-road route which encounters the full range of hill and walking slopes.

\begin{figure}[!h]
\begin{adjustwidth}{-2.25in}{0in} 
    \includegraphics[width=\linewidth]{Images/Paper/Fig4.eps}
    \captionsetup{width=1\linewidth}
    \caption[width=\textwidth]{{\bf Comparison of new model and existing hiking functions.}  Predicted walking speeds of the new model, Naismith's rule, Tobler's function and Campbell et al.'s function when: (A, C, E) travelling directly up or down hills of varying slope, (B, D, F) traversing across hills of varying slope.}
    \label{Fig4}
\end{adjustwidth}
\end{figure}

When comparing the performances of each of the models (Table \ref{tab:2comparison}), the predicted speeds for individual 50 m sections had a lower RMSE and percentage error, and a higher R squared value using our new model than in the existing ones. To isolate the impact of each of the slope variables, we filtered the results to look at the data where a slope was being directly climbed or traversed. Figs \ref{Fig5}A, B and \ref{Fig6}A, B show the RMSE and mean residuals for each of the models, for data which was within 5 degrees of directly climbing (A) or traversing (B) hills of varying slope. From this we can clearly see that Naismith's rule consistently overestimates walking speeds when descending a slope, and underestimates speeds when climbing a slope. When ascending or descending a slope, the RMSE of our GLM is similar to that of Tobler's hiking function. However, one of the main areas where we see an improvement using our model is on slight declines. Tobler's hiking function suggests that walking speed increases on mild descents up to a maximum of 6 km/h. It is clear from Fig \ref{Fig5}A, that Tobler's function overestimates the walking speed in this region. Campbell et al.'s function has a slightly lower RMSE value than our new model on the steepest walking slopes, however it underestimates the walking speeds on flat ground and mild slopes. Previous research has found that most walking takes place on low walking slopes \cite{Proffitt1995PerceivingSlant}, and this is evidenced by our data ($\sim$98\% of our data was from walking slopes of under 10 degrees). Improved walking speed predictions in this region therefore have the greatest impact in real-world situations. Within this region our model consistently has a lower RMSE than the existing functions, and a mean residual error close to 0 km/h. 

\begin{table}[!ht]
\centering
\caption{Comparison of new model against existing methods to calculate walking speeds.}
\begin{tabular}{|l|c|c|c|c|}
\hline
& New Model & Naismith & Tobler & Campbell\\
\hline
Average \% error & 23.68 & 26.36 & 26.17 & 25.33\\
\hline
MSE & 1.20 & 1.61 & 1.53 & 1.58\\
\hline
RMSE & 1.10 & 1.27 & 1.24 & 1.26\\
\hline
R\textsuperscript{2}  & 0.09 & -0.22 & -0.16 & -0.19\\
\hline
\end{tabular}
\label{tab:2comparison}  
\end{table}

\begin{figure}[!h]    
\begin{adjustwidth}{-2.25in}{0in} 
    \includegraphics[width=\linewidth]{Images/Paper/Fig5.eps}
    \captionsetup{width=1\linewidth}
    \caption[width=\textwidth]{{\bf Comparing RMSE values for the new model, Naismith's rule, Tobler's function and Campbell et al.'s function.} When: (A) travelling directly up or down hills of varying slope (all data), (B) traversing across hills of varying slope (all data), (C) travelling directly up or down hills of varying slope (off-road data only), (D) traversing across hills of varying slope (off-road data only). Campbell et al.'s function does not provide off-road speed estimates, so was not included in the off-road data comparisons.}
    \label{Fig5}
\end{adjustwidth}
\end{figure}

\begin{figure}[!h]
    \begin{adjustwidth}{-2.25in}{0in} 
    \includegraphics[width=\linewidth]{Images/Paper/Fig6.eps}
    \captionsetup{width=1\linewidth}
    \caption[width=\textwidth]{{\bf Comparing mean residual values for the new model, Naismith's rule, Tobler's function and Campbell et al.'s function.} When: (A) travelling directly up or down hills of varying slope, (B) traversing across hills of varying slope, (C)  travelling directly up or down hills of varying slope (off-road data only), (D) traversing across hills of varying slope (off-road data only). Campbell et al.'s function does not provide off-road speed estimates, so was not included in the off-road data comparisons.}
    \label{Fig6}
\end{adjustwidth}
\end{figure}

 We also see an improvement in RMSE when using our model to predict speeds for hill traversals (Fig \ref{Fig5}B). We can note from Fig \ref{Fig6}B that both Naismith's rule and Tobler's hiking function consistently overestimate the walking speed when traversing a slope, as they do not take into account the impact that the hill slope has on reducing walking speeds. The performance of Campbell et al's model improves as the hill slope increases, although we suggest this is more due to it underestimating the speed on shallow slopes. We do see that the average error in our model increases as the hill slope increases, but we believe that this is due to limited volumes of data at high hill slopes ($\sim$0.5\% of our data occurs on hill slopes steeper than 40 degrees). 

As well as looking at the overall performance of our new model, we looked to explore how well our model performed in off-road conditions, compared to the off-road adjustments for the existing functions (Naismith's reduced base speed of 4 km/h, and Tobler's correction factor of 0.6). Figs \ref{Fig5}C, D and \ref{Fig6}C, D show the RMSE and mean residuals, only considering data which was recorded in off-road conditions. From Figs \ref{Fig5}C and \ref{Fig6}C it is clear that Tobler's function consistently underestimates the walking speed when off-road. The factor of 0.6 is a larger reduction in walking speed than is observed in practice. As we found when looking at our data as a whole, Naismith's rule underestimates the walking speed when climbing a slope and overestimates when descending a slope. Our new model does not suffer from these problems, with both a lower RMSE and lower absolute mean residual value across all walking slopes. Both of these existing models also consistently underestimate walking speeds when traversing a slope, unlike our new model which has a mean residual of less than 0.4 km/h on slopes of up to 35 degrees. The error in predictions of our new model does increase as the hill slope increases, though the RMSE is generally lower than seen in the existing models. On the steepest hill slopes our model appears to perform less well than the existing ones, though only 0.2\% of our off-road data occurred on a hill slope steeper than 40 degrees. 

Although we have shown an improvement in walking speed predictions over short sections of routes, this did not translate to similar results when looking at predicted walking times for routes as a whole. Our model and all of the existing models which we have explored here had an average percentage error of 13.5\% - 15.5\% when predicting the time taken for a complete route. However, based on the errors seen in Figs \ref{Fig5} and \ref{Fig6}, we believe that this is a result of errors cancelling out over the course of a hike. For example while ascending a hill, Naismith's rule will underestimate the walking speed (and thus overestimate the walking time), but it will then overestimate the walking speed on the subsequent descent, leading to a relatively accurate total time estimate. The results here suggest that Naismith's rule, and other existing functions, are still a good rule of thumb to calculate route times as a whole, but time estimates for individual sections of a route will be less accurate than when using the new model found here.




% \newpage

\section{Summary}
% !TEX TS-program = pdflatex
% !TEX root = ../ArsClassica.tex

%*******************************************************
% Lay Summary
%*******************************************************
\pdfbookmark{Lay summary}{Lay summary}
\addcontentsline{toc}{chapter}{Lay summary}

\chapter*{Lay summary}

\looseness=-1 Over the past few decades, the volume of astronomical and cosmological data has increased substantially. In response to that, a variety of astrophysical models have been proposed to explain the plethora of observations. As the information provided by the data is always incomplete and uncertain, inferring the properties of a model, including the values of its parameters, given the observed data, generally requires us to reason in the face of uncertainty. In the context of \textit{Bayesian inference}, uncertainty is represented by the notion of probability. One usually starts by quantifying their state of knowledge about the possible values of the model parameters \textit{prior} to seeing the data, in the form of a probability distribution called the \textit{prior}. The next step is to use the so--called \textit{Bayes' theorem} in order to update one's degree of belief about the model parameters given the available data. The outcome of this updating process is the \textit{posterior} probability distribution of the model parameters given the data which quantifies the plausibility of different parameter values.

Approximating the \textit{posterior} generally requires the use of probabilistic computational methods. Standard practice in astronomy often employs conventional computational tools (e.g. \textit{Markov chain Monte Carlo}) despite their specific theoretical limitations or narrow range of validity. The aim of this thesis is to first introduce the basic principles of Bayesian inference along with the basic methods used for Bayesian computation and then present two novel algorithms and their respective software implementations. A common element of these newly developed tools is their ability to exploit the available information about the geometry of the posterior in order to approximate it more quickly. Finally, both methods are able to benefit from the possible availability of multiple CPUs in order to accelerate their computation.


\section*{Acknowledgments}
This work has been supported by the Department of Energy, Office of Science, Office of Advanced Scientific Computing under award number DE-SC0021395. 
The authors would like to express their gratitude to the CMS Collaboration, and in particular to the CMS ECAL community for making the crystal transparency data available, and to the CMS beam, radiation and integrated luminosity group for supplying the luminosity data. We would also like to thank our colleagues from the FAIR4HEP group for discussions and their invaluable inputs and suggestions for writing this paper.

%Bibliography
\bibliographystyle{unsrt}  
\bibliography{references}

\end{document}
