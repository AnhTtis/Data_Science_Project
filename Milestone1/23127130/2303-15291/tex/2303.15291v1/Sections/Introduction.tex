


The Compact Muon Solenoid (CMS) experiment at the CERN Large Hadron Collider (LHC) is designed to detect and measure particles produced in the proton-proton collisions at the LHC~\cite{CMS_2008}. 
One of the components of the CMS detector is an electromagnetic calorimeter (ECAL) made of lead tungstate crystals that is used to measure the energy of electrons and photons produced in the collisions. This was the principle detector used to detect the Higgs boson in the two-photon decay channel in the discovery of the Higgs boson in 2012~\cite{Chatrchyan:2012ufa}, in the precision measurement of the Higgs boson mass~\cite{HGG}, and in many other measurements made by the CMS Collaboration.
Changes in the crystal transparency caused by radiation from the collisions in the LHC lead to changes in the response to electrons and photons that needs to be tracked and corrected for. 
Predicting these changes both in the short term and in the long term is a critical question for the CMS experiment. We describe here the public release following FAIR principles~\cite{Chen_2022} of the crystal monitoring data collected by CMS between 2016 and 2018.  Besides describing the dataset and its access, and the problem to be addressed with it, we provide links to an example solution based on a long short-term memory (LSTM) neural network, developed to predict future behavior of the crystals. 

\subsection{Electromagnetic Calorimeter (ECAL)}

There are 75,848 lead tungstate crystals in the ECAL, of which 61,200 are arranged in a barrel surrounding the interaction point where the beams collide, and the barrel is capped by two endcap calorimeters, each consisting of 7,324 crystals~\cite{ECALTDR}. The crystals in CMS measure approximately 22 \cm in length with a $2\times2 \cm^2$ section and weigh $\approx 1.1 \unit{kg}$. Lead tungstate crystals are optically transparent and emit a short pulse of light when they absorb ionizing radiation.  Due to the intense radiation while the LHC is in operation, the optical transparency of the crystals is reduced over time. 
This reduction is due to the creation by the radiation of atomic-level impurities in the crystal that act as color centers~\cite{914439-xtal-color-centers} that absorb light propagating through the crystal. 
Deep well impurities are stable and persist for years and the radiation damage is permanent, while shallow impurities are metastable and are short-lived. During beam-beam collisions the optical transmission is reduced and it partially recovers when the beams are off as the meta-stable states decay.

When the LHC is operating, the beam-beam collisions are continuous, with a `fill' generally lasting for approximately 18 hours and a four- to six-hour interval between fills. There are longer intervals when there are no beams for maintenance of the LHC and the detectors. These typically last on the order of a day, and every year there are several-month-long shut downs. 

During a fill there are proton-proton collisions at a rate of about 1 GHz at the center of the CMS detector. These collisions produce a copious number of secondary particles that travel through the detector and are absorbed. It is these secondary particles that induce the production of color centers in the ECAL crystals, and the more particles produced in the proton-proton collisions: the more color centers are created in the crystals. Hence the number of color centers produced is proportional to the number of collisions in the LHC, which is measured in terms of the `luminosity'.  Luminosity has units of inverse centimeters, or, since these are sub-nuclear processes, in terms of inverse femtobarns (fb$^{-1}$), where a femtobarn is $1\times10^{-39} \rm{cm}^2$.\ \footnote{For one inverse femtobarn of delivered luminosity we expect about one event with a cross-section of a femtobarn to occur. For example, at the LHC the process where a Higgs boson is produced has a cross section of 55,400 femtobarns, then with one inverse femtobarn of luminosity we expect that 55,400 Higgs bosons will be produced. Unfortunately many fewer are detected.}

Since detailed knowledge of the crystal's light output due to ionizing radiation is essential for the physics measurements, the transparency of each crystal is carefully monitored with a laser monitoring system that injects pulses of light into each crystal at intervals of approximately every 40 minutes. 

The principle of operation of the crystal calorimeter, which is designed to detect electrons and photons with energies of 1\GeV or more, is as follows: When a high energy electron or photon ($>100 \MeV $) is incident on the crystal electron and positron pairs are produced, these in turn interact with the atomic nuclei and radiate (bremsstrahlung) photons. These in turn produce more electron-positron pairs, which in turn radiate photons, though always with less energy. This sets up a cascade where photons and electron-positron pairs are produced that continues until the photons have insufficient energy to create further electron-positron pairs ($< 1 \MeV$). During this process the electrons and positrons produced propagate in the crystal and ionize the atoms causing scintillation light (optical photons) to be produced that can be detected with a photodetector coupled to the crystal. The operating principle of the CMS ECAL is that the amount of ionizing radiation is linearly proportional to the energy of the incident electron or photon, and hence the light output is proportional to the energy of the incident particle. 

The cascade is a stochastic process, thus the energy measured with the crystals is measured with a resolution that can be parameterized as:

\begin{equation}
\dfrac{\sigma_{E}}{E}=\dfrac{S}{\sqrt{E}}\oplus\dfrac{N}{E}\oplus C
\end{equation}

where the first term on the right is the \textit{stochastic} term, which is the contribution to the resolution of natural fluctuations in the cascade; second is the \textit{noise} term due to electronic noise, and the third is the \textit{constant} term that accounts for energy leakage and other signal losses. The parameters $S$, $N$ and $C$ have been measured with a prototype in a dedicated particle beam. For electrons with $p_{T}$ greater than 10 GeV, the energy resolution is better than 1\%. 

A convenient approximation to the production angle is the pseudorapidity ($\eta$)~\cite{ECALTDR}, which is equal to zero in the center of the barrel and increases to $\pm 1.41$ at the end of the barrel.
%This difference in the level of radiation damage in the crystals due to different radiation levels can be seen in Figure \ref{fig:crystal_transparency_runII}. 
The changes in the crystal transparency for different regions as they are being irradiated for the barrel calorimeter can be seen in Figure \ref{fig:crystal_transparency_runII}.

\begin{figure}[htbp]
	\centering
	\includegraphics[width=0.8\linewidth]{img/crystal_transparency_runII.png}
%	\captionsetup{justification=centering}
	\caption{Relative response of crystal to laser light in different $\eta$ regions of ECAL between 2016 and 2018 when the total luminosity of the LHC was 151~fb$^{-1}$. The crystals closest to the center of the  barrel ($|\eta| = 0$)  all have less light loss compared with the crystals that are closest to the end of the barrel  ($|\eta| = 1.15$).}
	\label{fig:crystal_transparency_runII}
\end{figure}