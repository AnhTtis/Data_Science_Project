\section{Datasets}
\label{datasets}
The response to injected laser pulses for every crystal is measured approximately every 40 minutes and stored in an offline database. Data corresponding to the laser response of all 75848 crystals from 2016 through 2018 were extracted from the database.  Each entry has a timestamp corresponding to the time when the measurement was taken, and the current intensity of the beam-beam collisions (the instantaneous luminosity) which is a measure of the radiation dose being received by the crystals.  

The dataset consists of the following elements corresponding to a single measurement.

\begin{itemize}
    \item \textbf{xtal\_id}: Crystal identification number within ECAL ranging from [0, 75848]. 
    \item \textbf{start\_ts}: Start of the Interval of Validity (IOV). An Interval of Validity corresponds to a time during which a measurement is taken for a single crystal. In other words, each IOV contains one measurement per crystal.
    \item \textbf{stop\_ts}: End of the Interval of Validity (IOV).
    \item \textbf{laser\_datetime}: Timestamp of the measurement for a given crystal within an IOV. The timestamp lies between the start of IOV and end of IOV.
    \item \textbf{calibration}: APD/PD ratio taken at laser\_datetime. This value is used to quantify the transparency of the crystal at the time of measurement.
    \item \textbf{time}: Time corresponding to the luminosity measurement (obtained from BRIL) closest to the time when the laser measurement was taken.
    \item \textbf{int\_deliv\_inv\_ub}: Approximate integrated luminosity delivered up to the measurement, in the units of inverse microbarns.
\end{itemize}

To ensure the FAIR-ness of the publication of the dataset, it has been published~\cite{bhargav_joshi_2022_7510572} on Zenodo\footnote{\href{https://zenodo.org/}{https://zenodo.org/}} platform, which was launched in May 2013 as part of the OpenAIRE project, in partnership with CERN. 
%It is a repository funded by the European Commision and is widely used by the scientific community to publish large datasets. 
The dataset consists of 26 files in \textit{tar gzip} format, each file consisting of up to 10 csv files that contain measurements of up to 360 crystals. The files corresponding to the +z side of the ECAL are labelled as "plus" and those corresponding to -z side are labelled "minus". The list of $i\eta$ rings along with their position in terms of pseudo-rapidity ($\eta$) and azimuthal angle ($\phi$) in each tar file are included in form of json files under the metadata section in Zenodo.