\section{Machine Learning Solutions}

We describe in this and subsequent sections two ML solutions that we have developed to address this problem, their training and the results we have obtained. The source code is published on GitHub at \textcolor{blue}{\href{https://github.com/FAIR-UMN/fair_ecal_monitoring}{https://github.com/FAIR-UMN/fair\_ecal\_monitoring}}.

Neural networks (NNs) are machine learning (ML) methods that mimic the biological structure and functioning of neurons in a brain. A NN that does not involve any cyclic connections is called a Feedforward neural network (FNN). Deep neural networks (DNNs) are structures that consist of many stages of interconnected neurons. DNNs perform well for hard learning tasks, such as object identification and speech recognition. However, they require that the inputs and outputs of the task be encoded into vectors with fixed dimensionality \cite{sutskever2014sequence}. 

%It consists of units called \textit{neurons}, each activating to produce an output based on the inputs received. The inputs received by a neuron are weighted according to the weight assigned for each input connection and in some cases a bias is added. Following the input transformation, an activation function is applied to the input to produce an output. The model is \textit{trained} to tune the weights assigned to the input connections to maximize the accuracy of the final output. The number of neurons and their inter-connectivity can vary depending on the given problem. Due to the non-linearity introduced by the \textit{activation functions} at each node, determining the weights becomes analytically non-trivial. In NN models which are trained using teacher-based \textit{Supervised Learning}, an algorithm based on gradient descent method, known as \textit{backpropagation}\cite{backpropagation} efficiently determines the optimum set of weights.  

Many problems, like machine translation and speech recognition, have a sequential structure, since their input and output lengths are not known \textit{a-priori}~\cite{sutskever2014sequence}. 
%The same architectures have also been implemented to solve time series prediction problems. 
A class of neural networks called recurrent neural networks (RNNs) are a type of FNNs that pass the data sequentially between different nodes. This architecture allows the network to learn and to retain past knowledge when processing data points from a given data series. RNNs have some shortcomings---in particular, the vanishing gradient problem~\cite{vanishing-gradient} while training the network. Other architectures have been developed in the past to address these issues.

% This feature allows them learn the trends in data series over long ranges and make accurate forecasts. 
\subsection{Long Short Term Memory (LSTM) Models}

LSTM models are a type of RNNs that include feedback components. RNNs are good at tracking arbitrary long-term dependencies in a sequence but have a tendency to be unstable during training. LSTMs solve the vanishing-gradient problem through an additive gradient structure. The LSTM cell is shown in Figure \ref{fig:encoder-decoder}, which is the key part of the Seq2Seq model. For each element in the input sequence, each layer of the LSTM computes the following functions: 
\begin{align}
    \begin{split}
        & \vi_t = \sigma(\mW_{ii}\vx_t + \mathbf{b}_{ii} + \mW_{hi}\vh_{t-1} + \mathbf{b}_{hi} ) \\
        & \vf_t = \sigma(\mW_{if}\vx_t + \mathbf{b}_{if} + \mW_{hf}\vh_{t-1} + \mathbf{b}_{hf} ) \\ 
        & \vg_t = \tanh(\mW_{ig}\vx_t + \mathbf{b}_{ig} + \mW_{hg}\vh_{t-1} + \mathbf{b}_{hg} ) \\
        & \vo_t = \sigma(\mW_{io}\vx_t + \mathbf{b}_{io} + \mW_{ho}\vh_{t-1} + \mathbf{b}_{ho} ) \\
        & \vc_t = \vf_t \odot \vc_{t-1} + \vi_t \odot \vg_t \\
        & \vh_t = \vo_t \odot \tanh{\vc_t} 
    \end{split}
\end{align}

where $\vh_t,\vc_t$ are the hidden and cell state at time $t$, $\vx_t$ is the input at time $t$, $\vi_t,\vf_t,\vg_t,\vo_t$ are the input, forget, cell, and output gates, respectively. $\sigma$ is the sigmoid function and $\odot$ is the Hadamard product~\cite{lstm}.

% In multi layer LSTM, the input $\vx_t^{(l)}$ of the $l$-th layer ($l\geq 2$) is the hidden state $\vh_t^{(l-1)}$ of the previous layer multiplied by dropout $\delta_t^{(l-1)}$.
% \newcommand{\bbox}{\text{bbox}}
\newcommand{\alphapck}{\alpha_\bbox}
\newcommand{\kcycle}{\text{k-CyPCK}}
\newcommand{\cycle}{\text{-CyPCK}}

\newcommand{\I}{\mathbf{I}}
\newcommand{\Ia}{\I^\text{a}}
\newcommand{\Ib}{\I^\text{b}}
\newcommand{\Iatob}{\I^\text{a $\rightarrow$ b}}
\newcommand{\F}{\mathbf{F}}
\newcommand{\Fa}{\F^\text{a}}
\newcommand{\Fb}{\F^\text{b}}
\newcommand{\f}{\mathbf{f}}
\newcommand{\fa}{\f^\text{a}}
\newcommand{\fb}{\f^\text{b}}
\newcommand{\p}{\mathbf{p}}
\newcommand{\pa}{\p^\text{a}}
\newcommand{\pb}{\p^\text{b}}
\newcommand{\A}{\boldsymbol{\Phi}_\text{align}}
\newcommand{\G}{\mathbf{G}}
\newcommand{\C}{\mathbf{C}}
\newcommand{\Ca}{\C^\text{a}}
\newcommand{\Cb}{\C^\text{b}}
\newcommand{\cc}{\mathbf{c}}
\newcommand{\cca}{\cc^\text{a}}
\newcommand{\ccb}{\cc^\text{b}}
\newcommand{\Irec}{\I_\text{Recon}}
\newcommand{\M}{\mathbf{M}}
\newcommand{\Mrec}{\M_\text{Recon}}
\newcommand{\loss}{\mathcal{L}}
\newcommand{\T}{\mathcal{T}}
\newcommand{\W}{\mathcal{W}}
\newcommand{\Id}{\mathcal{I}}

% \input{Utils/ju_math.tex}

\subsection{Seq2Seq Model}

A sequence-to-sequence (Seq2Seq) model is an architecture that combines two or more LSTMs. It consists of two parts---the encoder and the decoder, each of which is built by using separate LSTMs. This type of model has been developed for automatic language translation, where a sentence from one language is translated to another language \cite{sutskever2014sequence}. The encoder is used to process each token in the input sentence, and encode all the input sequence information into a fixed-length vector. The transform vector, known as context vector, is a vector in a latent space and it encapsulates the whole meaning of the input sequence. The decoder reads the context vector and predicts the target sequence token by token.

Figure \ref{fig:encoder-decoder} shows the basic architecture of the encoder-decoder network used for this problem. The encoder block consists of LSTM units connected in series that take in a set of calibration values and the luminosity differences between those calibration values as input. All the information from the input sequence is encapsulated in the internal states $\vh_t$ (hidden state) and $\vc_t$ (cell state). The decoder block is another block of LSTM units connected in series. The final states $(\vh_t,\vc_t)$ of the encoder are used as the initial states $(\vh_0,\vc_0)$ of the decoder, which is the context vector used to predict the target sequence. The decoder network also takes input along with the initial states to predict the target sequence. The input to the decoder varies according to the method used for the training.

The Seq2Seq model can be trained using teacher forcing method, where the decoder is trained using the target output (ground truth output) instead of the output generated by the decoder in the previous step of the sequence. However, during the evaluation step, the decoder generates the output sequentially using the output generated in the previous step. Using teacher forcing during training has been shown to improve the training process. To determine if the training process has converged, the model was trained for 300 epochs and the validation e
%In this problem, the input $X$ is an English sentence (e.g., "nice to meet you"), and the Seq2Seq model is used to find a prediction $Y_{pred}$ that is closest to its corresponding French translation $Y_{true}$ (e.g., "ravi de vous rencontrer"). A token in this case is a single word in the sentence.

\begin{figure}[!ht]
  \centering
  \includegraphics[width=1.0\textwidth]{img/Seq2Seq_arch.png}
  \caption{Seq2Seq model for used for predicting future calibration values (\textbf{lower left}). The Encoder block (\textbf{upper left}) and Decoder block (\textbf{lower right}) are a set of sequentially connected LSTM units (\textbf{upper right}).}
  \label{fig:encoder-decoder}
\end{figure}