

Recurrent neural networks, such as long short-term memory and sequence-to-sequence models, can be used to predict the response of the CMS electromagnetic calorimeter crystals in the future. The reduction in the crystal transparency as a function of the luminosity of the CERN LHC, or their recovery, when there is no radiation, can be captured by these models. The dataset of the time development of the response to laser light for each crystal and of the luminosity of the LHC, between 2016 and 2018 is now available publicly. We have demonstrated how standard machine learning tools can be used to predict future changes to the crystals.  It is our intent that by publishing our data and our models, including the source code, it will generate interest in the wider computer science and physics communities leading to the development of novel tools to address the problem of how to predict the future performance of the CMS crystals in the both the short term,  and as far ahead as 2035, when the CMS detector will still be taking data.