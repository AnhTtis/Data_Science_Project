The details of the color center formation under radiation are poorly understood. It is thought that they are primarily determined by the atomic-level defects in the crystals when they were first grown and, thus are determined by the individual history of each crystal. From this it follows that the best way to predict the future behavior of a crystal is by examining their past behavior under radiation. 

It is valuable to the CMS collaboration to be able to accurately predict how the ensemble of crystals will respond to future irradiation cycles with damage and recovery. In particular there is interest in predicting the level of light-loss for a group of crystals after a duration of the order one day, when the integrated luminosity is $\approx 1 \rm{fb}^{-1}$. Another point of interest is to predict the performance of the crystals in the barrel ECAL up to the end of operation of the High-Luminosity LHC in 2040, when the total delivered luminosity is expected to be 3000 fb$^{-1}$

In the examples given below the short-term problem is tackled. The long term problem is left as an open challenge.