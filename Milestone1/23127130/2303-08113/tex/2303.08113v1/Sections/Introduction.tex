\section{Introduction}
Deformable image registration (DIR) aims to align two or more images by optimising a non-linear voxel-wise transformation  between given target and source images. DIR is a fundamental task in medical image analysis, and plays a key role in various clinical applications including image-guided interventions~\cite{tam2016image}, 
diagnosis~\cite{tekchandani2022computer} and treatment planning~\cite{krilavicius2016predicting}.
%
Traditional methods treat DIR as a pair-wise optimisation problem, which relies on specific modelling assumptions. 
As a result, flexibility is limited, and it is difficult to handle complex and unpredictable deformations. Furthermore, the selection of an appropriate model, that accurately represents the deformation, can be challenging.


Learning-based approaches have been widely studied for DIR and achieved promising results e.g.~\cite{cciccek20163d,heinrich2019closing,long2015fully,ronneberger2015u,liu2019probabilistic,zou2022deformable}.
Following this research line, several techniques and strategies have been proposed to improve registration performance. For example, cascaded networks~\cite{zhao2019recursive}, pyramid networks~\cite{mok2020large} or joint models~\cite{liu2021rethinking}.
However, the improvements are limited
due to two main challenges. Firstly, existing models assume to have, for the training stage, a large and well-represented dataset. However,  the representations learned from training data cannot obtain the optimal  complex deformations for all image pairs. Moreover, 
current learning-based methods cannot theoretically and strictly guarantee diffeomorphism of the deformation field.
Although $L^2$ regularisation \cite{balakrishnan2019voxelmorph,hering2021cnn}, cycle consistency constraint \cite{kim2021cyclemorph} and inverse consistency constraint \cite{zhu2022swin,liu2022pc} are utilised to improve the quality of deformation fields, 
these frameworks strictly cannot theoretically guarantee diffeomorphic transformations--this is reflected in reporting negative values in the Jacobian determinant.

To handle the aforementioned challenges, optimising pair-wise transformations has been explored from the variational and learning perspectives. This set of techniques mitigates the need of having a large dataset. 
A crucial factor, for any type of image registration methods, is the regulariser used to enforce plausible transformations.  For example, the community largely has used Laplacian constraints, incompressibility constraints, and fluid-like regularisers~\cite{passieux2012high,christensen1996deformable,mansi2011ilogdemons}. However, these regularisers fail to handle large deformations or do not capture the characteristics of the target tissues. Perhaps hyperelastic regularisation~\cite{veress2005measurement,phatak2009strain,burger2013hyperelastic} is the most widely used constraint in the medical domain; as this principle matches with the characteristics of biological tissues. Learning-based methods for pair-wise registration have also used such types of regularisers. For example, the work of that~\cite{wolterink2022implicit} used the existing hyperelastic regulariser of~\cite{burger2013hyperelastic}. A weak constraint has been imposed on the Jacobian determinant in~\cite{han2023diffeomorphic}. In contrast to those works that use existing regularisers. In this work, we focus on proposing a novel regulariser that is more flexible and can provide theoretical guarantees. 



\textbf{Contributions.} 1) We propose a novel pair-wise image registration framework that does not require pre-training or prior affine registration. 2) The key contribution of this paper is that we introduce a novel conformal-invariant regulariser. The proposed regulariser unifies the hyperelasticity with conformal and Beltrami-based approaches. It enables simultaneous control of changes in length, area, and volume and of distortion with conformal mappings, as well as smoothness of deformations. Most importantly, \textit{our regulariser yields to homeomorphic transformations enforcing plausible transformations and theoretical guarantees.} 3) A learned image deformation mapping that is driven by our novel regulariser via coordinate MLP. 4) We demonstrate, through extensive experimental results, that our proposed framework yields to better performance than existing implicit and explicit regularisers.







