\documentclass[10pt,twocolumn,letterpaper]{article}

\usepackage{iccv}
\usepackage{times}
\usepackage{epsfig}
\usepackage{graphicx}
\usepackage{amsmath}
\usepackage{amssymb}
\usepackage{multirow}
\usepackage{bbm}
\usepackage{enumitem}
\usepackage{booktabs}
\usepackage[table,dvipsnames]{xcolor}
\usepackage{array}
\newcolumntype{L}[1]{>{\raggedright\let\newline\\\arraybackslash\hspace{0pt}}p{#1}}

% Include other packages here, before hyperref.

% If you comment hyperref and then uncomment it, you should delete
% egpaper.aux before re-running latex.  (Or just hit 'q' on the first latex
% run, let it finish, and you should be clear).
\usepackage[breaklinks=true,bookmarks=false]{hyperref}
\usepackage[capitalize]{cleveref}
\crefname{section}{Sec.}{Secs.}
\Crefname{section}{Section}{Sections}
\Crefname{table}{Table}{Tables}
\crefname{table}{Tab.}{Tabs.}
\newcommand{\smallsec}[1]{\vspace{0.04in} \noindent {\bf #1.}}
\newcommand{\olga}[1]{{\color{magenta} Olga: #1}}
\newcommand{\todo}[1]{{\color{red} TODO: #1}}
\newcommand{\vikram}[1]{{\color{ForestGreen} Vikram: #1}}
\newcommand{\ruth}[1]{{\color{BurntOrange} Ruth: #1}}
\newcommand{\sunnie}[1]{{\color{Violet} Sunnie: #1}}
\newcommand{\nicole}[1]{{\color{Cerulean} Nicole: #1}}
\DeclareMathOperator*{\argmax}{arg\,max}
\DeclareMathOperator*{\argmin}{arg\,min}
\newcommand\rowincludegraphics[2][]{\raisebox{-0.5\height}{\includegraphics[#1]{#2}}}
\setlength{\belowcaptionskip}{-8pt}

%\newcommand{\MU}{$\bigcirc$\kern-0.75em{$\circ$\kern-0.38em$\cdot$\kern0.38em$\mathbf{U}$}} 
%\newcommand{\SU}{$\circ$\kern-0.38em$\cdot\mathbf{U}$  }
%\newcommand{\LU}{$\bullet\mathbf{U}$  }

%\newcommand{\MF}{$\bigcirc$\kern-0.75em{$\circ$\kern-0.38em$\cdot$\kern0.38em$\mathbf{F}$}}
%\newcommand{\SF}{$\circ$\kern-0.38em$\cdot\mathbf{F}$ }
%\newcommand{\LF}{$\bullet\mathbf{F}$}

%\newcommand{\MU}{\textcolor{black}{$\dddot{\textbf{U}}$}}

%\newcommand{\SU}{\textcolor{black}{$\ddot{\textbf{U}}$}}

%\newcommand{\LU}{\textcolor{black}{$\dot{\textbf{U}}$}}

%\newcommand{\MF}{\textcolor{black}{$\dddot{\textbf{F}}$}}
%\newcommand{\SF}{\textcolor{black}{$\ddot{\textbf{F}}$}}
%\newcommand{\LF}{\textcolor{black}{$\dot{\textbf{F}}$}}

\newcommand{\MU}{$\textbf{UUU}${}}

\newcommand{\SU}{$\textbf{UU}$}{}

\newcommand{\LU}{$\textbf{U}$}{}

\newcommand{\MF}{$\textbf{FFF}$}{}
\newcommand{\SF}{$\textbf{FF}$}{}
\newcommand{\LF}{$\textbf{F}$}{}



\iccvfinalcopy % *** Uncomment this line for the final submission

\def\iccvPaperID{3957} % *** Enter the ICCV Paper ID here
\def\httilde{\mbox{\tt\raisebox{-.5ex}{\symbol{126}}}}

% Pages are numbered in submission mode, and unnumbered in camera-ready
\ificcvfinal\pagestyle{empty}\fi

\begin{document}

%%%%%%%%% TITLE
% \title{Method Name: Unifying Concept-Based Explanations for CNNs}
% \title{UFO: Controlling Understandability and Faithfulness Objectives in Concept-based Explanations for CNNs}
\title{UFO: A unified method for controlling Understandability and Faithfulness Objectives in concept-based explanations for CNNs}

\author{Vikram V. Ramaswamy, Sunnie S. Y. Kim, Ruth Fong, Olga Russakovsky\\
Princeton University\\
{\tt\small \{vr23, suhk, ruthfong, olgarus\}@cs.princeton.edu}
}

\maketitle
% Remove page # from the first page of camera-ready.
\ificcvfinal\thispagestyle{empty}\fi

%%%%%%%%% ABSTRACT
\begin{abstract}
    Concept-based explanations for convolutional neural networks (CNNs) aim to explain model behavior and outputs using a pre-defined set of semantic concepts (e.g., the model recognizes scene class ``bedroom'' based on the presence of concepts ``bed'' and ``pillow'').
    However, they often do not faithfully (i.e., accurately) characterize the model's behavior and can be too complex for people to understand.
    Further, little is known about how faithful and understandable different explanation methods are, and how to control these two properties.
    In this work, we propose UFO, a unified method for controlling Understandability and Faithfulness Objectives in concept-based explanations. UFO formalizes understandability and faithfulness as mathematical objectives and unifies most existing concept-based explanations methods for CNNs. Using UFO, we systematically investigate how explanations change as we turn the knobs of faithfulness and understandability.
    Our experiments demonstrate a faithfulness-vs-understandability tradeoff: increasing understandability reduces faithfulness.
    We also provide insights into the ``disagreement problem'' in explainable machine learning, by analyzing when and how concept-based explanations disagree with each other.
\end{abstract}

%%%%%%%%% BODY TEXT
\section{Introduction}
\label{sec:intro}
\section{Introduction}
\label{sec:introduction}
% \begin{itemize}
%     % Diffusion of FL
%     \item {\st{Diffusion of FL}}
%     % Security threats to FL
%     \item {\st{Security threats to FL with particular focus on model poisoning}}
%     % Limitations of existing countermeasures
%     \item {\st{Current countermeasures (e.g., KRUM) and their limitations}}
%     % Proposed method and its advantages
%     \item {\st{Intuitive description of the proposed method and its difference (i.e., advantages) w.r.t. state of the art}}
%     % Main contributions
%     \item {\st{Summary of the main contributions of this work}}
%     % Paper's structure and organization
%     \item {\st{Paper's structure and organization}}
% \end{itemize}

% Diffusion of FL
Recently, {\em federated learning} (FL) has emerged as the leading paradigm for training distributed, large-scale, and privacy-preserving machine learning (ML) systems~\cite{mcmahan2017googleai,mcmahan2017aistats}. 
The core idea of FL is to allow multiple edge clients to collaboratively train a shared, global model without disclosing their local private training data.
%Specifically, an FL system consists of a central server and many edge clients; 
A typical FL round involves the following steps: {\em(i)} the server randomly picks some clients and sends them the current, global model; {\em(ii)} each selected client locally trains its model with its own private data; then, it sends the resulting local model to the server;\footnote{Whenever we refer to global/local model, we mean global/local model {\em parameters}.} {\em(iii)} the server updates the global model by computing an \emph{aggregation function}, usually the average (FedAvg), on the local models received from clients.
% \begin{enumerate}
%     \item[{\em(i)}] the server sends the current, global model to the clients and appoints some of them for training;
%     \item[{\em(ii)}] each selected client locally trains its copy of the global model with its own private data; then, it sends the resulting local model back to the server;\footnote{Whenever we refer to global/local model, we mean global/local model {\em parameters}.}
%     \item[{\em(iii)}] the server updates the global model by computing an \emph{aggregation function} on the local models received from clients (by default, the average, also referred to as FedAvg~\cite{mcmahan2017aistats}).
% \end{enumerate}
This process goes on until the global model converges. %(e.g., after a certain number of rounds or other similar stopping criteria).
%\\
% The advantages of FL over the traditional, centralized learning paradigm are undoubtedly clear in terms of flexibility/scalability (clients can join/disconnect from the FL network dynamically), network communications (only model weights\footnote{We will use \textit{parameters} and \textit{weights} interchangeably.} are exchanged between clients and server), and privacy (each client's private training data is kept local at the client's end and not uploaded to the server).
\\
% Security threats to FL
%However, the growing adoption of FL also raises security concerns~\cite{costa2022covert}, particularly about its confidentiality, integrity, and availability.
Although its advantages over standard ML, FL also raises security concerns~\cite{costa2022covert}. %, particularly about its confidentiality, integrity, and availability~\cite{costa2022covert}.
% OLD, LONG VERSION
% Indeed, some work deals with privacy leakage that may expose the local data of some clients~\cite{melis2019sp}. 
% A large body of work, instead, investigates attacks that usually aim to detriment the predictive accuracy of the learned global model. For instance, \emph{data poisoning} attacks achieve this goal by letting an adversary pollute the training set of some corrupt FL clients with maliciously crafted examples~\cite{jagielski2018sp}.
% Similarly, in \emph{model poisoning} the attacker attempts to tweak the global model weights~\cite{bhagoji2019pmlr} by directly perturbing the local model's weights of some infected FL clients before these are sent to the central server for aggregation, usually via so-called Byzantine attacks. 
% It turns out that Byzantine model poisoning attacks severely impact standard FedAvg; therefore, more robust aggregation functions must be designed to make FL systems secure.
Here, we focus on \emph{untargeted model poisoning} attacks~\cite{bhagoji2019pmlr}, where an adversary attempts to tweak the global model weights %\footnote{We will use the terms \textit{parameters} and \textit{weights} interchangeably.} 
by directly perturbing the local model's parameters of some infected clients before these are sent to the central server for aggregation.
In doing so, the adversary aims to jeopardize the global model \textit{indiscriminately} at inference time.
Such model poisoning attacks severely impact standard FedAvg; therefore, more robust aggregation functions must be designed to secure FL systems.
\\
% In this paper, we focus on designing a novel robust aggregation scheme at the server's end to contrast the effect of Byzantine model poisoning attacks.
%
% Current countermeasures and their limitations
%Several countermeasures have been proposed in the literature to combat model poisoning attacks on FL systems.
% Some methods use simple statistics more robust than plain average to smooth the impact of malicious updates (e.g., Trimmed Mean and FedMedian~\cite{yin2018icml}). 
% Other defenses implement outlier detection techniques to discard malicious updates from the aggregation performed at the server's end. Those are either based on heuristics (e.g., Krum/Multi-Krum~\cite{blanchard2017nips} and Bulyan~\cite{mhamdi2018pmlr}) or data-driven approaches (e.g., K-means clustering~\cite{shen2016acm} or DnC via spectral analysis~\cite{shejwalkar2021ndss}). 
% Finally, some strategies rely on a centralized ``source of trust'' to spot potential malicious updates (e.g., FLTrust~\cite{cao2020fltrust}).
% Several countermeasures have been proposed in the literature to combat model poisoning attacks on FL systems, i.e., to discard possible malicious local updates from the aggregation performed at the server's end. 
% These techniques range from simple statistics more robust than plain average (e.g., Trimmed Mean and FedMedian~\cite{yin2018icml}) to outlier detection heuristics (e.g., Krum/Multi-Krum~\cite{blanchard2017nips} and Bulyan~\cite{mhamdi2018pmlr}) or data-driven approaches (e.g., spectral analysis via K-means clustering~\cite{shen2016acm} or spectral analysis), or methods based on ``source of trust'' (e.g., FLTrust~\cite{cao2020fltrust}).
% OLD, LONG VERSION
%Several countermeasures have been proposed in the literature to combat Byzantine model poisoning attacks on FL systems.
% Descriptive statistics
% For example, Trimmed Mean and FedMedian aggregate local model updates using more robust statistics than standard average~\cite{yin2018icml}.
%
% % Heuristics for outlier detection
% Many existing Byzantine-resilient strategies implement some outlier detection heuristics to discard the model updates sent by potentially malicious clients from the input of the aggregation function.
% One of the most popular heuristics is Krum~\cite{blanchard2017nips}.
% This strategy tries to mitigate the impact of Byzantine attacks by selecting as a global model the local model with the smallest sum of Euclidean distances to {\em all} the other local models.
% Although powerful, Krum requires the server to know (or, at least, estimate) the number of malicious FL clients upfront, which is generally impossible in a realistic attack scenario. %
% Moreover, Krum may become ineffective for complex, high-dimensional model parameter spaces due to the curse of dimensionality.
% Bulyan~\cite{mhamdi2018pmlr} tries to overcome this issue by combining Krum with a variant of Trimmed Mean.
% % Data-driven outlier detection
% Other strategies use data-driven outlier detection techniques -- e.g., via K-means clustering~\cite{shen2016acm} -- to spot potential malicious local model updates. 
% %For instance, Shen et al. propose to cluster local model updates with K-means and thus identify outliers.
%
% % Other techniques
% As far as the server is concerned, any local model received can be from a potential malicious client. 
% FLTrust~\cite{cao2020fltrust} assumes the server acts as a client, i.e., trains a local model on an additional {\em trustworthy} dataset at the server's end and compares it against all the local models from other clients. 
% This way, the server can rely on some ``source of trust'' when discarding potentially malicious clients.
%\\
% Limitations of existing Byzantine-resilient strategies
Unfortunately, existing defense mechanisms either rely on simple heuristics (e.g., Trimmed Mean and FedMedian by~\cite{yin2018icml}) or need strong and unrealistic assumptions to work effectively (e.g., foreknowledge or estimation of the number of malicious clients in the FL system, as for Krum/Multi-Krum~\cite{blanchard2017nips} and Bulyan~\cite{mhamdi2018pmlr}, which, however, cannot exceed a fixed threshold).
Furthermore, outlier detection methods using K-means clustering~\cite{shen2016acm} or spectral analysis like DnC~\cite{shejwalkar2021ndss} do not directly consider the temporal evolution of local model updates received.
Finally, strategies like FLTrust~\cite{cao2020fltrust} require the server to collect its own dataset and act as a proper client, thereby altering the standard FL protocol.
\\
% OLD, LONG VERSION
% Overall, existing Byzantine-resilient strategies are either simple heuristics (e.g., FedMedian) or, if they are more complex, they rely on strong and unrealistic assumptions to work effectively (e.g., knowing the number of malicious clients in the FL system in advance, as for Krum and alike).
% Furthermore, data-driven outlier detection methods do not consider the temporary evolution of local model updates received (e.g., K-means clustering). 
% Finally, strategies like FLTrust requires the server to collect its own dataset and act as a proper client, thereby altering the standard FL protocol.
%
% Description of the proposed method
This work introduces a novel pre-aggregation \textit{filter} robust to untargeted model poisoning attacks. Notably, this filter $(i)$ operates without requiring prior knowledge or constraints on the number of malicious clients and $(ii)$ inherently integrates temporal dependencies. 
The FL server can employ this filter as a preprocessing step before applying \textit{any} aggregation function, be it standard like FedAvg or robust like Krum or Bulyan.
Specifically, we formulate the problem of identifying corrupted updates as a multidimensional (i.e., matrix-valued) time series anomaly detection task. 
The key idea is that legitimate local updates, resulting from well-calibrated iterative procedures like stochastic gradient descent (SGD) with an appropriate learning rate, show \textit{higher predictability} compared to malicious updates. This hypothesis stems from the fact that the sequence of gradients (thus, model parameters) observed during legitimate training exhibit regular patterns, as validated in Section~\ref{subsec:intuition}. %until convergence. 
%This regularity may be more pronounced for smooth convex loss functions, but it can still be captured within an appropriate time window, even for more complex and convoluted loss surfaces. 
%We provide evidence of this claim in Appendix~B, where we show that the average mutual information (i.e., ``predictability''), calculated over pairs of legitimate model updates sent at different FL rounds, is significantly higher than the corresponding computation for a malicious client.
\\
Inspired by the matrix autoregressive (MAR) framework for multidimensional time series forecasting~\cite{chen2021je}, we propose the FLANDERS ({\em \textbf{F}ederated \textbf{L}earning meets \textbf{AN}omaly \textbf{DE}tection for a \textbf{R}obust and \textbf{S}ecure}) filter.
The main advantages of FLANDERS over existing strategies like FLDetector~\cite{zhao2020multivariate} are its resilience to large-scale attacks, where $50\%$ or more FL participants are hostile, and the capability of working under realistic non-iid scenarios.
We attribute such a capability to two key factors: $(i)$ FLANDERS works without knowing a priori the ratio of corrupted clients, and $(ii)$ it embodies temporal dependencies between intra- and inter-client updates, quickly recognizing local model drifts caused by evil players. Below, we summarize our main contributions:

\begin{itemize}
\item[{\em(i)}]
We provide empirical evidence that the sequence of models sent by legitimate clients is more predictable than those of malicious participants performing untargeted model poisoning attacks.
\\
\item[{\em(ii)}] 
We introduce FLANDERS, the first pre-aggregation filter for FL robust to untargeted model poisoning based on multidimensional time series anomaly detection.
\\
\item[{\em(iii)}] 
We integrate FLANDERS into Flower,\footnote{\scriptsize{\url{https://flower.dev/}}} a popular FL simulation framework for reproducibility.
\\
\item[{\em(iv)}] 
We show that FLANDERS improves the robustness of the existing aggregation methods under multiple settings: different datasets, client's data distribution (non-iid), models, and attack scenarios.
\\
\item[{\em(v)}] 
We publicly release all the implementation code of FLANDERS along with our experiments.\footnote{\scriptsize{\url{https://anonymous.4open.science/r/flanders_exp-7EEB}}}
\end{itemize}

% Paper's structure and organization
The remainder of the paper is structured as follows. %some related work and the current state-of-the-art solutions to security issues that FL entails. 
Section~\ref{sec:background} covers background and preliminaries. 
In Section~\ref{sec:related}, we discuss related work.
Section~\ref{sec:problem} and Section~\ref{sec:method} describe the problem formulation and the method proposed. % to tackle it. 
Section~\ref{sec:experiments} gathers experimental results. %, and Section~\ref{sec:limitations} discusses some limitations of this work.
Finally, we conclude in Section~\ref{sec:conclusion}.
 %discusses the limitations of this work and draws future research directions.
%reports conclusions and draws perspectives for future research directions.

%%%%%%% OLD %%%%%%%
%to overcome the resilience of Byzantine failures in distributed Stochastic Gradient Descent computations. 
% The strength of Krum is its time complexity, which is linear in the gradient dimension. 
% However, the robustness of the approach is guaranteed for gradient-based learning applications only when the majority of the clients are not compromised. 
% Besides, the aggregation mechanism of Krum, as well as that of similar methods, is robust from a coarse-grained perspective and does not provide solutions to errors and perturbations that may occur at inference time.
%A related approach to~\cite{blanchard2017nips} is the work of Su et al.~\cite{su2016dc}. Here, the authors propose an iterated approximate agreement to tackle a multi-layer scenario attacked by Byzantine agents. 
%However, the method works efficiently on the sole discrete context and it is inapplicable to continuous state environments.
%\gabri{Maybe, we should just talk about the main limitations of existing countermeasures without digging into their details (or, we can just mention Krum as this is the most popular one). I will move the description of all these methods to the Related Work section.}

\section{Related work}
\label{sec:related}
\section{Related work}
\noindent \textbf{Video foundation models.}
With sufficient computational power and an abundant source of data, there have been attempts to build a single large-scale foundation model that can be adapted to diverse downstream tasks.
Along with the success of foundations models in the natural language processing domain~\cite{brown2020language,chen2021evaluating,devlin2019bert} and in computer vision~\cite{bertasius2021space,jia2021scaling,radford2021learning}, video data has become another data type of interest, as it has grown in scale due to numerous internet video-sharing platforms.
Accordingly, several methods to train a video foundation model have been proposed.
Due to the innate multi-modality of video data, \textit{i.e.}, a combination of visual $\cdot$ vocal $\cdot$ textual context, most works have centered around the variations of the cross-modal attention mechanism \cite{akbari2021vatt,bertasius2021space,gabeur2020multi,luo2020univl,neimark2021video,tan2021look,wei2020multi,yang2021taco}.
In addition, as most video data lack proper labels or descriptions, contrastive learning methods were studied to learn meaningful feature representations or enhance video-text alignment in a self-supervised manner \cite{akbari2021vatt,kuang2021video,luo2020univl,yang2021taco}.

More specifically, MERLOT \cite{zellers2021merlot} proposed a multi-modal representation learning method for visual commonsense reasoning, which also performed well in twelve video reasoning tasks.
VATT \cite{akbari2021vatt} introduced a multi-modal learning method via contrastive learning. 
The pre-trained model performed well in a variety of vision tasks from image classification to video action recognition and zero-shot video retrieval.
Another representative work, UniVL \cite{luo2020univl} proposed a straightforward pre-training method with auxiliary loss functions. 
After fine-tuning on a specific task, the pre-trained model performed outstandingly in a wide range of tasks of text-to-video retrieval, action segmentation, action step localization, video sentiment analysis, and video captioning.
Other foundation models for multiple video tasks include \cite{li2020hero,sun2019learning,sun2019videobert,zhu2020actbert,fu2021violet,wang2022all}. 

\noindent \textbf{Auxiliary learning.}
In order to enhance the performance of one or a multitude of primary tasks, auxiliary learning methods can be incorporated.
\cite{ruder2017overview} introduced Multi-task learning (MTL) to the deep neural networks by training a single model with multiple task losses to assist learning on the main task.
Such a method is generally adapted to pre-train the foundation models in the self-supervised manner~\cite{li2020hero,sun2019learning,sun2019videobert,zhu2020actbert,fu2021violet,wang2022all}.
However, these various pretext task losses used in the pre-training phase are ignored in the fine-tuning phase, and only the primary task loss is minimized.

Recently, meta-learning methods have been introduced for auxiliary learning.
\cite{liu2019self,navon2020auxiliary,shu2019meta} proposed a meta-learning method in which the model learns auxiliary tasks to generalize well to unseen data. 
In these settings, a separate subset of data is held out as the primary task, while the others are used as auxiliary tasks that aid the primary task's performance.
Similar methods were adopted for computer vision tasks such as semantic segmentation \cite{xu2021leveraging}.
Other domain applications include navigation tasks with reinforcement learning \cite{ye2021auxiliary}, or self-supervised learning methods on graph data \cite{hwang2020self}.

\section{UFO: A unified method for generating concept-based explanations for CNNs}
\label{sec:method}
\section{PoseRAC Model}
\label{sec4}

\begin{figure*}[t]
\centering
\includegraphics[width=1.0\textwidth]{figure5.pdf}
\caption{Overview of our proposed PoseRAC. For a input video, the repetitive count can be obtained through Pose Estimation, Transformer Encoder, Pose Mapping and Action-trigger, where only the Encoder and the Pose Mapping need to be trained. We use Triplet Margin Loss to train the Encoder while Binary Cross Entropy Loss to train both the Encoder and the Pose Mapping. In addition to achieving the state-of-the-art performance so far, the biggest highlight of our PoseRAC is that it is lightweight enough to be easily trained on a CPU.}
\label{fig5}
\end{figure*}

Given a video $V={\{x_i\}}^{T}_{1}\in \mathbb{R}^{C\times H\times W\times T}$ with $T$ RGB frames, repetitive action counting model aims to predict a certain value $Y$, which is the number of repetitive actions. In this section, we will introduce our PoseRAC in detail.

\subsection{Model Overview}

As shown in Figure \ref{fig5}, PoseRAC consists of four parts. 

\begin{itemize}

\item The first is a state-of-the-art and lightweight Pose Estimation Network~($\S\ref{first}$), which is used to estimate the poses represented by lots of human pose key points from each frame of the original video sequence. 

\item The second is a simple Transformer Encoder~($\S\ref{second}$) to embed the key points of poses into high-level feature space, where the same class have similar distances, while the distances of different classes are far apart.

\item The third is a Pose Mapping Module~($\S\ref{third}$), where the unique mapping relationship between the salient poses and the action classes can be learned. Each pose can be mapped to the action class with the highest probability after the previous encoding.

\item The fourth part is a lightweight Action-trigger Module~($\S\ref{fourth}$). When we get the salient action classification results of all frames of the entire video sequence, we can use this module to calculate the repetition count in a short time.

\end{itemize}

\subsection{Pose Estimation Network}
\label{first}
Our model first converts the video sequence into a sequence of human pose key points, which can be defined as: 
\begin{equation}
\begin{split}
&V={\{x_i\}}^{T}_{1}\in \mathbb{R}^{C\times H\times W\times T}\\
&V\xrightarrow{\mathrm{Pose Estimation}} P={\{p_i\}}^{T}_{1}\in \mathbb{R}^{D\times K\times T}
\end{split}
\label{eq1}
\end{equation}
where each $x_i$ represents a single RGB frame, and each $p_i$ represents the key points of each frame. To express the key points of each frame, we use $D\times K$ sequence, which includes two parts, one ($K$) is the number of key points to fully represent the current pose, the other ($D$) is the dimension of each key point, generally three, which are the two coordinates of the planes and the depth estimation.

Here we use state-of-the-art pose estimation models such as Vitpose\cite{xu2022vitpose} and BlazePose\cite{bazarevsky2020blazepose}. The pose estimation algorithms themselves are not designed by us, but we introduce pose information into the action counting task, which is a novel design not explored by previous work.

Moreover, our pose-level poses estimation processes the primitive information of video, which is similar to the feature extraction network in all video-level algorithms such as I3D\cite{carreira2017quo}, VideoSwinTransformer\cite{liu2022video}, and TSN\cite{wang2016temporal}. But the difference is that the result of video-level incorporates all information, while pose-level only produces core information, which greatly improves the performance. Additionally, using pose information can contribute to the lightweight of model. For instance, for a 1024-frame video, video-level feature extraction with an output dimension of 512 would produce a data volume of $1024\times 512=524288$, while using pose information with 33 key points produces a data volume of only $1024\times 33 \times 3=101376$.

\subsection{Encoding Poses with Transformer}
\label{second}
Here we specify our data representation for the Transformer Encoder, which requires input batch size, sequence length, and embedding dimensions. In our pose-level approach, each frame is a batch, the number of key points in each frame is the sequence length, and the feature dimension of each key point is the embedding dimension.

First we get the pose of each frame ${p_i}\in \mathbb{R}^{D\times K}$ through the Pose Estimation Network, where $i\in {1, 2, \dots, T}$ is the frame index, $K$ is the number of key points, and $D$ is the dimension of each key point. We further define $p_i = {\{k_j\}}^{K}_{1}$ to represent each key point, where $k_j\in \mathbb{R}^D$, and we embed it to obtain richer information. Our embedding projection $\mathrm{\bf{E}}$ is a simple MLP network with ReLU as the activation function. These calculations can be defined as:
\begin{equation}
\begin{split}
\mathrm{\bf{Z}}^0 = [\mathrm{\bf{E}}(k_1), \mathrm{\bf{E}}(k_2), \dots, \mathrm{\bf{E}}(k_K)]^T
\end{split}
\end{equation}
where $\mathrm{\bf{E}}(k_j)\in \mathbb{R}^{D^{\prime}}$ is the embedding feature. Then the next Transformer takes $\mathrm{\bf{Z}}^0$ as input and encodes it with self-attention. Given $\mathrm{\bf{Z}}^0\in \mathbb{R}^{K\times D^{\prime}}$ with $K$ key point features, each of which is $D^{\prime}$-dimensional, $\mathrm{\bf{Z}}^0$ is projected using $\mathrm{\bf{W}}_Q\in \mathbb{R}^{D^{\prime}\times D_q}$, $\mathrm{\bf{W}}_K\in \mathbb{R}^{D^{\prime}\times D_k}$, $\mathrm{\bf{W}}_V\in \mathbb{R}^{D^{\prime}\times D_v}$, where $D_k=D_q$, to extract feature representations query($\mathrm{\bf{Q}}$), key($\mathrm{\bf{K}}$) and value($\mathrm{\bf{V}}$), which can be defined as:
\begin{equation}
\begin{split}
&\mathrm{\bf{Q}}=\mathrm{\bf{Z}}^0\times \mathrm{\bf{W}}_Q\\
&\mathrm{\bf{K}}=\mathrm{\bf{Z}}^0\times \mathrm{\bf{W}}_K\\
&\mathrm{\bf{V}}=\mathrm{\bf{Z}}^0\times \mathrm{\bf{W}}_V
\end{split}
\end{equation}
and the output of self-attention can be computed as:
\begin{equation}
\begin{split}
\mathrm{\bf{Attn}}=\mathrm{Softmax}(\frac{\mathrm{\bf{Q}}\mathrm{\bf{K}}^T}{\sqrt{D_q}})\mathrm{\bf{V}}
\end{split}
\end{equation}
where $\mathrm{\bf{Attn}}\in \mathbb{R}^{K\times D^{\prime}}$. Also, we use common multi-head self-attention (MHSA) to make several self-attention operations calculate in parallel.

Now we introduce the overall architecture of Transformer Encoder, which has $L$ layers with each layer consisting of MHSA and MLP blocks. Also, LayerNorm and Residual Connection are applied before and after every MHSA or MLP block, respectively. Because the number of key points of each frame is  a bit less, so our encoder does not include the downsampling module that other models may have. The overall process can be defined as:
\begin{equation}
\begin{split}
&\mathrm{\bf{\hat{Z}}}^l = \mathrm{MHSA}(\mathrm{LN}(\mathrm{\bf{Z}}^{l-1})) + \mathrm{\bf{Z}}^{l-1}\\
&\mathrm{\bf{Z}}^l = \mathrm{MLP}(\mathrm{LN}(\mathrm{\bf{\hat{Z}}}^l)) + \mathrm{\bf{\hat{Z}}}^l
\end{split}
\end{equation}
where $\mathrm{\bf{Z}}^{l-1}$, $\mathrm{\bf{\hat{Z}}}^l$, $\mathrm{\bf{Z}}^l\in \mathbb{R}^{K\times D^{\prime}}$.


\subsection{Pose Mapping}
\label{third}
Taking the Encoder output $\mathrm{\bf{Z}}^L\in \mathbb{R}^{K\times D^{\prime}}$ as input, Pose Mapping module outputs probability scores $\mathrm{\bf{S}}\in \mathbb{R}^{C}$ of the current frame over all action classes. We perform binary classification after Sigmoid activation for each class, with the two salient poses of each class represented by the same bit data. To realize such a module, we use a very lightweight MLP network, which avoids the complexity. First, the two dimensions $K$ and $D^{\prime}$ of $\mathrm{\bf{Z}}^L$ are flattened into $\mathbb{R}^{KD^{\prime}}$, and then it passes through an MLP module, where the output channels is set to $C$, which can be defined as:
\begin{equation}
\begin{split}
\mathrm{\bf{S}} = \sigma(\mathrm{MLP}(\mathrm{Flatten}(\mathrm{\bf{Z}}^L)))
\end{split}
\end{equation}
where $\sigma$ represents the Sigmoid activation function.

With such Pose Mapping, we can obtain the scores of single frame. It should be noted that we extract the poses of all frames, and use the convenience of matrix operations to obtain scores in parallel, which is actually consistent with the idea of mini batch. So at last, we combine the scores of all frames to get the video score matrix $\mathrm{\bf{\hat{S}}}\in \mathbb{R}^{C\times T}$, where $T$ represents the number of frames in the current video. 


\subsection{Action-trigger Module}
\label{fourth}
We use the lightweight Action-trigger Module to obtain the final output $Y$, the repetitive action count, which has a time complexity of $\mathcal{O}(n)$. First, we get the scores $S_c\in \mathbb{R}^T$ of a given action class from $\mathrm{\bf{\hat{S}}}$. Then, we scan all frames and use the action-trigger mechanism to count when the two salient poses of the action class occur sequentially. We set upper and lower bounds to distinguish the scores of the two salient poses, which cluster non-salient poses in the middle and easily classify the salient poses to the two ends.

\subsection{Losses and Metric Learning}

The modules need to be trained are Embedding, Transformer Encoder and Pose Mapping, and because we perform binary classification for each class, so we use the Binary Cross Entropy Loss, which can be defined as follows:
\begin{gather}
\mathcal{L}_{bce} = -\frac{1}{N}\sum\limits_{i=1}^{N}(\frac{1}{C}\sum\limits_{j=1}^{C}loss(i,j))  \\
 loss(i,j)=y_{ij}\log p_{ij} + (1-y_{ij})\log(1-p_{ij})
\end{gather}
where $N$ represents the batch size (in our method, each frame is a batch), $C$ represents the number of classes, $y$ and $p$ are the labels and our predictions, respectively.

Moreover, we use Metric Learning to improve our Encoder and introduce the Pose Triplet Loss. Given a pose, Encoder produces higher-level features $\mathrm{\bf{Z}}^L$, which should be more representative. As shown in Figure \ref{fig5}, we achieve this with Triplet Margin Loss function, which selects anchors, same class positive samples, and different classes negative samples in a batch. It can be expressed as:
\begin{equation}
\begin{split}
\mathcal{L}_{tri} = \mathrm{max}(\mathrm{CS}(a,p)-\mathrm{CS}(a,n)+\mathrm{margin},0)
\end{split}
\end{equation}
where $a$, $p$, $d$ are anchors, positive and negative samples, and $\mathrm{CS}$ represents the Cosine Similarity to measure the distance between features. We pay more attention to hard samples, where the distances between anchors and negative samples are even smaller than those of positive samples. After Metric Learning, the poses of each action can be distinguishable, which cluster in the high-level space.

At last, our overall training combines these two losses:
\begin{equation}
\begin{split}
\mathcal{L} = \mathcal{L}_{bce} + \alpha\mathcal{L}_{tri}
\end{split}
\end{equation}
where $\alpha$ is the weight factor to control the two losses in the same numeric scale.
\subsection{Implementation Details}

\noindent{\bf Training.} We use the \emph{RepCount-pose} and \emph{UCFRep-pose} dataset we created to train our model. Only the frames with salient poses are inputted into the network instead of the entire video to speed up the fitting.

\noindent{\bf Inference.} During inference, the entire video sequence is inputted into the model. The poses of all frames pass through the Encoder and Pose Mapping, and then enter the Action-trigger Module to output the repetitive count.


\section{Experiments}
\label{sec:experiments}
We present in section~\ref{ssec:faces} an application of PnP-HVAE on face images, using a pretrained state-of-the-art hierarchical VAE. 
Next, we study the application of our framework to natural images. To that end, we introduce  in section~\ref{ssec:patchVDVAE}  a patch hierachical VAE architecture, that is able to model natural images of different resolutions. In section~\ref{ssec:app_nat}, we provide deblurring, super-resolution and inpainting experiments to demonstrate the relevance of the proposed method.

Additional results are presented in Appendix~\ref{app:add}. All experiments can be reproduced using the code available at \url{https://github.com/jprost76/PnP-HVAE}.



\subsection{Face Image restoration (FFHQ)}\label{ssec:faces}
We first demonstrate the effectiveness of PnP-HVAE on highly structured data, by performing face image restoration.
Latent variable generative models can accurately model structured images such as face images \cite{karras2019style,vahdat2020nvae,child2021very,kingma2018glow}, and then be used to produce high quality restoration of such data. 
In our experiments, we use the VDVAE model of~\cite{child2021very}, pre-trained on the FFHQ dataset~\cite{karras2019style}, as our hierarchical VAE prior.
VDVAE has $L=66$ latent variable groups in its hierarchy and generates images at resolution $256\times256$.

We compare PnP-HVAE with the intermediate layer optimization algorithm (ILO)~\cite{daras2021intermediate} that is based on a different class of generative models than HVAE. ILO is a GAN inversion method which optimizes the image latent code along with the intermediate layer representation of a StyleGAN to generate an image consistent with a degraded observation.
We use the official implementation of ILO, along with a StyleGAN2 model~\cite{karras2020analyzing, stylegan2pytorch}, that was trained for 550k iterations on images of resolution $256\times256$ from FFHQ.  
As VDVAE and StyleGAN models are not trained on the same train-test split of FFHQ, we chose to evaluate the methods on a subset of 100 images from the CelebA dataset~\cite{liu2018large}. 
For super-resolution, the degradation model corresponds to the application of a gaussian low-pass filter followed by a $\times 4$ sub-sampling, and the addition of a gaussian white noise with $\sigma=3$.
For the deblurring, we considered motion blur and  gaussian kernels, both with a noise level $\sigma=8$. %

We provide quantitative comparisons in table~\ref{table:comp_ILO}, along with a visual comparison of the results in figure~\ref{fig:face_restoration}.
PnP-HVAE has the best  PSNR and SSIM results for all the considered restoration tasks, while ILO provides better results  for the perceptual distance.
By jointly optimizing the image and its latent variable, PnP-HVAE provides  results that are both realistic and consistent with the degraded observation.
On the other hand,  ILO  only optimizes on an extended latent space. This method generates  sharp and realistic images with better LPIPS scores,   
but the results lack  of consistency with respect to the observation, which explains the overall lower PSNR performance. 






\subsection{PatchVDVAE: a HVAE for natural images}\label{ssec:patchVDVAE}
Available generative models in the literature operate on images of  fixed resolutions and
are either restrained to datasets of limited diversity, or even to registered face images~\cite{kingma2018glow,child2021very, vahdat2020nvae, karras2019style}, or requiring additional class information~\cite{brock2018large, dhariwal2021diffusion, song2020score, luhman2022optimizing}.
Fitting an unconditional model on natural images appears to be a more difficult task, as their resolution can change, and their content is highly diverse.
The complexity of the problem can be reduced by learning a prior model on patches of reduced dimension. 
For image restoration problems, the patch model can be reused on images of higher dimensions~\cite{zoran2011learning,prost2021learning,altekruger2022patchnr}. When the model is a full CNN, the prior on the set of the  patches can  be computed efficiently by applying the network on the full image~\cite{prost2021learning}.

We thus introduce  patchVDVAE, a fully convolutional hierarchical VAE.
Contrary to existing HVAE models whose resolution is constrained by the constant tensor at the input of the top-down block, patchVDVAE can generate images of different resolutions by controlling the dimension of the input latent. 
This amounts to defining a prior on patches whose dimension corresponds to the receptive field of the VAE. A similar model is used for image denoising in~\cite{prakash2021interpretable}.

 
For PatchVDVAE architecture, we use the same bottom-up and top-down blocks as VDVAE~\cite{child2021very}, and replace the constant trainable input in the first top-down block by a latent variable, to make the model fully convolutional (details on the  architecture are given in Appendix~\ref{app:details}). 
The training dataset is composed of $128\times 128$ patches extracted from a combination of DIV2K~\cite{agustsson2017ntire} and Flickr2K~\cite{Lim_2017_CVPR_workshops} datasets.
We perform data augmentation by extracting  patches at $3$ resolutions: HR-images and $\times 2$ and $\times 4$ downscaled images. 
The model is trained for $7.10^5$ iterations with a batch size of $64$. Following the recommendation of~\cite{hazami2022efficient}, we use Adamax optimizer with an exponential moving average and gradient smoothing of the variance.
We set the decoder model to be a gaussian with diagonal covariance, as in~\cite{luhman2022optimizing}.
PatchVDVAE is fully convolutional and can generate images of dimension that are multiples of $64$ as illustrated by
figure~\ref{fig:vdvae}.

\newlength{\patchwidth}
\setlength{\patchwidth}{0.135\columnwidth}
\begin{figure}[!ht]
    \centering
    \begin{subfigure}[t]{.34\columnwidth}\hspace{0.1cm}
        \setlength{\tabcolsep}{0.02pt}
\renewcommand{\arraystretch}{0}
        \begin{tabular}{*{2}{p{1.03\patchwidth}}}
            \includegraphics[width=\patchwidth]{figures_arxiv/patchVDVAE/samples/generated/64x64/setup-5-image-0018.png} &
            \includegraphics[width=\patchwidth]{figures_arxiv/patchVDVAE/samples/generated/64x64/setup-5-image-0016.png} \\
            \includegraphics[width=\patchwidth]{figures_arxiv/patchVDVAE/samples/generated/64x64/setup-5-image-0008.png} &
            \includegraphics[width=\patchwidth]{figures_arxiv/patchVDVAE/samples/generated/64x64/setup-5-image-0019.png}   
        \end{tabular}
    \end{subfigure}\hspace{-0.15cm}
    \begin{subfigure}[t]{.64\columnwidth}
\begin{tabular}{cc}\vspace{-0.1cm}
\includegraphics[width=2\patchwidth]{figures_arxiv/patchVDVAE/samples/generated/256x256/setup-2-image-0009.png}&
        \includegraphics[width=2\patchwidth]{figures_arxiv/patchVDVAE/samples/generated/256x256/setup-2-image-0002.png}\end{tabular}

    \end{subfigure}
    \caption{\label{fig:vdvae} Left: $64\times64$ patches samples from our patchVDVAE model trained on patches from natural images.
    Right: PatchVDVAE is fully convolutional and it can generate images of higher resolution (here: $128\times128$).\vspace{-0.2cm}}
\end{figure}

\subsection{Natural images restoration}\label{ssec:app_nat}
We  evaluate PnP-HVAE on natural image restoration.
For each task, we report the average value of the PSNR, the SSIM, and the LPIPS metrics on $20$ images from the test set of the BSD dataset~\cite{MartinFTM01}.\\


\noindent
{\bf Image deblurring.}
In the experiments, we consider $2$ gaussian kernels and $2$ motion blur kernels from~\cite{levin2009understanding}, with $3$ different noise levels 
$\sigma \in \{2.55, 7.65, 12.75\}$.
As a baseline we consider  EPLL~\cite{zoran2011learning}, which learns a prior on image patches with a gaussian mixture model.
We also compare PnP-HVAE  with PnP-MMO and GS-PnP, $2$ competing convergent Plug-and-Play methods based on CNN denoisers.
PnP-MMO~\cite{pesquet2021learning} restricts the denoiser to be contraction in order to guarantee the convergence of the PnP forward-backard algorithm. GS-PnP~\cite{hurault2022gradient} considers a gradient step denoiser and reaches state-of-the-art performances of non converging methods~\cite{zhang2021plug}.
We set the temperature $\tau$  in our method as $0.95$, $0.8$ and $0.6$ for noise levels $2.55$, $7.65$ and $12.75$ respectively, and we let it run for a maximum of $50$ iterations. 
For the three compared methods we use the official implementations and pre-trained models provided by the respective authors. 
Details on the choice of hyperparameters for the concurrent methods are provided in the Appendix~\ref{app:details}
Figure~\ref{fig:deblurring_bsd} illustrates that our method provides correct deblurring results. 

According to table~\ref{tab:deb}, the performance of PnP-HVAE is between those of EPLL and GS-PnP and it outperforms PnP-MMO for large noise levels.\\

\begin{table}
\begin{center}\footnotesize
    \begin{tabular}{>{\centering}m{.3cm}*{5}{c}}
    $\sigma$ &Method & PSNR$\uparrow$ & SSIM$\uparrow$ & LPIPS$\downarrow$  \\ 
    \hline
    \multirow{4}{*}{\vcell{$2.55$}}
    & PnP-HVAE & $27.75$ & $0.79$ & $0.31$\\
    & GS-PNP \cite{hurault2022gradient} & $\mathbf{29.59}$ & $\mathbf{0.84}$ & $\mathbf{0.22}$\\
    & EPLL \cite{zoran2011learning} & $26.49$ & $0.71$ & $0.36$\\ 
    & PnP-MMO \cite{pesquet2021learning} & $\underbar{29.50}$ & $\underbar{0.83}$ & $\underbar{0.20}$ \\ \hline
    \multirow{4}{*}{\vcell{$7.65$}}
    & PnP-HVAE & $\underbar{26.36}$ & $\underbar{0.72}$ & $\underbar{0.40}$\\
    & GS-PNP \cite{hurault2022gradient} & $\mathbf{27.33}$ & $\mathbf{0.77}$ & $\mathbf{0.31}$\\
    & EPLL \cite{zoran2011learning} & $24.04$ & $0.66$ & $0.45$ \\ 
    & PnP-MMO \cite{pesquet2021learning} & $25.34$ & $0.69$ & $0.34$\\
    \hline
    \multirow{4}{*}{\vcell{$12.75$}}
    & PnP-HVAE & $\underbar{25.12}$ & $\mathbf{0.73}$ & $\underbar{0.47}$\\
    & GS-PNP \cite{hurault2022gradient} & $\mathbf{26.32}$ & $\mathbf{0.73}$ & $\mathbf{0.37}$\\
    & EPLL \cite{zoran2011learning} & $23.28$ & $0.61$ & $0.51$ \\ 
    & PnP-MMO \cite{pesquet2021learning} & $22.42$ & $0.53$& $0.54$ \\
    \hline
    &\vspace*{-.3cm}\\
            \multicolumn{2}{c}{Blur and motion kernels}& \multicolumn{3}{c}{
        \includegraphics*[scale=1]{figures_arxiv/kernels/4.png}\;\includegraphics*[scale=1]{figures_arxiv/kernels/7.png}\;\includegraphics*[scale=1]{figures_arxiv/kernels/9.png}\;\includegraphics*[scale=1]{figures_arxiv/kernels/11.png}} 
    \end{tabular}
        \caption{\label{tab:deb}Comparison  of PnP-HVAE  and other restoration methods on deblurring. Results are averaged on $4$ kernels.\vspace{-0.2cm}}% on image deblurring.}
    \end{center}
\end{table}

\begin{figure}
    
    \begin{subfigure}[h]{\linewidth}
        \centering
        \includegraphics*[width=\columnwidth]{figures_arxiv/deb_s255_k7.pdf}\vspace{-0.1cm}
        \caption{Gaussian blur, $\sigma=2.55$}
    \end{subfigure}
    \begin{subfigure}[h]{\linewidth}
        \centering
        \includegraphics*[width=\columnwidth]{figures_arxiv/deb_s765_k11.pdf}\vspace{-0.1cm}
        \caption{Motion blur, $\sigma=7.65$}
    \end{subfigure}\vspace*{-0.1cm}
    \caption{\label{fig:deblurring_bsd} Natural image deblurring\vspace{-0.1cm}}
\end{figure}

\noindent {\bf Effect of the temperature.}
PnP-HVAE gives control on the temperature of the prior over the latent space.
In figure~\ref{fig:temp_effect}, we illustrate that reducing the temperature increases the strength of the regularization prior. In this example the tuning $\tau=0.7$ produces the best performance.\\
\begin{figure}[!ht]
   
    \includegraphics[width=\columnwidth]{figures_arxiv/demo_temp.pdf}\vspace{-0.15cm}
    \caption{ \label{fig:temp_effect} Effect of the temperature in PnP-VAE on a deblurring problem, with $\sigma=7.65$.\vspace{-0.15cm}}
\end{figure}


\noindent
{\bf Image inpainting.}
Next we consider the task of noisy image inpainting. 
We compose a test-set of 10 images from the validation set of BSD~\cite{MartinFTM01} and we create masks
  by occluding diverse objects of small size in the images. 
A gaussian white noise with $\sigma=3$ is added to the images.
As a comparaison, we still consider GS-PnP and EPLL.
For PnP-HVAE, the temperature is set to $\tau=0.6$, and the algorithm is run for a maximum of $200$ iterations, unless the residual $||\x_{k+1}-\x_k||$ is on a plateau.
We provide on Table~\ref{tab:inpainting_bsd} the distortion metrics with the ground truth, as well as a visual
\begin{table}



\begin{center}
    \begin{tabular}{cccc}
        & PSNR$\uparrow$ & SSIM$\uparrow$ &LPIPS$\downarrow$ \\\hline
        PnP-HVAE  & $\mathbf{29.54}$ & $\mathbf{0.93}$ & $\mathbf{0.06}$\\
        GS-PNP & $28.52$ & $\mathbf{0.93}$ & $0.09$\\
        EPLL & $\underline{29.16}$ & $\mathbf{0.93}$ & $\mathbf{0.06}$\\
    \end{tabular}
    \caption{\label{tab:inpainting_bsd}Quantitative evaluation for inpainting on BSD.}
    \end{center}
\end{table}
comparison on figure~\ref{fig:inpainting_bsd}. 
With its hierarchical structure,  PnP-HVAE outperforms the compared methods. \vspace{0.05cm}



\begin{figure}[!h]
    \includegraphics[width=\columnwidth]{figures_arxiv/demo_inp_bsd2.pdf}\vspace{-0.1cm}
    \caption{\label{fig:inpainting_bsd}Natural image inpainting\vspace{-0.3cm}}
\end{figure}













\section{Conclusion}
\label{sec:conclusion}
\section{Conclusion}\label{sec:conclusion}
In this work, we focus on addressing the fundamental challenge of OOD detection tasks, which is how to fully understand the semantic discrepancy between the ID/OOD samples. We reveal that the key to success in the realistic SCOOD task is to allocate as many ID samples in the unlabeled set correctly as possible. To this end, we propose a novel uncertainty-aware optimal transport scheme that introduces class-specific energy scores as guidance for effective label assignment. Experimental results show that our method achieves better performance than previous state-of-the-art methods on SCOOD benchmarks.

\textbf{Limitations.} In addition to temperature scaling, other techniques such as feature clipping applied in ReAct~\cite{sun2021react} also enhance the performance of energy score, so how to obtain an OOD score that best fits the SCOOD task can be further explored. Moreover, a setting highly related to SCOOD has been proposed in \cite{katz2022training} and formulated as a constrained optimization problem. We will also theoretically analyze these practical OOD settings in our feature work.

% \section*{Acknowledgments}
\textbf{Acknowledgments.} 
This work is supported by National Key R\&D Program of China under Grant 2020AAA0105701, National Natural Science Foundation of China (NSFC) under Grants 61872327, Major Special Science and Technology Project of Anhui, National Natural Science Foundation of China (62033012) and Ant Group through Ant Research Intern Program.


{\small
\bibliographystyle{ieee_fullname}
\bibliography{egbib}
}

\appendix

In this appendix, we provide more details about our method, as well as some additional results. 

\section{Additional details about comparisons with prior work}
\label{sec:netdissect}
%\iffalse
Here we provide more details about the different methods, and how they can be thought of within our framework. 

\smallsec{Net2Vec and TCAV}
For Net2Vec~\cite{fong2018net2vec} and TCAV~\cite{kim2018tcav}, the authors align the feature space with concepts, without considering the final output. This can be achieved with $\lambda_1 = 0$.
The authors allow the features $f$ to be any layer within the trained model $M$, and learn $h_\text{conc}$ as a series of individual indicator functions to the concepts for each neuron in $f$ or as a linear combination of the different neurons in $f$, i.e, $h_\text{conc} \colon \mathbb{R}^n \rightarrow \mathbb{R}^C$, where $n$ is the number of neurons in a layer of the CNN and $C$ is the number of concepts. The authors allow the selection function $S$ to select all the $C$ concepts, i.e, $K=C$. Thus, these works optimize $L_\text{align}$:
\begin{align}
    \label{eq:net2vec_tcav}
    L_\text{align}^{\text{Net2Vec}, \text{TCAV}} & = \sum_{j \in \{1, 2, \ldots C\}}\sum_{x\in X} CE(A(x)_j, (h_\text{conc}\circ f(x))_j)
\end{align}
Net2Vec also considers aligning individual neurons with concepts. This can be achieved by forcing $h_\text{conc}$ to be an indicator function: each neuron is aligned with exactly one concept.  

\smallsec{NetDissect}
NetDissect~\cite{bau2017netdissect} uses a slightly different framework compared to ours, however, we show that by rewriting $L_\text{align}$, we can consider NetDissect within our framework. We first rewrite NetDissect using the following notation. 
\begin{itemize}
    \item Suppose $A_{\text{seg}} \colon X \to \mathbb{R}^{C \times H\times W}$ is the segmentation map for $C$ concepts.
    \item $f \colon \mathcal{X} \to \mathbb{R}^{n \times H' \times W'}$ is the feature space. 
    \item $t: \mathbb{R}^{n \times H' \times W'} \rightarrow \{0,1\}^{D \times H \times W}$.
    This is an upsampling and thresholding function: first, the vector is bilinearly upsampled to size $H \times W$ for each neuron, and thresholded such that only the top $0.5\%$ for a neuron is activated.
\end{itemize}
Now, for each neuron $i \in \{1, 2, \ldots n\}$, they compute the concept $j$ that maximizes
\begin{align}
\label{eq:netdissect}
\text{IOU}_i 
   := & \argmax_{j \in \{1, 2, \ldots C\}} \left(\frac{\sum_{x \in X} ((A_{\text{seg}}(x))_j \cap (t \circ f(x))_i)}{\sum_{x \in X} ((A_{\text{seg}}(x))_j \cup (t \circ f(x))_i)}\right)
\end{align}

In order to consider NetDissect within our framework, we can rewrite $L_\text{align}$ as follows. We first consider $f_i$ at a single neuron $i$, i.e. $f_i \colon \mathcal{X} \rightarrow \mathbb{R}^{H' \times W'}$. Then, rather than using $\sum_x CE(A(x), h_\text{conc}\circ f(x))$, we can express $L_\text{align}$ in terms of \cref{eq:netdissect}, with $|S| = 1$:

\begin{align}
    L_{align} = - \sum_{j \in \{1, 2, \ldots C\}} \mathbbm{1}_S \circ \text{IOU}_i
\end{align}
% For the distance function $d$, they use $d = \text{IoU}$
%\fi

\smallsec{IBD}
For IBD~\cite{zhou2018ibd}, $h_\text{conc}$ is a linear combination of the activations of each neuron and $h_\text{pred}$ is a linear combination of the outputs of $h_\text{conc}$, very similar to our \textbf{sf,su} framework. The main difference is in an additional constraint imposed on $h_\text{pred}$: that the coefficients are all non-negative, and each target class is allowed to use exactly $K$ concepts (but these do not need to be the same across target classes). Thus, for IBD, $L_\text{mimic}$ can be written as:

\begin{align}
    \label{eq:ibd}
    \forall i \in \{1, 2, \ldots, D\} & \nonumber\\
 \left(L_\text{mimic}^\text{IBD}\right)_i & = 
   \sum_{x \in X} \|(g \circ f(x))_i -  \nonumber \\ & h_\text{pred}^i  \circ \mathbbm{1}_{S_i} \circ 
   p \circ h_\text{conc}\circ f(x)
   \|
   \\
   \text{such that } \nonumber\\
&   h_\text{pred}^i(x) = W_i^Tf(x) \nonumber \\
 & \textcolor{red}{ W_{i, k}\geq 0} \ \ \ \forall k \in \{1, 2, \ldots n\} \nonumber
\end{align}

As mentioned in the main text Section 5.2, the non-negative constraint changes the concepts chosen: examples of concepts chosen are in table~\ref{tab:ibd}

\begin{table*}[t]
\centering
\begin{tabular}{L{2cm} L{6.5cm}L{6.5cm}}
\toprule
scene & IBD & UFO(Ours) \\
\toprule
attic & heater, basket, stairway, breads, magazine, television camera, drum & backpack, grandstand, coffee maker, pitcher, microwave, sand, door, sculpture\\ \arrayrulecolor{black!30}\midrule
bathroom & screen door, village, water tower, tray, candelabrum, stands, drinking glass & grandstand, backpack, crt screen, bench, microwave, double door, sculpture, work surface\\\midrule
bedchamber & headboard, pillow, shade, vault, eiderdown, water tower, tent & coffee maker, pitcher, spotlight, microwave, cabinet, door, sky, sculpture\\\midrule
bedroom & headboard, pillow, eiderdown, shade, lower sash, shower, shirt & grandstand, coffee maker, doorframe, ladder, sculpture, spotlight, work surface, clock\\\midrule
conference-room & bulletin board, wineglass, trouser, escalator, mouse, button panel, mouse pad & grandstand, platform, counter, pitcher, microwave, ladder, floor, desk\\\midrule
crosswalk & vineyard, traffic light, autobus, chain wheel, trailer, skylight, cockpit & coffee maker, grandstand, land, platform, crt screen, backpack, doorframe, television\\\midrule
dining-room & chandelier, candelabrum, back pillow, skirt, carpet, cart, grand piano & coffee maker, pitcher, platform, chest, windowpane, road, door, table\\\midrule
downtown & skyscraper, paper towel, gas station, candelabrum, place mat, slot machine, crosswalk & grandstand, land, pitcher, platform, crt screen, television, sky, sofa\\\midrule
highway & catwalk, autobus, document, book stand, dashboard, slats, corner pocket & coffee maker, platform, plant, flag, sky, lamp, door, table\\\midrule
hotel-room & bed, tracks, candlestick, cushion, seat cushion, capital, candle & land, microwave, spotlight, cabinet, sculpture, dishwasher, clock, work surface\\\midrule
kitchen & stove, refrigerator, tray, kitchen island, container, screen door, microwave & grandstand, coffee maker, backpack, crt screen, pitcher, doorframe, cap, faucet\\\midrule
living-room & post, cushion, riser, fireplace, monitoring device, sconce, bumper & pitcher, doorframe, counter, spotlight, windowpane, cabinet, door, table\\\midrule
parking-garage/outdoor & paper towel, crane, windows, notebook, steam shovel, gym shoe, television & coffee maker, grandstand, land, backpack, platform, crt screen, doorframe, television\\\midrule
recreation-room & pool table, court, microwave, table football, slot machine, wire, grand piano & grandstand, land, chest, counter, spotlight, floor, windowpane, sky\\\midrule
residential-neighborhood & sill, balloon, trailer, metal shutters, flowerpot, switch, synthesizer & coffee maker, faucet, land, platform, doorframe, sculpture, dishwasher, floor\\\midrule
skyscraper & skyscraper, display board, workbench, manhole, paw, lighthouse, gas station & coffee maker, land, pitcher, television, sky, sofa, lamp, spotlight\\\midrule
street & slats, roundabout, crosswalk, beak, arcades, bus, parking & coffee maker, land, faucet, platform, crt screen, microwave, doorframe, television\\\midrule
television-room & seat base, brick, sash, inside arm, gravel, water wheel, pantry & pitcher, chest, cap, microwave, spotlight, counter, desk, door\\\midrule
waiting-room & armchair, sconce, shoe, console table, back pillow, canvas, dishrag & pitcher, chest, counter, spotlight, sky, doorframe, sofa, table\\\midrule
youth-hostel & sweater, towel, equipment, kettle, wardrobe, vent, partition & grandstand, doorframe, ladder, microwave, spotlight, sculpture, work surface, flag\\ \arrayrulecolor{black}\bottomrule
\end{tabular}
\caption{\textbf{Concepts chosen by IBD~\cite{zhou2018ibd} versus that chosen by our method.} We see that the non-negative constraint added by IBD changes the concepts chosen by quite a lot.}
\label{tab:ibd}
\end{table*}

%\smallsec{Concept Bottleneck}
%Concept Bottleneck models~\cite{koh2020conceptbottleneck} fall neatly within our framework: these models can be written as $F:= g \circ f$, where $f$ predicts concepts from an input $x$ and $g$ combines these concepts to form the model output. Thus, $h_\text{conc}$ can be expressed as an  identity function, with $h_\text{pred}$ being $g$ itself. 


\section{More results}
\label{sec:more_res}
In this section, we give more details about our experiment set up and highlight additional results. 

\smallsec{Experimental setup}
We use a greedy method to select uncorrelated concepts, using Pearson's correlation coefficient.  For each understandability setting, we compute the correlation coefficient between all pairs of concepts (using either the base rates or the learned concept vector). Next, we compute the 90\% percentile correlation coefficient and set that as a threshold. We add each concept to the list of chosen concepts if it is not more correlated than the threshold with any of the previously chosen concepts. 

When choosing the values of $\lambda_1$ and $\lambda_2$, we fix $\lambda_1$ to 1, and pick $\lambda_2$ from $\{1.0, 0.5, 0.1, 0.05, 0.01, 0.005\}$ that best explains the model on the validation set. 

\smallsec{Concepts for coarse grained scenes}
We first report the concepts chosen across the 6 faithfulness-understandability settings described in the main text. Similar to the fine-grained model, we see that the attributes chosen vary based on the setting (Tab.~\ref{tab:grouped_attr_comp}).

\begin{table*}[t]
    \centering
    \resizebox{0.82\linewidth}{!}{
    \begin{tabular}{L{2cm} l   L{4cm}  L{4cm}  L{4cm}}
 \toprule
%\multirow{2}{*}{scene} 
%& & Acc & concepts & Acc & concepts & Acc & concepts \\
%\cmidrule(l{3pt}r{3pt}){3-4} \cmidrule(l{3pt}r{3pt}){5-6} \cmidrule(l{3pt}r{3pt}){7-8}
& & {most understandable (\MU)} & {somewhat understandable (\SU)} & {least understandable (\LU)} \\ 
\midrule 
shopping-dining & \SF & \textit{\textcolor{red}{bed}}, \textcolor{NavyBlue}{bulletin board}, \textit{\textcolor{red}{sky}} & \textcolor{NavyBlue}{\textbf{cap}}, \textit{\textcolor{red}{footbridge}}, \textcolor{NavyBlue}{baby buggy} & \textit{\textcolor{red}{\textbf{faucet}}}, \textit{\textcolor{red}{land}}, \textcolor{NavyBlue}{\textbf{cap}}\\
 & \LF & \textit{\textcolor{red}{grandstand}}, \textit{\textcolor{red}{pillow}}, \textit{\textcolor{red}{bird}} & \textit{\textcolor{red}{table tennis}}, \textcolor{NavyBlue}{\textbf{cap}}, \textit{\textcolor{red}{loudspeaker}} & \textit{\textcolor{red}{\textbf{faucet}}}, \textcolor{NavyBlue}{pitcher}, \textcolor{NavyBlue}{counter}\\
\midrule
workplace & \SF & \textit{\textcolor{red}{\textbf{sky}}}, \textcolor{NavyBlue}{floor}, \textcolor{NavyBlue}{desk} & \textcolor{NavyBlue}{\textbf{dog}}, \textcolor{NavyBlue}{footbridge}, \textit{\textcolor{red}{\textbf{baby buggy}}} & \textcolor{NavyBlue}{platform}, \textit{\textcolor{red}{bread}}, \textit{\textcolor{red}{jar}}\\
 & \LF & \textit{\textcolor{red}{\textbf{sky}}}, \textit{\textcolor{red}{bed}}, \textit{\textcolor{red}{tree}} & \textit{\textcolor{red}{\textbf{baby buggy}}}, \textcolor{NavyBlue}{\textbf{dog}}, \textit{\textcolor{red}{lake}} & \textcolor{NavyBlue}{paper}, \textcolor{NavyBlue}{bulletin board}, \textcolor{NavyBlue}{runway}\\
\midrule
home-hotel & \SF & \textcolor{NavyBlue}{bed}, \textcolor{NavyBlue}{towel}, \textcolor{NavyBlue}{floor} & \textit{\textcolor{red}{\textbf{footbridge}}}, \textit{\textcolor{red}{cap}}, \textit{\textcolor{red}{railroad train}} & \textcolor{NavyBlue}{table tennis}, \textcolor{NavyBlue}{\textbf{blind}}, \textcolor{NavyBlue}{bread}\\
 & \LF & \textit{\textcolor{red}{grandstand}}, \textit{\textcolor{red}{road}}, \textit{\textcolor{red}{bird}} & \textit{\textcolor{red}{bulletin board}}, \textit{\textcolor{red}{text}}, \textit{\textcolor{red}{\textbf{footbridge}}} & \textcolor{NavyBlue}{\textbf{blind}}, \textcolor{NavyBlue}{faucet}, \textit{\textcolor{red}{paper}}\\
\midrule
indoor-transport & \SF & \textcolor{NavyBlue}{floor}, \textit{\textcolor{red}{tree}}, \textcolor{NavyBlue}{work surface} & \textit{\textcolor{red}{\textbf{table tennis}}}, \textcolor{NavyBlue}{\textbf{railroad train}}, \textit{\textcolor{red}{\textbf{cap}}} & \textcolor{NavyBlue}{\textbf{clock}}, \textit{\textcolor{red}{\textbf{cap}}}, \textcolor{NavyBlue}{\textbf{railroad train}}\\
 & \LF & \textit{\textcolor{red}{grandstand}}, \textit{\textcolor{red}{dog}}, \textit{\textcolor{red}{umbrella}} & \textit{\textcolor{red}{\textbf{table tennis}}}, \textit{\textcolor{red}{\textbf{cap}}}, \textit{\textcolor{red}{lake}} & \textcolor{NavyBlue}{backpack}, \textcolor{NavyBlue}{\textbf{clock}}, \textcolor{NavyBlue}{runway}\\
%\midrule
%indoor-sports-leisure & \SF & \textcolor{NavyBlue}{grandstand}, \textcolor{NavyBlue}{floor}, \textcolor{NavyBlue}{towel} & \textcolor{NavyBlue}{table tennis}, \textcolor{NavyBlue}{horse}, \textcolor{NavyBlue}{baby buggy} & \textcolor{NavyBlue}{net}, \textit{\textcolor{red}{trunk}}, \textit{\textcolor{red}{door}}\\
% & \LF & \textit{\textcolor{red}{road}}, \textit{\textcolor{red}{dog}}, \textit{\textcolor{red}{central reservation}} & \textcolor{NavyBlue}{table tennis}, \textit{\textcolor{red}{telephone booth}}, \textit{\textcolor{red}{microwave}} & \textcolor{NavyBlue}{net}, \textit{\textcolor{red}{door}}, \textcolor{NavyBlue}{television}\\
\midrule
indoor-cultural & \SF & \textit{\textcolor{red}{sky}}, \textit{\textcolor{red}{\textbf{work surface}}}, \textcolor{NavyBlue}{desk} & \textcolor{NavyBlue}{lake}, \textit{\textcolor{red}{footbridge}}, \textcolor{NavyBlue}{dog} & \textcolor{NavyBlue}{sculpture}, \textit{\textcolor{red}{trunk}}, \textit{\textcolor{red}{\textbf{runway}}}\\
 & \LF & \textit{\textcolor{red}{\textbf{work surface}}}, \textit{\textcolor{red}{towel}}, \textcolor{NavyBlue}{central reservation} & \textit{\textcolor{red}{land}}, \textit{\textcolor{red}{microwave}}, \textit{\textcolor{red}{exhaust hood}} & \textcolor{NavyBlue}{paper}, \textit{\textcolor{red}{\textbf{runway}}}, \textit{\textcolor{red}{television}}\\
\midrule
water, ice, snow & \SF & \textcolor{NavyBlue}{\textbf{lake}}, \textcolor{NavyBlue}{mountain}, \textit{\textcolor{red}{floor}} & \textcolor{NavyBlue}{\textbf{lake}}, \textcolor{NavyBlue}{footbridge}, \textcolor{NavyBlue}{cap} & \textcolor{NavyBlue}{\textbf{land}}, \textcolor{NavyBlue}{backpack}, \textcolor{NavyBlue}{refrigerator}\\
 & \LF & \textit{\textcolor{red}{grandstand}}, \textit{\textcolor{red}{door}}, \textit{\textcolor{red}{scaffolding}} & \textit{\textcolor{red}{\textbf{horse}}}, \textcolor{NavyBlue}{\textbf{lake}}, \textit{\textcolor{red}{text}} & \textcolor{NavyBlue}{counter}, \textcolor{NavyBlue}{\textbf{land}}, \textit{\textcolor{red}{\textbf{horse}}}\\
\midrule
mountains, hills, desert & \SF & \textcolor{NavyBlue}{\textbf{mountain}}, \textcolor{NavyBlue}{sky}, \textit{\textcolor{red}{building}} & \textcolor{NavyBlue}{lake}, \textcolor{NavyBlue}{cap}, \textcolor{NavyBlue}{\textbf{land}} & \textcolor{NavyBlue}{backpack}, \textcolor{NavyBlue}{\textbf{land}}, \textcolor{NavyBlue}{\textbf{mountain}}\\
 & \LF & \textit{\textcolor{red}{bird}}, \textit{\textcolor{red}{windowpane}}, \textit{\textcolor{red}{grandstand}} & \textit{\textcolor{red}{footbridge}}, \textit{\textcolor{red}{text}}, \textit{\textcolor{red}{signboard}} & \textit{\textcolor{red}{flag}}, \textcolor{NavyBlue}{\textbf{land}}, \textcolor{NavyBlue}{\textbf{mountain}}\\
\midrule
forest, field, jungle & \SF & \textcolor{NavyBlue}{tree}, \textit{\textcolor{red}{building}}, \textit{\textcolor{red}{road}} & \textit{\textcolor{red}{footbridge}}, \textcolor{NavyBlue}{\textbf{trunk}}, \textcolor{NavyBlue}{\textbf{horse}} & \textcolor{NavyBlue}{\textbf{horse}}, \textcolor{NavyBlue}{\textbf{trunk}}, \textit{\textcolor{red}{ladder}}\\
 & \LF & \textit{\textcolor{red}{central reservation}}, \textit{\textcolor{red}{\textbf{bulletin board}}}, \textit{\textcolor{red}{\textbf{spotlight}}} & \textcolor{NavyBlue}{\textbf{trunk}}, \textcolor{NavyBlue}{\textbf{horse}}, \textit{\textcolor{red}{\textbf{spotlight}}} & \textit{\textcolor{red}{\textbf{bulletin board}}}, \textcolor{NavyBlue}{\textbf{horse}}, \textit{\textcolor{red}{forecourt}}\\
%\midrule
%man-made elements & \SF & \textit{\textcolor{red}{floor}}, \textcolor{NavyBlue}{mountain}, \textcolor{NavyBlue}{sky} & \textcolor{NavyBlue}{footbridge}, \textit{\textcolor{red}{baby buggy}}, \textcolor{NavyBlue}{cap} & \textcolor{NavyBlue}{railroad train}, \textit{\textcolor{red}{blind}}, \textit{\textcolor{red}{platform}}\\
% & \LF & \textit{\textcolor{red}{grandstand}}, \textit{\textcolor{red}{dog}}, \textit{\textcolor{red}{umbrella}} & \textit{\textcolor{red}{text}}, \textit{\textcolor{red}{baby buggy}}, \textit{\textcolor{red}{spotlight}} & \textit{\textcolor{red}{blind}}, \textit{\textcolor{red}{bread}}, \textcolor{NavyBlue}{backpack}\\
\midrule
outdoor-transport & \SF & \textcolor{NavyBlue}{road}, \textcolor{NavyBlue}{sky}, \textcolor{NavyBlue}{building} & \textit{\textcolor{red}{\textbf{lake}}}, \textit{\textcolor{red}{\textbf{cap}}}, \textcolor{NavyBlue}{footbridge} & \textcolor{NavyBlue}{\textbf{runway}}, \textcolor{NavyBlue}{bulletin board}, \textcolor{NavyBlue}{ladder}\\
 & \LF & \textit{\textcolor{red}{desk}}, \textit{\textcolor{red}{rack}}, \textit{\textcolor{red}{towel}} & \textit{\textcolor{red}{\textbf{lake}}}, \textit{\textcolor{red}{\textbf{cap}}}, \textit{\textcolor{red}{\textbf{horse}}} & \textcolor{NavyBlue}{\textbf{runway}}, \textcolor{NavyBlue}{pitcher}, \textit{\textcolor{red}{\textbf{horse}}}\\
\midrule
cultural-historic & \SF & \textcolor{NavyBlue}{building}, \textit{\textcolor{red}{\textbf{lake}}}, \textcolor{NavyBlue}{sky} & \textcolor{NavyBlue}{baby buggy}, \textit{\textcolor{red}{\textbf{lake}}}, \textcolor{NavyBlue}{\textbf{trunk}} & \textcolor{NavyBlue}{forecourt}, \textit{\textcolor{red}{\textbf{trunk}}}, \textit{\textcolor{red}{table tennis}}\\
 & \LF & \textit{\textcolor{red}{fluorescent}}, \textit{\textcolor{red}{cabinet}}, \textit{\textcolor{red}{bed}} & \textit{\textcolor{red}{\textbf{lake}}}, \textit{\textcolor{red}{horse}}, \textit{\textcolor{red}{railroad train}} & \textit{\textcolor{red}{pitcher}}, \textit{\textcolor{red}{blind}}, \textit{\textcolor{red}{runway}}\\
\midrule
sports fields, parks & \SF & \textcolor{NavyBlue}{grandstand}, \textit{\textcolor{red}{desk}}, \textcolor{NavyBlue}{tree} & \textcolor{NavyBlue}{\textbf{baby buggy}}, \textit{\textcolor{red}{lake}}, \textit{\textcolor{red}{cap}} & \textcolor{NavyBlue}{\textbf{baby buggy}}, \textcolor{NavyBlue}{\textbf{net}}, \textit{\textcolor{red}{paper}}\\
 & \LF & \textit{\textcolor{red}{scaffolding}}, \textit{\textcolor{red}{air conditioner}}, \textit{\textcolor{red}{curtain}} & \textcolor{NavyBlue}{\textbf{baby buggy}}, \textit{\textcolor{red}{telephone booth}}, \textit{\textcolor{red}{lake}} & \textcolor{NavyBlue}{\textbf{net}}, \textcolor{NavyBlue}{table}, \textit{\textcolor{red}{double door}}\\
%\midrule
%industrial-construction & \SF & \textit{\textcolor{red}{lake}}, \textcolor{NavyBlue}{building}, \textcolor{NavyBlue}{sky} & \textit{\textcolor{red}{lake}}, \textit{\textcolor{red}{trunk}}, \textcolor{NavyBlue}{dog} & \textcolor{NavyBlue}{paper}, \textit{\textcolor{red}{cap}}, \textit{\textcolor{red}{desk}}\\
% & \LF & \textit{\textcolor{red}{grandstand}}, \textit{\textcolor{red}{dog}}, \textit{\textcolor{red}{table}} & \textit{\textcolor{red}{lake}}, \textit{\textcolor{red}{trunk}}, \textit{\textcolor{red}{horse}} & \textit{\textcolor{red}{desk}}, \textit{\textcolor{red}{floor}}, \textcolor{NavyBlue}{paper}\\
\midrule
cabins, gardens, farms & \SF & \textcolor{NavyBlue}{tree}, \textcolor{NavyBlue}{plant}, \textit{\textcolor{red}{bulletin board}} & \textit{\textcolor{red}{\textbf{table tennis}}}, \textit{\textcolor{red}{cap}}, \textcolor{NavyBlue}{dog} & \textit{\textcolor{red}{\textbf{table tennis}}}, \textcolor{NavyBlue}{blind}, \textit{\textcolor{red}{backpack}}\\
 & \LF & \textit{\textcolor{red}{bird}}, \textit{\textcolor{red}{spotlight}}, \textit{\textcolor{red}{scaffolding}} & \textit{\textcolor{red}{land}}, \textit{\textcolor{red}{lake}}, \textit{\textcolor{red}{railroad train}} & \textcolor{NavyBlue}{bread}, \textit{\textcolor{red}{counter}}, \textit{\textcolor{red}{poster}}\\
\midrule
comm-buildings/towns & \SF & \textcolor{NavyBlue}{building}, \textit{\textcolor{red}{\textbf{grandstand}}}, \textcolor{NavyBlue}{road} & \textcolor{NavyBlue}{telephone booth}, \textcolor{NavyBlue}{trunk}, \textit{\textcolor{red}{\textbf{land}}} & \textcolor{NavyBlue}{\textbf{forecourt}}, \textcolor{NavyBlue}{platform}, \textit{\textcolor{red}{net}}\\
 & \LF & \textit{\textcolor{red}{\textbf{grandstand}}}, \textit{\textcolor{red}{desk}}, \textit{\textcolor{red}{piano}} & \textit{\textcolor{red}{\textbf{land}}}, \textit{\textcolor{red}{horse}}, \textit{\textcolor{red}{lake}}& \textcolor{NavyBlue}{\textbf{forecourt}}, \textcolor{NavyBlue}{poster}, \textcolor{NavyBlue}{flag}\\
 \bottomrule

\end{tabular}}
    \caption{\textbf{Selected concepts vary based on faithfulness-understandability setting.} Similar to the main text Tab. 2., we examine the concepts chosen for each scene group across 6 settings (\{\MU, \SU, \LU\} $\times$ \{\SF, \LF\}. We report the 3 concepts with highest absolute weights within the explanation. Common concepts are \textbf{bolded}, \textit{\textcolor{red}{Red}} denotes that the coefficient is negative, whereas \textcolor{NavyBlue}{blue} denotes that the coefficient is positive. We note that the concepts highlighted are typically not shared among different explanations  }
    \label{tab:grouped_attr_comp}
\end{table*}




\end{document}
