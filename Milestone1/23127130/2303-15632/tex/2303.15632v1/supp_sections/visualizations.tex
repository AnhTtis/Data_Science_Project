In this section, we give more details about our experiment set up and highlight additional results. 

\smallsec{Experimental setup}
We use a greedy method to select uncorrelated concepts, using Pearson's correlation coefficient.  For each understandability setting, we compute the correlation coefficient between all pairs of concepts (using either the base rates or the learned concept vector). Next, we compute the 90\% percentile correlation coefficient and set that as a threshold. We add each concept to the list of chosen concepts if it is not more correlated than the threshold with any of the previously chosen concepts. 

When choosing the values of $\lambda_1$ and $\lambda_2$, we fix $\lambda_1$ to 1, and pick $\lambda_2$ from $\{1.0, 0.5, 0.1, 0.05, 0.01, 0.005\}$ that best explains the model on the validation set. 

\smallsec{Concepts for coarse grained scenes}
We first report the concepts chosen across the 6 faithfulness-understandability settings described in the main text. Similar to the fine-grained model, we see that the attributes chosen vary based on the setting (Tab.~\ref{tab:grouped_attr_comp}).

\begin{table*}[t]
    \centering
    \resizebox{0.82\linewidth}{!}{
    \begin{tabular}{L{2cm} l   L{4cm}  L{4cm}  L{4cm}}
 \toprule
%\multirow{2}{*}{scene} 
%& & Acc & concepts & Acc & concepts & Acc & concepts \\
%\cmidrule(l{3pt}r{3pt}){3-4} \cmidrule(l{3pt}r{3pt}){5-6} \cmidrule(l{3pt}r{3pt}){7-8}
& & {most understandable (\MU)} & {somewhat understandable (\SU)} & {least understandable (\LU)} \\ 
\midrule 
shopping-dining & \SF & \textit{\textcolor{red}{bed}}, \textcolor{NavyBlue}{bulletin board}, \textit{\textcolor{red}{sky}} & \textcolor{NavyBlue}{\textbf{cap}}, \textit{\textcolor{red}{footbridge}}, \textcolor{NavyBlue}{baby buggy} & \textit{\textcolor{red}{\textbf{faucet}}}, \textit{\textcolor{red}{land}}, \textcolor{NavyBlue}{\textbf{cap}}\\
 & \LF & \textit{\textcolor{red}{grandstand}}, \textit{\textcolor{red}{pillow}}, \textit{\textcolor{red}{bird}} & \textit{\textcolor{red}{table tennis}}, \textcolor{NavyBlue}{\textbf{cap}}, \textit{\textcolor{red}{loudspeaker}} & \textit{\textcolor{red}{\textbf{faucet}}}, \textcolor{NavyBlue}{pitcher}, \textcolor{NavyBlue}{counter}\\
\midrule
workplace & \SF & \textit{\textcolor{red}{\textbf{sky}}}, \textcolor{NavyBlue}{floor}, \textcolor{NavyBlue}{desk} & \textcolor{NavyBlue}{\textbf{dog}}, \textcolor{NavyBlue}{footbridge}, \textit{\textcolor{red}{\textbf{baby buggy}}} & \textcolor{NavyBlue}{platform}, \textit{\textcolor{red}{bread}}, \textit{\textcolor{red}{jar}}\\
 & \LF & \textit{\textcolor{red}{\textbf{sky}}}, \textit{\textcolor{red}{bed}}, \textit{\textcolor{red}{tree}} & \textit{\textcolor{red}{\textbf{baby buggy}}}, \textcolor{NavyBlue}{\textbf{dog}}, \textit{\textcolor{red}{lake}} & \textcolor{NavyBlue}{paper}, \textcolor{NavyBlue}{bulletin board}, \textcolor{NavyBlue}{runway}\\
\midrule
home-hotel & \SF & \textcolor{NavyBlue}{bed}, \textcolor{NavyBlue}{towel}, \textcolor{NavyBlue}{floor} & \textit{\textcolor{red}{\textbf{footbridge}}}, \textit{\textcolor{red}{cap}}, \textit{\textcolor{red}{railroad train}} & \textcolor{NavyBlue}{table tennis}, \textcolor{NavyBlue}{\textbf{blind}}, \textcolor{NavyBlue}{bread}\\
 & \LF & \textit{\textcolor{red}{grandstand}}, \textit{\textcolor{red}{road}}, \textit{\textcolor{red}{bird}} & \textit{\textcolor{red}{bulletin board}}, \textit{\textcolor{red}{text}}, \textit{\textcolor{red}{\textbf{footbridge}}} & \textcolor{NavyBlue}{\textbf{blind}}, \textcolor{NavyBlue}{faucet}, \textit{\textcolor{red}{paper}}\\
\midrule
indoor-transport & \SF & \textcolor{NavyBlue}{floor}, \textit{\textcolor{red}{tree}}, \textcolor{NavyBlue}{work surface} & \textit{\textcolor{red}{\textbf{table tennis}}}, \textcolor{NavyBlue}{\textbf{railroad train}}, \textit{\textcolor{red}{\textbf{cap}}} & \textcolor{NavyBlue}{\textbf{clock}}, \textit{\textcolor{red}{\textbf{cap}}}, \textcolor{NavyBlue}{\textbf{railroad train}}\\
 & \LF & \textit{\textcolor{red}{grandstand}}, \textit{\textcolor{red}{dog}}, \textit{\textcolor{red}{umbrella}} & \textit{\textcolor{red}{\textbf{table tennis}}}, \textit{\textcolor{red}{\textbf{cap}}}, \textit{\textcolor{red}{lake}} & \textcolor{NavyBlue}{backpack}, \textcolor{NavyBlue}{\textbf{clock}}, \textcolor{NavyBlue}{runway}\\
%\midrule
%indoor-sports-leisure & \SF & \textcolor{NavyBlue}{grandstand}, \textcolor{NavyBlue}{floor}, \textcolor{NavyBlue}{towel} & \textcolor{NavyBlue}{table tennis}, \textcolor{NavyBlue}{horse}, \textcolor{NavyBlue}{baby buggy} & \textcolor{NavyBlue}{net}, \textit{\textcolor{red}{trunk}}, \textit{\textcolor{red}{door}}\\
% & \LF & \textit{\textcolor{red}{road}}, \textit{\textcolor{red}{dog}}, \textit{\textcolor{red}{central reservation}} & \textcolor{NavyBlue}{table tennis}, \textit{\textcolor{red}{telephone booth}}, \textit{\textcolor{red}{microwave}} & \textcolor{NavyBlue}{net}, \textit{\textcolor{red}{door}}, \textcolor{NavyBlue}{television}\\
\midrule
indoor-cultural & \SF & \textit{\textcolor{red}{sky}}, \textit{\textcolor{red}{\textbf{work surface}}}, \textcolor{NavyBlue}{desk} & \textcolor{NavyBlue}{lake}, \textit{\textcolor{red}{footbridge}}, \textcolor{NavyBlue}{dog} & \textcolor{NavyBlue}{sculpture}, \textit{\textcolor{red}{trunk}}, \textit{\textcolor{red}{\textbf{runway}}}\\
 & \LF & \textit{\textcolor{red}{\textbf{work surface}}}, \textit{\textcolor{red}{towel}}, \textcolor{NavyBlue}{central reservation} & \textit{\textcolor{red}{land}}, \textit{\textcolor{red}{microwave}}, \textit{\textcolor{red}{exhaust hood}} & \textcolor{NavyBlue}{paper}, \textit{\textcolor{red}{\textbf{runway}}}, \textit{\textcolor{red}{television}}\\
\midrule
water, ice, snow & \SF & \textcolor{NavyBlue}{\textbf{lake}}, \textcolor{NavyBlue}{mountain}, \textit{\textcolor{red}{floor}} & \textcolor{NavyBlue}{\textbf{lake}}, \textcolor{NavyBlue}{footbridge}, \textcolor{NavyBlue}{cap} & \textcolor{NavyBlue}{\textbf{land}}, \textcolor{NavyBlue}{backpack}, \textcolor{NavyBlue}{refrigerator}\\
 & \LF & \textit{\textcolor{red}{grandstand}}, \textit{\textcolor{red}{door}}, \textit{\textcolor{red}{scaffolding}} & \textit{\textcolor{red}{\textbf{horse}}}, \textcolor{NavyBlue}{\textbf{lake}}, \textit{\textcolor{red}{text}} & \textcolor{NavyBlue}{counter}, \textcolor{NavyBlue}{\textbf{land}}, \textit{\textcolor{red}{\textbf{horse}}}\\
\midrule
mountains, hills, desert & \SF & \textcolor{NavyBlue}{\textbf{mountain}}, \textcolor{NavyBlue}{sky}, \textit{\textcolor{red}{building}} & \textcolor{NavyBlue}{lake}, \textcolor{NavyBlue}{cap}, \textcolor{NavyBlue}{\textbf{land}} & \textcolor{NavyBlue}{backpack}, \textcolor{NavyBlue}{\textbf{land}}, \textcolor{NavyBlue}{\textbf{mountain}}\\
 & \LF & \textit{\textcolor{red}{bird}}, \textit{\textcolor{red}{windowpane}}, \textit{\textcolor{red}{grandstand}} & \textit{\textcolor{red}{footbridge}}, \textit{\textcolor{red}{text}}, \textit{\textcolor{red}{signboard}} & \textit{\textcolor{red}{flag}}, \textcolor{NavyBlue}{\textbf{land}}, \textcolor{NavyBlue}{\textbf{mountain}}\\
\midrule
forest, field, jungle & \SF & \textcolor{NavyBlue}{tree}, \textit{\textcolor{red}{building}}, \textit{\textcolor{red}{road}} & \textit{\textcolor{red}{footbridge}}, \textcolor{NavyBlue}{\textbf{trunk}}, \textcolor{NavyBlue}{\textbf{horse}} & \textcolor{NavyBlue}{\textbf{horse}}, \textcolor{NavyBlue}{\textbf{trunk}}, \textit{\textcolor{red}{ladder}}\\
 & \LF & \textit{\textcolor{red}{central reservation}}, \textit{\textcolor{red}{\textbf{bulletin board}}}, \textit{\textcolor{red}{\textbf{spotlight}}} & \textcolor{NavyBlue}{\textbf{trunk}}, \textcolor{NavyBlue}{\textbf{horse}}, \textit{\textcolor{red}{\textbf{spotlight}}} & \textit{\textcolor{red}{\textbf{bulletin board}}}, \textcolor{NavyBlue}{\textbf{horse}}, \textit{\textcolor{red}{forecourt}}\\
%\midrule
%man-made elements & \SF & \textit{\textcolor{red}{floor}}, \textcolor{NavyBlue}{mountain}, \textcolor{NavyBlue}{sky} & \textcolor{NavyBlue}{footbridge}, \textit{\textcolor{red}{baby buggy}}, \textcolor{NavyBlue}{cap} & \textcolor{NavyBlue}{railroad train}, \textit{\textcolor{red}{blind}}, \textit{\textcolor{red}{platform}}\\
% & \LF & \textit{\textcolor{red}{grandstand}}, \textit{\textcolor{red}{dog}}, \textit{\textcolor{red}{umbrella}} & \textit{\textcolor{red}{text}}, \textit{\textcolor{red}{baby buggy}}, \textit{\textcolor{red}{spotlight}} & \textit{\textcolor{red}{blind}}, \textit{\textcolor{red}{bread}}, \textcolor{NavyBlue}{backpack}\\
\midrule
outdoor-transport & \SF & \textcolor{NavyBlue}{road}, \textcolor{NavyBlue}{sky}, \textcolor{NavyBlue}{building} & \textit{\textcolor{red}{\textbf{lake}}}, \textit{\textcolor{red}{\textbf{cap}}}, \textcolor{NavyBlue}{footbridge} & \textcolor{NavyBlue}{\textbf{runway}}, \textcolor{NavyBlue}{bulletin board}, \textcolor{NavyBlue}{ladder}\\
 & \LF & \textit{\textcolor{red}{desk}}, \textit{\textcolor{red}{rack}}, \textit{\textcolor{red}{towel}} & \textit{\textcolor{red}{\textbf{lake}}}, \textit{\textcolor{red}{\textbf{cap}}}, \textit{\textcolor{red}{\textbf{horse}}} & \textcolor{NavyBlue}{\textbf{runway}}, \textcolor{NavyBlue}{pitcher}, \textit{\textcolor{red}{\textbf{horse}}}\\
\midrule
cultural-historic & \SF & \textcolor{NavyBlue}{building}, \textit{\textcolor{red}{\textbf{lake}}}, \textcolor{NavyBlue}{sky} & \textcolor{NavyBlue}{baby buggy}, \textit{\textcolor{red}{\textbf{lake}}}, \textcolor{NavyBlue}{\textbf{trunk}} & \textcolor{NavyBlue}{forecourt}, \textit{\textcolor{red}{\textbf{trunk}}}, \textit{\textcolor{red}{table tennis}}\\
 & \LF & \textit{\textcolor{red}{fluorescent}}, \textit{\textcolor{red}{cabinet}}, \textit{\textcolor{red}{bed}} & \textit{\textcolor{red}{\textbf{lake}}}, \textit{\textcolor{red}{horse}}, \textit{\textcolor{red}{railroad train}} & \textit{\textcolor{red}{pitcher}}, \textit{\textcolor{red}{blind}}, \textit{\textcolor{red}{runway}}\\
\midrule
sports fields, parks & \SF & \textcolor{NavyBlue}{grandstand}, \textit{\textcolor{red}{desk}}, \textcolor{NavyBlue}{tree} & \textcolor{NavyBlue}{\textbf{baby buggy}}, \textit{\textcolor{red}{lake}}, \textit{\textcolor{red}{cap}} & \textcolor{NavyBlue}{\textbf{baby buggy}}, \textcolor{NavyBlue}{\textbf{net}}, \textit{\textcolor{red}{paper}}\\
 & \LF & \textit{\textcolor{red}{scaffolding}}, \textit{\textcolor{red}{air conditioner}}, \textit{\textcolor{red}{curtain}} & \textcolor{NavyBlue}{\textbf{baby buggy}}, \textit{\textcolor{red}{telephone booth}}, \textit{\textcolor{red}{lake}} & \textcolor{NavyBlue}{\textbf{net}}, \textcolor{NavyBlue}{table}, \textit{\textcolor{red}{double door}}\\
%\midrule
%industrial-construction & \SF & \textit{\textcolor{red}{lake}}, \textcolor{NavyBlue}{building}, \textcolor{NavyBlue}{sky} & \textit{\textcolor{red}{lake}}, \textit{\textcolor{red}{trunk}}, \textcolor{NavyBlue}{dog} & \textcolor{NavyBlue}{paper}, \textit{\textcolor{red}{cap}}, \textit{\textcolor{red}{desk}}\\
% & \LF & \textit{\textcolor{red}{grandstand}}, \textit{\textcolor{red}{dog}}, \textit{\textcolor{red}{table}} & \textit{\textcolor{red}{lake}}, \textit{\textcolor{red}{trunk}}, \textit{\textcolor{red}{horse}} & \textit{\textcolor{red}{desk}}, \textit{\textcolor{red}{floor}}, \textcolor{NavyBlue}{paper}\\
\midrule
cabins, gardens, farms & \SF & \textcolor{NavyBlue}{tree}, \textcolor{NavyBlue}{plant}, \textit{\textcolor{red}{bulletin board}} & \textit{\textcolor{red}{\textbf{table tennis}}}, \textit{\textcolor{red}{cap}}, \textcolor{NavyBlue}{dog} & \textit{\textcolor{red}{\textbf{table tennis}}}, \textcolor{NavyBlue}{blind}, \textit{\textcolor{red}{backpack}}\\
 & \LF & \textit{\textcolor{red}{bird}}, \textit{\textcolor{red}{spotlight}}, \textit{\textcolor{red}{scaffolding}} & \textit{\textcolor{red}{land}}, \textit{\textcolor{red}{lake}}, \textit{\textcolor{red}{railroad train}} & \textcolor{NavyBlue}{bread}, \textit{\textcolor{red}{counter}}, \textit{\textcolor{red}{poster}}\\
\midrule
comm-buildings/towns & \SF & \textcolor{NavyBlue}{building}, \textit{\textcolor{red}{\textbf{grandstand}}}, \textcolor{NavyBlue}{road} & \textcolor{NavyBlue}{telephone booth}, \textcolor{NavyBlue}{trunk}, \textit{\textcolor{red}{\textbf{land}}} & \textcolor{NavyBlue}{\textbf{forecourt}}, \textcolor{NavyBlue}{platform}, \textit{\textcolor{red}{net}}\\
 & \LF & \textit{\textcolor{red}{\textbf{grandstand}}}, \textit{\textcolor{red}{desk}}, \textit{\textcolor{red}{piano}} & \textit{\textcolor{red}{\textbf{land}}}, \textit{\textcolor{red}{horse}}, \textit{\textcolor{red}{lake}}& \textcolor{NavyBlue}{\textbf{forecourt}}, \textcolor{NavyBlue}{poster}, \textcolor{NavyBlue}{flag}\\
 \bottomrule

\end{tabular}}
    \caption{\textbf{Selected concepts vary based on faithfulness-understandability setting.} Similar to the main text Tab. 2., we examine the concepts chosen for each scene group across 6 settings (\{\MU, \SU, \LU\} $\times$ \{\SF, \LF\}. We report the 3 concepts with highest absolute weights within the explanation. Common concepts are \textbf{bolded}, \textit{\textcolor{red}{Red}} denotes that the coefficient is negative, whereas \textcolor{NavyBlue}{blue} denotes that the coefficient is positive. We note that the concepts highlighted are typically not shared among different explanations  }
    \label{tab:grouped_attr_comp}
\end{table*}

