%\iffalse
Here we provide more details about the different methods, and how they can be thought of within our framework. 

\smallsec{Net2Vec and TCAV}
For Net2Vec~\cite{fong2018net2vec} and TCAV~\cite{kim2018tcav}, the authors align the feature space with concepts, without considering the final output. This can be achieved with $\lambda_1 = 0$.
The authors allow the features $f$ to be any layer within the trained model $M$, and learn $h_\text{conc}$ as a series of individual indicator functions to the concepts for each neuron in $f$ or as a linear combination of the different neurons in $f$, i.e, $h_\text{conc} \colon \mathbb{R}^n \rightarrow \mathbb{R}^C$, where $n$ is the number of neurons in a layer of the CNN and $C$ is the number of concepts. The authors allow the selection function $S$ to select all the $C$ concepts, i.e, $K=C$. Thus, these works optimize $L_\text{align}$:
\begin{align}
    \label{eq:net2vec_tcav}
    L_\text{align}^{\text{Net2Vec}, \text{TCAV}} & = \sum_{j \in \{1, 2, \ldots C\}}\sum_{x\in X} CE(A(x)_j, (h_\text{conc}\circ f(x))_j)
\end{align}
Net2Vec also considers aligning individual neurons with concepts. This can be achieved by forcing $h_\text{conc}$ to be an indicator function: each neuron is aligned with exactly one concept.  

\smallsec{NetDissect}
NetDissect~\cite{bau2017netdissect} uses a slightly different framework compared to ours, however, we show that by rewriting $L_\text{align}$, we can consider NetDissect within our framework. We first rewrite NetDissect using the following notation. 
\begin{itemize}
    \item Suppose $A_{\text{seg}} \colon X \to \mathbb{R}^{C \times H\times W}$ is the segmentation map for $C$ concepts.
    \item $f \colon \mathcal{X} \to \mathbb{R}^{n \times H' \times W'}$ is the feature space. 
    \item $t: \mathbb{R}^{n \times H' \times W'} \rightarrow \{0,1\}^{D \times H \times W}$.
    This is an upsampling and thresholding function: first, the vector is bilinearly upsampled to size $H \times W$ for each neuron, and thresholded such that only the top $0.5\%$ for a neuron is activated.
\end{itemize}
Now, for each neuron $i \in \{1, 2, \ldots n\}$, they compute the concept $j$ that maximizes
\begin{align}
\label{eq:netdissect}
\text{IOU}_i 
   := & \argmax_{j \in \{1, 2, \ldots C\}} \left(\frac{\sum_{x \in X} ((A_{\text{seg}}(x))_j \cap (t \circ f(x))_i)}{\sum_{x \in X} ((A_{\text{seg}}(x))_j \cup (t \circ f(x))_i)}\right)
\end{align}

In order to consider NetDissect within our framework, we can rewrite $L_\text{align}$ as follows. We first consider $f_i$ at a single neuron $i$, i.e. $f_i \colon \mathcal{X} \rightarrow \mathbb{R}^{H' \times W'}$. Then, rather than using $\sum_x CE(A(x), h_\text{conc}\circ f(x))$, we can express $L_\text{align}$ in terms of \cref{eq:netdissect}, with $|S| = 1$:

\begin{align}
    L_{align} = - \sum_{j \in \{1, 2, \ldots C\}} \mathbbm{1}_S \circ \text{IOU}_i
\end{align}
% For the distance function $d$, they use $d = \text{IoU}$
%\fi

\smallsec{IBD}
For IBD~\cite{zhou2018ibd}, $h_\text{conc}$ is a linear combination of the activations of each neuron and $h_\text{pred}$ is a linear combination of the outputs of $h_\text{conc}$, very similar to our \textbf{sf,su} framework. The main difference is in an additional constraint imposed on $h_\text{pred}$: that the coefficients are all non-negative, and each target class is allowed to use exactly $K$ concepts (but these do not need to be the same across target classes). Thus, for IBD, $L_\text{mimic}$ can be written as:

\begin{align}
    \label{eq:ibd}
    \forall i \in \{1, 2, \ldots, D\} & \nonumber\\
 \left(L_\text{mimic}^\text{IBD}\right)_i & = 
   \sum_{x \in X} \|(g \circ f(x))_i -  \nonumber \\ & h_\text{pred}^i  \circ \mathbbm{1}_{S_i} \circ 
   p \circ h_\text{conc}\circ f(x)
   \|
   \\
   \text{such that } \nonumber\\
&   h_\text{pred}^i(x) = W_i^Tf(x) \nonumber \\
 & \textcolor{red}{ W_{i, k}\geq 0} \ \ \ \forall k \in \{1, 2, \ldots n\} \nonumber
\end{align}

As mentioned in the main text Section 5.2, the non-negative constraint changes the concepts chosen: examples of concepts chosen are in table~\ref{tab:ibd}

\begin{table*}[t]
\centering
\begin{tabular}{L{2cm} L{6.5cm}L{6.5cm}}
\toprule
scene & IBD & UFO(Ours) \\
\toprule
attic & heater, basket, stairway, breads, magazine, television camera, drum & backpack, grandstand, coffee maker, pitcher, microwave, sand, door, sculpture\\ \arrayrulecolor{black!30}\midrule
bathroom & screen door, village, water tower, tray, candelabrum, stands, drinking glass & grandstand, backpack, crt screen, bench, microwave, double door, sculpture, work surface\\\midrule
bedchamber & headboard, pillow, shade, vault, eiderdown, water tower, tent & coffee maker, pitcher, spotlight, microwave, cabinet, door, sky, sculpture\\\midrule
bedroom & headboard, pillow, eiderdown, shade, lower sash, shower, shirt & grandstand, coffee maker, doorframe, ladder, sculpture, spotlight, work surface, clock\\\midrule
conference-room & bulletin board, wineglass, trouser, escalator, mouse, button panel, mouse pad & grandstand, platform, counter, pitcher, microwave, ladder, floor, desk\\\midrule
crosswalk & vineyard, traffic light, autobus, chain wheel, trailer, skylight, cockpit & coffee maker, grandstand, land, platform, crt screen, backpack, doorframe, television\\\midrule
dining-room & chandelier, candelabrum, back pillow, skirt, carpet, cart, grand piano & coffee maker, pitcher, platform, chest, windowpane, road, door, table\\\midrule
downtown & skyscraper, paper towel, gas station, candelabrum, place mat, slot machine, crosswalk & grandstand, land, pitcher, platform, crt screen, television, sky, sofa\\\midrule
highway & catwalk, autobus, document, book stand, dashboard, slats, corner pocket & coffee maker, platform, plant, flag, sky, lamp, door, table\\\midrule
hotel-room & bed, tracks, candlestick, cushion, seat cushion, capital, candle & land, microwave, spotlight, cabinet, sculpture, dishwasher, clock, work surface\\\midrule
kitchen & stove, refrigerator, tray, kitchen island, container, screen door, microwave & grandstand, coffee maker, backpack, crt screen, pitcher, doorframe, cap, faucet\\\midrule
living-room & post, cushion, riser, fireplace, monitoring device, sconce, bumper & pitcher, doorframe, counter, spotlight, windowpane, cabinet, door, table\\\midrule
parking-garage/outdoor & paper towel, crane, windows, notebook, steam shovel, gym shoe, television & coffee maker, grandstand, land, backpack, platform, crt screen, doorframe, television\\\midrule
recreation-room & pool table, court, microwave, table football, slot machine, wire, grand piano & grandstand, land, chest, counter, spotlight, floor, windowpane, sky\\\midrule
residential-neighborhood & sill, balloon, trailer, metal shutters, flowerpot, switch, synthesizer & coffee maker, faucet, land, platform, doorframe, sculpture, dishwasher, floor\\\midrule
skyscraper & skyscraper, display board, workbench, manhole, paw, lighthouse, gas station & coffee maker, land, pitcher, television, sky, sofa, lamp, spotlight\\\midrule
street & slats, roundabout, crosswalk, beak, arcades, bus, parking & coffee maker, land, faucet, platform, crt screen, microwave, doorframe, television\\\midrule
television-room & seat base, brick, sash, inside arm, gravel, water wheel, pantry & pitcher, chest, cap, microwave, spotlight, counter, desk, door\\\midrule
waiting-room & armchair, sconce, shoe, console table, back pillow, canvas, dishrag & pitcher, chest, counter, spotlight, sky, doorframe, sofa, table\\\midrule
youth-hostel & sweater, towel, equipment, kettle, wardrobe, vent, partition & grandstand, doorframe, ladder, microwave, spotlight, sculpture, work surface, flag\\ \arrayrulecolor{black}\bottomrule
\end{tabular}
\caption{\textbf{Concepts chosen by IBD~\cite{zhou2018ibd} versus that chosen by our method.} We see that the non-negative constraint added by IBD changes the concepts chosen by quite a lot.}
\label{tab:ibd}
\end{table*}

%\smallsec{Concept Bottleneck}
%Concept Bottleneck models~\cite{koh2020conceptbottleneck} fall neatly within our framework: these models can be written as $F:= g \circ f$, where $f$ predicts concepts from an input $x$ and $g$ combines these concepts to form the model output. Thus, $h_\text{conc}$ can be expressed as an  identity function, with $h_\text{pred}$ being $g$ itself. 
