\section{Related Work}
\label{sec:related}

Diverse techniques exists for anomaly detection in literature. Approaches include density-based techniques like k-nearest neighbor \cite{knorr2000distance}, local outlier factor \cite{breunig2000lof}, isolation forests \cite{liu2012isolation} and other variations \cite{schubert2014local}. For higher dimensional data - subspace \cite{kriegel2009outlier}, correlation \cite{kriegel2012outlier},  single class support vector machines \cite{platt1999estimating} and tensor based methods \cite{fanaee2016tensor} have been proposed. Neural network approaches include replicators \cite{hawkins2002outlier}, Bayesian Networks \cite{platt1999estimating}, hidden markov models \cite{platt1999estimating}. Additionally, cluster analysis based \cite{campello2015hierarchical}, fuzzy logic and ensemble techniques with feature bagging \cite{nguyen2010mining} and score normalization \cite{schubert2012evaluation} have been applied. 

Although these methodologies have minimal systematic advantages across data sets and parameters, many are applied in the realm of static graphs. 
To narrow the scope in relation to our proposed method regarding anomaly detection algorithms for dynamic graph networks a deeper exploration will be applied to network embedding and streaming anomaly approaches. 

\subsection{Streaming Anomaly Detection}
Streaming anomaly detection attempts to detect suspicious or uncharacteristic behavior in continuous-time streams of relational data.
\subsubsection{Streaming Graphs}
Streaming graphs can be applied to anomalous node detection like dynamic tensor analysis (DTA) \cite{sun2006beyond} which approximates an adjacency matrix for a graph at time \textit{t} with incremental matrix factorization and utilizes a high reconstruction error to.  Anomalous subgraph detection is illustrated in methods like \cite{shin2018patterns} which uses \textit{k-core} which is the maximal subgraph in which all vertices have degree at least \textit{k} and use patterns related to \textit{k-core} to find anomalous subgraphs. Streaming graphs can also be utilized for anomalous event detection such as changes in first and second derivatives in PageRank \cite{yoon2019fast}.
\subsubsection{Streaming Edges}
Streaming anomaly detection can also be applied to streaming edges. Anomalous nodes can be detected using methods like \cite{yu2013anomalous} which uses an incremental eigenvector update algorithm based in von Mises interations and discover hotspots of local changes in dynamics streams.
    SedanSpot \cite{eswaran2018sedanspot} utilizes streaming edges to detect anomalous edges by exploiting edges that connect part of the graph that are sparsely connected and finding where bursts of activity indicative of anomalous behavior.

\subsection{Network Embedding}
Network embedding approaches consist of learning low-dimensional feature representation of the nodes or links within a network. Many advances \cite{chang2015heterogeneous,levy2014neural,wang2016structural, yu2013embedding} in the network embedding space present new ways to learn representations for networks which can be used to detect anomalies. Feature learning strategies for extracting network embedding have been widely used in language models such as Skip-gram \cite{mikolov2013distributed} in which the the defining neighborhood attributes for words in a sentence are preserved. In a similar manner, many popular methodologies such as DeepWalk \cite{perozzi2014deepwalk} and Node2vec \cite{grover2016node2vec} use sequences of vertices from the graphs and learn the representation of these vertices by maximizing the preservation of the structure inherent to a neighborhood of vertices in the network. A large portion of work has also been dedicated to representation learning in other manners for graph network architectures \cite{tian2014learning, wang2016structural, zhou2018dynamic} and compact representations of the graph have been used to find anomalies within the network \cite{manzoor2016fast}. 

These methods have also been specifically applied for anomaly detection in dynamic temporal graphs. Specifically, NetWalk \cite{yu2018netwalk} learns network representations which are dynamically updated as the network evolves. This is done by by learning the latent network representations by extracted sequences of vertices from the graph, otherwise termed as "walks" from the initial network and then extract deep representations by minimizing the pairwise distance of the vertices when encoding the vector representations and adding global regularization by using a deep autoencoder reconstruction error. NetWalk is updated over dynamic changes with reservoir sampling strategy and a dynamic clustering model is used to find anomalies. 

In \cite{hibshman2021sst}, dynamic graph evolution is modeled with subgraph to subgraph transitions (SST) by fitting linear SVM models to SST count vectors. This can be used for link prediction of static, temporal, directed and undirected graphs while providing interpretable results. The idea of counting SSTs inspires our proposed methodology, however, instead of utilizing an edge based approach as this paper did, we propose a neighborhood based approach for detecting anomalies based on clusters (\textit{cf.} \autoref{sec:model}).
