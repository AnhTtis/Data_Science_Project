%\documentclass[aps,prl,reprint,groupedaddress]{revtex4-1}
\documentclass[aps,prb,reprint,superscriptaddress,
raggedbottom]{revtex4-2}
%\documentclass[aps,prl,reprint,groupedaddress]{revtex4-1}

%\usepackage[english,russian]{babel}
\usepackage{hyperref}       % hyperlinks
\usepackage{url}            % simple URL typesetting
\usepackage{booktabs}       % professional-quality tables
\usepackage{amsfonts}       % blackboard math symbols
\usepackage{tablefootnote}
\usepackage{verbatim}
\usepackage{nicefrac}       % compact symbols for 1/2, etc.
\usepackage{microtype}      % microtypography
\usepackage{lipsum}
\usepackage{mathtools}

\usepackage{amsmath,amssymb}
\usepackage{algorithm,algorithmic}
\usepackage{pifont}
\usepackage{cases}
\usepackage{subcaption,graphicx}
\usepackage{stackengine}    % circled symbols
\usepackage{wrapfig}
\usepackage{enumitem}

%\newtheorem{theorem}{Theorem}[section]
%\newtheorem{corollary}[theorem]{Corollary}
%\newtheorem{lemma}[theorem]{Lemma}
\newtheorem{assumption}[theorem]{Assumption}
%\newtheorem{definition}[theorem]{Definition}
%\newtheorem{remark}[theorem]{Remark}
%\newtheorem{proposition}[theorem]{Proposition}

\newcommand*{\LargerCdot}{\raisebox{-0.25ex}{\scalebox{2.4}{$\cdot$}}}
\newcommand{\Sum}{\displaystyle\sum\limits}
\newcommand{\Max}{\max\limits}
\newcommand{\Min}{\min\limits}
\newcommand{\fromto}[3]{{#1}=\overline{{#2},\,{#3}}}
\newcommand{\floor}[1]{\left\lfloor{#1}\right\rfloor}
\newcommand{\ceil}[1]{\left\lceil{#1}\right\rceil}

\newcommand{\tild}{\widetilde}
\newcommand{\eps}{\varepsilon}
\newcommand{\lam}{\lambda}
\newcommand{\ol}{\overline}
\newcommand{\one}{\mathbf{1}}
\newcommand{\cset}{\mathcal{C}}
%\newcommand{\Breg}{\mathcal{D}_{h}}
%\newcommand{\PBreg}{\mathbb{D}_{h}}

%\newcommand{\EndProof}{\begin{flushright}$\square$\end{flushright}}

\newcommand{\circledOne}{\text{\ding{172}}}
\newcommand{\circledTwo}{\text{\ding{173}}}
\newcommand{\circledThree}{\text{\ding{174}}}
\newcommand{\circledFour}{\text{\ding{175}}}
\newcommand{\circledFive}{\text{\ding{176}}}
\newcommand{\circledSix}{\text{\ding{177}}}
\newcommand{\circledSeven}{\text{\ding{178}}}
\newcommand{\circledEight}{\text{\ding{179}}}
\newcommand{\circledNine}{\text{\ding{180}}}
\newcommand{\circledTen}{\text{\ding{181}}}
\newcommand{\balashstar}{\stackMath\mathbin{\stackinset{c}{0ex}{c}{0ex}{\text{\ding{83}}}{\bigcirc}}}
\renewcommand\balashstar{\stackMath\mathbin{\stackinset{c}{0ex}{c}{0ex}{\ast}{\bigcirc}}}


\renewcommand{\le}{\leqslant}
\renewcommand{\ge}{\geqslant}
\renewcommand{\hat}{\widehat}

\newcommand{\numberthis}{\addtocounter{equation}{1}\tag{\theequation}}


\DeclareMathOperator*{\argmin}{arg\,min}
\DeclareMathOperator*{\argmax}{arg\,max}
\DeclareMathOperator*{\Argmin}{Arg\,min}
\DeclareMathOperator*{\Argmax}{Arg\,max}
\DeclareMathOperator{\spn}{span}
\DeclareMathOperator{\kernel}{Ker}
\DeclareMathOperator{\image}{Im}
\DeclareMathOperator{\prox}{prox}
\DeclareMathOperator{\proj}{Proj}
\DeclareMathOperator{\col}{col}
\DeclareMathOperator{\diag}{diag}

\newcommand{\N}{\mathbb{N}}
\newcommand{\R}{\mathbb{R}}
\newcommand{\Z}{\mathbb{Z}}
\newcommand{\V}{\mathbb{V}}
\newcommand{\E}{\mathbb{E}}
%\newcommand{\P}{\mathbb{P}}
\newcommand{\I}{\mathbb{I}}
\newcommand{\F}{\mathbb{F}}

\newcommand{\mA}{{\bf A}}
\newcommand{\mB}{{\bf B}}
\newcommand{\mC}{{\bf C}}
\newcommand{\mD}{{\bf D}}
\newcommand{\mE}{{\bf E}}
\newcommand{\mF}{{\bf F}}
\newcommand{\mG}{{\bf G}}
\newcommand{\mH}{{\bf H}}
\newcommand{\mI}{{\bf I}}
\newcommand{\mJ}{{\bf J}}
\newcommand{\mK}{{\bf K}}
\newcommand{\mL}{{\bf L}}
\newcommand{\mM}{{\bf M}}
\newcommand{\mN}{{\bf N}}
\newcommand{\mO}{{\bf O}}
\newcommand{\mP}{{\bf P}}
\newcommand{\mQ}{{\bf Q}}
\newcommand{\mR}{{\bf R}}
\newcommand{\mS}{{\bf S}}
\newcommand{\mT}{{\bf T}}
\newcommand{\mU}{{\bf U}}
\newcommand{\mV}{{\bf V}}
\newcommand{\mW}{{\bf W}}
\newcommand{\mX}{{\bf X}}
\newcommand{\mY}{{\bf Y}}
\newcommand{\mZ}{{\bf Z}}

\newcommand{\cA}{{\mathcal{A}}}
\newcommand{\cB}{{\mathcal{B}}}
\newcommand{\cC}{{\mathcal{C}}}
\newcommand{\cD}{{\mathcal{D}}}
\newcommand{\cE}{{\mathcal{E}}}
\newcommand{\cF}{{\mathcal{F}}}
\newcommand{\cG}{{\mathcal{G}}}
\newcommand{\cH}{{\mathcal{H}}}
\newcommand{\cI}{{\mathcal{I}}}
\newcommand{\cJ}{{\mathcal{J}}}
\newcommand{\cK}{{\mathcal{K}}}
\newcommand{\cL}{{\mathcal{L}}}
\newcommand{\cM}{{\mathcal{M}}}
\newcommand{\cN}{{\mathcal{N}}}
\newcommand{\cO}{{\mathcal{O}}}
\newcommand{\cP}{{\mathcal{P}}}
\newcommand{\cQ}{{\mathcal{Q}}}
\newcommand{\cR}{{\mathcal{R}}}
\newcommand{\cS}{{\mathcal{S}}}
\newcommand{\cT}{{\mathcal{T}}}
\newcommand{\cU}{{\mathcal{U}}}
\newcommand{\cV}{{\mathcal{V}}}
\newcommand{\cW}{{\mathcal{W}}}
\newcommand{\cX}{{\mathcal{X}}}
\newcommand{\cY}{{\mathcal{Y}}}
\newcommand{\cZ}{{\mathcal{Z}}}

\newcommand{\ba}{{\bf a}}
\newcommand{\bb}{{\bf b}}
\newcommand{\bc}{{\bf c}}
\newcommand{\bd}{{\bf d}}
\newcommand{\be}{{\bf e}}
%\newcommand{\bf}{{\bf f}}
\newcommand{\bg}{{\bf g}}
\newcommand{\bh}{{\bf h}}
\newcommand{\bi}{{\bf i}}
\newcommand{\bj}{{\bf j}}
\newcommand{\bk}{{\bf k}}
\newcommand{\bl}{{\bf l}}
\newcommand{\bm}{{\bf m}}
\newcommand{\bn}{{\bf n}}
\newcommand{\bo}{{\bf o}}
\newcommand{\bp}{{\bf p}}
\newcommand{\bq}{{\bf q}}
\newcommand{\br}{{\bf r}}
\newcommand{\bs}{{\bf s}}
\newcommand{\bt}{{\bf t}}
\newcommand{\bu}{{\bf u}}
\newcommand{\bv}{{\bf v}}
\newcommand{\bw}{{\bf w}}
\newcommand{\bx}{{\bf x}}
\newcommand{\by}{{\bf y}}
\newcommand{\bz}{{\bf z}}

\newcommand{\ds}{\displaystyle}
\newcommand{\norm}[1]{\left\| #1 \right\|}
\newcommand{\normtwo}[1]{\left\| #1 \right\|_2}
\newcommand{\sqn}[1]{\norm{#1}_2^2}
\newcommand{\angles}[1]{\left\langle #1 \right\rangle}
\newcommand{\cbraces}[1]{\left( #1 \right)}
\newcommand{\sbraces}[1]{\left[ #1 \right]}
\newcommand{\braces}[1]{\left\{ #1 \right\}}
\def\<#1,#2>{\langle #1,#2\rangle}

\newcommand{\sigmamax}{\sigma_{\max}(\cA)}
\newcommand{\sigmamaxsqr}{\sigma_{\max}^2(\cA)}
\newcommand{\sigmaminplus}{\sigma_{\min}^+(\cA)}
\newcommand{\sigmaminplussqr}{(\sigma_{\min}^+(\cA))^2}

\usepackage[colorinlistoftodos,bordercolor=blue,backgroundcolor=blue!20,linecolor=blue,textsize=scriptsize]{todonotes}
\newcommand{\arogozin}[1]{\todo[inline]{{\textbf{Alexander R.:} \emph{#1}}}}
\newcommand{\schezhegov}[1]{\todo[inline]{{\textbf{Savelii C.:} \emph{#1}}}}
 % Paste back in later
\bibliographystyle{apsrev4-2}

% \usepackage{ulem}
\newcommand{\HR}[1]{\textcolor{red}{#1}}

\begin{document}

\title{Combined investigation of collective amplitude and phase modes in a quasi-one-dimensional charge-density-wave system over a wide spectral range}

\author{Konstantin Warawa}
\affiliation{\ffm}
\author{Nicolas Christophel}
\affiliation{\ffm}
\author{Sergei Sobolev}
\affiliation{\jgu}
\author{Jure Demsar}
\affiliation{\jgu}
\author{Hartmut G. Roskos}
\affiliation{\ffm}
\author{Mark D. Thomson}
\affiliation{\ffm}
%\date{January 2023}


\begin{abstract}
We investigate experimentally both the amplitude and phase channels of the collective modes in the quasi-1D charge-density-wave (CDW) system, \kmo, by combining (i) optical impulsive-Raman pump-probe and (ii) terahertz time-domain spectroscopy (THz-TDS), with high resolution and a detailed analysis of the full complex-valued spectra in both cases.
This allows an unequivocal assignment of the observed bands to CDW modes across the THz range up to $9\THz$.
We revise and extend a time-dependent Ginzburg-Landau model to account for the observed temperature dependence of the modes, where the combination of both amplitude and phase modes allows one to robustly determine the bare-phonon and electron-phonon coupling parameters.  
While the coupling is indeed strongest for the lowest-energy phonon, dropping sharply for the immediately subsequent phonons, it grows back significantly for the higher-energy phonons, demonstrating their important role in driving the CDW formation.
We also include a reassessment of our previous analysis of the lowest-lying phase modes, whereby assuming weaker electronic damping for the phase channel results in a qualitative picture more consistent with quantum-mechanical treatments of the collective modes, with a strongly coupled amplitudon and phason as the lowest modes.
\end{abstract}  

\maketitle

\section{Introduction}

Charge density waves (CDWs) constitute an important example of symmetry-broken ground states, arising in low-dimensional conductors and typically driven by electron-phonon (e-ph) coupling,
manifesting as an electron-density modulation and periodic lattice distortion (PLD) with wavevectors $q=2\kF$.  Their study continues to take on new relevance, especially as they can appear as co-existing/competing phases in complex solids, e.g. %
unconventional superconductors \cite{chang_direct_2012,lee19,ortiz20, chen21,yu21,liu_tas2_21}, nematic compounds \cite{sato_thermodynamic_2017, yao22} and Weyl semimetals \cite{gooth19}.
%including cuprates \cite{chang_direct_2012}, 
%122-pnictides \cite{lee19}, 
%kagome metals (?) \cite{ortiz20, chen21, yu21}, 
%TMDs \cite{liu_tas2_21}. 
%Nematicity \cite{sato_thermodynamic_2017, yao22}. 
%Weyl semimetal ("axion insulator") \cite{gooth19}. 
%Recent ultrafast studies on those systems (non-equilibrium studies on collective modes (BNA) \cite{pokharel22} and band structure (trARPES on TSI) \cite{crepaldi22}.
Here the low-energy excitations offer an important spectroscopic probe, for both ground-state and non-equilibrium studies \cite{schaefer10,yusupov10,hinton13,thomson17,pokharel22,crepaldi22}. 
%
In addition to the single-particle gap, corresponding to excitation of electron-hole pairs from the CDW condensate into the adjacent conduction band (and typically lying in the mid-infrared \cite{degiorgi94}),
coupling between phonons and the electronic modulation at $q=2\kF$ gives rise to collective modes at lower energies  -- typically in the terahertz (THz) range -- which serve as a sensitive probe of the CDW physics.
While the PLD alone may lead to zone-folding (and hence allow the appearance of conventional phonons at new energies below the CDW transition temperature $T_c$), the CDW collective modes arise specifically due to e-ph coupling and exhibit physical properties  comprising both the underlying bare phonons and coupled electronic wave, the latter characterized by a complex-valued electronic order parameter (EOP, $\Delta$).
These excitations manifest as both amplitude- and phase-modes (AMs, PMs), which are respectively Raman- and 
infrared-active in centrosymmetric materials. %
%
While the bare phonons may have a vanishing dipole response vs. their lattice displacements in the normal phase (and hence be only Raman-active), the PMs possess IR-activity as an electromagnetic field can drive them via the polarization of the electron density modulation \cite{rice76,rice78}.
Nevertheless, a reliable assignment of phonon-like bands appearing below $T_c$ to CDW modes is affected by the fact that in the quasi-1D systems, one also has a transition from a normal-phase metal to a semiconducting CDW phase, such that conventional IR-active phonons can also emerge below $T_c$ due to the lifting of screening in the normal phase, and any temperature dependence could, in principle, be due to interaction with the ($T$-dependent) free carriers \cite{rice1979,mihaly1989}.

In order to reliably assign CDW collective modes, a rigorous approach is to demonstrate the simultaneous appearance of both AMs and PMs (in their respective spectroscopies), and ideally also account for their $T$-dependence with an applicable physical model.  
This is the subject of the present report, applied to the well-established quasi-1D CDW system \kmo, using both impulsive-Raman pump-probe spectroscopy and THz-TDS to characterize the AMs and PMs, respectively, with both high spectral resolution and coverage, resolving modes up to $9\THz$ ($\simx 300\pcm$).  
This study extends our previous reports \cite{schaefer10,schaefer14,thomson17}, which were limited to the lowest-frequency modes (${<}3\THz$), 
and provides a comprehensive analysis beyond those in other earlier studies of the Raman-active \cite{travaglini_raman83,sagar_raman08} and IR-active modes \cite{travaglini_ir84,ng86,degiorgi91,beyer2012} in \kmo.
%

As previously, we employ a phenomenological time-dependent Ginzburg-Landau (TDGL) model, which we now apply to account for the full set of modes and their $T$-dependence, yielding estimates of the e-ph coupling for each bare mode contributing to the manifold.
%
An important outcome of this study is that while the lowest phonon indeed has the strongest electronic coupling (akin to certain notions in the literature that only a single phonon is involved in forming the CDW \cite{sagar_raman08}), the coupling for the higher-lying phonons first weakens abruptly, but then \textit{increases} with phonon energy, demonstrating their importance for driving the CDW state formation. 

Moreover, the new results for the PM spectra lead us to a significant revision of both the lowest fitted experimental mode and parameters in the TDGL model.
Our previous analysis \cite{thomson17} of the low-lying PMs was based on the differential reflectivity changes in
optical-pump THz-probe experiments, to follow the time evolution of the non-equilibrium response.  Here a strong spectral feature at about $\nu\simx 1.75\THz$ led us to fit the data assuming a PM in this region, which allowed a detailed quantitative fit of both real and imaginary parts of the differential spectra.
The presence of a PM at this location was indeed predicted from the  TDGL model used, assuming strong damping for both the amplitude and phase components of the EOP (discussed in Sec.~\ref{sec:modes}).
One main motivation of the current work was to scrutinize this assignment with high-resolution ground-state THz reflectivity measurements.  Here we find that while such a strong feature is present in the ground-state reflectivity spectrum at this frequency, this can be reproduced by a band model where no PM is present in this vicinity, due to the strong non-local behavior in how modes affect the reflectivity spectrum.
As presented below, this leads us to review the TDGL model, assuming  a significantly weaker damping ($\gamma_2$) for the electronic phase motion, which then predicts that the lowest AM does not have a closely lying PM.  Moreover, this yields predictions for a ``phason'' (i.e. the Goldstone mode, shifted slightly from zero-frequency due to impurity pinning) much more consistent with early experiments \cite{mihaly1989, degiorgi91}, and a qualitative AM/PM arrangement more consistent with quantum-mechanical models \cite{rice76,rice78}.


\section{Experimental details}\label{sec:details}

All experiments were performed on single crystals of \kmo in the incommensurate CDW phase below $T_c=183\Kel$, using complementary time-domain spectroscopy techniques, with radiation polarized along the $b$-axis of the crystal.
The coherent detection of (Raman-active) AMs was realized in all-optical reflective pump-probe experiments, where 40-fs pulses from a 250-kHz \tisa amplifier laser at 800~nm wavelength were used for both optical pump and probe pulses, as described previously \cite{schaefer10}.

To investigate the (IR-active) PMs, we utilized broadband THz-TDS, based on a 1-kHz \tisa{} amplifier laser, using a two-color air plasma for the source
\cite{roskos2007,blank2013,thomson18,thomson23}
and electro-optic sampling (EOS) with 30-fs optical detection pulses, covering a spectral range from $\simx$0.5-7~THz (see Sec.~\ref{sec:thz}, Fig.~\ref{fig:tds}, which includes a schematic of the setup). 
We used a 100-\mumx-thick $\langle 110 \rangle$-cut GaP crystal as the EOS sensor, supported by a 3-mm-thick $\langle 100 \rangle$-cut GaP substrate, to delay signal reflections and provide a time window ${>}40\psec$ for the main THz pulse, and hence an achievable spectral resolution of ${<}25$~GHz.
Additional measurements with air-biased coherent detection (ABCD \cite{karpowicz08}) were employed at $T=20\Kel$ to reach higher THz frequencies (although the signal-to-noise ratio was superior for EOS detection used for the majority of experiments).
The THz beam path comprised four off-axis paraboloidal mirrors for imaging the beam at the sample and detection focal planes.  In order to adapt this transmission-geometry setup for reflectivity measurements, we employed a Au-coated hyperboloidal mirror (of in-house construction) to divert the beam focus onto the sample (angle of incidence $28^\circ$) in a LHe cryostat (equipped with a 50-\mumx-thick polypropylene window), which preserved the subsequent beam propagation to the detector.
Multiple optical guide beams and a camera were used to aid alignment of the sample.  
A linear THz polarizer (vendor: Tydex) was placed in the beam before focusing on the sample to ensure p-polarized THz light along the $b$-axis of the \kmo sample, while the whole setup was purged with dry air in order to suppress the water vapor response in this THz range. 

\section{Amplitude modes: Impulsive Raman scattering}\label{sec:opop}
\begin{figure*}
\centering
\includegraphics[width=\textwidth]{Figures/AM_spectra1.pdf}
\caption{
AM spectra from all-optical pump-probe (impulsive Raman) experiments for selected temperatures: (a) Spectral amplitude $|S(\nu)|$ and (b) phase $\varphi(\nu)=\arg{S(\nu)}$, including fit results (red curves). Vertical magenta lines denote fitted mode frequencies (scaled by relative band strength in (a)). %
(c) Magnified range of spectrum in (a) for $T=10\Kel$, including also the 
``incoherent'' spectrum $\usr{S}{Ram}(\nu)$ (black curve) obtained by summing band intensities.
(d) Real ($S_r$, dashed) and imaginary ($S_i$, solid) parts of fitted band spectra for selected modes.
} \label{fig:opop}
\end{figure*}

We begin by presenting the new AM analysis results from previously published, all-optical pump-probe reflectivity experiments \cite{schaefer10,schaefer14}, which allow coherent detection of AMs via their impulsive excitation and subsequent modulation of the optical probe reflectivity (the term ``impulsive'' here implying the general case, covering both impulsive and displacive limits \cite{stevens2002}). 
As mentioned in the Introduction, our ability to perform a new, comprehensive analysis of the mode spectra, i.e. including higher-frequency/weak modes, relies on globally fitting the complex Fourier spectra, as opposed to approaches such as sequential fitting of bands in sub-ranges about their center frequencies.
A summary of our method is given in Appendix~\ref{app:opop}.  
Notable aspects include: (i) One can resolve and characterize (weak) bands even in the presence of significant overlap, (ii) the initial phases $\phin$ of each mode are inherently included via the spectral phase, and (iii) the regression delivers the complex mode amplitudes $A_n$ in each iteration, such that only the mode frequencies ($\wzn$) and bandwidths ($\Gamn$) must be searched, greatly improving the reliability and speed of the algorithm.
Superposed on the time-domain signals $S(t)=\Delta R(t)/R_0$ are non-oscillatory components $\usr{S}{el}(t)$ due to predominantly electronic responses to the excitation. %
%
While one can incorporate these in the complex spectral analysis (assuming exponential decay kinetics, these manifest as zero-frequency, i.e. Drude-like, bands), we found that the conventional approach of first fitting and subtracting these contributions in the time domain \cite{schaefer10} is advantageous for the subsequent mode analysis, as one must ensure that any (either broadband or low-frequency) spectral background is effectively suppressed to allow fitting of the weak/broad modes, including careful treatment of the initial signal around $t=0$.

Examples of the time-domain signals and the multi-exponential fitting analysis for $\usr{S}{el}$ are given in the Supplementary.
The spectra of the residual oscillatory signals are shown in Fig.~\ref{fig:opop} for selected temperatures below $T_c$, in terms of both (a) the Fourier spectral amplitude $|S(\nu)|$ and (b) phase $\varphi(\nu)$.
(We will explicitly use the terms ``spectral amplitude'' and ``spectral phase'' when discussing the experimental data, to avoid any confusion with the CDW amplitude and phase channels).
Also included are the fitted spectra from the mode analysis (based on modes at the frequencies $\nuzn(T)$ denoted by the vertical lines).
One can resolve modes extending out to $9\THz$, which can be fitted both in terms of their spectral amplitude and phase, with a clear broadening of the features with increasing $T$. One sees how numerous modes manifest in $|S(\nu)|$ not as symmetric peaks, but rather with a derivative-like structure or destructive dips in the cumulative background of the other modes.
Such features clearly hamper approaches to fit the spectra on the basis of $|S(\nu)|$ alone, but are handled naturally by the inclusion of the spectral phase in the analysis. Moreover, the correspondence between the experimental and fitted phase spectra in Fig.~\ref{fig:opop}(b) also provides additional support for the validity of the fitted mode spectra.

To examine these spectral structures more closely, in Fig.~\ref{fig:opop}(c) we show a magnified range for the modes in the range 2-3~THz for $T=10\Kel$.  One sees that we resolve a doublet of two narrow adjacent modes at 2.23 and 2.24~THz (as well as two relatively close modes at 2.56/2.61~THz).  The former doublet was previously analyzed by fitting a \textit{single} mode lineshape \cite{schaefer10,schaefer14} to $|S(\nu)|$ in 
that spectral region.
One sees that for both doublets, $|S(\nu)|$ shows a significant dip between the two adjacent modes.  In order to assess this, we also calculate the ``incoherent'' band sum spectrum, $\usr{S}{Ram}(\nu)$, i.e. corresponding to the intensity sum of each fitted mode lineshape $S_n(\nu)$, i.e. $\usr{S}{Ram}(\nu)=[\Sigma_n|S_n(\nu)|^2]^{1/2}$.  This is also included in Fig.~\ref{fig:opop}(c) (black curve), and corresponds to the signal one would measure in conventional spontaneous Raman scattering measurements (notwithstanding possibly different relative band strengths, due to the distinct matrix elements for spontaneous vs. impulsive Raman interaction \cite{stevens2002}).  Evidently, the incoherent spectrum does not exhibit such pronounced local minima between the bands (nor the derivative-like structures for the modes at 2.69 and 2.77~THz), further emphasizing how band interference arises and provides more detailed information for the coherent Raman approach here.
A cursory consideration of the spectral interference might lead one to conclude that the neighboring mode pairs are significantly out of phase.  
To address this, in Fig.~\ref{fig:opop}(d), we plot the real/imaginary parts ($S^{(n)}_{r,i}$) of the fitted mode-resolved spectra $S_n(\nu)$ (Eq.~\eqref{eq:opopband}) for the doublet modes.  An inspection of the real parts demonstrates that all modes have indeed a phase $\varphi_{0n}$ close to zero (corresponding to a displacive, cosine time dependence in coherent Raman pump-probe studies \cite{stevens2002}), i.e. each $S^{(n)}_{r}(\nu)$ 
corresponds to a peak with nearly symmetric shape,
while each $S^{(n)}_{i}$ possesses a derivative shape, as well known for Lorentzian profiles in spectral response functions.  Note that for $\varphi_{0n}\rightarrow \pm \pi/2$, the real and imaginary lineshapes would indeed exchange shapes, as known for the more general Fano lineshape \cite{fano61}.
An inspection of $S^{(n)}_{i}$ then clarifies why destructive interference is observed between the two bands, as one sees that these are inherently of opposite sign (while constructive interference indeed occurs at their respective peak frequencies $\nuzn$).
Correctly accounting for this effect is clearly vital, e.g. if one were to assess the relative phase of neighboring modes in terms of the intermediate spectral structure (e.g. two neighboring modes in \textit{anti}-phase would exhibit \textit{constructive} interference between the peaks).

While we defer a presentation and analysis of the mode frequencies and damping vs. $T$ to Sec.~\ref{sec:modes} (Fig.~\ref{fig:modes}), clearly we now have a much more comprehensive set of Raman-active modes (compared to the previous analyses, which concentrated on modes at 1.68, 2.22 and 2.55~THz), to assess as candidates for CDW AMs.
Nevertheless, as discussed in the Introduction, the presence of complementary PMs is decisive for an unequivocal assignment of these bands to collective CDW modes, as one could always consider that these are usual Raman-active phonon modes which arise purely from zone-folding in the CDW phase \cite{sagar_raman08}.  
Due to the centrosymmetry in \kmo, one expects Raman-IR exclusion in the selection rules 
for conventional phonons, such that the appearance of corresponding modes is compelling evidence for their assignment as CDW modes.

\section{Phase modes: Reflective THz time-domain spectroscopy}\label{sec:thz}

\begin{figure}[h!]
\centering
\includegraphics[width=\linewidth]{Figures/TDS_pulses_setup_inset3.pdf}
\caption{(a) Example of detected THz time-domain fields with reflection sample geometry (and EOS detection): \kmo sample in the metallic phase $T=220\Kel>T_c$ (red curve, used as reference) and in the CDW phase $T=20\Kel \ll T_c$ (blue curve).  Inset shows magnified range of oscillatory signatures after the main pulse for the \kmo sample at low $T$ (while the weak residual oscillations for $T=220\Kel$ are due to residual water-vapor absorption in the THz beam path).
(b) Corresponding intensity spectra, including an additional reference spectrum using ABCD detection (see Sec.~\ref{sec:details}, used to provide extended bandwidth coverage for $T=20\Kel$). 
Also included is a schematic of a selected portion of the THz-TDS setup.
}
%
\label{fig:tds}
\end{figure}

\begin{figure*}
\centering
\includegraphics[width=\textwidth]{Figures/kmo_refl_spectra3.pdf}
\caption{Reflection spectral intensity ($R(\nu)$, left) and phase ($\varphi(\nu)$, right) obtained from THz-TDS measurements at the indicated temperatures (blue points), and fitted spectra (red curves).  Magenta vertical lines denote position of fitted modes $\nuzn$
(see text and Appendix~\ref{app:tds} for details on baseline correction method).  Lowest mode was fixed at the nominal value 0.1~THz for all $T$ during fitting. 
} \label{fig:tds_spec}
\end{figure*}

We now proceed to the PM results using reflection THz-TDS (see Sec.~\ref{sec:details} for details).
Examples of the detected THz pulse's temporal electric fields are shown in Fig.~\ref{fig:tds}(a), measured after reflection from the \kmo sample at both $T=220\Kel$ (in the metallic phase) and $T=20\Kel$ (in the CDW phase), with the 
corresponding intensity spectra in Fig.~\ref{fig:tds}(b) obtained by Fourier transformation.
One sees clearly the appearance of reflective dips across the spectrum for \kmo in the CDW phase.
Equivalently, this manifests in the time-domain field as a long oscillatory tail in the reflected field (while the main pulse is only weakly affected).
As discussed in Sec.~\ref{sec:details}, for these experiments we ensured that all additional reflections in the THz beam path are significantly delayed (or 
their effect minimized, as in the case of the thin polymer cryostat window, 
which produces weak internal reflections with a small temporal delay).
This allows a long time range and hence fine spectral resolution ($\simx 25$~GHz), without introducing spectral modulation from signal echos.
%
Despite efforts to obtain reference spectra with a gold mirror at the sample position, due to issues with the baseline (depending sensitively on alignment), 
we rather employ the \kmo sample in the metallic phase at $T=220\Kel>T_c$ as the reference, where one has a broad metallic response (for fields polarized along the b-axis, with $R_0$ varying smoothly in the range $0.8-0.9$) and negligible additional spectroscopic features in our measured THz frequency range \cite{beyer2012}.
%
The reflectivity spectra for a set of temperatures are shown in Fig.~\ref{fig:tds_spec}, both in terms of the (a) intensity $R(\nu)$ and (b) phase $\varphi(\nu)$. 
In contrast to the last section, where one obtains the mode spectra directly from the impulsive Raman signals, for the reflectivity measurements one must retrieve these via the complex Fresnel field coefficient -- see Appendix~\ref{app:tds} for details, including the baseline correction method used to account for the non-ideal reference.
%
The fitted reflection intensity- and phase-spectra are included in Fig.~\ref{fig:tds_spec}, based on a set of Lorentzian conductivity bands (Eq.~\eqref{eq:lorentz}), with mode frequencies $\nuzn$ denoted by vertical dotted lines (see Supplementary for fitted conductivity spectra, and next section for mode parameters vs. $T$).
One sees that the model reproduces both the reflection intensity- and phase-spectra well across the full bandwidth and, as per the last section, the inclusion of the spectral phase is decisive here to achieve a robust fit with numerous modes, especially with the mode broadening for increasing $T$. 

Due to the low-frequency roll-off in the spectral intensity (Fig.~\ref{fig:tds}(b)), we cannot perform a quantitative analysis of any modes in the region below 1~THz.  
Nevertheless, as shown in the Supplementary, a series of fitting tests with a low-frequency mode fixed at positions $\nu_{01}$ in the range 0-1.7~THz indicate that such a mode is indeed required, in order to obtain the reflectivity dip at $1.75\THz$ (most pronounced at low temperatures). This dip essentially manifests due to the interference of the tails of this low-frequency mode and the next higher one at 2.14~THz.
%
As mentioned in the Introduction, this feature previously led us to fit a PM very close to 1.75~THz, on the basis of non-equilibrium differential reflectivity spectra \cite{thomson17}.
The analysis of our ground-state spectra here results in a smaller misfit as  $\nu_{01}$ is lowered towards 1~THz, 
with the misfit then remaining essentially independent of $\nu_{01}$ for values below 1~THz.
Hence, we tentatively fixed the position of this low-frequency mode to $\nu_{01}=0.1$~THz, as per the experimentally proposed position of the ``phason'' in previous studies \cite{mihaly1989,degiorgi91}, for fitting our spectra for all $T$.
%
In order to provide additional data for PMs at higher frequencies than covered with EOS detection (data in Fig.~\ref{fig:tds_spec}), we also carried out an additional measurement at $T=20\Kel$ with extended bandwidth using ABCD detection (see Sec.~\ref{sec:details}, Fig.~\ref{fig:tds}(b), and Supplementary).


\section{Combined mode analysis: Time-dependent Ginzburg-Landau model}\label{sec:modes}

\begin{figure}[!h]
\centering
\includegraphics[width=0.85\linewidth]{Figures/kmo_am_pm_combined2.pdf}
\caption{
Combined temperature dependence of the experimental AMs (blue) and PMs (red) (left and right panels cover the lower and higher frequency ranges, respectively).  Vertical bars denote the band half-maximum widths $\pm\Gamma_n/2\pi$ centered on each mode frequency point \nuzn.  Additional PM data at $T=20\Kel$ (magenta) from extended-bandwidth (ABCD) THz detection (see Sec.~\ref{sec:details}).
} \label{fig:modes}
\end{figure}

In this section, we present the combined AM and PM results, and apply the TDGL model to substantiate their assignment as CDW collective modes and account for their $T$-dependence.
In Fig.~\ref{fig:modes}, we plot the fitted Raman-/IR-active mode frequencies $\nuzn(T)$ (from the last two sections, respectively), with the respective band-half-widths $\Gamma_n/2\pi$ denoted by vertical bars on each data point to depict the broadening.
One indeed observes a close pair-wise correspondence between the frequencies 
of the AMs and PMs (\nuznA and \nuznP, respectively), with the understanding that 
for the lowest-frequency respective modes, conventionally referred to as the ``amplitudon'' and ``phason'' \cite{gruener_book}, one expects $\nuzP\rarr 0$ (fixed in our spectral analysis at $\nuzP=0.1\THz$ \cite{mihaly1989,degiorgi91}, see Sec.~\ref{sec:thz}) 
and hence $\nuzA \gg \nuzP$ (with $\nuzA=1.68\THz$ at low $T$).
Significant broadening for $T\rarr T_c$ is observed for many of the modes, particularly so for the lowest AM (as reported before \cite{schaefer10,schaefer14}), but also significantly for the newly analyzed AMs above 6~THz.  While the available PM data does not approach $T_c$ as closely, the onset of a similar degree of broadening is also observed for most of the PMs for $T\rarr 145\Kel$. 
%
Compared to the previously reported AM analysis results \cite{schaefer10}, our new analysis approach here allows estimates of \nuzA approaching closer to $T_c$, and shows a more 
pronounced softening, with \nuzA falling to $\simx 1\THz$ at $T=175\Kel$.
Also, here we took care to fit the \textit{damped} mode frequencies $\wzn=2\pi\nuzn$ (see Eq.s~\eqref{eq:opopt} and \eqref{eq:opopband}), which are also those directly yielded from the TDGL eigenvalues below.  
As shown in the Supplementary, the new $\nuzA(T)$ data are also more consistent with those from conventional Raman \cite{travaglini_raman83,sagar_raman08} and neutron-diffraction \cite{pouget91} studies for $T\rarr T_c$. 

As all fitted modes exhibit a correspondence compatible with CDW collective modes, we apply a revised TDGL model with a bare coupled phonon (i.e. originally at $q=2\kF$ for $T>T_c$ with frequency $\Wzn$) for each experimental AM/PM pair (excluding the sharp, weak Raman-active modes at 1.36~THz, and at 1.72~THz just above the amplitudon \cite{schaefer10}).
The implementation of the TDGL is similar to that in our previous reports \cite{schaefer10,schaefer14,thomson17} (summarized again in the Supplementary), with the major difference here that we use a significantly weaker damping parameter $\gamma_2=0.09\cdot\gamma_1$ for the EOP-phase ($\Delta_2$) compared to $\gamma_1$ for the EOP-amplitude ($\Delta_1$), and hence also maintain the general-damping form (and not the overdamped limit \cite{schaefer10}) of the TDGL equations of motion for the phase channel. 
Note that the choice of equal nominal damping used in our previous PM study \cite{thomson17} followed certain assertions in the literature \cite{hennion92,tutis91}, the notion being that classically, the 
charge-density-compression/rarefaction (EOP-amplitude) and translation (EOP-phase) involves motion of the same condensate carriers. 
As mentioned above, the use of an equally strong damping for the EOP-phase (in combination with a strong phase-pinning parameter $\Wpin$) results in a lowest PM $\nuzP$ at higher frequencies, closer to $\nuzA$. 
However, such a prediction is no longer consistent with our revised determination of the lowest experimental PM.
There are indeed assertions in the literature that phase damping/relaxation can be significantly slower than for the amplitude in the case of CDW, where the quasi-particle excitations are neutral 
(contrary to the case in superconductivity with charged quasi-particles, where the magnitudes of the amplitude/phase damping rates are reversed) \cite{yusupov10,dolgirev20}. %
%
We note though, that these assertions are generally made in the context of the \textit{resulting} collective modes, while we instead consider here the appropriate, inherent damping magnitudes to be used for the EOP as input to the TDGL model, which in turn predicts the damping of the collective modes.
Nevertheless, as shown in the following, we find that the choice $\gamma_2=0.09\cdot\gamma_1$ does lead to revised TDGL predictions where the lowest PM is now close to zero-frequency (consistent with a nearly gapless phason), while providing a reasonable description of the other PMs.

\begin{figure*}
\centering
\includegraphics[width=\textwidth]{Figures/kmo_am_pm_tdgl3b.pdf}
\caption{
Comparison of AMs (left) and PMs (right) (Fig.~\ref{fig:modes})  with fitted TDGL model predictions. Equally colored fit curves correspond to one respective bare mode (shown as dashed lines).
} \label{fig:tdgl}
\end{figure*}

The results of the revised TDGL model are shown in Fig.~\ref{fig:tdgl}, where we plot the AMs and PMs separately for clarity, along with the experimental data from Fig.~\ref{fig:modes} (with each channel plotted in three graph columns to allow better inspection of each frequency range).
The predicted modes are plotted as filled regions tracing out $\nuzn(T)\pm\Gamma_n(T)/2\pi$ for direct comparison with the experimental mode frequencies/damping, while 
the bare phonon frequencies \Wzn are included as horizontal dashed lines (the mode coupling parameters $m_n$ and their bare-mode frequency dependence are discussed in detail below and presented in Fig.~\ref{fig:coupling}, while a full account of the other parameters is given in the Supplementary).  
To achieve these TDGL results, we tuned the bare mode frequencies $\Wzn$ and coupling parameters $m_n$, in conjunction with the global Ginzburg-Landau parameter $\alpha$ (which determines the restoring force about the $T$-dependent equilibrium EOP amplitude $\Delta_0=\alpha(T_c-T)/\beta$) and impurity pinning potential via $\Wpin$, in order to best reproduce both the experimental AMs and PMs simultaneously. 
(Note that $\beta$ is the coefficient of the quartic term $\propto \Delta^4$ in the TDGL potential energy, and drops out of the equations for the AM/PM frequencies at the equilibrium).
Here we employ a $T$-independent model for the impurity pinning $\Wpin$ (see Supplementary), which yields a pinned phason with nearly constant frequency, as observed experimentally \cite{mihaly1989,degiorgi91}.  
The model results show overall a good qualitative agreement with experiment, near-quantitatively for many of the modes, although certain systematic deviations are evident, such as the precise PM frequencies and some of the trends approaching $T_c$, in particular the lack of PM broadening which is significant for several experimental PMs (including the phason at 0.1~THz, although here the fitting of the bandwidth is tentative, given that the mode peak lies outside the fitting range). Still, such deviations are not surprising, considering the simplicity of the phenomenological TDGL model.

The assumption of $T$-independent coupling parameters $m_n$ neglects any influence of, e.g., the presence of normal electrons (density \usr{N}{th}) thermally promoted across the CDW gap as $T$ approaches $T_c$, which could screen the e-ph coupling \cite{rice1979,mihaly1989}. 
As the relative fraction of charges in the CDW condensate should follow 
$\usr{N}{C}(T)/N_0=\delta_0(T)$ 
(where $\delta_0(T)=\Delta_0(T)/\Delta_0(0)=\sqrt{1-T/T_c}$), in a two-fluid model we then have 
$\usr{N}{th}(T)/N_0=1-\delta_0(T)$, and allow for a $T$-dependent coupling via 
$m_n(T)=m_n(0)(1-b_n\cdot \usr{N}{th}(T)/N_0)$.
Indeed this correction (applied sparingly to selected bare phonons $\Wzn$) allowed us to refine the correspondence for the PMs between 2-2.5~THz (applied to the bare phonon at $1.82\THz$), and the   
AMs between 5-6~THz (applied to the bare phonons at 5.31, 5.59 and 5.76~THz), each with 
moderate values $b_n=0.3(5)$ (see Supplementary) -- these extensions are incorporated in the TDGL results in Fig.~\ref{fig:tdgl}.

We assessed incorporating several other, physically plausible $T$-dependent effects into the TDGL model, to see if these might readily account for the remaining deviations, as discussed in the following. %
%
To investigate mechanisms which could lead to PM broadening, we considered the effects of 
(i) $T$-dependent EOP-phase damping ($\gamma_2(T)$ increasing for $T\rarr T_c$) and (ii) inherent bare-phonon damping with thermal anharmonic broadening \cite{schaefer10}.  Neither of these extensions provided a convincing improvement for describing the experimental trends, where we observed that the bare-phonon damping does not translate directly to the resultant PM damping.
%
While such directions to extend the TDGL description deserve further investigation in ongoing studies, it seems prudent to first develop an estimate of the expected magnitude of such corrections from microscopic models, before incorporating them in the phenomenological TDGL framework here.

To conclude this section, we focus on the magnitude of the coupling parameters $m_n$, in particular their dependence on their respective bare-mode frequencies $\Wzn$.
Within the TDGL model, at equilibrium, the amplitude of the $n$th phonon coordinate is given by
$\xi_{0n}=(m_n/\Wzn^2)\Delta_0$, 
which results in an elastic deformation energy cost of 
$U_{Ln}=+\tfrac{1}{2}\Wzn^2\xi_{0n}^2=\tfrac{1}{2}m_n^2\Delta_{0}^2/\Wzn^2$ but a stabilization energy of
$U_{Cn}=-m_n\Delta_0\cdot\xi_{0n}=-2 U_{Ln}$, i.e. twice the magnitude of the elastic energy cost.
Based on this $1/\Wzn^2$-dependence, one might infer that the contribution of each bare phonon to the CDW formation decreases with increasing $\Wzn$. However, we show here that, based on our TDGL parameters for \kmo, this effect is actually countered by the growth of $m_n$ vs. $\Wzn$ for the higher-energy phonons.

In Fig.~\ref{fig:coupling}, we plot $m_n$ vs. $\Wzn$ for the set of bare phonons employed in the TDGL analysis in Fig.~\ref{fig:tdgl}. 
As can be seen, there is a clear increasing trend vs. $\Wzn$.  To interpret this result more physically, one must transform the TDGL parameters to a measure which reflects the inherent e-ph coupling strength, as is the case for the dimensionless e-ph coupling parameters $\lambda_n$ in quantum-mechanical models \cite{rice76,rice78} (as was used tentatively in an early report of the phase modes in \kmo at a single temperature, $T=6\Kel$ \cite{degiorgi91}).
To this end, in Appendix~\ref{app:qm}, we derive a correspondence between $m_n$ and $\lambda_n$ (Eq.~\eqref{eq:lambda}), with the result that 
$\lambda_n\propto (m_n/\Wzn)^2$.  The relative calculated values of $\lambda_n$ are also shown in Fig.~\ref{fig:coupling}.  Ones sees that while the value of $\lambda_1$ indeed is significantly larger than the values of the subsequent modes, for $n\geq 2$ there is clear (roughly linear) increase in the dimensionless e-ph coupling, even after correcting for the inherent $\Wzn$-dependence in $m_n$.
This is in contrast to the treatment in \cite{degiorgi91}, where a constant nominal value of $\lambda_{n\geq 2}$ was assumed for the modes, and indicates that these stiffer phonons possess character which influence the electronic energy more significantly.  This result strongly motivates ab initio/DFT calculations to assign the structural character of the bare modes and investigate how they interact with the electronic orbitals in more detail.

\begin{figure}
\centering
\includegraphics[width=0.9\linewidth]{Figures/coupling_vs_bare1.pdf}
\caption{
Dependence of e-ph coupling parameters from TDGL analysis on bare-mode frequency: (a) $m_n$ parameters from TDGL model (at $T=0$ for modes where $T$-dependence was employed), (b) relative dimensionless e-ph coupling parameters $\lambda_n$ accounting for inherent frequency scaling of $m_n$.
%
} \label{fig:coupling}
\end{figure}

\section{Conclusion}

The combined study of the CDW collective modes in \kmo for both amplitude and phase channels provides strong support for their assignment, whereby the TDGL model applied here indicates that higher frequency modes are indeed strongly coupled to the electronic density wave and play an important role in promoting the CDW phase.
These results strongly motivate first-principles calculations of the phonons and e-ph coupling, although this remains challenging for relatively complex materials such as \kmo.
%
From the experimental side, in addition to returning to the non-equilibrium response of these modes \cite{thomson17,rabia19}, a rigorous determination of the ground-state phase response in the low-frequency range (${\lesssim} 1.5\THz$) is still lacking, being complicated by the inherent issue of resolving spectral features in reflection with $R\approx 1$.  Here we are pursuing THz transmission studies of thin exfoliated flakes, although the small lateral dimensions of sufficiently thin samples (thickness ${<}10\mum$) are limited, requiring specialized spectroscopic methods.
Here deposited \kmo thin films \cite{staresinic2012,dominko2021}, which exhibit near-crystalline AM response, could provide essential experimental results, if their morphological properties maintain the PM response for macroscopic field interaction.
%
The TDGL model can be readily applied to account for the collective modes, and serves as a versatile framework which can be applied to systems with multiple, coupled order parameters and for ultrafast non-equilibrium studies \cite{yusupov10,hinton13,zhou21}. However, its phenomenological basis necessitates further studies based on microscopic quantum-mechanical/many-body models \cite{rice76,rice78} to better establish its validity and estimate effective parameters.
Current efforts here require the extension of such quantum-mechanical treatments to rigorously treat the finite-temperature case and account for effects due to e.g. Coulomb interactions and impurities, where developments have already begun \cite{hansen23}.

\section*{Acknowledgments}

We gratefully acknowledge funding by the German Research Foundation (DFG) via the Collaborative Research Center TRR 288 (422213477, project B08).  We thank Viktor Kabanov, Max Hansen, Yash Palan, Viktor Hahn, Falko Pientka, Oleksandr Tsyplyatyev, and Peter Kopietz for helpful discussions, and Virna Kisi\v{c}ek for participating in preliminary experiments.

\appendix
\section{Spectral analysis of impulsive Raman signals}
\label{app:opop}

The ansatz for the model differential reflectivity signal $S(t)\equiv \Delta R(t)/R_0$ vs. delay time $t$ following the pump pulse is given by:
\beq
S(t)=A_n e^{-t/\tau_n}\cos(\wzn t - \phin)\Theta(t)
\label{eq:opopt}
\eeq
(the sum over mode index $n$ is implied, $\Theta$ the Heaviside function), where incoherent (i.e. purely electronic) signal components correspond to $\wzn=0$.  Fourier transformation of Eq.~\eqref{eq:opopt} yields the sum over a set of general Lorentzian (Fano) lineshapes
\begin{flalign}
S(\omega) &= \frac
{(\Gamma_n + i\omega)A_n \cos\phin + \wzn A_n \sin\phin}
{(\wzn^2+\Gamn^2)-\omega^2+2i\Gamn\omega}  \notag \\
&=c_n f_n(\omega) + s_n g_n(\omega),
\end{flalign}
with damping $\Gamn=\tau_n^{-1}$, $c_n=A_n \cos\phin$, $d_n=A_n \sin\phin$, and the basis functions are given by 
\begin{flalign}
f_n(\omega)&=(\Gamn+i\omega)/D_n(\omega), \qquad    
g_n(\omega)=\wzn/D_n(\omega),  \notag \\
D_n(\omega)&=(\wzn^2+\Gamn^2)-\omega^2+2i\Gamn\omega.
\label{eq:opopband}
\end{flalign}
Clearly $A_n^2=c_n^2+s_n^2$ and $\tan\phin=s_n/c_n$.
For the purely electronic components ($\wzn=\phin=0$), this reduces to the Drude form $S_n=A_n/(\Gamn+i\wzn)$.

Taking into account the experimental impulse response $H(t)$ (i.e. 
cross-correlation of the pump- and probe-pulse intensity profiles, taken as a Gaussian function with FWHM temporal width of $\simx 80$~fs here), one has $S(t)\rightarrow S(t)\ast H(t)$, or for the spectra, $S(\omega)\rightarrow S(\omega)\cdot H(\omega)$, such that this response can be simply multiplied into the basis functions $f_n, g_n$.  For each iteration in the optimization algorithm, one generates the basis functions for the current values of $\wzn$ and $\Gamn$, and performs linear regression to minimize the misfit $\Sigma_j |S_j-\hat{S}_j|^2$, where $S_j=S(\omega_j)$ and $\hat{S}$ denotes the complex experimental spectrum. 

As discussed in the main text, while this approach allows one to simultaneously fit both the incoherent ($\wzn=0$)  and oscillatory components ($\usr{S}{el}$, $\usr{S}{osc}$, respectively), in practice for the spectra here we rather first perform a fit of $\usr{S}{el}$ in the time domain, and subtract this result to fit the modes in the residual spectrum
$\usr{S}{osc}=S-\usr{S}{el}$.  This allows one to closely scrutinize (and take steps to further minimize) any residual from the electronic response before fitting the modes, which is particularly important to cleanly fit the higher-frequency modes.

While fitting the complex spectra may appear to be equivalent to fitting the time-domain signal directly, the essential difference is that any deviations from the ideal model signal in Eq.~\eqref{eq:opopt} (e.g. due to frequency chirp or a non-exponential decay envelope) are more robustly ameliorated by the spectral misfit function.
% Add note on NL search algorithms?
% We typically used a robust Nelder-Mead simplex algorithm, although the ...

\section{Spectral analysis of THz reflectivity spectra}\label{app:tds}

For THz-TDS reflectivity measurements, one must retrieve the complex relative permittivity $\epsr{r}(\nu)$ from the complex reflectivity field coefficient $r(\nu)$, which, for our case of oblique incidence ($\theta=28^\circ$) and p-polarised field, is given by \cite{dresselbook}:
\beq
r=\sqrt{R}\cdot e^{i\phi}=
-\frac{\epsr{r} C_i-C_t}{\epsr{r} C_i+C_t},
\label{eq:fresnel}
\eeq
where $C_i=\cos{\theta}$ and $C_t=(\epsr{r}-\sin^2{\theta})^{1/2}$.
The corresponding conductivity spectrum is then calculated via 
$\sigma=i\omega\epsilon_0(\epsr{r}-\epsr{\infty r})$ \cite{hangyo05,thomson18}.
Due to the sensitivity of the directly recovered conductivity spectra to the precise reference baseline, especially here with strongly reflecting samples (as well as an inadvertent reference pulse delay $\delta t$ which introduces a phase term $r\rarr r\cdot e^{-i\omega\delta t}$) we instead fit $r(\omega)$ directly, with a conductivity model comprising a standard Lorentzian band for each mode \cite{dresselbook},
%
\beq
\sigma(\omega) = \frac{i\omega \sigma_{0n}}{\wzn^2-\omega^2+i\Gamma_n\omega}.
\label{eq:lorentz}
\eeq
(again summing over mode index $n$) with $\epsr{\infty r}$ and $\delta t$ included in the fit parameters to minimise the misfit $\Sigma_j|r_j-\hat{r}_j|^2$ to the experimental data $\hat{r}_j=\hat{r}(\omega_j)$.  
In order to compensate the resulting intensity/phase baseline of the non-ideal reference (the metallic phase of the \kmo sample), for each iteration we applied a complex correction factor $r(\omega) \rarr P(i\omega)\cdot r(\omega)$ where $P$ was taken as a third-order complex polynomial determined adaptively via regression of the model reflectivity spectrum.
The experimental results in Fig.~\ref{fig:tds_spec} correspond to those following this correction, i.e. $\hat{r}(\omega) \rarr \hat{r}(\omega)/P(i\omega)$.
While this approach introduces additional uncertainty into the fitting analysis, it still maintains a reasonable degree of robustness as one fits the full complex spectra (Fig.~\ref{fig:tds_spec}).
%
\\
\section{Correspondence of coupling parameters between TDGL and quantum-mechanical models}
\label{app:qm}

In Sec.~\ref{sec:modes} we obtain estimates of the coupling parameters $m_n$ for each coupled phonon mode (with bare frequencies \Wzn) from fitting the experimental modes with the TDGL model, as presented in Fig.~\ref{fig:coupling}.
In order to obtain parameters with a more transparent physical interpretation, here we derive a correspondence between the TDGL $m_n$ parameters and the dimensionless e-ph coupling parameters $\lambda_n$ from quantum-mechanical models \cite{rice76,rice78,khomskii_book}, albeit in a simplified classical limit,  
to account for any inherent dependence on $\Wzn$.
Following the development in \cite{khomskii_book}, one can write the lattice displacement along the $n$th phonon coordinates with wavevector $q=2k_F$ as 
$u_n=(2M_n\Wzn/\hbar)^{-1}\cdot 2b$ ($M_n$ the reduced mass), where 
one takes $(b_{nq}^{\dag}+b_{n,-q})\rarr 2b_n \delta(q-2 k_F)$ for the phonon operators.  
The elastic deformation energy cost is then $U_{Ln}=\tfrac{1}{2}M_n \Wzn^2u_n^2=\hbar \Wzn b_n^2$ while the coupling energy is 
$U_{Cn}=-2g_n \rho b_n$, where $g\equiv g_{2k_F}$ is the e-ph constant in the Fröhlich coupling term, and 
$\rho \equiv \rho_{2k_F}=\langle \Sigma_{k,\sigma}c^{\dag}_{k-2k_F,\sigma}c_{k,\sigma}\rangle$ represents the electronic density modulation amplitude.

The corresponding expressions based on the TDGL potential energy \cite{schaefer10,schaefer14,thomson17} are 
$U_{Ln}=\tfrac{1}{2}\Wzn^2 \xi_n^2$ and $U_{Cn}=-m_n\Delta \cdot \xi_n$.  Hence we can associate $\xi_n = \sqrt{M}\cdot u_n$ and
\beq
g_n=m_n \sqrt{\frac{\hbar}{2\Wzn}}\frac{\Delta}{\rho}.
\eeq
Due to the arbitrary scaling of the EOP amplitude $\Delta$ in the TDGL equations, the ratio $\Delta/\rho$ is undetermined but constant.
One then obtains for the dimensionless electron phonon constant:
\beq
\lambda_n = \frac{g_n^2N_0}{\hbar \Wzn} = \frac{N_0}{2}
\left(\frac{\Delta}{\rho}\right)^2\frac{m_n^2}{\Wzn^2},
\label{eq:lambda}
\eeq
where $N_0$ is the electronic density of states at the Fermi energy (in the normal undistorted phase).  Eq.~\eqref{eq:lambda} shows that the inherent e-ph coupling depends on the ratio $m_n^2/\Wzn^2$, which we then take into account in Sec.~\ref{sec:modes} (Fig.~\ref{fig:coupling}) in assessing the dependence on $\Wzn$.

%\bibliography{shorttitles,references,refs_extra}
%\bibliography{CPA_Lab,refs_extra}
% This must be in the first 5 lines to tell arXiv to use pdfLaTeX, which is strongly recommended.
\pdfoutput=1
% In particular, the hyperref package requires pdfLaTeX in order to break URLs across lines.

\documentclass[11pt]{article}

% Remove the "review" option to generate the final version.
%\usepackage[review]{ACL2023}
\usepackage{ACL2023}

% Standard package includes
\usepackage{times}
\usepackage{latexsym}

% For proper rendering and hyphenation of words containing Latin characters (including in bib files)
\usepackage[T1]{fontenc}
% For Vietnamese characters
% \usepackage[T5]{fontenc}
% See https://www.latex-project.org/help/documentation/encguide.pdf for other character sets

% This assumes your files are encoded as UTF8
\usepackage[utf8]{inputenc}

% This is not strictly necessary, and may be commented out.
% However, it will improve the layout of the manuscript,
% and will typically save some space.
\usepackage{microtype}

% This is also not strictly necessary, and may be commented out.
% However, it will improve the aesthetics of text in
% the typewriter font.
\usepackage{inconsolata}


% If the title and author information does not fit in the area allocated, uncomment the following
%
%\setlength\titlebox{10cm}
%
% and set <dim> to something 5cm or larger.

%%%%%%%%%%%%%%%%%%%%%%%%%%%%%%%%%%
\usepackage{graphicx}
\usepackage{amsfonts}
\usepackage{amsmath}
\usepackage{bigdelim}
\usepackage{diagbox}
\usepackage{amsthm}
\usepackage{makecell}
\usepackage{mathtools}
\usepackage{booktabs}
\usepackage[shortlabels]{enumitem}
\graphicspath{ {figs/} }

\theoremstyle{remark}
\newtheorem*{question}{Question}

\newcommand{\tk}[1]{\textcolor{blue}{{#1}}}
\newcommand{\sy}[1]{\textcolor{red}{{#1}}}
\newcommand{\mg}[1]{\textcolor{purple}{{#1}}}
\newcommand{\lh}[1]{\textcolor{green}{{#1}}}
\newcommand{\lc}[1]{\textcolor{green}{{#1}}}

% Rounded color box
\definecolor{light_blue}{HTML}{cfdfff}
\usepackage[most]{tcolorbox}
\tcbset{on line, 
        boxsep=1pt, left=0pt,right=0pt,top=0pt,bottom=0pt,
        colframe=white,colback=light_blue,  
        highlight math style={enhanced}
        }

\newcommand{\quash}[1]{}  %Anything in \quash is ignored
\newcommand{\gpt}{\textsc{GPT-2}}
\newcommand{\bert}{\textsc{BERT}}
\newcommand{\bertlarge}{\textsc{BERT-large}}
\newcommand{\mask}{\texttt{[MASK]}}
\newcommand{\cls}{\texttt{[CLS]}}
\newcommand{\sep}{\texttt{[SEP]}}
\newcommand{\mat}{\texttt{mat}}
\newcommand{\id}{\texttt{id}}
\newcommand{\matl}{\texttt{mat}_{\ell \rightarrow \ell'}}
\newcommand{\matattnl}{\texttt{mat\_attn}_{\ell \rightarrow \ell'}}
\newcommand{\matffl}{\texttt{mat\_ffn}_{\ell \rightarrow \ell'}}
\newcommand{\matlnl}{\texttt{mat\_ln1\_ln2}_{\ell \rightarrow \ell'}}
\newcommand{\idl}{\texttt{id}_{\ell \rightarrow \ell'}}
\newcommand{\matlL}{\texttt{mat}_{\ell \rightarrow L}}
\newcommand{\matattnlL}{\texttt{mat\_attn}_{\ell \rightarrow L}}
\newcommand{\matfflL}{\texttt{mat\_ffn}_{\ell \rightarrow L}}
\newcommand{\matlnlL}{\texttt{mat\_ln1\_ln2}_{\ell \rightarrow L}}
\newcommand{\idlL}{\texttt{id}_{\ell \rightarrow L}}

\definecolor{blue(munsell)}{rgb}{0.0, 0.5, 0.69}
%%%%%%%%%%%%%%%%%%%%%%%%%%%%%%%%%%

\title{Jump to Conclusions: Short-Cutting Transformers\\With Linear Transformations}

% Author information can be set in various styles:
% For several authors from the same institution:
% \author{Author 1 \and ... \and Author n \\
%         Address line \\ ... \\ Address line}
% if the names do not fit well on one line use
%         Author 1 \\ {\bf Author 2} \\ ... \\ {\bf Author n} \\
% For authors from different institutions:
% \author{Author 1 \\ Address line \\  ... \\ Address line
%         \And  ... \And
%         Author n \\ Address line \\ ... \\ Address line}
% To start a seperate ``row'' of authors use \AND, as in
% \author{Author 1 \\ Address line \\  ... \\ Address line
%         \AND
%         Author 2 \\ Address line \\ ... \\ Address line \And
%         Author 3 \\ Address line \\ ... \\ Address line}

\author{Alexander Yom Din$^{1}$ ~~~~~ Taelin Karidi$^{1}$ ~~~~~ Leshem Choshen$^{1}$ ~~~~~
Mor Geva$^{2}$ 
\vspace{0.2cm} \\
$^1$Hebrew University of Jerusalem ~~~ $^2$Google Research \\
\small{\texttt{\{alexander.yomdin, taelin.karidi, leshem.choshen\}@mail.huji.ac.il}}, \small{\texttt{pipek@google.com}}}

\quash{
\author{Alexander Yom Din \\
  Hebrew University of Jerusalem \\ \texttt{alexander.yomdin@mail.huji.ac.il} \\\And
  Taelin Karidi \\
  Hebrew University of Jerusalem \\
  \texttt{taelin.karidi@mail.huji.ac.il} \\\And
  Leshem Choshen \\
  Hebrew University of Jerusalem \\ \texttt{leshem.choshen@mail.huji.ac.il} \\\And
  Mor Geva \\
  Google Research \\
  \texttt{pipek@google.com} \\}
}

\begin{document}
\maketitle



\begin{abstract}
% \vspace{-1em}
The diffusion-based generative models have achieved remarkable success in text-based image generation. However, since it contains enormous randomness in generation progress, it is still challenging to apply such models for real-world visual content editing, especially in videos. 
In this paper, we propose \texttt{FateZero}, a zero-shot text-based editing method on real-world videos without per-prompt training or use-specific mask. 
\RM{Specifically, different from a pipeline of two independent inversion and then generation stages, we find the intermediate attention maps during inversions store better structure and motion information. We thus reform them to temporally casual attention and replace them in the generation progress. To further reduce the unnecessary semantic leakage of source video and enhance the editing quality, we then remix the temporally casual attentions via the cross-attention features of the source prompt as the mask.}
To edit videos consistently, we propose several techniques based on the pre-trained models. Firstly, in contrast to the straightforward DDIM inversion technique, our approach captures intermediate attention maps during inversion, which effectively retain both structural and motion information. These maps are directly fused in the editing process rather than generated during denoising. To further minimize semantic leakage of the source video, we then fuse self-attentions with a blending mask obtained by cross-attention features from the source prompt. Furthermore, we have implemented a reform of the self-attention mechanism in denoising UNet by introducing spatial-temporal attention to ensure frame consistency.
Yet succinct, our method is the first one to show the ability of zero-shot text-driven video style and local attribute editing from the trained text-to-image model. We also have a better zero-shot shape-aware editing ability based on the text-to-video model~\cite{tuneavideo}. \RM{Besides video, our unified method also achieves state-of-the-art performance in zero-shot image editing.\chenyang{Need exp or remove the zero-shot image}} Extensive experiments demonstrate our superior temporal consistency and editing capability than previous works.
% The code will be released.
% \chenyang{emphasize: our observation at inversion time} \xiaodong{replacing the bold part to the actual pipeline: \textbf{Specifically, we work on replacing and mixing the attention maps between the inversion and generation since the self-attention map keeps the structure of the original natural image and the cross-attention is semantic-related, after remixing, we replace them in the corresponding generation steps for denoising.}}
% \footnote{Since there is no general video diffusion model is publicly available, we use one-shot video generation method~(Tune-A-Video~\cite{tuneavideo}) as the pretrained video diffusion model for zero-shot video editing\xiaodong{can be removed if we actually zero-shot on video}.}.
\end{abstract}
\section{Introduction}

The ability to reason about plans is critical for performing long-horizon tasks \citep{erol1996hierarchical, sohn2018hierarchical, sharma-etal-2022-skill}, compositional generalization \citep{corona-etal-2021-modular} and generalization to unseen tasks and environments \citep{shridhar2020alfred}.
Consider a simple long-horizon planning scenario where a robot is tasked with preparing a meal and serving it on the table. 
This presents a non-trivial planning problem since the agent needs to understand the sequence of operations required to perform the task and search for the relevant objects in the unfamiliar environment by interacting with various objects. %



Large language models have been recently shown to possess commonsense knowledge about the world such as object affordances and physical dynamics \citep{ouyang2022training,chowdhery2022palm}.
Early approaches considered text based environments and fine-tuned PLMs to predict actions given the history of past observations and actions \citep{jansen-2020-visually,micheli-fleuret-2021-language,yao-etal-2020-keep}.
Recent work has used this ability to reason about plans from text instructions in simulated household environments with simplifying assumptions such as text-only environment observations or feedback \citep{huang2022language,ahn2022can,li2022pre,logeswaran-etal-2022-shot}.


We focus on \emph{visually grounded planning} with PLMs --- the ability to adapt plans based on interaction and visual feedback from the environment.
While PLMs have strong planning commonsense priors, predictions from a PLM may not be directly realizable in the environment since the observation and action spaces are unknown.
This requires \emph{grounding} the PLM in the environment and adapting it to observe visual feedback, which is highly non-trivial.
Some prior works assume the availability of a pre-trained affordance function \citep{ahn2022can} or a success detector \citep{mirchandani2021ella}.
Notably, SayCan \citep{ahn2022can} completely decouples the PLM from observation information by selecting actions that have both high affordability (through a pre-trained affordance model) and high PLM likelihood.
Although this partially addresses the grounding problem, the use of visual feedback for action affordance alone is limited.
Often an agent must choose one of many affordable actions using information from observations.
For example, a driving agent should re-navigate and possibly turn around when encountering a ``road closed'' sign, but both turning around and driving forward are indistinguishable to SayCan because they are both affordable and the PLM is blind to observations.

Another workaround explored in prior work is translating the information in the visual observations to text using a pre-trained captioning system \citep{shridhar2021alfworld,huang2022language}.
However, it can be difficult to faithfully describe an image in words and information is lost in this inherently noisy process, which limits the information available to the planner.



Recent work shows that PLMs can be adapted for various natural language tasks by inserting tunable embeddings or soft prompts at the input of the PLM (also called prompt tuning or prefix tuning)~\citep{li-liang-2021-prefix,lester-etal-2021-power}.
This approach also extends to multi-modal understanding tasks such as image captioning \citep{mokady2021clipcap} and VQA \citep{tsimpoukelli2021multimodal} where images are encoded as soft prompts and finetuned for the target task.
Transformer based architectures have also been successfully applied to offline Reinforcement Learning in recent work \citep{chen2021decision,janner2021offline,li2022pre,reid2022can}.

Taking inspiration from these works, we propose the simple approach of embedding visual observations (`visual prompts') and \textit{directly inserting them as PLM input embeddings}.
The visual encoder and PLM are jointly trained for the target task, an approach we call \textbf{\oursfull}~(\ours).
By teaching the PLM to use observations for planning in an end to end manner, we remove the dependency on external data such as captions and affordability information that was used in prior work.
We show that this simple approach performs better than prior PLM-based planning approaches on two embodied planning benchmarks based on ALFWorld~\citep{shridhar2021alfworld} and Virtualhome~\cite{puig2018virtualhome}.



\section{Related Work}

%Here we summarize prior work on transfer learning and property inference.

%\shortsection{Transfer Learning}
%%Transfer learning reuses features learned by pre-trained models for new tasks, with the pretext that inherent similarities in the generic features will be useful for the downstream tasks and hence reducing their cost of downstream training. Specifically, the downstream model trainer will use a pre-trained upstream model as the starting point for the downstream training, with inclusion of (or replacement with) the task-specific classification layer/module. The downstream model is then trained by either updating all layers of the model (including ones reused from upstream model) or freezing some earlier layers of the reused parts as the ``feature extractor'' and only updating the rest. The latter approach is more popular as the reused feature extractors can already learn useful feature representations and the training cost is also much lower and affordable for individuals with limited computational resources. We study the vulnerability of the latter transfer learning approach in this paper. 


%\shortsection{Transfer Learning} 
Several works have demonstrated risks associated with transfer learning across a variety of attack goals. Wang et al.~\cite{wang2018great} and Yao et al.~\cite{yao2019latent} consider manipulating the upstream model such that the fine-tuned downstream models contain backdoors, misclassifying test inputs that contain predefined backdoor triggers. These transfer manipulations are tailored to their particular attack goals and cannot be applied for the property inference goal considered in this paper. Zou et al.~\cite{zou2020privacy} study the threat of membership inference attacks on transfer learning, but with normally trained upstream models.  
%\dnote{its clear that the goals are different for these attacks, but how similar are the methods?} \ynote{similarity of the methods? more details about the methods? do not know what is expected here}
%In contrast, we investigate the possibility of boosting the effectiveness of property inference by manipulating the upstream model training. % Schuster et al.~\cite{schuster2020humpty} show that the attacker can modify the corpus on which the word embedding is trained such that the downstream NLP models which use that embedding will behave abnormally.

%\shortsection{Property Inference}
The risk of property inference was introduced by Ateniese et al.~\cite{ateniese2015hacking}, % introduces the threat of inferring properties of the training data from pre-trained models, 
and several subsequent works have developed property inference (also known as distribution inference) attacks~\cite{Wang2022GroupPI, suri2022formalizing, Jurez2022BlackBoxAF, Hartmann2022DistributionIR}.
% Ganju et al.~\cite{ganju2018property} and Suri and Evans~\cite{suri2022formalizing} 
These works study property inference against normally trained models, and they launch attacks using a variety of black-box and white-box attacks. All the white-box attacks use meta-classifiers, which take the permutation-invariant representation~\cite{ganju2018property} of the model parameters as the features. We use the state-of-the-art white-box attack~\cite{suri2022formalizing} in our experiments.
%We will use the state-of-the-art white-box method proposed by Ganju et al.~\cite{ganju2018property} and later extended by suri et al.~\cite{suri2022formalizing} in this paper.
%\dnote{do we use these attacks?} 
Melis et al.~\cite{melis2019exploiting} and Zhang et al.~\cite{zhang2021leakage} focus on property inference in distributed training scenarios. In their settings, the attacker is a participant in the global model training and conducts property inference using meta-classifiers that are trained on model outputs or gradients. Similarly, Suri et al.~\cite{suri2022subject} focus on federated learning settings where the attacker is a participant (or the central server) that utilizes black-box attacks for inferring membership of data from particular subjects. %\dnote{if we use black-box attacks, explain which ones, or how ours are related to previous ones} 
For our experiments, We improve the black-box meta-classifier proposed by Zhang et al.~\cite{zhang2021leakage} using the ``query tuning'' technique in Xu et al.~\cite{xu2019detecting}. 

The closest works to ours are Chase et al.~\cite{saeed} and Chaudhari et al.~\cite{Chaudhari2022SNAPEE}, which both consider a scenario where the attacker can manipulate some of the training data of the model to induce a model that significantly increases property inference risk.
% \dnote{it enables precise property inference attacks?}.
These works assume an adversary with the ability to poison the victim's training data, while the adversary in our scenario has no access to the victim's training data, and therefore, their methods are not applicable.
% \dnote{example how different from ours, and why the methods are not applicable}
%Thus, their methods are not applicable to our transfer learning scenario.
%Their methods rely on inducing certain behavior correlated with the properties to be inferred, and thus are not applicable to our transfer learning scenario. \anote{Still a bit unclear why that is the case.}
%
There are also works similar to ours that leverage ``adversarial initializations'' for attack purposes.
% \cite{grosse2019adversarial, boenisch2021curious, wen2022fishing, fowl2021robbing}.
Grosse et al.~\cite{grosse2019adversarial} focus on scenarios where the attacker can control the parameter initialization of a model, and demonstrate that the attacker can use special initializations to damage the performance of the trained model. %This attack is orthogonal to ours.
Other works \cite{boenisch2021curious, wen2022fishing, fowl2021robbing} show that the malicious central server in a federated learning protocol can reconstruct some training samples via falsifying the global model in some training rounds and then analyzing the submitted gradients. These kinds of attacks do not apply to our transfer-learning scenario since the attacker cannot access the downstream gradients, and can only manipulate the upstream training.

\iffalse %%%%%%%%%%%%%%%%%%%%%%%%%%%%%%%%

In this section, we provide the background and also the summary of prior attacks on transfer learning (Section~\ref{sec:transfer_learning}) and property inference (Section~\ref{sec:property_inference}). Then, we introduce the closely related manipulation attacks against machine learning models to boost different privacy risks in Section~\ref{sec:active_inference_attacks}.

%\anote{Do we really need a dedicated section for this? It's barely 2 paragraphs right now.}

%\dnote{the most closely related work to ours are works that attempt to amplify inference attacks by poisoning models, the two most relevant I know of are \url{https://www.computer.org/csdl/proceedings-article/sp/2022/131600b569/1CIO8nmuota} and \url{https://arxiv.org/abs/2204.00032}, but need to look thoroughly for others. We should definitely be describing this and relating it to our work, probably in the introduction. Most of what is here is Background, but should be clear what this section is for (not muddling background and related work)}

\subsection{Transfer Learning} \label{sec:transfer_learning}
Transfer learning reuses features learned by pre-trained models for new tasks, with the pretext that inherent similarities in generic features can be useful for downstream tasks, thus reducing the cost of downstream training. Specifically, the downstream model trainer uses a pre-trained upstream model as the starting point for downstream training, with the inclusion (or replacement) of task-specific classification layers/modules. The downstream model is then trained by either updating all layers of the model (including ones reused from the upstream model) or freezing some earlier layers of the reused parts as the ``feature extractor'' and only updating the rest. The latter approach is more popular as the reused feature extractors can already learn useful feature representations and the training cost is also much lower and affordable for individuals with limited computational resources. We study the vulnerability of the latter transfer learning approach in this paper. 
%mainly in two ways:  1) all the layers (including ones reused from ) and tune the full model; the other one is to freeze some earlier layers of the model as the feature extractor and only tune the rest later layers. The second update strategy could achieve better efficiency since the frozen layers can already produce meaningful feature representations~\cite{wang2018great,yao2019latent}, and we will study the transfer learning using this strategy. 

Recently, various attacks have been proposed for the transfer learning setting, but with different attack goals from ours. Wang et al.~\cite{wang2018great} generate adversarial examples against black-box student models that transfer knowledge from publicly available teacher models without repeated queries. Yao et al.~\cite{yao2019latent} propose to manipulate the upstream model such that the downstream models derived from the upstream model contain backdoors, which would misclassify test inputs that contain some predefined backdoor triggers. Zou et al.~\cite{zou2020privacy} study the threat of membership inference attacks on transfer learning and the upstream models are trained normally. In contrast, we investigate the possibility of boosting the effectiveness of property inference by manipulating the upstream model training. Schuster et al.~\cite{schuster2020humpty} show that the attacker can modify the corpus on which the word embedding is trained such that the downstream NLP models which use that embedding will behave abnormally.

%This additionally allows model trainers to achieve satisfactory performance with limited training samples, leading to reduced computational costs. The most common approach reuses parameters in the earlier layers of the pre-trained model, either by fixing them as the feature extractor or just using them for initialization, to conduct downstream training.

\subsection{Property Inference} \label{sec:property_inference}

\shortsection{Property Inference Attacks} In property inference attacks, the adversary aims to infer some sensitive properties of some data, given a model trained on it. For example, the adversary may be interested in sensitive properties like the presence of people of a specific race in the dataset~\cite{ateniese2015hacking, melis2019exploiting}), or even be curious about the 
the statistics of the training set (e.g, the ratio of people with a specific gender~\cite{saeed, ganju2018property, suri2022formalizing, zhang2021leakage}).


Ateniese et al.~\cite{ateniese2015hacking} were the first to identify the threat of inferring properties of the training data from pre-trained models. Ganju et al.~\cite{ganju2018property} and Suri and Evans~\cite{suri2022formalizing} 
study property inference against normally trained models, and they launch attacks using white-box meta-classifiers, which utilize the permutation-invariance representation~\cite{ganju2018property} of the model parameters, while other works focus on distributed training~\cite{zhang2021leakage} where the attacker is a participant in the global model training and conducts property inference using meta-classifiers trained on model outputs. Similarly, Suri et al.~\cite{suri2022subject} focus on federated learning, where the attacker is a participant (or the central server) that utilizes black-box attacks for inferring membership of data from particular subjects. Chase et al.~\cite{saeed} propose an active property inference attack for data poisoning scenarios, which we will cover and compare to in Section~\ref{sec:active_inference_attacks}.

%The closest work to ours are by Chase et al.~\cite{saeed} and Tramer et al.~\cite{tramer2022truth}. In their work, the attacker can manipulate some of the training data of the model such that a model trained (from scratch) on the poisoned data has an increased inference risk. However, their methods are not applicable to the transfer learning scenario. 
%In this work, we will focus on the property inference in transfer learning scenarios in which the attacker releases the upstream model and infer sensitive properties of the downstream models tuned from that upstream model.
% 

\shortsection{Defenses}
Defending against property inference attacks is an open problem. There are no studies in the current literature on active adversaries, and only a couple on passive ones. Ma et. al.~\cite{ma2021nosnoop} propose a defense against property inference attacks on data batches in the  collaborative learning setting. However, adversaries in the transfer-learning setting do not have access to batch-wise gradients of the downstream trainer. Chen and Ohrimenko~\cite{chen2022protecting} utilize mechanisms that add carefully-crafted noise to features to provide theoretical guarantees against inference adversaries, but focus on query-based access to the underlying dataset, not a machine learning model trained on it. These existing defenses thus do not apply to our threat model.

%propose a framework that reduces property inference to Boolean functions of individual members, posing the ratio of members satisfying the given function in a dataset as the property. These property inference attacks have since then been proposed as distribution inference attacks~\cite{suri2022formalizing}, presenting such attacks as inferring properties of the distributions used to sample datasets, differentiating them from exact inference attacks like dataset inference~\cite{maini2021dataset}. Nearly all property inference attacks use meta-classifiers to perform inference: training models on versions of datasets with and without the target property, followed by training a meta-classifier on top of these classifiers's model representations. These representations can take several forms: using model weights themselves with permutation-invariance~\cite{ganju2018property}, or model activations or logits for a generated set of query points~\cite{xu2019detecting}. However, the capability of such approaches is limited: the most that these attacks have been shown to work is medium-sized convolutional networks on the CelebA dataset~\cite{suri2022formalizing}.


\subsection{Active Privacy Attacks} \label{sec:active_inference_attacks}
% Perhaps the closely related works to ours as ones that proactively enhance the effectiveness of privacy attacks by manipulating the model training process in certain ways~\cite{saeed, melis2019exploiting, nasr2019comprehensive, tramer2022truth}. 
%shown that the adversary can, by using proactive ways, achieve stronger attacks that infer private information from deep learning systems~\cite{nasr2019comprehensive, melis2019exploiting, tramer2022truth, saeed}. In this section, we introduce the ones that are close to ours.

In the decentralized federated learning training, by submitting specially crafted gradients to the central server, malicious agents can increase membership inference risk~\cite{nasr2019comprehensive} and property inference risks~\cite{melis2019exploiting} of other benign agents' training data. However, these attacks do not apply to transfer learning scenario, as the attacker cannot control model gradients of downstream training. In the centralized setting, researchers propose attacks to poison the victim's training data such that the impacts of attribute inference and membership inference~\cite{tramer2022truth} and property inference~\cite{saeed} attacks are amplified on the poisoned model.
The ability to poison the victim's data is a threat model orthogonal to ours, since we have no access to the victim's downstream data. While there is scope to combine such approaches for stronger attacks (albeit with stronger access assumptions), we choose to focus on the scenario with no read/write access to the victim's data.

\fi %%%%%%%%%%%%%%%%%%%%%%%%%%%%%%%%

\section{Linear Shortcut Across Blocks}
\label{sec:layer_jump}

To use a hidden representation from layer $\ell<L$ as a final representation, we propose to cast it using linear regression, while skipping the computation in-between these layers. More generally, this approach can be applied to cast any $\ell$-th hidden representation to any subsequent layer $\ell'>\ell$.


\subsection{Method}
\label{subsec:methodology_linear_shortcut}

Given a source layer $\ell$ and a target layer $\ell'$ such that $0 \leq \ell < \ell' \leq L$, our goal is to learn a mapping
%$A_{\ell', \ell} \in \mathbb{R}^{d_h \times d_h}$
from hidden representations at layer $\ell$ to those at layer $\ell'$. To this end, we first collect a set of corresponding hidden representation pairs $(h^\ell, h^{\ell'})$. Concretely, we run a set $\mathcal{T}$ of input sequences through the model, and for each input $s$, we extract the hidden representations $h_{i_s}^{\ell}, h_{i_s}^{\ell'}$, where $i_s$ is a random position in $s$.
Next, we learn a matrix $A_{\ell', \ell} \in \mathbb{R}^{d_h \times d_h}$ by fitting linear regression over $\mathcal{T}$, i.e., $A_{\ell', \ell}$ is a numerical minimizer for:
$$ A \mapsto \sum_{s \in \mathcal{T}} || A \cdot h_{i_s}^\ell - h_{i_s}^{\ell'} ||^2,$$ 
and define the mapping of a representation $h$ from layer $\ell$ to layer $\ell'$ as:
\begin{equation}
\label{eq:linear_jump}
    \matl{} (h) \coloneqq A_{\ell', \ell} \cdot h.
\end{equation}


\subsection{Baseline}
\label{subsec:baseline}

We evaluate 
% our method against 
the prevalent approach of ``reading'' hidden representations directly, without any transformation. 
Namely, the propagation of a hidden representation from layer $\ell$ to layer $\ell'$ is given by the identity function, dubbed \id{}:

$$ \idl{} (h) \coloneqq h.$$

% Notably, 
This baseline 
assumes that representations at different layers operate in the same linear space.

\subsection{Quality of Fit}
\label{subsec:experiments_r2}

We first evaluate our method by measuring how well the learned linear mappings approximate the representations at the target layer. To this end, we calculate the (coordinate-averaged) $r^2$-score of our mapping's outputs with respect to the representations obtained from a full inference pass, and compare to the same for the \id{} baseline.


\paragraph{Models.}

We use \gpt{} \cite{radford2019language}, a decoder-only auto-regressive LM, with $L = 48$, $d_h = 1600$, and \bert{} \cite{devlin-etal-2019-bert}, an encoder-only model trained with masked language modeling, with $L=24$, $d_h=1024$.
% \footnote{\label{footnote:hf}We use models and data from Huggingface \cite{wolf-etal-2020-transformers,lhoest-etal-2021-datasets}.}
%For masked token prediction, we use a masked LM head pre-trained for our \bert{} model.

% \footnote{Specifically, we use the Huggingface Transformers \cite{wolf-etal-2020-transformers} implementations of all these models.}

%\sy{We use \gpt{} \cite{radford2019language}, a decoder-only auto-regressive LM, coming in four scales; $\texttt{gpt2}$ ($L = 12$, $d_h = 768$), $\texttt{gpt2-medium}$ ($L = 24$, $d_h = 1024$), $\texttt{gpt2-large}$ ($L = 36$, $d_h = 1280$) and $\texttt{gpt2-xl}$ ($L = 48$, $d_h = 1600$). Also, we use \bert{} \cite{devlin-etal-2019-bert}, an encoder-only model trained with masked language modeling, coming in two scales;  \texttt{bert-base-uncased} ($L=12$, $d_h=768$) and \texttt{bert-large-uncased} ($L=24$, $d_h=1024$). For masked token prediction, we use masked LM heads pre-trained for our models. Specifically, we use the Huggingface Transformers \cite{wolf-etal-2020-transformers} implementations of all these models. The plots presented in this section are for $48$-layered \gpt{} and $24$-layered \bert{}.}

%\sy{We use \gpt{} \cite{radford2019language}, a decoder-only auto-regressive LM, in the Huggingface \cite{wolf-etal-2020-transformers} implementation\footnote{\url{https://huggingface.co/gpt2}}, coming in four scales; $\texttt{gpt2}$ ($L = 12$, $d_h = 768$), $\texttt{gpt2-medium}$ ($L = 24$, $d_h = 1024$), $\texttt{gpt2-large}$ ($L = 36$, $d_h = 1280$) and $\texttt{gpt2-xl}$ ($L = 48$, $d_h = 1600$). Also, we use \bert{} \cite{devlin-etal-2019-bert}, an encoder-only model trained with masked language modeling, in the Hugginface implementation, coming in two scales;  \texttt{bert-base-uncased}\footnote{\url{https://huggingface.co/bert-base-uncased}} ($L=12$, $d_h=768$) and \texttt{bert-large-uncased}\footnote{\url{https://huggingface.co/bert-large-uncased}} ($L=24$, $d_h=1024$). For masked token prediction, we use the \texttt{BertForMaskedLM} heads from Huggingface, pretrained for these models. The plots presented in this section are for $48$-layered \gpt{} and $24$-layered \bert{}.}

\paragraph{Data.}
We sample random sentences from Wikipedia,
% \footref{footnote:hf} 
collecting 9,000 (resp. 3,000) sentences for the training set $\mathcal{T}$ (resp. validation set $\mathcal{V}$).\footnote{We use sentences rather than full documents to simplify the analysis.}
%\sy{We use two data sources to evaluate our method. One is Wikiepdia \cite{lhoest-etal-2021-datasets}\footnote{\url{https://huggingface.co/datasets/wikipedia}}; we use \texttt{spaCy}\footnote{\url{https://spacy.io/}} to divide documents into sentences\footnote{We use sentences rather than full documents to simplify the analysis.}\footnote{We pick randomly a Wikipedia document and then pick randomly a sentence ending in a newline character in it. \sy{[maybe this footnote is not needed?]}}, collecting 9,000 (resp. 3,000) random sentences for the training set $\mathcal{T}$ (resp. validation set $\mathcal{V}$). The second is a news article sentences dataset, the 10K English 2020 news sentences corpus
% \footnote{\url{https://downloads.wortschatz-leipzig.de/corpora/eng_news_2020_10K.tar.gz}} from the Leipzig Corpora Collection \cite{goldhahn-etal-2012-building}, which we randomly divide into a training set $\mathcal{T}$ consisting of 9,000 examples and a validation set $\mathcal{V}$ consisting of 1,000 examples.
% We truncate sentences to the maximal token length allowed by the model \mg{do we ever need to truncate? a sentence has about 10 words and the max. input len is thousands} \sy{[I surely did not need to in Leipzig, but discovered (via a transformers runtime warning) that I do need to for some (probably a minority) of the Wikipedia sentences. This probably has to do with that it is not really ``sentences" necessarily, for example, I noticed that it has some listings or something like that (bulleted items)... So some minority might get very long I guess...]}.
For each example $s$, we select a random position $i_s$ and extract the hidden representations $h_{i_s}^{\ell}$ at that position from all the layers.
For \bert{}, we first replace the input token at position $i_s$ with a \mask{} token, as our motivation is interpreting predictions, which are obtained via masked tokens in \bert{} (see \S\ref{subsec:BERT}).
Thus, in this case, the hidden representations we consider
%in the case of \bert{}
are of \mask{} tokens only.
%As we observed highly similar results for the two data sources across all our experiments, throughout the paper we will mainly report results for Wikipedia (except for \S\ref{sec:robustness}, where we cross-validate).


\begin{figure}[t]
\includegraphics[scale=0.2]{figs/r2_scores_48.pdf}
% \includegraphics[width=\columnwidth]{figs/r2_scores_48.pdf}
\caption{The coordinate-averaged $r^2$-score of $\matl{}$ (left) and $\idl{}$ (right) (\gpt{}).}
\label{fig:r2_scores}
\end{figure}


\begin{figure}[t]
\setlength{\belowcaptionskip}{-10pt}
\includegraphics[scale=0.2]{figs/bertmask_r2_scores_24.pdf}
% \includegraphics[width=\columnwidth]{figs/bertmask_r2_scores_24.pdf}
\caption{The coordinate-averaged $r^2$-score of $\matl{}$ (left) and $\idl{}$ (right) (\bert{}).}
\label{fig:bertmask_r2_scores}
\end{figure}



\paragraph{Evaluation.}
For every pair of layers $\ell, \ell'$, such that $0 \leq \ell < \ell' \leq L$, we use the training set $\mathcal{T}$ to fit linear regression as described in \S\ref{subsec:methodology_linear_shortcut}, and obtain a mapping $\matl{}$. 
Next, we evaluate the quality of $\matl{}$ as well as of $\idl{}$ using the $r^2$-coefficient, uniformly averaged over all coordinates. Concretely, we compute the $r^2$-coefficient of each of the predicted representations $\matl{} (h_{i_s}^{\ell})$ and $\idl{} (h_{i_s}^{\ell})$ versus the true representations $h_{i_s}^{\ell'}$
over all $s \in \mathcal{V}$.
%as we vary $s \in \mathcal{V}$.
%for every $s \in \mathcal{V}$.



\paragraph{Results.}
Results for \gpt{} and \bert{} are presented in Figs.~\ref{fig:r2_scores} and~\ref{fig:bertmask_r2_scores}, respectively.
In both models, \mat{} consistently yields better approximations than \id{}, as it obtains higher $r^2$-scores (in blue) across the network. 
This gap between \mat{} and \id{} is especially evident in \bert{}, where \id{} completely fails to map the representations between most layers, suggesting that hidden representations are modified  substantially by every transformer block.
Overall, this highlights the shortcoming of existing practices to inspect representations in the same linear space, and the gains from using our method to approximate future layers.
% in the network.
\section{Linear Shortcut for Language Modeling}
\label{sec:prediction}

We saw that our method approximates future hidden representations substantially better than a naive propagation. 
In this section, we will show that this improvement also translates to better predictive abilities from earlier layers. Specifically, we will use our method to estimate how often intermediate representations encode the final prediction, in the context of two fundamental LM tasks; next token prediction and masked token prediction.

\paragraph{Evaluation Metrics.}
Let $h, h' \in \mathbb{R}^{d_h}$ be a final representation and a substitute final representation obtained by some mapping, and denote by $\delta (h), \delta (h') \in \mathbb{R}^{d_v}$ their corresponding output probability distributions (obtained through projection to the output vocabulary -- see details below). 
We measure the prediction quality of $h'$ with respect to $h$ using two metrics:
\begin{itemize}
[leftmargin=*,topsep=1pt,parsep=1pt]
    \item \textbf{Precision@$k$} ($\uparrow$ is better): This checks whether the token with the highest probability according to $\delta(h')$ appears in the top-$k$ tokens according to $\delta(h)$. Namely, we sort $\delta(h)$ and assign a score of $1$ if $\arg\max(\delta(h'))$ appears in the top-$k$ tokens by $\delta(h)$, and $0$ otherwise.
    
    \item \textbf{Surprisal} ($\downarrow$ is better): We measure the minus log-probability according to $\delta(h)$, of the highest-probability token according to $\delta(h')$. Intuitively, low values mean that the model sees the substitute result as probable and hence not surprising.
\end{itemize}

\noindent We report the average Precision@$k$ and Surprisal over the validation set $\mathcal{V}$.



\subsection{Next Token Prediction}
\label{subsec:next_token_prediction_task}

Auto-regressive LMs output for every position a probability distribution over the vocabulary for the next token. Specifically, the output distribution for every position $i$ is given by $\delta (h_i^L)$, where:
\begin{equation}\label{eq:output_distribution}
    \delta (h) = \texttt{softmax} ( E^\top \cdot h) \in \mathbb{R}^{d_v}
\end{equation}
For some LMs, including \gpt{}, a layer normalization $\texttt{ln\_f}$ is applied to the final layer representation before this conversion (i.e., computing $\delta (\texttt{ln\_f}(h))$ rather than $\delta (h)$).

Recall that our goal is to measure how well this distribution can be estimated from intermediate representations, i.e. estimating $\delta (h_i^L)$ from $\delta (h_i^\ell)$ where $\ell<L$. To this end, we first run examples from the validation set through the model, while extracting for each example $s$ the hidden representation of a random position $i_s$ at every layer. Next, we apply our mappings $\matlL{}$ and the $\idlL{}$ baseline to cast the hidden representations of every layer $\ell$ to final layer substitutes (see \S\ref{sec:layer_jump}). Last, for each layer, we convert its corresponding final-layer substitute to an output distribution (Eq.~\ref{eq:output_distribution}) and compute the average Precision@$k$ (for $k=1,5,10$) and Surprisal scores with respect to the final output distribution, over the validation set.

\paragraph{Results.}
Figs.~\ref{fig:pre} and~\ref{fig:surp} show the average Precision@$k$ and Surprisal scores per layer in $48$-layered \gpt{}, respectively (the plots for the other \gpt{} models are presented in \S\ref{sec:app_scale}). Across all layers, \mat{} outperforms \id{} in terms of both scores, often by a large margin (e.g. till layer $44$ the Precision@$1$ achieved by \mat{} is bigger than that of $\id{}$ by more than $0.2$). 
This shows that linear mappings enable not just better estimation of final layer representations, but also of the predictions they induce. Moreover, the relatively high Precision@$k$ scores of \mat{} in early layers ($0.62$-$0.82$ for $k=10$, $0.52$-$0.74$ for $k=5$, and $0.28$-$0.45$ for $k=1$) suggest that early representations already encode a good estimation of the final prediction. Also, the substantially lower Surprisal scores of \mat{} compared to \id{} imply that our method allows for a more representative reading into the layer-wise prediction-formation of the model than allowed through direct projection to the vocabulary.

\begin{figure}[t]
\centering
\includegraphics[scale=0.4]{figs/pre_48.pdf}
\caption{Precision@$k$ ($k = 1,5, 10$) of $\matlL{}$ and $\idlL{}$ for next token prediction in $48$-layered \gpt{}.}
\label{fig:pre}
\end{figure}

\begin{figure}[t]
\centering
\includegraphics[scale=0.35]{figs/surp_48.pdf}
\caption{Surprisal for $\matlL$ and the baseline $\idlL{}$ ($48$-layered \gpt{} next token prediction task). A 95\% confidence interval surrounds the lines.}
\label{fig:surp}
\end{figure}

\subsection{Masked Token Prediction}
\label{subsec:BERT}

We now conduct the same experiment for the task of masked language modeling, where the model predicts a probability distribution of a masked token in the input rather than the token that follows the input. Unlike next token prediction, where the output distribution is computed from representations of varying input tokens, in masked token prediction the output is always obtained from representations of the same input token (i.e. \texttt{[MASK]}).

For this experiment, we use \bert{}, on top of which we use a pretrained masked language model head $\delta$; given a token sequence $s$, a \mask{} token inside it and its final representation $h$, $\delta (h) \in \mathbb{R}^{d_v}$
 is a probability distribution over tokens giving the model's assessment
 of the likelihood of tokens to be fitting in place of the \mask{} token in $s$.


\begin{figure}[t]
\centering
\includegraphics[scale=0.4]{figs/bertmask_pre_24.pdf}
\caption{Precision@$k$ ($k = 1,5, 10$) for  $\matlL{}$ and the baseline $\idlL{}$ ($24$-layered \bert{} masked token prediction task).}
\label{fig:bertmask_pre}
\end{figure}

\begin{figure}[t]
\centering
\includegraphics[scale=0.35]{figs/bertmask_surp_24.pdf}
\caption{Surprisal for $\matlL{}$ and the baseline $\idlL{}$ ($24$-layered \bert{} masked token prediction task). A 95\% confidence interval surrounds the lines.}
\label{fig:bertmask_surp}
\end{figure}

\paragraph{Results.}
Figs.~\ref{fig:bertmask_pre} and~\ref{fig:bertmask_surp} present the average Precision@$k$ and Surprisal scores per layer in $24$-layered \bert{} (the plots for the $12$-layered \bert{} model are presented in \S\ref{sec:app_scale}), overall showing trends similar to those observed for next token prediction in \gpt{} (\S\ref{subsec:next_token_prediction_task}). This is despite the differences between the two tasks and the considerable architectural differences between \bert{} and \gpt{}.
Notably, the superiority of \mat{} over \id{} in this setting is even more prominent; 
while \mat{}'s precision is between $0.2-0.6$ in the first ten layers (Fig.~\ref{fig:bertmask_pre}), \id{}'s precision for all values of $k$ is close to zero, again strongly indicating that our method allows for better reading into early layer hidden representations. 
More generally, \mat{} improves the Precision@$1$ of \id{} by more than $17\%$ at most layers, and unveils that a substantial amount of predictions ($>25\%$ starting from layer $3$) appear already in the very first layers.
Interestingly, the (rough) divide between the first half of layers and last half of layers for $\id{}$ in Figs.~\ref{fig:bertmask_pre},~\ref{fig:bertmask_surp} seems to align with the two-hump shape of the blue region for $\mat{}$ in Fig.~\ref{fig:bertmask_r2_scores}.

\paragraph{Analysis.}
We manually compare the predictions of our mapping $\matlL{}$ with $\idlL{}$, for a $24$-layered \bert{} model.  Concretely, we select 50 random sentences from the Leipzig dataset. Next, for each layer $\ell$, we manually analyze how many of the top-$5$ tokens according to $\matlL{}$ and $\idlL{}$ fit into context. We consider a token to fit into context if it is grammatically plausible within the sentence (see Tab.~\ref{tab:manual} for concrete examples).
In the resulting $1250$ instances (i.e. $50$ sentences $\times$ $25$ representations), we observe a substantially higher plausibility rate of $85.36\%$ for \mat{} compared to $52.8\%$ for \id{}. In fact, only in less than $4.3\%$ of the instances there are more plausible tokens among the top-$5$ tokens according to \id{} than among the top-$5$ tokens according to \mat{}, further supporting the Surprisal results above.

\begin{table*}
\footnotesize
\setlength{\belowcaptionskip}{-15pt}
\begin{tabular}{p{0.3\linewidth}ccccc}
& $\texttt{id}_{4 \rightarrow 24}$ & $\texttt{mat}_{4 \rightarrow 24}$ & $\texttt{id}_{12 \rightarrow 24}$ & $\texttt{mat}_{12 \rightarrow 24}$ & $\texttt{id}_{24 \rightarrow 24}$ \\ \midrule
\multirow{5}{=}{aldridge had shoulder surgery in \mask{}.} & fellowship & \tcbox{time} & cyclist & \tcbox{2009} & \tcbox{september} \\
& employment & \tcbox{it} & emergencies & \tcbox{2008} & \tcbox{november} \\
& agreement & her & seniors & \tcbox{2010} & \tcbox{december} \\
& \#\#ostal & them & cycling & \tcbox{2006} & \tcbox{august} \\
& \#\#com & work & \tcbox{pennsylvania} & \tcbox{2007} & \tcbox{july} \\ \midrule
\multirow{5}{=}{on your next view you will be asked to \mask{} continue reading.} & \#\#com & be & be & be & \tcbox{please} \\
& accreditation & get & undergo & \tcbox{please} & \tcbox{simply} \\ 
& $	\copyright$ & go & spartans & help & \tcbox{also} \\ 
& fellowship & \tcbox{help} & seniors & \tcbox{simply} & \tcbox{again} \\ 
& summer & have & * & say & \tcbox{immediately} \\ \bottomrule
\end{tabular}
\caption{Examples of top-$5$ predictions at layers $4$, $12$ and $24$, under the mappings $\matlL{}$ and $\idlL{}$, for a $24$-layered \bert{} model. Grammatically plausible predictions (according to a human annotator) are marked in \tcbox{blue}. Note that at layer $24$ the predictions of $\matlL{}$ and $\idlL{}$ are the same (by definition).} 
\label{tab:manual}
\end{table*}

\section{Implication to Early Exiting}
\label{sec:applications}

%The fact that it is often possible to approximate
The possibility of approximating
the final prediction already in the early layers has important implications for efficiency; applying our linear mapping instead of executing transformer blocks of quadratic time complexity, could save a substantial portion of the computation. In this section, we demonstrate this in the context of early exiting.

When 
% performing transformer model inference under 
using an early exit strategy \cite{schwartz-etal-2020-right, xin-etal-2020-deebert, schuster2022confident}, one aims at deciding dynamically at which layer to stop the computation and ``read'' the prediction from the hidden representation of that layer.
More precisely, under a confidence measure paradigm, one decides to stop the computation for a position $i$ at layer $\ell$ based on a confidence criterion, that is derived from casting the hidden representation $h_i^\ell$ as a final-layer representation and converting it to an output probability distribution. Specifically, following \citet{schuster2022confident}, a decision to exit is made if the difference between the highest and the second highest probabilities is bigger than $$ 0.9 \cdot \lambda + 0.1 \cdot {\rm exp} (-4 i / N),$$
where $N$ is the average length of the input until position $i_s$ for $s \in \mathcal{V}$, and $\lambda$ is a hyper-parameter.

\begin{figure}[t]
\setlength{\belowcaptionskip}{-10pt}
\centering
\includegraphics[width=\columnwidth]{figs/ee_gpt2bert.pdf}
\caption{Precision@$1$ with early exit and ``fixed exit'', applied to the $24$-layer \gpt{} for next token prediction (left) and the $24$-layer \bert{} for masked token prediction (right). Varying the confidence parameter $\lambda$, the $x$-coordinate is the average number of layers processed before an early exit decision is reached.}
\label{fig:ee_gpt2bert}
\end{figure}

\quash{
\begin{figure}[t]
\setlength{\belowcaptionskip}{-10pt}
\centering
\includegraphics[scale=0.35]{figs/ee_pre1_24.pdf}
\caption{Precision@$1$ for the various early exit methods, and previous ``fixed exit'' methods for comparison ($24$-layer \gpt{} next token prediction task). Varying the confidence parameter $\lambda$, the $x$-coordinate is the average number of layers processed before an early exit decision is reached.}
\label{fig:ee_pre1}
\end{figure}
}

\paragraph{Experiment.}
We assess the utility of our mapping $\matlL{}$ for early exit as a plug-and-play replacement for $\idlL{}$, through which intermediate representations are cast into final-layer representations.
We use \gpt{} for the next token prediction and \bert{} for masked token prediction (both with 24 layers).
We run each of the models over the validation set examples, while varying the confidence parameter $\lambda$ and using either $\idlL{}$ or $\matlL{}$ for casting intermediate representations.
Furthermore, we compare these early exit variants to the ``fixed exit'' strategy from \S\ref{sec:prediction}, where the computation is stopped after a pre-defined number of layers rather than relying on a dynamic decision.
We evaluate each variant in terms of both prediction's accuracy, using the Precision@$1$ metric (see \S\ref{sec:prediction}), and efficiency, measured as the average number of transformer layers processed during inference.


\paragraph{Results.}
%Figs.~\ref{fig:ee_pre1} and~\ref{fig:bertmask_ee_pre1}
Fig.~\ref{fig:ee_gpt2bert}
plots the average Precision@$1$ score against the average number of layers processed, for $24$-layer \gpt{} and $24$-layer \bert{}. For both models, under an early exit strategy our mapping \mat{} again provides a substantial improvement over \id{}.
For example, aiming at $95\%$ average precision, \mat{} saves $\sim3.3$ ($13.8$\%) layers in \gpt{} compared to only $\sim1.4$ ($5.9$\%) layers by \id{}, and $\sim4.8$ ($20$\%) layers in \bert{} versus $\sim3.5$ ($14.6$\%) layers by \id{}.
These results highlight the potential gains prominent early exit methods can obtain by using our method.
Notably, in both models and for each of the mapping methods, early exit obtains better results than fixed layer exit, as expected. 

\quash{
\begin{figure}[t]
\setlength{\belowcaptionskip}{-10pt}
\centering
\includegraphics[scale=0.35]{figs/bertmask_ee_pre1_24.pdf}
\caption{Precision@$1$ for the various early exit methods, and previous ``fixed exit'' methods for comparison ($24$-layer \bert{} masked token prediction task). Varying the confidence parameter $\lambda$, the $x$-coordinate is the average number of layers processed before an early exit decision is reached.}
\label{fig:bertmask_ee_pre1}
\end{figure}
}
\section{Linear Shortcut Across Sub-Modules}
\label{sec:submodules}

% Our experiments show that
% , despite the commonly-applied simplification by interpretability works, transformer layers do not operate in the same linear space and 
% there is a major gap in approximating future representations using an identity mapping (\S\ref{sec:layer_jump}, \S\ref{sec:prediction}).
% Here, 
In this section, we investigate whether discrepancies across layers result from specific sub-modules or are a general behaviour of all sub-modules in the network.  
This is done by extending our approach to test how well particular components in transformer blocks can be linearly approximated. 


\paragraph{Method.}

Consider \gpt{} for definiteness, then:
% we have 
$$ \texttt{b}_{\ell} = \texttt{b}_{\ell}^{\texttt{ffn}} \circ \texttt{b}_{\ell}^{\texttt{attn}}$$ 
% with
\begin{equation}\label{eq:attn} \texttt{b}^{\texttt{attn}}_{\ell} (H) = \texttt{attn}_{\ell} (\texttt{ln1}_{\ell} (H)) + H,\end{equation} 
where $\texttt{attn}_{\ell}$ is
%a multi-head self-attention
a MHSA
layer and \texttt{ln1} is a layer normalization (LN), and 
$$ \texttt{b}^{\texttt{ffn}}_{\ell} (H) = \texttt{ffn}_{\ell} (\texttt{ln2}_{\ell} (H)) + H,$$  
where $\texttt{ffn}_{\ell}$ is
%a feed-forward network
an FFN
layer and $\texttt{ln2}$ is a LN.
\quash{
Given a block $\texttt{b}_\ell$ and one of its sub-modules $\texttt{ln1}_\ell, \ \texttt{attn}_\ell, \ \texttt{ln2}_\ell$, or $\texttt{ffn}_\ell$, we fit linear regression approximating the output of the sub-module given its input and then use it in order to define mappings, as we now describe.
}
Given a block $\texttt{b}_\ell$ and one of its sub-modules $\texttt{ln1}_\ell, \ \texttt{attn}_\ell, \ \texttt{ln2}_\ell$, or $\texttt{ffn}_\ell$, we fit linear regression approximating the output of the sub-module given its input, and then use it to define mappings $\matattnl{}$, $\matlnl{}$ and $\matffl{}$.
%We provide the definition of $\matattnl{}$ below, and that of the other two in App. \ref{sec:app_submodule_skip_description}.
We provide the formal definitions of these mappings in App. \ref{sec:app_submodule_skip_description}.
\iffalse
\paragraph{$\matattnl{}$.}
%Illustrating this on $\texttt{attn}_\ell$ for definiteness,
For an input $s$, let $v^\ell_{i_s}$ be the vector at position $i_s$ in the output of $\texttt{attn}_\ell (\texttt{ln1}_\ell (H^{\ell - 1}))$. We denote by $A_\ell^{\texttt{attn}} \in \mathbb{R}^{d_h \times d_h}$ the matrix numerically minimizing 
$$ A \mapsto \sum_{s \in \mathcal{T}} || A \cdot \texttt{ln1}_\ell (h^{\ell-1}_{i_s}) - v^\ell_{i_s}||^2,$$
and define an attention sub-module replacement (Eq.~\ref{eq:attn}) by $$
\texttt{b}^{\overline{\texttt{attn}}}_\ell (h) \coloneqq A_{\ell}^{\texttt{attn}} \cdot \texttt{ln1}_\ell (h) + h. $$
We then define a mapping between two layers ${\ell \rightarrow \ell'}$ by:
$$ \matattnl{} (h) \coloneqq $$
$$ \texttt{b}^{\texttt{ffn}}_{\ell'} ( \texttt{b}^{\overline{\texttt{attn}}}_{\ell'} ( \ldots (\texttt{b}^{\texttt{ffn}}_{\ell+1} ( \texttt{b}^{\overline{\texttt{attn}}}_{\ell+1} (h)))\ldots)).$$ 
Namely, when applying each $\ell''$-th block, $\ell < \ell'' \leq \ell'$, we replace its attention sub-module $\texttt{attn}_{\ell''}$ by its linear approximation.
%In an analogous way, we consider the mappings $\matffl{}$ and $\matlnl{}$, where in the latter we perform the linear shortcut both for \texttt{ln1} and for \texttt{ln2} (see~\S\ref{sec:app_submodule_skip_description} for precise descriptions).
Importantly, unlike the original attention module, the approximation $\texttt{b}^{\overline{\texttt{attn}}}_\ell$ operates on each position independently, and therefore applying $\matattnl{}$ disables any contextualization between the layers $\ell$ and $\ell'$. Note that this is not the case for $\matffl{}$ and $\matlnl{}$, which retain the self-attention sub-modules and operate contextually.
\fi

\paragraph{Evaluation.}


We analyze the $24$-layered \gpt{}, and proceed completely analogously to \S\ref{subsec:next_token_prediction_task}, evaluating the Precision@$1$ and Surprisal metrics for the mappings $\matattnlL{}$, $\matfflL{}$ and $\matlnlL{}$.

\begin{figure}[t]
\setlength{\belowcaptionskip}{-0pt}
\centering
%\includegraphics[scale=0.2]
\includegraphics[width=\columnwidth]{figs/parts_presurp_24.pdf}
\caption{Precision@$1$ and Surprisal for the various sub-module linear mappings, and $\matlL{}$ for comparison ($24$-layer \gpt{} next token prediction task). A 95\% confidence interval surrounds the Surprisal lines.}
\label{fig:parts_presurp}
\end{figure}

\quash{
\begin{figure}[t]
\centering
\includegraphics[scale=0.4]{figs/parts_pre1_24.pdf}
\caption{Precision@$1$ for the various sub-module linear shortcut mappings, and the mapping $\matlL{}$ for comparison (\gpt{} next token prediction task).}
\label{fig:parts_pre1}
\end{figure}

\begin{figure}[t]
\centering
\includegraphics[scale=0.35]{figs/parts_surp_24.pdf}
\caption{Surprisal for the various sub-module linear shortcut mappings, and the mapping $\matlL{}$ for comparison (\gpt{} next token prediction task). A 95\% confidence interval surrounds the lines.}
\label{fig:parts_surp}
\end{figure}
}

\paragraph{Results.}
Fig.~\ref{fig:parts_presurp} shows the average Precision@$1$ and Surprisal scores per layer.
From a certain layer (\textasciitilde$7$), all sub-module mappings achieve better results than the full-block mapping $\matlL{}$. Thus, it is not just the cumulative effect of all the sub-modules in the transformer block that is amenable to linear approximation, but also individual sub-modules can be linearly approximated. 
Furthermore, the linear approximation of attention sub-modules is less harmful than that of the FFN or LN sub-modules. 
% Hypothetically, 
A possible reason is that the linear replacement of FFN or LN ``erodes'' the self-attention computation after a few layers. 
Moreover, the good performance of $\matattnlL{}$ suggests that contextualization often exhausts itself in early layers; speculatively, it is only in more delicate cases that the self-attention of late layers adds important information. Last, remark the sharp ascent of the scores for layer normalization in layers $5$-$8$, for which we do not currently see a particular reason. To conclude, we see that the possibility of linear approximation permeates
%the various
transformer components.


\section{Related Work}

Recently, there was a lot of interest in utilizing intermediate representations in transformer-based LMs, both for interpretability and for efficiency.

In the direction of interpretability, one seeks to understand the prediction construction process of the model \cite{tenney-etal-2019-bert, voita-etal-2019-bottom}.

More recent works use mechanistic interpretability and view the inference pass as a residual stream of information \cite{dar2022analyzing,geva-etal-2022-transformer}. Additionally, there are works on probing, attempting to understand what features are stored in the hidden representations \cite{adi2017finegrained, conneau-etal-2018-cram,liu-etal-2019-linguistic}. Our work is different in that it attempts to convert intermediate representations into a final-layer form, which is interpretable by design.

In the direction of efficiency, there is the thread of work on early exit, where computation is cut at a dynamically-decided earlier stage \cite{schwartz-etal-2020-right,xin-etal-2020-deebert,schuster2022confident}. Other works utilize a fixed early stage network to parallelize inference \citep{leviathan2022fast, chen2023accelerating}. However, intermediate representations are directly propagated in these works, which we show is substantially worse than our approach. Moreover, our method requires training considerably less parameters than methods such as \citet{schuster-etal-2021-consistent}, that learn a different output softmax for each intermediate layer.  

More broadly, skipping transformer layers and analyzing the linearity properties of transformer components have been discussed in prior works \cite{Zhao2021of,mickus-etal-2022-dissect,wang-etal-2022-skipbert,lamparth2023analyzing}.


\section{Conclusion and Future Work}

We present a simple and effective method for enhancing utilization of hidden representations in transformer-based LMs, that uses 
pre-fitted context-free and token-uniform linear mappings.
Through a series of experiments on different data sources, model architectures and scales, we show that our method consistently outperforms the prevalent practice of interpreting representations in the final-layer space of the model, yielding better approximations of succeeding representations and the predictions they induce, thus allowing a more faithful interpretation of the model's prediction-formation.
We demonstrate the practicality of our method for improving computation efficiency, saving a substantial amount of compute on top of prominent early exiting approaches. 
Also, by extending our method to sub-modules, 
% more specifically the attention sub-modules, 
we observe that replacing a part of the transformer inference by a non-contextual linear computation often results in a small deterioration of the prediction.
This opens new research directions for improving model efficiency,
% and parallelizability.
% including breaking the computation into several parallelizable tasks.
including breaking the computation into parallel tasks.

\section*{Limitations}

Although we see in this work that there is more linear structure to transformer inference than could be explained solely by the residual connection, we do not elucidate a reason for that. We also do not try to formulate formal criteria according to which to judge, in principle, the quality of ways of short-cutting transformer inference in-between layers. In addition, our experiments cover only English data.


%\section*{Ethics Statement}
%Scientific work published at ACL 2023 must comply with the ACL Ethics Policy.\footnote{\url{https://www.aclweb.org/portal/content/acl-code-ethics}} We encourage all authors to include an explicit ethics statement on the broader impact of the work, or other ethical considerations after the conclusion but before the references. The ethics statement will not count toward the page limit (8 pages for long, 4 pages for short papers).

\section*{Acknowledgements}

We thank Tal Schuster for constructive comments.

% Entries for the entire Anthology, followed by custom entries
\bibliography{anthology,custom}
\bibliographystyle{acl_natbib}

\appendix

\section{Descriptions of $\matattn{}$, $\matff{}$ and $\matln{}$}
\label{sec:app_submodule_skip_description}

Here we detail the definitions of the mappings $\matattnl{}$, $\matffl{}$ and $\matlnl{}$ utilized in \S\ref{sec:submodules}.

\paragraph{Description of $\matattnl{}$.}
%Illustrating this on $\texttt{attn}_\ell$ for definiteness,
For an input $s$, let $v^\ell_{i_s}$ be the vector at position $i_s$ in the output of $\texttt{attn}_\ell (\texttt{ln1}_\ell (H^{\ell - 1}))$. We denote by $A_\ell^{\texttt{attn}} \in \mathbb{R}^{d_h \times d_h}$ the matrix numerically minimizing 
$$ A \mapsto \sum_{s \in \mathcal{T}} || A \cdot \texttt{ln1}_\ell (h^{\ell-1}_{i_s}) - v^\ell_{i_s}||^2,$$
and define an attention sub-module replacement (Eq.~\ref{eq:attn}) by $$
\texttt{b}^{\overline{\texttt{attn}}}_\ell (h) \coloneqq A_{\ell}^{\texttt{attn}} \cdot \texttt{ln1}_\ell (h) + h. $$
We then define a mapping between two layers ${\ell \rightarrow \ell'}$ by:
$$ \matattnl{} (h) \coloneqq $$
$$ \texttt{b}^{\texttt{ffn}}_{\ell'} ( \texttt{b}^{\overline{\texttt{attn}}}_{\ell'} ( \ldots (\texttt{b}^{\texttt{ffn}}_{\ell+1} ( \texttt{b}^{\overline{\texttt{attn}}}_{\ell+1} (h)))\ldots)).$$ 
Namely, when applying each $\ell''$-th block, $\ell < \ell'' \leq \ell'$, we replace its attention sub-module $\texttt{attn}_{\ell''}$ by its linear approximation.
%In an analogous way, we consider the mappings $\matffl{}$ and $\matlnl{}$, where in the latter we perform the linear shortcut both for \texttt{ln1} and for \texttt{ln2} (see~\S\ref{sec:app_submodule_skip_description} for precise descriptions).
Importantly, unlike the original attention module, the approximation $\texttt{b}^{\overline{\texttt{attn}}}_\ell$ operates on each position independently, and therefore applying $\matattnl{}$ disables any contextualization between the layers $\ell$ and $\ell'$. Note that this is not the case for $\matffl{}$ and $\matlnl{}$, which retain the self-attention sub-modules and operate contextually.

\paragraph{Description of $\matffl{}$.}
Let $v^\ell_{i_s}$ be the vector at position $i_s$ in the output of $\texttt{ln2}_{\ell} (\texttt{b}_\ell^{\texttt{attn}} (H^{\ell - 1}))$, for a given input $s$. We denote by $A_\ell^{\texttt{ffn}} \in \mathbb{R}^{d_h \times d_h}$ the matrix numerically minimizing 
$$ A \mapsto \sum_{s \in \mathcal{T}} || A \cdot v^{\ell}_{i_s} - \texttt{ffn}_{\ell} (v^\ell_{i_s})||^2,$$
and define a replacement of the feed-forward sub-module $\texttt{b}_{\ell}^{\texttt{ffn}}$ by $$ \texttt{b}^{\overline{\texttt{ffn}}}_\ell (H) \coloneqq A_{\ell}^{\texttt{ffn}} \cdot \texttt{ln2}_\ell (H) + H.$$
We then define a mapping between two layers ${\ell \rightarrow \ell'}$ by:
$$ \matffl{} (H) \coloneqq $$
$$ \texttt{b}^{\overline{\texttt{ffn}}}_{\ell'} ( \texttt{b}^{\texttt{attn}}_{\ell'} ( \ldots (\texttt{b}^{\overline{\texttt{ffn}}}_{\ell+1} ( \texttt{b}^{\texttt{attn}}_{\ell+1} (H))\ldots)).$$

\paragraph{Description of $\matlnl{}$.}
Let $v^\ell_{i_s}$ be the vector at position $i_s$ in the output of $\texttt{b}^{\texttt{attn}}_{\ell} (H^{\ell - 1})$, for a given input $s$. We denote by $A_\ell^{\texttt{ln1}} \in \mathbb{R}^{d_h \times d_h}$ the matrix numerically minimizing 
$$ A \mapsto \sum_{s \in \mathcal{T}} || A \cdot h^{\ell}_{i_s} - \texttt{ln1}_{\ell} (h^\ell_{i_s})||^2$$ and we denote by $A_\ell^{\texttt{ln2}} \in \mathbb{R}^{d_h \times d_h}$ the matrix numerically minimizing $$ A \mapsto \sum_{s \in \mathcal{T}} || A \cdot v^{\ell}_{i_s} - \texttt{ln2}_{\ell} (v^\ell_{i_s})||^2.$$ We define a replacement of the block $\texttt{b}^{\texttt{attn}}_{\ell}$ by \begin{equation} \texttt{b}^{\overline{\texttt{ln1}}}_\ell (H) \coloneqq \texttt{attn}_{\ell} (A_{\ell}^{\texttt{ln1}} \cdot H) + H\end{equation} and we define a replacement of the block $\texttt{b}^{\texttt{ffn}}_{\ell}$ by \begin{equation} \texttt{b}^{\overline{\texttt{ln2}}}_\ell (H) \coloneqq \texttt{ffn}_{\ell} (A_{\ell}^{\texttt{ln2}} \cdot H) + H.\end{equation}
We then define a mapping between two layers ${\ell \rightarrow \ell'}$ by:
$$ \matlnl{} (H) \coloneqq $$
$$ \texttt{b}^{\overline{\texttt{ln2}}}_{\ell'} ( \texttt{b}^{\overline{\texttt{ln1}}}_{\ell'} ( \ldots (\texttt{b}^{\overline{\texttt{ln2}}}_{\ell+1} ( \texttt{b}^{\overline{\texttt{ln1}}}_{\ell+1} (H))\ldots)).$$


\end{document}



\end{document}


