%\documentclass[aps,prl,reprint,groupedaddress]{revtex4-1}
\documentclass[aps,prb,preprint,superscriptaddress,
raggedbottom,tightenlines]{revtex4-2}
%\documentclass[aps,prl,reprint,groupedaddress]{revtex4-1}

%\usepackage[pdftex]{graphics}
\usepackage{graphicx}
%\usepackage{float}
\usepackage{amsmath}
\usepackage{amssymb}
%\usepackage{comment}
\usepackage{xspace}
\usepackage{xcolor} 
%\usepackage{cleveref}
\usepackage{upgreek}
\usepackage{multirow}

%\usepackage{jabbrv}

\usepackage[utf8]{inputenc}

\newcommand{\beq}{\begin{equation}}
\newcommand{\eeq}{\end{equation}}

% General Helpers
\usepackage{pgffor} % To support \foreach coommand in macros
\DeclareRobustCommand{\chem}[1]{%
\ensuremath{\foreach \x/\y in {#1} {{\mathrm{\x}_{\y}}}}\xspace}

\newcommand{\usr}[2]{\ensuremath{#1_{\mathrm{#2}}}\xspace} % 

% Materials
\newcommand{\kmo}{\chem{K/0.3,Mo/,O/3}}

% Units
\newcommand{\unit}[1]{\ensuremath{\,{\mathrm{#1}}}\xspace}
\newcommand{\pcm}{\ensuremath{\,\mathrm{cm}^{-1}}\xspace}
%\newcommand{\mum}{\ensuremath{\,\upmu\textrm{m}}\xspace}
%\newcommand{\muJ}{\unit{\upmu J}}
\newcommand{\mum}{\unit{\mu m}}
\newcommand{\muJ}{\unit{\mu J}}
\newcommand{\fsec}{\unit{fs}}
\newcommand{\psec}{\unit{ps}}
\newcommand{\Hz}{\unit{Hz}}
\newcommand{\kHz}{\unit{kHz}}
\newcommand{\GHz}{\unit{GHz}}
\newcommand{\THz}{\unit{THz}}
\newcommand{\Kel}{\ensuremath{\,\mathrm{K}}}
\newcommand{\nm}{\unit{nm}}
\newcommand{\mm}{\unit{mm}}

\newcommand{\tisa}{Ti:Al$_2$O$_3$\xspace}

% Misc
\newcommand{\rarr}{\ensuremath{\rightarrow}\xspace}
\newcommand{\simx}{{\sim}}

% Quantities
\newcommand{\kF}{\usr{k}{F}\xspace}
\newcommand{\epsr}[1]{\usr{\varepsilon}{#1}\xspace}
\newcommand{\nuzn}{\ensuremath{\nu_{0n}}\xspace}
\newcommand{\nuznA}{\ensuremath{\nu_{0n}^{(A)}}\xspace}
\newcommand{\nuznP}{\ensuremath{\nu_{0n}^{(P)}}\xspace}
\newcommand{\nuzA}{\ensuremath{\nu_{01}^{(A)}}\xspace}
\newcommand{\nuzP}{\ensuremath{\nu_{01}^{(P)}}\xspace}

%\newcommand{\wznA}{\ensuremath{\omega_{0n}^{(A)}}\xspace}
%\newcommand{\wznP}{\ensuremath{\omega_{0n}^{(P)}}\xspace}
%\newcommand{\wzA}{\ensuremath{\omega_{01}^{(A)}}\xspace}
\newcommand{\wzP}{\ensuremath{\omega_{01}^{(P)}}\xspace}


\newcommand{\wzn}{\ensuremath{\omega_{0n}}\xspace}
\newcommand{\Wzn}{\ensuremath{\Omega_{0n}}\xspace} % Bare mode frequencies
%\newcommand{\Wz}[1]{\ensuremath{\Omega_{0#1}}\xspace} % Bare mode frequencies
\newcommand{\Gamn}{\ensuremath{\Gamma_{n}}\xspace}
\newcommand{\phin}{\ensuremath{\varphi_{0n}}\xspace}
\newcommand{\Wpin}{\usr{\Omega}{p}}
\newcommand{\Upin}{\usr{U}{p}}
\newcommand{\Upinx}{\usr{U}{p}^{(1)}}


% Commments
% \newcommand{\HGR}[1]{\textcolor{orange}{[#1]}}
% \newcommand{\HR}[1]{\textcolor{red}{#1}}
\newcommand{\KW}[1]{\textcolor{blue}{[KW: #1]}}
\newcommand{\KWx}[1]{\textcolor{blue}{#1}}
\newcommand{\MDT}[1]{\textcolor{purple}{[MDT: #1]}}
%\newcommand{\toKW}[1]{\textcolor{\red}{[$\rightarrow$ KW: #1]}}

% Affiliations
\newcommand{\ffm}{Physikalisches Institut, J. W. Goethe-Universität, D-60438 Frankfurt am Main, Germany} 
\newcommand{\jgu}{Institute of Physics, Johannes Gutenberg-University Mainz, D-55128 Mainz, Germany}

% For Supplmentary
% Ensure everthing above can be updated by over-pasting from main paper
%\newcommand{\chi_n}{\ensuremath{\nu_{01}^{(A)}}\xspace}
\newcommand{\Tc}{\usr{T}{c}}
\newcommand{\Tcz}{\usr{T}{c0}} % Paste back in later
\bibliographystyle{apsrev4-2}

\usepackage{tabularx}
% Tabular with fixed width and specified horizontal justification
\newcolumntype{C}[1]{>{\centering}p{#1}}
% \newcolumntype{C1}[1]{>{\centering\arraybackslash}p{1mm}}
% >{\centering\arraybackslash}p{1cm}
\newcolumntype{L}[1]{>{\raggedright}p{#1}}
\newcolumntype{R}[1]{>{\raggedleft}p{#1}}

\begin{document}

\title{Supplementary Material to: ``Combined investigation of collective amplitude and phase modes in a quasi-one-dimensional charge-density-wave system over a wide spectral range''}

\author{Konstantin Warawa}
\affiliation{\ffm}
\author{Nicolas Christophel}
\affiliation{\ffm}
\author{Sergei Sobolev}
\affiliation{\jgu}
\author{Jure Demsar}
\affiliation{\jgu}
\author{Hartmut G. Roskos}
\affiliation{\ffm}
\author{Mark D. Thomson}
\affiliation{\ffm}
%\date{January 2023}

\maketitle

\newpage

\section{Amplitude modes: Impulsive-Raman all-optical pump-probe spectroscopy}
\label{sec:opopt}

\subsection{Time-domain signals and fitted incoherent signal components}

\begin{figure}[!h]
	\includegraphics[width=0.5\textwidth]{Figures/AM_timedomain1.pdf}
\caption{
Time-domain differential reflectivity from all-optical pump-probe (impulsive Raman) experiments for selected temperatures, including fits to incoherent (predominantly electronic) signal components, and oscillatory residual, the latter used to analyze the amplitude-mode (AM) spectra in main paper.
} \label{fig:opopt}
\end{figure}
\begin{figure}[!h]
	\includegraphics[width=0.65\textwidth]{Figures/elec_pars_vs_T1.pdf}
\caption{
(a) Fitted exponential decay time constants and (b) corresponding amplitudes for incoherent (predominantly electronic) signal components vs. temperature (see Fig.~\ref{fig:opopt} for example kinetics).  Also included are the data for the shortest decay time $\tau_1$ from our previous report \cite{schaefer10}.  Common color coding used in both figures (e.g. the component with $\tau_1$ in (a) has the largest magnitude for the amplitude $|A_1|$ in (b)).
} \label{fig:opopt2}
\end{figure}

\subsection{Fitted lowest amplitude mode vs. literature reports}

\begin{figure}[!h]
	\includegraphics[width=0.7\textwidth]{Figures/nu1_literature_tdgl_comparison2.pdf}
\caption{Comparison of the fitted temperature dependence of the first AM frequency $\nuzA$ (``$\nu_{01}$'') and full-width half maximum (FWHM) bandwidth $\Delta\nu_1$ obtained from different experimental studies, including neutron diffraction ($\blacktriangledown$) \cite{pouget91}, Raman scattering ($\triangledown$, $\diamond$) \cite{travaglini_raman83, sagar_raman08} and all-optical pump-probe ($\blacksquare$) \cite{schaefer10}. The original analysis results from \cite{schaefer10} including a TDGL fit (left) are compared with those from the current paper (right), with the latter showing stronger mode softening and broadening approaching $T_c$ and better agreement with neutron and Raman data.
} \label{fig:schaefercomp}
\end{figure}

\newpage

\section{Phase modes: Reflective THz time-domain spectroscopy}\label{sec:thz}

\subsection{Fitted conductivity spectra}

As described in detail in Sec.~IV and Appendix B of the main paper, we fitted the complex reflectivity spectra $r(\nu) = \sqrt{R(\nu)}\cdot e^{i\phi(\nu)}$ using a set of Lorentzian bands for the conductivity $\sigma(\nu)$
(which, due to the sensitivity to the precise reflectivity baseline, could not be inverted directly from $r(\nu)$ via the Fresnel reflection formula).
For completeness, we show these fitted conductivity spectra in Fig.~\ref{fig:fitsig}, where one can more readily visualize the relative band strengths and broadening with increasing $T$. 

\begin{figure}[htbp]
	% \includegraphics[width=1\textwidth]{Figures/fit_sig_spectra1.pdf}
    \includegraphics[width=1\textwidth]{Figures/fit_sig_spectra1_log.pdf}
\caption{Fitted complex conductivity spectra ($\sigma(\nu)=\sigma_1(\nu) + i\cdot\sigma_2(\nu)$) from THz reflectivity spectral analysis for each temperature $T$ (Fig. 3 in main paper). For clarity the real part $\sigma_1$ is plotted with a logarithmic vertical scale.
} \label{fig:fitsig}
\end{figure}

\subsection{Fitting of lowest phase modes}

As discussed in the main paper, the measured strong reflectivity signature at $\nu\simx 1.75\THz$ indicates the presence of a phase mode  (PM) at lower frequency $\nu_{01}$, although we cannot identify its position from our reflectivity spectra.
This is demonstrated in Fig.~\ref{fig:fixnu1spec}, where we show fitted spectra for a set of fixed values for $\nu_{01}$, with the corresponding misfit (calculated here for the subrange 0.8-2~THz) in Fig.~\ref{fig:fixnu1misfit}.
One can see the misfit is essentially flat for $\nu_{01}\lesssim 1.2\THz$, with reasonable fits to the spectra (except in the case where such a mode is omitted).
For this reason, in the main paper, we assume a PM at $\nu_{01}=0.1\THz$, corresponding to the ``pinned phason frequency'' found in previous experimental studies \cite{mihaly1989,degiorgi91}.

\begin{figure}[!t]
	\includegraphics[width=1\textwidth]{Figures/fixed_nu1_spectra1b.pdf}
\caption{Reflection spectral intensities ($R(\nu$)) and phases ($\varphi(\nu)$) obtained from THz-TDS measurements at $T=20\Kel$ (blue points), and fitted spectra for selected values of a fixed lowest-mode frequencies $\nu_{01}$ (red curves, as per the labels in the left column). Magenta vertical lines denote position of fitted modes $\nuzn$.
The first plot with $\nu_{01} = 0.1\THz$ is equivalent to the ``$T=20\Kel$''-plot in Fig. 3 from the main paper. A fit with no mode used below $1.75\THz$ is also shown (``$\nu_{01}$ omitted''), which cannot reproduce the strong reflectivity dip at that frequency.
} \label{fig:fixnu1spec}
\end{figure}

\begin{figure}[!t]
	\includegraphics[width=0.5\textwidth]{Figures/fixed_nu1_misfits_close1b.pdf}
\caption{Calculated root mean square (rms) misfit between experimental complex reflectivities ($r(\nu) = \sqrt{R(\nu)}\cdot e^{i\varphi(\nu)}$) and fit curves (examples shown in Fig.~\ref{fig:fixnu1spec}) as a function of fixed lowest-mode frequency $\nu_{01}$ (misfit here calculated for the subrange $0.8$-$2\THz$ to concentrate on how this specific spectral range is affected). Here the case of ``omitted $\nu_{01}$'' corresponds to a lowest-mode frequency of $2.2\THz$ (corresponding to $\nu_{02}$ in the other fits).
} \label{fig:fixnu1misfit}
\end{figure}

\newpage
\subsection{Extended-bandwidth TDS measurements with ABCD detection at $T=20\Kel$}

The majority of THz-TDS experiments here used EOS detection, providing a detection  bandwidth limited to $\simx 7\THz$ (see Fig.s~2 and 3 in main paper), due to the onset of phonon absorption in the GaP crystal.
To extend the bandwidth and detect higher-lying modes, we also employed the ABCD detection method \cite{karpowicz08}, which does not suffer from such phonon absorption, and allows here detection up to $\simx 10\THz$ (see Fig.~\ref{fig:abcdtds}). 
However, we found that the signal-to-noise ratio and sample alignment sensitivity were superior for EOS detection, and employed ABCD only for $T=20\Kel$ to resolve PMs in the range 7-$9\THz$.

\begin{figure}[!h]
	\includegraphics[width=1\textwidth]{Figures/TDS_pulses_abcd2.pdf}
\caption{(a) Example of detected THz time-domain fields with reflection sample geometry utilizing ABCD: reference (using a mirror in the sample position, red curve), and \kmo sample at $T=20\Kel$ (blue curve). Inset shows magnified range of oscillatory signatures after the main pulse for the \kmo data (while the weak residual oscillations of the mirror reference are due to residual water-vapor absorption in the THz beam path). (b) Corresponding intensity spectra.
} \label{fig:abcdtds}
\end{figure}

%\newpage

\section{Time-dependent Ginzburg-Landau model}

\subsection{Theoretical description}\label{sec:tdgl_theory}

The theoretical basis of the TDGL model is given in \cite{schaefer10,schaefer14,thomson17} (and their respective Supplementaries), so we summarize the aspects here only briefly to clarify the particular details/notation for the fitted model results in the present paper.

Writing the complex electronic order parameter (EOP) as 
$\tilde\Delta = \Delta e^{i\varphi} = \Delta_1 + i\Delta_2$ 
and complex bare-phonon coordinates
$\tilde\xi_n = \xi_n e^{i\chi_n} = \xi_{n1} + i\xi_{n2}$ ($n=1\dots N$) (where all coordinates refer to the complex wave amplitudes of the $q=2\kF$ components), we consider the potential function:
\beq
U(\tilde\Delta,\tilde\xi_1,\dots,\tilde\xi_N)=
U_{\Delta} + U_{\xi n} + \usr{U}{c} + \usr{U}{p},
\label{eq:tdglU}
\eeq
where 
$U_{\Delta}=-\tfrac{1}{2}\alpha(\Tcz-T)\Delta^2 + \tfrac{1}{4}\beta \Delta^4$ 
is the Mexican hat potential, 
$U_{\xi n}=\tfrac{1}{2}\Wzn^2 \xi_n^2$ 
represents the elastic energy stored in the bare phonon mode $n$ with frequency $\Wzn$, 
$\usr{U}{c}=-m_n(\Delta_1\xi_{n1}+\Delta_2\xi_{n2}) = -m_n\Delta\cdot\xi_{n}\cos(\varphi-\chi_n)$ is the linear coupling term (summations over $n$ are left implicit), and
$\Upin=-\Wpin^2\Delta^2\cos{\varphi}$ 
the impurity pinning energy \cite{tucker1989,wonneberger1999} (discussed below).

This has the equilibrium solution (neglecting a small correction for the pinning potential)
$$\Delta_0^2=\frac{\alpha(\Tc-T)}{\beta}, \qquad 
\xi_{0n}=\frac{m_n}{\Wzn^2}\Delta_0, \qquad 
\Tc=\Tcz+\frac{m_n^2}{\alpha\Wzn^2},
$$
where $\Tc$ is the renormalized critical temperature, and 
$\varphi_0=\chi_{n0}=0$  due to the impurity pinning minimum.

Calculating the Hessian matrix of the potential about $\tilde\Delta_0=\Delta_0$ yields the linearized equations of motion \cite{schaefer10,schaefer14,thomson17}:
\begin{subequations}
\begin{align}
\partial_t{\hat\Delta_1}&=
-\kappa_1 \left[ 2\alpha(\Tc-T) + \frac{m_n^2}{\Wzn^2} +2\Wpin^2 \right]\hat\Delta_1  +\kappa_1 m_n\hat\xi_{n1} \label{eq:motionDA}
%\partial_t^2{\hat\Delta_1}&=-\left[ 2\alpha(\Tc-T) + \frac{m_n^2}{\Wzn^2} \right]\hat\Delta_1 +m_n\hat\xi_{n1} - \gamma_1\partial_t{\hat\Delta_1} \label{eq:motionDA}
 \\
 %
\partial_t^2{\hat\Delta_2}&=-\left(\frac{m_n^2}{\Wzn^2}+\Wpin^2\right)\hat\Delta_2
 +m_n\hat\xi_{n2} - \gamma_2\partial_t{\hat\Delta_2} \label{eq:motionDP}
 \\  %
\partial_t^2{\hat\xi_{n1}}&=m_n\hat\Delta_{1} - \Wzn^2 \hat\xi_{n1} 
- \gamma_{\xi n}\partial_t \hat{\xi}_{n2}\\
%
\partial_t^2{\hat\xi_{n2}}&=m_n\hat\Delta_{2} - \Wzn^2 \hat\xi_{n2}
- \gamma_{\xi n}\partial_t \hat{\xi}_{n2}
\end{align} \label{eq:motion}%
\end{subequations}
where $\hat\Delta_1\approx \Delta-\Delta_0$ and $\hat\Delta_2\approx\Delta_0 \varphi$ represent the amplitude and phase deviations from equilibrium (likewise for $\hat \xi_{n1}$,$\hat \xi_{n2}$), and we have added phenomenological damping constants $\gamma_{1,2}$ for the EOP $\hat\Delta_{1,2}$ (and allow for $\gamma_1 \neq \gamma_2$), and $\gamma_{\xi n}$ for the $n$th bare phonon.
Note that in Eq.~\eqref{eq:motionDA} we have taken the overdamped limit 
$\partial_t^2\hat\Delta_1 \ll \gamma \partial_t\hat\Delta_1$ 
for the EOP-amplitude (as per \cite{schaefer10,schaefer14,thomson17})
while we retain the general damping case for the EOP-phase in Eq.~\eqref{eq:motionDP}, as appropriate in the present paper where we consider $\gamma_2\ll \gamma_1$.
Also, while tests were performed with different models for the bare-phonon damping $\gamma_{\xi n}> 0$, we found that these do not significantly assist fitting of the experimental modes and set $\gamma_{\xi n}\equiv 0$ for the analysis shown in the main paper.

The collective modes are found by substituting the ansatz $\propto e^{\lambda t}$ for all coordinates in Eq.s~\eqref{eq:motion}, yielding the eigenvalues $\lambda_n=-\Gamma_n/2+i\wzn$.  
%
For the amplitude channel, with eigenvector components in $(\hat\Delta_1,\hat\xi_{n1})$, one has $N$ modes with $\wzn>0$ and one overdamped mode ($\wzn=0$).
For the phase channel, with eigenvector components in $(\hat\Delta_2,\hat\xi_{n2})$, one has (i) a ``phason'' with $\omega_{01}$ close to zero 
(at finite frequency due to pinning when the damping $\gamma_2$ is sufficiently weak, as is the case here), (ii) $N-1$ modes close to their respective bare-phonon frequencies ($\Wzn$, $n \geq 2$) and (iii) one high-frequency mode (well above all $\Wzn$) with a much higher damping.
(All modes with $\wzn\neq 0$ also possessing a complex-conjugate eigenvalue with $\wzn<0$).
These results are depicted in Fig.~\ref{fig:tdgl_phase} for the TDGL parameters employed in the main paper to model the experimental data. %
%

An inspection of the eigenvector for the high-frequency PM above 20~THz shows that it involves almost purely the EOP phase ($\hat\Delta_2$), and only manifests here by not taking the overdamped limit for the phase channel in Eq.~\eqref{eq:motion}.
Given its energy, one might be tempted to connect it with the single-particle gap (also found close to this frequency \cite{degiorgi94}), although we stress here that the TDGL does not explicitly contain this gap energy, and so one cannot make this association. 
Indeed, the condensation energy in Eq.~\eqref{eq:tdglU} can be shown to be 
$U_C=\tfrac{1}{4}\alpha(T-T_c)\Delta_0^2$, where the arbitrary nominal scaling of $\Delta$ reflects this lack of energy-gap calibration. 
Given that this predicted mode has a frequency close to the single-particle gap, it may well be more strongly damped than in the TDGL prediction.  Also, its current position is based on a TDGL model accounting for bare modes only up to $\simx 9\THz$, and so such a mode may rather appear at even higher frequencies if additional bare modes were included.
Interestingly, a close inspection of this spectral region in previous 
mid-infrared reflectivity studies \cite{degiorgi91,degiorgi94,beyer2012} indicates that such a feature could be possibly present, but obscured due to overlap with other spectral features.

\begin{figure}[!h]
	\includegraphics[width=1\textwidth]{Figures/tdgl_exp_pm1.pdf}
\caption{PM parameters vs. $T$ from TDGL model of experimental modes: (a) Mode frequencies and (b) FWHM bandwidths.} \label{fig:tdgl_phase}
\end{figure}

The ansatz for our pinning potential $\Upin$ here warrants further discussion.  An intuitive formulation would be 
$\Upinx=-\usr{V}{i}\cdot\Delta\cdot\cos{\varphi}=\usr{V}{i}\cdot\Delta_1$ \cite{tucker1989}, with the factor $\Delta$ reflecting the CDW amplitude (and hence the magnitude of net charge interacting with the impurity) and $\cos{\varphi}$ describing how this interaction varies between attractive/repulsive as the CDW translates over the impurity.  Clearly, for the form $\Upinx$ (with $\usr{V}{i}$ constant), the Hessian terms $\partial^2 \Upinx/\partial \Delta_{1,2}^2$ vanish, and hence also the force terms about the equilibrium $\Delta_0$ in Eq.s~\eqref{eq:motion} (the pinning only causing a small positive shift in $\Delta_0$).
A more careful treatment of the relation between the phase $\varphi$ and Cartesian coordinate $\hat \Delta_2$ about $(\Delta_1,\Delta_2)=(\Delta_0,0)$ yields 
$\Upinx\approx -\usr{V}{i}\cdot\Delta_0(1-\tfrac{1}{2}\varphi^2)
\approx -\usr{V}{i}(\Delta_0-\tfrac{1}{2}\Delta_2^2/\Delta_0)$.
This indeed yields a finite value of
$\partial^2 \Upinx/\partial \Delta_2^2=-\usr{V}{i}/\Delta_0$.  However, this term diverges as $T\rarr T_c$ ($\Delta_0\rarr 0$), which for \kmo seems untenable as the experimental phason frequency is found to remain close to 
$\nuzP=0.1\THz$ for all $T$ \cite{degiorgi91} (although divergent pinning fields for $T\rarr T_c$ were indeed observed for other quasi-1D CDW system, e.g. \chem{Nb/,Se/3} \cite{mccarten1992}).

We instead propose a pinning potential of the form 
$\Upin=-\Wpin^2\Delta^2\cos{\varphi}=-\Wpin^2\Delta\cdot\Delta_1$.
This yields 
\beq
\frac{\partial^2\Upin}{\partial \Delta_1^2}=-\Wpin^2
\frac{\Delta_1(2\Delta_1^2+3\Delta_2^2)}{\Delta^2}
\xrightarrow[\Delta_0]{} - 2\Wpin^2, \qquad 
\frac{\partial^2\Upin}{\partial \Delta_2^2}=-\Wpin^2\frac{\Delta_1^3}{2\Delta_1^2+3\Delta_2^2}
\xrightarrow[\Delta_0]{} - \Wpin^2,
\eeq
and hence a $T$-independent pinning force along both $\Delta_{1,2}$.  Note that for our TDGL parameters, the pinning term only has a significant effect on the lowest PM (phason).

\newpage
\subsection{Fitted TDGL parameters}

%In Table~\ref{tab:tdglpars} we list the TDGL parameters used for fitting the experimental AM and PM.

\begin{table}[htbp]
% \def \colwidA{0.1\columnwidth}
% \def \colwidB{0.06\columnwidth}
    % \centering
    \begin{center}
        % \begin{tabular}{|c|c|c|c|c|c|c|c|}
        \begin{tabular}{|C{0.06\textwidth}|C{0.15\textwidth}|C{0.15\textwidth}|C{0.08\textwidth}|C{0.12\textwidth}|C{0.08\textwidth}|C{0.08\textwidth}|c|}
            \hline
            $n$ & $\Wzn$ & $\kappa_1 m_n^2$ & $b_n$ & $\alpha$ & $\gamma_1/2\pi$ & $\gamma_2/\gamma_1$ & $\Wpin$ \\
             &  (THz) & (THz$^3$) & & (ps$^{-2}$~K$^{-1}$) & (THz) & & (THz)\\
            \hline
            1 & 1.82 (1.79) & 1170 (580) & 0.30 (0) & \multirow{23}{*}{\shortstack{72.5 \\ (46)}} & \multirow{23}{*}{\shortstack{52.5 \\ (46.8)}} & \multirow{23}{*}{\shortstack{0.09 \\ (1)}} & \multirow{23}{*}{\shortstack{0.7 \\ (9.3)}}\\
            2 & 2.24 (2.25) & 240 (320) & 0 (0) & & & & \\
            3 & 2.55 (2.64) & 60 (1150) & 0 (0) & & & & \\
            4 & 2.59 & 250 & \multirow{10}{*}{0} & & & & \\
            5 & 2.69 & 160 &  & & & & \\
            6 & 2.79 & 610 &  & & & & \\
            7 & 3.20 & 410 &  & & & & \\
            8 & 3.46 & 360 &  & & & & \\
            9 & 3.61 & 340 &  & & & & \\
            10 & 3.77 & 800 &  & & & & \\
            11 & 4.01 & 1200 &  & & & & \\
            12 & 4.08 & 1700 &  & & & & \\
            13 & 4.69 & 3300 &  & & & & \\
            14 & 5.31 & 1880 & 0.35 & & & & \\
            15 & 5.59 & 2280 & 0.30 & & & & \\
            16 & 5.76 & 3750 & 0.30 & & & & \\
            17 & 5.94 & 5250 & \multirow{7}{*}{0} & & & & \\
            18 & 6.30 & 4050 &  & & & & \\
            19 & 6.82 & 6200 &  & & & & \\
            20 & 7.54 & 7350 &  & & & & \\
            21 & 8.32 & 8500 &  & & & & \\
            22 & 8.44 & 12000 &  & & & & \\
            23 & 8.67 & 15700 &  & & & & \\
         \hline
    \end{tabular}
    \end{center}

    \caption{Overview of TDGL model parameters used for obtaining the $T$-dependent fit curves in Fig. 5 in the main paper, compared to the values used in our previous work (in brackets) \cite{thomson17} (with $\kappa_1 = \gamma_1^{-1}$ and other parameters as defined above in Sec.~\ref{sec:tdgl_theory} and in the main paper). Additionally, the $T$-dependent parameters of both EOP-phase damping and impurity pinning potential (defined in \cite{thomson17}) were both set to zero in this work, as these did not improve the quality of fitting PMs vs $T$.}
    \label{tab:tdglpars}
\end{table}

%\bibliography{references}
%\bibliography{CPA_Lab}
\documentclass[aps, prx, twocolumn, superscriptaddress, longbibliography]{revtex4-1}
\usepackage[colorlinks, linkcolor=blue, anchorcolor=blue, citecolor=blue]{hyperref}
\usepackage{amsmath}
\usepackage{graphicx}
\usepackage{braket}
\usepackage{multirow}
\usepackage{color}
\usepackage{soul}
%\usepackage[smalltableaux,centertableaux,boxsize=5pt]{ytableau}

%\usepackage{ulem}%only for the command \sout = scrap

\renewcommand{\thefigure}{S\arabic{figure}}
\renewcommand{\theequation}{S\arabic{equation}}
\renewcommand{\thetable}{S\arabic{table}}
\renewcommand{\thesection}{S-\Roman{section}}
\newcommand{\aw}[1]{{\color[rgb]{.8,.4,.2}{#1}}}
\newcommand{\awc}[1]{{\color[rgb]{.8,.6,.6}{[AW: {\it #1}\,]}}}
\newcommand{\awx}[1]{{\color[rgb]{.8,.6,.6}{\sout{#1}}}}%
\newcommand{\ylw}[1]{\textcolor{red}{#1}}

\begin{document}

\title{Supporting Information for ``Electronic Correlation Effects on Stabilizing a Perfect Kagome Lattice and Ferromagnetic Fluctuation in LaRu$_3$Si$_2$''}
\author{Yilin Wang}     
\affiliation{Hefei National Laboratory for Physical Sciences at Microscale, University of Science and Technology of China, Hefei, Anhui 230026, China} 

\date{\today}

\begin{abstract}
\end{abstract}

\maketitle

%\section{Computational Details}

\begin{table*}
    \centering
    \caption{Values of Hubbard $U$ and Hund's coupling $J_H$ calculated by the code \emph{R\underline{ }Coulomb.py} in DFT+EDMFTF package. These values are used for both DFT+U and LDA+DMFT calculations.}
    \begin{ruledtabular}
    \begin{tabular}{cccccccccc}
    $U$ (eV)   & 1.1   & 1.5   & 2.0   & 3.0   & 4.0   & 4.5  & 5.0   & 5.5   & 6.0\\
    $J_H$ (eV) & 0.389 & 0.476 & 0.563 & 0.692 & 0.782 &0.817 & 0.848 & 0.874 & 0.897\\
    \end{tabular}
    \end{ruledtabular}
    \label{tab:multi}
    \end{table*}

\begin{figure*}
        \centering
        \includegraphics[width=0.9\textwidth]{x_ggasoc.pdf}
        \caption{Fractional coordinates $x$ of Ru sites as function of Hubbard $U$, relaxed by GGA+U with spin-orbital coupling.}
        \label{fig:ggasoc}
\end{figure*}

\begin{figure*}
    \centering
    \includegraphics[width=0.9\textwidth]{magnetic_conf.pdf}
    \caption{Magnetic configurations considered in the GGA+U calculations.}
    \label{fig:mag_conf}
\end{figure*}



\begin{figure*}
    \centering
    \includegraphics[width=0.9\textwidth]{XRD.pdf}
    \caption{Simulated XRD pattern of the possible distorted Kagome structure of LaRu$_3$Si$_2$. (a) For space group P6$_3$/mcm with Ru at (0.52, 0, 0.25). (b) For space group P6$_3$/m with Ru at (0.52, 0.01, 0.25). Lattice parameters are $a=5.676$\AA\ and $c=7.12$\AA. Their only difference is that there is an additional weak peak at (1 0 1) for P6$_3$/m.  }
    \label{fig:xrd}
\end{figure*}
    



%\pagebreak
\bibliography{suppl}

\end{document}


\end{document}


