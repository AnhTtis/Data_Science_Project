\documentclass[12pt,a4paper]{amsart}
\usepackage{fullpage}%{a4wide}
%%%%%%%%%%%%%
%\usepackage[latin1]{inputenc}
%\usepackage[english]{babel}
%\usepackage{tikz}
%\usepackage[utf8]{inputenc}
\usepackage[all]{xy}
\usepackage{amsmath, amsthm, amssymb, amsfonts}
%\usepackage{tikz-cd}

%%%%%
\usepackage[usenames]{xcolor}
 \usepackage{graphicx,scalerel}
%\usepackage[dvipdfmx]{graphicx}
%\usepackage{hyperref,cleveref}
\usepackage[colorlinks,
    linkcolor={red!50!black},
    citecolor={blue!50!black},
    urlcolor={blue!80!black}]{hyperref}
\usepackage[nameinlink,capitalise]{cleveref}
%\usepackage{url}
%\usepackage{ mathrsfs }
%Macros
\newcommand{\SU}{\operatorname{SU}}
\newcommand{\SL}{\operatorname{SL}}
\newcommand{\slf}{\operatorname{\mathfrak{sl}}}
\newcommand{\GL}{\operatorname{GL}}
\newcommand{\C}{\mathbb{C}}
\newcommand{\Z}{\mathbb{Z}}
\newcommand{\R}{\mathbb{R}}
\newcommand{\F}{\mathbb F}
\newcommand{\Q}{\mathbb Q}
\newcommand{\Id}{\mathrm{Id}}

\newcommand{\bma}{\begin{pmatrix}}
\newcommand{\ema}{\end{pmatrix}}
\newcommand{\bsm}{\left(\begin{smallmatrix}}
\newcommand{\esm}{\end{smallmatrix}\right)}
\newcommand{\Tr}{\operatorname{Tr}}
\newcommand{\Ad}{\operatorname{Ad}}
\newcommand{\Adr}{\Ad\circ\rho}
\newcommand{\Sym}{\operatorname{Sym}}
\newcommand{\rank}{\operatorname{rank}}
\newcommand{\tor}{\operatorname{tor}}
\newcommand{\Tor}{\operatorname{Tor}}
\newcommand{\Si}{{\Sigma}}
\newcommand{\ord}{\operatorname{ord}}
\newcommand{\Frac}{\operatorname{Frac}}
%Titre & divers

%\setcounter{section}{-1}
%\addbibresource{biblio.bib}
\newtheorem{theorem}{Theorem}[section]
\newtheorem{conj}{Conjecture}
\newtheorem{corollary}[theorem]{Corollary}
\newtheorem{lemma}[theorem]{Lemma}
\newtheorem{proposition}[theorem]{Proposition}
\newtheorem*{thmintro}{Theorem}
\newtheorem{question}[theorem]{Question}
\newtheorem*{claim}{Claim}
\newtheorem{notation}[theorem]{Notation}

\theoremstyle{definition}
\newtheorem{definition}[theorem]{Definition}
\newtheorem{terminology}{Terminology}
\newtheorem{example}[theorem]{Example}

\newtheorem{remark}[theorem]{Remark}
\newtheorem{construction}[theorem]{Construction}



\usepackage[pagewise, mathlines]{lineno} %\linenumbers 
\usepackage{time, enumerate, bm}

%\numberwithin{equation}{section}
\usepackage{etoolbox}          %% <- for \cspreto, \csappto and \patchcmd

%% Patch 'normal' math environments:
\newcommand*\linenomathpatch[1]{%
  \cspreto{#1}{\linenomath}%
  \cspreto{#1*}{\linenomath}%
  \csappto{end#1}{\endlinenomath}%
  \csappto{end#1*}{\endlinenomath}%
}
%% Patch AMS math environments:
\newcommand*\linenomathpatchAMS[1]{%
  \cspreto{#1}{\linenomathAMS}%
  \cspreto{#1*}{\linenomathAMS}%
  \csappto{end#1}{\endlinenomath}%
  \csappto{end#1*}{\endlinenomath}%
}

%% Definition of \linenomathAMS depends on whether the mathlines option is provided
\expandafter\ifx\linenomath\linenomathWithnumbers 
 \let\linenomathAMS\linenomathWithnumbers
 %% The following line gets rid of an extra line numbers at the bottom:
 \patchcmd\linenomathAMS{\advance\postdisplaypenalty\linenopenalty}{}{}{}
\else
  \let\linenomathAMS\linenomathNonumbers
\fi

\linenomathpatch{equation}
\linenomathpatchAMS{gather}
\linenomathpatchAMS{multline}
\linenomathpatchAMS{align}
\linenomathpatchAMS{alignat}
\linenomathpatchAMS{flalign}

% Disable line numbering during measurement step of multline
\makeatletter
\patchcmd{\mmeasure@}{\measuring@true}{
  \measuring@true
  \ifnum-\linenopenaltypar>\interdisplaylinepenalty
    \advance\interdisplaylinepenalty-\linenopenalty
  \fi
  }{}{}
\makeatother

\newcommand{\ol}{\overline}
\newcommand{\ul}{\underline}
\newcommand{\tc}[1]{\textcircled{\scriptsize {#1}}}
\newcommand{\ds}{\displaystyle}

\newcommand{\wt}[1]{{\widetilde{#1}}}
\newcommand{\wh}[1]{{\widehat{#1}}}

\newcommand{\mf}[1]{{\mathfrak{#1}}}
\newcommand{\mb}[1]{{\mathbf{#1}}}
\newcommand{\bb}[1]{{\mathbb{#1}}}
\newcommand{\mca}[1]{{\mathcal{#1}}}
\newcommand{\bs}[1]{{\boldsymbol{#1}}}

\newcommand{\To}{\longrightarrow}
\newcommand{\LR}{\longleftrightarrow}
\newcommand{\inj}{\hookrightarrow}
\newcommand{\surj}{\twoheadrightarrow}
\newcommand{\act}{\curvearrowright}
\newcommand{\congto}{\overset{\cong}{\to}}
\newcommand{\congTo}{\overset{\cong}{\To}}
\newcommand{\imp}{\Longrightarrow}

\newcommand{\iffu}{\underset{\rm iff}{\iff}}

%\usepackage[usenames]{color}
\newcommand{\red}[1]{\textcolor{red}{#1}} 
\newcommand{\blue}[1]{\textcolor{blue}{#1}} 
\newcommand{\green}[1]{\textcolor{green}{#1}} 
\newcommand{\violet}[1]{\textcolor{Violet}{#1}} 
\newcommand{\bred}[1]{\textcolor{brickred}{#1}} 
\newcommand{\magenta}[1]{\textcolor{magenta}{#1}} 
\newcommand{\cyan}[1]{\textcolor{cyan}{#1}}

%\newcommand{\red}{} \newcommand{\blue}{} \newcommand{\green}{} \newcommand{\violet}{} \newcommand{\bred}{}


\makeatletter
\@namedef{subjclassname@2020}{
\textup{2020} Mathematics Subject Classification}
\makeatother 

\subjclass[2020]{Primary 
57K10, 11R23; 
Secondary 57K31, 11S05
%, 11T99 
} %; Secondary 57M27} 

%57K10 knot theory 
%11R23 Iwasawa theory
%11R29 Class numbers 
%11S05 Polynomials (ANT local/p-adic fields) 
%11S99 local/p-adic fields その他
%57M10 Covering spaces and LDT 
%11G20 Curves over finite and local fields 

\keywords{knot, link, 3-manifold, non-acyclic representation, character variety, Reidemeister torsion, universal deformation, $L$-function, arithmetic topology.
%Weber's class number problem, number field, knot, elliptic curve, 
%$p$-adic torsion, Iwasawa $\nu$-invariant, Iwasawa theory, arithmetic topology. %\\[1mm] %
%\hspace{5mm}
%\framebox{%last updated:
%\today \ \now}
} 



\title[Non-acyclic $\SL_2$-representations of twisted Whitehead links]
{Multiplicity of non-acyclic $\SL_2$-representations and $L$-functions of 
the twisted Whitehead links}

%Divisor and multiplicity of the Reidemeister torsion and $L$-functions of 
%the twisted Whitehead links} }
%of the Whitehead link and surgeries.}

\author{B\'{e}nard L\'{e}o} 
\email{leo.benard@mathematik.uni-goettingen.de}
\address{Mathematisches Institut, Georg--August Universit\"at G\"ottingen, Bunsenstra$\ss$e\\ 
3-5, 37073 G\"ottingen, Germany} 

\author{Ryoto Tange} 
\email{rtange.math@gmail.com}
\address{Department of Mathematics, School of Education, Waseda University\\ 
1-104, Totsuka-cho, Shinjuku-ku, 169-8050, Tokyo, Japan} 

\author{Anh T. Tran} 
\email{att140830@utdallas.edu}
\address{Department of Mathematical Sciences, The University of Texas at Dallas\\ 
Richardson, TX 75080, USA} 

\author{Jun Ueki} 
\email{uekijun46@gmail.com}
\address{Department of Mathematics, Faculty of Science, Ochanomizu University; 
2-1-1 Otsuka, Bunkyo-ku, 112-8610, Tokyo, Japan}

\begin{document}
\maketitle
\begin{abstract} 
We consider a natural divisor on ${\rm SL}_2\C$-character varieties of knots and links, given by the so-called \emph{acyclic Reidemeister torsion}. We provide a geometric interpretation of this divisor. We focus on the particular family of odd twisted Whitehead links, where we show that this divisor has multiplicity two. Moreover, we {apply these results} to the study of the $L$-functions of the universal deformations of representations {over finite fields} of twisted Whitehead links. 
%We study the zeros of the $\SL_2$-acyclic torsion functions $\tau$ of links. 
%We first study that of the Whitehead link $W_1$ and deduce results for twist knots simultaneously. 
%Next, we extend the study to every twisted Whitehead link $W_k$ and prove that the zeros of $\tau$
%%the torsion functions 
%have the multiplicity two on the geometric component of the character variety.
%Finally, we clarify the notion of multiplicity of common zeros, and investigate the $L$-functions of the universal deformations of residual representations of twisted Whitehead links. %$W_k$. %and investigate their $L$-functions. 
%In addition, we precisely investigate the Whitehead link $W_1$ and deduce results for the twist knots. 
%We also deduce result of 
%We prove that the zeros of $\SL_2$-acyclic torsion function of every twisted Whitehead link $W_k$ have order two in the geometric component. 
\end{abstract} 

\tableofcontents

\section{Introduction}

In this paper, we will study the $\SL_2$-character varieties of some families of 3-manifolds. The  $\SL_2$-character variety of a manifold $M$ is the space of representations $\{\rho \colon \pi_1(M) \to \SL_2\C\}$, up to conjugation. 
%It is an affine complex algebraic set
%, and in this set we will focus on the components which contain one (and hence a dense Zariski open subset of) irreducible representation.

Given a representation $\rho \colon \pi_1(M) \to \SL_2\C$, one can construct a cellular homological complex $C_*(M, \rho)$ which only depends on the conjugacy class of the representation $\rho$. For 3-manifolds $M$, this complex is often \emph{acyclic}: its homology is trivial. More precisely, if $M$ is a hyperbolic 3-manifold or a Seifert fibered space, there is a non-empty Zariski open subset of the character variety of $M$ consisting of acyclic representations. 

For $\rho$ an acyclic representation, despite the homology of the complex $C_*(M, \rho)$ does not contain any interesting information on the topology of $M$, one can construct a numerical invariant, the \emph{Reidemeister torsion} (or just torsion for short), out from this complex. This invariant extends to a rational function on the character variety, whose zero locus -- its \emph{divisor} -- is precisely the set of non-acyclic representations. In \cite{Benard2020OJM}, the first author studied the torsion on the character variety of knots complement, and exhibited a criterion for the torsion to define a non-constant regular function on this algebraic set. In particular, he gave a sufficient condition so that the zero locus of the torsion is not void.

In \cite{TTU}, the three last authors studied the infinite family of \emph{twist knots} from this perspective. Any twist knot has a one-dimensional character variety. The Reidemeister torsion defines a function on this curve, which vanishes at a finite number of points. Interestingly, in \cite{TTU} they proved that all these zeros appear with vanishing multiplicity exactly two. Since all these data (character varieties, Reidemeister torsion...) are topological invariants, it is natural to try to understand why this phenomenon occurs. This is what we call the \emph{multiplicity two phenomenon}.

Our first purpose is to generalize these notions and results. It turns out that twists knots can all be obtained by surgery on a boundary component of a two-component link: the Whitehead link (see \cref{fig:Whitehead}). 




%We consider the Whitehead link $W$ in the 3-sphere (see \cref{Fig:W}), and its exterior $W$, a compact manifold with two toroidal boundary components. 
%The $\SL_2\C$-character variety $X(W)$ of the Whitehead link is essentially the set of conjugacy classes of representations $\rho \colon \pi_1(S^3 \setminus W) \to \SL_2\C$. As it is well-know, it is an algebraic surface in $\C^3$ with two irreducible components. One of them uniquely consists of reducible representations, and will be discarded in the rest of this paper. We will focus on the other component, it contains an open dense subset of irreducible representations.

%Given a conjugacy class of irreducible representation $\rho \colon \pi_1(S^3 \setminus W) \to \SL_2\C$, we can define a numerical invariant $\tor_W(\rho)$, the \emph{Reidemeister torsion} of the pair $(W, \rho)$. This is a complex valued invariant, in fact the torsion function $\tor_W(\cdot) \colon X(W) \to \C$ is regular, and we are interested in its divisor.

We prove the following:
\begin{theorem}
\label{theo:main}
The divisor of the torsion function on the character variety of the Whitehead link $W$ is $2L$, where $L$ is a complex line corresponding to the character variety of the closed manifold $W(-3,-3)$. This manifold is the connected sum of two lens spaces $L(3,1) \# L(3,1)$, obtained after $(-3,-3)$ surgery on the Whitehead link.
\end{theorem}

We stress two notable facts out of this statement: first, the divisor consists of one line \emph{of multiplicity two}. %In other words, the torsion vanishes with multiplicity two there.
%We don't have any explication for this fact that the torsion vanishes at order two, but we observe that it is not an exceptional phenomenon:

Second, the vanishing locus of the torsion has a geometric interpretation, since it can be described as the character variety of $W(-3,-3)$. We deduce the following corollary.
%A representation $\rho \colon \pi_1(M) \to \SL_2\C$ is called \emph{acyclic} if the twisted homology $H_*(M, \C^2_\rho)$ is trivial (see Section \cref{sec:Torsion} for definitions). It is not difficult to see that non-acyclic representations are precisely those for which the torsion function vanishes.
\begin{corollary}
\label{cor:-3}
The character variety $X(W(-3,-3))$ is a complex line, with no acyclic representations.
\end{corollary}
%We find this statement surprising: as we said before, acyclic representations are usually abundant. To our knowledge, this is the first example of a manifold whose character variety is a curve which does not contain any acyclic representation. 
%Note that the manifold with toroidal boundary obtained by $-3$ surgery on only one boundary component of the Whitehead has its character variety consisting of two parallel complex lines, one of which is the character variety of $M_{(-3,-3)}$.

%It is a Zariski open property on the character variety, and any hyperbolic $3$-manifold of finite volume possesses at least one acyclic representation, see \cite[Theorem 0.3 and Corollary 2.2]{MP10}. In fact \cite[Theorem 0.3]{MP10} is stated for only one cusp, but the same argument (applying Theorem 0.1) obviously works in the same way when $M$ has more cusps. For Seifert fibered $3$-manifolds, it is not difficult to see (see for instance \cite{kitano1994}) that acyclic representations form a non-empty open subset in $X(M)$. 
%In fact the manifold $M_{(-3,-3)}$ is a reducible manifold, a connected sum $L(3,1)\# L(3,1)$ of two lens space, see \cite{GuillouxWill} right after Proposition 4\footnote{Check this, and refer to Martelli}. 
The last corollary was guided by the fact (\cite{TTU}) that every non-acyclic representation of a twist knot group sends a specific element of the fundamental group on an elliptic of order $3$ in $\SL_2\C$. In fact, we show that these representations factor through the $-3$ surgery on one boundary component of the Whitehead link, and that along this surgery this specific curve has order 3.

Furthermore, we provide a geometric interpretation of this theorem. Let us denote by $W(-3)$ the manifold obtained by $-3$ surgery on \emph{one} component of the Whitehead link. This is a Seifert fibered manifold with boundary (see also \cite[Table 1]{MP}, and we show that its fundamental group is a central extension of $\Z/3 \ast \Z/3$ by $\Z$. In fact, this Seifert manifold is a circle bundle on a 2-dimensional orbifold, a sphere with three singularities -- two conical points of order 3, and one cusp -- obtained by gluing along their edges two hyperbolic triangles with angles $(\pi/3, \pi/3, \infty)$. 

We prove
\begin{proposition}
\label{prop:NAsurg}
Let $\rho \colon \pi_1(M) \to \SL_2\C$ an irreducible representation. Assume that $\rho$ factors through the fundamental group of a Seifert fibered manifold $N$ obtained by surgery on $M$ but is not trivial on this boundary component. If $\rho$ sends the generic fiber on the identity, then $\rho$ is non-acyclic.
\end{proposition}

One can see the divisor of the torsion on the character variety of the Whitehead link, described in \cref{theo:main}, as a special case of this proposition. The non-acyclic representations in this divisor factor through the $-3$ surgery, and then the second $-3$ surgery has the effect of killing the homotopy class of the fiber. Then any irreducible representation which factors through the $(-3,-3)$ surgery satisfies the hypotheses of the latter proposition.

We also recover and extend some of the results proved in \cite{TTU} for twist knots directly from the description of the Whitehead link. The acyclicity property for representations behaves particularly well under surgery operation, see \cref{subsec:surg}.
For $n\in \Z$, the manifold $W(-1/n)$ obtained by $-1/n$ surgery on one of the boundary components of the exterior of the Whitehead link $W$ is the exterior of the twist knot $J(2,2n)$.
%It is well known that $M_{-1/n}$ is the exterior of the twist knot $J(2,2n)$.
Our \cref{theo:main}, together with \cref{lem:intermult} has then the following corollary:
\begin{corollary}
\label{cor:twist}
The torsion function vanishes with multiplicity at least 2 on the character variety of $X(W(-1/n))$.
\end{corollary}

Note that a stronger result has already been obtained by the last three authors in \cite{TTU}. In fact, they proved that the vanishing multiplicity is exactly two. This may be seen as a transversality statement in $X(W)$ between the line $L$ and the character varieties of the twist knots.

We refine the properties of the representations of twist knots groups where the torsion vanishes which had been highlighted in \cite{TTU}:
\begin{theorem}\label{theo:prop}
The zeros of the torsion are representations that factor through the fundamental group of the manifold $W(-1/n,-2)$ obtained by $-3$ surgery on $W(-1/n)$. Each of those representations factor through to the triangle group $\Gamma(3,3,3n-1)$ if $n \ge 1$, and through $\Gamma(3,3, -3n+1)$ if $n \le 1$. These representations are  ${\rm SU}(2)$ representations which lie on a ``diagonal'' $D$,
%, and their traces satisfy the equation $\Tr \rho(y) = \Tr(\rho(y\upsilon))$ where $y, \upsilon$ are meridians, 
see after \cref{prop:twist}.
\end{theorem}

For any surgery on one component of the Whitehead link (with slope different from $-3$), the divisor of the torsion will have multiplicity at least two, as follows from \cref{lem:intermult}. In particular for integral surgeries, one gets the family of once-punctured torus bundles with tunnel number one which has been deeply studied in~\cite{BakPet}.

\begin{corollary}
The torsion function vanishes on the character variety of once-punctured torus bundles with tunnel number one with multiplicity at least two.
\end{corollary}


It turns out that the phenomenon observed for the Whitehead link (and for the manifolds obtained by surgery on it) is not an exception: we consider further examples of odd twisted Whitehead links $W_{2n-1}$ (see \cref{fig:TwistedW}). These are hyperbolic links with the linking number zero, whose character varieties and Reidemeister torsion have been computed by Nguyen and the third author in \cite{NguyenTran}. In particular, for each twisted Whitehead link, there is exactly one geometric component in the character variety, which contains the lifts of the holonomy representation. 

First, we prove the following proposition:
\begin{proposition}
\label{prop:smooth}
Each component of the character variety of the odd twisted Whitehead link $W_{2n-1}$ is a smooth surface in $\C^3$.
\end{proposition}
Then we study the vanishing multiplicity of the divisor of the torsion on these character varieties.
We prove the following:
\begin{theorem}
\label{theo:TW}
The divisor of the torsion has multiplicity two on the geometric component of the character variety of any twisted Whitehead link $W_{2n-1}$. On the other components, either it has multiplicity one, or the torsion vanishes identically.
\end{theorem}

Considering surgeries with slope $-1/m$ on twisted Whitehead links $W_{2n-1}$, we obtain the double twist knots $J(2m,2n)$. We deduce the following corollary from the fact that the representations in $X({W_{2n-1}})$ which factor to representations of the fundamental group of double twist knots lie in the geometric component:
\begin{corollary}
\label{coro:doubletwist}
The torsion vanishes with multiplicity at least two on the character variety of double twist knots $J(2m,2n)$.
\end{corollary}

The multiplicity of the divisor of the torsion is also related to the singularities of the $\SL_3$-character variety, as we explain now. 
The $\SL_2$-character variety of a knot exterior always contains a line of \emph{reducible} representation. This line may intersect the other components, in which irreducible representations are dense. 
It has been known since de Rham (\cite{DR67}) that the intersection points between the component of reducible representations and the other component correspond to roots of the Alexander polynomial of the knot. This has been generalized by Heusener and Porti (\cite{HP15}). They proved the following: if an $\SL_n(\C)$-representation is an intersection point between a component of reducible representations and another one, then a certain \emph{twisted} Alexander polynomial vanishes at some special value. Moreover, if this value is a simple root of the twisted Alexander polynomial, then the intersection is as mild as possible: a transverse intersection between two algebraic sets. 
In our setting, the torsion of a representation $\rho \colon \pi_1(M) \to \SL_2\C$ is the value at $t=1$ of the $\rho$-twisted Alexander polynomial $\Delta_\rho(t)$. Heusener--Porti's result suggests that the curve of non-acyclic representations of the twisted Whitehead links yield $\SL_3(\C)$-representations of the form $\bma \rho & \ast \\ 0 & 1 \ema$ which are highly singular intersection points in their $\SL_3$-character variety. Note that this is the case for one of the only $\SL_3(\C)$-character varieties explicitly computed: the figure-eight knot, see \cite{HMP}.

\medbreak

%\ 
%[Add description on deformations and $L$-functions] \ 
In the final section of this article, we pursue the study of the $L$-functions of universal deformations, raised in a viewpoint of number theory (\cite{Mazur2000}, \cite{MTTU2017}, \cite{KMTT2018}, \cite{TTU}), to interpret the ``multiplicity two'' phenomenon into the property of the $L$-functions of the odd twisted Whitehead links $W_{2n-1}$. 

{Given a finite field $\F$ with characteristic $p>2$ (or an algebraic closure of such a field), and a representation $\overline \rho \colon \pi \to \SL_2\F$, there is a natural Complete Discrete Valuation Ring (CDVR) $O$ whose residual field $O/ \mathfrak m_{O} = \F$ (a typical example is $O = \Z_p$ for $\F = \F_p$). 
\emph{The universal deformation} ${\bs \rho}$ of $\ol{\rho}$ over $O$ is a lift of $\ol{\rho}$ with a certain universal property, to a representation over a complete local $O$-algebra $\mca R_{\ol{\rho}}$. %This algebra is called the \emph{universal deformation ring} and 
The representation $\bs \rho$  is unique up to strict equivalence (fixing the residual representation $\ol \rho$). 
%There is a \emph{universal representation} $\bs{\rho} \colon \pi = \pi_1(S^3 \setminus W_{2n-1}) \to \mathcal R$, for $\mathcal R$ the \emph{universal deformation ring}.
%, which somehow encodes deformations of $\overline \rho$. 
Considering the homology of the complex of $\mathcal R_{\ol{\rho}}$-modules given by $\bs \rho$,  the analogy between knots and prime numbers designed as \emph{arithmetic topology} suggests that the order $L_{\bs \rho}$ of the first homology group of this complex should be of interest. It is called the $L$-function of $(\ol \rho, O)$, and has been studied by many authors (see references above).} 

{
Since the multiplicity of the zeros of the $L$-function $L_{\bs \rho}$ are intimately related to the vanishing of the Reidemeister torsion, we deduce from our study of the Reidemeister torsion of the twisted Whitehead links the following result.
Note that since the character variety is defined over $\Z$, one can reduce its defining equation $f_n$ modulo a prime $p$, and see new solutions of $\overline f_n = 0$ as the character variety over the algebraic closure of $\F_p$. 
\begin{theorem}
\label{theo:L}
Let $\ol{\rho}:\pi\to \SL_2\F$ be an absolutely irreducible representation of the group of the twisted Whitehead link $W_{2n-1}$ {at $(\ol{a},\ol{b},\ol{c})$ in $\F^3$} on the geometric component $f_n=0$ of the character variety and suppose that $\ol{\rho}$ is non-acyclic. 
{Let $(a,b)$ a lift of $(\ol{a},\ol{b})$ in $O^2$.}  
Suppose that $\partial_z f_n \neq 0$ {in $\F$} at $\ol{\rho}$, 
and let $z_{f_n}(x)$ denotes the implicit function given by Hensel's lemma. 
Then, the $L$-function $L_{\bs \rho}$ of the universal deformation is {given by} 
\[L_{\bs \rho}\,\dot{=}\, \tau(x,y,z_{f_n}(x))\] 
in $\mca{R}_{\ol{\rho}}=O[\![x-a, y-b]\!]$.  
%where $z_f(x)$ denotes the implicit function that Hensel's lemma gives. 
{In general, under the weaker hypothesis that $\ol{\rho}$ corresponds to a regular point of $\ol{f}=0$, then $L_{\bs \rho}$ is given in a similar way. In any of these cases,  
%\red{For a generic $\ol{\rho}$}, 
the multiplicities of the zeros of $L_{\bs \rho}$ are {at least} two.}  
%In addition, if $\ol{\rho}$ moves, $L_{\bs \rho}$ is simultaneously presented by a certain rational function. 
%{\footnotesize \red{(The precise statement of this theorem depends on that of \cref{theo:TW} and \cref{lem.mult-two}.)}} 
\end{theorem} 
%\begin{theorem}
%\label{theo:L}
%Let $\ol{\rho}:\pi\to SL_2\F$ be an absolutely irreducible representation of the group of the twisted Whitehead link $W_{2n-1}$ on the geometric component  of the character variety. Suppose that $\ol{\rho}$ is non-acyclic, and corresponds to a point $(\ol a, \ol b, \ol c) \in  \F^3$ such that $\overline f_n(\ol a, \ol b, \ol c)=0$.
%If $\partial_z {\ol f_n} \neq 0$ {in $\F$} at $\ol{\rho}$, there is a rational function $z_{f_n}(x,y)$ such that the $L$-function $L_{\bs \rho}$ of the universal deformation is {given by} 
%\[L_{\bs \rho}\,\dot{=}\, \tau(x,y,z_{f_n}(x,y))\] 
%in $\mca{R}_{\ol{\rho}}=O[\![x-\ol a, y-\ol b]\!]$. 
%A similar statement holds for $\partial_x \ol f_n\neq 0$ of $\partial_y \ol f_n \neq 0$. In any of these cases, 
%{the multiplicities of the zeros of $L_{\bs \rho}$ are {at least} two.}
%%\red{For a generic $\ol{\rho}$}, the multiplicities of the zeros of $L_{\bs \rho}$ are {at least} two. 
%%In addition, if $\ol{\rho}$ moves, $L_{\bs \rho}$ is simultaneously presented by a certain rational function. 
%%{\footnotesize \red{(The precise statement of this theorem depends on that of \cref{theo:TW} and \cref{lem.mult-two}.)}} 
%\end{theorem} 
}

\subsection*{Organization of the paper}
In \cref{sec:pre} we introduce character varieties, Reidemeister torsion, the notion of multiplicity of a divisor and we prove some preparatory lemmas.  In \cref{sec:Whitehead} we treat the case of the Whitehead link, proving \cref{theo:main}, \cref{theo:prop} and their corollaries. In \cref{sec:tW} we study the family of odd twisted Whitehead links. We prove \cref{prop:smooth} and \cref{theo:TW}. Then in \cref{sec:L} we prove \cref{theo:L} on the $L$-functions.

\subsection*{Acknowledgments}
The authors would like to express their sincere gratitude to 
Marco Maculan for very enlightening conversations on the notion of the multiplicity of a divisor. We also thank Rapha\"el Alexandre, Elisha Falbel, Antonin Guilloux, Michael Heusener, Tomoki Mihara, and Gauthier Ponsinet for useful discussions and references.% , Tomoki Mihara for useful information and fruitful conversations. Gauthier Ponsinet?

L.B. is partially funded by the Research Training Group 2491 ``Fourier Analysis and Spectral Theory'', University of G\"ottingen.
A.T. has been supported by a grant from the Simons Foundation (\#708778). 
J.U. 
%The fourth author 
has been partially supported by 
%Japan Society for the Promotion of Science
JSPS KAKENHI Grant Number JP19K14538 respectively. 

\section{Preliminaries} % on character varieties, twisted homology, and Reidemeister torsion.}
\label{sec:pre}
This section contains preliminary facts on character varieties (\cref{subsec:prechar}), twisted homology (\cref{subsec:homol}), and Reidemeister torsion (\cref{subsec:pretors}). We prove some technical lemmas on acyclicity and surgeries in \cref{subsec:surg}. Then we prove \cref{prop:NAsurg} in \cref{subsec:NAsurg}, and we introduce the notion of multiplicity of a divisor in \cref{subsec:mult}


\subsection{Character varieties}
\label{subsec:prechar}
Given a finitely generated group $\Gamma$, its representation variety $R(\Gamma)$ is the set of group homomorphisms $\rho \colon \Gamma \to \SL_2\C$.
The group $\SL_2\C$ acts by conjugation on $R(\Gamma)$, and the \emph{character variety} of $\Gamma$ is the algebraic quotient 
\[X(\Gamma) = R(\Gamma) /\!/\SL_2\C.\]
Two representations $\rho$ and $\rho'$ are identified in the quotient precisely if $\Tr \rho(\gamma) = \Tr \rho(\gamma')$ for all $\gamma$ in $\Gamma$. %A point in $X(\Gamma)$ will commonly be called a \emph{character}. 
In general $X(\Gamma)$ is an algebraic set: it is defined as the zero locus of a family of polynomials in some $\C^N$.

 In the following we will be interested mainly in \emph{irreducible} representations $\rho \colon \Gamma \to \SL_2\C$. A representation is irreducible if its image in $SL_2\C$ does not stabilize any proper subset of $\C^2$. The following lemma ensures that this notion is well-defined on the character variety $X(\Gamma)$:
\begin{lemma}{\cite[Lemma 1.2.1]{CS83}}
\label{lem:red}
A representation $\rho\colon \Gamma \to \SL_2\C$ is reducible if and only if 
$\Tr \rho(\gamma\delta\gamma^{-1}\delta^{-1}) = 2$ for all $\gamma, \delta \in \Gamma$.
\end{lemma}
This lemma says a bit more: the set of irreducible characters is Zariski open in $X(\Gamma)$. 
Let us write the character variety $X(\Gamma) = X^{\rm{red}}(\Gamma) \cup X^{\rm{irr}}(\Gamma)$. In the following, we will be only interested in the closure $\overline{X^{\rm{irr}}(\Gamma)}$. We will abuse notation and say \emph{the} character variety for this subset. Note that characters of irreducible representations now form a Zariski dense open subset.

Given a manifold $M$ with finitely generated fundamental group $\pi_1(M)$, the character variety $X(\pi_1(M))$ will be simply denoted by $X(M)$. 

\subsection{Twisted homology} \label{subsec:homol} 
Given a compact $3$-manifold $M$ with a CW decomposition $K$ and a representation $\rho \colon \pi_1(M) \to \SL_2\C$, we defined the complex of twisted homology 
\[C_*(M, \C^2_\rho) = \C^2 \otimes_{\Z[\pi_1(M)]} C_*(\widetilde M)\]
with right-action on $\C^2$ through $\rho$, and left action on lifts in $C_*(\widetilde M)$ of the cells in $K$ by deck transformation. If one numbers the cells $\{c_i^j\}_{\substack{i=0, \ldots, 3\\ j = 1, \ldots, n_i}}$ with a choice of lifts $\widetilde c_i^j$, and one picks any basis $v_1, v_2$ of $\C^2$,  then a basis of the vector space $C_i(M, \C_\rho^2)$ is 
\[ \{v_1 \otimes \widetilde c_i^1, \ldots, v_2 \otimes \widetilde c_i^{n_i}\}.\]
The boundary map acts by 
$\partial (v \otimes \widetilde c) = v \otimes \partial \widetilde c$.

The homology of this complex does not depend on the cell decomposition $K$, nor on the choice of lifts.
A representation $\rho \colon \pi_1(M) \to \SL_2\C$ is said \emph{acyclic} if 
$H_*(M, \C^2_\rho) = \{ 0 \}$.

\subsection{Reidemeister torsion}
\label{subsec:pretors}
In general, given an acyclic complex $C_*$ of $\C$-vector spaces with a basis $\bf c_* = \{ \bf c_1, \ldots, c_n \}$ (i.e., the set $\bf c_i$ is a basis for the vector space $C_i$), then the Reidemeister torsion is defined as follows.

Consider the exact sequences
\[0 \to Z_i \to C_i \xrightarrow{\partial_i} B_{i-1} \to 0.\]
One can pick arbitrary bases $\bf b_i$ of the vector spaces $B_i \simeq Z_i$ for each $i$, and arbitrary lifts $\bf{\overline b}_i$ to $C_{i+1}$: it yields new bases $\bf b_i \sqcup \bf{\overline b}_{i-1}$ of $C_i$. Now denoting by $[\bf b_i \sqcup \bf{\overline b}_{i-1} \colon \bf c_i]$ the determinant of the change of basis matrix, the Reidemeister torsion is defined by the alternating product
\[\tor(C_*, \bf c_*)  = \prod_i [\bf b_i \sqcup \bf{\overline b}_{i-1} \colon \bf c_i]^{(-1)^i} \in \C^*.\]
It does not depend on the choices $\bf b_*$, nor on the lifts $\bf{\overline b}_i$. Note that there is a sign indeterminacy since we did not choose \emph{ordered} bases. We won't be concerned by this indeterminacy in this paper, because we will be interested only in the vanishing of the Reidemeister torsion.

Given a $3$-manifold $M$ with an acyclic representation $\rho \colon \pi_1(M) \to \SL_2\C$, the Reidemeister torsion $\tor_M(\rho)$ is the Reidemeister torsion of the complex $C_*(M, \C^2_\rho)$ for some cellular decomposition of $M$. A deep theorem (\cite{Chapman, Cohen}) ensures that this does not depend on the choice of a cellular decomposition. It depends only on the conjugacy class of $\rho$, and hence yields a function on the character variety $X(M)$. For a non-acyclic representation $\rho$, we just define $\tor_M(\rho) = 0$. Since the torsion of an acyclic complex is a non-zero 
complex number, these representations are exactly the zeros of the torsion function.
\subsection{Acyclicity}
\label{subsec:surg}
We prove two technical lemmas.
\begin{lemma}
\label{lem:MV}
Let $N$ be a compact $3$-manifold with a knot $\gamma$ in $N$. We denote also by $\gamma$ its homotopy class in $\pi_1(N)$. Let $\rho \colon \pi_1(N) \to \SL_2\C$ be a representation such that $\Tr \rho(\gamma) \neq~2$. Denote by $M$ the exterior of $\gamma$ in $N$. The representation $\rho$ extends naturally to a representation $\varrho \colon \pi_1(M) \to \SL_2\C$ such that $\varrho$ is acyclic if and only if $\rho$ is acyclic.
\end{lemma}

\begin{proof}
Denote by $V(\gamma)$ a tubular neighborhood of $\gamma$ homeomorphic to a solid torus $D^2 \times S^1$, and by $T_\gamma$ the boundary $\partial V(\gamma)$. Note that $\pi_1(V(\gamma))$ is freely generated by $\gamma$, and let $c_\gamma \in \pi_1(T_\gamma)$ any element such that $\{\gamma, c_\gamma\}$ generates $\pi_1(T_\gamma)$.
Then the complex $C_*(V(\gamma), \C^2_\rho)$ is 
\[C_1(V(\gamma), \C^2_\rho)\simeq \C^2 \xrightarrow{I-\rho(\gamma)} \C^2\simeq  C_0(V(\gamma), \C^2_\rho),\]
in particular $H_*(V(\gamma), \C^2_\rho) = \{0\}$ since $\Tr(\rho(\gamma))\neq 2$.
Now the complex of the boundary torus $T_\gamma$ looks as follows:
\[C_2(T_\gamma, \C^2_\rho) \simeq \C^2 \xrightarrow{d_1} C_1(T_\gamma, \C^2_\rho) \simeq \C^4 \xrightarrow{d_0} C_0(T_\gamma, \C^2_\rho)\]
where 
\[d_1 = \bma I-\rho(\gamma) \\ I-\rho(c_\gamma) \ema, \quad d_0 = \bma \rho(c_\gamma) - I & \rho(\gamma)-I \ema.\]
Again because $\Tr \rho(\gamma)\neq 2$, one sees that $d_1$ is injective and $d_0$ is onto. Since $\chi(T_\gamma) = 0$, one deduces that $H_*(T_\gamma, \C^2_\rho) = \{0\}$ as well.

Now we use the Mayer--Vietoris sequence for the decomposition
\[N = M \cup_{T_\gamma} V(\gamma)\]
and it comes that $H_*(M, \C^2_\varrho) \simeq H_*(N, \C^2_\rho)$. The lemma follows.
\end{proof}

\begin{lemma}
\label{lem:open}
Let $M$ be a 3 manifold, and $X(M)$ be the closure of the subset of irreducible characters in its character variety. Then being acyclic is an open property on $X(M)$
\end{lemma}

\begin{proof}
If $\chi(M)\neq 0$, then there are no acyclic representations, and hence one can assume that $\chi(M) = 0$. Since irreducible representations form an open dense subset of $X(M)$, we shall prove that for $\varrho$ irreducible acyclic, a neighborhood of $\varrho$ is acyclic as well.
Let $\rho$ be a representation close enough to $\varrho$. Since it is irreducible, by \cref{lem:red} there exists a curve $\gamma_0$ in $\pi_1(M)$ such that $\Tr \rho(\gamma_0) \neq 2$. It is well-known (see \cite{Bro}) that the space $H^0(M, \C^2_\rho)$ is isomorphic to the space of invariant vectors $\{v \in \C^2 \mid \forall \gamma, \, \rho(\gamma)^{-1} v = v \}$, which is trivial because $\rho(\gamma_0)$ has no fixed non-zero vector. By the universal coefficients theorem, the space $H_0(M, \C^2_\rho)$ is then trivial as well. Since $\chi(M) = 0$, it follows from Poincar\'{e} duality that the representation $\rho$ is acyclic if and only if $H_1(M, \C^2_\rho) = \{0\}$. 
We claim that the latter is an open condition on $X(M)$. To see this, note that the dimension of the kernel of the map $d_0$ is constant, since $H_0(M, \C^2_\rho) = \{0\}$. Hence $\dim H_1(M, \C^2_\rho)$ is determined by the rank of the matrix $d_1$. This matrix varies polynomially with the entries of the representation $\rho$, and the rank of this matrix is a lower-semicontinuous function. Hence it cannot decrease around $\varrho$, and it proves the last claim.
%this follows from the lower semi-continuity of the dimension, see \cite[Theorem 12.8]{Hartshorne}.
\end{proof}

\subsection{Conventions for surgeries}
Let $M$ be the exterior of a link $L = K_1 \cup \ldots K_s$ in a homology sphere. For each boundary component (homeomorphic to a $2$-torus) $T_i = \partial V(K_i)$, there is a natural choice of peripheral curves $m_i ,l_i$ which generate $\pi_1(T_i)$, such that $m_i$ is a generator of $H_1(S^3 \setminus K_i) \simeq \Z$ which is trivial in $H_1(V(K_i))$, and $\ell_i$ is a generator $H_1(V(K_i))$ trivial in $H_1(S^3 \setminus K_i)$. The intersection $m_i \cdot l_i$ equals one in $H_1(T_i)$. This choice is unique up to simultaneous inversion $(m_i, l_i) \mapsto (-m_i, -l_i)$.

Fix a components $K_i$ of $L$, and $p/q$ in $\Q \cup \{1/0\}$, then the $p/q$ surgery along $K_i$ is the $3$-manifold obtained by gluing a solid torus $S^1 \times D^2$ to $M$ by an orientation reversing homeomorphism of $\partial (S^1 \times D^2) \simeq T_i$ such that the curve $c = \{ \ast \} \times \partial D^2$ is identified with the curve $p m + q l$ in $H_1(T_i)$. 

\subsection{Surgery and non-acyclic representations}
\label{subsec:NAsurg}
In this subsection, we prove \cref{prop:NAsurg}.

Take a 3-manifold $M$, and an irreducible representation $\rho \colon M \to \SL_2\C$. Assume that $\rho$ factors through $\pi_1(N)$, where $N$ is a Seifert fibered manifold obtained by surgery on $M$ along a curve $\gamma$ in a boundary component $T \subset \partial M$, such that $\rho$ is trivial on the generic fiber and non-trivial $\gamma$. We prove that $\rho$ in non-acyclic, that is $H_*(M, \rho)$ is non-trivial.

To see this, first consider the complex $C_*(N, \rho)$, where $N$ is Seifert fibered. A classical computation of twisted homology (see \cite{kitano1994} for instance) shows that $C_*(N, \rho)$ is acyclic if and only if the image of the fiber is non-trivial. Since $\rho$ factors through $\pi_1(N)$ but is not the identity on $\gamma$, applying \cref{lem:MV} we deduce that $\rho$ is non-acyclic.


\subsection{Definition of multiplicity}
\label{subsec:mult}
Here we follow \cite[Chapter 3]{Shaf}. Let $X$ be an irreducible algebraic variety. In this first paragraph, we assume that $X$ is non-singular in codimension one. Let $P \in \C[X]$ a regular function on $X$. The multiplicity of (the divisor of) $P$ is defined as follows. The set $\{P=0\}$ defines a union of codimension one irreducible subvarieties $C_1 \cup \ldots \cup C_n$. For each $i$, take an affine open subset $U$ of $X$ such that $C_i \cap U$ is locally defined by the equation $f_i = 0$, for some $f_i \in \C[U]$. Then the local ring $O_i$ of $C_i \cap U$ has $(f_i)$ as a unique maximal ideal, and Nakayama lemma implies that there exists an integer $k \ge 0$ such that $P \in (f_i^k)$ but $P \notin (f_i^{k+1})$. \\
We define this integer $k$ to be the \emph{multiplicity} of $\{P =0\}$ along the component  $C_i$ in $X$. Note that if $X$ is smooth (which will be the case in this paper, see \cref{prop:smooth}), then any irreducible divisor in $X$ corresponds to a discrete valuation on its local ring, and the multiplicity of $P$ is just its valuation.


It is not obvious from this definition that this does only depend on the local geometry of the divisor of $P$, and not on the ambient manifold $X$ itself.
To see this, we refer to \cite[Appendix 1]{Hartshorne} where an alternative definition of the multiplicity of a divisor is given by Serre's intersection formula:
if $X$ and $Y$  intersect properly (meaning that every irreducible component of $X\cap Y$ has the right codimension), then for $W$ an irreducible component of $X \cap Y$, we have that the intersection multiplicity of $X$ and $Y$ along $W$ is
\[i(X,Y;\, W) = \sum_i (-1)^i \,  \textrm{length}( \Tor_i^A(A/\mathfrak a, A/ \mathfrak b)).\]
In the setting we described above, $W$ is the underlying manifold given locally by $\{f_i=0\}$, $A$ is the local ring $O_i$, and $\mathfrak a, \mathfrak b$ are the ideals defining $X$ and $Y$ in $A$.

Note that  \cite[Appendix 1, Theorem 1.1]{Hartshorne} shows that the multiplicity is additive, and is equal to one if the varieties $X$ and $Y$  intersect transversally.
Moreover, in the case we will consider in this paper, we will be in the situation where the $\Tor$ part vanishes, hence the intersection multiplicity is just the length of the ideal defining $X \cap Y$ locally.

We prove that the multiplicity behaves well under restriction to codimension one subvarieties:
\begin{lemma}
\label{lem:intermult}
Let $D$ be a divisor in a variety $X$ (which we assume to have one single underlying irreducible component), with multiplicity $m_X(D)$. Let $Y$ be a hypersurface in $X$, such that $Y$ and $D$ intersect properly (namely they have no common component). Then 
\[m_Y(D\cap Y) \ge m_X(D).\]
\end{lemma}

\begin{proof}
Let $k = m_X(D)$, then locally the divisor $D$ is defined by a polynomial $P$ in $(f^k)$, for some regular function $f$ in $\C[U]$ (see the notations above). Assume that $Y = \{g=0\}$ in $U$, then  $D \cap Y$ is defined by $\overline P$ in $\C[U]/(g)$. Since $P = f^k Q$, we have $\overline P = \overline f^k \overline Q$, and if $\overline P \neq 0$, this proves that $m_Y(D\cap Y) \ge k$. Otherwise, $\overline P = 0$, and it contradicts the fact that $Y$ and $D$ intersect properly.
\end{proof}

%If we need, computation of Tor functors follows:

%, as we show now:
%\begin{lemma}
%The following hold:
%\begin{enumerate}
%\item
%$\Tor_0^A(A/\mathfrak a, A/\mathfrak b) = A/(\mathfrak a,\mathfrak b)$
%\item 
%$\Tor_1^A(A/\mathfrak a, A/\mathfrak b) = (\mathfrak a \cap \mathfrak b) / \mathfrak a \mathfrak b$
%\item If $\mathfrak b = (f)$, then
%$\Tor_i^A(A/\mathfrak a, A/\mathfrak b) = 0 \text{ for any } i \ge 2$
%\end{enumerate}
%\end{lemma}
%The proof is an exercise in commutative algebra, we omit it.



%We want to emphasize that this definition only depends on the local geometry of the divisor of $P$, and not of the ambient manifold $X$ itself. Moreover, the refinement we propose here allows to get rid of the codimension one smoothness assumption.
%To this purpose, let us assume that $X$ is an affine algebraic set. Note that it will be the case along this paper. Moreover the definition above is local, so it causes no loss of generality.
%The function $P$ defines an ideal $I$ in the function ring $\C[X]$ of  $X$. This ideal has a \emph{primary decomposition}
%\[I = Q_1 \cap \ldots \cap Q_n\]
%into \emph{primary ideals} $Q_i$. Their radicals $P_i = \sqrt{Q_i}$ are prime ideals, each of which defines an irreducible component $C_i$ of the algebraic variety underlying the set $\{ P=0\}$ in $X$. Given a component $C_i$, we want to define the multiplicity of the primary ideal $Q_i$. Note that $Q_i$ need not to be a power of its radical $P_i$, nevertheless, the following construction permits to make this intuition rigorous.
%
%First, consider the localization $\C[X]_{P_i}$ of the function ring $\C[X]$ at the ideal $P_i$. It is  a ring, containing $\C[X]$ and contained in the fraction field $\C(X)$, where any function which is not in $P_i$ (hence, which does not identically vanish on $C_i$) is invertible. As above, this ring has a unique maximal ideal $P_i$, and there is a unique integer $k$ such that $Q_i \subset P_i^k$ and not in $P_i^{k+1}$, which clearly is the multiplicity of the ideal $Q_i$ alluded above.
%
%\begin{remark}
%Note that the multiplicity defined that we have defined is not, for intersection of curves in general, the intersection multiplicity. For example, if one takes the two functions $f = xy$ and $g = y^2$ in $\C[x,y]$, the intersection multiplicity $\mu(f,g)= \dim_\C \left(\C[x,y]/(f,g)\right)_{(x,y)}$ is infinite (since there is a curve $y=0$ of common intersection), despite the multiplicity $m(f,g) = 2$. The latter corresponds to the mutiplicity of the point $(x,y)$, forgetting the fact that it is immerged in a higher dimensional component of the intersection. \\
%Nevetheless, in the case we are interested in (when $f$ is irreducible and reduced, i.e. considering the multiplicity of a divisor in an actual variety), one can show that the two definition coincide. For example, one can show that $m(f,g)$ satisfies the list of axioms of the intersection multiplicity of curves, see \cite[Chapter 3.3]{Ful}. The axiom which is not satisfied without the assumption that $f$ is irreducible is the first one, as we just have pointed out above.
%\end{remark}
%
%We conclude this section with an example, which will be used as a lemma throughout this paper.
%\begin{lemma}
%Let $X$ be an affine surface in $\C^3$, defined by the ideal $(f)$ for some $f \in \C[x,y,z]$, and let $g \in \C[x,y,z], \, g \notin (f)$. Assume that there exists polynomials $a,b \in \C[x,y,z]$ such that $(f,g) = (a,b^2)$, and $\mathcal J(a,b)= \bma \partial_x a & \partial_x b\\ \partial_y a & \partial_y b \\ \partial_z a & \partial_z b\ema$ has maximal rank. Then the multiplicity of $g$ on $X$ is two.
%\end{lemma}
%
%\begin{proof}
%First, we show that $g \in (b^2)$ mod $f$. Indeed, writing $g = \alpha a + \beta b^2$, and $a = Pf + Qg$, one gets in $\C[x,z]/(f)$:
%\[(1-\alpha Q) g \in (b^2)\]
%Now either $(1-\alpha Q) \notin (b)$, and we are done, either 
%$(1-\alpha Q) \in (b)$. In this case, we have $\alpha, Q \equiv 1$ mod $(b)$.
%
%Now we use $(f,g) = (a,b^2)$ to write 
%\[f = A(Pf+Qg) + B b^2.\]
%Reducing modulo $f$ yields
%\[Qg = -B b^2\]
%and we conclude that $g \in (b^2)$ from $Q \equiv 1$ mod $(b)$.
%

%Rewriting $g = = \alpha a + \beta b^2$, we see that $g \equiv a$ mod $b$, and hence, using $a = Pf + Qg$, we deduce $Pf \equiv 0$ mod $b$.
%
%Now we use $(f,g) = (a,b^2)$ to write 
%$$f = A(Pf+Qg) + B b^2.$$
%Reducing modulo $(b)$ yields
%$$f \equiv Qa$$
%

%\end{proof}
%In this paper we will deal with affine hypersurfaces of $\C^3$. We will say that two such hypersurfaces $X_a=\{a(x,y,z) = 0\}$ and $X_b = \{b(x,y,z)=0\}$ \emph{intersect transversally} if the Jacobian matrix 
%\[ \mathcal J(a,b) = \bma \partial_x a & \partial_x b\\ \partial_y a & \partial_y b \\ \partial_z a & \partial_z b\ema\]
%has maximal rank on $X_a \cap X_b$, equivalently there exists a non-zero $2\times 2$ minor of $\mathcal J(a,b)$ everywhere on $X_a \cap X_b$. In other words, $b$ vanishes on $X_a$ with multiplicity one. 
%
%For given hypersurfaces $X =\{f(x,y,z) = 0\} ,Y =\{g(x,y,z) = 0\} \subset \C^3$, we say that they intersect \emph{with multiplicity two}, or that $g$ vanishes on $X_f$ with multiplicity two, if there exists $a,b \in \C[x,y,z]$, $X_a$ and $X_b$ transverse, such that $(f,g) = (a,b^2)$.
%
%Note that because of the semi-continuity of the rank of a matrix, the transversality condition is an open condition on $X_a \cap X_b$. We can relax this condition as follows:
%we say that $X_a$ and $X_b$ are \emph{generically transverse} if they intersect transversally on a Zariski open subset (or equivalently, at least at one point in each irreducible component) of $X_1 \cap X_b$. Similarly, we say that $X$ and $Y$ \emph{intersect generically with multiplicity two} if $X\cap Y$ is given by an ideal of the form $(a,b^2)$ with $X_a$ generically transverse to $X_b$.
%
%\begin{remark}
%Note also that if $X\cap Y$ has generic multiplicity two, the multiplicity can only be higher on the exceptional locus.
%\end{remark}



\section{Reidemeister torsion of the Whitehead link}
\label{sec:Whitehead}
In this section, we describe the character variety of the Whitehead link exterior and its Reidemeister function (\cref{subsec:char} and \cref{subsec:Tors}). We prove there the first part of \cref{theo:main}. Then we consider surgeries on the Whitehead link and relate their representation spaces with the vanishing locus of the torsion.
In \cref{subsec:-3} we prove the second part of \cref{theo:main} and deduce \cref{cor:-3}. Then in \cref{subsec:Twist} we prove \cref{cor:twist}
and \cref{theo:prop}.
%and compute the Reidemeister torsion function (\cref{subsec:Tors}), where the first part of \cref{theo:main} is proved.

We follow the computations of \cite{NguyenTran} for the Whitehead link ($W=W_1$ in the notations therein).
\subsection{The character variety}
\label{subsec:char}
The fundamental group of the exterior $W$ of the Whitehead link has the presentation
\begin{equation*}
\pi_1(S^3 \setminus W) = \langle m,\mu \mid mw=w\mu \rangle
\end{equation*}
 where $w= \mu m \mu^{-1}m^{-1}\mu^{-1}m\mu$, with $m$ and $\mu$ standard meridians of each component (see \cref{fig:Whitehead}). 

\begin{figure}[h]
\def\svgwidth{0.5\columnwidth}
\input{Whitehead.pdf_tex}
%\includegraphics[width=50mm%60mm
%]{Whitehead.jpg}
\caption{\label{fig:Whitehead} 
A diagram of the Whitehead link, with meridians $m$ and $\mu$.}
\end{figure}
 We rewrite it as
\begin{align}
\label{eq:pi1}
\pi_1(S^3 \setminus W) &= \langle m,\mu \mid m \mu m\mu^{-1}m^{-1}\mu^{-1}m\mu =  \mu m\mu^{-1}m^{-1}\mu^{-1}m\mu m \rangle \nonumber \\
&=\langle m,\mu \mid [\mu,m][\mu^{-1},m] [\mu^{-1},m^{-1}][\mu,m^{-1}]  \rangle
\end{align}

Using the coordinate functions $x = \Tr m, y=\Tr \mu, y = \Tr m \mu$,
the character variety has the equation
\begin{multline}
\label{eq:character}
X(W) = \lbrace (x,y,z) \in \C^3 \mid \\
(xyz^2-z(x^2+y^2+z^2)+xy+2z)(x^2+y^2+z^2-xyz-4) = 0 \rbrace
\end{multline}
Both defining polynomials in \cref{eq:character} are irreducible. The zero locus of the second polynomial is the set of reducible characters. Denoting by $f= xyz^2-z(x^2+y^2+z^2)+xy+2z$, the set of irreducible characters is 
\[ \lbrace {f}=0 \rbrace \setminus \big\{ \{x=\pm 2, z=\pm y \} \cup \{y=\pm 2, z=\pm x \} \big\}.\]
\subsection{The Reidemeister torsion}
\label{subsec:Tors}
Evaluating the result of \cite[Theorem 1]{NguyenTran} at $t_1=t_2=1$, the Reidemeister torsion is the following regular function
\begin{equation}
\label{eq:tors}
\tau(x,y,z)= 2(2+z-x-y)
\end{equation}
	
We compute the zero locus of the torsion in the component $X=\{x,y,z \mid f(x,y,z)=0\}$ of $X(W)$ containing irreducible characters. It proves the first part of \cref{theo:main}.
\begin{proposition}
\label{prop:acycl}
The Reidemeister torsion vanishes with multiplicity two on the line 
\begin{equation}
\label{eq:zerotors}
L=\{ x+y-1=0, \, z=-1 \} \subset X.
\end{equation}
It describes every irreducible non-acyclic character in $X(W)$.
\end{proposition}
\begin{proof}
From \cref{eq:tors} we substitute $z=x+y-2$ in the polynomial ${f}$. 
Then ${f}=0$ becomes
\begin{equation}
\label{eq:P0}
(x+y-1)^2(x-2)(y-2) = 0.
\end{equation}
For if $x=2$, $\tau=0$ becomes $z=y$ in $X$, and similarly, $y=2$ implies $z=x$ in $X$, the last two factor consists of reducible characters in $X \cap \{\tau(x,y,z) = 0\}$. 
Hence the set of irreducible, non-acyclic representations on the surface $X$ is the line $\{ x+y-1=0, \, z=-1 \} $. The corresponding prime ideal is $P = (x+y-1, z+1)$, and the local ring $\left(\C[x,y,z]/(f,\tau)\right)_P$ is isomorphic to $\big(\C[x,y,z]/\left((x+y-1)^2, z+1\right)\big)_P$. 

Now the multiplicity of the torsion is twice the multiplicity of $P$, which is one since it is a transverse intersection of two hyperplanes in $\C^3$.
%The primary ideal $Q = \left((x+y-1)^2, z+1\right)$ has length $2$, and it proves the proposition.
%Moreover, the substitution shows that this line has algebraic multiplicity two, in other words, the torsion vanishes there with multiplicity two.
\end{proof}

%\section{Surgeries on the Whitehead link}
%In this section we consider various surgeries on the Whitehead link, and relate their representation spaces with the vanishing locus of the torsion.
%In \cref{subsec:-3} we prove the second part of \cref{theo:main} and deduce \cref{cor:-3}. Then in \cref{subsec:Twist} we prove \cref{cor:twist}


\subsection{The manifold $M_{(-3,-3)}$}
\label{subsec:-3}
First observe that the equation $z=-1$ for the line $L$ of non-acyclic representation can be interpreted as follows: since $z =  \Tr m\mu$, it means that any (class of) representation $\rho$ in $L$ satisfies $\rho(m\mu)^3=\Id$. Similarly, using the $\SL_2\C$-trace relation $\Tr uv + \Tr u^{-1}v = \Tr u \Tr v$ one sees easily that $-x-y = \Tr(m^2\mu) = -1$ on the line $L$, hence the curve $m^2\mu$ has order 3 as well.
In fact, $(a,b) = (m\mu, m^2\mu)$ is a new generating system for the group $\pi_1(S^3 \setminus W)$, and we further prove the second part of the statement of \cref{theo:main}:
\begin{proposition}
\label{prop:M-3}
The line $L$ is realized as the embedding in $X(W)$ of the character variety $X(W(-3,-3))$ of the closed manifold $W(-3,-3)$ obtained by $(-3,-3)$-surgery on $W$. The fundamental group $\pi_1(W(-3,-3))$ is isomorphic to the triangle group $\Gamma(3,3,\infty)= \langle a,b \mid a^3 = b^3 = 1\rangle \simeq \Z/3 \ast \Z/3$.
\end{proposition}
\begin{proof}
We use the new presentation
\begin{equation}
\label{eq:Pres}
\pi_1(S^3 \setminus W) = \langle a,b \mid  ba^{-3}bab^{-2}a^3b^{-1}a^{-1}b =1 \rangle
\end{equation}
obtained by replacing $(m,\mu)$ by $(ba^{-1}, ab^{-1}a)$ in the relation of \cref{eq:pi1} (compare with \cite[(4)]{GuillouxWill}), and as in \cite[Remark 1]{GuillouxWill} we have the following peripheral basis for $\pi_1(S^3 \setminus W)$:
\[ \mu'=a^{-2}b \,(= \mu^{-1}m^{-1}\mu^{-1} m \mu) , \quad
 \lambda' = a^{-2}bab^{-2}ab,\]
\[ m' = b^{-1}a \,(= \mu^{-1}m^{-1}\mu), \quad
 \ell'=b^{-1}ab^{-1}aba^{-3}ba.\]
We start by performing the $-3$ surgery on the second boundary component: we denote the resulting manifolds by $W(-3)$. For this purpose, 
we add the relation $m'^3=\ell'$ to the presentation \cref{eq:Pres}.
It yields 
\[\pi_1(W(-3)) = \pi_1(S^3 \setminus W)/ \langle\langle m'^3\ell'^{-1} \rangle\rangle = \langle a,b \mid a^3 = b^3 \rangle.\]
Note that the former relator from \cref{eq:Pres} becomes trivial there since $a^3 = b^3$ becomes central.
Now adding the relation $\mu'^3=\lambda'$ gives
\[ \pi_1(W(-3)) /\langle\langle \mu'^3\lambda'^{-1} \rangle\rangle \simeq \pi_1(W(-3,-3)) = \langle a,b \mid a^3 = b^3 = 1\rangle.\]
%Note that this group is the triangle group $\Gamma(3,3\infty)$. 
The character variety $X(\Gamma(3,3,\infty))$ is a complex line: it is known that the character variety of the free group $\langle a,b \rangle$ on two generators is isomorphic to $\C^3$, generated by the traces $\Tr a, \Tr b$ and $\Tr ab$. In $X(\Gamma(3,3,\infty))$ the traces of $a$ and $b$ are fixed to $-1$, and the trace of $ab$ varies freely.

The discussion before \cref{prop:M-3} shows that the traces of $a$ and $b$ in $L$ are precisely $-1$, and it proves the proposition.

\end{proof}

%\subsection{The triangle group $\Gamma(3,3,\infty)$.}
%As showed in \cite{GuillouxWill}, there is a surjective homomorphism $\pi_1(S^3 \setminus W) \to \Gamma(3,3,\infty) = \langle a,b \mid a^3 = b^3 = 1\rangle$.
%It works as follows: we use the automorphism of $\pi_1(S^3 \setminus W)$ given by 
%$$(a,b) = (xy,x^2y), \quad (x,y) = (ba^{-1}, ab^{-1}a) $$
%(note that our generator $x$ is the inverse $x^{-1}$ in \cite{GuillouxWill})
%and it yields the new presentation (equivalent to \cite[(4)]{GuillouxWill})
%\begin{equation}
%\label{eq:Pres}
%\pi_1(S^3 \setminus W) = \langle a,b \mid ba^{-3}bab^{-2}a^3b^{-1}a^{-1}b =1 \rangle
%\end{equation}
%and this group clearly surjects onto $\Gamma(3,3, \infty)$ by taking $a^3=b^3=1$. Note that the relator in the presentation of \cref{eq:Pres} becomes trivial in $\Gamma(3,3,\infty)$.
%
%\begin{proposition}
%The character variety of $\Gamma(3,3,\infty)$ embeds in $\C^3$ as $\{X+Y-1=0,Z=-1\}\cap \{P=0\}$.
%\end{proposition}
%
%\begin{proof}
%The character variety of $\Gamma(3,3,\infty)$ is a line parameterized by $\Tr a = \Tr b = -1$ and $\Tr ab$ is the parameter. Seen as a subvariety of $P=0$, we need to add the relations $\Tr a = \Tr b =-1$. Indeed, $\Tr a = Z$ and it yields the relation $Z=-1$ as claimed. The second relation $\Tr b=-1$ gives then $X+Y-1=0$.
%\end{proof}

\begin{corollary}
The irreducible, non-acyclic representations of the Whitehead link are exactly the irreducible representations that factor through $\Gamma(3,3,\infty)$.
\end{corollary}

%The $\SL_2\C$-character variety of $W$ can be embedded in the set of reducible characters of the $\SL_3(\C)$ character variety of the Whitehead (any $\SL_2\C$ representation acts on $\C^3$, trivially on the third copy of $\C$). The line $L$ consisting of non-acyclic representations lies in the intersection between a component of reducible characters in $X(W, \SL_3(\C))$ and the component consisting of characters of representations factorizing through $\Gamma(3,3,\infty)$ described in \cite{GuillouxWill}. 
%
%\begin{proposition}
%The embedding of the line $L$ in the $\SL_3(\C)$ character variety of $X(W)$ is the intersection of at least 3 irreducible components of $X(W)$:
%a component of reducible representations, the $\SL_3(\C)$ character variety of  $\Gamma(3,3,\infty)$ described in \cite{GuillouxWill} and the geometric component given by the irreducible embedding $\Sym^2\colon X(W, \SL_2\C) \to X(W, \SL_3(\C))$.
%\end{proposition}
%\begin{proof}
%\textcolor{red}{To do, use the computations for the figure-eight knot in \cite{HMP}}
%\end{proof}

%\subsection{Surgeries}
%Using \cite[Remark 1]{GuillouxWill}, we have the following basis for the first homology of the two boundary tori of the Whitehead link complement:
%$$m_1=a^{-2}b, \, l_1 = a^{-2}bab^{-2}ab, \quad m_2 = b^{-1}a, \, l_2=b^{-1}ab^{-1}aba^{-3}ba.$$
%
%\begin{lemma}{\cite[Proposition 4]{GuillouxWill}}
%\label{lem:surg}
%The group $\Gamma(3,3,\infty)$ is the fundamental group of manifold $M_{(-3,-3)}$ obtained by surgery on the exterior $W$ of the Whitehead link.
%
%The line of non-acyclic representations $\{L=0\}$ corresponds to the character variety of $M_{(-3,-3)}$.
%\end{lemma}
%
%\begin{proof}
%We start by performing the $-3$ surgery on the second boundary component: we denote the resulting manifolds by $M_{(\infty, -3)}$. To this purpose, 
%we add the relation $m_2^3=l_2$ to the presentation \cref{eq:Pres}.
%It yields 
%$$\pi_1(M_{(\infty,-3)}) = \pi_1(S^3 \setminus W)/ \langle\langle m_2^3l_2^{-1} \rangle\rangle = \langle a,b \mid a^3 = b^3 \rangle.$$
%Note that the former relator from \cref{eq:Pres} becomes trivial there since $a^3 = b^3$ becomes central.
%Now adding the relation $m_1^3=l_1$ gives
%$$\pi_1(M_{(\infty,-3)}) /\langle\langle m_1^3l_1^{-1} \rangle\rangle \simeq \pi_1(M_{(-3,-3)}) = \Gamma(3,3,\infty).$$
%\end{proof}

\begin{remark}
The character variety of the manifold $W(-3)$ has two components containing irreducible representations. Indeed, any irreducible representation of $\pi_1(W(-3))$ maps the central element $a^3=b^3$ to a central element $\pm I_2$ of $\SL_2\C$. The first component is the line $L$, which corresponds to representations factorizing through $\Gamma(3,3,\infty)$. The second component is described by the relations $\Tr a = \Tr b = 1$, it yields
\[ \{z=1, x-y-1 = 0\} \subset X.\]
\end{remark}

Finally we prove \cref{cor:-3}:
\begin{proof}[Proof of \cref{cor:-3}]
We already proved that $X(W(-3,-3))$ is a complex line. Assume that there is an acyclic irreducible representation $\rho \colon \pi_1(W(-3,-3)) \to \SL_2\C$. In other words, the twisted homology $H_*(W(-3,-3), \C^2_\rho) = \{ 0 \}$.  Now $\rho$ can be seen as a representation in $L$, still denoted by $\rho \colon \pi_1(S^3 \setminus W) \to \SL_2\C$, such that $\rho(m'^3\ell'^{-1}) = \rho(\mu'^3\lambda'^{-1}) = \Id$. Moreover, for all but a finite number of points in $L$, the traces of $\rho(m')$ and $\rho(\mu')$ (which are respectively equal to $x$ and $y$) are different from $2$. 
%It follows that the restriction of $\rho$ to both of the boundary components $\partial_i W$ of $W$ induce acyclic representations $\rho_i \colon \partial_iW \to \SL_2\C$. By a Mayer--Vietoris argument, the representation $\rho \colon \pi_1(S^3 \setminus W) \to \SL_2\C$ should then be acyclic as well, a contradiction since it lies in $L$. 
Using \cref{lem:MV} twice, one sees that any such representation $\rho$ induces an acyclic representation $\varrho \colon \pi_1(S^3 \setminus W) \to \SL_2\C$. But such a representation shall lie in $L$, a contradiction.

Since being acyclic is an open condition (\cref{lem:open}), this extends to all representations of $X(W(-3,-3))$.
\end{proof}
\subsection{Twist knots}
\label{subsec:Twist}
Exteriors of the twist knots $J(2,2n)$ are obtained by surgery $-1/n$ on one boundary component in the exterior of the Whitehead knot, and are denoted by $W(-1/n)$. The resulting character variety is a curve $X_n$ embedded in the character variety of the Whitehead. 
\begin{lemma}
\label{lem:twist}
Non-acyclic representations on $X_n$ are exactly intersection points $X_n \cap L$.
\end{lemma}
\begin{proof}
\cref{lem:MV} implies that irreducible, non-acyclic representations of $W(-1/n)$ are exactly irreducible, non-acyclic representations of $W$ which factor through the $-1/n$-surgery, ie whose character lies in $X_n$, and the lemma follows.
%Let $\rho \colon \pi_1(S^3-J(2,2n)) \to \SL_2\C$ be an irreducible representation of the twist knot $J(2,2n)$. It induces a representation $\rho_L \colon \pi_1(S^3 \setminus W) \to \SL_2\C$ just by composition with the surjective morphism $\pi_1(S^3 \setminus W) \to \pi_1(S^3-J(2,2n))$.
%Let $T_1 \subset \partial W$ denote the surgery torus, we have $\pi_1(T_1) = \langle m_1, l_1 \mid [m_1,l_1] \rangle$. The surgery curve is $m_1 + n\, l_1 \in \pi_1(T_1)$, in other words $\rho(m_1l_1^n) = I_2$. This curve is the meridian of a solid torus $S^1 \times D^2$ which performs the surgery. A longitude for this solid torus is $l_1$, in particular $\pi_1(S^2 \times D^2) = \langle l_1 \rangle$. First note that $\rho(l_1) \neq I_2$, else we would have $\rho(m_1) = I_2$ and $\rho$ would be reducible. It implies that $H_*(S^1\times D^2, \rho|_{\langle l_1 \rangle}) = H_*(T_1, \rho|_{\pi_1(T_1)}) = 0$.
%
%Using a Mayer--Vietoris argument, we show that 
%$$H_*(W, \rho_L) \simeq H_*(J(2,2n), \rho) $$
%in particular $\rho$ is acyclic as a representation of $\pi_1(S^3-J(2,2n))$ if and only if $\rho_L$ is an acyclic representation of $\pi_1(S^3 \setminus W)$. Since $\rho$ is irreducible, the only non-acyclic such representations occur on the line $\{L=0\}$ by \cref{prop:acycl}.
\end{proof}

Denote by $\upsilon = m^{-1}\mu m$ a different meridian of the second component of the Whitehead link, and $\lambda = \upsilon^{-1} \mu \upsilon \mu^{-1}$ the surgered longitude, such that 
\[ \pi_1(M_{-1/n}) = \langle y,\upsilon \mid \lambda^n y = \upsilon \lambda^n \rangle \]
is a presentation of the group of the twist knot $J(2,2n)$.
\begin{proposition}
\label{prop:twist}
The torsion function vanishes with multiplicity at least 2 on the character variety of $X(W(-1/n))$. The zeros correspond to representations which factor through the manifold $W(-1/n, -3)$ obtained by $-3$ surgery on $W(-1/n)$. Each of those representations factor through to the triangle group $\Gamma(3,3,3n-1)$ if $n \ge 1$, and through $\Gamma(3,3, -3n+1)$ if $n \le 1$. These representations are  ${\rm SU}(2)$ representations, and they satisfy $\Tr \rho(\mu) = \Tr \rho(\mu\upsilon)$.
\end{proposition}
Note that in the coordinates $\alpha = \Tr \mu = \Tr \upsilon, \, \beta = \Tr \mu \upsilon$, these points lie on the diagonal 
\[ D=\{ (\alpha, \beta) \mid \alpha = \beta\}.\]

\begin{proof}
The first part of the proposition comes combining \cref{prop:acycl} and \cref{lem:twist}: the vanishing multiplicity might be bigger than two if the intersection $X_n \cap L$ were not transverse.
For the second statement, a non-acyclic representation of $\pi_1(W(-1/n))$ satisfies 
\[ \rho(a^3) = \rho(b^3) = \Id, \quad \rho(\mu') = \rho(\lambda')^n\]
hence using $\rho(a^{-2}) = \rho(a), \rho(b^{-2}) = \rho(b)$, the second relation becomes
\[\rho(ab) = \rho(ab)^{3n}.\]
Such a representation takes indeed value in ${\rm SU}(2)$, since its image is generated by two elliptic elements with elliptic product.

Finally, $\Tr \rho(y) = \Tr \rho(y\upsilon)$ implies $D=xyz-y^2-z^2-x+2 = 0$. Using \verb+elimination_ideal+ in SageMath, we can eliminate the variable $z$ in the intersection $P\cap D$, it yields
\[(X+Y-1)(X-Y-1)(X-2)X.\]
In particular, the line $L=0$ corresponds to the first factor. It proves that representations of the line $L$ factorizing through twist knots lie satisfy $\Tr \rho(\mu) = \Tr \rho(\mu\upsilon)$, and the proposition is proved.
\end{proof}

\section{Twisted Whitehead links}
\label{sec:tW}
We now consider the family of odd twisted Whitehead links. We prove that the character variety is smooth in \cref{subsec:smooth}. In \cref{subsec:vanish} we compute the vanishing multiplicity of the Reidemeister torsion on the character variety of twisted Whitehead link, and prove \cref{theo:TW}. Then in \cref{subsec:double} we prove \cref{coro:doubletwist}.
\medbreak

Denote by $W_k$ the $k$-twisted Whitehead link, and by $M_{W_k}$ its exterior in the 3-sphere. These links, their character varieties, and Reidemeister torsion were studied in detail in~\cite{NguyenTran}.

 A presentation of the fundamental group $M_{W_k}$ is 
\[ \pi_1(M_{W_k})= \langle m, \mu \mid m\omega = \omega m \rangle \] 
where 
\[ \omega = \begin{cases}
(\mu m \mu^{-1} m^{-1})^n m (m^{-1} \mu^{-1} m \mu)^n &\text{ if } k = 2n-1\\
(\mu m \mu^{-1} m^{-1})^n \mu m \mu (m^{-1} \mu^{-1} m \mu)^n &\text{ if } k = 2n
\end{cases}
\] 
and $m,\mu$ are meridians of the two components of $W_k$, see \cref{fig:TwistedW}.
\begin{figure}[h]
\def\svgwidth{0.5\columnwidth}
\input{Twisted.pdf_tex}
\caption{\label{fig:TwistedW} A diagram of twisted Whitehead links, with meridians $m$ and $\mu$. For $k=0$ it is the torus link $T(2,4)$, for $k=1$ it is the Whitehead link.}
\end{figure}

Define the Chebychev polynomials of the second kind by
\begin{equation}
\label{eq:Cheb}
S_0(v) = 1, \, S_1(v) = v, \, S_{k+2}(v) = vS_{k+1}(v) - S_k(v), \quad \text{ for } k \ge 0.
\end{equation}
We start with a simple lemma:
\begin{lemma}
\label{lem:Cheb}
Write $v= w + w^{-1}$, then $S_k(v) = \frac {w^{k+1} - w^{-k-1}}{w-w^{-1}}$.
In particular, the roots of $S_k$ are $v = 2\cos (l \pi/(k+1))$, $1 \le l \le k$
\end{lemma}


In the sequel, we assume $n\ge 2$, since the case $n=1$ is the Whitehead link, which we already studied.

Let $v = x^2 + y^2 + z^2 - xyz - 2$. By \cite{NguyenTran}, the geometric component of the twisted Whitehead link $W_{2n-1}$ is {defined by the polynomial}
\[ {f_n=}\ xy S_{n-1}(v) - (xy-z) S_{n-2}(v) -  z S_n(v). \]
The $n-1$ other components are given by $S_{n-1}(v)=0$.  
\subsection{Smoothness}
\label{subsec:smooth}
In this subsection we prove \cref{prop:smooth}: that the character variety of the odd twisted Whitehead link is smooth.


\subsubsection{The geometric component}
We consider the system
\[\begin{cases}
f =xy S_{n-1}(v) - (xy-z) S_{n-2}(v) -  z S_n(v) =0\\
f_x = (xy S'_{n-1}(v) - (xy-z) S'_{n-2}(v) -  z S'_n(v))(2x-yz) + y (S_{n-1}(v) - S_{n-2}(v))=0\\
f_y = (xy S'_{n-1}(v) - (xy-z) S'_{n-2}(v) -  z S'_n(v))(2y-xz) + x (S_{n-1}(v) - S_{n-2}(v))=0\\
f_z= (xy S'_{n-1}(v) - (xy-z) S'_{n-2}(v) -  z S'_n(v))(2z-xy) -  (S_{n}(v) - S_{n-2}(v))=0\\
%f_v = xy S'_{n-1}(v) - (xy-z) S'_{n-2}(v) -  z S'_n(v) = 0
\end{cases}\]
where $f = f_n$ is the defining polynomial of the geometric component of the character variety, $f_x, f_y, f_z$ stand for the partial derivatives of $f$, and $S_k'(v)$ is the derivative of $S_k$ in the variable $v$. 

We show that this system has no solution.
The proof splits into a disjunction of several cases.

First remark the following:

\begin{equation}
\label{eq:equal}
S_n(v) = S_{n-1}(v) = S_{n-2}(v)
\end{equation}
cannot occur. 
Otherwise, inserting this in \cref{eq:Cheb} yields
\[(2-v) S_{n-1}(v) = 0. \]
One easily sees that $S_k(2)= k+1$, hence $v=2$ contradicts \cref{eq:equal}. 
So  $S_n(v) = S_{n-1}(v) = S_{n-2}(v) =0$. \cref{lem:Cheb} implies that this is impossible: there is no root of unity of simultaneous order $n, n-1$ and $n-2$.

\medbreak

\paragraph{\bf Case 1: $x^2-y^2 \neq 0$}
Then from $x f_x=y f_y=0$, one obtains
\[xy S'_{n-1}(v) - (xy-z) S'_{n-2}(v) -  z S'_n(v)=0 \text{ and } S_{n-1}(v) - S_{n-2}(v)=0,\]
 since $x(2x-yz) -y(2y-xz) = 2(x^2-y^2) \neq 0$.
Now inserting the first equality, one can simplify $f_z = 0$ into $(S_{n}(v) - S_{n-2}(v))=0$, together with the second equality $S_{n-1}(v) - S_{n-2}(v)=0$ we obtain \eqref{eq:equal}, which is impossible.

\medbreak

\paragraph{\bf Case 2: $(x,y) = (0,0)$}

The system becomes

$\begin{cases}
z(S_{n-2}(v) - S_n(v))=0\\
x=0\\
y=0\\
2z^2 (S'_{n-2}(v) - S'_n(v)) - (S_{n-2}(v) - S_n(v)) =0\\
\end{cases}$

If $z=0$, then $v = x^2+y^2+z^2-xyz-2=-2$. Note that $S_n(-2) = (-1)^n (n+1)$ for any $n$, hence $S_n(-2) - S_{n-2}(-2) \neq 0$, a contradiction.

Hence $S_{n-2}(v) - S_n(v)$ which then implies $S'_{n-2}(v) - S'_n(v)=0$. This is prohibited by the fact that the polynomial $S_n - S_{n-2}$ is separable, see \cref{lem:Sep}.

\medbreak

\paragraph{\bf Case 3: $x=y \neq 0$.}

From $f_x=0$ we deduce 
\[(x^2 S'_{n-1}(v) - (x^2-z) S'_{n-2}(v) -  z S'_n(v))(2-z) + (S_{n-1}(v) - S_{n-2}(v)) = 0.\]
From $f=0$ we have $S_{n-1}(v) - S_{n-2}(v) = z (S_{n}(v) - S_{n-2}(v))/x^2$. Hence 
$$
(x^2 S'_{n-1}(v) - (x^2-z) S'_{n-2}(v) -  z S'_n(v))(2-z) + z (S_{n}(v) - S_{n-2}(v))/x^2 =0.
$$ 
From $f_z=0$ we have $(x^2 S'_{n-1}(v) - (x^2-z) S'_{n-2}(v) -  z S'_n(v))(2z - x^2) - (S_{n}(v) - S_{n-2}(v))=0$. This implies that
$$
(x^2 S'_{n-1}(v) - (x^2-z) S'_{n-2}(v) -  z S'_n(v)) \left( (2-z) + (2z-x^2) z/x^2 \right) =0.
$$

If $x^2 S'_{n-1}(v) - (x^2-z) S'_{n-2}(v) -  z S'_n(v)=0$, then $S_{n}(v) - S_{n-2}(v)=0$. This implies that $S_{n-1}(v) - S_{n-2}(v) = z (S_{n}(v) - S_{n-2}(v))/x^2 =0$. But the equations $S_{n-1}(v) - S_{n-2}(v)=0$ and $S_{n}(v) - S_{n-2}(v)=0$ cannot occur simultaneously. 

Hence we have $(2-z) + (2z-x^2) z/x^2 =0$ (i.e. $z^2 - x^2 z + x^2=0$), and then 
\begin{eqnarray*}
&& (x^2 S'_{n-1}(v) - (x^2-z) S'_{n-2}(v) -  z S'_n(v))(2z - x^2) - (S_{n}(v) - S_{n-2}(v)) =0, \\
&& S_{n-1}(v) - S_{n-2}(v) = z (S_{n}(v) - S_{n-2}(v))/x^2.
\end{eqnarray*}
Note that $v = 2x^2 + z^2 - x^2 z -2 = x^2 -2$. Moreover $S_{n}(v) - S_{n-2}(v) \not= 0$. (Otherwise, $S_{n-1}(v) - S_{n-2}(v) = z (S_{n}(v) - S_{n-2}(v))/x^2=0$, same contradiction again.)

Since $x \not= 0$, we have $v \not=-2$. If $v=2$, then from $S_{n-1}(v) - S_{n-2}(v) = z (S_{n}(v) - S_{n-2}(v))/x^2$ we have $z=x^2/2 =2$. This implies that $(x^2 S'_{n-1}(v) - (x^2-z) S'_{n-2}(v) -  z S'_n(v))(2z - x^2) - (S_{n}(v) - S_{n-2}(v)) =0-2 \not=0$, a contradiction. Hence $v \not= \pm 2$.  

Write $x = a + a^{-1}$. Then $v = x^2-2 = a^2 + a^{-2}$. Note that $a^2 \not= \pm 1$. Since $S_k(v) = \frac{a^{2k+2} - a^{-2k-2}}{a^2-a^{-2}}$, we have 
$$z = x^2 \frac{S_{n-1}(v) - S_{n-2}(v)}{S_{n}(v) - S_{n-2}(v)} = x^2 \frac{a^{4n} + a^2}{(a^2+1)(a^{4n} +1)} = x \frac{a^{4n-1} + a}{a^{4n} +1}.$$
Then 
\begin{eqnarray*}
z^2 - x^2 z + x^2 &=& x^2 \left( \frac{(a^{4n-1} + a)^2}{(a^{4n} +1)^2} - (a+a^{-1}) \frac{a^{4n-1} + a}{a^{4n} +1}+ 1\right) \\
&=& x^2 \left( \frac{a^{4n-1} + a}{a^{4n} +1} - a^{-1}  \right)  \left( \frac{a^{4n-1} + a}{a^{4n} +1} - a\right) \\
&=& - x^2 \, \frac{a-a^{-1}}{a^{4n} +1} \, \frac{(a-a^{-1})a^{4n}}{a^{4n} +1} \\
&\not=& 0.
\end{eqnarray*}
This is a contradiction.

\medbreak

\paragraph{\bf Case 4: $x=- y \neq 0$.}

From $f_x=0$ we have  
\[(-x^2 S'_{n-1}(v) + (x^2+z) S'_{n-2}(v) -  z S'_n(v))(2+z) - (S_{n-1}(v) - S_{n-2}(v)) = 0.\]

%By the same argument as in Case 3, we have $x \not=0$.  Then $(-x^2 S'_{n-1}(v) + (x^2+z) S'_{n-2}(v) -  z S'_n(v))(2+z) - (S_{n-1}(v) - S_{n-2}(v)) = 0$. 
From $f=0$ we have $S_{n-1}(v) - S_{n-2}(v) = -z (S_{n}(v) - S_{n-2}(v))/x^2$. Hence 
$$
(-x^2 S'_{n-1}(v) + (x^2+z) S'_{n-2}(v) -  z S'_n(v))(2+z) + z (S_{n}(v) - S_{n-2}(v))/x^2 =0.
$$ 
From $f_z=0$ we have $(-x^2 S'_{n-1}(v) + (x^2+ z) S'_{n-2}(v) -  z S'_n(v))(2z + x^2) - (S_{n}(v) - S_{n-2}(v))=0$. This implies that
$$
(-x^2 S'_{n-1}(v) + (x^2+z) S'_{n-2}(v) -  z S'_n(v)) \left( (2+z) +(2z+x^2) z/x^2 \right) =0.
$$

By the same argument as in Case 3, we have $-x^2 S'_{n-1}(v) + (x^2+z) S'_{n-2}(v) -  z S'_n(v) \not=0$. Then $(2+z) + (2z+x^2) z/x^2 =0$, i.e. $z^2 + x^2 z + x^2=0$. 

We have $v = 2x^2 + z^2 + x^2 z -2 = x^2 -2$ and 
\begin{eqnarray*}
&& (-x^2 S'_{n-1}(v) +(x^2+z) S'_{n-2}(v) -  z S'_n(v))(2z+ x^2) - (S_{n}(v) - S_{n-2}(v)) =0, \\
&& S_{n-1}(v) - S_{n-2}(v) = -z (S_{n}(v) - S_{n-2}(v))/x^2.
\end{eqnarray*}
By the same argument as in Case 3, $v \not= \pm 2$. Write $x = a + a^{-1}$. Then $v= a^2 + a^{-2}$ and 
$z = - x^2 \frac{S_{n-1}(v) - S_{n-2}(v)}{S_{n}(v) - S_{n-2}(v)} = - x \frac{a^{4n-1} + a}{a^{4n} +1}$. 
This implies that 
$$
z^2 + x^2 z + x^2 
= x^2 \left( \frac{(a^{4n-1} + a)^2}{(a^{4n} +1)^2} - (a+a^{-1}) \frac{a^{4n-1} + a}{a^{4n} +1}+ 1\right) \not= 0.
$$
Hence the system $f = f_x = f_y = f_z =0$ has no solutions.


%Inserting $f_v=0$, the system becomes
%
%$\begin{cases}
%f=0\\
%f_v = 0\\
%y (S_{n-1}(v) - S_{n-2}(v))=0\\
%x (S_{n-1}(v) - S_{n-2}(v))=0\\
%S_{n}(v) - S_{n-2}(v)=0\\
%\end{cases}$
%
%There are now two cases:
%\begin{itemize}
%\item
%If $(x,y) \neq (0,0)$, then one has 
%\begin{equation}
%\label{eq:equal}
%S_n(v) = S_{n-1}(v) = S_{n-2}(v).
%\end{equation}
%Inserting this in \cref{eq:Cheb} yields
%\[(2-v) S_{n-1}(v) = 0. \]
%One easily sees that $S_k(2)= k+1$, hence $v=2$ contradicts \cref{eq:equal}. 
%So  $S_n(v) = S_{n-1}(v) = S_{n-2}(v) =0$. \cref{lem:Cheb} implies that this is impossible: there is no root of unity of simultaneous order $n, n-1$ and $n-2$.
%
%\item
%Otherwise $(x,y) = (0,0)$. Now the system becomes
%
%$\begin{cases}
%x=0\\
%y=0\\
%S_{n-2}(v) - S_n(v)=0\\
%z (S'_{n-2}(v) - S'_n(v))=0\\
%\end{cases}$
%If $z=0$, then $v = x^2+y^2+z^2-xyz-2=-2$. Note that $S_n(-2) = (-1)^n (n+1)$ for any $n$, hence $S_n(-2) - S_{n-2}(-2) \neq 0$, a contradiction.
%
%Hence $S'_{n-2}(v) - S'_n(v)=0$, and this together with $S_{n-2}(v) - S_n(v)=0$ is prohibited by the following lemma. This ends the proof that the character variety is smooth.

\begin{lemma}
\label{lem:Sep}
For any $n \ge 2$, the polynomial $S_{n} - S_{n-2}$ has $n$ distinct roots.
\end{lemma}
\begin{proof}
%Note that $S_{-k}(v) = - S_{k-2}(v)$ for all $k \in \Z$. Moreover $S_0(v) = 1$ and $S_1(v) =v$. 
%
%Since $P_0(v)=2$, $P_{1}(v)= v$ and $P_{-n}(v) = S_{-n}(v) - S_{-n-2}(v) = - S_{n-2}(v) + S_n(v) = P_n(v)$, it suffices to show that $P_n(v) \in \C[v]$ is separable for $n \ge 2$. 
%
When $n \ge 2$, by \cref{lem:Cheb}, $S_{n} - S_{n-2}$ is a monic polynomial of degree $n$ in $v$. 

Assume $v = 2\cos \frac{(2l+1)\pi}{2n}$, $0 \le l \le n-1$, then since 
\[S_k(v) = \frac{\sin \frac{(k+1)(2l+1)\pi}{2n}}{\sin \frac{(2l+1)\pi}{2n}}\] 
we have 
\[
S_{n}(v) - S_{n-2}(v) = \frac{\sin \frac{(n+1)(2l+1)\pi}{2n} - \sin \frac{(n-1)(2l+1)\pi}{2n} }{\sin \frac{(2l+1)\pi}{2n}} =0.
\]
Hence the polynomial \[S_{n}(v) - S_{n-2}(v) = \prod_{l=0}^{n-1} \left( v -  2\cos \frac{(2l+1)\pi}{2n} \right)\] is separable. 
\end{proof}
%\end{itemize}

\subsubsection{The other components}
There are $n-1$ other components, given by $S_{n-1}(v) = 0$. By \cref{lem:Cheb} this is equivalent to $v=2\cos (l \pi/(n))$, $1 \le l \le n-1$. 

Hence we consider the system

\[
\begin{cases}
v = x^2 + y^2 + z^2 -xyz-2 = \cos (l \pi/(n)) \\
2x-yz=0\\
2y-xz=0\\
2z-xy=0\\
\end{cases}
\]
A straightforward computation shows that it has no solutions.

\subsection{Vanishing multiplicity}
\label{subsec:vanish}
Now we compute the vanishing multiplicity of the torsion.
%
Let $P_k(v) = (S_{k+1}(v) - S_k(v)-1)/(v-2)$. 
%
The Reidemeister torsion is 
\[ \tau_n = (2-x-y+z)S_{n-1}(v) + (4-2x-2y+xy)P_{n-2}(v) \]
(see \cite[Corollary 2]{NguyenTran}).
%
The divisor of the Reidemeister torsion $\tau_n$ on the character variety of odd twisted Whitehead links is given by {the ideal %an ideal generated by the polynomials
%generated by the following 3 equations in 4 variables $x, y, z, v$:
\[(x^2 + y^2 + z^2 - xyz - 2 - v,  \quad %, \label{v}\\
f_n \, S_{n-1}(v),  \quad %,  \label{g}\\
\tau_n),\]
where $f_n$ and $\tau_n$ are regarded as elements of $\C[x,y,z,v]$. 
} %\label{t}.
%\begin{eqnarray}
%x^2 + y^2 + z^2 - xyz - 2 - v , \label{v}\\
%f_n(x,y,z,v) \, S_{n-1}(v) ,  \label{g}\\
%\tau_n(x,y,z,v) \label{t}.
%\end{eqnarray}

\subsubsection{The case $S_{n-1}(v)=0$}
\label{subsec:case1}
In this subsection we first consider the ideal 
\[(x^2 + y^2 + z^2 - xyz - 2 - v, \quad S_{n-1}(v), \quad \tau_n).\]
%case $S_{n-1}(v)=0$, i.e. 
Using \cref{lem:Cheb} we rewrite 
\[S_{n-1}(v) = \prod_{k=1}^{n-1} (v - 2\cos (k\pi/n))\]
It corresponds to non-geometric components of the character variety of twisted Whitehead links $W_{2n-1}$. We prove that the torsion vanishes there with multiplicity one. 
\medbreak
The non-geometric components of the character variety $X(M_{W_{2n-1}})$ are indexed by integers $1 \le l \le |n| -1$. For each such $l$, there is an algebraic surface of characters corresponding to representations that send the commutator $[m,\mu]$ on a non-trivial elliptic element of order dividing $2n$.

\begin{proposition}\label{prop:nongeo}
The torsion vanishes on the whole component when $k$ is even, and on a line in this component when $k$ is odd. The vanishing multiplicity is one.
\end{proposition}


Since $P_{n-2}(v) = (S_{n-1}(v) - S_{n-2}(v)-1)/(v-2)$ and 
\[ S_{n-2}(v) = \frac{\sin \frac{k(n-1)\pi}{n}}{\sin \frac{k\pi}{n}} = (-1)^{k-1},\] 
we have $P_{n-2}(v) = \frac{(-1)^k-1}{2-v}$. 
Note that, after localization if needed, $(2-v)$ is invertible since we exclude the component of reducible representations.

\begin{proof}[Proof of \cref{prop:nongeo}]
The proof divides into two cases depending on whether $k$ is even or odd.

\medbreak

\paragraph{\bf If $k$ is even,} 
then %polynomial \eqref{t} 
{$\tau_n$} 
becomes trivial and the ideal is presented by $x^2+y^2+z^2-xyz-v$ and $v - 2\cos (k\pi/n)$. Hence any representation on the component $v = 2\cos (k\pi/n)$, where $k$ is even, is non-acyclic. 

\medbreak

\paragraph{\bf If $k$ is odd,} 
then $P_{n-2}(v) \not=0$ and so %polynomial \eqref{t} 
{$\tau_n$} is equivalent to $(2-x)(2-y)$. 
If $x=2$, then the equation %\eqref{v} \red{$=0$} 
{$x^2 + y^2 + z^2 - xyz - 2 - v =0$} 
becomes $(y-z)^2=v-2 = - 4\sin^2(\frac{k\pi}{2n})$, which is equivalent to $y-z= \pm 2 \sin(\frac{k\pi}{2n})\sqrt{-1}$. 

Hence a representation on the component $v = 2\cos (k\pi/n)$, where $k$ is odd, is non-acylic iff it belongs to one of the lines
\[ \{x=2, \, y-z= \pm 2 \sin(\frac{k\pi}{2n})\sqrt{-1}, v = 2\cos (k\pi/n)\},\]
\[ \{y=2, \, x-z= \pm 2 \sin(\frac{k\pi}{2n})\sqrt{-1}, v = 2\cos (k\pi/n)\}.\]

Any non-acyclic representation on the component $v = 2\cos (k\pi/n)$, where $k$ is even, has multiplicity $1$.
\end{proof}

\subsubsection{The case $S_{n-1}(v) \neq 0$: the geometric component}
\label{subsec:case2}
We now consider the case 
\[%xy S_{n-1}(v) - (xy-z) S_{n-2}(v) -  z S_n(v)
{f_n}=0\]
(the geometric component) and $S_{n-1}(v) \not=0$.
We prove that the torsion vanishes with multiplicity 2 on this component. 
We consider the ideal generated by the polynomials
\begin{align*} x^2 + y^2 + z^2 - xyz - 2 - v,  \\
f_n = xy S_{n-1}(v) - (xy-z) S_{n-2}(v) -  z S_n(v),\\
\tau_n = (2-x-y+z)S_{n-1}(v) + (4-2x-2y+xy)P_{n-2}(v) )
\end{align*}
{in $\C[x,y,z,v]$.} 

For this, we work in the localization of $\C[x,y,z,v]$ along any prime ideal underlying the primary decomposition of this ideal. In particular, one can check that the polynomials $S_{n-1}(v), \, v-2, T_n(v)-2, \,  2 T_n(v) -v +2$ appearing in the following computation are invertible.

Let $\dot{=}$ denote the equality up to multiplication by units in the localized ring.

\medbreak

% Rewriting \eqref{t} we can express $z$ in terms of $x,y,v$:
%\[
%z = (x + y -2) \left(1+ \frac{2P_{n-2}(v)}{S_{n-1}(v)}\right)- xy\frac{P_{n-2}(v)}{S_{n-1}(v)} .
%\]
%
%We re-write \eqref{g} as 
%\begin{multline*}
%xy (S_{n-1}(v)  - S_{n-2}(v)) -  z (S_n(v) - S_{n-2}(v) )=0\\
%\Leftrightarrow xy (S_{n-1}(v)  - S_{n-2}(v)) -
% \left( (x + y -2)(1+ \frac{2P_{n-2}(v)}{S_{n-1}(v)})- xy\frac{P_{n-2}(v)}{S_{n-1}(v)} \right)(S_n(v) - S_{n-2}(v) )=0\\
%\Leftrightarrow xy \left( S_{n-1}(v)  - S_{n-2}(v) + \frac{P_{n-2}(v)}{S_{n-1}(v)}  (S_n(v) - S_{n-2}(v) )\right) \\ -
%(x + y -2)\left(1+ \frac{2P_{n-2}(v)}{S_{n-1}(v)}\right) (S_n(v) - S_{n-2}(v) )=0.
%\end{multline*}
%
%Write $v = a+ a^{-1}$. Then $S_k(v) = (a^{k+1} - a^{-k-1})/(a-a^{-1})$. By a direct calculation, the above equation becomes
%\[(2a^n + 2 a^{-n} - a - a^{-1} +2) xy - (a^n + a^{-n})(2+a+a^{-1})(x+y-2)=0.\]
%Hence 
%\[ xy = \frac{(a^n + a^{-n})(2+a+a^{-1}) }{2a^n + 2 a^{-n} - a - a^{-1} +2} (x+y-2).\]
%Then
%\[
%z = (x + y -2)\left(1+ \frac{2P_{n-2}(v)}{S_{n-1}(v)}\right)- xy\frac{P_{n-2}(v)}{S_{n-1}(v)}=\frac{a^n + a^{-n}+a^{n-1} + a^{1-n} }{2a^n + 2 a^{-n} - a - a^{-1} +2} (x+y-2).
%\]
%
%Equation \eqref{v} can be written as $(x+y)^2 + z^2 -2-v- xy (z+2)$, i.e. 
%\begin{eqnarray*}
%&& (x+y)^2 + \left(\frac{a^n + a^{-n}+a^{n-1} + a^{1-n} }{2a^n + 2 a^{-n} - a - a^{-1} +2}\right)^2 (x+y-2)^2 -2-(a+a^{-1}) \\
%&&- \frac{(a^n + a^{-n})(2+a+a^{-1}) }{2a^n + 2 a^{-n} - a - a^{-1} +2} (x+y-2) \left(\frac{a^n + a^{-n}+a^{n-1} + a^{1-n} }{2a^n + 2 a^{-n} - a - a^{-1} +2} (x+y-2)+2 \right) .
%\end{eqnarray*}
%This equation can be factored as
%$(a+a^{-1}-2) \left( x + y - \frac{a+a^{-1} +2}{a^n+a^{-n}+2}\right)^2$, i.e.
%\[(v-2)\left( x + y - \frac{v+2}{T_n(v)+2}\right)^2.\]

First, we have 
\[ \tau_n\,\dot{=}\, (x + y -2) \left(1+ \frac{2P_{n-2}(v)}{S_{n-1}(v)}\right)- xy\frac{P_{n-2}(v)}{S_{n-1}(v)} - z.\] 
Hence, modulo $\tau_n(x,y,z,v)$ in the local ring, 
%in the localization of $C[x,y,z,v]/(\tau_n(x,y,z,v))$, 
we have 
\begin{align*} f_n=&\ xy (S_{n-1}(v)  - S_{n-2}(v)) -  z (S_n(v) - S_{n-2}(v) )\\
=&\ xy (S_{n-1}(v)  - S_{n-2}(v)) -
 \left( (x + y -2)(1+ \frac{2P_{n-2}(v)}{S_{n-1}(v)})- xy\frac{P_{n-2}(v)}{S_{n-1}(v)} \right)(S_n(v) - S_{n-2}(v) )\\
= &\ xy \left( S_{n-1}(v)  - S_{n-2}(v) + \frac{P_{n-2}(v)}{S_{n-1}(v)}  (S_n(v) - S_{n-2}(v) )\right) \\ 
 & -
(x + y -2)\left(1+ \frac{2P_{n-2}(v)}{S_{n-1}(v)}\right) (S_n(v) - S_{n-2}(v) ).
\end{align*}
Write $v = a+ a^{-1}$. Then $S_k(v) = (a^{k+1} - a^{-k-1})/(a-a^{-1})$. By a direct calculation, we further obtain 
\begin{align*} f_n=&\ (2a^n + 2 a^{-n} - a - a^{-1} +2) xy - (a^n + a^{-n})(2+a+a^{-1})(x+y-2)\\
\dot{=}&\ xy- \frac{(a^n + a^{-n})(2+a+a^{-1}) }{2a^n + 2 a^{-n} - a - a^{-1} +2} (x+y-2).
\end{align*}
Thus, modulo this element, we have 
%Thus, in the localization of $\C[x,y,z,v]/(f_n)$, we have 
% $\tau_n(x,y,z,v)$ in the ideal of $\C[x,y,z]$ may be replaced by 
\[\tau_n(x,y,z,v)\,\dot{=}\,\frac{a^n + a^{-n}+a^{n-1} + a^{1-n} }{2a^n + 2 a^{-n} - a - a^{-1} +2} (x+y-2).\] 
Therefore, modulo these elements, we have 
\begin{align*}
\ \ \ &x^2 + y^2 + z^2 - xyz - 2 - v\\ 
&=(x+y)^2 + z^2 -2-v- xy (z+2)\\
&=x^2 + y^2 + z^2 - xyz - 2 - v\\
&=(x+y)^2 + \left(\frac{a^n + a^{-n}+a^{n-1} + a^{1-n} }{2a^n + 2 a^{-n} - a - a^{-1} +2}\right)^2 (x+y-2)^2 -2-(a+a^{-1})\\
&=- \frac{(a^n + a^{-n})(2+a+a^{-1}) }{2a^n + 2 a^{-n} - a - a^{-1} +2} (x+y-2) \left(\frac{a^n + a^{-n}+a^{n-1} + a^{1-n} }{2a^n + 2 a^{-n} - a - a^{-1} +2} (x+y-2)+2 \right) \\
&=(a+a^{-1}-2) \left( x + y - \frac{a+a^{-1} +2}{a^n+a^{-n}+2}\right)^2\\
&=(v-2)\left( x + y - \frac{v+2}{T_n(v)+2}\right)^2.
\end{align*}


Here $T_k(v)$'s are the Chebyshev polynomials of the first kind defined by $T_0(v) =2$, $T_1(v) =v$ and $T_k(v) = v T_{k-1}(v) - T_{k-2}(v)$ for all $k \in \Z$. Note that $T_k(a+a^{-1})=a^k + a^{-k}$. 

Note that $v-2=0$ corresponds to reducible representations. We have shown that non-acyclic representations on the geometric component (and not on other components $v= 2\cos(k\pi/n)$) are given by an ideal generated by the following 3 polynomials in 4 variables $x, y, z, v$:
\begin{eqnarray*}
z - \frac{T_n(v) + T_{n-1}(v) }{2T_n(v) - v+2} (x+y-2),\\
xy - \frac{T_n(v)(2+v) }{2T_n(v) - v+2} (x+y-2), \\
 \left( x + y - \frac{v+2}{T_n(v)+2}\right)^2.
\end{eqnarray*}


So we have expressed the subvariety $(f_n, \tau_n)$ of non-acyclic representations on the geometric component of $X(M_{W_{2n-1}})$ as the intersection $(g,h^2)$, for two polynomials $g,h \in \C[x,y,v]_{(f_n,\tau_n)}$, given by
\[g= \frac{T_n(v)(2+v) }{2T_n(v) - v+2} (x+y-2)-xy, \quad h = x + y - \frac{v+2}{T_n(v)+2}.\]

Since the multiplicity is additive \cite[Appendix A, Theorem 1.1]{Hartshorne}, the multiplicity of the ideal $(g,h^2)$ along any irreducible component is twice the multiplicity of $(g,h)$. Hence we are led to show that the latter is one.
%To conclude the proof, we need to show that the ideal $(a,b)=P$ of the local ring $\C[x,y,v]_P$ is the intersection of prime ideals $P_1\cap \ldots \cap P_i$\footnote{We don't necessarily have a prime ideal $P$, since the vanishing locus of the torsion need not to be irreducible. But the notion of multiplicity can be applied for each irreducible components of the intersection.}. This will imply that $Q=(a, b^2) = Q_1\cap \ldots Q_i$ is the intersection of primary ideals, each of them being of multiplicity two.
To this purpose, we show that the intersection $X_g \cap X_h$ is generically transverse.

We compute the Jacobian matrix $\mathcal J(g,h)$, note that $T_n'(v) = n S_{n-1}(v)$. After multiplying by some polynomial inversible in the local ring $(C[x,y,v]/(g,h))_{(g,h)}$, we obtain
\[ \mathcal J(g,h)= \bma (2T_n(v) - v+2)y - T_n(v)(2+v) &T_n(v)+2 \\
 (2T_n(v) - v+2)x - T_n(v)(2+v) & T_n(v)+2 \\
 \partial_v g & \partial_v h\ema\] 
and the first $2\times 2$ minor is
$M = (T_n(v) +2) (x-y)(2T_n(v)-v+2)$.
One can check that this is non-zero apart on a strict subvariety of $X_g \cap X_h$, and this proves that the intersection $X_g \cap X_h$ is generically transverse.

Hence the torsion vanishes on the geometric component of the twisted Whitehead link with multiplicity two.

\subsection{Double twist knots $J(2m,2n)$}
\label{subsec:double}
In this section we consider $(-1/m)$-surgery on the twisted Whitehead link $W_{2n-1}$. The resulting knots are the genus one two-bridge knot $J(2m,2n)$. We deduce from \cref{theo:TW} that the torsion vanishes with multiplicity at least two on their character variety, proving \cref{coro:doubletwist}.

\medbreak

Suppose $\rho: \pi_1(S^3 \setminus W_{2n-1})  \to \mathrm{SL}_2(\C)$ is a nonabelian representation of the form 
\[
\rho(a) = \left[ \begin{array}{cc}
s_1 & 1 \\
0 & s_1^{-1} \end{array} \right] \quad \text{and} \quad 
\rho(b) = \left[ \begin{array}{cc}
s_2 & 0 \\
u & s_2^{-1} \end{array} \right]
\]
where 
\[
\big( xy S_{n-1}(v) -  (xy-z)  S_{n-2}(v) - z S_n(v) \big) S_{n-1}(v) = 0.
\]
Here $x = s_1 + s_1^{-1}$, $y = s_2 + s_2^{-1} $, $z = s_1 s_2 + s_1^{-1} s_2^{-1} + u$ and $v = x^2 + y^2 + z^2 - xyz-2$ are the traces of the images of $a$, $b$, $ab$ and $bab^{-1}a^{-1}$ respectively.

We first claim that if $\rho$ satisfies $S_{n-1}(v)=0$, then it cannot be extended to a representation $\pi_1(S^3 \setminus J(2m,2n)) \to \mathrm{SL}_2(\C)$. Indeed, let $\lambda_a$ be the canonical longitude corresponding to the meridian $a$ of $W_{2n-1}$. By \cite{Tran1}  we have
\[ \rho(\lambda_a) = \left[ \begin{array}{cc}
l_a & * \\
0 & l_a^{-1} \end{array} \right],\]
where 
\begin{equation} \label{l_a}
l_a  = 
\begin{cases}
1 & \text{if } \, S_{n-1}(v) = 0,\\
\frac{s_1 y - z}{-s_1^{-1}y+z} & \text{if } \, xy S_{n-1}(v) -  (xy-z)  S_{n-2}(v) - z S_n(v) = 0 \text{ and } S_{n-1}(v) \not=  0,
\end{cases}
\end{equation} 
and
\begin{equation} \label{*}
*  = 
\begin{cases}
0 & \text{if } \, S_{n-1}(v) = 0,\\
\frac{y(xy-2z)}{xyz-y^2-z^2} & \text{if } \, xy S_{n-1}(v) -  (xy-z)  S_{n-2}(v) - z S_n(v) = 0 \text{ and } S_{n-1}(v) \not=  0.
\end{cases}
\end{equation} 
Note that there is a small error in \cite{Tran1} : the canonical longitude in \cite{Tran1}  is actually the inverse of the canonical longitude. 

If $S_{n-1}(v) = 0$, then $l_a =1$ and $* = 0$ by \eqref{l_a} and \eqref{*} respectively. This means that $\rho(\lambda_a) = \Id$. Since $\rho(a) \not=\Id$, we obtain $\rho(a \lambda^{-m}_a) \not= \Id$, which means that $\rho: \pi_1(S^3 \setminus W_{2n-1}) \to \mathrm{SL}_2(\C)$ on the components $v= 2\cos(k\pi/n)$ cannot be extended to a representation $\pi_1(S^3 \setminus J(2m,2n)) \to \mathrm{SL}_2(\C)$.

Since any non-acyclic representation $\rho: \pi_1(S^3 \setminus W_{2n-1}) \to \mathrm{SL}_2(\C)$ satisfying $xy S_{n-1}(v) -  (xy-z)  S_{n-2}(v) - z S_n(v) = 0$ and $S_{n-1}(v) \not=0$ has multiplicity $2$, it proves the corollary. 

%\begin{theorem}
%Any non-acyclic representation $\tilde{\rho}: \pi_1(S^3 \setminus J(2m,2n)) \to \mathrm{SL}_2(\C)$ has multiplicity $2$. 
%\end{theorem}

Note that $\Tr \rho(\lambda_a) \not= 2$, since $l_a \not= 1$. (If $l_a =1$ then $\frac{s_1 y - z}{-s_1^{-1}y+z} =1$, which means that $xy-2z=0$. So $*=0$ and $\rho(a) \not=\Id$, a contradiction.)



\section{$L$-functions of universal deformations} 
\label{sec:L}
In this section, we pursue the study of the $L$-functions of universal deformations, raised in a viewpoint of number theory. 
The main goal is to interpret the multiplicity two phenomenon to the $L$-functions of the odd twisted Whitehead links $W_{2n-1}$.  

\subsection{Multiplicity two} 
First we show the following paraphrasing for ``multiplicity at least two''. 
\begin{lemma} \label{lem.mult-two}
Let $f, g\in \C[x,y,z]$, assume that $f$ is irreducible, and let $(a,b,c)$ be a common zero of $f$ and $g$. 
Assume $\partial_z f (a,b,c)\neq 0$ and let $z_f$ denote the implicit function of $f=0$ at $(a,b,c)$. 
If the point $(a,b,c)$ is a common zero of the divisors $(f)$ and $(g)$ with multiplicity two, 
then the point $(a,b)$ is a zero of $g(x,y,z_f(x,y))$ with at least multiplicity two 
in that sense that the 2nd Taylor polynomial at $(a,b)$ vanishes. 
%\footnote{It would be better if we could remove ``at least'' and insert ``but the 3rd does not''.} 
\end{lemma} 
\begin{proof}
Consider the divisor given by $(f,g)$ in the variety $\{f=0\}$. Since this divisor has multiplicity two, locally $g$ can be written as a square mod $f$. In particular its second Taylor polynomial vanishes on $\{f=g=0\}$.
\end{proof}

%\begin{comment}
%Then a common zero $(a,b,c)$ of the divisors $(f)$ and $(g)$ has multiplicity exactly two 
%iff the following equivalent conditions hold 
%\begin{itemize}
%\item The 2nd Taylor polynomials of $f$ and $g$ are parallel but the 3rd do not. 
%\item $(f,g)=(h^2,k)$ for some $h,k$ such that $(h)$ and $(k)$ intersect transversely at $(a,b,c)$.  
%\item (Assume that $\partial_z f (a,b,c)\neq 0$ and let $z_f$ denote the implicit function of $f=0$ at $(a,b,c)$. Then,) the point $(a,b)$ is a zero of $g(x,y,z_f(x,y))$ with multiplicity exactly two. 
%\end{itemize} 
%\end{lemma} 
%\end{comment} 


\subsection{Universal deformations}

Let $\pi$ be the group of $W_{2n-1}$. Let $\F$ be any finite field with characteristic $p>2$ and $O$ a CDVR with $O/\mf{m}_O=\F$. 
Then the conjugacy classes of absolutely irreducible $\SL_2\F$-representations
correspond to  points of a component of the character variety $\ol f_n(x,y,z)=0$ in $\F^3$. 
(This assertion holds by the fact that the Brauer group of a finite field is trivial and \cite[Proposition 3.4]{Marche-RIMS2016}.) 

Let {$\ol{\rho}:\pi\to \SL_2\F$ be at} a zero $(\ol{a}, \ol{b}, \ol{c}) \in \F^3$ of $\ol f_n$, %\frac{\partial f}{\partial z} 
%(\ol{a}, \ol{b}, \ol{c})\neq 0$, 
let $(a,b)\in O^2$ be a lift of $(\ol{a},\ol{b})$. %and put $\mca{R}=O[\![x-a,y-b]\!]$. 
%By \cite[Corollary 3.2.3]{MTTU2017}, $\mathcal R$ is the universal deformation ring of $\ol \rho$.
Suppose in addition that $\partial_z  f_n (\ol{a}, \ol{b}, \ol{c})\neq 0$. 
Then, Hensel's lemma for multivariable functions yields the implicit function $z_{f_n}(x,y)$ of $ f_n(x,y,z)=0$ around $(a,b)$ and a natural map $O[x,y,z]\mapsto O[\![x-a,y-b]\!]; z\mapsto z_{f_n}(x,y)$. 

\begin{lemma} %Let $\ol{\rho}:\pi\to \SL_2\F$ be as above and
%If $\ol{\rho}$ is at $(\ol{a},\ol{b},\ol{c})$, then 
{There is a} universal deformation ${\bs \rho}:\pi\to \SL_2\mca{R}_{\ol{\rho}}$ of $\ol{\rho}$ over $O$ with $\mca{R}_{\ol{\rho}}=O[\![x-a,y-b]\!]$. 
\end{lemma} 
\begin{proof}
{The image $\bs{\rho}^R$} of so-called Riley's representation via Hensel's map is a representation over a quadratic extension of $O[\![x-a,y-b]\!]$ such that every deformation of $\ol{\rho}$ over $O$ factors through it. 
Since the image of ${\rm tr}{\bs \rho}^R$ is in $O[\![x-a,y-b]\!]$, Nyssen and Carayol's theorems (\cite[Theorem 1]{Nyssen1996}, \cite[Theorem 1]{Carayol1994}) assures that ${\bs \rho}^R$ is strictly equivalent to a representation ${\bs \rho}$ over $O[\![x-a,y-b]\!]$. By \cite[Corollary 3.2.3]{MTTU2017}, the universal ring is $\mca{R}_{\ol{\rho}}=O[\![x-a,y-b]\!]$ and this ${\bs \rho}$ is a universal deformation of $\ol{\rho}$ over $O$. 
\end{proof} 
%The image ρ^R of Riley's rep via Hensel's map is a rep over a quad ext of O[[…]] s.t. every deformation of \ol{rho} factors through it. Since tr ρ^R is in O[[…]], Nyssen's thm and Carayol's thm yield that ρ^R is strictly equivalent to a rep {\bs \rho} over O[[…]]. By [MTTU 3.2.3], this {\bs \rho} is a universal deformation. 

%By \cite[Theorem 2.2.2]{MTTU2017}, %\footnote{I was a bit confused; The existence of a universal deformation is proved for any finitely generated group by \cite[Theorem 2.2.2]{MTTU2017}.} 
%Riley's representation assures that there exists a universal deformation ${\bs \rho}:\pi\to \SL_2(\mca{R}_{\ol{\rho}})$ of $\ol{\rho}$ with $\mca{R}_{\ol{\rho}}=O[\![x-a,y-b]\!]$. 
%We remark that ${\bs \rho}$ is strictly equivalent to the image ${\bs \rho}^R$ of so-called Riley's representation via {Hensel's} map, % in a generic case, while 
%Note that ${\bs \rho}^R$ 
%which is defined over a quadratic extension of $\mca{R}$ in general. 
Another construction of ${\bs \rho}$ may be given in a similar way to \cite{MTTU2017}, \cite{KMTT2018}, or \cite{RTange2023ProcLDTNT} using the tautological representation introduced in \cite{Benard2020OJM}. 

%Another idea for numerical study would be to use the tautological representation (cf. \cite{Benard2020OJM, RTange2023ProcLDTNT}. 

%Let ${\bs T}:\pi\to \mca{R}$ denote the image of Riley's character via the natural map. Then by Nyssen and Carayol's theorems (\cite[Theorem 1]{Nyssen1996}, \cite[Theorem 1]{Carayol1994}), there exists a deformation ${\bs \rho}:\pi\to \SL_2(\mca{R})$ of $\ol{\rho}$ such that ${\bs \rho}$ is equivalent to the image of Riley's representation and ${\rm tr}{\bs \rho}={\bs T}$ holds (cf. \cite{MTTU2017}). 
%{\footnotesize \blue{(Exactly the same argument as in the previous papers.)}} 
%By a similar argument to \cite[Theorem 2.2.4]{KMTT2018}, this ${\bs \rho}$ is in fact a universal deformation of $\ol{\rho}$. %{\footnotesize \blue{(I should check whether we should contain a proof.)}} %\footnote{If we contain a proof, then we need to define the (universal) deformations, and it can be just troublesome.} 

%\begin{remark} Since the Brauer group of $\mca{R}$ is not trivial, a criterion \cite[Proposition 3.4]{Marche-RIMS2016} is not applicable to obtain ${\bs \rho}$. In addition, the deformation problem is not necessarily unobstructed in the sense of Mazur; in the case of a knot group representation $\ol{\rho}$, we have $H^2({\rm Ad}(\ol{\rho}))\neq 0$ \cite[Theorem 2.3.2]{KMTT2018}. Thus, the existence of such a universal deformation is theoretically non-trivial. \end{remark} 

\subsection{$L$-functions}
%Above. 
%\blue{\ul{Claim.} 
{
We start with the following lemma:
\begin{lemma}
Let 
%$\rho \colon \pi \to SL_2\C$ irreducible be an integral point of the geometric component, and 
$\ol \rho \colon \pi \to \SL_2\F$ an absolutely irreducible representation in the geometric component.
The universal representation $\bs {\rho} \colon \pi \to \SL_2\mathcal R$ is rationally acyclic, that is the $\mathcal R$-modules $H_i(\pi, \bs \rho)$ are torsion modules.
\end{lemma}
Note that the homology $H_i(\pi, \bs \rho)$ is naturally isomorphic to the cellular homology $H_i(W_{2n-1}, \bs \rho)$, in this section we use the former notation.
\begin{proof}
Assume that there is a free factor isomorphic to $\mathcal R^k$ in $H_i(\pi,\bs \rho)$, for some $k>0$.
This yields a free factor in the $\Frac \mathcal R$-vector space $H_i(\pi, \bs \rho \otimes \Frac \mathcal R)$, where $\bs \rho \otimes \Frac \mathcal R$ is the same representation, but with coefficients seen in $\Frac \mathcal R$.

Now consider the $\F$-character variety (precisely the geometric component). It contains the set of non-acyclic representations as a strict algebraic subset. Take any acyclic representation $\ol \rho' : \pi \to \SL_2\F$, and consider its universal deformation $(\bs \rho', \mca R')$.
Note that the fraction fields $\Frac \mca R$ and $\Frac \mca R'$ are isomorphic (both $\mca R$ and $\mca R'$ are power series in two variables over $O$), and that $\bs \rho \otimes \Frac \mathcal R$ and $\bs \rho \otimes \Frac \mathcal R'$ are then conjugate representations.
Hence $\bs \rho'$ is non-acyclic. Tensoring by $\F$ and applying the universal coefficients theorem, we deduce that $\ol \rho'$ is non-acyclic, a contradiction.
%Since there is a dense Zariski open subset of the geometric component which consists of acyclic repr\UTF{00E9}sentations, there is a deformation $\varrho \colon \pi \to R$ of $\ol \rho$, with $R$ an $O$ algebra, which is acylic \footnote{\red{This is what one should find an argument for, but I'm confident it is true}}. Then one can apply \cite[Theorem 3.2.4]{KMTT2018}, since $\mathcal R$ is clearly a noetherian integral domain (it is isomorphic to the ring of formal power series in two variables on $O$) and the representation $\ol \rho$ is irreducible. The Lemma follows directly.
\end{proof}
}

%The calculation of the $\SL_2$-acyclic torsion function assures that the twisted homology groups $H_i(\bs{\rho})$ are torsion $\mca{R}_{\bs{\rho}}$-modules. \red{Give a precise ref?}
%{\footnotesize (Maybe we should contain a minimal proof for this.)}} 

%\magenta{We claim that the twisted homology groups $H_i(\bs{\rho})$ are torsion modules.} 
%{\footnotesize \blue{(Maybe we should contain a minimal proof for this.)}} 
Following \cite{KMTT2018}, we define \emph{the $L$-function} $L_{\bs \rho}$ of the universal deformation $\bs{\rho}$ of $\ol{\rho}$ over $O$ to be the order of the 1st twisted homology group $H_1(\pi, \bs{\rho})$. 
We may easily verify that the 0th twisted homology $H_0(\pi, \bs{\rho})$ is trivial. %Delta_0(\bs{\rho})=1$. 
By the standard exact sequence of the Fox derivative, 
%if the 0th twisted homology $H_0(\bs{\rho})$ is trivial, which is easily verified for $W_k$, then 
%if the 0-th Alexander polynomial satisfies $\Delta_{{\bs \rho},0}\,\dot{=}\,1$, then 
we have 
%the $L$-function of the universal deformation of $\ol{\rho}$ is 
$L_{\bs \rho}\,\dot{=}\,\tau_n(x,y,z_{f_n}(x,y))\neq 0$ in $O[\![x-a,y-b]\!]$, where $\dot{=}$ denotes the equality up to multiplication by units. 
%\red{Please explain the notation $\dot{=}$}
%\blue{Assume that the 0-th Alexander polynomial satisfies $\Delta_{{\bs \rho},0}\,\dot{=}\,1$, which holds for most cases.}\footnote{\blue{Do we have this for $\ol{\rho}$ of $W_k$?}} 
In this case, since 
%By the procedure of the calculation with use of Fox derivative we see that 
$L_{\bs \rho}\, {\rm mod}\, \mf{m}_\mca{R}\,\dot{=}\, \ol \tau_n$ in $\F[\![x-\ol{a},y-\ol{b}]\!]$, we have $L_{\bs \rho}\,\dot{\neq}\, 1$ iff $\ol{\rho}$ is non-acyclic, that is, $\ol{\rho}$ lies in the intersection of $\ol \tau_n=0$ and $\ol f_n=0$ in $\F^3$. 

%\blue{\ul{Claim.} We have $H_0(\bs{\rho})=0$ for $W_k$. {\footnotesize(We want a proof ore some mild condition for this.)}}  

Note that the Taylor expansion of $\tau_n(x,y,z_f(x,y))$ is defined by the formal derivatives, and hence the multiplicities of zeros of the $L$-function $L_\bs \rho$ as a formal series coincide with those in \cref{lem.mult-two}. 
Thus, \cref{lem.mult-two} and \cref{theo:TW} %(theo:TW) 
on the multiplicities of common zeros of the character variety $f_n$ and the torsion function $\tau_n$ %and \cref{lem.mult-two} 
yield the following. %$\tau$ together with an elementary calculus yield the following. 


\begin{theorem} 
Let $\ol{\rho}:\pi\to \SL_2\F$ be an absolutely irreducible representation of the group of the twisted Whitehead link $W_{2n-1}$ {at $(\ol{a},\ol{b},\ol{c})$ in $\F^3$} on the geometric component $f_n=0$ of the character variety and suppose that $\ol{\rho}$ is non-acyclic. 
{Let $(a,b)$ a lift of $(\ol{a},\ol{b})$ in $O^2$.}  
Suppose that $\partial_z f_n \neq 0$ {in $\F$} at $\ol{\rho}$, 
and let $z_{f_n}(x)$ denotes the implicit function given by Hensel's lemma. 
Then, the $L$-function $L_{\bs \rho}$ of the universal deformation is {given by} 
\[L_{\bs \rho}\,\dot{=}\, \tau(x,y,z_{f_n}(x))\] 
in $\mca{R}_{\ol{\rho}}=O[\![x-a, y-b]\!]$.  
%where $z_f(x)$ denotes the implicit function that Hensel's lemma gives. 
{In general, under the weaker hypothesis that $\ol{\rho}$ corresponds to a regular point of $\ol{f}=0$, then $L_{\bs \rho}$ is given in a similar way. In any of these cases,  
%\red{For a generic $\ol{\rho}$}, 
the multiplicities of the zeros of $L_{\bs \rho}$ are {at least} two.}  
\end{theorem}
%Let $\ol{\rho}:\pi\to SL_2\F$ be an absolutely irreducible representation of the group of the twisted Whitehead link $W_k$ on the geometric component $\ol f_n=0$ of the character variety and suppose that $\ol{\rho}$ is non-acyclic. 
%If $\partial_z \ol f_n \neq 0$ at $\ol{\rho}$, then the $L$-function $L_{\bs \rho}$ of the universal deformation is {given by} 
%\[L_{\bs \rho}\,\dot{=}\, \tau_n(x,y,z_f(x))\] 
%in $\mca{R}_{\ol{\rho}}= O[\![x-a, y-b]\!]$. 
%%For a generic $\ol{\rho}$, the multiplicities of the zeros of $L_{\bs \rho}$ are {at least} two. 
%%In addition, if $\ol{\rho}$ moves, $L_{\bs \rho}$ is simultaneously presented by a certain rational function. 
%%{\footnotesize \red{(The precise statement of this theorem depends on that of \cref{theo:TW} and \cref{lem.mult-two}.)}} 
%A similar statement holds for $\partial_x \ol f_n\neq 0$ of $\partial_y \ol f_n \neq 0$. In any of these cases, 
%{the multiplicities of the zeros of $L_{\bs \rho}$ are {at least} two.}
%\end{theorem} 

%{Note that} $\partial_z f_n = 0$ may hold over $\F$. \red{In this case, by the assumption of absolute irreducible} 
{Note that despite $f_n$ being smooth, there is no reason why $\overline f_n$ should be smooth after reduction along any prime. Nevertheless, this is true for almost any prime. The primes where the character variety $\overline f_n$ is not smooth should be of particular interest.}

Also note that, by the definition of multiplicity, the result is independent of the choice of coordinate. 

%\begin{remark} Since the Brauer group of $\mca{R}$ is not trivial, \cite[Proposition 3.4]{Marche-RIMS2016} is not applicable to obtain ${\bs \rho}$. Instead, the existence of Riley's character and \cite[Theorem 1]{Nyssen1996}, \cite[Theorem 1]{Carayol1994}, \cite[Theorem 2.2.4]{KMTT2018} yield the universal deformation.
%\end{remark}

%If there is surjective homomorphism $\pi\surj \pi_K$ to a knot group and $\ol{\rho}:\pi_K\to \SL_2\F$ is an absolutely irreducible non-acyclic representation, then since 

If there is surjective homomorphism $\pi\surj \pi_K$ to a knot group and $\ol{\rho}:\pi_K\to \SL_2\F$ is an absolutely irreducible non-acyclic representation, then the above theorem %should 
yields a similar assertion on the multiplicity of $L_{\bs \rho}$ of $K$, recovering the result on twist knots \cite[Theorem E]{TTU} to some extent. 
A numerical study of explicit presentations of $L_{\bs \rho}$, such as in \cite[Section 5]{TTU}, would be of further interest. 
%\section{Borromean link and twisted Whitehead links}
%The twisted Whitehead link $W_{2k-1}$ can be seen as $-1/k$ surgeries on a component of the Borromean link (the trivial link with two component is the $\infty=-1/0$ surgery and the Whitehead link $W_1$ is the $-1=-1/1$-surgery)%\footnote{The indices differ with the indices with the paper of Anh, where $W_{-1}$ is the unlink, and $W_1$ is the Whitehead link. Our $W_n$ are the $W_{2n-1}$ in \cite{NguyenTran}}. 
%%In this section we focus on 1/2k surgeries, $k\ge 0$ (the trivial link with two component is the $\infty$ surgery, the Whitehead link is the 1/2 surgery)\footnote{There is a shift of +1 with the paper of Anh.}. 
%\begin{figure}[h]
%\begin{center}
%\def\svgwidth{0.3\columnwidth}
%%\def\svgscale{0.3}
%\input{borromean.pdf_tex}
%\caption{\label{Fig:borromean}}.
%\end{center}
%\end{figure}
%Let $B$ be the exterior of the Borromean link in $S^3$, using \cref{Fig:borromean} we get the Wirtinger presentation
%%$$\pi_1(B) = \langle a,b,c \mid [[A,c],b] = [[B,a],c] = [[C,b],a] = 1\rangle $$
%%where $a,b,c$ are meridians of each of the components, and capital letters denote the inverses.
%%The longitudes of the corresponding components are given by 
%%$$\ell_a = [b,C], \, \ell_b = [c,A], \, \ell_c= [a,B].$$
%\begin{multline*}
%\pi_1(B) = \langle x,y,z \mid xy^{-1}x^{-1}z^{-1}xyx^{-1}z = y^{-1}x^{-1}z^{-1}xyx^{-1}zx, \\
% y[x^{-1},z] = [x^{-1},z]y, \quad z[y,x] = [y,x]z \rangle 
%\end{multline*}
%with longitudes 
%$$\ell_x = xy^{-1}x^{-1}z^{-1}xyx^{-1}z, \quad \ell_y = [x^{-1},z], \quad \ell_z = z[x,y]z^{-1}$$
%
%The presentation of the fundamental group of the twisted Whitehead link $W_{2k-1}$ obtained by $-1/k$ surgery on the component with meridian $z$ is
%\begin{equation*}\pi_1(W_{2k-1}) = \langle x,y \mid [x,y^{-1}]^k[x,y]^k[x^{-1},y]^k[x^{-1},y^{-1}]^k=1
%\rangle
%\end{equation*}
%
%On the other hand, performing $-3$ surgeries on the components of $B$ with meridians $x$ and $y$ yields
%\begin{multline*}
%\pi_1(B(-3,-3)) = \langle x,y,z \mid x^3 = xy^{-1}x^{-1}z^{-1}xyx^{-1}z,\\
%y^3 = [x^{-1},z], \quad z [y,x] = [y,x] z \rangle
%\end{multline*}
%
%Finally, performing $-1/k$ surgery on the remaining component, we get
%\begin{equation*}
%\pi_1(B(-3,-3, -1/k)) = \langle x,y \mid x^3 = [x,y^{-1}]^k [x,y]^k, \, 
%y^3 = [y,x^{-1}]^k [y,x]^k\rangle
%\end{equation*}
%
%Notice that the relation $[x,y^{-1}]^k[x,y]^k[x^{-1},y]^k[x^{-1},y^{-1}]^k=1
%$ is redondant in the group $\pi_1(B(-3,-3, -1/k))$.
%
%Also, note the manifold $B(-3,-3,-1/k)$ can be seen as the $(-3,-3)$ surgery on the twisted Whitehead link.
%
%\begin{proposition}
%\label{prop:Z3}
%The group $\pi_1(B(-3,-3, -1/k))$ is isomorphic to $\Z/3 \ast \Z/3$ for $k=0$ and $k=1$.
%\end{proposition}
%
%\begin{proof}
%For $k=0$ it is obvious. For $k=1$ it is the content of \cref{lem:surg}. 
%
%As a side remark, we include a direct proof that $xy$ has order 3 in this case (reminder $Z=-1$ in \cref{sec:Whitehead}).
%First, we claim that the relation 
%\begin{equation}\label{eq:rel}
%[x,y^{-1}][x,y][x^{-1},y][x^{-1},y^{-1}]=1
%\end{equation}
%holds in $\pi_1(B(-3,-3, -1))$ (it works for any $-1/k$).
%The relation $x^3 = [x,y^{-1}] [x,y]$ shows that $x$ commutes with $[x,y^{-1}] [x,y]$. Using this fact, it comes
%\begin{equation}
%[x,y^{-1}][x,y][x^{-1},y][x^{-1},y^{-1}] = x^{-1} [x,y^{-1}][x,y] yxy^{-1}[x^{-1},y^{-1}]
%\end{equation}
%which can be rewritten as
%\begin{equation}
%[x,y^{-1}][x,y][x^{-1},y][x^{-1},y^{-1}] = x^{-1} [x,y^{-1}][x,y] [y,x][y^{-1},x] x
%\end{equation}
%and the claim follows since $[a,b]^{-1} = [b,a]$ for all $a,b$.
%
%On the other hand, using $x^3 = [x,y^{-1}] [x,y]$ we obtain
%\begin{align}\label{eq:8}
%[x,y^{-1}][x,y][x^{-1},y][x^{-1},y^{-1}] &= x^2 yxy^{-1}[x^{-1},y^{-1}]\\
%&=x^2  [y,x] y^{-1} x y \nonumber
%\end{align}
%and we insert then the relation 
%$[y,x] = [x^{-1},y] y^3$ in \cref{eq:8} to obtain
%\begin{align}
%[x,y^{-1}][x,y][x^{-1},y][x^{-1},y^{-1}] & =x^2 [x^{-1},y] y^3y^{-1} x y \\
%&= xyx yxy = (xy)^3   \nonumber
%\end{align}
%which together with \cref{eq:rel} implies that $(xy)^3=1$.
%\end{proof}
%
%
%
%\begin{question}
%\label{quest:1}
%\begin{enumerate}
%\item \label{quest:Z3}
%Does the statement of \cref{prop:Z3} still hold true for $k>1$? (Snappy says yes for $k=2$ as well, but I cannot find the order 3 curves in our presentation)
%\item \label{quest:torsvanish} At least, can we compute the image of the character variety of  $B(-3,-3,-1/k)$ in the character variety of the twisted Whitehead link $W_{2k-1}$? If yes, does it coincides with the vanishing locus of the torsion?
%\item Even more, could we compute the character variety of the manifold $B(-3,-3)$, which is the exterior of a trivial knot in a connected sum of two lens spaces $L(3,1)$.
%If the answer of \cref{quest:torsvanish} is yes,  this manifold should have no acyclic irreducible $\SL_2\C$ representations, which would be interesting by itself.
%\end{enumerate}
%\end{question}
%
%\begin{question}
%The manifolds $B(-3,-3,-1/k)$ are Dehn fillings on the exterior of a (trivial) knot $K$ in the manifold $B(-3;-3,\infty)$ which is a connected sum of two lens spaces $L(3,1)$. For $k=1$, it gives back the same manifold $L(3,1)\# L(3,1)$, as said in \cite{GuillouxWill} right after Proposition 4. 
%
%Are the two knots $K$ and $K_{-1,1}$ (which are corresponding to the filling curves for $B(-3,-3,\infty)$ and $B(-3,-3,-1)$) equivalent, meaning that there exists a homeomorphism of the manifold $B(-3,-3)$ sending $K$ onto $K_{-1,-1}$?
%If not, this would be what is known as a \emph{cosmetic surgery}. Few example of such are known, and it would be interesting to establish some. Caution: one should distinguish between cosmetic and \emph{purely} cosmetic surgery. Existence of the second is open, as far as I know. The question of purely cosmetic surgery is then: if the knots $K$ onto $K_{-1,-1}$ are not equivalent, but the manifolds $B(-3,-3,\infty)$ and $B(-3,-3,-1)$ are homeomorphic, is this homeomorphism orientation preserving?
%
%Even more, it could be that all the $-1/k$ surgeries on $B(-3,-3)$ are cosmetic, but we should answer \cref{quest:1} \cref{quest:Z3} above before that. Note that in this case, necessarily some of them would be purely cosmetic. See \cite{Kirby} Problem 1.81 for more on cosmetic surgeries.
%\end{question}
%Again, for $k=0$ one obtains the free group on the two generators $a,b$ and for $n=1$ one obtains (twice) the relation $[A,B][A,b][a,b][a,B]$ of the Whitehead link.

%We prove the following:
%\begin{proposition}
%For any $n$, the fundamental group of the manifold $M_{(-3,-3,\frac 1{n})}$ obtained by $(-3,-3)$-surgery on the twisted Whitehead link $W_{n}$ is isomorphic to the triangle group $(3,3,\infty)$. In particular there are infinitely many (different) surjections $\phi_{n}\colon  \pi_1(B) \to (3,3,\infty)$.
%\end{proposition}
%
%\begin{proof}
%As usual, we use the fact that surgery is a commutative operation: let us denote by $B_{(-3,-3)}$ the $(-3,-3)$ surgery on the components $a$ and $b$ of the Borromean link.
%
%We have
%$$\pi_1(B_{(-3,-3)}) = \langle a,b,c \mid a^3 = [b,C], \,  b^3 = [c,A], \, [[B,a],c]=1 \rangle.$$
%Let $f_n$ be the automorphism of $\pi_1(B_{(-3,-3)})$ defined by
%$$f_n(a,b,c) = (a,b,c\,[a,B]^n),$$ and let 
%$$K_n = \langle\langle c\,[a,B]^n\UTF{00A0}\rangle\rangle \vartriangleleft \pi_1(B_{(-3,-3)}) $$
%denote the subgroup normally generated by $c\,[a,B]^n$ in $\pi_1(B_{(-3,-3)})$.
%The quotient $\pi_1(B_{(-3,-3)})/K_n$ is the fundamental group of the compact manifold obtained by $1/n$-surgery on $B(-3,-3)$.
%Obviously the automorphism $f_n$ maps isomorphically $K_i$ to $K_{i+n}$.
%
%We defined the map $\phi_n \colon \pi_1(B) \to \pi_1(B_{(-3,-3)}) \to \pi_1(B_{(-3,-3)})/K_n$ as induced by the $(-3,-3,\frac 1{n})$-surgery on $B$.
%
%The proof will be complete once the following claim will be proved:
%\begin{claim}
%For any $n \ge 0$, the group $\pi_1(B_{(-3,-3)})/K_{n}$ is isomorphic to $(3,3,\infty)$.
%\end{claim}
%
%\begin{proof}[Proof of the claim]
%It is easily seen to be true for $n=0$. For $n=1$ it is the content of the first statement in \cref{lem:surg}. 
%
%Now for arbitrary $n$, consider the commutative diagram
%
%\[\begin{tikzcd}
%		1 \rar &K_0 \rar \dar["f_n"] &\pi_1(B_{(-3,-3)}) \rar \dar["f_n"] & \pi_1(B_{(-3,-3)})/K_0 \dar \rar &1 \\
%		1 \rar &K_n \rar& \pi_1(B_{(-3,-3)}) \rar& \pi_1(B_{(-3,-3)})/K_n\rar &1
%	\end{tikzcd}\]
%Because $f_n$ is a group automorphism sending $K_0$ onto $K_n$, the vertical arrow on the right is an isomorphism, and we conclude since $\pi_1(B_{(-3,-3)})/K_0 \simeq (3,3,\infty)$.
%\end{proof}
%
%\end{proof}


%Let $M$ be the exterior of the Whitehead link (Figure \cref{fig:White}). A presentation for its fundamental group is 
%$$\pi_1(M)= \langle a,b, \mu\UTF{00A0}\mid b=\mu^{-1}a\mu, \, [\mu^{-1},a^{-1}][\mu^{-1},a][\mu,a][\mu,a^{-1}] =1 \rangle.$$
%We will denote the relation by $r$.
%\begin{figure}[h]
%\begin{center}
%\def\svgwidth{0.5\columnwidth}
%%\def\svgscale{0.3}
%\input{whitehead.pdf_tex}
%\caption{\label{fig:White} Whitehead link.}
%\end{center}
%\end{figure}
%Let $\ell$ denote the longitude corresponding to $a$, 
%and $\lambda$ corresponding to $\mu$, we have 
%$$\ell = a^{-1} \mu a \mu^{-1} a \mu^{-1} a^{-1} \mu, \quad \lambda = ab^{-1} a^{-1} b  =  a \mu^{-1} a^{-1} \mu a^{-1} \mu^{-1} a \mu $$
%
%We denote by 
%$$X=  \Tr a, \, Y = \Tr \mu, Z = \Tr a \mu$$
%Computing 
%$$\Tr [\mu^{-1},a^{-1}][\mu^{-1},a][\mu,a][\mu,a^{-1}]  -2=0$$
%we obtain the following polynomial $P$ defining the component of the character variety that contains irreducible representations:
%$$P =XYZ^2 - (X^2+Y^2)Z - Z^3 +XY +2Z.$$
%Note that irreducible representations are given by 
%$$\lbrace P=0 \rbrace \setminus \big\{ \{X=\pm 2, Z=\pm Y \} \cup \{Y=\pm 2, Z=\pm X \} \big\}$$ 
%We compute the Reidemeister torsion with the help of the tautological representation
%%$$\rho(a) = \bma a & 1 \\ 0 & a^{-1} \ema, \quad \rho(\mu) = \bma b & 0 \\ c & b^{-1} \ema$$
%%where 
%%$$a+a^{-1} = x, \quad b+b^{-1} = y, \quad ab+a^{-1}b^{-1}+c = z$$
%and from the formula
%$$\tor(M) = \frac{\det \rho (\partial_\mu r)}{\det \rho (a) -1}$$
%we obtain as in \cite[Section 4.2]{DY}
%$$ \tor(M) = \det \rho( a\mu a^{-1} \mu^{-1} - a - \mu^{-1} + \mu^{-1} a)$$
%Simplifying this yields:
%$$\tor(M) = 2 \, (2+Z-X-Y).$$
%
%Note that the non-acyclic representations of the Whitehead are obtained by substituting $Z=X+Y-2$ in the polynomial $P$, one obtains
%$${\left(X+Y - 1\right)}^{2} {\left(X - 2\right)} {\left(Y - 2\right)}$$
%but the components $(x-2)(y-2)$ contain only reducible representations, hence the zero locus of the torsion on the irreducible part of the character variety is the line
%$$\{ (X+Y-1)^2=0, \, Z=-1 \}$$
%The square explains the \emph{order two phenomenon} (since the character variety of $M$ is tangent to the hyperplan $\{\tor(M) = 0\}$ in $\C^3$, hence the torsion of the Whitehead vanishes at order 2), together with the following section.
%
%\section{Surgeries}
%\subsection{Diagonal}
%After an $-1/n$-surgery on the component $\lambda$, one obtains a twist knot $J(2,2n)$ with fundamental group
%$$\pi_1(S^3-J(2,2n))= \langle a,b \mid \lambda^n a = b \lambda^n\rangle,$$
%where $\lambda$ is the surgered longitude (recall $\lambda=a b^{-1} a^{-1} b$, and $b= \mu^{-1} a \mu$)\footnote{it is $\omega$ in other notations}.
%%\begin{remark}
%%In the notations of \cite{TTU}, $m=b, m'=a$ and $\lambda = \omega$.
%%\end{remark}
%The "diagonal" $\Tr a = \Tr ab$ (corresponding to $\{ x=y \}$ in \cite{TTU}) has the equation $D=0$ in the character variety of $M$, where
%$$D = XYZ-Y^2-Z^2-X+2$$
%
%\textcolor{blue}{
%Substituting $Z = X+Y-2$, one gets
%$${\left(X+Y - 1\right)} {\left(X - 2\right)} {\left(Y - 1\right)}$$
%and removing reducible representations (corresponding to $\{X=2, Y=Z\}$), one has
%$$D' = {\left(X + Y - 1\right)}  {\left(Y - 1\right)}.$$
%The curve $D'$ consists of points of the "diagonal" that are zeros of the function $\tor(M)$. 
%Only the first factor yields (irreducible) representations of the Withehead: if $Y=1$ we have $Z=X-1$, and $P$ becomes $(X-2)X^2$. If $X=2$, then the representation is reducible, hence $X=0$ and the representation lies as well in the first factor $X+Y-1$.}
%
%\textcolor{red}{A cleaner argument:}
%Using elimination ideal in sageMath, on can eliminate the variable $Z$ in the ideal generated by $P$ and $D$, it yields
%$$-{\left(X + Y - 1\right)} {\left(X - Y - 1\right)} {\left(X - 2\right)} X$$
%and in the character variety of the Whitehead, the zero locus of the torsion is contained in the diagonal as the first factor shows.
%%Consequently, any irreducible non-acyclic representation of the Whitehead link is contained in the "diagonal" $D' = X+Y-1$.
%\subsection{Order 3 phenomenon}
%There is the following interpretation of the line $(X+Y-1)$: consider the curve $\gamma =  a^{-1} \mu$ in $\pi_1(M)$. Its trace is
%$$\Tr \gamma = XY-Z$$
%and this curve has order 3 exactly when $\Tr \gamma = - 1$. 
%Substituting $Z = 1+XY$ in $P$, one gets
%$$P'' = -{\left(X+Y - 1\right)} {\left(Y + 1\right)}$$
%the curve of representations such that $\rho(\gamma)^3 = 1$.
%
%Now the curve $\gamma$ becomes $a^{-1}\lambda^n$ in the surgered manifold $J(2,2n)$ since the surgery relation is $\mu =\lambda^n $. 
%
%%Now observe that $ \Tr \gamma = \Tr b^{-1} \mu$ as well, hence $ \gamma^3=1$ becomes $m'^{-1} \lambda^n $ has order $3$ in $J(2,2n)$ after surgery (since the surgery relation is $\mu =\lambda^n $). 
%
%Again, only the first factor yields (irreducible) representations of the Withehead: if $Y=-1$ we have $Z=1-X$, and $P$ becomes $(X-2)X^2$. If $X=2$, then the representation is reducible, hence $X=0$ and the representation lies as well in the first factor $X+Y-1$.
%
%This explains the fact that the curve denoted by $a^{-1}\lambda^n$ has order 3 in your paper\footnote{recall your $\omega$ is my $\lambda$.}
%
%\subsection{The $-3$ surgery}
%Recall the longitude $\ell$ is the curve $a^{-1} \mu a \mu^{-1} a \mu^{-1} a^{-1} \mu$.
%We compute the character variety of the $-3$-surgery manifold given by $a^3\ell=1$, it yields
%$$\{(XYZ-Y^2-Z^2-X+2) (X+1) = 0 \}$$
%Note that the first factor we already saw: it is the diagonal $D$.
%It explains why all non-acyclic irreps of the Whitehead are representations of the $-3$-surgered manifold.
%
%\begin{remark}
%Note that computing the character variety of the surgered manifold requires luck, as far as I know. Indeed, its character variety is given by an ideal in $\C^3$, generated by two elements. Computing a relation of type $\Tr(a^3\ell) = 2$ yields an element of this ideal, and there are infinitely many distinct possibilities. Getting the "right one" is quite delicate. Here I was lucky, and the relation I've computed is 
%$$\Tr(a^2\mu a \mu^{-1}) = \Tr (\mu^{-1}a\mu a^{-1})$$
%and I had this nice answer. 
%Other relations give other elements of the given ideal, less nice...
%It is the main part of the difficulty, and a bit frustrating.
%\end{remark}
%
%\subsection{Recap}
%We proved:
%\begin{proposition}
%\begin{itemize}
%\item
%The torsion function vanishes at order two on the character variety of the Whitehead link.
%\item
%Any irreducible, non-acyclic representation of the Whitehead link lies in the diagonal $D$, which is itself contained in the character variety of the $-3$ surgery.
%\item
%A non-acyclic, irreducible representation of the Whitehead link lies in the diagonal $D'$ iff the curve $m \mu^{-1}$ has order 3.
%\end{itemize}
%\end{proposition}
%
%\begin{question}
%Find a geometric interpretation of this proposition. 
%\end{question}

%\newpage 
%
%\section{The multiplicity %(order) 
%of common zeros} 
%In this section, we clarify the notion of the multiplicity of a common zero of polynomials by stating elementary facts of multivariable calculus, \blue{as well as invoking the intersection theory.}  
%
%\begin{definition}
%%Let $f,g\in \C[x,y,z]$ with a common zero $(a,b,c) \in \C^3$. 
%Let $f,g\in \C[x,y,z]$. %or $\C[[x-a,y-b,z-c]]$. 
%We say that a common zero $(a,b,c)$ of $f$ and $g$ is \emph{of multiplicity %(multiplicity) 
%at least} $n$ if 
%the $n$-th Taylor polynomials satisfy $f_n=\alpha g_n$ for some $\alpha \neq 0$. 
%We say that $(a,b,c)$ is of multiplicity $n$ if in addition $f_{n+1}\neq \alpha g_{n+1}$ holds. 
%
%We define similar notions for $f,g\in \C[[x-a,y-b,z-c]]$ in an obvious way. 
%\end{definition} 
%
%\begin{lemma} 
%Let $f,g\in \C[x,y,z]$ with a common zero $(a,b,c) \in \C^3$ 
%and suppose that $(a,b,c)$ is a regular point of both $f=0$ and $g=0$,
%that is, we have $\nabla f (a,b,c)\neq \bm{0}$ and $\nabla g(a,b,c)\neq \bm{0}$. 
%For simplicity, suppose $\frac{\partial f}{\partial z}(a,b,c)\neq 0$. 
%Then, %$\bma{p}$.
%
%{\rm (1)} The following conditions are equivalent. 
%\begin{enumerate}[{\rm (i)}] 
%\item $(a,b,c)$ is a common zero of multiplicity at least two. 
%\item $\nabla f (a,b,c)$ and $\nabla g(a,b,c)$ are parallel. 
%\item The varieties $f=0$ and $g=0$ have a common tangent plane at $(a,b,c)$. 
%\item Let $z_f(x,y,z)$ be an implicit function of $f=0$ at $(a,b,c)$. 
%Then, the Taylor expansion of $g(x,y,z_f(x,y))$ at $(a,b)$ has 
%trivial degree 0 and 1 terms. 
%\item The equality of ideals $(f,g)=(h,k)$ holds for some $h,k\in \C[x,y,z]$ and 
%the Taylor expansion of $h$ at $(a,b,c)$ has trivial degree 0 and 1 terms. 
%\end{enumerate} 
%
%{\rm (2)} The following conditions are equivalent.\footnote{Finding a condition for multiplicity exactly $n$ seems slightly troublesome, but elementary example suggests that it should be hold under some mild assumption. 
%I (ju) will continue to think about it.}
%
%\begin{enumerate}[{\rm (i)}]
%\item $(a,b,c)$ is a common zero of multiplicity at least $n$. 
%\item Let $z_f(x,y,z)$ be an implicit function of $f=0$ at $(a,b,c)$. 
%Then, the $n$-th Taylor polynomial of $g(x,y,z_f(x,y))$ at $(a,b)$ is trivial. 
%\end{enumerate} 
%
%{\rm (3)} If the equality of ideals $(f,g)=(k^2,r)$ holds for some $0\neq k,r\in \C[x,y,z]$
%and $(a,b,c)$ is a zero of $k$ of multiplicity $n$, then $(a,b,c)$ is a common zero of multiplicity $2n$. 
%%\begin{enumerate}[{\rm (i)}]
%%\item $(a,b,c)$ is a common zero of multiplicity $2n$. 
%%\item The equality of ideals $(f,g)=(k,h^2)$ for some $0\neq k,h\in \C[x,y,z]$
%%and $(a,b,c)$ is a zero of $h$ of multiplicity $n$. %and the Taylor expansion of $h$ at $(a,b,c)$ has trivial degree 0 and 1 terms. 
%%\end{enumerate} 
%\end{lemma} 
%
%
%\begin{lemma} \label{lem.fgh} 
%Let $f,g,h\in \C[x,y,z,v]$ with a common zero $\bm{a}=(a,b,c,d) \in \C^4$ and suppose that $\bm{a}$ is a regular point of all of $f=0$, $g=0$, and $h=0$. 
%Suppose $\frac{\partial h}{\partial v}(\bm{a})\neq 0$. 
%Then, 
%
%{\rm (1)} The following conditions are equivalent. 
%\begin{enumerate}[{\rm (i)}]
%\item Let $v_h(x,y,z)$ be an implicit at $\bm{a}$. Then $(a,b,c)$ is a common zero of $f(x,y,z,v_h(x,y,z))$ and $g(x,y,z,v_h(x,y,z))$ of multiplicity at least two. %, that is, the 1st Taylor polynomials of $f$ and $g$ at $(a,b,c)$ are parallel. 
%%\item $\nabla f(\bm{a})$ and $\nabla g(\bm{a})$ are parallel under the restriction $h=0$. 
%\item $\alpha \nabla f(\bm{a})+\beta \nabla g(\bm{a}) + \gamma  \nabla h(\bm{a})=0$ holds for some $\alpha, \beta, \gamma \in \C$ with $\alpha \beta \neq 0$. 
%%\item The equality of ideals $(f,g)=(k,h^2)$ for some $0\neq k,h\in \C[x,y,z]$
%%and $(a,b,c)$ is a zero of $h$ of multiplicity $n$. %and the Taylor expansion of $h$ at $(a,b,c)$ has trivial degree 0 and 1 terms. 
%\end{enumerate} 
%
%{\rm (2)} Suppose that the equality $(f,g,h)=(k^2,r_1,r_2)$ of ideals in $\C[x,y,z,v]$ holds for some $0\neq k, r_1, r_2\in \C[x,y,z,v]$,  
%$k^2,$ $r_1,$ $r_2$ are independent, 
%and $(a,b,c)$ is a zero of $k$ with multiplicity $n$. 
%%the conditions in {\rm (1)} holds. 
%Then, 
%$(a,b,c)$ is a common zero of $f(x,y,z,v_h(x,y,z))$ and $g(x,y,z,v_h(x,y,z))$ of multiplicity $2n$. 
%\end{lemma} 
%
%
%\footnote{Caution! I have not yet explicitly written down the proofs of these statements. 
%%Some statements in above are not yet proved. 
%Perhaps polynomials should be irreducible. 
%By using the intersection theory, we may have a more useful paraphrasing of the notion of the multiplicity.}  
%
%\newpage 
%
%\section{$L$-functions of universal deformations} 
%In this section, we pursue the study of $L$-functions of universal deformation, raised in a viewpoint of number theory (\cite{Mazur2000}, \cite{MTTU2017}, \cite{KMTT2018}, \cite{TTU}). 
%The main goal is to interpret the ``multiplicity two'' phenomenon to the $L$-functions of the twisted Whitehead links $W_k$. 
%
%Let $\pi$ be the group of $W_k$. Let $\F$ be any finite field with characteristic $p>2$ and $O$ a CDVR with $O/\mf{m}_O=\F$. 
%Then the conjugacy classes of absolutely irreducible $\SL_2\F$-representations
%correspond to regular points of a component of the character variety $f(x,y,z)=0$ in $\F^3$. 
%(This assertion holds by the fact that the Brauer group of a finite field is trivial and \cite[Proposition 3.4]{Marche-RIMS2016}.) 
%
%Suppose $\frac{\partial f}{\partial z} (\ol{a}, \ol{b}, \ol{c})\neq 0$, let $(a,b)\in O^2$ be a lift of $(\ol{a},\ol{b})$, and put $\mca{R}=O[[x-a,y-b]]$. 
%Hensel's lemma for multivariable functions yields the implicit function $z_f(x,y)$ of $f(x,y,z)=0$ around $(a,b)$ and a natural map $O[x,y,z]\mapsto O[[x-a,y-b]]; z\mapsto z_f(x,y)$. 
%
%Let ${\bs T}:\pi\to \mca{R}$ denote the image of Riley's character via the natural map. 
%Then by Nyssen and Carayol's theorems (\cite[Theorem 1]{Nyssen1996}, \cite[Theorem 1]{Carayol1994}), there exists a deformation ${\bs \rho}:\pi\to \SL_2(\mca{R})$ of $\ol{\rho}$ such that ${\bs \rho}$ is equivalent to the image of Riley's representation and ${\rm tr}{\bs \rho}={\bs T}$ holds. 
%By a similar argument to \cite[Theorem 2.2.4]{KMTT2018}, this ${\bs \rho}$ is indeed a universal deformation of $\ol{\rho}$. 
%(Since the Brauer group of $\mca{R}$ is not trivial, \cite[Proposition 3.4]{Marche-RIMS2016} is not applicable to obtain ${\bs \rho}$. In addition, the deformation problem is not necessarily unobstructed in the sense of Mazur; in the case of a knot group representation $\ol{\rho}$, we have $H^2({\rm Ad}(\ol{\rho}))\neq 0$ \cite[Theorem 2.3.2]{KMTT2018}. Thus, the existence of such a universal deformation is non-trivial.) 
%
%The standard exact sequence given by the Fox derivative yields that the $L$-function of the universal deformation of $\ol{\rho}$ is 
%$L_{\bs \rho}=\tau(x,y,z_f(x,y)) \in O[[x-a,y-b]]$. 
%
%\blue{Assume that the 0-th Alexander polynomial satisfies $\Delta_{{\bs \rho},0}\,\dot{=}\,1$, which holds for most cases.}\footnote{\blue{Do we have this for $\ol{\rho}$ of $W_k$?}} 
%Since 
%%By the procedure of the calculation with use of Fox derivative we see that 
%$L_{\bs \rho}\, {\rm mod}\, \mf{m}_\mca{R}\,\dot{=}\, \tau_{\ol{\rho}}$ in $f$, $L_{\bs \rho}$ is non-trivial iff $\ol{\rho}$ is non-acyclic, that is, $\ol{\rho}$ corresponds to the intersection of $\tau=0$ and $f=0$ in $\F^3$. 
%
%\cref{theo:TW} on the multiplicitys of common zeros of the character variety $f=0$ and the torsion function $\tau$ together with an elementary calculus (Lemma \ref{lem.fgh} (2)) yield the following. 
%\begin{theorem} Let $\ol{\rho}:\pi\to \SL_2\F$ be an absolutely irreducible representation of the group of the twisted Whitehead link $W_k$ on the geometric component $f_n=0$ of the character variety 
%and suppose that $\ol{\rho}$ is non-acyclic. 
%If $\partial f_n/\partial z\neq 0$ at $\ol{\rho}$, then 
%the $L$-function $L_{\bs \rho}$ of the universal deformation is defined in $\mca{R}_{\ol{\rho}}=O[[x-a, y-b]]$ and the multiplicity is even. 
%\end{theorem} 
%
%%\begin{remark} Since the Brauer group of $\mca{R}$ is not trivial, \cite[Proposition 3.4]{Marche-RIMS2016} is not applicable to obtain ${\bs \rho}$. Instead, the existence of Riley's character and \cite[Theorem 1]{Nyssen1996}, \cite[Theorem 1]{Carayol1994}, \cite[Theorem 2.2.4]{KMTT2018} yield the universal deformation.
%%\end{remark}
%
%%If there is surjective homomorphism $\pi\surj \pi_K$ to a knot group and $\ol{\rho}:\pi_K\to \SL_2\F$ is an absolutely irreducible non-acyclic representation, then since 
%
%If there is surjective homomorphism $\pi\surj \pi_K$ to a knot group and $\ol{\rho}:\pi_K\to \SL_2\F$ is an absolutely irreducible non-acyclic representation, then the above theorem \blue{should}\footnote{to be checked} yield the similar assertion on the multiplicity of $L_{\bs \rho}$, recovering the result on twist knots \cite[Theorem E]{TTU}. 
%Finding explicit presentations of $L_{\bs \rho}$ will be of further interest. 
%
%\newpage 
%
%
%
%\newpage 

\bibliography{biblio}
\bibliographystyle{amsplain} \ 
%\bibliography{ref}

%\begin{thebibliography}{9}
%\bibitem{NguyenTran}
%Nguyen, Hoang-An, and Tran, Anh T., "Twisted Alexander polynomials of twisted Whitehead links",  New York J. Math. 25 (2019), 1240?1258. 
%\bibitem{DY}
%Dubois, Jerome, and Yoshikazu Yamaguchi. "Twisted Alexander invariant and non-abelian Reidemeister torsion for hyperbolic three-dimensional manifolds with cusps." arXiv preprint arXiv:0906.1500 (2009).
%\bibitem{TTU}
%Tange, Ryoto, Anh T. Tran, and Jun Ueki. "Non-acyclic $\SL _2 $-representations of twist knots." arXiv preprint arXiv:2005.13246 (2020).
%
%\end{thebibliography}
\end{document}
