%!TEX root=../Draft1.tex

\subsection{List of features} \label{s.app.list_features}


\textbf{Static features.}
For each patient, we include the following static covariates: weight, height, age, sex, language, county, HIV positive, and extrapulmonary TB.
There were 6 different counties where we used a one-hot encoding, which resulted in 13 features in total.

% 'weightkgs',
%  'heightmtrs',
%  'age_in_years',
%  'sex_male',
%  'language_swahili',
%  'county_KAKAMEGA',
%  'county_KIAMBU',
%  'county_KISUMU',
%  'county_MACHAKOS',
%  'county_MOMBASA',
%  'county_NAIROBI',
%  'extrapulmonary',
%  'hiv_positive',

\textbf{Condensed history.}
For patient $i$ at time $t$, we summarize their history, $H_{it} = (V_{i1}, A_{i1}, \dots, V_{i, t-1}, A_{i,t-1}, V_{it}) \in \bR^{2t-1}$, using the following features:
\begin{itemize}
	\item Verifications: total so far, total percentage, total last week, X days ago for the last $X \in \{1, \dots, 7\}$ days.
	\item Verification/non-verification streaks (how many days in a row a patient verifies / does not verify): current streak, longest streak
	\item Interventions: total so far, total last week, X days ago for the last $X \in \{1, 2, 3\}$ days.
	\item Number of days on the platform, number of days of treatment left.
\end{itemize}
This results in 21 features in total.


% History summary 
% - total num calls, last week num calls
% - total num verified, percent verified, last week verified, 
% - verified X days ago, where X is 1 to 7
% - called X days ago, where X is 1 to 3
% - past verified/unverified streak
% - longest verified/unverified streak
% - average length of verification streak
% - day on platform, days left

\subsection{Simulation Model Details} \label{s.app.simulation}

We briefly describe the double ML method of \cite{chernozhukov2018double}.
Let $Y \in \bR$ be the outcome variable, $T \in \{0, 1\}$ the treatment, and $X \in \bR^{d}$ the observable features.
The model makes the following structural assumptions:
\begin{align*}
Y &= \tau(X) \cdot T + g(X) + \eps, \\ 
T &= f(X) + \eta,
\end{align*}
where $\bE[\eps | X] = 0, \bE[\eta | X] = 0$, and $\bE[\eps \cdot \eta | X] = 0$.
The goal is to estimate the conditional average treatment effect, $\tau(X)$.
% The goal is to estimate $\tau(X)$, the conditional treatment effect 
We estimate two functions:
\begin{align*}
q(X) = \bE[Y \;|\; X], \quad
f(X) = \bE[T \;|\; X].
\end{align*}
Then, we compute the residuals
\begin{align*}
\tilde{Y} = Y - q(X), \quad \tilde{T} = T - f(X).
\end{align*}
% Lastly, we write $\tau(X) = \theta^\top X$ as a linear function, and estimate
Lastly, we estimate
\begin{align*}
% \hat{\tau} = \argmin_{\tau} \bE[(\tilde{Y} - \tau(X) \cdot \tilde{T})^2].
\hat{\tau} = \argmin_{\tau} \bE[(\tilde{Y} - \tau(X) \cdot \tilde{T})^2].
\end{align*}

For the Keheala case study, we used gradient boosting to estimate $q$ and $f$, and we assumed a linear function for $\tau(X) = \theta^\top X$.





\iffalse
\subsubsection{Learning $f$.} \label{s.learningf}
We use the same state space $\cS$ as described in \cref{s.advg_keheala}, and we construct the function $f: \cS \times \{0, 1\} \to [0, 1]$ using data from the RCT.
% We construct the function $f(s, a)$ using data from the RCT, and we use the same state space $\cS$ as described in \cref{s.advg_keheala}.
% Recall that we have data of the form $\{(H_{it}, a_{it}, v_{it})\}_{i \in N_0, t \in \cT_i}$.
% We use the same state space as described in \cref{s.advg_keheala}, and we construct the next-day verification probability function, , using the dataset from the RCT.
We learn the two functions $f(s, 0)$ and $f(s, 1)$ separately:
\begin{itemize}
	\item For $f(s, 0)$, we train a gradient boosting classifier on the dataset $\{(S_{it}, v_{i,t+1})\}_{i \in N_0, t \in \cT_i : A_{it} = 0}$, using $v_{it}$ as the outcome variable.
	\item For $f(s, 1)$, we write the function as $f(s, 1) = f(s, 0) + \tau(s)$, and we learn $\tau(s)$ using the double machine learning method of estimating heterogeneous treatment effects \citep{chernozhukov2018double}.
\end{itemize}
\fi












