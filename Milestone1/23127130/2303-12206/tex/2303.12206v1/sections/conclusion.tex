%!TEX root=../Draft1.tex

We tackle a problem that is ubiquitous in the growing industry of digital support systems for behavioral health. 
Namely, how to personalize and optimize the timing of costly interventions to maximize their cumulative impact. 

We approach this problem from a practical perspective. 
Specifically, we consider a case in which a behavioral health system has been piloted using some baseline policy for the timing of interventions and the system operator wishes to leverage the data collected to improve the outreach policy. 
For such a setting, our aim is to develop an approach that is simultaneously safe (i.e., guaranteed to outperform current practice), requires little data (since a pilot implementation only results in a limited number of observations), is model free (to avoid relying on a specific model of patient behavior), and does not require online experimentation (since operators can be reluctant to experiment in healthcare settings).

With these constraints in mind, we present a novel Decomposed Policy Iteration $(\DPI)$ approach. This approach satisfies the above considerations and has multiple desirable properties. First, for a reasonable model of patient behavior in the context of behavioral health, it admits a strong performance guarantee. Second, for a general case, it is practical to implement since it relies only on offline data and decomposes to the patient level. Third, it exhibits strong empirical performance. In a validated simulation model of a representative behavioral health setting, $\DPI$ achieves the same performance as a baseline heuristic used in a field implementation, using less than half as much capacity.  