\documentclass[nonblindrev]{workingpaperJJB} %

\usepackage[normalem]{ulem}
\usepackage{natbib}
\usepackage{amsmath,bbm}
\usepackage[capitalise]{cleveref}
 \bibpunct[, ]{(}{)}{,}{a}{}{,}%
 \def\bibfont{\small}%
 \def\bibsep{\smallskipamount}%
 \def\bibhang{24pt}%
 \def\newblock{\ }%
 \def\BIBand{and}%
 \usepackage{bibspacing}
 \setlength{\bibitemsep}{.0\baselineskip plus .05\baselineskip minus .05\baselineskip}
 
 \setlength{\pdfpagewidth}{8.5in}
 \setlength{\pdfpageheight}{11in}

 \TheoremsNumberedThrough     %
 \ECRepeatTheorems

 \EquationsNumberedThrough    %


\usepackage{amsfonts}
\usepackage{amsmath}  
\usepackage{amssymb}
\usepackage{changepage}
\usepackage{graphicx}
\usepackage{multirow}
\usepackage{enumerate}
\usepackage{enumitem}
\usepackage{booktabs}
\usepackage{color}
\usepackage{filecontents}
\usepackage{subfig}
\usepackage[customcolors]{hf-tikz}
\usepackage{colortbl}
\usepackage{threeparttable}


\usepackage{algorithm}%
\usepackage{algpseudocode}%

\def\sym#1{\ifmmode^{#1}\else\(^{#1}\)\fi}
\def\jb#1{{\bf\color{blue}#1}}
\def\jnote#1{{\color{cyan}[Jackie: #1]}}


\newcommand{\added}[1]{{\leavevmode#1}}
\newcommand{\edit}[1]{{\leavevmode#1}}
\newcommand{\jedit}[1]{{\leavevmode#1}}


\DeclareMathOperator*{\veccat}{%
    \mathchoice%
        {\Bigg\Vert}%
        {\Big\Vert}%
        {\Vert}%
        {\Vert}%
}%


\NewDocumentEnvironment{myproof}{o}
  {\IfNoValueTF{#1}{\paragraph{{Proof.} }} {\paragraph{{#1.} }} }
  {\hfill$\Halmos$}


\begin{document}

 \RUNAUTHOR{Baek et al.}

\RUNTITLE{Policy Optimization for Personalized Interventions}

\EquationsNumberedThrough    %



\TITLE{
Policy Optimization for Personalized Interventions in Behavioral Health}
\ARTICLEAUTHORS{
\AUTHOR{Jackie Baek}\AFF{NYU Stern School of Business, \EMAIL{baek@stern.nyu.edu}}
\AUTHOR{Justin J. Boutilier}\AFF{Department of Industrial and Systems Engineering, University of Wisconsin-Madison, \EMAIL{j.boutilier@wisc.edu}}
\AUTHOR{Vivek F. Farias}\AFF{MIT Sloan School of Management, \EMAIL{vivekf@mit.edu}}
\AUTHOR{J\'{o}nas Oddur J\'{o}nasson}\AFF{MIT Sloan School of Management, \EMAIL{joj@mit.edu}}
\AUTHOR{Erez Yoeli}\AFF{MIT Sloan School of Management, \EMAIL{eyoeli@mit.edu}}
}
\HISTORY{\today}


\ABSTRACT{

Behavioral health interventions, delivered through digital platforms, have the potential to significantly improve health outcomes, through education, motivation, reminders, and outreach. 
We study the problem of optimizing personalized interventions for patients to maximize a long-term outcome, where interventions are costly and capacity-constrained. 
\edit{We assume there exists a dataset collected from an initial pilot study that we can leverage.}
We present a new approach for this problem that we dub $\DPI$, which approximates one step of policy iteration.
Implementing $\DPI$ simply consists of a prediction task using the dataset, alleviating the need for online experimentation.
$\DPI$ is a generic model-free algorithm that can be used irrespective of the underlying patient behavior model.
We derive theoretical guarantees on a simple, special case of the model 
that is representative of our problem setting.
We establish an approximation ratio for $\DPI$ with respect to the \textit{improvement} beyond a null policy that does not allocate interventions.
Specifically, when the initial policy used to collect the data is randomized, the approximation ratio of the improvement approaches 1/2 as the intervention capacity of the initial policy decreases. 
We show that this guarantee is robust to estimation errors.
We conduct a rigorous empirical case study using real-world data from a mobile health platform for improving treatment adherence for tuberculosis.
Using a validated simulation model, we demonstrate that $\DPI$ can provide the same efficacy as the status quo approach with approximately \textit{half} the capacity of interventions.
$\DPI$ is simple and easy to implement for organizations aiming to improve long-term behavior through targeted interventions, and this paper demonstrates its strong performance both theoretically and empirically.
}

\KEYWORDS{Health Analytics, Behavioral Health, Policy Optimization, Reinforcement Learning, Tuberculosis, Global Health}

\HISTORY{\today}

\maketitle

\section{Introduction}\label{s.intro}
\section{Introduction}
\label{sec:introduction}
% \begin{itemize}
%     % Diffusion of FL
%     \item {\st{Diffusion of FL}}
%     % Security threats to FL
%     \item {\st{Security threats to FL with particular focus on model poisoning}}
%     % Limitations of existing countermeasures
%     \item {\st{Current countermeasures (e.g., KRUM) and their limitations}}
%     % Proposed method and its advantages
%     \item {\st{Intuitive description of the proposed method and its difference (i.e., advantages) w.r.t. state of the art}}
%     % Main contributions
%     \item {\st{Summary of the main contributions of this work}}
%     % Paper's structure and organization
%     \item {\st{Paper's structure and organization}}
% \end{itemize}

% Diffusion of FL
Recently, {\em federated learning} (FL) has emerged as the leading paradigm for training distributed, large-scale, and privacy-preserving machine learning (ML) systems~\cite{mcmahan2017googleai,mcmahan2017aistats}. 
The core idea of FL is to allow multiple edge clients to collaboratively train a shared, global model without disclosing their local private training data.
%Specifically, an FL system consists of a central server and many edge clients; 
A typical FL round involves the following steps: {\em(i)} the server randomly picks some clients and sends them the current, global model; {\em(ii)} each selected client locally trains its model with its own private data; then, it sends the resulting local model to the server;\footnote{Whenever we refer to global/local model, we mean global/local model {\em parameters}.} {\em(iii)} the server updates the global model by computing an \emph{aggregation function}, usually the average (FedAvg), on the local models received from clients.
% \begin{enumerate}
%     \item[{\em(i)}] the server sends the current, global model to the clients and appoints some of them for training;
%     \item[{\em(ii)}] each selected client locally trains its copy of the global model with its own private data; then, it sends the resulting local model back to the server;\footnote{Whenever we refer to global/local model, we mean global/local model {\em parameters}.}
%     \item[{\em(iii)}] the server updates the global model by computing an \emph{aggregation function} on the local models received from clients (by default, the average, also referred to as FedAvg~\cite{mcmahan2017aistats}).
% \end{enumerate}
This process goes on until the global model converges. %(e.g., after a certain number of rounds or other similar stopping criteria).
%\\
% The advantages of FL over the traditional, centralized learning paradigm are undoubtedly clear in terms of flexibility/scalability (clients can join/disconnect from the FL network dynamically), network communications (only model weights\footnote{We will use \textit{parameters} and \textit{weights} interchangeably.} are exchanged between clients and server), and privacy (each client's private training data is kept local at the client's end and not uploaded to the server).
\\
% Security threats to FL
%However, the growing adoption of FL also raises security concerns~\cite{costa2022covert}, particularly about its confidentiality, integrity, and availability.
Although its advantages over standard ML, FL also raises security concerns~\cite{costa2022covert}. %, particularly about its confidentiality, integrity, and availability~\cite{costa2022covert}.
% OLD, LONG VERSION
% Indeed, some work deals with privacy leakage that may expose the local data of some clients~\cite{melis2019sp}. 
% A large body of work, instead, investigates attacks that usually aim to detriment the predictive accuracy of the learned global model. For instance, \emph{data poisoning} attacks achieve this goal by letting an adversary pollute the training set of some corrupt FL clients with maliciously crafted examples~\cite{jagielski2018sp}.
% Similarly, in \emph{model poisoning} the attacker attempts to tweak the global model weights~\cite{bhagoji2019pmlr} by directly perturbing the local model's weights of some infected FL clients before these are sent to the central server for aggregation, usually via so-called Byzantine attacks. 
% It turns out that Byzantine model poisoning attacks severely impact standard FedAvg; therefore, more robust aggregation functions must be designed to make FL systems secure.
Here, we focus on \emph{untargeted model poisoning} attacks~\cite{bhagoji2019pmlr}, where an adversary attempts to tweak the global model weights %\footnote{We will use the terms \textit{parameters} and \textit{weights} interchangeably.} 
by directly perturbing the local model's parameters of some infected clients before these are sent to the central server for aggregation.
In doing so, the adversary aims to jeopardize the global model \textit{indiscriminately} at inference time.
Such model poisoning attacks severely impact standard FedAvg; therefore, more robust aggregation functions must be designed to secure FL systems.
\\
% In this paper, we focus on designing a novel robust aggregation scheme at the server's end to contrast the effect of Byzantine model poisoning attacks.
%
% Current countermeasures and their limitations
%Several countermeasures have been proposed in the literature to combat model poisoning attacks on FL systems.
% Some methods use simple statistics more robust than plain average to smooth the impact of malicious updates (e.g., Trimmed Mean and FedMedian~\cite{yin2018icml}). 
% Other defenses implement outlier detection techniques to discard malicious updates from the aggregation performed at the server's end. Those are either based on heuristics (e.g., Krum/Multi-Krum~\cite{blanchard2017nips} and Bulyan~\cite{mhamdi2018pmlr}) or data-driven approaches (e.g., K-means clustering~\cite{shen2016acm} or DnC via spectral analysis~\cite{shejwalkar2021ndss}). 
% Finally, some strategies rely on a centralized ``source of trust'' to spot potential malicious updates (e.g., FLTrust~\cite{cao2020fltrust}).
% Several countermeasures have been proposed in the literature to combat model poisoning attacks on FL systems, i.e., to discard possible malicious local updates from the aggregation performed at the server's end. 
% These techniques range from simple statistics more robust than plain average (e.g., Trimmed Mean and FedMedian~\cite{yin2018icml}) to outlier detection heuristics (e.g., Krum/Multi-Krum~\cite{blanchard2017nips} and Bulyan~\cite{mhamdi2018pmlr}) or data-driven approaches (e.g., spectral analysis via K-means clustering~\cite{shen2016acm} or spectral analysis), or methods based on ``source of trust'' (e.g., FLTrust~\cite{cao2020fltrust}).
% OLD, LONG VERSION
%Several countermeasures have been proposed in the literature to combat Byzantine model poisoning attacks on FL systems.
% Descriptive statistics
% For example, Trimmed Mean and FedMedian aggregate local model updates using more robust statistics than standard average~\cite{yin2018icml}.
%
% % Heuristics for outlier detection
% Many existing Byzantine-resilient strategies implement some outlier detection heuristics to discard the model updates sent by potentially malicious clients from the input of the aggregation function.
% One of the most popular heuristics is Krum~\cite{blanchard2017nips}.
% This strategy tries to mitigate the impact of Byzantine attacks by selecting as a global model the local model with the smallest sum of Euclidean distances to {\em all} the other local models.
% Although powerful, Krum requires the server to know (or, at least, estimate) the number of malicious FL clients upfront, which is generally impossible in a realistic attack scenario. %
% Moreover, Krum may become ineffective for complex, high-dimensional model parameter spaces due to the curse of dimensionality.
% Bulyan~\cite{mhamdi2018pmlr} tries to overcome this issue by combining Krum with a variant of Trimmed Mean.
% % Data-driven outlier detection
% Other strategies use data-driven outlier detection techniques -- e.g., via K-means clustering~\cite{shen2016acm} -- to spot potential malicious local model updates. 
% %For instance, Shen et al. propose to cluster local model updates with K-means and thus identify outliers.
%
% % Other techniques
% As far as the server is concerned, any local model received can be from a potential malicious client. 
% FLTrust~\cite{cao2020fltrust} assumes the server acts as a client, i.e., trains a local model on an additional {\em trustworthy} dataset at the server's end and compares it against all the local models from other clients. 
% This way, the server can rely on some ``source of trust'' when discarding potentially malicious clients.
%\\
% Limitations of existing Byzantine-resilient strategies
Unfortunately, existing defense mechanisms either rely on simple heuristics (e.g., Trimmed Mean and FedMedian by~\cite{yin2018icml}) or need strong and unrealistic assumptions to work effectively (e.g., foreknowledge or estimation of the number of malicious clients in the FL system, as for Krum/Multi-Krum~\cite{blanchard2017nips} and Bulyan~\cite{mhamdi2018pmlr}, which, however, cannot exceed a fixed threshold).
Furthermore, outlier detection methods using K-means clustering~\cite{shen2016acm} or spectral analysis like DnC~\cite{shejwalkar2021ndss} do not directly consider the temporal evolution of local model updates received.
Finally, strategies like FLTrust~\cite{cao2020fltrust} require the server to collect its own dataset and act as a proper client, thereby altering the standard FL protocol.
\\
% OLD, LONG VERSION
% Overall, existing Byzantine-resilient strategies are either simple heuristics (e.g., FedMedian) or, if they are more complex, they rely on strong and unrealistic assumptions to work effectively (e.g., knowing the number of malicious clients in the FL system in advance, as for Krum and alike).
% Furthermore, data-driven outlier detection methods do not consider the temporary evolution of local model updates received (e.g., K-means clustering). 
% Finally, strategies like FLTrust requires the server to collect its own dataset and act as a proper client, thereby altering the standard FL protocol.
%
% Description of the proposed method
This work introduces a novel pre-aggregation \textit{filter} robust to untargeted model poisoning attacks. Notably, this filter $(i)$ operates without requiring prior knowledge or constraints on the number of malicious clients and $(ii)$ inherently integrates temporal dependencies. 
The FL server can employ this filter as a preprocessing step before applying \textit{any} aggregation function, be it standard like FedAvg or robust like Krum or Bulyan.
Specifically, we formulate the problem of identifying corrupted updates as a multidimensional (i.e., matrix-valued) time series anomaly detection task. 
The key idea is that legitimate local updates, resulting from well-calibrated iterative procedures like stochastic gradient descent (SGD) with an appropriate learning rate, show \textit{higher predictability} compared to malicious updates. This hypothesis stems from the fact that the sequence of gradients (thus, model parameters) observed during legitimate training exhibit regular patterns, as validated in Section~\ref{subsec:intuition}. %until convergence. 
%This regularity may be more pronounced for smooth convex loss functions, but it can still be captured within an appropriate time window, even for more complex and convoluted loss surfaces. 
%We provide evidence of this claim in Appendix~B, where we show that the average mutual information (i.e., ``predictability''), calculated over pairs of legitimate model updates sent at different FL rounds, is significantly higher than the corresponding computation for a malicious client.
\\
Inspired by the matrix autoregressive (MAR) framework for multidimensional time series forecasting~\cite{chen2021je}, we propose the FLANDERS ({\em \textbf{F}ederated \textbf{L}earning meets \textbf{AN}omaly \textbf{DE}tection for a \textbf{R}obust and \textbf{S}ecure}) filter.
The main advantages of FLANDERS over existing strategies like FLDetector~\cite{zhao2020multivariate} are its resilience to large-scale attacks, where $50\%$ or more FL participants are hostile, and the capability of working under realistic non-iid scenarios.
We attribute such a capability to two key factors: $(i)$ FLANDERS works without knowing a priori the ratio of corrupted clients, and $(ii)$ it embodies temporal dependencies between intra- and inter-client updates, quickly recognizing local model drifts caused by evil players. Below, we summarize our main contributions:

\begin{itemize}
\item[{\em(i)}]
We provide empirical evidence that the sequence of models sent by legitimate clients is more predictable than those of malicious participants performing untargeted model poisoning attacks.
\\
\item[{\em(ii)}] 
We introduce FLANDERS, the first pre-aggregation filter for FL robust to untargeted model poisoning based on multidimensional time series anomaly detection.
\\
\item[{\em(iii)}] 
We integrate FLANDERS into Flower,\footnote{\scriptsize{\url{https://flower.dev/}}} a popular FL simulation framework for reproducibility.
\\
\item[{\em(iv)}] 
We show that FLANDERS improves the robustness of the existing aggregation methods under multiple settings: different datasets, client's data distribution (non-iid), models, and attack scenarios.
\\
\item[{\em(v)}] 
We publicly release all the implementation code of FLANDERS along with our experiments.\footnote{\scriptsize{\url{https://anonymous.4open.science/r/flanders_exp-7EEB}}}
\end{itemize}

% Paper's structure and organization
The remainder of the paper is structured as follows. %some related work and the current state-of-the-art solutions to security issues that FL entails. 
Section~\ref{sec:background} covers background and preliminaries. 
In Section~\ref{sec:related}, we discuss related work.
Section~\ref{sec:problem} and Section~\ref{sec:method} describe the problem formulation and the method proposed. % to tackle it. 
Section~\ref{sec:experiments} gathers experimental results. %, and Section~\ref{sec:limitations} discusses some limitations of this work.
Finally, we conclude in Section~\ref{sec:conclusion}.
 %discusses the limitations of this work and draws future research directions.
%reports conclusions and draws perspectives for future research directions.

%%%%%%% OLD %%%%%%%
%to overcome the resilience of Byzantine failures in distributed Stochastic Gradient Descent computations. 
% The strength of Krum is its time complexity, which is linear in the gradient dimension. 
% However, the robustness of the approach is guaranteed for gradient-based learning applications only when the majority of the clients are not compromised. 
% Besides, the aggregation mechanism of Krum, as well as that of similar methods, is robust from a coarse-grained perspective and does not provide solutions to errors and perturbations that may occur at inference time.
%A related approach to~\cite{blanchard2017nips} is the work of Su et al.~\cite{su2016dc}. Here, the authors propose an iterated approximate agreement to tackle a multi-layer scenario attacked by Byzantine agents. 
%However, the method works efficiently on the sole discrete context and it is inapplicable to continuous state environments.
%\gabri{Maybe, we should just talk about the main limitations of existing countermeasures without digging into their details (or, we can just mention Krum as this is the most popular one). I will move the description of all these methods to the Related Work section.}


\section{Literature Review}\label{s.litreview}
\edit{
Our work relates to the expansive streams of literature on reinforcement learning and (approximate) dynamic programming, as well as the applied operations research literature focusing on improving healthcare delivery in resource-limited settings. Methodologically speaking, most existing solution approaches to problems similar to ours can be classified as either (a) developing policies for a \textit{known} underlying model of behavior or (b) \textit{learning} a policy using data. We summarize these two streams of work in Sections \ref{ss.knownmod} and \ref{ss.unknownmod}, respectively, before discussing prior work on TB treatment as an application area in \cref{ss.litreviewhealth}.

\edit{\subsection{Known Model}\label{ss.knownmod}
Our model (described in detail in \cref{s.gmodel}) assumes that every patient behaves according to a Markov decision process (MDP). Even if all the MDP parameters of these models were known exactly, the size of the system state space would be \textit{exponential} in the number of patients: the size of the state space is $|\cS|^N$, where $\cS$ is the state space for one patient and $N$ is the number of patients. As a result, a direct application of dynamic programming techniques such as backwards induction or policy/value iteration would take exponential time and hence is practically infeasible.

As exact methods are infeasible, one can resort to \textit{approximate} dynamic programming (ADP) techniques developed for weakly-coupled MDPs \citep[e.g.,][]{meuleau1998solving,adelman2008relaxations,d2023optimal}. Specifically, the model we study is a \textit{restless bandit}: each patient corresponds to an arm, there is a budget on the number of arms that can be `pulled' (given an intervention) at each time step. The state of each arm evolves as a Markov chain, where its transition probabilities depend on the action taken. It is known that finding the optimal policy to a restless bandit is PSPACE-hard \citep{papadimitriou1994complexity}. There is a large literature on developing algorithms for this problem \citep[e.g., ][]{whittle1988restless,glazebrook2002index,ansell2003whittle,glazebrook2006some,liu2010indexability,guha2010approximation}.
A commonly used policy is the Whittle's index \citep{whittle1988restless}, which is known to be asymptotically optimal under certain conditions \citep{weber1990index} and has been shown to have good empirical performance 
\citep{ansell2003whittle,glazebrook2002index,glazebrook2006some}.
All of the above methods assume that the model is known, whereas our problem has the further nontrivial complication that the model is unknown.
}


\subsection{Unknown Model}\label{ss.unknownmod}
\edit{Deriving an optimal policy for an MDP with unknown parameters is corresponds to reinforcement learning (RL), a rapidly expanding area of research \citep{sutton2018reinforcement}.
However, a naive application of the RL framework onto our problem results in an exponentially large state space --- this correspondingly results in an exponential blowup in the data requirements to deploy generic RL methods such as Q-learning \citep{jin2018q}.
Therefore, more tailored approaches are required to fit to this regime.}


\paragraph{Greedy and multi-armed bandits.}
One approach, as described in the introduction, is to greedily maximize the immediate reward at each time step.
For example, in the treatment adherence setting, one can reach out to patients with the highest increase in their probability of adhering in the next day from the intervention.
A multi-armed bandit is one natural framework that can be used to learn such a policy.
This is the approach used in HeartSteps \citep{lei2017actor,liao2020personalized}, a program to promote physical activity using real-time data collected from wristband sensors.
These papers use a contextual bandit model, where the context represents a user at a particular time step, and they develop a bandit algorithm whose goal is to increase the immediate activity level of the user.
Given the vast literature on contextual bandits, there are a wide variety of algorithms that one can easily plug in.

A fundamental downside of this greedy approach is that it does not capture any potential long-term effects of an action---that is, an action may not only impact a patient's immediate behavior, but their behavior for all future time steps.
One way to address this is to specifically model the type of long-term effect it can have.
For example, \cite{liao2020personalized} introduce a `dosage' variable that models the phenomenon that the treatment effect of an action is often smaller when an action was recently given to that patient in the past.
Similarly, \cite{mintz2020nonstationary} incorporate habituation and recovery dynamics into the bandit framework.
However, these approaches captures only particular types of long-term effects that are explicitly modeled, and there could be other, complex factors that affect the downstream behavior of patients.

Our work does not take this greedy approach, and we aim to learn a policy that maximizes long-term rewards, without specifically modeling the type of long-term effect that an action can have.
We benchmark against a contextual bandit policy, and we observe that incorporating the long-term effects is critical in the behavioral health setting that we study. 

\paragraph{Learning for restless bandits.}
\edit{A non-greedy approach to this problem corresponds to the restless bandit model in the unknown parameter regime.
There is a nascent literature develops learning algorithms in this setting. One approach is to adapt algorithms from the multi-armed bandit literature such as UCB \citep{wang2020restless} or Thompson Sampling \citep{jung2019regret}. 
Recent approaches similarly adapt reinforcement learning methods such as Q-learning \citep[e.g., ][]{fu2019towards,avrachenkov2022whittle}. Importantly, these methods are \textit{online} learning algorithms that require continuous exploration and assume that the state space is known and finite. This literature focuses on providing theoretical guarantees of the proposed algorithms {eventually} converging to the {optimal} policy.
}

Our work differs from the aforementioned literature in a couple of ways.
First, we take an \textit{offline} approach, where we leverage existing data to derive a new policy.
This removes the need for exploration, as well as the dependence on a long horizon to \edit{obtain} an improvement over a baseline policy.
Second, our work does not focus on deriving an \textit{optimal} policy; we derive a practical policy that can be implemented with limited data, and our theoretical results are approximation guarantees to the optimal policy.
Lastly, we do not postulate a simple model of patient behavior, and rather, take a model-free approach.
For example, \cite{mate2022field} and \cite{biswas2021learn} posit a simple MDP with two and three states respectively for each user, where a state represents the engagement level of the user. 
\cite{aswani2019behavioral} take a slightly different model-based approach in studying weight loss interventions, where they model user behavior via utility functions.
Then, the policy is developed based on these posited models.
These approaches rely heavily on the correctness of the models, and cannot take other \textit{non-modeled} factors into account, such as non-stationarity of patient behaviors.
Moreover, it is unclear how policies such as the Whittle's index behave under model misspecification.
In contrast, our model-free approach allows us to incorporate as much information as available (at the time) into the `state' of a patient, and our policy then operations under the assumption that patients in similar states will behave similarly.
We rely on the prediction algorithm that estimates the state-action values to identify the most relevant features of the state.
In \cref{s.theory}, we use a simple 2-state MDP for the purposes of proving a theoretical guarantee for our policy, but the policy is defined irrespective of the underlying patient behavior model.
}


\subsection{Healthcare systems} \label{ss.litreviewhealth}
Finally, from an application perspective, our work contributes to a growing literature focusing on improving healthcare delivery systems in resource-limited settings. Most related to the paper at hand are recent papers focusing on improving TB outcomes in resource-limited settings. Much of this work has been on the policy level, with \cite{Suen14Disease} evaluating strategic alternatives for disease control of multi-drug resistant TB (MDR TB) in India and finding that with MDR TB transitioning from treatment-generated to transmission-generated, rapid diagnosis of MDR TB becomes increasingly important. Similarly, \cite{Suen18Optimal} optimize the timing of sequential tests for TB drug resistance, a necessary step for transitioning patients to second-line treatment. 

Two papers focus on medication adherence. \cite{Suen22Design} tackle the problem of designing patient-level incentives to motivate medication adherence, in situations where adherence is observable but patients have unobserved and heterogeneous preferences for adherence. They first take a modeling approach to design an optimal incentive scheme and then demonstrate that deploying it would be cost effective in the context of TB control in India. Similar to us, \cite{Boutilier22Improving} focus on a behavioral intervention, demonstrating that data describing patient behavior (e.g., patterns of self-verification of treatment adherence, like in the case of Keheala) can be leveraged to predict short-term behavior as well as long-term outcomes. They use such predictions to assign patients to risk groups and demonstrate empirically that outreach can be effective, even for patients who are classified as at risk. However, they stop short of prescribing an actionable policy for assigning patient outreach, which is the topic of this paper.







\section{Full Model and Policy}\label{s.model}
\section{Proposed Framework: {\ourmodel}}
\label{model}


In this section, we introduce a novel self-supervised co-training framework {\ourmodel}.
The proposed framework is illustrated in Figure~\ref{fig:intro_model} and works in three phases.
Phase one automatically generates two sets of pseudo labels.
We use a combination of off-the-shelf pre-trained POS and NER taggers, knowledge graph, and GPT-2 scorer for generating the first set of pseudo labels automatically without any hand-crafted rules for matching the slot values.
The other set of pseudo labels is acquired through a zero-shot slot filling model~\cite{liu2020coach}, trained on the out-of-domain dataset.
It is critical to emphasize that both sets of labels are noisy and incomplete which poses serious challenges to training effective models for the task of open-domain slot filling.
Phase two fine-tunes the pre-trained BERT to the slot filling task that effectively transfers the knowledge from the pre-trained language model~(LM) to overcome the issue of label incompleteness to some extent. 
Further, we employ the early stopping technique to minimize the noise in the labels.
The output of this phase is two BERT models that can generate soft labels for self-supervision during co-training in phase three.
Phase three leverages the fine-tuned models and further trains them in an iterative fashion.
Specifically, the proposed peer training approach facilitates high-confidence soft label selection for the other peer to perform training. This phase progressively reduces the noise in the labels and enables effective model fitting. 



\subsection{Phase One: Automatic Label Generation}
To acquire the first set of labels, we perform the following steps.
First of all, off-the-shelf trained POS and NER taggers are used to predict initial estimates of the slot values irrespective of the slot types. Then, the type information of the slot values is queried from the KG and the slot value is tagged for the most appropriate slot in the target domain.
This approach, however, produces low recall. 
To expand the candidate slot values, we generate n-grams of the natural language text and employ a partial matching scheme to query the KG for type information (e.g., \myspecial{Jason} \myspecial{Aldean} = \myspecial{American} \myspecial{singer}) of the n-grams if the entry exists.
This process generates multiple overlapping hypotheses about the slot values.
We replace a span of text that corresponds to a slot value by its type information and a GPT-2 based scorer (see Section~\ref{sec:nlpmodels}) is used to select the best candidate based on the fluency of the text.
Naturally, if a token (or span of tokens) is replaced by its type, the sentence should score higher as compared to the case where an inappropriate substitution is performed. 
We select the best hypothesis if the score is greater than the threshold.
Intuitively, the candidate selection threshold can automatically be searched based on a small validation set from the target domain, making the label generation process fully automatic. 
The other set of noisy labels is acquired by the zero-shot slot filling model~\cite{liu2020coach} that has been trained using an out-of-domain dataset. It is important to highlight that the zero-shot slot filling model does not require any labeled in-domain training example. 
To summarize the automatic label generation phase, both sets of labels are acquired in a fully automatic fashion without any hand-crafting.


In contrast to previous work in weak supervision~\cite{ren2015clustype,he2017autoentity,fries2017swellshark,giannakopoulos2017unsupervised} that obtains a single set of noisy labels and then propose techniques to overcome the challenge of fitting an effective model to the noisy labels, we acquire two sets of complementary labels.
The choice of these two sets of labels is guided by the intuition that they should be complementary and the models trained on these sets of labels should be able to share complementary information with the other to improve the performance in the later phases of the framework.
Essentially, the first set of labels carries information from external knowledge sources, whereas the labels generated through the pre-trained zero-shot slot filling model capture how the slot values are mentioned in other domains.
%
To further elaborate on the motivation and our process for the first set of labels (i.e., labels using KG and other NLP models), the pre-trained LMs have been shown to have a great deal of knowledge~\cite{petroni2019language}, thus should be capable of generating automatic labels with no need of external KG. 
To the best of our knowledge, there exists no work that shows that accurate token-level automatic labeling (e.g., slot filling task) is possible with pre-trained LMs. 
Moreover, such approaches would require heavy prompting in each new target domain, whereas our label generation process is fully automatic and only relies on the readily-available pre-trained NLP models and external KG.

\subsection{Phase Two: LM-assisted Weak Supervision}
Since we do not have access to dataset $\{(\mathbf{X}_n,\mathbf{Y}_n)\}_{n=1}^N$ with true ground-truth labels.
We use pseudo labels generated in phase one, $\{(\mathbf{X}_n,\mathbf{D}_n)\}_{n=1}^N$, to learn 
$f_{m,c}(\cdot; \cdot)$ that outputs the probability of the $m$-th token to take on class $c$. 
We learn $f_{m,c}(\cdot; \cdot)$ by minimizing the following loss over the noisy dataset $\{(\mathbf{X}_n,\mathbf{D}_n)\}_{n=1}^N$: 
$$
\hat\theta = \argmin_{\theta}\frac{1}{N}\sum_{n=1}^{N} \ell(\mathbf{D}_n, f(\mathbf{X}_{n}; \theta)),
\label{eq:stage1}
$$
where $\ell(\mathbf{D}_n, f(\mathbf{X}_{n}; \theta)) = \frac{1}{M} \sum_{m=1}^{M} -\log{f_{m,d_{n, m}}(\mathbf{X}_{n}; \theta)}$. 
We employ the pre-trained multilingual BERT with token-level classification head that uses Adam optimizer \cite{kingma2014adam,Liu2019} with early stopping and multiple random initializations. 


Since slot filling task is similar to the MLM training objective of the BERT, we employ pre-trained BERT as the backbone model.
That is, MLM's goal is to predict the masked tokens using bidirectional contexts. Similarly, slot filling tries to predict the label for a token leveraging both left and right contexts simultaneously, which makes the pre-trained BERT an ideal model of choice that greatly facilitates minimizing incomplete labels.
It is important to highlight that our automatically generated labels are not only incomplete but also potentially wrong.
The training strategies employed in this phase minimize the noise in the label to some extent. 
Specifically, early stopping can provide a strong regularization and would not let the model overfit to the noisy labels, especially wrong labels. 
Moreover, early stopping does not let the model forget the knowledge in the pre-trained model.
Similarly, multiple random initializations enforce robustness. 
Since the model is fine-tuned on the noisy labels, averaging the predictions of multiple models for each token ensures that wrong labels end up with low probabilities and true labels consistently achieve high probabilities.
Using the above-mentioned strategies, we train two slot filling models, which we call the peers. The peer one is trained on the first set of pseudo labels that were generated using POS and NER taggers, KG, and the GPT-2 scorer in phase one. Similarly, peer two is trained using the predictions of the zero-shot slot filling model~\cite{liu2020coach}.
Both models have the same architecture and follow the same training procedures.

\begin{table*}[t!]
\centering
\caption{Dataset statistics.}
\vspace{-7pt}
\label{tab:dataset}
\begin{tabular}{lccccc}
\toprule
\textbf{Dataset}  & \textbf{Dataset Size} & \textbf{Vocab. Size} & \textbf{Avg. Length} & \textbf{\# of Domains} & \textbf{\# of Slots} \\ \hline
\textbf{SGD}      & 188K                  & 33.6K                & 13.8                 & 20                     & 240                  \\
\textbf{MultiWoZ} & 67.4K                 & 10.5K                & 13.3                 & 8                      & 61 \\
\bottomrule
\end{tabular}
\vspace{-7pt}
\end{table*}

\subsection{Phase Three: Self-supervised Co-training}
We introduce an iterative peer training algorithm where both peers generate high-confidence soft labels for training the other peer in the next iteration. 
Theoretically, these peers can be anything, but in this work, 
we explore two of the most promising directions that have shown the promise to minimize the need for manual labeling for the task: zero-shot learning and distant supervision.
This phase uses a self-supervised co-training scheme to exploit the patterns of slot values from other domains through the labels generated by the zero-shot filling model (i.e., peer two)~\cite{liu2020coach} as well as utilize the knowledge in external KGs and pre-trained models via labels provided by the peer one.
Specifically, we initialize the peers trained in phase two and use their pseudo labels to kick-start training in this phase.
Specifically, peer one $f_{m,c}(\cdot; \theta_{\textrm{p1}})$ would generate labels $\{\tilde{\mathbf{Y}}^{(t)}_n = [\tilde{y}_{n,1}^{(t)}, ..., \tilde{y}_{n,m}^{(t)}]\}_{n=1}^{N}$ for peer two $f_{m,c}(\cdot; \theta_{\textrm{p2}})$ at the $t$-th iteration by:
$$
\tilde{y}_{n,m}^{(t)} = \argmax_{c}{f_{m,c}(\mathbf{X}_n; \theta_{\textrm{p1}}^{(t)})}. 
\label{eq:pseudo}
$$

Based on these labels, the peer two can be fine-tuned by: 
$$
\hat\theta_{\textrm{p2}}^{(t+1)} = \argmin_{\theta}\frac{1}{N}\sum_{n=1}^N \ell(\tilde{\mathbf{Y}}_n^{(t)}, f(\mathbf{X}_{n}; \theta)).
\label{eq:self_train1}
$$

Similarly, peer two $f_{m,c}(\cdot; \theta_{\textrm{p2}})$ would generate pseudo labels for peer one $f_{m,c}(\cdot; \theta_{\textrm{p1}})$ that are used to fine-tune peer one. 
We also notice that it is beneficial to stop early during this phase as well, to improve the model fitting and gradually reduce the noise associated with the automatically generated labels.
Since pseudo labels are refined gradually in an iterative way, both peers can benefit from the knowledge contained within the labels of the other while avoiding overfitting.
Furthermore, as an alternative to pseudo labels, we also generate soft labels that are used for confidence re-weighting. 
The high-confidence soft label selection strategy enables better model fitting and efficient learning via better quality of the automatic labels.
Specifically, for the given $m$-th token in the $n$-th training example, the probability for all classes $C$ is $[f_{m,1}(\mathbf{X}_n;\theta),...,f_{m,C}(\mathbf{X}_n;\theta)]$. 
Following ~\cite{xie2016unsupervised}, at $t$-th iteration, peer one generates soft labels, $\{\mathbf{S}_n^{(t)} = [\mathbf{s}_{n,m}^{(t)}]_{m=1}^M \}_{n=1}^N$, as given below:
$$
\mathbf{s}_{n,m}^{(t)} = [s_{n,m,c}^{(t)}]_{c=1}^{C} = \Bigg[  \frac{f_{m,c}^2(\mathbf{X}_n;\theta_{\textrm{peer1}}^{(t)})/p_{c}}{\sum_{c'=1}^C f_{m,c'}^2(\mathbf{X}_n;\theta_{\textrm{peer1}}^{(t)})/p_{c'}}\Bigg]_{c=1}^{C}
\label{eq:soft}
$$ 
where $p_{c} = \sum_{n=1}^N \sum_{m=1}^M f_{m,c}(\mathbf{X}_n;\theta_{\textrm{p1}}^{(t)})$ computes the frequency of the tokens for the $c$-th class. 
Then, peer two $f(\cdot; \theta_{\textrm{p2}}^{(t+1)})$ is fine-tuned by:
$$
\theta_{\textrm{p2}}^{(t+1)} = \argmin_{\theta} \frac{1}{N} \sum_{n=1}^{N} \ell_{\rm KL}(\mathbf{S}_n^{(t)}, f(\mathbf{X}_{n}; \theta)),
$$
where $\ell_{\rm KL}(\cdot,\cdot)$ is the KL-divergence-based loss:
$$
\ell_{\rm KL}(\mathbf{S}_n^{(t)}, f(\mathbf{X}_{n}; \theta))=\frac{1}{M}\sum_{m=1}^M\sum_{c=1}^C - s_{n,m,c}^{(t)} \log f_{m,c}(\mathbf{X}_{n}; \theta).
\label{eq:klloss}
$$

Moreover, we also investigate selecting tokens that have high confidence. 
For instance, we pick high-confidence tokens from the $m$-th input example at the $t$-th iteration by  
$
H^{(t)}_n = \{m : \max_{c} s_{n,m,c}^{(t)} > \epsilon \},
$
where $\epsilon\in [0,1]$ is a threshold that can be searched based on a small validation set. 
Then, peer two $f(\cdot; \theta_{\textrm{p2}}^{(t+1)})$ is fine-tuned by:
$$
\theta_{\textrm{p2}}^{(t+1)} %&= \argmin_{\theta} \frac{1}{N} \sum_{n=1}^{N} \ell_{\rm S-KL}(\bS_n^{(t)}, f(\bX_{n}; \theta)) \\
= \argmin_{\theta} \frac{1}{N|H^{(t)}_n|}\sum_{n=1}^{N} \sum_{m\in H^{(t)}_n}\sum_{c=1}^C - s_{n,m,c}^{(t)} \log f_{m,c}(\mathbf{X}_{n}; \theta).
$$

This phase improves the robustness to effectively fit the model for tokens with high confidence. 
Both peers keep sharing information and their confidence by producing soft labels for their counterparts until they approximate to the true labels while employing early stopping and scheduled learning rates.
It is important to remind that phase three is the most important phase that progressively reduces noise from the labels to a great extent and enables superior performance for the task of open-domain slot filling.


\section{Case Study: TB Treatment Adherence}\label{s.casestudy}
In this section, we revisit the full problem (introduced in \cref{s.model}) and conduct numerical experiments to evaluate the performance of $\DPI$ for its intended use case. Our analysis is motivated by our partner organization, Keheala, which operates a digital health platform to support medication adherence among TB patients in highly resource-constraint settings. Here, we first summarize the state of the global TB epidemic and the Keheala behavioral intervention (\cref{ss.keheala}). We then describe our data sources and the validated simulation model we have developed to test outreach policies (\cref{ss.data} and \cref{ss.sim}). Next, we discuss how our policy, as well as some benchmark policies, can be implemented using Keheala's data (\cref{ss.policies}), before presenting our numerical results (\cref{ss.results}).

\subsection{The global TB epidemic and the Keheala intervention}\label{ss.keheala}
TB remains one of the deadliest communicable diseases in the world, causing 1.6 million deaths in 2021. This is remarkable in light of the fact that effective treatment has been available for over 80 years, with the current WHO guidelines recommending a 6 month regimen of antibiotics for drug-susceptible TB and a more intense regiment for drug-resistant strains \citep{World22Global}. A key limiting factor for curbing the epidemic is lack of patient adherence to these treatment regiments, which increases the probability of infection spreading, drug resistance, and poor health outcomes \citep{garfein2019synchronous}. 

Keheala was designed to provide treatment adherence support to TB patients in resource-limited settings. Their platform operates on the Unstructured Supplementary Service Data (USSD) mobile phone protocol, which importantly allows phones without smart capabilities to access the service. Once a patient has enrolled with Keheala, they are meant to self-verify treatment adherence every day, using their mobile phone. In addition, they have access to a range of services. Some are on-demand, for example educational material about TB or leaderboards for verification rates. Others are automatic, such as adherence reminders, which are sent to patients daily (at their pre-determined medication time) in the absence of verification. In addition, the Keheala protocol is to escalate outreach interventions when patients do not self-verify adherence. It states that after one day of non-adherence patients should receive a customized message to encourage resumed adherence and after two days of non-adherence patients should receive a phone call from a support sponsor. While these support sponsors are full-time employees, they are not healthcare professionals. They are members of the local community who have experience with TB treatment and are therefore familiar with the many contributing factors associated with low treatment adherence, such as side-effects, societal stigma against TB patients, and challenges with refilling prescriptions.

The overall effectiveness of Keheala's combination of services was evaluated in a randomized controlled trial (RCT) in Nairobi, Kenya. The trial demonstrated that Keheala reduced unsuccessful TB treatment outcomes—a composite of loss to follow-up, treatment failure, and death—by roughly two-thirds, as compared to a control group that received the standard of care \citep{Yoeli19Digital}. Given this success, Keheala's primary practical objective is to ensure that enrolled patients remain engaged with the platform through adherence verification.

In this paper, we focus on the final level in Keheala's escalation protocol---support sponsors making phone calls to patients. This part of the outreach was organized through populating a daily list of patients who had not verified treatment adherence for 48 hours. Support sponsors had many responsibilities in operating the platform, but would make phone calls to as many patients on the list as possible on a given day. Since hiring support sponsors is a costly aspect of operating the service, Keheala is interested in implementing a more personalized and targeted approach for prioritizing which patients should receive a phone call on a given day. Being able to maintain a similar performance with fewer support sponsors (or equivalently, serve more patients with the same number of support sponsors) is desirable for any future scale-up of the system.








\subsection{Data sources.}\label{ss.data}
Based on the success of the first RCT, the effectiveness of Keheala was further evaluated in a second RCT\footnote{The trial was approved by the institutional review board of Kenyatta National Hospital and the University of Nairobi. Trial participants or their parents or guardians provided written informed consent. The trial’s protocol and statistical analysis plan were registered in advance with ClinicalTrials.gov (\#NCT04119375).} during 2018-2020. The RCT was conducted in partnership with 902 health clinics distributed across each of Kenya's eight regions, representing a mix of rural and urban clinics. The study included four treatment arms and enrolled over 15,000 patients. We obtained data for 5,433 patients enrolled in the Keheala intervention arm (other arms aimed to independently test specific components of the Keheala intervention). 

As part of the RCT, the study team collected socio-demographic information from all patients. This information includes static covariates such as age, gender, language preferences, location, as well as limited clinical history (see \cref{s.app.list_features} for a full list). In addition, Keheala collected engagement data about each patient during their enrollment in the service. This includes whether a patient verified on a given day, how many reminders they received, and whether they were contacted by a support sponsor. 

After filtering out patients with missing information or not enough data, we conducted all our analysis on 3594 patients. The average patient was enrolled on the platform for 118 days. On an average day, 608 patients were enrolled and 210 of those were eligible for a support sponsor call according to the protocol (i.e., having not verified treatment adherence for the preceding 48 hours). The support sponsors, employed by Keheala, had a range of responsibilities in operating the platform, including making outreach phone calls to the eligible patients. 
The average number of calls made per day was 25.5. Hence, in our analysis, we use a budget of $B$ = 26 as our main point of comparison.
	

\subsection{Simulation Model.}\label{ss.sim}
We build a simulation model that we use to estimate the counterfactual outcomes of different outreach approaches.
The simulator is effectively represented by a single function, $f(S, A) \in [0, 1]$, which denotes the probability that a patient in state $S$ with action $A$ verifies in the next time step.
This function is used to simulate one step transitions for every patient.
We first describe the state space of the patients, describe the exact simulation procedure, and then discuss how we learn $f$ from data and validate the simulator.

\subsubsection{Patient state space.}\label{ss.simstatespace}
For patient $i$, let $X_i \in \bR^{13}$ be their static covariates. 
Let $V_{it} \in \{0, 1\}$ denote whether patient $i$ verified at time $t$, and let $A_{it}\in \{0, 1\}$ denote whether the patient received the intervention at time $t$.
Let $H_{it} = (V_{i1}, A_{i1}, \dots, V_{i, t-1}, A_{i,t-1}, V_{it}) \in \bR^{2t-1}$ be the history of verifications and interventions up to time $t$.
We define a \textit{condensed} history $\tH_{it} \in \bR^{21}$ by summarizing the history $H_{it}$ into 21 features, aiming to capture as much relevant information as possible.
The condensed history contains the patient's recent and overall behavior.
For statistics such as the number of times the patient verified and the number of interventions they have received, we aggregate them over the past week, as well as in total.
We also include information on their verification and non-verification streaks, as well as how long they have been in the platform.
See \cref{s.app.list_features} for a full list of these features.
Then, we define the state of patient $i$ at time $t$ to be $S_{it} = (X_i, \tH_{it}) \in \bR^{34}$.

\subsubsection{Simulation procedure.} 
Given $f$ and an intervention policy $\pi$, we `mimic' the RCT by simulating patient behavior day by day. 
In total, we simulate $T=700$ time steps, where each $t \in [T]$ corresponds to one day between April 2018 to March 2020. We let $T_{s}(i)$ and $T_{e}(i)$ denote the starting and ending time steps that patient $i$ was enrolled in Keheala. Each patient $i$ is then introduced into the system at time $t=T_s(i)$, and removed at time $t=T_e(i)$. We use the observed data from the RCT for their first 7 days in the system to initialize their state. 
Then, in each time period, given the set of patients that were active in the RCT for more than 7 days, we use a policy $\pi$ on these patients to determine who receives sponsor outreach. If a patient $i$ was in state $S_{it}$ at time $t$ and the policy $\pi$ chose action $A_{it}$, we let $V_{i, t+1}$ be 1 with probability $f(S_{it}, A_{it})$, and 0 otherwise (where the randomness is independent across patients and time steps).
Finally, we use $V_{i, t+1}$ to update their state for the next time step. 

\subsubsection{Estimating the $f$ function.} \label{s.learnf} 
Using the state space $\cS$ as described above, we construct the function $f: \cS \times \{0, 1\} \to [0, 1]$ using data from the RCT. Specifically, we learn the two functions $f(S, 0)$ and $f(S, 1)$ separately. For $f(S, 0)$, we train a gradient boosting classifier on the dataset $\{(S_{it}, V_{i,t+1})\}_{i \in N, t \in [T] : A_{it} = 0}$, using $V_{it}$ as the outcome variable. For $f(S, 1)$, we write the function as $f(S, 1) = f(S, 0) + \tau(S)$ and we learn $\tau(S)$ using the double machine learning method of estimating heterogeneous treatment effects \citep{chernozhukov2018double}. See \cref{s.app.simulation} for details on implementing this method.

\subsubsection{Train and test split.} \label{s.traintestsplit}
Importantly, we use a different set of patients to estimate the $f$ function (and to run our simulations) from the set of patients we use to train our policies. In particular, we randomly split all patients from the RCT into two groups, which we call \textit{train} and \textit{test}. We use the \textit{test} set to estimate the $f$ function that forms the basis for the simulation model. We keep the \textit{train} set of patients separate and use it to train policies (see \cref{ss.policies}). This ensures that the policies we evaluate are not learned off of the same dataset that was used to learn the simulator. 
The simulation itself uses the test patients, and we duplicated each patient so that we maintain a similar total number of patients as in the original study.


\subsubsection{Simulation validation.}
We validate the performance of the simulator on a \textit{different} intervention policy than the simulator was trained on, by leveraging the fact that there was variability in the number of interventions given throughout the RCT.
In particular, the average number of interventions given during the first half of the RCT was around double of that of the latter half (45.9 vs. 21.4),
and this variation induces a change in the intervention assignment policy.
Then, dividing the data into halves produces two datasets that are generated using effectively different intervention policies.

To validate the simulator, we use the method from \cref{s.learnf} to learn $f$ using the first dataset, and then validate its performance on the second dataset. 
Using this procedure, the AUCs on the second dataset for $f(S, 0)$ and $f(S, 1)$ were 0.918 and 0.745, respectively.
We also check the calibration of both of these functions, by grouping the samples into bins based on their predicted probability of verifying the next day, and checking whether their actual verification rates.


We group the samples based on the simulated probability of verification into bins with a 10\% range, and we compute the expected calibration error (ECE) \citep{naeini2015obtaining}. For bin $i$, let $o_i$ be the true fraction of positive instances in bin $i$, $e_i$ be the mean of the predicted probabilities of the instances in bin $i$, and $N_i$ be the number of samples in bin $i$. Then, the ECE is defined as
\begin{align}
\text{ECE}	 = \frac{1}{N} \sum_{i=1}^{10} N_i |o_i - e_i|,
\end{align}
where $N$ is the total number of samples.
The expected calibration error was 0.0066 for $f(S, 0)$ and 0.0308 for $f(S, 1)$.
Figure~\ref{f.calibration} displays these bins.

These results demonstrate that the simulator has good performance in mimicking patient behavior. 
As expected, the AUC and the ECE is worse for $f(S, 1)$ compared to $f(S, 0)$; 
this is due to the \textit{significantly} fewer number samples with an intervention in the training data, as well as the increase in variance of doing off-policy estimation.
The training data used for $f(S, 0)$ had 300K samples, while the one used for $f(S, 1)$ had 4.5K samples.
For $f(S, 1)$, the calibration is slightly off for samples with a high probability of verification (bins 0.7-0.9); however, we note that the 0.7-0.9 bins only contain 11.3\% of all samples for $f(S, 1)$.


\begin{figure}[h]%
	\centering
	\subfloat[Calibration for $f(S, 0)$]{{\includegraphics[width=0.48\linewidth]{figs/bar_pred0} }}%
	\subfloat[Calibration for $f(S, 1)$]{{\includegraphics[width=0.48\linewidth]{figs/bar_pred1} }}%
	\vspace{2mm}
	\caption{Calibration plots for $f(S, 0)$ and $f(S, 1)$ for simulation validation. 
		We group the samples based on the simulated probability of verification
		into bins with a 10\% range, which we label by the lower number. 
		For example, the 0.3 bin on the x-axis represents the samples whose probability of verification according to $f$ is in $[0.3, 0.4)$; hence we should expect the actual number of verifications of those samples to be close to 0.35.}
	\label{f.calibration}%
	\vspace{-2mm}
\end{figure}



\subsection{Outreach Policies and Experimental Design}\label{ss.policies}
Using the simulation model described above, we compare the performance of three main policies. For each policy, we vary the budget for outreach interventions per day between 10 and 40. As mentioned before, the average number of sponsor outreaches during a given day of the trial was 26. Importantly, we restrict all policies so that they can only provide an outreach to patients who have not verified for at least two days in a row. This is because that was what was done in the RCT, hence there is no data for how an outreach affects behaviors for those who do not meet this criterion (thus we would not be able to accurately evaluate policies that do not follow this restriction).

We note that attaining improved performance with lower outreach capacity is particularly important for the resource-limited setting at hand as it speaks to the performance achievable during a future scale-up of the system, in which the ratio of patients to support sponsors is likely to be much higher. 
Next, we describe the implemented policies.

\subsubsection{$\DPI$ for Keheala.} 
The first step in operationalizing $\DPI$ is defining the state space for each patient. For this, we use the same condensed state space as described in \cref{ss.simstatespace}, i.e., we define the state of patient $i$ at time $t$ to be $S_{it} = (X_i, \tH_{it}) \in \bR^{34}$ (a full list of these features is included in \ref{s.app.list_features}). Importantly, we note that all of the features of this state space are observable to Keheala at any time $t$, once a patient has been enrolled on the platform for seven days. 

The second step is estimating the $\hz_{it}(S_{it})$ score for each patient at each time period, which is ultimately used to prioritize patients. 
As before, we let $T_{s}(i)$ and $T_{e}(i)$ be the starting and ending time steps that patient $i$ was enrolled in Keheala. Using this notation, we can represent the future verification \emph{rate} for patient $i$ at time $t$ by $y_{it} = \frac{1}{T_{\text{e}}(i)-t} \sum_{r=t+1}^T V_{ir}$.
Then, the data from the RCT can be written in the form $\{(S_{it}, A_{it}, y_{it})\}_{i \in [N], t \in \{T_{s}(i), \dots, T_{e}(i)\}}$, and we can estimate the function $q_{it}^{\baseline}(S, A)$ using this data.
In our implementation, we use a linear function approximation for the verification rate, assuming the form 
\begin{align*}
q_{it}^{\baseline}(S, A) = \langle \theta_A, S \rangle \cdot (T_{\text{e}}(i)-t),
\end{align*}
for each of the two actions $A \in \{0, 1\}$.
The $\langle \theta_A, S \rangle$ term represents the future verification rate, and $T_{\text{e}}(i)-t$ represents the number of days left; combined, $q_{it}^{\baseline}(S, A)$ represents the total number of future verifications.
We note that the state contains information regarding the number of days the patient has been enrolled in Keheala, hence the verification rate is also a function of the time step.

We estimate $\theta_a$ using least squares with an $\ell_2$ regularizer:
\begin{align} \label{eq:leastsquares}
	\hat{\theta}_a &\in \argmin_{\theta \in \bR^{34}} \bigg( \sum_{i \in N} \sum_{t=T_{\text{s}}(i)}^{T_{\text{e}}(i)}  \bI(A_{it} = A)(y_{it} - \theta^\top S_{it})^2 + ||\theta||^2_2 \bigg).
\end{align}

Finally, we compute a patient's estimate of their intervention value at time $t$ as
\begin{align*}
	\hz_{it}(S_{it}) = \langle \htheta_1 - \htheta_0,S_{it} \rangle \cdot (T_{\text{e}}(i)-t), 
\end{align*}
and the resulting policy is to give the intervention to up to $B$ patients with the highest positive $\hz_{it}(S_{it})$ values.





\subsubsection{Bandit.}
The bandit policy aims to choose patients with the highest increase in the probability of next-day verification, using a linear contextual bandit model. In terms of the two-state model from \cref{ss.2statemodel}, the goal is to choose patients with the highest value of $\tau$.
We essentially use the same state space and linear model as was used for $\DPI$, except that the outcome variable is next-day verification, rather than total future verifications.
We first learn a prior using the offline data, and then we run a Thompson sampling style policy, which continually updates the policy with online data.

Specifically, we assume the linear form $V_{i,t+1} = \langle \beta_a, S_{it} \rangle$ for action $a \in \{0, 1\}$, with unknown parameters $\beta_0, \beta_1 \in \bR^{34}$.
The prior on $(\beta_0, \beta_1)$ is initialized as the output of a least-squares regression using the offline data, the same data that was used to train $\DPI$.
At each time step, $(\tilde{\beta_0}, \tilde{\beta_1})$ is sampled from the posterior. 
Then, the policy chooses the $B$ patients with the highest value of $\langle \tilde{\beta_1}, S_{it} \rangle - \langle \tilde{\beta_0}, S_{it} \rangle$.
After the outcome is observed at each time step, the posterior is updated accordingly.
The detailed description on the algorithm can be found in Section~\ref{sec:app:bandit}.

This policy makes use of strictly more data than $\DPI$, since $\DPI$ only uses the offline data. 
In the results, we confirm that this policy indeed learns myopic rewards correctly.
Therefore, this is a very strong benchmark algorithm for optimizing myopic rewards.


\subsubsection{Whittle's index (QWI).} \label{sec:qwi_description}
The next benchmark is the Whittle's index.
The advantage of this method compared to the bandit benchmark is that is is non-myopic.
However, the downside is that computing the Whittle's index requires the model to be known.
To implement Whittle's index in our setting where the model is unknown, we leverage the recent work of \cite{avrachenkov2022whittle} who propose a Q-learning approach to learn the Whittle's index, which we refer to as $\QWI$.
$\QWI$ is an online learning method that simultaneously learns the Q-values as well as the Whittle's index for each state.

There are two main challenges in implementing $\QWI$ in our setting. The first is that the algorithm is an online learning method, and the second is that it requires a finite state space as it learns the Whittle's index for each state separately. 
For the first point, we adapt the algorithm from \cite{avrachenkov2022whittle} to an offline setting so that it can use the same data that is used to train $\DPI$.
For the second point, we define a smaller, finite state space so that $\QWI$ can be implemented.
We define a patient's state at a point in time to be a 3-tuple $(s_1, s_2, s_3)$, where $s_1$ represents the number of times the patient verified in the last week, $s_2$ is the patient's historical total verification rate, and $s_3$ is the number of times that the patient received an intervention in the last week. The values of each of these terms are bucketed into a small number of bins (3 bins for $s_1$ and $s_3$, 5 bins for $s_2$), resulting in 45 states in total. Specifically, the bins for both $s_1$ and $s_2$ are $0, 1$ and $2-7$. For $s_2$, the bins are $0-1\%$, $1-5\%$, $5-20\%$, $20-45\%$, and $45-100\%$. The bin values were chosen to balance the number of samples in each bin. 
This results in 45 states in total.


Based on this state space, we learn the Whittle's index, $\lambda(s) \in \bR$, for each state $s$.
Then, at each point in time, $\QWI$ chooses the patients in states with the highest Whittle's index to give the intervention to.
Further details of the learning algorithm is deferred to Appendix~\ref{sec:app:qwi}. 

\added{
We note that the state space for $\QWI$ is different from that of $\DPI$, due to the computational limitation of $\QWI$. In \cref{ss.state_space_results}, we run simulations where we modify the state space for $\DPI$ to be the same as $\QWI$, so that we can isolate the performance difference to the algorithm rather than the state space.
That said, we believe that the ease of working with an infinite state space is a substantial advantage of $\DPI$.
}




\subsubsection{Baseline.}
The $\baseline$ policy approximates the heuristic followed by Keheala in the two RCTs that have been implemented. In both cases, the protocol was that patients were added to the support sponsor call queue after not verifying for 48 hours. As a result, the order of patients in the queue is effectively random, determined by a combination of their designated medication time (which prompts automated reminders to take the medicine and verify) and the timing of their self-verification. We approximate the resulting outreach policy by selecting $B$ patients out of all those who have not verified for 48 hours, at random. 

\added{
The $\RAND$ policy used in the theoretical results is aimed to be an approximation of $\baseline$. 
The discrepancy between these policies is solely for technical convenience, 
as $\RAND$ is easier to analyze due to the independence across patients.
}


\subsubsection{Null policy.}
\added{
Lastly, we simulate the $\NULL$ policy which does not give any interventions. 
We note that this policy does not depend on the budget parameter $B$.
}

\subsection{Results}\label{ss.results}

The results are shown in Figure~\ref{fig:main_results}. 

\begin{figure}[h]
\begin{center}
\vspace{-3mm}
  \includegraphics[width=1\linewidth]{figs/results_June14_whittle}
  \caption{Average overall verification rate over 50 runs for each policy and budget. 
  The overall verification rate for the $\NULL$ policy was 54.2\%.
  The shaded region indicates a 95\% confidence interval. The star represents the operating point for Keheala.}
  \label{fig:main_results}
\vspace{-6mm}
\end{center}
\end{figure}

\subsubsection{Overall performance.}
The average performance for each budget and policy over 50 runs are shown in Figure~\ref{fig:main_results}, which shows that $\DPI$ clearly outperforms the other policies over a wide range of budget values.
For a practical interpretation of the results, consider \textsf{Baseline} at a budget of 26, the policy and budget that Keheala was operating during the RCT, which results in an overall verification rate of 62.0\%.
By using less than \textit{half} of the budget, $B=12$, $\DPI$ achieves the same verification rate at 62.2\%.
As the costliest aspect of Keheala's system is in hiring staff to provide the interventions, these results imply that they can cut these costs by half to achieve the same outcome.
\added{
The $\NULL$ policy (no interventions) results in a verification rate of 54.2\% (we did not plot this for readability of the figure).
One can interpret this number as a reference benchmark to compare the effectiveness of interventions.
When the budget is 26, $\baseline$ improves over $\NULL$ by 14.6\%, while $\DPI$ improves over $\NULL$ by 20.3\%. 
Therefore, $\DPI$ improves the effectiveness of the interventions over $\baseline$ by 38.3\%.
}

Additionally, we observe that the improvement of $\DPI$ compared to the other policies is especially substantial for smaller budgets. 
This implies that when the number of patients that can be targeted is small, $\DPI$ can correctly identify the set of patients to target that result in the largest gains.
This is especially valuable for scaling up the system.
Indeed, if Keheala wanted to expand to include more patients without linearly increasing their staff costs, then the ratio of budget to the number of patients would decrease, resulting in the regime where $\DPI$ offers major improvements.

The fact that the performance of $\bandit$ policy improves over $\baseline$ as the budget increases is caused by the increase in relevant data.
Note that the number of interventions is small ($\sim 26$) relative to the number of patients in the system at once ($\sim 600$), implying that the number of data points with $A=1$ is much smaller than that of $A=0$. 
Therefore, the main bottleneck in estimation is learning patient behaviors after receiving an intervention.
As the budget increases, the $\bandit$ has access to more data from patients with an intervention, and hence is able to improve its learning. 

$\QWI$ has a slightly inconsistent performance curve relative to the other policies.
Its performance is always better than or similar to $\baseline$, but compared to $\bandit$, it over-performs in the mid-budget regime, but under-performs as the budget increases.
We dive deeper into the types of patients each policy targets in \cref{sss.targetedpatients}, where we provide an explanation for this behavior.  


\added{
One factor that may be contributing to the poor performance of $\QWI$ is the state space that is used.
$\QWI$ uses a discretized state space as described in \cref{sec:qwi_description}, different than the infinitely-sized state space used by $\DPI$.
In \cref{ss.state_space_results}, we run additional experiments where we run $\DPI$ with the same state space as $\QWI$, so that the performance differences can be purely attributed to the algorithm rather than the state space. 
}


\subsubsection{Patient-level verification rates.} 
The overall number of verifications increases under $\DPI$, but how do these rates get impacted at the patient-level?
Fixing the budget to be 26, we compute the verification rate of \textit{each} patient, and we examine the distribution of these patient-level rates. \added{
In Figure~\ref{fig:diff_vrates_all}, we plot how the distribution of patient verification rates shift under the $\DPI$, $\bandit$, and $\QWI$ algorithms, compared to $\baseline$.
We see that under $\DPI$, the distribution shifts in a way that there are fewer patients with verification rates under 50\%, and more patients with a verification rate higher than 50\%.
We see a similar phenomenon for $\QWI$, but the magnitude of the shift is smaller.
$\bandit$ also observes an increase in $>50\%$ verification rates, but there is also an increase of those with very low (0-10\%) verification rates.
These results for $\DPI$ represent a desirable type of shift, where main improvement of $\DPI$ comes from an increase in the number of patients with a high verification rate.
We also provide absolute numbers in \cref{tab.verification_rates}, where we show the percentage of patients whose verification rate is higher than 50\% and 70\% for each of the four algorithms.
Under $\DPI$, the number of patients whose verification rate is above 50\% and 70\% increased relatively by 6.7\% and 5.2\% respectively compared to $\baseline$.
}


\begin{figure}
\centering
\begin{subfigure}{.47\textwidth}
  \centering
  \includegraphics[width=1\linewidth]{figs/2024_dpi-keheala}
  \caption{Comparing $\DPI$ to $\baseline$.}
\end{subfigure}%
\begin{subfigure}{.47\textwidth}
  \centering
  \includegraphics[width=1\linewidth]{figs/2024_bandit-keheala}
  \caption{Comparing $\bandit$ to $\baseline$.}
\end{subfigure} \\
\begin{subfigure}{.47\textwidth}
  \centering
  \includegraphics[width=1\linewidth]{figs/2024_qwi-keheala}
  \caption{Comparing $\QWI$ to $\baseline$.}
\end{subfigure}
\caption{
\added{
Differences in the distribution of patient verification rates compared to $\baseline$. 
  The bins represent the difference in the number of patients whose overall verification rate is between $0-10\%$, $10-20\%, \dots, 90-100\%$.
For example, the first bin in (a) shows that there were 28 fewer patients whose verification rate was between 0 and 10\% under $\DPI$, compared to $\baseline$.
 There were 3594 patients in total, and the budget was fixed at 26.
 }
}
  \label{fig:diff_vrates_all}
\end{figure}


\begin{table}[h]
\TableSpaced %
\caption{\added{
The average percentage of patients whose verification rate was over 50\% and 70\% across the four algorithms.
 There were 3594 patients in total, and the budget was fixed at 26.
}} \label{tab.verification_rates}
\vspace{2mm}
\begin{center}
\begin{tabular}{@{}c|cccc@{}}
\toprule
\% patients with \\ verification rate      & \quad $\baseline$ \;  & $\DPI$ & $\bandit$& $\QWI$ \\ \midrule
$\ge 50\%$  &  61.7\%   & 66.1\%  & 63.8\% & 63.9\% \\ 
$\ge 70\%$  &  38.2\%  & 40.2\%  & 39.7\%  & 39.5\% \\ \bottomrule
\end{tabular}
\end{center}
\end{table}



\subsubsection{Description of the targeted patients.} \label{sss.targetedpatients}
In \cref{tab.stats}, we fix the budget to be 26 and we show statistics regarding the state of the targeted patients for each of the four policies.
For example, under $\baseline$, on average, the patient that received an intervention had a treatment effect of 8.8\% with respect to the probability that they will verify the next day.
8.8\% is the `true' average treatment effect, in the sense that the numbers that are averaged are taken directly from the simulation model.


\begin{table}[h]
\TableSpaced %
\caption{
Average statistics of the state of patients who were given an intervention, across the three policies that were run for $B=26$.
(a) is the average value of $f(x, 1) - f(x, 0)$, the increase in probability that the patient verifies the next day when they are given an intervention. 
(b) is the average $f(x, 0)$, the probability that a patient verifies the next day \textit{without} an intervention. 
(c) is the average number of remaining days the patient will be on TB treatment for.
} \label{tab.stats}
\vspace{2mm}
\begin{center}
\begin{tabular}{@{}ccccc@{}}
\toprule
                                                 & $\baseline$ & $\DPI$ & $\bandit$ & $\QWI$ \\ \midrule
\multicolumn{1}{l}{(a) \added{$f(x, 1) - f(x, 0)$}}  & 8.8\%   & 10.6\%    & 13.2\%    & 6.9\%    \\ 
\multicolumn{1}{l}{(b) \added{$f(x, 0)$}}   & 18.2\%  & 22.2\%  & 35.2\%   & 12.2\%      \\ 
\multicolumn{1}{l}{(c) Days on TB treatment remaining} & 68.3     & 109.3   & 92.4    & 69.7      \\ \bottomrule
\end{tabular}
\end{center}
\end{table}



Statistic (a) represents exactly what the $\bandit$ policy optimizes for, the increase in probability of the patient verifying the next day.
The fact that $\bandit$ yields the highest value confirms that indeed, the policy correctly learns what it is supposed to learn.
$\DPI$ chooses patients with a higher one-step treatment effect than $\baseline$, but lower than that of $\bandit$.
Then, the fact $\DPI$ outperforms $\bandit$ in terms of overall verification implies that a myopic strategy of looking only one step ahead is not sufficient.
The next two statistics shed light on why this may be.


Statistic (b) represents the probability that the targeted patient would have verified anyway without an intervention, and we see that the $\bandit$ targets patients with a much higher verify probability than the other policies. 
We plot the entire distribution of this quantity in Figure~\ref{fig:base_probs}, where we see that $\bandit$ often targets those with a relatively high probability ($>45\%$), while $\DPI$ targets those with a relatively low probability ($<15\%$).
This may contribute to the improved performance of $\DPI$, and the reasoning for this can be seen through the two-state model from \cref{ss.2statemodel}, where statistic (b) corresponds to the parameter $p$.
If two patients have the same values of the parameters $g$ and $\tau$ but differing values for $p$, the intervention value is higher when $p$ is smaller (see \cref{prop:z_clean_form}).
This is because the patient with a high value of $p$ is more likely to switch to state 1 at the current time step as well as all future time steps.
As an extreme example, for a patient with $p=0$, they \textit{need} an intervention to switch to state 1, whereas a patient with $p>0$ may switch to state 1 (either now or in the future), without an intervention.
Therefore, an intervention is more likely to be helpful for those with a smaller value of $p$, which $\DPI$ targets.

On the other hand, $\QWI$ takes the above strategy to an extreme, where it targets those with a very low probability of verifying (12.2\%), but on average these patients also do not have a high next-day treatment effect (6.9\%).
This may explain the inconsistent behavior of $\QWI$ as the budget increases. 
The strategy of targeting these patients with a low verify probability and a low treatment effect is reasonably effective in the mid-budget regime; however, as the budget increases, one may also need to judiciously target other types of patients, which $\QWI$ does not do. 





\begin{figure}[h]
\begin{center}
\vspace{-5mm}
  \includegraphics[width=0.54\linewidth]{figs/base_probs_June2024_2}
\vspace{-2mm}
  \caption{
  Histogram of the value of $f(x, 0)$ of targeted patients, the probability that the patients would verify without an intervention.
  This is the entire distribution of the statistic (b) in \cref{tab.stats} for $\bandit$ and $\DPI$.}
  \label{fig:base_probs}
\vspace{-6mm}
\end{center}
\end{figure}

Lastly, statistic (c) is the average number of days a targeted patient has remaining on the platform.
If an intervention positively affects patients for all of their future time steps, then targeting those with longer time left in the system would result in higher benefits. 
The results show that $\DPI$ targets those with the longest days of treatment left.

\subsubsection{Prominent features for $\DPI$.}
\cref{tab.coefs} displays the five most predictive features of the intervention values that $\DPI$ uses to target patients.
These features were found by using Lasso regression with a tuned parameter -- see Section~\ref{s.app.coefficients} for details on the method used.
The results show that the intervention value is lower when the number of previous interventions is higher (first two features), which is intuitive since patients may become fatigued and less receptive when there are too many interventions.
The intervention value is lower when the patient's past verifications is higher (third and fourth features). This is consistent with the analysis in \cref{tab.stats}, where $\DPI$ targets those with a smaller value of $f(x, 0)$.
Lastly, the intervention value increases for patients who are older.


\begin{table}[h]
\TableSpaced %
\caption{
The most predictive features of higher intervention values for $\DPI$, as well as the sign of their coefficient. 
} \label{tab.coefs}
\vspace{2mm}
\begin{center}
\begin{tabular}{@{}lc@{}}
\toprule
\; Feature & Sign of Coefficient  \\ \midrule
\; Interventions: total number & $-$ \\ 
\; Interventions: \# previous week &  $-$ \\
\; Verifications: overall percentage &  $-$ \\
\; Verifications: \# previous week & $-$ \\
\; Age & $+$ \\
\bottomrule
\end{tabular}
\end{center}
\end{table}



\subsection{\added{Robustness of State Space Representation}} \label{ss.state_space_results}
\added{One of the factors attributing to the performance gap between $\DPI$ and $\QWI$ is the differences in state space representation. $\QWI$ requires a finite state space and its computation time scales with the number of states. Hence, we use a relatively small state space for $\QWI$ for our numerical experiments. The ease of using a larger and infinite size state space is an inherent advantage of $\DPI$ over $\QWI$; however, in order to isolate the performance difference caused by the algorithm itself, we run $\DPI$ using the same state space as $\QWI$.

We try two variants of $\DPI$ that differ in the state space used:
\begin{itemize}
	\item \textsf{DecompPI-3}: This uses the same three features used for $\QWI$ (number of times verified in the last week, total historical verification rate, and number of interventions received in the last week), but these features are \textit{not discretized}, and hence the size of the state space is still infinite.
	\item \textsf{DecompPI-3-discrete}: This uses the exact same state space as $\QWI$ --- three features that are discretized in the same way, resulting in 45 states.
\end{itemize}


\begin{figure}[h]
\begin{center}
\vspace{-3mm}
  \includegraphics[width=1\linewidth]{figs/2024_May13_results3D}
  \caption{Average overall verification rate over 50 runs for each policy and budget. The shaded region indicates a 95\% confidence interval. The star represents the operating point for Keheala.}
  \label{fig:DPI3}
\vspace{-6mm}
\end{center}
\end{figure}

The performance of these two algorithms, along with the original $\DPI$ and $\QWI$ policies are shown in Figure~\ref{fig:DPI3}.
The policies \textsf{DecompPI-3-discrete} and $\QWI$ use the exact same state space, and we see that the former consistently outperforms the latter.
This comparison isolates the performance gap induced by the \emph{algorithms}, and the results provide robust evidence on the strength of $\DPI$.
Lastly, we see that \textsf{DecompPI-3} consistently has a strong performance, comparable to that of $\DPI$.
This demonstrates the robustness of $\DPI$ with respect to the feature space, and it also exemplifies the benefit of employing an infinite state space compared to a discretized one.  
}















\section{Conclusion and Future Directions}\label{s.conclusion}
\section{Conclusion}\label{sec:conclusion}
In this work, we focus on addressing the fundamental challenge of OOD detection tasks, which is how to fully understand the semantic discrepancy between the ID/OOD samples. We reveal that the key to success in the realistic SCOOD task is to allocate as many ID samples in the unlabeled set correctly as possible. To this end, we propose a novel uncertainty-aware optimal transport scheme that introduces class-specific energy scores as guidance for effective label assignment. Experimental results show that our method achieves better performance than previous state-of-the-art methods on SCOOD benchmarks.

\textbf{Limitations.} In addition to temperature scaling, other techniques such as feature clipping applied in ReAct~\cite{sun2021react} also enhance the performance of energy score, so how to obtain an OOD score that best fits the SCOOD task can be further explored. Moreover, a setting highly related to SCOOD has been proposed in \cite{katz2022training} and formulated as a constrained optimization problem. We will also theoretically analyze these practical OOD settings in our feature work.

% \section*{Acknowledgments}
\textbf{Acknowledgments.} 
This work is supported by National Key R\&D Program of China under Grant 2020AAA0105701, National Natural Science Foundation of China (NSFC) under Grants 61872327, Major Special Science and Technology Project of Anhui, National Natural Science Foundation of China (62033012) and Ant Group through Ant Research Intern Program.




\vspace*{-0.5em}
\bibliographystyle{ormsv080}
\bibliography{Keheala}

\clearpage

\setcounter{page}{1} \renewcommand{\theequation}{A\arabic{equation}}
\setcounter{equation}{0}
\renewcommand{\thelemma}{A\arabic{lemma}}
\setcounter{lemma}{0}
\renewcommand{\theproposition}{A\arabic{proposition}}
\setcounter{proposition}{0}
\renewcommand{\thesection}{A\arabic{section}}
\setcounter{section}{0}
\renewcommand{\thefigure}{A\arabic{figure}}
\setcounter{figure}{0}
\renewcommand{\theremark}{A\arabic{remark}}
\setcounter{remark}{0}
\renewcommand{\thetable}{A\arabic{table}}
\setcounter{table}{0}

\section*{E-companion for ``Targeted interventions for TB treatment adherence''}
~


\section{Table of Notation} \label{s.app.tablenotation}

\begin{table}[h]
\TableSpaced %
\caption{
List of notation used in the paper and the proofs in \cref{s.app.proof}.
} \label{tab.coefs}
\vspace{2mm}
\begin{center}
\begin{tabular}{@{}cl@{}}
\toprule
Notation & Meaning\\ \midrule
$N$ & Number of patients \\ 
$T$ & Number of time steps \\ 
$B$ & Budget \\ 
$\cS$ & State space for a single patient\\ 
$\cA$ & Action space for a single patient\\ 
$\cA^N$ & Action space for the system \\ 
$P_i(s, s', a)$ & Transition probability function for patient $i$ \\
$R_i(s, s', a)$ & Reward function for patient $i$ \\
$S_{it}^\pi$ & State of patient $i$  at time $t$ under policy $\pi$ \\
$A_{it}^\pi$ & Action for patient $i$  at time $t$ under policy $\pi$ \\
$q_{it}^{\pi}(s, a)$ & Patient-level $q$-value defined in \eqref{eq:q} \\
$z_{it}^{\pi}(s)$ & Intervention value used for $\DPI(\pi)$ \\
$p_i, q_i, \tau_i$ & Parameters for the 2-state model of \cref{ss.2statemodel} \\
$\gamma$ & Probability of an intervention for the policy $\RAND(\gamma)$ \\
$\bar{M}$ & Instance-dependent parameter used in \cref{thm:approx_ratio_b} \\
$V_{it}$ & Whether patient $i$ verifies at time $t$ \\
$P_{it}, Q_{it}, K_{it}, W_{it}$ & Bernoulli variables used in the sample path coupling of \cref{sec:coupling} \\
$\bar{z}_{it}(s)$ & Null intervention values defined as $\lim_{\gamma \to 0} z_{it}^{\gamma}(s)$ \\
$\DPI_0$ & Index policy that ranks patients by the null intervention values \\
$I_{t}$ & Subset of patients in state 0 at time $t$ \\
$D_t(I)$ & Subset of patients that $\DPI_0$ would choose out of $I \subseteq [N]$ \\
$\ONE(i, t)$ & Policy that chooses patient $i$ once at time $t$ \\
$\tilde{S}_{ir|t}$ & State of patient $i$ at time $r$ under $\ONE(i, t)$ \\
$Z_{it}$ & Number of times state of patient $i$ differ in $\ONE(i, t)$ compared to $\NULL$, defined in \eqref{eq:define_Y} \\
$\tau_i(t)$ & Last time before $t$ that $i$ was in state 0, defined in \eqref{eq:definetau} \\
\bottomrule
\end{tabular}
\end{center}
\end{table}

\section{Proof of \cref{thm:approx_ratio_b} and \cref{corr:approx}}  \label{s.app.proof}

We first prove \cref{thm:approx_ratio_b}, where we break down the proof into four steps, where the bulk of the proof lies in the first two steps.
The proof of \cref{corr:approx} follows from steps 3 and 4, which we describe in \cref{sec:app:pf_approx}.
The proofs of all of the steps make use a specific sample path coupling procedure which we describe in the next subsection.

\paragraph{Step 1.}
First, we define \textit{null intervention values} as $\nz_{it}(s) = \lim_{\gamma \to 0^+} z^{\gamma}_{it}(s)$. 
We show that for any algorithm $\ALG$, the difference $\ALG - \NULL$ can be written as a sum of the null intervention values of the patients that were chosen.
\begin{proposition} \label{prop:alg_minus_base}
For any algorithm $\ALG$,
\begin{align} \label{eq:diff_char}
\ALG - \NULL = \bE\left[\sumT \sum_{i \in A^{\ALG}_t} \nz_{it}(0) \right].
\end{align}
\end{proposition}

\paragraph{Step 2.}
Then, we define the policy $\DPI_0$ to be the policy which orders patients in state 0 with respect to the null intervention values, $\nz_{it}(0)$, gives interventions to the $B$ patients with the highest values.
We show that this policy achieves at least half of the optimal improvement over $\NULL$.
\begin{proposition} \label{prop:dpinull_half_result}
For any instance of the two-state model,
\begin{align} \label{eq:approx_ratio_b}
\DPI_0  - \NULL \;\geq\; \frac{1}{2} \; ( \OPT - \NULL ).
\end{align}	
\end{proposition}

\paragraph{Step 3.}
Next, we show that if a policy uses an index rule with indices that approximate the null intervention values, this policy also yields a performance guarantee. 
\begin{proposition} \label{prop:approx_index}
Fix $\alpha_1 \in (0, 1]$ and $\alpha_2 \geq 1$.
Let $\ALG$ be an index policy that uses indices $z^{\ALG}_{it}(0)$ that satisfy $\alpha_1 \nz_{it}(0) \leq z^{\ALG}_{it}(0) \leq \alpha_2 \nz_{it}(0)$ for all $i$ and $t$. Then,
\begin{align} \label{eq:approx_ratio_b}
\ALG  - \NULL \;\geq\; \frac{\alpha_1}{2\alpha_2} \; ( \OPT - \NULL ).
\end{align}	
\end{proposition}

\paragraph{Step 4.}
Lastly, we show that the intervention values $z^{\gamma}_{it}(0)$ satisfy the above, where the corresponding $\alpha$ is a function of both $\gamma$ and the underlying parameters.
\begin{proposition} \label{lemma:rand_null}
Fix a patient $i$, and let $M_i = \frac{\tau_i (1-p_i-g_i)}{(p_i + g_i)(1-p_i)} > 0$.
For any $t \in [T]$, the intervention values $z^{\gamma}_{it}(0)$ and $\nz_{it}(0)$ satisfy the following relationship:
\begin{align}
\frac{1}{1 + \gamma M_i} \nz_{it}(0) \; \leq \; z^{\gamma}_{it}(0) \; \leq \; \nz_{it}(0).
\end{align}
\end{proposition}

Using \cref{lemma:rand_null}, we can apply \cref{prop:approx_index} using $\alpha_1 = 1/(1+\gamma M)$ and $\alpha_2 = 1$ for $M = \max M_i$, which completes the proof of \cref{thm:approx_ratio_b}.
The proofs of Propositions \ref{prop:alg_minus_base}-\ref{lemma:rand_null} can be found in Sections \ref{s.app.p1}-\ref{s.app.p4}.

\paragraph{Terms and notation.} We define a couple of terms and notation that we will use for the proofs.
We say that a patient is `chosen' to mean that the patient received an intervention, and we say that a patient is `available' to mean they are in state 0.
Let $I_t \subseteq [N]$ be the set of patients that are in state  0 at time $t$. Then, $[N] \setminus I_t$ are the patients in state 1 at time $t$.
We use $A_t \subseteq [N]$ to refer to the subset of patients who are chosen (instead of using the notation $A_{it} \in \{0, 1\}$ to denote whether patient $i$ is chosen).
Without loss of generality, we assume that $A_t \subseteq I_t$. 
This is because giving an intervention to a patient in state 1 has no impact on their transitions.
We put $\pi$ in the superscript, $I^\pi_t$ and $A^\pi_t$, to refer to the random variables induced by running policy $\pi$.

We now describe the sample path coupling, and then prove each of the above four propositions in the following subsections.

\subsection{Sample path coupling} \label{sec:coupling}
Fix $\gamma \in (0, 1]$.
We specify a new set of model dynamics that couples different policies through shared random variables.
We will show that these new dynamics are equivalent to the original dynamics specified in \cref{ss.2statemodel}.

\noindent
\textbf{New dynamics.}
At each time $t=1, \dots, T-1$, the following occurs:
\begin{enumerate}
	\item States $S_{it}$ are observed for all patients $i$.
	\item A policy selects a subset of patients $A_{t} \subseteq I_t$ for an intervention.
	\begin{itemize}
		\item For the policy $\RAND(\gamma)$, draw $W_{it} \sim \Bern(\gamma)$ independently for each patient $i$.
		Then, $i \in A_t$  if and only if $S_{it} = 0$ and $W_{it} = 1$.
	\end{itemize}
	\item For every patient $i \in [N]$, draw independent, Bernoulli variables
	$G_{it} \sim \Bern(g_i / (1-p_i))$, $P_{it} \sim \Bern(p_i)$, $K_{it} \sim \Bern(\tau_i/(1-p_i))$.
	\item Patient transitions occur in the following order:
\begin{enumerate}[label=(\roman*)]
	\item \textbf{Return to state 0:} For each patient $i \in I_t(1)$ and $G_{it} = 1$, the patient returns to state 0.
	Let $I_t' = I_t \cup \{i : G_{it} = 1, i \in I_t(1)\}$ be the new set of patients in state 0.
	\item \textbf{Passive transitions to state 1:}  For each patient $i \in I_t'$ and $P_{it} = 1$, the patient goes to state 1.
	Let $I_t'' = I_t' \setminus \{i : P_{it} = 1, i \in I_t'\}$ be the new set of patients in state 0.
	\item \textbf{Active transitions to state 1:}  For each patient $i \in I_t'' \cap A_{t}$ with $K_{it} = 1$, the patient goes to state 1.
	Then, $I_{t+1} = I_t'' \setminus \{i : K_{it} = 1, i \in I_t'' \cap A_{t}\}$ are the set of patients in state 0 at the next time step, and $I_{t+1}(1) = [N] \setminus I_{t+1}$.
\end{enumerate}
	\item We collect the reward $I_{t+1}(1)$.
\end{enumerate}

\textbf{Equivalence.}
We claim that the original model dynamics (from \cref{ss.2statemodel}) is equivalent to these new model dynamics.
To show this, we need to show that the transition probabilities between states are equal under both models.

In the new model, suppose a patient is in state 0.
If they were not chosen, they can only transition to 1 under step (ii), which occurs with probability $p_i$.
If they were chosen, they can transition to 1 under either step (ii) or (iii).
In total, they transition with probability $p_i + (1-p_i) \cdot \tau_i/(1-p_i) = p_i + \tau_i$.
Next, suppose a patient is in state 1.
For the patient to transition to state 0, it must be that $G_{it} = 1$ and $P_{it} = 0$.
This occurs with probability $g_i / (1-p_i) \cdot (1-p_i) = g_i$.
Therefore, the transition probabilities between the two models are equal for all patients.

\textbf{Coupling sample paths of policies.} 
From now on, we assume that all policies are coupled through the variables $P_{it}, G_{it}, K_{it}$.
This coupling immediately gives us the following properties.

\begin{lemma} \label{prop:coupling_property0}
For any policy, if $P_{it} = 1$, then $S_{i, t+1} = 1$.
\end{lemma}

\begin{myproof}
Suppose $P_{it} = 1$ for some $i$ and $t$.
In step 4(ii) of the new dynamics, the set $I_t''$ is defined so that $i \in I_t''$.
In step 4(iii), $I_{t+1}$ is a subset of $I_t''$,  and $I_{t+1}(1) = [N] \setminus I_{t+1}$.
Hence $S_{i, t+1} = 1$.
\end{myproof}

The next property says that if a patient is in state 1 under the $\NULL$ policy, they must also be in state 1 under any other policy.

\begin{lemma} \label{prop:coupling_property}
Let $\ALG$ be any policy. If $S_{it}^{\NULL} = 1$, then $S_{it}^{\ALG} = 1$.
Equivalently, if $S_{it}^{\ALG} = 0$, then $S_{it}^{\NULL} = 0$.
\end{lemma}

\begin{myproof} %
Let $i, t$ be such that $S_{it}^{\NULL} = 1$.
Let $t' = \max\{t' < t : P_{it'} = 1\}$ be the most recent time $P_{it'}$ was 1.
Since $S_{it}^{\NULL} = 1$, $G_{is} = 0$ for every $s \in \{t'+1, \dots, t-1\}$.
Since $P_{it'} = 1$, \cref{prop:coupling_property0} implies $S^{\ALG}_{i, t'+1} = 1$.
Since $G_{is} = 0$ for every $s \in \{t'+1, \dots, t-1\}$, $S_{it}^{\ALG} = 1$.
\end{myproof}

Lastly, using this sample path coupling, we can write an expression for the intervention values $z^{\gamma}_{it}(0)$ and $\nz_{it}(0)$.
\begin{lemma} \label{lemma:zgamma}
The intervention value with respect to $\RAND(\gamma)$ can be written as 
\begin{align} \label{eq:zformula}
z^{\gamma}_{it}(0) = \tau_i \cdot \bE[\min\{\text{Geometric}(p_i + g_i + \gamma \tau_i (1-p_i-g_i)/(1-p_i)), T-t+1\}].
\end{align}
Moreover, the null intervention value can be written as
\begin{align} \label{eq:nzformula}
\nz_{it}(0) = \tau_i \cdot \bE[\min\{\text{Geometric}(p_i + g_i), T-t+1\}].
\end{align}
\end{lemma}


\begin{myproof}[Proof of \cref{lemma:zgamma}]
Recall that $z^{\gamma}_{it}(0) = g_{it}^{\RAND(\gamma)}(0, 1) - g_{it}^{\RAND(\gamma)}(0, 0)$.
Let $E_0 = \{S_{it} = 0, A_{it} = 0\}$ be the event where patient $i$ is in state 0 at time $t$ and $\RAND(\gamma)$ does not choose the patient and let $E_1 = \{S_{it} = 0, A_{it} = 1\}$ be the event where patient $i$ is in state 0 at time $t$ and $\RAND(\gamma)$ does choose the patient.

Conditioned on either $E_0$ or $E_1$, the distribution of the variables $W_{it'}$ for $t' > t$ and  $P_{it'}, G_{it'}, K_{it'}$ for $t' \geq t$ are the same, due to independence.
Therefore, to compute $z^{\gamma}_{it}(0)$, we consider two hypothetical sample paths for patient $i$, one where $E_0$ holds and one where $E_1$ holds, but where all future variables are exactly the same.
We refer to $S^0_{it'},S^1_{it'} \in \{0, 1\}$ as the states under the two sample paths respectively at time $t' > t$.
Let $\zeta = \min\{T+1, \min\{t' > t: S^0_{it'} = S^1_{it'}\}\}$ be the first time after time $t$ that the two states converge, where $\zeta = T+1$ if they never converge. 
The states will always be the same after time $\zeta$ due to the sample path coupling.
Then, $z^{\gamma}_{it}(0) = \bE[\zeta] - t - 1$.

We now write an expression for $z^{\gamma}_{it}(0)$.
For the states to differ at time $t+1$, it must be that $P_{it} = 0$ and $K_{it} = 1$.
After that, the states will converge at time $t'+1$ if (i) $P_{it'} = 1$,  (ii) $G_{it'} = 1$, or (iii) $W_{it'} = 1$ and $K_{it'} = 1$.
Let $\Gamma = \min\{t' > t : P_{it'} = 1 \} - t$ be the length of time from $t$ until $P_{it'} = 1$.
Let $\Gamma' = \min\{t' > t : G_{it'} = 1 \} - t$ be length of time from $t$ until $G_{it'} = 1$.
Let $\Gamma'' = \min\{t' > t : W_{it'} = 1, K_{it'} = 1 \} - t$ be length of time from $t$ until $W_{it'} = 1$ and $K_{it'} = 1$.
Then, $z^{\gamma}_{it}(0)$ can be written as
\begin{align*}
z^{\gamma}_{it}(0) = \Pr(K_{it} = 1) \cdot \Pr(P_{it} = 1) \cdot \bE[\min\{\Gamma, \Gamma', \Gamma'', T-t+1\}].
\end{align*}
The term $\min\{\Gamma, \Gamma', \Gamma''\}$ is a geometric random variable with parameter 
\begin{align*}
&1-(1-\Pr(P_{it} = 1))(1-\Pr(G_{it} = 1))(1-\Pr(W_{it} = 1, K_{it} = 1)) \\
=& 1-(1-p)(1-g/(1-p))(1-\gamma \tau / (1-p)) \\
=& p + g + \gamma \tau (1-p-g)/(1-p).
\end{align*}
Therefore,
\begin{align*} 
z^{\gamma}_{it}(0) = \tau \cdot \bE[\min\{\text{Geometric}(p + g + \gamma \tau (1-p-g)/(1-p)), T-t+1\}].
\end{align*}
\cref{eq:nzformula} follows from taking the limit of the above as $\gamma \to 0$, using the dominated convergence theorem.
\end{myproof}



\subsection{Step 1: Proof of \cref{prop:alg_minus_base}} \label{s.app.p1}
We start with analyzing the left-hand side, $\ALG - \NULL$.
\cref{prop:coupling_property} says that whenever $S^{\ALG}_{it} = 0$, it must be that $S^{\NULL}_{it} = 0$.
Therefore, $\ALG-\NULL$ can be written as the number of times when $S^{\ALG}_{it} = 1$ while $S^{\NULL}_{it} = 0$:
\begin{align}
\ALG - \NULL = \bE\bigg[\sumT \bI(S^{\ALG}_{it'} =1, S^{\NULL}_{it'} = 0) \bigg]
\end{align}

Due to the sample path coupling, $S_{it}^{\ALG} \neq S_{it}^{\NULL}$ can only occur if
$\ALG$ chose patient $i$ at a prior time step, and the states have been different since then (if the states converged, it will stay the same unless $\ALG$ chose the patient again).
Therefore, each time $\bI(S^{\ALG}_{it'} =1, S^{\NULL}_{it'} = 0)$ occurs, it is associated with an intervention by $\ALG$ at a previous time step.
Hence, we will instead represent $\ALG-\NULL$ by summing over all interventions given by $\ALG$, and relating each intervention to how long the states $S^{\ALG}_{it'}$ and $S^{\NULL}_{it'}$ deviate.

\textbf{Defining counterfactual state.}
To formalize this notion, we need to define a couple of new quantities that will play a important role in both this step and step 2 of the proof.
We define a policy $\ONE(i, t)$ to be the same as the $\NULL$ policy, except that it chooses patient $i$ once at time $t$.
Define $\tilde{S}_{i r | t} = S^{\ONE(i, t)}_{ir}$ to be the state of patient $i$ at time $r$ under this policy, which we call the \textit{counterfactual state}.
Then, let $Z_{it}$ be the number of times that the counterfactual state is not equal to the state under $\NULL$:
\begin{align} \label{eq:define_Y}
Z_{it} &= |\{t' \in [T] \; : \; S_{it'}^{\NULL} \neq \tilde{S}_{it' | t} \}|.
\end{align}

Note the following properties:
\begin{itemize}
	\item The two states are always equal before time $t$.
	($S_{it'}^{\NULL} =\tilde{S}_{it' | t}$ for any $t' \leq t$.)
	\item Once the two states converge at some time $t' > t$, they will never diverge again since the policies are the same. (If $S_{it'}^{\NULL} =\tilde{S}_{it' | t}$ for some $t' > t$, then the same holds for any $r > t'$.)
	\item The only way that the states can be different is if the state is 0 under $\NULL$ and 1 under $\ONE(i, t)$, due to \cref{prop:coupling_property}.
\end{itemize}
Therefore, $Z_{it}$ represents exactly the increase in total reward from patient $i$ caused by the intervention at time $t$, compared to $\NULL$.
More specifically, when $S^{\NULL}_{it} = 0$, $Z_{it}$ is the number of time steps that the patient was in state 1 right after time $t$, before it transitioned back to state 0 or the state under the $\NULL$ policy also moved to state 1 (or we reached the last time step).
The above logic allows us to write out an expression for the expected value of $Z_{it}$, conditioned on $S^{\NULL}_{it} = 0$.

\begin{lemma} \label{lemma:zequalsy}
$\bE\big[Z_{it} \;|\; S^{\NULL}_{it} = 0 \big] = \tau_i \cdot \bE[\min\{\text{Geometric}(p_i + g_i, T-t+1\}] = \nz_{it}(0)$.
\end{lemma}

Now, we can write $\ALG-\NULL$ as the sum of $Z_{it}$ values at the times when $i$ was chosen:
\begin{align}
\ALG - \NULL 
&= \bE\bigg[\sumT \sum_{i \in A_t^{\ALG}} Z_{it}\bigg] \label{eq:propRandom} \\
&= \sumT \sum_{i \in [N]} \bE\big[\bI(i \in A_t^{\ALG}) Z_{it}\big] \nonumber \\
&= \sumT \sum_{i \in [N]} \bE\big[\bI(i \in A_t^{\ALG})\big]\; \bE\big[Z_{it} \;|\; i \in A_t^{\ALG}\big]  \nonumber \\
&= \sumT \sum_{i \in [N]} \bE\big[\bI(i \in A_t^{\ALG})\big] \nz_{it}(0) \nonumber \\
&= \bE\bigg[\sumT \sum_{i \in A_t^{\ALG}}  \nz_{it}(0) \bigg], \label{eq:propDet} 
\end{align}
as required.


\begin{myproof}[Proof of \cref{lemma:zequalsy}]
Let $\Gamma_{it} = \min\{t' > t : P_{it'} = 1 \} - t$ be the length of time from $t$ until $P_{it'} = 1$.
Let $\Gamma'_{it} = \min\{t' > t : G_{it'} = 1 \} - t$ be length of time from $t$ until $G_{it'} = 1$.
Then, 
\begin{align} \label{eq:Z_explicit}
Z_{it} = \bI(S_{it}^\NULL = 0, P_{it} = 0, K_{it} = 1) \min\{\Gamma_{it}, \Gamma'_{it}, T-t+1\}.
\end{align}
The indicator represents the fact that $i$ would not transition to state 1 under $\NULL$ ($P_{it} = 0$), but it would transition under $\ONE(i, t)$ ($K_{it} = 1$).
That is, at time $t+1$, the patient is in state 1 under $\ONE$ but in state 0 under $\NULL$.
The $\min\{\Gamma_{it}, \Gamma'_{it}\}$ term counts how long this is the case.
This could end either because patient transitions to state 1 under $\NULL$ (captured by $B_{it}$), or it could be that the patient in $\ONE$ transitions back to state 0 (captured by $L_{it}$).

Note that $\min\{\Gamma_{it}, \Gamma'_{it}\}$ is a geometric random variable with parameter $1-(1-\Pr(P_{it} = 1))(1-\Pr(G_{it} = 1)) = p_i+g_i$.
Therefore,
\begin{align*}
\bE[Z_{it} \;|\; S_{it}^\NULL = 0] 
&= \Pr(P_{it} = 0) \cdot \Pr(K_{it} = 1) \; \bE[\min\{\text{Geometric}(p_i+g_i), T-t+1\}] \\
&= \tau_i \cdot \bE[\min\{\text{Geometric}(p_i+g_i), T-t+1\}].
\end{align*}
Note that the above expression is the same as the one for $\nz_{it}(0)$ from \cref{lemma:zgamma}.
\end{myproof}





\subsection{Step 2: Proof of \cref{prop:dpinull_half_result}}
We prove a more general and stronger version of \cref{prop:dpinull_half_result}, which we state as \cref{thm:step2stronger}.

Define the policy $\DPI_0$ to be the policy which orders patients in state 0 with respect to the null intervention values, $\nz_{it}(0)$, gives interventions to the $B$ patients with the highest values.
Denote by $D_t(I_t^{\ALG}) \subseteq I_t^{\ALG}$ the subset of patients that $\DPI_0$ \textit{would choose} out of $I_t^{\ALG}$, the $B$ patients with the highest null intervention values, $\nz_{it}(0)$.
The next result shows that the sum of null intervention values of the patients in $D_t(I_t^{\ALG})$, will lead to at least half of total sum intervention values for an optimal policy.
\begin{proposition} \label{thm:step2stronger}
For any $\ALG$, 
\begin{align} \label{eq.main_result}
\OPT - \NULL \leq 2 \bE\left[ \sumT \sum_{i \in D_t(I_t^{\ALG})} \yit(0) \right].
\end{align}
\end{proposition}
Note that if $\ALG = \DPI_0$, then $D_t(I_t^{\ALG})$ is simply the patients that $\DPI_0$ chooses at time $t$; in that case, the RHS of \eqref{eq.main_result} equals the RHS of \eqref{eq:diff_char}.
Then, by \cref{prop:alg_minus_base}, the RHS of \eqref{eq.main_result} equals $\DPI_0 - \NULL$, which corresponds exactly to the statement of \cref{prop:dpinull_half_result}.
Hence, \cref{thm:step2stronger}  implies that $\DPI_0$ achieves at least half of the optimal improvement over $\NULL$.


We now prove \cref{thm:step2stronger}.
This proof makes use of the quantity $Z_{it}$, which was defined in the proof of \cref{prop:alg_minus_base}.
Fix any policy $\ALG$.
Recall that $I^{\ALG}_t$ and $I^*_t$ are the set of patients that are in state 0 under $\ALG$ and $\OPT$ respectively at time $t$.
Additionally, $D_t(I_t^{\ALG}) \subseteq I^{\ALG}_t$ are the patients that $\DPI_0$ would choose, out of patients in $I^{\ALG}_t$.
We first decompose the rewards based on whether or not a patient that was chosen in $\OPT$ was available to be chosen under $\ALG$.
\begin{align}
\OPT - \NULL
&= \bE\left[ \sumT \sum_{i \in A^*_t} Z_{it} \right]\nonumber \\
&= \bE\left[ \sumT \sum_{i \in A^*_t \cap I^{\ALG}_t} Z_{it} \right]
+ \bE\left[ \sumT \sum_{i \in A^*_t \setminus I^{\ALG}_t} Z_{it} \right].
\label{eq:opt_minus_base_decomp}
\end{align}

For the first term in \eqref{eq:opt_minus_base_decomp},
by definition of $D_t(I_t^{\ALG})$, we have
$\sum_{i \in A^*_t \cap I^{\ALG}_t} \nz_{it}(0) \leq \sum_{i \in D_t(I_t^{\ALG})} \nz_{it}(0)$.
Therefore, 
\begin{align*}
\bE\left[ \sumT \sum_{i \in A^*_t \cap I^{\ALG}_t} Z_{it} \right]
 = \bE\left[ \sumT \sum_{i \in A^*_t \cap I^{\ALG}_t} \nz_{it}(0) \right]
 \leq 
 \bE\left[ \sumT \sum_{i \in D_t(I_t^{\ALG})}  \nz_{it}(0) \right],
\end{align*}
where the first equality follows the same reasoning as \eqref{eq:propRandom}-\eqref{eq:propDet}.


\cref{thm:step2stronger} then follows from the following result which bounds the second term in \eqref{eq:opt_minus_base_decomp}. This term represents rewards from patients who are in state 0 under $\OPT$ but not under $\ALG$.
\begin{proposition} \label{prop:second_term}
\begin{align} \label{eq:second_term_claim}
\bE\left[ \sumT \sum_{i \in A^*_t \setminus I^{\ALG}_t} Z_{it} \right]	
\leq 
\bE\left[ \sumT \sum_{i \in A^{\ALG}_t} Z_{it} \right]	
\end{align}
\end{proposition}

\subsubsection{Proof of \cref{prop:second_term}.}
The main idea of this result is that for every $Z_{it}$ term that contributes to the LHS of \eqref{eq:second_term_claim}, since that patient is not available under $\ALG$, we have already collected this reward under $\ALG$.
We will show a one-to-one mapping from every $Z_{it}$ term on the LHS to a $Z_{i \nu_i(t)}$ term on the RHS of \eqref{eq:second_term_claim}.

Let $\cI = \{(i, t) : i \in A^*_t \setminus I^{\ALG}_t, K_{it} = 1\}$ be the set of (patient, time) tuples in which patient $i$ is chosen under $\OPT$ but not available under $\ALG$, and moreover, $K_{it} = 1$.
Note that from \eqref{eq:Z_explicit}, $K_{it} = 1$ is a necessary condition for $Z_{it} > 0$. 
Therefore, the LHS of \eqref{eq:second_term_claim} can be written as
\begin{align*}
\bE\left[ \sumT \sum_{i \in A^*_t \setminus I^{\ALG}_t} Z_{it} \right]	
= \bE\left[ \sum_{(i, t) \in \cI} Z_{it} \right].
\end{align*}

Fix $(i, t) \in \cI$.
We have that $S_{it}^{\ALG} = 1$ (since $i \notin I^{\ALG}_t$) and $S_{it}^{\OPT} = 0$ (since $i \in A^*_{t}$).
Define $\nu_i(t) < t$ to be the last time that $i$ was in state 0 under $\ALG$:
\begin{align} \label{eq:definetau}
\nu_i(t) = \max\{ t' < t : S_{it'}^{\ALG} = 0\}.
\end{align}
That is, under $\ALG$, patient $i$ transitioned from state 0 to state 1 between time $\nu_i(t)$ and $\nu_i(t)+1$ and stayed in state 1 since.
We can show that $\nu_i(t)$ satisfies the following property: it must be that $i$ was chosen at time $\nu_i(t)$ under $\ALG$, and moreover, $Z_{i \nu_i(t)}$ is at least $t - \nu_i(t)$.

\begin{claim} \label{claim:properties_of_y}
Let $(i, t) \in \cI$.
Then, $i \in A^{\ALG}_{\nu_i(t)}$, and $Z_{i \nu_i(t)} \geq t - \nu_i(t)$.
\end{claim}	
\begin{myproof}[Proof of \cref{claim:properties_of_y}]
Let $(i, t) \in \cI$.
By definition of $\nu_i(t)$, $S_{i, \nu_i(t)}^{\ALG} = 0$ and $S_{i, \nu_i(t)+1}^{\ALG} = 1$.
To the contrary, suppose $i \notin A_{\nu_i(t)}^{\ALG}$.
Then, $i$ transitioned to state 1 without being chosen, and therefore $P_{i\nu_i(t)} = 1$.
Then, it must be that $i$ transitions to state 1 under $\OPT$ as well; $S_{i, \nu_i(t)+1}^{\OPT} = 1$.
But since $S_{it}^{\OPT} = 0$, $i$ switches back to state 0 before state $t$, which means that this should also happen under $\ALG$ (due to the sample path coupling).
This is a contradiction by the definition of $\nu_i(t)$.
Therefore, $i \in A_{\nu_i(t)}^{\ALG}$.

Moreover, under the same reasoning, it must be that $K_{i\nu_i(t)} = 1$, and that $S_{is}^{\NULL} = 0$ for all $s \in \{\nu_i(t)+1, \dots, t\}$.
Therefore, $Z_{i \nu_i(t)} \geq t - \nu_i(t)$.
\end{myproof}


We next show that the mapping $\nu_i$ is one-to-one.
\begin{claim} \label{claim:1-1}
If $(i, t), (i, s) \in \cI$ for $t \neq s$, then $\nu_i(t) \neq \nu_i(s)$.
\end{claim}
\begin{myproof}[Proof of \cref{claim:1-1}]
Suppose $t < s$ such that $(i, t), (i, s) \in \cI$.
That is, under $\OPT$, patient $i$ was chosen both at time $t$ and $s$, and the patient was already in state 1 under $\ALG$ at both of these times.
Suppose, for contradiction, $\nu_i(t) = \nu_i(s)$.
This means that under $\ALG$, patient $i$ was chosen at time $\nu_i(t)$, and the patient has stayed in state 1 from time $\nu_i(t)+1$ until at least time $s$.

$(i, t), (i, s) \in \cI$ implies $S^{\OPT}_{it} = S^{\OPT}_{is} = 0$.
Since $K_{it} = 1$, the patient transitioned to state 1 at time $t+1$.
Since $S^{\OPT}_{is} = 0$, it must be that there exists a $t' \in \{t+1, \dots, s-1\}$ such that $G_{it'} = 1, P_{it'} = 0$.
But since $S^{\ALG}_{it'} = 1$, the fact that $G_{it'} = 1$ and $P_{it'} = 0$ implies that $S^{\ALG}_{i, t'+1} = 0$, which is a contradiction.
\end{myproof}

Claims \ref{claim:properties_of_y} and \ref{claim:1-1} show that every $(i, t) \in \cI$ maps to one term in the RHS of \eqref{eq:second_term_claim}.
Therefore, the right hand side of \eqref{eq:second_term_claim} is at least
\begin{align*}
\bE\left[ \sumT \sum_{i \in A^{\ALG}_t} Z_{it} \right]	
\geq 
\bE\left[ \sum_{(i, t) \in \cI} Z_{i\nu_i(t)} \right].
\end{align*}
Then, we are done if we can show
\begin{align*} 
\bE\left[ \sum_{(i, t) \in \cI} Z_{it} \right]
\leq
\bE\left[ \sum_{(i, t) \in \cI} Z_{i\nu_i(t)} \right],
\end{align*}
which we can write as
\begin{align}\label{eq:simpler_goal}
\sumT \sumN \bE[Z_{it} \;|\; (i, t) \in \cI\;] \Pr((i, t)\in \cI)
\leq 
\sumT \sumN \bE[Z_{i\nu_i(t)} \;|\; (i, t) \in \cI\;] \Pr((i, t)\in \cI).
\end{align}

The following claim implies \cref{eq:simpler_goal} and finishes the proof of \cref{prop:second_term}.
\begin{claim} \label{claim:compare_ys}
For every $i \in [N]$ and $t \geq 1$,
\begin{align} \label{eq:inner_term}
\bE[Z_{it} \;|\; (i, t)  \in \cI\;] 
\leq 
\bE[Z_{i\nu_i(t)} \;|\; (i, t)  \in \cI\;]
\end{align}
\end{claim}


\begin{myproof}[Proof of \cref{claim:compare_ys}]
Fix $i, t$.
First, we upper bound the LHS of \cref{eq:inner_term}.
\begin{align*}
\bE[Z_{it} \;|\; (i, t)  \in \cI\;]  
&= \bE[Z_{it} \;|\; Z_{it} \geq 1,  (i, t)  \in \cI\;] \Pr(Z_{it} \geq 1 \;|\; (i, t)  \in \cI) \\
&\leq \bE[Z_{it} \;|\; Z_{it} \geq 1,  (i, t)  \in \cI\;].
\end{align*}
Note that conditioned on $Z_{it} \geq 1$, $Z_{it}$ is only a function of $(P_{it'}, G_{it'})_{t' > t}$, the `future' with respect to $t$, and the event $\{(i, t)  \in \cI\} = \{i \in A^*_t \setminus I^{\ALG}_t,  K_{it} = 1\}$ is independent of these future random variables.
Therefore, conditioned on $Z_{it} \geq 1$, $Z_{it}$ is independent of $(i, t)  \in \cI$, and hence
\begin{align}\label{eq:lhs_ub} 
\bE[Z_{it} \;|\; (i, t)  \in \cI\;]  
\leq
\bE[Z_{it} \;|\; Z_{it} \geq 1].
\end{align}

Next, consider the RHS of \eqref{eq:inner_term}.
From \cref{claim:properties_of_y}, $(i, t) \in \cI$ implies $Z_{i \nu_i(t)} \geq t - \nu_i(t)$.
Therefore, 
\begin{align*}
\bE[Z_{i\nu_i(t)} \;|\; (i, t)  \in \cI\;]
= \bE[Z_{i\nu_i(t)} \;|\; (i, t)  \in \cI, Z_{i \nu_i(t)} \geq t - \nu_i(t)\;].
\end{align*}
Similar to before, conditioned on $Z_{i \nu_i(t)} \geq t - \nu_i(t)$, $Z_{it}$ is only a function of $(P_{it'}, G_{it'})_{t' > t}$, the `future' with respect to $t$.
Therefore, conditioned on $Z_{i \nu_i(t)} \geq t - \nu_i(t)$, $Z_{it}$ is independent of $(i, t) \in \cI$, and hence
\begin{align*}
\bE[Z_{i\nu_i(t)} \;|\; (i, t)  \in \cI\;]
&= \bE[Z_{i\nu_i(t)} \;|\; Z_{i \nu_i(t)} \geq t - \nu_i(t)\;] \\
&= \sum_{t' < t} \Pr(\nu_i(t) = t' \;|\; Z_{i \nu_i(t)} \geq t - \nu_i(t)) \bE[Z_{it'} \;|\; Z_{i t'} \geq t - t'] \\ 
&\geq \sum_{t' < t} \Pr(\nu_i(t) = t' \;|\; Z_{i \nu_i(t)} \geq t - \nu_i(t)) \bE[Z_{it'} \;|\; Z_{i t'} \geq 1].
\end{align*}


Note that for $s < t$, $\bE[Z_{is} | Z_{is} \geq 1] \geq \bE[Z_{it} | Z_{it} \geq 1]$.
This is because given the equation for $Z_{it}$ in \eqref{eq:Z_explicit}, the only difference between $Z_{it}$ and $Z_{is}$ is that $Z_{it}$ has a smaller maximum value of $T-t$.
Therefore,
\begin{align}
\bE[Z_{i\nu_i(t)} \;|\; (i, t)  \in \cI\;] 
&\geq  \bE[ Z_{i t}  | Z_{i t} \geq 1] \sum_{t' < t} \Pr(\nu_i(t) = t' \;|\; Z_{i \nu_i(t)} \geq t - \nu_i(t)) \nonumber \\
&= \bE[ Z_{it}  | Z_{it} \geq 1]. \label{eq:rhs_lb} 
\end{align}

Combining \eqref{eq:rhs_lb} and \eqref{eq:lhs_ub} proves the result.
\end{myproof}



\subsection{Step 3: Proof of \cref{prop:approx_index}}

Fix $\alpha_1, \alpha_2$,
and let $\ALG$ be a policy that uses indices that satisfies $\alpha_1 \nz_{it}(0) \leq z^{\ALG}_{it}(0) \leq \alpha_2 \nz_{it}(0)$.
Then, if patient $i$ and $j$ satisfy $z^{\ALG}_{it}(0) \geq z^{\ALG}_{jt}(0)$, then we have $\nz_{it}(0) \geq \frac{\alpha_1}{\alpha_2} \nz_{jt}(0)$.
Therefore, if one chooses patients with the highest values of $z^{\ALG}_{it}(0)$, their null intervention values will be at least an $\alpha_1/\alpha_2$ factor of the patients that have the highest null intervention values.
That is, at any time $t$, we have
\begin{align*}
\sum_{i \in A_t^{\ALG}} \nz_{it}(0) 
\geq 
\frac{\alpha_1}{\alpha_2} \sum_{i \in D_t(I_t^{\ALG}(0))} \nz_{it}(0),
\end{align*}
where $A_t^{\ALG}$ are the patients that $\ALG$ chooses at time $t$, $I_t^{\ALG}$ are the patients in state 0 at time $t$ under $\ALG$, and $D_t(I_t^{\ALG}) \subseteq I_t^{\ALG}$ are the $B$ patients with the largest null intervention values.
Summing over time steps and taking an expectation results in
\begin{align} \label{eq:comparez}
\bE\left[ \sumT \sum_{i \in A_t^{\ALG}} \nz_{it}(0) \right]
\geq 
\alpha \cdot \bE\left[ \sumT \sum_{i \in D_t(I_t^{\ALG})} \nz_{it}(0) \right].
\end{align}
\cref{prop:alg_minus_base} states that the LHS of \cref{eq:comparez} is equal to $\ALG - \NULL$.
Next, \cref{thm:step2stronger} implies that the RHS is at least $\frac{1}{2} (\OPT - \NULL)$.
Combining yields
\begin{align*}
\ALG - \NULL \geq \frac{\alpha}{2} (\OPT - \NULL)
\end{align*}
as desired.


\subsection{Step 4: Proof of \cref{lemma:rand_null}}  \label{s.app.p4}
Fix a patient $i$ and time $t$. In this proof, we remove the subscript $i$ on $p$, $g$ and $\tau$ for convenience.
\cref{lemma:zgamma} gives us an expression for both $z^{\lambda}_{it}(0)$ and $\nz_{it}(0)$:
\begin{align} 
z^{\lambda}_{it}(0) &= \tau \cdot \bE[\min\{\text{Geometric}(p + g + \gamma \tau (1-p-g)/(1-p)), T-t+1\}] \label{eq:z1}\\
\nz_{it}(0) &= \tau \cdot \bE[\min\{\text{Geometric}(p + g, T-t+1\}]. \label{eq:z2}
\end{align}
Since $p+g < 1$, $p + g + \gamma \tau (1-p-g)/(1-p)) > p+g$, hence $z^{\pi}_{it}(0)/\nz_{it}(0) \leq 1$.
Next, to lower bound the ratio $z^{\pi}_{it}(0)/\nz_{it}(0)$, we use the following lemma:
\begin{lemma} \label{lemma:trunc_geometric}
Suppose $X = \text{Geometric}(P)$, $Y = \text{Geometric}(G)$ with $P > G$, and let $T > 0$ be a positive integer.
Then, 
\begin{align*}
\frac{\bE[\min\{X, T\}]}{\bE[\min\{Y, T\}]} \geq \frac{\bE[X]}{\bE[Y]}.
\end{align*}
\end{lemma}

This lemma allows us to lower bound $z^{\pi}_{it}(0)/\nz_{it}(0)$ by considering the expressions \eqref{eq:z1} and \eqref{eq:z2} without the $\min$ with $T-t+1$.
That is, we have
\begin{align*}
\frac{\bE[\text{Geometric}(p + g + \gamma \tau(1-p-g)/(1-p) )]}{\bE[\text{Geometric}(p + g)]}
&= \frac{1}{1 + \gamma \frac{\tau (1-p-g)}{(p+g)(1-p)}}.
\end{align*}
Using \cref{lemma:trunc_geometric} implies
\begin{align*}
z^{\pi}_{it}/\nz_{it} 
&\geq \frac{1}{1 + \gamma \frac{\tau (1-p-g)}{(p+g)(1-p)}}.
\end{align*}

\begin{myproof}[Proof of \cref{lemma:trunc_geometric}]
Let $X = \text{Geometric}(P)$, $Y = \text{Geometric}(G)$ with $P > G$.
We can write an explicit expression for $\bE[\min\{X, T\}]$ as the following:
\begin{align*}
\bE[\min\{X, T\}] 
&= \sum_{k=1}^{T-1} k (1-P)^{k-1} P + T (1-P)^{T-1} \\
&= -P \frac{d}{dP} \left(\sum_{k=1}^{T-1} (1-P)^{k} \right) + T (1-P)^{T-1} \\
&= -P \frac{d}{dP} \left(\frac{1-(1-P)^T}{P}-1 \right) + T (1-P)^{T-1} \\
&= -P  \left(\frac{ T(1-P)^{T-1} P - (1-(1-P)^T) }{P^2}\right) + T (1-P)^{T-1} \\
&= \frac{1}{P} -  \frac{  (1-P)^T }{P} 
\end{align*}

Using this, we get the desired result:
\begin{align*}
	\frac{\bE[\min\{X, T\}] }{\bE[\min\{Y, T\}] } 
	&= \frac{\frac{1}{P} -  \frac{  (1-P)^T }{P}}{\frac{1}{G} -  \frac{  (1-G)^T }{G}} \\
	&= \frac{\frac{1}{P}(1-(1-P)^T)}{ \frac{1}{G}(1-(1-G)^T)} \\
	&\geq \frac{1/P}{1/G} \\
	&=\frac{\bE[X]}{ \bE[Y] },
\end{align*}
where the inequality follows since $P > G$.
\end{myproof}


\subsection{Proof of \cref{corr:approx}} \label{sec:app:pf_approx}

This result follows from applying \cref{prop:approx_index} and \cref{lemma:rand_null}.
Let $\ALG$ be a policy that satisfies \eqref{eq.approxvalues2}.
Then, by \cref{lemma:rand_null}, we have that the following holds for all $i$ and $t$:
\begin{align}
\frac{1}{1 + \gamma M_i} c_1 \nz_{it}(0) \leq z^{\ALG}_{it}(0)\leq c_2 \nz_{it}(0).
\end{align}
Then, we apply \cref{prop:approx_index} with $\alpha_1 = \frac{1}{1 + \gamma M_i} c_1$ and $\alpha_2 = c_2$, and the result follows.















\section{Decomposition Bias Example} \label{s.app.decompbias}

\subsection{Proof of \cref{prop:dpi_can_be_worse}} \label{s.proof_dpi_worse}
We describe an instance where $\pi_0$ is an optimal policy, but $\DPI(\pi_0)$ is a suboptimal policy.
Consider the following instance of the two-state MDP model from \cref{ss.2statemodel}.
There are $N=3$ patients, where the first two patients have parameters $p_i = 0, \tau_i = 0.01, g_i = 0$ for $i=1, 2$, and the third patient has parameters $p_3 = 0, \tau_3 = 0.01, g_3 = 0.1$.
When patients 1 or 2 goes to state 1, they stay there indefinitely, while patient 3 does not (since $g_3 > 0$). 
Since all other parameters are the same, the value of an intervention is strictly higher for patient 1 and 2, versus patient 3.

Suppose all patients start at state 0, and there are $T=5$ time steps.
Let $\pi_0$ be a policy that assigns at most one intervention per time step defined using the following rules:
\begin{itemize}
  \item If both patients 1 and 2 are in state 0, assign it to one of them uniformly at random.
  \item Otherwise, if either patient 1 or 2 is in state 0, assign them an intervention.
  \item If neither patient 1 or 2 are in state 0, then assign an intervention to patient 3.
\end{itemize}
Notice that $\pi_0$ is the optimal policy.
However, $\DPI(\pi_0)$ ends up being a suboptimal policy.
At time $t=3$, the intervention value for patient 3 is $z_{33}^{\pi_0}(0)=0.029$, whereas the intervention value for patient 1 and patient 2 is $z_{13}^{\pi_0}(0)=z_{23}^{\pi_0}(0)=0.010$.
Therefore, at time $t=3$, the intervention value for patient 3 is higher than that of patient 1 or 2, and hence $\DPI(\pi_0)$ will (suboptimally) prioritize patient 3 at time 3.

The reason for this behavior is that under $\pi_0$, the only time that patient 3 receives an intervention at time 3 is in the unlikely event where both patient 1 and 2 are in state 1 at time 3.
When this event occurs, since patients 1 and 2 stay in state 1 indefinitely, patient 3 is also guaranteed to receive an intervention at time $t=4$ (if they did not switch to state 1 by then).
Then, the $q$-value for patient 3 at time 3, $q^{\pi_0}_{33}(s=0, a=1)$, 
incorporates the fact that they will \textit{also receive an intervention at time 4}.
Therefore, the intervention value, $z_{33}^{\pi_0}(0)$ effectively represents the increase in reward when patient 3 is given interventions at both time 3 and 4, and hence is higher than the intervention value for patient 1 and 2.
This behavior stems from the fact that $q^{\pi_0}_{33}(s=0, a=1)$ does not contain information regarding the fact that under $\pi_0$, patient 3 only receives an intervention under a very specific \textit{system} state. 


\added{
\subsection{Proof of \cref{thm:improvement_random}} \label{s.proof_improvement_random}
Let $\pi_0 = \RAND(\gamma)$ and let $\pi_1 = \DPI(\RAND(\gamma))$.
For any system policy $\pi$, let $V_t^{\pi}$ and $Q_t^{\pi}$ denote the system value function and the system Q-function respectively for the system MDP under policy $\pi$:
\begin{align*} 
V_{t}^\pi(\tilde{\bS}) &= \bE_{\pi}\left[\sum_{t'=t}^T \sumN R(S_{it'}^\pi, S_{i,t'+1}^\pi, A_{it'}^\pi) \;\bigg|\; \bS_t^\pi = \tilde{\bS} \right], \\
Q_{t}^\pi(\tilde{\bS}, \tilde{\bA}) &= \bE_{\pi}\left[\sum_{t'=t}^T \sumN R(S_{it'}^\pi, S_{i,t'+1}^\pi, A_{it'}^\pi) \;\bigg|\;  \bS_t^\pi = \tilde{\bS},  \bA_t^\pi = \tilde{\bA} \right].
\end{align*}
With this notation, we have that $\RAND(\gamma) = V_1^{\pi_0}(\bS_1)$ and $\DPI(\RAND(\gamma)) = V_1^{\pi_1}(\bS_1)$, where $\bS_1$ is the initial state that specified by the instance.
Our goal is to show $V_1^{\pi_1}(\bS_1) \geq V_1^{\pi_0}(\bS_1)$.

Note that under the policy $\pi_0 = \RAND(\gamma)$, the actions $A_{it}$ are independently chosen for each patient $i$. 
Therefore, the system value function and the Q-functions can be decomposed as:
\begin{align*} 
V_{t}^{\pi_0}(\tilde{\bS}) = \sum_{i=1}^N \big(\gamma q_{it}^{\pi_0}(\tilde{S}_{i}, 1) + (1-\gamma) q_{it}^{\pi_0}(\tilde{S}_{i}, 0) \big), \quad
Q_{t}^{\pi_0}(\tilde{\bS}, \tilde{\bA}) = \sum_{i=1}^N q_{it}^{\pi_0}(\tilde{S}_{i}, \tilde{A}_{i}).
\end{align*}
For a system state $\bS$, let $\bA_t^{\pi_1}(\bS) \in \{0, 1\}^N$ be the action that $\pi_1$ chooses at state $\bS$ at time $t$.
We claim that under any state, it is better to take the action from $\pi_1$ and then continue with $\pi_0$ henceforth, compared to just following $\pi_0$.
\begin{claim} \label{claim:improvement}
For any state $\tilde{\bS}$ and any $t$, $Q_{t}^{\pi_0}(\tilde{\bS}, \bA_t^{\pi_1}(\tilde{\bS}) ) \geq V_{t}^{\pi_0}(\tilde{\bS})$.
\end{claim}

\begin{myproof}[Proof of \cref{claim:improvement}]
Recall that $z^{\pi_0}_{it}(s) = q_{it}^{\pi_0}(s, 1) - q_{it}^{\pi_0}(s, 0)$.
Then, we can write the value function $V_{t}^{\pi_0}(\tilde{\bS})$ as
\begin{align*}
V_{t}^{\pi_0}(\tilde{\bS}) &= \sum_{i=1}^N q_{it}^{\pi_0}(\tilde{S}_{i}, 0) + \gamma \sum_{i=1}^N  z^{\pi_0}_{it}(\tilde{S}_i).
\end{align*}
Similarly, we can write $Q_{t}^{\pi_0}(\tilde{\bS}, \bA_t^{\pi_1}(\tilde{\bS}))$ as
\begin{align*}
Q_{t}^{\pi_0}(\tilde{\bS}, \bA_t^{\pi_1}(\tilde{\bS})) &= \sum_{i=1}^N q_{it}^{\pi_0}(\tilde{S}_{i}, 0) + \sum_{i: \bA_t^{\pi_1}(\tilde{\bS})_i =1}  z^{\pi_0}_{it}(\tilde{S}_i).
\end{align*}
Note that by definition of $\pi_1$, $\bA_t^{\pi_1}(\tilde{\bS})_i =1$ for patients $i$ with the $B$ highest values of $z^{\pi_0}_{it}(\tilde{S}_i)$.
Since $B \geq \gamma N$, we have that $\sum_{i: \bA_t^{\pi_1}(\tilde{\bS})_i =1}  z^{\pi_0}_{it}(\tilde{S}_i) \geq \gamma \sum_{i=1}^N  z^{\pi_0}_{it}(\tilde{S}_i)$.
Therefore, $Q_{t}^{\pi_0}(\tilde{\bS}, \bA_t^{\pi_1}(\tilde{\bS})) \geq V_{t}^{\pi_0}(\tilde{\bS})$.
\end{myproof}

We leverage the claim to show that $V_1^{\pi_0}(\bS) \leq V_1^{\pi_1}(\bS)$ for any $\bS$, which finishes the proof.
\begin{align*}
V_1^{\pi_0}(\bS) 
&\leq Q_{1}^{\pi_0}(\bS, \bA^{\pi_1}(\bS)) \\
&=\bE_{\bS' \sim P(\bS, \cdot, \pi_1)}\left[ \sum_{i=1}^N R_i(S_{i}, S'_{i}, \bA_1^{\pi_1}(\bS)_i) +  V_{2}^{\pi_0}(\bS') \right] \\
&\leq \bE_{\bS' \sim P(\bS, \cdot, \pi_1)}\left[ \sum_{i=1}^N R_i(S_{i}, S'_{i}, \bA_1^{\pi_1}(\bS)_i) +  Q_{2}^{\pi_0}(\bS', \bA^{\pi_1}(\bS')) \right] \\
&= \bE_{\bS' \sim P(\bS, \cdot, \pi_1), \bS'' \sim P(\bS', \cdot, \pi_1)}\left[ \sum_{i=1}^N R_i(S_{i}, S'_{i}, \bA_1^{\pi_1}(\bS)_i) +  \sum_{i=1}^N R_i(S'_{i}, S''_{i}, \bA_2^{\pi_1}(\bS')_i) +  V_{3}^{\pi_0}(\bS'') \right] \\
&\leq \ldots \\
&\leq \bE_{\bS' \sim P(\bS, \cdot, \pi_1), \bS'' \sim P(\bS', \cdot, \pi_1), \dots}\left[ \sum_{i=1}^N R_i(S_{i}, S'_{i}, \bA_1^{\pi_1}(\bS)_i) +  \sum_{i=1}^N R_i(S'_{i}, S''_{i}, \bA_2^{\pi_1}(\bS')_i) +  \dots \right] \\
&= V^{\pi_1}_1(\bS).
\end{align*}
}



\section{Details on Keheala Case Study} \label{s.app.keheala}

\subsection{List of features} \label{s.app.list_features}


\textbf{Static features.}
For each patient, we include the following static covariates: weight, height, age, sex, language, county, HIV positive, and extrapulmonary TB.
There were 6 different counties where we used a one-hot encoding, which resulted in 13 features in total.


\textbf{Condensed history.}
For patient $i$ at time $t$, we summarize their history, $H_{it} = (V_{i1}, A_{i1}, \dots, V_{i, t-1}, A_{i,t-1}, V_{it}) \in \bR^{2t-1}$, using the following features:
\begin{itemize}
	\item Verifications: total so far, total percentage, total last week, X days ago for the last $X \in \{1, \dots, 7\}$ days.
	\item Verification/non-verification streaks (how many days in a row a patient verifies / does not verify): current streak, longest streak
	\item Interventions: total so far, total last week, X days ago for the last $X \in \{1, 2, 3\}$ days.
	\item Number of days on the platform, number of days of treatment left.
\end{itemize}
This results in 21 features in total for the condensed history.
A similar structure of features was used in \cite{Boutilier22Improving} which analyzed the same dataset.



\subsection{Simulation Model Details} \label{s.app.simulation}

We briefly describe the double ML method of \cite{chernozhukov2018double}.
Let $Y \in \bR$ be the outcome variable, $T \in \{0, 1\}$ the treatment, and $X \in \bR^{d}$ the observable features.
The model makes the following structural assumptions:
\begin{align*}
Y &= \tau(X) \cdot T + g(X) + \eps, \\ 
T &= f(X) + \eta,
\end{align*}
where $\bE[\eps | X] = 0, \bE[\eta | X] = 0$, and $\bE[\eps \cdot \eta | X] = 0$.
The goal is to estimate the conditional average treatment effect, $\tau(X)$.
We estimate two functions:
\begin{align*}
q(X) = \bE[Y \;|\; X], \quad
f(X) = \bE[T \;|\; X].
\end{align*}
Then, we compute the residuals
\begin{align*}
\tilde{Y} = Y - q(X), \quad \tilde{T} = T - f(X).
\end{align*}
Lastly, we estimate
\begin{align*}
\hat{\tau} = \argmin_{\tau} \bE[(\tilde{Y} - \tau(X) \cdot \tilde{T})^2].
\end{align*}

For the Keheala case study, we used gradient boosting to estimate $q$ and $f$, and we assumed a linear function for $\tau(X) = \theta^\top X$.

Lastly, we cap $\hat{\tau}(s)$ so that the resulting function $f(s, 1)$ is between 0 and 1.
That is, we let $\tau(s) = \max\{\min\{\hat{\tau}(s), 1 - f(s, 0)\}, -f(s, 0)\}$.







\subsection{Details on the Bandit Algorithm} \label{sec:app:bandit}
The pseudocode for the bandit algorithm used in the experiments can be found in \cref{alg:bandit}.
The bandit is initialized with the offline dataset, the same data that was used to train $\DPI$ -- the $\mathbf{S}(a)$  and $\mathbf{V}(a)$ that is an input in \cref{alg:bandit} comes from the offline data. 
The algorithm uses a Thompson Sampling approach, where a parameter is sampled from the posterior, and the intervention is given to those with the highest intervention values with respect to the sampled parameter. 
In our experiments, we used $\sigma^2 = 1/4$ \added{(we tried several different parameters values and chose the best one)}. 

\begin{algorithm}
\caption{Thompson Sampling Bandit Policy}
\label{alg:bandit}
\begin{algorithmic}[1] %
\Require Budget $B$, $\mathbf{S}(a), \mathbf{V}(a)$ for $a \in \{0, 1\}$, noise $\sigma^2$.
	\For{$t=1, \dots, T$}
	\For{$a \in \{0, 1\}$}
		\State $\hat{\beta}(a) = (\mathbf{S}(a)^{\top} \mathbf{S}(a))^{-1} \mathbf{S}(a)^{\top} \mathbf{V}(a)$ 
		\State $\hat{\Sigma}(a) = \sigma^2 (\mathbf{S}(a)^{\top} \mathbf{S}(a))^{-1}$
		\State $\tilde{\beta}(a) \sim \text{Normal}(\hat{\beta}(a), \hat{\Sigma}(a))$
			\Comment Sample parameter from posterior
	\EndFor
	\For{$i=1, \dots, N$}
	\State $\tilde{z}_{it} = (\tilde{\beta}(1)-\tilde{\beta}(0))^{\top} S_{it}$
	\Comment Sampled intervention values for each patient
	\EndFor
	\State Assign interventions to $B$ patients with the largest $\tilde{z}_{it}$ values.
	\For{$i=1, \dots, N$}
		\Comment Update parameters
		\State $V_{i,t+1} = 1 \text{ if patient $i$ verified at $t+1$, otherwise } 0$
		\If{$A_{it} = 1$}
		\State $\mathbf{S}(1) = \text{Append}(\mathbf{S}(1), S_{it})$
		\State $\mathbf{V}(1) = \text{Append}(\mathbf{V}(1), V_{i,t+1})$
		\Else
		\State $\mathbf{S}(0) = \text{Append}(\mathbf{S}(0), S_{it})$
		\State $\mathbf{V}(0) = \text{Append}(\mathbf{V}(0), V_{i,t+1})$
		\EndIf
    \EndFor
    \EndFor
 \end{algorithmic}
\end{algorithm}


\added{We note that action-centered contextual bandits \citep{greenewald2017action} is a variant of Thompson Sampling that has been used in a similar setting (e.g., HeartSteps study of \cite{liao2020personalized}).
However, this method does not readily apply to our setting due to the budget constraint, hence we use \cref{alg:bandit}.
}

\subsection{Details on the QWI algorithm} \label{sec:app:qwi}

\paragraph{Learning the index.}
For a given $\lambda \in \bR$, we consider a process in which every time action 0 is taken, a reward of $\lambda$ is awarded.
We define $Q(S, A; \lambda)$ as the Q-value from state $S$ and action $A$ under this process.
Then, the $Q$-values satisfy:
\begin{align}
	Q(S, A; \lambda) = r(S, A) + (1-A) \lambda - \beta + \bE_{S' \sim P(S, \cdot, A)}[\max_{a'} Q(S', A'; \lambda)],
\end{align}
where $\beta \in \bR$ is the optimal reward.
Then, the Whittle's index for state $s$ is the value $\lambda(s)$ that satisfies
\begin{align}
	Q(S, 0; \lambda(S)) = Q(S, 1; \lambda(S)).
\end{align}

Recall that we have offline samples $\{(S_i, A_i, v_i, \ell_i, S_i')\}$, where $v_i \in \{0, 1\}$ is the outcome of whether the patient verified, $\ell_i \in \bN$ is the number of days the patient has left, and $S_i'$ is the next state of the patient.
Fix a number $\lambda$, and we will aim to learn $Q(S, A; \lambda)$, which represents the future verification rate.
Initialize $Q_0(\cdot, \cdot; \lambda) = 0$.
Let $\cI(S, A) = \{i : S_i = S, A_i = A\}$ be the data points with state $S$ and action $A$.
For $t = 1, 2, \dots$, update the $Q$-values as:
\begin{align} \label{eq:q_iteration}
Q_t(S, A; \lambda) = \frac{1}{|\cI(S, A)|} \sum_{i \in \cI(S, A)} \frac{1}{\ell_i} \big(v_i  + (1-A_i) \lambda + (\ell_i-1) \max_{a'}Q_{t-1}(S_i', A')\big).
\end{align}	
Then, for every state $u$, to find the $\lambda(S)$ such that $Q(S, 0; \lambda(S)) = Q(S, 1; \lambda(S))$, 
we did a binary search across $\lambda$. 
At each iteration of the binary search, for a given value of $\lambda$, we ran the above method \eqref{eq:q_iteration} for $t=100$ iterations.

\subsection{Details on Prominent Features for $\DPI$} \label{s.app.coefficients}
We describe how \cref{tab.coefs} was generated. 
First, we compute $\hat{\theta}_0$ using the least square regression in \eqref{eq:leastsquares}.
Then, for each sample $(S_{it}, A_{it}, y_{it})$ where $A_{it} = 1$, we create a new target $\tilde{y}_{it} = y_{it} - \hat{\theta}_0^{\top} S_{it}$, which simply subtracts off the prediction from the first regression.
Then, the intervention value is essentially the regression of $S_{it}$ onto the new target values.
Instead of performing the usual least-squares, we first normalize each feature so that they have a standard deviation of 1, and then we perform a Lasso regression \citep{tibshirani1996regression}:
\begin{align*} 
	\tilde{\theta} &\in \argmin_{\theta \in \bR^{34}} \bigg( \sum_{i \in N} \sum_{t=T_{\text{s}}(i)}^{T_{\text{e}}(i)}  \bI(A_{it} = 1)(\tilde{y}_{it} - \theta^\top \tilde{S}_{it})^2 + \lambda ||\theta||^2_1 \bigg),
\end{align*}
where $\tilde{S}$ represent the states after column normalization.
We chose the value of $\lambda$ so that the output $\tilde{\theta}$ has five non-zero entries.
These are the features and their respective sign of the coefficient that is shown in \cref{tab.coefs}.

















\end{document}