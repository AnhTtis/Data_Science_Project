
We walk through an example where the decomposition of the $q$ values leads $\DPI$ to an undesirable policy.
Specifically, in this example, we start with $\pi_0$ being the optimal policy, but $\DPI(\pi_0)$ is a suboptimal policy.
Consider the following instance of the two-state MDP model from \cref{ss.2statemodel}.
There are $N=3$ patients, where the first two patient has parameters $p_i = 0, \tau_i = 0.01, q_i = 0$ for $i=1, 2$, and the third patient has parameters $p_3 = 0, \tau_3 = 0.01, q_3 = 0.1$.
When patient 1 or 2 goes to state 1, they stay there indefinitely, while patient 3 does not (since $q_3 > 0$). 
Since all other parameters are the same, the value of an intervention is strictly higher for patient 1 and 2, versus patient 3.

Suppose all patients start at state 0, and there are $T=5$ time steps.
Let $\pi_0$ be a policy that assigns at most one intervention per time step defined using the following rules:
\begin{itemize}
  \item If both patients 1 and 2 are in state 0, assign it to one of them uniformly at random.
  \item Otherwise, if either patient 1 or 2 is in state 0, assign them an intervention.
  \item If neither patient 1 or 2 are in state 0, then assign an intervention to patient 3.
\end{itemize}
Notice that $\pi_0$ is the optimal policy.
However, $\DPI(\pi_0)$ ends up being a suboptimal policy.
At time $t=3$, the intervention value for patient 3 is $z_{33}^{\pi_0}(0)=0.029$, whereas the intervention value for patient 1 and patient 2 is $z_{13}^{\pi_0}(0)=z_{23}^{\pi_0}(0)=0.010$.
Therefore, at time $t=3$, the intervention value for patient 3 is higher than that of patient 1 or 2, and hence $\DPI(\pi_0)$ will (suboptimally) prioritize patient 3 at time 3.

The reason for this behavior is that under $\pi_0$, the only time that patient 3 receives an intervention at time 3 is in the unlikely event where both patient 1 and 2 are in state 1 at time 3.
When this event occurs, since patients 1 and 2 stay in state 1 indefinitely, patient 3 is also guaranteed to receive an intervention at time $t=4$ (if they did not switch to state 1 by then).
Then, the $q$-value for patient 3 at time 3, $q^{\pi_0}_{33}(s=0, a=1)$, 
incorporates the fact that they will \textit{also receive an intervention at time 4}.
Therefore, the intervention value, $z_{33}^{\pi_0}(0)$ effectively represents the increase in reward when patient 3 is given interventions at both time 3 and 4, and hence is higher than the intervention value for patient 1 and 2.
This behavior stems from the fact that $q^{\pi_0}_{33}(s=0, a=1)$ does not contain information regarding the fact that under $\pi_0$, patient 3 only receives an intervention under a very specific \textit{system} state. 

