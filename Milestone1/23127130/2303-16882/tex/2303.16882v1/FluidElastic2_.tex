

% Modif. June 2, 2010
% Send comments to publ@impan.pl

\documentclass[12pt, twoside]{article}
\usepackage{amsmath,amsthm,amssymb}
\usepackage{times}
\usepackage{color}
\usepackage{enumerate}
\usepackage{wrapfig}
\usepackage{graphicx}

\pagestyle{myheadings}
%\markboth{Th. Jankuhn, M. Olshanskii, and A. Reusken}{Fluid problems on surfaces}


%% Numbered objects (text of theorems etc. is NOT italicized).
%% The optional parameters indicate that all objects are numbered together, and "by section".
%% However, you are welcome to use any other numbering system of your choice.

\theoremstyle{definition}
\newtheorem{thm}{Theorem}[section]
\newtheorem{cor}[thm]{Corollary}
\newtheorem{lem}[thm]{Lemma}
\newtheorem{defin}[thm]{Definition}
\newtheorem{rem}[thm]{Remark}
\newtheorem{exa}[thm]{Example}

%% A numbered theorem with a fancy name:

\newtheorem{mainthm}[thm]{Main Theorem}

%% An unnumbered remark:

\newtheorem*{xrem}{Remark}


%% Equations numbered by section:

\numberwithin{equation}{section}


%%%%%%%%%%% For IFB
\frenchspacing

\textwidth=167mm
\textheight=23cm
\parindent=16pt
\oddsidemargin=-0.5cm
\evensidemargin=-0.5cm
\topmargin=-0.5cm

\newcommand{\subjclass}[1]{\bigskip\noindent\emph{2010 Mathematics Subject Classification:}\enspace#1}
\newcommand{\keywords}[1]{\noindent\emph{Keywords:}\enspace#1}
%%%%%%%%%%%%%%%%%%%%%%%%%%%%%%%%%%%
%%%%%%%%%%%%%%%%%%%%%%%%%%%%%%%%%%%

%%%% Put your macros here:

\let\pa\partial
\let\eps\varepsilon
\newcommand{\R}{\mathbb R}
\newcommand{\bA}{\mathbf A}
\newcommand{\bB}{\mathbf B}
\newcommand{\bC}{\mathbf C}
\newcommand{\bD}{\mathbf D}
\newcommand{\bE}{\mathbf E}
\newcommand{\bH}{\mathbf H}
\newcommand{\bI}{\mathbf I}
\newcommand{\bJ}{\mathbf J}
\newcommand{\bP}{\mathbf P}
\newcommand{\bK}{\mathbf K}
\newcommand{\bM}{\mathbf M}
\newcommand{\bQ}{\mathbf Q}
\newcommand{\bS}{\mathbf S}
\newcommand{\cV}{\mathcal{V}_{\rm div}}
\newcommand{\bV}{\mathbf V}
\newcommand{\ga}{\gamma}
\newcommand{\ba}{\mathbf a}
\newcommand{\bh}{\mathbf h}
\newcommand{\bb}{\mathbf b}
\newcommand{\bg}{\mathbf g}
\newcommand{\bn}{\mathbf n}
\newcommand{\be}{\mathbf e}
\newcommand{\bp}{\mathbf p}
\newcommand{\br}{\mathbf r}
\newcommand{\bs}{\mathbf s}
\newcommand{\bT}{\mathbf T}
\newcommand{\bu}{\mathbf u}
\newcommand{\bv}{\mathbf v}
\newcommand{\bw}{\mathbf w}
\newcommand{\bx}{\mathbf x}
\newcommand{\by}{\mathbf y}
\newcommand{\bz}{\mathbf z}
\newcommand{\bbf}{\mathbf f}
\newcommand{\A}{\mathcal A}
\newcommand{\F}{\mathcal F}
\newcommand{\G}{\mathcal G}
\newcommand{\I}{\mathcal I}
\newcommand{\K}{\mathcal K}
\newcommand{\M}{\mathcal M}
\newcommand{\N}{\mathcal N}
\newcommand{\Q}{\mathcal Q}
\newcommand{\J}{\mathcal J}
\newcommand{\T}{\mathcal T}
\newcommand{\V}{\mathcal V}
\newcommand{\Div}{\mathop{\rm div}}
\newcommand{\divG}{{\mathop{\,\rm div}}_{\Gamma}}
\newcommand{\gradG}{\nabla_{\Gamma}}
\newcommand{\cH}{\mathcal H}
\newcommand{\cD}{\mathcal D}
\newcommand{\cF}{\mathcal F}
\newcommand{\cE}{\mathcal E}
\newcommand{\cO}{\mathcal O}
\newcommand{\cT}{\mathcal T}
\newcommand{\cP}{\mathcal P}
\newcommand{\cR}{\mathcal R}
\newcommand{\cS}{\mathcal S}
\newcommand{\cL}{\mathcal L}
\newcommand{\rr}{\mathbb{R}}
\newcommand{\cc}{\mathbb{C}}
\newcommand{\kk}{\mathbb{K}}
\newcommand{\nn}{\mathbb{N}}
\newcommand{\dd}{\,d}
\newcommand{\zz}{\mathbb{Z}}
\newcommand{\Gs}{\mathcal{S}} %{\Gamma_\ast}
%\newcommand{\Gsn}{{\Gamma_\ast^n}}
%\newcommand{\normal}{n}
\newcommand{\wn}{w_{\!\perp}}
\newcommand{\ddp}{\partial}
\newcommand{\ati}{\tilde{a}}
\newcommand{\wtang}{\bw_\parallel}
%\newcommand{\grad}{\nabla}
\newcommand{\jumpleft}{[\![}
\newcommand{\jumpright}{]\!]}
\newcommand{\averageleft}{\{\!\!\{}
\newcommand{\averageright}{\}\!\!\}}
\newcommand{\divergence}{\textrm{div}\ \!}
\renewcommand{\div}{\mathop{\rm div}}
\newcommand{\HSobolev}{H}
\newcommand{\Lzwei}{L^2}
\newcommand{\Rn}{\Bbb{R}^n}
\newcommand{\Rm}{\Bbb{R}^m}
\newcommand{\tr}{{\rm tr}}
%\newcommand{\alert}[1]{{\bf #1}}
\def\Fh{\mathcal{F}_h}
\def\Eh{\mathcal{E}_h}
\newcommand{\la}{\left\langle}
\newcommand{\ra}{\right\rangle}
\newcommand{\Yo}{\overset{\circ}{Y}}
%\usepackage{showkeys}
\newcommand{\deriv}[2]{\frac{\partial #1}{\partial #2}}
\newcommand{\bsigma}{\boldsymbol{\sigma}}
\newcommand{\bphi}{\boldsymbol{\phi}}
\newcommand{\DG}{D_\Gamma}
\newcommand{\DGh}{D_{\Gamma_h}}
\newcommand{\bxi}{\mbox{\boldmath$\xi$\unboldmath}}

\newcommand{\mo}[1]{{\color{red}#1}}
\newcommand{\yp}[1]{{\color{blue}#1}}

\usepackage{accents}
\renewcommand*{\dot}[1]{%
	\accentset{\mbox{\large\bfseries .}}{#1}}
\usepackage{cite}

%%%%%%%%%%%%%


\begin{document}

%%%%% To ease editing, add:

%\baselineskip=17pt

%%%%%%%%%%%%%%%%

\title{On equilibrium states of fluid membranes}

\author{
	Maxim A. Olshanskii\thanks{Department of Mathematics, University of Houston, Houston, Texas 77204-3008; { maolshanskiy@uh.edu}}
}
\date{}
\maketitle






%%%%%%%%

\begin{abstract}	The paper studies equilibrium configurations of inextensible elastic membranes exhibiting lateral  fluidity.  Differential equations governing the mechanical equilibrium are derived using a continuum description of the membrane motions given by the surface Navier--Stokes equations  with bending forces. 
Equilibrium conditions are found to be independent of lateral viscosity and relate  tension,  pressure and tangential velocity of the fluid. The conditions yield that only surfaces with Killing vector fields, such as axisymmetric shapes, can support non-zero stationary flow of mass. A shape equation is derived that extends a classical Helfrich model with area constraint to membranes of non-negligible  mass. A simple numerical method to compute solutions of the shape equation is suggested. Numerical experiments reveal a diverse family of equilibrium configurations.
\end{abstract}

%\subjclass{37E35, 76A20, 35Q35, 35Q30, 76D05}

%\keywords{Fluids on surfaces,  viscous  material  interface, fluid membrane, Navier--Stokes equations on manifolds}

%
\section{Introduction}
Motivated by applications in cell biology there has  been an extensive research on studying equilibrium configurations of fluid membranes, their stability and transformations; see e.g. \cite{deuling1976curvature,jenkins1977static,peterson1985instability,seifert1991shape,seifert1995morphology,seifert1997configurations,deserno2015fluid}.  A now classical energetic approach to describe the statics and dynamics of fluid membranes
was pioneered by Canham~\cite{canham1970minimum}  and Helfrich~\cite{helfrich1973elastic}.
%Pioneered by Canham~\cite{canham1970minimum}  and Helfrich~\cite{helfrich1973elastic} a now classical energetic approach to describe the statics and dynamics of fluid membranes relies on the concept of bending elasticity. 
According to Canham--Helfrich theory an equilibrium shape of membrane  minimizes a curvature energy functional subject to possible constraints.  
More complex models  account  for in-plane fluidity exhibited by the membranes. In a continuum based modeling approach, the membrane is represented by a  \emph{material surface} which supports a density flow and may deform driven by both elastic and hydrodynamic forces. The development, analysis and application of such models to numerical simulation  of  fluid membranes dynamics is an area of active recent research~\cite{arroyo2009,hu2007continuum,rangamani2013interaction,barrett2015numerical,jankuhn2018incompressible,nitschke2019hydrodynamic,torres2019modelling,reuther2020numerical,krause2022numerical}.  A system of equations governing the motion of fluid membrane is then consists of the surface Navier--Stokes equations coupled with an elasticity model and posed on a time-dependent surface, which evolution is defined by the hydrodynamic part of the solution; see Sec.~\ref{sectmodel} for further details.
The governing equations represent the conservation of momentum  so that a steady state of the system is the mechanical equilibrium. While the existing research on fluid-elastic deformable surfaces mainly addresses the system evolution, in the present study we are interested in equilibrium configurations.

Assuming a steady state solution to the surface Navier--Stokes equations with vanishing external lateral forces, we deduce three conditions for such solutions to exist: the radial motions are zero, the lateral motions correspond to a Killing field on manifold, and the third condition requires that a specific surface pressure  (defined in \eqref{pressure}) is constant. 
The first two conditions imply that only two scenarios of equilibrium  are possible: Either the fluid motion ceases and the problem reduces to the well studied one of finding a shape of minimal curvature energy under the area constraint, or an additional geometrical constraint arises for  the equilibrium shape to support a non-decaying lateral fluid flow.
To explore the second scenario, we make use of the third condition and  a specific elasticity model (the simplest Helfrich model in this paper) to deduce the \emph{shape equation}, an equation satisfied by geometrical quantities of the system in equilibrium. A numerical approach to solve the shape equation is introduced, which exploits the axial symmetry of the unknown surface.  Further, the shape equation is solved numerically to find branches of shapes with a fixed surface area and varying interior  volume for several sets of physical parameters.  In particular, we find steady states of the surface fluid equations with vanishing elastic forces, which correspond to equilibrium configurations of `pure fluidic' membranes.  

The remainder of the paper is organized in four sections. Section~\ref{sectmodel} recalls the  continuum model of a  fluid--elastic membrane and derives the equilibrium conditions and the shape equation. Section~\ref{sec:num} introduces numerical solver. The computed shapes are discussed in section~\ref{sec:shapes}. Section~\ref{sec:concl} offers a few concluding remarks. 
For model setup and analysis, the paper uses elementary tangential calculus in embedding $\R^3$ space and thus avoiding any calculations in local surface coordinates.   

\section{A deforming fluid--elastic membrane} \label{sectmodel}
Following the continuum--mechanical description, we 
represent membrane by a smooth closed time-dependent surface $\Gamma(t)\subset \mathbb{R}^3$ with a density distribution $\rho(x,t)$~\cite{GurtinMurdoch75,MurdochCohen79}. Denote by $\bu$ a smooth velocity field  of the density flow on $\Gamma$,  i.e. $\bu(x,t)$ is the velocity of the material point  $x\in\Gamma(t)$. In general, $\bu$ is not necessarily tangential to $\Gamma$ and its normal component defines the geometric evolution of $\Gamma$.

 

To formulate equations governing the membrane motion, we need a few surface quantities and tangential differential operators: Let $\bn$ be an outward pointing normal vector on $\Gamma$ and let $\bP=\bI- \bn\bn^T$ denote  the normal projector. The surface gradient  $\nabla_\Gamma p$ of a scalar function $p:\, \Gamma \to \mathbb{R}$ can be defined as $\nabla_\Gamma p=\nabla p^e$, where $p^e$ is an arbitrary smooth extension of $p$ in a neighborhood of $\Gamma$.   Also  $\nabla_\Gamma \bu =\bP(\nabla_\Gamma u_1, \nabla_\Gamma u_2, \nabla_\Gamma u_3)^T$   is a surface  gradient of a  vector field $\bu=(u_1,u_2,u_3)^T:\, \Gamma \to \mathbb{R}^3$, i.e. $\nabla_\Gamma \bu$ is covariant gradient if $\bu$ is tangential to $\Gamma$, 
  $\divG \bu = \tr (\gradG \bu)$ is surface divergence, and $\Delta_\Gamma p= \divG\nabla_\Gamma p$ is the Laplace--Beltrami operator. For a tensor field $\bA=[\ba_1,\ba_2,\ba_3]: \Gamma \to \mathbb{R}^{3\times 3}$ the surface divergence $\divG \bA$  is defined row-wise. 


The conservation of mass and linear momentum for an arbitrary material area $\gamma(t)\subset\Gamma(t)$ together with the Boussinesq–Scriven constitutive relation for the surface stress tensor and the membrane inextensibility condition  leads to the evolving surface Navier-Stokes equations for the viscous thin material layer (see, e.g. \cite{jankuhn2018incompressible}):
\begin{equation} \label{momentum}
\left\{
\begin{split}
 \rho \dot \bu & =- \nabla_\Gamma  p + 2\mu \divG (\bD_\Gamma(\bu))  + \bb +   p \kappa\bn, \\
 \divG \bu  & =0,\\
 \dot{\rho} & =0,
\end{split}\right.\quad\text{on}~\Gamma(t),
\end{equation}
where  $\kappa$ is the double mean curvature, $ p$ is the surface pressure, $\mu$ is the viscosity,  $\bD_\Gamma(\bu)=\frac12(\nabla_\Gamma \bu + \nabla_\Gamma \bu^T)$ is the surface rate-of-strain tensor \cite{GurtinMurdoch75} and $\dot \bu$, $\dot \rho$ stand for material derivatives of $\bu$ and $\rho$.  If  $\rho$ is smoothly extended to a space-time neighborhood of $\Gamma(t)$, then a common representation  holds 
\[
\dot{\rho} = \frac{\partial \rho}{\partial t}+(\bu\cdot\nabla)\rho,
\]
which is independent of the particular extension. Same identity holds for $\bu$ componentwise. 

 The geometrical evolution of the surface is defined by the normal velocity $V_\Gamma$ of $\Gamma$  given by
the normal component of the material velocity
\begin{equation} \label{Gamma_evol}
V_\Gamma=\bu\cdot \bn \quad\text{on}~\Gamma(t).
\end{equation}
If $\bb$ is given or defined through other unknowns, then \eqref{momentum}--\eqref{Gamma_evol} form a closed system of six equations for six unknowns $\bu$, $p$, $\rho$, and $V_\Gamma$, subject to suitable initial conditions. Assuming $\rho=\mbox{const}$ at $t=0$,  $\dot{\rho} =0$ implies the density stays constant for all times.


We further distinct between area forces $\bb$ coming from the adjacent inner--outer media and elastic forces generated by the bending and stretching of the membrane,
\begin{equation} \label{forces}
\bb=\bb^{\rm ext}+\bb^{\rm elst}.
\end{equation}
For the purpose of this paper, we assume the external force given by the constant pressure difference across the membrane, 
\begin{equation} \label{force1}
	\bb^{\rm ext}= p^{\rm ext}\bn,\quad \text{with}~~p^{\rm ext}=\mbox{const}.
\end{equation}
For the elasticity, we consider the Helfrich  model with the Willmore energy functional~\cite{canham1970minimum,helfrich1973elastic}:
\begin{equation} \label{Willmore}
H=\frac{c_\kappa}2\int_{\Gamma}(\kappa-\kappa_0)^2\, ds+c_K\int_{\Gamma}K\, ds,
\end{equation}
where $K$ is the Gauss curvature, and  material  parameters $c_\kappa>0$, $c_K>0$, $\kappa_0$  have the meaning of bending rigidity, Gaussian bending rigidity, and spontaneous curvature, respectively.  In this paper we restrict our interest to the simplest model with 
\[
\kappa_0=0.
 \]

For closed surfaces that does not change its topology during evolution, the Gauss--Bonnet theorem implies that the second term in the Willmore energy functional equals to the $2\pi$ times the surface Euler characteristic. Hence this term does not contribute to the variation of the energy and so to the elastic forces.
By the principle of virtual work one gets
\[
\int_{\Gamma(t)}\bb^{\rm elst}\cdot\bv\,ds= %\int_{\Gamma(t)}
-\left.\frac{dH}{d\Gamma}\right|_{\bv},%\,ds,
\]
where $\frac{dH}{d\Gamma}|_{\bv}$ is the variation of the energy functional on the (infinitesimal) displacement of $\Gamma$ given by the vector field $\bv$.

The shape derivative of $H$ can be computed to take the form of
\begin{equation}\label{BE1}
\left.\frac{dH}{d\Gamma}\right|_{\bv} = c_\kappa\int_{\Gamma(t)}(-\Delta_\Gamma \kappa -\frac12 \kappa^3+2K\kappa)(\bv\cdot\bn)\,ds.
\end{equation}
The result in \eqref{BE1} is well-known due to Willmore, see e.g. \cite{willmore1996riemannian}. For completeness, we give in Appendix a short  proof using elementary tangential calculus. 
From \eqref{BE1} it is clear that the release of the bending energy produces a force  in the normal direction to the surface:
\begin{equation} \label{bN}
	\bb^{\rm elst}=c_\kappa(\Delta_\Gamma \kappa +\frac12 \kappa^3-2K\kappa)\bn.
\end{equation}


\begin{rem}\rm The fluid system \eqref{momentum}--\eqref{Gamma_evol} with a generic force $\bb$ was independently derived in \cite{jankuhn2018incompressible} from balance laws of continuum mechanics and  in  \cite{Gigaetal} from energetic principles (with $\bb=0$). 
Equations for moving fluid membrane derived  in local (curvilinear) coordinates in \cite{hu2007continuum} and \cite{nitschke2019hydrodynamic} were shown~\cite{brandner2022derivations} to also  yield \eqref{momentum}--\eqref{Gamma_evol} if they are translated into the language of tangential calculus. %Fluid equations for a deforming surface with elastic forces as in \eqref{bN} are  found in \cite{arroyo2009}. These equations need an updated treatment of inertia terms~\cite{yavari2016nonlinear} to lead  \eqref{momentum}--\eqref{Gamma_evol}, \eqref{bN}.
\end{rem}



\subsection{Conditions of geometric equilibrium} \label{splitting} 
We are interested in the equilibrium state solutions to \eqref{momentum}--\eqref{Gamma_evol} with external and bending forces \eqref{force1}, \eqref{bN}. The \emph{geometric} equilibrium requires $\Gamma(t)$ to be time-independent (in the sense of a shape). This and \eqref{Gamma_evol} immediately implies the  \emph{first equilibrium condition}:
\begin{equation}\label{cond1}
\bu\cdot\bn=0.
\end{equation}


\textbf{Case $\bu=0$.} Let us start with considering the case of no-flow, $\bu=0$. The momentum equation in \eqref{momentum} yields
$\nabla_\Gamma  p=	\bb^{\rm elst} +   (p^{\rm ext}+ p \kappa)\bn$. The left hand side of this identity is tangential to $\Gamma$, while the right hand side is orthogonal, and so both are zero yielding  $ p=\mbox{const}$ (the implication of $\nabla_\Gamma  p=0$ along $\Gamma$). Denote this constant surface pressure by $ p_0$. Then 
$0=	\bb^{\rm elst} +   (p^{\rm ext}+ p \kappa)\bn$ and \eqref{bN} imply
\begin{equation}\label{equl1}
	c_\kappa(\Delta_\Gamma \kappa +\frac12 \kappa^3-2K\kappa) +  p_0 \kappa +p^{\rm ext}=0.
\end{equation}
Equation \eqref{equl1} was also derived in \cite{zhong1989bending} as the  optimality condition for finding the minimum of Willmore energy \eqref{Willmore} with conserved surface area and enclosed volume  and constants $ p_0$, $p^{\rm ext}$ playing a role of Lagrange multipliers for the area and volume constraints, respectively. 
As the constrained minimization  problem, it has been extensively addressed in the literature  in the context of finding shapes of vesicles,  see e.g.~\cite{jenkins1977static,luke1982method,peterson1985instability,seifert1991shape,seifert1997configurations}.% 


%Two branches of axisymmetric solution   start from  the spherical shape as  $\widehat V$ decreases~\cite{seifert1991shape}: one branch  consists of oblate shapes which  become biconcave discocytes for smaller  $\widehat V$, while another branch consists of prolate shapes which acquire a dumbbell form for smaller  $\widehat V$ (see also Fig.~\ref{fig2} below).
%For $\widehat V\lesssim0.51$ the first branch continuous with mechanically irrelevant  self-intersecting shapes.  A third branch of ``stomatocyte'' shapes without reflection symmetry bifurcates  from oblite shapes at $\widehat V\simeq 0.59$ and exists up to  $\widehat V\simeq 0.51$~\cite{seifert1997configurations}.
%No stable non-axisymmetric solutions to \eqref{equl1} are known~\cite{seifert1997configurations}. \textbf{Check that this is still the case!}

We conclude that for the static equilibrium ($\bu=0$) the system \eqref{momentum}--\eqref{Gamma_evol}, \eqref{force1}, \eqref{bN} coincides with a well-studied problem of Willmore energy constrained minimization. The membrane fluidity does not play a role in this scenario. We now consider the case of dynamic equilibrium.  

\textbf{Case $\bu\neq0$.} The geometric equilibrium condition still implies $\bu\cdot\bn=0$ (only lateral motions are allowed). 
To deduce other conditions, we  consider the tangential part of the momentum equation \eqref{momentum}. 
This can be done by applying the orthogonal projection $\bP$ to the first equation in  \eqref{momentum} and noting that $\bP\bn=0$ and hence
$\bP\bb=0$. We get
\[
\rho \bP\dot \bu =- \nabla_\Gamma  p + 2\mu \bP\divG (\bD_\Gamma(\bu)).
\]
For tangential field $\bu$ we have $\bP\bu=\bu$  and so $\bP(\bu\nabla \bu)= \bP(\nabla \bu)\bu=\bP(\nabla \bu)\bP\bu =(\nabla_\Gamma \bu) \bu$.  This and $\frac{\partial\bP}{\partial t}=0$ imply the following identity for the projection of material derivative:
\[ 
	\bP\dot \bu  = \bP\Big(\frac{\partial \bu}{\partial t} + (\nabla \bu) \bu\Big) = \frac{\partial \bu}{\partial t} + (\nabla_\Gamma \bu) \bu = \frac{\partial \bu}{\partial t}+(\bu\cdot\nabla_\Gamma)\bu.
\]
We therefore get the following system satisfied by $\bu$ and $ p$:
\begin{equation}\label{NSstat}
	\left\{
	\begin{aligned}
		\rho \left( \frac{\partial \bu}{\partial t}+(\bu\cdot\nabla_\Gamma)\bu \right)&= -\nabla_\Gamma  p + 2\mu\bP \divG \bD(\bu)\\
		\divG \bu  &= 0
	\end{aligned}\right.
\end{equation}
{on} a geometrically stationary $\Gamma$.

Multiplying the first equation in \eqref{NSstat} with $\bu$, integrating over $\Gamma$ and integrating by parts brings us to the energy equality
\[
\frac\rho2\frac{d}{dt}\int_{\Gamma} |\bu|^2\,ds=-2\mu \int_{\Gamma} |\bD_\Gamma(\bu)|^2ds.
\]
We see that the kinetic energy of the lateral flow decays for all motions with $\bD_\Gamma(\bu)\neq0$. 
Therefore,  the equilibrium flow must satisfy  the  \emph{second equilibrium condition}:
\begin{equation}\label{Killing}
	\bD_\Gamma(\bu)=0.
\end{equation}
`Tangentially rigid' motions satisfying \eqref{cond1} and \eqref{Killing} correspond to Killing vector fields on manifolds~\cite{eisenhart1997riemannian,sakai1996riemannian}. 
A non-zero Killing  field  generates a continuous one-parameter group of transformations $\Gamma\to\Gamma$ which are isometries, and  the ability of $\Gamma$ to support it is a \emph{geometric} constraint. 
In particular,  among 2D compact closed surfaces only those of genus 0 and 1 may have non-zero  Killing fields and  the corresponding   group of transformations is 1 parameter with the exception of surfaces of constant  curvature (i.e. those isometric to a sphere), which have the 3 parameter group~\cite{myers1936isometries}. Moreover, the intrinsic geometry of such surfaces is rotationally symmetric, see e.g.  \cite{eisenhart1997riemannian} and \cite[lemma~0.1]{chen2006note}. Additional assumptions on the Gauss curvature ensure (see \cite{nirenberg1953weyl} where the proof is given if $K>0$ on $\Gamma$ or more recent treatment in \cite{engman2004note})  that there is a unique smooth isometric embedding of such surface into $\mathbb{R}^3$ as a classical surface of revolution. We have not found results in the literature from which one may conclude that without additional assumptions on $K$ the classical surface of revolution is the only representation in $\R^3$ of a connected compact closed smooth surface with Killing field, although such conclusion looks very plausible. 



%We conclude that  non-zero  Killing field is supported by  a surface of revolution.
%a 3D closed surface supporting a \textit{non-zero} Killing vector field is a {surface of revolution}. 
%We conclude that $\Gamma$ must be \emph{axisymmetric} in an equilibrium state with $\bu\neq0$.

With the help of $\gradG |\bu|^2 = 2 (\gradG \bu)^T\bu$ and $(\bu\cdot\nabla_\Gamma)\bu= (\gradG \bu)\bu$ one verifies the identity
\[
(\bu\cdot\nabla_\Gamma)\bu= 2\bD_\Gamma(\bu)\bu-\frac12\gradG |\bu|^2.
\]
Using this identity in \eqref{NSstat} we see that for steady flow fields satisfying \eqref{Killing}  the momentum equation reduces to $\gradG  p-\frac\rho2\gradG |\bu|^2=0$. We thus get our \emph{third equilibrium condition}:
\begin{equation}\label{pressure}
	 p-\frac\rho2 |\bu|^2= p_0%\footnote{Opposite to the Bernoulli law, the 'dynamic pressure' $\frac\rho2 |\bu|^2$ appears in \eqref{pressure} with negative sign.}
	\qquad\text{with some}\quad p_0:=const.
\end{equation}
According to \eqref{pressure} the in-surface pressure in an equilibrium state splits into a constant term and a term representing the kinetic energy density. For a pure fluid membrane ($c_\kappa=0$), $ p$ can be interpreted as  the surface tension coefficient, which is found to depend on the in-plane flow.  

Summarizing, we obtain three conditions  for the velocity and pressure  of the fluid membrane in an equilibrium. These conditions \eqref{cond1}, \eqref{Killing}, and \eqref{pressure} are independent of an elasticity model and we use them below together with the particular elasticity model to derive the shape equation.   

\subsection{Shape equations} The  Weingarten mapping (shape operator) $\bH:\Gamma\to\mathbb{R}^{3\times3}$ is given by $\bH=\nabla_\Gamma\bn$.
Note that $\bH=\bH^T$, $\bH\bn=0$. Eigenvectors of $\bH$ orthogonal to $\bn$ are the principle directions on $\Gamma$ and the corresponding eigenvectors are the curvatures $\kappa_1$ and $\kappa_2$. In particular, $\kappa=\kappa_1+\kappa_2:=\tr(\bH)$. 
We also need the following identity  for the material derivative of $\bn$  (see eq.~(2.14) in \cite{jankuhn2018incompressible}): 
\begin{equation} \label{aux442}
	\dot{\bn}=\bH \bu - \gradG (\bu\cdot\bn) .
\end{equation}
To deduce the shape equation, we first take the normal part of the momentum equation \eqref{momentum},
\begin{equation} \label{aux372}
\rho \bn\cdot\dot\bu  = 2 \mu \bn\cdot \divG \bD_\Gamma(\bu) + p \kappa + \bn\cdot \bb.
\end{equation}
The first term on the right-hand side vanishes due to \eqref{Killing}. For the normal projection of the material derivative we compute with the help of  $\bn\cdot\bu=0$ and \eqref{aux442}
\[
0=\dot{(\bn\cdot\bu)}= \bn\cdot\dot\bu+\bu\cdot\dot\bn=\bn\cdot\dot\bu+\bu^T\bH \bu.
\] 
Substituting this and \eqref{pressure}, \eqref{force1}, \eqref{bN} in \eqref{aux372} gives the \emph{shape equation}
\begin{equation} \label{reaction2}
  -\rho  \bu^T\bH \bu-\frac\rho2\kappa|\bu|^2=  p_0  \kappa+ c_\kappa(\Delta_\Gamma \kappa +\frac12 \kappa^3-2K\kappa)+p^{\rm ext}
\end{equation}
with some $p_0=\mbox{const}$. The $-\rho  \bu^T\bH \bu$ term on the left hand side can be interpreted as a normal component centrifugal force generated by the material flow along a curved trajectory. This interpretation becomes evident when we restrict to axisymmetric shapes below.   Therefore the shape equation \eqref{ShapeEq} represents the balance between the normal component of the centrifugal force, effective membrane tension $( p_0+\frac\rho2|\bu|^2) \kappa$, bending force, and osmotic pressure $p^{\rm ext}$. In turn, the effective membrane tension splits into the `static' $ p_0\kappa$ and `dynamic' $\frac\rho2|\bu|^2\kappa$ ones.

Summarizing, the problem of finding dynamic equilibrium of a fluid--elastic membrane can be formulated as follows: 
For the given density $\rho$, bending rigidity $c_\kappa$, osmotic pressure $p^{\rm ext}$,   and  surface area $A=\mbox{area}(\Gamma)$ find a shape $\Gamma$, tangential flow $\bu$ and  parameter $ p_0$ that solve \eqref{Killing} and \eqref{reaction2}. 
Alternatively, one may ask to find $\Gamma$, $\bu$, $ p_0$, and  $p^{\rm ext}$ such that \eqref{Killing} and \eqref{reaction2} hold with given $\rho$,  $\kappa$, $A=\mbox{area}(\Gamma)$ \emph{and} $V=\mbox{vol}(\Gamma)$.
\smallskip

Any surface of revolution supports a non-zero Killing field. Moreover, it looks plausible that {only} surfaces of revolution support  non-zero Killing fields among closed compact smooth surfaces embedded in $\R^3$; see the discussion following \eqref{Killing}. 
This motivates us to restrict further considerations to such surfaces.
 Without loss of generality, we let $Oz$ to be the axis of symmetry for $\Gamma$. Then tangential $\bu$ satisfying \eqref{Killing} is a field of rigid rotations given by 
\begin{equation}\label{velocity}
\bu(\bx)=w\,\be_z\times\bx,\quad\bx\in\Gamma,\quad 
\end{equation}
with the angular velocity $w\,\be_z$. It holds
\[
|\bu(\bx)| = |w|\,r,\quad\text{with}~r=\mbox{dist}(\bx,Oz). 
\] 
For an axisymmetric surface, the first principle direction is tangential to the generating curve and the second one is the azimuthal direction and coincides with the direction of $\bu_T$. Since the principle directions are given by the eigenvectors of $\bH$, the later observation implies $\bu^T\bH \bu=\kappa_2 |\bu|^2$.
Now \eqref{Killing} yields the  \emph{shape equation for an axisymmetric surface}:
\begin{equation} \label{ShapeEq}
	-\rho\Big(\kappa_2+\frac\kappa2\Big)(w\,r)^2=  p_0  \kappa+c_\kappa(\Delta_\Gamma \kappa +\frac12 \kappa^3-2K\kappa)+p^{\rm ext},
\end{equation}
with some $ p_0=\mbox{const}$. 
Thus, further in the paper we are interested in the following problem:  \emph{Find an axisymmetric $\Gamma$, $ p_0$, and  $p^{\rm ext}$ such that \eqref{ShapeEq} holds with given $\rho$,  $\kappa$, $|w|$, $A=\mbox{\rm area}(\Gamma)$ \emph{and} $V=\mbox{\rm vol}(\Gamma)$.} 
We  remark that instead of prescribing $w$ one may consider the prescribed angular momentum (a conserved quantity). 
In such formulation, $w$ should be treated as unknown. 

\begin{rem}\rm For $\bu=0$ eqs. \eqref{reaction2} and \eqref{ShapeEq} naturally simplifies to \eqref{equl1}, which is the  optimality condition for  constrained minimization of the energy functional \eqref{Willmore} with conserved surface area and enclosed volume. For the general case  of $\bu\neq0$ we don't see how the shape equation  can be related to an energy minimization problem.
\end{rem}
	
\begin{rem}[scaling]
A scaling property well-known for \eqref{equl1} extends to  \eqref{ShapeEq}: If a triple $\{\Gamma,  p_0, p_{\rm ext}\}$ solves   \eqref{ShapeEq}, then  for any $R>0$ the triple $\{R^{-1}\Gamma, R^2 p_0, R^3p_{\rm ext}\}$  solves   \eqref{ShapeEq} with 
$ w\to R^2 w$. %The bending energy in \eqref{Willmore} is invariant with respect to this scaling.
Choosing  a representative solution with $\mbox{area}(\Gamma)=4\pi^2$, it is therefore convenient to parameterize solutions by their reduced volume
	\begin{equation}\label{Vref}
		\widehat V=\widehat V(\Gamma):= 3\mbox{vol}(\Gamma)/(4\pi),\quad  	\widehat V(\Gamma)\in (0,1],
	\end{equation}  
	where $\widehat V=1$  corresponds to the unit sphere, a trivial solution of \eqref{ShapeEq}  for $ w=0$ and $ p_0$, $p^{\rm ext}$ satisfying $2 p_0+ Rp^{\rm ext}=0$. Same scaling argument holds for solutions of \eqref{Killing}, \eqref{reaction2}.
\end{rem}




 

\section{Parametrization of \eqref{ShapeEq} and a numerical solve}\label{sec:num}   %Since we are mainly interested in dynamic equilibrium ($w\neq0$) configurations, the axial symmetry is further assumed. 
An axisymmetric $\Gamma$ can be described by its profile curve 
\[
s\to (r(s), z(s)),
\]
so that $\Gamma$ is generated by rotating the profile curve around the $z$-axis in $\mathbb{R}^3$. Assuming $s$ is the arc-length parameter, one computes~\cite[Section 3C]{kuhnel2015differential} principle curvatures to be $\kappa_1=-r_{ss}z_s+r_{s}z_{ss}$, $\kappa_2=\frac{z_s}r$.
It is convenient to introduce the tilt angle $\psi(s)$ (an angle between the $Or$-axis and tangent vector to the profile curve). Writing geometric quantities in terms of $\psi$, one gets
\[
r_s=\cos\psi,\quad z_s=\sin\psi,\qquad
\kappa_1=\psi_s,\quad \kappa_2=\tfrac{\sin\psi}{r}.
\]
One also computes $\Delta_\Gamma \kappa=\frac1r(r\kappa_s)_s$.
Denote the length of the profile  curve by $L$. Then the  boundary conditions at $s=0$ and $s=L$ are obviously 
$
r(0)=0$, $r(L)=0$, $\psi(0)=0$, $\psi(L)=\pi.
$
The area and volume of the surface  $\Gamma$ can be computed as $2\pi \int_{0}^L r\,ds$ and $\pi \int_{0}^L r^2\sin\psi\,ds$, respectively.  
Now, we can formulate the problem of finding a stationary shape as follows:\\
   Given  an angular velocity $ w\ge0$,  surface area $A>0$ and  volume $V>0$ (such that $V\le 1/(6\pi^2)A^{\frac32}$),  find $L\in\R_+$, ~$\psi(s), r(s) :[0,L]\to\R$, ~ $ p_0,p^{\rm ext}\in\R$  satisfying
   the following system of ODEs, integral and boundary conditions:
   \begin{align}\label{eq1}
   	 -\rho w^2r(\tfrac{1}2r\psi_s+\tfrac32\sin\psi)=& p_0  \kappa+c_\kappa \big(r^{-1}(r\kappa_s)_s +\tfrac12\kappa^3-2K\kappa\big)+p^{\rm ext},\\  &\quad{\small \text{with}~~ \kappa=(\psi_s+\frac{\sin\psi}{r}),~ K=\frac{\psi_s\sin\psi}{r}}, \notag\\ 
   	 r_s=&\cos\psi, \label{eq2}\\
   	 2\pi\int_{0}^{L} r\,ds=& A,\quad
   	 \pi \int_{0}^L r^2\sin\psi\,ds=V\label{eq4}\\
   	 r(0)=&\,0,\quad r(L)=0,\quad\psi(0)=0,\quad\psi(L)=\pi.	\label{eq5}
   \end{align}  

The system \eqref{eq1}--\eqref{eq5} is further discretized using a staggered grid for $\psi$ and $r$ with a uniform  
mesh step $\Delta s=L/N$.
 We prescribed $r$-unknowns to nodes $x_i=i\Delta s$, {\small $i=0,\dots,N$} and $\psi$-unknowns to nodes $\hat x_j=(j-\tfrac12)\Delta s$, {\small $j=0,\dots,N+1$}. Then equations  \eqref{eq1}--\eqref{eq2} are discretized (using standard finite differences)
in the inner $\psi$-nodes, integrals \eqref{eq4} are computed with the help of composite trapezoid and rectangular (using averaging for $r$ unknowns), respectively. After we approximate 
\begin{wrapfigure}{r}{0.45\textwidth}
	\vskip-1ex 		
	\includegraphics[width=0.435\textwidth]{./figs/convergence.png}
	\vskip-2ex 		
	\caption{\label{fig1} Convergence of the numerical solutions for refined meshes.}
	\vskip-1ex 		
\end{wrapfigure}
boundary conditions in \eqref{eq5} by $r(x_0)=r(x_N)=0$, $\phi(\hat x_0)+\phi(\hat x_1)=0$, and
$\phi(\hat x_N)+\phi(\hat x_{N+1})=2\pi$, we obtain a non-linear system of $2N+6$ algebraic equations for $2N+6$ unknowns: $L$, $ p_0$, $p^{\rm ext}$,
$r(x_i)$, {\small $i=0,\dots,N$}, and $\psi(\hat x_j)$, {\small $j=0,\dots,N+1$}.  The system of algebraic equations is then solved by
 a non-linear least-square method with the trust-region-dogleg algorithm for finding search directions realized in the `fsolve()' Matlab\texttrademark  procedure.  The accuracy of numerical method  is verified by computing solutions for a sequence of refined meshes,  $N\in\{40,80,160,320,640\}$, and checking the error  reduction factor. The error is computed as a difference  of $N\in\{40,80,160,320\}$ solutions with the finest grid solution. The method demonstrates the  second order of convergence as shown in Fig.~\ref{fig1} for two examples of shapes (prolate and oblate).  


\section{Stationary shapes}\label{sec:shapes}
To minimize the number of parameters we let 
\[
c_\kappa\in\{0,1\},\quad \rho/2=1, \quad A=4\pi^2.
\]
This can be always ensured by a proper re-scaling of $ w$, $ p_0$, and $p^{\rm ext}$. We then vary $ w$ and $\widehat{V}$ and solve \eqref{ShapeEq} to find $\Gamma$, $ p_0$, $p^{\rm ext}$.

\begin{figure}[h]
\begin{center}
	\includegraphics[width=0.45\textwidth]{./figs/OblateW0.png}
	\includegraphics[width=0.45\textwidth]{./figs/ProlateW0.png}
\end{center}
\vskip-3ex	
	\caption{\label{fig2} Branches of oblate--biconcave and prolate--dumbbell shapes for pure elastic membrane.}
\end{figure}

\subsection*{Case $w=0$, $c_\kappa=1$.}
Setting $w=0$ (pure elasticity, no fluidity) we recover two branches of solutions to  \eqref{equl1} consisting of oblate and prolate shapes, cf.~Fig~\ref{fig2}. To start each branch, we perturb the unit sphere by  the second spherical harmonic as an initial guess for our non-linear solver. The branch of oblate shapes continuous with biconcave discocytes until approximately $\hat V\simeq 0.51$, while the branch of prolate shapes continuous with increasingly elongated dumbbell forms. The recovered steady shapes and corresponding $ p_0$ and $p^{\rm ext}$ are in perfect agreement with results known in the literature~\cite{jenkins1977static,seifert1991shape,seifert1997configurations}.



\subsection*{Case $w=4$, $c_\kappa=1$.}

\begin{figure}[h]
	\begin{center}
		\includegraphics[width=0.45\textwidth]{./figs/Oblate3DW4.png}
		\includegraphics[width=0.45\textwidth]{./figs/OblateW4.png}
	\end{center}
\vskip-3ex
	\caption{\label{fig3}\small  A branch of oblate--biconcave shapes for fluid--elastic membrane with $ w=4$. Left panel visualizes the 3D shape for  $\widehat V=0.51$.}
\end{figure}

\begin{figure}[h]
	\begin{center}
		\begin{minipage}[c]{0.7\textwidth}
			\includegraphics[width=0.32\textwidth]{./figs/dagger3D.png}	\hskip-2ex	
			\includegraphics[width=0.7\textwidth]{./figs/dagger_w4.png}
		\end{minipage}\hskip-3ex
		\begin{minipage}[c]{0.2\textwidth}		\small
			\begin{tabular}{l|cc}	
				$\widehat{V}$& 	$ p_0$&	$p^{\rm ext}$\\ \hline
				0.35&   -174 & 29.0 \\    
				0.45&  -95.7 & 18.9 \\
				0.55&  -74.8 & 17.0  \\
				0.65&  -39.9  & 8.25  \\
				0.75&   -38.6 & 9.11 \\
				0.85&   -42.6 & 11.8 \\
				0.95&   -56.1 & 18.6 \\
			\end{tabular}
		\end{minipage}
	\end{center}
\vskip-3ex
	\caption{\label{fig8}\small The first branch of oblique shapes for $ w=4$.  Left panel visualizes the 3D shape with  $\widehat V=0.66$.}
\end{figure}

\begin{figure}[h]
	\begin{center}
		\hskip-2ex\begin{minipage}[c]{0.7\textwidth}		
			\includegraphics[width=0.34\textwidth]{./figs/star3D_0.666.png}
			\includegraphics[width=0.61\textwidth]{./figs/star_w4.png}
		\end{minipage}\hskip-1ex
		\begin{minipage}[c]{0.2\textwidth}		\small
			\begin{tabular}{l|cc}	
				$\widehat{V}$& 	$ p_0$&	$p^{\rm ext}$\\ \hline
				0.625&  72.2 & -198 \\    
				0.698&  89.7 & -229 \\
				0.771&  102 & -260  \\
				0.844&  115 & -294  \\
				0.917&  130 & -330 \\
				0.99&   10.3 & -57.5 \\
			\end{tabular}
		\end{minipage}
	\end{center}
\vskip-3ex
	\caption{\label{fig9}\small The second branch of oblique shapes for $ w=4$,  Left panel visualizes the 3D shape with  $\widehat V=0.66$.}
\end{figure}


We now let $w=4$ to see how the balance between bending forces and forces
generated by fluid motion affects the equilibrium state. Starting from an oblate perturbation of the unit sphere, we find a branch of oblate ellipsoids which continue with  biconcave shapes for decreasing reduced volume $\widehat {V}$; see Fig.~\ref{fig3}. However, the transition to biconcave forms happens later than for $w=0$. The surface starts self-intersecting for  $\widehat {V}\lesssim0.41$.    

Starting with a prolate perturbation of the sphere, we were not able to find a branch of prolate ellipsoid shapes in the vicinity of the unit sphere. Instead we found two branches of  oblique forms illustrated in Figs.~\ref{fig8} and~\ref{fig9}. We were not able to compute shapes on these branches much further beyond the smallest reduced volumes reported, i.e. $\widehat {V}=0.35$ and $\widehat {V}=0.625$, respectively. 


\begin{figure}[h]
	\begin{center}
		\hskip-1ex\begin{minipage}[c]{0.7\textwidth}		
			\includegraphics[width=0.34\textwidth]{./figs/sandwatch3D_0.52.png}
			\includegraphics[width=0.64\textwidth]{./figs/sandwatch_w4.png}
		\end{minipage}\hskip-2ex
		\begin{minipage}[c]{0.2\textwidth}		\small
			\begin{tabular}{l|cc}	
				$\widehat{V}$& 	$ p_0$&	$p^{\rm ext}$\\ \hline
				0.62&   2.88 & -31.5 \\    
				0.666&  3.25 & -28.8 \\
				0.712&  3.30 & -26.0  \\
				0.758&  4.34 & -27.5  \\
				0.804&  5.73 & -30.5 \\
				0.85&   7.82 & -35.9 \\
			\end{tabular}
		\end{minipage}
	\end{center}
\vskip-3ex
	\caption{\label{fig7}\small  A branch of sand watch shapes for $ w=4$.  Left panel visualizes the 3D shape with  $\widehat V=0.66$.}
\end{figure}

Another branch of solutions was found for reduced volumes $\widehat {V}\in[0.62,0.85]$. The branch consists of sand watch shapes shown in Fig.~\ref{fig7}. For $\widehat {V}<0.62$ the neck of the shape is closing.
The non-linear solver also failed to converge to any solution for reduced volume  larger than  $\widehat {V}\approx 0.85$.  


\begin{figure}[h]
	\begin{center}
\begin{minipage}[c]{0.68\textwidth}		
		\includegraphics[width=0.18\textwidth]{./figs/dumbbell3D_0.48.png}
		\includegraphics[width=0.75\textwidth]{./figs/Dumbbell_w4.png}
\end{minipage}\hskip-5ex
\begin{minipage}[c]{0.2\textwidth}		\small
\begin{tabular}{l|cc}	
	$\widehat{V}$& 	$ p_0$&	$p^{\rm ext}$\\ \hline
	0.40&  21.8 & -115 \\    
	0.44&  18.3 & -89.8 \\
	0.48&  15.7 & -72.9  \\
	0.52&  13.8 & -61.4  \\
	0.56&  12.5 & -53.7 \\
	0.60&  11.4 & -48.3 \\
\end{tabular}
\end{minipage}
	\end{center}
\vskip-3ex
	\caption{\label{fig5}\small  The first branch of dumbbell shapes for $ w=4$. Left panel visualizes the 3D shape with $\widehat V=0.48$.}
\end{figure}

\begin{figure}[h]
	\begin{center}
	\begin{minipage}[c]{0.7\textwidth}
	\includegraphics[width=0.25\textwidth]{./figs/bone3D_0.48.png}
	\includegraphics[width=0.7\textwidth]{./figs/bone_w4.png}
	\end{minipage}\hskip-5ex
\begin{minipage}[c]{0.2\textwidth}		\small
	\begin{tabular}{l|cc}	
		$\widehat{V}$& 	$ p_0$&	$p^{\rm ext}$\\ \hline
		0.40&  12.8 & -57.2 \\    
		0.44&  14.0 & -64.6 \\
		0.48&  16.7 & -78.2  \\
		0.52&  18.2 & -89.5  \\
		0.56&  20.7 & -107 \\
		0.60&  23.8 & -130 \\
	\end{tabular}
\end{minipage}
\vskip-3ex
	\end{center}
	\caption{\label{fig6}\small  The second branch of dumbbell shapes  at $ w=4$.  Left panel visualizes the 3D shape with $\widehat V=0.48$.}
\end{figure}

 
Two branches of dumbbell shapes were found for  $\widehat V\le0.6$; see Figs.~\ref{fig5} and~\ref{fig6}. The first branch somewhat resembles dumbbell shapes for $w=0$; compare shape profiles in Fig.~\ref{fig5} and Fig.~\ref{fig2}. In the second branch, the dumbbell surfaces are distinctly different, featuring flatter concave discs for decreasing  reduced volume values. 
Results suggest that around $\widehat V\simeq0.61$ we may have transition points, where oblique II and sand watch shapes yield two branches of dumbbell shape solutions.  

\subsection*{Larger $w$  and $c_\kappa=0$ cases.}


\begin{figure}[h]
	\begin{center}
		\begin{minipage}[c]{0.45\textwidth}
			\includegraphics[width=\textwidth]{./figs/oblate_w_0,128_.png}
		\end{minipage}
		\begin{minipage}[c]{0.5\textwidth} \small		
			\begin{tabular}{l|cccl}	
				$ w$ & 0 &2 & 4 &~~8  \\ \hline
				$ p_0$ &   4.10 &  0.69 & -10.3  & -57.6  \\
				$p^{\rm ext}$ &  (-0.016 &   -0.017 &   -0.018&   -0.016)$\times10^{3}$ \\
			\end{tabular}
			\vskip1ex
			\begin{tabular}{l|cccl}	
				$ w$ & 16 &32 &64 &~~128  \\ \hline
				$ p_0$ &   -251 &  -1036  & -4183 &   -16775\\
				$p^{\rm ext}$ &  (   0.011 &    0.148 &    0.728 &    ~~3.081)$\times10^{3}$ \\
			\end{tabular}
		\end{minipage}		
	\end{center}
\vskip-3ex
	\caption{\label{fig4}\small  Evolution of shapes with increasing $ w$ for $\widehat V=0.51$.}
\end{figure}

\begin{figure}[h]
	\begin{center}
		\includegraphics[width=0.45\textwidth]{./figs/PureFluidPro.png}
		\includegraphics[width=0.5\textwidth]{./figs/PureFluidicOblVV.png}
	\end{center}
\vskip-3ex
	\caption{\label{fig10}\small Two branches of shapes for a pure fluid membrane, $c_\kappa=0$.}
\end{figure}
We finish the section by taking a look into some shape transformations
for the case where fluid inertia forces dominate over bending forces.
Figure~\ref{fig4} shows a branch of disc-type shapes for reduced volume $\widehat{V}=0.51$. The branch starts with the bi-concave shape solving \eqref{ShapeEq} for $w=0$ and continuous with solutions for a sequence of increasing $w$. We see that for larger $w$ the discs become less concave and converge to a  oblate ellipsoid type shape.

If the  limit  (for $w\to\infty$) smooth surface exists, it solves the shape equation \eqref{ShapeEq} for the ``pure fluid'' case, when one neglects the elastic forces letting $c_\kappa=0$.  Solving \eqref{ShapeEq} for $c_\kappa$ we found two branches of solutions consisting of oblate and prolate shapes; see Fig.~\ref{fig10}. In this limit case we did not find any equilibrium states with concave or saddle shapes (both principle curvatures were always found positive). We also did not find other solutions rather than two branches illustrated in Fig.~\ref{fig10}. Coefficient $ p_0$ and $p^{\rm ext}$ corresponding to the shapes in Fig.~\ref{fig10} are reported in Table~\ref{tab1}.

\begin{table}[h!]
	\footnotesize
	\begin{tabular}{l|ccccccc|ccccccc}	
		& \multicolumn{7}{c}{Oblate shapes} & \multicolumn{7}{c}{Prolate shapes}\\
		$\widehat{V}$& 0.51  &  0.59 &   0.67&   0.75&  0.83&  0.91&  0.99 
		&0.40 &   0.498&   0.597  &  0.695 &0.793  &0.892  &0.99 \\ \hline
		&\multicolumn{14}{c}{$ w=0$, $c_\kappa=1$}\\
		$ p_0$  &  4.11&   4.64&    5.12&    5.55&    5.91&  6.17&  6.17&
		21.1  & 13.6  & 9.46  &  6.93  &  5.15 & 5.08& 5.74\\
		$p^{\rm ext}$ &  -16.1& -15.7&  -15.3&  -14.8&  -14.2&  -13.6&  -12.5&
		-105&  -54.5 & -31.7 & -19.9&  -13.0 & -11.4&  -11.6\\ \hline
		&\multicolumn{14}{c}{$ w=1$, $c_\kappa=0$}\\
		$ p_0$  &  -1.02&   -1.04&   -1.07&   -1.11&   -1.19&   -1.39&   -3.06&
		2.17  &  0.424  & 0.232  &  0.147  &  0.097   & 0.063  &0.038\\
		$p^{\rm ext}$ &  0.19&  0.26&    0.36&   0.50&  0.72&  1.21&  4.70 &
		-5.57  & -1.879  & -1.367 & -1.086 & -0.881 & -0.709   &-0.555 \\ 
	\end{tabular}
	\caption{\label{tab1}\small Surface tension coefficient $ p_0$ and osmotic pressure $p^{\rm ext}$ recovered for branches of oblate and prolate shapes for pure elastic and pure fluid shapes.}
\end{table}




When we were seeking for solutions to \eqref{ShapeEq} we always used initial guess in the non-linear solver which is symmetric with respect to $xy$-plane. Hence  non-symmetric solutions, if exist, are not reported here, although such solution were reported as steady shapes for the pure elastic model for example in \cite{seifert1997configurations,jenkins1977static}.

\section{Conclusions} \label{sec:concl}
Mechanical equilibrium of a fluid inextensible membrane with non-negligible mass is given by steady state solutions of the surface Navier--Stokes equations coupled with an out-of-plain elasticity model.  Assuming the  Boussinesq–Scriven constitutive law for the viscous  stresses we derived three conditions for the membrane equilibrium, i.e. \eqref{cond1}, \eqref{Killing} and \eqref{pressure}, which are independent of the elasticity model.
The second condition implies that only two scenarios are possible: either membrane lateral motions completely cease or the equilibrium shape supporting a stationary flow is the surface with Killing field. %must be axisymmetric and the flow is given by rotation around the axis of symmetry. 
Accounting for a particular elasticity model brings us to the shape equation.  For elasticity models involving energy functional, the shape equation under the first scenario reduces to the optimality condition for the functional with area and volume constraints. For the second scenario, the shape equation  represents a balance between (normal components of) centrifugal, elastic, tension and external forces. For axisymmetric surfaces, the equation can be efficiently parameterized and solved numerically.  Numerical studies with the simplest   Helfrich elasticity model show that    equilibrium shapes depend on lateral motions. In particular, new branches of solution appear. We also found some equilibrium states for pure fluid membrane, which correspond to stationary solutions of the  evolving--surface Navier--Stokes equations with no elastic forces and external forces given by a constant force acting in normal direction (i.e. the constant osmotic pressure). 

\subsection*{Acknowledgments}
The author was supported in  part by the U.S. National Science Foundation under awards DMS-2011444 and DMS-1953535. It is a pleasure
to thank Robert Bryant and Gordon Heier for their help in understanding surfaces with Killing fields.  


\bibliographystyle{acm}
\bibliography{literatur}{}



\appendix
\section{Appendix}
%To compute the shape derivative of $H$, 
Consider a smooth closed  $\Gamma$ embedded in $\mathbb{R}^3$. For a smooth vector field $\bv:\mathcal{O}(\Gamma)\to\mathbb{R}^3$
define 
$
	\Gamma(t)=\{\by\in\R^3~|~\by=\bx(t,\bz),~\bz \in \Gamma \},
$
$t\in[0,\eps)$,
where the trajectories $\bx(t,\bz)$ solve the Cauchy problem
$
		\frac{d\bx}{dt} = \bv(\bx), \bx(0,\bz)=\bz\in\Gamma,
$
and a small $\eps>0$ such that $\Gamma(t)\in\mathcal{O}(\Gamma)$ for all  $t\in[0,\eps)$. Obviously, we have $\Gamma(0)=\Gamma$.  
%Since for the geometric deformations only normal motions matters, we can consider  $\bv=v_N\bn$.
Applying the surface Reynolds transport theorem, one computes
\begin{equation}\label{aux696a}
\left.\frac{dH}{d\Gamma}\right|_{\bv} =\left.\left(\frac{d}{dt}\frac{c_\kappa}2\int_{\Gamma(t)}\kappa^2\, ds\right)\right|_{t=0}
= \frac{c_\kappa}2\int_{\Gamma}\big(\dot{\kappa^2} +\kappa^2\divG\bv \big)\, ds
= \frac{c_\kappa}2\int_{\Gamma}\big(2\kappa\dot{\kappa} +\kappa^2\divG\bv \big)\, ds.
\end{equation}
Let $d=d(t):\mathcal{O}(\Gamma)\to\mathbb{R}$ be a sign distance function for $\Gamma(t)$ which is smooth in the sufficiently small neighborhood  $\mathcal{O}(\Gamma)$. Then  $\bn=\nabla d$, $\kappa=\Div_\Gamma\bn$ are extensions of the normal vector field and mean curvature to  $\mathcal{O}(\Gamma)$. 
We split $\bv$ into tangential and normal component:
\[
\bv=\bv_T+ v_N\bn.
\]
Since $\kappa$ is defined in a neighborhood of $\Gamma$, we can expand 
\begin{equation}\label{aux719}
\dot{\kappa}=\tfrac{\partial\kappa}{\partial t}+\bv\cdot\nabla\kappa= \tfrac{\partial\kappa}{\partial t}+\bv_T\cdot\nabla_\Gamma\kappa+v_N(\bn\cdot\nabla)\kappa\qquad\text{on}~\Gamma.
\end{equation}
Integration by parts along $\Gamma$ proves the identity 
\begin{equation}\label{aux723}
2\int_{\Gamma}\kappa\bv_T\cdot\nabla_\Gamma\kappa\, ds=-\int_{\Gamma}\kappa^2\divG\bv_T\, ds.
\end{equation}
The identity  $\Div_\Gamma\bn=\kappa$ yields 
\begin{equation}\label{aux723b}
\divG\bv=\div\bv_T+\divG(v_N\bn)=\div\bv_T+\bn\cdot\nabla_\Gamma v_N +v_N\kappa=\div\bv_T+v_N\kappa.
\end{equation}
Using \eqref{aux719}, \eqref{aux723} and \eqref{aux723b} in \eqref{aux696a} gives
\begin{equation}\label{aux696}
	\left.\frac{dH}{d\Gamma}\right|_{\bv} 
	= \frac{c_\kappa}2\int_{\Gamma}\big(2\kappa(\tfrac{\partial\kappa}{\partial t}+v_N(\bn\cdot\nabla)\kappa) +\kappa^3 v_N \big)\, ds
\end{equation}
We assume that  the neighborhood $\mathcal{O}(\Gamma)$ is sufficiently small such that the closest point projection  $p:\mathcal{O}(\Gamma)\to\Gamma(t)$, $p(x,t)=x-d\bn$ is well defined. We then have 
\begin{equation}\label{aux696b}
\frac{\partial d}{\partial t}=-v_N^e\quad \text{in}~\mathcal{O}(\Gamma)
\end{equation}
where $v_N^e(x,t)=v_N(p(x,t),t)$. With the help of \eqref{aux696b} and $\kappa=\Div_\Gamma \bn=\Div \bn=\Delta d$, we compute
\begin{equation}\label{aux703}
\begin{split}
\tfrac{\partial\kappa}{\partial t}+v_N(\bn\cdot\nabla)\kappa&
=\Delta\tfrac{\partial d}{\partial t}+v_N\bn\cdot\nabla\Div\bn=-\Delta v_N^e+v_N\bn\cdot\nabla\Div\bn\\
&=-\Delta_\Gamma v_N+v_N\bn\cdot\nabla\Div\bn
\end{split}\qquad \text{on}~\Gamma(t).
\end{equation}
Taking the divergence of the identity $\nabla\bn^2=0$, we get $0=\bn\cdot\Delta\bn+\nabla\bn:\nabla\bn$ implying that 
$-\bn\cdot\Delta\bn=\mbox{tr}((\nabla\bn)^2)=\mbox{tr}(\bH^2)=\kappa_1^2+\kappa_2^2$. We use this  and $\Delta=\nabla\Div-\nabla\times\nabla\times$ to handle the last term in the right-hand side of \eqref{aux703}:
\begin{equation}\label{aux710}
\bn\cdot\nabla\Div\bn= \bn\cdot\Delta\bn+\bn\cdot(\nabla\times\nabla\times\bn)
=-(\kappa_1^2+\kappa_2^2) +\bn\cdot(\nabla\times\nabla\times\bn)=-(\kappa_1^2+\kappa_2^2).
\end{equation}
For the last equality we used $\nabla\times\bn=\nabla\times(\nabla d)=0$.
 Substituting \eqref{aux703}--\eqref{aux710} in \eqref{aux696}
we obtain
\begin{equation*}
	\left.\frac{dH}{d\Gamma}\right|_{\bv}
	= \frac{c_\kappa}2\int_{\Gamma}2\kappa(-\Delta_\Gamma v_N -(\kappa_1^2+\kappa_2^2)v_N) +\kappa^3 v_N \, ds
	= \frac{c_\kappa}2\int_{\Gamma}2\kappa(-\Delta_\Gamma v_N -(\kappa^2 - 2K)v_N) +\kappa^3 v_N \, ds.
\end{equation*}
Integration by parts yields the result in \eqref{bN}.
\end{document}
