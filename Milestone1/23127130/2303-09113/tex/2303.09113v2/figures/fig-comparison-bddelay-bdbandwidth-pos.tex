\begin{figure}[tb]%
    \centering%
    
    \begin{tikzpicture}[]%
        \footnotesize
        \begin{axis}[
                mysimpleresilienceplot01,
                name=plot1,
                xmode=log,
                xlabel={Block production rate $\lambda$},
                legend columns=1
                %
            ]

            %
            %
            %
            %

            \addlegendimage{empty legend}
            \addlegendentry{%
                \tikz[x=0.75em,y=0.75em,baseline=-0.35em]{%
                    \draw [black,dashed,thick] (0,0) -- (2,0);
                }%
                \hspace{0.3em} Private attack%
            };

            \addlegendimage{empty legend}
            \addlegendentry{%
                \tikz[x=0.75em,y=0.75em]{%
                    \draw [myparula71,thick] (0,0) -- (2,0);
                    \draw [myparula72,thick] (0,-0.5) -- (2,-0.5);
                    \draw [myparula73,thick] (0,-1.0) -- (2,-1.0);
                }%
                \hspace{0.3em} \PoSTeaserattack (\cref{sec:pos-teaser-attack}): decreasing success prob. $\epsilon$%
            };

            \addlegendimage{empty legend}
            \addlegendentry{%
                \tikz[x=0.75em,y=0.75em]{%
                    \draw [myparula51,thick] (0,0) -- (2,0);
                    \draw [myparula52,thick] (0,-0.5) -- (2,-0.5);
                    \draw [myparula53,thick] (0,-1.0) -- (2,-1.0);
                }%
                \hspace{0.3em} PoS NC security \cite{bwlimitedposlc}: decreasing security error prob. $\epsilon$%
            };

            \addlegendimage{empty legend}
            \addlegendentry{%
                \tikz[x=0.75em,y=0.75em,baseline=-0.35em]{%
                    \draw [myParula01Blue,thick] (0,0) -- (2,0);
                }%
                \hspace{0.3em} \SaPoS security (this work)%
            };

            \addplot [draw=none,name path=xaxis,domain={1e-8:1e8}] {0};
            \addplot [draw=none,name path=xaxisplus1,domain={1e-8:1e8}] {1};


            %
            
            \addplot [myparula71,thick,no marks,name path=posattack10] table [x=lbyC,y=beta] {figures/fig-comparison-bddelay-bdbandwidth-bdbandwidth-newresult-posattack-logprob10.txt};
            \addplot [myparula72,thick,no marks,name path=posattack100] table [x=lbyC,y=beta] {figures/fig-comparison-bddelay-bdbandwidth-bdbandwidth-newresult-posattack-logprob100.txt};
            \addplot [myparula73,thick,no marks,name path=posattack1000] table [x=lbyC,y=beta] {figures/fig-comparison-bddelay-bdbandwidth-bdbandwidth-newresult-posattack-logprob1000.txt};

            \addplot [black,dashed,no marks,name path=resiliencebddelay] table [x=relblockfrequency,y=beta] {figures/fig-comparison-bddelay-bdbandwidth-bddelay.txt};
            \label{leg:comparison-bddelay-bdbandwidth-privateattack-pos}

            \addplot [myparula71,fill opacity=0.2] fill between [of=posattack10 and xaxisplus1];
            \addplot [myparula71,fill opacity=0.2] fill between [of=posattack100 and xaxisplus1];
            \addplot [myparula71,fill opacity=0.2] fill between [of=posattack1000 and xaxisplus1];

            \addplot [jnSUDigitalRedDark,pattern=north east lines,pattern color=jnSUDigitalRedDark] fill between [of=resiliencebddelay and xaxisplus1];

            \addplot [myParula01Blue,no marks,name path=resiliencebdbandwidth] table [x=lbyC,y=beta] {figures/fig-comparison-bddelay-bdbandwidth-bdbandwidth-newresult.txt};

            \addplot [myParula01Blue,fill opacity=0.2] fill between [of=resiliencebdbandwidth and xaxis];
            
            \addplot [myparula51,thick,no marks,name path=resilience10] table [x=lbyC,y=beta] {figures/fig-comparison-bddelay-bdbandwidth-bdbandwidth-oldresult-kappa10.txt};
            \addplot [myparula52,thick,no marks,name path=resilience100] table [x=lbyC,y=beta] {figures/fig-comparison-bddelay-bdbandwidth-bdbandwidth-oldresult-kappa100.txt};
            \addplot [myparula53,thick,no marks,name path=resilience1000] table [x=lbyC,y=beta] {figures/fig-comparison-bddelay-bdbandwidth-bdbandwidth-oldresult-kappa1000.txt};

            \addplot [myparula51,fill opacity=0.2] fill between [of=resilience10 and xaxis];
            \addplot [myparula51,fill opacity=0.2] fill between [of=resilience100 and xaxis];
            \addplot [myparula51,fill opacity=0.2] fill between [of=resilience1000 and xaxis];

            
            %
            %
            %

            %
            %
            %
            %
            %
            %

            %
            %

            %
            %

            %
            %
            %
        \end{axis}
    \end{tikzpicture}%
    \vspace{-0.5em}%
    \caption[]{%
        Regions of fraction $\beta$ of adversarial nodes and
        block production rate $\lambda$
        with security proofs \cite{bwlimitedposlc} 
        (\tikz[x=0.75em,y=0.75em]{ \draw [draw=myParula05Green,thick,fill=myParula05Green,fill opacity=0.3] (0,0) rectangle (1,1); })
        and attacks
        (\tikz[x=0.75em,y=0.75em]{ \draw [draw=myParula07Red,thick,fill=myParula07Red,fill opacity=0.3] (0,0) rectangle (1,1); })
        for PoS Nakamoto consensus under a fixed processing capacity of $C=1$ block per second.
        The private attack succeeds with overwhelming probability for parameters in \tikz[x=0.75em,y=0.75em]{ \draw [draw=none,fill opacity=1,pattern=north east lines,pattern color=jnSUDigitalRedDark] (0,0) rectangle (1,1); } while it fails with overwhelming probability otherwise.
        In contrary, our \PoSteaserattack continues to succeed at a lower adversarial fraction as the required attack success probability is decreased (in order, \tikz[x=0.75em,y=0.75em,baseline=-0.35em]{%
        \draw [myparula71,thick] (0,0) -- (0.4cm,0cm);
        }, \tikz[x=0.75em,y=0.75em,baseline=-0.35em]{%
        \draw [myparula72,thick] (0,0) -- (0.4cm,0cm);
        }, \tikz[x=0.75em,y=0.75em,baseline=-0.35em]{%
        \draw [myparula73,thick] (0,0) -- (0.4cm,0cm);
        }), as analyzed in \cref{sec:pos-attack}.
        This explains why the security region (against all attacks) proven in \cite{bwlimitedposlc} shrinks as the desired security error probability decreases (in order, \tikz[x=0.75em,y=0.75em,baseline=-0.35em]{%
        \draw [myparula51,thick] (0,0) -- (0.4cm,0cm);
        }, \tikz[x=0.75em,y=0.75em,baseline=-0.35em]{%
        \draw [myparula52,thick] (0,0) -- (0.4cm,0cm);
        }, \tikz[x=0.75em,y=0.75em,baseline=-0.35em]{%
        \draw [myparula53,thick] (0,0) -- (0.4cm,0cm);
        }).
        Our new protocol \SaPoS, by using \equivocationremoval, achieves a security region (\tikz[x=0.75em,y=0.75em]{ \draw [draw=myParula01Blue,thick,fill=myParula01Blue,fill opacity=0.3] (0,0) rectangle (1,1); }) that is
        independent of
        the security error probability.
    }%
    \label{fig:comparison-bddelay-bdbandwidth-pos}%
\end{figure}%