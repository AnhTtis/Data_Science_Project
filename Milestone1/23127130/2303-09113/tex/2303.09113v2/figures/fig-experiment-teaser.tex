\begin{figure}[tb]%
    \centering%
    \begin{tikzpicture}[]
        \footnotesize
        \begin{axis}[
                mysimpleplot,
                %
                xlabel={Bandwidth: $\bwtime$ [blocks per second]},
                ylabel={Honest chain growth rate\\ ($\blkratetimeGrowth/\blkratetimeHon$)},
                legend columns=3,
                xmin=0, xmax=2,
                ymin=0, ymax=0.7,
                height=0.5\linewidth,
                width=\linewidth,
                yticklabel style={
                        /pgf/number format/fixed,
                        /pgf/number format/precision=2
                },
                scaled y ticks=false,
                %
                %
                legend columns=2,
                legend style={
                        at={(0.5,1)},
                        %
                        xshift=-1em,
                        anchor=south,
                        draw=none,
                        /tikz/every even column/.append style={
                                column sep=0.3em
                            },
                        cells={
                                align=left
                            }
                    },
            ]

            \addplot [myparula11, %
                    only marks, mark size=1.5pt] table [x=bandwidth,y=chain_growth] {figures/fig-experiment-teaser-noattacker-data.txt};
            \addlegendentry{No/private attack};
            \label{plt:experiment-teaser-noattacker};

            \addplot [myparula22,thin,solid,mark size=1.5pt] table [x=inverse_delay,y=chain_growth] {figures/fig-experiment-growth-delay-data.txt};
            \addlegendentry{No/private attack under $\Delta = 1/\bwtime$};
            \label{plt:experiment-growth-delay};

            \addplot [myparula73, mark size=1.5pt,%
            only marks] table [x=bandwidth,y=chain_growth] {figures/fig-experiment-teaser-activeattacker-data.txt};
            \addlegendentry{\Teaserattack (PoW+PoS)};
            \label{plt:experiment-teaser-attacker};

            \addplot [myparula54, mark size=1.5pt,%
            only marks] table [x=bandwidth,y=chain_growth] {figures/fig-experiment-teaserequiv-equivteasingattacker-data.txt};
            \addlegendentry{\PoSTeaserattack (PoS)};
            \label{plt:experiment-teaser-equivattacker};


        \end{axis}
    \end{tikzpicture}%
    \vspace{-0.5em}%
    \caption{%
    Simulation results: The rate of chain growth relative to honest block production rate, when nodes prioritize downloads towards the longest header chain, for various bandwidths. When the attacker does not release any blocks (no attack or private attack), we already see $\blkratetimeGrowth < \blkratetimeHon$ due to natural congestion (\ref{plt:experiment-teaser-noattacker}).
    As described in \cref{sec:private-attack}, the honest chain growth rate under the private attack is approximately the same whether simulated on a network with finite processing capacity $\bwtime$ (\ref{plt:experiment-teaser-noattacker}), or on an idealized network with bounded delay $\Delta = 1/\bwtime$ (\ref{plt:experiment-growth-delay}).
    %
    With a \teaserattack, processing is slowed roughly by a factor of $2$, which lowers the growth rate of the chain further (\ref{plt:experiment-teaser-attacker}).
    %
    This lowers security compared to a private attack, \cf \cref{fig:comparison-bddelay-bdbandwidth}.
    %
    Due to equivocations in PoS,
    the honest chain grinds to a halt under the \PoSteaserattack
    (\ref{plt:experiment-teaser-equivattacker}),
    implying vulnerability even to adversaries with close to zero stake.
    }%
    
    \label{fig:experiment-teaser}%
\end{figure}%
