\begin{figure}[tb]%
    \centering%
    \begin{tikzpicture}[]%
        \begin{axis}[
                mybandwidthplot01,
                name=plot1,
                ymode=log,
                ymin=1e-3,ymax=1e5,
                ytick={1e-3,1e-2,1e-1,1e0,1e1,1e2,1e3,1e4,1e5},
                legend columns=2,
                %
                minor tick style={draw=none},
            ]
            
            \addlegendimage{area legend,myparula21,fill=myParula02Orange,thin,draw=none}
            \addlegendentry{Bitcoin (PoW)};
            
            \addlegendimage{area legend,myparula11,fill=myParula01Blue,thin,draw=none}
            \addlegendentry{Cardano, with \equivocationremoval (PoS)};
            
            \addlegendimage{gray,mark=none}
            \addlegendentry{Our analysis};
            
            \addlegendimage{gray,densely dotted,mark=none}
            \addlegendentry{Earlier bounded delay analysis};

            \addplot [myparula21,no marks,mark=none,name path=bandwidth] table [x=beta,y=C(Mbps)] {figures/fig-bitcoin-resilience-bandwidth.txt};
            %
            \label{leg:bitcoin-cardano-resilience-bandwidth-pow-bw};

            \addplot [myparula11,no marks,mark=none,name path=bandwidth] table [x=beta,y=C(Mbps)] {figures/fig-cardano-resilience-bandwidth.txt};
            %
            \label{leg:bitcoin-cardano-resilience-bandwidth-pos-bw};

            \addplot [myparula21,densely dotted,no marks,mark=none,name path=bandwidth] table [x=beta,y=C(Mbps)] {figures/fig-bitcoin-resilience-delay.txt};
            %
            \label{leg:bitcoin-cardano-resilience-bandwidth-pow-bd};

            \addplot [myparula11,densely dotted,no marks,mark=none,name path=bandwidth] table [x=beta,y=C(Mbps)] {figures/fig-cardano-resilience-delay.txt};
            %
            \label{leg:bitcoin-cardano-resilience-bandwidth-pos-bd};
            
            \addplot [myparula23,draw=none,line width=0,only marks] coordinates { (0.4831, 0.39402665088757344) };
            \label{leg:bitcoin-cardano-resilience-bandwidth-mark};
            
        \end{axis}
    \end{tikzpicture}%
    \vspace{-0.5em}%
    \caption[]{%
        Calculation based on \cref{thm:safety-and-liveness-pow,thm:safety-and-liveness-pos} of the bandwidth per node that is sufficient to ensure security of NC with the parametrizations used by two major blockchains: Bitcoin (PoW, $\blkratetime = 1/600\;\mathrm{blocks/s}$, max. block size $4\;\mathrm{MB}$, \ref{leg:bitcoin-cardano-resilience-bandwidth-pow-bw}), and Cardano (PoS, $\blkratetime=1/20\;\mathrm{blocks/s}$, max. block size $88\;\mathrm{KB}$, \ref{leg:bitcoin-cardano-resilience-bandwidth-pos-bw}). Dotted lines show the corresponding predictions from the bounded delay analysis, which holds for the private attack only (\ref{leg:bitcoin-cardano-resilience-bandwidth-pow-bd}, \ref{leg:bitcoin-cardano-resilience-bandwidth-pos-bd}). This suggests that while the commonly recommended $0.4\;\mathrm{Mbps}$~\cite{bitcoin_requirements} is enough to secure Bitcoin against a 48\% private attack adversary (\ref{leg:bitcoin-cardano-resilience-bandwidth-mark}), higher bandwidth may be required to defend against all attacks.%
    }%
    \label{fig:bitcoin-cardano-resilience-bandwidth}%
\end{figure}%