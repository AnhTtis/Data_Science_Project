%
\section{Proof Details}
\label{sec:appendix-full-version-proofs}

\subsection{Proof Details for \Cref{sec:analysis-definitions}}
\label{sec:appendix-full-version-proofs-definitions}

\begin{proof}[Proof of \cref{prop:X_i-is-iid}]
    First, for any $k$,
    \begin{IEEEeqnarray}{rCl}
        \Prob{\Gat{k} = 1} &=& \Prob{ \predGood{t_k} \mid \lnot \predEmpty{t_k} } \\
        &=& \frac{\Prob{\predGood{t_k}}}{\Prob{\predEmpty{t_k}}}
        %
        %
        = \frac{(1-\beta)\blkrateslot e^{-\rho(\goodsep+1)}}{1-e^{-\blkrateslot}}. \IEEEeqnarraynumspace
    \end{IEEEeqnarray}
    Take an \iid random process $\{T_k\}$ with $\Prob{T_k = t} = (1-\probEmpty)\probEmpty^t$ for $t \geq 0$ where $\probEmpty = \Prob{\Hat{t}+\Aat{t}=0}$.
    The random variables $\{T_k\}$ describe the inter-arrival times between non-empty slots.
    Take another \iid random process $\{\Gat{k}'\}$, independent of $\{T_k\}$, such that $\Gat{k}' = 1$ with probability $\Prob{\Hat{t} = 1 \land \Aat{t} = 0 \mid \Hat{t}+\Aat{t}>0}$ and $\Gat{k}' = 0$ otherwise.
    The random process $\{\Gat{k}\}$ can be equivalently defined as $G_k = 1$ iff $G_k' = 1$ and $T_k \geq \goodsep$.
    
    The independence of the random variables $\{\Gat{k}\}$ then follows from the independence of the random variables $\{(T_k, \Gat{k}')\}$.
\end{proof}



\subsection{Proof Details for \Cref{sec:analysis-details-cps-stabilize}}
\label{sec:appendix-full-version-proofs-cps-stabilize}

\begin{proof}[Proof of \cref{lem:cps-stabilize}]
    Note that $\dC_p(t)$ is a valid chain at \timeslot $t$ and $\len{\dC_p(t)} = L_p(t) \geq L_{\min}(t)$. Therefore, it suffices to show the first claim of the lemma.
    
    For contradiction, let $s \geq t_k + \goodsep$ be the first \timeslot in which 
    there is a valid header chain $\Chain'$ such that 
    $\len{\Chain'} \geq L_{\min}(s)$ and $b^* \not\in \Chain'$.
    %
    
    Let $b'$ be the block with maximum height on the chain $\Chain'$, such that $b'$ was produced in a \timeslot $t_i$ with $D_i = 1$.
    For $\Chain'$ to be a valid chain at \timeslot $s$, we need $t_i \leq s$.
    Since the block $b'$ is produced by an honest node, $b'$ extends $\dC_q(t_i-1)$ for some honest node $q$.
    Therefore, $\dC_q(t_i-1)$ is a prefix of $\Chain'$.
    This means that $b^* \not\in \dC_q(t_i-1)$.
    Moreover, $\len{\dC_q(t_i-1)} = L_q(t_i-1) \geq L_{\min}(t_i-1)$.
    If $i > k$, then $t_i-1 \geq t_k + \goodsep$ (since $D_k = 1$) and $t_i - 1 < s$ (shown above). 
    This is a contradiction because we assumed that $s$ is the first \timeslot such that $s \geq t_k + \goodsep$ and 
    %
    $b^* \notin \Chain'$ and $\len{\Chain'} \geq L_{\min}(s)$ for some valid chain $\Chain'$.
    Since $b^*$ is the only block produced in slot $t_k$, $i=k$ is also not possible.
    We conclude that $i < k$.
    
    Since $D_i = 1$ and $b'$ is produced in \timeslot $t_i$,
    \begin{IEEEeqnarray}{C}
    \label{eq:block-i-download}
        L_{\min}(t_i + \goodsep) \geq \len{b'}.
    \end{IEEEeqnarray}
    %
    By assumption,
    \begin{IEEEeqnarray}{C}
    \label{eq:block-j-switch}
        \len{\Chain'} \geq L_{\min}(s).
    \end{IEEEeqnarray}
    
    Let $t_j$ be the last non-\sltempty \timeslot such that $t_j \leq s$. Note that $j \geq k > i$. 
    We must consider two cases: 
    %
    
    \begin{enumerate}
    \item Case 1: $s \geq t_j + \goodsep$ or $\Dat{j}=0$.
    If $\Dat{j}=0$, we don't have to worry about whether the block from slot $t_j$ was downloaded by all honest nodes.
    If $\Dat{j} = 1$ but $s \geq t_j + \goodsep$, then we know that all honest nodes have downloaded the block from slot $t_j$ before the end of \timeslot $s$. That is,
    \begin{IEEEeqnarray}{rCl}
        L_{\min}(s) 
        &\geq& L_{\min}(t_j + \goodsep)  \IEEEeqnarraynumspace \\
        &\geq& L_{\min}(t_{i+1}-1) + \Din{i}{j} \quad \text{(from \cref{prop:chain-growth-interval})}  \IEEEeqnarraynumspace \\
        \label{eq:chain-growth-case1}
        &\geq& L_{\min}(t_{i} + \goodsep) + \Din{i}{j}. \IEEEeqnarraynumspace 
    \end{IEEEeqnarray}
    By definition of $b'$, all blocks in $\Chain'$ appearing after $b'$ correspond to \ydowns. These blocks must be from distinct \iindices greater than $i$ but at most $j$. So,
    \begin{IEEEeqnarray}{C}
    \label{eq:adv-chain-case1}
        \len{\Chain'} \leq \len{b'} + \Nin{i}{j}.
    \end{IEEEeqnarray}
    From \eqref{block-i-download,block-j-switch,chain-growth-case1,adv-chain-case1}, we derive
    \begin{IEEEeqnarray}{rCl}
    \label{eq:pivot-contra-case1}
        \Din{i}{j} \leq \Nin{i}{j} \implies \Yin{i}{j} \leq 0 \implies \Yin{0}{i} < \Yin{0}{j} \IEEEeqnarraynumspace
    \end{IEEEeqnarray}
    where $i < k \leq j$.
    
    \item Case 2: $t_j \leq s < t_j + \goodsep$ and $\Dat{j} = 1$.
    In this case, the block from slot $t_j$ may not have enough time to be downloaded by all honest nodes before the end of slot $s$.
    However, for any $l < j$ such that $\Dat{l} = 1$, $t_l + \goodsep < t_j \leq s$, so there is enough time to download the block from \timeslot $t_l$.
    Let $l \in\intvl{i}{j-1}$ be the greatest index such that $\Dat{l} = 1$. Then, $t_j > t_l + \goodsep$, and $\Din{i}{l} = \Din{i}{j-1}$.
    \begin{IEEEeqnarray}{rCl}
        \label{eq:chain-growth-case2}
        L_{\min}(s) 
        &\geq& L_{\min}(t_j) \\
        &\geq& L_{\min}(t_l + \goodsep) \\
        &\geq& L_{\min}(t_{i+1} - 1) + \Din{i}{l} \quad \text{(from \cref{prop:chain-growth-interval})} \IEEEeqnarraynumspace \\
        &\geq& L_{\min}(t_{i} + \goodsep) + \Din{i}{j-1}.
    \end{IEEEeqnarray}
    Note that since $\Dat{j}=1$, $\Nin{i}{j} = \Nin{i}{j-1}$. Therefore, as in the previous case,
    \begin{IEEEeqnarray}{C}
    \label{eq:adv-chain-case2}
        \len{\Chain'} \leq \len{b'} + \Nin{i}{j-1}.
    \end{IEEEeqnarray}
    From \eqref{block-i-download,block-j-switch,chain-growth-case2,adv-chain-case2},
    \begin{IEEEeqnarray}{rCl}
    \label{eq:pivot-contra-case2}
        \Din{i}{j-1} \leq \Nin{i}{j-1} &\implies& \Yin{i}{j-1} \leq 0 \nonumber \\ 
        &\implies& \Yin{0}{i} < \Yin{0}{j-1}. \IEEEeqnarraynumspace
    \end{IEEEeqnarray}
    Note that since we assumed $s \geq t_k + \goodsep$ and $s < t_j + \goodsep$, we know that $j > k$. Therefore, $i < k \leq j-1$.
    \end{enumerate}
    %
    In either case, \eqref{pivot-contra-case1} or \eqref{pivot-contra-case2} contradict the assumption $\predCP{k}$ (\cref{def:cp}).
    %
\end{proof}





\subsection{Proof Details for \Cref{sec:analysis-details-many-pps}}
\label{sec:appendix-full-version-proofs-many-pps}


\begin{proposition}[Hoeffding's inequality~{\cite{doi:10.1080/01621459.1963.10500830} \cite[Thm.~4]{duchi-hoeffding}}]
    \label{prop:hoeffding}
    Let $Z_1, ..., Z_n$ be independent bounded random variables with
    $\forall i: Z_i \in [a,b]$, where $-\infty < a \leq b < \infty$.
    Then, $\forall t\geq0$:
    \begin{IEEEeqnarray}{rCl}
        \Prob{\left(\sum_{i=1}^n Z_i\right) \geq \Exp{\sum_{i=1}^n Z_i} + t n}
        &\leq&
        \exp\left(\frac{-2 n t^2}{(b-a)^2} \right) \IEEEeqnarraynumspace
        \\
        \Prob{\left(\sum_{i=1}^n Z_i\right) \leq \Exp{\sum_{i=1}^n Z_i} - t n}
        &\leq&
        \exp\left(\frac{-2 n t^2}{(b-a)^2} \right) \IEEEeqnarraynumspace
    \end{IEEEeqnarray}
\end{proposition}




\begin{proof}[Proof of \cref{prop:ppivot-randomwalk}]
    \Eqref{pivot-conditions-equivalence-randomwalks}
    characterizes $\predPP{k}$
    as the intersection of three independent events:
    \begin{IEEEeqnarray}{rCl}
        \Event_1
        &\triangleq&
        \{ \Xat{k} = 1 \}
        \\
        \Event_2
        &\triangleq&
        \{ \forall\ell\colon \Xin{k}{k+\ell} \geq 0 \}
        \\
        \Event_3
        &\triangleq&
        \{ \forall\ell\colon \Xin{k-1-\ell}{k-1} \geq 0 \}
    \end{IEEEeqnarray}
    Their probabilities are easily calculated~\cite{stackexchange-math-rwreturnto0}:
    \begin{IEEEeqnarray}{C}
        \Prob{\Event_1}
        = \probGood
        \qquad
        \Prob{\Event_2} = \Prob{\Event_3}
        = (2\probGood - 1) / \probGood
        \IEEEeqnarraynumspace
    \end{IEEEeqnarray}
\end{proof}






\begin{proof}
    Let
    $\Event \triangleq \{\forall \intvl{i}{j} \intvlgeq K_1\colon \Xin{i}{j} > 0\}$.
    From \cref{prop:lower-tailbound-X} with $\delta=1$,
    and a union bound over all intervals
    ($\leq \Khorizon^2$ many),
    we get
    \begin{IEEEeqnarray}{C}
        \Prob{\lnot\Event} \leq \Khorizon^2 \exp(-\alphaLowerTailX K_1).
    \end{IEEEeqnarray}

    For any given index $k$, we can
    partition
    the intervals of
    \eqref{pivot-conditions-equivalence-intervals}
    into `long'
    %
    and `short'
    %
    intervals (length at least vs.\ less than $K_1$):
    \begin{IEEEeqnarray}{rCl}
        \Event_k
        &\triangleq&
        \{ \predPP{k} \}
        = \Event_k^{\mathrm{L}} \land \Event_k^{\mathrm{S}}
        \\
        \Event_k^{\mathrm{L}}
        &\triangleq&
        \{\forall k \in \intvl{i}{j} \intvlgeq K_1\colon \Xin{i}{j} > 0\}
        \IEEEeqnarraynumspace
        \\
        \Event_k^{\mathrm{S}}
        &\triangleq&
        \{\forall k \in \intvl{i}{j} \intvll K_1\colon \Xin{i}{j} > 0\}.
    \end{IEEEeqnarray}
    Note that $\Event_k^{\mathrm{L}} \supseteq \Event$.
    Thus, for any two given \iindices $k_1, k_2$,
    if $k_1, k_2$ are `far apart',
    \ie,
    if $\abs{k_1 - k_2} \geq 2 K_1$,
    then
    $\Event_{k_1}$ and $\Event_{k_2}$ are conditionally independent
    given $\Event$
    (since $\Event_{k_1}^{\mathrm{S}}$ and $\Event_{k_2}^{\mathrm{S}}$ are).

    We decompose $I^* \triangleq \intvl{i}{j} = \intvl{i}{i + 2 K_1 K_2} = \bigcup_{\ell=1}^{2K_1} I_{\ell}$:
    \begin{IEEEeqnarray}{rCl}
        \forall\ell\in\{1,...,2K_1\}\colon\
        I_{\ell}
        &\triangleq&
        \{ i+0\cdot 2K_1+\ell, ...
        \nonumber\\
        && \quad{} ..., i+(K_2-1)\cdot 2K_1+\ell \}.
        \IEEEeqnarraynumspace
    \end{IEEEeqnarray}
    We define corresponding events, $\forall\ell\in\{1,...,2K_1\}$:
    \begin{IEEEeqnarray}{rCl}
        \Event^*
        &\triangleq&
        \left\{ \Pat{I^*} \leq (1-\delta) \probPP 2 K_1 K_2 \right\}
        \\
        \Event_{\ell}
        &\triangleq&
        \left\{ \Pat{I_{\ell}} \leq (1-\delta) \probPP K_2 \right\}.
    \end{IEEEeqnarray}
    Clearly, $\Event^* \subseteq \bigcup_{\ell=1}^{2 K_1} \Event_\ell$.
    Thus, by a union bound,
    \begin{IEEEeqnarray}{rCl}
        \Prob{ \Event^* \cond \Event }
        &\leq&
        \sum_{\ell=1}^{2 K_1} \Prob{ \Event_\ell \cond \Event }.
        \IEEEeqnarraynumspace
    \end{IEEEeqnarray}
    Furthermore, $\forall\ell\in\{1,...,2K_1\}$,
    and with $\mu_\ell \triangleq \Exp{ \Pat{I_{\ell}} \cond \Event}$:
    \begin{IEEEeqnarray}{rCl}
        \IEEEeqnarraymulticol{3}{l}{
            \Prob{ \Event_\ell \cond \Event }
            =
            \Prob{ \Pat{I_{\ell}} \leq (1-\delta) \probPP K_2 \cond \Event }
        }
        \IEEEeqnarraynumspace
        \\\quad
        &\leqA&
        \Prob{ \Pat{I_{\ell}} \leq (1-\delta) \mu_\ell \cond \Event }
        \IEEEeqnarraynumspace
        \\
        &\leqB&
        \exp(-2 \delta^2 \mu_\ell^2 / K_2)
        \leqC
        \exp(-2 \probPP^2 \delta^2 K_2),
        \IEEEeqnarraynumspace
        %
    \end{IEEEeqnarray}
    where
    (a) and (c)~use 
    \begin{IEEEeqnarray}{rCl}
        \mu_\ell = K_2 \Exp{\Ind{\predPP{k}} \cond \Event} &\geq& K_2 \Exp{\Ind{\predPP{k}}} \nonumber \\ &\geq& K_2 \probPP \IEEEeqnarraynumspace
    \end{IEEEeqnarray}
    (\cref{prop:ppivot-randomwalk}),
    %
    and
    (b)~uses that
    $\{\predPP{k_1}\}$ and
    $\{\predPP{k_2}\}$
    are conditionally independent given $\Event$
    for $k_1, k_2 \in I_\ell$,
    and
    Hoeffding's inequality (\cref{prop:hoeffding}).
    %

    Thus, we complete the proof by observing, that with $\alphaLowerTailPP = 2 \probPP^2$,
    \begin{IEEEeqnarray}{rCl}
        \Prob{ \Event^* }
        &=&
        \Prob{ \Event^* \cap \Event } + \Prob{ \Event^* \cap \lnot\Event }
        \\
        &\leq&
        \Prob{ \Event^* \cond \Event } + \Prob{ \lnot\Event }
        \\
        &\leq&
        2 K_1 \exp(-\alphaLowerTailPP \delta^2 K_2)
        + \Khorizon^2 \exp(-\alphaLowerTailX K_1).
        \IEEEeqnarraynumspace
    \end{IEEEeqnarray}
\end{proof}





\subsection{Proof Details for \cref{sec:analysis-details-many-pps-one-cps}}
\label{sec:appendix-full-version-proofs-many-pps-one-cps}

\begin{proof}[Proof of \cref{lem:one-cp-induction-full}]
    This will be proved through induction.
    For the base case ($m=0$), \cref{lem:one-cp-induction-base} shows 
    %
    that $\exists k_1^* \in \intvl{0}{\Kcp} \colon \predCP{k_1^*}$.
    
    For $m \geq 1$, assume that $\exists k_{m-1}^* \in \intvl{(m-1)\Kcp}{m\Kcp}$ such that  $\predCP{k_{m-1}^*}$.
    Now we want to show that $\exists k_{m}^* \in \intvl{m\Kcp}{(m+1)\Kcp}$ such that  $\predCP{k_{m}^*}$.
    %
    %
    Suppose for contradiction that there is no \sltcp in $\intvl{m\Kcp}{(m+1)\Kcp}$.
    %
    %
    %
    %
    As in the proof of \cref{lem:one-cp-induction-base}, 
    there is a set of intervals $\intvlset$ such that:
    \begin{IEEEeqnarray}{Cr}
        \label{eq:induction-full-intervals-cover-ppivots}
        \bigcup_{I \in \intvlset} I \supseteq \left\{ k \in \intvl{m\Kcp}{(m+1)\Kcp} \colon \predPP{k} \right\} & \\
        \label{eq:induction-full-intervals-y-condition}
        \forall I \in \intvlset \colon \Yat{I} \leq 0. &
        %
    \end{IEEEeqnarray}
    Without loss of generality, each interval $I \in \intvlset$ contains at least one \sltpp.
    %
    Then if $\intvl{i}{j} \in \intvlset$, $i < (m+1)\Kcp$ and $j > m\Kcp$.
    
    First, consider the large intervals with $|I| \geq \Kcp$.
    Consider \iindices $k \in I$ for which $\Gat{k}=1$ (\sltgood) but $\Dat{k}=0$ (\ydown).
    %
    %
    %
    From \cref{prop:download-or-spend-budget},
    for each such \iindex $k$, all honest nodes download $\goodsepbw$ blocks that are produced
    %
    in the interval $\intvl{k_{m-1}^*}{k}$.
    %
    %
    %
    %
    %
    %
    The number of \iindices $k \in I$ with  $\Gat{k} = 1$ and $\Dat{k} = 0$ is exactly $\Gat{I} - \Dat{I}$.
    For each such index, there must exist $\goodsepbw$ distinct blocks from distinct \BPOs
    %
    that are downloaded by honest nodes.
    Therefore if $I = \intvl{i}{j}$,
    \begin{IEEEeqnarray}{rClr}
        \Qin{k_{m-1}^*}{j} &\geq& \goodsepbw \left( \Gin{i}{j} - \Din{i}{j} \right) & \IEEEeqnarraynumspace \\
        &\geq& \frac{\goodsepbw}{2} \left( \Gin{i}{j} - \Bin{i}{j} \right) & \quad \text{(from \cref{prop:not-cp-interval-properties}).} \IEEEeqnarraynumspace
    \end{IEEEeqnarray}
    But $k_{m-1}^* > (m-1)\Kcp$ and $i < (m+1)\Kcp$. Therefore $\Qin{k_{m-1}^*}{j} \leq \Qin{i-2\Kcp}{j}$.
    Then we have a contradiction to \eqref{cp-induction-full-margin-condition}.
    
    Therefore all intervals $I \in \intvlset$ are small ($|I| < \Kcp$).
    %
    %
    %
    %
    Then for each $I \in \intvlset$, $I \subset \intvl{(m-1)\Kcp}{(m+1)\Kcp}$.
    Also, 
    \begin{IEEEeqnarray}{Cr}
        \label{eq:failed-more-than-ppivots-induction}
        \Gat{I} - \Dat{I} \geq \frac{1}{2} \left( \Gat{I} - \Bat{I} \right) \geq \frac{1}{2} \Pat{I} & \quad \text{(\cref{prop:not-cp-interval-properties,prop:ppivots-imply-honest-margin})} \IEEEeqnarraynumspace
    \end{IEEEeqnarray}
    
    Consider the \iindices $k \in \intvl{(m-1)\Kcp}{(m+1)\Kcp}$ with $\Gat{k} = 1$ and $\Dat{k} = 0$.
    %
    %
    %
    %
    %
    %
    %
    Following the arguments in the proof of \cref{lem:one-cp-induction-base},
    we can reduce the set $\intvlset$
    so that in the resulting set $\intvlset$, each such \iindex $k$ is contained in at most two intervals.
    Then,
    \begin{IEEEeqnarray}{Cl}
        & \sum_{k \in \intvl{(m-1)\Kcp}{(m+1)\Kcp} \colon \Gat{k} = 1, \Dat{k} = 0} |\intvlset_k| \nonumber \\
        \leq& 
        %
        %
        2\left( \Gin{(m-1)\Kcp}{(m+1)\Kcp} - \Din{(m-1)\Kcp}{(m+1)\Kcp} \right). \IEEEeqnarraynumspace
    \end{IEEEeqnarray}
    This sum can be rewritten as
    \begin{IEEEeqnarray}{rCl}
        && \sum_{k \in \intvl{(m-1)\Kcp}{(m+1)\Kcp} \colon \Gat{k} = 1, \Dat{k} = 0} |\intvlset_k| \\
         &=& \sum_{I \in \intvlset} \left( \Gat{I} - \Dat{I} \right) \\
        &\geq& \sum_{I \in \intvlset} \frac{1}{2} \Pat{I} \\
        &\geq& \frac{1}{2} \Pin{m\Kcp}{(m+1)\Kcp}.
        \IEEEeqnarraynumspace
        %
    \end{IEEEeqnarray}
    Therefore,
    \begin{IEEEeqnarray}{rCl}
        &&\Gin{(m-1)\Kcp}{(m+1)\Kcp} - \Din{(m-1)\Kcp}{(m+1)\Kcp} \nonumber \\
        &\geq& \frac{1}{4} \Pin{m\Kcp}{(m+1)\Kcp}.
    \end{IEEEeqnarray}
    
    Finally, for each \index $k$ with $\Gat{k}=1$ and $\Dat{k}=0$, all honest nodes download at least $\goodsepbw$ distinct blocks produced in or before 
    the most recent \sltcp before $(m-1)\Kcp$.
    By induction assumption, we have a \sltcp $k_{m-2}^* \in \intvl{(m-2)\Kcp}{(m-1)\Kcp}$.
    %
    %
    This gives
    \begin{IEEEeqnarray}{rCl}
        && \Qin{(m-2)\Kcp}{(m+1)\Kcp} \nonumber \\
        &\geq& \goodsepbw \left( \Gin{(m-1)\Kcp}{(m+1)\Kcp} - \Din{(m-1)\Kcp}{(m+1)\Kcp} \right) \IEEEeqnarraynumspace \\
        &\geq& \frac{\goodsepbw}{4} \Pin{m\Kcp}{(m+1)\Kcp}
    \end{IEEEeqnarray}
    which is a contradiction.
    %
    \end{proof}




\subsection{Proof Details for \Cref{sec:pow}}
\label{sec:appendix-full-version-proofs-pow}

\begin{proposition}
    \label{prop:index-time-bridge-pow}
    \begin{IEEEeqnarray}{C}
        \forall k, K \in \IN \colon
        \Prob{ \slotduration(t_{k+K} - t_{k}) \geq \frac{K}{\blkratetime(1-\delta)} } \leq e^{ \frac{-K\delta^2}{2(1+\delta)}}.
        \IEEEeqnarraynumspace
    \end{IEEEeqnarray}
\end{proposition}
\begin{proof}
    This results from a Poisson tail bound for the number of \BPOs in real time $K/\blkratetime$, and noting that 
    each non-\sltempty \timeslot has exactly one \BPO.
    %
\end{proof}



\begin{proof}[Proof of \cref{lem:safety-and-liveness-comb-pow}]
    Safety: For an arbitrary \timeslot $t$, let $k$ be the largest \iindex such that $t_k \leq t$.
    From \eqref{condition-one-cp-m-pow-safety}, every interval of $2\Kcp$ \iindices contains at least one \sltcp. Therefore, there exists $k^* \in \intvl{k-2\Kcp-1}{k-1}$ such that $\predCP{k^*}$.
    Let $b^*$ be the block from \iindex $k^*$.
    Due to \cref{lem:cps-stabilize}, for all honest nodes $p,q$ and $t' \geq t$,
    $b^* \in \dC_p(t)$ and $b^* \in \dC_q(t')$.
    But $k^* \geq k-\confDepth$, so the block $b^*$ cannot be $\confDepth$-deep in any chain at \timeslot $t$ Therefore, $\LOG{p}{t}$ is a prefix of $b^*$ which in turn is a prefix of $\dC_q(t')$.
    We can thus conclude that
    either 
    $\LOG{p}{t} \preceq \LOG{q}{t'}$ or $\LOG{q}{t'} \preceq \LOG{p}{t}$.
    Therefore, safety holds.

    %
    %
    %
    %
    %
    %
    %
    %

    Liveness: Assume a transaction $\tx$ is received by all honest nodes before \timeslot $t$. Again let $k$ be the largest \iindex such that $t_k \leq t$.
    We know that there exists $k^* \in (k,k+2\Kcp]$ such that $\predCP{k^*}$.
    The honest block $b^*$ from \iindex $k^*$ or its prefix must contain $\tx$ since $\tx$ is seen by all honest nodes at time $t < t_{k^*}$.
    Since $k^*$ is a \sltcp, for all $\intvl{i}{j} \ni k^*$, $\Din{i}{j} > \Nin{i}{j}$ (\cref{def:cp} and \eqref{random_walks_X_and_Y}), and hence $\Din{i}{j} > \frac{j-i}{2}$.
    Particularly,
    \begin{IEEEeqnarray}{rCl}
        \Din{k^*-1}{k^* + 2\confDepth-1} &>& \confDepth \\
        \implies \Din{k^*}{k^* + 2\confDepth - 1} &>& \confDepth - 1.
        \IEEEeqnarraynumspace
    \end{IEEEeqnarray}
    Then from \cref{prop:chain-growth-interval},
    \begin{IEEEeqnarray}{rCl}
        L_{\min}(t_{k^*+2\confDepth-1}+\goodsep) - L_{\min}(t_{k^*+1}-1) &\geq& \Din{k^*}{k^* + 2\confDepth - 1} \nonumber \\
        &\geq& \confDepth.
        \IEEEeqnarraynumspace
    \end{IEEEeqnarray}
    Due to \cref{lem:cps-stabilize},
    $b^*\in\dC_p(t')$ for all honest nodes $p$ and $t'\geq t_{k^*} + \goodsep$, and $L_{\min}(t_{k^*+1}-1) \geq \len{b^*}$. This means that $b^*$ is $\confDepth$-deep in $\dC_p(t')$ for all honest nodes $p$ and all $t' \geq t_{k^*+2\confDepth-1}+\goodsep$.
    Finally, with $k^* \leq k+2\Kcp$ and \eqref{cond-index-time-bridge-pow},
    \begin{IEEEeqnarray}{rCl}
        t_{k^*+2\confDepth-1}+\goodsep - t &\leq& t_{k + 6\Kcp + 1} + \goodsep - t_{k} \nonumber \\
        &\leq& t_{k + 6\Kcp + 2} - t_{k} \nonumber \\
        &<& \frac{6\Kcp + 2}{\blkratetime\slotduration(1-\delta)}.
        \IEEEeqnarraynumspace
    \end{IEEEeqnarray}
    Therefore, $\mathsf{tx}\in\LOG{p}{t'}$ for all $t'\geq t+\Tlive$.
\end{proof}


%
%
%
%
%
%
%
%
%
%
%
%
%
%
%
%
%
%
%
%
%
%
%
%
%
%
%
%
%

%
%
%

%
%
%
%

%
%
%
%
%
%
%
%
%
%
%
%
%
%
%
%
%
%

%
%
%
%
%
%
%
%
%

%
%
%
%
%
%
%
%







\begin{proof}[Proof of \cref{thm:safety-and-liveness-pow}]

    First, we show that the conditions of \cref{lem:one-cp-induction-full} hold, and therefore \sltcps occur.
    Define the event
    \begin{IEEEeqnarray}{rCl}
        \Event_1 &=& \left\{\forall \intvl{i}{j} \intvlgeq \Kcp \colon \Pin{i}{j} > (1-\delta)\probPP(j-i) \right\} \IEEEeqnarraynumspace
    \end{IEEEeqnarray}
    Suppose that $\Event_1$ occurs, and $\frac{\goodsepbw}{16}\probPP (1-\delta) > 1$.
    Then,
    %
    \begin{IEEEeqnarray}{rCl}
        \forall \intvl{i}{j} \intvlgeq \Kcp \colon \frac{\goodsepbw}{4} \Pin{i}{j} &>& \frac{\goodsepbw}{4} (1-\delta)\probPP(j-i) \IEEEeqnarraynumspace \\
        &>& 4(j-i) \\
        &\eqA& \Qin{i-2\Kcp}{j+\Kcp}
    \end{IEEEeqnarray}
    where (a) is because as $\slotduration \to 0$, each non-\sltempty \timeslot has exactly one \BPO. This satisfies \eqref{cp-induction-full-ppivots-condition} in \cref{lem:one-cp-induction-full}. Further,
    \begin{IEEEeqnarray}{C}
        \frac{\goodsepbw}{2} \left( \Gin{i}{j} - \Bin{i}{j} \right)
        \geq \frac{\goodsepbw}{2} \Pin{i}{j} > 3 (j-i) > \Qin{i-2\Kcp}{j}\IEEEeqnarraynumspace
    \end{IEEEeqnarray}
    which satisfies condition \eqref{cp-induction-full-margin-condition} in \cref{lem:one-cp-induction-full}. 
    %
    Therefore there is one \sltcp in every interval of the form $\intvl{m\Kcp}{(m+1)\Kcp}$.
    %
    Also suppose the following event occurs:
    \begin{IEEEeqnarray}{rCl}
        \Event_2 &=& \left\{\forall k \in \IN, K \geq \Kcp \colon t_{k+K} - t_k < \frac{K}{\blkratetime\slotduration(1-\delta)} \right\}. \IEEEeqnarraynumspace
    \end{IEEEeqnarray}
    %
    %
    %
    %
    %
    %
    %
    Then \cref{lem:safety-and-liveness-comb-pow} guarantees safety and liveness with $\confDepth = 2\Kcp$ and $\Tlive = \frac{6\Kcp+2}{\blkratetime\slotduration(1-\delta)}$.
    
    %
    By choosing $\Kcp = \Omega(\kappa^2)$, $\Khorizon = \poly(\kappa)$, and using \cref{lem:many-pps}, \cref{prop:index-time-bridge-pow} and a union bound,
    %
    %
    %
    %
    %
    %
    %
    %
    %
    the probability of failure of either $\Event_1$ or $\Event_2$ is $\negl(\kappa)$.
    \Iindices are mapped to real time as $\TliveReal \triangleq \Tlive \slotduration$.
    
    Finally, we take the limit $\slotduration \to 0$.
    %
    With the relations $\blkratetime = \blkrateslot/\slotduration$, $\BWEquation$, and 
    $\probPP = \probPPFormula$,
    \begin{IEEEeqnarray}{C}
        \label{eq:prob-good-equation-pow}
        \probGood = \probGoodFormula \to (1-\beta)e^{-\blkratetime\left(\DeltaHeader + \goodsepbw/\bwtime\right)}, \IEEEeqnarraynumspace \\
        \label{eq:C-equation-pow}
        \frac{\goodsepbw}{16} \frac{(2\probGood-1)^2}{\probGood} (1-\delta) > 1
    \end{IEEEeqnarray}
    %
    
    %
    %
    %
    %
    %
    
    
    Note that $\goodsepbw$ is an analysis parameter whose value is arbitrarily.
    To find the maximum block production rate $\blkratetime$ that the protocol can achieve, we should optimize over $\goodsepbw$.
    To find the maximum achievable $\blkratetime$,
    we can take $\delta \to 0$ as we can increase the latency through increasing $\Kcp$ to still satisfy the error bounds.
    %
    %
    %
    %
    %
    %
    %
    %
    Maximizing over $\goodsepbw$ from \eqref{C-equation-pow,prob-good-equation-pow} gives the resulting threshold.
    
    
    
    %
    %
    %
\end{proof}





\subsection{Proof Details for \cref{sec:pos}}
\label{sec:appendix-full-version-proofs-pos}

The proposition below helps to derive the confirmation latency in units of real time from the analysis.
\begin{proposition}
    \label{prop:index-time-bridge-pos}
    For all $\delta \in (0,1)$, $k, K \in \IN$,
    \begin{IEEEeqnarray}{C}
        \Prob{ t_{k+K} - t_{k} \geq \frac{K/(1-e^{-\blkrateslot})}{1-\delta} } \leq 
        %
        e^{- 2K(1-e^{-\blkrateslot})\delta^2}, \IEEEeqnarraynumspace
        %
        %
        %
    \end{IEEEeqnarray}
\end{proposition}
\begin{proof}
    This results from a Hoeffding bound for the number of non-\sltempty \timeslots in $K/(1-e^{-\blkrateslot})$ \timeslots.
\end{proof}

We also need another proposition to bound the number of \BPOs in a given number of slots, in order to bound $\Qin{i}{j}$.
\begin{proposition}
\label{prop:Q-bound-pos}
\begin{IEEEeqnarray}{C}
    \forall t, T \in \IN \colon
    \Prob{ \sum_{r=t}^{t+T} (H_t + A_t) \geq \blkrateslot T (1+\delta) } \leq 
    e^{\frac{-\blkrateslot T \delta^2}{2(1+\delta)}}. \IEEEeqnarraynumspace
\end{IEEEeqnarray}
\end{proposition}
\begin{proof}
This results from a Poisson tail bound since $\sum_{r=t}^{t+T} (H_t + A_t) \sim \operatorname{Poisson}(\blkrateslot T)$.
\end{proof}





\begin{proof}[Proof of \cref{thm:safety-and-liveness-pos}]

First, we show that the conditions of \cref{lem:one-cp-induction-full} hold, and therefore \sltcps occur.
Suppose that the following three events occur.
\begin{IEEEeqnarray}{rCl}
    \Event_1 &=& \left\{\forall \intvl{i}{j} \intvlgeq \Kcp \colon \Pin{i}{j} > (1-2\delta)\probPP(j-i) \right\}, \nonumber
    \\
    \Event_2 &=& \left\{\forall t \in \IN, T \geq \frac{\Kcp}{1-e^{-\blkrateslot}} \colon \sum_{r=t}^{t+T} (H_t + A_t) < \blkrateslot T (1+\delta) \right\},
    \nonumber
    \\
    \Event_3 &=& \left\{ \forall k \in \IN, K \geq \Kcp \colon t_{k+K} - t_{k} < \frac{K/(1-e^{-\blkrateslot})}{1-\delta} \right\}. \nonumber
\end{IEEEeqnarray}

From $\Event_2$ and $\Event_3$, we get
\begin{IEEEeqnarray}{rCl}
    \forall \intvl{i}{j} \intvlgeq \Kcp \colon \quad 
    \Qin{i}{j} &\triangleq& \sum_{k=i+1}^{j} (H_{t_k} + A_{t_k}) \\
    \text{with } T = \frac{j-i}{(1-e^{-\blkrateslot})(1-\delta)}, \quad &\leq& \sum_{t=t_i}^{t_i + T} (H_{t} + A_{t}) \\
    &<& \frac{\blkrateslot (j-i) (1+\delta)}{(1-e^{-\blkrateslot})(1-\delta)} \\
    &\leq&  \frac{\blkrateslot (j-i)}{(1-e^{-\blkrateslot})(1-2\delta)}.
    \IEEEeqnarraynumspace
\end{IEEEeqnarray}
Then if $\frac{\goodsepbw}{16}\probPP (1-4\delta) > \frac{\blkrateslot}{(1-e^{-\blkrateslot})}$,
%
\begin{IEEEeqnarray}{rCl}
    \forall \intvl{i}{j} \intvlgeq \Kcp \colon \quad \frac{\goodsepbw}{4} \Pin{i}{j} &>& \frac{\goodsepbw}{4} (1-2\delta)\probPP(j-i)
    \IEEEeqnarraynumspace
    \\
    &>& \frac{4\blkrateslot(j-1)(1-2\delta)}{(1-e^{-\blkrateslot})(1-4\delta)}
    \\
    &>& \frac{4\blkrateslot(j-1)}{(1-e^{-\blkrateslot})(1-2\delta)}
    \\
    &>& \Qin{i-2\Kcp}{j+\Kcp}.
\end{IEEEeqnarray}
This satisfies \eqref{cp-induction-full-ppivots-condition} in \cref{lem:one-cp-induction-full}. Further,
\begin{IEEEeqnarray}{rCl}
    \frac{\goodsepbw}{2} \left( \Gin{i}{j} - \Bin{i}{j} \right)
    &\geq& \frac{\goodsepbw}{2} \Pin{i}{j} \\
    &>& \frac{3\blkrateslot(j-1)}{(1-e^{-\blkrateslot})(1-2\delta)}  \\
    &>& \Qin{i-2\Kcp}{j}
\end{IEEEeqnarray}
which satisfies condition \eqref{cp-induction-full-margin-condition} in \cref{lem:one-cp-induction-full}. 
%
Therefore there is one \sltcp in every interval of the form $\intvl{m\Kcp}{(m+1)\Kcp}$.
%
%
%
%
%
%
%
Then by \cref{lem:safety-and-liveness-comb-pos}, the protocol achieves safety and liveness with appropriately chosen $\confDepth, \keqproof, \Tlive$.
%
%
%
%

By using \cref{lem:many-pps}, $\Kcp = \Omega(\kappa^2)$, $\Khorizon = \poly(\kappa)$, \cref{prop:index-time-bridge-pos,prop:Q-bound-pos},
%
%
%
%
%
%
and union bounds, 
%
the probability of failure of either $\Event_1$, $\Event_2$ or $\Event_3$ is $\negl(\kappa)$.
%
%

The required security threshold is obtained from $\BWEquation$,
$\probGood = \probGoodFormula$,
$\frac{C}{16}\probPP > \frac{\blkrateslot}{1-e^{-\blkrateslot}}$,
and $\probPP = \probPPFormula$,.
As in the case of PoW, $\goodsepbw$ is a free parameter that can be optimized to find the best set of parameters.
%
%
%
%
%
%
%
%
%
%
\end{proof}