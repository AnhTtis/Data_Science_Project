\section{Attacks and Experiments}
\label{sec:experiments}

In this section, we describe the \teaserattack and \PoSteaserattack. We simulate both attacks 
%
on a network of $100$ nodes.\footnote{Source code: \gitSourceUrl}
Setup details are in \Cref{sec:attacks-details}.
Honest nodes collectively produce blocks at a rate $\blkratetimeHon = 1$ block per second.
Each node has a limited processing rate of $\bwtime$ blocks per second.
Blocks consist of content (transactions) and a header (block production lottery information and parent block pointer).
The header contains information to verify the block production lottery, thus nodes only process validly created blocks.
Given a tree of valid block headers, nodes run the \rulelc policy: nodes attempt to process (download and verify) the first unprocessed block along the longest header chain.
If the longest chain is already processed, or if the content of any block on that chain is unavailable or invalid, then the rule considers the next longest header chain, and so on.
%
%
%


\subsection{Recap of the Private Attack}
\label{sec:private-attack}

In the private attack, the adversary produces a chain of blocks that it keeps private until it becomes longer than the honest nodes' longest chain.
Subsequently, releasing the chain causes honest nodes to switch their longest chain. 
Recall that in Nakamoto consensus, honest nodes confirm transactions which are at least $\confDepth$ blocks deep in their longest downloaded chain, for some parameter $\confDepth$ chosen by the node.
Therefore, if the adversary's chain differs from the honest nodes' original longest chain by $\confDepth$ blocks, then the attack causes honest nodes to alter their ledger, which is a safety violation.

Note that during this attack, the adversary does not release any blocks. So, honest nodes undergo processing delay due to honest blocks only. 
%
The growth of the honest nodes' chain in this case is illustrated in \cref{fig:attack-no}.
Due to processing delays, the chain growth rate $\blkratetimeGrowthSilent$ is less than the honest mining rate $\blkratetimeHon$ (`no attack' in \cref{fig:experiment-teaser}).
In this case, the honest chain growth rate is approximately the same as that under a network where all blocks are processed within $\Delta=1/\bwtime$ time after they are announced (the bounded delay model), as seen in \cref{fig:experiment-teaser}.

The success of the private attack (and the attacks we describe ahead) depend crucially on the race between the honest chain growth rate $\blkratetimeGrowth$ and the adversary's block production rate $\blkratetimeAdv$.
If $\blkratetimeAdv > \blkratetimeGrowth$, then with high probability, in the long run, the adversary's private chain is longer than the honest chain, so the attack succeeds.
Thus, the honest chain growth rate determines the adversarial fraction $\beta$ required for the attack.%
\footnote{Precisely, the attack succeeds if $\beta \triangleq \frac{\blkratetimeAdv}{\blkratetimeAdv+\blkratetimeHon} > \frac{\blkratetimeGrowth}{\blkratetimeGrowth+\blkratetimeHon}$}
Conversely, if $\blkratetimeAdv < \blkratetimeGrowth$, then with high probability, in the long run, the adversary's chain is shorter than the honest chain.
In the short run however, even if $\blkratetimeAdv < \blkratetimeGrowth$, with some probability, the adversary's chain can get longer than the honest chain. Our attacks exploit this to save up blocks which will be released during the attack.

\import{./figures/}{fig-attack-no.tex}

\import{./figures/}{fig-attack-teaser.tex}

\import{./figures/}{fig-attack-pos-teaser.tex}


\subsection{The \TeaserAttack (PoW and PoS)}
\label{sec:teaser-attack}

%
In \cref{fig:attack-teaser}, we describe the \teaserattack. We see that the adversary utilizes the chain that it constructs not only to later overtake the public chain and break safety, but also to induce processing of one extra block for every block that grows the length of the honest chain. It therefore effectively doubles the processing invested per growth event of the public main chain.
%
%

Before the attack starts, the honest chain grows at the rate $\blkratetimeGrowthSilent$ just as in the private attack, because the adversary does not release any blocks.
%
%
To start the attack, the adversary mines a short private chain, which succeeds with some probability even if the adversary's mining rate is $\blkratetimeAdv < \blkratetimeGrowthSilent$.
The attacker then releases blocks from this chain during the \teaserattack.
Thereafter, the \teaserattack 
%
slows down the honest chain growth rate to $\blkratetimeGrowthTeaser < \blkratetimeGrowthSilent$ (`\teaserattack' in \cref{fig:attack-teaser}).
If the adversary's mining rate $\blkratetimeAdv$ exceeds $\blkratetimeGrowthTeaser$, then the adversary can maintain a chain that is longer than the honest chain and continue the attack forever.
Since $\blkratetimeGrowthTeaser < \blkratetimeGrowthSilent$, the \teaserattack succeeds with lower adversarial mining power than the private attack. This attack works in both PoW and PoS.

\subsection{The \PoSTeaserAttack (PoS)}
\label{sec:pos-teaser-attack}

%

In PoS, 
%
the adversary can greatly increase the network's processing load using equivocations. The \PoSteaserattack, described in \cref{fig:attack-pos-teaser}, uses equivocations to announce a whole new chain at every instance when the \teaserattack would have announced a single new block.
As the attack goes on, the length of the new announced chain increases. This increases the time honest nodes spend downloading this chain, and \emph{decelerates} the honest chain growth until it comes to a halt. As a result, in \cref{fig:experiment-teaser}, the chain growth rate under the \PoSteaserattack is nearly zero (the small nonzero value is because there is some chain growth at the start). 

As in the \teaserattack, the adversary starts by producing a private chain. Assuming the adversary's block production rate $\blkratetimeAdv$ is less than the honest chain growth rate before the attack ($\blkratetimeGrowthSilent$), the probability that the adversary produces a chain of length $L$ before the honest chain reaches length $L$ is $e^{-O(L)}$ \cite{nakamoto_paper,dem20}. This means that with probability $\epsilon$, the adversary eventually produces a private chain of length $L = O(\log(1/\epsilon))$, of which it can announce equivocations during the attack.
Since this chain is longer than the honest chain, it has higher download priority.
It takes honest nodes $L/\bwtime$ time to download such a chain, during which time, honest nodes do not download blocks on the honest chain. So, any honest blocks produced within $L/\bwtime$ time after the first honest block at height $h$ do not grow the honest chain (\cref{fig:attack-teaser}(e)). If $\blkratetimeHon L/\bwtime$ is large, then there are many honest blocks that do not lead to chain growth, causing the chain growth rate $\blkratetimeGrowth$ to drop (\cref{fig:experiment-teaser}). As in the \teaserattack, if the adversary's block production rate $\blkratetimeAdv$ exceeds $\blkratetimeGrowth$, then the adversary succeeds in maintaining the number of block productions required for the attack to go on forever. This eventually slows honest chain growth to a halt. Thus, if $\blkratetimeHon L/\bwtime$ is large, \ie, $\blkratetimeHon = \Omega(1/L) = \Omega\left(\frac{1}{\log(1/\epsilon)}\right)$, then the attack succeeds with probability $\epsilon$.
A more detailed analysis can be found in 
%
\fullVersionRef{\cref{sec:pos-attack}}.


\import{./figures/}{fig-experiment-teaser.tex}

It may seem at first that the above attacks exploit the specific \rulelc scheduling policy to tease honest nodes into downloading adversarial blocks. However, even for other scheduling policies, it is possible to devise attack strategies which exploit increased queuing delays and thereby succeed for parameter regimes where the private attack does not succeed.
For example, the \teaserattack would not succeed if honest nodes `greedily' prioritized processing blocks that extend their longest chain. But this is vulnerable to a forking attack in which, following a short network split, honest nodes build two separate chains and fail to ever catch up with the other chain.
Some such generalizations of our attacks are described in \fullVersionRef{\cref{sec:general-attacks}}.
The overarching conclusion 
%
is that models for security analysis must capture effects of adversarial queuing delay.


%
%

%

%

%

%

%

%

%
%
%

%