\section{Security Analysis Proofs}
\label{sec:appendix-security-proofs}

%


%
%
%
%
%
%
%
%
%
%
%
%
%
%
%
%
%
%
%
%
%
%
%
%
%
%
%
%
%
%
%
%
%
%
%
%
%
%
%
%





\subsection{Combinatorial Pivots Stabilize}
\label{sec:appendix-security-proofs-cps-stabilize}

The following proposition is helpful for our proofs. The proof of \cref{lem:cps-stabilize} is similar to that of \cite[Lem.~5]{sleepy}, \cite[Thm.~3.2]{dem20}, or \cite[Lem.~11]{bwlimitedposlc} and is hence deferred to \fullVersionRef{\cref{sec:appendix-security-proofs-cps-stabilize}}.

\begin{proposition}
\label{prop:chain-growth-interval}
For any $i < j$,
\begin{IEEEeqnarray}{C}
    L_{\min}(t_j + \goodsep) \geq L_{\min}(t_{i+1} - 1) + \Din{i}{j}.
\end{IEEEeqnarray}
\end{proposition}
\begin{proof}
%
%
%
%
%
%
%
%
%
By noting that if $\Dat{k} = 1$, then $t_{k+1} > t_k + \goodsep$, and adding the result of \cref{prop:chain-growth} for each \iindex with $\Dat{k}=1$.
\end{proof}











\subsection{Probabilistic Pivots Are Abundant}
\label{sec:appendix-security-proofs-many-pps}

These alternative characterizations
of \sltpps are insightful:
\begin{proposition}
    \label{prop:pivot-conditions-equivalence}
    \begin{IEEEeqnarray}{rCl}
        \predPP{k}
        &\iff&  (\forall \intvl{i}{j} \ni k\colon  \Xin{i}{j} > 0)
        \label{eq:pivot-conditions-equivalence-intervals}
        \IEEEeqnarraynumspace\\
        &\iff&  (\forall \intvl{i}{j} \ni k\colon  \Gin{i}{j} > \Bin{i}{j})
        \IEEEeqnarraynumspace\\
        &\iff&  (\Xat{k} = 1) \land (\forall j\geq k: \Xin{k}{j} \geq 0)
        \IEEEeqnarraynumspace\nonumber\\
        && \quad {}\land{} (\forall i<(k-1): \Xin{i}{k-1} \geq 0)
        \label{eq:pivot-conditions-equivalence-randomwalks}
        \IEEEeqnarraynumspace
    \end{IEEEeqnarray}
\end{proposition}
\begin{proof}
    Elementary, using $\Xin{i}{j} = \Xin{0}{j} - \Xin{0}{i}$.
\end{proof}
%
%
%
%
%
%
%
%
%
%
In particular, \eqref{pivot-conditions-equivalence-randomwalks}
characterizes a \sltpp as an \iindex $k$
such that $\Gat{k} = 1$
and the simple random walks
$\ell\mapsto \Xin{k}{k+\ell}$
and
$\ell\mapsto \Xin{k-1-\ell}{k-1}$
starting at
%
$0$
remain non-negative forever
%
(\cref{fig:pivot-randomwalk}).
%
The process $\{\Pat{k}\}$ counts \sltpps,
with increments $\Pat{k} \triangleq \Ind{\predPP{k}}$.


\import{./figures/}{fig-pivot-randomwalk.tex}

We build up to the proof of \cref{lem:many-pps} through a series of propositions.
Assume that
$\probGood = \frac{1}{2} + \epsGood$
with $\epsGood \in (0,1/2]$

%
%
\begin{proposition}
    \label{prop:lower-tailbound-X}
    With $\alphaLowerTailX \triangleq 2 \epsGood^2$, $\forall \intvl{i}{j}\colon \forall \delta \geq 0\colon$
    \begin{IEEEeqnarray}{C}
        \Prob{\Xin{i}{j} \leq (1-\delta) 2 \epsGood (j-i)}
        \leq \exp( - \alphaLowerTailX \delta^2 (j-i)).
        \IEEEeqnarraynumspace
    \end{IEEEeqnarray}
\end{proposition}


\begin{proposition}
    \label{prop:ppivot-randomwalk}
    \begin{IEEEeqnarray}{C}
        \forall k\colon
        \Prob{\predPP{k}}
        \geq \probPPFormula
        %
        \triangleq \probPP
    \end{IEEEeqnarray}
\end{proposition}


\begin{proposition}
    \label{prop:lower-tailbound-ppivots}
    With $\alphaLowerTailPP \triangleq 2 \probPP^2$,
    \begin{IEEEeqnarray}{l}
        \forall \intvl{i}{j} \intvleq 2 K_1 K_2\colon
        %
        %
        %
        \Prob{\Pin{i}{j} \leq (1-\delta) \probPP 2 K_1 K_2}
        \nonumber
        \\
        \qquad\qquad {}\leq{} 2 K_1 \exp(- \alphaLowerTailPP \delta^2 K_2) + \Khorizon^2 \exp(-\alphaLowerTailX K_1).
        \IEEEeqnarraynumspace
    \end{IEEEeqnarray}
\end{proposition}

Proof of \cref{prop:lower-tailbound-X} is by Hoeffding's inequality ~\cite{doi:10.1080/01621459.1963.10500830} ~\cite[Thm.~4]{duchi-hoeffding}.
\Cref{prop:ppivot-randomwalk} is proved using the probability of a random walk never returning to zero ~\cite{stackexchange-math-rwreturnto0}.
The proof of \cref{prop:lower-tailbound-ppivots} is similar to that of ~\cite[Thm.~5]{sleepy}, except that we use a concentration bound to show that there are many \sltpps in $\intvl{i}{j}$ and not just one as shown in \cite{sleepy}.
Details are in \fullVersionRef{\cref{sec:appendix-security-proofs-many-pps}}.



\begin{proof}[Proof of \cref{lem:many-pps}]
From \cref{prop:lower-tailbound-ppivots} by setting $K_1,K_2 = \Omega(\kappa)$ and $\Kcp = 2K_1K_2$.
\end{proof}







\subsection{Many Probabilistic Pivots Imply One Combinatorial Pivot}
\label{sec:appendix-security-proofs-many-pps-one-cp}


\begin{proof}[Proof of \cref{prop:download-or-spend-budget}]
%

%

In \timeslot $t_k$, there is exactly one block $b$ produced by an honest node, and 
the block header is made public at the beginning of the \timeslot,
and is seen by all honest nodes within $\DeltaHeader$ time.
Thereafter, each node has enough time to download $\goodsepbw$ blocks during \timeslots $[t_k, t_k + \goodsep]$.

%
%

%
%
%
%
%
%
    %
    %
%
%
%


Under the download rule $\dlrulelong$, all honest nodes download content for their longest header chain.
If $\Dat{k} = 0$
\ie an honest node did not download content for the block $b$ before the end of \timeslot $t_k + \goodsep$,
then
%
that honest node must download the content for at least $\goodsepbw$ blocks on chains longer than the height of the block $b$ or in the prefix of the block $b$.
%
Since honest nodes produce blocks extending their longest chain, $b$ extends $\dC_p(t_k-1)$ for some $p$.
Let $b^*$ be the block produced in \timeslot $t_i$ where $\predCP{i}$ (suppose $i$ exists).
$\predCP{i} \implies \Yat{i} = 1$, therefore this block is unique, and also $t_k > t_i + \goodsep$.
Due to \cref{lem:cps-stabilize}, any valid header chain longer than $b$ at time slot $t_k$ must contain $b^*$.
%
Therefore, the only blocks 
%
that are downloaded by an honest node during \timeslots $[t_k, t_k + \goodsep]$
\begin{enumerate}
    \item must be produced after $t_i$ because they extend $b^*$, and
    \item must be produced no later than $t_k$ because there are no blocks produced in $\intvl{t_k}{t_k+\goodsep}$.
\end{enumerate}
%
In case a \sltcp $i<k$ does not exist, the claim is trivial.
%
%
\end{proof}



\begin{proposition}
\label{prop:not-cp-exists-interval}
\begin{IEEEeqnarray}{C}
    \lnot \predCP{k} \implies \exists \intvl{i}{j} \ni k \colon \Yin{i}{j} \leq 0.
\end{IEEEeqnarray}
\end{proposition}

\begin{proof}
From \cref{def:cp}, $\lnot \predCP{k}$ implies that either there exists $i < k$ such that $\Yin{0}{i} \geq \Yin{0}{k}$ or there exists $j \geq k$ such that $\Yin{0}{k} > \Yin{0}{j}$.
In the first case, $\intvl{i}{k} \ni k$ and $\Yin{i}{k} \leq 0$.
In the second case, $\intvl{k-1}{j} \ni k$ and $\Yin{k-1}{j} \leq \Yin{k}{j} + 1 \leq 0$.
\end{proof}


\begin{proposition}
\label{prop:not-cp-interval-properties}
If $\Yin{i}{j} \leq 0$, then
\begin{IEEEeqnarray}{rCl}
    \label{eq:not-cp-interval-property1}
    \Nin{i}{j} &\geq& \Din{i}{j}, \\
    \label{eq:not-cp-interval-property2}
    \Gin{i}{j} - \Din{i}{j} &\geq& \frac{1}{2} \left( \Gin{i}{j} - \Bin{i}{j} \right).
\end{IEEEeqnarray}
\end{proposition}

\begin{proof}
\Eqref{not-cp-interval-property1} is by the definition $\Yat{i} = \Dat{i} - \Nat{i}$.
%
Then,
\begin{IEEEeqnarray}{rCl}
    \Gin{i}{j} + \Bin{i}{j} &=& \Din{i}{j} + \Nin{i}{j} \\
    \Gin{i}{j} + \Bin{i}{j} &\geq& 2 \Din{i}{j} \\
    2 \Gin{i}{j} - 2 \Din{i}{j} &\geq& \Gin{i}{j} - \Bin{i}{j}.
\end{IEEEeqnarray}
\end{proof}



\begin{proposition}
\label{prop:ppivots-imply-honest-margin}
If $\Pin{i}{j} > 0$, then $\Gin{i}{j} - \Bin{i}{j} \geq \Pin{i}{j}$.
\end{proposition}

\begin{proof}
Let $n = \Pin{i}{j}$.
First, consider the case $n=1$.
There is exactly one \sltpp $k \in \intvl{i}{j}$.
From \cref{def:pp}, $\Xin{0}{i} < \Xin{0}{j}$. Therefore, $\Xin{i}{j} > 0$, hence $\Gin{i}{j} - \Bin{i}{j} \geq 1$.

For the general case, let $k_1,...,k_n$ be the \sltpps in $\intvl{i}{j}$. Then, we can apply the $n=1$ case on the disjoint intervals $\intvl{i}{k_1}$, $\intvl{k_1}{k_2}, ...$, $\intvl{k_{n-1}}{j}$ and then sum them up.

%
%
%
%
\end{proof}




\begin{lemma}
\label{lem:one-cp-induction-base}
If all honest nodes use
the download rule $\dlrulelong$,
%
%
%
%
%
and if
\begin{IEEEeqnarray}{C}
    \label{eq:cp-induction-base-margin-condition}
    \forall \intvl{i}{j} \intvlgeq \Kcp, i < \Kcp \colon \frac{\goodsepbw}{2} \left( \Gin{i}{j} - \Bin{i}{j} \right) > \Qin{0}{j}, \IEEEeqnarraynumspace \\
    %
    \label{eq:cp-induction-base-ppivots-condition}
    \frac{\goodsepbw}{4} \Pin{0}{\Kcp} > \Qin{0}{2\Kcp},
    %
\end{IEEEeqnarray}
then $\exists k_1^* \in \intvl{0}{\Kcp} \colon \predCP{k_1^*}$.
\end{lemma}

\begin{proof}
Due to \eqref{cp-induction-base-ppivots-condition}, there is at least one \sltpp in $\intvl{0}{\Kcp}$ (otherwise $\Pin{0}{\Kcp}=0$).
Suppose for contradiction that there is no \sltcp in $\intvl{0}{\Kcp}$.
Since \sltcps are also \sltpps, it is enough to consider that
none of the \sltpps is a \sltcp.
Then around each \sltpp, there must be at least one interval which violates the combinatorial pivot condition.
Formally, there is a set of intervals $\intvlset$ such that:
\begin{IEEEeqnarray}{C}
    \label{eq:intervals-cover-ppivots}
    \bigcup_{I \in \intvlset} I \supseteq \left\{ k \in \intvl{0}{\Kcp} \colon \predPP{k} \right\} \IEEEeqnarraynumspace \\
    \label{eq:intervals-y-condition}
    \forall I \in \intvlset \colon \Yat{I} \leq 0 \quad \text{(from \cref{prop:not-cp-exists-interval})}. \IEEEeqnarraynumspace
\end{IEEEeqnarray}
Without loss of generality, each interval $I \in \intvlset$ contains at least one \sltpp (removing all intervals that do not contain a \sltpp maintains \eqref{intervals-cover-ppivots,intervals-y-condition}).
Then if $\intvl{i}{j} \in \intvlset$, $i < \Kcp$.

First, let's consider the large intervals with $|I| \geq \Kcp$.
Consider \iindices $k \in I$ for which $\Gat{k}=1$ (\sltgood) but $\Dat{k}=0$ (\ydown).
From \cref{prop:download-or-spend-budget}, for each such \iindex, all honest nodes download $\goodsepbw$ blocks that are produced no later than $t_k$.
%
%
%
%
%
%
The number of \iindices $k \in I$ with  $\Gat{k} = 1$ and $\Dat{k} = 0$ is exactly $\Gat{I} - \Dat{I}$.
For each such index, there must exist $\goodsepbw$ distinct blocks produced in or before the interval $I$. Therefore if $I = \intvl{i}{j}$,
\begin{IEEEeqnarray}{rClr}
    \Qin{0}{j} &\geq& \goodsepbw \left( \Gin{i}{j} - \Din{i}{j} \right) & \\
    &\geq& \frac{\goodsepbw}{2} \left( \Gin{i}{j} - \Bin{i}{j} \right) & \quad \text{(from \cref{prop:not-cp-interval-properties}).}
\end{IEEEeqnarray}
This is a contradiction to \eqref{cp-induction-base-margin-condition}.

Therefore all intervals $I \in \intvlset$ are small ($|I| < \Kcp$).
%
%
%
%
Then for each $I \in \intvlset$, $I \subset \intvl{0}{2\Kcp}$.
Also, 
\begin{IEEEeqnarray}{rClr}
    \Gat{I} - \Dat{I} &\geq& \frac{1}{2} \left( \Gat{I} - \Bat{I} \right) & \quad \text{(from \cref{prop:not-cp-interval-properties})} \\
    \label{eq:failed-more-than-ppivots}
    &\geq& \frac{1}{2} \Pat{I} & \quad \text{(from \cref{prop:ppivots-imply-honest-margin}).}
\end{IEEEeqnarray}

Consider the \iindices $k \in \intvl{0}{2\Kcp}$ with $\Gat{k} = 1$ and $\Dat{k} = 0$.
%
Let $\intvlset_k = \{ I \in \intvlset \colon k \in I\}$ be the set of intervals that contain \index $k$.
Let $I^L_k$ be an interval in $\intvlset_k$ that stretches farthest to the left, and let $I^R_k$ be an interval that stretches farthest to the right (these may also be the same).
%
Note that all other intervals in $\intvlset_k$ are contained in $I^L_k \cup I^R_k$.
Therefore, all intervals in $\intvlset_k$ except $I^L_k$ and $I^R_k$ can be removed from $\intvlset$ while maintaining \eqref{intervals-cover-ppivots,intervals-y-condition} (see \cref{fig:one-cp-proof-figures}(a)).
This process is repeated for all $k \in \intvl{0}{2\Kcp}$ with $\Gat{k} = 1$ and $\Dat{k} = 0$, so that in the resulting set $\intvlset$, each such \iindex $k$ is contained in at most two intervals.
Then,
\begin{IEEEeqnarray}{rCl}
    \sum_{k \in \intvl{0}{2\Kcp} \colon \Gat{k} = 1, \Dat{k} = 0} |\intvlset_k|  &\leq& \sum_{k \in \intvl{0}{2\Kcp} \colon \Gat{k} = 1, \Dat{k} = 0} 2 \IEEEeqnarraynumspace \\
    &=& 2\left( \Gin{0}{2\Kcp} - \Din{0}{2\Kcp} \right). \IEEEeqnarraynumspace
\end{IEEEeqnarray}
This sum can be rewritten as
\begin{IEEEeqnarray}{rCl}
    \sum_{k \in \intvl{0}{2\Kcp} \colon \Gat{k} = 1, \Dat{k} = 0} |\intvlset_k| &=& \sum_{I \in \intvlset} \left( \Gat{I} - \Dat{I} \right) \IEEEeqnarraynumspace \\
    &\geq& \sum_{I \in \intvlset} \frac{1}{2} \Pat{I} \IEEEeqnarraynumspace \\
    &\geq& \frac{1}{2} \Pin{0}{\Kcp} \quad \text{(\eqref{intervals-cover-ppivots})}.
    \IEEEeqnarraynumspace
\end{IEEEeqnarray}
Therefore,
\begin{IEEEeqnarray}{rCl}
    \Gin{0}{2\Kcp} - \Din{0}{2\Kcp} &\geq& \frac{1}{4} \Pin{0}{\Kcp}.
\end{IEEEeqnarray}
This can also be seen from \cref{fig:one-cp-proof-figures}(b).

Finally, as shown before, for each \index $k$ with $\Gat{k}=1$ and $\Dat{k}=0$, all honest nodes download at least $\goodsepbw$ distinct blocks produced in or before \iindex $k$ (\cref{prop:download-or-spend-budget}). This gives
\begin{IEEEeqnarray}{rCl}
    \Qin{0}{2\Kcp} &\geq& \goodsepbw \left( \Gin{0}{2\Kcp} - \Din{0}{2\Kcp} \right) \\
    &\geq& \frac{\goodsepbw}{4} \Pin{0}{\Kcp}
\end{IEEEeqnarray}
which is a contradiction to \eqref{cp-induction-base-ppivots-condition}.
%
\end{proof}



\import{./figures/}{fig-one-cp-proof-figures.tex}


\begin{proof}[Proof of \cref{lem:one-cp-induction-full}]
This is proved by induction. For the base case ($m=0$), \cref{lem:one-cp-induction-base} shows that $\exists k_1^* \in \intvl{0}{\Kcp} \colon \predCP{k_1^*}$.
For $m \geq 1$, assume that $\exists k_{m-1}^* \in \intvl{(m-1)\Kcp}{m\Kcp}$ such that $\predCP{k_{m-1}^*}$.
Now we want to show that $\exists k_{m}^* \in \intvl{m\Kcp}{(m+1)\Kcp}$ such that $\predCP{k_{m}^*}$.
The proof for this follows the same steps as the proof of \cref{lem:one-cp-induction-base}, except that we make use of \cref{prop:download-or-spend-budget} to note that honest nodes only download blocks proposed after the most recent \sltcp. Then due to the induction assumption, during an interval of \iindices $\intvl{i}{j}$ where $i < (m+1)\Kcp$, honest nodes only download blocks produced in \iindices $\intvl{i-2\Kcp}{j}$.
The details of this proof are in 
%
\fullVersionRef{\cref{sec:appendix-full-version-proofs-many-pps-one-cps}}.
\end{proof}