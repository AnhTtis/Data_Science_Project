\subsection{Unified Model for PoW and PoS}
\label{sec:analysis-probmodel}

We develop a unified probabilistic model for the block production 
%
of both PoW and PoS as per \cref{alg:hdrtree-pow,alg:hdrtree-pos}.
This enables us to prove properties of the block production process and block tree structure that are common to both variants (\cref{sec:analysis-details}).
We then use these properties to prove security of PoW NC (\cref{sec:pow}) and PoS NC (\cref{sec:pos}).
%

Recall that the protocol runs in discrete units of time of duration $\slotduration$ called \timeslots, and that we consider $\slotduration \to 0$ to model PoW.
A \emph{block production opportunity} (\BPO) is a pair $(p,t)$ where according
%
to the PoW/PoS block production lottery,
node $p$ is eligible to produce a block in \timeslot $t$.
A \BPO is called \emph{honest} (resp. \emph{adversarial}) if node $p$ is honest (resp. adversarial).
%
%
The random variables $H_t$ and $A_t$ 
%
denote
the number of honest and adversarial 
%
\BPOs
in \timeslot $t$, respectively.
When the number of nodes $N\to\infty$ and each node holds an equal rate of block production, by the Poisson approximation of a binomial random variable,
we have $\Hat{t}\overset{\text{i.i.d.}}{\sim}\mathrm{Poisson}((1-\beta)\rho)$
and $\Aat{t}\overset{\text{i.i.d.}}{\sim}\mathrm{Poisson}(\beta\rho)$, independent of each other and across \timeslots.
The total number of \BPOs per \timeslot is $\Qat{t} = \Hat{t} + \Aat{t}$.
An \emph{execution} refers to a particular realization of the random process $\{(\Hat{t}, \Aat{t})\}$.

In PoW, as we take $\slotduration \to 0$, the block production process 
%
converges to a \emph{Poisson point process}.
As noted in \cref{sec:modelprotocol-protocolfeatures}, each \BPO corresponds to a different \timeslot, and thus in PoW, blocks in one chain must come from increasing \timeslots. In PoS the latter property is by design (\cref{alg:hdrtree-pos}, \cref{loc:hdrtree-pos-check-lottery}).

%
%

In this unified model, we make the adversary's powers the strongest of both PoW and PoS.
Specifically, we allow the adversary to create multiple blocks from the same \BPO (equivocations) which is only possible in PoS but not in PoW.
%
%
%
However, we assume in the unified analysis that honest nodes use a download rule which downloads at most one block per \BPO.
From a bandwidth perspective, this puts both PoW and PoS on an equal footing.
Then as seen in \cite{dem20, tight_bitcoin}, the additional ability to equivocate does not change the 
block tree properties
and therefore allows us to use similar techniques in our unified analysis.
%
%
The assumption of downloading at most one block per \BPO clearly holds for any download rule in PoW, but we define an \equivocationremoval policy to achieve this in PoS, so that the unified model applies to PoS as well.


%
%
%
%


%
%



%
%
%
%
%
%
%
%


%
%

%
%

%

%


%
%

%

%
%
%

