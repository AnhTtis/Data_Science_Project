\subsection{Definitions}
\label{sec:analysis-definitions}

%
%
%
%
%
%
%
%
%
%
%
%
%
%
%
%
%
%

`\sltGood' \timeslots are
\timeslots with
%
%
%
exactly one honest \BPO and no adversarial \BPOs in that \timeslot,
and
%
%
no \BPOs in $\goodsep$ \timeslots after.
This definition is inspired by convergence opportunities \cite{pss16,sleepy,kiffer2018better}, loners \cite{dem20}, and laggers \cite{ren}.
%
Here, $\goodsep$ is an analysis parameter whose value is chosen such that each honest node can 
receive the block header from the honest \BPO, and
download content for $\goodsepbw$ blocks within $\goodsep+1$ \timeslots, \ie, 
\begin{IEEEeqnarray}{C}
    %
    \label{eq:goodsep-bw-equation}
    (\goodsep+1)\slotduration \triangleq \DeltaHeader + \goodsepbw / \bwtime.
\end{IEEEeqnarray}

%
%
%
%
%

%
%

%


\begin{definition}
    \label{def:slots}
    We call a \timeslot $t$ \emph{\sltgood}, \emph{\sltbad}, \emph{\sltempty},
    %
    %
    %
    respectively,
    denoted as $\predGood{t}$, $\predBad{t}$, $\predEmpty{t}$, respectively, iff:
    \begin{IEEEeqnarray}{rCl}
        \predGood{t} &\;\triangleq\;& (\Hat{t} = 1) \land (\Aat{t} = 0) \nonumber \\ && \quad \land (\Hin{t}{t+\goodsep} + \Ain{t}{t+\goodsep} = 0)
        \IEEEeqnarraynumspace\\
        \predBad{t} &\;\triangleq\;& (\Hat{t} + \Aat{t} > 0) \land \lnot\predGood{t}
        \IEEEeqnarraynumspace\\
        \predEmpty{t} &\;\triangleq\;& (\Hat{t} + \Aat{t} = 0).
        \IEEEeqnarraynumspace
    \end{IEEEeqnarray}
\end{definition}
Note that $\predEmpty{t} = \lnot\predGood{t} \land \lnot\predBad{t}$.
We denote by $t_k$ the $k$-th non-\sltempty \timeslot.
Then, we can introduce random processes over \emph{\iindices},
with \iindex $k$ corresponding
to the $k$-th non-\sltempty \timeslot $t_k$.
The process $\{\Gat{k}\}$ counts good \timeslots,
with $\Gat{k} \triangleq 1$ if $\predGood{t_k}$,
and $\Gat{k} \triangleq 0$ otherwise (\ie, if $\predBad{t_k}$).
Correspondingly, $\{\Bat{k}\}$ counts bad \timeslots,
$\Bat{k} \triangleq 1 - \Gat{k}$.
%

\begin{proposition}
    \label{prop:X_i-is-iid}
    The random variables $\{\Gat{k}\}$ are independent and identically distributed (\iid) with
    \begin{IEEEeqnarray}{C}
        \Prob{\Gat{k} = 1} \triangleq \probGood = \probGoodFormula.
        %
    \end{IEEEeqnarray}
\end{proposition}
The proof is by noting that the inter-arrival times between non-\sltempty \timeslots are \iid and independent of how many and what kind (honest/adversarial) \BPOs occur in that non-\sltempty \timeslot. Details are in \fullVersionRef{\cref{sec:appendix-full-version-proofs-definitions}}.
Throughout the analysis, we will assume that
$\probGood > \frac{1}{2}$ (`honest majority' assumption).

A special role is played by good \timeslots $t_k$
%
as these are candidate \timeslots in which
the block produced
at $t_k$ is `soon' downloaded by all honest nodes.
We count these \timeslots with $\{\Dat{k}\}$,
and all other non-\sltempty \timeslots with $\{\Nat{k}\}$.
Specifically,
$\Dat{k} \triangleq 1$ if $\predGood{t_k}$
and the block produced at $t_k$
has been downloaded by all honest nodes by the end
of \timeslot $t_k + \goodsep$,
$\Dat{k} \triangleq 0$ otherwise,
and $\Nat{k} \triangleq 1 - \Dat{k}$.
We call slots $k$ with $\Dat{k}=1$ as \yups and those with $\Nat{k}=1$ as \ydowns.

Finally, we define two random walks
on \iindices of non-\sltempty \timeslots
with increments
$\{\Xat{k}\}$ and $\{\Yat{k}\}$
that will come in handy for the definition of probabilistic
and combinatorial pivots:
\begin{IEEEeqnarray}{rClCrCl}
    \label{eq:random_walks_X_and_Y}
    \Xat{k} &\triangleq& \Gat{k} - \Bat{k}
    &\qquad\qquad&
    \Yat{k} &\triangleq& \Dat{k} - \Nat{k}
    \IEEEeqnarraynumspace
\end{IEEEeqnarray}
Note that the increments $\{\Xat{k}\}$
are \iid, and not affected by adversarial action,
while the increments $\{\Yat{k}\}$ \emph{do depend}
on the adversarial action and are thus in particular
\emph{not} \iidPERIOD.
Also note that $\forall k\colon Y_k \leq X_k$ since $D_k = 1 \implies G_k = 1$.


\begin{definition}
    \label{def:pp}
    We call an \iindex $k$ a \emph{\sltpp} (short for \emph{probabilistic pivot}),
    denoted as $\predPP{k}$, iff:
    \begin{IEEEeqnarray}{rCl}
        \predPP{k} &\;\triangleq\;&  (\forall \intvl{i}{j} \ni k\colon  \Xin{0}{i} < \Xin{0}{k} \leq \Xin{0}{j})
        \IEEEeqnarraynumspace
    \end{IEEEeqnarray}
\end{definition}
This definition of \sltpps captures
the \emph{probabilistic} aspects of~\cite[Def.~5]{sleepy}
used in~\cite[Sec.~5.6.3]{sleepy}
and casts them as conditions
on a random walk,
inspired by~\cite{dem20, close-latency-security-ling-ren}, to simplify the analysis.
%

%


\begin{definition}
    \label{def:cp}
    We call an \iindex $k$ a \emph{\sltcp} (short for \emph{combinatorial pivot}),
    denoted as $\predCP{k}$, iff:
    \begin{IEEEeqnarray}{rCl}
        \predCP{k} &\;\triangleq\;&  (\forall \intvl{i}{j} \ni k\colon  \Yin{0}{i} < \Yin{0}{k} \leq \Yin{0}{j})
        \IEEEeqnarraynumspace
    \end{IEEEeqnarray}
\end{definition}
This definition of \sltcps captures
the \emph{combinatorial} aspects of~\cite[Def.~5]{sleepy}
used in~\cite[Sec.~5.6.2]{sleepy}
and casts them as conditions
on a random walk,
inspired by~\cite{dem20}, to simplify the analysis.
%
%
Note that a \sltcp is also a \sltpp because $Y_i \leq X_i$.


We denote by $\dC_p(t)$ the longest fully downloaded chain of an honest node $p$ at the end of \timeslot $t$, and let $\len{b}$ denote the height of a block $b$. We use the same notation $\len{\Chain}$ to denote the length of a chain $\Chain$, define $L_p(t)=\len{\dC_p(t)}$ and $L_{\min}(t) = \min_p L_p(t)$.
%


%
%
%
%
%
%
%
%
%
%
%
%
%
%



%

%

%

%

%

%

%


