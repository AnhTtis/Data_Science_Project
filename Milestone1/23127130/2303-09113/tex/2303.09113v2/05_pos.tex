\section{Sanitizing-Proof-of-Stake (\sapos)}
\label{sec:pos}

\subsection{\EquivocationRemoval}
\label{sec:pos-equivocations}

%

%
%

%




%
%
%
%
%
For PoS,
due to spamming by equivocations,
we need a policy to ensure 
that nodes download at most one block from each \BPO.
%
We therefore propose the Sanitizing-Proof-of-Stake (\sapos) protocol, in which the contents of provably equivocating blocks are sanitized from the blockchain.
Pseudocodes \cref{alg:posequivblank-lc} and \cref{alg:posequivblank-hdrtree} are in \cref{sec:appendix-pos-eqremoval-pseudocodes}.

\paragraph{The Download Rule in \sapos}
%
On top of any existing download rule (such as \rulelc), we add another rule that an honest node does not download content for a header $\Chain$ if it has seen 
another equivocating header
%
from the same \BPO (same producing node and \timeslot) as $\Chain$.
Instead of downloading content for such a header, the node considers that content to be ``downloaded'' and sets it to be empty
%
(\myalgref{alg:posequivblank-lc}{loc:posequivblank-lc-downloadrule}).
This means that the node can continue to download content for headers that extend $\Chain$, and these blocks will be candidates for the node's longest downloaded chain $\dC$.
%

\paragraph{Equivocation Proofs}
With only the above download rule,
one honest node may download content for a header while another 
%
may not (depending on when each node saw an equivocating header).
In order to output a consistent ledger that all honest nodes have downloaded, 
%
reaching consensus on just the header chain is not enough.
%
For nodes to later \emph{catch up} to the confirmed header chain's contents, the content must be available in the network. Unfortunately, verifying \emph{data availability} \cite{DBLP:conf/fc/Al-BassamSBK21} comes with several challenges.

Instead, we ensure that honest nodes agree on which blocks had an equivocation, and unilaterally blank their contents.
For this, when an honest node produces a new block header,
it adds an `equivocation proof' against any equivocating blocks among the recent blocks in its downloaded longest chain.
Specifically, the node picks from among the last $\keqproof$ block headers in its longest downloaded chain $\dC$, block headers $\Chain'$ for which the node has seen an equivocating block header $\Chain'$,
%
%
and there is no equivocation proof against it in any block header in $\dC$.
The node then creates an equivocation proof which consists of the two block headers $\Chain$ and $\Chain'$ and adds the equivocation proof to the header of the block that it creates (\myalgref{alg:posequivblank-lc}{loc:posequivblank-lc-construct-eq-proof}).

The deadline $\keqproof$ for adding equivocation proofs exists so that the adversary cannot release an equivocation after its block has been confirmed, and force honest nodes to then blank the content for that block, thereby altering the ledger.
The deadline also keeps the size of equivocation proofs in a header limited.
We also don't want an equivocation proof to be repeated in several headers in a chain.
%
Therefore, a block header $\Chain$ is considered invalid if it contains an equivocation proof against a block not in the prefix of $\Chain$, a block more than $\keqproof$ blocks above $\Chain$, or contains an equivocation proof that has already been proven in the prefix of $\Chain$ (\myalgref{alg:posequivblank-hdrtree}{loc:posequivblank-hdrtree-valid-eq-proof}).


\paragraph{Ledger Construction in \sapos}
%
To create the ledger at the end of \timeslot $t$, an honest node takes all blocks on its longest header chain that 
%
are $\confDepth$-deep,
then blanks the contents of any block 
%
against which there is an equivocation proof in
a block header following it
(\myalgref{alg:posequivblank-lc}{loc:posequivblank-lc-confirmation-blanking}).



%
%
    %
    
    %
    
    %
    %
    %
    
    %
    
    %
    %
    %
    
    %
    
    %
    %
    
    %
    
    %
    
    %
    
%


%
%


\subsection{Security Theorem}
\label{sec:pos-result}

Recall that the analysis in \cref{sec:analysis-details-many-pps-one-cps} uses four properties of the download rule. It is easy to see that with the addition of \equivocationremoval, the `download longest header chain' rule satisfies these properties in PoS.
The \equivocationremoval rule in \sapos clearly satisfies the property that each honest node never downloads the same block twice \ref{item:good-download-rule-no-repeat}, and downloads at most one block from each \BPO \ref{item:good-download-rule-one-per-bpo}.
The rule will never prohibit download of an honest block because it has no equivocations, and blocks in its prefix will either be downloaded or blanked.
Moreover, the rule never remains idle as long as there are block headers remaining with $\BLOCKUNKNOWN$ content \ref{item:good-download-rule-honest-block}.
Finally, \sapos does not spend bandwidth on any more blocks than the base download rule does, and since the base download rule $\dlrulelong$ does not download blocks before the most recent \sltcp (\cref{prop:download-or-spend-budget}), the rule with \equivocationremoval also does not \ref{item:good-download-rule-cutoff}.
This means that the analysis of \cref{sec:analysis-details-many-pps-one-cps} works for \sapos.
%
Just like in PoW, this 
%
leads to liveness and consistency of the confirmed header chains of all honest nodes.
Therefore, to ensure consistency of the ledger, we only need to show that the ledger construction process 
%
in \sapos
retains consistency. That is, if one honest node blanks the content of a block in its ledger, then all honest nodes do. Conversely, if one honest node does not blank the content for a block in its ledger, no honest node does. Proof details are in \cref{sec:appendix-pos-proofs}.

%
%
%
    
%
    
%
    
%
%


\begin{theorem}
\label{thm:safety-and-liveness-pos}
%
%
%
%
%
For all $\beta < 1/2$,
$\goodsepbw \in \IN$,
and $\blkrateslot, \slotduration$
satisfying
\begin{IEEEeqnarray}{C}
    %
    %
    \frac{\goodsepbw}{16} \frac{(2\probGood-1)^2}{\probGood} > \frac{\blkrateslot}{1-e^{-\blkrateslot}},
    \probGood = (1-\beta)e^{-\frac{\blkrateslot}{\slotduration} \left(\DeltaHeader + \goodsepbw/\bwtime\right)}, \IEEEeqnarraynumspace
\end{IEEEeqnarray}
there exists $\keqproof, \confDepth = \Theta(\kappa^2)$
such that the \sapos protocol
$\protocolPoSEquivBlank$ 
with
the download rule $\dlrulelong$,
%
%
%
%
%
is secure
with liveness latency $\TliveReal = \Theta(\kappa^2)$ \timeslots
over a time horizon of $\Khorizon = \poly(\kappa)$ block productions.
%
\end{theorem}
By choosing $\blkrateslot, \slotduration \to 0$ such that $\blkrateslot/\slotduration = \blkratetime$ (small slot approximation), we get the same security region as PoW, which is shown in \cref{fig:comparison-bddelay-bdbandwidth}. However, in PoS, we do have the additional freedom to choose a larger $\slotduration$, offering a potentially larger set of secure parameters.
The exact confirmation depth and liveness latency are larger for \sapos than for PoW NC (details in \cref{sec:appendix-pos-proofs} and \fullVersionRef{\cref{sec:appendix-full-version-proofs-pos}}).


%
%
%
%
%
%
%
%
%
%
%
%

%

%

