
\section{Simulation Details}
\label{sec:attacks-details}

%

Nodes in our simulation generate blocks in a Poisson process with rate proportional to their mining power. We assume the mining difficulty is fixed, and do not include any adjustment by a difficulty adjustment algorithm (DAA). In fact, DAAs tend to worsen processing problems as they increase the block creation rate if the chain does not grow fast enough---which in turn requires more download from nodes. 

Nodes process blocks one at a time according to the priority dictated by the processing policy, at a rate determined by their capacity. They are allowed to preempt their current task if new information (headers that are published, blocks that they mined) presents them with a higher priority target. Since queues can grow large if nodes do not manage to process all blocks in a timely manner, we maintain priority queues of bounded size (typically 100) and evict low priority tasks from the queue as needed.
%
As preemption of downloads may cause nodes to alternate between downloads, we 
%
allow nodes to retain partial work in an LRU cache of size 10.
%

Except where we note otherwise, headers are assumed to propagate instantly in the simulations.
To simulate an idealized bounded delay network, we set the header propagation delay to $\Delta$ and the capacity of each node to be $\infty$.
Block headers in both PoW and PoS contain the relevant lottery information which can be easily validated. We therefore assume the adversary never publishes headers it did not actually mine.

To remain close to the theoretical analysis, we model all processing tasks as dependent only on the resources available to the node itself. In reality, things are much more complex: nodes typically propagate blocks in a P2P network, which means both the overlay network topology and the underlying internet topology both greatly impact block download rates and performance. 
%
Our simplified setting allows us to focus more on the congestion effects in isolation from the effects of topology and other P2P related issues.

%
%
%

%
%

%

%
%
%

%

%



%
%

%

%
%
%
%
%
%
%

%

%
%

%
%





%

%

%

%



%

%


