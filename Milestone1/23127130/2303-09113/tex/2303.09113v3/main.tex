\pdfoutput=1
%
%
\documentclass[sigconf,nonacm]{acmart}
%
%
%
%
%
%
\setcopyright{none}
\settopmatter{printacmref=false}
%
%
%
%
%
%
%


%
%
%
%
%
%
%
%

%
%
%
%
%


\usepackage{import}
\import{./lib/}{boilerplate_pdflatex.tex}
\import{./lib/}{colors.tex}
\import{./lib/}{tikzpgfplot.tex}
\import{./lib/}{tikzpgfplot_parulalinestyles.tex}
\import{./lib/}{tikzpgfplot_plotstyles.tex}
\import{./lib/}{tikzpgfplot_blockchainstructures.tex}
\import{./lib/}{tables.tex}
\import{./lib/}{algorithms.tex}
%
\import{./lib/}{lateplate_pdflatex.tex}

\usepackage{color}
\renewcommand\UrlFont{\color{blue}\rmfamily}

\def\eg{\emph{e.g.}}
\def\Eg{\emph{E.g.}}
\def\ie{\emph{i.e.}}
\def\Ie{\emph{I.e.}}
\def\etal{et al.}

\newcommand{\m}{\mathbf{m}}



%

%
%
%
%
%





%
%
%
%
%
%
%


%
%
%
%
%
%
%
%
%
%
%
%
%
%


%
%
%
%
%
%
%
%
%
%
%
%
%
%
%
%
\title[Security--Throughput Tradeoff of Nakamoto Consensus under Bandwidth Constraints]{Security--Throughput Tradeoff of Nakamoto Consensus\\under Bandwidth Constraints}
%



%
%

%
%
%
%
%
%
%
%
%
%
%
%
%
%
%
%
%
%
%
%
%
%
%



\author{Lucianna Kiffer}
\email{lkiffer@ethz.ch}
\affiliation{\country{}}
%
\author{Joachim Neu}
\email{jneu@stanford.edu}
\affiliation{\country{}}
%
\author{Srivatsan Sridhar}
\email{svatsan@stanford.edu}
\affiliation{\country{}}
%
\author{Aviv Zohar}
\email{avivz@cs.huji.ac.il}
\affiliation{\country{}}
%
\author{David Tse}
\email{dntse@stanford.edu}
\affiliation{\country{}}

\thanks{LK, JN, SS and AZ are listed alphabetically.}


%
%
%
%
%
%
%
%
%
%
%
%
%
%
%
%
%
%


%
\newcommand{\gitSourceUrl}[0]{\url{https://github.com/avivz/finitebwlc}}
%

%
%

%


%
\makeatletter
\def\ps@headings{%
\def\@oddhead{\mbox{}\scriptsize\rightmark \hfil \thepage}%
\def\@evenhead{\scriptsize\thepage \hfil \leftmark\mbox{}}}
\makeatother
\pagestyle{headings}



\setlength{\floatsep}{0.65\baselineskip}
\setlength{\textfloatsep}{1.15\baselineskip}
%
%

%
%

\widowpenalty=0
\clubpenalty=0
%
%

%
%
%

%
%
%
%
%


%
%

%
\setcounter{secnumdepth}{3}
\renewcommand\thesubsubsection{\thesubsection.\arabic{subsubsection}}
\crefname{section}{Sec.}{Secs.}
\Crefname{section}{Sec.}{Secs.}
\crefname{appendix}{App.}{Apps.}
\Crefname{appendix}{App.}{Apps.}
\crefname{subsection}{Sec.}{Secs.}
\Crefname{subsection}{Sec.}{Secs.}
\crefname{subappendix}{App.}{Apps.}
\Crefname{subappendix}{App.}{Apps.}
\crefname{subsubsection}{Sec.}{Secs.}
\Crefname{subsubsection}{Sec.}{Secs.}
\crefname{subsubappendix}{App.}{Apps.}
\Crefname{subsubappendix}{App.}{Apps.}
\crefname{subsubsubsection}{Sec.}{Secs.}
\Crefname{subsubsubsection}{Sec.}{Secs.}
\crefname{subsubsubappendix}{App.}{Apps.}
\Crefname{subsubsubappendix}{App.}{Apps.}

%
\newcommand{\myparagraph}[1]{\smallskip\noindent\textbf{#1.}~~}

%
%
%
%
%
%
%
%
%
%
%
%
%
%
%

%
%
\renewcommand{\floatpagefraction}{0.9}
\renewcommand{\topfraction}{0.9}
\renewcommand{\textfraction}{0.1}

%
\setlist[itemize]{topsep=0.25em}
\setlist[enumerate]{topsep=0.25em}

%
\setcounter{tocdepth}{4}

%
\usepackage{thmtools}

%
\newcommand\blfootnote[1]{%
  \begingroup%
  \renewcommand\thefootnote{}\footnote{#1}%
  \addtocounter{footnote}{-1}%
  \endgroup%
}




\begin{document}
%
\begin{abstract}
    %
    %
    %
    %
    %
    %
    %
    For Nakamoto's longest-chain consensus protocol, whose proof-of-work (PoW) and proof-of-stake (PoS) variants power major blockchains such as Bitcoin and Cardano,
    %
    %
    %
    %
    %
    we revisit the classic problem of the security--performance tradeoff:
    Given a network of nodes with limited capacities, against what fraction of adversary power is Nakamoto consensus (NC) secure for a given block production rate?
    %
    State-of-the-art analyses of Nakamoto's protocol
    fail to answer this question
    because their \emph{bounded-delay} model
    does not capture realistic constraints such as 
    limited communication- and computation-resources.
    We develop a new analysis technique
    to prove
    a refined security--performance tradeoff
    for PoW NC in a
    \emph{bounded-bandwidth}
    model.
    In this model, we show that, in contrast to the classic bounded-delay model, Nakamoto's private attack is no longer the worst attack, and a new attack strategy we call the \emph{\teaserattack}, that exploits the network congestion caused by limited bandwidth, is strictly worse.
    %
    In PoS, equivocating blocks can exacerbate congestion, making 
    %
    traditional PoS NC 
    %
    insecure 
    %
    except at very low block production rates. To counter such equivocation spamming, 
    we present a 
    variant of 
    %
    PoS NC
    %
    we call \emph{\ProtMid} (\emph{\ProtShort}), 
    %
    which achieves the \emph{same resilience} as PoW NC.
    %
    %
    %
    %
    %
    %
\end{abstract}

\maketitle
%
\begingroup%
\renewcommand\thefootnote{1--3}%
%
\endgroup%

\section{Introduction}

The ability to reason about plans is critical for performing long-horizon tasks \citep{erol1996hierarchical, sohn2018hierarchical, sharma-etal-2022-skill}, compositional generalization \citep{corona-etal-2021-modular} and generalization to unseen tasks and environments \citep{shridhar2020alfred}.
Consider a simple long-horizon planning scenario where a robot is tasked with preparing a meal and serving it on the table. 
This presents a non-trivial planning problem since the agent needs to understand the sequence of operations required to perform the task and search for the relevant objects in the unfamiliar environment by interacting with various objects. %



Large language models have been recently shown to possess commonsense knowledge about the world such as object affordances and physical dynamics \citep{ouyang2022training,chowdhery2022palm}.
Early approaches considered text based environments and fine-tuned PLMs to predict actions given the history of past observations and actions \citep{jansen-2020-visually,micheli-fleuret-2021-language,yao-etal-2020-keep}.
Recent work has used this ability to reason about plans from text instructions in simulated household environments with simplifying assumptions such as text-only environment observations or feedback \citep{huang2022language,ahn2022can,li2022pre,logeswaran-etal-2022-shot}.


We focus on \emph{visually grounded planning} with PLMs --- the ability to adapt plans based on interaction and visual feedback from the environment.
While PLMs have strong planning commonsense priors, predictions from a PLM may not be directly realizable in the environment since the observation and action spaces are unknown.
This requires \emph{grounding} the PLM in the environment and adapting it to observe visual feedback, which is highly non-trivial.
Some prior works assume the availability of a pre-trained affordance function \citep{ahn2022can} or a success detector \citep{mirchandani2021ella}.
Notably, SayCan \citep{ahn2022can} completely decouples the PLM from observation information by selecting actions that have both high affordability (through a pre-trained affordance model) and high PLM likelihood.
Although this partially addresses the grounding problem, the use of visual feedback for action affordance alone is limited.
Often an agent must choose one of many affordable actions using information from observations.
For example, a driving agent should re-navigate and possibly turn around when encountering a ``road closed'' sign, but both turning around and driving forward are indistinguishable to SayCan because they are both affordable and the PLM is blind to observations.

Another workaround explored in prior work is translating the information in the visual observations to text using a pre-trained captioning system \citep{shridhar2021alfworld,huang2022language}.
However, it can be difficult to faithfully describe an image in words and information is lost in this inherently noisy process, which limits the information available to the planner.



Recent work shows that PLMs can be adapted for various natural language tasks by inserting tunable embeddings or soft prompts at the input of the PLM (also called prompt tuning or prefix tuning)~\citep{li-liang-2021-prefix,lester-etal-2021-power}.
This approach also extends to multi-modal understanding tasks such as image captioning \citep{mokady2021clipcap} and VQA \citep{tsimpoukelli2021multimodal} where images are encoded as soft prompts and finetuned for the target task.
Transformer based architectures have also been successfully applied to offline Reinforcement Learning in recent work \citep{chen2021decision,janner2021offline,li2022pre,reid2022can}.

Taking inspiration from these works, we propose the simple approach of embedding visual observations (`visual prompts') and \textit{directly inserting them as PLM input embeddings}.
The visual encoder and PLM are jointly trained for the target task, an approach we call \textbf{\oursfull}~(\ours).
By teaching the PLM to use observations for planning in an end to end manner, we remove the dependency on external data such as captions and affordability information that was used in prior work.
We show that this simple approach performs better than prior PLM-based planning approaches on two embodied planning benchmarks based on ALFWorld~\citep{shridhar2021alfworld} and Virtualhome~\cite{puig2018virtualhome}.



\section{The \TeaserAttack}
    \label{sec:experiments}

\import{./figures/}{fig-attack-race-concept.tex}

We begin by exploring a strategy that the attacker can adopt which forces honest nodes to waste capacity on blocks that do not contribute to chain growth. This strategy demonstrates that the well-studied \emph{private attack} \cite{nakamoto_paper,dem20} is not the worst case behavior of the attacker, and that the previously established security bounds of the bounded-delay model do not hold in the bounded-capacity setting. 
We go on to simulate our \teaserattack and to show how it compares to the private attack (summarized in \cref{fig:attack-race-concept}%
). 
%
%
%
%
%

%
Previous analyses concluded that the private attack (\cref{fig:attack-race-concept}) is worst-case
based on the false assumption that delays, and hence the honest chain growth rate, do not depend on whether the adversary releases blocks and causes congestion. We exploit congestion to develop the \teaserattack.


\import{./figures/}{fig-attack-teaser.tex}

\import{./figures/}{fig-experiment-teaser.tex}


\myparagraph{Description of the \TeaserAttack}
\label{sec:teaser-attack}
%
The key idea in the \teaserattack (summarized in \cref{fig:attack-teaser}) is that the adversary can strategically time the release of blocks it had mined in order to take up some of the capacity of honest nodes.\footnote{While similar 
%
to the BDoS attack of \cite{mirkin2020bdos}, we note that while they exploit miner incentives to depress honest mining, our \teaserattack exploits network and processing congestion to attack safety.}
In a nutshell, while the adversary continues to mine a private chain, every time an honest node announces a block at a new height,
the adversary releases the headers of a segment of its longer withheld chain and the contents of only the first block.
Due to \rulelc scheduling, honest nodes prioritize processing blocks on the chain announced by the adversary.
%
Only after an honest node has processed the first adversary block and realizes that the content for the remaining blocks in the announced adversary chain segment are unavailable, does the \rulelc rule switch back to processing the newly created honest block.
Therefore, the adversary \emph{`teased'} the honest nodes to spend some of their resources processing the adversary chain,  but without actually gaining a longer chain of blocks compared to the chain they already possessed.
The result of this strategy is delayed processing of honest blocks that extend the longest honest chain. Processing is delayed by a factor of $2$ compared to the private attack.
This in turn results in more honest blocks forking, thus slowing down the honest chain growth rate 
(\cref{fig:experiment-teaser}~\ref{plt:experiment-teaser-attacker}) to $\blkratetimeGrowthTeaser < \blkratetimeGrowthSilent$.

\myparagraph{Conditions for success of the \teaserattack}
Formally, in both the private attack and the \teaserattack, the length difference between the adversary chain and the honest chain is a random walk \cite{dem20} which increases
at the rate $\blkratetimeAdv$
%
and decreases at the rate $\blkratetimeGrowth$.
%
If $\blkratetimeAdv > \blkratetimeGrowth$, the random walk has a positive drift, so
%
in the long run, the adversary chain will outgrow the honest chain indefinitely and the attack succeeds.
Conversely, if $\blkratetimeAdv < \blkratetimeGrowth$, the random walk has a negative drift and the attack will eventually fail. 
%
%
%
%
Thus, $\blkratetimeGrowth$
%
determines the fraction $\beta$ of total mining power that the adversary needs for the attack, \ie, the attack succeeds if
\begin{IEEEeqnarray}{c}
    \label{eq:chain-growth-rate-beta}
    \beta \triangleq \frac{\blkratetimeAdv}{\blkratetimeAdv+\blkratetimeHon} > \frac{\blkratetimeGrowth}{\blkratetimeGrowth+\blkratetimeHon}.
\end{IEEEeqnarray}
%
\import{./figures/}{fig-experiment-teaser-start3.tex}

Note that the \teaserattack requires the adversary to maintain a lead of at least two blocks with respect to the honest chain at all times (to proceed in steps (a), (e) in \cref{fig:attack-teaser}). If this fails, then the adversary must give up and try the attack again.
We show a sample plot of the adversary's lead for different mining rates in \cref{fig:experiment-teaser-start}. For a large enough adversary (if $\blkratetimeAdv > \blkratetimeGrowthSilent$), it is clear the lead has a positive drift and eventually stays positive. However, the \teaserattack succeeds even when the lead has a negative drift initially (\eg for $\blkratetimeGrowthTeaser < \blkratetimeAdv < \blkratetimeGrowthSilent$), as it only needs a random lucky short burst to kickstart step (a). The resulting congestion then decreases the average growth rate of the honest chain to $\blkratetimeGrowthTeaser$, and the adversary with mining power $\blkratetimeAdv > \blkratetimeGrowthTeaser$ can positively bias the random walk, thus eventually maintaining a positive lead, and succeed. We see this process in \cref{fig:experiment-teaser-start}~\ref{plt:teaser-medium2}: the adversary's lead rises and drops to zero a few times, causing the adversary to try again. However, eventually, the adversary manages
to maintain a permanent lead. On the other hand, when $\blkratetimeAdv < \blkratetimeGrowthTeaser$, the adversary's lead has a negative drift even after the congestion effects kick in (\cref{fig:experiment-teaser-start}~\ref{plt:teaser-weak2}), and therefore the \teaserattack is bound to run out of blocks and fail.


With the combined mining rate $\blkratetime \triangleq \blkratetimeHon + \blkratetimeAdv$ of honest nodes and adversary,
and the honest chain growth rates from \cref{fig:experiment-teaser}, we use \eqref{chain-growth-rate-beta} to calculate the adversary fraction $\beta$
required for each attack
and plot it in \cref{fig:comparison-bddelay-bdbandwidth}.


\myparagraph{Simulation details}
We simulate\footnote{Source code: \gitSourceUrl} both the private attack and the \teaserattack
on a network of $100$ nodes.
%
Honest nodes collectively 
mine
blocks at a rate $\blkratetimeHon = 1$ block per second.
Each node has a limited processing rate of $\bwtime$ blocks per second.
Blocks consist of content (transactions) and a header
(PoW and parent block pointer).
Since the header contains all information necessary to verify the PoW,
nodes only process validly created blocks.
All honest nodes and the adversary can directly send valid block headers to one another.
Given a tree of valid block headers, nodes run the \emph{\rulelc policy}, \ie,
nodes attempt to process (download and verify) the first unprocessed block along the longest header chain.
If the longest chain is already processed, or if the content of any block on that chain is unavailable or invalid, then the rule considers the next longest header chain, and so on.
We further elaborate on the setup and other simulation details in \appendixRef{\cref{sec:attacks-details}}.

\myparagraph{Practical aspects of the \teaserattack}
The \teaserattack may not acutely break specific real-world implementations of PoW NC,
%
%
%
mainly because 
%
miners have over-provisioned capacity. 
%
%
%
%
Although the \teaserattack is specific to the \rulelc policy,
%
%
%
it is possible to devise attacks that exploit congestion
even for other policies
%
(see \appendixRef{\cref{sec:greedy-attack}}).
%
%
%
%
%
We also note that in basic PoS NC, the adversary can 
%
exacerbate the \teaserattack by 
equivocating the whole adversary chain 
%
every time before it releases a block.
%
%
As the attack goes on, the length of the new announced chain increases. This increases the time honest nodes spend processing this chain, and \emph{decelerates} the honest chain growth until it comes to a halt. As a result, 
%
the chain growth rate under the \PoSteaserattack is nearly zero (details in \appendixRef{\cref{sec:pos-teaser-attack}}).
The key takeaway from the \teaserattack is 
that exploiting congestion results in attacks that are more severe than the private attack, 
%
%
%
even in PoW where the block production is limited, and even when the block production rate is below the capacity of nodes. This invalidates the bounded-delay model's predictions and emphasizes the need for a security analysis under models that capture the effects of congestion, especially for protocols that aim to saturate physical performance limits.


\begin{figure}[tb]%
    \centering%
    \begin{tikzpicture}[]
        \footnotesize
        \begin{axis}[
                mysimpleplot,
                %
                xlabel={Capacity: $\bwtime$ [blocks per second]},
                ylabel={Honest chain growth\\rate ($\blkratetimeGrowth/\blkratetimeHon$)},
                xmin=0, xmax=2,
                ymin=0, ymax=0.7,
                height=0.5\linewidth,
                width=\linewidth,
                yticklabel style={
                        /pgf/number format/fixed,
                        /pgf/number format/precision=2
                },
                scaled y ticks=false,
                %
                %
                legend columns=2,
                %
                %
                %
                %
                %
                %
                %
                %
                %
                %
                %
                %
                %
            ]

            \addplot [myparula11, %
                    only marks, mark size=1.5pt] table [x=bandwidth,y=chain_growth] {figures/fig-experiment-teaser-noattacker-data.txt};
            \addlegendentry{No attack or private attack ($\blkratetimeSPV = 0$)};
            \label{plt:experiment-teaser-spv-noattacker};

            \addplot [myparula73, mark size=1.5pt,%
            only marks] table [x=bandwidth,y=chain_growth] {figures/fig-experiment-teaser-activeattacker-data.txt};
            \addlegendentry{\Teaserattack ($\blkratetimeSPV = 0$)};
            \label{plt:experiment-teaser-spv0-attacker};

            %
            %
            %
            %
            %
            %
            %
            %
            %
            %

            \addplot [
            myparula45, 
            mark size=1.5pt,
            only marks
            ] table [x=bandwidth,y=chain_growth] {figures/fig-experiment-teaser-spv50new-activeattacker-data.txt};
            \addlegendentry{\Teaserattack ($\blkratetimeSPV = 0.5$)};
            \label{plt:experiment-teaser-spv50new-attacker};


        \end{axis}
    \end{tikzpicture}%
    \vspace{-0.5em}%
    \caption{%
    \Teaserattack in the presence of SPV miners, compared with \teaserattack and private attack without SPV miners (same as in \cref{fig:experiment-teaser}). 
    Total mining rate of SPV miners is $\blkratetimeSPV$ blocks per second.
    Total mining rate of honest miners is 
    %
    $\blkratetimeHon = 1$ block per second. 
    SPV miners are not counted as honest. 
    %
    %
    %
    %
    %
    %
    %
    %
    %
    %
    %
    \Teaserattack still succeeds with lower adversary power than private attack
(\cref{fig:experiment-teaser-spv}).
    }%
    
    \label{fig:experiment-teaser-spv}%
\end{figure}%



\myparagraph{Effect of SPV miners}
Rational miners in PoW NC face a \emph{verifier's dilemma}~\cite{verifier-dilemma,tuxedo,demystifying-incentives}:
there is no incentive to download and verify a block's content before mining to extend it.
Some so called \emph{SPV miners}
(named after simple-payment-verification clients who download only block headers)
mine empty blocks without verifying the parent block's content first,
and thus
%
%
get more time to mine, increasing their chances of being rewarded for mining the next block.
%
%
%
Since SPV miners are immune to congestion (as they do not process block content), how does their presence affect the \teaserattack?
Under the \teaserattack, SPV miners would mine on the adversary’s longer header chain (red block $3$ in \cref{fig:attack-teaser}(d)) without waiting for its contents. However, the remaining honest miners (who we assume still outnumber the SPV miners) still do not consider this chain valid (due to unavailable content). They continue mining on the honest chain, and would still be slowed down by the \teaserattack just as before.
We added SPV miners to our simulation and verified that the \teaserattack still succeeds with lower adversary power than the private attack
(\cref{fig:experiment-teaser-spv}).
%
%
Thus, the qualitative insight from the \teaserattack, that congestion enables worse attacks than the private attack, persists.



\section{Protocol \& Model}
\label{sec:modelprotocol}

%



%
We briefly recap Nakamoto consensus (NC)
%
and the bounded-ca\-pa\-ci\-ty model of~\cite{bwlimitedposlc}.
Detailed pseudocode of the protocol is provided in \appendixRef{\cref{sec:algos-reference-pseudocode}}.
Technical details about the model are provided in \appendixRef{\cref{sec:algos-reference-environment}}.
%
%
%
%
%
%
%
%
%
%
For ease of exposition, the execution features a \emph{static} set of $N$ \emph{equipotent} \emph{nodes}, each of which runs an independent instance of the protocol.
%
Temporary crash faults (`sleepiness') of nodes, heterogeneous distribution of hash power,
or difficulty adjustment
are left to be addressed with techniques from~\cite{backbone,sleepy,garay2017bitcoin}.
%
We are interested in the large system regime $N\to\infty$.
Nodes interact with each other and with the adversary $\Adv$ through an environment $\Env$ that models the network.
$\Adv$ and $\Env$ are summarized below.

%

\myparagraph{Nakamoto's Longest Chain Consensus Protocol}
%
%
%
%
%
%
%
%
%
%
%
%
%
%
For ease of analysis, we consider the protocol 
(pseudocode in \appendixRef{\cref{alg:generic-lc-protocol}})
to proceed in discrete \emph{\timeslots} of duration $\slotduration$.
%
Consider $\slotduration$ to be a small quantum of time where $\slotduration \to 0$.
At each \timeslot $t$, the protocol 
queries the PoW block production (`mining') oracle
(idealized functionality in \appendixRef{\cref{alg:hdrtree-pow}})
%
%
in an attempt to extend the \emph{longest processed chain} $\dC$ in the node's view with a new block 
of
%
pending transactions $\txs$.
Each block production attempt is committed to a parent block and block content,
%
and only a single block is produced when the attempt is successful.
Per \timeslot, each node can make one block production attempt that will be successful with probability $\blkrateslot/N$ where $\blkrateslot = \Theta(\slotduration)$, independently of other nodes and \timeslots.
%
If successful, the node disseminates both the resulting \emph{(block) header} $\Chain'$ and the associated \emph{(block) content} $\txs$ via the environment $\Env$ to all nodes.
%
%
Finally, the protocol identifies the $\confDepth$-deep prefix $\dC\trunc{\confDepth}$ containing all but the last $\confDepth$ blocks of $\dC$.
The transactions along $\dC\trunc{\confDepth}$ are concatenated to produce the \emph{output ledger} $\LOG{}{t}$.
%
%
%


When a node $p$ receives a new valid block header $\Chain$ from $\Env$ (push-based header broadcasting), 
%
then $p$ adds $\Chain$ to its \emph{header tree} $\hT$ 
%
and relays $\Chain$ to all other nodes via $\Env$.
Throughout the execution, the protocol requests from $\Env$
(pull-based content downloading)
the content 
for
%
block headers 
%
decided by a \emph{scheduling policy}.
%
As a concrete example, we use the \rulelc rule
(pseudocode in \appendixRef{\cref{alg:longest-header-chain-rule}})
in which
a node downloads content for the first block header with unknown content on the longest header chain it sees.
%
%
Once a block's content is received and verified by executing its transactions, 
%
the node makes it available to other nodes via $\Env$, and updates its 
%
$\dC$.


%
%
%
%
%
%
%
%
%
%
%
%
%
%
%


%
%
%
%
%
%
%
%
%
%
%
%
%
%
%
%



\myparagraph{Bounded-Capacity Network}
%
%
%
%
We borrow the bounded-capacity network model of~\cite{bwlimitedposlc} (see \appendixRef{\cref{fig:model}} for an illustration).
In this model, $\Env$ abstracts \emph{push-based flooding of `small' block headers} and \emph{pull-based downloading of `large' block contents} from peers.
%
Broadcasted block header chains 
%
are delivered by $\Env$ to every node,
%
%
%
with a per-node per-header delay determined by $\Adv$, up to a commonly known delay upper bound $\DeltaHeader$.
%
Block content made available for download
%
is kept by $\Env$ in what can be thought of as a `cloud'.
Nodes can request the content associated with a particular header.
If content matching the header is available, then it is delivered by $\Env$ to the node.
%
%
Content download and verification is subject to a per-node capacity constraint of~$\bwtime$.
%
Blocks have a fixed maximum size, hence $C$ is measured in blocks per second.
See \appendixRef{\cref{sec:algos-reference-environment}} for a more formal description of $\Env$.

The `cloud' captures key properties of pull-based
peer-to-peer 
%
downloading. At first, content matching a particular header might not be available (\eg, $\Adv$ produced a block and disseminated its header, but withheld its content). Later, such content can become available (\eg, $\Adv$ releases the content to one node). Thus, the `cloud' ensures neither data availability nor strong consistency of query outcomes, unlike stronger primitives such as
verifiable information dispersal
%
\cite{avid,avidfp,dispersedledger,semiavidpr}. However, once content for a header does become available, it is unique and remains available. This captures the header's binding commitment to the content, and the fact that honest nodes share content 
%
with peers. Requests for unavailable content do not count towards the processing budget.

Also note that the adversary can push additional headers and contents to nodes at will.
%
This models non-uniform capacity (higher than the lower bound $C$)
and non-uniform delay (lower than the upper bound $\DeltaHeader$)
across nodes (analogous to adversary delay up to maximum $\Delta$ in the bounded-delay model).


\myparagraph{The Adversary}
%
The \emph{static} adversary $\Adv$ chooses a set of nodes (up to a fraction $\beta$ of all $N$ nodes, where $\beta$ is common knowledge) to corrupt before the randomness of the execution is drawn and the execution commences. Uncorrupted \emph{honest} nodes follow the protocol at all times. Corrupted \emph{adversary} nodes have arbitrary computationally-bounded \emph{Byzantine} behavior, coordinated by $\Adv$ in an attempt to break consensus.
%
Among other things, the adversary can:
withhold block headers and contents, or release them late or selectively to honest nodes;
push headers and contents to nodes while bypassing the delay and capacity constraints;
break ties in the chain selection and schdeuling policy.
%
%
%
%
Note that all miners that deviate from the honest protocol (including crash faults and SPV miners) are modeled as adversary.



\myparagraph{Security}
%
%
For an execution of 
PoW NC
%
where every honest node $p$ at every \timeslot $t$ outputs a ledger $\LOG{p}{t}$, we recall the security desiderata.

\begin{itemize}
    \item \emph{Safety:}
          For all adversary strategies,
          %
          all \timeslots $t,t'$, and 
          %
          all honest nodes $p, q$ (same or different): $\LOG{p}{t}\preceq\LOG{q}{t'}$ or $\LOG{q}{t'}\preceq\LOG{p}{t}$.
    \item \emph{$\Tlive$-Liveness:}
          For all adversary strategies, if a transaction $\tx$ is received by all honest nodes by \timeslot $t$,
          then for every honest node $p$ and for all \timeslots $t' \geq t+\Tlive$, $\tx \in \LOG{p}{t'}$.
    %
    %
    %
\end{itemize}

Note that since blocks have a fixed maximum size, liveness is expected only if transactions are received at a bounded rate. The following definition captures this.

\begin{definition}
\label{def:env-bounded-tx}
    The environment $\Env$ is \emph{$(\tput,\Ttput)$-tx-limited}, if the cumulative size of all transactions received by honest nodes during any interval of $\Ttput$ \timeslots is at most $\tput\cdot\Ttput$ times the maximum block size.
\end{definition}

Liveness will be proved under transaction-limited environments. The parameter $\tput$ is thus the worst-case throughput ($\blkratetime$ being the best-case throughput).
The burstiness of transaction arrival is measured by $\Ttput$; large $\Ttput$ may increase confirmation latency $\Tlive$.

%
A consensus protocol is \emph{secure over time horizon $\Thorizon$ \timeslots with transaction rate $\tput$} iff for some finite $\Ttput,\Tlive$, for all $(\tput,\Ttput)$-tx-limited environments, it satisfies safety, and $\Tlive$-liveness 
with overwhelming probability\footnote{As is customary, we denote by $\kappa$ the security parameter. 
Event $\Event_{\kappa}$ occurs \emph{with overwhelming probability} if $\Prob{\Event_{\kappa}} \geq 1 - \negl(\kappa)$.
Here, a function $f(\kappa)$ is \emph{negligible} $\negl(\kappa)$, if for all $n>0$, there exists $\kappa_n^*$ such that for all $\kappa > \kappa_n^*$, $f(\kappa) < \frac{1}{\kappa^n}$.} over executions of time horizon $\Thorizon$ \timeslots.
The properties 
%
can also be redefined in terms of real-time units instead of \timeslots.

%
%
%
%
%
%
%
%
%



%
%

%

%
%
%
%
%
%
%
%
%

%
%
%

%
%
%
%
%
%
%
%
%
%
%
%
%
%
%
%
%
%
%
%
%
%
%
%


\section{Security Proof}
\label{sec:proof}

Due to space constraints, we
focus on the intuition for the proof.
%
%
%
The security theorem for PoW NC is \cref{thm:safety-and-liveness-pow}.
The detailed full proof is provided in \cref{sec:fullproof}.


%
\import{./}{04_analysis_02_definitions.tex}
\import{./}{04_analysis_03_analysis_overview.tex}
\import{./}{04_analysis_04_analysis_cps_stabilize.tex}
\import{./}{04_analysis_05_analysis_many_pps.tex}
\import{./}{04_analysis_06_analysis_many_pps_one_cps.tex}
\import{./}{04_analysis_07_pow.tex}


\section{Proof-of-Stake Nakamoto Consensus}

Nakamoto consensus has been adapted to proof-of-stake in protocols of the Ouroboros~\cite{kiayias2017ouroboros,david2018ouroboros,badertscher2018ouroboros} and Sleepy Consensus~\cite{sleepy,snowwhite} families.
The protocol is identical to what was described in \cref{sec:modelprotocol} and formalized in \appendixRef{\cref{alg:generic-lc-protocol}}, except for a few key differences.
The block production oracle for proof-of-stake (idealized in \appendixRef{\cref{alg:hdrtree-pos}}) behaves differently.
As in PoW, each node can make one block production attempt per \timeslot that will be successful with probability $\blkrateslot/N$, independently of other nodes and \timeslots%
%
\footnote{There may be multiple blocks in one \timeslot, as in
%
%
the Ouroboros~\cite{kiayias2017ouroboros,david2018ouroboros,badertscher2018ouroboros} and Sleepy Consensus~\cite{sleepy,snowwhite} protocols.}%
, modeling uniform stake.
In PoS, however, (even past) block production opportunities can be `reused' to produce multiple blocks with different parents and/or content, \ie, to equivocate. 
%
%
%

\subsection{\ProtLong (\ProtShort)}
\label{sec:sapos}

%

%


%
%
%

%

%
%
In the classic bounded-delay analysis, the tradeoff between 
%
$\beta$ and $\lambda$
%
%
is the same for 
%
PoW and PoS NC~\cite{dem20,tight_bitcoin},
because, conceptually,
%
NC security depends only on 
a 
%
race between
the honest chain and adversary chains.
%
%
%
%
%
Even in PoS,
the adversary cannot 
use
%
equivocations
%
%
%
to boost
its chain growth rate,
because blocks within one chain must be from strictly increasing \timeslots, \ie, different \BPOs.
%
%
Under bounded capacity, however, as observed in~\cite{bwlimitedposlc}, honest nodes may waste their limited capacity processing 
equivocations
%
rather than staying up-to-date with the longest chain.
Thus, blocks they produce may not contribute to honest chain growth.
%
%
%
As a result,
%
%
the honest chain growth rate decreases, 
and with it
%
%
PoS NC security~\cite{bwlimitedposlc} (compared to PoW).
%
%
%
%
The key idea in \ProtShort is to modify the scheduling policy of PoS NC 
such that per \BPO at most one block is processed.
This restores the one assumption of the bounded-capacity PoW NC analysis (\cref{sec:proof-analysis-many-pps-one-cps}~(P1))
that was previously violated in PoS NC due to equivocations.
With the modification of \ProtShort,
the analysis from \cref{sec:proof} carries over to PoS.
%
%
%
%
%
%
%
%
%

%
%
One 
may consider 
this
%
alternative:
%
defer content processing until after consensus has been reached on a header chain.
This, however, requires
%
ensuring
%
%
%
that the contents belonging to headers
will be available for download.
%
%
%
Sampling-based approaches~\cite{DBLP:conf/fc/Al-BassamSBK21}
to check \emph{data availability}
come with various challenges~\cite{paradigmdas}
and
%
VID-based
approaches~\cite{dispersedledger,poar}   %
%
%
do not
%
%
scale to the large $N$ 
%
found in PoS NC.
%

%
%
%




\subsubsection{Protocol}
\label{sec:sapos_protocol}

%

%
%
%
%
%
%
%

%
%


\ProtShort is PoS NC (\cf\cite{kiayias2017ouroboros,sleepy}),
%
with the following modifications.

\myparagraph{The Scheduling Policy in \ProtShort}
%
\ProtShort uses
%
any scheduling policy that is secure for PoW NC (such as \rulelc),
modified as follows:
%
a node does not process content for a header
(denoted by the corresponding header chain $\Chain$)
if it has seen 
another equivocating header
%
from the same \BPO as $\Chain$.
%
%
Instead,
%
the node \emph{pretends} that content was ``processed'' and sets it to be empty.
%
%
%
The node can then continue processing content for headers that extend $\Chain$, and these blocks will be candidates for the node's longest processed chain.
%
%
With only the above scheduling policy,
one honest node may process
the real
content for a header while another 
%
may 
%
set it to be empty
(depending on when each node saw an equivocating header).
In order to output a consistent 
transaction ledger,
%
%
reaching consensus on the header chain is 
no longer
%
enough.
%
%
Instead, we ensure that honest nodes also agree on which blocks had an equivocation, through equivocation proofs, 
so that they can consistently \emph{blank} their contents.


\myparagraph{Equivocation Proofs}
An equivocation proof consists of two headers $\Chain, \Chain'$ from the same \BPO.
Whenever a node produces a new block header extending its longest processed chain,
%
%
it 
includes an equivocation proof
for any header $\Chain$ among the last $\keqproof$ headers
%
(on the new block's prefix)
for which it has seen an equivocating header $\Chain'$
but no equivocation proof 
%
was
%
recorded on chain yet.
%
%
%
%
%
%
%
%
%
%
%
%


\myparagraph{Equivocation Proof Deadline}
The deadline $\keqproof$ for adding equivocation proofs 
ensures
%
that 
%
%
%
the adversary cannot 
use equivocations or equivocation proofs
to make honest nodes blank the content of an old block
whose transactions they have already confirmed.
%
%
%
%
%
%
%
%
%
%
%
A header $\Chain$ is thus \emph{invalid}
if it contains an equivocation proof against 
a block 
that is 
not within $\keqproof$ blocks above $\Chain$.
%
%
%


\myparagraph{Ledger Construction in \ProtShort}
%
%
At the end of each \timeslot,
each node confirms 
%
%
all blocks on its longest processed chain that 
%
are $\confDepth$-deep,
except it blanks
%
%
the contents of 
blocks
%
%
against which there is an equivocation proof
on chain.
%
%
%




\subsubsection{Security Proof}
\label{sec:security}

%

The scheduling policy of \ProtShort ensures that,
just like in PoW NC,
honest nodes process at most one block per \BPO.
%
This eliminates additional block processing delays caused by equivocations, allowing the honest chain growth rate
to match that of PoW NC.
%
%
Given this, the security proof of PoW NC in \cref{sec:proof}
can be adapted to \ProtShort to 
%
show that
the $\confDepth$-deep \emph{header chains} of all nodes are consistent.
%
%
%
%
%
%


To ensure that their \emph{ledgers} are consistent, and complete the security proof, we need two more steps.
First, liveness of \ProtShort follows easily because the contents of blocks produced by honest nodes will never be blanked.
%
Second, for safety,
%
we show (a) honest nodes have processed the content for all blocks against which there is no equivocation proof on chain (these blocks must not be blanked), and (b) honest nodes blank content in their ledger consistently, that is, any honest node blanks the contents of a block in its ledger iff all honest nodes do so.
%
%
%
We prove (a) and (b) in \cref{thm:safety-and-liveness-pos} by choosing appropriate values for $\confDepth$ and $\keqproof$.




%
%
%
%
%
%
%
%
%




%
%
%
%
%
%
%
%
%
%
%
%
%
%
%
%
%
%

%
%
%
%
%
%
%
%
%
%
%
%
%
%
%
%
%
%
%
%
%
%
%
%
%
%
%
%
%
%
%
%
%
%
%
%
%
%
%
%
%
%
%



Since the analysis of PoW NC from 
\cref{sec:proof} (details in \appendixRef{\cref{sec:fullproof}}) applies to \ProtShort as well, \cref{lem:cps-stabilize-informal} (\sltcps stabilize the longest processed chains of all nodes) and \cref{lem:many-pps-one-cps-informal} (\sltcps recur) hold for \ProtShort.
%
Thus, \eqref{pow-max-tp} also determines the parameters under which \ProtShort is secure, \ie,
%
%
%
%
%
%
%
the security--throughput tradeoff of \ProtShort.%
\footnote{Technically, since PoS protocols run in \timeslots of fixed duration, unlike PoW, $\slotduration$ must match the \timeslot duration. If $\slotduration$ is small relative to the block production and processing times (such as $1$ second in Cardano), we can still use the approximation $\slotduration \to 0$, just like in PoW. We calculate the parameters for general $\slotduration$ in \appendixRef{\cref{sec:fullproof-analysis-many-pps-one-cps}}.}
In \cref{fig:comparison-bddelay-bdbandwidth-pos},
we plot the solutions of \eqref{pow-max-tp} with 
$\bwtime = 1$ and
$\DeltaHeader \approx 0$ (approximation for block content much larger than headers).
Since \eqref{pow-max-tp} does not depend on $\kappa$, for any given $\beta$, the block rate $\lambda$ is non-vanishing.
Only latency scales with $\kappa$, similar to PoW NC.
%
%
%

%

In both \cref{thm:safety-and-liveness-pos} 
(for \ProtShort) and \cref{thm:safety-and-liveness-pow} (for PoW NC), we prove an upper bound on the confirmation latency that scales with the security parameter $\kappa$ as $O(\kappa^2)$.
Concretely, our bound on \ProtShort's latency (\cref{thm:safety-and-liveness-pos}) is $3\times$ our bound for PoW NC (\cref{lem:safety-and-liveness-comb-pow}).


\begin{theorem}
\label{thm:safety-and-liveness-pos}
For all $\beta < 1/2$,
%
$\bwtime$, $\DeltaHeader$, $\blkrateslot$, $\slotduration$
satisfying \eqref{pow-max-tp},
there exists $\keqproof, \confDepth = \Theta(\kappa^2)$
such that the \emph{\ProtShort protocol
%
%
%
%
%
%
%
%
is secure}
with
transaction rate $(1 - 2\beta)\blkratetime$,
confirmation latency $\Tlive = \Theta(\kappa^2)$ \timeslots
over a time horizon of 
%
$\Thorizon = \poly(\kappa)$.
%
\end{theorem}

\begin{proof}
\import{./figures/}{fig-pos-safety-proof.tex}

%

Set $\confDepth \triangleq 6\Kcp + 1, \keqproof \triangleq 4\Kcp$.
%
Denote the longest processed chain of node $p$ at \timeslot $t$ as $\dC_p(t)$ and its $\confDepth$-deep prefix as $\dC_p(t)\trunc{\confDepth}$.
\emph{Safety} holds if the following three properties hold for all \timeslots $t \leq t'$ and for all honest nodes $p,q$:
%
(1) $\dC_p(t)\trunc{\confDepth} \preceq \dC_q(t')\trunc{\confDepth}$ or $\dC_q(t')\trunc{\confDepth} \preceq \dC_p(t)\trunc{\confDepth}$.
%
(2) If $b \in \dC_p(t)\trunc{\confDepth}$ and there is no equivocation proof in a block header following it, then node $p$ must have processed the content of $b$ before \timeslot $t$.
%
(3) 
%
If $b \in \dC_p(t)\trunc{\confDepth}$ and $b \in \dC_q(t')\trunc{\confDepth}$, then $p$ blanks the content of $b$ in $\LOG{p}{t}$ iff $q$ blanks it in $\LOG{q}{t'}$.

Consider an arbitrary block $b_i$ (produced at some \iindex $i$) that is confirmed by an honest node $p$ at \timeslot $t$, \ie, $b_i \in \dC_p(t)\trunc{\confDepth}$.
Since $b_i$ is $\confDepth$-deep, there must have been at least $6\Kcp + 1$ \iindices after $i$.
%
Due to \cref{lem:many-pps-one-cps-informal}, there must have been at least three \sltcps $j,k,l$ after \iindex $i$.
Due to \cref{lem:cps-stabilize-informal}, the blocks produced at these \iindices, $b_j,b_k,b_l$ are in $\dC_q(t')$ for all $t' \geq t$ and for all $q$ (see \cref{fig:pos-safety-proof}).
Therefore, $\dC_p(t)\trunc{\confDepth} \preceq \dC_q(t')$, and from this, we can prove (1).

To prove (2), suppose that node $p$ did not process the content of block $b_i$.
Since block $b_j$, and hence also $b_i$, is in every honest node's longest processed chain at \timeslot $t_{k+1} - 1$ (\cref{lem:cps-stabilize-informal}), it must have been that $p$ saw an equivocation for $b_i$ before \timeslot $t_{k+1} - 1$ (otherwise it must have actually processed the content of $b_i$).
Due to synchrony, all honest nodes see the headers of $b_i$ and its equivocation.
Since the block $b_k$ is produced by an honest node, and $k < i + 4\Kcp = i + \keqproof$, $b_k$ must contain an equivocation proof against $b_i$ (see \cref{fig:pos-safety-proof}).

To prove (3), we show that while confirming the block $b_i$, either all nodes see an equivocation proof against $b_i$ or none of them do.
The latest that an equivocation proof against $b_i$ can be included is $\keqproof$ blocks below $b_i$.
Since $\confDepth > \keqproof + 2\Kcp$, due to \cref{lem:many-pps-one-cps-informal}, the \sltcp $l$ must have occurred after $b_i$ became $\keqproof$-deep and before it became $\confDepth$-deep (see \cref{fig:pos-safety-proof}).
Thus, for all $p$ and $t$, if $b_i \in \dC_p(t)\trunc{\confDepth}$, then $b_l \in \dC_p(t)$, hence all nodes agree on whether or not an equivocation proof was included.

\emph{Liveness} follows similarly to PoW NC: Within $2\Kcp$ \iindices, there are enough honestly produced blocks to include new transactions, and their contents will never be blanked. In $\Theta(\Kcp)$ \timeslots, these blocks will become $\confDepth$-deep and will be confirmed.
\end{proof}



%
%
%
%
%
%
%
%
%
%
%
%
%
%
%
%
%
%
%
%
%
%
%
%
%
%
%
%
%
%
%
%
%
%
%
%
%
%
%
%
%
%
%
%
%
%
%
%
%
%
%
%
%
%
%
%
%
%
%
%
%
%
%
%
%
%
%
%
%
%
%
%
%
%
%
%
%
%
%
%
%
%
%
%
%
%
%
%
%
%
%
%
%
%
%
%
%
%
%
%
%
%
%
%
%
%
%
%
%
%
%

\subsection{Handling Loss of Predictable Validity}
\label{sec:predictablevalidity}

\subsubsection{Predictable Transaction Validity}
\label{sec:predictablevalidity-transaction}

%
%
In UTXO-based chains like Bitcoin (account-based like Ethereum), a transaction is \emph{valid} iff its inputs are unspent (its execution succeeds and fees are paid).
%

\begin{definition}
\label{def:predictablevalidity-transaction}
    %
    A transaction has \emph{predictable validity} iff: 
    validity
    %
    at the time an honest node adds it to a block
    implies validity when that block gets executed.
\end{definition}

The blanking in \ProtShort leads to a loss of \emph{predictable transaction validity}. An honest block $B$ may include a transaction that depends on the contents of a previous block $A$ whose equivocations were not known at the time. 
After block $B$ is produced, the adversary could release an equivocation for the block $A$, forcing honest nodes to blank block $A$'s contents, which may invalidate the transaction in block $B$. Such invalidated transactions take up free space in honest blocks and lower the effective throughput (valid confirmed transactions) of the ledger.


%


We propose a simple solution to recover predictable validity for \ProtShort:
If nodes limit transactions they include in a block to those that don't depend on any \emph{recently changed} state, then they can be sure that all equivocations that could affect the validity 
%
of a transaction already have a corresponding equivocation proof included on chain. 
This is because at the time of creating a block, honest nodes \emph{have seen all transactions which will be executed}, however, \emph{not all transactions nodes have seen will be executed}. The following lemma follows easily.
%
%

\begin{lemma}
    \label{lem:pred-valid-1}
    If a node produces a block whose transactions do not share state with any transaction included in the last $\keqproof$ blocks, then the block satisfies 
    \cref{def:predictablevalidity-transaction}.
    %
\end{lemma}





\subsubsection{Predictable Fee Validity}
\label{sec:predictablevalidity-fee}

In practice, %
in popular DeFi-ecosystems, which consist of very interdependent transactions~%
%
\cite{guo2019graph,chen2020understanding}, it may not always be practical to limit the interaction between transactions. %
We propose instead %
preserving the minimum requirement that each transaction \textit{pays its fee}, regardless of the outcome of its execution. This guarantees that miners are compensated for space used in their blocks, and also makes it costly for the adversary to take up space with invalid transactions.

\begin{definition}
\label{def:predictablevalidity-fee}
    A transaction has \emph{predictable fees} 
    iff:
    ability to pay fees when an honest node adds it to a block
    implies ability to pay fees when the block executes.
    %
\end{definition}

In systems like Ethereum, transactions have a \emph{max gas} value set by the sender which limits the computation allowed by the transaction and ultimately its fee. We consider a protocol with this gas mechanism, as well as a base transaction cost that covers the block space taken up by the transaction. 
We introduce a notion of \emph{gas deposit accounts} to \ProtShort that can only be used for transaction fees (transactions internally do not have access to these accounts).
When a miner includes a transaction, it checks that the account funding the transaction has enough funds to cover the maximum gas, even if all transactions in its recent ancestor blocks make it to the blanked ledger and consume their maximum gas. Users thus need to maintain a balance proportional to the complexity and frequency of the transactions they make. 
%
%
%
We also require that any deposit to the account is not considered in the balance until $\keqproof$ blocks after the deposit transaction. 
Withdrawals can take place immediately, as direct transactions.
%
%

\begin{lemma}
    \label{lem:pred-valid-2}
    If a node produces a block whose transactions are funded by gas deposit accounts with sufficient balance (balance before $\keqproof$ blocks minus any fees since),
    then all transactions in the block satisfy 
    \cref{def:predictablevalidity-fee}.
    %
\end{lemma}

%
%

The solutions in \cref{sec:predictablevalidity-transaction,sec:predictablevalidity-fee} are complementary and could each be adopted as per-validator heuristics (\ie, not a consensus rule), or by the system based on the use-case (\eg, expected inter-dependency of transactions).


%
%



In this work, we present the simple yet effective Graph Attentive Vectors (GAV) link prediction framework. GAV relies on the idea of modeling simplified physical flow in spatial networks by updating vector embeddings in a constrained manner.
GAV achieves 97.99 to 99.44 AUC on the link prediction task, outperforming the previous state-of-the-art by an impressive margin on all metrics across multiple whole-brain vessel and road network datasets while requiring a significantly smaller amount of trainable parameters. This indicates the importance of developing link prediction algorithms tailored to flow-driven spatial networks.
%of developing link prediction approaches adopting considerations of known functional properties, such as physical flow., defined by the structural properties of the network.
GAV's imitation of the dynamics of physical flow represents a simplified concept, which is not entirely representative of physical principles from, \eg, fluid dynamics (see Fig.~\ref{fig:inter}). Future work should, therefore, aim to extend GAV's simplistic assumptions by incorporating different physical principles, such as conservation of mass and momentum, resulting in vector embeddings highly representative of physical flow in flow-driven spatial networks.
% to recover directionality lost in generation process.

% \paragraph{Limitations}
% one key limitation in our framework is that our flow-inspired embedding only considers the directional vector between nodes. It does not consider the actual flow-trajectory which can be quite different. E.g a vessel connecting two bifurcation points can have a large curvature. This motivates future work blablabla 

% We would like to particularly highlight GAV's limitations in this section. 

% more challenging benchmarks
% spatial networks are underexplored, since most graph networks dont contain spatial coordinates

% future work: experiment with pseudo-spatial networks by experimenting with energy-based layout functions (spring layout), try to make use of more hops in a more advanced manner; ensure that directionality is consistent across whole brain; make use of this direction info for downstream tasks (AV cls, prior for flow simulations); more advanced spatial network-specific labeling tricks; use directionality for mpn-layers


\section*{Acknowledgment}
%
We thank
%
Lei Yang, Mohammad Alizadeh,
Sundararajan Renganathan, David Mazières,
Ertem Nusret Tas,
Ghassan Karame,
Florian Tschorsch,
and
George Danezis
for fruitful discussions.
The work of LK, JN, and AZ
was conducted in part
during Dagstuhl Seminar \#22421.
JN is supported by the Protocol Labs PhD Fellowship.
JN and SS are supported by a gift from
the Ethereum Foundation and a research hub funded by Input Output Global Inc.
SS is suported by NSF grant CCF-1563098.
LK is supported by the
armasuisse Science and Technology CYD Distinguished Postdoctoral Fellowship.
AZ is supported by grant \#1443/21 from the Israel Science Foundation.
%


%
%
%

%
%

%
%
\bibliographystyle{ACM-Reference-Format}
\bibliography{references}


%
\appendix
\crefalias{section}{appendix}
\crefalias{subsection}{subappendix}
\crefalias{subsubsection}{subsubappendix}
\crefalias{subsubsubsection}{subsubsubappendix}

%
\input{A01_attacks_experiment_details.tex}
\iffalse
\section{\TeaserAttack Adversary Lead}
\label{sec:teasing-lead}

%

To demonstrate that the adversary in the \teaserattack eventually succeeds in maintaining a 2 block lead over the honest chain, we plot the adversary's lead for different adversary mining rates in \cref{fig:experiment-teaser-start}.
If the adversary has a very large mining power (if $\blkratetimeAdv > \blkratetimeGrowthSilent$), then it is clear that the adversary's lead eventually stays positive 
(recall the lead is a random walk with positive drift,
\cref{fig:experiment-teaser-start}~\ref{plt:teaser-strong2}).
However, the \teaserattack is stronger: it succeeds even when $\blkratetimeGrowthTeaser < \blkratetimeAdv < \blkratetimeGrowthSilent$. %
Though the random walk representing the adversary's lead initially has a negative drift, %
due to random fluctuations, %
it occasionally stays positive for short periods of time (\cref{fig:experiment-teaser-start}~\ref{plt:teaser-medium2}).
The occurrence of such events is enough for the \teaserattack to kickstart, because it
%
allows the adversary to create the setup (step (a)), and to repeat steps (b)--(e) of the \teaserattack (\cref{fig:attack-teaser}) at consecutive block heights. The resulting congestion then decreases the average growth rate of the honest chain to $\blkratetimeGrowthTeaser$. Thereafter, the adversary only needs mining power $\blkratetimeAdv > \blkratetimeGrowthTeaser$ in order to positively bias the random walk and thus eventually maintain a positive lead.
%
We see this process in \cref{fig:experiment-teaser-start}~\ref{plt:teaser-medium2}: the adversary's lead rises and drops to zero a few times, causing the adversary to try again. However, eventually, the adversary manages
to maintain a permanent lead.

On the other hand, when $\blkratetimeAdv < \blkratetimeGrowthTeaser$, the adversary's lead has a negative drift even after the congestion effects kick in (\cref{fig:experiment-teaser-start}~\ref{plt:teaser-weak2}), and therefore the \teaserattack is bound to run out of blocks and fail.






\section{Private Attack}
\label{sec:attacks-private}

On a high level, in the notorious so-called \emph{private attack} \cite{nakamoto_paper,dem20} (\cref{fig:attack-race-concept}), the adversary's objective is to
break the safety property of the blockchain
(\ie, de-confirm transactions).
For this purpose, 
the adversary
uses its hash power
to mine a chain that it keeps private (`adversary chain'),
with the goal of eventually
outgrowing the public longest chain built jointly by all honest nodes (`honest chain').
Once that happens, the adversary reveals its longer adversary
chain, thereby displacing the shorter honest chain as the longest
chain, and de-confirming transactions included in
the honest chain.
Note that if the adversary chain falls behind the honest chain,
the adversary restarts its attack from the tip of the current longest chain.

%
%
The success of the private attack %
%
depends crucially on
the growth rate 
of each of the chains.
The adversary's chain grows every time the adversary mines a block, and hence with the rate $\blkratetimeAdv$ of adversary block production.
The honest chain grows at a rate
$\blkratetimeGrowth$ which is generally
lower than
the total honest block production rate $\blkratetimeHon$,
due to \emph{forking} caused
by \emph{processing delays} of blocks.
%
\import{./figures/}{fig-attack-no.tex}
%
Specifically,
since honest nodes mine on their longest \emph{downloaded} chain,
when some honest node mines a new block,
the other nodes continue mining on the same tip as before
(\cref{fig:attack-no}(a))
until they have finished processing 
the newly mined block (\cref{fig:attack-no}(c)).
If another honest node mines a new block during this time,
it does not extend the longest chain 
(\cref{fig:attack-no}(b)).
Thus, only part of honest mining contributes to growing the honest chain,
and the honest chain growth rate $\blkratetimeGrowth$ is less than the honest block production rate $\blkratetimeHon$.

During the private attack, honest nodes undergo delays
only due to processing \emph{honest} blocks, because until completion of the attack, the adversary does not release any blocks.
%
As described in \cref{fig:attack-no} and as seen empirically in \cref{fig:experiment-teaser} (\ref{plt:experiment-teaser-noattacker} vs.\ \ref{plt:experiment-growth-delay}),
the honest chain growth rate $\blkratetimeGrowthSilent$
under private attack is approximated well by the growth rate that would occur
in an idealized network where all blocks are processed within $\Delta=1/\bwtime$ time
after they are announced
(the bounded-delay model).
Therefore, previous analyses' calculations of the adversary mining rates at which the private attack succeeds, even though based on the bounded-delay model, continue to hold
under bounded bandwidth.
\fi
\section{Other Congestion-Based Attacks}
\label{sec:general-attacks}


%
%


%
%
%
%
%
%
%
%


%
%
%
%
%
%

%




\subsection{\GreedyAttack}
%
\label{sec:greedy-attack}
%
The \teaserattack relied on the fact that the adversary could entice nodes with a long header chain that is later discovered to be unavailable for processing. It is natural in this case to consider adjusting the scheduling policy to one that prefers the proverbial `bird in the hand over two birds in the bush', \ie, to extend the blocks we already processed over the illusive promise of a longer chain that the adversary may withhold from us. 

%
\emph{\ruleGreedy policy.}\;\;
This policy prioritizes processing blocks that extend the chain a node has already processed. If a header of a block at height $h$ is announced, and we already have $h_i$ blocks from that chain,
%
we set the priority of the block to be $(h_i,h)$ and compare between the two priorities lexicographically.

While the \rulegreedy policy performs well at high processing rates, we unfortunately find that it performs poorly in the low processing rate regime. Specifically, if a fork in the chain occurs, and nodes are split evenly between the two alternatives, the fork may never resolve. This is because nodes extend their own chain, and prioritize processing on their side of the split while having insufficient processing power to catch up with the other alternative chain. A fork in the chain can result from a deliberate attack by an adversary that releases blocks selectively to different nodes, by a network split, or worse, by an unlucky timing of honest node mining events. In this case, the blockchain fails even for small adversaries. 
Importantly, a fork that never resolves is either a safety or a liveness failure, as no transaction on either side of the split can be safely accepted.

\import{./figures/}{fig-experiment-greedy.tex}

To demonstrate this scheduling policy in action, we simulate a network of 100 nodes that are split evenly between two partitions for only 15 seconds, \ie, for an expected time required to produce 15 blocks.%
\footnote{Such short splits are relatively easy to induce in reality (transient problems with Internet routing, denial-of-service on the network, etc.) and thus a practical scheduling rule must recover from such splits.}
Once the network split ends, the simulation continues for another 4000 seconds, allowing nodes the opportunity to 
converge on a chain.
%
We 
%
measure the height of the latest block all nodes agree upon. If nodes do not recover from the partition, 
this block will be the genesis and the liveness of the protocol has failed. Otherwise, nodes quickly agree on the main chain and the height of the latest agreed block is 
just a little behind the longest tip of the chain.

We simulate the evolution after a brief partition for both the \rulelc policy as well as for the \rulegreedy policy. Our results (\cref{fig:experiment-greedy}) show that in settings where capacity is greater than $1/2$, nodes manage to catch up with the chain and the rate of growth matches for both scheduling policies. In lower capacity settings, however, nodes never catch up.
Note that this attack requires no adversary mining,
yet the protocol is insecure (\cf \cref{fig:comparison-bddelay-bdbandwidth}(c)).
This is in stark contrast to the bounded-delay analysis
which suggests that the protocol retains security
against a non-mining adversary
%
at any capacity (\cf \cref{fig:comparison-bddelay-bdbandwidth}(a)),
and highlights again the need to study the security of blockchains at capacity.











\subsection{The \PoSTeaserAttack (PoS)}
\label{sec:pos-teaser-attack}

\import{./figures/}{fig-experiment-teaserequiv.tex}

\import{./figures/}{fig-attack-pos-teaser.tex}

In PoS, 
%
the adversary can greatly increase the network's processing load using equivocations. The \PoSteaserattack, described in \cref{fig:attack-pos-teaser}, uses equivocations to announce a whole new chain at every instance when the \teaserattack would have announced a single new block.
As the attack goes on, the length of the new announced chain increases. This increases the time honest nodes spend processing this chain, and \emph{decelerates} the honest chain growth until it comes to a halt. As a result,
in \cref{fig:experiment-teaser-pos}, 
the chain growth rate under the \PoSteaserattack is nearly zero.
%

As in the \teaserattack, the adversary starts by producing a private chain. Assuming the adversary's block production rate $\blkratetimeAdv$ is less than the honest chain growth rate before the attack ($\blkratetimeGrowthSilent$), the probability that the adversary produces a chain of length $k$ before the honest chain reaches length $k$ is $e^{-O(k)}$ \cite{nakamoto_paper,dem20}. This means that with probability $\epsilon$, the adversary eventually produces a private chain of length $k = O(\log(1/\epsilon))$, of which it can announce equivocations during the attack.
Since this chain is longer than the honest chain, it has higher scheduling priority.
It takes honest nodes $k/\bwtime$ time to process such a chain, during which time, honest nodes do not process blocks on the honest chain. So, any honest blocks produced within $k/\bwtime$ time after the first honest block at height $h$ do not grow the honest chain (\cref{fig:attack-teaser}(e)). If $\blkratetimeHon k/\bwtime$ is large, then there are many honest blocks that do not lead to chain growth, causing the chain growth rate $\blkratetimeGrowth$ to drop
(\cref{fig:experiment-teaser-pos}). 
As in the \teaserattack, if the adversary's block production rate $\blkratetimeAdv$ exceeds $\blkratetimeGrowth$, then the adversary succeeds in maintaining the number of block productions required for the attack to go on forever. This eventually slows honest chain growth to a halt. Thus, if $\blkratetimeHon k/\bwtime$ is large, \ie, $\blkratetimeHon = \Omega(1/k) = \Omega\left(\frac{1}{\log(1/\epsilon)}\right)$, then the attack succeeds with probability $\epsilon$.
%
%
%
\section{Protocol \& Model Details}
\label{sec:algos-reference}

\subsection{Nakamoto Consensus Pseudocode}
\label{sec:algos-reference-pseudocode}

\import{./figures/}{alg-generic-lc-protocol.tex}
\import{./figures/}{alg-hdrtree-pow.tex}
%
\import{./figures/}{alg-longest-header-chain-rule.tex}

Pseudocode of an idealized NC protocol $\protocol$ is provided in \cref{alg:generic-lc-protocol}.
Details of the PoW-based block production lottery, \ie, of production and verification of blocks, are abstracted through an idealized functionality $\FtreePoW$ whose pseudocode is provided in \cref{alg:hdrtree-pow} (\cf~\cite[Fig.~2]{sleepy}, \cite[Alg.~3]{bwlimitedposlc}).
Pseudocode for the \rulelc block download/processing rule $\dlrulelong$ is provided in \cref{alg:longest-header-chain-rule}.
%
%
Helper functions used in the pseudocode are detailed in \cref{sec:algos-reference-helperfunctions}.



\subsection{Helper Functions for Nakamoto Consensus Pseudocode}
\label{sec:algos-reference-helperfunctions}

\begin{itemize}
      \item $\operatorname{Hash}(\txs)$:\;\;
%
            %
            Cryptographic hash function to produce
            a binding commitment to $\txs$
            (modelled as a random oracle)

      \item $\Chain' \preceq \Chain$, $\Chain \succeq \Chain'$:\;\;
%
            %
            Relation that $\Chain'$ is a prefix of $\Chain$

      \item $\Chain \| \Chain'$:\;\;
%
            %
            Concatenation of $\Chain$ and $\Chain'$

      \item $\len{\Chain}$:\;\;
%
            %
            Length of $\Chain$

      \item $(\TRUE \text{ with probability $x$, else } \FALSE)$:\;\;
%
            %
            Bernoulli random variable with success probability $x$

      \item $\operatorname{prefixChainsOf}(\Chain)$:\;\;
%
            %
            Set of prefixes of $\Chain$, \ie, all $\Chain'$ with $\Chain' \preceq \Chain$

      \item $\operatorname{newBlock}(\mathsf{txsHash}\colon \operatorname{Hash}(\txs))$ and
      \\
      $\operatorname{newBlock}(\mathsf{time}\colon t, \mathsf{node}\colon P, \mathsf{txsHash}\colon \operatorname{Hash}(\txs))$:\;\;
%
            %
            Produce a new PoW and PoS block header with
            given parameters, respectively

      \item $\operatorname{txsLedger}(\TxsMap, \Chain)$:\;\;
%
            %
            Concatenates the block contents stored in $\TxsMap$ for the blocks along the chain $\Chain$, to obtain the corresponding transaction ledger
\end{itemize}

%




\subsection{Bounded-Bandwidth Model Environment $\Env$}
\label{sec:algos-reference-environment}

\import{./figures/}{fig-model.tex}

We study PoW NC (\cref{sec:algos-reference-pseudocode})
%
using the following model
%
for a network $\Env$ with finite bandwidth (\cref{fig:model}), and for the powers and limits of an adversary $\Adv$.

%
%
The environment $\Env$ initializes $N$ nodes and lets $\Adv$ corrupt up to $\beta N$ nodes at the beginning of the execution. Corrupted nodes are controlled by the adversary. Honest nodes run $\protocol$.
The environment maintains a mapping $\Env.\TxsMap$ from block headers to the block content (transactions). This mapping is referred to as the `cloud' in 
%
\cref{fig:model}.
%
$\Env$ also maintains for each node a queue of pending block headers
to be delivered after a delay determined by the adversary.
If $\Adv$ has not instructed $\Env$ to deliver a header $\DeltaHeader$ real time after it was added to the queue of pending block headers,
then $\Env$ delivers it to the node.
%

Honest nodes and $\Adv$ interact with $\Env$ via the following functions:
\begin{itemize}
      \item $\Env.\Call{broadcastHeaderChain}{\Chain}$:

            \noindent If called by an honest node, $\Env$
            enqueues $\Chain$ in the queue of pending block headers for each node, and notifies $\Adv$.
            %
            Then, for each node $P$, on receiving $\Call{deliver}{\Chain,P}$ from $\Adv$,
            or when $\DeltaHeader$ time has passed since $\Chain$
            was added to the queue of pending headers, $\Env$ triggers $P.\Call{receivedHeaderChain}{\Chain}$.

      \item $\Env.\Call{uploadContent}{\Chain, \txs}$:

            \noindent $\Env$ stores a mapping from the header chain $\Chain$ to the content $\txs$ of its last block by setting $\Env.\TxsMap[\Chain] = \txs$.
            $\Env$ only stores the content $\txs$
            if $\mathrm{Hash}(\txs) = \Chain.\mathsf{txsHash}$.

      \item $\Env.\Call{receivePendingTxs}{\null}$:

            \noindent $\Env$ generates a set of pending
            transactions and returns them.
    
    \item
        If node $P$ at \timeslot $t$ requests the content associated with a block header $\Chain$, $\Env$ acts as follows.
        %
        %
If $\Env.\TxsMap[\Chain]$ is set, then let $\txs = \Env.\TxsMap[\Chain]$ (if not, $\Env$ ignores the request).
%
If the request was received from an honest node $P$,
if $\Env$ has recently triggered $P.\Call{receivedContent}{.}$ at a rate below $\bwtime$,
%
then $\Env$ triggers  $P.\Call{receivedContent}{\Chain, \txs}$ (else, $\Env$ ignores the request).
If the request was received from $\Adv$, $\Env$ sends $(\Chain, \txs)$ to $\Adv$.
\end{itemize}

At all times, $\Adv$ can trigger $P.\Call{receivedHeaderChain}{\Chain}$
and $P.\Call{receivedContent}{\Chain, \txs}$
for honest nodes $P$ (bypassing header delay and bandwidth constraint in an adversarially chosen way).

%
%
%
%
%
%


\section{Full Security Proof}
\label{sec:fullproof}


%

This section provides a self-contained
proof of the argument
developed in \cref{sec:proof}.

Nodes are identified using $p, q$.
%
%
We distinguish between three notions of `time':
\emph{\Timeslots} of $\protocol$ are indicated by $r, s, t$.
\Timeslots in which one or more blocks are produced form a sub-sequence $\{t_k\}$, defined in \cref{sec:fullproof-definitions}.
\emph{\Iindices} into this sub-sequence are denoted by $i, j, k$.
%
The physical parameters of our model,
%
%
%
header propagation delay $\DeltaHeader$ and bandwidth $\bwtime$, as well as the mining rate $\blkratetime$, are specified in units of \emph{real time}.
%

We denote by $\dC_p(t)$ the longest fully downloaded chain of an honest node $p$ at the end of \timeslot $t$, and let $\len{b}$ denote the height of a block $b$. We use the same notation $\len{\Chain}$ to denote the length of a chain $\Chain$, define $L_p(t)=\len{\dC_p(t)}$ and $L_{\min}(t) = \min_p L_p(t)$
(where ``$\min_p$'' ranges only over honest nodes).
%

We denote intervals of \iindices (or \timeslots) as $\intvl{i}{j} \triangleq \{i+1,...,j\}$, with the convention that $\intvl{i}{j} \triangleq \emptyset$ for $j \leq i$.
We study executions over a finite horizon of $\Thorizon$ \timeslots (or $\Khorizon$ \iindices), and any interval $\intvl{i}{j}$ with $i < 0$ or $j > \Khorizon$ considered truncated accordingly.
The notation $\intvl{i}{j} \intvlg K$ (resp.\ $\intvlgeq, \intvll, \intvlleq, \intvleq$) is short for $j-i > K$ (resp.\ $\geq, <, \leq, =$).
In the analysis, we denote with upper-case Latin letters several random processes over \iindices (\eg, $\Xat{k}$) or \timeslots (\eg, $\Hat{t}$).
For any set $I$ of \iindices (analogously for \timeslots), we define $\Xat{I} \triangleq \sum_{k \in I} X_k$.

%
We denote by $\kappa$ the security parameter. An event $\Event_{\kappa}$ occurs \emph{with overwhelming probability} if $\Prob{\Event_{\kappa}} \geq 1 - \negl(\kappa)$.
Here, a function $f(\kappa)$ is \emph{negligible} $\negl(\kappa)$, if for all $n>0$, there exists $\kappa_n^*$ such that for all $\kappa > \kappa_n^*$, $f(\kappa) < \frac{1}{\kappa^n}$.





\import{./}{A04_analysis_01_probmod.tex}
\import{./}{A04_analysis_02_definitions.tex}
\import{./}{A04_analysis_03_analysis_overview.tex}
\import{./}{A04_analysis_04_analysis_cps_stabilize.tex}
\import{./}{A04_analysis_05_analysis_many_pps.tex}
\import{./}{A04_analysis_06_analysis_many_pps_one_cps.tex}
\import{./}{A04_analysis_07_pow.tex}


\section{Proof-of-Stake Model Details}
\label{sec:pos-model-details}

\import{./figures/}{alg-hdrtree-pos.tex}

Details of the PoS-based block production and verification are abstracted through an idealized functionality $\FtreePoS$ whose pseudocode is provided in \cref{alg:hdrtree-pow} (\cf~\cref{alg:hdrtree-pow}, \cite[Fig.~2]{sleepy}, \cite[Alg.~3]{bwlimitedposlc}).

As in PoW, each node can make one block production attempt per \timeslot that will be successful with probability $\blkrateslot/N$, independently of other nodes and \timeslots
(\myalgref{alg:hdrtree-pos}{loc:hdrtree-pos-blockproductionlottery})%
\footnote{There may be multiple blocks in one \timeslot, as in
%
%
the Ouroboros~\cite{kiayias2017ouroboros,david2018ouroboros,badertscher2018ouroboros} and Sleepy Consensus~\cite{sleepy,snowwhite} protocols.}%
, modeling uniform stake.
In PoS, however, (even past) block production opportunities can be `reused' to produce multiple blocks with different parents and/or content, \ie, to equivocate
(\myalgref{alg:hdrtree-pos}{loc:hdrtree-pos-leader,loc:hdrtree-pos-binding}).
%
%
%
%
%
%
%
%
%





%
%
%


%
%
%
%



%

%
%
%
%
%
%
%
%
%


\end{document}
