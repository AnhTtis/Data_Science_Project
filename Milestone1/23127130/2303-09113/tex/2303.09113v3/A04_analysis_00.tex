\section{Full Security Proof}
\label{sec:fullproof}


%

This section provides a self-contained
proof of the argument
developed in \cref{sec:proof}.

Nodes are identified using $p, q$.
%
%
We distinguish between three notions of `time':
\emph{\Timeslots} of $\protocol$ are indicated by $r, s, t$.
\Timeslots in which one or more blocks are produced form a sub-sequence $\{t_k\}$, defined in \cref{sec:fullproof-definitions}.
\emph{\Iindices} into this sub-sequence are denoted by $i, j, k$.
%
The physical parameters of our model,
%
%
%
header propagation delay $\DeltaHeader$ and bandwidth $\bwtime$, as well as the mining rate $\blkratetime$, are specified in units of \emph{real time}.
%

We denote by $\dC_p(t)$ the longest fully downloaded chain of an honest node $p$ at the end of \timeslot $t$, and let $\len{b}$ denote the height of a block $b$. We use the same notation $\len{\Chain}$ to denote the length of a chain $\Chain$, define $L_p(t)=\len{\dC_p(t)}$ and $L_{\min}(t) = \min_p L_p(t)$
(where ``$\min_p$'' ranges only over honest nodes).
%

We denote intervals of \iindices (or \timeslots) as $\intvl{i}{j} \triangleq \{i+1,...,j\}$, with the convention that $\intvl{i}{j} \triangleq \emptyset$ for $j \leq i$.
We study executions over a finite horizon of $\Thorizon$ \timeslots (or $\Khorizon$ \iindices), and any interval $\intvl{i}{j}$ with $i < 0$ or $j > \Khorizon$ considered truncated accordingly.
The notation $\intvl{i}{j} \intvlg K$ (resp.\ $\intvlgeq, \intvll, \intvlleq, \intvleq$) is short for $j-i > K$ (resp.\ $\geq, <, \leq, =$).
In the analysis, we denote with upper-case Latin letters several random processes over \iindices (\eg, $\Xat{k}$) or \timeslots (\eg, $\Hat{t}$).
For any set $I$ of \iindices (analogously for \timeslots), we define $\Xat{I} \triangleq \sum_{k \in I} X_k$.

%
We denote by $\kappa$ the security parameter. An event $\Event_{\kappa}$ occurs \emph{with overwhelming probability} if $\Prob{\Event_{\kappa}} \geq 1 - \negl(\kappa)$.
Here, a function $f(\kappa)$ is \emph{negligible} $\negl(\kappa)$, if for all $n>0$, there exists $\kappa_n^*$ such that for all $\kappa > \kappa_n^*$, $f(\kappa) < \frac{1}{\kappa^n}$.





\import{./}{A04_analysis_01_probmod.tex}
\import{./}{A04_analysis_02_definitions.tex}
\import{./}{A04_analysis_03_analysis_overview.tex}
\import{./}{A04_analysis_04_analysis_cps_stabilize.tex}
\import{./}{A04_analysis_05_analysis_many_pps.tex}
\import{./}{A04_analysis_06_analysis_many_pps_one_cps.tex}
\import{./}{A04_analysis_07_pow.tex}

