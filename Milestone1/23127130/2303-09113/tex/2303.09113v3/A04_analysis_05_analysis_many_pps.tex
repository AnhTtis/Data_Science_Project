\subsubsection{Probabilistic Pivots Are Abundant}
\label{sec:fullproof-analysis-many-pps}

Previous analyses of NC~\cite{sleepy,dem20} show that sufficiently long intervals contain at least \emph{one \sltpp} (\cref{fig:analysis-comparison-sleepy}(a)). This was enough for the bounded-delay analysis because in the bounded-delay setting, every \sltpp is also a \sltcp. 
However, in the bounded-bandwidth setting, not every \sltcp is a \sltpp, because not every \sltgood \timeslot results in 
growth of the longest downloaded chain of honest nodes (\cref{fig:analysis-comparison-sleepy}(b)).
%
Thus, existence of one \sltpp in every large interval is not enough to conclude existence of one \sltcp in every large interval. 
%
%
Instead,
we prove,
using a concentration bound on the number of \sltpps (\cref{prop:lower-tailbound-ppivots}),
that long intervals of \iindices in fact contain
\emph{a number of \sltpps proportional
to the interval length} (\cref{lem:many-pps}).
Then, in \cref{sec:fullproof-analysis-many-pps-one-cps}, we prove that out of those many \sltpps, at least one must also be a \sltcp, which allows us to continue with the safety and liveness proofs from~\cite{sleepy}.



%

The key challenge in proving that there are many \sltpps is that for two \iindices $k_1, k_2$, the events that $k_1$ is a \sltpp and that $k_2$ is a \sltpp are dependent, because both events depend on overlapping intervals. But a key observation is that since the \sltpp condition
(\cref{def:pp})
already holds for large intervals with high probability (\cref{prop:lower-tailbound-X}), we only need to look at the small intervals. Then, for two \iindices $k_1,k_2$ that are sufficiently far apart, these short intervals are disjoint, and thus the corresponding \sltpp conditions are independent. Therefore, we decompose a long interval of \iindices into several groups of far-apart \iindices. This is illustrated in \cref{fig:ppivot-tailbound-illustration}, each group indicated by a different color. Within each group, by a concentration bound for \iid random variables, there are many \sltpps. Further, by a union bound, the concentration holds in all the groups simultaneously with high probability. This summarizes the proof of \cref{prop:lower-tailbound-ppivots}, which culminates in \cref{lem:many-pps} showing that with overwhelming probability, there are many \sltpps in every long enough interval.




\import{./figures/}{fig-pivot-randomwalk.tex}

We first identify insightful
alternative characterizations
of \sltpps, and a few propositions to help prove \cref{prop:lower-tailbound-ppivots}. \cref{lem:many-pps} follows from there.
\begin{proposition}
    \label{prop:pivot-conditions-equivalence}
    \begin{IEEEeqnarray}{rCl}
        \predPP{k}
        &\iff&  (\forall \intvl{i}{j} \ni k\colon  \Xin{i}{j} > 0)
        \label{eq:pivot-conditions-equivalence-intervals}
        \IEEEeqnarraynumspace\\
        &\iff&  (\forall \intvl{i}{j} \ni k\colon  \Gin{i}{j} > \Bin{i}{j})
        \IEEEeqnarraynumspace\\
        &\iff&  (\Xat{k} = 1) \land (\forall j\geq k: \Xin{k}{j} \geq 0)
        \IEEEeqnarraynumspace\nonumber\\
        && \quad {}\land{} (\forall i<(k-1): \Xin{i}{k-1} \geq 0)
        \label{eq:pivot-conditions-equivalence-randomwalks}
        \IEEEeqnarraynumspace
    \end{IEEEeqnarray}
\end{proposition}
\begin{proof}
    Elementary, using $\Xin{i}{j} = \Xin{0}{j} - \Xin{0}{i}$.
\end{proof}
%
In particular, \eqref{pivot-conditions-equivalence-randomwalks}
characterizes a \sltpp as an \iindex $k$
such that $\Gat{k} = 1$
and the simple random walks
$\ell\mapsto \Xin{k}{k+\ell}$
and
$\ell\mapsto \Xin{k-1-\ell}{k-1}$
starting at
%
$0$
remain non-negative forever
%
(\cref{fig:pivot-randomwalk}).
Due to this, we easily see that the probability that any given \iindex is a \sltpp is the probability that the \iindex is \sltgood and the two random walks never return to zero (\cref{prop:ppivot-randomwalk}).
In \cref{prop:lower-tailbound-X} by a simple concentration bound over \iid random variables, we show that in all large intervals, with high probability, the random walk $\Xat{k}$ advances proportionally to the interval length (due to its positive drift).


Throughout this section, assume that
$\probGood = \frac{1}{2} + \epsGood$
with $\epsGood \in (0,1/2]$.

\begin{proposition}
    \label{prop:lower-tailbound-X}
    With $\alphaLowerTailX \triangleq 2 \epsGood^2$, $\forall \intvl{i}{j}$, $\forall \delta \geq 0$:
    \begin{IEEEeqnarray}{C}
        \Prob{\Xin{i}{j} \leq (1-\delta) 2 \epsGood (j-i)}
        \leq \exp( - \alphaLowerTailX \delta^2 (j-i)).
        \IEEEeqnarraynumspace
    \end{IEEEeqnarray}
\end{proposition}
%
\begin{proof}
By Hoeffding's inequality~\cite{doi:10.1080/01621459.1963.10500830}~\cite[Thm.~4]{duchi-hoeffding}.
\end{proof}
%
\begin{proposition}[Hoeffding's inequality~{\cite{doi:10.1080/01621459.1963.10500830} \cite[Thm.~4]{duchi-hoeffding}}]
    \label{prop:hoeffding}
    Let $Z_1, ..., Z_n$ be independent bounded random variables with
    $\forall i: Z_i \in [a,b]$, where $-\infty < a \leq b < \infty$.
    Then, $\forall t\geq0$:
    \begin{IEEEeqnarray}{rCl}
        \Prob{\left(\sum_{i=1}^n Z_i\right) \geq \Exp{\sum_{i=1}^n Z_i} + t n}
        &\leq&
        \exp\left(\frac{-2 n t^2}{(b-a)^2} \right) \IEEEeqnarraynumspace
        \\
        \Prob{\left(\sum_{i=1}^n Z_i\right) \leq \Exp{\sum_{i=1}^n Z_i} - t n}
        &\leq&
        \exp\left(\frac{-2 n t^2}{(b-a)^2} \right) \IEEEeqnarraynumspace
    \end{IEEEeqnarray}
\end{proposition}


\begin{proposition}
    \label{prop:ppivot-randomwalk}
    \begin{IEEEeqnarray}{C}
        \forall k\colon\quad
        \Prob{\predPP{k}}
        \geq \probPPFormula
        %
        \triangleq \probPP
    \end{IEEEeqnarray}
\end{proposition}
%
\begin{proof}
    In \eqref{pivot-conditions-equivalence-randomwalks},
    $\predPP{k}$ is characterized
    as the intersection of three independent events:
    \begin{IEEEeqnarray}{rCl}
        \Event_1
        &\triangleq&
        \{ \Xat{k} = 1 \}
        \\
        \Event_2
        &\triangleq&
        \{ \forall\ell\colon \Xin{k}{k+\ell} \geq 0 \}
        \\
        \Event_3
        &\triangleq&
        \{ \forall\ell\colon \Xin{k-1-\ell}{k-1} \geq 0 \}
    \end{IEEEeqnarray}
    Their probabilities are easily calculated~\cite{stackexchange-math-rwreturnto0}:
    \begin{IEEEeqnarray}{C}
        \Prob{\Event_1}
        = \probGood
        \qquad
        \Prob{\Event_2} = \Prob{\Event_3}
        = (2\probGood - 1) / \probGood
        \IEEEeqnarraynumspace
    \end{IEEEeqnarray}
\end{proof}


The process $\{\Pat{k}\}$ counts \sltpps,
with $\Pat{k} \triangleq \Ind{\predPP{k}}$.
\begin{proposition}
    \label{prop:lower-tailbound-ppivots}
    With $\alphaLowerTailPP \triangleq 2 \probPP^2$,
    \begin{IEEEeqnarray}{l}
        \forall \intvl{i}{j} \intvleq 2 K_1 K_2\colon\quad
        %
        %
        %
        \Prob{\Pin{i}{j} \leq (1-\delta) \probPP 2 K_1 K_2}
        \nonumber
        \\
        \qquad\qquad\qquad\qquad\qquad {}\leq{} 2 K_1 \exp(- \alphaLowerTailPP \delta^2 K_2) + \Khorizon^2 \exp(-\alphaLowerTailX K_1).
        \IEEEeqnarraynumspace
    \end{IEEEeqnarray}
\end{proposition}
\begin{proof}
    Let
    $\Event \triangleq \{\forall \intvl{i}{j} \intvlgeq K_1\colon \Xin{i}{j} > 0\}$.
    From \cref{prop:lower-tailbound-X} with $\delta=1$,
    and a union bound over all intervals
    ($\leq \Khorizon^2$ many),
    we get
    \begin{IEEEeqnarray}{C}
        \Prob{\lnot\Event} \leq \Khorizon^2 \exp(-\alphaLowerTailX K_1).
    \end{IEEEeqnarray}

    For any given index $k$, we can
    partition
    the intervals of
    \eqref{pivot-conditions-equivalence-intervals}
    into `long'
    %
    and `short'
    %
    intervals (length at least vs.\ less than $K_1$):
    \begin{IEEEeqnarray}{rCl}
        \Event_k
        &\triangleq&
        \{ \predPP{k} \}
        = \Event_k^{\mathrm{L}} \land \Event_k^{\mathrm{S}}
        \\
        \Event_k^{\mathrm{L}}
        &\triangleq&
        \{\forall \intvl{i}{j} \ni k, \intvl{i}{j} \intvlgeq K_1\colon \Xin{i}{j} > 0\}
        \IEEEeqnarraynumspace
        \\
        \Event_k^{\mathrm{S}}
        &\triangleq&
        \{\forall \intvl{i}{j} \ni k, \intvl{i}{j} \intvll K_1\colon \Xin{i}{j} > 0\}.
    \end{IEEEeqnarray}
    Note that $\Event_k^{\mathrm{L}} \supseteq \Event$.
    Also, 
    %
    for any two given \iindices $k_1, k_2$ that are `far apart',
    \ie,
    if $\abs{k_1 - k_2} \geq 2 K_1$,
    then
    $\Event_{k_1}$ and $\Event_{k_2}$ are conditionally independent
    given $\Event$
    (since $\Event_{k_1}^{\mathrm{S}}$ and $\Event_{k_2}^{\mathrm{S}}$ are).

    We decompose $I^* \triangleq \intvl{i}{j} = \intvl{i}{i + 2 K_1 K_2} = \bigcup_{\ell=1}^{2K_1} I_{\ell}$:
    \begin{IEEEeqnarray}{rCl}
        \forall\ell\in\{1,...,2K_1\}\colon\quad\
        I_{\ell}
        &\triangleq&
        \{ i+0\cdot 2K_1+\ell, ...
        \nonumber\\
        && \quad{} ..., i+(K_2-1)\cdot 2K_1+\ell \}.
        \IEEEeqnarraynumspace
    \end{IEEEeqnarray}
    See \cref{fig:ppivot-tailbound-illustration}
    for illustration.
    We define corresponding events, $\forall\ell\in\{1,...,2K_1\}$:
    \begin{IEEEeqnarray}{rCl}
        \Event^*
        &\triangleq&
        \left\{ \Pat{I^*} \leq (1-\delta) \probPP 2 K_1 K_2 \right\}
        \\
        \Event_{\ell}
        &\triangleq&
        \left\{ \Pat{I_{\ell}} \leq (1-\delta) \probPP K_2 \right\}.
    \end{IEEEeqnarray}
    Clearly, $\Event^* \subseteq \bigcup_{\ell=1}^{2 K_1} \Event_\ell$.
    Thus, by a union bound,
    \begin{IEEEeqnarray}{rCl}
        \Prob{ \Event^* \cond \Event }
        &\leq&
        \sum_{\ell=1}^{2 K_1} \Prob{ \Event_\ell \cond \Event }.
        \IEEEeqnarraynumspace
    \end{IEEEeqnarray}
    Furthermore, $\forall\ell\in\{1,...,2K_1\}$,
    and with $\mu_\ell \triangleq \Exp{ \Pat{I_{\ell}} \cond \Event}$:
    \begin{IEEEeqnarray}{rCl}
        \IEEEeqnarraymulticol{3}{l}{
            \Prob{ \Event_\ell \cond \Event }
            =
            \Prob{ \Pat{I_{\ell}} \leq (1-\delta) \probPP K_2 \cond \Event }
        }
        \IEEEeqnarraynumspace
        \\\quad
        &\leqA&
        \Prob{ \Pat{I_{\ell}} \leq (1-\delta) \mu_\ell \cond \Event }
        \IEEEeqnarraynumspace
        \\
        &\leqB&
        \exp(-2 \delta^2 \mu_\ell^2 / K_2)
        \leqC
        \exp(-2 \probPP^2 \delta^2 K_2),
        \IEEEeqnarraynumspace
        %
    \end{IEEEeqnarray}
    where
    (a) and (c)~use 
    \begin{IEEEeqnarray}{rCl}
        \mu_\ell = K_2 \Exp{\Ind{\predPP{k}} \cond \Event} &\geq& K_2 \Exp{\Ind{\predPP{k}}} \nonumber \\ &\geq& K_2 \probPP \IEEEeqnarraynumspace
    \end{IEEEeqnarray}
    (\cref{prop:ppivot-randomwalk}),
    %
    and
    (b)~uses that
    $\{\predPP{k_1}\}$ and
    $\{\predPP{k_2}\}$
    are conditionally independent given $\Event$
    for $k_1, k_2 \in I_\ell$,
    and
    Hoeffding's inequality (\cref{prop:hoeffding}).
    %

    To complete the proof, with $\alphaLowerTailPP = 2 \probPP^2$,
    \begin{IEEEeqnarray}{rCl}
        \Prob{ \Event^* }
        &=&
        \Prob{ \Event^* \cap \Event } + \Prob{ \Event^* \cap \lnot\Event }
        \\
        &\leq&
        \Prob{ \Event^* \cond \Event } + \Prob{ \lnot\Event }
        \\
        &\leq&
        2 K_1 \exp(-\alphaLowerTailPP \delta^2 K_2)
        + \Khorizon^2 \exp(-\alphaLowerTailX K_1).
        \IEEEeqnarraynumspace
    \end{IEEEeqnarray}
\end{proof}

\begin{lemma}
    \label{lem:many-pps}
    For
    $\Kcp = \Omega(\kappa^2)$,
    and $\Khorizon = \poly(\kappa)$,
    \begin{IEEEeqnarray}{C}
        \Prob{
            \forall \intvl{i}{j} \intvlgeq \Kcp\colon
            \Pin{i}{j} \geq (1-\delta) \probPP \Kcp}
        \nonumber
        \\
        \qquad{}\geq{} 1 - \exp(- \Omega(\kappa)) = 1 - \negl(\kappa).
    \end{IEEEeqnarray}
    %
\end{lemma}
%
\begin{proof}
From \cref{prop:lower-tailbound-ppivots} by setting $K_1,K_2 = \Omega(\kappa)$ and $\Kcp = 2K_1K_2$.
\end{proof}


