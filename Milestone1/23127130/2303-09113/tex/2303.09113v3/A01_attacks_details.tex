\iffalse
\section{\TeaserAttack Adversary Lead}
\label{sec:teasing-lead}

%

To demonstrate that the adversary in the \teaserattack eventually succeeds in maintaining a 2 block lead over the honest chain, we plot the adversary's lead for different adversary mining rates in \cref{fig:experiment-teaser-start}.
If the adversary has a very large mining power (if $\blkratetimeAdv > \blkratetimeGrowthSilent$), then it is clear that the adversary's lead eventually stays positive 
(recall the lead is a random walk with positive drift,
\cref{fig:experiment-teaser-start}~\ref{plt:teaser-strong2}).
However, the \teaserattack is stronger: it succeeds even when $\blkratetimeGrowthTeaser < \blkratetimeAdv < \blkratetimeGrowthSilent$. %
Though the random walk representing the adversary's lead initially has a negative drift, %
due to random fluctuations, %
it occasionally stays positive for short periods of time (\cref{fig:experiment-teaser-start}~\ref{plt:teaser-medium2}).
The occurrence of such events is enough for the \teaserattack to kickstart, because it
%
allows the adversary to create the setup (step (a)), and to repeat steps (b)--(e) of the \teaserattack (\cref{fig:attack-teaser}) at consecutive block heights. The resulting congestion then decreases the average growth rate of the honest chain to $\blkratetimeGrowthTeaser$. Thereafter, the adversary only needs mining power $\blkratetimeAdv > \blkratetimeGrowthTeaser$ in order to positively bias the random walk and thus eventually maintain a positive lead.
%
We see this process in \cref{fig:experiment-teaser-start}~\ref{plt:teaser-medium2}: the adversary's lead rises and drops to zero a few times, causing the adversary to try again. However, eventually, the adversary manages
to maintain a permanent lead.

On the other hand, when $\blkratetimeAdv < \blkratetimeGrowthTeaser$, the adversary's lead has a negative drift even after the congestion effects kick in (\cref{fig:experiment-teaser-start}~\ref{plt:teaser-weak2}), and therefore the \teaserattack is bound to run out of blocks and fail.






\section{Private Attack}
\label{sec:attacks-private}

On a high level, in the notorious so-called \emph{private attack} \cite{nakamoto_paper,dem20} (\cref{fig:attack-race-concept}), the adversary's objective is to
break the safety property of the blockchain
(\ie, de-confirm transactions).
For this purpose, 
the adversary
uses its hash power
to mine a chain that it keeps private (`adversary chain'),
with the goal of eventually
outgrowing the public longest chain built jointly by all honest nodes (`honest chain').
Once that happens, the adversary reveals its longer adversary
chain, thereby displacing the shorter honest chain as the longest
chain, and de-confirming transactions included in
the honest chain.
Note that if the adversary chain falls behind the honest chain,
the adversary restarts its attack from the tip of the current longest chain.

%
%
The success of the private attack %
%
depends crucially on
the growth rate 
of each of the chains.
The adversary's chain grows every time the adversary mines a block, and hence with the rate $\blkratetimeAdv$ of adversary block production.
The honest chain grows at a rate
$\blkratetimeGrowth$ which is generally
lower than
the total honest block production rate $\blkratetimeHon$,
due to \emph{forking} caused
by \emph{processing delays} of blocks.
%
\import{./figures/}{fig-attack-no.tex}
%
Specifically,
since honest nodes mine on their longest \emph{downloaded} chain,
when some honest node mines a new block,
the other nodes continue mining on the same tip as before
(\cref{fig:attack-no}(a))
until they have finished processing 
the newly mined block (\cref{fig:attack-no}(c)).
If another honest node mines a new block during this time,
it does not extend the longest chain 
(\cref{fig:attack-no}(b)).
Thus, only part of honest mining contributes to growing the honest chain,
and the honest chain growth rate $\blkratetimeGrowth$ is less than the honest block production rate $\blkratetimeHon$.

During the private attack, honest nodes undergo delays
only due to processing \emph{honest} blocks, because until completion of the attack, the adversary does not release any blocks.
%
As described in \cref{fig:attack-no} and as seen empirically in \cref{fig:experiment-teaser} (\ref{plt:experiment-teaser-noattacker} vs.\ \ref{plt:experiment-growth-delay}),
the honest chain growth rate $\blkratetimeGrowthSilent$
under private attack is approximated well by the growth rate that would occur
in an idealized network where all blocks are processed within $\Delta=1/\bwtime$ time
after they are announced
(the bounded-delay model).
Therefore, previous analyses' calculations of the adversary mining rates at which the private attack succeeds, even though based on the bounded-delay model, continue to hold
under bounded bandwidth.
\fi