\section{Introduction}
\label{sec:introduction}



In order to remain secure against adversaries controlling up to $50\%$ of the network,
blockchains that utilize Nakamoto's proof-of-work (PoW) longest-chain consensus protocol~\cite{nakamoto_paper,backbone} 
%
have been parameterized
to leave a security margin between the 
%
throughput
under normal operation and each node's capacity limits.
For instance, in expectation, Bitcoin produces only one block of transactions every ten minutes, though it usually only takes a few seconds for a node to download and process a block \cite{decker}.
%
%
%
On the other hand, Bitcoin Cash forked off to increase the block size for better throughput, a proposal whose security implications were hotly debated~\cite{btc-blocksize-war}.

The fundamental question that protocol designers face is: \emph{%
    What is the security--performance trade-off between the block production rate
    (relative to the nodes' capacity limit)
    and the fraction of adversary power that the protocol tolerates?}
%
In this work, we 
%
show the inadequacy of the bounded-delay model that most previous works utilized to analyze the security of Nakamoto consensus (NC)~\cite{kiayias2017ouroboros,backbone,dem20,sleepy,ren,tight_bitcoin,pss16,kiffer2018better}, and instead use a bounded-bandwidth model that better captures real-world effects such as congestion due to a backlog of blocks that need to be sent and processed by nodes.

%
%
%
%
%
%
%
%
In PoW NC,
collectively starting with a well-known ``genesis'' block,
each node continuously works to solve a computational
puzzle to extend the longest chain of blocks 
it sees with a new block containing pending transactions (``\emph{mining} a new block'').
%
%
%
%
%
%
%
%
%
%
%
%
%
%
%
%
%
%
When successful, the node pushes the new block's \emph{header} to the network,
and makes its \emph{content} available for download.
In order to extend a chain, nodes must first \emph{process}, \ie, download and verify, the content of blocks in that chain, to ensure that the content is both \emph{available} and \emph{valid}.
Downloading blocks may take time, especially if blocks are extremely large~\cite{decker}, but in systems that contain smart contracts, 
%
even smaller blocks may take a while to process---mostly due to the time it takes to execute and validate smart contracts~\cite{demystifying-incentives}.


In PoW mechanisms, block production occurs at random times, which makes the processing load of the network bursty.
Moreover, the adversary can selectively withold its own mined blocks and release them opportunistically.
Both these factors further stress 
%
the bandwidth and processing capacities of nodes.
With limited processing capacity, during times of high load, blocks will be \emph{queued} for processing. 
%
%
Since nodes cannot mine new blocks extending chains that they have not yet fully processed, queueing further delays the growth of the honest nodes' chain. As the security of NC is based on the honest chain outgrowing any adversary chain,
the reduced growth of the honest nodes' chain makes it easier for an adversary to attack the system.
%
\import{./figures/}{fig-comparison-bddelay-bdbandwidth.tex}
%
%
%
%
%
%
To analyze security under such effects,
it is important to consider the  \emph{scheduling policy} that nodes use in deciding which blocks to download and process first, given a set of new block headers.
%
%
%
Since a node extends its longest chain to produce new blocks, the most obvious policy is to first process blocks along the longest chain that the node has seen. Indeed, this policy can be found in the Bitcoin implementation \cite{btcdevp2pnetworkheadersfirst}.
%

%

%


%

\myparagraph{Limitations of the Bounded-Delay Model}
%
%
%
Previous work has focused on the security analysis of Bitcoin in the synchronous setting: All messages are assumed to arrive after a maximum delay of $\Delta$~\cite{kiayias2017ouroboros,backbone,dem20,sleepy,ren,tight_bitcoin,pss16,kiffer2018better}.
%
Using this model, 
%
\cite{dem20,tight_bitcoin} calculate a tight bound on the fraction $\beta$ of adversary nodes, for given block production rate $\blkratetime$ and delay bound $\Delta$, for which the protocol is secure against all attacks.
%
However, the $\Delta$-delay model assumes that the delay is the same \emph{irrespective of the total processing load}, and specifically, that the load cannot be manipulated by an attacker. Thus, the model fails to capture the effects of bursty release of blocks by an adversary or due to the stochastic nature of PoW mining even by honest nodes alone.
%
%
%
%
%
%
%

The bounded-delay analysis \cite{dem20,tight_bitcoin} 
%
%
concludes that the
well-known
\emph{private attack}~\cite{nakamoto_paper} (along with delaying every message by $\Delta$) is the worst-case attack strategy since 
its attack threshold matches
%
the
security threshold,
%
%
%
%
\ie,
under parameters where the private attack fails, the analysis concludes that all other attacks must fail, too.
If we only consider the private attack and low block production rates, then the bounded-delay analysis, with $\Delta$ taken as the time to process one block, is a good approximation to calculate the fraction of adversary power with which the attack succeeds (see \cref{fig:comparison-bddelay-bdbandwidth}, validated by simulations in \cref{sec:experiments}).
This is because during the private attack, the adversary does not release any blocks (only ``benign'' random congestion), and the effect of bursty honest mining is less significant at low block rates.
%

%
%
%
%
%
%
%
%
%
%
%


%
%
%
%
%
%
%

%
%
%
%
%
%
%
%
%
%
%

%
%
%

However, there are other strategies in which the adversary adds to the network load to increase queuing delays.
We simulate one such strategy, the \teaserattack (\cref{sec:teaser-attack}), that is stronger than the private attack, \ie, it succeeds in regions of $(\lambda, \beta)$ where the private attack does not succeed (\cref{fig:comparison-bddelay-bdbandwidth}).
In the \teaserattack, the adversary ``teases'' honest nodes to process a longer chain it announces, but makes this effort ``useless'' by not releasing the block contents for the entire chain.
The adversary effectively doubles the network load and queuing delays, thus slowing the growth of the honest nodes' chain. The adversary builds a longer chain to break security. This halves the maximum secure block rate $\blkratetime$ for any given $\beta$ (\cref{fig:comparison-bddelay-bdbandwidth}).
While the concrete \emph{quantitative} impact of this attack may be considered modest, it highlights \emph{conceptual} limitations of earlier analyses and emphasizes the need for security analysis under more realistic network models,
especially to rule out that unbeknown to us there could be even more serious queuing-based attacks.

\myparagraph{Security Bounds under Bandwidth Constraints}
%
%
To
re-establish the security of NC in a more realistic model,
%
%
%
%
%
%
%
we adopt the \emph{bounded-bandwidth} model from~\cite{bwlimitedposlc}.
Under this model, we consider the scheduling policy as a part of the protocol description as it affects the security of the protocol.
\emph{Henceforth, though we adopt the word ``bandwidth'', we continue to mean
    ``capacity'' in the wider sense, \ie, rate
    limits on communication, computation,
    and storage access. Similarly, we use ``download'' to mean ``process'' in the
    wider sense.}

\begin{result}
%
\label{res:result-pow-security}
    Using the bounded-bandwidth model and a novel analysis technique, we characterize a region of block mining rate $\blkratetime$ and adversary fraction $\beta$ for which we prove that PoW NC, with a wide range of suitable scheduling policies, is secure (\cref{thm:safety-and-liveness-pow}).
    This region is shown in \cref{fig:comparison-bddelay-bdbandwidth}.
    Specifically, this analysis expands the set of adversary strategies
    captured by earlier bounded-delay analyses to include adversary strategies
    that exploit effects from bounded capacity.
\end{result}


%



%
%
%
%

%

%
%
%
%

%
%
%
%
%
%

%
%
%

%

%
%
%
%
In \cref{fig:bitcoin-cardano-resilience-bandwidth}, we plot the adversary resilience versus bandwidth requirement for PoW NC with cautious (\eg Bitcoin) and ambitious (\eg Bitcoin Cash) parameters.
%
%
%
%
%
%
%
%
%
%
%
%
%
%
%
%
%
It shows the importance of modeling and studying congestion effects on security,
in particular, for protocols that aim for maximum performance,
%
and our analysis provides tools to do so.
While our work demonstrates that earlier analyses
%
have failed to capture some security-critical phenomena,
%
%
the quantitative gap between our best-known attack
(\cref{fig:comparison-bddelay-bdbandwidth}~\tikz[x=0.75em,y=0.75em]{ \draw [draw=myParula07Red,thick,fill=myParula07Red,fill opacity=0.3] (0,0) rectangle (1,1); })
%
and our best-known security analysis
(\cref{fig:comparison-bddelay-bdbandwidth}~\tikz[x=0.75em,y=0.75em]{ \draw [draw=myParula05Green,thick,fill=myParula05Green,fill opacity=0.3] (0,0) rectangle (1,1); })
%
points to a need for future
work.\footnote{%
%
We focus on the security--throughput tradeoff
of NC, \ie, for what tuples $(\lambda, \beta)$
%
%
does the consensus-failure probability $\varepsilon$
%
decay exponentially to zero
under \emph{some appropriate}
scaling of the confirmation latency $\Tlive$.
On a finer point, in our 
bounded-bandwidth analysis,
%
latency scales 
%
%
$\Tlive \sim (\log(1/\varepsilon))^2$ (\cref{thm:safety-and-liveness-pow}),
in contrast to earlier bounded-delay analyses
that required only
%
%
$\Tlive \sim \log(1/\varepsilon)$~\cite{tight_bitcoin,linear-cons-pos,ren}.
%
Exploring the possibility of tighter latency scaling under bounded bandwidth
requires future work.
%
%
%
%
%
}

\import{./figures/}{fig-bitcoin-cardano-resilience-bandwidth.tex}

%
%
%
%

%
%
%


\myparagraph{Proof-of-Stake (PoS) NC}
Nakamoto consensus has been adapted to proof-of-stake in protocols of the Ouroboros~\cite{kiayias2017ouroboros,david2018ouroboros,badertscher2018ouroboros} and Sleepy Consensus~\cite{sleepy,snowwhite} families.
In PoS NC, the block production lottery is independent of the block's content or parent~\cite{pos_paper}.
This, unlike in PoW NC, allows an adversary to \emph{reuse} a ``winning PoS lottery ticket'' to create infinitely many valid blocks (called \emph{equivocations}).
As observed in~\cite{bwlimitedposlc}, the adversary can 
\emph{spam} nodes with many \emph{equivocating} blocks, aggravating the problem of congestion.
%
While slashing~\cite{weaksubjectivity,casper,aadilemma,bftforensics} may deter \emph{rational} adversaries to \emph{some} extent,
protocols need to tolerate equivocations to 
handle plausibly irrational \emph{Byzantine} adversaries~\cite{bwlimitedposlc}.

Analytical work \cite{bwlimitedposlc} gives a security proof for PoS NC in the bounded-bandwidth model.
%
%
%
%
%
%
%
However,
\cite{bwlimitedposlc} proves security only when
%
%
%
%
%
%
%
nodes have enough bandwidth so
that for each block, 
they can download potentially different versions of its $k$ predecessors, where $k$ is the confirmation depth chosen for the chain.
%
%
This increases the network load by $k$ times, thus reducing the maximum secure block rate $\blkratetime$ by $k$ times
(\cref{fig:comparison-bddelay-bdbandwidth-pos}).
Decreasing the probability of consensus failure requires increasing $k$, which means that for security with overwhelming probability, the throughput must approach zero.
This is not merely an artifact of the security analysis of \cite{bwlimitedposlc}:
Augmenting our \teaserattack with equivocations demonstrates this behavior (we discuss this in \cref{sec:experiments} and \cref{sec:pos-teaser-attack}).
%
On the other hand, PoW NC does not suffer from such vanishing throughput (\cref{fig:comparison-bddelay-bdbandwidth}).

\import{./figures/}{fig-comparison-bddelay-bdbandwidth-pos}

\begin{result}
    We propose and prove the security of a new PoS protocol we call \emph{\ProtMid} (\emph{\ProtShort}), a variant of PoS NC, that is secure in the same region of block production rate $\blkratetime$ and adversary fraction $\beta$ as PoW NC. Thus, similar to PoW NC, security with overwhelming probability requires increasing the confirmation depth, which affects latency, but not decreasing the block production rate, which affects throughput (see \cref{fig:comparison-bddelay-bdbandwidth-pos}).
\end{result}

On a high level, in \ProtShort,
honest nodes establish consensus on PoS lottery tickets for which
they have seen 
equivocations.
%
The contents of blocks from those equivocating PoS lottery tickets
are then \emph{blanked}, \ie, all blocks from those tickets are treated as empty blocks.
This absolves honest nodes from downloading more than one block
per PoS lottery ticket, restoring the non-equivocation
behavior of PoW from a bandwidth point of view.
From there,
the security proof closely follows that of PoW NC.\footnote{%
The confirmation latency of \ProtShort under bounded bandwidth scales quadratically with the security parameter, just like PoW NC's latency.}

Blanking block contents
undermines predictability of transaction validity
(\cf \cref{sec:introduction-methods-throughputloss}).
%
%
%
%
%
%
%
%
%
%
%
%
%
%
%
%
%
%
%
%
%
%
%
%
%
%
%
%
%
%
%
In particular, it is harder to ensure,
at the time of composing a block, whether
transactions
%
are able to pay their fees.
%
%
%
Many modern consensus protocols share this problem
(\eg, \cite{spiegelman2022bullshark,danezis2022narwhal,dispersedledger,honeybadger,al2019lazyledger}).
We suggest some solutions in \cref{sec:predictablevalidity}.





\subsection{Related Works}
\label{sec:introduction-relatedworks}

%

%
Earlier works have analyzed the security of
PoW~\cite{backbone,nakamoto_paper,dem20,pss16,kiffer2018better,ren,tight_bitcoin} and
PoS~\cite{kiayias2017ouroboros,david2018ouroboros,badertscher2018ouroboros,sleepy,snowwhite,dem20,pos_paper} NC
in the bounded-delay model.
Our analysis builds on tools from several of these works, primarily pivots~\cite{sleepy} (Nakamoto blocks~\cite{dem20}) and convergence opportunities~\cite{pss16,sleepy,kiffer2018better} (or similar~\cite{dem20,ren}).
%
%
%
%
%
%
%
%
%
%
%
%
%
%
%
%
%
%
%
%
%
%
%
%
%
%
%
Markov decision processes
were used~\cite{sompolinsky2016bitcoins, gervais2016security} 
to computationally find optimal attack strategies,
%
%
assuming honest nodes do not suffer any delay.
%

Limitations of the bounded-delay model have been observed in previous work~\cite{prism,near-optimal-thruput,bwlimitedposlc}.
To use the bounded-delay model to set the protocol's block production rate, one needs to find the value of the bound $\Delta$.
%
This is tricky because unlike the bandwidth limit, which is a physical limit of the hardware used, delay depends on the network load.
%
One approach is to set the delay to the time taken to process one block, \ie, $\Delta = 1/\bwtime$.
While this may be reasonable at rates much smaller than the bandwidth (as processing queues are mostly empty), queuing delay breaks this bound otherwise.
%
%
%
%
%
%
In \cite{near-optimal-thruput}, a queuing model is used to 
%
calculate a delay bound that holds throughout the execution with overwhelming probability.
However, such a tail bound is too pessimistic because the queuing delays cannot always be large, due to limited block production.
In contrast, our finer-grained analysis captures limited block production.
%
%
Another work \cite{longest-chain-random-delay} analyzes security in a random (\iid) delay model.
%
%
However,
the network load, hence queuing delay, is not purely a random process, but is controlled by the adversary.
Network experiments~\cite{decker,kiffer2021under,revisiting-asynchronous} help estimate the delay distribution but cannot show us the impact of all possible adversary manipulations.
%

In analytical work \cite{bwlimitedposlc}, the bounded-bandwidth model captures adversarial manipulations.
%
Our paper's bounded-bandwidth model is that of~\cite{bwlimitedposlc}. Our paper differs from~\cite{bwlimitedposlc} two-fold:
(a) Only PoS is studied in \cite{bwlimitedposlc}. Due to equivocations in PoS, the security bounds of \cite{bwlimitedposlc} are too pessimistic for PoW NC. We develop \emph{new analysis machinery} (\cf~\cref{sec:introduction-methods-analysis}) to prove security \emph{for PoW}. Furthermore, the attack in \cite{bwlimitedposlc} does not apply to PoW, while our teasing strategy does.
(b) PoS NC with the freshest-block policy proposed in \cite{bwlimitedposlc} is secure only when its throughput approaches zero. In contrast, our \emph{new PoS protocol}, \ProtShort, is secure \emph{with non-vanishing throughput}.
%
%
%
%
%
%
%
%
%
%
%
%
%
%
%
%
%
%
%
%
%
%
%
%
%
%
%
%
%
%
%
%
%
%
%
%
%
%
%
%
%
%
%
%
%
%
%
%
%
%
%
%
%
%
%
%
%
%
%
%
%
%
%
%
%
%
%
%
%
%

Concurrently, \cite{dag-pow-bandwidth} analyzes specific congestion-based attacks on PoW DAG protocols but does not provide a security proof against all attacks.
%
%
%
Propagation delays also exacerbate selfish mining strategies~\cite{zhang-slower-block}, and congestion is another way to increase propagation delays.
However, the goal of this work is to analyze \emph{security} under bounded bandwidth, and selfish mining does not affect the two security properties of consensus: safety and liveness. It affects incentives and fairness, which are orthogonal.

Capacity limits apply not only to downloads but also to computational processing of transactions and smart contracts.
%
%
For instance,
%
earlier works \cite{brokenmetre} have shown that
execution times can vary by orders of magnitude
between benign 
%
and maliciously crafted transactions
(with equal gas consumption).
%
While download and validation are similar in that the time taken increases with the number of transactions, they are different in some aspects.
Validation is harder to parallelize due to transactions that depend on each other.
Methods to parallelize execution of smart contracts are studied in \cite{adding-concurrency-smart-contracts,speculative-concurrency-eth}.
Additionally, validating transactions can be delayed until after confirmation, such as in~\cite{al2019lazyledger,tuxedo}, but delaying downloads could lead to data availability attacks (\cf~\cref{sec:sapos}).
%
%
%




%
%
%
%
%
%
%
%
%
%
%
%
%
%
%
%
%
%
%
%
%
%
%
%
%
%
%
%
%
%
%
%
%
%
%
%
%
%
%
%
%
%
%
%
%
%
%
%
%
%
%
%
%
%
%
%
%
%
%
%
%
%
%
%
%
%
%
%
%
%
%
%
%
%
%
%
%
%
%
%
%
%
%
%
%
%
%
%
%
%
%
%
%
%
%








%

%

%
%
%
%
%
%
%
%
%
%
%
%
%
%
%
%
%
%
%
%
%
%
%
%
%
%
%
%
%
%
%
%
%
%

%
%
%
%
%
%
%
%
%
%
%
%
%
%
%
%
%
%
%
%
%
%
%






%
%
%
%
%
%
%
%
%
%
%
%
%
%
%
%
%
%
%


%
%
%
%
%
%
%
%
%
%
%
%
%
%
%
%
%
%
%
%
%
%
%
%
%
%
%
%
%
%
%






%
%
%
%
%
%
%
%
%
%
%
%
%
%
%








\subsection{Overview of Key Ideas and Methods}
\label{sec:introduction-methods}

%
\subsubsection{New Analysis Technique}
\label{sec:introduction-methods-analysis}

%

\import{./figures/}{fig-analysis-comparison-sleepy.tex}

%
%
Our key contribution is a new analysis technique for PoW NC
under bounded bandwidth.
%
%
Traditional NC security analysis (\cref{fig:analysis-comparison-sleepy}(a)) is based on the notion of a \emph{pivot}~\cite{sleepy}.
%
%
%
%
%
%
%
%
%
Pivots are special \emph{honest} blocks 
($\Rightarrow$~liveness)
%
%
%
which by a combinatorial argument
%
%
remain in the chain forever 
($\Rightarrow$~safety),
%
and by a probabilistic argument happen frequently.
%
%
%
%
Safety and liveness of NC with suitable parameters follow swiftly.

%
%
%
%
%
%
Under bounded delay,
the qualities required for the probabilistic and combinatorial argument, respectively,
are equivalent. 
As a result, it has not been 
widely
noted that these properties are 
%
not identical.
Under bounded bandwidth,
%
these properties are no longer equivalent.
Our \textbf{first} conceptual \textbf{contribution} is to
%
decompose pivots' probabilistic/combinatorial qualities into \emph{\sltpps} and \emph{\sltcps} (\cref{fig:analysis-comparison-sleepy}(b)).
\sltPps are honest block production events where in every time interval around them there are more
%
honest than adversary block production opportunities (same as pivots in the bounded-delay analysis).
\sltCps are honest block production events where in every time interval around them there are more \emph{chain growth} events than non-chain-growth events (chain growth occurs
only when an honest block is produced \emph{and soon downloaded} by
%
honest nodes).
%

Some \sltpps no longer turn into \sltcps under bounded bandwidth, because adversary block
%
release can delay the download of honestly produced blocks, and thus some honest block production opportunities might not translate to chain growth.
Previous bounded-bandwidth analysis~\cite{bwlimitedposlc} side-stepped this difference 
by choosing a specific scheduling policy 
and such a low block production rate 
%
that every \sltpp becomes a \sltcp.
Instead,
our \textbf{second} technical \textbf{contribution} is a combinatorial argument to show that if there is a sufficiently \emph{high density} of \sltpps over a long time interval, then at least one of these \sltpps is typically a \sltcp.
This relies on the adversary's limited budget of blocks it can spam with, and holds for a wide range of scheduling policies (including longest-header-chain and freshest-block~\cite{bwlimitedposlc}).

%
The original probabilistic argument of~\cite{sleepy}
%
guarantees only a fairly \emph{low density} of \sltpps.
Proving a high density is challenging because the occurrence of \sltpps are dependent events, so standard Chernoff-style tail bounds are not enough.
Our \textbf{third} technical \textbf{contribution} is to show, by leveraging the weak dependence of \sltpp occurrences, that long time intervals typically have a \emph{high density} of \sltpps.
%
This completes the analysis for PoW NC.








        %

        %
        %
        %
        %
        %
        %
        %
        %
        %
        %
        %
        %
        %
        %
        %
        %
        %
        %








%
\subsubsection{\ProtLong}
\label{sec:introduction-methods-pos}

%
%

%
%
%
%
%
%

%
%
In \ProtShort, every honest node downloads at most one out of several equivocations, and instead considers equivocating blocks to be \emph{blank}.
This makes honest nodes immune to the effects of equivocation spamming.
However, we need to ensure that honest nodes can still switch from one chain to another longer chain, both of which might contain different equivocating blocks.
%
For this, note that headers of two equivocating blocks from the same 
PoS lottery
%
can serve as a \emph{succinct equivocation proof} to convince other nodes that an equivocation was committed.
Therefore, in \ProtShort, if an honest node sees an equivocation for a block in its longest chain, it publishes an equivocation proof in the block that it produces, which allows all nodes to consistently treat the equivocating block's content as \emph{blank} without downloading it.
%

%
A caveat so far is that an adversary could reveal an equivocation late and cause inconsistent ledgers across honest nodes and/or time.
%
%
To avoid this, we enforce a deadline for how late an equivocation proof can be included in the chain.
%
%
Our security proof shows how to parameterize the 
%
deadline
and the protocol's confirmation depth such that if any honest node has blanked the content of any equivocating block on its longest chain, then an appropriate equivocation proof is timely included on-chain, and all honest nodes blank the block's content before it reaches the output ledger.


%
\subsubsection{Ensuring Fees Get Paid despite Lack of Predictable Validity}
\label{sec:introduction-methods-throughputloss}

%

%
%
%

%

%

Blanking of blocks in \ProtShort leads to \emph{lack of predictable transaction validity},
\ie, honest nodes do not know whether transactions they include in their block will be valid, since the content of blocks in the prefix may later be blanked due to an equivocation. 
Many modern consensus protocols in which consensus proceeds without executing transactions~\cite{spiegelman2022bullshark,danezis2022narwhal,dispersedledger,honeybadger,al2019lazyledger} also lack predictable transaction validity.
This risks that the adversary gets to spam the ledger with invalid transactions for free.
%
In one solution to prevent this, we focus on guaranteeing \textit{transaction fees} are always paid regardless of equivocations, by introducing 
%
%
%
%
 \emph{gas deposit accounts} that can only be used to pay transaction fees.
Any deposit to such an account takes effect only after the deadline has passed for the inclusion of any equivocation proof that might lead to removal of transactions from the deposit's prefix.
This gives honest block producers a lower bound on the account's balance 
%
which they can use to reliably determine whether a transaction can pay fees.
%
%
