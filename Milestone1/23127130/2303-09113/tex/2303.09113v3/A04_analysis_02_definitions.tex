\subsection{Definitions}
\label{sec:fullproof-definitions}


\myparagraph{\sltGood, \sltBad, and \sltEmpty \Timeslots}
%
%
\Timeslots without a \BPO are called \emph{`\sltempty'}.
%
A \timeslot is \emph{`\sltgood'} iff
it has 
exactly one honest \BPO and no adversary \BPOs,
and is followed by $\goodsep$ \sltempty \timeslots.
This definition is inspired by convergence opportunities \cite{pss16,sleepy,kiffer2018better}, loners \cite{dem20}, and laggers \cite{ren}.
Here, $\goodsep$ is an analysis parameter.
We define another analysis parameter $\goodsepbw$
which is related to $\goodsep$ as 
\begin{IEEEeqnarray}{C}
    %
    \label{eq:goodsep-bw-equation}
    (\goodsep+1)\slotduration \triangleq \DeltaHeader + \goodsepbw / \bwtime.
\end{IEEEeqnarray}
%
Thus, $\goodsep, \goodsepbw$ are chosen such
that for a \sltgood \timeslot, every honest node can 
receive the block header for the honest \BPO, and
download content for $\goodsepbw$ blocks, before the next \BPO. 
%
%
Any non-\sltempty \timeslot which is not \sltgood is called \emph{`\sltbad'}.
%
\begin{definition}
    \label{def:slots}
    We call a \timeslot $t$ \emph{\sltgood}, \emph{\sltbad}, \emph{\sltempty},
    respectively,
    denoted as $\predGood{t}$, $\predBad{t}$, $\predEmpty{t}$, respectively, iff:
    \begin{IEEEeqnarray}{rCl}
        \predGood{t} &\;\triangleq\;& (\Hat{t} = 1) \land (\Aat{t} = 0) \nonumber \\ && \quad {}\land{} (\Hin{t}{t+\goodsep} + \Ain{t}{t+\goodsep} = 0)
        \IEEEeqnarraynumspace\\
        \predBad{t} &\;\triangleq\;& (\Hat{t} + \Aat{t} > 0) \land \lnot\predGood{t}
        \IEEEeqnarraynumspace\\
        \predEmpty{t} &\;\triangleq\;& (\Hat{t} + \Aat{t} = 0).
        \IEEEeqnarraynumspace
    \end{IEEEeqnarray}
\end{definition}
%
Note that $\predEmpty{t} = \lnot\predGood{t} \land \lnot\predBad{t}$.


We denote by $t_k$ the $k$-th non-\sltempty \timeslot.
Then, we can introduce random processes over \emph{\iindices},
with \iindex $k$ corresponding
to the $k$-th non-\sltempty \timeslot $t_k$.
Considering only \iindices simplifies analysis by not having to deal with \sltempty \timeslots.
The process $\{\Gat{k}\}$ counts good \timeslots,
with $\Gat{k} \triangleq \Ind{ \predGood{t_k} }$.
%
%
Correspondingly, $\{\Bat{k}\}$ counts bad \timeslots,
$\Bat{k} \triangleq 1 - \Gat{k}$.

The following fact shows the distribution of \sltgood \iindices.
%
%
%
%
%
%
%
%
%
\RestatePropXiIsIid*
%
\begin{proof}
    First, for any $k$,
    \begin{IEEEeqnarray}{rCl}
        \Prob{\Gat{k} = 1} &=& \Prob{ \predGood{t_k} \mid \lnot \predEmpty{t_k} } \\
        &=& \frac{\Prob{\predGood{t_k}}}{\Prob{\lnot \predEmpty{t_k}}}
        %
        %
        = \frac{(1-\beta)\blkrateslot e^{-\rho(\goodsep+1)}}{1-e^{-\blkrateslot}}. \IEEEeqnarraynumspace
    \end{IEEEeqnarray}
    Let $\probEmpty \triangleq \Prob{\Hat{t}+\Aat{t}=0}$.
    Take an \iid random process $\{T_k\}$ with $\Prob{T_k = t} = (1-\probEmpty)\probEmpty^t$ for $t \geq 0$.
    The random variables $\{T_k\}$ describe the inter-arrival times between non-empty slots.
    Take another \iid random process $\{\Gat{k}'\}$, independent of $\{T_k\}$, such that $\Gat{k}' = 1$ with probability $\Prob{\Hat{t} = 1 \land \Aat{t} = 0 \mid \Hat{t}+\Aat{t}>0}$ and $\Gat{k}' = 0$ otherwise.
    The random process $\{\Gat{k}\}$ can be equivalently defined as $G_k = 1$ iff $G_k' = 1$ and $T_k \geq \goodsep$.
    The independence of the random variables $\{\Gat{k}\}$ then follows from the independence of the random variables $\{(T_k, \Gat{k}')\}$.
\end{proof}
%
Throughout the analysis, we assume
$\probGood > \frac{1}{2}$ (`honest majority' assumption).


\myparagraph{Some \sltGood \Timeslots Imply Growth}
%
%
A special role is played by \sltgood \timeslots $t_k$
with the additional property that 
the block produced
at $t_k$ is `soon' downloaded by all honest nodes.
Intuitively, these lead to \emph{chain growth},
the cornerstone of NC security~\cite{sleepy,dem20}.
We count these \timeslots with $\{\Dat{k}\}$,
and all other non-\sltempty \timeslots with $\{\Nat{k}\}$.
Specifically,
$\Dat{k} \triangleq 1$ if $\predGood{t_k}$
\emph{and} the block produced at $t_k$
has been downloaded by all honest nodes by the end
of \timeslot $t_k + \goodsep$,
$\Dat{k} \triangleq 0$ otherwise,
and $\Nat{k} \triangleq 1 - \Dat{k}$.
%
%
Finally, we define two random walks
on \iindices of non-\sltempty \timeslots
with increments
$\{\Xat{k}\}$ and $\{\Yat{k}\}$
that are handy for the definition of probabilistic
and combinatorial pivots:
\begin{IEEEeqnarray}{rClCrCl}
    \label{eq:random_walks_X_and_Y}
    \Xat{k} &\triangleq& \Gat{k} - \Bat{k}
    &\qquad\qquad&
    \Yat{k} &\triangleq& \Dat{k} - \Nat{k}
    \IEEEeqnarraynumspace
\end{IEEEeqnarray}
Note that the increments $\{\Xat{k}\}$
are \iid, and not affected by adversary action,
while the increments $\{\Yat{k}\}$ \emph{do depend}
on the adversary action and are thus in particular
\emph{not} \iidPERIOD.
Also note that $\forall k\colon Y_k \leq X_k$ since $D_k = 1 \implies G_k = 1$.


\myparagraph{Probabilistic and Combinatorial Pivots}
%
%
\begin{definition}
    \label{def:pp}
    We call an \iindex $k$ a \emph{\sltpp} (short for \emph{probabilistic pivot}),
    denoted as $\predPP{k}$, iff:
    \begin{IEEEeqnarray}{rCl}
        \predPP{k} &\;\triangleq\;&  ( \forall \intvl{i}{j} \ni k\colon  \Xin{0}{i} < \Xin{0}{k} \leq \Xin{0}{j} )
        \IEEEeqnarraynumspace
    \end{IEEEeqnarray}
\end{definition}
%
\begin{definition}
    \label{def:cp}
    We call an \iindex $k$ a \emph{\sltcp} (short for \emph{combinatorial pivot}),
    denoted as $\predCP{k}$, iff:
    \begin{IEEEeqnarray}{rCl}
        \predCP{k} &\;\triangleq\;&  ( \forall \intvl{i}{j} \ni k\colon  \Yin{0}{i} < \Yin{0}{k} \leq \Yin{0}{j} )
        \IEEEeqnarraynumspace
    \end{IEEEeqnarray}
\end{definition}
%
This definition of \sltpps and \sltcps decouples \cite[Def.~5]{sleepy} into its \emph{probabilistic} aspects~\cite[Sec.~5.6.3]{sleepy} and \emph{combinatorial} aspects~\cite[Sec.~5.6.2]{sleepy},
and casts them as conditions
on a random walk,
inspired by~\cite{dem20,close-latency-security-ling-ren}, to simplify the analysis.
The decoupling is one of the key differences from the analysis in \cite{sleepy} (see \cref{fig:analysis-comparison-sleepy}).
Note that a \sltcp is also a \sltpp because $Y_i \leq X_i$.
%
Also, \cref{def:pp} is equivalent to \cref{def:pp-informal},
and \cref{def:cp} is equivalent to \cref{def:cp-informal}
(\cf proof of \cref{prop:pivot-conditions-equivalence}).
