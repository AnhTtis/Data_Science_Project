\begin{figure}[tb]%
    \centering%
    
    \begin{tikzpicture}[]%
        \footnotesize
        \begin{axis}[
                mysimpleresilienceplot01,
                name=plot1,
                xmode=log,
                xlabel={Block production rate $\lambda$},
                %
                ylabel={Adversary res.\ $\beta$},
                legend columns=2,
                %
                %
                %
                xmax=1e-2, xmin=2e-6,
                %
                %
                %
                %
                %
                %
                %
                %
                %
                %
                %
                %
                %
                %
                %
                %
                %
                %
            ]


            %

            %
            %
            %
            %
            %
            %
            %
            %
            %


            %

            \addlegendimage{empty legend}
            \addlegendentry{%
                \tikz[x=2em,y=0.75em]{%
                    \draw [myparula21,thick] (0,0) -- (1,0);
                    \draw [myparula22,thick] (0,-0.5) -- (1,-0.5);
                    \draw [myparula23,thick] (0,-1.0) -- (1,-1.0);
                }%
                \hspace{0.5em}PoS NC security \cite{bwlimitedposlc}%
            };


            %

            \addlegendimage{empty legend}
            \addlegendentry{%
                \tikz[x=2em,y=0.75em,baseline=-0.35em]{%
                    \draw [myParula01Blue,thick] (0,0) -- (1,0);
                }%
                \hspace{0.5em}\ProtShort security (this work)%
            };


            %

            %
            %
            %
            %
            %
            %
            %



            

            \addplot [draw=none,name path=xaxis,domain={1e-8:1e8}] {0};
            \addplot [draw=none,name path=xaxisplus1,domain={1e-8:1e8}] {1};




            %
            
            %
            %
            %

            %
            %
            %


            %

            %
            %

            %


            %

            \addplot [myParula01Blue,no marks,name path=resiliencebdbandwidth] table [x=lbyC,y=beta] {figures/fig-comparison-bddelay-bdbandwidth-bdbandwidth-newresult.txt};

            \addplot [myParula01Blue,fill opacity=0.2] fill between [of=resiliencebdbandwidth and xaxis];


            %
            
            \addplot [myparula21,thick,no marks,name path=resilience10] table [x=lbyC,y=beta] {figures/fig-comparison-bddelay-bdbandwidth-bdbandwidth-oldresult-kappa10.txt};
            \addplot [myparula22,thick,no marks,name path=resilience100] table [x=lbyC,y=beta] {figures/fig-comparison-bddelay-bdbandwidth-bdbandwidth-oldresult-kappa100.txt};
            \addplot [myparula23,thick,no marks,name path=resilience1000] table [x=lbyC,y=beta] {figures/fig-comparison-bddelay-bdbandwidth-bdbandwidth-oldresult-kappa1000.txt};

            \addplot [myparula21,fill opacity=0.2] fill between [of=resilience10 and xaxis];
            \addplot [myparula21,fill opacity=0.2] fill between [of=resilience100 and xaxis];
            \addplot [myparula21,fill opacity=0.2] fill between [of=resilience1000 and xaxis];

        \end{axis}
    \end{tikzpicture}%
    \vspace{-0.5em}%
    \caption[]{%
        The region of 
        fraction $\beta$ of adversary nodes and
        block production rate $\lambda$
        where
        PoS NC is secure according to~\cite{bwlimitedposlc}
        (\tikz[x=0.75em,y=0.75em]{ \draw [draw=none,thick,fill=myParula02Orange,fill opacity=0.3] (0,0) rectangle (1,1); })
        shrinks as
        the NC confirmation depth increases, \ie,
        the desired consensus failure probability decreases
        %
        %
        %
        %
        (in order:
        \tikz[x=2em,y=0.75em,baseline=-0.35em]{%
        \draw [myparula21,thick] (0,0) -- (1,0);
        } 
        %
        %
        %
        %
        %
        to
        \tikz[x=2em,y=0.75em,baseline=-0.35em]{%
        \draw [myparula23,thick] (0,0) -- (1,0);
        }).
        Thus, for the PoS NC protocol of~\cite{bwlimitedposlc},
        security requires vanishing throughput.
        %
        %
        %
        %
        %
        %
        %
        %
        %
        %
        %
        %
        %
        %
        %
        %
        %
        %
        %
        %
        In contrast, our new \ProtShort protocol
        %
        achieves a security region (\tikz[x=0.75em,y=0.75em]{ \draw [draw=myParula01Blue,thick,fill=myParula01Blue,fill opacity=0.3] (0,0) rectangle (1,1); }) 
        that is
        independent of
        the desired consensus failure probability.
        Thus, \ProtShort is secure with non-vanishing constant throughput.
        (For all lines, processing capacity is fixed to $C=1$ block/s.)
    }%
    \label{fig:comparison-bddelay-bdbandwidth-pos}%
\end{figure}%