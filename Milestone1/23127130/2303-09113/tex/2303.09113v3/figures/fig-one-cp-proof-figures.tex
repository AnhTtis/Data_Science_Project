\begin{figure}[tb]%
    \centering%
    %
    \begin{tikzpicture}[x=2em,y=2em]%
    
        \begin{scope}[]
            
            \node at (-1.5,0) {\textbf{(a)}};
            \draw (-1,0) -- (4,0);
            
            \node [circle,fill=myParula01Blue,inner sep=2pt] (circle1) at (0,0) {};
            \node [circle,fill=myParula01Blue,inner sep=2pt] (circle2) at (1,0) {};
            \node [circle,fill=myParula01Blue,inner sep=2pt] (circle3) at (2,0) {};
            \node [circle,fill=myParula01Blue,inner sep=2pt] (circle4) at (3,0) {};
        
            %
            %
            
            \draw [draw=myParula07Red,rounded corners] ([xshift=-3pt,yshift=-3pt]circle1.south west) rectangle ([xshift=3pt,yshift=3pt]circle2.north east) node [midway,yshift=11pt,xshift=-0.75em] {\textcolor{myParula07Red}{$A$}};
            \draw [draw=myParula07Red,rounded corners] ([xshift=-3pt,yshift=-5pt]circle2.south west) rectangle ([xshift=3pt,yshift=5pt]circle4.north east) node [midway,yshift=13pt,xshift=1.75em] {\textcolor{myParula07Red}{$B$}};
            \draw [draw=myParula07Red,rounded corners] ([xshift=-5pt,yshift=-7pt]circle2.south west) rectangle ([xshift=5pt,yshift=7pt]circle3.north east) node [midway,yshift=15pt] {\textcolor{myParula07Red}{$C$}};
        
        \end{scope}
    
        \begin{scope}[xshift=4.6cm]
            
            \node at (-1.5,0) {\textbf{(b)}};
            \draw (-1,0) -- (4,0);
            
            \node [circle,fill=myParula01Blue,inner sep=2pt] at (0,0) {};
            \node [circle,fill=myParula01Blue,inner sep=2pt] at (0.8,0) {};
            \node [circle,fill=myParula01Blue,inner sep=2pt] at (2.2,0) {};
            \node [circle,fill=myParula01Blue,inner sep=2pt] at (3,0) {};
        
            \draw (-0.5,0) ++(0,0.2) -- ++(0,-0.4);
            \draw (3.5,0) ++(0,0.2) -- ++(0,-0.4);
            
            \draw [draw=myParula07Red,ultra thick] (1.5,0) ++(-0.15,-0.15) -- ++(0.3,0.3);
            \draw [draw=myParula07Red,ultra thick] (1.5,0) ++(-0.15,0.15) -- ++(0.3,-0.3);
            
            \draw [draw=myParula07Red,rounded corners] (-0.3,-0.3) rectangle (1.8,0.3);
            \draw [draw=myParula07Red,rounded corners] (1.2,-0.4) rectangle (3.3,0.4);
            
        
        \end{scope}
        
    \end{tikzpicture}%
    \caption{%
    Blue circles represent \sltpps, red crosses represent \iindices with $\Gat{k}=1$ and $\Dat{k}=0$. 
    (a) Given intervals $A,B,C$ all containing the 2nd blue circle from left, interval $C$ is redundant.
    (b) Given $n$ blue circles, the adversary needs at least $n/4$ red crosses to draw a set of intervals satisfying \eqref{intervals-cover-ppivots,intervals-y-condition}. Here is a placement of red crosses relative to blue circles that achieves the minimum number of red crosses.%
    %
    }
    \label{fig:one-cp-proof-figures}
\end{figure}
