\subsection{Definitions}
\label{sec:proof-definitions}

%
%
%
%
%
%
%
%
%
%
%
%
%
%
%
%
%
%
For any sequence $\{\Xat{k}\}$ and index set $I$,
let $\Xat{I} \triangleq \sum_{k \in I} X_k$.
%
%
%
%

\myparagraph{Probabilistic Model for PoW NC Executions}
%
A \emph{block production opportunity} (\BPO) is a pair $(p,t)$ where according
to the PoW block production lottery,
node $p$ is eligible to produce a block in \timeslot $t$.
A \BPO is \emph{honest} (resp.\ \emph{adversary}) if $p$ is honest (resp.\ adversary).
%
%
%
%
%
Since $N\to\infty$, and mining power is homogeneous,
honest (resp.\ adversary) \BPOs per slot are Poisson distributed
with parameter $(1-\beta)\rho$ (resp.\ $\beta\rho$).
%
%
%
%
%
An \emph{execution} refers to a particular realization of the block production lottery
for all \timeslots.

\myparagraph{\sltGood, \sltBad, and \sltEmpty \Timeslots}
%
%
\Timeslots without a \BPO are called \emph{`\sltempty'}.
%
A \timeslot is \emph{`\sltgood'} iff
it has 
exactly one honest \BPO and no adversary \BPOs,
and is followed by $\goodsep$ \sltempty \timeslots
(inspired by convergence opportunities \cite{pss16,sleepy,kiffer2018better}, loners \cite{dem20}, and laggers \cite{ren}).
Here, $\goodsep$ is an analysis parameter.
We define another analysis parameter $\goodsepbw$
which is related to $\goodsep$ as 
%
%
%
%
%
$(\goodsep+1)\slotduration \triangleq \DeltaHeader + \goodsepbw / \bwtime$.
%
Thus, $\goodsep, \goodsepbw$ are chosen so
that for a \sltgood \timeslot, every honest node can 
receive the block header for the honest \BPO, and
process content for $\goodsepbw$ blocks, before the next \BPO. 
%
%
Any non-\sltempty \timeslot which is not \sltgood is called \emph{`\sltbad'}.
%
%
%
%
%
%
%
%
%
%
%
%
%
%
%
%
%


We denote by $t_k$ the $k$-th non-\sltempty \timeslot.
Then, we can introduce random processes over \emph{\iindices},
with \iindex $k$ corresponding
to the $k$-th non-\sltempty \timeslot $t_k$.
Considering only \iindices simplifies notation considerably.
%
The process $\{\Gat{k}\}$  (`$G$' for \emph{good})
counts good \timeslots,
with $\Gat{k} \triangleq \Ind{ \predGood{t_k} }$.
%
%
%
Correspondingly, let $\Bat{k} \triangleq 1 - \Gat{k}$.
The following fact shows the distribution of \sltgood \iindices.
%
\begin{restatable}{proposition}{RestatePropXiIsIid}
%
    \label{prop:X_i-is-iid}
    The 
    %
    $\{\Gat{k}\}$ are independent and identically distributed (\iid) with
    %
        $\Prob{\Gat{k} = 1} \triangleq \probGood = \probGoodFormula$.
        %
    %
%
\end{restatable}
%
%
%
%
%
%
%
%
%
%
%
%
%
%
%
%
%
%
Throughout the analysis, we assume
$\probGood > \frac{1}{2}$ (`honest majority' assumption).


\myparagraph{Some \sltGood \Timeslots Imply Growth}
%
%
A special role is played by \sltgood \timeslots $t_k$
with the additional property that 
\emph{the block produced
at $t_k$ is `soon' processed by all honest nodes}.
Intuitively, these lead to \emph{chain growth},
the cornerstone of NC security~\cite{sleepy,dem20}.
We count these \timeslots with $\{\Dat{k}\}$ 
(`$D$' for \emph{downloaded}).
%
%
Specifically,
$\Dat{k} \triangleq 1$ if $t_k$ is good
%
\emph{and} the block produced at $t_k$
has been processed by all honest nodes by the end
of \timeslot $t_k + \goodsep$,
$\Dat{k} \triangleq 0$ otherwise,
and $\Nat{k} \triangleq 1 - \Dat{k}$.
%
%
%
%
%
%
%
    %
    %
    %
%
%
Note that 
%
%
$\{\Gat{k}\}$
are \iid, and not affected by adversary action,
while 
%
$\{\Dat{k}\}$ \emph{do depend}
on the adversary action and are thus in particular
\emph{not} \iidPERIOD.
%


\myparagraph{Probabilistic and Combinatorial Pivots}
%
%
\begin{definition}
    \label{def:pp-informal}
    We call an \iindex $k$ a \emph{\sltpp} (\emph{probabilistic pivot}),
    denoted as $\predPP{k}$, iff
    %
        $\predPP{k} \triangleq  
        %
        (\forall \intvl{i}{j} \ni k\colon  \Gin{i}{j} > \Bin{i}{j})$.%
        %
    %
    \footnote{We denote intervals 
    %
    as $\intvl{i}{j} \triangleq \{i+1,...,j\}$, with
    %
    $\intvl{i}{j} \triangleq \emptyset$ if $j \leq i$.}
\end{definition}
%
\begin{definition}
    \label{def:cp-informal}
    We call an \iindex $k$ a \emph{\sltcp} (\emph{combinatorial pivot}),
    denoted as $\predCP{k}$, iff
    %
        $\predCP{k} \triangleq 
        %
        (\forall \intvl{i}{j} \ni k\colon  \Din{i}{j} > \Nin{i}{j})$.
        %
    %
\end{definition}
%
This definition of \sltpps and \sltcps decouples \cite[Def.~5]{sleepy} into its \emph{probabilistic} aspects~\cite[Sec.~5.6.3]{sleepy} and \emph{combinatorial} aspects~\cite[Sec.~5.6.2]{sleepy},
and casts them as conditions
on a random walk,
inspired by~\cite{dem20,close-latency-security-ling-ren}, to simplify the analysis.
The decoupling is one of the key differences from the analysis in \cite{sleepy} (see \cref{fig:analysis-comparison-sleepy}).
Note that a \sltcp is also a \sltpp because 
%
$\Dat{i} = 1$ implies $\Gat{i} = 1$.

%