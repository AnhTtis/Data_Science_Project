\begin{figure}[tb]
    \centering
    \begin{tikzpicture}[x=0.7cm,y=0.4cm]
        \footnotesize

        \draw [draw=none,fill=myParula05Green,opacity=0.3] (0,0) rectangle (5,3);
        \draw [draw=none,fill=myParula07Red,opacity=0.3] (0,3) rectangle (5,6.5);
        \draw [draw=none,fill=myParula07Red,opacity=0.3] (6,0) rectangle (10.5,4);
        \draw [draw=none,fill=myParula05Green,opacity=0.3] (6,4) rectangle (10.5,6.5);
        \draw [draw=none,fill=myParula01Blue,opacity=0.3] (5,0) rectangle (6,6.5);

        \draw [-Latex] (0,0) -- (10.75,0) node [below] {$k$};
        \draw [-Latex] (0,0) -- (0,6.75) node [left] {$\Xin{0}{k}$};

        \foreach \x in {0,...,10} {
                \draw (\x,0) -- ++(0,0.1) -- ++(0,-0.2) node [below] {$\x$};
            }
        \foreach \x in {1,...,10} {
                \draw (\x,0) ++(-0.5,0) node [below=1em] {$\Xat{\x}$};
            }
        \foreach \y in {0,...,6} {
                \draw (0,\y) -- ++(0.1,0) -- ++(-0.2,0) node [left] {$\y$};
            }

        \pgfmathsetmacro{\sumY}{0}
        \foreach \d [count=\di from 0] in {1,1,1,-1,1,1,1,-1,1,1} {
                \draw [-latex,thick,shorten >=2pt,shorten <=2pt] (\di,\sumY) -- ++(1,\d);
                \pgfmathsetmacro{\tmp}{\sumY+\d}
                \global\let\sumY\tmp
            }

        \draw [thick,densely dotted] (10,6) -- ++(0.5,0.5);

    \end{tikzpicture}%
    \vspace{-0.5em}%
    \caption[]{Illustration of \sltpp
        (\eqref{pivot-conditions-equivalence-randomwalks}):
        A \sltpp as an \iindex $k$
        so that
        $\Xat{k} = 1$ (\tikz[x=0.75em,y=0.75em]{ \draw [draw=none,fill=myParula01Blue,opacity=0.3] (0,0) rectangle (1,1); })
        and
        $\Xin{0}{.}$ is strictly below
        $\Xin{0}{k}$ left of $k$
        and
        weakly above
        $\Xin{0}{k}$ right of $k$
        (\tikz[x=0.75em,y=0.75em]{ \draw [draw=none,fill=myParula05Green,opacity=0.3] (0,0) rectangle (1,1); }, \tikz[x=0.75em,y=0.75em]{ \draw [draw=none,fill=myParula07Red,opacity=0.3] (0,0) rectangle (1,1); }).%
    }
    \label{fig:pivot-randomwalk}
\end{figure}