\begin{figure}[tb]%
    \centering%
    \begin{tikzpicture}[]%
        %
        \scriptsize%
        \begin{axis}[
            mysimpleplot,
            grid = both,
            %
            xlabel={Time (seconds)},
            ylabel={Attacker lead (blocks)},
            %
            %
            xmin=0, xmax=500,
            ymin=0, ymax=100, 
            %
            height=0.5\linewidth,
            width=\linewidth,
            yticklabel style={
                    /pgf/number format/fixed,
                    /pgf/number format/precision=2
            },
            scaled y ticks=false,
            %
            %
            legend columns=4,  %
            %
            %
            %
            %
            %
            %
            legend style={
                    %
                    %
                    xshift=-3mm,
                    %
                    %
                    %
                            %
                        %
                    %
                            %
                        %
                },
        ]
            
            \addlegendimage{empty legend}
            \addlegendentry{\Teaserattack:};

            \addplot [myparula11, no markers,mark=none] table [col sep=comma, x=time,y=height_delta,restrict expr to domain={\thisrow{beta}}{0.75:0.75}]
            {figures/fig-experiment-teaser-start-data3.txt};
            \addlegendentry{$\blkratetimeAdv=0.75$};
            \label{plt:teaser-strong2};


            \addplot [myparula71, no markers,mark=none] table [col sep=comma, x=time,y=height_delta,restrict expr to domain={\thisrow{beta}}{0.6:0.6}]
            {figures/fig-experiment-teaser-start-data3.txt};
            \addlegendentry{$\blkratetimeAdv=0.60$};
            \label{plt:teaser-medium2};

            \addplot [myparula51, no markers, solid,mark=none] table [col sep=comma, x=time,y=height_delta,restrict expr to domain={\thisrow{beta}}{0.45:0.45}]
            {figures/fig-experiment-teaser-start-data3.txt};
            \addlegendentry{$\blkratetimeAdv=0.45$};
            \label{plt:teaser-weak2};

            \addlegendimage{empty legend}
            \addlegendentry{Private attack:};

            \addplot [myparula11, opacity=0.3, no markers,mark=none] table [col sep=comma, x=time,y=height_delta,restrict expr to domain={\thisrow{beta}}{0.75:0.75}]
            {figures/fig-experiment-private-start-data3.txt};
            \addlegendentry{$\blkratetimeAdv=0.75$};
            \label{plt:private-strong2};

            \addplot [myparula71, opacity=0.3, no markers,mark=none] table [col sep=comma, x=time,y=height_delta,restrict expr to domain={\thisrow{beta}}{0.6:0.6}]
            {figures/fig-experiment-private-start-data3.txt};
            \addlegendentry{$\blkratetimeAdv=0.60$};
            \label{plt:private-medium2};

            %
            %
            %
            %
        \end{axis}%
    \end{tikzpicture}%
    \vspace{-0.5em}%
    \caption{%
    Adversary lead (difference between adversary and honest  chain lengths) under private attack and \teaserattack.
    %
    The simulation consists of $100$ honest nodes with capacity $C=2$ blocks per second, collectively mining $\blkratetimeHon=1$ block per second, and one adversary node with variable mining rate $\blkratetimeAdv$ blocks per second.
    With these parameters, $\blkratetimeGrowthSilent = 0.67$ and $\blkratetimeGrowthTeaser = 0.50$ (from \cref{fig:experiment-teaser}). 
    A weak adversary ($\blkratetimeAdv = 0.45$, \ref{plt:teaser-weak2}) is unable to mine fast enough to gain a lead on the network. A stronger adversary ($\blkratetimeAdv = 0.60$) fails to gain a permanent lead through the private attack (\ref{plt:private-medium2}). But, through the \teaserattack, after repeatedly retrying during the first 200 seconds, eventually manages to maintain a lead (\ref{plt:teaser-medium2}). An even stronger adversary ($\blkratetimeAdv = 0.75$) succeeds almost at once under both strategies (\ref{plt:teaser-strong2},\ref{plt:private-strong2}).
    %
    }%
    \label{fig:experiment-teaser-start}%
\end{figure}%