\begin{figure}[tb]%
    \centering%
    \begin{tikzpicture}[x=58pt,y=0.4cm]
        %
        \scriptsize
        %
        \draw ([xshift=-10pt]0,0) -- (0,0);
        \draw [dashed] ([xshift=116pt]0,0) -- ([xshift=150pt]0,0);
        \draw ([xshift=208pt]0,0) -- ([xshift=218pt]0,0);
        %
        \begin{scope}
            \draw (0,0) -- (0.35,0);
            \draw [densely dotted] (0.35,0) -- (0.75,0);
            \draw (0.75,0) -- (1,0);
            \node [circle,draw=none,fill=myParula01Blue,inner sep=2pt] at (0.05,0) {};
            \node [circle,draw=none,fill=myParula02Orange,inner sep=2pt] at (0.15,0) {};
            \node [circle,draw=none,fill=myParula03Yellow,inner sep=2pt] at (0.25,0) {};
            \node [circle,draw=none,fill=myParula04Purple,inner sep=2pt] at (0.35,0) {};
            \node [circle,draw=none,fill=myParula05Green,inner sep=2pt] at (0.75,0) {};
            \node [circle,draw=none,fill=myParula06LightBlue,inner sep=2pt] at (0.85,0) {};
            \node [circle,draw=none,fill=myParula07Red,inner sep=2pt] at (0.95,0) {};
        \end{scope}
        \begin{scope}[xshift=58pt]
            \draw (0,0) -- (0.35,0);
            \draw [densely dotted] (0.35,0) -- (0.75,0);
            \draw (0.75,0) -- (1,0);
            \node [circle,draw=none,fill=myParula01Blue,inner sep=2pt] at (0.05,0) {};
            \node [circle,draw=none,fill=myParula02Orange,inner sep=2pt] at (0.15,0) {};
            \node [circle,draw=none,fill=myParula03Yellow,inner sep=2pt] at (0.25,0) {};
            \node [circle,draw=none,fill=myParula04Purple,inner sep=2pt] at (0.35,0) {};
            \node [circle,draw=none,fill=myParula05Green,inner sep=2pt] at (0.75,0) {};
            \node [circle,draw=none,fill=myParula06LightBlue,inner sep=2pt] at (0.85,0) {};
            \node [circle,draw=none,fill=myParula07Red,inner sep=2pt] at (0.95,0) {};
        \end{scope}
        \begin{scope}[xshift=150pt]
            \draw (0,0) -- (0.35,0);
            \draw [densely dotted] (0.35,0) -- (0.75,0);
            \draw (0.75,0) -- (1,0);
            \node [circle,draw=none,fill=myParula01Blue,inner sep=2pt] at (0.05,0) {};
            \node [circle,draw=none,fill=myParula02Orange,inner sep=2pt] at (0.15,0) {};
            \node [circle,draw=none,fill=myParula03Yellow,inner sep=2pt] at (0.25,0) {};
            \node [circle,draw=none,fill=myParula04Purple,inner sep=2pt] at (0.35,0) {};
            \node [circle,draw=none,fill=myParula05Green,inner sep=2pt] at (0.75,0) {};
            \node [circle,draw=none,fill=myParula06LightBlue,inner sep=2pt] at (0.85,0) {};
            \node [circle,draw=none,fill=myParula07Red,inner sep=2pt] at (0.95,0) {};
        \end{scope}
        %
        \draw ([yshift=3pt]0,0) -- (0,2);
        \draw ([yshift=3pt]1,0) -- (1,1);
        \draw ([yshift=3pt]2,0) -- (2,1);
        \draw ([yshift=3pt,xshift=34pt]2,0) -- ([xshift=34pt]2,1);
        \draw ([xshift=34pt,yshift=3pt]3,0) -- ([xshift=34pt]3,2);
        \draw [latex-latex] ([yshift=-3pt]0,1) -- ([yshift=-3pt]1,1) node [midway,fill=white] {$2 K_1$};
        \draw [latex-latex] ([yshift=-3pt]1,1) -- ([yshift=-3pt]2,1) node [midway,fill=white] {$2 K_1$};
        \draw [latex-latex] ([yshift=-3pt,xshift=34pt]2,1) -- ([yshift=-3pt,xshift=34pt]3,1) node [midway,fill=white] {$2 K_1$};
        \draw [latex-latex] ([yshift=-3pt]0,2) -- ([xshift=34pt,yshift=-3pt]3,2) node [midway,fill=white] {$2 K_1 K_2$};
    \end{tikzpicture}
    \caption{%
        An illustration for the proof of abundance of \sltpps (\appendixRef{\cref{prop:lower-tailbound-ppivots}}). Given a long interval of size $2K_1K_2$, we partition it into $K_2$ intervals of size $2K_1$ each, and we group the \iindices as indicated by different colors. All \iindices of the same color are at least $2K_1$ apart, so that intervals of size at most $K_1$ surrounding two \iindices from the same group are disjoint, and hence the corresponding \sltpp conditions are independent (conditioned on the fact that the \sltpp condition holds for all long intervals).
    }%
    \label{fig:ppivot-tailbound-illustration}%
\end{figure}%