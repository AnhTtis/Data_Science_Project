\begin{figure}[tb]%
    \centering%
    \begin{tikzpicture}[]%
        %
        \scriptsize
        \begin{axis}[
                mysimpleresilienceplot01,
                name=plot1,
                xmode=log,
                xlabel={Block production rate $\lambda$},
                height=0.5\linewidth,
            ]

            \addplot [draw=none,name path=xaxis,domain={1e-8:1e8}] {0};
            \addplot [draw=none,name path=xaxisplus1,domain={1e-8:1e8}] {1};

            \addplot [black,dashed,no marks,name path=resiliencebddelay] table [x=relblockfrequency,y=beta] {figures/fig-comparison-bddelay-bdbandwidth-bddelay.txt};
            \label{leg:comparison-bddelay-bdbandwidth-privateattack}

            \addplot [myparula51,fill opacity=0.5,pattern=crosshatch dots,pattern color=myParula05Green] fill between [of=resiliencebddelay and xaxis];
            \addplot [myparula71,pattern=north east lines,pattern color=myParula07Red] fill between [of=resiliencebddelay and xaxisplus1];

            \addplot [myparula71,thick,no marks,name path=teaserattack] table [x=lbyC,y=beta] {figures/fig-comparison-bddelay-bdbandwidth-exp_teaser-attack.txt};
            \addplot [myparula71,fill opacity=0.3] fill between [of=teaserattack and xaxisplus1];

            \addplot [myparula51,no marks,name path=resiliencebdbandwidth] table [x=lbyC,y=beta] {figures/fig-comparison-bddelay-bdbandwidth-bdbandwidth-newresult.txt};

            \addplot [myparula51,fill opacity=0.45] fill between [of=resiliencebdbandwidth and xaxis];

            \node [align=center] (attack) at (axis cs:1e-1,0.1) {\Teaserattack\\{}(\cref{sec:teaser-attack})};
            \draw [-latex,shorten >=1pt,shorten <=-2pt] (attack) -- (axis cs:1.5,0.2);

            \node [align=center] (security) at (axis cs:5e-3,0.40) {Our security proof\\{}(\cref{thm:safety-and-liveness-pow})};
            \draw [-latex,shorten >=2pt] (security) -- (axis cs:1e-3,0.27);

            \node [align=center] (bddelay) at (axis cs:1e1,0.40) {Private attack\\{}\cite{dem20,tight_bitcoin}};
            \draw [-latex] (bddelay) -- (axis cs:4e0,0.21);
            
            %
            %
        \end{axis}
    \end{tikzpicture}%
    \vspace{-0.5em}%
    \caption[]{%
        Regions of
        fraction $\beta$ of adversary nodes and
        block production rate $\lambda$
        with
        security proofs
        (\tikz[x=0.75em,y=0.75em]{ \draw [draw=myParula05Green,thick,fill=myParula05Green,fill opacity=0.3] (0,0) rectangle (1,1); })
        and attacks
        (\tikz[x=0.75em,y=0.75em]{ \draw [draw=myParula07Red,thick,fill=myParula07Red,fill opacity=0.3] (0,0) rectangle (1,1); })
        for NC under a fixed processing capacity of $C=1$ block per second.
        Analysis in the bounded-delay model \cite{dem20,tight_bitcoin} (with $\Delta = 1\;\mathrm{s}$) proves that the private attack
        (\ref{leg:comparison-bddelay-bdbandwidth-privateattack})
        succeeds
        (\tikz[x=0.75em,y=0.75em]{ \draw [draw=none,fill opacity=1,pattern=north east lines,pattern color=myParula07Red] (0,0) rectangle (1,1); })
        %
        iff
        $\beta \geq \frac{1-\beta}{1+(1-\beta)\blkratetime}$, and that for all other values of $\beta,\lambda$, no attack succeeds
        (\tikz[x=0.75em,y=0.75em]{ \draw [draw=none,fill opacity=1,pattern=crosshatch dots,pattern color=myParula05Green] (0,0) rectangle (1,1); }).
        %
        %
        Our \teaserattack exploits congested block processing and succeeds at lower adversary $\beta$ than the private attack
        (\tikz[x=0.75em,y=0.75em]{ \draw [draw=myParula07Red,thick,fill=myParula07Red,fill opacity=0.3] (0,0) rectangle (1,1); }, \tikz[x=0.75em,y=0.75em]{ \draw [draw=none,fill opacity=1,pattern=north east lines,pattern color=myParula07Red] (0,0) rectangle (1,1); }).
        Our analysis in a bounded-capacity model 
        %
        yields
        a security region
        (\tikz[x=0.75em,y=0.75em]{ \draw [draw=myParula05Green,thick,fill=myParula05Green,fill opacity=0.3] (0,0) rectangle (1,1); })
        for PoW NC.%
    }%
    \label{fig:comparison-bddelay-bdbandwidth}%
\end{figure}%