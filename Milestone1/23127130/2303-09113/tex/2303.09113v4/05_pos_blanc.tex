\section{Proof-of-Stake Nakamoto Consensus}

Nakamoto consensus has been adapted to proof-of-stake in protocols of the Ouroboros~\cite{kiayias2017ouroboros,david2018ouroboros,badertscher2018ouroboros} and Sleepy Consensus~\cite{sleepy,snowwhite} families.
The protocol is identical to what was described in \cref{sec:modelprotocol} and formalized in \appendixRef{\cref{alg:generic-lc-protocol}}, except for a few key differences.
The block production oracle for proof-of-stake (idealized in \appendixRef{\cref{alg:hdrtree-pos}}) behaves differently.
As in PoW, each node can make one block production attempt per \timeslot that will be successful with probability $\blkrateslot/N$, independently of other nodes and \timeslots%
%
\footnote{There may be multiple blocks in one \timeslot, as in
%
%
the Ouroboros~\cite{kiayias2017ouroboros,david2018ouroboros,badertscher2018ouroboros} and Sleepy Consensus~\cite{sleepy,snowwhite} protocols.}%
, modeling uniform stake.
In PoS, however, (even past) block production opportunities can be `reused' to produce multiple blocks with different parents and/or content, \ie, to equivocate. 
%
%
%

\subsection{\ProtLong (\ProtShort)}
\label{sec:sapos}

%

%


%
%
%

%

%
%
In the classic bounded-delay analysis, the tradeoff between 
%
$\beta$ and $\lambda$
%
%
is the same for 
%
PoW and PoS NC~\cite{dem20,tight_bitcoin},
because, conceptually,
%
NC security depends only on 
a 
%
race between
the honest chain and adversary chains.
%
%
%
%
%
Even in PoS,
the adversary cannot 
use
%
equivocations
%
%
%
to boost
its chain growth rate,
because blocks within one chain must be from strictly increasing \timeslots, \ie, different \BPOs.
%
%
Under bounded capacity, however, as observed in~\cite{bwlimitedposlc}, honest nodes may waste their limited capacity processing 
equivocations
%
rather than staying up-to-date with the longest chain.
Thus, blocks they produce may not contribute to honest chain growth.
%
%
%
As a result,
%
%
the honest chain growth rate decreases, 
and with it
%
%
PoS NC security~\cite{bwlimitedposlc} (compared to PoW).
%
%
%
%
The key idea in \ProtShort is to modify the scheduling policy of PoS NC 
such that per \BPO at most one block is processed.
This restores the one assumption of the bounded-capacity PoW NC analysis (\cref{sec:proof-analysis-many-pps-one-cps}~(P1))
that was previously violated in PoS NC due to equivocations.
With the modification of \ProtShort,
the analysis from \cref{sec:proof} carries over to PoS.
%
%
%
%
%
%
%
%
%

%
%
One 
may consider 
this
%
alternative:
%
defer content processing until after consensus has been reached on a header chain.
This, however, requires
%
ensuring
%
%
%
that the contents belonging to headers
will be available for download.
%
%
%
Sampling-based approaches~\cite{DBLP:conf/fc/Al-BassamSBK21}
to check \emph{data availability}
come with various challenges~\cite{paradigmdas}
and
%
VID-based
approaches~\cite{dispersedledger,poar}   %
%
%
do not
%
%
scale to the large $N$ 
%
found in PoS NC.
%

%
%
%




\subsubsection{Protocol}
\label{sec:sapos_protocol}

%

%
%
%
%
%
%
%

%
%


\ProtShort is PoS NC (\cf\cite{kiayias2017ouroboros,sleepy}),
%
with the following modifications.

\myparagraph{The Scheduling Policy in \ProtShort}
%
\ProtShort uses
%
any scheduling policy that is secure for PoW NC (such as \rulelc),
modified as follows:
%
a node does not process content for a header
(denoted by the corresponding header chain $\Chain$)
if it has seen 
another equivocating header
%
from the same \BPO as $\Chain$.
%
%
Instead,
%
the node \emph{pretends} that content was ``processed'' and sets it to be empty.
%
%
%
The node can then continue processing content for headers that extend $\Chain$, and these blocks will be candidates for the node's longest processed chain.
%
%
With only the above scheduling policy,
one honest node may process
the real
content for a header while another 
%
may 
%
set it to be empty
(depending on when each node saw an equivocating header).
In order to output a consistent 
transaction ledger,
%
%
reaching consensus on the header chain is 
no longer
%
enough.
%
%
Instead, we ensure that honest nodes also agree on which blocks had an equivocation, through equivocation proofs, 
so that they can consistently \emph{blank} their contents.


\myparagraph{Equivocation Proofs}
An equivocation proof consists of two headers $\Chain, \Chain'$ from the same \BPO.
Whenever a node produces a new block header extending its longest processed chain,
%
%
it 
includes an equivocation proof
for any header $\Chain$ among the last $\keqproof$ headers
%
(on the new block's prefix)
for which it has seen an equivocating header $\Chain'$
but no equivocation proof 
%
was
%
recorded on chain yet.
%
%
%
%
%
%
%
%
%
%
%
%


\myparagraph{Equivocation Proof Deadline}
The deadline $\keqproof$ for adding equivocation proofs 
ensures
%
that 
%
%
%
the adversary cannot 
use equivocations or equivocation proofs
to make honest nodes blank the content of an old block
whose transactions they have already confirmed.
%
%
%
%
%
%
%
%
%
%
%
A header $\Chain$ is thus \emph{invalid}
if it contains an equivocation proof against 
a block 
that is 
not within $\keqproof$ blocks above $\Chain$.
%
%
%


\myparagraph{Ledger Construction in \ProtShort}
%
%
At the end of each \timeslot,
each node confirms 
%
%
all blocks on its longest processed chain that 
%
are $\confDepth$-deep,
except it blanks
%
%
the contents of 
blocks
%
%
against which there is an equivocation proof
on chain.
%
%
%




\subsubsection{Security Proof}
\label{sec:security}

%

The scheduling policy of \ProtShort ensures that,
just like in PoW NC,
honest nodes process at most one block per \BPO.
%
This eliminates additional block processing delays caused by equivocations, allowing the honest chain growth rate
to match that of PoW NC.
%
%
Given this, the security proof of PoW NC in \cref{sec:proof}
can be adapted to \ProtShort to 
%
show that
the $\confDepth$-deep \emph{header chains} of all nodes are consistent.
%
%
%
%
%
%


To ensure that their \emph{ledgers} are consistent, and complete the security proof, we need two more steps.
First, liveness of \ProtShort follows easily because the contents of blocks produced by honest nodes will never be blanked.
%
Second, for safety,
%
we show (a) honest nodes have processed the content for all blocks against which there is no equivocation proof on chain (these blocks must not be blanked), and (b) honest nodes blank content in their ledger consistently, that is, any honest node blanks the contents of a block in its ledger iff all honest nodes do so.
%
%
%
We prove (a) and (b) in \cref{thm:safety-and-liveness-pos} by choosing appropriate values for $\confDepth$ and $\keqproof$.




%
%
%
%
%
%
%
%
%




%
%
%
%
%
%
%
%
%
%
%
%
%
%
%
%
%
%

%
%
%
%
%
%
%
%
%
%
%
%
%
%
%
%
%
%
%
%
%
%
%
%
%
%
%
%
%
%
%
%
%
%
%
%
%
%
%
%
%
%
%



Since the analysis of PoW NC from 
\cref{sec:proof} (details in \appendixRef{\cref{sec:fullproof}}) applies to \ProtShort as well, \cref{lem:cps-stabilize-informal} (\sltcps stabilize the longest processed chains of all nodes) and \cref{lem:many-pps-one-cps-informal} (\sltcps recur) hold for \ProtShort.
%
Thus, \eqref{pow-max-tp} also determines the parameters under which \ProtShort is secure, \ie,
%
%
%
%
%
%
%
the security--throughput tradeoff of \ProtShort.%
\footnote{Technically, since PoS protocols run in \timeslots of fixed duration, unlike PoW, $\slotduration$ must match the \timeslot duration. If $\slotduration$ is small relative to the block production and processing times (such as $1$ second in Cardano), we can still use the approximation $\slotduration \to 0$, just like in PoW. We calculate the parameters for general $\slotduration$ in \appendixRef{\cref{sec:fullproof-analysis-many-pps-one-cps}}.}
In \cref{fig:comparison-bddelay-bdbandwidth-pos},
we plot the solutions of \eqref{pow-max-tp} with 
$\bwtime = 1$ and
$\DeltaHeader \approx 0$ (approximation for block content much larger than headers).
Since \eqref{pow-max-tp} does not depend on $\kappa$, for any given $\beta$, the block rate $\lambda$ is non-vanishing.
Only latency scales with $\kappa$, similar to PoW NC.
%
%
%

%

In both \cref{thm:safety-and-liveness-pos} 
(for \ProtShort) and \cref{thm:safety-and-liveness-pow} (for PoW NC), we prove an upper bound on the confirmation latency that scales with the security parameter $\kappa$ as $O(\kappa^2)$.
Concretely, our bound on \ProtShort's latency (\cref{thm:safety-and-liveness-pos}) is $3\times$ our bound for PoW NC (\cref{lem:safety-and-liveness-comb-pow}).


\begin{theorem}
\label{thm:safety-and-liveness-pos}
For all $\beta < 1/2$,
%
$\bwtime$, $\DeltaHeader$, $\blkrateslot$, $\slotduration$
satisfying \eqref{pow-max-tp},
there exists $\keqproof, \confDepth = \Theta(\kappa^2)$
such that the \emph{\ProtShort protocol
%
%
%
%
%
%
%
%
is secure}
with
transaction rate $(1 - 2\beta)\blkratetime$,
confirmation latency $\Tlive = \Theta(\kappa^2)$ \timeslots
over a time horizon of 
%
$\Thorizon = \poly(\kappa)$.
%
\end{theorem}

\begin{proof}
\import{./figures/}{fig-pos-safety-proof.tex}

%

Set $\confDepth \triangleq 6\Kcp + 1, \keqproof \triangleq 4\Kcp$.
%
Denote the longest processed chain of node $p$ at \timeslot $t$ as $\dC_p(t)$ and its $\confDepth$-deep prefix as $\dC_p(t)\trunc{\confDepth}$.
\emph{Safety} holds if the following three properties hold for all \timeslots $t \leq t'$ and for all honest nodes $p,q$:
%
(1) $\dC_p(t)\trunc{\confDepth} \preceq \dC_q(t')\trunc{\confDepth}$ or $\dC_q(t')\trunc{\confDepth} \preceq \dC_p(t)\trunc{\confDepth}$.
%
(2) If $b \in \dC_p(t)\trunc{\confDepth}$ and there is no equivocation proof in a block header following it, then node $p$ must have processed the content of $b$ before \timeslot $t$.
%
(3) 
%
If $b \in \dC_p(t)\trunc{\confDepth}$ and $b \in \dC_q(t')\trunc{\confDepth}$, then $p$ blanks the content of $b$ in $\LOG{p}{t}$ iff $q$ blanks it in $\LOG{q}{t'}$.

Consider an arbitrary block $b_i$ (produced at some \iindex $i$) that is confirmed by an honest node $p$ at \timeslot $t$, \ie, $b_i \in \dC_p(t)\trunc{\confDepth}$.
Since $b_i$ is $\confDepth$-deep, there must have been at least $6\Kcp + 1$ \iindices after $i$.
%
Due to \cref{lem:many-pps-one-cps-informal}, there must have been at least three \sltcps $j,k,l$ after \iindex $i$.
Due to \cref{lem:cps-stabilize-informal}, the blocks produced at these \iindices, $b_j,b_k,b_l$ are in $\dC_q(t')$ for all $t' \geq t$ and for all $q$ (see \cref{fig:pos-safety-proof}).
Therefore, $\dC_p(t)\trunc{\confDepth} \preceq \dC_q(t')$, and from this, we can prove (1).

To prove (2), suppose that node $p$ did not process the content of block $b_i$.
Since block $b_j$, and hence also $b_i$, is in every honest node's longest processed chain at \timeslot $t_{k+1} - 1$ (\cref{lem:cps-stabilize-informal}), it must have been that $p$ saw an equivocation for $b_i$ before \timeslot $t_{k+1} - 1$ (otherwise it must have actually processed the content of $b_i$).
Due to synchrony, all honest nodes see the headers of $b_i$ and its equivocation.
Since the block $b_k$ is produced by an honest node, and $k < i + 4\Kcp = i + \keqproof$, $b_k$ must contain an equivocation proof against $b_i$ (see \cref{fig:pos-safety-proof}).

To prove (3), we show that while confirming the block $b_i$, either all nodes see an equivocation proof against $b_i$ or none of them do.
The latest that an equivocation proof against $b_i$ can be included is $\keqproof$ blocks below $b_i$.
Since $\confDepth > \keqproof + 2\Kcp$, due to \cref{lem:many-pps-one-cps-informal}, the \sltcp $l$ must have occurred after $b_i$ became $\keqproof$-deep and before it became $\confDepth$-deep (see \cref{fig:pos-safety-proof}).
Thus, for all $p$ and $t$, if $b_i \in \dC_p(t)\trunc{\confDepth}$, then $b_l \in \dC_p(t)$, hence all nodes agree on whether or not an equivocation proof was included.

\emph{Liveness} follows similarly to PoW NC: Within $2\Kcp$ \iindices, there are enough honestly produced blocks to include new transactions, and their contents will never be blanked. In $\Theta(\Kcp)$ \timeslots, these blocks will become $\confDepth$-deep and will be confirmed.
\end{proof}



%
%
%
%
%
%
%
%
%
%
%
%
%
%
%
%
%
%
%
%
%
%
%
%
%
%
%
%
%
%
%
%
%
%
%
%
%
%
%
%
%
%
%
%
%
%
%
%
%
%
%
%
%
%
%
%
%
%
%
%
%
%
%
%
%
%
%
%
%
%
%
%
%
%
%
%
%
%
%
%
%
%
%
%
%
%
%
%
%
%
%
%
%
%
%
%
%
%
%
%
%
%
%
%
%
%
%
%
%
%
%
