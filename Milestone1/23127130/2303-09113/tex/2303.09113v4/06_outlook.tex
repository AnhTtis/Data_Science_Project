\section{Discussion}
\label{sec:conclusion}
%
%


%
%
%
%



%
%
%
\myparagraph{Tightening the Analysis}
%
Our \teaserattack and security analysis (\cf \cref{fig:comparison-bddelay-bdbandwidth}) serve as the first lower and upper bounds on nodes' minimum capacity required to ensure security in the bounded-capacity PoW setting. A question remains on how to tighten the gap.
%
One avenue for future work is whether the adversary has better strategies
than the \teaserattack, which we believe may be optimal in the bounded-capacity model.
%
\begin{conjecture}
    For the PoW NC protocol with the \rulelc scheduling policy,
    for all $\beta < 1/2, \blkratetime$ for which the \teaserattack is unsuccessful,
    the protocol is secure (against all attacks).
\end{conjecture}

Conversely, we expect that the security analysis can be improved in multiple ways. The current analysis uses only a few basic properties \ref{item:good-download-rule-no-repeat}, \ref{item:good-download-rule-honest-block}, \ref{item:good-download-rule-cutoff} of the scheduling policy. As a result, we assume that any valid block can be used by the adversary to spam honest nodes. However, when using the \rulelc policy, the adversary can only force honest nodes to process blocks that are on their longest header chain, which is already hard for the adversary given that the honest chain has been growing so far. An improved analysis should account not just for the number of block productions in the adversary's budget but also their blocktree structure. Further, \sltgood \timeslots are sufficient but not necessary for chain growth. Improved analysis of the chain growth rate using techniques such as blocktree partitioning \cite{dem20} can further tighten the analysis.

%

%
%
%
%
%
%
%
%
%
%
%
%

%
%
%


%
%
\myparagraph{Variable Difficulty}
%
In practice, PoW blockchains implement a difficulty adjustment algorithm (DAA) to maintain the target block rate as players join and leave the system.
%
This introduces new avenues for attack \cite{bahack2013theoretical}.
%
The variable difficulty protocol has so far been proven secure only in the lock-step synchronous 
model (\ie, messages delivered in exactly one round) 
%
\cite{garay2017bitcoin}. Security in the bounded-delay and bounded-capacity models remains an open problem.
We note, however, that the DAA seems to apply even more stress to limited capacity nodes,
as 
it would
lower the difficulty to compensate for chain growth rate lost due to congestion,
leading to an increase in the overall block rate of the system.
In turn, this would increase congestion, in particular
if honestly produced orphaned blocks are processed by honest parties, leading to a vicious cycle.
%
%
%
%
%
%
%
%
%
%
%
Under the \rulelc scheduling policy that we consider in this work,
honest nodes do not prioritize processing orphaned blocks,
but this appears to be the case for scheduling policies that 
%
allow for processing multiple blocks in parallel. The nuance in this analysis is left for future work.



%

%
%
%


%

%


%
%
%
%


%


%

