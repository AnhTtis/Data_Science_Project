\subsubsection{Combinatorial Pivots Stabilize}
\label{sec:fullproof-analysis-cps-stabilize}

We now show that the honest block produced in a \timeslot corresponding to a \sltcp persists in the longest processed chain of all honest nodes forever after $\goodsep$ \timeslots after it was produced.
Towards this, we first show that if $\Dat{k}=1$, \ie, if all honest nodes process the block produced in the \sltgood \timeslot $t_k$,
then the length of the longest processed chain of honest nodes increases, \ie, a \emph{chain growth event} (made precise in \appendixRef{\cref{prop:chain-growth}}).
Due to this, since, by \appendixRef{\cref{def:cp}}, all intervals around a \sltcp contain more \iindices with $\Dat{k}=1$ than those with $\Dat{k}=0$,
there can never be some honest node with a longest processed chain that does not contain the block corresponding to the \sltcp (\appendixRef{\cref{lem:cps-stabilize}}).
This is because there are not enough blocks for any other chain to outnumber the \emph{chain growth events} that contributed to the growth of the processed chain containing the \sltcp's block.
Thus, the block corresponding to the \sltcp remains in all honest nodes' longest processed chains forever.
%
\appendixRef{\cref{lem:cps-stabilize}} is proven
analogously to the combinatorial argument of~\cite{sleepy}.


Recall that $\dC_p(t)$ is the longest processed chain of node $p$ at the end of \timeslot $t$, $\len{\Chain}$ denotes the length of chain $\Chain$, $L_p(t) = \len{\dC_p(t)}$ and the length of the ``shortest (across honest nodes) longest processed chain'' is $L_{\min}(t) = \min_p L_p(t)$ (where ``$\min_p$'' ranges only over honest nodes).
%
The following proposition says that 
%
$L_{\min}(t)$
grows for every \iindex $k$ with $\Dat{k}=1$,
\ie, these are ``chain growth events''.
\begin{proposition}
\label{prop:chain-growth}
If $\Dat{k} = 1$, then $L_{\min}(t_k + \goodsep) \geq L_{\min}(t_k-1) + 1$.
\end{proposition}
\begin{proof}
Since $\Dat{k} = 1$, \timeslot $t_k$ is a \sltgood \timeslot.
Let $b$ be the unique honest block produced in \timeslot $t_k$, and let honest node $p$ be its producer.
Since honest nodes produce blocks on their longest processed chain, $\len{b} = L_p(t_k-1) + 1 \geq L_{\min}(t_k-1) + 1$.
Further, $\Dat{k} = 1$ means that the block $b$ is processed by all honest nodes by the end of \timeslot $t_k + \goodsep$. Therefore, $L_{\min}(t_k + \goodsep) \geq \len{b}$.
\end{proof}

\begin{lemma}
\label{lem:cps-stabilize}
Let $b^*$ be the block produced in a non-\sltempty \timeslot $t_k$ such that $\predCP{k}$. 
Then, for all header chains $\Chain'$ that are valid at \timeslot $t \geq t_k + \goodsep$ and $\len{\Chain'} \geq L_{\min}(t)$: $b^* \in \Chain'$.
Also then, for all honest nodes $p$ and for all \timeslots $t \geq t_k + \goodsep$: $b^* \in \dC_p(t)$.
\end{lemma}

The following proposition is helpful for proving \appendixRef{\cref{lem:cps-stabilize}}.
\begin{proposition}
    \label{prop:chain-growth-interval}
    For any $i < j$,
    \begin{IEEEeqnarray}{C}
        L_{\min}(t_j + \goodsep) \geq L_{\min}(t_{i+1} - 1) + \Din{i}{j}.
    \end{IEEEeqnarray}
\end{proposition}
\begin{proof}
By noting that if $\Dat{k} = 1$, then $t_{k+1} > t_k + \goodsep$, and adding the result of \appendixRef{\cref{prop:chain-growth}} for each \iindex with $\Dat{k}=1$.
\end{proof}

\begin{proof}[Proof of \appendixRef{\cref{lem:cps-stabilize}}]
    Note that $\dC_p(t)$ is a valid chain at \timeslot $t$ and $\len{\dC_p(t)} = L_p(t) \geq L_{\min}(t)$. Therefore, it suffices to show the first claim of the lemma.
    
    For contradiction, let $s \geq t_k + \goodsep$ be the first \timeslot in which 
    there is a valid header chain $\Chain'$ such that 
    $\len{\Chain'} \geq L_{\min}(s)$ and $b^* \not\in \Chain'$.
    
    Let $b'$ be the block with maximum height on the chain $\Chain'$, such that $b'$ was produced in a \timeslot $t_i$ with $D_i = 1$.
    For $\Chain'$ to be a valid chain at \timeslot $s$, we need $t_i \leq s$.
    Since the block $b'$ is produced by an honest node, $b'$ extends $\dC_q(t_i-1)$ for some honest node $q$.
    Therefore, $\dC_q(t_i-1)$ is a prefix of $\Chain'$.
    This means that $b^* \not\in \dC_q(t_i-1)$.
    Moreover, $\len{\dC_q(t_i-1)} = L_q(t_i-1) \geq L_{\min}(t_i-1)$.
    If $i > k$, then $t_i-1 \geq t_k + \goodsep$ (since $D_k = 1$) and $t_i - 1 < s$ (shown above). 
    This is a contradiction because we assumed that $s$ is the first \timeslot such that $s \geq t_k + \goodsep$ and 
    %
    $b^* \notin \Chain'$ and $\len{\Chain'} \geq L_{\min}(s)$ for some valid chain $\Chain'$.
    Since $b^*$ is the only block produced in slot $t_k$, $i=k$ is also not possible.
    We conclude that $i < k$.
    
    Since $D_i = 1$ and $b'$ is produced in \timeslot $t_i$,
    \begin{IEEEeqnarray}{C}
    \label{eq:block-i-download}
        L_{\min}(t_i + \goodsep) \geq \len{b'}.
    \end{IEEEeqnarray}
    %
    By assumption,
    \begin{IEEEeqnarray}{C}
    \label{eq:block-j-switch}
        \len{\Chain'} \geq L_{\min}(s).
    \end{IEEEeqnarray}
    
    Let $t_j$ be the last non-\sltempty \timeslot such that $t_j \leq s$. Note that $j \geq k > i$. 
    We must consider two cases:
    %
    \begin{enumerate}
    \item Case 1: $s \geq t_j + \goodsep$ or $\Dat{j}=0$.
    If $\Dat{j}=0$, we don't have to worry about whether the block from slot $t_j$ was processed by all honest nodes.
    If $\Dat{j} = 1$ but $s \geq t_j + \goodsep$, then we know that all honest nodes have processed the block from slot $t_j$ before the end of \timeslot $s$. That is,
    \begin{IEEEeqnarray}{rCl}
        L_{\min}(s) 
        &\geq& L_{\min}(t_j + \goodsep)  \IEEEeqnarraynumspace \\
        &\geq& L_{\min}(t_{i+1}-1) + \Din{i}{j} \quad \text{(from \cref{prop:chain-growth-interval})}  \IEEEeqnarraynumspace \\
        \label{eq:chain-growth-case1}
        &\geq& L_{\min}(t_{i} + \goodsep) + \Din{i}{j}. \IEEEeqnarraynumspace 
    \end{IEEEeqnarray}
    By definition of $b'$, all blocks in $\Chain'$ appearing after $b'$ correspond to \iindices $l$ with $\Dat{l}=0$. These blocks must be from distinct \iindices greater than $i$ but at most $j$. So,
    \begin{IEEEeqnarray}{C}
    \label{eq:adv-chain-case1}
        \len{\Chain'} \leq \len{b'} + \Nin{i}{j}.
    \end{IEEEeqnarray}
    From \eqref{block-i-download,block-j-switch,chain-growth-case1,adv-chain-case1}, we derive
    \begin{IEEEeqnarray}{rCl}
    \label{eq:pivot-contra-case1}
        \Din{i}{j} \leq \Nin{i}{j} \implies \Yin{i}{j} \leq 0 \implies \Yin{0}{i} < \Yin{0}{j} \IEEEeqnarraynumspace
    \end{IEEEeqnarray}
    where $i < k \leq j$.
    
    \item Case 2: $t_j \leq s < t_j + \goodsep$ and $\Dat{j} = 1$.
    In this case, the block from slot $t_j$ may not have enough time to be processed by all honest nodes before the end of slot $s$.
    However, for any $l < j$ such that $\Dat{l} = 1$, $t_l + \goodsep < t_j \leq s$, so there is enough time to process the block from \timeslot $t_l$.
    Let $l \in\intvl{i}{j-1}$ be the greatest index such that $\Dat{l} = 1$. Then, $t_j > t_l + \goodsep$, and $\Din{i}{l} = \Din{i}{j-1}$.
    \begin{IEEEeqnarray}{rCl}
        \label{eq:chain-growth-case2}
        L_{\min}(s) 
        &\geq& L_{\min}(t_j) \\
        &\geq& L_{\min}(t_l + \goodsep) \\
        &\geq& L_{\min}(t_{i+1} - 1) + \Din{i}{l} \quad \text{(from \cref{prop:chain-growth-interval})} \IEEEeqnarraynumspace \\
        &\geq& L_{\min}(t_{i} + \goodsep) + \Din{i}{j-1}.
    \end{IEEEeqnarray}
    Note that since $\Dat{j}=1$, $\Nin{i}{j} = \Nin{i}{j-1}$. Therefore, as in the previous case,
    \begin{IEEEeqnarray}{C}
    \label{eq:adv-chain-case2}
        \len{\Chain'} \leq \len{b'} + \Nin{i}{j-1}.
    \end{IEEEeqnarray}
    From \eqref{block-i-download,block-j-switch,chain-growth-case2,adv-chain-case2},
    \begin{IEEEeqnarray}{rCl}
    \label{eq:pivot-contra-case2}
        \Din{i}{j-1} \leq \Nin{i}{j-1} &\implies& \Yin{i}{j-1} \leq 0 \nonumber \\ 
        &\implies& \Yin{0}{i} < \Yin{0}{j-1}. \IEEEeqnarraynumspace
    \end{IEEEeqnarray}
    Note that since we assumed $s \geq t_k + \goodsep$ and $s < t_j + \goodsep$, we know that $j > k$. Therefore, $i < k \leq j-1$.
    \end{enumerate}
    %
    In either case, \eqref{pivot-contra-case1} or \eqref{pivot-contra-case2} contradict the assumption $\predCP{k}$ (\appendixRef{\cref{def:cp}}).
    %
\end{proof}

