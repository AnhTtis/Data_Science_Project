\subsection{Handling Loss of Predictable Validity}
\label{sec:predictablevalidity}

\subsubsection{Predictable Transaction Validity}
\label{sec:predictablevalidity-transaction}

%
%
In UTXO-based chains like Bitcoin (account-based like Ethereum), a transaction is \emph{valid} iff its inputs are unspent (its execution succeeds and fees are paid).
%

\begin{definition}
\label{def:predictablevalidity-transaction}
    %
    A transaction has \emph{predictable validity} iff: 
    validity
    %
    at the time an honest node adds it to a block
    implies validity when that block gets executed.
\end{definition}

The blanking in \ProtShort leads to a loss of \emph{predictable transaction validity}. An honest block $B$ may include a transaction that depends on the contents of a previous block $A$ whose equivocations were not known at the time. 
After block $B$ is produced, the adversary could release an equivocation for the block $A$, forcing honest nodes to blank block $A$'s contents, which may invalidate the transaction in block $B$. Such invalidated transactions take up free space in honest blocks and lower the effective throughput (valid confirmed transactions) of the ledger.


%


We propose a simple solution to recover predictable validity for \ProtShort:
If nodes limit transactions they include in a block to those that don't depend on any \emph{recently changed} state, then they can be sure that all equivocations that could affect the validity 
%
of a transaction already have a corresponding equivocation proof included on chain. 
This is because at the time of creating a block, honest nodes \emph{have seen all transactions which will be executed}, however, \emph{not all transactions nodes have seen will be executed}. The following lemma follows easily.
%
%

\begin{lemma}
    \label{lem:pred-valid-1}
    If a node produces a block whose transactions do not share state with any transaction included in the last $\keqproof$ blocks, then the block satisfies 
    \cref{def:predictablevalidity-transaction}.
    %
\end{lemma}





\subsubsection{Predictable Fee Validity}
\label{sec:predictablevalidity-fee}

In practice, %
in popular DeFi-ecosystems, which consist of very interdependent transactions~%
%
\cite{guo2019graph,chen2020understanding}, it may not always be practical to limit the interaction between transactions. %
We propose instead %
preserving the minimum requirement that each transaction \textit{pays its fee}, regardless of the outcome of its execution. This guarantees that miners are compensated for space used in their blocks, and also makes it costly for the adversary to take up space with invalid transactions.

\begin{definition}
\label{def:predictablevalidity-fee}
    A transaction has \emph{predictable fees} 
    iff:
    ability to pay fees when an honest node adds it to a block
    implies ability to pay fees when the block executes.
    %
\end{definition}

In systems like Ethereum, transactions have a \emph{max gas} value set by the sender which limits the computation allowed by the transaction and ultimately its fee. We consider a protocol with this gas mechanism, as well as a base transaction cost that covers the block space taken up by the transaction. 
We introduce a notion of \emph{gas deposit accounts} to \ProtShort that can only be used for transaction fees (transactions internally do not have access to these accounts).
When a miner includes a transaction, it checks that the account funding the transaction has enough funds to cover the maximum gas, even if all transactions in its recent ancestor blocks make it to the blanked ledger and consume their maximum gas. Users thus need to maintain a balance proportional to the complexity and frequency of the transactions they make. 
%
%
%
We also require that any deposit to the account is not considered in the balance until $\keqproof$ blocks after the deposit transaction. 
Withdrawals can take place immediately, as direct transactions.
%
%

\begin{lemma}
    \label{lem:pred-valid-2}
    If a node produces a block whose transactions are funded by gas deposit accounts with sufficient balance (balance before $\keqproof$ blocks minus any fees since),
    then all transactions in the block satisfy 
    \cref{def:predictablevalidity-fee}.
    %
\end{lemma}

%
%

The solutions in \cref{sec:predictablevalidity-transaction,sec:predictablevalidity-fee} are complementary and could each be adopted as per-validator heuristics (\ie, not a consensus rule), or by the system based on the use-case (\eg, expected inter-dependency of transactions).


%
%


