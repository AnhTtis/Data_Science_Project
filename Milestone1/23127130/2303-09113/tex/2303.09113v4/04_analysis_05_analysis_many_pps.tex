\subsubsection{Probabilistic Pivots Are Abundant}
\label{sec:proof-analysis-many-pps}

Previous analyses of NC~\cite{sleepy,dem20} show that sufficiently long intervals contain at least \emph{one \sltpp} (\cref{fig:analysis-comparison-sleepy}(a)). This was enough for the bounded-delay analysis because in the bounded-delay setting, every \sltpp is also a \sltcp~\cite[Fact 1]{DBLP:journals/iacr/BentovPS16}. 
However, in the bounded-capacity setting, not every \sltcp is a \sltpp, because not every \sltgood \timeslot results in 
growth of the longest processed chain of honest nodes (\cref{fig:analysis-comparison-sleepy}(b)).
%
Thus, existence of one \sltpp in every large interval is not enough to conclude existence of one \sltcp in every large interval. 
%
%
Instead,
we prove,
using a concentration bound on the number of \sltpps,
%
that long intervals of \iindices in fact contain
\emph{a number of \sltpps proportional
to the interval length} (\cref{lem:many-pps-informal}).
Then, in \cref{sec:proof-analysis-many-pps-one-cps}, we prove that out of those many \sltpps, at least one must also be a \sltcp, which allows us to continue with the safety and liveness proofs from~\cite{sleepy}.



\import{./figures/}{fig-ppivot-tailbound-illustration.tex}

The key challenge in proving that there are many \sltpps is that for two \iindices $k_1, k_2$, the events that $k_1$ is a \sltpp and that $k_2$ is a \sltpp are dependent, because both events depend on overlapping intervals. But a key observation is that since the \sltpp condition
(\cref{def:pp-informal})
already holds for large intervals with high probability (\appendixRef{\cref{prop:lower-tailbound-X}}), we only need to look at the small intervals. Then, for two \iindices $k_1,k_2$ that are sufficiently far apart, these short intervals are disjoint, and thus the corresponding \sltpp conditions are independent. Therefore, we decompose a long interval of \iindices into several groups of far-apart \iindices. This is illustrated in \cref{fig:ppivot-tailbound-illustration}, each group indicated by a different color. Within each group, by a concentration bound for \iid random variables, there are many \sltpps. Further, by a union bound, the concentration holds in all the groups simultaneously with high probability (\appendixRef{\cref{prop:lower-tailbound-ppivots}}). Using this, we show:
%




%

%
%
%
%
%
%
%
%
%
%
%
%
%
%
%
%
%
%
%
%
%
%
%
%
%
%
%
%
%
%
%
%
%
%
%
%
%
%


%
%
%

%
%
%
%
%
%
%
%
%
%
%
%
%
%
%
%
%
%
%
%
%
%
%
%
%
%
%
%
%


%
%
%
%
%
%
%
%
%
%
%
%
%
%
%
%
%
%
%
%
%
%
%
%
%
%
%
%
%
%
%
%
%
%
%
%
%
%


%
%
%
%
%
%
%
%
%
%
%
%
%
%
%
%
%
%
%
%
%
%
%
%
%
%
%
%

%
%
%
%
%
%
%
%
%
%
%
%
%
%
%
%
%
%
%
%
%
%
%
%
%
%
%
%
%
%
%
%
%
%

%
%
%
%
%
%
%
%
%
%
%
%
%
%
%
%
%
%
%
%
%
%
%
%
%
%
%
%
%
%
%
%
%
%
%
%
%
%
%
%
%
%
%
%
%
%
%
%
%
%
%
%
%
%
%
%
%
%
%
%
%
%
%
%
%
%
%

%
%
%
%
%
%
%
%
%
%
%
%
%
%
%

%
%
%
%
%
%
%
%
%
%
%
%
%
%
%
%
%
%
%


\begin{lemma}[\FormalVersion{\appendixRef{\cref{lem:many-pps}}}]
    \label{lem:many-pps-informal}
    For $\Kcp = \Omega(\kappa^2)$,
    with overwhelming probability,
    in every interval of size at least $\Kcp$,
    at least $(1-\delta)\probPP$ fraction of the \iindices in the interval are \sltpps.
\end{lemma}
