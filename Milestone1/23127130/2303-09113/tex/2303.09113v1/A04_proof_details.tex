\section{Transaction Validity Proofs}
\label{sec:proof-details}

%
%
%
\begin{proof}[Proof of \lemref{pred-valid-1}] %
    In \secref{pos-equivocations}, we have that equivocation proofs against a block need to be included within the next $\keqproof$ blocks. A node creating a block thus knows all equivocation proofs that will ever be included in their header chain against blocks that are $\keqproof$-deep, thus the state of the $\keqproof$-deep chain is determined. Since equivocations for the last $\keqproof$ blocks can only remove transactions, the node knows all transactions that \emph{may} be included in the final chain. From this, the node can determine all states $\mathcal{S}$ that could be touched by any transaction in the last $\keqproof$ blocks.\footnote{Note that this includes all states a transaction could have changed if it executed differently. This could be achieved by transactions needing to include an access list of all states they are allowed to change. One can imagine a DOS attack where a transaction's access list could prevent future transactions.} A transaction $tx$ that does not depend on any state in $\mathcal{S}$ for its execution, can thus be executed on the state of the $\keqproof$-deep chain, therefore, satisfying predictable transaction validity. A node then only includes transactions that don't rely on a state in $\mathcal{S}$. Note that transactions in the same block could depend on the same state.
\end{proof}

%
%
%
%
\begin{proof}[Proof of \lemref{pred-valid-2}]
    Consider a funding gas account $\mathsf{acc}$ with balance $b$ before the last $\keqproof$ blocks in the chain.
    This balance is set for that account as no equivocation proofs against blocks that are $\keqproof$-deep are allowed by the protocol.
    Note that any transactions in the last $\keqproof$ blocks that fund the account can still be sanitized from the ledger so we do not consider them in the balance yet. The node instead considers all transactions $\mathcal{T}_{\mathsf{acc}(\keqproof)}$ in the last $\keqproof$ which use the funds from the account (including any withdrawals). Since the transactions funded by $\mathsf{acc}$ that end up in the ledger are a subset of $\mathcal{T}_{\mathsf{acc}(\keqproof)}$, and all fees are extracted regardless of how a transaction executes, the node will at worst underestimate the balance of $\mathsf{acc}$ at the tip of the chain.  
\end{proof}