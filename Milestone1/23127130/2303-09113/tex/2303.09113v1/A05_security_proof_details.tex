\section{Security Analysis Proofs}
\label{sec:appendix-security-proofs}

Refer to \tabref{notation} for a recap of notation and definitions.


\begin{table}[tb]
    \caption{Summary of notation (\cf~\secref{modelprotocol-notation,analysis-definitions})}
    \label{tab:notation}
    \vspace{-0.5em}
    \begin{tabular}{cl}
        \toprule
        \multicolumn{2}{c}{\textbf{Protocol parameters}} \\
        \midrule
        $\slotduration$ & \Timeslot duration (seconds) \\
        $\blkrateslot$ & Avg. no. of \BPOs per \timeslot \\
        %
        $\confDepth$ & Confirmation depth \\
        \bottomrule
        \toprule
        \multicolumn{2}{c}{\textbf{Model parameters}} \\
        \midrule
        $\beta$ & Fraction of adversarial nodes \\
        $\DeltaHeader$ & Header propagation delay (seconds) \\
        $\bwtime$ & Bandwidth (blocks/second) \\
        \bottomrule
        \toprule
        \multicolumn{2}{c}{\textbf{Analysis variables}} \\
        \midrule
        $\goodsep$ & No. of \sltempty \timeslots after a \sltgood \timeslot \\
        $\goodsepbw$ & No. of blocks downloaded in $\goodsep$ \timeslots \\
        $t_k$ & $k$-th non-\sltempty \timeslot \\
        $\Gat{k}$ & $1$ iff \timeslot $t_k$ is \sltgood \\
        %
        $\Dat{k}$ & $1$ iff $\Gat{k}=1$ and block in $t_k$ downloaded \\
        %
        $\Pat{k}$ & $1$ iff \iindex $k$ is a \sltpp \\
        \bottomrule
    \end{tabular}
\end{table}



\begin{proof}[Proof of \Propref{X_i-is-iid}]
    First, for any $k$,
    \begin{IEEEeqnarray}{rCl}
        \Prob{\Gat{k} = 1} &=& \Prob{ \predGood{t_k} \mid \lnot \predEmpty{t_k} } \\
        &=& \frac{\Prob{\predGood{t_k}}}{\Prob{\predEmpty{t_k}}}
        %
        %
        = \frac{(1-\beta)\blkrateslot e^{-\rho(\goodsep+1)}}{1-e^{-\blkrateslot}}.
    \end{IEEEeqnarray}
    Take an \iid random process $\{T_k\}$ with $\Prob{T_k = t} = (1-\probEmpty)\probEmpty^t$ for $t \geq 0$ where $\probEmpty = \Prob{\Hat{t}+\Aat{t}=0}$.
    The random variables $\{T_k\}$ describe the inter-arrival times between non-empty slots.
    Take another \iid random process $\{\Gat{k}'\}$, independent of $\{T_k\}$, such that $\Gat{k}' = 1$ with probability $\Prob{\Hat{t} = 1 \land \Aat{t} = 0 \mid \Hat{t}+\Aat{t}>0}$ and $\Gat{k}' = 0$ otherwise.
    The random process $\{\Gat{k}\}$ can be equivalently defined as $G_k = 1$ iff $G_k' = 1$ and $T_k \geq \goodsep$.
    
    The independence of the random variables $\{\Gat{k}\}$ then follows from the independence of the random variables $\{(T_k, \Gat{k}')\}$.
\end{proof}


\subsection{Combinatorial Pivots Stabilize}
\label{sec:appendix-security-proofs-cps-stabilize}


\begin{proposition}
\label{prop:chain-growth-interval}
For any $i < j$,
\begin{IEEEeqnarray}{C}
    L_{\min}(t_j + \goodsep) \geq L_{\min}(t_{i+1} - 1) + \Din{i}{j}.
\end{IEEEeqnarray}
\end{proposition}
\begin{proof}
For each $k \in \{i+1, ..., j\}$,
if $\Dat{k} = 1$,
\begin{IEEEeqnarray}{rCl}
    L_{\min}(t_{k+1}-1) &\geq& L_{\min}(t_k + \goodsep) \quad (\Dat{k} = 1 \implies t_{k+1} > t_k + \goodsep) \IEEEeqnarraynumspace \\
    &\geq& L_{\min}(t_k - 1) + 1 \quad \text{(from \propref{chain-growth})}.
\end{IEEEeqnarray}
If $\Dat{k}=0$, clearly $L_{\min}(t_{k+1}-1) \geq L_{\min}(t_k - 1)$.
Adding these up gives the required result.
\end{proof}



\begin{proof}[Proof of \lemref{cps-stabilize}]
Note that $\dC_p(t)$ is a valid chain at \timeslot $t$ and $\len{\dC_p(t)} = L_p(t) \geq L_{\min}(t)$. Therefore, it suffices to show the first claim of the lemma.

For contradiction, let $s \geq t_k + \goodsep$ be the first \timeslot in which 
there is a valid header chain $\Chain'$ such that 
$\len{\Chain'} \geq L_{\min}(s)$ and $b^* \not\in \Chain'$.
%

Let $b'$ be the block with maximum height on the chain $\Chain'$, such that $b'$ was produced in a \timeslot $t_i$ with $D_i = 1$.
For $\Chain'$ to be a valid chain at \timeslot $s$, we need $t_i \leq s$.
Since the block $b'$ is produced by an honest node, $b'$ extends $\dC_q(t_i-1)$ for some honest node $q$.
Therefore, $\dC_q(t_i-1)$ is a prefix of $\Chain'$.
This means that $b^* \not\in \dC_q(t_i-1)$.
Moreover, $\len{\dC_q(t_i-1)} = L_q(t_i-1) \geq L_{\min}(t_i-1)$.
If $i > k$, then $t_i-1 \geq t_k + \goodsep$ (since $D_k = 1$) and $t_i - 1 < s$ (shown above). 
This is a contradiction because we assumed that $s$ is the first \timeslot such that $s \geq t_k + \goodsep$ and 
%
$b^* \notin \Chain'$ and $\len{\Chain'} \geq L_{\min}(s)$ for some valid chain $\Chain'$.
Since $b^*$ is the only block produced in slot $t_k$, $i=k$ is also not possible.
We conclude that $i < k$.

Since $D_i = 1$ and $b'$ is produced in \timeslot $t_i$,
\begin{IEEEeqnarray}{C}
\label{eq:block-i-download}
    L_{\min}(t_i + \goodsep) \geq \len{b'}.
\end{IEEEeqnarray}
%
By assumption,
\begin{IEEEeqnarray}{C}
\label{eq:block-j-switch}
    \len{\Chain'} \geq L_{\min}(s).
\end{IEEEeqnarray}

Let $t_j$ be the last non-\sltempty \timeslot such that $t_j \leq s$. Note that $j \geq k > i$. 
We must consider two cases: 
%

\begin{enumerate}
\item Case 1: $s \geq t_j + \goodsep$ or $\Dat{j}=0$.
If $\Dat{j}=0$, we don't have to worry about whether the block from slot $t_j$ was downloaded by all honest nodes.
If $\Dat{j} = 1$ but $s \geq t_j + \goodsep$, then we know that all honest nodes have downloaded the block from slot $t_j$ before the end of \timeslot $s$. That is,
\begin{IEEEeqnarray}{rCl}
    L_{\min}(s) 
    &\geq& L_{\min}(t_j + \goodsep) \\
    &\geq& L_{\min}(t_{i+1}-1) + \Din{i}{j} \quad \text{(from \propref{chain-growth-interval})} \\
    \label{eq:chain-growth-case1}
    &\geq& L_{\min}(t_{i} + \goodsep) + \Din{i}{j}.
\end{IEEEeqnarray}
By definition of $b'$, all blocks in $\Chain'$ appearing after $b'$ correspond to \ydowns. These blocks must be from distinct \iindices greater than $i$ but at most $j$. So,
\begin{IEEEeqnarray}{C}
\label{eq:adv-chain-case1}
    \len{\Chain'} \leq \len{b'} + \Nin{i}{j}.
\end{IEEEeqnarray}
From \eqref{block-i-download, block-j-switch, chain-growth-case1, adv-chain-case1}, we derive
\begin{IEEEeqnarray}{rCl}
\label{eq:pivot-contra-case1}
    \Din{i}{j} \leq \Nin{i}{j} \implies \Yin{i}{j} \leq 0 \implies \Yin{0}{i} < \Yin{0}{j}
\end{IEEEeqnarray}
where $i < k \leq j$.

\item Case 2: $t_j \leq s < t_j + \goodsep$ and $\Dat{j} = 1$.
In this case, the block from slot $t_j$ may not have enough time to be downloaded by all honest nodes before the end of slot $s$.
However, for any $l < j$ such that $\Dat{l} = 1$, $t_l + \goodsep < t_j \leq s$, so there is enough time to download the block from \timeslot $t_l$.
Let $l \in\intvl{i}{j-1}$ be the greatest index such that $\Dat{l} = 1$. Then, $t_j > t_l + \goodsep$, and $\Din{i}{l} = \Din{i}{j-1}$.
\begin{IEEEeqnarray}{rCl}
    \label{eq:chain-growth-case2}
    L_{\min}(s) 
    &\geq& L_{\min}(t_j) \\
    &\geq& L_{\min}(t_l + \goodsep) \\
    &\geq& L_{\min}(t_{i+1} - 1) + \Din{i}{l} \quad \text{(from \propref{chain-growth-interval})} \\
    &\geq& L_{\min}(t_{i} + \goodsep) + \Din{i}{j-1}.
\end{IEEEeqnarray}
Note that since $\Dat{j}=1$, $\Nin{i}{j} = \Nin{i}{j-1}$. Therefore, as in the previous case,
\begin{IEEEeqnarray}{C}
\label{eq:adv-chain-case2}
    \len{\Chain'} \leq \len{b'} + \Nin{i}{j-1}.
\end{IEEEeqnarray}
From \eqref{block-i-download, block-j-switch, chain-growth-case2, adv-chain-case2},
\begin{IEEEeqnarray}{rCl}
\label{eq:pivot-contra-case2}
    \Din{i}{j-1} \leq \Nin{i}{j-1} \implies \Yin{i}{j-1} \leq 0 \implies \Yin{0}{i} < \Yin{0}{j-1}. \IEEEeqnarraynumspace
\end{IEEEeqnarray}
Note that since we assumed $s \geq t_k + \goodsep$ and $s < t_j + \goodsep$, we know that $j > k$. Therefore, $i < k \leq j-1$.
\end{enumerate}
%
In either case, \eqref{pivot-contra-case1} or \eqref{pivot-contra-case2} contradict the assumption $\predCP{k}$ (\defref{cp}).
%
\end{proof}











\subsection{Probabilistic Pivots are Abundant}
\label{sec:appendix-security-proofs-many-pps}

We build up to the proof of \lemref{many-pps} through a series of propositions,
starting with recalling a versatile tail bound.
\begin{proposition}[Hoeffding's inequality~{\cite{doi:10.1080/01621459.1963.10500830} \cite[Thm.~4]{duchi-hoeffding}}]
    \label{prop:hoeffding}
    Let $Z_1, ..., Z_n$ be independent bounded random variables with
    $\forall i: Z_i \in [a,b]$, where $-\infty < a \leq b < \infty$.
    Then, $\forall t\geq0$:
    \begin{IEEEeqnarray}{rCl}
        \Prob{\left(\sum_{i=1}^n Z_i\right) - \Exp{\sum_{i=1}^n Z_i} \geq t n}
        &\leq&
        \exp\left(-\frac{2 n t^2}{(b-a)^2} \right)
        \\
        \Prob{\left(\sum_{i=1}^n Z_i\right) - \Exp{\sum_{i=1}^n Z_i} \leq -t n}
        &\leq&
        \exp\left(-\frac{2 n t^2}{(b-a)^2} \right).
    \end{IEEEeqnarray}
\end{proposition}

%
%
\begin{proposition}
    \label{prop:lower-tailbound-X}
    With $\alphaLowerTailX \triangleq 2 \epsGood^2$,
    \begin{IEEEeqnarray}{l}
        \forall \intvl{i}{j}\colon
        \forall \delta \geq 0\colon
        \nonumber
        \\
        \qquad
        \Prob{\Xin{i}{j} \leq (1-\delta) 2 \epsGood (j-i)}
        \leq \exp( - \alphaLowerTailX \delta^2 (j-i)).
        \IEEEeqnarraynumspace
    \end{IEEEeqnarray}
\end{proposition}
\begin{proof}
    By Hoeffding's inequality (\propref{hoeffding}).
\end{proof}


\begin{proposition}
    \label{prop:ppivot-randomwalk}
    \begin{IEEEeqnarray}{C}
        \forall k\colon
        \Prob{\predPP{k}}
        \geq \probPPFormula
        %
        \triangleq \probPP
    \end{IEEEeqnarray}
\end{proposition}
\begin{proof}
    \Eqref{pivot-conditions-equivalence-randomwalks}
    characterizes $\predPP{k}$
    as the intersection of three independent events:
    \begin{IEEEeqnarray}{rCl}
        \Event_1
        &\triangleq&
        \{ \Xat{k} = 1 \}
        \\
        \Event_2
        &\triangleq&
        \{ \forall\ell\colon \Xin{k}{k+\ell} \geq 0 \}
        \\
        \Event_3
        &\triangleq&
        \{ \forall\ell\colon \Xin{k-1-\ell}{k-1} \geq 0 \}
    \end{IEEEeqnarray}
    Their probabilities are easily calculated~\cite{stackexchange-math-rwreturnto0}:
    \begin{IEEEeqnarray}{C}
        \Prob{\Event_1}
        = \probGood
        \qquad
        \Prob{\Event_2} = \Prob{\Event_3}
        = (2\probGood - 1) / \probGood
        \IEEEeqnarraynumspace
    \end{IEEEeqnarray}
\end{proof}


\begin{proposition}
    \label{prop:lower-tailbound-ppivots}
    With $\alphaLowerTailPP \triangleq 2 \probPP^2$,
    \begin{IEEEeqnarray}{l}
        \forall \intvl{i}{j} \intvleq 2 K_1 K_2\colon
        \quad
        %
        %
        %
        \Prob{\Pin{i}{j} \leq (1-\delta) \probPP 2 K_1 K_2}
        \nonumber
        \\
        \qquad\qquad\qquad\quad {}\leq{} 2 K_1 \exp(- \alphaLowerTailPP \delta^2 K_2) + \Khorizon^2 \exp(-\alphaLowerTailX K_1).
        \IEEEeqnarraynumspace
    \end{IEEEeqnarray}
\end{proposition}
\begin{proof}
    Let
    $\Event \triangleq \{\forall \intvl{i}{j} \intvlgeq K_1\colon \Xin{i}{j} > 0\}$.
    From \propref{lower-tailbound-X} with $\delta=1$,
    and a union bound over all intervals
    ($\leq \Khorizon^2$ many),
    we get
    \begin{IEEEeqnarray}{C}
        \Prob{\lnot\Event} \leq \Khorizon^2 \exp(-\alphaLowerTailX K_1).
    \end{IEEEeqnarray}

    For any given index $k$, we can
    partition
    the intervals of
    \eqref{pivot-conditions-equivalence-intervals}
    into `long'
    %
    and `short'
    %
    intervals (length at least vs.\ less than $K_1$):
    \begin{IEEEeqnarray}{rCl}
        \Event_k
        &\triangleq&
        \{ \predPP{k} \}
        = \Event_k^{\mathrm{L}} \land \Event_k^{\mathrm{S}}
        \\
        \Event_k^{\mathrm{L}}
        &\triangleq&
        \{\forall k \in \intvl{i}{j} \intvlgeq K_1\colon \Xin{i}{j} > 0\}
        \IEEEeqnarraynumspace
        \\
        \Event_k^{\mathrm{S}}
        &\triangleq&
        \{\forall k \in \intvl{i}{j} \intvll K_1\colon \Xin{i}{j} > 0\}.
    \end{IEEEeqnarray}
    Note that $\Event_k^{\mathrm{L}} \supseteq \Event$.
    Thus, for any two given \iindices $k_1, k_2$,
    if $k_1, k_2$ are `far apart',
    \ie,
    if $\abs{k_1 - k_2} \geq 2 K_1$,
    then
    $\Event_{k_1}$ and $\Event_{k_2}$ are conditionally independent
    given $\Event$
    (since $\Event_{k_1}^{\mathrm{S}}$ and $\Event_{k_2}^{\mathrm{S}}$ are).

    We bound and decompose $I^* \triangleq \intvl{i}{j} = \intvl{i}{i + 2 K_1 K_2} = \bigcup_{\ell=1}^{2K_1} I_{\ell}$:
    \begin{IEEEeqnarray}{rCl}
        \forall\ell\in\{1,...,2K_1\}\colon
        \quad
        I_{\ell}
        &\triangleq&
        \{ i+0\cdot 2K_1+\ell, ...
        \nonumber\\
        && \qquad{} ..., i+(K_2-1)\cdot 2K_1+\ell \}.
        \IEEEeqnarraynumspace
    \end{IEEEeqnarray}
    We define corresponding events, $\forall\ell\in\{1,...,2K_1\}$:
    \begin{IEEEeqnarray}{rCl}
        \Event^*
        &\triangleq&
        \left\{ \Pat{I^*} \leq (1-\delta) \probPP 2 K_1 K_2 \right\}
        \\
        \Event_{\ell}
        &\triangleq&
        \left\{ \Pat{I_{\ell}} \leq (1-\delta) \probPP K_2 \right\}.
    \end{IEEEeqnarray}
    Clearly, $\Event^* \subseteq \bigcup_{\ell=1}^{2 K_1} \Event_\ell$.
    Thus, by a union bound,
    \begin{IEEEeqnarray}{rCl}
        \Prob{ \Event^* \cond \Event }
        &\leq&
        \sum_{\ell=1}^{2 K_1} \Prob{ \Event_\ell \cond \Event }.
        \IEEEeqnarraynumspace
    \end{IEEEeqnarray}
    Furthermore, $\forall\ell\in\{1,...,2K_1\}$,
    and with $\mu_\ell \triangleq \Exp{ \Pat{I_{\ell}} \cond \Event}$:
    \begin{IEEEeqnarray}{rCl}
        \IEEEeqnarraymulticol{3}{l}{
            \Prob{ \Event_\ell \cond \Event }
            =
            \Prob{ \Pat{I_{\ell}} \leq (1-\delta) \probPP K_2 \cond \Event }
        }
        \IEEEeqnarraynumspace
        \\\quad
        &\leqA&
        \Prob{ \Pat{I_{\ell}} \leq (1-\delta) \mu_\ell \cond \Event }
        \IEEEeqnarraynumspace
        \\
        &\leqB&
        \exp(-2 \delta^2 \mu_\ell^2 / K_2)
        \leqC
        \exp(-2 \probPP^2 \delta^2 K_2),
        \IEEEeqnarraynumspace
        %
    \end{IEEEeqnarray}
    where
    (a) and (c)~use 
    \begin{IEEEeqnarray}{C}
        \mu_\ell = K_2 \Exp{\Ind{\predPP{k}} \cond \Event} \geq K_2 \Exp{\Ind{\predPP{k}}} \geq K_2 \probPP \IEEEeqnarraynumspace
    \end{IEEEeqnarray}
    (\propref{ppivot-randomwalk}),
    %
    and
    (b)~uses that
    $\{\predPP{k_1}\}$ and
    $\{\predPP{k_2}\}$
    are conditionally independent given $\Event$
    for $k_1, k_2 \in I_\ell$,
    and
    Hoeffding's inequality (\propref{hoeffding}).
    %

    Thus, we complete the proof by observing, as desired, that
    \begin{IEEEeqnarray}{rCl}
        \Prob{ \Event^* }
        &=&
        \Prob{ \Event^* \cap \Event } + \Prob{ \Event^* \cap \lnot\Event }
        \\
        &\leq&
        \Prob{ \Event^* \cond \Event } + \Prob{ \lnot\Event }
        \\
        &\leq&
        2 K_1 \exp(-2 \probPP^2 \delta^2 K_2)
        + \Khorizon^2 \exp(-\alphaLowerTailX K_1).
        \IEEEeqnarraynumspace
    \end{IEEEeqnarray}
\end{proof}


\begin{proof}[Proof of \lemref{many-pps}]
From \propref{lower-tailbound-ppivots} by setting $K_1,K_2 = \Omega(\kappa)$ and $\Kcp = 2K_1K_2$.
\end{proof}







\subsection{Many Probabilistic Pivots Imply One Combinatorial Pivot}
\label{sec:appendix-security-proofs-many-pps-one-cp}


\begin{proof}[Proof of \Propref{download-or-spend-budget}]
%

%

In \timeslot $t_k$, there is exactly one block $b$ produced by an honest node, and 
the block header is made public at the beginning of the \timeslot,
and is seen by all honest nodes within $\DeltaHeader$ time.
Thereafter, each node has enough time to download $\goodsepbw$ blocks during \timeslots $[t_k, t_k + \goodsep]$.

%
%

%
%
%
%
%
%
    %
    %
%
%
%


Under the download rule $\dlrulelong$, all honest nodes download content for their longest header chain.
If $\Dat{k} = 0$
\ie an honest node did not download content for the block $b$ before the end of \timeslot $t_k + \goodsep$,
then
%
that honest node must download the content for at least $\goodsepbw$ blocks on chains longer than the height of the block $b$ or in the prefix of the block $b$.
%
Since honest nodes produce blocks extending their longest chain, $b$ extends $\dC_p(t_k-1)$ for some $p$.
Let $b^*$ be the block produced in \timeslot $t_i$ where $\predCP{i}$ (suppose $i$ exists).
$\predCP{i} \implies \Yat{i} = 1$, therefore this block is unique, and also $t_k > t_i + \goodsep$.
Due to \lemref{cps-stabilize}, any valid header chain longer than $b$ at time slot $t_k$ must contain $b^*$.
%
Therefore, the only blocks 
%
that are downloaded by an honest node during \timeslots $[t_k, t_k + \goodsep]$
\begin{enumerate}
    \item must be produced after $t_i$ because they extend $b^*$, and
    \item must be produced no later than $t_k$ because there are no blocks produced in $\intvl{t_k}{t_k+\goodsep}$.
\end{enumerate}
%
In case a \sltcp $i<k$ does not exist, the claim is trivial.
%
%
\end{proof}



\begin{proposition}
\label{prop:not-cp-exists-interval}
\begin{IEEEeqnarray}{C}
    \lnot \predCP{k} \implies \exists \intvl{i}{j} \ni k \colon \Yin{i}{j} \leq 0.
\end{IEEEeqnarray}
\end{proposition}

\begin{proof}
From \defref{cp}, $\lnot \predCP{k}$ implies that either there exists $i < k$ such that $\Yin{0}{i} \geq \Yin{0}{k}$ or there exists $j \geq k$ such that $\Yin{0}{k} > \Yin{0}{j}$.
In the first case, $\intvl{i}{k} \ni k$ and $\Yin{i}{k} \leq 0$.
In the second case, $\intvl{k-1}{j} \ni k$ and $\Yin{k-1}{j} \leq \Yin{k}{j} + 1 \leq 0$.
\end{proof}


\begin{proposition}
\label{prop:not-cp-interval-properties}
If $\Yin{i}{j} \leq 0$, then
\begin{IEEEeqnarray}{rCl}
    \label{eq:not-cp-interval-property1}
    \Nin{i}{j} &\geq& \Din{i}{j}, \\
    \label{eq:not-cp-interval-property2}
    \Gin{i}{j} - \Din{i}{j} &\geq& \frac{1}{2} \left( \Gin{i}{j} - \Bin{i}{j} \right).
\end{IEEEeqnarray}
\end{proposition}

\begin{proof}
\Eqref{not-cp-interval-property1} follows from the definition $\Yat{i} = \Dat{i} - \Nat{i}$.
%
Then,
\begin{IEEEeqnarray}{rCl}
    \Gin{i}{j} + \Bin{i}{j} &=& \Din{i}{j} + \Nin{i}{j} \\
    \Gin{i}{j} + \Bin{i}{j} &\geq& 2 \Din{i}{j} \\
    2 \Gin{i}{j} - 2 \Din{i}{j} &\geq& \Gin{i}{j} - \Bin{i}{j}.
\end{IEEEeqnarray}
\end{proof}



\begin{proposition}
\label{prop:ppivots-imply-honest-margin}
If $\Pin{i}{j} > 0$, then $\Gin{i}{j} - \Bin{i}{j} \geq \Pin{i}{j}$.
\end{proposition}

\begin{proof}
Let $n = \Pin{i}{j}$.
First, consider the case $n=1$.
There is exactly one \sltpp $k \in \intvl{i}{j}$.
From \defref{pp}, $\Xin{0}{i} < \Xin{0}{j}$. Therefore, $\Xin{i}{j} > 0$, hence $\Gin{i}{j} - \Bin{i}{j} \geq 1$.

For the general case, let $k_1,...,k_n$ be the \sltpps in $\intvl{i}{j}$. Then, we can apply the $n=1$ case on the disjoint intervals $\intvl{i}{k_1}$, $\intvl{k_1}{k_2}, ...$, $\intvl{k_{n-1}}{j}$ and then sum them up.

This can also be seen from \figref{pivot-randomwalk}.
Each \sltpp corresponds to a height that the random walk $\Xat{k}$ attains exactly once.
This means that in any interval containing $n$ \sltpps, the random walk $\Xat{k}$ `moves up' by at least $n$ units, and this is possible only if there are $n$ more `ups' than `downs'.
%
\end{proof}




\begin{lemma}
\label{lem:one-cp-induction-base}
If all honest nodes use
the download rule $\dlrulelong$,
%
%
%
%
%
and if
\begin{IEEEeqnarray}{C}
    \label{eq:cp-induction-base-margin-condition}
    \forall \intvl{i}{j} \intvlgeq \Kcp, i < \Kcp \colon \frac{\goodsepbw}{2} \left( \Gin{i}{j} - \Bin{i}{j} \right) > \Qin{0}{j}, \text{ and} \\
    %
    \label{eq:cp-induction-base-ppivots-condition}
    \frac{\goodsepbw}{4} \Pin{0}{\Kcp} > \Qin{0}{2\Kcp},
    %
\end{IEEEeqnarray}
then $\exists k_1^* \in \intvl{0}{\Kcp} \colon \predCP{k_1^*}$.
\end{lemma}

\begin{proof}
Due to \eqref{cp-induction-base-ppivots-condition}, there is at least one \sltpp in $\intvl{0}{\Kcp}$ (otherwise $\Pin{0}{\Kcp}=0$).
Suppose for contradiction that there is no \sltcp in $\intvl{0}{\Kcp}$.
Since \sltcps are also \sltpps, it is enough to consider that
none of the \sltpps is a \sltcp.
Then around each \sltpp, there must be at least one interval which violates the combinatorial pivot condition.
Formally, there is a set of intervals $\intvlset$ such that:
\begin{IEEEeqnarray}{Cr}
    \label{eq:intervals-cover-ppivots}
    \bigcup_{I \in \intvlset} I \supseteq \left\{ k \in \intvl{0}{\Kcp} \colon \predPP{k} \right\} & \\
    \label{eq:intervals-y-condition}
    \forall I \in \intvlset \colon \Yat{I} \leq 0 &\quad \text{(from \propref{not-cp-exists-interval})}.
\end{IEEEeqnarray}
Without loss of generality, each interval $I \in \intvlset$ contains at least one \sltpp (removing all intervals that do not contain a \sltpp maintains \eqref{intervals-cover-ppivots, intervals-y-condition}).
Then if $\intvl{i}{j} \in \intvlset$, $i < \Kcp$.

First, let's consider the large intervals with $|I| \geq \Kcp$.
Consider \iindices $k \in I$ for which $\Gat{k}=1$ (\sltgood) but $\Dat{k}=0$ (\ydown).
From \propref{download-or-spend-budget}, for each such \iindex, all honest nodes download $\goodsepbw$ blocks that are produced no later than $t_k$.
%
%
%
%
%
%
The number of \iindices $k \in I$ with  $\Gat{k} = 1$ and $\Dat{k} = 0$ is exactly $\Gat{I} - \Dat{I}$.
For each such index, there must exist $\goodsepbw$ distinct blocks produced in or before the interval $I$. Therefore if $I = \intvl{i}{j}$,
\begin{IEEEeqnarray}{rClr}
    \Qin{0}{j} &\geq& \goodsepbw \left( \Gin{i}{j} - \Din{i}{j} \right) & \\
    &\geq& \frac{\goodsepbw}{2} \left( \Gin{i}{j} - \Bin{i}{j} \right) & \quad \text{(from \propref{not-cp-interval-properties}).}
\end{IEEEeqnarray}
This is a contradiction to \eqref{cp-induction-base-margin-condition}.

Therefore all intervals $I \in \intvlset$ are small ($|I| < \Kcp$).
%
%
%
%
Then for each $I \in \intvlset$, $I \subset \intvl{0}{2\Kcp}$.
Also, 
\begin{IEEEeqnarray}{rClr}
    \Gat{I} - \Dat{I} &\geq& \frac{1}{2} \left( \Gat{I} - \Bat{I} \right) & \quad \text{(from \propref{not-cp-interval-properties})} \\
    \label{eq:failed-more-than-ppivots}
    &\geq& \frac{1}{2} \Pat{I} & \quad \text{(from \propref{ppivots-imply-honest-margin}).}
\end{IEEEeqnarray}

Consider the \iindices $k \in \intvl{0}{2\Kcp}$ with $\Gat{k} = 1$ and $\Dat{k} = 0$.
%
Let $\intvlset_k = \{ I \in \intvlset \colon k \in I\}$ be the set of intervals that contain \index $k$.
Let $I^L_k$ be an interval in $\intvlset_k$ that stretches farthest to the left, and let $I^R_k$ be an interval that stretches farthest to the right (these may also be the same).
%
Note that all other intervals in $\intvlset_k$ are contained in $I^L_k \cup I^R_k$.
Therefore, all intervals in $\intvlset_k$ except $I^L_k$ and $I^R_k$ can be removed from $\intvlset$ while maintaining \eqref{intervals-cover-ppivots, intervals-y-condition} (see \figref{one-cp-proof-figures}(a)).
This process is repeated for all $k \in \intvl{0}{2\Kcp}$ with $\Gat{k} = 1$ and $\Dat{k} = 0$, so that in the resulting set $\intvlset$, each such \iindex $k$ is contained in at most two intervals.
Then,
\begin{IEEEeqnarray}{rCl}
    \sum_{k \in \intvl{0}{2\Kcp} \colon \Gat{k} = 1, \Dat{k} = 0} |\intvlset_k| &\leq& \sum_{k \in \intvl{0}{2\Kcp} \colon \Gat{k} = 1, \Dat{k} = 0} 2 \\
    &=& 2\left( \Gin{0}{2\Kcp} - \Din{0}{2\Kcp} \right).
\end{IEEEeqnarray}
This sum can be rewritten as
\begin{IEEEeqnarray}{rCl}
    \sum_{k \in \intvl{0}{2\Kcp} \colon \Gat{k} = 1, \Dat{k} = 0} |\intvlset_k| &=& \sum_{I \in \intvlset} \left( \Gat{I} - \Dat{I} \right) \\
    &\geq& \sum_{I \in \intvlset} \frac{1}{2} \Pat{I} \\
    &\geq& \frac{1}{2} \Pin{0}{\Kcp} \quad \text{(due to \eqref{intervals-cover-ppivots})}.
    \IEEEeqnarraynumspace
\end{IEEEeqnarray}
Therefore,
\begin{IEEEeqnarray}{rCl}
    \Gin{0}{2\Kcp} - \Din{0}{2\Kcp} &\geq& \frac{1}{4} \Pin{0}{\Kcp}.
\end{IEEEeqnarray}
This can also be seen from \figref{one-cp-proof-figures}(b).

Finally, as shown before, for each \index $k$ with $\Gat{k}=1$ and $\Dat{k}=0$, all honest nodes download at least $\goodsepbw$ distinct blocks produced in or before \iindex $k$ (\propref{download-or-spend-budget}). This gives
\begin{IEEEeqnarray}{rCl}
    \Qin{0}{2\Kcp} &\geq& \goodsepbw \left( \Gin{0}{2\Kcp} - \Din{0}{2\Kcp} \right) \\
    &\geq& \frac{\goodsepbw}{4} \Pin{0}{\Kcp}
\end{IEEEeqnarray}
which is a contradiction to \eqref{cp-induction-base-ppivots-condition}.
%
\end{proof}



\import{./figures/}{fig-one-cp-proof-figures.tex}




\begin{proof}[Proof of \lemref{one-cp-induction-full}]
This will be proved through induction.
For the base case ($m=0$), \lemref{one-cp-induction-base} shows 
%
that $\exists k_1^* \in \intvl{0}{\Kcp} \colon \predCP{k_1^*}$.

For $m \geq 1$, assume that $\exists k_{m-1}^* \in \intvl{(m-1)\Kcp}{m\Kcp}$ such that  $\predCP{k_{m-1}^*}$.
Now we want to show that $\exists k_{m}^* \in \intvl{m\Kcp}{(m+1)\Kcp}$ such that  $\predCP{k_{m}^*}$.
%
%
Suppose for contradiction that there is no \sltcp in $\intvl{m\Kcp}{(m+1)\Kcp}$.
%
%
%
%
As in the proof of \lemref{one-cp-induction-base}, 
there is a set of intervals $\intvlset$ such that:
\begin{IEEEeqnarray}{Cr}
    \label{eq:induction-full-intervals-cover-ppivots}
    \bigcup_{I \in \intvlset} I \supseteq \left\{ k \in \intvl{m\Kcp}{(m+1)\Kcp} \colon \predPP{k} \right\} & \\
    \label{eq:induction-full-intervals-y-condition}
    \forall I \in \intvlset \colon \Yat{I} \leq 0. &
    %
\end{IEEEeqnarray}
Without loss of generality, each interval $I \in \intvlset$ contains at least one \sltpp.
%
Then if $\intvl{i}{j} \in \intvlset$, $i < (m+1)\Kcp$ and $j > m\Kcp$.

First, consider the large intervals with $|I| \geq \Kcp$.
Consider \iindices $k \in I$ for which $\Gat{k}=1$ (\sltgood) but $\Dat{k}=0$ (\ydown).
%
%
%
From \propref{download-or-spend-budget},
for each such \iindex $k$, all honest nodes download $\goodsepbw$ blocks that are produced
%
in the interval $\intvl{k_{m-1}^*}{k}$.
%
%
%
%
%
%
The number of \iindices $k \in I$ with  $\Gat{k} = 1$ and $\Dat{k} = 0$ is exactly $\Gat{I} - \Dat{I}$.
For each such index, there must exist $\goodsepbw$ distinct blocks from distinct \BPOs
%
that are downloaded by honest nodes.
Therefore if $I = \intvl{i}{j}$,
\begin{IEEEeqnarray}{rClr}
    \Qin{k_{m-1}^*}{j} &\geq& \goodsepbw \left( \Gin{i}{j} - \Din{i}{j} \right) & \\
    &\geq& \frac{\goodsepbw}{2} \left( \Gin{i}{j} - \Bin{i}{j} \right) & \quad \text{(from \propref{not-cp-interval-properties}).}
\end{IEEEeqnarray}
But $k_{m-1}^* > (m-1)\Kcp$ and $i < (m+1)\Kcp$. Therefore $\Qin{k_{m-1}^*}{j} \leq \Qin{i-2\Kcp}{j}$.
Then we have a contradiction to \eqref{cp-induction-full-margin-condition}.

Therefore all intervals $I \in \intvlset$ are small ($|I| < \Kcp$).
%
%
%
%
Then for each $I \in \intvlset$, $I \subset \intvl{(m-1)\Kcp}{(m+1)\Kcp}$.
Also, 
\begin{IEEEeqnarray}{Cr}
    \label{eq:failed-more-than-ppivots-induction}
    \Gat{I} - \Dat{I} \geq \frac{1}{2} \left( \Gat{I} - \Bat{I} \right) \geq \frac{1}{2} \Pat{I} & \quad \text{(from \propref{not-cp-interval-properties, ppivots-imply-honest-margin})}
\end{IEEEeqnarray}

Consider the \iindices $k \in \intvl{(m-1)\Kcp}{(m+1)\Kcp}$ with $\Gat{k} = 1$ and $\Dat{k} = 0$.
%
%
%
%
%
%
%
Following the arguments in the proof of \lemref{one-cp-induction-base},
we can reduce the set $\intvlset$
so that in the resulting set $\intvlset$, each such \iindex $k$ is contained in at most two intervals.
Then,
\begin{IEEEeqnarray}{Cl}
    & \sum_{k \in \intvl{(m-1)\Kcp}{(m+1)\Kcp} \colon \Gat{k} = 1, \Dat{k} = 0} |\intvlset_k| \nonumber \\
    \leq& 
    %
    %
    2\left( \Gin{(m-1)\Kcp}{(m+1)\Kcp} - \Din{(m-1)\Kcp}{(m+1)\Kcp} \right).
\end{IEEEeqnarray}
This sum can be rewritten as
\begin{IEEEeqnarray}{rCl}
    \sum_{k \in \intvl{(m-1)\Kcp}{(m+1)\Kcp} \colon \Gat{k} = 1, \Dat{k} = 0} |\intvlset_k| &=& \sum_{I \in \intvlset} \left( \Gat{I} - \Dat{I} \right) \\
    &\geq& \sum_{I \in \intvlset} \frac{1}{2} \Pat{I} \\
    &\geq& \frac{1}{2} \Pin{m\Kcp}{(m+1)\Kcp}.
    \IEEEeqnarraynumspace
    %
\end{IEEEeqnarray}
Therefore,
\begin{IEEEeqnarray}{rCl}
    &&\Gin{(m-1)\Kcp}{(m+1)\Kcp} - \Din{(m-1)\Kcp}{(m+1)\Kcp} \nonumber \\
    &\geq& \frac{1}{4} \Pin{m\Kcp}{(m+1)\Kcp}.
\end{IEEEeqnarray}

Finally, for each \index $k$ with $\Gat{k}=1$ and $\Dat{k}=0$, all honest nodes download at least $\goodsepbw$ distinct blocks produced in or before 
the most recent \sltcp before $(m-1)\Kcp$.
By induction assumption, we have a \sltcp $k_{m-2}^* \in \intvl{(m-2)\Kcp}{(m-1)\Kcp}$.
%
%
This gives
\begin{IEEEeqnarray}{rCl}
    && \Qin{(m-2)\Kcp}{(m+1)\Kcp} \nonumber \\
    &\geq& \goodsepbw \left( \Gin{(m-1)\Kcp}{(m+1)\Kcp} - \Din{(m-1)\Kcp}{(m+1)\Kcp} \right) \\
    &\geq& \frac{\goodsepbw}{4} \Pin{m\Kcp}{(m+1)\Kcp}
\end{IEEEeqnarray}
which is a contradiction.
%
\end{proof}