\section{Introduction}
\label{sec:introduction}

%

\import{./figures/}{fig-comparison-bddelay-bdbandwidth.tex}

%
%
%
%


The goal of a blockchain protocol is to create a \emph{secure} and \emph{decentralized} ledger of transactions.
This protocol is run by a network of nodes, each with certain capabilities in terms of communication rates and computing power.
In this work, we study the connection between these processing capacities (in the wider sense) of individual nodes, and the security of the system.
%

In order to remain secure under adversarial conditions,
blockchain protocols have been parameterized 
to leave a `security margin' between the transaction rate under normal operation, and each node's capacity limits.
%
%
For instance, Bitcoin only produces one block of transactions per ten minutes, even though it usually only takes a few seconds for a node to download and process each block \cite{decker}.
%
%
%
%
%
%
%
%
%
%
On the other hand, protocols that push close to the limits
of their nodes,
%
become insecure when the processing capacities of nodes are overwhelmed (such as Solana \cite{solanastop, solanastop2, solanastopagain}).
%
%
%
The natural question to ask then is: 
%
%
given a capacity limit of nodes, what is the maximum block rate under which a blockchain remains secure?
%
%
%
%
%
%
%

In this work, we focus on longest chain (LC) protocols (a.k.a.\ Nakamoto consensus~\cite{nakamoto_paper})---a well-studied class of blockchain protocols
that can be instantiated using various Sybil resistance mechanisms such as proof-of-work (PoW)~\cite{nakamoto_paper,backbone} and proof-of-stake (PoS)~\cite{david2018ouroboros,badertscher2018ouroboros,sleepy,snowwhite}.
%
%
%
This protocol selects the nodes that will mine the next block based on a continuously running lottery.
A selected node collects pending transactions, creates a new block extending the longest chain of blocks it sees, and sends the block to the network.
Nodes must download and process the transactions in a chain before they can extend it.
Attackers prevail if they manage to grow a chain at a faster pace than the honest nodes do, which they can then use to double-spend or to censor transactions.

It seems at first that it is sufficient for nodes to have enough processing capacity to keep up with block production.
However, in 
%
LC, the production of blocks occurs at random times,
%
which means the network and computing load is bursty.
With limited processing capacity, nodes must queue blocks for processing. Even without adversarial activity, the resulting queuing delays increase the time it takes to process blocks.
Moreover, a malicious node can selectively delay the release of blocks that it produces, so that the processing load is not just purely random but is to some degree determined adversarially.
In PoS, 
%
%
%
%
the adversary can additionally
produce equivocations---conflicting versions of the block it was allowed to produce---and send them to different nodes.
Nodes cannot always predict which of two conflicting blocks will eventually be part of the chain and may thus waste processing capacity on blocks that are later discarded.
Attackers can use this to increase load and queuing delays.
While blocks are waiting to be processed, nodes cannot mine on top of them, and the honest nodes' chain slows down.
%
%
%
%
%
This makes it easier for
%
an adversary to attack the system. 


%
%
%
%
%
%

\import{./figures/}{fig-experiment-trace.tex}

\import{./figures/}{fig-bitcoin-cardano-resilience-bandwidth.tex}

\Figref{experiment-trace} shows a sample trace of a simulation of the proof-of-work longest chain protocol.
%
The figure presents both block creation events, and block processing activity. Queuing effects are evident. For example, node $0$ processes blocks almost without pause, occasionally preempting for higher over lower blocks.
Several times, blocks are created
on the same height
%
%
%
due to delays in processing. 


Due to queuing of blocks, nodes need a carefully chosen `scheduling rule' to determine which blocks to process first.
We observe that choosing the right scheduling policy is challenging. Different attacks can be carried out by the adversary, depending on this scheduling rule,
slowing down the growth of the honest chain and breaking the blockchain's security (\figref{experiment-teaser, experiment-greedy}, details in \secref{experiments}).
%
%


To analyze the security of Nakamoto consensus,
previous work~\cite{kiayias2017ouroboros, backbone, dem20, sleepy, ren, tight_bitcoin} has considered 
%
the `bounded delay' model.
In this model, blocks are processed by all honest nodes within a fixed time of $\Delta$ seconds after they are published.
%
The works~\cite{dem20, tight_bitcoin} give a tight characterization of the tradeoff between the fraction $\beta$ of adversarial nodes, delay bound $\Delta$, and the block production rate $\blkratetime$ (\figref{comparison-bddelay-bdbandwidth}(a)).
However, this model assumes the processing time of each block to be independent of the processing load.
Thus, this model fails to capture
%
the effect of queuing delays.
%
The bounded delay model is only a suitable approximation for limited capacity when the block rate is much smaller than the capacity,
%
and newly produced blocks typically find the queue empty.
This leads to absurd conclusions, such as that the protocol remains secure against a non-zero adversary for arbitrarily high block rates (\figref{comparison-bddelay-bdbandwidth}(a)).
%
%

%

To study the security of blockchains `at capacity',
%
we adopt the `bounded bandwidth' model
%
proposed in \cite{bwlimitedposlc}.
%
\emph{Thus, henceforth, we adopt the word `\emph{bandwidth}', but continue to mean `\emph{capacity}' in the wider sense,
intending to model nodes' rate-limits across domains such as
communication, computation, or storage.
We also use `\emph{download}' to mean `\emph{process}' in the wider sense.}
The work \cite{bwlimitedposlc} 
%
analyzed suitable download rules to secure PoS LC where the adversary can spam nodes with equivocating blocks, and waste their download bandwidth. 
%
However, their rules and analysis result in an undesirable scaling of the block rate with the desired security parameter, \ie, logarithm of the security error probability (\figref{comparison-bddelay-bdbandwidth}(b)). We solve this by introducing a new variant proof-of-stake protocol that we call \sapos.
%
%
%
%
%
%
%
To the best of our knowledge, there is no security analysis of PoW LC under bounded bandwidth.
%
It stands to reason that the analysis of~\cite{bwlimitedposlc} carries over to PoW `in some form', but this analysis'
%
undesirable 
scaling of the block rate
%
might be too pessimistic for PoW, where the adversary cannot equivocate.
%




%
%
%
%
%
%
%
%

%

%
%
%
%
%
%
%
%
%
%

%
%
%
%
%

%

%
%
%
%

%
%
%
%
%
%
%
%
%
%
%

%
%
%
%
%
%
%
%
%
%
%

%

%
%
%
%
%


%
%
\subsection{Our Results}
\label{sec:introduction-question}



%
%
%
%
%
%

%


%
\noindent\textbf{The PoS LC protocols studied in \cite{bwlimitedposlc} cannot have block rate independent of the security parameter (\figref{comparison-bddelay-bdbandwidth}(b)).}\;\;
We show this with an attack (\secref{pos-attack}).
%
%
%
%
This indicates that overcoming 
the 
%
dependence of block rate on security parameter
%
requires not just tighter analysis, but 
%
a change to protocol and/or scheduling policy.


\smallskip\noindent\textbf{PoW LC is secure with block rate independent of the security parameter (\figref{comparison-bddelay-bdbandwidth}(c)).}\;\;
On a high level, bandwidth-related attacks require the adversary to release withheld blocks
to distract honest nodes from downloading honestly produced blocks.
In PoW, blocks spent for an attack today cannot be spent tomorrow, and vice versa.
Thus, the adversary is subject to an overall budget constraint.
The analysis of~\cite{bwlimitedposlc} ignores this constraint.
Instead, it assumes that at every moment the adversary uses the maximum number of blocks
it has available in \emph{any} of its strategies (which is possible in PoS).
Thus, \cite{bwlimitedposlc} replaces the overall worst-case adversary with a fictitious one that acts worst-case \emph{point-wise}.
%
This makes the analysis of~\cite{bwlimitedposlc} overly pessimistic for PoW.
We provide a \emph{new analysis technique} (\secref{introduction-methods-analysis}) that might be of independent interest
and with which we can capture
%
the budget constraint of the adversary.
%
%



\smallskip\noindent\textbf{\sapos, a variant of PoS LC that is secure with block rate independent of the security parameter (\figref{comparison-bddelay-bdbandwidth}(c)).}\;\;
We learn from the PoW result to modify PoS LC to achieve this.
%
%
%
Due to
equivocations in PoS,
%
the budget constraint of PoW does not readily carry over to PoS. Rather, the adversary can produce many blocks per block production opportunity, and use these blocks
%
%
to attack at different points in time.
In fact, 
%
\cite{bwlimitedposlc} explicitly gives this reasoning for their approach, and our attack in \secref{pos-attack} exploits this effect.
%

To re-introduce the budget constraint of PoW LC into PoS LC,
we propose \emph{\equivocationremoval} (\secref{introduction-methods-pos}).
Thereby, we preserve the LC protocol's simple blockchain structure, but modify it so that per block production opportunity, honest nodes download at most one of possibly many equivocating blocks.
To this end, honest nodes collectively remove the content of equivocating blocks before they reach the ledger of confirmed transactions.
We call the PoS LC protocol with this modification \emph{\sapos}, for \emph{Sanitizing-Proof-of-Stake}.
Based on our analysis, we calculate the minimum sufficient bandwidth to secure PoW LC and \sapos with the parameters of major PoW/PoS blockchain implementations (\figref{bitcoin-cardano-resilience-bandwidth}).



%
%
%
%

%

%
%
%
%
%
%


%
%
%
%
%
%
%

%
%
%
%

%
%
%
%
%
%
%
%
%
%

%
%
%
%


%
%
%
%


%
%
%
%
%
%
%
%
%

%
%
%
%
%

%


%
%
%

\smallskip\noindent\textbf{Ensure all transactions have their fee paid.}\;\;
\Equivocationremoval comes with a drawback:
At the time of block production, an honest node might not yet have learned about equivocating blocks in its prefix, and as a result might add transactions to the newly produced block that at execution turn out invalid, due to \equivocationremoval. This \emph{lack of predictable transaction validity} 
%
leads to
attacks where the adversary spams the ledger with transactions whose funding source is later invalidated, so no fees can be claimed for the resources they occupy.
We present a mechanism (\secref{introduction-methods-throughputloss}) to ensure appropriate fees get paid.


%

%
%





\subsection{Related Works}
\label{sec:introduction-relatedworks}

Several earlier works have analyzed the security of 
%
PoW~\cite{backbone,nakamoto_paper,dem20,pss16,kiffer2018better,ren,tight_bitcoin} and 
%
PoS~\cite{kiayias2017ouroboros,david2018ouroboros,badertscher2018ouroboros,sleepy,snowwhite,dem20,pos_paper} LC protocols 
in the bounded delay model.
%
%
%
%
%
%
Our analysis builds on tools from several of these works, primarily pivots~\cite{sleepy} (or Nakamoto blocks~\cite{dem20}), and convergence opportunities~\cite{pss16, sleepy, kiffer2018better} (or similar~\cite{dem20,ren}).
%
%

%
%
%
%
%
%
%
%


%

%
%
%
%
%
%
%
%
%
%


In the bounded delay model, what is the value of the bound~$\Delta$?
This is an important question because the parameters of the protocol, such as block production rate, must be tuned according to the delay bound.
It is a tricky question because unlike the bandwidth limit, which is a physical limit of the hardware used, delay depends on the network load.
%
One approach is to set the delay to the `time taken to process one block', \ie, $\Delta = 1/\bwtime$.
While this may be reasonable at rates much smaller than the bandwidth (as processing queues are mostly empty), queuing delay breaks this bound otherwise.
A more conservative approach is to set the delay to be at the tail of the probability distribution of the delay.
In theory, given an enqueuing and dequeuing process, it is possible to characterize the distribution of the queuing delay, and this approach is taken in \cite{near-optimal-thruput}.
In practice, the delay distribution can be estimated through network experiments \cite{decker,kiffer2021under}.
Another work \cite{longest-chain-random-delay} analyzes security in a random (\iid) delay model.

The problem here is that the network load, hence queuing delay, is not purely a random process, but it is controlled by the adversary.
This effect is hard to see in experiments.
%
The analysis in \cite{bwlimitedposlc}, although in a bounded bandwidth model, parameterizes the protocol according to a delay bound that holds under the worst-case adversary at all times with overwhelming probability.
The above approaches that choose high-probability delay bounds lead to a conservative parameterization where the block rate must decrease as the error probability of the delay bounds decreases (increasing security parameter) as in \cite{bwlimitedposlc}.
As mentioned in \secref{introduction-question}, our analysis exploits the limited supply of blocks in PoW (and in PoS after \equivocationremoval) to show that while not all honest blocks may be downloaded in time, at least some of them will be, enough to overcome the adversary's power.


%
%
%
%
%
%

%
%
%
%
%
%

%
%



%

%
%
%
%
%
%
%
%
%
%
%

%
%
%
%
The unsuitability of the bounded delay model at high throughput
led to a more careful modeling of limited communication capacities of nodes.
%
%
Increase in delay due to increase in the block size was pointed out in early Bitcoin discussions~\cite{btc-blocksize-war} and also in experiments with the Bitcoin network~\cite{decker}. 
Works~\cite{ghost,prism} model and analyze this effect, but the model still assumes that processing delays are bounded as long as the average network load is below the capacity, thus failing to capture increasing queuing delays while operating near capacity.
Increase in delay due to increasing rate of block production was analyzed in \cite{near-optimal-thruput}, capturing the relationship between the bandwidth constraint and queuing delays.
Work \cite{bwlimitedposlc} extends and formalizes the model from \cite{near-optimal-thruput} to consider adversarial spamming, particularly due to equivocations in proof-of-stake.



%


%

%
%
%

%
Capacity limits apply not only to downloads, but also to processing of blocks.
%
For instance, to validate a block, an Ethereum validator must execute all smart contracts in it.
While download and processing are similar in that the time taken increases with the number of transactions, they are different in that processing is hard to parallelize due to transactions that depend on each other.
A line of work \cite{adding-concurrency-smart-contracts, speculative-concurrency-eth} studies methods to parallelize execution of smart contracts to 
%
make use of multi-core architectures.


%






%
\subsection{Overview of Methods}
\label{sec:introduction-methods}

%
\subsubsection{New Analysis Technique}
\label{sec:introduction-methods-analysis}

%

\import{./figures/}{fig-analysis-comparison-sleepy.tex}

%
Traditional LC security analysis (\figref{analysis-comparison-sleepy}(a)) is based on the notion of a \emph{pivot}~\cite{sleepy} (or \emph{Nakamoto block}~\cite{dem20}).
A pivot is a point of time in which a block is produced by an honest node (\ie, it includes pending transactions) with 
an additional property that in every interval around the pivot, there are more honest than adversarial block production opportunities.
A probabilistic argument shows that typically pivots happen frequently.
A combinatorial argument shows that the pivot block remains in the longest chains of all honest nodes forever.
%
%
%
%
Safety and liveness of LC with suitable parameters follow swiftly.

%
%
%
%
%
In the bounded delay network model,
the qualities required for the probabilistic and combinatorial argument, respectively, 
are equivalent. As a result, it has not been widely observed that these properties are actually not identical.
In the bounded bandwidth model, these properties are no longer equivalent.
Our first conceptual contribution is to
%
decompose pivots' probabilistic/combinatorial qualities into \emph{\sltpps} and \emph{\sltcps} (\figref{analysis-comparison-sleepy}(b)).
\sltPps are honest block production events where in every time interval around them there are more
%
honest than adversarial block production opportunities (same as pivots in the bounded delay analysis).
\sltCps are honest block production events where in every time interval around them there are more \emph{chain growth} events than non-chain-growth events, where chain growth occurs 
only when an honest block is produced \emph{and downloaded soon} by all honest nodes. 
%

Some \sltpps no longer turn into \sltcps under bounded bandwidth, because adversarial block 
%
release can delay the download of honestly produced blocks, and thus some honest block production opportunities might not translate to chain growth.
Our first technical contribution is a combinatorial argument to show that if there is a sufficiently high density of \sltpps over a sufficiently long time interval, then one of these \sltpps is typically a \sltcp.
This relies on the adversary's limited budget of blocks it can spam with.

%
The original probabilistic argument of Sleepy~\cite{sleepy} guarantees only a fairly low density of \sltpps.
Thus, our second technical contribution is to show, using a Chernoff-style tail bound for weakly dependent random processes, that long time intervals typically have a high density of \sltpps.
%
This completes the analysis for PoW LC.


%
\subsubsection{\EquivocationRemoval}
\label{sec:introduction-methods-pos}

%
%
%
%
%
%

%
%
To control bandwidth consumption, we stipulate that in PoS, per block production opportunity, every honest node downloads at most one equivocating block.
To ensure that honest nodes can still switch from one chain to another longer chain, both of which might contain a different equivocating block from the same block production opportunity,
%
we allow honest nodes to not download, but treat as empty, any block for which they see an equivocation.
Note that headers of two equivocating blocks from the same 
block production opportunity
%
can serve as a \emph{succinct equivocation proof} in that they suffice for honest nodes to convince one another that an equivocation was committed.
Therefore, if an honest node that sees an equivocation for a block in its longest chain, publishes an equivocation proof in the block that it produces,
nodes can agree on which blocks were equivocated and hence consistently treat them as empty.
%

%
A caveat so far is that an adversary could reveal an equivocation late and cause inconsistent ledgers across honest nodes and/or time.
%
%
To avoid this, we enforce a deadline for how late an equivocation proof can be included in the chain.
%
%
Our analysis shows how to parameterize the 
%
deadline
and the LC protocol's confirmation time such that, if any honest node has removed the content of any equivocating block on its longest chain, then an appropriate equivocation proof is timely included on-chain, and all honest nodes remove the block's content before it reaches the output ledger.


%
\subsubsection{Ensuring Fees Get Paid Despite Lack Of Predictable Validity}
\label{sec:introduction-methods-throughputloss}

%

%
%
%

%

%

\Equivocationremoval leads to \emph{lack of predictable transaction validity},
which risks that the adversary gets to spam the ledger with transactions `for free'.
To ensure that honest block producers only include transactions that pay for their blockspace,
we propose to introduce \emph{gas deposit accounts} that can only be used for transaction fees. We also require that any deposit to such an account is not reflected in the balance until the deadline has passed for the inclusion of any equivocation proof that might lead to removal of transactions from the deposit's prefix.
This gives honest block producers a lower bound on the account's balance (\eg, more funds than the lower bound might be available if a transaction spending from the gas account gets removed due to an equivocation) which they can use to reliably determine whether a transaction can pay fees.
Withdrawals from these accounts can take place immediately.




%
%
%
%
%

%
%
%

%
%
%
%
%
%

%


%
%
%
%
%

%



%

%
%
%
%
%
%
%
%
%
