\section{Conclusion}
\label{sec:conclusion}
%
In this work we focused on the security of the longest chain protocol both in the PoW and PoS settings. While block downloading and processing is usually implemented in an ad-hoc manner and is not typically discussed in the context of the protocol's security analysis, our work highlights the importance of correctly prioritizing block download and processing. 
In addition to providing a security proof using new techniques, and attacks on natural prioritization rules in the PoW setting, we also propose \sapos, a new proof-of-stake variant. Several important open questions remain:
\begin{itemize}
    \item There remain gaps between security bounds we provide in the PoW setting and the known attacks in this case (\cf \figref{comparison-bddelay-bdbandwidth}(b)). Can better attacks be found? What are the optimal prioritization rules for which security is achieved?
    \item In the PoW setting, the attacker is unable to equivocate, but in \sapos we were forced to deal with equivocations. This came at a cost to the latency of transaction execution, and with decreased certainty about the state at which the transaction is eventually executed. Can these costs be avoided, so that PoS based LC is on par with the PoW variant?
    \item The difficulty adjustment algorithm (DAA) seems to apply even more stress to limited capacity nodes. Can DAAs be designed for this setting and incorporated into the security analysis?
    \item Can processing and download parallelization, pre-processing and pre-fetching of blocks be utilized more efficiently in order to securely improve the throughput of LC based protocols?
    \item In \sapos, the 
    %
    deadline for including equivocation proofs
    is not user-dependent, but baked into the protocol.
    %
    A user cannot increase this to achieve lower error probability.
    This is a drawback compared to traditional Nakamoto consensus. Can it be avoided?
\end{itemize}



%
%
%
%


%


%



