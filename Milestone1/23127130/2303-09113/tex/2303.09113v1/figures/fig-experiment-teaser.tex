\begin{figure}[tb]%
    \centering%
    \begin{tikzpicture}[]
        \footnotesize
        \begin{axis}[
                mysimpleplot,
                %
                xlabel={Bandwidth: $\bwtime$ [blocks per second]},
                ylabel={Honest chain growth rate\\ ($\blkratetimeGrowth/\blkratetimeHon$)},
                legend columns=2,
                xmin=0, xmax=2,
                ymin=0, ymax=0.7,
                height=0.5\linewidth,
                width=\linewidth,
                yticklabel style={
                        /pgf/number format/fixed,
                        /pgf/number format/precision=2
                },
                scaled y ticks=false,
                grid = major,
                major grid style={solid,draw=gray!10},
            ]

            \addplot [myparula11, %
                    only marks, mark size = 1pt] table [x=bandwidth,y=chain_growth] {figures/fig-experiment-teaser-noattacker-data.txt};
            \addlegendentry{Silent attacker};
            \label{plt:experiment-teaser-noattacker};

            \addplot [myparula73, mark size=1pt,%
            only marks,
            mark=o] table [x=bandwidth,y=chain_growth] {figures/fig-experiment-teaser-activeattacker-data.txt};
            \addlegendentry{Teasing attacker};
            \label{plt:experiment-teaser-attacker};


        \end{axis}
    \end{tikzpicture} %
    \vspace{-0.5em}%
    \caption{%
    The rate of chain growth relative to honest block production, when nodes prioritize downloads towards the longest known header chain, for various bandwidths. With a \teaserattack (\ref{plt:experiment-teaser-attacker}), processing is effectively slowed by a factor of $2$, which lowers the growth rate of the chain (and hence lowers security, \cf \figref{comparison-bddelay-bdbandwidth}(c)) compared to a silent attacker (\ref{plt:experiment-teaser-noattacker}). 
    }%
    
    \label{fig:experiment-teaser}%
\end{figure}%