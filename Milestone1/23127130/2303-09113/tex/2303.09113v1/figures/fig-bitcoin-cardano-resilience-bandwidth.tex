\begin{figure}[tb]%
    \centering%
    \begin{tikzpicture}[]%
        %
        \begin{axis}[
                mybandwidthplot01,
                name=plot1,
                ymode=log,
                ymin=1,ymax=1e4,
                ytick={1,10,100,1000,10000},
            ]

            \addplot [myparula21,no marks,name path=bandwidth] table [x=beta,y=C(Mbps)] {figures/fig-bitcoin-resilience-bandwidth.txt};
            \addlegendentry{Bitcoin (PoW)};

            \addplot [myparula11,no marks,name path=bandwidth] table [x=beta,y=C(Mbps)] {figures/fig-cardano-resilience-bandwidth.txt};
            \addlegendentry{Cardano, with \equivocationremoval (PoS)};
        \end{axis}
    \end{tikzpicture}%
    \vspace{-0.5em}%
    \caption[]{%
        Calculation based on \thmref{safety-and-liveness-pow,safety-and-liveness-pos} of the minimum bandwidth per node that is sufficient to ensure security of LC with the parametrizations used by two major blockchains: Bitcoin (PoW, $\blkratetime = 1/600\;\mathrm{blocks/s}$, max. block size $1\;\mathrm{MB}$), and Cardano (PoS, $\slotduration=1\;\mathrm{s}$, $\blkrateslot=1/20\;\mathrm{blocks/slot}$, max. block size $88\;\mathrm{KB}$). This suggests that to defend against worst-case attacks, Bitcoin might need more per-node bandwidth than commonly recommended ($0.4\;\mathrm{Mbps}$~\cite{bitcoin_requirements}).%
    }%
    \label{fig:bitcoin-cardano-resilience-bandwidth}%
\end{figure}%