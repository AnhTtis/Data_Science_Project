\begin{figure}[tb]%
    \centering%
    \begin{tikzpicture}[]
        \footnotesize
        \begin{axis}[
                mysimpleplot,
                xmode=log, 
                xlabel={Bandwidth: $\bwtime$ [blocks per second]},
                ylabel={Growth of agreed chain\\ ($\mathrm{height}/\blkratetimeHon\cdot T$)},
                legend columns=2,
                xmin=1e-2, xmax=1e3,
                ymin=0, ymax=1.005,
                height=0.5\linewidth,
                width=\linewidth,
                yticklabel style={
                        /pgf/number format/fixed,
                        /pgf/number format/precision=2
                    },
                scaled y ticks=false,
                grid = major,
                major grid style={solid,draw=gray!10},
            ]

            %
            %
            %
            %

            %
            %
            %
            %

            %
            %
            %

	
            \addplot [myparula11, only marks,
                     mark size = 1pt] table [x=bandwidth,y=max_ancestor_height] {figures/fig-experiment-greedy-longestdl-data.txt};
            \addlegendentry{\ruleLc policy};
            \label{plt:experiment-greedy-longestdl};

            \addplot [myparula73, mark size=1pt,%
            dotted, 
            mark=o ] table [x=bandwidth,y=max_ancestor_height] {figures/fig-experiment-greedy-greedydl-data.txt};
            \addlegendentry{\ruleGreedy policy};
            \label{plt:experiment-greedy-greedydl};



        \end{axis}
    \end{tikzpicture}%
    \vspace{-0.5em}%
    \caption{%
        The rate nodes grow the agreed chain after the network splits into two sets of 50 nodes for 15 secs, when the download rule is ``longest-header-chain'' (\ref{plt:experiment-greedy-longestdl}) or ``\rulegreedy'' (\ref{plt:experiment-greedy-greedydl}).
        %
        %
        %
        %
        %
        %
        %
        %
        %
        Nodes using the \rulegreedy policy prioritize downloads on their current chain.
        Under low bandwidth, they do not recover from the split, resulting in two chains forking at genesis, providing no growth of the agreed chain. Thus, longest chain is insecure \emph{without an adversary}
        (\cf \figref{comparison-bddelay-bdbandwidth}(c)).
    }%
    \label{fig:experiment-greedy}%
\end{figure}%