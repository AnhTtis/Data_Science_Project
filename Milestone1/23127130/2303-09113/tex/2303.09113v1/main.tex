\pdfoutput=1
\documentclass[sigconf,nonacm,natbib=false]{acmart}
%
%
%
%
%
%
\setcopyright{none}
\settopmatter{printacmref=false}
\settopmatter{printfolios=true}


\usepackage{import}
\import{./lib/}{boilerplate_pdflatex.tex}
\import{./lib/}{colors.tex}
\import{./lib/}{tikzpgfplot.tex}
\import{./lib/}{tikzpgfplot_parulalinestyles.tex}
\import{./lib/}{tikzpgfplot_plotstyles.tex}
\import{./lib/}{tikzpgfplot_blockchainstructures.tex}
\import{./lib/}{tables.tex}
\import{./lib/}{algorithms.tex}
\import{./lib/}{spacinghacks.tex}
\import{./lib/}{lateplate_pdflatex.tex}

\def\eg{\emph{e.g.}}
\def\Eg{\emph{E.g.}}
\def\ie{\emph{i.e.}}
\def\Ie{\emph{I.e.}}
\def\etal{et al.}

\newcommand{\m}{\mathbf{m}}


%
\usepackage[datamodel=acmdatamodel,style=acmnumeric]{biblatex}
\addbibresource{references.bib}


\begin{document}

%
%
%
%
%
%
%
%
%
%
%
%
\title{Security of Blockchains at Capacity}
%


\author{Lucianna Kiffer}
\email{lkiffer@ethz.ch}
\affiliation{\country{}}
%
\author{Joachim Neu}
\email{jneu@stanford.edu}
\affiliation{\country{}}
%
\author{Srivatsan Sridhar}
\email{svatsan@stanford.edu}
\affiliation{\country{}}
%
\author{Aviv Zohar}
\email{avivz@cs.huji.ac.il}
\affiliation{\country{}}
%
\author{David Tse}
\email{dntse@stanford.edu}
\affiliation{\country{}}

\thanks{LK, JN, SS and AZ are listed alphabetically.}

%
\newcommand{\gitSourceUrl}[0]{\url{https://github.com/avivz/finitebwlc}}


\begin{abstract}
    Given a network of nodes with certain communication and computation capacities,
    what is the maximum rate at which a blockchain can run securely?
    We study this question for proof-of-work (PoW) and proof-of-stake (PoS) longest chain protocols under a `bounded bandwidth' model which
    %
    captures queuing and processing delays due to high block rate relative to capacity, bursty release of adversarial blocks, and in PoS, spamming due to equivocations.
    %
    %

    We demonstrate that security of both PoW and PoS longest chain, when operating at capacity, requires carefully designed scheduling policies that correctly prioritize which blocks are processed first,
    as we show attack strategies tailored to such policies.
    In PoS, we show an attack exploiting equivocations,
    %
    which highlights that
    %
    the
    throughput
    %
    of the PoS longest chain protocol
    with a broad class of scheduling policies
    %
    must decrease as the desired
    %
    security error probability
    decreases.
    %
    %
    At the same time, through an improved analysis method,
    our work is the first to
    %
    identify block production rates under which PoW longest chain is secure in the bounded bandwidth setting.
    %
    We also present the first PoS longest chain protocol, \sapos, which is secure with
    a block production rate
    %
    independent of the security error probability, by using an `equivocation removal' policy to prevent equivocation spamming.
    %
    %
    %
    %
    %
    %
    %
    %
    %
    %
    %
    %
    %
    %
    %
    %
    %
    %
    %
    %
    %
    %
    %
    %
    %
    %
    %
    %
    %
    %
    %
    %
    %
\end{abstract}


\maketitle

\section{Introduction}

The ability to reason about plans is critical for performing long-horizon tasks \citep{erol1996hierarchical, sohn2018hierarchical, sharma-etal-2022-skill}, compositional generalization \citep{corona-etal-2021-modular} and generalization to unseen tasks and environments \citep{shridhar2020alfred}.
Consider a simple long-horizon planning scenario where a robot is tasked with preparing a meal and serving it on the table. 
This presents a non-trivial planning problem since the agent needs to understand the sequence of operations required to perform the task and search for the relevant objects in the unfamiliar environment by interacting with various objects. %



Large language models have been recently shown to possess commonsense knowledge about the world such as object affordances and physical dynamics \citep{ouyang2022training,chowdhery2022palm}.
Early approaches considered text based environments and fine-tuned PLMs to predict actions given the history of past observations and actions \citep{jansen-2020-visually,micheli-fleuret-2021-language,yao-etal-2020-keep}.
Recent work has used this ability to reason about plans from text instructions in simulated household environments with simplifying assumptions such as text-only environment observations or feedback \citep{huang2022language,ahn2022can,li2022pre,logeswaran-etal-2022-shot}.


We focus on \emph{visually grounded planning} with PLMs --- the ability to adapt plans based on interaction and visual feedback from the environment.
While PLMs have strong planning commonsense priors, predictions from a PLM may not be directly realizable in the environment since the observation and action spaces are unknown.
This requires \emph{grounding} the PLM in the environment and adapting it to observe visual feedback, which is highly non-trivial.
Some prior works assume the availability of a pre-trained affordance function \citep{ahn2022can} or a success detector \citep{mirchandani2021ella}.
Notably, SayCan \citep{ahn2022can} completely decouples the PLM from observation information by selecting actions that have both high affordability (through a pre-trained affordance model) and high PLM likelihood.
Although this partially addresses the grounding problem, the use of visual feedback for action affordance alone is limited.
Often an agent must choose one of many affordable actions using information from observations.
For example, a driving agent should re-navigate and possibly turn around when encountering a ``road closed'' sign, but both turning around and driving forward are indistinguishable to SayCan because they are both affordable and the PLM is blind to observations.

Another workaround explored in prior work is translating the information in the visual observations to text using a pre-trained captioning system \citep{shridhar2021alfworld,huang2022language}.
However, it can be difficult to faithfully describe an image in words and information is lost in this inherently noisy process, which limits the information available to the planner.



Recent work shows that PLMs can be adapted for various natural language tasks by inserting tunable embeddings or soft prompts at the input of the PLM (also called prompt tuning or prefix tuning)~\citep{li-liang-2021-prefix,lester-etal-2021-power}.
This approach also extends to multi-modal understanding tasks such as image captioning \citep{mokady2021clipcap} and VQA \citep{tsimpoukelli2021multimodal} where images are encoded as soft prompts and finetuned for the target task.
Transformer based architectures have also been successfully applied to offline Reinforcement Learning in recent work \citep{chen2021decision,janner2021offline,li2022pre,reid2022can}.

Taking inspiration from these works, we propose the simple approach of embedding visual observations (`visual prompts') and \textit{directly inserting them as PLM input embeddings}.
The visual encoder and PLM are jointly trained for the target task, an approach we call \textbf{\oursfull}~(\ours).
By teaching the PLM to use observations for planning in an end to end manner, we remove the dependency on external data such as captions and affordability information that was used in prior work.
We show that this simple approach performs better than prior PLM-based planning approaches on two embodied planning benchmarks based on ALFWorld~\citep{shridhar2021alfworld} and Virtualhome~\cite{puig2018virtualhome}.



\section{Scheduling Policies \& Attacks}
\label{sec:experiments}
Since download and processing resources are constrained, it becomes increasingly important to correctly prioritize the blocks that are downloaded and validated.  In this section we describe two possible scheduling policies for nodes running Nakamoto consensus. We show attacks tailored to each such policy and thus show that the choice of policy has a high impact on security. The attacks in this section apply to both PoW and PoS Nakamoto consensus, as the attacks only exploit the block production process that is common to both. The precise setup of the attacks is described in \secref{attacks-details}.

In the PoW setting, since headers contain all information needed to verify that enough work has been spent to mine the block, invalid block headers can be ignored, and the attacker is unable to produce blocks without spending computation. Similarly in the PoS setting, the attacker cannot produce blocks for a \timeslot where it is not elected a leader as per the PoS lottery. We restrict ourselves to process only blocks whose parent block is already fully validated. Thus, when we describe the priority of some header block as high, we actually start to process its first unprocessed ancestor.



\smallskip\noindent\textbf{The \ruleLC policy.}\;\;
This policy aims to match Nakamoto consensus' confirmation rule. It prioritizes the processing of blocks that are on the longest announced header chain, regardless of which blocks we already have. We assign each unprocessed header a priority $h$ if it is on a header chain of height $h$.


\smallskip\noindent\textbf{The \ruleGreedy policy.}\;\;
This policy prioritizes downloading blocks that extend the chain a node has already processed. If a header of a block at height $h$ is announced, and we already have $h_i$ blocks from that chain, then we set the priority of the block to be $(h_i,h)$ and compare between the two priorities lexicographically.

%
%

%

\subsection{Attacking the \ruleLC Policy: The \TeaserAttack}
\label{sec:teaser-attack}

\import{./figures/}{fig-attack-teaser.tex}
The \ruleLC policy seems to be a natural policy when considering the longest-chain protocol. We would like to consider attacks that break the safety or liveness of the chain using as little mining power as possible. The naive attack strategy, 
%
that bounded-delay analysis suggests to be worst-case~\cite{dem20},
is to have the attacker mine a secret chain of blocks without releasing any blocks to the network. If the attacker is able to outpace the rate of growth of the honest chain, it can publish its blocks at will, and undo all transactions in the blockchain.
Since honest nodes take time to process each block, the rate of growth of the chain is slower than the honest nodes' block creation rate and the attacker can more easily succeed in this naive attack if bandwidth is low.
%
This effect is detailed in \secref{experiment-growth}.
But can we do better than this attack?

\smallskip\noindent\textbf{The \TeaserAttack.}\;\;
We show that in fact, an adversary can strategically announce headers and release blocks in a way that will exploit the \rulelc policy to waste some processing done by the honest nodes. In \figref{attack-teaser}, we describe the \teaserattack that achieves this. We see that this attack utilizes the chain that the attacker constructs not only to later overtake the public chain, but also to induce processing of one extra block for every block that grows the length of the honest chain. It therefore effectively doubles the processing invested per growth event of the public main chain. To entice nodes to process blocks needlessly, the attacker reveals a long header chain from its secret chain, teasing the nodes to download a block from that chain, but makes the rest of the blocks unavailable for download.
%

\import{./figures/}{fig-experiment-teaser.tex}

To demonstrate the attack, we simulate a network with 100 nodes that together create 1 block per second.
The simulations were written as event-driven simulations using Python's simpy package.\footnote{Source code: \gitSourceUrl}
See \secref{attacks-details} for details on the implementation and setup.
To start the attack, the attacker must pre-mine a short private chain longer than the honest chain. Even if the attacker's mining rate is lower than the honest mining rate, the attacker still succeeds in doing this with some probability. Thereafter, it starts the \teaserattack and the subsequent increased processing delays slow down the honest chain growth rate, which allows the adversary to maintain its lead.

We compare the rate of growth of the honest chain when the attacker mines silently and does not publish blocks (private attack) to the scenario in which a \teaserattack (\figref{attack-teaser}) is being carried out. The chain's rate of growth then sets the bound on the system's security: if the attacker manages to mine faster than the honest network is growing (\ie $\blkratetimeAdv > \blkratetimeGrowth$), then it is able to continue the attack indefinitely.

\Figref{experiment-teaser} depicts the result of our simulation. It shows
%
that with the \teaserattack, the network's processing power is slowed roughly by a factor of $2$. As a result, the effective delay of block propagation increases, and the attacker succeeds with greater ease compared to the naive attack. 
Since the silent attack is the worst-case attack 
for Nakamoto consensus according to bounded-delay analysis~\cite{dem20},
%
%
the \teaserattack demonstrates that scheduling policies must be taken into account when considering the security of the protocol, and sufficient capacity needs to be provisioned to ensure security.


\subsection{The \ruleGreedy Policy and the \GreedyAttack}
\label{sec:greedy-attack}

The \teaserattack relied strongly on the fact that the attacker could entice nodes with a long header chain that is later discovered to be unavailable for download. It is natural in this case to consider adjusting the download rule to one that prefers the proverbial `bird in the hand over two birds in the bush', \ie, to extend the blocks we already downloaded over the illusive promise of a longer chain that the attacker may withhold from us. 

While the \rulegreedy policy performs well at high processing rates, we unfortunately find that it preforms poorly in the low processing rate regime. Specifically, if a fork in the chain occurs, and nodes are split evenly between the two alternatives, the fork may never resolve. This is because nodes extend their own chain, and prioritize download on their side of the split, while having insufficient processing power to catch up with the other alternative chain. A fork in the chain can result from a deliberate attack by an attacker that releases blocks selectively to different nodes, by a network split, or worse, by an unlucky timing of honest node mining events. In this case, the blockchain fails even for small attackers. 
Importantly, a fork that never resolves is either a safety or a liveness failure, as no transaction on either side of the split can be safely accepted.

\import{./figures/}{fig-experiment-greedy.tex}

To demonstrate this download rule in action, we simulate a network of 100 nodes that are split evenly between two partitions for only 15 seconds, \ie, for an expected time required to produce 15 blocks.%
\footnote{Such short splits are relatively easy to induce in reality (transient problems with Internet routing, denial-of-service on the network, etc.) and thus a practical scheduling rule must recover from such splits.}%
Once the network split ends, the simulation continues for another 4000 seconds, allowing nodes the opportunity to 
converge on a chain.
%
We 
%
measure the height of the latest block all nodes agree upon. If nodes do not recover from the partition, 
this block will be the genesis and the liveness of the protocol has failed. Otherwise, nodes quickly agree on the main chain and the height of the latest agreed block is 
just a little behind the longest tip of the chain. 

%

We simulate the evolution after a brief partition for both the \rulelc policy as well as for the \rulegreedy policy. Our results (\figref{experiment-greedy}) show that in settings where bandwidth is greater than $1/2$, nodes manage to catch up with the chain and the rate of growth matches for both scheduling policies. In lower bandwidth settings, however, nodes never catch up.
Note that this attack requires no adversarial mining,
yet the protocol is insecure (\cf \figref{comparison-bddelay-bdbandwidth}(c)).
This is in stark contrast to the bounded-delay analysis
which suggests that the protocol retains security
against a non-mining adversary
%
at any bandwidth (\cf \figref{comparison-bddelay-bdbandwidth}(a)),
and highlights again the need to study the security of blockchains at capacity.



%
%



\section{Protocol \& Model}
\label{sec:modelprotocol}

%
Pseudocode of an idealized LC protocol $\protocol$ is provided in \algref{generic-lc-protocol}.
Details of the protocol's resource-based block production lottery, \ie, of production and verification of blocks, are abstracted through an idealized functionality $\Ftree$ (\cf~\cite[Fig.~2]{sleepy}, \cite[Alg.~3]{bwlimitedposlc}).
Pseudocode for instantiations $\FtreePoW$ and $\FtreePoS$ modeling proof-of-work (PoW) and proof-of-stake (PoS) are provided in \algref{hdrtree-pow} and \algref{hdrtree-pos}, respectively.
Helper functions used in the pseudocode are detailed in \secref{algos-reference-helperfunctions}.
We study these protocols in a unified model for a network $\Env$ with finite bandwidth (\figref{model}), and for the powers and limits of an adversary $\Adv$.



\subsection{Longest Chain Protocols}
\label{sec:modelprotocol-protocolfeatures}

\import{./figures/}{alg-generic-lc-protocol.tex}
\import{./figures/}{alg-hdrtree-pow.tex}
\import{./figures/}{alg-hdrtree-pos.tex}
\import{./figures/}{alg-longest-header-chain-rule.tex}

%
%
%
%

For ease of exposition, the execution features a \emph{static} set of $N$ \emph{equipotent} \emph{nodes}, each of which runs an independent instance of $\protocol$.
Temporary crash faults (`sleepiness') of nodes (in PoW and PoS), heterogeneous distribution of hash power (in PoW) or stake (in PoS), and stake shift (in PoS)
or difficulty adjustment (in PoW),
are left to be addressed with techniques from~\cite{david2018ouroboros,snowwhite,sleepy,DBLP:conf/crypto/GarayKL17}.
%
We are interested in the large system regime $N\to\infty$.
Nodes interact with each other and with the adversary $\Adv$ through an environment $\Env$ that models the network and is detailed in \secref{modelprotocol-modelfeatures,algos-reference-environment}.
The protocol proceeds in \emph{\timeslots} of duration $\slotduration$ (\myalgref{generic-lc-protocol}{generic-lc-protocol-mainloop}).
At each \timeslot $t$, the protocol queries the block production lottery $\Ftree$ in an attempt to extend the longest downloaded chain $\dC$ in the node's view with a new block 
of
%
pending transactions $\txs$.
If successful, the node disseminates both the resulting \emph{block header} $\Chain'$ and the associated \emph{block content} $\txs$ via the environment $\Env$ to all nodes.
%
%
Finally, the protocol identifies the $\confDepth$-deep prefix $\dC\trunc{\confDepth}$ containing all but the last $\confDepth$ blocks of $\dC$.
The transactions along $\dC\trunc{\confDepth}$ are concatenated to produce the output ledger $\LOG{}{t}$.
%
%
%

When a node $p$ receives a new valid block header $\Chain$ (\myalgref{generic-lc-protocol}{generic-lc-protocol-receiveheader}), $p$ adds $\Chain$ to its header tree $\hT$, records $\Chain$ as first seen at the current \timeslot, and relays $\Chain$ to all other nodes via $\Env$.
Throughout the execution, the protocol requests from $\Env$ the content 
for
%
block headers 
%
decided by a download priority rule (\myalgref{generic-lc-protocol}{generic-lc-protocol-downloadrule}).
As a concrete example, we use the `download longest header chain' rule (\algref{longest-header-chain-rule})
in which
a node downloads content for the first block header with unknown content on the longest header chain it has seen.
%
%
Once a valid block's content is received (\myalgref{generic-lc-protocol}{generic-lc-protocol-receivecontent}), the node makes it available to other nodes via $\Env$, and updates its 
%
$\dC$.


\paragraph{Proof-of-Work}
%
The characteristics of PoW-based block production, \eg, in Bitcoin~\cite{nakamoto_paper,backbone}, are captured by the idealized functionality $\FtreePoW$ (\algref{hdrtree-pow}).
Each block production attempt is committed to a parent block and block content (\myalgref{hdrtree-pow}{hdrtree-pow-binding}), and only a single block is produced when the attempt is successful.
Per \timeslot, each node can make one block production attempt that will be successful with probability $\blkrateslot/N$, independently of other nodes and \timeslots (\myalgref{hdrtree-pow}{hdrtree-pow-blockproductionlottery}).
This model, for ease of exposition, assumes uniform hash power across all nodes.
Since each \timeslot represents a single PoW evaluation, we study PoW in the regime $\blkrateslot = \Theta(\slotduration)$, $\slotduration \to 0$.
In turn as $\blkrateslot \to 0$, with probability $1$, each \timeslot produces at most one block across all nodes.
%
%
The PoW model thus implies that every block must be produced in a \timeslot \emph{strictly after} its parent block.


\paragraph{Proof-of-Stake}
%
PoS LC protocols such as from the Ouroboros~\cite{kiayias2017ouroboros,david2018ouroboros,badertscher2018ouroboros} or Sleepy Consensus~\cite{sleepy,snowwhite} families can be modeled using $\FtreePoS$ (\algref{hdrtree-pos}).
As in PoW, each node can make one block production attempt per \timeslot that will be successful with probability $\blkrateslot/N$, independently of other nodes and \timeslots (\myalgref{hdrtree-pos}{hdrtree-pos-blockproductionlottery})%
\footnote{There may be multiple blocks in one \timeslot, as in
%
%
the Ouroboros~\cite{kiayias2017ouroboros,david2018ouroboros,badertscher2018ouroboros} and Sleepy Consensus~\cite{sleepy,snowwhite} protocols.}%
, modeling uniform stake.
In PoS, however, (even past) block production opportunities can be `reused' to produce multiple blocks with different parents and/or content, \ie, to equivocate (\myalgref{hdrtree-pos}{hdrtree-pos-leader,hdrtree-pos-binding}).
The regime of interest is $\slotduration = \Theta(1)$.
%
%
%
%
%



\subsection{Bandwidth Constrained Network}
\label{sec:modelprotocol-modelfeatures}

\import{./figures/}{fig-model.tex}

We borrow the bandwidth constrained network model of~\cite{bwlimitedposlc} (\figref{model}).
In this model, $\Env$ abstracts push-based flooding of `small' block headers and pull-based downloading of `large' block contents from peers.
%
Block header chains sent via $\Env.\Call{broadcastHeaderChain}{}$ are eventually delivered by $\Env$ to every node,
%
\cf~\myalgref{generic-lc-protocol}{generic-lc-protocol-receiveheader}.
Headers are delivered with a per-node per-header delay determined by $\Adv$, up to a commonly known delay upper bound $\DeltaHeader$.
%
Block content made available via $\Env.\Call{uploadContent}{}$ is kept by $\Env$ in what can be thought of as a `cloud'.
Nodes can request the content associated with a particular header.
If content matching the header is available, then it is delivered by $\Env$ to the node,
%
\cf~\myalgref{generic-lc-protocol}{generic-lc-protocol-receivecontent}.
Content download is subject to a per-node bandwidth constraint of~$\bwtime$.
%
See \secref{algos-reference-environment} for a more formal description of $\Env$.

The `cloud' captures key properties of pull-based
peer-to-peer 
%
downloading. At first, content matching a particular header might not be available (\eg, $\Adv$ produced a block and disseminated its header, but withheld its content). Later, such content can become available (\eg, $\Adv$ releases the content to one honest node). Thus, the `cloud' ensures neither data availability nor strong consistency of query outcomes, unlike stronger primitives such as
verifiable information dispersal
%
\cite{avid,avidfp,dispersedledger,semiavidpr}. However, once content for a header does become available, it is unique and remains available. This captures the header's binding commitment to the content, and the fact that honest nodes share content 
%
with peers. Requests for unavailable content do not count towards the download budget.
Also note that the adversary can push headers and content bypassing bandwidth and delay constraints, and this models non-uniform bandwidth across nodes, and additional effects (analogous to adversarially controlled delay up to maximum $\Delta$ in the bounded delay model).


\paragraph{Powers and Limits of the Adversary}
%
The \emph{static} adversary $\Adv$ chooses a set of nodes (up to a fraction $\beta$ of all $N$ nodes, where $\beta$ is common knowledge) to corrupt before the randomness of the execution is drawn and the execution commences. Uncorrupted \emph{honest} nodes follow $\protocol$ at all times. Corrupted \emph{adversarial} nodes follow arbitrary computationally bounded \emph{Byzantine} behavior, coordinated by $\Adv$ in an attempt to break consensus.
%
Among other things, the adversary can:
withhold block headers and content, or release them late or selectively to honest nodes;
push headers and content to nodes while bypassing the delay and bandwidth constraints;
break ties in $\protocol$'s chain selection and content download policy;
in PoS, reuse block production opportunities to produce multiple blocks (\emph{equivocations}, \cf~$\FtreePoS.\Call{extend}{}$), and extend chains using past opportunities as long as the purported block production \timeslots along every chain remain strictly increasing.



\subsection{Security of Ledger Protocols}
\label{sec:modelprotocol-ledgersecurity}

For an execution of $\protocol$ where every honest node $p$ at every \timeslot $t$ outputs a ledger $\LOG{p}{t}$, we recall the security desiderata:
\begin{itemize}
    \item \emph{Safety:}
          For all adversarial strategies, for all \timeslots $t,t'$, and for all honest nodes $p, q$: $\LOG{p}{t}\preceq\LOG{q}{t'}$ or $\LOG{q}{t'}\preceq\LOG{p}{t}$.
    \item \emph{Liveness with parameter $\Tlive$:}
          For all adversarial strategies, if a transaction $\tx$ is received by all honest nodes by \timeslot $t$,
          then for every honest node $p$ and for all \timeslots $t' \geq t+\Tlive$, $\tx \in \LOG{p}{t'}$.
\end{itemize}
A consensus protocol is \emph{secure over \timeslot horizon $\Thorizon$ with parameter $\Tlive$} iff it satisfies safety, and liveness with parameter $\Tlive$, with overwhelming probability over executions of \timeslot horizon $\Thorizon$.

%
%
%
%
%
%
%
%
%



\subsection{Notation}
\label{sec:modelprotocol-notation}

Nodes are identified using $p, q$.
%
%
Our notation distinguishes between three notions of `time':
\emph{\Timeslots} of $\protocol$ are indicated by $r, s, t$.
\Timeslots in which one or more blocks are produced form a sub-sequence $\{t_k\}$, defined in \secref{analysis-definitions}.
\emph{\Iindices} into this sub-sequence are denoted by $i, j, k$.
%
Physical parameters of our model,
%
%
%
header propagation delay $\DeltaHeader$ and bandwidth $\bwtime$, are specified in units of \emph{real time}.
%

We denote intervals of \iindices (or \timeslots) as $\intvl{i}{j} \triangleq \{i+1,...,j\}$, with the convention that $\intvl{i}{j} \triangleq \emptyset$ for $j \leq i$.
We study executions over a finite horizon of $\Thorizon$ \timeslots (or $\Khorizon$ \iindices), and any interval $\intvl{i}{j}$ with $i < 0$ or $j > \Khorizon$ considered truncated accordingly.
The notation $\intvl{i}{j} \intvlg K$ (resp.\ $\intvlgeq, \intvll, \intvlleq, \intvleq$) is short for $j-i > K$ (resp.\ $\geq, <, \leq, =$).
In the analysis, we denote with upper-case Latin letters several random processes over \iindices (\eg, $\Xat{k}$) or \timeslots (\eg, $\Hat{t}$).
For any set $I$ of \iindices (analogously for \timeslots), we define $\Xat{I} \triangleq \sum_{k \in I} X_k$.

%
We denote by $\kappa$ the security parameter. An event $\Event_{\kappa}$ occurs \emph{with overwhelming probability} (\emph{\wop}) if $\Prob{\Event_{\kappa}} \geq 1 - \negl(\kappa)$.
Here, a function $f(\kappa)$ is \emph{negligible} $\negl(\kappa)$, if for all $n>0$, there exists $\kappa_n^*$ such that for all $\kappa > \kappa_n^*$, $f(\kappa) < \frac{1}{\kappa^n}$.


\section{Security Analysis}
\label{sec:analysis}

\subsection{Unified Model for PoW and PoS}
\label{sec:analysis-probmodel}

We develop a unified probabilistic model for the block production 
%
of both PoW and PoS as per \cref{alg:hdrtree-pow,alg:hdrtree-pos}.
This enables us to prove properties of the block production process and block tree structure that are common to both variants (\cref{sec:analysis-details}).
We then use these properties to prove security of PoW NC (\cref{sec:pow}) and PoS NC (\cref{sec:pos}).
%

Recall that the protocol runs in discrete units of time of duration $\slotduration$ called \timeslots, and that we consider $\slotduration \to 0$ to model PoW.
A \emph{block production opportunity} (\BPO) is a pair $(p,t)$ where according
%
to the PoW/PoS block production lottery,
node $p$ is eligible to produce a block in \timeslot $t$.
A \BPO is called \emph{honest} (resp. \emph{adversarial}) if node $p$ is honest (resp. adversarial).
%
%
The random variables $H_t$ and $A_t$ 
%
denote
the number of honest and adversarial 
%
\BPOs
in \timeslot $t$, respectively.
When the number of nodes $N\to\infty$ and each node holds an equal rate of block production, by the Poisson approximation of a binomial random variable,
we have $\Hat{t}\overset{\text{i.i.d.}}{\sim}\mathrm{Poisson}((1-\beta)\rho)$
and $\Aat{t}\overset{\text{i.i.d.}}{\sim}\mathrm{Poisson}(\beta\rho)$, independent of each other and across \timeslots.
The total number of \BPOs per \timeslot is $\Qat{t} = \Hat{t} + \Aat{t}$.
An \emph{execution} refers to a particular realization of the random process $\{(\Hat{t}, \Aat{t})\}$.

In PoW, as we take $\slotduration \to 0$, the block production process 
%
converges to a \emph{Poisson point process}.
As noted in \cref{sec:modelprotocol-protocolfeatures}, each \BPO corresponds to a different \timeslot, and thus in PoW, blocks in one chain must come from increasing \timeslots. In PoS the latter property is by design (\cref{alg:hdrtree-pos}, \cref{loc:hdrtree-pos-check-lottery}).

%
%

In this unified model, we make the adversary's powers the strongest of both PoW and PoS.
Specifically, we allow the adversary to create multiple blocks from the same \BPO (equivocations) which is only possible in PoS but not in PoW.
%
%
%
However, we assume in the unified analysis that honest nodes use a download rule which downloads at most one block per \BPO.
From a bandwidth perspective, this puts both PoW and PoS on an equal footing.
Then as seen in \cite{dem20, tight_bitcoin}, the additional ability to equivocate does not change the 
block tree properties
and therefore allows us to use similar techniques in our unified analysis.
%
%
The assumption of downloading at most one block per \BPO clearly holds for any download rule in PoW, but we define an \equivocationremoval policy to achieve this in PoS, so that the unified model applies to PoS as well.


%
%
%
%


%
%



%
%
%
%
%
%
%
%


%
%

%
%

%

%


%
%

%

%
%
%


\input{03_analysis_02_highlevel.tex}
\subsection{Definitions}
\label{sec:analysis-definitions}

%
%
%
%
%
%
%
%
%
%
%
%
%
%
%
%
%
%

`\sltGood' \timeslots are
\timeslots with
%
%
%
exactly one honest \BPO and no adversarial \BPOs in that \timeslot,
and
%
%
no \BPOs in $\goodsep$ \timeslots after.
This definition is inspired by convergence opportunities \cite{pss16,sleepy,kiffer2018better}, loners \cite{dem20}, and laggers \cite{ren}.
%
Here, $\goodsep$ is an analysis parameter whose value is chosen such that each honest node can 
receive the block header from the honest \BPO, and
download content for $\goodsepbw$ blocks within $\goodsep+1$ \timeslots, \ie, 
\begin{IEEEeqnarray}{C}
    %
    \label{eq:goodsep-bw-equation}
    (\goodsep+1)\slotduration \triangleq \DeltaHeader + \goodsepbw / \bwtime.
\end{IEEEeqnarray}

%
%
%
%
%

%
%

%


\begin{definition}
    \label{def:slots}
    We call a \timeslot $t$ \emph{\sltgood}, \emph{\sltbad}, \emph{\sltempty},
    %
    %
    %
    respectively,
    denoted as $\predGood{t}$, $\predBad{t}$, $\predEmpty{t}$, respectively, iff:
    \begin{IEEEeqnarray}{rCl}
        \predGood{t} &\;\triangleq\;& (\Hat{t} = 1) \land (\Aat{t} = 0) \nonumber \\ && \quad \land (\Hin{t}{t+\goodsep} + \Ain{t}{t+\goodsep} = 0)
        \IEEEeqnarraynumspace\\
        \predBad{t} &\;\triangleq\;& (\Hat{t} + \Aat{t} > 0) \land \lnot\predGood{t}
        \IEEEeqnarraynumspace\\
        \predEmpty{t} &\;\triangleq\;& (\Hat{t} + \Aat{t} = 0).
        \IEEEeqnarraynumspace
    \end{IEEEeqnarray}
\end{definition}
Note that $\predEmpty{t} = \lnot\predGood{t} \land \lnot\predBad{t}$.
We denote by $t_k$ the $k$-th non-\sltempty \timeslot.
Then, we can introduce random processes over \emph{\iindices},
with \iindex $k$ corresponding
to the $k$-th non-\sltempty \timeslot $t_k$.
The process $\{\Gat{k}\}$ counts good \timeslots,
with $\Gat{k} \triangleq 1$ if $\predGood{t_k}$,
and $\Gat{k} \triangleq 0$ otherwise (\ie, if $\predBad{t_k}$).
Correspondingly, $\{\Bat{k}\}$ counts bad \timeslots,
$\Bat{k} \triangleq 1 - \Gat{k}$.
%

\begin{proposition}
    \label{prop:X_i-is-iid}
    The random variables $\{\Gat{k}\}$ are independent and identically distributed (\iid) with
    \begin{IEEEeqnarray}{C}
        \Prob{\Gat{k} = 1} \triangleq \probGood = \probGoodFormula.
        %
    \end{IEEEeqnarray}
\end{proposition}
The proof is by noting that the inter-arrival times between non-\sltempty \timeslots are \iid and independent of how many and what kind (honest/adversarial) \BPOs occur in that non-\sltempty \timeslot. Details are in \fullVersionRef{\cref{sec:appendix-full-version-proofs-definitions}}.
Throughout the analysis, we will assume that
$\probGood > \frac{1}{2}$ (`honest majority' assumption).

A special role is played by good \timeslots $t_k$
%
as these are candidate \timeslots in which
the block produced
at $t_k$ is `soon' downloaded by all honest nodes.
We count these \timeslots with $\{\Dat{k}\}$,
and all other non-\sltempty \timeslots with $\{\Nat{k}\}$.
Specifically,
$\Dat{k} \triangleq 1$ if $\predGood{t_k}$
and the block produced at $t_k$
has been downloaded by all honest nodes by the end
of \timeslot $t_k + \goodsep$,
$\Dat{k} \triangleq 0$ otherwise,
and $\Nat{k} \triangleq 1 - \Dat{k}$.
We call slots $k$ with $\Dat{k}=1$ as \yups and those with $\Nat{k}=1$ as \ydowns.

Finally, we define two random walks
on \iindices of non-\sltempty \timeslots
with increments
$\{\Xat{k}\}$ and $\{\Yat{k}\}$
that will come in handy for the definition of probabilistic
and combinatorial pivots:
\begin{IEEEeqnarray}{rClCrCl}
    \label{eq:random_walks_X_and_Y}
    \Xat{k} &\triangleq& \Gat{k} - \Bat{k}
    &\qquad\qquad&
    \Yat{k} &\triangleq& \Dat{k} - \Nat{k}
    \IEEEeqnarraynumspace
\end{IEEEeqnarray}
Note that the increments $\{\Xat{k}\}$
are \iid, and not affected by adversarial action,
while the increments $\{\Yat{k}\}$ \emph{do depend}
on the adversarial action and are thus in particular
\emph{not} \iidPERIOD.
Also note that $\forall k\colon Y_k \leq X_k$ since $D_k = 1 \implies G_k = 1$.


\begin{definition}
    \label{def:pp}
    We call an \iindex $k$ a \emph{\sltpp} (short for \emph{probabilistic pivot}),
    denoted as $\predPP{k}$, iff:
    \begin{IEEEeqnarray}{rCl}
        \predPP{k} &\;\triangleq\;&  (\forall \intvl{i}{j} \ni k\colon  \Xin{0}{i} < \Xin{0}{k} \leq \Xin{0}{j})
        \IEEEeqnarraynumspace
    \end{IEEEeqnarray}
\end{definition}
This definition of \sltpps captures
the \emph{probabilistic} aspects of~\cite[Def.~5]{sleepy}
used in~\cite[Sec.~5.6.3]{sleepy}
and casts them as conditions
on a random walk,
inspired by~\cite{dem20, close-latency-security-ling-ren}, to simplify the analysis.
%

%


\begin{definition}
    \label{def:cp}
    We call an \iindex $k$ a \emph{\sltcp} (short for \emph{combinatorial pivot}),
    denoted as $\predCP{k}$, iff:
    \begin{IEEEeqnarray}{rCl}
        \predCP{k} &\;\triangleq\;&  (\forall \intvl{i}{j} \ni k\colon  \Yin{0}{i} < \Yin{0}{k} \leq \Yin{0}{j})
        \IEEEeqnarraynumspace
    \end{IEEEeqnarray}
\end{definition}
This definition of \sltcps captures
the \emph{combinatorial} aspects of~\cite[Def.~5]{sleepy}
used in~\cite[Sec.~5.6.2]{sleepy}
and casts them as conditions
on a random walk,
inspired by~\cite{dem20}, to simplify the analysis.
%
%
Note that a \sltcp is also a \sltpp because $Y_i \leq X_i$.


We denote by $\dC_p(t)$ the longest fully downloaded chain of an honest node $p$ at the end of \timeslot $t$, and let $\len{b}$ denote the height of a block $b$. We use the same notation $\len{\Chain}$ to denote the length of a chain $\Chain$, define $L_p(t)=\len{\dC_p(t)}$ and $L_{\min}(t) = \min_p L_p(t)$.
%


%
%
%
%
%
%
%
%
%
%
%
%
%
%



%

%

%

%

%

%

%



\subsection{Unified Analysis in the Probabilistic Model}
\label{sec:analysis-details}

In this section, we develop all the tools needed to prove the safety and liveness of the PoW and PoS longest chain protocols.

In \secref{analysis-details-cps-stabilize} we show that a block produced in a \timeslot corresponding to a \sltcp stabilizes, \ie, remains in the longest downloaded chain of all honest nodes. 
This is useful because if transactions in a block are confirmed after waiting long enough so at least one \sltcp occurs, 
the prefix of the \sltcp stabilizes and so
those transactions remain in every honest node's ledger (safety).
%
%
The occurrence of \sltcps also guarantees liveness because the block from a \sltcp 
is honest, so it adds new valid transactions to the ledger.
%

Further, we show that \sltpps occur very often (\secref{analysis-details-many-pps}) and the adversary cannot prevent all \sltpps from becoming \sltcps (\secref{analysis-details-many-pps-one-cps}). Thus, at least one \sltcp occurs in a long enough time interval, the length of which can be set as the confirmation time.




%
%
%
%

%
%
%
%
%
%
%
%
%
%
%
%

%
%
%
%
%


\subsubsection{Combinatorial Pivots Stabilize}
\label{sec:analysis-details-cps-stabilize}

In this section, we show that the honest block produced in a \timeslot corresponding to a \sltcp persists in the longest downloaded chain of all honest nodes after $\goodsep$ \timeslots.
Towards this, we first show that if $\Dat{k}=1$, \ie, if all honest nodes download the block produced in the \sltgood \timeslot $t_k$, 
%
then the length of the longest downloaded chain of honest nodes increases (made precise in \cref{prop:chain-growth}).
Due to this, since all intervals around a \sltcp contain more \iindices with $\Dat{k}=1$ than those with $\Dat{k}=0$,
there can never be a chain which is longer than an honest node's longest downloaded chain and does not contain the block corresponding to the \sltcp (\cref{lem:cps-stabilize}).
%
In turn, this means that the block corresponding to the \sltcp remains in all honest nodes' longest downloaded chains forever.
\Cref{lem:cps-stabilize} is proved in \fullVersionRef{\cref{sec:appendix-full-version-proofs-cps-stabilize}} using tools similar to \cite{dem20,sleepy}.

\begin{proposition}
\label{prop:chain-growth}
If $\Dat{k} = 1$, then $L_{\min}(t_k + \goodsep) \geq L_{\min}(t_k-1) + 1$.
\end{proposition}
\begin{proof}
Since $\Dat{k} = 1$, \timeslot $t_k$ is a \sltgood \timeslot.
Let $b$ be the unique honest block produced in \timeslot $t_k$, and let honest node $p$ be its producer.
Since honest nodes produce blocks on their longest downloaded chain, $\len{b} = L_p(t_k-1) + 1 \geq L_{\min}(t_k-1) + 1$.
Further, $\Dat{k} = 1$ means that the block $b$ is downloaded by all honest nodes by the end of \timeslot $t_k + \goodsep$. Therefore, $L_{\min}(t_k + \goodsep) \geq \len{b}$.
\end{proof}


\begin{lemma}
\label{lem:cps-stabilize}
Let $b^*$ be the block produced in a non-\sltempty \timeslot $t_k$ such that $\predCP{k}$. 
Then for all header chains $\Chain'$ that are valid at \timeslot $t \geq t_k + \goodsep$ and $\len{\Chain'} \geq L_{\min}(t)$, $b^* \in \Chain'$.
Further, for all honest nodes $p$ and for all \timeslots $t \geq t_k + \goodsep$, $b^* \in \dC_p(t)$.
\end{lemma}




\subsubsection{Probabilistic Pivots Are Abundant}
\label{sec:analysis-details-many-pps}

%
    %
%
%
%
%
Sufficiently long intervals of \iindices contain
a number of \sltpps proportional
to the interval length.
%
Recall that throughout,
$\probGood = \frac{1}{2} + \epsGood$
with $\epsGood \in (0,1/2]$.
%
\begin{lemma}
    \label{lem:many-pps}
    For
    $\Kcp = \Omega(\kappa^2)$,
    and $\Khorizon = \poly(\kappa)$,
    \begin{IEEEeqnarray}{C}
        \Prob{
            \forall \intvl{i}{j} \intvlgeq \Kcp\colon
            \Pin{i}{j} \geq (1-\delta) \probPP \Kcp}
        \nonumber
        \\
        \qquad\qquad {}\geq{} 1 - \exp(- \Omega(\kappa)) = 1 - \negl(\kappa).
    \end{IEEEeqnarray}
    %
    %
\end{lemma}
The proof is in \secref{appendix-security-proofs-many-pps}.


\subsubsection{Many Probabilistic Pivots Imply One Combinatorial Pivot}
\label{sec:analysis-details-many-pps-one-cps}

%

%
%
%
%
%
%
%
%
%
%
%
%
%
%
%

%
%
%
%
%
%


%
%
%
%

%
%
%
%
%
%
%
%
%
%
%
%
%
    
%
%
%
%
%
%
%
    
%
%
%
%
%
%
%
%
%
%

%
%


%
%
%
%
%
The \rulelc rule $\dlrulelong$
has some useful properties.
Intuitively, nodes using this rule 
\begin{enumerate}[(P1)]
    %
    \item \label{item:good-download-rule-no-repeat} do not download the same block twice,
    \item \label{item:good-download-rule-one-per-bpo} download at most one block from each \BPO (in PoW),
    \item \label{item:good-download-rule-honest-block} either download the most recent honest block, or fully utilize their bandwidth to download other blocks (don't stay idle), and
    \item \label{item:good-download-rule-cutoff} download only blocks that were produced `recently'.
    %
\end{enumerate}
%
%

%
\ref{item:good-download-rule-no-repeat} clearly holds as this rule only downloads content for headers whose content is yet $\BLOCKUNKNOWN$, hence was not downloaded before.
\ref{item:good-download-rule-one-per-bpo} holds in PoW because there is only one block per \BPO. In PoS, the download rule is modified to satisfy this property (\cref{sec:pos}).
\ref{item:good-download-rule-honest-block} holds because the download rule $\dlrulelong$
is never idle, and 
will always download towards an honest block
when it has downloaded all longer chains
%
and there is bandwidth remaining.
%
Moreover, we expect that under a secure execution,
\ref{item:good-download-rule-cutoff} holds because the longest header chain can not fork off too much from the longest downloaded chain of an honest node, otherwise it would cause a safety violation.
%
More precisely, due to \cref{lem:cps-stabilize},
any longest header chain in any honest node's view must extend the block produced in the most recent \sltcp, and therefore blocks with the highest download priority must have been produced after the most recent \sltcp (\cref{prop:download-or-spend-budget}).
%
%


\begin{proposition}
    \label{prop:download-or-spend-budget}
    If $G_k = 1$ and $D_k = 0$, then during \timeslots $[t_k, t_k + \goodsep]$, all honest nodes 
    using the download rule $\dlrulelong$
    download content of at least $\goodsepbw$ blocks that are produced in $\intvl{i}{k}$, where $i<k$ is the largest index such that $\predCP{i}$ (if such an $i$ does not exist, $i=0$).
\end{proposition}


\import{./figures/}{fig-ppivot-cpivot-intuitive.tex}


Given the above properties of the download rule, we now want to show that \sltcps occur often.
\Cref{fig:ppivot-cpivot-intuitive} illustrates the key argument for this.
To start, let us show that there is at least one \sltcp in $\intvl{0}{\Kcp}$.
From \cref{lem:many-pps}, there are many \sltpps in $\intvl{0}{\Kcp}$.
If there were no \sltcps in $\intvl{0}{\Kcp}$, then the adversary must prevent each \sltpp from turning into a \sltcp.
We know that in any interval around a \sltpp, there are more \sltgood \iindices than \sltbad \iindices (see top row in \cref{fig:ppivot-cpivot-intuitive}). In fact, \sltgood \iindices outnumber \sltbad \iindices by a margin that increases linearly with the size of the interval.
Therefore, for a \sltpp to not be a \sltcp, the adversary must prevent an honest node from downloading the most recent honest block in several of these \sltgood \iindices (so that the $\Gat{k}=1$ \iindices have $\Dat{k}=0$).
\Cref{fig:ppivot-cpivot-intuitive} shows an example where the adversary prevented download of the honest block in one \sltgood \iindex, and as a result, two of the \sltpps fail to become a \sltcp.
In the proof of \cref{lem:one-cp-induction-full}, through a combinatorial argument, we show that to prevent all of $n$ \sltpps in $\intvl{0}{\Kcp}$ from becoming \sltcps, the adversary must prevent download of the honest block in at least $n/4$ \sltgood \iindices in $\intvl{0}{2\Kcp}$.
From \cref{prop:download-or-spend-budget}, for each such \iindex, the adversary must `spend' at least $\goodsepbw$ blocks that the honest node downloads.
These blocks must come from a `budget' that can contain at most all blocks mined during $\intvl{0}{2\Kcp}$.
%
%
If this `budget' falls short of the number of blocks required to overthrow all \sltcps, then there must be at least one \sltcp in $\intvl{0}{\Kcp}$.
%


%
Next, we would like to show that there is at least one \sltcp in $\intvl{m\Kcp}{(m+1)\Kcp}$ for all $m \geq 0$ (where 
%
we just saw the base case $m=0$).
%
%
Here,
the adversary might save up many blocks from the past and attempt to make honest nodes download these blocks at a particular target \timeslot $t_k$.
This is where the property of the download rule proven in \cref{prop:download-or-spend-budget} becomes useful.
%
%
%
%
Given that one \sltcp occurred in $\intvl{(m-1)\Kcp}{m\Kcp}$, \cref{prop:download-or-spend-budget} ensures that honest nodes will only download blocks that are produced after $(m-1)\Kcp$.
This allows us to bound the `budget' of blocks that the adversary can use to overthrow \sltcps, and therefore show that there is at least one \sltcp in $\intvl{m\Kcp}{(m+1)\Kcp}$.
The above arguments are formalized in \cref{lem:one-cp-induction-full}.





%

%
%
%
%
%
%
%
%
%
%
%

%
%
%
%
%

%
%
%
%
%
%
%
%
%
%
%
%
%
%
%
%
%
%
%
%
%

%
%
%
%
%
%
%
%
%
%
%
%
%
%
%
%
%
%
%
%

%
%
%
%
%
%
%
%
%
%
%

%
%
%
%
%
%
%
%
%
%
%
%
%
%
%
%
%
%
%
%
%
%
%
%
%
%
%
%
%
%
%
%

%
%
%
%
%
%
%

%
%




%


\begin{lemma}
\label{lem:one-cp-induction-full}
\label{lem:many-pps-one-cps}
%
If all honest nodes use the download rule $\dlrulelong$,
%
and if
%
%
%
%
%
%
%
\begin{IEEEeqnarray}{C}
    \label{eq:cp-induction-full-margin-condition}
    \forall \intvl{i}{j} \intvlgeq \Kcp
    %
    \colon 
    %
    \frac{\goodsepbw}{2} \left( \Gin{i}{j} - \Bin{i}{j} \right) > \Qin{i-2\Kcp}{j}, \IEEEeqnarraynumspace \\
    \label{eq:cp-induction-full-ppivots-condition}
    \forall m \geq 0 \colon 
    \frac{\goodsepbw}{4} \Pin{m\Kcp}{(m+1)\Kcp} > \Qin{(m-2)\Kcp}{(m+2)\Kcp}, \IEEEeqnarraynumspace
\end{IEEEeqnarray}
then 
$\forall m \geq 0 \colon$ 
$\exists k_m^* \in \intvl{m\Kcp}{(m+1)\Kcp} \colon \predCP{k_m^*}$.
\end{lemma}


Here, $\Qin{.}{.}$ is the adversary's block budget, and the expressions on the left in \eqref{cp-induction-full-margin-condition,cp-induction-full-ppivots-condition} are the minimum number of blocks the adversary needs to produce to ensure that there are no \sltcps, in terms of the number of \sltpps $\Pin{.}{.}$ and number of \sltgood \iindices $\Gin{.}{.}$.
\cref{lem:many-pps-one-cps} is proven inductively
%
using \cref{prop:download-or-spend-budget}. The proofs of \cref{prop:download-or-spend-budget} and \cref{lem:one-cp-induction-full} are in \cref{sec:appendix-security-proofs-many-pps-one-cp}.



While all the analysis below is done for the download rule $\dlrulelong$, the proofs only use the properties \ref{item:good-download-rule-no-repeat}, \ref{item:good-download-rule-one-per-bpo}, \ref{item:good-download-rule-honest-block}, \ref{item:good-download-rule-cutoff}
and thus apply to several other simple download rules.
%
%
A few examples are
i) ``download towards the freshest block'' \cite{bwlimitedposlc},
ii) ``download only blocks that are consistent with the node's confirmed chain'', or iii) ``at \timeslot $t_k$, only download blocks produced in \timeslots $\intvl{t_k-\Tdl}{t_k}$'' for some $\Tdl$.
In fact, iii) gives an alternative definition of the property \ref{item:good-download-rule-cutoff} instead of the one in \cref{prop:download-or-spend-budget}.
In this work, we did not adopt i) because `freshness' cannot be determined in PoW, and ii) and iii) because they would fail to recover from a network split (as demonstrated in the \greedyattack briefly mentioned in \cref{sec:experiments}).
In \cref{sec:pos}, we modify the `download longest header chain rule' to remove equivocations in PoS.
We show that this rule satisfies the above properties, and hence the analysis of this section carries over in PoS as well.








%

%
%
%
%

%

%

%

%

%

%

%

%


%

%
%

%
%


%



%




\section{Proof-of-Work}
\label{sec:pow}


For PoW, we use the simple download rule `download the longest header chain'.
%
%
%
%
%
%
%
In \cref{lem:one-cp-induction-full}, we showed that under this download rule, \sltcps occur in every $\Kcp$-interval. 
We will use this to prove safety and liveness 
%
and identify the protocol parameters for which this holds with overwhelming probability in \cref{thm:safety-and-liveness-pow}.




%
%
%
%
%
As noted in \cref{sec:modelprotocol-protocolfeatures},
it is most appropriate for 
PoW to set $\slotduration \to 0$, and
%
to state its security properties in terms of real time.
%
In order to use the results from \cref{sec:analysis}, we must bridge between \iindices and real time.
This is easy to do as the number of \iindices or non-\sltempty \timeslots is proportional to the time interval.
In fact, as $\slotduration \to 0$, the block production process converges to a Poisson point process with rate $\blkratetime \triangleq \blkrateslot/\slotduration$.
Moreover, each non-\sltempty \timeslot has exactly one \BPO (arrivals of a Poisson point process do not coincide).

Proof details in \cref{sec:appendix-pow-proofs}. Result with $\DeltaHeader \approx 0$ (reasonable approximation for large block sizes) plotted in \cref{fig:comparison-bddelay-bdbandwidth}.

%


%
\begin{theorem}
\label{thm:safety-and-liveness-pow}
%
%
%
%
%
For all $\beta < 1/2$,
%
$\blkratetime > 0$,
such that
%
%
%
%
%
%
%
\begin{IEEEeqnarray}{C}
\label{eq:pow-max-tp}
    \blkratetime < \max_{\goodsepbw} \frac{1}{\DeltaHeader + \goodsepbw/\bwtime} \ln\left( \frac{2(1-\beta)\goodsepbw}{\goodsepbw+4 + \sqrt{8\goodsepbw+16}} \right),
\end{IEEEeqnarray}
the PoW Nakamoto consensus protocol $\protocol$ 
with 
the download rule $\dlrulelong$,
$\slotduration \to 0$,
%
%
$\blkrateslot = \blkratetime \slotduration$,
%
%
and
%
$\confDepth = \Theta(\kappa^2)$
%
is secure
with 
liveness latency
$\TliveReal \triangleq \Tlive \slotduration = \Theta(\kappa^2)$ 
over a time horizon of 
$\Khorizon = \poly(\kappa)$ block productions.
%
\end{theorem}

%



%
%
%
%
%
%

%
%
%
%
%
%
%

%

%

%




%
%
%
%
%

%
%
%
%
%
%
%
%
%
%
%
%
%
%
%
\section{Sanitizing-Proof-of-Stake (\sapos)}
\label{sec:pos}

\subsection{\EquivocationRemoval}
\label{sec:pos-equivocations}

%

%
%

%




%
%
%
%
%
For PoS,
due to spamming by equivocations,
we need a policy to ensure 
that nodes download at most one block from each \BPO.
%
We therefore propose the Sanitizing-Proof-of-Stake (\sapos) protocol, in which the contents of provably equivocating blocks are sanitized from the blockchain.
Pseudocodes \cref{alg:posequivblank-lc} and \cref{alg:posequivblank-hdrtree} are in \cref{sec:appendix-pos-eqremoval-pseudocodes}.

\paragraph{The Download Rule in \sapos}
%
On top of any existing download rule (such as \rulelc), we add another rule that an honest node does not download content for a header $\Chain$ if it has seen 
another equivocating header
%
from the same \BPO (same producing node and \timeslot) as $\Chain$.
Instead of downloading content for such a header, the node considers that content to be ``downloaded'' and sets it to be empty
%
(\myalgref{alg:posequivblank-lc}{loc:posequivblank-lc-downloadrule}).
This means that the node can continue to download content for headers that extend $\Chain$, and these blocks will be candidates for the node's longest downloaded chain $\dC$.
%

\paragraph{Equivocation Proofs}
With only the above download rule,
one honest node may download content for a header while another 
%
may not (depending on when each node saw an equivocating header).
In order to output a consistent ledger that all honest nodes have downloaded, 
%
reaching consensus on just the header chain is not enough.
%
For nodes to later \emph{catch up} to the confirmed header chain's contents, the content must be available in the network. Unfortunately, verifying \emph{data availability} \cite{DBLP:conf/fc/Al-BassamSBK21} comes with several challenges.

Instead, we ensure that honest nodes agree on which blocks had an equivocation, and unilaterally blank their contents.
For this, when an honest node produces a new block header,
it adds an `equivocation proof' against any equivocating blocks among the recent blocks in its downloaded longest chain.
Specifically, the node picks from among the last $\keqproof$ block headers in its longest downloaded chain $\dC$, block headers $\Chain'$ for which the node has seen an equivocating block header $\Chain'$,
%
%
and there is no equivocation proof against it in any block header in $\dC$.
The node then creates an equivocation proof which consists of the two block headers $\Chain$ and $\Chain'$ and adds the equivocation proof to the header of the block that it creates (\myalgref{alg:posequivblank-lc}{loc:posequivblank-lc-construct-eq-proof}).

The deadline $\keqproof$ for adding equivocation proofs exists so that the adversary cannot release an equivocation after its block has been confirmed, and force honest nodes to then blank the content for that block, thereby altering the ledger.
The deadline also keeps the size of equivocation proofs in a header limited.
We also don't want an equivocation proof to be repeated in several headers in a chain.
%
Therefore, a block header $\Chain$ is considered invalid if it contains an equivocation proof against a block not in the prefix of $\Chain$, a block more than $\keqproof$ blocks above $\Chain$, or contains an equivocation proof that has already been proven in the prefix of $\Chain$ (\myalgref{alg:posequivblank-hdrtree}{loc:posequivblank-hdrtree-valid-eq-proof}).


\paragraph{Ledger Construction in \sapos}
%
To create the ledger at the end of \timeslot $t$, an honest node takes all blocks on its longest header chain that 
%
are $\confDepth$-deep,
then blanks the contents of any block 
%
against which there is an equivocation proof in
a block header following it
(\myalgref{alg:posequivblank-lc}{loc:posequivblank-lc-confirmation-blanking}).



%
%
    %
    
    %
    
    %
    %
    %
    
    %
    
    %
    %
    %
    
    %
    
    %
    %
    
    %
    
    %
    
    %
    
%


%
%


\subsection{Security Theorem}
\label{sec:pos-result}

Recall that the analysis in \cref{sec:analysis-details-many-pps-one-cps} uses four properties of the download rule. It is easy to see that with the addition of \equivocationremoval, the `download longest header chain' rule satisfies these properties in PoS.
The \equivocationremoval rule in \sapos clearly satisfies the property that each honest node never downloads the same block twice \ref{item:good-download-rule-no-repeat}, and downloads at most one block from each \BPO \ref{item:good-download-rule-one-per-bpo}.
The rule will never prohibit download of an honest block because it has no equivocations, and blocks in its prefix will either be downloaded or blanked.
Moreover, the rule never remains idle as long as there are block headers remaining with $\BLOCKUNKNOWN$ content \ref{item:good-download-rule-honest-block}.
Finally, \sapos does not spend bandwidth on any more blocks than the base download rule does, and since the base download rule $\dlrulelong$ does not download blocks before the most recent \sltcp (\cref{prop:download-or-spend-budget}), the rule with \equivocationremoval also does not \ref{item:good-download-rule-cutoff}.
This means that the analysis of \cref{sec:analysis-details-many-pps-one-cps} works for \sapos.
%
Just like in PoW, this 
%
leads to liveness and consistency of the confirmed header chains of all honest nodes.
Therefore, to ensure consistency of the ledger, we only need to show that the ledger construction process 
%
in \sapos
retains consistency. That is, if one honest node blanks the content of a block in its ledger, then all honest nodes do. Conversely, if one honest node does not blank the content for a block in its ledger, no honest node does. Proof details are in \cref{sec:appendix-pos-proofs}.

%
%
%
    
%
    
%
    
%
%


\begin{theorem}
\label{thm:safety-and-liveness-pos}
%
%
%
%
%
For all $\beta < 1/2$,
$\goodsepbw \in \IN$,
and $\blkrateslot, \slotduration$
satisfying
\begin{IEEEeqnarray}{C}
    %
    %
    \frac{\goodsepbw}{16} \frac{(2\probGood-1)^2}{\probGood} > \frac{\blkrateslot}{1-e^{-\blkrateslot}},
    \probGood = (1-\beta)e^{-\frac{\blkrateslot}{\slotduration} \left(\DeltaHeader + \goodsepbw/\bwtime\right)}, \IEEEeqnarraynumspace
\end{IEEEeqnarray}
there exists $\keqproof, \confDepth = \Theta(\kappa^2)$
such that the \sapos protocol
$\protocolPoSEquivBlank$ 
with
the download rule $\dlrulelong$,
%
%
%
%
%
is secure
with liveness latency $\TliveReal = \Theta(\kappa^2)$ \timeslots
over a time horizon of $\Khorizon = \poly(\kappa)$ block productions.
%
\end{theorem}
By choosing $\blkrateslot, \slotduration \to 0$ such that $\blkrateslot/\slotduration = \blkratetime$ (small slot approximation), we get the same security region as PoW, which is shown in \cref{fig:comparison-bddelay-bdbandwidth}. However, in PoS, we do have the additional freedom to choose a larger $\slotduration$, offering a potentially larger set of secure parameters.
The exact confirmation depth and liveness latency are larger for \sapos than for PoW NC (details in \cref{sec:appendix-pos-proofs} and \fullVersionRef{\cref{sec:appendix-full-version-proofs-pos}}).


%
%
%
%
%
%
%
%
%
%
%
%

%

%


%
\section{Predictable Validity}
\label{sec:throughputloss}


%
%

\subsection{Predictable Transaction Validity}
\label{sec:throughputloss:predictranvad}

\begin{definition}
    A transaction has \emph{predictable validity} if it is valid\footnote{In UTXO-based systems (e.g., Bitcoin), \textit{valid} means the inputs of the transaction have not been spent. In account-based systems (e.g., Ethereum), \textit{valid} means the transaction execution succeeds and fees are paid.} both at the time an honest node adds it to a block and when that block is executed.
\end{definition}

The sanitization in \sapos leads to a loss of \emph{predictable transaction validity}. An honest block $B$ may include a transaction that depends on the contents of a previous block $A$ whose equivocations were not known at the time. 
After block $B$ is produced, the adversary could release an equivocation for the block $A$, forcing honest nodes to sanitize block $A$'s contents, which may invalidate the transaction in block $B$. Such invalidated transactions take up free space in honest blocks and lower the effective throughput (valid confirmed transactions) of the ledger.

Traditional NC protocols, require a node to download and validate blocks before building on them, satisfying predictable validity. On the other hand, protocols which lack state determinism, such as DAG-based \cite{spiegelman2022bullshark,danezis2022narwhal} and LazyLedger \cite{al2019lazyledger} blockchain protocols, choose to forego predictable validity and accept that not all transactions will ultimately be executed.

We propose a simple solution to recover predictable validity for \sapos:
If nodes limit transactions included in a block to those that don't depend on any \emph{recent} state, then they can be sure that all equivocations that could affect the validity state of a transaction have a corresponding equivocation proof included in the chain. This is because at the time of creating a block, honest nodes \emph{have seen all transactions which will be executed}, however, \emph{not all transactions nodes have seen will be executed}. The following lemma follows naturally. See \fullVersionRef{\cref{sec:proof-details}} for proof details.

\begin{lemma}
    \label{lem:pred-valid-1}
    If a node produces a block whose transactions do not share state with any transaction included in the last $\keqproof$ blocks, then the block satisfies predictable transaction validity.
\end{lemma}


\subsection{Predictable Fee Validity}
\label{sec:throughputloss:predicfeevad}

In practice, for instance in popular Defi-ecosystems which consist of very interdependent transactions (e.g., transactions interacting with major token exchanges and other prominent smart contracts) \cite{guo2019graph,chen2020understanding}, it may not always be practical to limit the interaction between transactions. As an alternative to predictable transaction validity, we would like to preserve the minimum requirement that each transaction pays its fee, regardless of the outcome of its execution. This guarantees that miners are compensated for space used in their blocks, and also makes it costly for the adversary to take up space with invalid transactions.

\begin{definition}
    A transaction has predictable \emph{fee validity} if its fee can be paid both when an honest node adds it to a block and when that block is executed.
\end{definition}

In systems like Ethereum, transactions have a \emph{max gas} value set by the sender, which limits the computation allowed by the transaction and ultimately its fee. We consider a protocol with this gas mechanism, as well as a base transaction cost that covers the block space taken up by the transaction. 
We introduce a notion of \emph{gas deposit accounts} to \sapos that can only be used for transaction fees (transactions internally do not have access to these accounts).
When a miner includes a transaction, it checks that the account funding the transaction has enough funds to cover the maximum gas, even if all transactions in its recent ancestor blocks make it to the sanitized ledger and consume their maximum gas. Users thus need to maintain a balance proportional to the complexity and frequency of the transactions they make. Thus, users who primarily make simple transactions (direct transfers having low max gas) or transact infrequently (few transactions in recent ancestor blocks) need to maintain smaller balances than those who are spending more on fees. We also require that any deposit to the account is not considered in the balance until $\keqproof$ blocks after the deposit transaction. Withdraws however can take place immediately, as direct transactions.

\begin{lemma}
    \label{lem:pred-valid-2}
    If a node produces a block whose transactions are funded by gas deposit accounts with sufficient balance (balance before $\keqproof$ blocks minus any fees since),
    then all transactions in the block satisfy predictable fee validity.
\end{lemma}
See \fullVersionRef{\cref{sec:proof-details}} for proof details.
Thus, by sanitizing the contents of equivocating blocks and using our gas deposit scheme, we ensure that nodes download a maximum of one block per \timeslot and that honest block creators only include transactions that pay for their spot in the block.

The solutions in \cref{sec:throughputloss:predictranvad} and \cref{sec:throughputloss:predicfeevad} are complementary and could each be adopted as per-miner heuristics (i.e., not a consensus rule), or by the system based on the use-case (e.g., expected inter-dependency of transactions). Note that there are user-side complexities both schemes do not directly address. Since transactions can be sanitized, we can no longer rely on transaction nonce schemes that are strictly incremental but instead must relax them to strictly increasing. In lieu of stronger validity guarantees, it is the onus of the user to make sure their transactions behave correctly in the event some get sanitized. Sanitizing block content also opens up the potential for the adversary to perform free options (for a limited amount of time) by including transactions in a block that they can later decide to cancel (by revealing an equivocation at no cost within the allowed window).



\section{Conclusion}
\label{sec:conclusion}
%
In this work we focused on the security of the longest chain protocol both in the PoW and PoS settings. While block downloading and processing is usually implemented in an ad-hoc manner and is not typically discussed in the context of the protocol's security analysis, our work highlights the importance of correctly prioritizing block download and processing. 
In addition to providing a security proof using new techniques, and attacks on natural prioritization rules in the PoW setting, we also propose \sapos, a new proof-of-stake variant. Several important open questions remain:
\begin{itemize}
    \item There remain gaps between security bounds we provide in the PoW setting and the known attacks in this case (\cf \figref{comparison-bddelay-bdbandwidth}(b)). Can better attacks be found? What are the optimal prioritization rules for which security is achieved?
    \item In the PoW setting, the attacker is unable to equivocate, but in \sapos we were forced to deal with equivocations. This came at a cost to the latency of transaction execution, and with decreased certainty about the state at which the transaction is eventually executed. Can these costs be avoided, so that PoS based LC is on par with the PoW variant?
    \item The difficulty adjustment algorithm (DAA) seems to apply even more stress to limited capacity nodes. Can DAAs be designed for this setting and incorporated into the security analysis?
    \item Can processing and download parallelization, pre-processing and pre-fetching of blocks be utilized more efficiently in order to securely improve the throughput of LC based protocols?
    \item In \sapos, the 
    %
    deadline for including equivocation proofs
    is not user-dependent, but baked into the protocol.
    %
    A user cannot increase this to achieve lower error probability.
    This is a drawback compared to traditional Nakamoto consensus. Can it be avoided?
\end{itemize}



%
%
%
%


%


%






%
\begin{acks}
    We thank
    %
    Lei Yang, Mohammad Alizadeh,
    Ertem Nusret Tas,
    Ghassan Karame,
    Florian Tschorsch,
    and
    George Danezis
    for fruitful discussions.
    The work of LK, JN, and AZ
    was conducted in part
    during Dagstuhl Seminar \#22421.
    JN is supported by the Protocol Labs PhD Fellowship.
    JN and SS are supported by a gift from
    the Ethereum Foundation.
    LK is supported by the
    armasuisse Science and Technology CYD Distinguished Postdoctoral Fellowship.
    AZ is supported by grant \#1443/21 from the Israel Science Foundation.
\end{acks}


%
%
%

%
\printbibliography%


\appendix

\section{Protocol Algorithms Reference}
\label{sec:algos-reference}

\subsection{Helper Functions for Pseudocode}
\label{sec:algos-reference-helperfunctions}

\begin{itemize}
      \item $\operatorname{Hash}(\txs)$:\;\;
%
            %
            Cryptographic hash function to produce
            a binding commitment to $\txs$
            (modelled as a random oracle)

      \item $\Chain' \preceq \Chain$, $\Chain \succeq \Chain'$:\;\;
%
            %
            Relation describing that $\Chain'$ is a prefix of $\Chain$

      \item $\Chain \| \Chain'$:\;\;
%
            %
            Concatenation of $\Chain$ and $\Chain'$

      \item $\len{\Chain}$:\;\;
%
            %
            Length of $\Chain$

      \item $(\TRUE \text{ with probability $x$, else } \FALSE)$:\;\;
%
            %
            Bernoulli random variable with success probability $x$

      \item $\operatorname{prefixChainsOf}(\Chain)$:\;\;
%
            %
            Set of prefixes of $\Chain$, \ie, all $\Chain'$ with $\Chain' \preceq \Chain$

      \item $\operatorname{newBlock}(\mathsf{txsHash}\colon \operatorname{Hash}(\txs))$ and
      \\
      $\operatorname{newBlock}(\mathsf{time}\colon t, \mathsf{node}\colon P, \mathsf{txsHash}\colon \operatorname{Hash}(\txs))$:\;\;
%
            %
            Produce a new PoW and PoS block header with
            given parameters, respectively

      \item $\operatorname{txsLedger}(\TxsMap, \Chain)$:\;\;
%
            %
            Concatenates the block contents stored in $\TxsMap$ for the blocks along the chain $\Chain$, to obtain the corresponding transaction ledger
        
        \item $(\Chain \eqEquivocation \Chain') \triangleq (\Chain \neq \Chain') \land (\Chain.\mathsf{node} = \Chain'.\mathsf{node}) \land (\Chain.\mathsf{time} = \Chain'.\mathsf{time})$:\;\;
        Relation for distinct headers from the same \BPO
\end{itemize}

%


\subsection{Environment $\Env$}
\label{sec:algos-reference-environment}

%
%
The environment $\Env$ initializes $N$ nodes and lets $\Adv$ corrupt up to $\beta N$ nodes at the beginning of the execution. Corrupted nodes are controlled by the adversary. Honest nodes run $\protocol$.
The environment maintains a mapping $\Env.\TxsMap$ from block headers to the block content (transactions). This mapping is referred to as the `cloud' in \secref{modelprotocol} and \figref{model}.
%
$\Env$ also maintains for each node a queue of pending block headers
to be delivered after a delay determined by the adversary.
If $\Adv$ has not instructed $\Env$ to deliver a header $\DeltaHeader$ real time after it was added to the queue of pending block headers,
then $\Env$ delivers it to the node.
%

Honest nodes and $\Adv$ interact with $\Env$ via the following functions:
\begin{itemize}
      \item $\Env.\Call{broadcastHeaderChain}{\Chain}$:

            \noindent If called by an honest node, $\Env$
            enqueues $\Chain$ in the queue of pending block headers for each node, and notifies $\Adv$.
            %
            Then, for each node $P$, on receiving $\Call{deliver}{\Chain,P}$ from $\Adv$,
            or when $\DeltaHeader$ time has passed since $\Chain$
            was added to the queue of pending headers, $\Env$ triggers $P.\Call{receivedHeaderChain}{\Chain}$.

      \item $\Env.\Call{uploadContent}{\Chain, \txs}$:

            \noindent $\Env$ stores a mapping from the header chain $\Chain$ to the content $\txs$ of its last block by setting $\Env.\TxsMap[\Chain] = \txs$.
            $\Env$ only stores the content $\txs$
            if $\mathrm{Hash}(\txs) = \Chain.\mathsf{txsHash}$.

      \item $\Env.\Call{receivePendingTxs}{\null}$:

            \noindent $\Env$ generates a set of pending
            transactions and returns them.
    
    \item
        If node $P$ at \timeslot $t$ requests the content associated with a block header $\Chain$, $\Env$ acts as follows.
        %
        %
If $\Env.\TxsMap[\Chain]$ is set, then let $\txs = \Env.\TxsMap[\Chain]$ (if not, $\Env$ ignores the request).
%
If the request was received from an honest node $P$,
if $\Env$ has recently triggered $P.\Call{receivedContent}{.}$ at a rate below $\bwtime$,
%
then $\Env$ triggers  $P.\Call{receivedContent}{\Chain, \txs}$ (else, $\Env$ ignores the request).
If the request was received from $\Adv$, $\Env$ sends $(\Chain, \txs)$ to $\Adv$.
\end{itemize}

At all times, $\Adv$ can trigger $P.\Call{receivedHeaderChain}{\Chain}$
and $P.\Call{receivedContent}{\Chain, \txs}$
for honest nodes $P$ (bypassing header delay and bandwidth constraint in an adversarially chosen way).

%
%
%
%
%
%


\section{Simulation Details}
\label{sec:attacks-details}

To complement the theoretical analysis, we conducted simulations of a PoW blockchain with bandwidth constraints. We evaluated several download rules with and without the presence of attackers. The simulations were written as event-driven simulations using Python's simpy package.\footnote{Source code: \gitSourceUrl}

Nodes in our simulation generate blocks in a Poisson process with rate proportional to their mining power. We assume the mining difficulty is fixed, and do not include any adjustment by a difficulty adjustment algorithm (DAA). In fact, DAAs tend to worsen processing problems as they increase the block creation rate if the chain does not grow fast enough---which in turn requires more download from nodes. 

Nodes process blocks one at a time according to the priority dictated by the processing policy, at a rate determined by their capacity. They are allowed to preempt their current task if new information (headers that are published, blocks that they mined) presents them with a higher priority targets. Since queues can grow large if nodes do not manage to process all blocks in a timely manner, we maintain priority queues of bounded size (typically 100) and evict low priority tasks from the queue as needed. If nodes do keep up, queues remain small, and all is well. If however queues grow large, it is usually safe to discard low priority tasks, since higher priority alternatives are arriving at a fast pace, advertised by peers that continue to mine. The high rate of incoming header announcements implies the node will never manage to process all low priority blocks unless their priority changes (in which case they will be re-advertised).

As preemption of downloads may cause nodes to alternate between downloads, we run the risk of wasting work if we discard partially processed information. We therefore allow nodes to retain partial work in an LRU cache of size 10. Cached entries allow nodes to resume processing where they left off. (We note that in practice, it may be difficult to cache information, and that in realistic settings such caching mechanisms may be targeted by an adversary that will flood nodes with incorrect information that they cannot validate prior to completing the processing of the entire block.)

Except where we note otherwise, headers are assumed to propagate instantly in the simulations. Block headers in the PoW settings contain the proof-of-work itself, which can be easily validated. We therefore assume the adversary never publishes headers it did not actually mine.
To remain close to the theoretical analysis, we model all processing tasks as dependent only on the resources available to the node itself. In reality, things are much more complex: nodes typically propagate blocks in a P2P network, which means both the overlay network topology and the underlying internet topology both greatly impact block download rates and performance. Block processing in turn, behaves differently and does not depend on the topology. With bandwidth nodes need to decide on ways to balance incoming and outgoing bandwidth between their peers, and attackers may try to isolate nodes via eclipse attacks~\cite{eclipse,eclipse2,eclipse3,eclipse4,eclipse1}. Our simplified setting allows us to focus more on the priority rules in isolation from the effects of topology and other P2P related issues that are bandwidth-specific.

\paragraph{An Example Run}
%
\Figref{experiment-trace} is an example of a trace generated by our simulation for a simple setting with only 5 nodes. The x-axis is time, and each node's timeline is represented horizontally at a different height along the y-axis. Blocks that are created are shown as squares, placed at the time of their creation, and arrows point to their parent blocks. Each block is named $h.j$ to denote that it is the $j$'th block of height $h$ to be created. 

The timeline of each node also depicts the blocks it is processing at any particular time. For example, block $1.1$ is created within the first second of the simulation by Node $3$ and other nodes begin to process it immediately. This work concludes before the next block is mined. However, that is not always the case. Block $3.2$ for example, is mined by Node $1$ at around time $4$, but a previous block at this height (block $3.1$) was mined earlier. Node $1$ had not finished validating it, and therefore did not mine on top of it. 

Finally, it is possible to see processing tasks that are preempted and resumed later. For example, Node $0$ is in the process of validating block $3.2$ when block $4.1$ is advertized. It stops its current download since $4.1$ represents a longer chain. Node $0$ resumes the download of block $3.2$ one second later. 

Each point in our simulation result graphs is typically computed from multiple repetitions of the same experiment. 
We normalize time so that a block is created by the honest nodes once every time unit in expectation. We thus consider the basic time unit as $1$ second, and the total block creation rate as $\blkratetimeHon=1$.
Each time we start the chain at the genesis block and run for a period of $5,000$ to $10,000$ seconds (depending on the experiment). Bandwidth is measured in units of blocks per second.
The standard deviation of values plotted is typically well below 1\% of the values themselves. Error bars are thus too small to properly appear in the plot, and were not added.



\subsection{Chain Growth Rate at Capacity}
\label{sec:experiment-growth}

The rate at which the chain grows without the presence of an attacker sets a bound on the security of the system: if the chain grows at a rate $\blkratetimeGrowth$, then an attacker mining at that rate or above is able to overtake the blockchain at will. 

As a baseline comparison for other simulations, we consider a a network of 100 honest nodes without the presence of an attacker, and measure the rate of growth of the chain $\blkratetimeGrowth$ in two scenarios:
(a) Non-zero header delay $\DeltaHeader>0$, and infinite processing speed $\bwtime = \infty$.
(b) No header delay $\DeltaHeader=0$, and a constant processing rate $\bwtime < \infty$.
%
The first case is equivalent to the bounded delay model, and blocks arrive at all nodes exactly $\DeltaHeader$ seconds after they are created. The second scenario is in our model and includes queueing delays of blocks.
%
To properly compare between the two models, we note that $\DeltaHeader^{-1}$ can be considered as an effective rate at which a single block is propagated. 

\import{./figures/}{fig-experiment-growth.tex}

\Figref{experiment-growth} depicts the results of the simulations. It shows that indeed as bandwidth decreases (or header propagation time increases) the chain grows at a slower pace. 
It is perhaps surprising to see that some growth of the chain occurs even when download rates are below the rate needed to download all blocks produced (\eg, a processing rate of $1/2$ allows nodes to download at most $1/2$ of the blocks that are created). The reason progress still takes place at extremely low rates is that nodes that are behind create blocks that others do not need to process, and hence their blocks, which do not contribute to the height of the chain, at least do not waste resources. 

\Figref{experiment-growth} also shows that the limited capacity case is slightly worse than the bounded delay setting for comparable delays. 
We note that in our simulation, even in the limitted capacity case, if blocks of the same height are created, the first one is advertised to all nodes instantly and is downloaded first. This results in this block being downloaded in a coordinated manner by all nodes, and thus most likely extended. This is the same as in the bounded delay setting. Queuing delays (\ie, delaying a block while we are downloading its parent) occur only on rare occasions---when a miner has single-handedly managed to mine several consecutive blocks (a rare occurence in our highly decentralized setting).





%

%

%

%



%

%



\section{Security Analysis Proofs}
\label{sec:appendix-security-proofs}

Refer to \tabref{notation} for a recap of notation and definitions.


\begin{table}[tb]
    \caption{Summary of notation (\cf~\secref{modelprotocol-notation,analysis-definitions})}
    \label{tab:notation}
    \vspace{-0.5em}
    \begin{tabular}{cl}
        \toprule
        \multicolumn{2}{c}{\textbf{Protocol parameters}} \\
        \midrule
        $\slotduration$ & \Timeslot duration (seconds) \\
        $\blkrateslot$ & Avg. no. of \BPOs per \timeslot \\
        %
        $\confDepth$ & Confirmation depth \\
        \bottomrule
        \toprule
        \multicolumn{2}{c}{\textbf{Model parameters}} \\
        \midrule
        $\beta$ & Fraction of adversarial nodes \\
        $\DeltaHeader$ & Header propagation delay (seconds) \\
        $\bwtime$ & Bandwidth (blocks/second) \\
        \bottomrule
        \toprule
        \multicolumn{2}{c}{\textbf{Analysis variables}} \\
        \midrule
        $\goodsep$ & No. of \sltempty \timeslots after a \sltgood \timeslot \\
        $\goodsepbw$ & No. of blocks downloaded in $\goodsep$ \timeslots \\
        $t_k$ & $k$-th non-\sltempty \timeslot \\
        $\Gat{k}$ & $1$ iff \timeslot $t_k$ is \sltgood \\
        %
        $\Dat{k}$ & $1$ iff $\Gat{k}=1$ and block in $t_k$ downloaded \\
        %
        $\Pat{k}$ & $1$ iff \iindex $k$ is a \sltpp \\
        \bottomrule
    \end{tabular}
\end{table}



\begin{proof}[Proof of \Propref{X_i-is-iid}]
    First, for any $k$,
    \begin{IEEEeqnarray}{rCl}
        \Prob{\Gat{k} = 1} &=& \Prob{ \predGood{t_k} \mid \lnot \predEmpty{t_k} } \\
        &=& \frac{\Prob{\predGood{t_k}}}{\Prob{\predEmpty{t_k}}}
        %
        %
        = \frac{(1-\beta)\blkrateslot e^{-\rho(\goodsep+1)}}{1-e^{-\blkrateslot}}.
    \end{IEEEeqnarray}
    Take an \iid random process $\{T_k\}$ with $\Prob{T_k = t} = (1-\probEmpty)\probEmpty^t$ for $t \geq 0$ where $\probEmpty = \Prob{\Hat{t}+\Aat{t}=0}$.
    The random variables $\{T_k\}$ describe the inter-arrival times between non-empty slots.
    Take another \iid random process $\{\Gat{k}'\}$, independent of $\{T_k\}$, such that $\Gat{k}' = 1$ with probability $\Prob{\Hat{t} = 1 \land \Aat{t} = 0 \mid \Hat{t}+\Aat{t}>0}$ and $\Gat{k}' = 0$ otherwise.
    The random process $\{\Gat{k}\}$ can be equivalently defined as $G_k = 1$ iff $G_k' = 1$ and $T_k \geq \goodsep$.
    
    The independence of the random variables $\{\Gat{k}\}$ then follows from the independence of the random variables $\{(T_k, \Gat{k}')\}$.
\end{proof}


\subsection{Combinatorial Pivots Stabilize}
\label{sec:appendix-security-proofs-cps-stabilize}


\begin{proposition}
\label{prop:chain-growth-interval}
For any $i < j$,
\begin{IEEEeqnarray}{C}
    L_{\min}(t_j + \goodsep) \geq L_{\min}(t_{i+1} - 1) + \Din{i}{j}.
\end{IEEEeqnarray}
\end{proposition}
\begin{proof}
For each $k \in \{i+1, ..., j\}$,
if $\Dat{k} = 1$,
\begin{IEEEeqnarray}{rCl}
    L_{\min}(t_{k+1}-1) &\geq& L_{\min}(t_k + \goodsep) \quad (\Dat{k} = 1 \implies t_{k+1} > t_k + \goodsep) \IEEEeqnarraynumspace \\
    &\geq& L_{\min}(t_k - 1) + 1 \quad \text{(from \propref{chain-growth})}.
\end{IEEEeqnarray}
If $\Dat{k}=0$, clearly $L_{\min}(t_{k+1}-1) \geq L_{\min}(t_k - 1)$.
Adding these up gives the required result.
\end{proof}



\begin{proof}[Proof of \lemref{cps-stabilize}]
Note that $\dC_p(t)$ is a valid chain at \timeslot $t$ and $\len{\dC_p(t)} = L_p(t) \geq L_{\min}(t)$. Therefore, it suffices to show the first claim of the lemma.

For contradiction, let $s \geq t_k + \goodsep$ be the first \timeslot in which 
there is a valid header chain $\Chain'$ such that 
$\len{\Chain'} \geq L_{\min}(s)$ and $b^* \not\in \Chain'$.
%

Let $b'$ be the block with maximum height on the chain $\Chain'$, such that $b'$ was produced in a \timeslot $t_i$ with $D_i = 1$.
For $\Chain'$ to be a valid chain at \timeslot $s$, we need $t_i \leq s$.
Since the block $b'$ is produced by an honest node, $b'$ extends $\dC_q(t_i-1)$ for some honest node $q$.
Therefore, $\dC_q(t_i-1)$ is a prefix of $\Chain'$.
This means that $b^* \not\in \dC_q(t_i-1)$.
Moreover, $\len{\dC_q(t_i-1)} = L_q(t_i-1) \geq L_{\min}(t_i-1)$.
If $i > k$, then $t_i-1 \geq t_k + \goodsep$ (since $D_k = 1$) and $t_i - 1 < s$ (shown above). 
This is a contradiction because we assumed that $s$ is the first \timeslot such that $s \geq t_k + \goodsep$ and 
%
$b^* \notin \Chain'$ and $\len{\Chain'} \geq L_{\min}(s)$ for some valid chain $\Chain'$.
Since $b^*$ is the only block produced in slot $t_k$, $i=k$ is also not possible.
We conclude that $i < k$.

Since $D_i = 1$ and $b'$ is produced in \timeslot $t_i$,
\begin{IEEEeqnarray}{C}
\label{eq:block-i-download}
    L_{\min}(t_i + \goodsep) \geq \len{b'}.
\end{IEEEeqnarray}
%
By assumption,
\begin{IEEEeqnarray}{C}
\label{eq:block-j-switch}
    \len{\Chain'} \geq L_{\min}(s).
\end{IEEEeqnarray}

Let $t_j$ be the last non-\sltempty \timeslot such that $t_j \leq s$. Note that $j \geq k > i$. 
We must consider two cases: 
%

\begin{enumerate}
\item Case 1: $s \geq t_j + \goodsep$ or $\Dat{j}=0$.
If $\Dat{j}=0$, we don't have to worry about whether the block from slot $t_j$ was downloaded by all honest nodes.
If $\Dat{j} = 1$ but $s \geq t_j + \goodsep$, then we know that all honest nodes have downloaded the block from slot $t_j$ before the end of \timeslot $s$. That is,
\begin{IEEEeqnarray}{rCl}
    L_{\min}(s) 
    &\geq& L_{\min}(t_j + \goodsep) \\
    &\geq& L_{\min}(t_{i+1}-1) + \Din{i}{j} \quad \text{(from \propref{chain-growth-interval})} \\
    \label{eq:chain-growth-case1}
    &\geq& L_{\min}(t_{i} + \goodsep) + \Din{i}{j}.
\end{IEEEeqnarray}
By definition of $b'$, all blocks in $\Chain'$ appearing after $b'$ correspond to \ydowns. These blocks must be from distinct \iindices greater than $i$ but at most $j$. So,
\begin{IEEEeqnarray}{C}
\label{eq:adv-chain-case1}
    \len{\Chain'} \leq \len{b'} + \Nin{i}{j}.
\end{IEEEeqnarray}
From \eqref{block-i-download, block-j-switch, chain-growth-case1, adv-chain-case1}, we derive
\begin{IEEEeqnarray}{rCl}
\label{eq:pivot-contra-case1}
    \Din{i}{j} \leq \Nin{i}{j} \implies \Yin{i}{j} \leq 0 \implies \Yin{0}{i} < \Yin{0}{j}
\end{IEEEeqnarray}
where $i < k \leq j$.

\item Case 2: $t_j \leq s < t_j + \goodsep$ and $\Dat{j} = 1$.
In this case, the block from slot $t_j$ may not have enough time to be downloaded by all honest nodes before the end of slot $s$.
However, for any $l < j$ such that $\Dat{l} = 1$, $t_l + \goodsep < t_j \leq s$, so there is enough time to download the block from \timeslot $t_l$.
Let $l \in\intvl{i}{j-1}$ be the greatest index such that $\Dat{l} = 1$. Then, $t_j > t_l + \goodsep$, and $\Din{i}{l} = \Din{i}{j-1}$.
\begin{IEEEeqnarray}{rCl}
    \label{eq:chain-growth-case2}
    L_{\min}(s) 
    &\geq& L_{\min}(t_j) \\
    &\geq& L_{\min}(t_l + \goodsep) \\
    &\geq& L_{\min}(t_{i+1} - 1) + \Din{i}{l} \quad \text{(from \propref{chain-growth-interval})} \\
    &\geq& L_{\min}(t_{i} + \goodsep) + \Din{i}{j-1}.
\end{IEEEeqnarray}
Note that since $\Dat{j}=1$, $\Nin{i}{j} = \Nin{i}{j-1}$. Therefore, as in the previous case,
\begin{IEEEeqnarray}{C}
\label{eq:adv-chain-case2}
    \len{\Chain'} \leq \len{b'} + \Nin{i}{j-1}.
\end{IEEEeqnarray}
From \eqref{block-i-download, block-j-switch, chain-growth-case2, adv-chain-case2},
\begin{IEEEeqnarray}{rCl}
\label{eq:pivot-contra-case2}
    \Din{i}{j-1} \leq \Nin{i}{j-1} \implies \Yin{i}{j-1} \leq 0 \implies \Yin{0}{i} < \Yin{0}{j-1}. \IEEEeqnarraynumspace
\end{IEEEeqnarray}
Note that since we assumed $s \geq t_k + \goodsep$ and $s < t_j + \goodsep$, we know that $j > k$. Therefore, $i < k \leq j-1$.
\end{enumerate}
%
In either case, \eqref{pivot-contra-case1} or \eqref{pivot-contra-case2} contradict the assumption $\predCP{k}$ (\defref{cp}).
%
\end{proof}











\subsection{Probabilistic Pivots are Abundant}
\label{sec:appendix-security-proofs-many-pps}

We build up to the proof of \lemref{many-pps} through a series of propositions,
starting with recalling a versatile tail bound.
\begin{proposition}[Hoeffding's inequality~{\cite{doi:10.1080/01621459.1963.10500830} \cite[Thm.~4]{duchi-hoeffding}}]
    \label{prop:hoeffding}
    Let $Z_1, ..., Z_n$ be independent bounded random variables with
    $\forall i: Z_i \in [a,b]$, where $-\infty < a \leq b < \infty$.
    Then, $\forall t\geq0$:
    \begin{IEEEeqnarray}{rCl}
        \Prob{\left(\sum_{i=1}^n Z_i\right) - \Exp{\sum_{i=1}^n Z_i} \geq t n}
        &\leq&
        \exp\left(-\frac{2 n t^2}{(b-a)^2} \right)
        \\
        \Prob{\left(\sum_{i=1}^n Z_i\right) - \Exp{\sum_{i=1}^n Z_i} \leq -t n}
        &\leq&
        \exp\left(-\frac{2 n t^2}{(b-a)^2} \right).
    \end{IEEEeqnarray}
\end{proposition}

%
%
\begin{proposition}
    \label{prop:lower-tailbound-X}
    With $\alphaLowerTailX \triangleq 2 \epsGood^2$,
    \begin{IEEEeqnarray}{l}
        \forall \intvl{i}{j}\colon
        \forall \delta \geq 0\colon
        \nonumber
        \\
        \qquad
        \Prob{\Xin{i}{j} \leq (1-\delta) 2 \epsGood (j-i)}
        \leq \exp( - \alphaLowerTailX \delta^2 (j-i)).
        \IEEEeqnarraynumspace
    \end{IEEEeqnarray}
\end{proposition}
\begin{proof}
    By Hoeffding's inequality (\propref{hoeffding}).
\end{proof}


\begin{proposition}
    \label{prop:ppivot-randomwalk}
    \begin{IEEEeqnarray}{C}
        \forall k\colon
        \Prob{\predPP{k}}
        \geq \probPPFormula
        %
        \triangleq \probPP
    \end{IEEEeqnarray}
\end{proposition}
\begin{proof}
    \Eqref{pivot-conditions-equivalence-randomwalks}
    characterizes $\predPP{k}$
    as the intersection of three independent events:
    \begin{IEEEeqnarray}{rCl}
        \Event_1
        &\triangleq&
        \{ \Xat{k} = 1 \}
        \\
        \Event_2
        &\triangleq&
        \{ \forall\ell\colon \Xin{k}{k+\ell} \geq 0 \}
        \\
        \Event_3
        &\triangleq&
        \{ \forall\ell\colon \Xin{k-1-\ell}{k-1} \geq 0 \}
    \end{IEEEeqnarray}
    Their probabilities are easily calculated~\cite{stackexchange-math-rwreturnto0}:
    \begin{IEEEeqnarray}{C}
        \Prob{\Event_1}
        = \probGood
        \qquad
        \Prob{\Event_2} = \Prob{\Event_3}
        = (2\probGood - 1) / \probGood
        \IEEEeqnarraynumspace
    \end{IEEEeqnarray}
\end{proof}


\begin{proposition}
    \label{prop:lower-tailbound-ppivots}
    With $\alphaLowerTailPP \triangleq 2 \probPP^2$,
    \begin{IEEEeqnarray}{l}
        \forall \intvl{i}{j} \intvleq 2 K_1 K_2\colon
        \quad
        %
        %
        %
        \Prob{\Pin{i}{j} \leq (1-\delta) \probPP 2 K_1 K_2}
        \nonumber
        \\
        \qquad\qquad\qquad\quad {}\leq{} 2 K_1 \exp(- \alphaLowerTailPP \delta^2 K_2) + \Khorizon^2 \exp(-\alphaLowerTailX K_1).
        \IEEEeqnarraynumspace
    \end{IEEEeqnarray}
\end{proposition}
\begin{proof}
    Let
    $\Event \triangleq \{\forall \intvl{i}{j} \intvlgeq K_1\colon \Xin{i}{j} > 0\}$.
    From \propref{lower-tailbound-X} with $\delta=1$,
    and a union bound over all intervals
    ($\leq \Khorizon^2$ many),
    we get
    \begin{IEEEeqnarray}{C}
        \Prob{\lnot\Event} \leq \Khorizon^2 \exp(-\alphaLowerTailX K_1).
    \end{IEEEeqnarray}

    For any given index $k$, we can
    partition
    the intervals of
    \eqref{pivot-conditions-equivalence-intervals}
    into `long'
    %
    and `short'
    %
    intervals (length at least vs.\ less than $K_1$):
    \begin{IEEEeqnarray}{rCl}
        \Event_k
        &\triangleq&
        \{ \predPP{k} \}
        = \Event_k^{\mathrm{L}} \land \Event_k^{\mathrm{S}}
        \\
        \Event_k^{\mathrm{L}}
        &\triangleq&
        \{\forall k \in \intvl{i}{j} \intvlgeq K_1\colon \Xin{i}{j} > 0\}
        \IEEEeqnarraynumspace
        \\
        \Event_k^{\mathrm{S}}
        &\triangleq&
        \{\forall k \in \intvl{i}{j} \intvll K_1\colon \Xin{i}{j} > 0\}.
    \end{IEEEeqnarray}
    Note that $\Event_k^{\mathrm{L}} \supseteq \Event$.
    Thus, for any two given \iindices $k_1, k_2$,
    if $k_1, k_2$ are `far apart',
    \ie,
    if $\abs{k_1 - k_2} \geq 2 K_1$,
    then
    $\Event_{k_1}$ and $\Event_{k_2}$ are conditionally independent
    given $\Event$
    (since $\Event_{k_1}^{\mathrm{S}}$ and $\Event_{k_2}^{\mathrm{S}}$ are).

    We bound and decompose $I^* \triangleq \intvl{i}{j} = \intvl{i}{i + 2 K_1 K_2} = \bigcup_{\ell=1}^{2K_1} I_{\ell}$:
    \begin{IEEEeqnarray}{rCl}
        \forall\ell\in\{1,...,2K_1\}\colon
        \quad
        I_{\ell}
        &\triangleq&
        \{ i+0\cdot 2K_1+\ell, ...
        \nonumber\\
        && \qquad{} ..., i+(K_2-1)\cdot 2K_1+\ell \}.
        \IEEEeqnarraynumspace
    \end{IEEEeqnarray}
    We define corresponding events, $\forall\ell\in\{1,...,2K_1\}$:
    \begin{IEEEeqnarray}{rCl}
        \Event^*
        &\triangleq&
        \left\{ \Pat{I^*} \leq (1-\delta) \probPP 2 K_1 K_2 \right\}
        \\
        \Event_{\ell}
        &\triangleq&
        \left\{ \Pat{I_{\ell}} \leq (1-\delta) \probPP K_2 \right\}.
    \end{IEEEeqnarray}
    Clearly, $\Event^* \subseteq \bigcup_{\ell=1}^{2 K_1} \Event_\ell$.
    Thus, by a union bound,
    \begin{IEEEeqnarray}{rCl}
        \Prob{ \Event^* \cond \Event }
        &\leq&
        \sum_{\ell=1}^{2 K_1} \Prob{ \Event_\ell \cond \Event }.
        \IEEEeqnarraynumspace
    \end{IEEEeqnarray}
    Furthermore, $\forall\ell\in\{1,...,2K_1\}$,
    and with $\mu_\ell \triangleq \Exp{ \Pat{I_{\ell}} \cond \Event}$:
    \begin{IEEEeqnarray}{rCl}
        \IEEEeqnarraymulticol{3}{l}{
            \Prob{ \Event_\ell \cond \Event }
            =
            \Prob{ \Pat{I_{\ell}} \leq (1-\delta) \probPP K_2 \cond \Event }
        }
        \IEEEeqnarraynumspace
        \\\quad
        &\leqA&
        \Prob{ \Pat{I_{\ell}} \leq (1-\delta) \mu_\ell \cond \Event }
        \IEEEeqnarraynumspace
        \\
        &\leqB&
        \exp(-2 \delta^2 \mu_\ell^2 / K_2)
        \leqC
        \exp(-2 \probPP^2 \delta^2 K_2),
        \IEEEeqnarraynumspace
        %
    \end{IEEEeqnarray}
    where
    (a) and (c)~use 
    \begin{IEEEeqnarray}{C}
        \mu_\ell = K_2 \Exp{\Ind{\predPP{k}} \cond \Event} \geq K_2 \Exp{\Ind{\predPP{k}}} \geq K_2 \probPP \IEEEeqnarraynumspace
    \end{IEEEeqnarray}
    (\propref{ppivot-randomwalk}),
    %
    and
    (b)~uses that
    $\{\predPP{k_1}\}$ and
    $\{\predPP{k_2}\}$
    are conditionally independent given $\Event$
    for $k_1, k_2 \in I_\ell$,
    and
    Hoeffding's inequality (\propref{hoeffding}).
    %

    Thus, we complete the proof by observing, as desired, that
    \begin{IEEEeqnarray}{rCl}
        \Prob{ \Event^* }
        &=&
        \Prob{ \Event^* \cap \Event } + \Prob{ \Event^* \cap \lnot\Event }
        \\
        &\leq&
        \Prob{ \Event^* \cond \Event } + \Prob{ \lnot\Event }
        \\
        &\leq&
        2 K_1 \exp(-2 \probPP^2 \delta^2 K_2)
        + \Khorizon^2 \exp(-\alphaLowerTailX K_1).
        \IEEEeqnarraynumspace
    \end{IEEEeqnarray}
\end{proof}


\begin{proof}[Proof of \lemref{many-pps}]
From \propref{lower-tailbound-ppivots} by setting $K_1,K_2 = \Omega(\kappa)$ and $\Kcp = 2K_1K_2$.
\end{proof}







\subsection{Many Probabilistic Pivots Imply One Combinatorial Pivot}
\label{sec:appendix-security-proofs-many-pps-one-cp}


\begin{proof}[Proof of \Propref{download-or-spend-budget}]
%

%

In \timeslot $t_k$, there is exactly one block $b$ produced by an honest node, and 
the block header is made public at the beginning of the \timeslot,
and is seen by all honest nodes within $\DeltaHeader$ time.
Thereafter, each node has enough time to download $\goodsepbw$ blocks during \timeslots $[t_k, t_k + \goodsep]$.

%
%

%
%
%
%
%
%
    %
    %
%
%
%


Under the download rule $\dlrulelong$, all honest nodes download content for their longest header chain.
If $\Dat{k} = 0$
\ie an honest node did not download content for the block $b$ before the end of \timeslot $t_k + \goodsep$,
then
%
that honest node must download the content for at least $\goodsepbw$ blocks on chains longer than the height of the block $b$ or in the prefix of the block $b$.
%
Since honest nodes produce blocks extending their longest chain, $b$ extends $\dC_p(t_k-1)$ for some $p$.
Let $b^*$ be the block produced in \timeslot $t_i$ where $\predCP{i}$ (suppose $i$ exists).
$\predCP{i} \implies \Yat{i} = 1$, therefore this block is unique, and also $t_k > t_i + \goodsep$.
Due to \lemref{cps-stabilize}, any valid header chain longer than $b$ at time slot $t_k$ must contain $b^*$.
%
Therefore, the only blocks 
%
that are downloaded by an honest node during \timeslots $[t_k, t_k + \goodsep]$
\begin{enumerate}
    \item must be produced after $t_i$ because they extend $b^*$, and
    \item must be produced no later than $t_k$ because there are no blocks produced in $\intvl{t_k}{t_k+\goodsep}$.
\end{enumerate}
%
In case a \sltcp $i<k$ does not exist, the claim is trivial.
%
%
\end{proof}



\begin{proposition}
\label{prop:not-cp-exists-interval}
\begin{IEEEeqnarray}{C}
    \lnot \predCP{k} \implies \exists \intvl{i}{j} \ni k \colon \Yin{i}{j} \leq 0.
\end{IEEEeqnarray}
\end{proposition}

\begin{proof}
From \defref{cp}, $\lnot \predCP{k}$ implies that either there exists $i < k$ such that $\Yin{0}{i} \geq \Yin{0}{k}$ or there exists $j \geq k$ such that $\Yin{0}{k} > \Yin{0}{j}$.
In the first case, $\intvl{i}{k} \ni k$ and $\Yin{i}{k} \leq 0$.
In the second case, $\intvl{k-1}{j} \ni k$ and $\Yin{k-1}{j} \leq \Yin{k}{j} + 1 \leq 0$.
\end{proof}


\begin{proposition}
\label{prop:not-cp-interval-properties}
If $\Yin{i}{j} \leq 0$, then
\begin{IEEEeqnarray}{rCl}
    \label{eq:not-cp-interval-property1}
    \Nin{i}{j} &\geq& \Din{i}{j}, \\
    \label{eq:not-cp-interval-property2}
    \Gin{i}{j} - \Din{i}{j} &\geq& \frac{1}{2} \left( \Gin{i}{j} - \Bin{i}{j} \right).
\end{IEEEeqnarray}
\end{proposition}

\begin{proof}
\Eqref{not-cp-interval-property1} follows from the definition $\Yat{i} = \Dat{i} - \Nat{i}$.
%
Then,
\begin{IEEEeqnarray}{rCl}
    \Gin{i}{j} + \Bin{i}{j} &=& \Din{i}{j} + \Nin{i}{j} \\
    \Gin{i}{j} + \Bin{i}{j} &\geq& 2 \Din{i}{j} \\
    2 \Gin{i}{j} - 2 \Din{i}{j} &\geq& \Gin{i}{j} - \Bin{i}{j}.
\end{IEEEeqnarray}
\end{proof}



\begin{proposition}
\label{prop:ppivots-imply-honest-margin}
If $\Pin{i}{j} > 0$, then $\Gin{i}{j} - \Bin{i}{j} \geq \Pin{i}{j}$.
\end{proposition}

\begin{proof}
Let $n = \Pin{i}{j}$.
First, consider the case $n=1$.
There is exactly one \sltpp $k \in \intvl{i}{j}$.
From \defref{pp}, $\Xin{0}{i} < \Xin{0}{j}$. Therefore, $\Xin{i}{j} > 0$, hence $\Gin{i}{j} - \Bin{i}{j} \geq 1$.

For the general case, let $k_1,...,k_n$ be the \sltpps in $\intvl{i}{j}$. Then, we can apply the $n=1$ case on the disjoint intervals $\intvl{i}{k_1}$, $\intvl{k_1}{k_2}, ...$, $\intvl{k_{n-1}}{j}$ and then sum them up.

This can also be seen from \figref{pivot-randomwalk}.
Each \sltpp corresponds to a height that the random walk $\Xat{k}$ attains exactly once.
This means that in any interval containing $n$ \sltpps, the random walk $\Xat{k}$ `moves up' by at least $n$ units, and this is possible only if there are $n$ more `ups' than `downs'.
%
\end{proof}




\begin{lemma}
\label{lem:one-cp-induction-base}
If all honest nodes use
the download rule $\dlrulelong$,
%
%
%
%
%
and if
\begin{IEEEeqnarray}{C}
    \label{eq:cp-induction-base-margin-condition}
    \forall \intvl{i}{j} \intvlgeq \Kcp, i < \Kcp \colon \frac{\goodsepbw}{2} \left( \Gin{i}{j} - \Bin{i}{j} \right) > \Qin{0}{j}, \text{ and} \\
    %
    \label{eq:cp-induction-base-ppivots-condition}
    \frac{\goodsepbw}{4} \Pin{0}{\Kcp} > \Qin{0}{2\Kcp},
    %
\end{IEEEeqnarray}
then $\exists k_1^* \in \intvl{0}{\Kcp} \colon \predCP{k_1^*}$.
\end{lemma}

\begin{proof}
Due to \eqref{cp-induction-base-ppivots-condition}, there is at least one \sltpp in $\intvl{0}{\Kcp}$ (otherwise $\Pin{0}{\Kcp}=0$).
Suppose for contradiction that there is no \sltcp in $\intvl{0}{\Kcp}$.
Since \sltcps are also \sltpps, it is enough to consider that
none of the \sltpps is a \sltcp.
Then around each \sltpp, there must be at least one interval which violates the combinatorial pivot condition.
Formally, there is a set of intervals $\intvlset$ such that:
\begin{IEEEeqnarray}{Cr}
    \label{eq:intervals-cover-ppivots}
    \bigcup_{I \in \intvlset} I \supseteq \left\{ k \in \intvl{0}{\Kcp} \colon \predPP{k} \right\} & \\
    \label{eq:intervals-y-condition}
    \forall I \in \intvlset \colon \Yat{I} \leq 0 &\quad \text{(from \propref{not-cp-exists-interval})}.
\end{IEEEeqnarray}
Without loss of generality, each interval $I \in \intvlset$ contains at least one \sltpp (removing all intervals that do not contain a \sltpp maintains \eqref{intervals-cover-ppivots, intervals-y-condition}).
Then if $\intvl{i}{j} \in \intvlset$, $i < \Kcp$.

First, let's consider the large intervals with $|I| \geq \Kcp$.
Consider \iindices $k \in I$ for which $\Gat{k}=1$ (\sltgood) but $\Dat{k}=0$ (\ydown).
From \propref{download-or-spend-budget}, for each such \iindex, all honest nodes download $\goodsepbw$ blocks that are produced no later than $t_k$.
%
%
%
%
%
%
The number of \iindices $k \in I$ with  $\Gat{k} = 1$ and $\Dat{k} = 0$ is exactly $\Gat{I} - \Dat{I}$.
For each such index, there must exist $\goodsepbw$ distinct blocks produced in or before the interval $I$. Therefore if $I = \intvl{i}{j}$,
\begin{IEEEeqnarray}{rClr}
    \Qin{0}{j} &\geq& \goodsepbw \left( \Gin{i}{j} - \Din{i}{j} \right) & \\
    &\geq& \frac{\goodsepbw}{2} \left( \Gin{i}{j} - \Bin{i}{j} \right) & \quad \text{(from \propref{not-cp-interval-properties}).}
\end{IEEEeqnarray}
This is a contradiction to \eqref{cp-induction-base-margin-condition}.

Therefore all intervals $I \in \intvlset$ are small ($|I| < \Kcp$).
%
%
%
%
Then for each $I \in \intvlset$, $I \subset \intvl{0}{2\Kcp}$.
Also, 
\begin{IEEEeqnarray}{rClr}
    \Gat{I} - \Dat{I} &\geq& \frac{1}{2} \left( \Gat{I} - \Bat{I} \right) & \quad \text{(from \propref{not-cp-interval-properties})} \\
    \label{eq:failed-more-than-ppivots}
    &\geq& \frac{1}{2} \Pat{I} & \quad \text{(from \propref{ppivots-imply-honest-margin}).}
\end{IEEEeqnarray}

Consider the \iindices $k \in \intvl{0}{2\Kcp}$ with $\Gat{k} = 1$ and $\Dat{k} = 0$.
%
Let $\intvlset_k = \{ I \in \intvlset \colon k \in I\}$ be the set of intervals that contain \index $k$.
Let $I^L_k$ be an interval in $\intvlset_k$ that stretches farthest to the left, and let $I^R_k$ be an interval that stretches farthest to the right (these may also be the same).
%
Note that all other intervals in $\intvlset_k$ are contained in $I^L_k \cup I^R_k$.
Therefore, all intervals in $\intvlset_k$ except $I^L_k$ and $I^R_k$ can be removed from $\intvlset$ while maintaining \eqref{intervals-cover-ppivots, intervals-y-condition} (see \figref{one-cp-proof-figures}(a)).
This process is repeated for all $k \in \intvl{0}{2\Kcp}$ with $\Gat{k} = 1$ and $\Dat{k} = 0$, so that in the resulting set $\intvlset$, each such \iindex $k$ is contained in at most two intervals.
Then,
\begin{IEEEeqnarray}{rCl}
    \sum_{k \in \intvl{0}{2\Kcp} \colon \Gat{k} = 1, \Dat{k} = 0} |\intvlset_k| &\leq& \sum_{k \in \intvl{0}{2\Kcp} \colon \Gat{k} = 1, \Dat{k} = 0} 2 \\
    &=& 2\left( \Gin{0}{2\Kcp} - \Din{0}{2\Kcp} \right).
\end{IEEEeqnarray}
This sum can be rewritten as
\begin{IEEEeqnarray}{rCl}
    \sum_{k \in \intvl{0}{2\Kcp} \colon \Gat{k} = 1, \Dat{k} = 0} |\intvlset_k| &=& \sum_{I \in \intvlset} \left( \Gat{I} - \Dat{I} \right) \\
    &\geq& \sum_{I \in \intvlset} \frac{1}{2} \Pat{I} \\
    &\geq& \frac{1}{2} \Pin{0}{\Kcp} \quad \text{(due to \eqref{intervals-cover-ppivots})}.
    \IEEEeqnarraynumspace
\end{IEEEeqnarray}
Therefore,
\begin{IEEEeqnarray}{rCl}
    \Gin{0}{2\Kcp} - \Din{0}{2\Kcp} &\geq& \frac{1}{4} \Pin{0}{\Kcp}.
\end{IEEEeqnarray}
This can also be seen from \figref{one-cp-proof-figures}(b).

Finally, as shown before, for each \index $k$ with $\Gat{k}=1$ and $\Dat{k}=0$, all honest nodes download at least $\goodsepbw$ distinct blocks produced in or before \iindex $k$ (\propref{download-or-spend-budget}). This gives
\begin{IEEEeqnarray}{rCl}
    \Qin{0}{2\Kcp} &\geq& \goodsepbw \left( \Gin{0}{2\Kcp} - \Din{0}{2\Kcp} \right) \\
    &\geq& \frac{\goodsepbw}{4} \Pin{0}{\Kcp}
\end{IEEEeqnarray}
which is a contradiction to \eqref{cp-induction-base-ppivots-condition}.
%
\end{proof}



\import{./figures/}{fig-one-cp-proof-figures.tex}




\begin{proof}[Proof of \lemref{one-cp-induction-full}]
This will be proved through induction.
For the base case ($m=0$), \lemref{one-cp-induction-base} shows 
%
that $\exists k_1^* \in \intvl{0}{\Kcp} \colon \predCP{k_1^*}$.

For $m \geq 1$, assume that $\exists k_{m-1}^* \in \intvl{(m-1)\Kcp}{m\Kcp}$ such that  $\predCP{k_{m-1}^*}$.
Now we want to show that $\exists k_{m}^* \in \intvl{m\Kcp}{(m+1)\Kcp}$ such that  $\predCP{k_{m}^*}$.
%
%
Suppose for contradiction that there is no \sltcp in $\intvl{m\Kcp}{(m+1)\Kcp}$.
%
%
%
%
As in the proof of \lemref{one-cp-induction-base}, 
there is a set of intervals $\intvlset$ such that:
\begin{IEEEeqnarray}{Cr}
    \label{eq:induction-full-intervals-cover-ppivots}
    \bigcup_{I \in \intvlset} I \supseteq \left\{ k \in \intvl{m\Kcp}{(m+1)\Kcp} \colon \predPP{k} \right\} & \\
    \label{eq:induction-full-intervals-y-condition}
    \forall I \in \intvlset \colon \Yat{I} \leq 0. &
    %
\end{IEEEeqnarray}
Without loss of generality, each interval $I \in \intvlset$ contains at least one \sltpp.
%
Then if $\intvl{i}{j} \in \intvlset$, $i < (m+1)\Kcp$ and $j > m\Kcp$.

First, consider the large intervals with $|I| \geq \Kcp$.
Consider \iindices $k \in I$ for which $\Gat{k}=1$ (\sltgood) but $\Dat{k}=0$ (\ydown).
%
%
%
From \propref{download-or-spend-budget},
for each such \iindex $k$, all honest nodes download $\goodsepbw$ blocks that are produced
%
in the interval $\intvl{k_{m-1}^*}{k}$.
%
%
%
%
%
%
The number of \iindices $k \in I$ with  $\Gat{k} = 1$ and $\Dat{k} = 0$ is exactly $\Gat{I} - \Dat{I}$.
For each such index, there must exist $\goodsepbw$ distinct blocks from distinct \BPOs
%
that are downloaded by honest nodes.
Therefore if $I = \intvl{i}{j}$,
\begin{IEEEeqnarray}{rClr}
    \Qin{k_{m-1}^*}{j} &\geq& \goodsepbw \left( \Gin{i}{j} - \Din{i}{j} \right) & \\
    &\geq& \frac{\goodsepbw}{2} \left( \Gin{i}{j} - \Bin{i}{j} \right) & \quad \text{(from \propref{not-cp-interval-properties}).}
\end{IEEEeqnarray}
But $k_{m-1}^* > (m-1)\Kcp$ and $i < (m+1)\Kcp$. Therefore $\Qin{k_{m-1}^*}{j} \leq \Qin{i-2\Kcp}{j}$.
Then we have a contradiction to \eqref{cp-induction-full-margin-condition}.

Therefore all intervals $I \in \intvlset$ are small ($|I| < \Kcp$).
%
%
%
%
Then for each $I \in \intvlset$, $I \subset \intvl{(m-1)\Kcp}{(m+1)\Kcp}$.
Also, 
\begin{IEEEeqnarray}{Cr}
    \label{eq:failed-more-than-ppivots-induction}
    \Gat{I} - \Dat{I} \geq \frac{1}{2} \left( \Gat{I} - \Bat{I} \right) \geq \frac{1}{2} \Pat{I} & \quad \text{(from \propref{not-cp-interval-properties, ppivots-imply-honest-margin})}
\end{IEEEeqnarray}

Consider the \iindices $k \in \intvl{(m-1)\Kcp}{(m+1)\Kcp}$ with $\Gat{k} = 1$ and $\Dat{k} = 0$.
%
%
%
%
%
%
%
Following the arguments in the proof of \lemref{one-cp-induction-base},
we can reduce the set $\intvlset$
so that in the resulting set $\intvlset$, each such \iindex $k$ is contained in at most two intervals.
Then,
\begin{IEEEeqnarray}{Cl}
    & \sum_{k \in \intvl{(m-1)\Kcp}{(m+1)\Kcp} \colon \Gat{k} = 1, \Dat{k} = 0} |\intvlset_k| \nonumber \\
    \leq& 
    %
    %
    2\left( \Gin{(m-1)\Kcp}{(m+1)\Kcp} - \Din{(m-1)\Kcp}{(m+1)\Kcp} \right).
\end{IEEEeqnarray}
This sum can be rewritten as
\begin{IEEEeqnarray}{rCl}
    \sum_{k \in \intvl{(m-1)\Kcp}{(m+1)\Kcp} \colon \Gat{k} = 1, \Dat{k} = 0} |\intvlset_k| &=& \sum_{I \in \intvlset} \left( \Gat{I} - \Dat{I} \right) \\
    &\geq& \sum_{I \in \intvlset} \frac{1}{2} \Pat{I} \\
    &\geq& \frac{1}{2} \Pin{m\Kcp}{(m+1)\Kcp}.
    \IEEEeqnarraynumspace
    %
\end{IEEEeqnarray}
Therefore,
\begin{IEEEeqnarray}{rCl}
    &&\Gin{(m-1)\Kcp}{(m+1)\Kcp} - \Din{(m-1)\Kcp}{(m+1)\Kcp} \nonumber \\
    &\geq& \frac{1}{4} \Pin{m\Kcp}{(m+1)\Kcp}.
\end{IEEEeqnarray}

Finally, for each \index $k$ with $\Gat{k}=1$ and $\Dat{k}=0$, all honest nodes download at least $\goodsepbw$ distinct blocks produced in or before 
the most recent \sltcp before $(m-1)\Kcp$.
By induction assumption, we have a \sltcp $k_{m-2}^* \in \intvl{(m-2)\Kcp}{(m-1)\Kcp}$.
%
%
This gives
\begin{IEEEeqnarray}{rCl}
    && \Qin{(m-2)\Kcp}{(m+1)\Kcp} \nonumber \\
    &\geq& \goodsepbw \left( \Gin{(m-1)\Kcp}{(m+1)\Kcp} - \Din{(m-1)\Kcp}{(m+1)\Kcp} \right) \\
    &\geq& \frac{\goodsepbw}{4} \Pin{m\Kcp}{(m+1)\Kcp}
\end{IEEEeqnarray}
which is a contradiction.
%
\end{proof}
\section{Proof-of-Work Security Proofs}
\label{sec:appendix-pow-proofs}


\begin{proposition}
    \label{prop:index-time-bridge-pow}
    \begin{IEEEeqnarray}{C}
        \forall k, K \in \IN \colon
        \Prob{ \slotduration(t_{k+K} - t_{k}) \geq \frac{K}{\blkratetime(1-\delta)} } \leq \exp\left( - \frac{K\delta^2}{2(1+\delta)}\right)
        \IEEEeqnarraynumspace
    \end{IEEEeqnarray}
\end{proposition}
\begin{proof}
    This results from a Poisson tail bound for the number of \BPOs in real time $K/\blkratetime$, and noting that 
    each non-\sltempty \timeslot has exactly one \BPO.
    %
\end{proof}


\begin{lemma}
    \label{lem:safety-and-liveness-comb-pow}
    If for some $\Kcp > 0$,
    \begin{IEEEeqnarray}{C}
        \label{eq:condition-one-cp-m-pow-safety}
        \forall m \geq 0 \colon \exists k_m^* \in \intvl{m\Kcp}{(m+1)\Kcp} \colon \predCP{k_m^*},
        \IEEEeqnarraynumspace
    \end{IEEEeqnarray}
    then the PoW longest chain protocol $\protocol$ with $\confDepth = 2\Kcp + 1$ satisfies safety.
    Further, if
    \begin{IEEEeqnarray}{C}
        \label{eq:cond-index-time-bridge-pow}
        \forall k \in \IN, K \geq \Kcp \colon t_{k+K} - t_{k} < \frac{K}{\blkratetime\slotduration(1-\delta)},
        \IEEEeqnarraynumspace
    \end{IEEEeqnarray}
    then it also satisfies liveness with $\Tlive = \frac{6\Kcp + 2}{\blkratetime\slotduration(1-\delta)}$.
\end{lemma}
\begin{proof}
    Safety: For an arbitrary \timeslot $t$, let $k$ be the largest \iindex such that $t_k \leq t$.
    From \eqref{condition-one-cp-m-pow-safety}, every interval of $2\Kcp$ \iindices contains at least one \sltcp. Therefore, there exists $k^* \in \intvl{k-2\Kcp-1}{k-1}$ such that $\predCP{k^*}$.
    Let $b^*$ be the block from \iindex $k^*$.
    Due to \lemref{cps-stabilize}, for all honest nodes $p,q$ and $t' \geq t$,
    $b^* \in \dC_p(t)$ and $b^* \in \dC_q(t')$.
    But $k^* \geq k-\confDepth$, so the block $b^*$ cannot be $\confDepth$-deep in any chain at \timeslot $t$ Therefore, $\LOG{p}{t}$ is a prefix of $b^*$ which in turn is a prefix of $\dC_q(t')$.
    We can thus conclude that
    either 
    $\LOG{p}{t} \preceq \LOG{q}{t'}$ or $\LOG{q}{t'} \preceq \LOG{p}{t}$.
    Therefore, safety holds.

    %
    %
    %
    %
    %
    %
    %
    %

    Liveness: Assume a transaction $\tx$ is received by all honest nodes before \timeslot $t$. Again let $k$ be the largest \iindex such that $t_k \leq t$.
    We know that there exists $k^* \in (k,k+2\Kcp]$ such that $\predCP{k^*}$.
    The honest block $b^*$ from \iindex $k^*$ or its prefix must contain $\tx$ since $\tx$ is seen by all honest nodes at time $t < t_{k^*}$.
    Since $k^*$ is a \sltcp, for all $\intvl{i}{j} \ni k^*$, $\Din{i}{j} > \Nin{i}{j}$ (\defref{cp} and \eqref{random_walks_X_and_Y}), and hence $\Din{i}{j} > \frac{j-i}{2}$.
    Particularly,
    \begin{IEEEeqnarray}{rCl}
        \Din{k^*-1}{k^* + 2\confDepth-1} &>& \confDepth \\
        \implies \Din{k^*}{k^* + 2\confDepth - 1} &>& \confDepth - 1.
        \IEEEeqnarraynumspace
    \end{IEEEeqnarray}
    Then from \propref{chain-growth-interval},
    \begin{IEEEeqnarray}{rCl}
        L_{\min}(t_{k^*+2\confDepth-1}+\goodsep) - L_{\min}(t_{k^*+1}-1) &\geq& \Din{k^*}{k^* + 2\confDepth - 1} \nonumber \\
        &\geq& \confDepth.
        \IEEEeqnarraynumspace
    \end{IEEEeqnarray}
    Due to \lemref{cps-stabilize},
    $b^*\in\dC_p(t')$ for all honest nodes $p$ and $t'\geq t_{k^*} + \goodsep$, and $L_{\min}(t_{k^*+1}-1) \geq \len{b^*}$. This means that $b^*$ is $\confDepth$-deep in $\dC_p(t')$ for all honest nodes $p$ and all $t' \geq t_{k^*+2\confDepth-1}+\goodsep$.
    Finally, with $k^* \leq k+2\Kcp$ and \eqref{cond-index-time-bridge-pow},
    \begin{IEEEeqnarray}{rCl}
        t_{k^*+2\confDepth-1}+\goodsep - t &\leq& t_{k + 6\Kcp + 1} + \goodsep - t_{k} \nonumber \\
        &\leq& t_{k + 6\Kcp + 2} - t_{k} \nonumber \\
        &<& \frac{6\Kcp + 2}{\blkratetime\slotduration(1-\delta)}.
        \IEEEeqnarraynumspace
    \end{IEEEeqnarray}
    Therefore, $\mathsf{tx}\in\LOG{p}{t'}$ for all $t'\geq t+\Tlive$.
\end{proof}


%
%
%
%
%
%
%
%
%
%
%
%
%
%
%
%
%
%
%
%
%
%
%
%
%
%
%
%
%

%
%
%

%
%
%
%

%
%
%
%
%
%
%
%
%
%
%
%
%
%
%
%
%
%

%
%
%
%
%
%
%
%
%

%
%
%
%
%
%
%
%



\begin{proof}[Proof of \Thmref{safety-and-liveness-pow}]

First, we show that the conditions of \lemref{one-cp-induction-full} hold, and therefore \sltcps occur.
Define the event
\begin{IEEEeqnarray}{rCl}
    \Event_1 &=& \left\{\forall \intvl{i}{j} \intvlgeq \Kcp \colon \Pin{i}{j} > (1-\delta)\probPP(j-i) \right\}
\end{IEEEeqnarray}
Suppose that $\Event_1$ occurs, and $\frac{\goodsepbw}{16}\probPP (1-\delta) > 1$.
Then,
%
\begin{IEEEeqnarray}{rCl}
    \forall \intvl{i}{j} \intvlgeq \Kcp \colon \qquad \frac{\goodsepbw}{4} \Pin{i}{j} &>& \frac{\goodsepbw}{4} (1-\delta)\probPP(j-i) \\
    &>& 4(j-i) \\
    &\eqA& \Qin{i-2\Kcp}{j+\Kcp}
\end{IEEEeqnarray}
where (a) is because as $\slotduration \to 0$, each non-\sltempty \timeslot has exactly one \BPO. This satisfies \eqref{cp-induction-full-ppivots-condition} in \lemref{one-cp-induction-full}. Further,
\begin{IEEEeqnarray}{rCl}
    \frac{\goodsepbw}{2} \left( \Gin{i}{j} - \Bin{i}{j} \right)
    &\geq& \frac{\goodsepbw}{2} \Pin{i}{j} \\
    &>& 3 (j-i) \\
    &>& \Qin{i-2\Kcp}{j}
\end{IEEEeqnarray}
which satisfies condition \eqref{cp-induction-full-margin-condition} in \lemref{one-cp-induction-full}. 
%
Therefore there is one \sltcp in every interval of the form $\intvl{m\Kcp}{(m+1)\Kcp}$.
%
Also suppose the following event occurs:
\begin{IEEEeqnarray}{rCl}
    \Event_2 &=& \left\{\forall k \in \IN, K \geq \Kcp \colon t_{k+K} - t_k < \frac{K}{\blkratetime\slotduration(1-\delta)} \right\}.
\end{IEEEeqnarray}
%
%
%
%
%
%
%
Then \lemref{safety-and-liveness-comb-pow} guarantees safety and liveness with $\confDepth = 2\Kcp$ and $\Tlive = \frac{6\Kcp+2}{\blkratetime\slotduration(1-\delta)}$.

%
By choosing $\Kcp = \Omega(\kappa^2)$, $\Khorizon = \poly(\kappa)$, and using \lemref{many-pps}, \propref{index-time-bridge-pow},
\begin{IEEEeqnarray}{rCl}
    \Prob{\lnot \Event_1 } &=& \negl(\kappa) \\
    %
    \Prob{\lnot \Event_2 } 
    &\leq& \Khorizon^2 e^{ - \Kcp \delta^2 /(2(1+\delta))}
    = \negl(\kappa).
\end{IEEEeqnarray}
By a union bound, 
%
the probability of failure of either $\Event_1$ or $\Event_2$ is $\negl(\kappa)$.
Finally, indices are mapped to real time as $\TliveReal \triangleq \Tlive \slotduration$.

Finally, we take the limit $\slotduration \to 0$.
%
With the relations $\blkratetime = \blkrateslot/\slotduration$, $\BWEquation$, and 
$\probPP = \probPPFormula$,
\begin{IEEEeqnarray}{C}
    \label{eq:prob-good-equation-pow}
    \probGood = \probGoodFormula \to (1-\beta)e^{-\blkratetime\left(\DeltaHeader + \goodsepbw/\bwtime\right)}, \\
    \label{eq:C-equation-pow}
    \frac{\goodsepbw}{16} \frac{(2\probGood-1)^2}{\probGood} (1-\delta) > 1
\end{IEEEeqnarray}
%

%
%
%
%
%


Note that $\goodsepbw$ is an analysis parameter whose value is arbitrarily.
To find the maximum block production rate $\blkratetime$ that the protocol can achieve, we should optimize over $\goodsepbw$.
To find the maximum achievable $\blkratetime$,
we can take $\delta \to 0$ as we can increase the latency through increasing $\Kcp$ to still satisfy the error bounds.
Then solving for $\probGood$ from \eqref{C-equation-pow},
\begin{IEEEeqnarray}{C}
    \probGood > \frac{\goodsepbw+4 + \sqrt{8\goodsepbw+16}}{2\goodsepbw}.
\end{IEEEeqnarray}
Then from \eqref{prob-good-equation-pow},
\begin{IEEEeqnarray}{C}
    \blkratetime\left(\DeltaHeader + \goodsepbw/\bwtime\right) > \ln\left( \frac{2(1-\beta)\goodsepbw}{\goodsepbw+4 + \sqrt{8\goodsepbw+16}} \right).
\end{IEEEeqnarray}
Maximizing over $\goodsepbw$ gives the resulting threshold.



%
%
%
\end{proof}
\section{Proof-of-Stake}
\label{sec:appendix-pos}





\subsection{Pseudocodes for Equivocation Removal}
\label{sec:appendix-pos-eqremoval-pseudocodes}

\import{./figures/}{alg-posequivblank-lc.tex}
\import{./figures/}{alg-posequivblank-hdrtree.tex}

\algref{posequivblank-lc,posequivblank-hdrtree}






\subsection{Security Proofs}
\label{sec:appendix-pos-proofs}


\begin{proposition}
    \label{prop:index-time-bridge-pos}
    For all $\delta \in (0,1)$, $k, K \in \IN$,
    \begin{IEEEeqnarray}{C}
        \Prob{ t_{k+K} - t_{k} \geq \frac{K/(1-e^{-\blkrateslot})}{1-\delta} } \leq 
        %
        e^{- 2K(1-e^{-\blkrateslot})\delta^2}, \IEEEeqnarraynumspace
        %
        %
        %
    \end{IEEEeqnarray}
\end{proposition}
\begin{proof}
    This results from a Hoeffding bound for the number of non-\sltempty \timeslots in $K/(1-e^{-\blkrateslot})$ \timeslots.
\end{proof}



\begin{lemma}
    \label{lem:safety-and-liveness-comb-pos}
    If for some $\Kcp > 0$,
    \begin{IEEEeqnarray}{C}
        \label{eq:condition-one-cp-m-pos-safety}
        \forall m \geq 0 \colon \exists k_m^* \in \intvl{m\Kcp}{(m+1)\Kcp} \colon \predCP{k_m^*},
        \IEEEeqnarraynumspace \\
        \label{eq:cond-index-time-bridge-pos}
        \forall k \in \IN, K \geq \Kcp \colon t_{k+K} - t_{k} < \frac{K/(1-e^{-\blkrateslot})}{1-\delta},
        \IEEEeqnarraynumspace
    \end{IEEEeqnarray}
    then \sapos
    with $\confDepth = 6\Kcp + 1$
    and $\keqproof = 4\Kcp$
    %
    satisfies safety and liveness with
    $\Tlive = \frac{14\Kcp+2}{(1-e^{-\blkrateslot})(1-\delta)}$.
    %
\end{lemma}

\begin{proof}
%
%
%
%
%
%

First, we prove safety.
Consider arbitrary \timeslots $t \leq t'$ and let $h$ be the largest \iindex such that $t_h \leq t$.
Consider a block $b_i \in \dC_p(t)\trunc{\confDepth}$ which was produced in \iindex $i \leq h - \confDepth$.
%
%
%
%
%
%
%
%
%
From \eqref{condition-one-cp-m-pos-safety}, every interval of $2\Kcp$ \iindices contains at least one \sltcp.
%
Therefore
for any $i$,
%
there exist \sltcps $j,k$ %
such that %
\begin{IEEEeqnarray}{C}
    \label{eq:cond-i-j-k-pos}
    %
    %
    %
    i < j < k \leq i + 4\Kcp.
\end{IEEEeqnarray}
Also, let $l$ be the last \sltcp before (excluding) \iindex $h$. Then
\begin{IEEEeqnarray}{C}
    \label{eq:cond-i-k-l-pos}
    l \geq h - 2\Kcp \geq i + \confDepth - 2\Kcp > i + 4\Kcp.
    \IEEEeqnarraynumspace
\end{IEEEeqnarray}
%
%
%
From \eqref{cond-i-j-k-pos} and \eqref{cond-i-k-l-pos}, we have
\begin{IEEEeqnarray}{C}
    \label{eq:cond-i-j-k-l-pos}
    i < j < k \leq i + \keqproof < l < h.
\end{IEEEeqnarray}
These are shown in \figref{pos-safety-proof}.
%
%
%
%
%
Let $b_j, b_k, b_l$ be the blocks corresponding to the respective \sltcps (see \figref{pos-safety-proof}).
Due to \lemref{cps-stabilize} and $t \geq t_h > t_l + \goodsep$,
\begin{IEEEeqnarray}{C}
    b_i \preceq b_j \preceq b_k \preceq b_l \preceq \dC_p(t) \cap \dC_q(t').
\end{IEEEeqnarray}
%
%
Since the above holds for all $b_i \in \dC_p(t)\trunc{\confDepth}$, we obtain that $\dC_p(t)\trunc{\confDepth} \preceq \dC_q(t')$.
We can thus conclude that
\begin{IEEEeqnarray}{rCl}
    %
    %
    %
    \label{eq:consistent-confDepth-chains-pos}
    \dC_p(t)\trunc{\confDepth} &\consistent& \dC_q(t')\trunc{\confDepth}
    %
\end{IEEEeqnarray}
%
%
where $\Chain_1 \consistent \Chain_2$ denotes $\Chain_1 \preceq \Chain_2$ or $\Chain_2 \preceq \Chain_1$.
%
Due to \eqref{consistent-confDepth-chains-pos},
the $\confDepth$-deep header chains of $p$ at $t$ and of $q$ at $t'$ are consistent. 
Without \equivocationremoval, this was enough to show safety of the corresponding ledgers.
Now to show that the two ledgers $\LOG{p}{t}$ and $\LOG{q}{t'}$ are consistent, we only need to show that if the content of a block is blanked in $\LOG{p}{t}$, it is also blanked in $\LOG{q}{t'}$, and conversely if it is not blanked in $\LOG{p}{t}$, it is not blanked in $\LOG{q}{t'}$.




Suppose that the content of $b_i$ is blanked in $\LOG{p}{t}$.
This means that either there was an equivocation for $b_i$ in node $p$'s view (hence node $p$ did not download the content), or there is an equivocation proof against $b_i$ in
a header 
%
in $\dC_p(t)$.
The header of $b_i$ must be seen by all honest nodes $p$ before the end of \timeslot $t_j+\goodsep$ (since $b_j \in \dC_p(t_j + \goodsep)$).
Then since block $b_k$ is honest, $t_k > t_j + \goodsep$, and $k \leq i + \keqproof$, either $b_k$ or another block in its prefix must include an equivocation proof against $b_i$.
%
%
We know that $b_k \in \dC_q(t')$,
so the content of block $b_i$ will be blanked in 
%
$\LOG{q}{t'}$ as well.

Suppose that the content of $b_i$ is not blanked in $\LOG{p}{t}$.
This means that there is no equivocation proof against $b_i$ in $\dC_p(t)$.
Since $l \geq h - 2\Kcp$, the block $b_l$ cannot be more than $2\Kcp$-deep in $\dC_p(t)$, \ie,
\begin{IEEEeqnarray}{C}
    \len{b_l} \geq \len{\dC_p(t)} - 2\Kcp.
    \IEEEeqnarraynumspace
\end{IEEEeqnarray}
But $b_i$ is $\confDepth$-deep in $\dC_p(t)$ (as assumed), so 
\begin{IEEEeqnarray}{C}
    \len{b_i} \leq \len{\dC_p(t)} - \confDepth.    
    \IEEEeqnarraynumspace
\end{IEEEeqnarray}
Together, we have 
\begin{IEEEeqnarray}{C}
    \len{b_l} \geq \len{b_i} + \confDepth - 2\Kcp > \len{b_i} + \keqproof.
    \IEEEeqnarraynumspace
\end{IEEEeqnarray}
%
%
%
Therefore $b_l$ or any block extending it cannot contain an equivocation proof against $b_i$.
Since $b_l \in \dC_p(t)$ and $b_l \in \dC_q(t')$, there cannot be any block in $\dC_q(t')$ before $b_l$, that is not in $\dC_p(t)$.
Therefore, there is no equivocation proof against $b_i$ in $\dC_q(t')$.
Also, node $q$ must have downloaded block $b_i$, otherwise there must have been an equivocation proof in $b_k$ or its prefix as discussed in the previous paragraph.
So, the content of $b_i$ is not blanked in $\LOG{q}{t'}$.
%
%
We can thus conclude that
either 
%
$\LOG{p}{t} \preceq \LOG{q}{t'}$ or $\LOG{q}{t'} \preceq \LOG{p}{t}$.
Therefore, safety holds.

%
%
%
%
%
%
%
%
%

We next prove liveness. Assume a transaction $\tx$ is received by all honest nodes before \timeslot $t$. Again let $h$ be the largest \iindex such that $t_h \leq t$.
We know that there exists $k^* \in (h,h+2\Kcp]$ such that $\predCP{k^*}$.
The honest block $b^*$ from \iindex $k^*$ or its prefix must contain $\tx$ since $\tx$ is seen by all honest nodes at time $t < t_{k^*}$.
Since $k^*$ is a \sltcp, for all $\intvl{i}{j} \ni k^*$, $\Din{i}{j} > \Nin{i}{j}$ (\defref{cp} and \eqref{random_walks_X_and_Y}), and hence $\Din{i}{j} > \frac{j-i}{2}$.
Particularly,
\begin{IEEEeqnarray}{rCl}
    \Din{k^*-1}{k^* + 2\confDepth-1} &>& \confDepth \\
    \implies \Din{k^*}{k^* + 2\confDepth - 1} &>& \confDepth - 1.
    \IEEEeqnarraynumspace
\end{IEEEeqnarray}
Then from \propref{chain-growth-interval},
\begin{IEEEeqnarray}{rCl}
    L_{\min}(t_{k^*+2\confDepth-1}+\goodsep) - L_{\min}(t_{k^*+1}-1) &\geq& \Din{k^*}{k^* + 2\confDepth - 1} \nonumber \\
    &\geq& \confDepth.
    \IEEEeqnarraynumspace
\end{IEEEeqnarray}
Due to \lemref{cps-stabilize},
$b^*\in\dC_p(t')$ for all honest nodes $p$ and $t'\geq t_{k^*} + \goodsep$, and $L_{\min}(t_{k^*+1}-1) \geq \len{b^*}$. This means that $b^*$ is $\confDepth$-deep in $\dC_p(t')$ for all honest nodes $p$ and all $t' \geq t_{k^*+2\confDepth-1}+\goodsep$.
Further, the content of an honest block will never be blanked out in any honest node's ledger.
Finally, with $k^* \leq h+2\Kcp$ and \eqref{cond-index-time-bridge-pow},
\begin{IEEEeqnarray}{rCl}
    t_{k^*+2\confDepth-1}+\goodsep - t &\leq& t_{h+2\Kcp+2\confDepth-1} + \goodsep - t_{h} \nonumber \\
    &\leq& t_{h+2\Kcp+2\confDepth} - t_{h} \nonumber \\
    &<& \frac{2\Kcp+2\confDepth}{(1-e^{-\blkrateslot})(1-\delta)}.
    \IEEEeqnarraynumspace
\end{IEEEeqnarray}
Therefore, $\mathsf{tx}\in\LOG{p}{t'}$ for all $t'\geq t+\Tlive$ with $\Tlive$ as in the lemma statement.
%
%
%
%
%
%
%
%
\end{proof}


\import{./figures/}{fig-pos-safety-proof.tex}



We also need another proposition to bound the number of \BPOs in a given number of slots, in order to bound $\Qin{i}{j}$.
\begin{proposition}
\label{prop:Q-bound-pos}
\begin{IEEEeqnarray}{C}
    \forall t, T \in \IN \colon
    \Prob{ \sum_{r=t}^{t+T} (H_t + A_t) \geq \blkrateslot T (1+\delta) } \leq 
    \exp\left( - \frac{\blkrateslot T \delta^2}{2(1+\delta)} \right). \IEEEeqnarraynumspace
\end{IEEEeqnarray}
\end{proposition}
\begin{proof}
This results from a Poisson tail bound since $\sum_{r=t}^{t+T} (H_t + A_t) \sim \operatorname{Poisson}(\blkrateslot T)$.
\end{proof}





\begin{proof}[Proof of \Thmref{safety-and-liveness-pos}]

First, we show that the conditions of \lemref{one-cp-induction-full} hold, and therefore \sltcps occur.
Suppose that the following three events occur.
\begin{IEEEeqnarray}{rCl}
    \Event_1 &=& \left\{\forall \intvl{i}{j} \intvlgeq \Kcp \colon \Pin{i}{j} > (1-2\delta)\probPP(j-i) \right\},
    \\
    \Event_2 &=& \left\{\forall t \in \IN, T \geq \frac{\Kcp}{1-e^{-\blkrateslot}} \colon \sum_{r=t}^{t+T} (H_t + A_t) < \blkrateslot T (1+\delta) \right\},
    \IEEEeqnarraynumspace
    \\
    \Event_3 &=& \left\{ \forall k \in \IN, K \geq \Kcp \colon t_{k+K} - t_{k} < \frac{K/(1-e^{-\blkrateslot})}{1-\delta} \right\}. \IEEEeqnarraynumspace
\end{IEEEeqnarray}

From $\Event_2$ and $\Event_3$, we get
\begin{IEEEeqnarray}{rCl}
    \forall \intvl{i}{j} \intvlgeq \Kcp \colon \quad 
    \Qin{i}{j} &\triangleq& \sum_{k=i+1}^{j} (H_{t_k} + A_{t_k}) \\
    \text{with } T = \frac{j-i}{(1-e^{-\blkrateslot})(1-\delta)}, \quad &\leq& \sum_{t=t_i}^{t_i + T} (H_{t} + A_{t}) \\
    &<& \frac{\blkrateslot (j-i) (1+\delta)}{(1-e^{-\blkrateslot})(1-\delta)} \\
    &\leq&  \frac{\blkrateslot (j-i)}{(1-e^{-\blkrateslot})(1-2\delta)}.
    \IEEEeqnarraynumspace
\end{IEEEeqnarray}
Then if $\frac{\goodsepbw}{16}\probPP (1-4\delta) > \frac{\blkrateslot}{(1-e^{-\blkrateslot})}$,
%
\begin{IEEEeqnarray}{rCl}
    \forall \intvl{i}{j} \intvlgeq \Kcp \colon \quad \frac{\goodsepbw}{4} \Pin{i}{j} &>& \frac{\goodsepbw}{4} (1-2\delta)\probPP(j-i)
    \IEEEeqnarraynumspace
    \\
    &>& \frac{4\blkrateslot(j-1)(1-2\delta)}{(1-e^{-\blkrateslot})(1-4\delta)}
    \\
    &>& \frac{4\blkrateslot(j-1)}{(1-e^{-\blkrateslot})(1-2\delta)}
    \\
    &>& \Qin{i-2\Kcp}{j+\Kcp}.
\end{IEEEeqnarray}
This satisfies \eqref{cp-induction-full-ppivots-condition} in \lemref{one-cp-induction-full}. Further,
\begin{IEEEeqnarray}{rCl}
    \frac{\goodsepbw}{2} \left( \Gin{i}{j} - \Bin{i}{j} \right)
    &\geq& \frac{\goodsepbw}{2} \Pin{i}{j} \\
    &>& \frac{3\blkrateslot(j-1)}{(1-e^{-\blkrateslot})(1-2\delta)}  \\
    &>& \Qin{i-2\Kcp}{j}
\end{IEEEeqnarray}
which satisfies condition \eqref{cp-induction-full-margin-condition} in \lemref{one-cp-induction-full}. 
%
Therefore there is one \sltcp in every interval of the form $\intvl{m\Kcp}{(m+1)\Kcp}$.
%
%
%
%
%
%
%
Then by \lemref{safety-and-liveness-comb-pos}, the protocol achieves safety and liveness with appropriately chosen $\confDepth, \keqproof, \Tlive$.
%
%
%
%

By using \lemref{many-pps}, $\Kcp = \Omega(\kappa^2)$, $\Khorizon = \poly(\kappa)$, \propref{index-time-bridge-pos, Q-bound-pos},
%
%
%
%
%
%
and union bounds, 
%
the probability of failure of either $\Event_1$, $\Event_2$ or $\Event_3$ is $\negl(\kappa)$.
%
%

The required security threshold is obtained from $\BWEquation$,
$\probGood = \probGoodFormula$,
$\frac{C}{16}\probPP > \frac{\blkrateslot}{1-e^{-\blkrateslot}}$,
and $\probPP = \probPPFormula$,.
As in the case of PoW, $\goodsepbw$ is a free parameter that can be optimized to find the best set of parameters.
%
%
%
%
%
%
%
%
%
%
\end{proof}
\section{Transaction Validity Proofs}
\label{sec:proof-details}

%
%
%
\begin{proof}[Proof of \lemref{pred-valid-1}] %
    In \secref{pos-equivocations}, we have that equivocation proofs against a block need to be included within the next $\keqproof$ blocks. A node creating a block thus knows all equivocation proofs that will ever be included in their header chain against blocks that are $\keqproof$-deep, thus the state of the $\keqproof$-deep chain is determined. Since equivocations for the last $\keqproof$ blocks can only remove transactions, the node knows all transactions that \emph{may} be included in the final chain. From this, the node can determine all states $\mathcal{S}$ that could be touched by any transaction in the last $\keqproof$ blocks.\footnote{Note that this includes all states a transaction could have changed if it executed differently. This could be achieved by transactions needing to include an access list of all states they are allowed to change. One can imagine a DOS attack where a transaction's access list could prevent future transactions.} A transaction $tx$ that does not depend on any state in $\mathcal{S}$ for its execution, can thus be executed on the state of the $\keqproof$-deep chain, therefore, satisfying predictable transaction validity. A node then only includes transactions that don't rely on a state in $\mathcal{S}$. Note that transactions in the same block could depend on the same state.
\end{proof}

%
%
%
%
\begin{proof}[Proof of \lemref{pred-valid-2}]
    Consider a funding gas account $\mathsf{acc}$ with balance $b$ before the last $\keqproof$ blocks in the chain.
    This balance is set for that account as no equivocation proofs against blocks that are $\keqproof$-deep are allowed by the protocol.
    Note that any transactions in the last $\keqproof$ blocks that fund the account can still be sanitized from the ledger so we do not consider them in the balance yet. The node instead considers all transactions $\mathcal{T}_{\mathsf{acc}(\keqproof)}$ in the last $\keqproof$ which use the funds from the account (including any withdrawals). Since the transactions funded by $\mathsf{acc}$ that end up in the ledger are a subset of $\mathcal{T}_{\mathsf{acc}(\keqproof)}$, and all fees are extracted regardless of how a transaction executes, the node will at worst underestimate the balance of $\mathsf{acc}$ at the tip of the chain.  
\end{proof}
\section{Congestion-Based Attacks}
\label{sec:general-attacks}


\subsection{\GreedyAttack}
\label{sec:greedy-attack}

The \teaserattack relied strongly on the fact that the attacker could entice nodes with a long header chain that is later discovered to be unavailable for download. It is natural in this case to consider adjusting the download rule to one that prefers the proverbial `bird in the hand over two birds in the bush', \ie, to extend the blocks we already downloaded over the illusive promise of a longer chain that the attacker may withhold from us. 

\smallskip\noindent\textbf{The \ruleGreedy policy.}\;\;
This policy prioritizes downloading blocks that extend the chain a node has already processed. If a header of a block at height $h$ is announced, and we already have $h_i$ blocks from that chain, then we set the priority of the block to be $(h_i,h)$ and compare between the two priorities lexicographically.

While the \rulegreedy policy performs well at high processing rates, we unfortunately find that it preforms poorly in the low processing rate regime. Specifically, if a fork in the chain occurs, and nodes are split evenly between the two alternatives, the fork may never resolve. This is because nodes extend their own chain, and prioritize download on their side of the split, while having insufficient processing power to catch up with the other alternative chain. A fork in the chain can result from a deliberate attack by an attacker that releases blocks selectively to different nodes, by a network split, or worse, by an unlucky timing of honest node mining events. In this case, the blockchain fails even for small attackers. 
Importantly, a fork that never resolves is either a safety or a liveness failure, as no transaction on either side of the split can be safely accepted.

\import{./figures/}{fig-experiment-greedy.tex}

To demonstrate this download rule in action, we simulate a network of 100 nodes that are split evenly between two partitions for only 15 seconds, \ie, for an expected time required to produce 15 blocks.%
\footnote{Such short splits are relatively easy to induce in reality (transient problems with Internet routing, denial-of-service on the network, etc.) and thus a practical scheduling rule must recover from such splits.}
Once the network split ends, the simulation continues for another 4000 seconds, allowing nodes the opportunity to 
converge on a chain.
%
We 
%
measure the height of the latest block all nodes agree upon. If nodes do not recover from the partition, 
this block will be the genesis and the liveness of the protocol has failed. Otherwise, nodes quickly agree on the main chain and the height of the latest agreed block is 
just a little behind the longest tip of the chain. 

%

We simulate the evolution after a brief partition for both the \rulelc policy as well as for the \rulegreedy policy. Our results (\cref{fig:experiment-greedy}) show that in settings where bandwidth is greater than $1/2$, nodes manage to catch up with the chain and the rate of growth matches for both scheduling policies. In lower bandwidth settings, however, nodes never catch up.
Note that this attack requires no adversarial mining,
yet the protocol is insecure (\cf \cref{fig:comparison-bddelay-bdbandwidth}(c)).
This is in stark contrast to the bounded-delay analysis
which suggests that the protocol retains security
against a non-mining adversary
%
at any bandwidth (\cf \cref{fig:comparison-bddelay-bdbandwidth}(a)),
and highlights again the need to study the security of blockchains at capacity.



%
%



\subsection{\PoSTeaserAttack}
\label{sec:pos-attack}

In this section, we present an attack to establish a bound (as a function of the security level) on the block production rate (and hence, throughput, or bandwidth requirement) of a single chain PoS NC protocol without an equivocation removal policy.
For concreteness, we demonstrate this attack on PoS NC using any one of three download rules: `download the longest header chain', `download towards the freshest block', and `equivocation avoidance'.
%
For the `download the longest header chain' rule, \cite[Figure~3]{bwlimitedposlc} showed one attack
%
and the attack in this section generalizes that.
%
On the other hand, \cite{bwlimitedposlc} proved PoS NC secure under the other two download rules by setting the duration of a \timeslot proportional to the security parameter $\kappa$, to achieve security with probability $1-\negl(\kappa)$.
Hence the block production rate (and throughput) decays as $O\left(\frac{1}{\kappa}\right)$.
In this section, we show an attack which succeeds if the block production rate is $\Omega\left(\frac{1}{\log(\kappa)}\right)$.




Furthermore, while the attack in \cite[Figure~3]{bwlimitedposlc} required that the PoS NC protocol rejects blocks with invalid transactions after downloading them, this attack does not require that. Therefore, this attack works even if the PoS NC protocol accepts blocks with invalid transactions into the output ledger (\eg, to subsequently clean them up deterministically across honest nodes).
This is because as noted in \cref{sec:throughputloss}, even if the protocol accepts blocks with invalid transactions, honest nodes must download the block content (to ensure data availability). 
This is why we require an equivocation removal policy so that honest nodes can unilaterally discard content for blocks that they do not download. This is what allows us to overcome the $\Omega\left(\frac{1}{\log(\kappa)}\right)$ throughput bound in this work.



Before describing the attack, we briefly recap the download rules analyzed in this section.
In the `download towards the freshest block' rule (\cf \cite[Alg.~2]{bwlimitedposlc}), a node chooses the block produced in the most recent time slot (`freshest'), and if it not yet downloaded, downloads the first unknown block in the chain containing that block.
One the node downloads the freshest block, it stops downloading any blocks until a block header from a more recent \timeslot shows up.
In the `equivocation avoidance' rule (\cite[Alg.~4]{bwlimitedposlc}), the node first filters the tree of its headers by keeping only one leaf per \BPO (ties broken by the adversary). From among the remaining headers, the node picks a block to download as per the `download longest header chain' rule.
The `download longest header chain' rule is as described in \cref{alg:longest-header-chain-rule}.



%
%
%
%
%
%
%
%

%
%
%
%
%

%
%

%

%

\subsection{Attack Strategy}
\label{sec:tp-attack-strategy}
%
The attack works in two phases. See \cref{fig:tp-log-kappa-attack} for reference.
$\bwtime$ is the bandwidth constraint (in blocks per second), $\slotduration$ is the slot duration, and $\kappa$ is the security parameter.

\paragraph{Setup phase} \label{item:tp-attack-setup} At time slot $t_0$, the adversary creates a chain $\Chain$ which forks off the honest chain $\Chain_0$ by at least $L = \log(\kappa)$ blocks, and is at least as long as $\Chain_0$.
%
The prefix length $L$ is chosen so that the setup succeeds with non-negligible probability.
The adversary initially keeps $\Chain$ private.
%

%

%
%
%
%

\paragraph{Execution phase} \label{item:tp-attack-exec}
The adversary creates different chains $\Chain_1, \Chain_2, ...$ which contain equivocations of the blocks in $\Chain$, and pushes one chain to each honest node.
\begin{enumerate}[(1)]
    \item \label{item:tp-attack-exec-start} Let $t_1 > t_0$ be the first time slot with a block production. For any block
    $b_1$ produced in slot $t_1$, if
    %
    $b_1$ is produced by an honest node, then, 
    %
    the adversary breaks ties such that $b_1$ extends one of the equivocating chains $\Chain_i$.
    If 
    %
    $b_1$ is produced by the adversary, the adversary produces $b_1$ at the tip of 
    another chain made of equivocations of the blocks in
    %
    $\Chain$.
    %
    Regardless, any block $b_1$ produced in $t_1$ extends 
    a chain that forks off the downloaded longest chain by $L$ new blocks that need to be downloaded, hence it will take a long time for an honest node to download up to
    the block $b_1$.
    %
    %
    %
    
    \item \label{item:tp-attack-exec-mid} The adversary repeats step~\ref{item:tp-attack-exec-start} in all time slots $t_2,t_3,...$ with a block production.
    %
    Assuming there are many honest nodes, each block extends a different equivocating chain and is too long to catch up with.
    The adversary continues this until the following condition occurs.
    
    \item \label{item:tp-attack-exec-end} Let $t^*$ be the first time slot since $t_0$ in which an honest block $b^*$ is produced, such that there are no other blocks produced in slots $[t^*,t^* + L/(\bwtime\slotduration))$. This condition ensures that there is enough time for $b^*$ to be downloaded by all honest nodes.
    
    If the adversary had at least one block production opportunity $t' \in [t_0, t^* + L/(\bwtime\slotduration))$, then the adversary attaches a block $b'$ produced in slot $t'$ to 
    %
    the chain $\Chain$.
    The adversary makes the following updates,
    \begin{itemize}
        \item $\Chain_0 \gets \text{ chain ending in } b^*$,
        \item $\Chain \gets \text{ chain ending in } b'$,
        \item $t_0 \gets t^*$,
        \item $L \gets L+1$,
    \end{itemize}
    and thereafter repeats steps~%
    %
    \ref{item:tp-attack-exec-start}--\ref{item:tp-attack-exec-end}.
    
    If the adversary failed to get one block production opportunity in $[t_0, t^* + L/(\bwtime\slotduration))$, then the adversary gives up.
    %
\end{enumerate}

%
%
%
%
%
%


\begin{figure}
    \centering
    %
    \begin{tikzpicture}
        \footnotesize
        \def\blockinterval{0.75}
        \begin{scope}[blockchainold,x=1cm,y=0.6cm]

            \coordinate (G) at (-1*\blockinterval,0);
            
            \node [block-gray] (b0) at (0,0) {};
            \draw [link] (b0) -- (G);

            \node [block-gray] (C01) at (\blockinterval,0) {};
            \draw [link] (C01) -- (b0);
            \node [block-gray] (C02) at (2*\blockinterval,0) {};
            \draw [link] (C02) -- (C01);
            \node [block-gray] (C03) at (3*\blockinterval,0) {};
            \draw [link] (C03) -- (C02);
            \node [block-gray] (C04) at (4*\blockinterval,0) {};
            \draw [link] (C04) -- (C03);
            \node [block-gray] (C05) at (5*\blockinterval,0) {};
            \draw [link] (C05) -- (C04);
            \node [anchor=north west, inner sep=0] at (C05.south east) {$\Chain_0$};

            \foreach \i in {1,...,4} {
                \node [block-red] (C\i1) at (1*\blockinterval,\i) {};
                \draw [link] (C\i1.west) -- (b0);
                \node [block-red] (C\i2) at (2*\blockinterval,\i) {};
                \draw [link] (C\i2) -- (C\i1);
                \node [block-red] (C\i3) at (3*\blockinterval,\i) {};
                \draw [link] (C\i3) -- (C\i2);
                \node [block-red] (C\i4) at (4*\blockinterval,\i) {};
                \draw [link] (C\i4) -- (C\i3);
                \node [block-red] (C\i5) at (5*\blockinterval,\i) {};
                \draw [link] (C\i5) -- (C\i4);
            }
            
            \node [inner sep=0,anchor=north west] at (C15.south east) {$\Chain$};
            %

            
            \node [block-gray] (b1) at (6*\blockinterval,1) {};
            \draw [link] (b1) -- (C15);
            \node [anchor=west, inner sep=0, xshift=2pt, yshift=-1pt] at (b1.east) {$b_1$};

            \node [inner sep=0,anchor=north west] at (C25.south east) {$\Chain'$};
            \node [block-gray] (b2) at (7*\blockinterval,2) {};
            \draw [link] (b2) -- (C25);
            \node [anchor=west, inner sep=0, xshift=2pt, yshift=-1pt] at (b2.east) {$b_2$};
            %

            %
            \node [block-gray] (b3) at (8*\blockinterval,3) {};
            \draw [link] (b3) -- (C35);
            \node [anchor=west, inner sep=0, yshift=-1pt, xshift=2pt] at (b3.east) {$b^*$ (new $\Chain_0$)};

            \node [block-red] (b') at (6.5*\blockinterval,4) {};
            \draw [link] (b') -- (C45);
            \node [anchor=west, inner sep=0, yshift=-1pt, xshift=2pt] at (b'.east) {$b'$ (new $\Chain$)};
            
            \draw [decorate,decoration={brace,amplitude=4pt,raise=5pt}]
                (C41.north west) -- (C45.north east)
                node [midway,above=9pt,anchor=south,align=center]
                {$L = \log(\kappa)$ blocks};

            \def\timeheight{-1}
            \begin{scope}%
                \draw [Latex-] (10*\blockinterval,\timeheight) -- (-1*\blockinterval,\timeheight) node [below right] {\emph{Time}};
                
                \draw [] (5*\blockinterval,\timeheight) ++(0,0.2) -- ++(0,-0.4) node [below] {$t_0$};
                \draw [] (6*\blockinterval,\timeheight) ++(0,0.2) -- ++(0,-0.4) node [below] {$t_1$};
                \draw [] (7*\blockinterval,\timeheight) ++(0,0.2) -- ++(0,-0.4) node [below] {$t_2$};
                \draw [] (8*\blockinterval,\timeheight) ++(0,0.2) -- ++(0,-0.4) node [below] {$t^*$};
                \draw [] (9.5*\blockinterval,\timeheight) ++(0,0.2) -- ++(0,-0.4) node [below] {$t^* + \frac{L}{\bwtime\slotduration}$};
                %
                %
                \draw [|-|] (8*\blockinterval,\timeheight+0.5) -- (9.5*\blockinterval,\timeheight+0.5) node [midway, above, anchor=south] {No blocks};
            \end{scope}

        \end{scope}
    \end{tikzpicture}%
    \vspace{-1em}%
    \caption{Illustration of 
    the
    new attack of Section~\ref{sec:tp-attack-analysis-overview}.
    At time $t_0$, $\Chain_0$ is the longest downloaded chain of all honest nodes, and the adversary produces a chain $\Chain$ that forks off $\Chain_0$ by $L=\log(\kappa)$ blocks.
    Blocks produced in time slots $t_1,t_2,...$ (whether honest or adversarial) extend the chain $\Chain$ or a chain $\Chain'$ containing equivocation of the blocks in $\Chain$,
    and are not downloaded by all honest nodes in time before
    the next block production opportunity.
    Time slot $t^*$ is the first slot such that there are no block productions in the $\frac{L}{\bwtime\slotduration}$ slots after $t^*$.
    The block $b^*$ produced in slot $t^*$ therefore gets downloaded. If the adversary had at least one block production opportunity $t' \in [t_0, t^* + L/(\bwtime\slotduration)]$, then the adversary sets the chain ending in $b^*$ as new $\Chain_0$ and the chain ending in $b'$ as new $\Chain$, and repeats the attack.}
    \label{fig:tp-log-kappa-attack}
\end{figure}

%

%

%
%


\subsection{Analysis Overview}
\label{sec:tp-attack-analysis-overview}
The analysis below reuses notation defined in \cref{sec:analysis-definitions}.
For the attack to succeed, we assume the following:
\begin{itemize}
    %
    \item The protocol parameters $\blkrateslot,\slotduration$ satisfy $\frac{\blkrateslot}{\slotduration} > \frac{\bwtime}{\log\kappa}\log\frac{1-\beta}{\beta}$, where $\beta$ is the fraction of adversarial nodes and $\bwtime$ is the bandwidth constraint of each honest node in blocks per second.
    \item The total number of nodes $N$ is large.
    \item The adversary is allowed to break ties among equally long chains in the fork choice rule.
    \item The adversary is allowed to break ties among equal priority chains in the download rule.
    %
\end{itemize}

The fork length $L=\log(\kappa)$ is chosen such that the adversary can succeed in the setup phase with probability at least $e^{-O(L)} = 1/\poly(\kappa)$ at any given time, even with a minority stake. Hence this setup can be achieved by the adversary with non-negligible probability eventually during an execution of length $\poly(\kappa)$.

Now consider the execution phase. The key vulnerability exploited in this attack is that if the highest priority chain according to the download rule is on a long fork of which honest nodes have not downloaded any blocks, it will take a long time for honest nodes to download up to the tip of this chain. If the next block arrival happens before this chain is downloaded, the adversary makes honest nodes shift their download priority to a different chain, which is also on an equally long fork. This keeps repeating and honest nodes never finish downloading a chain that would help grow their longest downloaded chain.

Honest nodes get some respite when there is an honest block produced in slot $t^*$ such that there are no other blocks produced in slots $[t^*, t^*+L/(\bwtime\slotduration))$. The three download rules `download longest header chain', `download towards the freshest block', and `equivocation avoidance'
%
ensure that the honest block $b^*$ produced in slot $t^*$ remains the highest priority chain to download during the slots $[t^*, t^*+L/(\bwtime\slotduration))$. Given a bandwidth constraint of $\bwtime$ blocks per second, \ie, $\bwtime\slotduration$ blocks per time slot, honest nodes can completely download a fork of length $L$ in $L/(\bwtime\slotduration)$ time slots.

However, this does not end the attack. While waiting for one honest block production opportunity with $L/(\bwtime\slotduration)$ empty slots following it, if the adversary gets one block production opportunity, this allows the adversary to create a new chain whose length matches the longest downloaded chain of honest nodes.
The situation now looks just like at the start of the execution phase, except that the adversary's chain now forks from the honest nodes' new downloaded longest chain by $L+1$ blocks (one more than before).
The adversary then and repeats the execution phase all over again with the new chain it has produced, and with $L \gets L+1$.
%
As the adversary's fork length $L$ increase, it takes more time for honest nodes to download up to the tip of a newly produced block extending that fork.
This means it takes even longer for honest nodes to produce a block after which there are $L/(\bwtime\slotduration)$ empty slots such that the block gets downloaded.
Thereby, it becomes more likely that the adversary produces one block before honest downloads download a new chain, and continue the attack for another iteration with a larger fork length $L$.
%
As a result of this vicious cycle, 
%
%
the adversary can continue this attack forever with non-negligible probability!
%

This attack breaks safety of the protocol because the downloaded longest chain of honest nodes switches to a different chain every time the condition in \cref{sec:tp-attack-strategy}~\ref{item:tp-attack-exec-end} occurs.
%




\subsection{Analysis Details}
\label{sec:tp-attack-analysis-details}
Building up on the definitions from \cref{sec:analysis-definitions}, define a time slot $t$ to be \emph{honest} if $H_t>0$, and \emph{attacking} if $A_t>0$. 
%
%
Also define $\Hint{r,s}$ and $\Aint{r,s}$ as the number of honest and attacking slots respectively in the interval $(r,s]$.
$\Bint{r,s}$ is the number of slots $t \in (r,s]$ such that $H_t + A_t > 0$. 

%
%
%

\begin{definition}
For all $t$, define the event  
\begin{IEEEeqnarray}{C}
    %
    F_t := \left\{ \exists r < t \colon (\Hat{r} > 0) \land (\Aint{r,t} \geq \Hint{r,t}) \land (\Aint{r,t} \geq L) \right\}. \nonumber\IEEEeqnarraynumspace
\end{IEEEeqnarray}
\end{definition}

\begin{lemma}
If $F_{t}$ occurs, then the setup phase of the attack succeeds at time slot $t$, \ie, there exists an adversarial strategy which creates a chain $\Chain$ that forks off the longest downloaded chain of all honest nodes at time $t$ by $L$ blocks and is at least as long as the longest downloaded chain.
\end{lemma}
\begin{proof}
Let $b$ be an honest block produced in slot $r<t$ where $r$ satisfies $(\Hat{r} > 0) \land (\Aint{r,t} \geq \Hint{r,t}) \land (\Aint{r,t} \geq L)$.
The adversary's strategy is as follows.
In time slot $r$, the adversary pushes the block $b$ to all nodes irrespective of bandwidth, so that $\len{\dC_p(r)} = \len{b}$ for all honest nodes $i$.
The adversary then creates a private chain using its own blocks, extending the block $b$ (all these blocks are kept hidden).
The adversary can add one block to this chain in every slot in which the adversary produces a block, therefore the length of the adversary's chain at time $t$ is $\len{b}+ \Aint{r,t}$.
On the other hand, in every time slot that an honest block is produced, at most one block is added to the longest chain of all honest nodes, therefore the length of the honest chain at time $t$ is at most $\len{b} + \Hint{r,t}$.
Since $\Aint{r,t} \geq \Hint{r,t}$, the adversary's chain has the same or greater length compared to the honest chain at time slot $t$.
Since the last block that is common between the honest and adversary's chain is $b$, and $\Aint{r,t} \geq L$, the adversary's chain forks off the honest chain by at least $L$ blocks.
Therefore, we have the required conditions for the attack setup.
Note that the adversary does not need to be able to predict when the event $F_t$ would occur. Since creating blocks in proof-of-stake does not require computation time, the adversary can create this chain after it observes that the event $F_t$ occurred.
\end{proof}

\begin{lemma}
\label{lem:no-download}
%
Let $t>t_0$ be a successful time slot (\ie, $\Hat{t}+\Aat{t} > 0$) such that there exists another successful time slot $t' \in (t, t+L/(\bwtime\slotduration)]$. 
Then none of the blocks produced in slot $t$ are ever downloaded by any honest node. Hence for all honest nodes $p$, $L_p(t'-1) = L_p(t)$.
\end{lemma}
\begin{proof}
For all blocks $b$ that are produced in slot $t$, the attack strategy in \cref{sec:tp-attack-strategy}
ensures that the number of blocks to be downloaded in the prefix of $b$ (including $b$) is $L+1$.
Since each honest node can download at most $\bwtime\slotduration$ blocks per time slots, no honest node can download the entire prefix within $L/(\bwtime\slotduration)$ time slots (the adversary does not push any blocks to honest nodes during this period). 
At time slot $t'$, either an adversarial block or an honest block (or both) are produced.
In either case, step \ref{item:tp-attack-exec-mid} of the execution phase ensures that at slot $t'$, this new block has the highest priority under all three download rules.
This is because i) it is clearly the freshest block at slot $t'$,
ii) it is one of the longest chains (and the adversary breaks ties), and
iii) it has a non-equivocating tip and has length $L+1$, which is one of the longest chains (and the adversary breaks ties), 
%
Therefore, at time slot $t'$, all honest nodes switch to download a different block, and therefore the block $b$ is not downloaded.
Since for all honest nodes $p$, no block is downloaded, it is clear that $L_p(t'-1) = L_p(t)$.
\end{proof}

%

\begin{lemma}
\label{lem:yes-download}
%
Let $t$ be a successful time slot such that for all time slots $t' \in (t, t+L/(\bwtime\slotduration)]$, there are no blocks produced in slot $t'$ (\ie, $\Hat{t'}+\Aat{t'} = 0$).
Then, each honest node downloads at least one block produced in slot $t$, and for all honest nodes $p$, $L_p(t+L/(\bwtime\slotduration)) = L_p(t)+1$.
\end{lemma}
\begin{proof}
Since $t$ is an honest time slot, let $b$ be one of the honestly produced blocks in this time slot. At time slot $t$, $b$ is one of the freshest blocks. It remains one of the freshest blocks until time slot $t+L/(\bwtime\slotduration)$ because there are no other blocks produced in this interval.
%
As per the attack strategy, for both honest and adversarial blocks $b$, the block $b$ is on the longest chain in every node's view, and is not an equivocation. 
%

In case of a tie in the download rules, we assume that all honest nodes break the tie in favour of the same block $b$ (as this is chosen by the adversary). Therefore, the block $b$ has the highest download priority for all honest nodes in slots $[t,t+L/(\bwtime\slotduration)]$. Since the number of blocks to be downloaded in the prefix of $b$ (including $b$) is $L+1$, these blocks can be downloaded before the end of slot $t+L/(\bwtime\slotduration)$. We know that $b$ is longer than all honest nodes' longest downloaded chains at slot $t$ because of the attack strategy and \cref{lem:no-download}. Therefore the length of the longest downloaded chain of every honest node grows by $1$.
\end{proof}

\begin{lemma}
\label{lem:tp-attack-safety-viol}
Let $t_{\mathrm a}$ be the first time slot such that $t_{\mathrm a} > t_0 + \Tconf$, $t_{\mathrm a}$ is a successful time slot, and there are no blocks produced in slots $(t_{\mathrm a},t_{\mathrm a}+L'/(\bwtime\slotduration)]$ where $L'$ is the value of the attacker's parameter $L$ at time slot $t_0+\Tconf$. If the attacker does not terminate before slot $t_{\mathrm a}$, then there is a safety violation.
\end{lemma}
\begin{proof}
At the end of time slot $t_0 + \Tconf$, let $\LOG{p}{t_0+\Tconf}$ denote the ledger output by an honest node $p$. Note that this ledger contains all blocks mined before slot $t_0$ in the longest downloaded chain of node $p$, $\dC_p(t_0+\Tconf)$. As per the attack strategy \cref{sec:tp-attack-strategy} steps~\ref{item:tp-attack-exec-start} and \ref{item:tp-attack-exec-mid}, the block produced in time slot $t_{\mathrm a}$ extends a different equivocating chain that forks off $\dC_p(t_0+\Tconf)$ by $L'$ blocks. Since there are no blocks produced in slots $(t_{\mathrm a},t_{\mathrm a}+L'/(\bwtime\slotduration)]$, all honest nodes download this new chain and hence update their longest downloaded chain. However, note that at least $L'$ blocks that were in $\LOG{p}{t_0+\Tconf}$ are replaced by different blocks in $\dC_p(t_{\mathrm a})$, and therefore $\LOG{p}{t_0+\Tconf}$ and $\LOG{p}{t_{\mathrm a}}$ are not prefixes of each other. This causes a safety violation. 
%
\end{proof}

\begin{lemma}
%
For all $t$,
\begin{IEEEeqnarray}{C}
    \Prob{ F_t } \geq \pu \left( 1 - 2e^{-L/9} \right) \frac{1}{\sqrt{8L}} e^{-4\constAttackSetup L},
\end{IEEEeqnarray}
where $\constAttackSetup = \frac{1}{4} \ln\left(\frac{p}{4\pu}\right) + \frac{3}{4} \ln \left(\frac{3p}{4\pu}\right)$
and 
%
$\pu \triangleq \Prob{\Hat{t} > 0} = 1 - e^{-(1-\beta)\blkrateslot}$.
%

\end{lemma}
\begin{proof}
Let $T = \frac{2L}{p}(1+\epsilon)$ for some $\epsilon>0$ and let $s=t-T$.
\begin{IEEEeqnarray}{rCl}
    && \Prob{ F_t} \nonumber \\
    &=& \Prob{ \exists r < t \colon (\Hat{r} > 0) \land (\Aint{r,t} \geq \Hint{r,t}) \land (\Aint{r,t} \geq L) } \nonumber \\
    &\geq& \Prob{ \Hat{s} > 0 \land \Aint{s,t} \geq \Hint{s,t} \land \Aint{s,t} \geq L } \nonumber \\
    &=& \Prob{ \Hat{s} > 0 } \Prob{ \Aint{s,t} \geq \Hint{s,t} \land \Aint{s,t} \geq L } \nonumber \\
    &\geq& \Prob{ \Hat{s} > 0 } \Prob{ \Hint{s,t} \leq L \land \Aint{s,t} \geq L } \nonumber \\
    &\geqA& \Prob{ \Hat{s} > 0 } \Prob{ \Hint{s,t} \leq L \land \Bint{s,t} \geq 2L } \nonumber \\
    &\geq& \pu \Prob{ \Hint{s,t} \leq L \land 2L \leq \Bint{s,t} \leq  2L (1+2\epsilon) } \nonumber \\
    %
    &\geq& \pu \Prob{ 2L \leq \Bint{s,t} \leq  2L (1+2\epsilon) } \nonumber \\
    && \> \Prob{ \Hint{s,t} \leq L \mid \Bint{s,t} = 2L (1+2\epsilon) } %
    %
\end{IEEEeqnarray}
where (a) is because $\Hint{s,t} + \Aint{s,t} \geq \Bint{s,t}$.
By Chernoff bounds for $\delta\in(0,1)$,
\begin{IEEEeqnarray}{rCl}
    \Prob{ \Bint{s,t} < p(t-s)(1-\delta) } &\leq& \exp\left( -\frac{p(t-s)\delta^2}{2} \right), \IEEEeqnarraynumspace \\
    \Prob{ \Bint{s,t} > p(t-s)(1+\delta) } &\leq& \exp\left( -\frac{p(t-s)\delta^2}{3} \right). \IEEEeqnarraynumspace
\end{IEEEeqnarray}
%
%
%
%
%
where $p \triangleq \Prob{\Hat{t} + \Aat{t} > 0} = 1 - e^{-\blkrateslot}$.
With $t-s = \frac{2L}{p}(1+\epsilon)$ and $\delta = \frac{\epsilon}{1+\epsilon}$,
\begin{IEEEeqnarray}{rCl}
    \Prob{ \Bint{s,t} < 2L } &\leq& \exp\left( \frac{-2L\epsilon^2}{2(1+\epsilon)} \right), \nonumber \\
    \Prob{ \Bint{s,t} > 2L(1+2\epsilon) } &\leq& \exp\left( \frac{-2L\epsilon^2}{3(1+\epsilon)} \right) \nonumber \\
    \Prob{ 2L \leq \Bint{s,t} \leq 2L(1+2\epsilon) } &\geq& 1 - 2\exp\left( \frac{-2L\epsilon^2}{3(1+\epsilon)} \right). \IEEEeqnarraynumspace
\end{IEEEeqnarray}
%
%
%
%
%
Each non-empty time slot ($\Hat{t}+\Aat{t}>0$) is an honest slot ($\Hat{t}>0$) independently with probability $\frac{\pu}{p}$. Therefore conditional on $\Bint{s,t}=2L(1+2\epsilon)$, $\Hint{s,t}$ has a binomial distribution. Then we can use tail bounds for the binomial distribution
%
to show that
\begin{IEEEeqnarray}{rCl}
    && \Prob{ \Hint{s,t} \leq L \mid \Bint{s,t} = 2L (1+\epsilon)} \nonumber \\
    &\geq& \frac{1}{\sqrt{4L(1+2\epsilon)}} \exp\left( -2\constAttackSetup L(1+2\epsilon) \right)
\end{IEEEeqnarray}
where $\constAttackSetup = D\left(\frac{1}{2(1+2\epsilon)} || \frac{\pu}{p} \right)$ and
\begin{IEEEeqnarray}{rCl}
    D(x || y) &=& x \ln\left(\frac{x}{y}\right) + (1-x)\ln\left(\frac{1-x}{1-y}\right).
\end{IEEEeqnarray}
%
%
Putting these together,
\begin{IEEEeqnarray}{rCl}
    \Prob{ F_t } \geq \pu \left(1 - 2e^{\frac{-2L\epsilon^2}{3(1+\epsilon)}} \right) \frac{1}{\sqrt{4L(1+2\epsilon)}} e^{-2\constAttackSetup L(1+2\epsilon)}.   \IEEEeqnarraynumspace
\end{IEEEeqnarray}
%
%
%
%
Since $\epsilon$ is arbitrary, we may choose $\epsilon = \frac{1}{2}$ to get a lower bound on the required probability.
\end{proof}

\begin{corollary}
\label{cor:attack_setup_prob}
    For large $\kappa$, if $L=\Theta(\log \kappa)$ and $\blkrateslot=\Omega\left(\frac{1}{n}\right)$, then $\Prob{F_t} \geq \frac{1}{\poly(\kappa)}$.
\end{corollary}

Recall that the attack goes on forever if the attacker gets one block production opportunity before the honest nodes download a longer chain. We have seen that honest nodes download a longer chain if and only if a non-empty slot is followed by at least $L/\bwtime$ empty time slots.

\begin{definition}
A successful time slot $t$ is called a $T$-loner if no blocks are produced in the $T$ slots following $t$, \ie, $\Bint{t+1,t+T}=0$.
The predicate $\TLoner{T}{t}$ is true iff slot $t$ is a $T$-loner.
\end{definition}
We observe that
\begin{IEEEeqnarray}{c}
    \Prob{\TLoner{T}{t} \mid \Hat{t} + \Aat{t} > 0} = (1-p)^{T} .
\end{IEEEeqnarray}
%

\begin{lemma}
%
\label{lem:tp-attack-t-loner-prob}
If $(1-\beta)e^{-\blkrateslot T} < \beta$, then the probability that the adversary gets one block production opportunity before a $T$-loner occurs is at least $1 - \frac{(1-\beta)
e^{-\blkrateslot T}}{\beta} > 0$.
\end{lemma}
\begin{proof}
We begin by calculating the probability there is at least one attacking slot before there is an $T$-loner. This ensures that the final step of the attack in \cref{sec:tp-attack-strategy} is successful and that the adversary can updates it state and continue the attack.
Let $t_1,t_2,...$ be the sequence of successful slots since the start of the attack. Let $t_N$ be the first $T$-loner in this sequence (note that $N$ is a random variable).
\begin{IEEEeqnarray}{rCl}
    &&\Prob{\exists i \leq N \colon \Aat{t_i} > 0} \nonumber \\
    &=& \sum_{k=1}^{\infty} \Prob{N=k} \Prob{\exists i \leq k \colon \Aat{t_i} > 0 \mid N=k}. \IEEEeqnarraynumspace
\end{IEEEeqnarray}
Here,
\begin{IEEEeqnarray}{rCl}
    \Prob{N=k} &=& \prod_{i=1}^{k-1} \Prob{ \lnot \TLoner{T}{t_i} \mid \Hat{t_i}+\Aat{t_i}>0} \nonumber \\
    && \> \Prob{ \lnot \TLoner{T}{t_k} \mid \Hat{t_k}+\Aat{t_k}>0} \nonumber \\
    &=& \left(1 - (1-p)^{T}\right)^{k-1} (1-p)^T.
\end{IEEEeqnarray}
Moreover, conditioned on $t$ being a successful slot, the events $\TLoner{T}{t}$ and $A_t>0$ are independent. Therefore,
\begin{IEEEeqnarray}{rCl}
    && \Prob{\exists i \leq k \colon \Aat{t_i} > 0 \mid N=k} \nonumber \\
    &=& 1 - \prod_{i=1}^{k} \Prob{\Aat{t_i} = 0 \mid \Aat{t_i}+\Hat{t_i}>0} \nonumber \\
    &=& 1 - \left( \frac{e^{-\beta\blkrateslot}(1-e^{-(1-\beta)\blkrateslot})}{(1-e^{-\blkrateslot})} \right)^k \nonumber \\
    &=& 1 - \left( 1 - \frac{1-e^{-\beta\blkrateslot}}{1-e^{-\blkrateslot}} \right)^k \nonumber \\
    %
    &\geq& 1 - (1-\beta)^k
\end{IEEEeqnarray}
Putting them together,
\begin{IEEEeqnarray}{rCl}
    && \Prob{\exists i \leq N \colon \Aat{t_i} > 0} \\
    &\geq& \sum_{k=1}^{\infty} \left(1 - (1-p)^{T}\right)^{k-1} (1-p)^T \left(1 - (1-\beta)^k \right) \nonumber \\
    &=& 1 - \frac{(1-p)^T(1-\beta)}{1-(1-(1-p)^T)(1-\beta)} \nonumber \\
    &\geq& 1 - \frac{(1-p)^T(1-\beta)}{\beta}.
\end{IEEEeqnarray}
Finally, we substitute $p = 1-e^{-\blkrateslot T}$.
\end{proof}

\begin{lemma}
\label{lem:tp-attack-never-terminates}
If the protocol parameters $\blkrateslot,\slotduration$ satisfy $\frac{\blkrateslot}{\slotduration} > \frac{\bwtime}{L}\log\frac{1-\beta}{\beta}$, then with probability non-negligible in $\kappa$, the attack never terminates.
\end{lemma}
\begin{proof}
From \cref{cor:attack_setup_prob}, for $L=\log(\kappa)$ and large enough $\blkrateslot$, the attack setup occurs with non-negligible probability.

If the adversary gets one block production opportunity before an $L/(\bwtime\slotduration)$-loner, then the adversary can continue the attack by upgrading $L$ to $L+1$. This means that in the next iteration of the attack, the adversary needs one block production opportunity before an $(L+1)/(\bwtime\slotduration)$-loner. Since an $(L+1)/(\bwtime\slotduration)$-loner is rarer than an $L/(\bwtime\slotduration)$-loner, the adversary has increased chances of getting one block production before an $(L+1)/(\bwtime\slotduration)$-loner, and therefore upgrading the attack to $L+2$. This process repeats whereby if the adversary upgrades the attack to the next phase, it increases the chance that the attacker can further upgrade the attack to the next phase, and so forth.

The probability that the attack continues forever is therefore
\begin{IEEEeqnarray}{rCl}
    \Prob{\text{attack continues forever}}
    &\geq& \prod_{l=L}^{\infty} \left( 1 - \frac{(1-\beta) e^{-\frac{\blkrateslot l}{\bwtime\slotduration}}}{\beta} \right) \nonumber \\
    &\geq& \prod_{l=L}^{\infty} \left( 1 - e^{-\frac{\blkrateslot (l-L)}{\bwtime\slotduration}} \right) \nonumber \\
    &=& \prod_{l=1}^{\infty} \left( 1 - e^{-\frac{\blkrateslot l}{\bwtime\slotduration}}\right) \nonumber \\
    &=& \left( e^{-\frac{\blkrateslot}{\bwtime\slotduration}} ; e^{-\frac{\blkrateslot}{\bwtime\slotduration}} \right)_\infty. \label{eq:tp-attack-prob-lb}
\end{IEEEeqnarray}
Here, $\left(x;x\right)_{\infty}$ is called the $q$-Pochhammer symbol and $\left(x;x\right)_{\infty} \in (0,1)$ for all $x\in(0,1)$ \cite{pochhammer}.
The condition $\frac{\blkrateslot}{\slotduration} > \frac{\bwtime}{L}\log\frac{1-\beta}{\beta}$ is derived from the condition in \cref{lem:tp-attack-t-loner-prob} with $T = \frac{L}{\bwtime\slotduration}$.
\end{proof}

%

\begin{corollary}
For the protocol $\protocol$ to satisfy safety and liveness, the throughput of the protocol must be $O\left(\frac{1}{\log\kappa}\right)$.
\end{corollary}
This is seen by noting that the throughput is $\frac{1-e^{-\blkrateslot}}{\slotduration} \leq \frac{\blkrateslot}{\slotduration} \leq \frac{\bwtime}{\log\kappa}\log\frac{1-\beta}{\beta}$. If this is not true, then the attacker never terminates, hence there is a safety violation as per \cref{lem:tp-attack-safety-viol}. The maximum block production rate $\lambda = \frac{\blkrateslot}{\slotduration}$ calculated from \cref{lem:tp-attack-never-terminates} is plotted in \cref{fig:comparison-bddelay-bdbandwidth}(b).

%



%
%
%


%
%
%
%


\end{document}