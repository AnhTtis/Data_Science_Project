\pdfoutput=1
\documentclass[%
reprint, 
superscriptaddress,
%groupedaddress,
%unsortedaddress,
%runinaddress,
%frontmatterverbose, 
%preprint,
%preprintnumbers,
%nofootinbib,
%nobibnotes,
%bibnotes,
 amsmath,amssymb,
 aps,
prl,
%pra,
%prb,
%rmp,
%prstab,
%prstper,
floatfix,
]{revtex4-1}

\usepackage{graphicx}% Include figure files
\graphicspath{{/Users/sebastianwill/Documents/Columbia/Papers/2021-Carbon2/Tex/09-08-2021/}}
\usepackage{epsfig} %Works faster than the graphicx package}
\usepackage{dcolumn}% Align table columns on decimal point
\usepackage{bm}% bold math
\usepackage{epstopdf}
\usepackage{multirow}
\usepackage{amsmath}
\usepackage{booktabs}
\usepackage{color}
\usepackage{gensymb}
%\usepackage{dblfloatfix}
\usepackage{kantlipsum}  
\usepackage{hyperref}
\usepackage{ulem}
\usepackage{braket}
\usepackage{physics}
\usepackage{url}
\usepackage[utf8]{inputenc}
\usepackage[T1]{fontenc}
\usepackage{mathptmx}
\usepackage{lineno}

%\linenumbers
%% New color commands %%%

\newcommand*{\red}{\textcolor{red}} %% SW
\newcommand*{\blue}{\textcolor{blue}} %%
\newcommand*{\green}{\textcolor{green}} %%

\begin{document}
\title{Supplementary Materials for "Collisionally Stable Gas of Bosonic Dipolar Ground State Molecules"}

\author{Niccol\`{o} Bigagli}
\affiliation{Department of Physics, Columbia University, New York, New York 10027, USA}
\author{Claire Warner}
\affiliation{Department of Physics, Columbia University, New York, New York 10027, USA}
\author{Weijun Yuan}
\affiliation{Department of Physics, Columbia University, New York, New York 10027, USA}
\author{Siwei Zhang}
\affiliation{Department of Physics, Columbia University, New York, New York 10027, USA}
\author{Ian Stevenson}
\affiliation{Department of Physics, Columbia University, New York, New York 10027, USA}
\author{Tijs Karman}
\affiliation{Institute for Molecules and Materials, Radboud University, 6525 AJ Nijmegen, Netherlands}
\author{Sebastian Will}
\affiliation{Department of Physics, Columbia University, New York, New York 10027, USA}

\date{\today}

\maketitle


\section{Sample preparation and detection}

First, NaCs Feshbach molecules are assembled from overlapping ultracold gases of Na and Cs via a magnetic field ramp across a Fesh\-bach resonance at $B_\mathrm{res} = 864.1(1)$ G \cite{lam2022high}. The points magnetic field is vertical $z$-direction and sets the quantization axis. The samples are held in a crossed optical dipole trap (ODT) with trap frequencies $\omega / (2 \pi) = (60, \ 65, \ 140)$~Hz (measured for NaCs ground state molecules). The $x$-dipole trap is elliptical and focused to waists of 127(5)~$\mu$m (horizontal) and 56(3)~$\mu$m (vertical); the $y$-dipole trap is circular with a waist of 106(5)~$\mu$m. Optical trapping light is provided by a 1064 nm narrow-line single-mode Nd:YAG laser (Coherent Mephisto MOPA).

Then, NaCs Feshbach molecules are transferred to their electronic, vibrational, and rotational ground state $X^1\Sigma^+ \ket{v = 0, \ J = 0}$ via stimulated Raman adiabatic passage (STIRAP) \cite{stevenson2023ultracold, warner2023pathway}. The specific hyperfine state of the molecules is $\ket{m_{I_{\rm Na}}, \ m_{I_{\rm Cs}}} = \ket{3/2,\ 5/2}$, where $m_{I_{\rm Na}}$ $(m_{I_{\rm Cs}})$ is the projection of the nuclear spin of sodium (cesium) onto the quantization axis. For STIRAP, mutually coherent lasers at 937~nm (generated by a Ti:Sapphire laser) and 642~nm (generated by an external-cavity-diode-laser) are used. Both lasers are stabilized to the resonance of an ultralow-expansion (ULE) glass cavity with a finesse of 22,000; their absolute frequency is stabilized to better than 1~kHz. The STIRAP lasers propagate vertically upwards in our setup with $\sigma^+$ polarization. The 937~nm beam is focused to a waist of 165~$\mu$m and has 150~mW of optical power.  The 642~nm beam is focused to a waist of 265~$\mu$m and has 2~mW of power. The one-photon detuning of the STIRAP lasers is +80~MHz from the intermediate state $c \ ^3\Sigma^+_1  \ket{v = 22, \ J = 1, \ m_J = +1}$ state~\cite{warner2023pathway}.   
The one-way STIRAP transfer efficiency is $74(3)\%$. Due to the vertical STIRAP beams, NaCs molecules can undergo time-of-flight expansion while in the ground state and in the presence of microwave shielding, which is important for precise thermometry of the molecular gases (see below). 

After time-of-flight, NaCs molecules are detected by reversing the STIRAP process, optically dissociating them with a pulse of light that is resonant on the Cs $6^2 S_{1/2} \ket{F = 3, \ m_F = 3} \rightarrow 6^2 P_{3/2} \ket{F = 4, \ m_F = 4}$ at high magnetic field, immediately followed by absorption imaging of Cs atoms on the $6^2 S_{1/2} \ket{F = 4, \ m_F = 4} \rightarrow 6^2 P_{3/2} \ket{F = 5, \ m_F = 5}$ transition at high field.

\section{Temperature measurement via Time-of-flight}

\begin{figure*} 
    \centering
    \includegraphics[width = 14.2 cm]{SI_Figure_1.pdf}\\
    \caption{Time-of-flight expansion of a gas of ground state molecules with shielding at $\Omega/(2\pi) = 4$ MHz, $\Delta/(2\pi) = 6$ MHz (A) and without shielding (B). Upper (lower) panels correspond to the $x$ ($y$) direction. Insets in the lower panels illustrate the respective experimental sequence. The fitted mean temperature is 290(10) nK for A and 330(20) nK for B. }
    \label{fig:SI1}
\end{figure*} 

The precise measurement of temperature of NaCs ground state molecular gases is critical in this work. We rely on thermometry using time-of-flight expansion, which has been well-established for ultracold atoms. By fitting the increase of cloud radius due to ballistic expansion, the in-trap temperature can be extracted \cite{ketterle1999making}. Using time-of-flight expansion for molecular thermometry requires careful consideration. Molecules cannot be imaged directly and need to be dissociated prior to imaging. There are several possible approaches: (1) Dissociation of molecules into atoms prior to time-of-flight expansion. Here, an energy offset is incurred from the reverse Feshbach ramp. We have quantified this effect in earlier work, but this approach is prone to systematic shifts, especially at cold temperatures \cite{lam2022high}. (2) Reverse STIRAP to Feshbach molecules, time-of-flight expansion, followed by optical dissociation of Feshbach molecules and immediate absorption imaging. This method suffers from a momentum kick provided by reverse STIRAP that distorts the cloud expansion. (3) Time-of-flight expansion of ground state molecules, followed by reverse STIRAP and dissociation right before absorption imaging. For unshielded molecules, two-body loss rates approach $10^{-9}$ cm$^3$/s, which leads to significant loss and heating in the initial phase of time of flight expansion. (4) Identical to (3) with microwave shielding switched on during time-of-flight. Here, losses are minimized. %For small $\Delta$ of the microwave field, leading to a $d_\mathrm{eff}$ up to 1.3~D, dipolar interactions may be significant especially for cold and dense molecular clouds. However, we do not believe this effect to be significant for the bulk of our data as temperatures are far above the dipolar interaction energy scale, which is only few 10s of nK even for small $\Delta$. 
The impact of dipolar interaction energy, especially in the initial phase of the time-of-flight expansion, may be a concern. We find that dipolar interaction energies are well below 500 Hz $\approx 25$ nK for our parameters and therefore the impact on the temperature measurement should be negligible. 

We realize approach (4) utilizing vertical STIRAP beams (i.e.,~collinear with the direction of gravity), allowing reverse transfer of ground state molecules into the Feshbach state after time-of-flight expansion to up to 50~$\mu$m cloud size, corresponding to about 7~ms time-of-flight at 750~nK. Absorption images record the column-density of the molecular cloud. For extract the cloud size in the $x$-direction, we integrate along the $y$-direction, fit the profile to a 1D Gaussian, $n(x) = A e^{-x^2 / (2 \sigma_x^2)}$, and extract the $\sigma_{x}$-radius at each $t_{\rm TOF}$. For the $y$-direction, an analogous procedure is followed. To obtain the temperature, $T$, from the extracted cloud sizes, we fit the data by $\sigma_{x} (t_{\rm TOF}) = \sqrt{ \sigma_0^2 + (k_B T/ M) t_{\rm TOF}^2   }$, where $\sigma_0$ is the initial cloud radius, $k_B$ is the Boltzmann constant, and $M$ is the mass of the molecule. In Fig.~\ref{fig:SI1}, 
we compare the systematic difference between approach (3) and (4) for otherwise identically-prepared clouds. 

\section{Quantum numbers}

We denote the relevant quantum states in this work by quantum number $J$, the total angular momentum of the molecules excluding nuclear spin, and its projection $m_J$ onto the quantization axis. Due to the high magnetic field in the vicinity of the Feshbach resonance at $B_\mathrm{res}$, the molecules are in the Paschen-Back regime and nuclear spin is decoupled from all other angular momenta. Therefore, we omit nuclear spin quantum numbers. The nuclear spin stays unchanged at $\ket{m_{I_{\rm Na}}, \ m_{I_{\rm Cs}}} = \ket{3/2,\ 5/2}$, even in the presence of the microwave field (see below).

\section{Microwave antenna}

Circularly polarized microwave fields are generated with a phased-array microwave antenna. The antenna consists of four individual loops, arranged in star-shape, that are one-wavelength resonant for a frequency of 3.5~GHz. Each loop is fed by a 15~W radio-frequency amplifier (MiniCircuits ZHL-15W-422). The frequency is generated with an ultralow noise signal generator (Rohde~\&~Schwarz SMA100B) that is split into four channels via a power splitter.  Each channel is given a differential phase shift of 90$\degree$ to generate $\sigma^+$ microwave polarization.  Before the power splitter, a voltage controlled attenuator (General Microwave D1954) allows control of microwave power and a stack of three pin-diode switches provides a 135~dB suppression of the source when off. Additional details of the microwave system are described in Ref.~\cite{yuan2023circular}. 

\section{Rotational spectroscopy}

In order to identify the $\sigma^+$ transition for microwave shielding, we performed rotational spectroscopy of NaCs. We prepared a gas of ground state molecules in the $\ket{J, \ m_J} = \ket{0, \ 0}$ state and exposed it to a microwave field with mixed polarization while the magnetic bias field was close to $B_\mathrm{res}$. The corresponding spectrum is shown in Fig.~\ref{fig:SI2}. To assign quantum numbers to each transition, we repeated the measurement with polarized microwave fields, showing a single transition depending on the polarization used. The resonance frequency of the $\sigma^+$ transition between $\ket{0, \ 0}$ and $\ket{1, \ +1}$ was measured to be 3.471323(2) GHz.  

\begin{figure} 
    \centering
    \includegraphics[width = 8.6 cm]{SI_Figure_2.pdf}\\
    \caption{Spectrum of the $|J=0\rangle \rightarrow |J=1\rangle$ rotational transitions of NaCs using a microwave field with mixed polarization. Different polarization couples different $m_J$ states, as marked. The vertical dashed lines indicate the assigned transitions.}
    \label{fig:SI2}
\end{figure} 

\section{Dressed state preparation}

The molecules are prepared in the dressed state $|+\rangle$ via an adiabatic increase of the intensity of the blue-detuned microwave field. The intensity is ramped up within $40$ $\mu$s using a ramp following a quadratic power law, $\Omega(t) \propto t^2$. To confirm adiabaticity of this ramp, we compared the molecule number in state $\ket{J, \ m_J, \ m_{I_{\rm Na}}, \ m_{I_{\rm Cs}}} = \ket{0, \ 0, \ 3/2, \ 5/2}$ before the ramp, $N_i$, to the molecule number $N_f$ after a microwave ramp into and out of state $|+\rangle$. For a non-adiabatic ramp, we would expect loss of population into other states and $N_f/N_i$ should be smaller than 1. We performed this measurement for microwave Rabi frequency $\Omega/(2 \pi)=4$ MHz and various detunings $\Delta$, as shown in Fig.~\ref{fig:SI3}. While the data point at the lowest detuning does not meet this criterion, all data with $\Delta/\Omega > 0.1$, which is the case for all data in the main text, fulfills the criterion. 

\begin{figure} 
    \centering
    \includegraphics[width = 8.6 cm]{SI_Figure_3.pdf}\\
    \caption{Adiabaticity of dressed state preparation for $\Omega / (2\pi) = 4$ MHz. The ratios of molecule numbers before ($N_i$) and after ($N_i$) a round trip through the dressed state $|+\rangle$ is shown. The inset shows a schematic of the experimental sequence.}
    \label{fig:SI3}
\end{figure} 

\section{Ellipticity}

We measured the ellipticity of the microwave field by driving resonant Rabi oscillations on the transitions $\ket{J, \ m_J } = \ket{ 0, \ 0} $ to $\ket{ 1, \ -1 }$ and $\ket{ 1, \ +1 }$. For fixed microwave power, we determine the respective Rabi frequencies and obtain an ellipticity $\xi = \arctan(\Omega_{\sigma^-} / \Omega_{\sigma^+}) = 3(2) \degree$. An attempt to drive a resonant transition to $\ket{ 1, \ 0 }$ under identical conditions was consistent with no Rabi coupling. We note that the axis of molecular rotation is well aligned with the axis of the magnetic field.

\section{One-body loss}

\begin{figure} 
    \centering
    \includegraphics[width = 8.6 cm]{SI_Figure_4.pdf}\\
    \caption{Measured one-body-limited lifetime of the shielded NaCs gas for the same Rabi frequency $\Omega / 2\pi = 4$ MHz and $\Delta/(2\pi) = 4$ MHz using different levels of attenuation of the 15 W amplifiers.}
    \label{fig:SI4}
\end{figure}

The dominant source of one-body loss in the microwave-shielded molecular gas stems from noise of the microwave field away from the carrier frequency. Such noise can drive transitions to the anti-shielded dressed state $|-\rangle$ and unshielded spectator states $\ket{0}$, limiting the lifetime of the molecular gas. Such noise can be generated or amplified by any active component of the microwave chain, e.g.~the microwave source and amplifiers. To reduce this noise we employ a high quality microwave source (Rohde \& Schwarz SMA100B with ultralow phase-noise option). We find that the phase-noise of the source is so low that it does not limit the observed one-body loss. Instead, we find thermal noise for the fixed-gain microwave amplifiers to contribute dominantly. To reduce this noise, we set the output power of the microwave source to maximum and attenuate the amplifiers' output to the desired power level via external attenuators. We have measured one-body loss at constant overall output power, corresponding to  $\Omega / 2\pi = 4$ MHz, for different combinations of source power level and amplifier attenuation (Fig.~\ref{fig:SI4}). For the largest attenuation of 16 dB, we find the longest one-body-limited lifetime of $\tau_{1\text{B}} \sim 4.4(4)$ s. 

\section{Optimum Rabi Frequency}

\begin{figure} 
    \centering
    \includegraphics[width = 8.6 cm]{SI_Figure_5.pdf}\\
    \caption{$1/e$-lifetime of microwave-shielded molecules as a function of $\Omega$. The ratio $\Delta / \Omega = 1.5$ is kept fixed.}
    \label{fig:SI5}
\end{figure} 

All the data of this work was taken for a Rabi frequency of $\Omega / 2\pi = 4$ MHz. For this Rabi frequency we were able to make use of the full 16 dB post-amplifier attenuation to limit one-body loss from microwave noise (see above). Fig.~\ref{fig:SI5} shows the measured $1/e$-lifetimes of the molecular cloud for different Rabi frequencies $\Omega$, while keeping the ratio $\Delta/\Omega = 1.5$ fixed. To access higher Rabi frequencies the post-amplifier attentuation was gradually reduced, which reduced the lifetimes due to higher microwave noise. Theoretically, higher Rabi frequency is expected to lead to better shielding than lower Rabi frequency. Within the technical limitations of the experimental setup, we observe peak shielding performance for $\Omega / 2\pi \sim 4$ MHz.

\section{Kinetic Model}

In order to extract elastic and inelastic loss rates from lifetime data of the shielded molecular gases, we employ a fitting model that includes one-body, two-body, and evaporative losses. The following coupled differential equations describe the rate of change of molecule number and energy in the molecular gas  \cite{warner2021overlapping}:
\begin{align}
\dot{N} & = \dot{N}_\text{1B} + \dot{N}_\text{2B} + \dot{N}_\text{ev}  \\
\dot{E} & = \dot{E}_\text{1B} + \dot{E}_\text{2B} + \dot{E}_\text{ev}.
\end{align}
In our formalism, the total energy of the gas is $E = 3 N k_B T$.  The one-body terms take the usual form $\dot{N}_\text{1B} = - N / \tau_\text{1B}$ and $\dot{E}_\text{1B} = - E / \tau_\text{1B}$, where $\tau_\text{1B}$ is the one-body lifetime. $\tau_\text{1B}$ is measured directly by observing low density loss curves in which other losses are negligible. The quantity $\tau_\text{1B}$ is kept fixed at the measured value. 

The two-body term in the number differential equation is given by \cite{olson2013optimizing} $ \dot{N}_\text{2B}  = -\beta_\text{2B} \bar{n} N $. Here, $\beta_\text{2B}$ is the two-body loss rate coefficient and $\bar{n}$ the average density of the molecular cloud. The average density is related to the peak density, $n_0$, by $\bar{n} = n_0 / (2 \sqrt{2})$, with $n_0 = N \left(\bar{\omega}^2M/(2\pi k_B T)\right)^{3/2}$, $\bar{\omega} = (\omega_x \omega_y \omega_z)^{1/3}$ the mean trap frequency, and $M$ the molecular mass. The two-body loss contribution to the energy differential equation is given by $ \dot{E}_\text{2B}  = - (3/4)\beta_\text{2B} \bar{n} E$.  The $(3/4)$ prefactor comes from integrating the product of the energy density and the number density over the volume of the cloud.

The effects of evaporation are included via the term $\dot{N}_\text{ev} = -N \nu(\eta) \Gamma_{\rm el} / N_{\rm col} $  \cite{luiten1996kinetic, olson2013optimizing}. Here, $\nu(\eta)$ is the fraction of elastically scattered molecules with kinetic energy higher than the trap depth and $\Gamma_{\rm el} / N_{\rm col}$ is the thermalization rate.  $\Gamma_{\rm el}$ is the elastic scattering rate, $N_{\rm col}$ is the number of collisions to produce a $1/e$ change in the molecule temperature, $\eta = U_\text{min} / ( k_B T) $ is the truncation parameter and $U_\text{min}$ is the trap depth. From Ref.~\cite{davis1995analytical}, $\nu(\eta) = (2 + 2 \eta + \eta^2) / (2 e^{\eta})$.  The elastic scattering rate is $\Gamma_{\rm el} = \bar{n} \sigma_\text{el} v_{\rm th}$, where $\sigma_\text{el}$ is the elastic scattering cross-section, and $v_{\rm th} = 4\sqrt{k_B T / (\pi M)}$ is the thermal velocity. In our fitting routine, we cap $\Gamma_{\rm el} < \bar{\omega} / ( 2 \pi )$, in order to account for the hydrodynmaic limit.  Because our gas is highly anisotropic, $N_{\rm col}$ is not a number, but rather a matrix accounting for the number of collisions for thermalization for every pair of trap axes, i.e. $N_{\rm col}^{\rm xx}$, $N_{\rm col}^{\rm xy}$, $N_{\rm col}^{\rm xz}$, ect.  In our fitting routine, we fit the product $\sigma_\text{el} / N_{\rm col}$ and then use the calculated maximum $N_{\rm col}$ element to extract $\sigma_\text{el}$.

The evaporative term in the energy differential equation is $\dot{E}_\text{ev} = - (1/3) E \alpha(\eta) \Gamma_{\rm el} / N_{\rm col}$, where $\alpha(\eta) = (6 + 6\eta + 3\eta^2 + \eta^3) / (2 e^{\eta})$~\cite{davis1995analytical}.  $(1/3) E \alpha(\eta)$ is the energy of the molecules with kinetic energy larger than the trap depth and $\Gamma_{\rm el} / N_{\rm col}$ is the rate at which the energy will leave the system.

We can recast the differential equations in the form:
\begin{align}
\dot{N} & =  -N [ 1 / \tau_\text{1B} + \beta_{\rm 2B} \bar{n} + \nu(\eta) \Gamma_{\rm ev} \bar{n} ] \\
\dot{E} & = -E [ 1 / \tau_\text{1B} + (3/4) \beta_{\rm 2B} \bar{n} + (1/3) \alpha(\eta) \Gamma_{\rm ev} \bar{n} ],
\label{eq:de}
\end{align}
where we used $\Gamma_{\rm ev} = \sigma_\text{el} v_{\rm th} / N_\text{col}$ for clarity. 

Experimentally, we measure the number and temperature of the molecular cloud as a function of hold time. Then the data is fitted with this model to extract the initial number, $N_0$, the initial temperature, $T_0$, $\beta_{\rm 2B}$ and $\sigma_\text{el} / N_{\rm col}$.  In practice, we first fit cross-thermalization data, as shown in Fig.~3 in the main text, to obtain $\sigma_\text{el} / N_{\rm col}$; then, we fit lifetime data, as shown in Fig.~2 in the main text, using $\sigma_\text{el} / N_{\rm col}$ from the first fit.
   
\section{Coupled-channel calculations}

We perform coupled-channel scattering calculations as described in Refs.~\cite{karman2018microwave,karman2019microwave}. Here, we give a brief description of the numerical details.

Our coupled-channels calculations describe the collision between two molecules that interact with a magnetic field, an elliptical microwave field~\cite{karman2019microwave}, and with each other through the dipole-dipole interaction.
The NaCs molecules are described as rigid rotors with $J=0,1$.
The end-over-end rotation of the molecules about one another is described by a partial wave expansion up to $L=12$.
Coupling to non-initial hyperfine states is neglected, after exploratory calculations that found no effect of hyperfine structure.
The solutions to the coupled-channels equations are propagated numerically for internuclear distances between 50 and $10^5~a_0$.
At short range, an absorbing boundary condition is imposed that models loss that occurs when the molecules come close together~\cite{karman2018microwave}.
The calculations are repeated for 31 logarithmically-spaced collision energies between 1~nK and 10~$\mu$K.
From the energy dependent cross sections we determine thermally averaged elastic and inelastic scattering rates.

In addition to overall elastic and inelastic collision rates, averaged over all incoming directions in a thermal gas, we also compute the differential cross section for elastic scattering
\begin{align}
\frac{d\sigma}{d\Omega}(\bm{k},\bm{k}') = \frac{4\pi^2}{k^2} \left| \sum_{L,M_L,L',M_L'} i^{L-L'} Y_{L',M'}(\bm{k}') T_{L',M_L';L,M_L} Y_{L,M}^\ast(\bm{k}) \right|^2
\end{align}
where $\bm{k}$ and $\bm{k}'$ are the initial and final wavenumber,
$Y_{L,M}(\bm{k})$ is a spherical harmonic depending on the polar angles of $\bm{k}$,
and the $T$-matrix elements are obtained from our coupled-channels calculations.
In the next section, these elastic cross sections are used to model the rate of thermalization in our dipolar gas.

\section{Dipolar thermalization}

We follow earlier work \cite{wang2021anisotropic,gueryodelin1999collective} on thermalization in a harmonically confined ultracold gas and derive equations of motion by computing moments of the Boltzmann equation
\begin{align}
\frac{d \langle q_j^2\rangle}{dt} - \frac{2}{M} \langle q_j p_j\rangle &= 0, \nonumber \\
\frac{d \langle q_j p_j\rangle}{dt} - \frac{1}{M} \langle p_j^2\rangle + M\omega_j^2 \langle q_j^2\rangle  &= 0, \nonumber \\
\frac{d \langle p_j^2\rangle}{dt} + 2M\omega_j^2 \langle q_jp_j\rangle &= \mathcal{C}[\Delta p_j^2],
\label{eq:EOM}
\end{align}
that is, equations of motion for the nine dynamical properties $\{ \langle x^2 \rangle, \langle x p_x \rangle, \langle p_x^2 \rangle, \langle y^2 \rangle, \langle y p_y \rangle, \langle p_y^2 \rangle, \langle z^2 \rangle, \langle z p_z \rangle, \langle p_z^2 \rangle \}$.
Collisions are described by the term
\begin{align}
\mathcal{C}[\Delta p_i^2] &= \mathcal{C}_{ix} \langle p_x^2\rangle + \mathcal{C}_{iy} \langle p_y^2\rangle + \mathcal{C}_{iz} \langle p_z^2\rangle, \nonumber \\
\mathcal{C}_{ij} &= -\frac{\bar{n}}{(Mk_BT)^2} \int d\bm{k}\ k\ c^\mathrm{eq}(k) \int d^2\Omega\ \frac{d\sigma}{d\Omega}\ \Delta\bm{k}_{i}^2\ \Delta\bm{k}_{j}^2,
\label{eq:colintfinalij}
\end{align}
where $\bm{k}$ is the relative momentum,
$\Delta\bm{k}_{i}$ is the $i$ Cartesian component of the change in momentum,
and
\begin{align}
c^\mathrm{eq}(k) &= \frac{1}{\left( \pi M k_B T \right)^{3/2}} \exp\left(-\frac{k^2}{M k_B T}\right)
\end{align}
is the thermal distribution of relative momenta.

Thermalization has been studied previously for $s$-wave collisions \cite{monroe1993measurement,wu1996direct,gueryodelin1999collective},
which results in an energy-independent isotropic cross section,
and for threshold dipolar interactions \cite{wang2021anisotropic}, which results in an energy-independent but anisotropic cross section that is known analytically \cite{bohn14differential}.
The threshold dipolar results describe dipolar gases with $k_B T \ll E_\mathrm{d}$.
This applies to ultracold magnetic atoms, but not necessarily to a strongly dipolar molecular gas.
For resonant dressing of NaCs the dipolar length can be as large as 40\,000~$a_0$, and the dipolar energy scale as low as 0.7~nK.
To describe thermalization in a strongly dipolar molecular gas, we here evaluate Eq.~\eqref{eq:colintfinalij} numerically using elastic differential cross sections from our coupled-channels calculations.

To determine the rate of thermalization we define pseudo-temperatures $T_i = [\langle p_i^2\rangle + M^2 \omega_i^2 \langle x^2\rangle] / 2 M k_B$, for each cartesian direction,
and an equilibrium temperature $T_\mathrm{eq} = (T_x+T_y+T_z) / 3$.
Then, at short times we have
\begin{align}
\frac{\partial \langle p_i^2\rangle}{\partial t} &= \mathcal{C}_{ix} \langle p_x^2\rangle + \mathcal{C}_{iy} \langle p_y^2\rangle + \mathcal{C}_{iz} \langle p_z^2\rangle.
\end{align}
If we bring the pseudo-temperature in the $j$ direction out of equilibrium, the pseudo-temperature in $i$ direction responds as
\begin{align}
\frac{\partial T_i}{\partial t} &= \frac{3}{2} \mathcal{C}_{ij} \left[ T_j - T_i\right],
\end{align}
where we used $\mathcal{C}_{ix}+\mathcal{C}_{iy}+\mathcal{C}_{iz}=0$.
Thus, at short times, the pseudo-temperatures approach equilibrium exponentially with time constant $k_{ij} = \frac{3}{2} \mathcal{C}_{ij}$.
If the collision rates, $\mathcal{C}_{ij}$, become comparable to the trap frequencies, the short-time approximation breaks down, and we instead determine the 1/$e$ thermalization time by a full simulation of the equations of motion Eqs.~\eqref{eq:EOM} as described in Ref.~\cite{wang2021anisotropic}.

Since an overall scaling of the elastic cross section will increase the rate of both thermalization and elastic collisions,
the effectiveness of thermalization is often characterized by their ratio
\begin{align}
N_\mathrm{col}^{ij} = \frac{\bar{n} \langle v_\mathrm{th} \sigma_\mathrm{el} \rangle}{k_{ij}},
\end{align}
known as the number of elastic collisions per thermalization.
For $s$-wave collisions, this ratio is $N_\mathrm{col} = 5/2$,
whereas it has been shown that threshold dipolar collisions can lead to a smaller value of $N_\mathrm{col}$,
and indeed it has been observed for fermionic microwave-shielded NaK that $N_\mathrm{col}$ is between 1 and 2 depending on the microwave polarization \cite{schindewolf2022evaporation,chen2023field}.
We note that these measurements for NaK were in fact in the threshold regime,
and the semiclassical behavior is masked by the hydrodynamic thermalization \cite{schindewolf2022evaporation,chen2023field}.

For strongly dipolar bosonic NaCs molecules, we find substantially different behavior of $N_\mathrm{col}$, as seen in Fig.~3C of the main text.
The effect of dipolar collisions is large and can increase the number of collisions for thermalization by almost an order of magnitude above the bare $s$-wave result of $N_\mathrm{col} = 5/2$.
The increase of $N_\mathrm{col}$ results from two effects.
First, in the semi-classical regime the dipolar elastic cross section decreases as $1/\sqrt{E}$,
which emphasizes low-energy collisions that lead to less momentum transfer.
Second, in the semi-classical regime the cross section also becomes more forward scattered,
which further reduces the amount of momentum transferred.
The dipolar thermalization is also strongly anisotropic, leading to substantially different thermalization rates for collisions in and perpendicular to the plane of the microwave polarization, $N_\mathrm{col}^{xy}$ and $N_\mathrm{col}^{xz}$.

\section{s-wave scattering length}

\begin{figure} 
    \centering
    \includegraphics[width = 8.6 cm]{SI_Figure_6.pdf}\\
    \caption{Fitting an $s$-wave scattering length.  \textbf{A}, Comparison of experimentally extracted collision cross-section, assuming $N_{\rm col} = 1$, to the calculated ratio of the elastic cross-section and $N_{\rm col}^{\rm xz}$.  Solid line corresponds to $a_s = 1200$~$a_0$ while the dashed line is obtained directly from our coupled-channels calculations.  \textbf{B}, Comparison of calculated $N_{\rm col}^{\rm xz}$ for $a_s = 1200$~$a_0$, orange solid line, and $a_s = \bar{a}$, orange dashed line.  The black dashed line marks $N_{\rm col} = 2.5$}
    \label{fig:SI6}
\end{figure} 

For parameters in which our bosonic molecules are in the threshold scattering regime, $k_\mathrm{B} T \lesssim E_\text{d}$, the scattering cross-section is given by the sum of its dipolar and $s$-wave contributions, $32 \pi \, a_\text{d}^2/ 45 + 8 \pi a_s^2$ \cite{bohn2009quasi}. While the dipolar part is well-defined by the microwave parameters, the $s$-wave scattering length of the NaCs molecules is not known a priori. Neither the scattering length from our coupled-channels calculations nor the universal $s$-wave scattering length, $a_s = (1-i)\bar{a}$ where $\bar{a}= [2 \pi / \Gamma(1/4)^2] (M C_6 / \hbar^2)^{1/4} \approx 580(30)$~$a_0$~\cite{julienne2011universal}, reproduce the measured elastic scattering cross section at large detunings (corresponding to $a_\mathrm{d} \approx 0$), as shown in  Fig.~\ref{fig:SI6}A.  
We directly add an $s$-wave contribution to the scattering matrix, which we use to fit the observed behavior at large detuning.  Our fit best agrees with the experiment for $a_s = 1200$~$a_0$.  The $s$-wave contribution affects both the scattering cross-section (dominantly at large detunings), and the anisotropy of the collisions, as shown in Fig.~\ref{fig:SI6}B.

%merlin.mbs apsrev4-1.bst 2010-07-25 4.21a (PWD, AO, DPC) hacked
%Control: key (0)
%Control: author (8) initials jnrlst
%Control: editor formatted (1) identically to author
%Control: production of article title (-1) disabled
%Control: page (0) single
%Control: year (1) truncated
%Control: production of eprint (0) enabled
\begin{thebibliography}{20}%
\makeatletter
\providecommand \@ifxundefined [1]{%
 \@ifx{#1\undefined}
}%
\providecommand \@ifnum [1]{%
 \ifnum #1\expandafter \@firstoftwo
 \else \expandafter \@secondoftwo
 \fi
}%
\providecommand \@ifx [1]{%
 \ifx #1\expandafter \@firstoftwo
 \else \expandafter \@secondoftwo
 \fi
}%
\providecommand \natexlab [1]{#1}%
\providecommand \enquote  [1]{``#1''}%
\providecommand \bibnamefont  [1]{#1}%
\providecommand \bibfnamefont [1]{#1}%
\providecommand \citenamefont [1]{#1}%
\providecommand \href@noop [0]{\@secondoftwo}%
\providecommand \href [0]{\begingroup \@sanitize@url \@href}%
\providecommand \@href[1]{\@@startlink{#1}\@@href}%
\providecommand \@@href[1]{\endgroup#1\@@endlink}%
\providecommand \@sanitize@url [0]{\catcode `\\12\catcode `\$12\catcode
  `\&12\catcode `\#12\catcode `\^12\catcode `\_12\catcode `\%12\relax}%
\providecommand \@@startlink[1]{}%
\providecommand \@@endlink[0]{}%
\providecommand \url  [0]{\begingroup\@sanitize@url \@url }%
\providecommand \@url [1]{\endgroup\@href {#1}{\urlprefix }}%
\providecommand \urlprefix  [0]{URL }%
\providecommand \Eprint [0]{\href }%
\providecommand \doibase [0]{http://dx.doi.org/}%
\providecommand \selectlanguage [0]{\@gobble}%
\providecommand \bibinfo  [0]{\@secondoftwo}%
\providecommand \bibfield  [0]{\@secondoftwo}%
\providecommand \translation [1]{[#1]}%
\providecommand \BibitemOpen [0]{}%
\providecommand \bibitemStop [0]{}%
\providecommand \bibitemNoStop [0]{.\EOS\space}%
\providecommand \EOS [0]{\spacefactor3000\relax}%
\providecommand \BibitemShut  [1]{\csname bibitem#1\endcsname}%
\let\auto@bib@innerbib\@empty
%</preamble>
\bibitem [{\citenamefont {Lam}\ \emph {et~al.}(2022)\citenamefont {Lam},
  \citenamefont {Bigagli}, \citenamefont {Warner}, \citenamefont {Yuan},
  \citenamefont {Zhang}, \citenamefont {Tiemann}, \citenamefont {Stevenson},\
  and\ \citenamefont {Will}}]{lam2022high}%
  \BibitemOpen
  \bibfield  {author} {\bibinfo {author} {\bibfnamefont {A.~Z.}\ \bibnamefont
  {Lam}}, \bibinfo {author} {\bibfnamefont {N.}~\bibnamefont {Bigagli}},
  \bibinfo {author} {\bibfnamefont {C.}~\bibnamefont {Warner}}, \bibinfo
  {author} {\bibfnamefont {W.}~\bibnamefont {Yuan}}, \bibinfo {author}
  {\bibfnamefont {S.}~\bibnamefont {Zhang}}, \bibinfo {author} {\bibfnamefont
  {E.}~\bibnamefont {Tiemann}}, \bibinfo {author} {\bibfnamefont
  {I.}~\bibnamefont {Stevenson}}, \ and\ \bibinfo {author} {\bibfnamefont
  {S.}~\bibnamefont {Will}},\ }\href@noop {} {\bibfield  {journal} {\bibinfo
  {journal} {Phys. Rev. Res.}\ }\textbf {\bibinfo {volume} {4}},\ \bibinfo
  {pages} {L022019} (\bibinfo {year} {2022})}\BibitemShut {NoStop}%
\bibitem [{\citenamefont {Stevenson}\ \emph {et~al.}(2023)\citenamefont
  {Stevenson}, \citenamefont {Lam}, \citenamefont {Bigagli}, \citenamefont
  {Warner}, \citenamefont {Yuan}, \citenamefont {Zhang},\ and\ \citenamefont
  {Will}}]{stevenson2023ultracold}%
  \BibitemOpen
  \bibfield  {author} {\bibinfo {author} {\bibfnamefont {I.}~\bibnamefont
  {Stevenson}}, \bibinfo {author} {\bibfnamefont {A.~Z.}\ \bibnamefont {Lam}},
  \bibinfo {author} {\bibfnamefont {N.}~\bibnamefont {Bigagli}}, \bibinfo
  {author} {\bibfnamefont {C.}~\bibnamefont {Warner}}, \bibinfo {author}
  {\bibfnamefont {W.}~\bibnamefont {Yuan}}, \bibinfo {author} {\bibfnamefont
  {S.}~\bibnamefont {Zhang}}, \ and\ \bibinfo {author} {\bibfnamefont
  {S.}~\bibnamefont {Will}},\ }\href@noop {} {\bibfield  {journal} {\bibinfo
  {journal} {Phys. Rev. Lett.}\ }\textbf {\bibinfo {volume} {130}},\ \bibinfo
  {pages} {113022} (\bibinfo {year} {2023})}\BibitemShut {NoStop}%
\bibitem [{\citenamefont {Warner}\ \emph {et~al.}(2023)\citenamefont {Warner},
  \citenamefont {Bigagli}, \citenamefont {Lam}, \citenamefont {Yuan},
  \citenamefont {Zhang}, \citenamefont {Stevenson},\ and\ \citenamefont
  {Will}}]{warner2023pathway}%
  \BibitemOpen
  \bibfield  {author} {\bibinfo {author} {\bibfnamefont {C.}~\bibnamefont
  {Warner}}, \bibinfo {author} {\bibfnamefont {N.}~\bibnamefont {Bigagli}},
  \bibinfo {author} {\bibfnamefont {A.~Z.}\ \bibnamefont {Lam}}, \bibinfo
  {author} {\bibfnamefont {W.}~\bibnamefont {Yuan}}, \bibinfo {author}
  {\bibfnamefont {S.}~\bibnamefont {Zhang}}, \bibinfo {author} {\bibfnamefont
  {I.}~\bibnamefont {Stevenson}}, \ and\ \bibinfo {author} {\bibfnamefont
  {S.}~\bibnamefont {Will}},\ }\href@noop {} {\bibfield  {journal} {\bibinfo
  {journal} {arXiv:2302.12293}\ } (\bibinfo {year} {2023})}\BibitemShut
  {NoStop}%
\bibitem [{\citenamefont {Ketterle}\ \emph {et~al.}(1999)\citenamefont
  {Ketterle}, \citenamefont {Durfee},\ and\ \citenamefont
  {Stamper-Kurn}}]{ketterle1999making}%
  \BibitemOpen
  \bibfield  {author} {\bibinfo {author} {\bibfnamefont {W.}~\bibnamefont
  {Ketterle}}, \bibinfo {author} {\bibfnamefont {D.~S.}\ \bibnamefont
  {Durfee}}, \ and\ \bibinfo {author} {\bibfnamefont {D.}~\bibnamefont
  {Stamper-Kurn}},\ }\href@noop {} {\bibfield  {journal} {\bibinfo  {journal}
  {Proceedings of the International School of Physics "Enrico Fermi", Course
  CXL, p.~67-176}\ } (\bibinfo {year} {1999})}\BibitemShut {NoStop}%
\bibitem [{\citenamefont {Yuan}\ \emph {et~al.}(2023)\citenamefont {Yuan},
  \citenamefont {Zhang}, \citenamefont {Bigagli}, \citenamefont {Warner},
  \citenamefont {Stevenson},\ and\ \citenamefont {Will}}]{yuan2023circular}%
  \BibitemOpen
  \bibfield  {author} {\bibinfo {author} {\bibfnamefont {W.}~\bibnamefont
  {Yuan}}, \bibinfo {author} {\bibfnamefont {S.}~\bibnamefont {Zhang}},
  \bibinfo {author} {\bibfnamefont {N.}~\bibnamefont {Bigagli}}, \bibinfo
  {author} {\bibfnamefont {C.}~\bibnamefont {Warner}}, \bibinfo {author}
  {\bibfnamefont {I.}~\bibnamefont {Stevenson}}, \ and\ \bibinfo {author}
  {\bibfnamefont {S.}~\bibnamefont {Will}},\ }\href@noop {} {\bibfield
  {journal} {\bibinfo  {journal} {(in preparation)}\ } (\bibinfo {year}
  {2023})}\BibitemShut {NoStop}%
\bibitem [{\citenamefont {Warner}\ \emph {et~al.}(2021)\citenamefont {Warner},
  \citenamefont {Lam}, \citenamefont {Bigagli}, \citenamefont {Liu},
  \citenamefont {Stevenson},\ and\ \citenamefont
  {Will}}]{warner2021overlapping}%
  \BibitemOpen
  \bibfield  {author} {\bibinfo {author} {\bibfnamefont {C.}~\bibnamefont
  {Warner}}, \bibinfo {author} {\bibfnamefont {A.~Z.}\ \bibnamefont {Lam}},
  \bibinfo {author} {\bibfnamefont {N.}~\bibnamefont {Bigagli}}, \bibinfo
  {author} {\bibfnamefont {H.~C.}\ \bibnamefont {Liu}}, \bibinfo {author}
  {\bibfnamefont {I.}~\bibnamefont {Stevenson}}, \ and\ \bibinfo {author}
  {\bibfnamefont {S.}~\bibnamefont {Will}},\ }\href {\doibase
  10.1103/PhysRevA.104.033302} {\bibfield  {journal} {\bibinfo  {journal}
  {Phys. Rev. A}\ }\textbf {\bibinfo {volume} {104}},\ \bibinfo {pages}
  {033302} (\bibinfo {year} {2021})}\BibitemShut {NoStop}%
\bibitem [{\citenamefont {Olson}\ \emph {et~al.}(2013)\citenamefont {Olson},
  \citenamefont {Niffenegger},\ and\ \citenamefont
  {Chen}}]{olson2013optimizing}%
  \BibitemOpen
  \bibfield  {author} {\bibinfo {author} {\bibfnamefont {A.~J.}\ \bibnamefont
  {Olson}}, \bibinfo {author} {\bibfnamefont {R.~J.}\ \bibnamefont
  {Niffenegger}}, \ and\ \bibinfo {author} {\bibfnamefont {Y.~P.}\ \bibnamefont
  {Chen}},\ }\href@noop {} {\bibfield  {journal} {\bibinfo  {journal} {{Phys.
  Rev. A}}\ }\textbf {\bibinfo {volume} {87}},\ \bibinfo {pages} {053613}
  (\bibinfo {year} {2013})}\BibitemShut {NoStop}%
\bibitem [{\citenamefont {Luiten}\ \emph {et~al.}(1996)\citenamefont {Luiten},
  \citenamefont {Reynolds},\ and\ \citenamefont
  {Walraven}}]{luiten1996kinetic}%
  \BibitemOpen
  \bibfield  {author} {\bibinfo {author} {\bibfnamefont {O.}~\bibnamefont
  {Luiten}}, \bibinfo {author} {\bibfnamefont {M.}~\bibnamefont {Reynolds}}, \
  and\ \bibinfo {author} {\bibfnamefont {J.}~\bibnamefont {Walraven}},\
  }\href@noop {} {\bibfield  {journal} {\bibinfo  {journal} {{Phys. Rev. A}}\
  }\textbf {\bibinfo {volume} {53}},\ \bibinfo {pages} {381} (\bibinfo {year}
  {1996})}\BibitemShut {NoStop}%
\bibitem [{\citenamefont {Davis}\ \emph {et~al.}(1995)\citenamefont {Davis},
  \citenamefont {Mewes},\ and\ \citenamefont {Ketterle}}]{davis1995analytical}%
  \BibitemOpen
  \bibfield  {author} {\bibinfo {author} {\bibfnamefont {K.~B.}\ \bibnamefont
  {Davis}}, \bibinfo {author} {\bibfnamefont {M.~O.}\ \bibnamefont {Mewes}}, \
  and\ \bibinfo {author} {\bibfnamefont {W.}~\bibnamefont {Ketterle}},\
  }\href@noop {} {\bibfield  {journal} {\bibinfo  {journal} {{Appl. Phys. B}}\
  }\textbf {\bibinfo {volume} {60}},\ \bibinfo {pages} {155} (\bibinfo {year}
  {1995})}\BibitemShut {NoStop}%
\bibitem [{\citenamefont {Karman}\ and\ \citenamefont
  {Hutson}(2018)}]{karman2018microwave}%
  \BibitemOpen
  \bibfield  {author} {\bibinfo {author} {\bibfnamefont {T.}~\bibnamefont
  {Karman}}\ and\ \bibinfo {author} {\bibfnamefont {J.~M.}\ \bibnamefont
  {Hutson}},\ }\href@noop {} {\bibfield  {journal} {\bibinfo  {journal} {Phys.
  Rev. Lett.}\ }\textbf {\bibinfo {volume} {121}},\ \bibinfo {pages} {163401}
  (\bibinfo {year} {2018})}\BibitemShut {NoStop}%
\bibitem [{\citenamefont {Karman}\ and\ \citenamefont
  {Hutson}(2019)}]{karman2019microwave}%
  \BibitemOpen
  \bibfield  {author} {\bibinfo {author} {\bibfnamefont {T.}~\bibnamefont
  {Karman}}\ and\ \bibinfo {author} {\bibfnamefont {J.~M.}\ \bibnamefont
  {Hutson}},\ }\href@noop {} {\bibfield  {journal} {\bibinfo  {journal} {Phys.
  Rev. A}\ }\textbf {\bibinfo {volume} {100}},\ \bibinfo {pages} {052704}
  (\bibinfo {year} {2019})}\BibitemShut {NoStop}%
\bibitem [{\citenamefont {Wang}\ and\ \citenamefont
  {Bohn}(2021)}]{wang2021anisotropic}%
  \BibitemOpen
  \bibfield  {author} {\bibinfo {author} {\bibfnamefont {R.~R.}\ \bibnamefont
  {Wang}}\ and\ \bibinfo {author} {\bibfnamefont {J.~L.}\ \bibnamefont
  {Bohn}},\ }\href@noop {} {\bibfield  {journal} {\bibinfo  {journal} {Phys.
  Rev. A}\ }\textbf {\bibinfo {volume} {103}},\ \bibinfo {pages} {063320}
  (\bibinfo {year} {2021})}\BibitemShut {NoStop}%
\bibitem [{\citenamefont {Gu\'ery-Odelin}\ \emph {et~al.}(1999)\citenamefont
  {Gu\'ery-Odelin}, \citenamefont {Zambelli}, \citenamefont {Dalibard},\ and\
  \citenamefont {Stringari}}]{gueryodelin1999collective}%
  \BibitemOpen
  \bibfield  {author} {\bibinfo {author} {\bibfnamefont {D.}~\bibnamefont
  {Gu\'ery-Odelin}}, \bibinfo {author} {\bibfnamefont {F.}~\bibnamefont
  {Zambelli}}, \bibinfo {author} {\bibfnamefont {J.}~\bibnamefont {Dalibard}},
  \ and\ \bibinfo {author} {\bibfnamefont {S.}~\bibnamefont {Stringari}},\
  }\href {\doibase 10.1103/PhysRevA.60.4851} {\bibfield  {journal} {\bibinfo
  {journal} {Phys. Rev. A}\ }\textbf {\bibinfo {volume} {60}},\ \bibinfo
  {pages} {4851} (\bibinfo {year} {1999})}\BibitemShut {NoStop}%
\bibitem [{\citenamefont {Monroe}\ \emph {et~al.}(1993)\citenamefont {Monroe},
  \citenamefont {Cornell}, \citenamefont {Sackett}, \citenamefont {Myatt},\
  and\ \citenamefont {Wieman}}]{monroe1993measurement}%
  \BibitemOpen
  \bibfield  {author} {\bibinfo {author} {\bibfnamefont {C.}~\bibnamefont
  {Monroe}}, \bibinfo {author} {\bibfnamefont {E.~A.}\ \bibnamefont {Cornell}},
  \bibinfo {author} {\bibfnamefont {C.}~\bibnamefont {Sackett}}, \bibinfo
  {author} {\bibfnamefont {C.}~\bibnamefont {Myatt}}, \ and\ \bibinfo {author}
  {\bibfnamefont {C.}~\bibnamefont {Wieman}},\ }\href@noop {} {\bibfield
  {journal} {\bibinfo  {journal} {Phys. Rev. Lett.}\ }\textbf {\bibinfo
  {volume} {70}},\ \bibinfo {pages} {414} (\bibinfo {year} {1993})}\BibitemShut
  {NoStop}%
\bibitem [{\citenamefont {Wu}\ and\ \citenamefont {Foot}(1996)}]{wu1996direct}%
  \BibitemOpen
  \bibfield  {author} {\bibinfo {author} {\bibfnamefont {H.}~\bibnamefont
  {Wu}}\ and\ \bibinfo {author} {\bibfnamefont {C.~J.}\ \bibnamefont {Foot}},\
  }\href@noop {} {\bibfield  {journal} {\bibinfo  {journal} {J.~Phys.~B}\
  }\textbf {\bibinfo {volume} {29}},\ \bibinfo {pages} {L321} (\bibinfo {year}
  {1996})}\BibitemShut {NoStop}%
\bibitem [{\citenamefont {Bohn}\ and\ \citenamefont
  {Jin}(2014)}]{bohn14differential}%
  \BibitemOpen
  \bibfield  {author} {\bibinfo {author} {\bibfnamefont {J.~L.}\ \bibnamefont
  {Bohn}}\ and\ \bibinfo {author} {\bibfnamefont {D.~S.}\ \bibnamefont {Jin}},\
  }\href {\doibase 10.1103/PhysRevA.89.022702} {\bibfield  {journal} {\bibinfo
  {journal} {Phys. Rev. A}\ }\textbf {\bibinfo {volume} {89}},\ \bibinfo
  {pages} {022702} (\bibinfo {year} {2014})}\BibitemShut {NoStop}%
\bibitem [{\citenamefont {Schindewolf}\ \emph {et~al.}(2022)\citenamefont
  {Schindewolf}, \citenamefont {Bause}, \citenamefont {Chen}, \citenamefont
  {Duda}, \citenamefont {Karman}, \citenamefont {Bloch},\ and\ \citenamefont
  {Luo}}]{schindewolf2022evaporation}%
  \BibitemOpen
  \bibfield  {author} {\bibinfo {author} {\bibfnamefont {A.}~\bibnamefont
  {Schindewolf}}, \bibinfo {author} {\bibfnamefont {R.}~\bibnamefont {Bause}},
  \bibinfo {author} {\bibfnamefont {X.-Y.}\ \bibnamefont {Chen}}, \bibinfo
  {author} {\bibfnamefont {M.}~\bibnamefont {Duda}}, \bibinfo {author}
  {\bibfnamefont {T.}~\bibnamefont {Karman}}, \bibinfo {author} {\bibfnamefont
  {I.}~\bibnamefont {Bloch}}, \ and\ \bibinfo {author} {\bibfnamefont {X.-Y.}\
  \bibnamefont {Luo}},\ }\href@noop {} {\bibfield  {journal} {\bibinfo
  {journal} {Nature}\ }\textbf {\bibinfo {volume} {607}},\ \bibinfo {pages}
  {677} (\bibinfo {year} {2022})}\BibitemShut {NoStop}%
\bibitem [{\citenamefont {Chen}\ \emph {et~al.}(2023)\citenamefont {Chen},
  \citenamefont {Schindewolf}, \citenamefont {Eppelt}, \citenamefont {Bause},
  \citenamefont {Duda}, \citenamefont {Biswas}, \citenamefont {Karman},
  \citenamefont {Hilker}, \citenamefont {Bloch},\ and\ \citenamefont
  {Luo}}]{chen2023field}%
  \BibitemOpen
  \bibfield  {author} {\bibinfo {author} {\bibfnamefont {X.-Y.}\ \bibnamefont
  {Chen}}, \bibinfo {author} {\bibfnamefont {A.}~\bibnamefont {Schindewolf}},
  \bibinfo {author} {\bibfnamefont {S.}~\bibnamefont {Eppelt}}, \bibinfo
  {author} {\bibfnamefont {R.}~\bibnamefont {Bause}}, \bibinfo {author}
  {\bibfnamefont {M.}~\bibnamefont {Duda}}, \bibinfo {author} {\bibfnamefont
  {S.}~\bibnamefont {Biswas}}, \bibinfo {author} {\bibfnamefont
  {T.}~\bibnamefont {Karman}}, \bibinfo {author} {\bibfnamefont
  {T.}~\bibnamefont {Hilker}}, \bibinfo {author} {\bibfnamefont
  {I.}~\bibnamefont {Bloch}}, \ and\ \bibinfo {author} {\bibfnamefont {X.-Y.}\
  \bibnamefont {Luo}},\ }\href@noop {} {\bibfield  {journal} {\bibinfo
  {journal} {Nature}\ }\textbf {\bibinfo {volume} {614}},\ \bibinfo {pages}
  {59} (\bibinfo {year} {2023})}\BibitemShut {NoStop}%
\bibitem [{\citenamefont {Bohn}\ \emph {et~al.}(2009)\citenamefont {Bohn},
  \citenamefont {Cavagnero},\ and\ \citenamefont {Ticknor}}]{bohn2009quasi}%
  \BibitemOpen
  \bibfield  {author} {\bibinfo {author} {\bibfnamefont {J.}~\bibnamefont
  {Bohn}}, \bibinfo {author} {\bibfnamefont {M.}~\bibnamefont {Cavagnero}}, \
  and\ \bibinfo {author} {\bibfnamefont {C.}~\bibnamefont {Ticknor}},\
  }\href@noop {} {\bibfield  {journal} {\bibinfo  {journal} {{New J. Phys.}}\
  }\textbf {\bibinfo {volume} {11}},\ \bibinfo {pages} {055039} (\bibinfo
  {year} {2009})}\BibitemShut {NoStop}%
\bibitem [{\citenamefont {Julienne}\ \emph {et~al.}(2011)\citenamefont
  {Julienne}, \citenamefont {Hanna},\ and\ \citenamefont
  {Idziaszek}}]{julienne2011universal}%
  \BibitemOpen
  \bibfield  {author} {\bibinfo {author} {\bibfnamefont {P.~S.}\ \bibnamefont
  {Julienne}}, \bibinfo {author} {\bibfnamefont {T.~M.}\ \bibnamefont {Hanna}},
  \ and\ \bibinfo {author} {\bibfnamefont {Z.}~\bibnamefont {Idziaszek}},\
  }\href@noop {} {\bibfield  {journal} {\bibinfo  {journal} {Phys. Chem. Chem.
  Phys.}\ }\textbf {\bibinfo {volume} {13}},\ \bibinfo {pages} {19114}
  (\bibinfo {year} {2011})}\BibitemShut {NoStop}%
\end{thebibliography}%



\end{document}