 \documentclass[%
 reprint,
superscriptaddress,
%groupedaddress,
%unsortedaddress,
%runinaddress,
%frontmatterverbose, 
%preprint,
%preprintnumbers,
%nofootinbib,
%nobibnotes,
%bibnotes,
 amsmath,amssymb,
 aps,
 prl,
%prb,
%rmp,
%prstab,
%prstper,
floatfix,
]{revtex4-2}


\usepackage{graphicx}% Include figure files
\usepackage{epsfig} %Works faster than the graphicx package}
\usepackage{dcolumn}% Align table columns on decimal point
\usepackage{bm}% bold math
\usepackage{epstopdf}
\usepackage{multirow}
\usepackage{amsmath}
\usepackage{booktabs}
\usepackage{color}
\usepackage{gensymb}
%\usepackage{dblfloatfix}
\usepackage{kantlipsum}  
\usepackage{hyperref}
\usepackage{ulem}
\usepackage{braket}
\usepackage{physics}
\usepackage[utf8]{inputenc}
\usepackage[T1]{fontenc}
\usepackage{mathptmx}


%\linenumbers
%% New color commands %%%

\begin{document}
\title{Supplementary Information for \\ "Collisionally Stable Gas of Bosonic Dipolar Ground State Molecules"}

\author{Niccol\`{o} Bigagli}
\affiliation{Department of Physics, Columbia University, New York, New York 10027, USA}
\author{Claire Warner}
\affiliation{Department of Physics, Columbia University, New York, New York 10027, USA}
\author{Weijun Yuan}
\affiliation{Department of Physics, Columbia University, New York, New York 10027, USA}
\author{Siwei Zhang}
\affiliation{Department of Physics, Columbia University, New York, New York 10027, USA}
\author{Ian Stevenson}
\affiliation{Department of Physics, Columbia University, New York, New York 10027, USA}
\author{Tijs Karman}
\affiliation{Institute for Molecules and Materials, Radboud University, 6525 AJ Nijmegen, Netherlands}
\author{Sebastian Will}\email{Corresponding author. Email: sebastian.will@columbia.edu}
\affiliation{Department of Physics, Columbia University, New York, New York 10027, USA}

\date{\today}

\setcounter{figure}{0}
\makeatletter 
\renewcommand{\thefigure}{S\@arabic\c@figure}
\makeatother

\maketitle

\section{Quantum numbers}

We denote the internal molecular quantum states relevant to this work with labels $|J,m_J\rangle$, where $J$ is the total angular momentum of the molecules excluding nuclear spin, and $m_J$ its projection onto the quantization axis. Due to the high magnetic field in the vicinity of the Feshbach resonance at $B_\mathrm{res}= 864.1(1)$ G, the molecules are in the Paschen-Back regime and nuclear spin is decoupled from all other angular momenta. Therefore, we omit nuclear spin quantum numbers. The nuclear spin stays unchanged at $\ket{m_{I_{\rm Na}}, \ m_{I_{\rm Cs}}} = \ket{3/2,\ 5/2}$, even in the presence of the microwave field.

\section{Rotational spectroscopy}

\begin{figure} 
    \centering
    \includegraphics[width = 6.9 cm]{SI_Figure_2.pdf}\\
    \caption{Spectrum of the $|J=0\rangle \rightarrow |J=1\rangle$ rotational transitions of NaCs using a microwave field with mixed polarization. Different polarization couples different $m_J$ states, as marked. The vertical dashed lines indicate the assigned transitions. Error bars show the 1$\sigma$ standard-error-of-the-mean from three repetitions of the measurement.}
    \label{fig:SI2}
\end{figure} 

In order to identify the $\sigma^+$ transition for microwave shielding, we performed rotational spectroscopy of NaCs. We prepared a gas of ground state molecules in the $\ket{J, \ m_J} = \ket{0, \ 0}$ state and exposed it to a microwave field with mixed polarization while the magnetic bias field was close to $B_\mathrm{res}$. The corresponding spectrum is shown in Fig.~\ref{fig:SI2}. To assign quantum numbers to each transition, we repeated the measurement with polarized microwave fields, showing a single transition depending on the polarization used. The resonance frequency of the $\sigma^+$ transition between $\ket{0, \ 0}$ and $\ket{1, \ +1}$ was measured to be 3.471323(2) GHz.  

\section{Dressed state preparation}

\begin{figure} 
    \centering
    \includegraphics[width = 6.4 cm]{SI_Figure_3.pdf}\\
    \caption{Adiabaticity of dressed state preparation for $\Omega / (2\pi) = 4$ MHz. The ratios of molecule numbers before ($N_i$) and after ($N_f$) a round trip through the dressed state $|+\rangle$ is shown. The inset shows a schematic of the experimental sequence. Error bars show the 1$\sigma$ standard-error-of-the-mean from three repetitions of the measurement.}
    \label{fig:SI3}
\end{figure} 

The molecules are prepared in the dressed state $|+\rangle$ via an adiabatic increase of the intensity of the blue-detuned microwave field. The intensity is ramped up within $40$ $\mu$s using a ramp following a quadratic power law, $\Omega(t) \propto t^2$. To confirm adiabaticity of this ramp, we compared the molecule number in state $\ket{J, \ m_J, \ m_{I_{\rm Na}}, \ m_{I_{\rm Cs}}} = \ket{0, \ 0, \ 3/2, \ 5/2}$ before the ramp, $N_i$, to the molecule number $N_f$ after a microwave ramp into and out of state $|+\rangle$. For a non-adiabatic ramp, we would expect loss of population into other states and $N_f/N_i$ should be smaller than 1. We performed this measurement for microwave Rabi frequency $\Omega/(2 \pi)=4$ MHz and various detunings $\Delta$, as shown in Fig.~\ref{fig:SI3}. While the data point at the lowest detuning does not meet this criterion, all data with $\Delta/\Omega > 0.1$, which is the case for all data in the main text, fulfills the criterion. 

\section{Microwave Ellipticity}

We measured the ellipticity of the microwave field by driving resonant Rabi oscillations on the transitions $\ket{J, \ m_J } = \ket{ 0, \ 0} $ to $\ket{ 1, \ -1 }$ and $\ket{ 1, \ +1 }$. For fixed microwave power, we determine the respective Rabi frequencies and obtain an ellipticity $\xi = \arctan(\Omega_{\sigma^-} / \Omega_{\sigma^+}) = 3(2) \degree$. An attempt to drive a resonant transition to $\ket{ 1, \ 0 }$ under identical conditions was consistent with no Rabi coupling. We note that the axis of molecular rotation is well-aligned with the axis of the magnetic field.

\section{One-body loss}

\begin{figure} 
    \centering
    \includegraphics[width = 6.4 cm]{SI_Figure_4.pdf}\\
    \caption{Measured one-body-limited lifetime of the shielded NaCs gas for the same Rabi frequency $\Omega / 2\pi = 4$ MHz and $\Delta/(2\pi) = 4$ MHz using different levels of attenuation of the 15 W amplifiers. The error bars show the 1$\sigma$ error from the fit of the loss curves.}
    \label{fig:SI4}
\end{figure}

\begin{figure}
    \centering
    \includegraphics[width = 6.6 cm]{SI_Figure_5.pdf}\\
    \caption{$1/e$-lifetime of microwave-shielded molecules as a function of $\Omega$. The ratio $\Delta / \Omega = 1.5$ is kept fixed. The error bars show the 1$\sigma$ error from the fit of the loss curves.}
    \label{fig:SI5}
\end{figure} 

The dominant source of one-body loss in the microwave-shielded molecular gas stems from noise of the microwave field away from the carrier frequency. Such noise can drive transitions to the anti-shielded dressed state $|-\rangle$ and unshielded spectator states $\ket{0}$, limiting the lifetime of the molecular gas. Such noise can be generated or amplified by any active component of the microwave chain, e.g.~the microwave source and amplifiers. To reduce this noise we employ a high quality microwave source (Rohde \& Schwarz SMA100B with ultralow phase-noise option). We find that the phase-noise of the source is so low that it does not limit the observed one-body loss. Instead, we find that thermal white noise of the fixed-gain microwave amplifiers (several MHz away from the carrier frequency) contributes dominantly. To reduce this noise, we set the output power of the microwave source to maximum and attenuate the amplifiers' output to the desired power level via external attenuators. We have measured one-body loss at constant overall output power, corresponding to  $\Omega / 2\pi = 4$ MHz, for different combinations of source power level and amplifier attenuation (Fig.~\ref{fig:SI4}). For the largest attenuation of 16 dB, we find the longest one-body-limited lifetime of $\tau_{1\text{B}} \sim 4.4(4)$ s. 


\section{Optimum Rabi Frequency}

All the data of this work was taken for a Rabi frequency of $\Omega / 2\pi = 4$ MHz. For this Rabi frequency we were able to make use of the full 16 dB post-amplifier attenuation to limit one-body loss from microwave noise (see above). Fig.~\ref{fig:SI5} shows the measured $1/e$-lifetimes of the molecular cloud for different Rabi frequencies $\Omega$, while keeping the ratio $\Delta/\Omega = 1.5$ fixed. To access higher Rabi frequencies the post-amplifier attentuation was gradually reduced, which reduced the lifetimes due to higher microwave noise. Theoretically, higher Rabi frequency is expected to lead to better shielding than lower Rabi frequency. Within the technical limitations of the experimental setup, we observe peak shielding performance for $\Omega / 2\pi \sim 4$ MHz.

\section{Thermometry via Time-of-flight expansion}

Precise thermometry of the ultracold gas of ground state molecules requires careful consideration. In the Methods section we point out a systematic shift between the temperatures measured for the time-of-flight expansion for shielded and unshielded molecules. Unshielded molecules undergo rapid loss especially at the beginning of time-of-flight expansion when the density is still high (with a two-body loss rate of up to $10^{-9}$ cm$^3$/s) which leads to a systematic overestimation of temperature by $10\%$. A comparison of time-of-flight expansion for shielded and unshielded molecules is shown in Fig. \ref{fig:SI1}. All thermometry in this work is performed using time-of-flight expansion of shielded molecules.

\begin{figure*}
    \centering
    \includegraphics[width = 14.2 cm]{SI_Figure_1_full.pdf}\\
    \caption{Time-of-flight expansion of a gas of ground state molecules with shielding at $\Omega/(2\pi) = 4$ MHz, $\Delta/(2\pi) = 6$ MHz (a) and without shielding (b). Upper (lower) panels correspond to the $x$ ($y$) direction. Insets in the lower panels illustrate the respective experimental sequence. The fitted mean temperature is 290(10) nK for (a) and 330(20) nK for (b). Error bars show 1$\sigma$ standard-error-of-the-mean from three repetitions of the measurement.}
    \label{fig:SI1}
\end{figure*} 

\end{document}