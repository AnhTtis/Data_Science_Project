% Information that is shared between the article and the supplement
% (title and author information, macros, packages, etc.) goes into
% ex_shared.tex. If there is no supplement, this file can be included
% directly.

% SIAM Shared Information Template
% This is information that is shared between the main document and any
% supplement. If no supplement is required, then this information can
% be included directly in the main document.


% Packages and macros go here
\usepackage{lipsum}
\usepackage{amsfonts, amsmath}
\usepackage{graphicx}
\usepackage{epstopdf}
\usepackage{algorithmic}
\usepackage{amssymb}
\usepackage{cleveref}
\ifpdf
  \DeclareGraphicsExtensions{.eps,.pdf,.png,.jpg}
\else
  \DeclareGraphicsExtensions{.eps}
\fi

\newsiamremark{example}{Example}

\newcommand{\newassumption}[2]{
  \theoremstyle{plain}
  \theoremheaderfont{\normalfont\sc}
  \theorembodyfont{\normalfont\itshape}
  \theoremseparator{.}
  \theoremsymbol{}
  \newtheorem{#1}{#2}
}

\newcommand{\rev}[1]{{\color{blue}#1}}

\newassumption{assumption}{Assumption}




% Add a serial/Oxford comma by default.
\newcommand{\creflastconjunction}{, and~}

% Used for creating new theorem and remark environments
\newsiamremark{remark}{Remark}
\newsiamremark{hypothesis}{Hypothesis}
\crefname{hypothesis}{Hypothesis}{Hypotheses}
\newsiamthm{claim}{Claim}


\def\del{\delta}

\newcommand{\aff}{\mathrm{aff}\,}
\def\argmin{ \mathop{{\rm argmin}}}
\def\argmax{ \mathop{{\rm argmax}}}
\newcommand{\bi}{\binom}
\newcommand{\bd}{\mathrm{bd}\,}
\newcommand{\rbd}{\mathrm{rbd}\,}
\newcommand{\co}{\mathrm{conv}\,}
\newcommand{\cl}{\mathrm{cl}\,}
\newcommand{\clco}{\mathrm{\overline{conv}}\,}
\newcommand{\diag}{\mathrm{diag}\,}
\newcommand{\dom}{\mathrm{dom}\,}
\newcommand{\epi}{\mathrm{epi}\,}
\def\elim{{\rm e-}\hspace{-0.5pt}\lim}
\def\clim{{\rm c-}\hspace{-0.5pt}\lim}
\def\plim{{\rm p-}\hspace{-0.5pt}\lim}
\def\glim{{\rm g-}\hspace{-0.5pt}\lim}
\newcommand{\gph}{\mathrm{gph}\,}
\newcommand{\id}{\mathrm{id}\,}
% \newcommand{\inter}{\mathrm{int}\,}
\DeclareMathOperator{\inter}{int}
\newcommand{\pos}{\mathrm{pos}\,}
\newcommand{\ri}{\mathrm{ri}\,}
\newcommand{\barr}{\mathrm{bar}\,}
\newcommand{\para}{\mathrm{par}\,}
\newcommand{\rge}{\mathrm{rge}\,}
% \newcommand{\Ker}{\mathrm{Ker}}
% \newcommand{\Rge}{\mathrm{Rge}}
\newcommand{\sgn}{\mathrm{sgn}}
\newcommand{\hzn}{\mathrm{hzn}\,}
\newcommand{\lin}{\mathrm{span}\,}
\def\Lim{\mathop{{\rm Lim}\,}}
\def\Liminf{\mathop{{\rm Lim}\,{\rm inf}}}
\def\Limsup{\mathop{{\rm Lim}\,{\rm sup}}}
\newcommand{\nul}{\mathrm{nul}\,}
\newcommand{\p}{\partial}
\def\plim{{\rm p-}\hspace{-0.5pt}\lim}

\newcommand{\lev}[2]{\mathrm{lev}_{#1}#2}

\newcommand{\R}{\mathbb{R}}
\newcommand{\Rn}{\R^n}
\newcommand{\Rnm}{\R^{n\times m}}
\newcommand{\Rm}{\R^m}
\newcommand{\bS}{\mathbb{S}}
\newcommand{\Sn}{\bS^n}


\newcommand{\supp}{\mathrm{supp}}

\newcommand{\rank}{\mathrm{rank}\,}
\newcommand{\rbar}{\overline{\mathbb R}}
\newcommand{\Rbar}{\overline{\mathbb R}}
\newcommand{\rp}{\mathbb R\cup\{+\infty\}}
\newcommand{\bR}{\mathbb{R}}
\def\Rn{\bR^n}
\def\Rm{\bR^m}
\def\Rk{\bR^k}
\def\Rmn{\bR^{m\times n}}
\def\Rnm{\bR^{n\times m}}
\def\bS{\mathbb{S}}
\def\Sn{\bS^n}
\def\Snp{\bS^n_+}
\def\Snpp{\bS_{++}^n}

\newcommand{\Iff}{\quad\Leftrightarrow\quad}
\newcommand{\IFF}{\quad\Longleftrightarrow\quad}


\newcommand{\tr}{\mathrm{tr}\,}
\def\pto {\overset{p}{\to}} 
\def\eto {\overset{e}{\to}} 
\def\cto {\overset{c}{\to}} 
\def\gto {\overset{g}{\to}} 

\newcommand{\cone}{\mathrm{cone}}

%\newcommand{\too}[3]{#1\overset{#2}{\to}#3}
\newcommand{\too}[3]{#1\to_{#2}#3}
%\newcommand{\bR}{\mathbb{R}}
\newcommand{\bB}{\mathbb{B}}
\newcommand{\bN}{\mathbb{N}}
\newcommand{\bE}{\mathbb{E }}
\newcommand{\bW}{\mathbb{W }}
\newcommand{\bZ}{\mathbb{Z }}
%\newcommand{\eR}{\overline{\R}}
\newcommand{\map}[3]{#1 :#2\rightarrow #3}
\newcommand{\ip}[2]{\left\langle #1,\, #2\right\rangle}
\newcommand{\half}{\frac{1}{2}}
\newcommand{\norm}[1]{\left\Vert #1\right\Vert}
\newcommand{\tnorm}[1]{\left\Vert #1\right\Vert_2}
\newcommand{\snorm}[1]{\left\Vert #1\right\Vert_\infty}
\newcommand{\onorm}[1]{\left\Vert #1\right\Vert_1}
\newcommand{\nnorm}[1]{\left\Vert #1\right\Vert_{\star}}
\newcommand{\dnorm}[1]{\left\Vert #1\right\Vert_*}
\newcommand{\fnorm}[1]{\left\Vert #1\right\Vert_{F}}
\newcommand{\set}[2]{\left\{#1\,\left\vert\; #2\right.\right\}}
%\newcommand{\cone}[2]{\mathrm{cone}\left\{#1\,\left\vert\; #2\right.\right\}}
\newcommand{\ncone}[2]{N_{#2}\left(#1\right)}
\newcommand{\sncone}[2]{N\left(#1\,\left|\, #2\right.\right)}
\newcommand{\tcone}[2]{T\left(#1\,\left|\, #2\right.\right)}
%\newcommand{\support}[2]{\sig\left(#1\;\left|\; #2\right.\right)}
\newcommand{\support}[2]{\sig_{#2}\left(#1\right)}
\newcommand{\dist}[2]{\mathrm{dist}\left(#1\;\left|\; #2\right.\right)}
%\newcommand{\indicator}[2]{\del\left(#1\;\left|\; #2\right.\right)}
\newcommand{\indicator}[2]{\del_{#2}\left(#1\right)}
\newcommand{\bindicator}[2]{\del\left(#1\,\left|\, #2\right.\right)}
%\newcommand{\gauge}[2]{\gam\left(#1\;\left|\; #2\right.\right)}
\newcommand{\gauge}[2]{\gam_{#2}\left(#1\right)}
\newcommand{\barrier}{{\mathrm{bar}\,}}
\def\rk{\mathrm{rk}}
\newcommand{\clcone}[1]{\overline{\mathrm{cone}}\; #1}

\def\sd{\partial}

\newcommand{\eplqs}{{\scriptscriptstyle{(U,B,R,b)}}}

\newcommand{\bu}{{\bar{u}}}
\newcommand{\bv}{{\bar{v}}}
\newcommand{\bx}{{\bar{x}}}
\newcommand{\by}{{\bar{y}}}
\newcommand{\bz}{{\bar{z}}}
\newcommand{\bw}{{\bar{w}}}

\newcommand{\bt}{{\bar{t}}}
\newcommand{\barX}{{\bar X}}
\newcommand{\barY}{{\bar Y}}
\newcommand{\barU}{{\bar U}}
\newcommand{\barV}{{\bar V}}
\newcommand{\barW}{{\bar W}}
\newcommand{\barT}{{\bar T}}

\newcommand{\ty}{{\tilde{y}}}
\newcommand{\tx}{{\tilde{x}}}

\newcommand{\si}{\dagger}
\newcommand{\alf}{\alpha}
\newcommand{\gam}{\gamma}
\newcommand{\Gam}{\Gamma}
\newcommand{\sig}{\sigma}
\newcommand{\Sig}{\Sigma}
\newcommand{\lam}{\lambda}
\newcommand{\Lam}{\Lambda}
\def\eps{\epsilon}


\newcommand{\GMF}{\varphi_{A,B}}
\newcommand{\gV}{g_{\bar V}^{A,B}}

\newcommand{\cC}{\mathcal{C}}
\newcommand{\cD}{\mathcal{D}}
\newcommand{\cE}{\mathcal{E}}
\newcommand{\cF}{\mathcal{F}}
\newcommand{\cK}{\mathcal{K}}
\newcommand{\cL}{\mathcal{L}}
\newcommand{\cN}{\mathcal{N}}
\newcommand{\cO}{\mathcal{O}}
\newcommand{\cP}{\mathcal{P}}
\newcommand{\cS}{\mathcal{S}}
\newcommand{\cT}{\mathcal{T}}
\newcommand{\cU}{\mathcal{U}}
\newcommand{\cX}{\mathcal{X}}
\newcommand{\cV}{\mathcal{V}}
\newcommand{\cZ}{\mathcal{Z}}

\newcommand{\hY}{{\widehat{Y}}}
\newcommand{\hX}{{\widehat{X}}}
\newcommand{\hZ}{{\widehat{Z}}}
\newcommand{\hW}{{\widehat{W}}}
\newcommand{\hU}{{\widehat{U}}}
\newcommand{\hQ}{{\widehat{Q}}}
\newcommand{\hP}{{\widehat{P}}}
\newcommand{\hV}{{\widehat{V}}}
\def\hN{{\widehat{N}}}
\def\hn{{\hat n}}

\newcommand{\hy}{{\hat{y}}}
\newcommand{\hz}{{\hat{z}}}

\def\tS{{\widetilde{S}}}
\def\tW{{\widetilde{W}}}

\newcommand{\AND}{\ \mbox{ and }\ }
\newcommand{\st}{\ \mbox{s.t.}\ }

\newcommand{\Adag}{{A^\dagger}}
\newcommand{\Vdag}{{\hV^\dagger}}
\newcommand{\MVdag}{{M(V)^\dagger}}

\newcommand{\infconv}{\mathbin{\mbox{\small$\square$}}}
\def\ssquare{\mathbin{\mbox{\small$\square$}}}

\newcommand{\kron}{\otimes}
\newcommand{\one}{\mathbf{1}}

\newcommand{\hcVp}{{\cV^{1/2}_A}}

\newcommand{\vphi}{\varphi}


\newcommand{\argsup}{\operatornamewithlimits{\mbox{arg\;sup}}}


\newcommand{\Epi}[2]{{#1}\mbox{-}\mathrm{epi}{\,#2}} 
\newcommand{\Epii}[2]{{#1}\mbox{-}\mathrm{epi}{\,#2}_<} 
\newcommand{\Kp}{\mbox{-}K^\circ}
\newcommand{\Kpp}{\mbox{-}K^\circ_+}




% definition of Cartesian products
\def\XX{\operatornamewithlimits{\mbox{\sf\la\rge X}}}
\def\XXX{\operatornamewithlimits{\mbox{\sf\La\rge X}}}


%color

\newcommand{\blue}[1]{\textcolor{blue}{#1}}
\newcommand{\red}[1]{\textcolor{red}{#1}}
\newcommand{\edit}[1]{\textcolor{orange}{#1}}
\newcommand{\comment}[1]{\textcolor{blue}{(#1)}}

\newcommand{\rrcomment}[1]{\textcolor{blue}{\textsc{[#1 ---rr]}}}

\newcommand{\ls}[1][\tau]{\text{$(\mathrm{LS}_{#1})$}}
\newcommand{\bp}[1][\sigma]{\text{$(\mathrm{BP}_{#1})$}}
\newcommand{\qp}[1][\lambda]{\text{$(\mathrm{QP}_{#1})$}}
\newcommand{\reals}{\ensuremath{\mathbb{R}}}
\newcommand{\nats}{\ensuremath{\mathbb{N}}}
\newcommand{\complex}{\ensuremath{\mathbb{C}}}
\newcommand{\ints}{\ensuremath{\mathbb{Z}}}
\newcommand{\sph}{\ensuremath{\mathbb{S}}}
\newcommand{\sphn}[1][n]{\ensuremath{\mathbb{S}^{{#1}-1}}}

\DeclareMathOperator{\iid}{\overset{\text{iid}}{\sim}}
\DeclareMathOperator{\proj}{P}
\DeclareMathOperator{\projperp}{P^{\perp}}
\DeclareMathOperator{\identity}{\mathbb{I}}
\DeclareMathOperator{\rmd}{d}

\definecolor{codegray}{gray}{0.95}
\newcommand{\code}[1]{\fcolorbox{codegray}{codegray}{\small\texttt{#1}}}
\newcommand{\cvxpy}{\code{CVXPY}}
\newcommand{\cvxopt}{\code{CVXOPT}}
\newcommand{\mosek}{\code{MOSEK}}
\newcommand{\sklearn}{\code{sklearn}}


\newcommand{\eg}{\emph{e.g.,}}
\newcommand{\etc}{\emph{etc.}}
\newcommand{\etal}{\emph{et al.}}
\newcommand{\ie}{\emph{i.e.,}}
\newcommand{\vs}{\emph{vs.\@}}

\newcommand{\lasso}{LASSO}
\newcommand{\srlasso}{SR-LASSO}


% Sets running headers as well as PDF title and authors
\headers{Square Root LASSO}{A. Berk, S. Brugiapaglia, and T. Hoheisel}

% Title. If the supplement option is on, then "Supplementary Material"
% is automatically inserted before the title.
\title{Square Root {LASSO}: well-posedness,
  Lipschitz stability and the tuning trade off\thanks{\today\funding{The first author was partially supported by a postdoc stipend from the Centre de Recherche Mathématiques (CRM) as well as the Institut de valorisation des données (IVADO) and NSERC. The second author acknowledges the support of NSERC through grant RGPIN-2020-06766, the Fonds de Recherche du Québec Nature et Technologies
(FRQNT) through grant 313276, the Faculty of Arts and Science of Concordia University and the CRM. The third author was partially supported by the NSERC discovery grant RGPIN-2017-04035.}}}

% Authors: full names plus addresses.
\author{Aaron Berk\thanks{Department of Mathematics and Statistics, McGill University\newline\indent{}\indent{}(\email{aaron.berk@mcgill.ca}, \email{tim.hoheisel@mcgill.ca}).}
\and Simone Brugiapaglia\thanks{Department of Mathematics and Statistics, Concordia University\newline\indent{}\indent{}(\email{simone.brugiapaglia@concordia.ca}).}
\and Tim Hoheisel\footnotemark[2]}

\usepackage{amsopn}
\usepackage{mathabx}
\usepackage{subcaption}
\usepackage{tikz-cd}

\usepackage{listings}
\usepackage{inconsolata}
\lstset{
  tabsize=1,
  language=python,
  backgroundcolor=\color[rgb]{0.97,0.97,0.99},
  basicstyle=\small\ttfamily, %\ttfamily,
  % upquote=true,
  aboveskip={\baselineskip},
  columns=flexible,
  showstringspaces=false,
  extendedchars=true,
  breaklines=true,
  prebreak = \raisebox{0ex}[0ex][0ex]{\ensuremath{\hookleftarrow}},
  frame=false,
  % numbers=left,
  showtabs=false,
  showspaces=false,
  showstringspaces=false,
  identifierstyle=\color[rgb]{0,0,.01},
  keywordstyle=\bfseries\color[rgb]{.2,0.29,.7},
  commentstyle=\color[rgb]{0.2,0.3,0.3},
  stringstyle=\color[rgb]{0.627,0.126,0.941},
  % identifierstyle=\ttfamily,
  % keywordstyle=\color[rgb]{0,0,1},
  % commentstyle=\color[rgb]{0.07,0.472,0.07},
  % stringstyle=\color[rgb]{0.627,0.126,0.941},
}

\usepackage{newfloat}
\DeclareFloatingEnvironment[
    fileext=loalg,
    listname={List of Algorithms},
    name=Algorithm,
    placement=tbhp,
    within=section,
    ]{lstalgorithm}
\newcommand{\lstalgoc}[2]{\begin{lstalgorithm}\vskip-16pt\begin{center}
      \begin{minipage}{35em}
        \lstinputlisting[mathescape=true]{#1}
      \end{minipage}
    \end{center}
    \vskip-12pt
  \caption{#2}\end{lstalgorithm}}
\newcommand{\lstalgo}[1]{\begin{lstalgorithm}\vskip-16pt\begin{center}
      \begin{minipage}{35em}
        \lstinputlisting[mathescape=true]{#1}
      \end{minipage}
    \end{center}\vskip-12pt\end{lstalgorithm}}
\crefname{lstalgorithm}{Algorithm}{Algorithms}
    


\newcommand{\bmat}[2]{\ensuremath{\left[
    \begin{array}{{#1}}
      #2
    \end{array}\right]}}


\makeatletter
\newcommand{\crefnames}[3]{%
  \@for\next:=#1\do{%
    \expandafter\crefname\expandafter{\next}{#2}{#3}%
  }%
}
\makeatother

\crefnames{section,subsection,subsubsection}{\S\!}{\S\!}
\setlength{\abovecaptionskip}{0pt plus 3pt minus 2pt}
\usepackage{enumitem}

%%% Local Variables:
%%% mode: latex
%%% TeX-master: "SRLASSO_SIOPT.tex"
%%% reftex-default-bibliography: ("bibliography.bib")
%%% End: