% Template for ICIP-2022 paper; to be used with:
%          spconf.sty  - ICASSP/ICIP LaTeX style file, and
%          IEEEbib.bst - IEEE bibliography style file.
% --------------------------------------------------------------------------
\documentclass{article}
\usepackage{spconf,amsfonts,amsmath,graphicx}
\usepackage{booktabs}
\usepackage{multirow}
\usepackage{CJKutf8}
\usepackage{makecell}

% Example definitions.
% --------------------
\def\x{{\mathbf x}}
\def\L{{\cal L}}

% Title.
% ------
% \title{The enhanced AUDIO-VISUAL HUBERT }
% enhanced AV-HuBERT指的是glu+fbank80+win15+mobilenet,在此基础上再加入confromer的叫做conformer enhanced AV-HuBERT
\title{Practice of the conformer enhanced AUDIO-VISUAL HUBERT on Mandarin and English}
%
% Single address.
% ---------------
% \name{Xiaoming Ren\thanks{Thanks to XYZ agency for funding.}}
\name{Xiaoming Ren,  Chao Li, Shenjian Wang, Biao Li}
\address{Beijing OPPO telecommunications corp., ltd., Beijing, China}
%
% For example:
% ------------
%\address{School\\
%	Department\\
%	Address}
%
% Two addresses (uncomment and modify for two-address case).
% ----------------------------------------------------------
%\twoauthors
%  {A. Author-one, B. Author-two\sthanks{Thanks to XYZ agency for funding.}}
%	{School A-B\\
%	Department A-B\\
%	Address A-B}
%  {C. Author-three, D. Author-four\sthanks{The fourth author performed the work
%	while at ...}}
%	{School C-D\\
%	Department C-D\\
%	Address C-D}
%
\begin{document}
\ninept
%
\maketitle
%
\begin{abstract}
Considering the bimodal nature of human speech perception, lips, and teeth movement has a pivotal role in automatic speech recognition. Benefiting from the correlated and noise-invariant visual information, audio-visual recognition systems enhance robustness in multiple scenarios. In previous work, audio-visual HuBERT appears to be the finest practice incorporating modality knowledge. This paper outlines a mixed methodology, named conformer enhanced AV-HuBERT, boosting the AV-HuBERT system's performance a step further. Compared with baseline AV-HuBERT, our method in the one-phase evaluation of clean and noisy conditions achieves 7\% and 16\%  relative WER reduction on the English AVSR benchmark dataset LRS3. Furthermore, we establish a novel 1000h Mandarin AVSR dataset CSTS. On top of the baseline AV-HuBERT, we exceed the WeNet ASR system by 14\% and 18\% relatively on MISP and CMLR by pre-training with this dataset. The conformer-enhanced AV-HuBERT we proposed brings 7\% on MISP and 6\% CER reduction on CMLR, compared with the baseline AV-HuBERT system.
\end{abstract}


% Our approach outperforms baseline AV-HuBERT by CWER 7\% and  NWER 16\% on the English AVSR benchmark dataset LRS3 for the first phase. 
%We propose: 1) an informative audio feature representation method  2) a more delicate visual extractor to capture visual stream  3) a gated approach leading to better modality fusion 4) a conformer-based architecture evolved from AV-HuBERT.
%
\begin{keywords}
AV-HuBERT, AVSR, Conformer, Modality Fusion
\end{keywords}
%
\section{Introduction}
\label{sec:intro}
In recent years, automatic speech recognition (ASR) systems have seen considerable improvements with the help of innumerable neural networks and models \cite{Hinton2012DeepNN}, which reach or exceed mankind in several scenarios, especially low-noise, near-field situations \cite{Nguyen2021SuperHumanPI}. A considerable amount of literature has been published on end-to-end approaches \cite{Kim2017JointCB}\cite{Graves2012SequenceTW}\cite{Chan2015ListenAA}. Among those attempts, attention-based architectures seem to become prevailing and receive rave reviews, such as conformer \cite{Gulati2020ConformerCT}, which is able to learn local context and long-period connections when modeling sequences.  

% Audio stream is vulnerable to the presence of noise. 
Nevertheless, ASR performance degrades inevitably due to various external disturbances, considering the vast applicability range of speech recognition technology. However, lip movements highly correlated to human speech are not benefited from the correlated and noise-invariant visual information, audio-visual recognition systems (AVSR) enhance robustness in multiple scenarios. By capturing the delicate relationship between audio and visual information, the treasure behind would be exploited.

With the availability of well-designed end-to-end architectures, the help of a growing number of multimedia datasets \cite{Afouras2018LRS3TEDAL}\cite{Chung2018VoxCeleb2DS} and the fusion of visual and audio modalities \cite{Sterpu2018AttentionbasedAF}\cite{Sterpu2020HowTT}, AVSR has remarkably made significant progress in recent years. Audio-Visual Hidden Unit BERT (AV-HuBERT) \cite{Shi2022LearningAS}, a self-supervised AVSR framework with a pre-training and fine-tuning stage, seems to bring AVSR performance to a new level.  

In this paper, we present a mixed methodology to boost AV-HuBERT system CER performance further. Firstly, by adopting the 80-dim filterbank feature, audio knowledge is more thoroughly and effectively captured. Secondly, a modified version of ResNet Encoder \cite{Stafylakis2017CombiningRN} which is pre-trained deliberately with an abundant number of multimedia resources, is employed to extract visual information. Then, we innovatively suggest a modality fusion method with a gating mechanism. Audio along with visual information is adopted in the design of the fusion gate, balancing audio knowledge going through the system. Finally, we investigate the usage of the conformer instead of the transformer, as mentioned above, which reduces deletion error in speech recognition and thus helps our system evolve further. 
% ()我们还提出了一个中文多模态数据集,包含了1000h左右的中文音视频数据。我们使用AV-HuBERT在这个数据集训练了一个中文预训练模型,在CMLR和MISP数据集上finetune得到的结果比wenet ASR的CER相对降低了18%。我们进一步使用提出的enhanced AV-HuBERT,相比baseline AV-HuBERT在在CMLR和MISP数据集上CER相对下降了3.3%。
We also establish a Mandarin multimodal dataset containing 1000h Mandarin audio and visual data. A Mandarin model based on AV-HuBERT is pre-trained using this dataset, which outperforms WeNet by 14\% and 18\% on MISP and CMLR after fine-tuning.
Furthermore, we propose the enhanced AV-HuBERT, which boosts baseline AV-HuBERT performance by a relative 7\% and 6\% on MISP and CMLR.


The rest of this paper is organized as follows. Section 2 reviews recent works in the AVSR domain. Section 3 profoundly presents our methodology. Section 4 introduces the dataset and how we conduct those experiments. The result and related analysis are presented in Section 5. Section 6 concludes and also discusses further work.



\section{Related Work}
\label{sec:format}
\noindent\textbf{AV-HuBERT.}
It has previously been observed that modern novel neural networks are craving hand-labeled data for training since they are fully supervised. Most AVSR systems were no exception. Encouragingly, \cite{Shi2022LearningAS}\cite{Shi2022RobustSA} proposed a robust self-supervised AVSR framework AV-HuBERT, capturing modality information from human speech and lip movement at the same time and achieved state-of-the-art on AVSR benchmark dataset LRS3 \cite{Afouras2018LRS3TEDAL}. Feature clustering and masked prediction are two impressive aspects of the pre-training stage. In this paper, our work is built on top of AV-HuBERT, continuing to evolve and grow to a more robust AVSR system.

\noindent\textbf{Conformer.}
Attention-based transformer plays a critical role and seemed to become prevailing. Nonetheless, its variant conformer, which can capture local context through convolution layers and long-term relationships, is even better. In this paper, we adopt a conformer encoder instead of the transformer in the pre-training stage, hoping to learn the nuanced correlation and local information. 

Standard conformer block \cite{Gulati2020ConformerCT} is composed of four modules, including a feed-forward module, a self-attention module, a convolution module, and a second feed-forward module. The two feed-forward modules sandwich the multi-headed self-attention module and the convolution module.  

Formally, for input \(\mathbf{x}_{i}\)
to a conformer block \(i\), the output \(\mathbf{y}_{i}\) of the block is computed as below:
% 

\begin{align}
\widetilde{\mathbf{x}}_{i} &= \mathrm{LN}\left(\mathbf{x}_{i} + \frac{1}{2}\mathrm{FFN}(\mathbf{x}_{i})\right) \\
{\mathbf{x}_{i}}' &= \mathrm{LN}(\widetilde{\mathbf{x}}_{i} + \mathrm{MHSA}(\widetilde{\mathbf{x}}_{i})) \\
{\mathbf{x}_{i}}'' &= \mathrm{LN}({\mathbf{x}_{i}}' + \mathrm{Conv}({\mathbf{x}_{i}}')) \\
{\mathbf{y}_{i}} &= \mathrm{LN}\left({\mathbf{x}_{i}}'' + \frac{1}{2}\mathrm{FFN}({\mathbf{x}_{i}}'')\right)
\end{align}

% \begin{equation}
%   \widetilde{\mathbf{x}}_{i} = \mathrm{LN}\left(\mathbf{x}_{i} + \frac{1}{2}\mathrm{FFN}(\mathbf{x}_{i})\right)
%   \label{eq1}
% \end{equation}
% % 
% % 
% \begin{equation}
%   {\mathbf{x}_{i}}\\' = \mathrm{LN}(\widetilde{\mathbf{x}}_{i} + \mathrm{MHSA}(\widetilde{\mathbf{x}}_{i}))
%   \label{eq2}
% \end{equation}
% % 
% % 
% \begin{equation}
%   {\mathbf{x}_{i}}\\'\\' = \mathrm{LN}({\mathbf{x}_{i}}\\' + \mathrm{Conv}({\mathbf{x}_{i}}\\'))
%   \label{eq3}
% \end{equation}
% % 
% % 
% \begin{equation}
%   {\mathbf{y}_{i}} = \mathrm{LN}\left({\mathbf{x}_{i}}\\'\\' + \frac{1}{2}\mathrm{FFN}({\mathbf{x}_{i}}\\'\\')\right)
%   \label{eq4}
% \end{equation}
% 
where FFN refers to the feed-forward module, MHSA refers to the multi-head self-attention module, Conv refers to the
convolution module and LN refers to the layer norm module.

\begin{figure}[t]
  \centering
  \includegraphics[width=\linewidth]{main_img3.png}
  \caption{Attention information visualization between different layers and heads, indicates the diversity of information, and complementarity.}
  \label{fig:speech_production}
\end{figure}

\section{THE PROPOSED METHOD}
% \label{sec:pagestyle}
\label{ssec:subhead}


\subsection{Audio Feature}
As we repeat the AV-HuBERT research and reproduce the result on speech recognition,  it is interesting to note that in the previous experiments, the audio feature employed in the pre-training process is 26-dim filterbanks. Empirically, a higher dimension gives a better representation of knowledge in several demanding and complex situations. Hence, we adopt 80-dim filterbanks as many end-to-end systems do. Concerning this adjustment, the performance sees a stable improvement. Additionally, when carrying out the research, features computed from a 15ms window outperform those from a 25ms window. 
\subsection{Visual Feature Extractor}
As the filterbank feature is popular for the audio stream, the visual information is extracted with front-end 3D ResNet18, whose parameters are learned during the model training stage. This visual extractor seems to more or less rely on the quality and quantity of its data set, which tends to be vulnerable and ephemeral in certain cases. We employ light-weighted MobileNet-V2 \cite{Sandler2018MobileNetV2IR} instead of ResNet.

\subsection{Gated Fusion}
Enlightened by \cite{Yu2020AudioVisualRO}, a mixed modality fusion methodology is introduced to our enhanced AV-HuBERT system. The paper above demonstrates a comparison between straight concatenation of modality features and fusion with gating mechanism \cite{Dauphin2017LanguageMW}. We believe simple concatenation may lead to unreliable results when the visual flow is of low quality or not synchronized with audio. Hence, the audio and visual information are combined to control the audio information flow with the help of a GLU gate. At this point, the chosen audio knowledge is then incorporated with its visual part as the input of the pre-trained encoder. 
We incorporate this scheme into the AV-HuBERT model. The structure of fusion is shown in Fig.1. FusionNet output is \(\mathbf{m}_{t}\).

\begin{align}
\mathbf{m}_{t} &= (\mathrm{concat}(\mathbf{v}_t,\mathbf{a}_t))\ast \mathbf{U} + \mathbf{a} \\
\mathbf{h}_{t} &= (\mathbf{a}_t\ast \mathbf{W}+\mathbf{b} )\otimes  \sigma (\mathbf{m}_t\ast \mathbf{V}+\mathbf{c})
\end{align}

% % 
% \begin{equation}
%   \mathbf{m}_{t} = (\mathrm{concat}(\mathbf{v}_t,\mathbf{a}_t))\ast \mathbf{U} + \mathbf{a}
% \end{equation}
% % 

% % 
% \begin{equation}
%   \mathbf{h}_{t} = (\mathbf{a}_t\ast \mathbf{W}+\mathbf{b} )\otimes  \sigma (\mathbf{m}_t\ast \mathbf{V}+\mathbf{c})
% \end{equation}
where \(\mathbf{U}\in \mathbb{R}^{2D\times D}\),\(\mathbf{W}\in \mathbb{R}^{D\times D}\),\(\mathbf{V}\in \mathbb{R}^{D\times D}\), \(\{\mathbf{a},\mathbf{b},\mathbf{c}\} \in \mathbb{R}^{D}\)  are the bias, \(\sigma\) refers to the sigmoid function, \(\otimes\) denotes the Hadamard product.

\subsection{Conformer Encoder}
Focusing back on the encoder in the original AV-Hubert architecture, one thing to note is that the transformer encoder employs convolution for its position encoding. We find it not compatible with conformer structure, and hence we employ the relative sinusoidal positional encoding scheme which proved to be an important technique from Transformer-XL \cite{Dai2019TransformerXLAL}. The relative positional encoding scheme supports the self-attention module on different input lengths, leading to robustness towards various utterance lengths. The effectiveness of this adjustment is demonstrated later. Moreover, blockformer \cite{Ren2022ImprovingMS} supplies the last output layer with adequate previous block output, which indicates the potential to be of benefit. Surprisingly, we also find that the conformer system reduces the reduction error in speech recognition, compared with the transformer system.

\begin{CJK*}{UTF8}{gbsn}
\begin{table}[t]\footnotesize
  \caption{\noindent{Composition of the CSTS dataset.} }
  \label{tab:word_styles2}
  \centering
  \begin{tabular}{|c|c|c|c|c|}
    \hline


    source & \makecell[c]{raw \\ hours} & \makecell[c]{processed \\ hours}  & utt & spks                 \\
    \hline
     CC Forum   & {215}    & 101 & 50670 & 791 \\
     Topics in Focus   & {243}    & 69 & 33141 & 3601  \\
     Yixi Lecture   & {39}    & 13 & 6043 & 91  \\
     Du Talk-Show   & {161}    & 117 & 59017 & 694  \\
     Legal Report   & {1997}    & 598 & 317154 & 14209  \\
     Rock\&Roast   & {73}    & 7 & 3451 & 69  \\
     The News Broadcasting   & {482}    & 82 & 38023 & 672  \\
     News Live Room & 298 & 69 & 33321          & 548        \\
      \hline
     Total & 3508 & 1056 & 540820 & 20675 \\


    \hline
  \end{tabular}
\end{table}
\end{CJK*}
\section{ Experimental setting}

Our experiments are based on four datasets, including:

\noindent\textbf{LRS3 .}
The dataset consists of over 400 hours of video, extracted
from 5594 TED and TEDx talks in English, downloaded from
YouTube. The dataset is organized into three sets: pre-train, train-val
and test. The first two overlap in terms of content but the last is
completely independent. It is the largest publicly available labeled audio-visual speech recognition dataset.

\noindent\textbf{CSTS.} We collect a 1000h unsupervised Chinese audio-visual dataset containing 200k individuals, which will release with the paper. Since it contains only one speaker at a time, we name this dataset Chinese Solo Talk Show (CSTS). The CSTS dataset is organized into three sets: pre-train, train-val, and test. %()现在我们已经完成了pre-train的数据,train-val and test的数据还在处理中。
% The pre-training dataset is now ready to release while train-val and test are on the way.

The cropped face tracks are provided as .mp4 files. The audio tracks are provided in single-channel 16-bit 16kHz format.
% ()Table 1 展示了CSTS数据集的来源、时长、句数还有说话人数量。我们从中文视频网站拉取了演讲访谈类和新闻类节目的视频,经过face detection,face clustering,face recognition,VAD等处理后得到了该数据集。
Table 1 demonstrates the source, period, total sentence, and number of speakers. We collect speeches, interviews, and news programs from Chinese websites. After face detection, face clustering, face recognition, and VAD, the dataset is available and ready for use. 

\noindent\textbf{CMLR.} 
It is a large Chinese Mandarin Lip Reading dataset (CMLR), designed to facilitate research on visual speech recognition, sometimes also referred to as automatic lip reading. More than 100,000 spoken sentences from 11 speakers of the national news program “News Broadcast” are included, up to an estimated total length of 61+h. More than 3,000 Chinese characters and 20,000 phrases appear in CMLR.
This dataset includes train, dev, and test sets. Following regular data allocation, we use the train and dev set(up to 61h) for fine-tuning and the 17h test set for decoding.

\noindent\textbf{MISP.}
This dataset is from the Multimodal Information Based Speech Processing Challenge 2021 (MISP), which contains 100+h of audio-visual data. The dataset is divided into the near, middle, and far scenarios, and recorded with 30 rooms, and 248 participants condition. The near-field audio is recorded individually via high-fidelity hardware and the middle scenario video is synchronized with the audio. Thus, we use near-scenario audio along with middle-range video, excluding far scenarios.

Generally speaking, our experiments are consistent with AV-HuBERT \cite{Shi2022LearningAS}, including pre-training and fine-tuning stages. The pre-training stage learns feature representation through unsupervised audio and visual data, while fine-tuning uses labeled audio-visual pairs, according to the parameters of the labeled data.

\subsection{Pre-Training}
\noindent\textbf{Pseudo labels.}
%() 我们的实验分为一阶段验证试验和全流程实验,一阶段验证试验我们使用的伪标签数量分别是100。全流程的实验和AV-HuBERT相同。
We divide our experiments into the one-phase evaluation and the five-phase operation. We use 100 pseudo labels in one-phase evaluation, while the five-phase is the same as AV-HuBERT. 

In five-phase, we iteratively increase the label number from 100, 100, 500, 1000, to 2000 at last. Note that in the first phase, we only use MFCC-39 from speech frames, in the other phases the transformer encoder's output from the previous phase is the input for feature clustering. 

\noindent\textbf{Audio feature.}
Filterbanks are for feature extraction. We compare fbank26 with fbank80, window length of 25ms with 15ms. If not mentioned, we adopt fbank26+win25. 

\noindent\textbf{Visual feature.}
We experiment with two methods of extracting visual features: resnet refers to ResNet18; mobilenet refers to MobileNet-v2.

\noindent\textbf{Modality fusion.}
In feature fusion experiments, we compare simple concatenation with the gated fusion method. In Table 3 the GLU refers to the gated way.

\noindent\textbf{Backbone.}
Transformer vs conformer backbone is compared in this paper. The Transformer encoder shares the same structure with AV-HuBERT while the conformer encoder uses relative position encoding, stacking 12 conformer blocks. In detail, the encoder embedding dimension is set to 768, the layer drop is 0.05, and the remaining keep the same with the transformer. The total parameter in the transformer is 103M and 183M for the conformer.

% ()我们在LRS3数据集上的实验都是一阶段验证试验,使用433h数据预训练,30h数据finetune,训练过程基于8 A100 GPUs with max tokens 1000。一阶段总共花费64h,包含400k steps。
% 我们在中文数据集上与wenet asr 对比的实验是全流程实验。enhanced AV-HuBERT的消融实验都是一阶段验证试验。
We conduct one-phase evaluations on LRS3 with 433h pre-training and 30h fine-tuning. The training stage is based on 8 A100 GPUs with max tokens of 1000. It takes 64h, containing 400k steps.

The comparison with WeNet ASR on the Mandarin set is based on five-phase operations. Enhanced AV-HuBERT ablation studies are just one phase. Our Mandarin pre-training models are all built upon a 1000h CSTS dataset and fine-tuned on 100h MISP and 61h CMLR afterward. Our model is trained on 4 A100 GPUs with max tokens 2500, one batch size of data containing at most 100 seconds of audio per GPU. It takes around 32 hours to finish a phase, where 400k steps are. We do not use accumulated gradient, update freq is set to 1.


%我们的中文预训练模型都是在1000h的CSTS数据集上pre-train得到的,然后分别在100h的MISP数据集和61h的CMLR数据集上进行finetune。
 
%()我们所有实验的预训练和finetune过程都没有加噪,解码的时候加入了随机噪声。(The noise audio clips in the categories of ”natural”, ”music” and ”babble” are sampled from MUSAN dataset \cite{Snyder2015MUSANAM})

One thing to note is that, among all pre-training and fine-tuning stages, there is no extra noise added. Random noise is included in the decoding stage. The noise audio clips in the categories of ”natural”, ”music” and ”babble” are sampled from the MUSAN dataset \cite{Snyder2015MUSANAM}.

\subsection{Supervised Fine-Tuning and Decoding}

As for the fine-tuning stage, we use the attention-based sequence-to-sequence cross-entropy
loss \cite{Bahdanau2016EndtoendAL}. Two modalities are included in fine-tuning. The modeling unit here for LRS3 is unigram-based subword units \cite{Kudo2018SubwordRI}. The vocabulary size is 1000.
% ()我们使用音视频双模态数据进行finetune。
% 对于LRS3数据集我们使用的建模单元是unigram-based subword units \cite{Subword regularization: Improving neural network translation models with multiple subword candidates.}(to do).The vocabulary size is 1000.


As for the Mandarin dataset, Chinese characters are employed as the modeling unit. Linguistic knowledge is learned by the seq2seq transformer decoder of its own accord. 
% ()中文和英文数据集的finetune过程都是使用以下训练参数。
Parameters for Mandarin, as well as English datasets in the fine-tuning stage, are listed below.

No extra language model is added during inference. It takes 30k steps to fine-tune the six-layer transformer decoder where in the first 24k steps, parameters in the pre-trained model are frozen. In the remaining 6k steps, we unfreeze the parameters. Other settings are similar to AV-HuBERT.

Warming up is set to 10k and learning is 0.001, without accumulated gradient. Adam \cite{Kingma2015AdamAM} optimizer is used with \(\beta\)  = (0.9, 0.98).
We use 8 A100 GPUs in fine-tuning process, with the max token set to 1000, which means a batch size of at most 40 seconds. 

\begin{table}[t]\footnotesize
  \caption{\noindent{Ablation study of the enhanced AV-HuBERT on LRS3 with update freq 1 for first phase (WER:\%). The baseline is AV-HuBERT. All is composition of fbank80, win15, GLU and mobilenet.}}
  \label{tab:word_styles2}
  \centering
  \begin{tabular}{|c|c|c|c|c|c|}
    \hline
  \multirow{2}*{\textbf{Method}} &  \multicolumn{2}{|c|}{\textbf{Audio-only}} &  \multicolumn{2}{|c|}{\textbf{Audio-visual}} \\
   ~ & \textbf{CWER}& \textbf{NWER} & \textbf{CWER}& \textbf{NWER}   \\
    \hline
    baseline             & 23.8   & 87.52  & 15.88  & 46.97          \\
   % baseline+conformer   & 19.04  & 57.72  & 16.75  & 42.03          \\
    baseline+fbank80     & /      & /      & 15.24  & 46.26          \\
    baseline+win15       & 28.65  & 82.74  & 15.66  & 44.43          \\
    baseline+GLU         & 21.07  & 71.42  & 16.31  & 49.59          \\
    baseline+mobilenet   & 21.84  & 69.25  & 16.36  & 48.15          \\
    baseline+all        & 19.81  & 68.70  & 14.34  & 43.95          \\

    \hline
  \end{tabular}
\end{table}

\begin{table}[t]\footnotesize
  \caption{\noindent{Ablation study of the enhanced AV-HuBERT on LRS3 with update freq 4 for first phase (WER:\%). The implication of baseline and all is the same as Table 2.} }
  \label{tab:word_styles2}
  \centering
  \begin{tabular}{|c|c|c|c|c|c|}
    \hline
  \multirow{2}*{\textbf{Method}} &  \multicolumn{2}{|c|}{\textbf{Audio-only}} &  \multicolumn{2}{|c|}{\textbf{Audio-visual}} \\
   ~ & \textbf{CWER}& \textbf{NWER} & \textbf{CWER}& \textbf{NWER}   \\
    \hline
    baseline             & 17.59  & 82.04  & 12.56  & 44.39          \\
    baseline+conformer   & 15.08  & 59.54  & 13.09  & 39.25          \\
    baseline+conformer+all         & 14.93  & 82.18  & 11.66  & 37.09          \\

    \hline
  \end{tabular}
\end{table}

\section{Experimental results \& Analysis}

% ()在表二中,我们在LRS3数据集上验证了每一个增强方法的效果。其中Audio-only指的是解码只使用audio的特征,Audio-visual指的是解码使用了Audio-visual特征。CWER指的是clean情况下的WER,NWER指的是加入了随机噪声解码的WER。我们发现fbank80和win15的AV的WER在clean和noisy的情况都有降低。GLU和mobilent的A的WER在clean和noisy的情况都有降低,但是AV的WER有所升高。
% 最后一行我们综合了以上所有的增强方法,发现A和AV的WER在clean的情况下相对下降了16.8%和9.7%,在noisy的情况下相对下降了21.5%和6.4%。
In Table 2, every boosting method is verified on LRS3, where 'audio-only' refers to the case when the audio feature is employed for decoding. By 'audio-visual', we use both audio and visual features at the same time. CWER refers to CER in a clean environment and NWER is the WER with random noise in the decoding stage. We find that, in fbank80, win15 conditions, WER reduces for both clean and noisy cases. With GLU and mobilenet, audio-only WER receives a reduction in both cases. However, the word error rate grows with audio-visual in clean or noisy experiments.
We combine all the boosting methods and the result is shown in the last row. Clean WER drops up to 16.8\% and 9.7\% relatively with audio and audio-visual features respectively. Noisy WER drops relatively by 21.5\% and 6.4\% respectively.


\begin{table}[t]\footnotesize
  \caption{\noindent{Ablation study of the enhanced AV-HuBERT pre-training on CSTS for the first phase and fine-tuning on MISP (CER:\%).}  }
  \label{tab:word_styles2}
  \centering
  \begin{tabular}{|c|c|c|c|}
    \hline
    \textbf{Backbone} & \textbf{Method} & \textbf{CER} \\
    \hline
    \multirow{5}*{transformer}     & fbank26+win25+concat+resnet & 18.93                  \\
    ~    & \textbf{fbank80}+win25+concat+resnet & 18.20                 \\
    ~    & fbank80+\textbf{win15}+concat+resnet & 17.68                  \\
    ~    & fbank80+win15+\textbf{GLU}+resnet & 17.12                  \\
    ~    & fbank80+win15+GLU+\textbf{mobilenet} & 16.92               \\
    \hline
    \multirow{5}*{conformer}     & fbank26+win25+concat+resnet & 16.12                    \\
    ~    & \textbf{fbank80}+win25+concat+resnet & 15.64                 \\
    ~    & fbank80+\textbf{win15}+concat+resnet & 15.41                \\
    ~    & fbank80+win15+\textbf{GLU}+resnet & 14.25                 \\
    ~    & fbank80+win15+GLU+\textbf{mobilenet} & 14.26                 \\

    \hline
  \end{tabular}
\end{table}



\begin{table}[t]\footnotesize
\centering
  \caption{\noindent{Comparison of models, which is pre-training on CSTS for the first phase and fine-tuning on MISP, with different noise type (CER:\%).}}
  \label{tab:word_styles2}
  \begin{tabular}{|c|c|c|c|c|}
     \hline
 \multirow{2}*{\textbf{Model}} & \multirow{2}*{\textbf{Mode}} & \multicolumn{2}{|c|}{\textbf{SNR=5}}&\multirow{2}*{\textbf{clean}}  \\
 \cline{3-4}
  ~ & ~ &  \ \textbf{All} & \textbf{Babble}&~  \\
 \hline
AV-HuBERT   & A  & 92.40 & 116.75 & 34.56  \\
AV-HuBERT   & AV   & 45.34 & 54.80& 18.93  \\
conformer enhanced AV-HuBERT   & A  & 55.19 & 74.78  & 17.25 \\
conformer enhanced AV-HuBERT   & AV   & 30.48 & 38.40& 14.26 \\
 \hline
  \end{tabular}
\end{table}

\begin{table}[t]\footnotesize
  \caption{\noindent{Comparison between WeNet, AV-HuBERT, and enhanced AV-HuBERT. The last two models are pre-training for five phases and fine-tuning on MISP and CMLR (CER:\%).}  }
  \label{tab:word_styles2}
  \centering
  \begin{tabular}{|cc|c|c|c|}
    \hline
    \textbf{Model} & \textbf{Backbone} & \textbf{Labeled} &  \textbf{Unlabeled} & \textbf{CER} \\
    \hline

    {WeNet}    & {conformer} & 100hrs & - & 15.04                 \\
    AV-HuBERT    & transformer & 100hrs & 1000hrs & 12.95                  \\
    enhanced AV-HuBERT    & transformer & 100hrs & 1000hrs & 12.05                  \\
    \hline
    {WeNet}    & {conformer} & 61hrs & - & 3.65                  \\
    AV-HuBERT   & transformer & 61hrs & 1000hrs & 3                  \\
    enhanced AV-HuBERT    & transformer & 61hrs & 1000hrs & 2.82                  \\

    \hline
  \end{tabular}
\end{table}

% 通过表三我们发现将update freq设置为4后,由于epoch数增加了4倍,对比表2,baseline 的A和AV的WER都有明显的下降。同时我们发现加入conformer和所有的增强方法后除了A的NWER,其他的WER有7~16%的相对下降。
In Table 3, we set the update frequency to 4, and leads to 4 times epoch growth. Compared with Table 2, the baseline WER in audio-only and audio-visual cases drops obviously. At the same time, we find that by incorporating conformer and all other boosting methods, WER can decrease around 7-16\% except NWER in the audio-only experiment. 

In Table 4, we conduct a series of experiments on the transformer and conformer backbone group, where 4 methods are presented for each backbone, including filterbank dimension, frame length, visual extractor category, and audio-visual fusion type. Our baseline sets filterbank to 26-dim, and widow size to 25ms. Only one factor is compared to each line in each backbone experiment. For the reader's convenience, we present our results in an increasing way of performance. Conclusions are:

1) Filterbank with a higher dimension is relatively better(80-dim vs 26-dim) with 3.8\% relative CER improvement. Increasing the number of filters when extracting audio features leads to more sufficient knowledge and thus better feature representation.

2) Decrease frame length boosts performance, where 15ms outperforms 25ms window by 2.8\% relative CER improvement. One unanticipated finding was that our outcome is contrary to the instinct longer frame length was considered to provide better feature representation where more information lies in there.

3) Gated fusion purposefully accepts audio-visual information flow, which exceeds vanilla concatenation by 3\%-7\% on relative CER performance, from transformer and conformer experiments. 

4) Mobilenet improves lightly and we think the main reason is that mobilenet has 7M fewer parameters than resnet18.

In Table 5, we compare the robustness of the conformer-enhanced AV-HuBERT and AV-HuBERT under various noise types with a 5dB signal-to-noise ratio (SNR) in the first phase. All means noise will be selected randomly from babble, music, and speech noise. The conformer-enhanced AV-HuBERT uses an 80-dim fbank, 15ms window size, gated fusion, and MobileNet. It can be seen that the conformer-enhanced AV-HuBERT outperforms AV-HuBERT no matter in clean or different noisy conditions. AV is better than A by adding visual modality. The experimental results show that our proposed conformer-enhanced AV-HuBERT does have a significant improvement over AV-HuBERT.

In Table 6, we compare the performance of the AV-HuBERT baseline model, the enhanced AV-HuBERT model, and the single audio modality-trained WeNet model.
% ()可以看出AV-HuBERT由于使用了1000h数据预训练并且多了一个模态,其在两个测试集对比wenet CER相对下降14%和17%。The proposed enhanced AV-HuBERT相比baseline AV-HuBERT在两个测试集CER分别相对下降了7%和6%,可见我们构建的数据集和提出的增强方法都是有效的。
It can be seen that, due to extra modality in AV-HuBERT and 1000h data for pre-training, AV-HuBERT surpasses WeNet CER by relatively 14\% and 18\% on MISP and CMLR. Moreover, the proposed enhanced AV-HuBERT outperforms the baseline AV-HuBERT by 7\% and 6\%, which shows the effectiveness of our proposed boosting methods and the generalization of the CSTS dataset we established.


Looking into CER results, performance on CMLR is admirably up to 2.82\%, compared with that of MISP, which is 12.05\%. The logic behind this could be ascended to the nature of the dataset. More specifically, CMLR contains clear News reports and high-quality audio leading to better performance while MISP is relatively more challenging, with HeFei accent chatting data, background noise sometimes, and blur visual resources. 




% \begin{CJK*}{UTF8}{gbsn}
% \begin{table}[t]\footnotesize
%   \caption{\noindent\textbf{Badcase (CER\%).} Examples showing improvements in long utterances
% in our model, vs the baseline model. Blue indicates deletions,
% pink indicates substitutions and yellow indicate insertions }
%   \label{tab:word_styles2}
%   \centering
%   \begin{tabular}{cc}
%     \toprule


%     REF & {说 给 她 走 走 舞 蹈 这 一 条 路 学 习 如 果 加 文 化 课 文 化 课 学 习 好}                   \\
%     \hline
%      TRS   & {说 给 他 做    \ \ *\ \ *\ \ *\ \ *\ \ *\ \ *\ \ *\ \ *\ \ *                         如 果 加 文 化 课 文 化 课 学 习 好}                   \\
%       & Transformer                 \\

%     \bottomrule
%   \end{tabular}
% \end{table}
% \end{CJK*}


\section{CONCLUSION \& FUTURE WORK}
\label{sec:majhead}
% ()这篇论文主要研究了多模态预训练框架在中文和英文数据集上的应用。我们构建1000h的中文多模态数据集CSTS,并在此数据集上验证了AV-HuBERT加入预训练和视觉模态能够超过单一模态wenet ASR,CER在中文开源数据集MISP和CMLR分别相对下降了14%和17%。我们基于AV-HuBERT提出了conformer enhanced AV-HuBERT,相比前者也分别取得了相对7%和6%的CER下降。
This paper dives into the practice of multi-modality pre-training framework on Mandarin and English datasets. We establish a 1000h Mandarin multi-modality dataset, CSTS. With the help of CSTS, we verify that AV-HuBERT with one extra modality and pre-training stage can outperform WeNet with one modality. CER drops 14\% and 18\% relatively on the Mandarin dataset MISP and CMLR. Based on AV-HuBERT, we propose the conformer enhanced AV-HuBERT, which also surpasses the baseline by a relative 7\% and 6\% in CER.  
Further work will focus on:

\noindent\textbf{Attention-based AV alignment.}
Attention-based approaches are used in \cite{Chang2021MultiChannelTT} to automatically align audio with video. Referring to the methods in the paper above, self-attention will perform over audio and video respectively. Then, the matrix containing video information will play the role of key and value while the audio matrix will be treated as the query. Finally, they will learn the alignment through the cross-attention approach.

\noindent\textbf{Trainable audio convolution network.}
A trainable audio convolution network has the potential to take the place of a filterbank to extract audio features. Experiments from google \cite{Sainath2015LearningTS} also indicate a decent adjustment.

\noindent\textbf{AV-Confidence.}
\cite{Yu2021FusingIS} suggests the usage of audio-visual confidence, besides current features from lip movement and speech. A decision fusion net combines all modality information, including confidence, which would help our AVSR system we believe.

% References should be produced using the bibtex program from suitable
% BiBTeX files (here: strings, refs, manuals). The IEEEbib.bst bibliography
% style file from IEEE produces an unsorted bibliography list.
% -------------------------------------------------------------------------
\bibliographystyle{IEEEbib}
\bibliography{refs}

\end{document}


