\section{Problem Setup}
\label{sec:problem-setup}

In this section, we first clarify the definition of Data-free Meta-Learning (DFML) problem. We then discuss the meta testing procedure for DFML problem.


\subsection{Data-free Meta Learning Setup}

We are given a collection of pre-trained models. Each pre-trained model is trained to solve a specific task without accessing their training data. The goal of 
DFML is to learn general prior knowledge (\eg, sensitive  initialization) which can be adapted fast to new unseen tasks.
Here, we emphasize that the pre-trained models may have different architectures. Our method can work with arbitrary architectures of pre-trained models.


\subsection{Meta Testing}

During meta testing, several $N$-way $K$-shot tasks arrive together, which are called the target tasks. The classes appearing in the target tasks have never been seen during both pre-training and meta training. Each task contains a support set with $N$ classes and $K$ instances per class. 
The support set is used for adapting the meta initialization to the specific task. The query set is what the model actually needs to predict. 
Our goal is to adapt the model to the support set so that it can perform well on the query set. The final accuracy is measured by the average accuracy for these target tasks.
